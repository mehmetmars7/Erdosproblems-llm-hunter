\section{MO \#328165: Do two integral matrices generate a free group?}
\MO{328165}{Do two integral matrices generate a free group?}

\subsection*{1) FORMAL RESTATEMENT}
Fix $n\ge 1$. Consider the decision problem:
\begin{quote}
Input: two matrices $A,B\in \mathrm{GL}(n,\mathbb Z)$.\
Question: does the subgroup $\langle A,B\rangle\le \mathrm{GL}(n,\mathbb Z)$ isomorphic to the free group $F_2$?
\end{quote}
Formally, is there an algorithm which halts on every input $(A,B)$ and outputs YES iff $\langle A,B\rangle\cong F_2$?

\subsection*{2) QUICK LITERATURE/CONTEXT CHECK}
A closely related MO question (2011) states: for $n=1,2$ decidable because $\mathrm{GL}(n,\mathbb Z)$ is virtually free, while for $n\ge 3$ it is open.

\subsection*{3) ATTACK PLAN}
\begin{itemize}[leftmargin=*]
\item \textbf{Proof track:} attempt to reduce to known decidable problems in hyperbolic/virtually free groups for $n=2$.
\item \textbf{Disproof track:} attempt to encode an undecidable problem into freeness in $\mathrm{GL}(n,\mathbb Z)$ for some $n$.
\end{itemize}

\subsection*{4) WORK}
\paragraph{Trivial case $n=1$ (fully resolved).}
$\mathrm{GL}(1,\mathbb Z)=\{\pm 1\}$ is finite, so no subgroup generated by two elements is isomorphic to $F_2$. Algorithm: always output NO.

\paragraph{What is known but not reproved here.}
For $n=2$, literature indicates decidability of related ``freeness'' problems for $\mathrm{GL}(2,\mathbb Z)$ and decidability in virtually free groups. For $n\ge 3$, no general algorithm is known.

\subsection*{5) VERIFICATION}
The $n=1$ algorithm is correct and terminates. Extending beyond $n=1$ requires substantial algorithmic group theory (e.g. subgroup graphs/foldings), which is not reconstructed gap-free in this report.

\subsection*{6) FINAL}
\textbf{UNRESOLVED.}
\begin{itemize}[leftmargin=*]
\item (i) Fully proved partial result: decidable for $n=1$ (always NO).
\item (ii) First gap: a halting algorithm for $n\ge 3$ or an undecidability proof.
\item (iii) Next moves: (1) attempt to adapt virtually-free-group folding methods to $\mathrm{GL}(2,\mathbb Z)$ and then investigate embeddings for $n\ge 3$; (2) explore reductions from known undecidable problems in $\mathrm{GL}(4,\mathbb Z)$-subgroups; (3) seek a structural characterization of free subgroups in $\mathrm{GL}(n,\mathbb Z)$ that is checkable.
\item (iv) Minimal counterexample structure: smallest $n$ for which undecidability could plausibly occur is often $n=4$ in matrix semigroup problems; for groups this remains unclear.
\end{itemize}

