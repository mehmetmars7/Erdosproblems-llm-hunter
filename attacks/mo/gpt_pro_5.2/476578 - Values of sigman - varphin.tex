\section{MO 476578 --- Values of $\sigma(n)/\varphi(n)$}
\label{sec:mo476578}
\noindent\textbf{MathOverflow link:} \href{https://mathoverflow.net/questions/476578}{https://mathoverflow.net/questions/476578}.

\subsection*{1) FORMAL RESTATEMENT}
Let $\sigma(n)=\sum_{d\mid n} d$ and $\varphi(n)=|\{1\le k\le n: \gcd(k,n)=1\}|$.
Define
\[
R := \left\{\frac{\sigma(n)}{\varphi(n)} : n\in\mathbb{Z}_{\ge 1}\right\} \subset \mathbb{Q}_{\ge 1}.
\]
\textbf{Question.} Is $R=\mathbb{Q}\cap[1,\infty)$?

\subsection*{2) QUICK LITERATURE/CONTEXT CHECK}
The ratio $\sigma(n)/\varphi(n)$ is multiplicative and admits an Euler product expansion over prime powers. The surjectivity of such multiplicative sets onto all rationals is subtle.
As of writing, the MO question appears unanswered.

\subsection*{3) ATTACK PLAN}
\textbf{Proof track.}
\begin{itemize}
  \item Use the prime-power factorization
  \[\frac{\sigma(p^k)}{\varphi(p^k)} = \frac{p^{k+1}-1}{p^{k-1}(p-1)^2},\]
  and try to realize an arbitrary rational by choosing suitable prime powers.
\end{itemize}
\textbf{Disproof track.}
\begin{itemize}
  \item Find an arithmetic obstruction: a rational $r$ whose prime factorization cannot be matched by the constrained local factors above.
  \item Search computationally for ``missing'' rationals among small denominators.
\end{itemize}

\subsection*{4) WORK}
\textbf{PHASE 1 (tiny values).}
\begin{itemize}
  \item $n=1$ gives value $1$.
  \item For prime $p$, $\sigma(p)/\varphi(p)=(p+1)/(p-1)$, yielding infinitely many rationals close to $1$.
\end{itemize}


\subsection*{5) VERIFICATION}
All stated formulas for prime powers are standard and easily checked.
No claim of surjectivity or a counterexample is made.

\subsection*{6) FINAL}
\textbf{UNRESOLVED}.
\begin{itemize}
  \item (i) Partial results: explicit multiplicative formula; infinitely many values (e.g. $(p+1)/(p-1)$ for primes $p$) accumulating at $1$.
  \item (ii) First gap: either construct $n$ for an \emph{arbitrary} target rational $r\ge 1$, or prove an obstruction.
  \item (iii) Next moves: (1) attempt to solve $\sigma(n)/\varphi(n)=a/b$ by choosing primes with prescribed congruences; (2) study the image semigroup generated by local factors; (3) search systematically for a ``forbidden'' rational using modular constraints.
  \item (iv) A minimal counterexample, if it exists, might be a rational with a prime in the denominator that can only appear via $(p-1)^2$-factors in a way that forces extra numerator primes.
\end{itemize}

% -----------------------------------------------------------------------------
