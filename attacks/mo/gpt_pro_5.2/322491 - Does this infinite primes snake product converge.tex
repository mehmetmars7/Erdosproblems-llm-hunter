\section{MO \#322491: Does this infinite primes snake product converge?}
\MO{322491}{Does this infinite primes snake product converge?}

\subsection*{1) FORMAL RESTATEMENT}
Let $p_1=2<p_2=3<p_3=5<\cdots$ be the increasing sequence of primes. Define
\[
P:=\prod_{m=1}^{\infty} r_m,\qquad
r_m:=\begin{cases}
\dfrac{p_{2m-1}}{p_{2m}}, & m\text{ odd},\\[4pt]
\dfrac{p_{2m}}{p_{2m-1}}, & m\text{ even}.
\end{cases}
\]
Question: does the infinite product $P$ converge to a nonzero finite limit?

\subsection*{2) QUICK LITERATURE/CONTEXT CHECK}
No settled answer is indicated in the attached MO snapshot. The product is close to $1$ termwise because consecutive primes are close, but convergence requires cancellation in an alternating series of small terms.

\subsection*{3) ATTACK PLAN}
\begin{itemize}[leftmargin=*]
\item \textbf{Proof track:} analyze $\log P$ as an alternating series of $\log(p_{2m}/p_{2m-1})$; attempt to use alternating-series/Dirichlet/Abel tests with prime number theorem (PNT) error control.
\item \textbf{Disproof track:} attempt to show divergence by producing large oscillations in partial sums using irregular prime gaps.
\end{itemize}

\subsection*{4) WORK}
\paragraph{Lemma 1 (logarithmic reformulation).}
Define $b_m:=\log\big(p_{2m}/p_{2m-1}\big)>0$. Then the product $P$ converges to a nonzero finite limit if and only if the series
\[
S:=\sum_{m=1}^{\infty} (-1)^m b_m
\]
converges in $\mathbb R$; in that case $P=e^S$.
\emph{Proof.} The $m$th factor is $r_m=\exp\big((-1)^m b_m\big)$, so partial products satisfy $\log\prod_{m\le M} r_m=\sum_{m\le M}(-1)^m b_m$. Exponentiating gives the equivalence. \qed

\paragraph{Asymptotic size of $b_m$ (heuristic from PNT).}
Write $p_{k+1}=p_k+g_k$ with prime gap $g_k$. Then
\[
 b_m=\log\Big(1+\frac{g_{2m-1}}{p_{2m-1}}\Big)=\frac{g_{2m-1}}{p_{2m-1}}+O\Big(\frac{g_{2m-1}^2}{p_{2m-1}^2}\Big).
\]
Using the heuristic $g_k\approx \log p_k$ and $p_k\approx k\log k$, one expects $b_m\approx 1/m$, making $\sum (-1)^m b_m$ reminiscent of the alternating harmonic series.

\paragraph{Numerical check (non-rigorous).}
Direct computation of partial products (implemented by summing $\sum_{m\le M}(-1)^m\log(p_{2m}/p_{2m-1})$ using the first $400{,}000$ primes, i.e. up to $p_{400000}=5{,}800{,}079$) gives:\\
\[
\begin{array}{r|l}
M & \prod_{m=1}^M r_m \\
\hline
50{,}000 & 0.905191285755\dots \\
100{,}000 & 0.905738306775\dots \\
150{,}000 & 0.906545485333\dots \\
200{,}000 & 0.906751967752\dots
\end{array}
\]
This is consistent with convergence to a constant near $0.906\text{--}0.907$, but is not a proof.

\subsection*{5) VERIFICATION}
Lemma 1 is exact. The asymptotic discussion is only heuristic: even with known bounds on prime gaps, establishing convergence of $\sum (-1)^m b_m$ appears to require fine control of the variation of $b_m$.

\subsection*{6) FINAL}
\textbf{UNRESOLVED.}
\begin{itemize}[leftmargin=*]
\item (i) Fully proved partial result: Lemma 1 (exact equivalence to an alternating series).
\item (ii) First gap: a proof that $\sum (-1)^m \log(p_{2m}/p_{2m-1})$ converges (or diverges).
\item (iii) Next moves: (1) establish bounded variation or eventual monotonicity of $b_m$ (likely false); (2) prove an Abel/Dirichlet-type criterion using cancellations in prime gaps; (3) use explicit formula or deep results on correlations of prime gaps.
\item (iv) Minimal counterexample: large blocks where $b_m$ fluctuates so that partial sums drift (requires structured bias in prime gaps at odd indices).
\end{itemize}

