\section{``Three great cocycles'' in complex analysis as cohomology generators (MO 231823)}

\subsection*{1) FORMAL RESTATEMENT}
\paragraph{Ambiguity.}
The problem does not specify:
\begin{itemize}[leftmargin=2em]
\item the group/groupoid of maps $f$ (germs? local biholomorphisms of a domain? global automorphisms?),
\item the coefficient module (holomorphic functions? smooth functions? tensor densities on what manifold?),
\item regularity/locality assumptions on $D$ (continuous? local differential operator of bounded order? polynomial in jets?).
\end{itemize}
Without such constraints, ``only'' is not a well-posed claim.

\paragraph{Minimal corrected statement we can rigorously verify.}
We extract and prove the concrete functional identities:
\begin{enumerate}[leftmargin=2em]
\item[(S0)] $D_0(f):=\log f'$ satisfies the cocycle equation with $\lambda=0$.
\item[(S1)] $D_1(f):=(\log f')'\,dz=\frac{f''}{f'}\,dz$ satisfies it with $\lambda=1$.
\item[(S2)] $D_2(f):=S(f)\,(dz)^2$ with $S(f)=\dfrac{f'''}{f'}-\dfrac32\left(\dfrac{f''}{f'}\right)^2$ satisfies it with $\lambda=2$.
\end{enumerate}

\subsection*{2) QUICK LITERATURE/CONTEXT CHECK}
The question points to the circle-diffeomorphism case where $H^1(\mathrm{Diff}(S^1),F_\lambda)$ is one-dimensional for $\lambda=0,1,2$ and trivial otherwise (as discussed e.g. by Ovsienko--Tabachnikov).
A comment on the MO thread suggests Wagemann's paper ``Explicit formulae for cocycles of holomorphic vector fields with values in $\lambda$ densities'' as relevant background.

\subsection*{3) ATTACK PLAN}
\begin{itemize}[leftmargin=2em]
\item Prove (S0)--(S2) by direct computation (chain rules).
\item For the classification question, attempt either:
  (a) Lie algebra cohomology (Gelfand--Fuchs/Witt algebra) under locality assumptions; or
  (b) jet-group functional equations.
\item If full classification cannot be closed without additional hypotheses, report it as unresolved and isolate the first missing input (e.g. a theorem classifying local cocycles).
\end{itemize}

\subsection*{4) WORK}
We work in the setting of (sufficiently smooth) local diffeomorphisms of a 1-dimensional manifold or domain, so that derivatives up to order 3 exist and $f'\neq 0$.

\begin{lemma}[Derivative cocycle]
For composable maps $f,g$ with nonvanishing derivatives,
\[\log (g\circ f)'=\log g'\circ f + \log f'.\]
Hence $D_0(f)=\log f'$ satisfies
\[D_0(g\circ f)=D_0(g)\circ f + D_0(f),\]
which is the cocycle equation with $\lambda=0$.
\end{lemma}

\begin{proof}
By the chain rule, $(g\circ f)'=(g'\circ f)\,f'$. Taking logs gives the identity.
\end{proof}

\begin{lemma}[Nonlinearity cocycle]
Let $D_1(f)=\dfrac{f''}{f'}\,dz$. Then
\[D_1(g\circ f)=(f')\,(D_1(g)\circ f)+D_1(f),\]
which is the cocycle equation with $\lambda=1$.
\end{lemma}

\begin{proof}
Differentiate $\log (g\circ f)'=\log g'\circ f+\log f'$ to get
\[(\log (g\circ f)')'=(\log g')'\circ f\cdot f' + (\log f')'.
\]
Multiplying by $dz$ gives
\[\frac{(g\circ f)''}{(g\circ f)'}\,dz = f'\cdot \Big(\frac{g''}{g'}\circ f\Big)\,dz + \frac{f''}{f'}\,dz.
\]
This is precisely the claimed cocycle identity.
\end{proof}

\begin{lemma}[Schwarzian cocycle]
Let $S(f)=\dfrac{f'''}{f'}-\dfrac32\left(\dfrac{f''}{f'}\right)^2$ and $D_2(f)=S(f)(dz)^2$.
Then for composable $f,g$,
\[D_2(g\circ f)=(f')^2\,(D_2(g)\circ f)+D_2(f),\]
which is the cocycle equation with $\lambda=2$.
\end{lemma}

\begin{proof}
Write $h=g\circ f$.
We compute $S(h)$ in terms of $f$ and $g$.
By the chain rule:
\[h'=(g'\circ f)f',\quad h''=(g''\circ f)(f')^2+(g'\circ f)f'',
\]
\[h'''=(g'''\circ f)(f')^3+3(g''\circ f)f'f''+(g'\circ f)f'''.
\]
Divide by $h'=(g'\circ f)f'$ to obtain
\[\frac{h'''}{h'}=\Big(\frac{g'''}{g'}\circ f\Big)(f')^2+3\Big(\frac{g''}{g'}\circ f\Big)f''+\frac{f'''}{f'}.
\]
Similarly,
\[\frac{h''}{h'}=\Big(\frac{g''}{g'}\circ f\Big)f' + \frac{f''}{f'}.
\]
Square the latter and expand:
\[\left(\frac{h''}{h'}\right)^2=\Big(\frac{g''}{g'}\circ f\Big)^2 (f')^2 + 2\Big(\frac{g''}{g'}\circ f\Big)f'\cdot\frac{f''}{f'} + \left(\frac{f''}{f'}\right)^2.
\]
Now form
\begin{align*}
S(h)
&=\frac{h'''}{h'}-\frac32\left(\frac{h''}{h'}\right)^2\\
&=\Big(\frac{g'''}{g'}\circ f\Big)(f')^2+3\Big(\frac{g''}{g'}\circ f\Big)f''+\frac{f'''}{f'}\\
&\quad-\frac32\Big[\Big(\frac{g''}{g'}\circ f\Big)^2 (f')^2 + 2\Big(\frac{g''}{g'}\circ f\Big)f'' + \left(\frac{f''}{f'}\right)^2\Big]\\
&=\Big[\Big(\frac{g'''}{g'}-\frac32\Big(\frac{g''}{g'}\Big)^2\Big)\circ f\Big](f')^2 + \Big(\frac{f'''}{f'}-\frac32\left(\frac{f''}{f'}\right)^2\Big)\\
&=(S(g)\circ f)(f')^2 + S(f).
\end{align*}
Multiplying by $(dz)^2$ yields the desired cocycle equation for $D_2$.
\end{proof}

\subsection*{5) VERIFICATION}
All three computations require only that the relevant derivatives exist and that $f',g'\neq 0$ (so that divisions by $f'$ and $g'$ make sense). No additional analytic structure is used.

\subsection*{6) FINAL}
\textbf{UNRESOLVED.}

\paragraph{(i) Strongest fully proved partial result obtained here.}
We proved rigorously that the three displayed expressions satisfy the 1-cocycle equation for $\lambda=0,1,2$.

\paragraph{(ii) First gap.}
A full classification ``these are the only cocycles up to coboundary'' requires additional hypotheses (choice of group/groupoid, coefficient module, and regularity/locality conditions on $D$). Without those, the classification problem is not well-posed, and with natural hypotheses it appears to require nontrivial cohomology computations not carried out here.

\paragraph{(iii) Top 3 next moves.}
\begin{enumerate}[leftmargin=2em]
\item Fix the groupoid of germs of holomorphic local diffeomorphisms and the module of holomorphic $\lambda$-densities; assume $D$ is a local differential operator in the jet of $f$ of bounded order.
\item Pass to Lie algebra cohomology (holomorphic vector fields) by differentiating at the identity and invoke/establish the $H^1$ computation for density modules (Gelfand--Fuchs/Witt-type).
\item Compare with known explicit classifications (e.g. Wagemann) and check that their hypotheses match the intended ``complex analysis'' setting.
\end{enumerate}

\paragraph{(iv) What a minimal counterexample would likely look like.}
If one omits locality/regularity constraints on $D$, there may exist exotic (highly discontinuous or nonlocal) cocycles not cohomologous to these classical ones. Any counterexample would likely exploit such freedom.
