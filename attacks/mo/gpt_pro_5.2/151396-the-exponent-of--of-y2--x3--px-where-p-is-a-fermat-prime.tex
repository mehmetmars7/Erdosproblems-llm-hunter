\section{(MO 151396) Exponent of $\Sha$ for $y^2=x^3+px$ where $p$ is a Fermat prime}

\subsection{1) FORMAL RESTATEMENT}
Let $k\ge 1$ and let $p=F_k=2^{2^k}+1$ be a Fermat prime. Let
\[
E_p/\mathbb Q:\quad y^2 = x^3 + p x.
\]
Let $\Sha(E_p)$ denote the Tate--Shafarevich group of $E_p$ over $\mathbb Q$.

A natural precise conjectural statement consistent with the excerpt is:
\begin{quote}
(*) If $p=F_k$ is a Fermat prime, then $E_p(\mathbb Q)$ has rank $0$ and, assuming the Birch--Swinnerton-Dyer conjecture for $E_p$, one has
\[
\Sha(E_p)\ \cong\ (\mathbb Z/2^{k-1}\mathbb Z)^2.
\]
\end{quote}
Stress points:
\begin{itemize}
\item The claim is conditional on BSD unless independent arithmetic arguments are given.
\item For $k>4$, it is unknown whether $F_k$ is prime, so the quantifier ``for all Fermat primes'' is vacuous beyond the known list unless new Fermat primes are discovered.
\end{itemize}

\subsection{2) QUICK LITERATURE/CONTEXT CHECK}
This is a complex multiplication (CM) family: $E_p$ has $j$-invariant $1728$ and CM by $\mathbb Z[i]$. Special values $L(E_p,1)$ and Selmer groups are often accessible via Hecke characters and $2$-adic considerations. The MO page excerpt gives computational evidence under BSD for the four known Fermat primes $>3$.

I did not locate, during a quick web search, a definitive theorem proving (*) in general.

\subsection{3) ATTACK PLAN}
\paragraph{Proof track.} Use CM theory:
\begin{itemize}
\item Identify $E_p$ as a quadratic (or quartic) twist of the CM curve $E_1:y^2=x^3+x$.
\item Express $L(E_p,1)$ in terms of a Hecke $L$-series over $\mathbb Q(i)$ and analyze its $2$-adic valuation for $p=1+2^{2^k}$.
\item Combine with Kolyvagin/Coates--Wiles type results for CM curves to control rank and $\Sha$ (potentially requiring Iwasawa theory for the $2$-primary part).
\end{itemize}

\paragraph{Disproof track.} Find a Fermat prime $p$ for which BSD predicts a different $\Sha$ structure, or find a theoretical obstruction.

Given current knowledge (only five Fermat primes known and the data matches the pattern for the four $>3$), disproof by counterexample is not currently feasible.

\subsection{4) WORK (partial results)}
\subsubsection*{Elementary observations}
\begin{itemize}
\item $E_p$ has CM by $\mathbb Z[i]$ (since $j=1728$).
\item $E_p$ is a quadratic twist of $E_1:y^2=x^3+x$ by $p$.
\item The $2$-primary behavior of $\Sha(E_p)$ is naturally governed by $2$-descent and by the arithmetic of $\mathbb Q(i)$, where $p=1+2^{2^k}$ splits as $p=(1+2^{2^{k-1}}i)(1-2^{2^{k-1}}i)$.
\end{itemize}

\subsubsection*{Computational sanity checks}
I do not have Sage/Magma in this environment, so I did not recompute $\Sha$ for $p=5,17,257,65537$ here. The MO excerpt reports the observed pattern under BSD.

\subsection{5) VERIFICATION}
Any proof of (*) would need to verify:
\begin{itemize}
\item analytic rank $0$ (or algebraic rank $0$) for each such $p$,
\item an exact evaluation of $L(E_p,1)/\Omega$ and its $2$-adic valuation,
\item the Tamagawa factors and torsion subgroup,
\item and then deduce the precise group structure of $\Sha(E_p)$, not just its order.
\end{itemize}
I did not produce these steps here.

\subsection{6) FINAL}
\textbf{UNRESOLVED.}

\paragraph{(i) Strongest proved partial result.}
I established only the basic structural facts (CM by $\mathbb Z[i]$, splitting of $p$ in $\mathbb Z[i]$). No nontrivial bound/proof on $\Sha$ exponent was obtained.

\paragraph{(ii) First crisp gap.}
A precise theorem giving the $2$-adic valuation of $L(E_p,1)$ (or equivalently the BSD-predicted order of $\Sha(E_p)$) in terms of $k$ for $p=2^{2^k}+1$, proved unconditionally or under clearly stated hypotheses.

\paragraph{(iii) Top 3 next moves.}
\begin{enumerate}
\item Reduce $L(E_p,1)$ to Hecke $L$-values of characters on $\mathbb Q(i)$ and analyze the $2$-adic valuation using explicit Gauss/Jacobi sum formulas.
\item Apply CM-Iwasawa theory at $2$ to control the $2$-primary Selmer and $\Sha$.
\item For the group-structure claim $(\mathbb Z/2^{k-1})^2$, refine from order to structure via Cassels pairing or explicit $2^n$-descents.
\end{enumerate}

\paragraph{(iv) Likely structure of a minimal counterexample.}
A Fermat prime $p$ (if one exists beyond $65537$) where either rank$(E_p)>0$ or the $2$-primary part of $\Sha(E_p)$ has different exponent/order than predicted by (*).

