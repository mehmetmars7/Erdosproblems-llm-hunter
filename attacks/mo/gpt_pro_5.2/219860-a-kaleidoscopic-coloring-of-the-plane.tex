\section{Problem 219860: A kaleidoscopic coloring of the plane}

\subsection*{1) FORMAL RESTATEMENT}

\paragraph{Definitions.}
Let $D\subset\mathbb R^2$ be the closed unit disk centered at the origin, and let $\chi_D$ be its indicator function.
A measurable partition $\mathbb R^2=A\sqcup B$ induces the measurable function
\[
f := \chi_A-\chi_B,
\]
which satisfies $f(x)\in\{+1,-1\}$ for all $x$.

For $g\in L^1(\mathbb R^2)$ and measurable bounded $f$, the convolution is
\[
(f*g)(x)=\int_{\mathbb R^2} f(x-y)g(y)\,dy,
\]
which is well-defined since $g\in L^1$ and $f$ is bounded.

\paragraph{Question (equivalent forms).}
Does there exist a measurable $f:\mathbb R^2\to\{\pm 1\}$ such that
\[
(f*\chi_D)(x)=0\quad\text{for all }x\in\mathbb R^2?
\]
Equivalently, does there exist a measurable partition $\mathbb R^2=A\sqcup B$ such that every translate of $D$ contains equal Lebesgue measure of $A$ and $B$:
\[
\forall x\in\mathbb R^2,\quad
\lambda\bigl((x+D)\cap A\bigr)=\lambda\bigl((x+D)\cap B\bigr)=\frac12\lambda(D)?
\]

\paragraph{Stress points.}
\begin{itemize}
\item The condition is extremely rigid: every translate of the same disk has perfect ``balance''.
\item The operator $T(f)=f*\chi_D$ is a Fourier multiplier whose symbol has zeros (Pompeiu-type phenomena), so nontrivial solutions exist in other function classes (e.g.\ trigonometric functions), but the $\{\pm 1\}$ constraint is much stronger.
\end{itemize}

\subsection*{2) QUICK LITERATURE/CONTEXT CHECK (web available)}
The attached MO post frames the question in terms of existence of a measurable $\{\pm 1\}$-valued function annihilated by convolution with the unit disk indicator. This is closely related to the classical Pompeiu problem and to spectral properties of the disk's Fourier transform (Bessel function zeros). I did not find a definitive resolution of the $\{\pm 1\}$ measurable case in a quick scan.

\subsection*{3) ATTACK PLAN}

\paragraph{Proof-track (construct such an $f$).}
\begin{enumerate}[label=\textbf{P\arabic*.}]
\item Try periodic constructions: if $f$ is periodic, Fourier series reduces the equation to a frequency support condition $\widehat f(k)\widehat{\chi_D}(k)=0$ on the dual lattice; seek a $\{\pm 1\}$ periodic function whose spectrum lies in the zero set.
\item Try probabilistic constructions with hard constraints, perhaps using limiting arguments.
\end{enumerate}

\paragraph{Disproof-track (show impossible).}
\begin{enumerate}[label=\textbf{D\arabic*.}]
\item Prove rigidity under mild regularity: e.g.\ show no solution exists in $L^2(\mathbb R^2)$, then attempt to extend to bounded measurable $\{\pm 1\}$-valued by truncation/ergodic arguments.
\item Use Fourier-analytic obstructions: the zero set of $\widehat{\chi_D}$ has measure zero (union of circles), so any $L^2$ solution must vanish.
\end{enumerate}

\paragraph{Chosen path.}
I can give a complete and gap-free disproof in the \emph{$L^2$ setting} (no nonzero $L^2$ solution), and exhibit smooth bounded nontrivial (non-$\{\pm 1\}$) solutions. I cannot bridge the gap to measurable $\{\pm 1\}$-valued solutions.

\subsection*{4) WORK}

\begin{lemma}[Fourier transform of the unit disk indicator has measure-zero zero set]\label{lem:disk-ft-zeros}
Let $\chi_D\in L^1(\mathbb R^2)$ be the indicator of the unit disk.
Then its Fourier transform $\widehat{\chi_D}:\mathbb R^2\to\mathbb C$ (in any standard convention) is a continuous, radial, real-analytic function of $r=|\xi|$ that is not identically zero. Consequently, the set
\[
Z:=\{\xi\in\mathbb R^2:\widehat{\chi_D}(\xi)=0\}
\]
has Lebesgue measure $0$ in $\mathbb R^2$.
\end{lemma}

\begin{proof}
It is standard (and follows by converting the Fourier integral to polar coordinates) that $\widehat{\chi_D}$ is radial and can be written in terms of a Bessel function; in particular it is real-analytic in $r=|\xi|$.

Also $\widehat{\chi_D}(0)=\int_{\mathbb R^2}\chi_D(x)\,dx=\lambda(D)>0$, so $\widehat{\chi_D}$ is not identically zero.

A nonzero real-analytic function of one variable has a discrete zero set (isolated zeros) on $\mathbb R$.
Therefore the set of radii $r\ge 0$ for which $\widehat{\chi_D}(\xi)=0$ when $|\xi|=r$ is a discrete subset of $[0,\infty)$. Hence $Z$ is contained in a countable union of circles $\{|\xi|=r_k\}$, which has Lebesgue measure $0$ in $\mathbb R^2$.
\end{proof}

\begin{theorem}[No nontrivial $L^2$ solution]\label{thm:L2-no-solution}
If $f\in L^2(\mathbb R^2)$ and $f*\chi_D=0$ almost everywhere, then $f=0$ almost everywhere.
\end{theorem}

\begin{proof}
Because $\chi_D\in L^1(\mathbb R^2)$, convolution with $\chi_D$ defines a bounded linear operator on $L^2$, and the Fourier transform (Plancherel theorem) gives
\[
\widehat{f*\chi_D} = \widehat f\cdot \widehat{\chi_D}
\quad\text{in }L^2(\mathbb R^2).
\]
The assumption $f*\chi_D=0$ a.e.\ implies $\widehat f(\xi)\,\widehat{\chi_D}(\xi)=0$ for almost every $\xi$.

By Lemma~\ref{lem:disk-ft-zeros}, $\widehat{\chi_D}(\xi)\neq 0$ for almost every $\xi$ (its zero set has measure $0$). Therefore $\widehat f(\xi)=0$ for almost every $\xi$.

By Plancherel, $\|\widehat f\|_{L^2}=\|f\|_{L^2}$, so $\widehat f=0$ a.e.\ implies $f=0$ a.e.
\end{proof}

\begin{lemma}[Smooth bounded nontrivial solutions exist if we drop the $\{\pm 1\}$ constraint]\label{lem:smooth-solution}
There exist nonzero bounded smooth functions $g:\mathbb R^2\to\mathbb R$ such that $g*\chi_D\equiv 0$.
\end{lemma}

\begin{proof}
Let $\xi_0\in\mathbb R^2\setminus\{0\}$ be such that $\widehat{\chi_D}(\xi_0)=0$ (such $\xi_0$ exist because the Bessel-function expression has infinitely many zeros in $r$). Define
\[
g(x)=\cos(2\pi \xi_0\cdot x).
\]
Then $g$ is smooth and bounded. Its Fourier transform is supported on $\{\pm \xi_0\}$ (in the distributional sense). Multiplication by the Fourier symbol gives
\[
\widehat{g*\chi_D} = \widehat g\cdot \widehat{\chi_D} = 0
\]
because $\widehat{\chi_D}(\pm \xi_0)=0$. Taking inverse Fourier transform yields $g*\chi_D\equiv 0$.
\end{proof}

\subsection*{5) VERIFICATION}

\paragraph{Check that Theorem~\ref{thm:L2-no-solution} does not answer the question.}
A measurable $\{\pm 1\}$-valued function on $\mathbb R^2$ is not in $L^2(\mathbb R^2)$ (it has infinite $L^2$ norm), so Theorem~\ref{thm:L2-no-solution} is strictly a partial result.

\paragraph{Lemma~\ref{lem:disk-ft-zeros} rigor.}
The key point is $\widehat{\chi_D}(0)=\lambda(D)>0$, so it is nonzero analytic radial, hence zeros occur only on discrete radii, yielding a measure-zero union of circles.

\subsection*{6) FINAL}

\textbf{UNRESOLVED.}

\paragraph{Strongest fully proved partial results.}
\begin{itemize}
\item There are no nontrivial solutions in $L^2(\mathbb R^2)$ (Theorem~\ref{thm:L2-no-solution}).
\item Nontrivial bounded smooth solutions exist if the $\{\pm 1\}$ constraint is removed (Lemma~\ref{lem:smooth-solution}).
\end{itemize}

\paragraph{Exact first gap.}
Bridge the gap between $L^2$ (or smooth) solutions and measurable $\{\pm 1\}$-valued solutions: either construct such an $f$, or prove that bounded measurable solutions must satisfy additional regularity/decay that forces $f=0$.

\paragraph{Top 3 next moves.}
\begin{enumerate}[label=\arabic*.]
\item Analyze the equation in the framework of \emph{translation-bounded measures} or \emph{Besicovitch almost periodic} functions, which can accommodate bounded non-$L^2$ functions while still allowing Fourier analysis.
\item Attempt a rigidity/discrepancy argument using overlaps of disks: compare $x$ and $x+h$ to deduce constraints on the coloring of thin crescents and derive a contradiction with measurability.
\item Attempt a periodic/discrete approximation: search for large-period $\{\pm 1\}$ patterns on a lattice whose disk-averages vanish on all lattice translates, then pass to a limit.
\end{enumerate}

\paragraph{What a minimal counterexample would likely look like.}
If such a coloring exists, it is likely highly nonregular (no local structure) yet measurable; one might expect it to be nonperiodic and to have ``spectral support'' concentrated near the Bessel-zero circles in a generalized Fourier sense.
