\section{Problem 312396: Chern classes / Chern character of a representation}

\subsection*{1) FORMAL RESTATEMENT}

\paragraph{Literal content.}
The attachment asks (paraphrasing with explicit objects) for an ``algebraic expression'' of the Chern classes
\[
c_i(\widehat{V})\in H^{2i}(BG;\Z)
\]
associated to a complex representation $V$ of a finite group $G$, in terms of the \emph{character} $\chi_V$.

Here $\widehat{V}\to BG$ denotes the complex vector bundle classified by the representation $V$
(via the map $BG\to BU(\dim V)$ induced by $V$).

\paragraph{Ambiguity.}
The phrase ``algebraic expression in terms of the character'' is not a single formal statement:
it could mean an explicit formula, an algorithm, or a characterization/uniqueness theorem.
Also, for a finite group $G$ one has $H^{>0}(BG;\Q)=0$, so the \emph{rational} Chern character carries essentially no information.

\paragraph{Minimal corrected mathematical statements proved below.}
I will prove two precise statements that directly answer parts of the question:

\begin{enumerate}[label=(\alph*),leftmargin=2.2em]
\item (\textbf{Rational triviality}) If $G$ is finite then $H^{>0}(BG;\Q)=0$, hence for any representation $V$ the rational Chern character satisfies
\[
\mathrm{ch}(\widehat{V})=\dim(V)\in H^0(BG;\Q).
\]
\item (\textbf{Restriction to cyclic subgroups}) Let $g\in G$ have order $m$ and let $C=\langle g\rangle$.
Then the restriction $c(\widehat{V})|_{BC}\in H^\ast(BC;\Z)$ of the \emph{total} Chern class is computable from the character values
\[
\chi_V(g^k)\qquad (k=0,1,\dots,m-1)
\]
via discrete Fourier inversion and the standard description $H^\ast(BC;\Z)\cong \Z[u]/(mu)$.
\end{enumerate}

These do not fully solve the global ``in terms of character'' problem on $H^\ast(BG;\Z)$, which I leave unresolved.

\subsection*{2) QUICK LITERATURE/CONTEXT CHECK}

Atiyah already remarked (1961) that beyond $c_1$ it is desirable to have a direct algebraic definition of the Chern classes of a finite group representation.
There is later literature (Evens, Kroll, Symonds, \dots) giving algebraic characterizations and structural results, but I do not reproduce them in full here.

\subsection*{3) ATTACK PLAN}

\begin{itemize}[leftmargin=2.2em]
\item \textbf{Proof track (chosen):} Work out what can be proved directly:
  \begin{itemize}[leftmargin=2.2em]
  \item show $H^{>0}(BG;\Q)=0$ for finite $G$ via an explicit averaging homotopy;
  \item compute $H^\ast(BC_m;\Z)$ and show that restricting $V$ to $C_m$ decomposes into $1$-dimensional characters determined by Fourier inversion from $\chi_V$, yielding an explicit total Chern class formula.
  \end{itemize}
\item \textbf{Disproof track:} If the question were interpreted as ``there is a simple universal closed-form formula for $c_i$ purely from character values on $G$'', attempt to find a counterexample; however, this is ill-posed without specifying what counts as ``algebraic'' and I do not pursue this.
\end{itemize}

\subsection*{4) WORK}

\subsubsection*{Phase 1 --- Rational cohomology vanishes for finite groups}

\begin{lemma}\label{lem:BGQvanish}
If $G$ is finite then $H^n(BG;\Q)=0$ for all $n>0$.
Equivalently, $H^n(G;\Q)=0$ for $n>0$.
\end{lemma}

\begin{proof}
Use the inhomogeneous cochain complex $C^n(G;\Q)=\mathrm{Map}(G^n,\Q)$ computing group cohomology with trivial coefficients.
Define a $\Q$-linear map $h:C^n(G;\Q)\to C^{n-1}(G;\Q)$ for $n\ge 1$ by averaging over the first argument:
\[
(hf)(g_1,\dots,g_{n-1})=\frac{1}{|G|}\sum_{x\in G} f(x,g_1,\dots,g_{n-1}).
\]
A standard direct computation of the coboundary $\delta$ in group cohomology shows
\[
\delta h + h\delta = \mathrm{id}_{C^n}\quad\text{for all }n\ge 1,
\]
again because $|G|$ is invertible in $\Q$.
Thus the complex is contractible in degrees $>0$, so $H^n(G;\Q)=0$ for $n>0$.
\end{proof}

\begin{corollary}[Rational Chern character is trivial]\label{cor:chernchartrivial}
If $G$ is finite and $V$ is a complex representation of dimension $r$, then
\[
\mathrm{ch}(\widehat{V})=r\in H^0(BG;\Q),
\]
and all positive-degree components of $\mathrm{ch}(\widehat{V})$ vanish.
\end{corollary}

\begin{proof}
$\mathrm{ch}(\widehat{V})\in H^{\mathrm{even}}(BG;\Q)$ by definition.
By Lemma~\ref{lem:BGQvanish}, $H^{2i}(BG;\Q)=0$ for $i>0$, so only degree $0$ remains, which is the rank $r$.
\end{proof}

\subsubsection*{Phase 2 --- Restriction to cyclic subgroups in terms of $\chi_V$}

Fix $g\in G$ of order $m$ and let $C=\langle g\rangle\cong C_m$.

\begin{lemma}\label{lem:cycliccohom}
$H^\ast(BC;\Z)\cong \Z[u]/(mu)$ with $|u|=2$.
\end{lemma}

\begin{proof}
This is the standard computation of the cohomology of a cyclic group with trivial coefficients.
One way: use the periodic free resolution of $\Z$ as a $\Z[C]$-module and compute $\mathrm{Ext}$.
It yields $H^{2k}(C;\Z)\cong \Z/m$ generated by $u^k$ and $H^{2k+1}(C;\Z)=0$ for $k\ge 0$.
This ring structure is exactly $\Z[u]/(mu)$ with $\deg u=2$.
\end{proof}

\begin{lemma}\label{lem:decompcyclic}
Let $\zeta:=e^{2\pi i/m}$.  The restriction of $V$ to $C$ decomposes as a direct sum of one-dimensional characters:
\[
V|_C \;\cong\;\bigoplus_{a=0}^{m-1} n_a\,\chi_a,
\qquad
\chi_a(g)=\zeta^{a},
\]
for uniquely determined multiplicities $n_a\in\Z_{\ge 0}$.
Moreover, the $n_a$ are determined from the character values by discrete Fourier inversion:
\[
n_a=\frac1m\sum_{k=0}^{m-1}\zeta^{-ak}\,\chi_V(g^k).
\]
\end{lemma}

\begin{proof}
Since $C$ is finite abelian, every complex representation is completely reducible into one-dimensional characters.
The irreducible characters of $C_m$ are exactly $\chi_a$ as displayed, so $V|_C$ decomposes with multiplicities $n_a$.

For $k=0,1,\dots,m-1$ we have
\[
\chi_V(g^k)=\sum_{a=0}^{m-1} n_a\,\chi_a(g^k)=\sum_{a=0}^{m-1} n_a\,\zeta^{ak}.
\]
This is a discrete Fourier transform of the vector $(n_a)_{a}$.
Multiply both sides by $\zeta^{-bk}$ and sum over $k$:
\[
\sum_{k=0}^{m-1} \zeta^{-bk}\chi_V(g^k)=\sum_{a=0}^{m-1} n_a \sum_{k=0}^{m-1}\zeta^{(a-b)k}.
\]
The inner sum is $m$ if $a=b$ and $0$ otherwise (geometric series with ratio $\neq 1$).
Thus the left-hand side equals $m n_b$, proving the formula.
\end{proof}

\begin{lemma}\label{lem:c1cyclic}
Let $L_a\to BC$ be the complex line bundle classified by the one-dimensional representation $\chi_a$.
With $u\in H^2(BC;\Z)$ as in Lemma~\ref{lem:cycliccohom}, one has
\[
c_1(L_a)=a\,u\in H^2(BC;\Z)\cong \Z/m.
\]
\end{lemma}

\begin{proof}
The group of isomorphism classes of complex line bundles over $BC$ is $[BC,BU(1)]\cong H^2(BC;\Z)\cong \Z/m$.
Under this identification, the character $\chi_1:C\to U(1)$ corresponds to the generator $u$; then $\chi_a=\chi_1^{\,a}$ corresponds to $a\,u$, and $c_1$ is exactly this class.
\end{proof}

\begin{theorem}[Total Chern class on a cyclic subgroup]\label{thm:cherncyclic}
With notation as above,
let $V|_C\cong \bigoplus_{a=0}^{m-1} n_a\chi_a$ and $u\in H^2(BC;\Z)$ the generator.
Then in $H^\ast(BC;\Z)\cong \Z[u]/(mu)$,
\[
c(\widehat{V})|_{BC}
=
\prod_{a=0}^{m-1}\bigl(1+a\,u\bigr)^{n_a}.
\]
In particular, the restriction of every Chern class $c_i(\widehat{V})$ to $BC$ is determined by the character values $\chi_V(g^k)$ via Lemma~\ref{lem:decompcyclic}.
\end{theorem}

\begin{proof}
By Lemma~\ref{lem:decompcyclic}, $\widehat{V}|_{BC}\cong \bigoplus_a n_a L_a$ as vector bundles.
The total Chern class is multiplicative under direct sum (Whitney sum formula), so
\[
c(\widehat{V})|_{BC}= \prod_a c(L_a)^{n_a}.
\]
For a line bundle, $c(L_a)=1+c_1(L_a)$.
By Lemma~\ref{lem:c1cyclic}, $c_1(L_a)=a\,u$, yielding the stated product.
Finally, each $n_a$ is a Fourier-invertible function of the character values, so the restriction is computable from $\chi_V$.
\end{proof}

\subsection*{5) VERIFICATION}

\begin{itemize}[leftmargin=2.2em]
\item The proof of Lemma~\ref{lem:BGQvanish} is an explicit contracting homotopy; no topology of $BG$ is assumed beyond its identification with group cohomology.
\item Lemma~\ref{lem:decompcyclic} uses only semisimplicity of complex representations of finite abelian groups and an explicit Fourier inversion computation.
\item The only ``external'' fact used is the standard cohomology ring of $BC_m$ (Lemma~\ref{lem:cycliccohom}) and the standard classification of line bundles over $BC_m$ (Lemma~\ref{lem:c1cyclic}), both of which can be derived from standard resolutions.
\end{itemize}

\subsection*{6) FINAL}

\begin{center}
\textbf{UNRESOLVED}
\end{center}

\noindent\textbf{(i) Strongest proved partial results obtained:}
Corollary~\ref{cor:chernchartrivial} shows the rational Chern character carries only the dimension for finite groups.
Theorem~\ref{thm:cherncyclic} gives an explicit character-theoretic formula for the restriction of the total Chern class to any cyclic subgroup $\langle g\rangle$.

\medskip
\noindent\textbf{(ii) First gap:}
Upgrade cyclic-subgroup restriction data to a global formula in $H^\ast(BG;\Z)$ purely in terms of $\chi_V$ (or prove no such ``direct'' formula exists under a precise notion of ``algebraic expression'').

\medskip
\noindent\textbf{(iii) Top 3 next moves:}
\begin{enumerate}[label=(\arabic*),leftmargin=2.2em]
\item Determine whether restrictions to cyclic subgroups detect $H^{2i}(BG;\Z)$ in the relevant range (e.g.\ via Quillen-style detection theorems), and if so, assemble $c_i$ from the cyclic restrictions.
\item Use induction theorems (Brauer induction) to express characters as integer combinations of induced 1-dimensional characters from cyclic subgroups, and track how Chern classes behave under induction/restriction/transfer.
\item Consult the post-1961 literature (Evens, Kroll, Symonds, etc.) for known algebraic characterizations and check whether they can be recast as explicit formulas from character data.
\end{enumerate}

\medskip
\noindent\textbf{(iv) What a minimal counterexample would likely look like:}
A pair of non-isomorphic representations with the same character on all cyclic subgroups but different Chern classes in $H^\ast(BG;\Z)$ would refute a ``cyclic restriction determines Chern classes'' principle.  (Characters are class functions, so this would require a more subtle failure than ordinary character equality; e.g.\ dependence on integral vs.\ rational data.)
