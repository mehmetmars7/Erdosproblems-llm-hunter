\section{MO 366772: ``A question related to the Hofstadter--Conway $10000$ sequence''}
\label{sec:mo366772}
\noindent\textbf{MathOverflow link:} \url{https://mathoverflow.net/questions/366772/a-question-related-to-the-hofstadter-conway-10000-sequence} (accessed 2026-01-16).

\subsection*{1) FORMAL RESTATEMENT}
Define two sequences $c, c^*:\NN\to\NN$ by
\[
 c(1)=c(2)=1,\qquad c(n)=c(c(n-1)) + c\bigl(n - c(n-1)\bigr) \ \text{for }n\ge 3,
\]
(the Hofstadter--Conway $10000$ sequence), and
\[
 c^*(1)=c^*(2)=1,\qquad c^*(n)= n - c^*(c^*(n-1)) - c^*\bigl(n-c^*(n-1)\bigr)\ \text{for }n\ge 3.
\]

\noindent\textbf{Conjecture (Ineq).}
For all $n\in\NN$,
\[
 c(n) - \frac{n}{2} \ \ge\ \left|c^*(n) - \frac{n}{2}\right|.
\]

\subsection*{2) QUICK LITERATURE/CONTEXT CHECK}
The MathOverflow page has no posted answers as of 2026-01-16.
The Hofstadter--Conway sequence is classical and has a substantial literature; however, the inequality involving the variant $c^*$ appears to be a new observation in the MO post.

\subsection*{3) ATTACK PLAN}
\textbf{Proof-track ideas.}
\begin{enumerate}[leftmargin=2em]
\item Attempt induction bounds of the form $c(n)\ge n/2$ and $|c^*(n)-n/2|\le c(n)-n/2$ using monotonicity/recursion.
\item Seek a coupling or identity relating $c$ and $c^*$.
\end{enumerate}

\textbf{Disproof-track ideas.}
\begin{enumerate}[leftmargin=2em]
\item Compute both sequences for large $n$ and directly test the inequality; look for the first counterexample.
\end{enumerate}

We executed the computation up to $n=2\times 10^6$ and found no counterexample; no general proof.

\subsection*{4) WORK}
\subsubsection*{4.1 Tiny cases}
Direct computation gives:
\[
\begin{array}{c|cccccccccc}
 n & 1&2&3&4&5&6&7&8&9&10\\\hline
 c(n) & 1&1&2&2&3&4&4&4&5&6\\
 c^*(n) & 1&1&2&2&2&2&3&4&4&4
\end{array}
\]
The inequality holds for these values.

\subsubsection*{4.2 Computation}
We computed $c(n)$ and $c^*(n)$ iteratively for $n\le 2\cdot 10^6$ (time $<1$ second in Python) and checked the inequality at each $n$.
\medskip
\noindent\textbf{Result:} No counterexample for $1\le n\le 2\cdot 10^6$.

\subsection*{5) VERIFICATION}
\begin{itemize}[leftmargin=2em]
\item The recurrence is well-defined because $c(n-1)\le n-1$ holds empirically and is known for the classical sequence; similarly $c^*(n-1)\le n-1$ held in the computation, ensuring all referenced indices are $\ge 1$.
\item The computational check is finite and cannot prove the universal quantifier.
\end{itemize}

\subsection*{6) FINAL}
\begin{center}
\textbf{UNRESOLVED}
\end{center}


