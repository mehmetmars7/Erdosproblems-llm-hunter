\section{Problem 1: $\ell^2$, $\ell^3$, and $\ell^4$ fixed point?}

\subsection*{1) FORMAL RESTATEMENT}
Fix a ground field $\Bbb F\in\{\R,\C\}$ and let $\ell^p=\ell^p(\N;\Bbb F)$ denote the Banach space of sequences
$x=(x_n)_{n\in\N}$ with $\sum_{n=1}^\infty |x_n|^p<\infty$, equipped with norm
\[
\normp{x}{p} := \Big(\sum_{n=1}^\infty |x_n|^p\Big)^{1/p}.
\]
Let
\[
K:=\{x\in \ell^2:\ \normp{x}{2}\le 1\}.
\]
Let $T:K\to K$ be any (not assumed linear) mapping such that for all $x,y\in K$,
\[
\normp{T(x)-T(y)}{4}\le \normp{x-y}{3}. \tag{$\ast$}
\]
Question: must there exist $x\in K$ with $T(x)=x$?

\medskip
\noindent\textbf{Conventions and stress points.}
\begin{itemize}[leftmargin=2em]
\item $\N=\{1,2,3,\dots\}$.
\item $K$ is closed, bounded, convex in $\ell^2$, but not compact.
\item $(\ast)$ implies $T$ is $1$-Lipschitz from $(K,\normp{\cdot}{3})$ to $(K,\normp{\cdot}{4})$, hence continuous for the $\ell^3$-topology on $K$ and the $\ell^4$-topology on $K$.
\item $K$ is \emph{not} closed in $\ell^3$ (limits in $\ell^3$ can leave $\ell^2$), so one cannot directly apply fixed point results in $\ell^3$.
\item The classical Browder--G\"ohde--Kirk theorem applies to maps nonexpansive in the \emph{same} uniformly convex norm; here the domain metric and codomain metric differ.
\end{itemize}

\subsection*{2) QUICK LITERATURE/CONTEXT CHECK (web)}
The MO post explicitly relates the question to the Browder--G\"ohde--Kirk fixed point theorem for nonexpansive maps on closed bounded convex subsets of uniformly convex Banach spaces, and notes (in comments) that certain ``mixed-norm'' fixed point properties can fail in other regimes. As of \date{}, the MO page shows no posted answer.

\subsection*{3) ATTACK PLAN}
\textbf{Proof-track candidates.}
\begin{enumerate}[label=(P\arabic*),leftmargin=2.5em]
\item Try to deduce compactness/condensing behavior: show $T(K)$ is relatively compact in $\ell^2$ (or $T$ is completely continuous), then apply Schauder.
\item Try to find an equivalent metric $d$ on $K$ for which $T$ is nonexpansive and $(K,d)$ is complete and uniformly convex in the metric sense, then apply a metric fixed point theorem.
\item Prove existence of an approximate fixed point sequence and then a compactness/weak lower semicontinuity argument to upgrade to an actual fixed point.
\end{enumerate}
\textbf{Disproof-track candidates.}
\begin{enumerate}[label=(D\arabic*),leftmargin=2.5em]
\item Adapt a known fixed-point-free map on the Hilbert ball (e.g.\ the classical shift/``Kakutani'' type map) and check whether one can tune it to satisfy $(\ast)$ by creating ``slack'' on one-coordinate perturbations.
\item Search for an explicit Lipschitz (in $\ell^3$) coordinate map with no fixed point and verify $(\ast)$.
\end{enumerate}
Best initial path: attempt (D1) with weighted shifts; if that fails systematically, look for a structural lemma that forces a fixed point.

\subsection*{4) WORK}

\paragraph{PHASE 1: tiny-case sanity check (finite-dimensional truncations).}
Let $n\in\N$ and consider the analogous problem in $\Bbb F^n$ with $\ell^p$-norms. Then the closed $\ell^2$-unit ball is compact and convex.
Any map $T:K_n\to K_n$ satisfying $(\ast)$ is Lipschitz (hence continuous), so by Brouwer's fixed point theorem it has a fixed point.
Thus any counterexample must use infinite-dimensional noncompactness.

\paragraph{PHASE 1: a natural counterexample template and why it is constrained.}
A standard fixed-point-free map on the infinite-dimensional Hilbert ball is
\[
\widetilde T(x) := \big(\sqrt{1-\normp{x}{2}^2},\, x_1, x_2, \dots\big),
\]
which maps $K$ into the unit sphere and has no fixed point.
However, $(\ast)$ fails for pairs $x,y$ differing in only one coordinate: in that case $\normp{x-y}{3}=\normp{x-y}{4}$ leaves no slack for the additional first-coordinate variation.

A common idea is to scale the ``shift part'' by a factor $a\in(0,1)$ to create slack:
\[
T_a(x) :=\big(\psi(x),\, a x_1, a x_2,\dots\big).
\]
Then
\[
\normp{T_a(x)-T_a(y)}{4}^4
=|\psi(x)-\psi(y)|^4 + a^4\normp{x-y}{4}^4.
\]
If $\psi$ is $\ell^3$-Lipschitz with constant $(1-a^4)^{1/4}$, then
\[
|\psi(x)-\psi(y)|^4 \le (1-a^4)\,\normp{x-y}{3}^4,\qquad
a^4\normp{x-y}{4}^4\le a^4\normp{x-y}{3}^4,
\]
so $(\ast)$ holds. Hence a large class of maps satisfy $(\ast)$.

\emph{But:} for any such $T_a$ that also maps $K$ into itself, the constraint $|\psi(x)|\le \sqrt{1-a^2}$ for all $x\in K$ forces a fixed point by a one-dimensional intermediate value argument along the curve
$c\mapsto x(c)=(c,ac,a^2c,\dots)$.
Indeed, if $\psi$ is continuous along this curve, the function $d(c):=\psi(x(c))-c$ is continuous on $[-\sqrt{1-a^2},\sqrt{1-a^2}]$, and
\[
d(-m)=\psi(x(-m))+m\ge -m+m=0,\qquad
d(m)=\psi(x(m))-m\le m-m=0,
\]
with $m=\sqrt{1-a^2}$. Thus $d(c_0)=0$ for some $c_0$, giving a fixed point.
So this entire template cannot yield a counterexample.

\paragraph{Strongest fully proved partial result obtained here.}
\begin{itemize}[leftmargin=2em]
\item Any counterexample must exploit genuinely infinite-dimensional phenomena (finite-dimensional truncations always have fixed points).
\item A large natural class of maps satisfying $(\ast)$ (weighted shifts plus one new coordinate) is \emph{forced} to have fixed points by an explicit intermediate value obstruction; thus a counterexample, if it exists, must be structurally different.
\end{itemize}

\paragraph{First hard gap.}
I do not have a proof that \emph{every} $T$ satisfying $(\ast)$ has a fixed point, and I also do not have an explicit fixed-point-free $T$ satisfying $(\ast)$. The obstacle is that $(\ast)$ controls the image only in $\ell^4$-distance, which is too weak to give compactness or $\ell^2$-nonexpansiveness.

\subsection*{5) VERIFICATION (attempt to break the partial analysis)}
\begin{itemize}[leftmargin=2em]
\item The intermediate value argument for the template $T_a$ uses only: (i) $|\psi(x)|\le \sqrt{1-a^2}$ on $K$, (ii) continuity of $\psi$ along $c\mapsto x(c)$. Since $(\ast)$ implies $\psi$ is $\ell^3$-Lipschitz (hence continuous), and $c\mapsto x(c)$ is continuous in $\ell^3$, the continuity hypothesis is satisfied.
\item The proof does \emph{not} claim that all $T$ reduce to this template; it only eliminates this common construction route.
\end{itemize}

\subsection*{6) FINAL}
\textbf{UNRESOLVED.}

\noindent (i) Strongest partial: the ``weighted-shift plus one coordinate'' family cannot produce a counterexample; finite-dimensional analogues always have fixed points.

\noindent (ii) First gap: either prove a global fixed point theorem from $(\ast)$, or produce a genuinely different explicit fixed-point-free construction.

\noindent (iii) Top 3 next moves:
\begin{enumerate}[leftmargin=2.5em]
\item Try to show $(\ast)$ implies $T$ is \emph{demiclosed} or condensing in $\ell^2$ (e.g.\ decreases Kuratowski measure of noncompactness), enabling Darbo/Schauder.
\item Seek a counterexample built from a fixed-point-free map that is constant on ``one-coordinate lines'' to respect the no-slack phenomenon for $\ell^3=\ell^4$ on 1-sparse differences.
\item Investigate whether $(\ast)$ implies existence of an $\ell^2$-approximate fixed point sequence and whether uniform convexity of $\ell^2$ forces an actual fixed point under additional compactness derived from $(\ast)$.
\end{enumerate}

\noindent (iv) Likely structure of a minimal counterexample: a highly nonlocal map that (a) avoids adding new $\ell^4$-mass on 1-sparse perturbations, but (b) still moves points globally enough to eliminate fixed points, exploiting infinite-dimensional noncompactness.
