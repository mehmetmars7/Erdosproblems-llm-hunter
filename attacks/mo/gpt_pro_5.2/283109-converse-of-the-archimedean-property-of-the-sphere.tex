\section*{Problem 283109: Converse Archimedean ``zone area'' property for a fixed $h$}
\addcontentsline{toc}{section}{Problem 283109: Converse Archimedean zone area property for a fixed h}

\subsection*{1) FORMAL RESTATEMENT}
Let $S\subset\mathbb{R}^3$ be the boundary of a convex body (assume at least $C^1$ so that surface area is well-defined without ambiguity; otherwise interpret area as $2$-dimensional Hausdorff measure).
Let $d$ be the diameter of the convex body.
Fix $h$ with $0<h<d$.

Assumption (fixed-thickness Archimedean property):
for every pair of parallel planes $P,P'$ in $\mathbb{R}^3$ with distance $\mathrm{dist}(P,P')=h$ such that both planes intersect $S$,
the area of the subset of $S$ lying between $P$ and $P'$ is a constant depending only on $h$ (not on the planes).

Question: must $S$ be a round sphere?

\subsection*{2) QUICK LITERATURE/CONTEXT CHECK}
The MO post notes that the ``all $h$'' version is a classical characterization of the sphere (attributed to Blaschke as an exercise).
The fixed-$h$ version is asked as a potential converse and appears open in the post.

\subsection*{3) ATTACK PLAN}
\emph{Proof-track ideas:}
(1) For each direction $u$, write the band-area as an integral of an area density function along $u$; the fixed-$h$ condition forces $h$-periodicity of that density.
Try to combine periodicity with boundary/convexity constraints to deduce constancy of density, hence sphere.
(2) Use convex-geometric curvature measures (support measures) to convert the slab condition into a statement about the surface area measure on $S^2$.

\emph{Disproof-track ideas:}
(1) Construct a non-spherical convex body whose ``area density'' in every direction is $h$-periodic but not constant (if such a density is realizable by a convex body).

\subsection*{4) WORK (partial results)}

\begin{proposition}[Periodicity of the directional area density]
Fix a unit vector $u$ and define $t(x)=u\cdot x$.
Assume $S$ is sufficiently regular that there is a measurable function $f_u(t)\ge 0$ such that for any interval $[a,b]$
the area of $\{x\in S: a\le t(x)\le b\}$ equals $\int_a^b f_u(t)\,dt$.
If the fixed-$h$ Archimedean property holds for slabs orthogonal to $u$, then
\[
f_u(t+h)=f_u(t)\quad\text{for a.e.\ }t\text{ in the range where both sides are defined.}
\]
\end{proposition}

\begin{proof}
For slabs orthogonal to $u$, define $A_u(a)=\mathrm{Area}\{x\in S: a\le t(x)\le a+h\}=\int_a^{a+h}f_u(t)\,dt$.
The hypothesis says $A_u(a)$ is constant (for all $a$ where the slab intersects $S$).
Whenever $A_u$ is differentiable,
\[
0=A_u'(a)=f_u(a+h)-f_u(a).
\]
Thus $f_u(a+h)=f_u(a)$ for a.e.\ $a$.
\end{proof}

\begin{proposition}[Integer multiples of $h$ also have constant band area]
Under the fixed-$h$ Archimedean property, for any integer $m\ge 1$ and any pair of parallel planes distance $mh$ apart that both intersect $S$,
the area of the portion of $S$ between them is $m$ times the constant area for thickness $h$ (provided the intermediate slabs are defined).
\end{proposition}

\begin{proof}
Insert $m-1$ intermediate planes parallel to the given ones, equally spaced by distance $h$.
Convexity ensures each intermediate plane intersects $S$ whenever the endpoints do, and the region between endpoints is a disjoint union of $m$ regions of thickness $h$.
Each has the same area by hypothesis, hence the total is $m$ times that constant.
\end{proof}

\subsection*{5) VERIFICATION}
The periodicity argument depends on existence of an area density $f_u(t)$; this can be justified under reasonable regularity assumptions via the coarea formula.
These partial results do not force $f_u$ to be constant, hence do not characterize the sphere for a \emph{single} $h$.

\subsection*{6) FINAL: \textbf{UNRESOLVED}}
(i) Strongest proved partial results: periodicity of directional area density and extension to thickness $mh$ (Section 4).\\
(ii) First gap: show that the only convex surfaces whose directional area densities are $h$-periodic for \emph{all} directions are spheres, or else exhibit a non-spherical convex body realizing such periodic densities.\\
(iii) Next moves:
\begin{itemize}
\item Make the coarea formula setup fully rigorous for convex (possibly nonsmooth) bodies and identify $f_u(t)$ in terms of curvature measures.
\item Investigate whether $h$-periodicity of $f_u$ combined with endpoint behavior (near support planes) forces $f_u$ constant.
\item Attempt construction: prescribe $f_u$ periodic and solve a Minkowski-type problem to realize it as a convex body.
\end{itemize}
(iv) Minimal counterexample structure (if the conjectured converse is false): a convex body whose surface area distribution along any direction repeats with period $h$, yet whose curvature measure is not that of a sphere.
