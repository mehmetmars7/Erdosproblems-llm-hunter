## 1) FORMAL RESTATEMENT

### Definitions/conventions

* For integers (n\ge 1), let (S^n) be the standard (n)-sphere with a chosen basepoint.
* For integers (k\ge 0), (\pi_k(S^n)) denotes the *based* homotopy group.
* For (k\ge 2), (\pi_k(S^n)) is abelian.
* Standard facts (used only for “edge-case hygiene”):

  * If (0<k<n), then (\pi_k(S^n)=0).
  * If (k=n), then (\pi_n(S^n)\cong \mathbb Z) (infinite).
  * Thus a **finite** homotopy group of a sphere can only occur for (k>n\ge 2) (or (k=1,n\ge 2) gives (0), which is finite but trivial).

### What “arise as a homotopy group of spheres” means

For a finite abelian group (G), say that **(G) arises** if
[
\exists,n\ge 2;\exists,k>n\text{ such that }\pi_k(S^n)\cong G.
]

### The conjectural statement embedded in the prompt

The prompt contains the conjecture (I’ll label it):

[
(\star)\quad\forall\text{ odd primes }p>3,;\forall n\ge 2,;\forall k>n:;\pi_k(S^n)\not\cong \mathbb Z/p.
]

The “(\mathbb Z_5) is the smallest one” claim is the special case (p=5) together with the assertion that (\mathbb Z/p) occurs for (p=2,3) (which it does, e.g. (\pi_4(S^3)\cong\mathbb Z/2) and (\pi_9(S^3)\cong\mathbb Z/3) appear in standard tables).

### Stress points / edge cases

* Any “proof of nonexistence” must quantify over **all** pairs ((n,k)) with (k>n\ge 2), not merely small tables.
* Stable vs unstable: for fixed (m=k-n), (\pi_{n+m}(S^n)) stabilizes for (n\ge m+2). ([Wikipedia][1])
  So if (\mathbb Z/5) were to occur, it could in principle occur either in the stable range (as (\pi_m^S\cong\mathbb Z/5)) or in some genuinely unstable spot.

---

## 2) QUICK LITERATURE/CONTEXT CHECK (browsing enabled)

* The basic “stable range” fact and the current computational landscape: (\pi_{n+k}(S^n)) is independent of (n) once (n\ge k+2); these stable homotopy groups have been computed for (k) up to (90) (per the Wikipedia summary), and unstable groups have been tabulated for (k<20). ([Wikipedia][1])
* There are explicit published/tabulated values (e.g. based on Toda’s tables) for (\pi_{n+k}(S^n)) with (k<20). One such tabulation is the Bielefeld table explicitly stating the notation “(n) means (\mathbb Z/n)”, “(n+m)” means direct sum, etc.
* I did **not** find (in the browsing I was able to do here) a theorem in the cited sources that either proves ((\star)) or produces an explicit ((n,k)) with (\pi_k(S^n)\cong\mathbb Z/5). The available tabulations for (k<20) show no (\mathbb Z/5) as an entire group; see the computation in §4.

So: as far as I can validate from accessible standard tables and summaries, **(\mathbb Z/5) does not appear in the known (k<20) tabulations**, but I cannot promote that to “never occurs” without additional theory beyond what these sources provide.

---

## 3) ATTACK PLAN

### Disproof/construction track (preferred first)

1. **Exhaustive search in published tables**: scan the standard tabulations of (\pi_{n+k}(S^n)) for small (k) (difference) and (n) for an entry exactly equal to (5) (meaning (\mathbb Z/5)).

   * If found: explicit counterexample to ((\star)).
   * If not found: yields a rigorous *finite-range nonexistence statement* (“no (\mathbb Z/5) for (k<20)”, etc.).

2. **If needed, extend to stable stems**: search stable homotopy stem data for a stem whose *entire* stable group is (\mathbb Z/5), which would imply (\pi_{n+m}(S^n)\cong\mathbb Z/5) for all (n\ge m+2).

### Proof track (harder)

Try to prove a structural obstruction of the form:

* “If (5)-torsion occurs in (\pi_k(S^n)), then (2)- or (3)-torsion also occurs,”
  which would rule out (\mathbb Z/5).
  This would likely require deep input (Adams SS / (J)-homomorphism / Toda brackets / unstable-to-stable comparison), and I do not have a complete route that is gap-free using only the accessible references.

**Chosen path:** run the disproof/construction track via rigorous table scan. If no counterexample appears, report the strongest *proved finite-range* result and identify the precise missing ingredient for a full proof.

---

## 4) WORK

### 4.1 Reality check on small cases (hand + table sanity)

From the standard tabulation (Toda-based) of (\pi_{n+k}(S^n)) for (k<20) (difference (k)), you indeed see:

* (\pi_4(S^3)\cong 2) (i.e. (\mathbb Z/2)),
* (\pi_9(S^3)\cong 3) (i.e. (\mathbb Z/3)),
* (\pi_8(S^4)\cong 2^2) (i.e. (\mathbb Z/2\oplus\mathbb Z/2)),
  and so on, exactly in the notation described.

So the “smaller groups occur” evidence is consistent with the standard tables in this range.

### 4.2 Exhaustive finite-range disproof attempt: search for (\mathbb Z/5) in the (k<20) tables

**Data source.** I used the explicit tabulation “(\pi_{n+k}(S^n))” (Toda-based) that states:

* an integer (n) denotes the cyclic group (\mathbb Z/n),
* (n+m) denotes a direct sum,
* (\infty) denotes (\mathbb Z),
* superscripts denote repeated direct sums, etc.

This table covers (k\le 19), (n\le 20), and a “(n>k+1)” stable column (so it covers the stable groups for these (k)). ([Wikipedia][1])

**Computation.** I parsed the HTML table and searched for any cell whose value is exactly “5” (meaning (\mathbb Z/5)).

* Result: **no cell equals “5”** in the entire tabulation with (k\le 19).

Moreover, the scan shows that every occurrence of “(+5)” in this tabulation appears in a larger direct sum such as (3+5), (2+3+5), (16+3+5), etc., never as “5” alone. (For example, in the (k=7) row one sees entries like (3+5), (2+3+5), (8+3+5), (16+3+5), and in the stable column (16+3+5).)

#### Lemma 4.2 (fully rigorous, finite-range)

Let (T) be the Toda-based table ({\pi_{n+k}(S^n)}) for (k\le 19) and (n\le 20) (including the “(n>k+1)” stable column). Then there is **no** pair ((n,k)) represented in (T) such that (\pi_{n+k}(S^n)\cong \mathbb Z/5).

**Proof.** In (T), the entry “5” denotes (\mathbb Z/5) by the table’s legend.
A brute-force scan of all entries in (T) finds zero occurrences of the string “5” as a standalone group; all entries involving (5) are of the form (A+5) with (A\ne 0) (e.g. (3+5), (2+3+5), (16+3+5), …), hence not isomorphic to (\mathbb Z/5). Therefore no listed (\pi_{n+k}(S^n)) is isomorphic to (\mathbb Z/5). ∎

#### Corollary 4.3 (quantified consequence)

If (\pi_k(S^n)\cong \mathbb Z/5) for some (n\ge 2), (k>n), then necessarily either

* (k-n\ge 20), or
* (n>20) in a way not covered by the tabulation (and also outside the stable column regime for (k-n\le 19)).

This follows because all cases with (k-n\le 19) and (n\le 20) (including the stable (n>k+1) cases for those (k-n)) are explicitly listed and none equals (\mathbb Z/5). ([Wikipedia][1])

### 4.3 Proof track attempt and the first hard barrier

To prove the global conjecture ((\star)), one would need an argument that rules out (\mathbb Z/5) in **every** remaining case ((n,k)) beyond the tabulated range, i.e. especially in unstable regions with (k-n\ge 20).

A natural route would be to prove a structural statement such as:
[
(\dagger)\quad \text{If }5\text{-torsion occurs in }\pi_k(S^n),\text{ then }2\text{- or }3\text{-torsion also occurs in }\pi_k(S^n).
]
Then (\pi_k(S^n)) could never be (\mathbb Z/5).

However, with only the tabulated data and the general stable/unstable framework cited above, I do **not** have a gap-free proof of ((\dagger)) (nor an explicit counterexample to it).

So the proof track stalls at establishing such a global “companionship of torsion primes” phenomenon.

---

## 5) VERIFICATION (adversarial self-check)

### Quantifier checks

* Lemma 4.2 is **not** a claim about all ((n,k)); it is explicitly about the finite set of ((n,k)) covered by the Toda-based table for (k-n\le 19), (n\le 20), and the stable-column cases (n>k+1) in those stems. This is exactly what the cited table provides.
* Corollary 4.3 correctly states only a necessary condition for any hypothetical (\mathbb Z/5) occurrence: it must lie outside the covered range.

### Boundary cases

* (n=1): irrelevant for finite abelian groups as homotopy groups (since (\pi_1(S^1)\cong\mathbb Z), (\pi_k(S^1)=0) for (k>1)).
* (k\le n): either (0) or (\mathbb Z), not (\mathbb Z/5).
* Stable range: accounted for in the table’s “(n>k+1)” column for (k\le 19). ([Wikipedia][1])

### Attempt to break Lemma 4.2

The only way Lemma 4.2 could fail is:

* the table’s legend/notation is misread, or
* the table actually contains an entry “5” somewhere in the covered range.

But the legend is explicit about interpretation of integers and sums, and the visible entries involving (5) appear as sums (3+5), (2+3+5), etc., not as “5”.
So the finite-range nonoccurrence is solid.

---

## 6) FINAL

**UNRESOLVED**

(i) **Strongest fully proved partial result obtained:**
Using the Toda-based tabulation of (\pi_{n+k}(S^n)) for (k<20) (including the stable-column values), there is **no** instance with (\pi_k(S^n)\cong \mathbb Z/5). In fact, within that entire tabulated range, every group that has (5)-torsion also has additional summands (e.g. (3+5), (2+3+5), (16+3+5), …), so it is never (\mathbb Z/5). ([Wikipedia][1])

(ii) **Exact first gap:**
A global proof of “(\mathbb Z/5) never occurs” would require a theorem controlling torsion in *all* (especially highly unstable) (\pi_k(S^n)) beyond (k-n\ge 20); concretely, something like ((\dagger)): “(5)-torsion in (\pi_k(S^n)) forces (2)- or (3)-torsion in the same group.” I do not have a gap-free proof of such a statement from the accessible sources/data.

(iii) **Top 3 next moves (concrete):**

1. **Extend the computational search** beyond (k-n<20): use published tables / modern computations (stable stems known through at least (90) per the summary) to check whether any stable stem is exactly (\mathbb Z/5), and/or whether any unstable entry past the Toda range is exactly (\mathbb Z/5). ([Wikipedia][1])
2. **Attack the “companionship” lemma ((\dagger))** using serious machinery: compare (p)-local homotopy groups and analyze whether integral groups in degrees supporting (5)-torsion can have vanishing (2)- and (3)-primary parts.
3. **Search for a constructive positive example**: look for a stem where the (5)-primary part is known to be (\mathbb Z/5) and attempt to prove the (2)- and (3)-primary parts vanish in that same stem (either stably or in a specific unstable range).

(iv) **What a minimal counterexample would likely look like (based on the verified finite-range scan):**
If (\pi_k(S^n)\cong \mathbb Z/5) exists at all, it must lie outside the Toda (k-n<20) tabulation; so it would have to occur in a *more highly unstable* region (roughly (k-n\ge 20), or in dimensions not covered) where the “extra” (2)- and (3)-torsion that accompanies (5)-torsion in low stems somehow disappears.

If you want, I can take one of these next-move directions (e.g. “stable stem search for a pure (\mathbb Z/5)”) and push it as far as the publicly available computed stable-stem data allows—but I cannot, from the material I can presently certify, give a complete proof or an explicit (\mathbb Z/5) occurrence.

[1]: https://en.wikipedia.org/wiki/Homotopy_groups_of_spheres "Homotopy groups of spheres - Wikipedia"
