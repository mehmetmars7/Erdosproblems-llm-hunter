## 1) FORMAL RESTATEMENT

### 1.1  Ambiguity in the literal question

The prompt asks:

> “Are there interesting periodicities in (n) with odd period? If not, are there good reasons not to expect them? For example, are there interesting manifold invariants which are naturally only defined when (n) is divisible by (3)?”

This is **not a theorem statement**: it is a meta-question, and “periodicity phenomenon” is ambiguous. Two standard, distinct meanings are:

* **(SPECTRUM PERIODICITY)** A (homotopy-)commutative ring spectrum (E) is *(d)-periodic* if there exists an invertible homogeneous element (u\in \pi_d(E)) such that multiplication by (u) gives isomorphisms (\pi_k(E)\xrightarrow{\cong}\pi_{k+d}(E)) for all (k).
* **(DIMENSION-CONGRUENCE EXISTENCE)** A geometric invariant is “only defined in dimensions divisible by (m)” if its very definition uses the equation (n=mk) (e.g. “middle-degree” forms on (H^k) whose target is (\mathbb Z) via evaluation on the fundamental class).

These lead to different answers.

### 1.2  Minimal corrected statement(s)

A minimal precise statement matching your example “only defined when (3\mid n)” is:

> **(C1)** For every integer (m\ge 2), every closed, connected, oriented smooth manifold (M^n) with (n=mk) has a canonical (m)-linear form
> [
> \mu_M^{(m)}:H^k(M;\mathbb Z)^{\times m}\to \mathbb Z,\qquad
> \mu_M^{(m)}(a_1,\dots,a_m)=\langle a_1\smile\cdots\smile a_m,,[M]\rangle,
> ]
> invariant under orientation-preserving homotopy equivalences. In particular, when (m=3) this gives a canonical **cubic** invariant defined exactly when (3\mid n).

A minimal precise “good reason not to expect odd periods” statement in the **spectrum periodicity** sense is:

> **(C2)** Let (R=\bigoplus_{i\in\mathbb Z}R_i) be a graded-commutative ring with unit (1) such that (2\cdot 1\ne 0\in R_0). Then **no homogeneous unit can have odd degree**. Consequently, if (E) is a homotopy-commutative ring spectrum with (2\ne 0\in \pi_0(E)), then any periodicity degree (d) (coming from a unit (u\in\pi_d(E))) must be **even**, so (E) cannot be odd-periodic.

I will **prove (C1) and (C2)**; together they answer both interpretations of your question.

---

## 2) QUICK LITERATURE/CONTEXT CHECK  (web available: YES)

* In **dimension (6)** (a first nontrivial case of (3\mid n)), Wall/Jupp-style classifications of simply connected (spin) (6)-manifolds use the **cubic form** on (H^2) given by triple cup product evaluated on ([M]), along with other data such as (p_1). For example, Schreieder–Tasin explicitly state that such manifolds are classified by a tuple including “the cubic form on (H^2), given by cup product” and (p_1). 
  Puppe’s exposition likewise lists among Wall’s invariants a symmetric trilinear form (\mu:H\times H\times H\to \mathbb Z) “corresponding to the cup product” and a linear form corresponding to (p_1). 
  A MathOverflow answer summarizing Wall’s invariants includes exactly the same “symmetric trilinear form (\mu(x,y,z)=\langle xyz,[M]\rangle)” description. ([MathOverflow][1])

* For the “why odd-period Bott-type periodicity is unlikely” part: in homotopy-commutative ring spectra, the homotopy groups form a graded-commutative ring with Koszul signs; Hesselholt explicitly notes that due to the sign rule, ordinary commutativity forces concentration in even degrees or characteristic (2). 
  (I will give a self-contained algebraic proof of the “odd-degree unit forces (2=0)” implication.)

---

## 3) ATTACK PLAN

### Proof track (chosen)

1. **Construct an explicit “(3\mid n)” invariant**: define the “middle” triple cup-product form on (H^{n/3}(M;\mathbb Z)) and prove it is well-defined and functorial (hence a manifold invariant).
2. **Show it is genuinely tied to (3\mid n)** (in the middle-degree sense): the construction requires (n=3k) to use the *same* degree (k) three times and land in (H^n\cong \mathbb Z).
3. **Explain why odd-periodicities of Bott/chromatic flavor are not expected**: prove the graded-commutative lemma that an odd-degree unit forces (2=0); apply to commutative ring spectra with (\pi_0) containing (\mathbb Z) (or just (2\ne 0)).

### Disproof/construction track (not needed)

If the implicit claim were “no odd-period phenomena exist,” then (C1) already provides a counterexample: the triple intersection/cup-product form.

---

## 4) WORK

### 4.1  Preliminaries and conventions

* A **closed** manifold means compact without boundary.
* A **connected oriented** (n)-manifold (M) has a fundamental class ([M]\in H_n(M;\mathbb Z)).
* For any space (X), singular cohomology (H^*(X;\mathbb Z)) is a graded ring with cup product (\smile), and evaluation pairing
  [
  \langle\cdot,\cdot\rangle:;H^n(M;\mathbb Z)\times H_n(M;\mathbb Z)\to \mathbb Z.
  ]
* For a connected oriented closed (n)-manifold (M), (H^n(M;\mathbb Z)\cong \mathbb Z), and evaluation on ([M]) identifies (H^n(M;\mathbb Z)) with (\mathbb Z).

I will use only functoriality and bilinearity of cup product and the evaluation pairing.

---

### 4.2  Odd-period (divisibility) invariant from Poincaré duality algebras

#### Lemma 4.2.1 (Definition and well-definedness of the middle (m)-fold form)

Let (m\ge 2) and (k\ge 0) be integers, and let (n=mk).
Let (M^n) be a closed, connected, oriented smooth manifold. Define
[
\mu_M^{(m)}:H^k(M;\mathbb Z)^{\times m}\to \mathbb Z
]
by
[
\mu_M^{(m)}(a_1,\dots,a_m):=\big\langle a_1\smile a_2\smile \cdots \smile a_m,,[M]\big\rangle.
]
Then (\mu_M^{(m)}) is a well-defined (m)-multilinear map of abelian groups.

**Proof.**

* Since each (a_i\in H^k(M;\mathbb Z)), the cup product (a_1\smile\cdots\smile a_m) lies in
  [
  H^{k+\cdots+k}(M;\mathbb Z)=H^{mk}(M;\mathbb Z)=H^n(M;\mathbb Z).
  ]
* Evaluation (\langle -, [M]\rangle) sends (H^n(M;\mathbb Z)\to \mathbb Z), hence (\mu_M^{(m)}(a_1,\dots,a_m)\in\mathbb Z) is defined.
* Cup product is bilinear in each variable, and evaluation is (\mathbb Z)-linear. Therefore (\mu_M^{(m)}) is multilinear in the (m) arguments. ∎

#### Lemma 4.2.2 (Homotopy invariance, orientation-sensitive)

Let (M^n,N^n) be closed, connected, oriented manifolds with (n=mk).
Let (f:M\to N) be a homotopy equivalence that is **orientation-preserving**, i.e.
[
f_*([M])=[N]\in H_n(N;\mathbb Z).
]
Then for all (a_1,\dots,a_m\in H^k(N;\mathbb Z)),
[
\mu_M^{(m)}(f^*a_1,\dots,f^*a_m)=\mu_N^{(m)}(a_1,\dots,a_m).
]

**Proof.**
Using naturality of the cup product,
[
f^*(a_1\smile\cdots\smile a_m)=f^*(a_1)\smile\cdots\smile f^*(a_m)\in H^n(M;\mathbb Z).
]
Therefore
[
\mu_M^{(m)}(f^*a_1,\dots,f^*a_m)
=\big\langle f^*(a_1\smile\cdots\smile a_m),[M]\big\rangle.
]
Now apply the defining naturality relation between pullback and pushforward under evaluation:
[
\langle f^*\alpha, x\rangle=\langle \alpha, f_*x\rangle
\quad\text{for }\alpha\in H^n(N;\mathbb Z),,x\in H_n(M;\mathbb Z),
]
to (\alpha=a_1\smile\cdots\smile a_m) and (x=[M]). This gives
[
\big\langle f^*(a_1\smile\cdots\smile a_m),[M]\big\rangle
=\big\langle a_1\smile\cdots\smile a_m, f_*[M]\big\rangle.
]
Since (f) is orientation-preserving, (f_*[M]=[N]), so the right-hand side equals
[
\big\langle a_1\smile\cdots\smile a_m,[N]\big\rangle=\mu_N^{(m)}(a_1,\dots,a_m).
]
This proves the claim. ∎

#### Corollary 4.2.3 (A “(3\mid n)” invariant)

Take (m=3). For every closed, connected, oriented (n)-manifold (M) with (n=3k), there is a canonical trilinear map
[
\mu_M^{(3)}:H^k(M;\mathbb Z)^{\times 3}\to \mathbb Z,
\qquad
\mu_M^{(3)}(a,b,c)=\langle a\smile b\smile c,[M]\rangle,
]
which is invariant under orientation-preserving homotopy equivalence.

In particular, this is a “middle-degree triple intersection form,” directly analogous to the ordinary intersection form in even dimensions ((m=2)).

**Justification.** Immediate from Lemmas 4.2.1–4.2.2 with (m=3). ∎

#### Tiny-case sanity checks (PHASE 1)

* (n=3), (k=1), (M=T^3=S^1\times S^1\times S^1).
  (H^1(T^3;\mathbb Z)\cong \mathbb Z^3) generated by (x,y,z) with (x\smile y\smile z) a generator of (H^3). Evaluating on ([T^3]) gives (\mu(x,y,z)=1). So (\mu_{T^3}^{(3)}) is nontrivial.
* (n=6), (k=2), (M=\mathbb{CP}^3).
  (H^*(\mathbb{CP}^3;\mathbb Z)\cong \mathbb Z[h]/(h^4)) with (\deg(h)=2). Then (h^3) generates (H^6) and (\langle h^3,[\mathbb{CP}^3]\rangle=1). Hence the cubic form (\mu(h,h,h)=1).

These computations show the (3\mid n) invariant is not vacuous.

#### “Interesting?” (link to classification)

In dimension (6) (real dimension (=3\cdot 2)), this cubic form on (H^2) is not just definable; it is a **primary piece of classification data** for simply connected (spin) (6)-manifolds in Wall/Jupp theory. Schreieder–Tasin explicitly list “the cubic form on (H^2), given by cup product” as part of the classification tuple, alongside (b_3,H^2,p_1). 
Puppe also lists Wall’s classification invariants including a “trilinear, symmetric form corresponding to the cup product in (H^*(M;\mathbb Z)).” 
The same trilinear form is described in a MathOverflow summary of Wall’s invariants as (\mu(x,y,z)=\langle xyz,[M]\rangle). ([MathOverflow][1])

So the (m=3) “middle triple product” is a genuine, structurally important odd-divisibility phenomenon.

---

### 4.3  Why Bott/chromatic-type periodicities don’t have odd period (when (2\neq 0))

#### Lemma 4.3.1 (Odd-degree unit forces (2=0) in a graded-commutative ring)

Let (R=\bigoplus_{i\in\mathbb Z}R_i) be a graded-commutative ring with unit (1).
Assume there exists a homogeneous unit (u\in R_d) of **odd** degree (d).
Then (2\cdot 1=0\in R_0). Equivalently, if (2\cdot 1\neq 0), then **every homogeneous unit has even degree**.

**Proof.**
Graded-commutativity says that for homogeneous (x\in R_p), (y\in R_q),
[
xy = (-1)^{pq}yx.
]
Apply this with (x=y=u), where (\deg(u)=d). Then
[
u^2 = (-1)^{d\cdot d}u^2 = (-1)^{d^2}u^2.
]
Since (d) is odd, (d^2) is odd, so ((-1)^{d^2}=-1). Hence
[
u^2 = -u^2 \quad\Longrightarrow\quad 2u^2=0.
]
Because (u) is a unit, so is (u^2), with inverse ((u^{-1})^2). Multiply (2u^2=0) by ((u^{-1})^2) to obtain
[
2\cdot 1 = 0.
]
This proves the lemma. ∎

#### Lemma 4.3.2 (Application to commutative ring spectra)

Let (E) be a homotopy-commutative (e.g. (E_\infty)) ring spectrum. Then (\pi_*(E)) is graded-commutative with Koszul signs. 
If furthermore (2\neq 0\in \pi_0(E)), then (\pi_*(E)) has **no odd-degree units**.

**Proof.**
The graded-commutativity of (\pi_*(E)) for commutative ring spectra is standard and is exactly the Koszul sign rule (as discussed e.g. in Hesselholt’s “Dirac ring” framework). 
Now apply Lemma 4.3.1 with (R=\pi_*(E)). If an odd-degree unit existed, it would force (2=0\in \pi_0(E)), contradicting the hypothesis. ∎

#### Corollary 4.3.3 (No odd periodicity degree for “integral” periodic theories)

Suppose (E) is a periodic commutative ring spectrum in the usual sense: there exists a unit (u\in\pi_d(E)) such that multiplication by (u) yields (\pi_k(E)\cong \pi_{k+d}(E)) for all (k).
If (2\neq 0\in \pi_0(E)) (e.g. (\pi_0(E)) contains (\mathbb Z)), then (d) must be **even**.

**Proof.**
The periodicity element (u) is, by definition, an invertible homogeneous element in degree (d). By Lemma 4.3.2, (d) cannot be odd. ∎

**Interpretation.**
This gives a clean structural reason that Bott-type periodicities (and more generally periodicities coming from commutative ring spectra whose (\pi_0) is not characteristic (2)) almost automatically have **even** period: odd-periodicity would require (2=0), i.e. essentially a characteristic-(2) context.

---

## 5) VERIFICATION

### 5.1  Check the “(3\mid n)” invariant definition

* **Degree check:** If (n=3k), then three degree-(k) classes multiply to degree (3k=n), so evaluation on ([M]) is defined.
* **Torsion inputs:** If any (a_i) is torsion, then (a_1\smile\cdots\smile a_m) is torsion in (H^n(M;\mathbb Z)). But (H^n(M;\mathbb Z)\cong\mathbb Z) is torsion-free, so the product must be (0). Hence (\mu) is still well-defined over (\mathbb Z) without quotienting out torsion.
* **Orientation sensitivity:** If we reverse the orientation, ([M]\mapsto -[M]), then (\mu\mapsto -\mu). That is expected; the invariant is of *oriented* manifolds.

### 5.2  Check invariance lemma

* The only needed facts were: (f^*) is a ring homomorphism on cohomology and (\langle f^*\alpha,[M]\rangle=\langle \alpha,f_*[M]\rangle). Both are standard naturality properties of singular (co)homology.
* Orientation-preserving condition (f_*[M]=[N]) is exactly what is needed; otherwise one gets a sign.

### 5.3  Check the “no odd periodicity unit” lemma

* The argument crucially uses that (u^2) is invertible if (u) is invertible. This is true because ((u^{-1})^2) is a two-sided inverse of (u^2).
* If (2=0), the obstruction disappears, so odd periodicity is not ruled out in characteristic (2). This matches the algebra.

Everything is quantifier-clean and boundary cases are accounted for.

---

## 6) FINAL

**PROOF**

### Theorem (odd-divisibility invariants exist; odd Bott-type periods do not when (2\neq 0))

1. (**Odd-period divisibility phenomenon**)
   For every integer (m\ge 2), every closed connected oriented manifold (M^n) with (n=mk) has a canonical multilinear invariant
   [
   \mu_M^{(m)}:H^k(M;\mathbb Z)^{\times m}\to \mathbb Z,\qquad
   \mu_M^{(m)}(a_1,\dots,a_m)=\langle a_1\smile\cdots\smile a_m,[M]\rangle,
   ]
   invariant under orientation-preserving homotopy equivalence. In particular, for (m=3) this gives a natural manifold invariant defined exactly when (3\mid n) (the “middle triple cup-product form”), and it is a key datum in the classification of simply connected (6)-manifolds. 

2. (**Reason odd Bott/chromatic-type periods are not expected**)
   If (R) is a graded-commutative ring with (2\cdot 1\neq 0), then (R) has no homogeneous unit in odd degree. Consequently, any periodic commutative ring spectrum (E) with (2\neq 0\in\pi_0(E)) cannot be odd-periodic; any periodicity degree must be even. 

### Proof

(1) Lemma 4.2.1 defines (\mu_M^{(m)}) and proves it is well-defined and multilinear. Lemma 4.2.2 proves invariance under orientation-preserving homotopy equivalence.

(2) Lemma 4.3.1 proves that an odd-degree unit forces (2=0) in any graded-commutative ring; Lemma 4.3.2 applies this to (\pi_*(E)) for commutative ring spectra, yielding Corollary 4.3.3: periodicity degrees must be even when (2\neq 0). ∎

[1]: https://mathoverflow.net/questions/279931/a-search-for-a-sequence-of-6-manifolds "at.algebraic topology - A search for a sequence of $6$-manifolds - MathOverflow"
