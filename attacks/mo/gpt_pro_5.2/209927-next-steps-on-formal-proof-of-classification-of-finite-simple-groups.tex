\section{Problem 209927: Next steps on formal proof of classification of finite simple groups}

\subsection*{1) FORMAL RESTATEMENT}

\paragraph{Ambiguity/misstatement.}
The attachment is not a single mathematical statement with quantifiers; it is a cluster of informal, time-dependent questions:
\begin{itemize}
\item whether there are efforts/plans to formalize more of CFSG,
\item whether components from the Odd Order formalization can be reused/ported,
\item an informal estimate of the time scale for formalizing CFSG.
\end{itemize}
Such questions are not well-posed for a mathematical proof/disproof as written, because they depend on changing historical facts and on subjective forecasting.

\paragraph{Minimal corrected statement (objective and checkable).}
I replace the informal questions by the following concrete, time-stamped existence claim.

\medskip
\noindent\textbf{Corrected Statement (CS).}
\emph{As of January 14, 2026, there exists at least one publicly documented project or work item in a mainstream proof assistant ecosystem (e.g.\ Lean or Coq) whose stated goal is to formalize nontrivial pieces directly connected to the Classification of Finite Simple Groups (CFSG) beyond the already-formalized Feit--Thompson Odd Order Theorem; moreover, the libraries produced for the Odd Order Theorem have been explicitly described as reusable infrastructure for further finite group formalization work.}

\subsection*{2) QUICK LITERATURE/CONTEXT CHECK (web available)}

The following publicly accessible sources support (CS):
\begin{itemize}
\item Floris van Doorn's slides \emph{``Formal Abstracts: Classification of Finite Simple Groups''} (JMM 2019) explicitly present a Lean project to formalize the \emph{statements} of major theorems in CFSG and to build infrastructure aimed at later formalization work.
\item The ``Formal Abstracts'' project site lists the CFSG among its targets.
\item The Lean Zulip transcript ``Statement of classification of finite simple groups'' documents discussion and work toward formalizing the CFSG statement and related infrastructure.
\item The 2025 arXiv paper \emph{``Classifying the groups of order $pq$ in Lean''} explicitly discusses the gap between current formalization and CFSG, and mentions ongoing formalization work on the O'Nan--Scott theorem.
\item Mathematical Components (Coq/SSReflect) describes itself as reusable library infrastructure used in the Odd Order formalization (and other large formalizations).
\end{itemize}

\subsection*{3) ATTACK PLAN}

\paragraph{Proof track.}
To prove (CS) it suffices to exhibit explicit, publicly documented examples (projects/papers/slides/repos) matching the corrected statement, and to verify they indeed claim CFSG relevance and go beyond Odd Order.

\paragraph{Disproof track.}
To disprove (CS) one would need to show there are \emph{no} such publicly documented projects as of the date, which is falsified by a single explicit example.

\paragraph{Chosen path.}
Provide explicit examples; conclude (CS) is true.

\subsection*{4) WORK}

\begin{theorem}[Corrected Statement (CS) holds]\label{thm:cfsg-cs}
The corrected statement (CS) is true.
\end{theorem}

\begin{proof}
We verify the two parts of (CS).

\paragraph{(i) Existence of a publicly documented CFSG-related formalization effort beyond Odd Order.}

\emph{Example 1: ``Formal Abstracts'' for CFSG in Lean.}
A publicly available PDF titled \emph{``Formal Abstracts: Classification of Finite Simple Groups''} (slides by Floris van Doorn) describes a project in Lean that aims to formalize ``formal abstracts'' of major mathematical results, including the CFSG.\footnote{\url{https://florisvandoorn.com/talks/2019_jmm_formal_abstracts_cfsg.pdf}}
This is beyond Odd Order in scope because it targets the classification theorem itself (at least at the level of formal statement and supporting definitions).

The ``Formal Abstracts'' project website lists the classification of finite simple groups among its topics.\footnote{\url{https://formalabstracts.github.io/}}

\emph{Example 2: Lean Zulip discussion/work on the statement of CFSG.}
A Zulip transcript titled \emph{``Statement of classification of finite simple groups''} documents detailed discussion toward formalizing the statement and relevant background.\footnote{\url{https://leanprover.zulipchat.com/stream/113488-general/topic/Statement.20of.20classification.20of.20finite.20simple.20groups.html}}

\emph{Example 3: Formalization work explicitly mentioning O'Nan--Scott.}
The arXiv paper \emph{``Classifying the groups of order $pq$ in Lean''} (2025) states that ``the formalization of CFSG is currently far out of reach'', and also reports that the O'Nan--Scott theorem is ``currently being formalised''.\footnote{\url{https://arxiv.org/abs/2501.10152}}
This constitutes a publicly documented effort on nontrivial finite group classification infrastructure, distinct from (and in addition to) the Odd Order Theorem.

These examples establish existence of CFSG-relevant formalization efforts beyond Odd Order.

\paragraph{(ii) Reusability of Odd Order infrastructure.}

Mathematical Components is described as a Coq/SSReflect library that ``constitutes the infrastructure for the machine checked proofs of the Four Color Theorem and of the Odd Order Theorem.''\footnote{\url{https://math-comp.github.io/mcb/}}
This is explicit documentation that the Odd Order formalization produced (and relied on) substantial reusable libraries. Moreover, the existence of a separately packaged finite group theory component (e.g.\ \texttt{mathcomp.fingroup}) with ongoing releases is public evidence that the library is maintained and reusable beyond the original proof effort.\footnote{\url{https://rocq-prover.org/p/rocq-mathcomp-fingroup/2.5.0}}

Together, (i) and (ii) prove (CS).
\end{proof}

\subsection*{5) VERIFICATION}

\paragraph{Check of scope.}
The corrected statement (CS) is carefully time-stamped and existential: it asks only for \emph{at least one} publicly documented effort beyond Odd Order, and for an explicit claim of reusability of the Odd Order infrastructure. The cited sources provide exactly that.

\paragraph{No forecasting.}
The original MO question asked for a time estimate. This is not mathematically provable and is omitted from (CS).

\subsection*{6) FINAL}

\textbf{PROOF.}
