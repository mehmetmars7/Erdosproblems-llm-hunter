\section{MO 64146: Grothendieck--Teichm\"uller conjecture}

\subsection*{1) FORMAL RESTATEMENT}
\paragraph{Literal statement and ambiguity.}
The literal statement refers to a ``category'' $\widehat{T}$ with objects $T_{g,n}$ and ``natural morphisms between moduli spaces'' but does not specify:
(i) which basepoints are used to define profinite \'{e}tale fundamental \emph{groupoids},
(ii) which pairs $(g,n)$ are included (usually $2g-2+n>0$),
(iii) what structure the morphisms are required to preserve (forgetful maps, gluing maps, symmetric group actions, etc.),
(iv) whether $\mathrm{Aut}(\widehat{T})$ means automorphisms of the underlying abstract category, or automorphisms of the tower/functor together with its specified generating morphisms, and
(v) whether automorphisms are taken up to inner automorphisms (usually \emph{outer} automorphisms appear because of basepoint choices).

\paragraph{Minimal corrected statement (standard convention).}
Fix the \emph{Teichm\"uller tower} (also called the profinite Teichm\"uller groupoid/tower): for each $(g,n)$ with $2g-2+n>0$, let $\Mgn{g}{n}$ be the Deligne--Mumford moduli stack of smooth $n$-pointed genus $g$ curves over $\mathbb{Q}$.
Choose compatible tangential/geometric basepoints so that one has profinite \'{e}tale fundamental groupoids
\[
T_{g,n}:=\pi_1^{\mathrm{\acute et}}(\Mgn{g}{n}\times_{\mathbb{Q}}\Qbar),
\]
viewed as profinite groupoids. The collection $\widehat{T}:=\{T_{g,n}\}$ is equipped with the functoriality induced by the standard morphisms of moduli spaces (forgetting marked points, gluing/pair-of-pants operations, and symmetric group actions on markings).

Since $\Mgn{g}{n}$ is defined over $\mathbb{Q}$, the absolute Galois group $\GQ=\mathrm{Gal}(\Qbar/\mathbb{Q})$ acts on $\Mgn{g}{n}\times \Qbar$ and hence on each $T_{g,n}$ by functoriality of \'{e}tale $\pi_1$, compatibly with the tower structure. This yields a natural continuous homomorphism
\[
\rho:\GQ \longrightarrow \mathrm{Aut}_{\mathrm{tower}}(\widehat{T}),
\]
where $\mathrm{Aut}_{\mathrm{tower}}(\widehat{T})$ denotes automorphisms of the Teichm\"uller tower preserving the specified structural morphisms.

\begin{quote}
\textbf{Corrected Grothendieck--Teichm\"uller conjecture.} The map $\rho$ is an isomorphism of profinite groups.
\end{quote}

\paragraph{Edge cases.}
For $(g,n)=(0,3)$, $\Mgn{0}{3}$ is a point and $T_{0,3}$ is trivial; the first nontrivial case is $(0,4)$ where $\Mgn{0}{4}\cong \mathbb{P}^1\setminus\{0,1,\infty\}$ and $T_{0,4}\cong \widehat{F}_2$ (profinite completion of a free group on two generators), up to basepoint choices.

\subsection*{2) QUICK LITERATURE/CONTEXT CHECK}
The MathOverflow question itself reports the conjecture and the associated Deligne--Drinfeld freeness conjecture.\footnote{See the MO page \url{https://mathoverflow.net/questions/64146/grothendieck-teichm\%C3\%BCller-conjecture}.}
A standard modern package of references includes:
\begin{itemize}[leftmargin=2em]
\item Grothendieck's \emph{Esquisse d'un Programme} (English translation available) where the ``Lego--Teichm\"uller'' program and Galois action on fundamental groups of moduli spaces are proposed.\footnote{\url{https://webusers.imj-prg.fr/~leila.schneps/grothendieckcircle/EsquisseEng.pdf}}
\item Bakalov--Kirillov's rigorous ``Lego--Teichm\"uller game'' proof of a presentation of surface decompositions (related to Moore--Seiberg), giving substance to the ``generated in dimension 1, relations in dimension 2'' philosophy.\footnote{arXiv:math/9809057 \url{https://arxiv.org/abs/math/9809057}}
\item Schneps' lecture notes/surveys on Grothendieck--Teichm\"uller theory.\footnote{Example notes: \url{https://swc-math.github.io/notes/files/05SchnepsNotes.pdf}}
\item Brown's paper on mixed Tate motives over $\mathbb{Z}$, relevant to the motivic avatar discussed in the MO post.\footnote{F. Brown, \emph{Mixed Tate motives over $\mathbb{Z}$}, arXiv:1102.1312, \url{https://arxiv.org/abs/1102.1312}}
\item Recent work constructs approximations/``shadows'' of elements of (variants of) $\widehat{GT}$ and studies inverse limit descriptions (evidence/structure rather than a proof of surjectivity).\footnote{V. Dolgushev--J. Guynee, \emph{GT-shadows for the gentle version of the Grothendieck--Teichm\"uller group}, arXiv:2401.06870, \url{https://arxiv.org/abs/2401.06870}}
\item Numerical verification of the Deligne--Drinfeld conjecture in low weight has been pushed to weight $29$.\footnote{F. Naef--T. Willwacher, \emph{Numerical computation of linearized KV and the Deligne--Drinfeld and Broadhurst--Kreimer conjectures}, arXiv:2508.08081, \url{https://arxiv.org/abs/2508.08081}}
\end{itemize}
As of these references, the \emph{surjectivity} part of $\rho$ (i.e.\ $\GQ \cong \mathrm{Aut}(\widehat{T})$) remains an open problem in mainstream accounts; the literature emphasizes partial results and related structures rather than a complete proof.

\subsection*{3) ATTACK PLAN}
\paragraph{Problem type.}
Anabelian/algebraic geometry + profinite/Teichm\"uller theory + Galois actions; it is a \emph{reconstruction/identification} problem for a profinite group from its actions on geometric fundamental groups.

\paragraph{Likely tools (why relevant).}
\begin{itemize}[leftmargin=2em]
\item \textbf{Belyi's theorem / dessins d'enfants:} gives faithfulness of Galois action on $\pi_1^{\acute et}(\mathbb{P}^1\setminus\{0,1,\infty\})$ and therefore injectivity.
\item \textbf{Drinfeld's definition of $\widehat{GT}$:} reduces the automorphism problem to explicit relations on $\widehat{F}_2$.
\item \textbf{Anabelian geometry / reconstruction:} to recover $\GQ$ from its action on a tower of fundamental groups.
\item \textbf{Operads and Grothendieck--Teichm\"uller Lie algebra $\mathfrak{grt}_1$:} relates to deformation theory/graph complexes and motivic aspects.
\item \textbf{Congruence/level structures and finite quotients:} attempt to characterize the image via actions on finite-level Teichm\"uller/Modular data.
\item \textbf{Profinite rigidity phenomena:} to show any tower automorphism comes from arithmetic.
\end{itemize}

\paragraph{Proof-track strategies.}
\begin{enumerate}[leftmargin=2em]
\item Show $\rho$ is injective (known; can be taken as a proved theorem in the literature).
\item Identify $\mathrm{Aut}_{\mathrm{tower}}(\widehat{T})$ with Drinfeld's $\widehat{GT}$ and prove $\GQ=\widehat{GT}$ by reconstructing a Galois element from its effect on a generating set of morphisms (genus $0$ suffices by ``Lego--Teichm\"uller'').
\end{enumerate}

\paragraph{Disproof-track strategies.}
\begin{enumerate}[leftmargin=2em]
\item Construct an explicit automorphism of the tower satisfying all Moore--Seiberg/operadic constraints but not induced by any Galois element.
\item More weakly: find two distinct elements of $\mathrm{Aut}_{\mathrm{tower}}(\widehat{T})$ acting identically on all finite quotients arising from $\GQ$; this would violate injectivity or compatibility.
\end{enumerate}

\paragraph{Best path chosen.}
Given the state of the art, a complete proof/disproof of the conjecture is not currently accessible here. The best feasible output is: (i) isolate the conjecture precisely, (ii) prove fully some \emph{provable} partial statements (notably injectivity and consistency checks), and (iii) identify the first genuine gap (surjectivity).

\subsection*{4) WORK}
\subsubsection*{Phase 1: tiny-case sanity checks}
\begin{itemize}[leftmargin=2em]
\item For $(g,n)=(0,3)$, the tower object is trivial, giving no information.
\item For $(g,n)=(0,4)$, $\Mgn{0}{4}\cong \mathbb{P}^1\setminus\{0,1,\infty\}$ and $T_{0,4}$ is the profinite completion of a free group on two generators, up to basepoint choices. Hence $\rho$ factors through a homomorphism $\GQ\to \mathrm{Out}(\widehat{F}_2)$ (and in fact through $\widehat{GT}\subset \mathrm{Out}(\widehat{F}_2)$).
\end{itemize}

\subsubsection*{Lemma 1 (Injectivity of the Galois action in the first nontrivial level)}
\begin{quote}
\textbf{Claim.} The natural homomorphism $\GQ \to \mathrm{Out}(\pi_1^{\acute et}(\mathbb{P}^1_{\Qbar}\setminus\{0,1,\infty\}))\cong \mathrm{Out}(\widehat{F}_2)$ is injective.
\end{quote}

\paragraph{Justification.}
This is a classical consequence of Belyi's theorem and Grothendieck's dessins d'enfants program: $\GQ$ acts faithfully on finite \'{e}tale covers of $\mathbb{P}^1\setminus\{0,1,\infty\}$ (equivalently dessins), and these covers are sufficiently rich to detect any nontrivial $\sigma\in \GQ$. A standard reference path is through surveys/notes in Grothendieck--Teichm\"uller theory.\footnote{See e.g.\ Schneps' notes \url{https://swc-math.github.io/notes/files/05SchnepsNotes.pdf} and Grothendieck's Esquisse translation \url{https://webusers.imj-prg.fr/~leila.schneps/grothendieckcircle/EsquisseEng.pdf}.}
\hfill$\square$

\subsubsection*{Lemma 2 (Surjectivity is the first hard gap)}
\begin{quote}
\textbf{Claim.} The conjecture reduces (in standard formulations) to showing the image of $\GQ$ in $\mathrm{Aut}_{\mathrm{tower}}(\widehat{T})$ is all of $\mathrm{Aut}_{\mathrm{tower}}(\widehat{T})$ (or equivalently all of $\widehat{GT}$ once the identification is made).
\end{quote}

\paragraph{Justification.}
By construction $\rho$ is a homomorphism into the full tower automorphism group. Lemma 1 gives a strong piece of evidence: $\rho$ is injective already at the $(0,4)$ level. The remaining direction is surjectivity, i.e.\ that every tower automorphism preserving the Teichm\"uller structure is arithmetic. Existing references emphasize partial results, approximations (e.g.\ GT-shadows) and motivic/graded avatars (e.g.\ $\mathfrak{grt}_1$), but do not claim a complete surjectivity proof.\footnote{E.g.\ Dolgushev--Guynee \url{https://arxiv.org/abs/2401.06870} develop approximations; Naef--Willwacher \url{https://arxiv.org/abs/2508.08081} provide numerical checks of the Deligne--Drinfeld conjecture in bounded weight.}
\hfill$\square$

\subsubsection*{Motivic/graded avatar (Deligne--Drinfeld) sanity}
The Deligne--Drinfeld conjecture concerns the graded Lie algebra $\mathfrak{grt}_1$ attached to the Grothendieck--Teichm\"uller Lie algebra (in characteristic zero). Numerical evidence is available up to weight $29$.\footnote{\url{https://arxiv.org/abs/2508.08081}}
This supports, but does not imply, the profinite group-level surjectivity conjecture.

\subsection*{5) VERIFICATION}
\begin{itemize}[leftmargin=2em]
\item \textbf{Quantifiers:} the conjecture is an \emph{isomorphism} statement (injective + surjective) between profinite groups; we isolated that injectivity is known at low level.
\item \textbf{Ambiguity check:} the statement depends on the precise definition of $\mathrm{Aut}(\widehat{T})$ (tower automorphisms vs raw category automorphisms; inner vs outer). The corrected statement explicitly fixes ``tower automorphisms preserving the specified structural maps''.
\item \textbf{Self-attack:} The partial ``Lemma 1'' relies on deep theorems (Belyi, dessins faithfulness) which are standard but not reproved here; this is acceptable only as a cited external theorem. The main conjecture remains out of reach.
\end{itemize}

\subsection*{6) FINAL}
\begin{center}
\textbf{UNRESOLVED}
\end{center}

\paragraph{(i) Strongest proved partial result here.}
Injectivity of $\GQ$ into $\mathrm{Out}(\widehat{F}_2)$ (hence into the genus-$0$ Teichm\"uller tower automorphisms) is established in the literature via Belyi/dessins.

\paragraph{(ii) First gap.}
Surjectivity: prove every tower automorphism (equivalently every element of $\widehat{GT}$ in the standard genus-$0$ reduction) arises from $\GQ$.

\paragraph{(iii) Top 3 next moves.}
(1) Make the identification $\mathrm{Aut}_{\mathrm{tower}}(\widehat{T})\cong \widehat{GT}$ precise and reduce to genus $0$ via a rigorous ``Lego--Teichm\"uller'' theorem; (2) attempt a reconstruction theorem from the action on finite quotients/level structures; (3) analyze whether GT-shadows/finite approximations detect non-Galois automorphisms.

\paragraph{(iv) What a minimal counterexample would look like.}
An explicit $\varphi\in \widehat{GT}$ (or tower automorphism) not lying in the image of $\GQ$, detectable on some finite quotient or on a specific configuration/dessin.
