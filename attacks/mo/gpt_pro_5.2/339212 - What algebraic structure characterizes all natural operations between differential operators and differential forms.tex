\section{MO 339212: ``What algebraic structure characterizes all natural operations between differential operators and differential forms?''}
\label{sec:mo339212}
\noindent\textbf{MathOverflow link:} \url{https://mathoverflow.net/questions/339212/what-algebraic-structure-characterizes-all-natural-operations-between-differenti} (accessed 2026-01-16).

\subsection*{1) FORMAL RESTATEMENT}
\textbf{Literal issue.} The MO post is a \emph{classification/characterization question}, not a yes/no proposition. As written, it is not a statement with a truth value.

\medskip
\noindent\textbf{Minimal corrected statement (one precise subclaim consistent with standard conventions).}
A foundational piece of ``natural operations'' between vector fields and differential forms is the classification of degree $-1$ derivations of the de~Rham algebra.

\medskip
\noindent\textbf{Theorem (T).}
Let $M$ be a smooth (say $C^\infty$) manifold. Let $\Omega^*(M)$ be the graded-commutative algebra of smooth differential forms.
Suppose $D: \Omega^*(M)\to\Omega^{*-1}(M)$ is an $\RR$-linear graded derivation of degree $-1$ such that $D(f)=0$ for all functions $f\in C^\infty(M)=\Omega^0(M)$.
Then there exists a unique smooth vector field $X$ on $M$ such that
\[
D(\omega)=\iota_X\omega \quad \text{for all }\omega\in\Omega^*(M),
\]
where $\iota_X$ denotes interior contraction.

\subsection*{2) QUICK LITERATURE/CONTEXT CHECK}
The MathOverflow question itself has no posted answers as of 2026-01-16.
The theorem (T) is standard differential geometry: contraction with a vector field is a degree $-1$ derivation, and conversely any degree $-1$ derivation vanishing on functions is determined by its action on $1$-forms, hence by a vector field.

\subsection*{3) ATTACK PLAN}
\textbf{Proof track.} Prove (T) directly:
\begin{enumerate}[leftmargin=2em]
\item Show that $D$ is determined by its restriction to $\Omega^1(M)$ by the derivation property.
\item Use $\omega\mapsto D(\omega)$ on $1$-forms to define a $C^\infty(M)$-linear map $\Omega^1(M)\to C^\infty(M)$.
\item Identify such maps with vector fields via duality: $\mathfrak{X}(M) \cong \mathrm{Hom}_{C^\infty(M)}(\Omega^1(M), C^\infty(M))$.
\end{enumerate}

\textbf{Disproof track.} Not applicable for (T) since it is a concrete statement.

\subsection*{4) WORK}
\subsubsection*{Lemma 4.1 (Derivation determined by degree $0$ and $1$)}
Let $D$ be a graded derivation of degree $-1$ on a graded-commutative algebra $A=\bigoplus_{k\ge 0} A^k$.
If $D$ vanishes on $A^0$ and is given on $A^1$, then $D$ is uniquely determined on all of $A$.

\textbf{Proof.}
Every homogeneous element of $A^k$ is a sum of products of $k$ elements of degree $1$ (in $\Omega^*(M)$ this holds locally and hence globally by partition of unity; more directly, differential forms are generated as an algebra by functions and $1$-forms).
Using the graded Leibniz rule,
\[
D(\alpha\wedge\beta)=D(\alpha)\wedge\beta + (-1)^{|\alpha|}\alpha\wedge D(\beta),
\]
we can reduce evaluation of $D$ on any wedge product to values on $A^0$ and $A^1$. Since $D\vert_{A^0}=0$ is known, $D$ is determined by its values on degree $1$. \qed

\subsubsection*{Lemma 4.2 ($C^\infty$-linearity on $1$-forms)}
For $D$ as in (T), the restriction $D\vert_{\Omega^1(M)}: \Omega^1(M)\to C^\infty(M)$ is $C^\infty(M)$-linear.

\textbf{Proof.}
Let $f\in C^\infty(M)$ and $\eta\in\Omega^1(M)$. Then
\[
D(f\eta) = D(f)\eta + fD(\eta) = 0\cdot \eta + fD(\eta) = fD(\eta),
\]
using the derivation property and the assumption $D(f)=0$. \qed

\subsubsection*{Lemma 4.3 (Vector fields as $C^\infty$-linear maps)}
There is a canonical isomorphism of $C^\infty(M)$-modules
\[
\mathfrak{X}(M) \cong \mathrm{Hom}_{C^\infty(M)}(\Omega^1(M), C^\infty(M)),
\]
given by $X\mapsto (\eta\mapsto \eta(X))$.

\textbf{Proof.}
A smooth vector field $X$ assigns to each point $p\in M$ a tangent vector $X_p\in T_pM$. Evaluation of a $1$-form $\eta$ on $X$ yields a function $p\mapsto \eta_p(X_p)$.
This is $C^\infty(M)$-linear in $\eta$.
Conversely, a $C^\infty(M)$-linear functional $\ell: \Omega^1(M)\to C^\infty(M)$ defines at each $p$ a linear functional on $T_p^*M$ by $\eta_p\mapsto \ell(\eta)(p)$; by finite-dimensional duality this corresponds to a unique tangent vector $X_p\in T_pM$.
Smoothness of $p\mapsto X_p$ follows from local coordinate expressions: in a chart with coordinates $(x^i)$, the $dx^i$ generate $\Omega^1$ as a $C^\infty$-module and $\ell(dx^i)$ are smooth functions giving the components of $X$.
These constructions are inverse. \qed

\subsubsection*{Proof of Theorem (T)}
By Lemma 4.2, $D\vert_{\Omega^1(M)}$ is a $C^\infty(M)$-linear map $\Omega^1(M)\to C^\infty(M)$. By Lemma 4.3, there exists a unique vector field $X$ such that
\[
D(\eta)=\eta(X)=\iota_X\eta \quad \text{for all }\eta\in\Omega^1(M).
\]
Both $D$ and $\iota_X$ are graded derivations of degree $-1$ and both vanish on functions.
Since they agree on $1$-forms, Lemma 4.1 implies they agree on all of $\Omega^*(M)$.
Uniqueness of $X$ follows from Lemma 4.3. \qed

\subsection*{5) VERIFICATION}
\begin{itemize}[leftmargin=2em]
\item The only subtle point is the claim that $\Omega^*(M)$ is generated by functions and $1$-forms; this is standard: $k$-forms are $C^\infty$-linear combinations of wedges of $1$-forms.
\item Uniqueness and smoothness of the vector field reconstructed from $\ell$ are checked in coordinates.
\end{itemize}

\subsection*{6) FINAL}
\begin{center}
\textbf{PROOF}
\end{center}


