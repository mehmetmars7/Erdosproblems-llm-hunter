\section{MO 459428 --- ``1089'' number trick and symmetric group/cohomology}
\label{sec:mo459428}
\noindent\textbf{MathOverflow link:} \href{https://mathoverflow.net/questions/459428}{https://mathoverflow.net/questions/459428}.

\subsection*{1) FORMAL RESTATEMENT}
\textbf{Ambiguity / misstatement.}
The MO post asks for an ``interpretation'' in terms of a symmetric group action on an Ext/cohomology group. This is not a single formal proposition.

\medskip
\textbf{Minimal corrected statement (concrete, checkable).}
We formalize the classical ``1089 trick'' as an identity:
\begin{quote}
Let $N$ be a three-digit base-$10$ integer with digits $(a,b,c)$ and $a\ne c$. Let $\mathrm{rev}(N)$ be the integer with digits reversed. Define
\[
D := |N-\mathrm{rev}(N)|,
\]
and interpret $D$ as a three-digit string allowing a leading zero. Then
\[
D + \mathrm{rev}(D) = 1089.
\]
\end{quote}
Additionally, we exhibit a symmetric-group (digit-permutation) viewpoint explaining why the digit contribution cancels and only carries contribute.

\subsection*{2) QUICK LITERATURE/CONTEXT CHECK}
The MO post points to D.~Isaksen's ``carrying cocycle'' viewpoint on arithmetic.

\subsection*{3) ATTACK PLAN}
\textbf{Proof strategy.}
\begin{itemize}
  \item Reduce $N-\mathrm{rev}(N)$ to $99|a-c|$.
  \item Write $99d$ in decimal digits and compute reverse-plus-original explicitly.
\end{itemize}
\textbf{Cohomology/group-ring interpretation (secondary).}
\begin{itemize}
  \item View digit reversal as the transposition $\tau=(1\ 3)$ acting on digit vectors.
  \item Note the group-ring identity $(1-\tau)+\tau(1-\tau)=0$ forces cancellation of the ``no-carry'' part; the constant comes from carries/borrows.
\end{itemize}

\subsection*{4) WORK}
\begin{theorem}[1089 trick]
Let $a,b,c\in\{0,1,\dots,9\}$ with $a\ne 0$ and $a\ne c$. Let
\[
N := 100a+10b+c,\qquad \mathrm{rev}(N):=100c+10b+a.
\]
Let $D:=|N-\mathrm{rev}(N)|$. Write $D$ as a three-digit decimal string (allowing a leading zero), and let $\mathrm{rev}(D)$ be the integer obtained by reversing those three digits. Then
\[
D+\mathrm{rev}(D)=1089.
\]
\end{theorem}
\begin{proof}
Compute the difference:
\[
N-\mathrm{rev}(N) = (100a+10b+c)-(100c+10b+a) = 99(a-c).
\]
Let $d:=|a-c|\in\{1,2,\dots,9\}$. Then $D=99d$.
We now compute the three decimal digits of $99d$.
Observe that
\[
99d = 100(d-1) + 90 + (10-d).
\]
Indeed the right-hand side equals $100d-100+100-d=99d$.
Since $1\le d\le 9$, the integers
\[
(d-1)\in\{0,\dots,8\},\qquad 9\in\{0,\dots,9\},\qquad (10-d)\in\{1,\dots,9\}
\]
are valid base-$10$ digits.
Therefore the three-digit decimal string for $D$ is exactly
\[
D = \overline{(d-1)\ 9\ (10-d)}.
\]
Reversing gives
\[
\mathrm{rev}(D)=\overline{(10-d)\ 9\ (d-1)}.
\]
Add these two integers:
\begin{align*}
D+\mathrm{rev}(D)
&= \bigl(100(d-1)+90+(10-d)\bigr) + \bigl(100(10-d)+90+(d-1)\bigr)\\
&= 100\bigl((d-1)+(10-d)\bigr) + 180 + \bigl((10-d)+(d-1)\bigr)\\
&= 100\cdot 9 + 180 + 9 = 1089.
\end{align*}
This computation is independent of $b$, and depends on $a,c$ only through $d=|a-c|$.
\end{proof}

\medskip
\textbf{Digit-permutation / group-ring viewpoint (precise but informal-to-formal bridge).}
Let $\tau=(1\ 3)\in S_3$ act on digit triples by swapping the first and third digit.
The operation ``subtract the reverse'' corresponds (ignoring borrows) to applying $(1-\tau)$.
The operation ``add the reverse'' corresponds (ignoring carries) to applying $(1+\tau)$.
In the group ring $\mathbb{Z}[S_3]$ we have $(1+\tau)(1-\tau)=1-\tau^2=0$, explaining why the ``digit part'' cancels; the nonzero constant $1089=9\cdot 111$ arises from systematic carries/borrows in base $10$.

\subsection*{5) VERIFICATION}
Boundary cases:
\begin{itemize}
  \item If $a=c$, then $D=0$ and the statement fails unless one explicitly excludes this case (as done).
  \item Allowing a leading zero in $D$ is necessary (e.g. $a-c=1$ gives $D=99$ which is treated as $099$).
\end{itemize}
All algebraic steps are explicit.

\subsection*{6) FINAL}
\textbf{PROOF}.

% -----------------------------------------------------------------------------
