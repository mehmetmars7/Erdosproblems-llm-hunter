\section{MO 58793: Nontrivial tangent bundle with total space diffeomorphic to $M\times\mathbb{R}^n$}

\subsection*{1) Formal restatement}
\paragraph{Ambiguities to fix.}
The statement should specify whether $M$ is assumed connected/closed, and what notion of ``trivial bundle'' is used (smooth vector bundle isomorphism over $M$).

\paragraph{Minimal corrected question.}
Does there exist a smooth connected $n$-manifold $M$ such that:
\begin{enumerate}[label=(\alph*)]
\item $TM$ is \emph{nontrivial} as a rank-$n$ smooth real vector bundle over $M$ (i.e.\ $TM\not\cong M\times\mathbb{R}^n$ over $M$);
\item There exists a \emph{diffeomorphism of manifolds} (not necessarily fiber-preserving)
\[
\Phi:TM \xrightarrow{\ \cong\ } M\times\mathbb{R}^n.
\]
\end{enumerate}

\subsection*{2) Quick literature/context check}
The MathOverflow post includes comments indicating:
\begin{itemize}[leftmargin=*]
\item For spheres, such an example does not occur (tangent bundle total space is not diffeomorphic to the trivial product).
\item There are results comparing diffeomorphisms of total spaces of vector bundles to equivalence of bundles under additional hypotheses on the induced map on the base (e.g.\ if the induced homotopy equivalence is homotopic to a diffeomorphism).
\end{itemize}

\subsection*{3) Attack plan}
\begin{itemize}[leftmargin=*]
\item \textbf{Proof track (no examples):} Show that any diffeomorphism $\Phi:TM\to M\times\mathbb{R}^n$ forces $TM$ to be trivial by extracting a bundle isomorphism from $\Phi$ (e.g.\ using the image of the zero section and its normal bundle).
\item \textbf{Disproof track (construct examples):} Seek an $M$ with a self-homotopy equivalence not homotopic to a diffeomorphism; try to realize it as the map induced on $M$ by restricting $\Phi$ to the zero section and projecting to $M$. Such exotic behavior might allow a diffeomorphism of total spaces without bundle triviality.
\end{itemize}

\subsection*{4) Work: necessary conditions}
Let $s:M\hookrightarrow TM$ be the zero section. If $\Phi:TM\to M\times\mathbb{R}^n$ is a diffeomorphism, define
\[
h := \mathrm{pr}_1\circ \Phi\circ s : M\to M.
\]
\begin{lemma}
$h$ is a homotopy equivalence.
\end{lemma}
\begin{proof}
The zero section $s(M)\subset TM$ is a deformation retract of $TM$, and $M\times\{0\}\subset M\times\mathbb{R}^n$ is a deformation retract of $M\times\mathbb{R}^n$. Hence $\Phi$ induces a homotopy equivalence between these deformation retracts, and $h$ is homotopic to the induced map on retracts, hence is a homotopy equivalence.
\end{proof}

Thus any counterexample must at least produce a self-homotopy equivalence $h$ of $M$ arising this way.

\subsection*{5) Verification / first gap}
A standard strategy would be: show that $\Phi$ forces $h$ to be homotopic to a diffeomorphism (or to have some rigidity property), and then apply a theorem that a diffeomorphism of total spaces covering such an $h$ implies the bundle is trivial. However, establishing that rigidity for $h$ (or extracting a bundle isomorphism from a non-fiber-preserving $\Phi$) is the key gap.

\subsection*{6) FINAL}
\textbf{UNRESOLVED.}
(i) Strongest fully proved partial result: existence of the induced self-homotopy equivalence $h$.\\
(ii) First gap: show $h$ must be homotopic to a diffeomorphism (or otherwise force bundle triviality), or construct an explicit $M$ where a suitable $h$ can occur and build $\Phi$.\\
(iii) Next moves: analyze the normal bundle of the embedded submanifold $\Phi(s(M))\subset M\times\mathbb{R}^n$, and compare it with $TM$; study end-structure invariants of $TM$ vs $M\times\mathbb{R}^n$.

