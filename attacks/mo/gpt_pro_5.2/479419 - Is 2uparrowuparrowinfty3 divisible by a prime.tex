\section{MO 479419 --- Is $2\uparrow\uparrow\infty+3$ divisible by a prime?}
\label{sec:mo479419}
\noindent\textbf{MathOverflow link:} \href{https://mathoverflow.net/questions/479419}{https://mathoverflow.net/questions/479419}.

\subsection*{1) FORMAL RESTATEMENT}
Define the power-tower sequence
\[
T_1=2,\qquad T_{k+1}=2^{T_k}\quad (k\ge 1).
\]
For a modulus $m\ge 1$, assume (as stated in the MO question) that the limit
\[
T_\infty(m):=\lim_{k\to\infty} T_k\pmod m
\]
exists.
For a prime $p$, write $T_\infty(p)\in\Z/p\Z$ for that limiting residue.

\medskip
\textbf{Question.} Does there exist a prime $p$ such that
\[
T_\infty(p)\equiv -3 \pmod p?
\]
Equivalently: is there a prime divisor $p$ of the profinite integer $T_\infty+3\in\widehat{\Z}$?

\subsection*{2) QUICK LITERATURE/CONTEXT CHECK}
The MO post reports extensive computational checks (first up to $p<3\cdot 10^{10}$, later updated further) without finding a prime $p$ with $T_\infty(p)\equiv -3$.
I did not verify a full proof/disproof in the literature as of writing.

\subsection*{3) ATTACK PLAN}
\textbf{Proof track (existence).}
\begin{itemize}
  \item Analyze the recursion defining $T_\infty(p)$ via the totient/carmichael chain for $p-1$.
\end{itemize}
\textbf{Disproof track (nonexistence).}
\begin{itemize}
  \item Derive necessary congruence conditions for a residue to be $T_\infty(p)$ and show $-3$ fails them for all $p$.
\end{itemize}

\subsection*{4) WORK}
\textbf{A fully proved necessary condition: a high $2$-power condition.}
\begin{proposition}
\label{prop:2power-condition}
Let $p$ be an odd prime and write
\(p-1=2^s m\) with $m$ odd and $s=v_2(p-1)\ge 1$.
Assume the limit $T_\infty(p)$ exists.
Then
\begin{equation}
\label{eq:Tinfty-2s-power}
T_\infty(p)\in (\Fp^\times)^{2^s}\cup\{0\},
\end{equation}
i.e. $T_\infty(p)$ is either $0$ or a $2^s$-th power in $\Fp^\times$.
In particular, if $T_\infty(p)\equiv -3\pmod p$, then $-3$ must be a $2^s$-th power modulo $p$.
\end{proposition}
\begin{proof}
Fix $p$ and write $p-1=2^s m$ with $m$ odd.
For any integer exponent $E$ divisible by $2^s$ we have
\(2^E\equiv (2^{E/2^s})^{2^s}\pmod p\), so $2^E$ is a $2^s$-th power in $\Fp^\times$.

By definition, $T_k=2^{T_{k-1}}$.
Since $T_{k-1}=2^{T_{k-2}}$ for $k\ge 3$, we have
\(v_2(T_{k-1})=T_{k-2}\), which tends to $+\infty$ as $k\to\infty$.
Hence there exists $k_0$ such that $2^s\mid T_{k-1}$ for all $k\ge k_0$.
For such $k$,
\(T_k=2^{T_{k-1}}\) is a $2^s$-th power modulo $p$.

Now assume the limit $T_\infty(p)$ exists. Then there is $k_1$ such that
\(T_k\equiv T_\infty(p)\pmod p\) for all $k\ge k_1$.
For $k\ge \max(k_0,k_1)$, we have $T_k$ a $2^s$-th power modulo $p$ and also $T_k\equiv T_\infty(p)\pmod p$, so $T_\infty(p)$ is a $2^s$-th power in $\Fp$ (or $0$).
This proves \eqref{eq:Tinfty-2s-power}.
The final sentence is immediate.
\end{proof}

\medskip
\textbf{Remark.} Condition \eqref{eq:Tinfty-2s-power} is often strong: the subgroup of $2^s$-th powers has index $2^s$ in $\Fp^\times$.
However, this condition alone does not settle whether $-3$ is excluded for all primes.

\subsection*{5) VERIFICATION}
\begin{itemize}
  \item The proof uses only: divisibility $2^s\mid T_{k-1}$ for large $k$ (true because $v_2(T_{k-1})=T_{k-2}\to\infty$) and the assumption that $T_\infty(p)$ is the eventual stable value of $T_k\bmod p$.
  \item Quantifiers: the necessary condition is for each fixed odd prime $p$.
\end{itemize}

\subsection*{6) FINAL}
\textbf{UNRESOLVED}.
\begin{itemize}
  \item (i) Strongest proved partial result: Proposition~\ref{prop:2power-condition} (a necessary $2^{v_2(p-1)}$-power condition).
  \item (ii) First gap: show that $-3$ fails this (or a stronger) necessary condition for every prime $p$, or else exhibit a prime where it passes and then verify the full tetration limit equals $-3$.
  \item (iii) Next moves: (1) refine necessary conditions using the full totient/carmichael chain for $p-1$; (2) analyze the dynamical fixed point equation $x\equiv 2^x\pmod p$ together with exponent reduction constraints; (3) combine subgroup conditions (power subgroup, subgroup generated by $2$, etc.).
  \item (iv) A minimal counterexample would be the smallest prime $p$ for which the totient-chain recursion stabilizes at an exponent giving $2^e\equiv -3\pmod p$.
\end{itemize}

% -----------------------------------------------------------------------------
