\section{MO 392283 --- \texorpdfstring{Cubic function $\Z^2\to\Z$ cannot be injective}{Cubic function Z^2 -> Z cannot be injective}}
\label{sec:mo392283}
\noindent\textbf{MathOverflow link:} \href{https://mathoverflow.net/questions/392283}{https://mathoverflow.net/questions/392283}.

\subsection*{1) FORMAL RESTATEMENT}
\textbf{Literal statement (as a yes/no claim).} Every ``cubic function'' $f\colon\Z^2\to\Z$ fails to be injective.

\medskip
\textbf{Ambiguity / misstatement.} The phrase ``cubic function'' is ambiguous. The standard interpretation in number theory is:
\begin{itemize}
  \item $f(x,y)\in\Z[x,y]$ is a polynomial of total degree $\le 3$.
\end{itemize}

\medskip
\textbf{Minimal corrected statement (standard convention).}
\begin{quote}
\emph{Conjecture.} For every polynomial $f\in\Z[x,y]$ of total degree $\le 3$, the map
\[ f\colon\Z^2\to\Z,\qquad (m,n)\mapsto f(m,n) \]
is not injective.
\end{quote}

\medskip
\textbf{Stress points.}
\begin{itemize}
  \item Coefficient ring ($\Z$ vs $\Q$) and domain ($\Z^2$ vs $\N^2$) matter.
  \item The difficult case is genuinely \emph{inhomogeneous}: lower-order terms may destroy simple symmetries.
\end{itemize}

\subsection*{2) QUICK LITERATURE/CONTEXT CHECK}
This question is in the general circle of ``polynomial encodings'' of lattices and Diophantine geometry on level sets
\(f(x,y)=f(x',y')\).
I did not verify any definitive resolution posted on MO as of writing.

\subsection*{3) ATTACK PLAN}
\textbf{Proof track.}
\begin{itemize}
  \item Analyze the leading homogeneous cubic part $f_3$; try to force a collision from the projective geometry of the cubic.
  \item View injectivity as a Diophantine constraint on the surface $f(x,y)-f(x',y')=0$.
\end{itemize}
\textbf{Disproof track.}
\begin{itemize}
  \item Attempt to construct an injective cubic by ``encoding digits'' in one variable and correcting with lower-order terms.
\end{itemize}

\subsection*{4) WORK}
\textbf{PHASE 1 (tiny cases).}
\begin{itemize}
  \item Degree $0$ (constant): not injective.
  \item Degree $1$: never injective $\Z^2\to\Z$ (rank drop).
\end{itemize}

\medskip
\textbf{Fully proved partial result: every nonzero homogeneous cubic is non-injective on $\Z^2$.}

\begin{lemma}[Odd homogeneous + nontrivial zero forces non-injectivity]
Let $F\in\Z[x,y]$ be homogeneous of odd degree. If there exists $(m,n)\in\Z^2\setminus\{(0,0)\}$ with $F(m,n)=0$, then $F\colon\Z^2\to\Z$ is not injective.
\end{lemma}
\begin{proof}
Homogeneity of odd degree gives
\(F(-m,-n)=(-1)^{\deg F}F(m,n)=-F(m,n)=0\).
Since $(m,n)\neq(-m,-n)$ for a nonzero integer pair, two distinct inputs map to $0$.
\end{proof}

\medskip
\begin{lemma}[Hessian vanishing characterizes cubes]
Let
\[F(x,y)=ax^3+bx^2y+cxy^2+dy^3\in\Q[x,y]\]
be a binary cubic form (not all coefficients zero).
Let $\mathrm{Hess}(F)$ denote the Hessian determinant
\[
\mathrm{Hess}(F):=F_{xx}F_{yy}-F_{xy}^2\in\Q[x,y].
\]
Then $\mathrm{Hess}(F)\equiv 0$ as a polynomial if and only if $F$ is the cube of a linear form over $\Q$, i.e.
\(F=\lambda(ux+vy)^3\) for some $\lambda,u,v\in\Q$.
\end{lemma}
\begin{proof}
Write out the second derivatives:
\[
F_{xx}=6ax+2by,\quad F_{xy}=2bx+2cy,\quad F_{yy}=2cx+6dy.
\]
Hence
\begin{align*}
\mathrm{Hess}(F)
&=(6ax+2by)(2cx+6dy)-(2bx+2cy)^2\\
&=4\bigl((3ac-b^2)x^2+(9ad-bc)xy+(3bd-c^2)y^2\bigr).
\end{align*}
Thus $\mathrm{Hess}(F)\equiv 0$ if and only if
\begin{equation}
\label{eq:hess-relations}
3ac=b^2,\qquad 9ad=bc,\qquad 3bd=c^2.
\end{equation}
Assume \eqref{eq:hess-relations} holds.
If $a=0$, then $b^2=3ac=0$ so $b=0$; then $c^2=3bd=0$ so $c=0$; hence $F=dy^3=d(0\cdot x+1\cdot y)^3$ is a cube.
Similarly if $d=0$ then $F=ax^3$ is a cube.
Now assume $a\neq 0$ and $d\neq 0$.
From $3ac=b^2$ we have $c=b^2/(3a)$.
Plugging into $3bd=c^2$ gives
\[
3bd=\frac{b^4}{9a^2}\quad\Longrightarrow\quad 27a^2bd=b^4.
\]
If $b=0$ then $c=0$ and $9ad=bc=0$ forces $ad=0$, contradiction. So $b\neq 0$.
Define
\(u:=a\) and \(v:=b/3\).
Then using $c=b^2/(3a)$ and $9ad=bc$ one checks
\[
F(x,y)=a\left(x+\frac{b}{3a}y\right)^3
=\frac{1}{a^2}(ax+\tfrac{b}{3}y)^3
=\frac{1}{u^2}(ux+vy)^3,
\]
so $F$ is a cube of a linear form.

Conversely, if $F=\lambda(ux+vy)^3$ then $F$ has rank $1$ as a cubic and a direct computation of second derivatives shows $\mathrm{Hess}(F)\equiv 0$.
\end{proof}

\medskip
\begin{theorem}[Homogeneous cubics are never injective on $\Z^2$]
\label{thm:homog-cubic-noninj}
Let $F\in\Z[x,y]$ be a nonzero homogeneous polynomial of total degree $3$.
Then the map $F\colon\Z^2\to\Z$ is not injective.
\end{theorem}
\begin{proof}
If $F(m,n)=0$ for some nonzero $(m,n)\in\Z^2$, we are done by the previous lemma.
So assume from now on that
\begin{equation}
\label{eq:no-nontrivial-zero}
F(m,n)=0\ \Rightarrow\ (m,n)=(0,0).
\end{equation}

\smallskip
\emph{Step 1: reduce away the ``cube'' case.}
If $F$ is the cube of a linear form in $\Z[x,y]$, say $F=L^3$ with $L\in\Z[x,y]$ linear and nonzero, then $L\colon\Z^2\to\Z$ is not injective (its kernel is an infinite rank-$1$ subgroup of $\Z^2$), and therefore $F=L^3$ is not injective either.
So we may assume $F$ is \emph{not} a cube of a linear form.
By the Hessian lemma, its Hessian polynomial
\(H:=\mathrm{Hess}(F)\in\Z[x,y]\)
(not identically zero) is a nonzero homogeneous quadratic.

\smallskip
\emph{Step 2: choose an integer point where both $F$ and $H$ are nonzero.}
Consider the univariate polynomials $F(t,1)$ (degree $3$) and $H(t,1)$ (degree $2$) in $\Q[t]$.
Each has only finitely many rational roots, hence there exists $t\in\Q$ such that
\(F(t,1)\neq 0\) and \(H(t,1)\neq 0\).
Write $t=u/v$ in lowest terms with $u,v\in\Z$ and $v\neq 0$.
By homogeneity,
\(F(u,v)=v^3F(u/v,1)\neq 0\) and \(H(u,v)=v^2H(u/v,1)\neq 0\).
Fix such an integer pair $(u,v)$ and set
\(T:=F(u,v)\in\Z\setminus\{0\}\).

\smallskip
\emph{Step 3: a tangent-line construction producing a second rational point on the same level set.}
Let $F_x,F_y$ denote the first partial derivatives.
At $(u,v)$ define
\[
A:=F_x(u,v),\qquad B:=F_y(u,v).
\]
Because $F$ is homogeneous of degree $3$, Euler's identity gives
\(uA+vB=3F(u,v)=3T\neq 0\), so $(A,B)\neq(0,0)$.
Define the direction vector
\(d:=(B,-A)\in\Z^2\setminus\{(0,0)\}\).
Consider the polynomial in one variable
\[
P(\lambda):=F\bigl(u+\lambda B,\ v-\lambda A\bigr)\in\Z[\lambda].
\]
Since $F$ is cubic homogeneous, $P$ has degree at most $3$ and satisfies $P(0)=T$.
Differentiate $P$ using the chain rule:
\[
P'(\lambda)=F_x(u+\lambda B,v-\lambda A)\cdot B+F_y(u+\lambda B,v-\lambda A)\cdot(-A).
\]
In particular,
\(P'(0)=A\cdot B+B\cdot(-A)=0\).
Thus $\lambda=0$ is a root of multiplicity at least $2$ of $P(\lambda)-T$.
Write
\begin{equation}
\label{eq:P-expansion}
P(\lambda)=T+c_2\lambda^2+c_3\lambda^3
\end{equation}
for some integers $c_2,c_3$.

\smallskip
\emph{Step 4: compute $c_2$ and show it is nonzero.}
A standard Taylor expansion for a polynomial (terminating at degree $3$) gives
\(c_2=\tfrac12 d^\top (\mathrm{Hess\,matrix\ of\ }F)(u,v)\, d\).
For a binary cubic one can compute this coefficient explicitly in terms of the Hessian determinant:
\begin{equation}
\label{eq:c2-identity}
 c_2\;=\;3\,T\,Q(u,v)\;=\;\frac{3}{4}\,T\,H(u,v),
\end{equation}
where $H=\mathrm{Hess}(F)=4Q$ as in the previous lemma.
(Identity \eqref{eq:c2-identity} is obtained by expanding $P(\lambda)$ using the explicit formulas for $F_x,F_y,F_{xx},F_{xy},F_{yy}$; equivalently, it is the polynomial identity
\(d^\top \mathrm{Hess}(F)\, d =6F\cdot Q\),
which can be verified by collecting the coefficients of the six monomials $x^5,x^4y,\dots,y^5$.)
Since $T\neq 0$ and $H(u,v)\neq 0$, we get $c_2\neq 0$.

\smallskip
\emph{Step 5: show $c_3\neq 0$ under the standing assumption \eqref{eq:no-nontrivial-zero}.}
By homogeneity, the coefficient of $\lambda^3$ in $P(\lambda)$ is
\begin{equation}
\label{eq:c3-homog}
 c_3 = F(B,-A).
\end{equation}
If $c_3=0$, then $(B,-A)\neq (0,0)$ is a nontrivial integer zero of $F$, contradicting \eqref{eq:no-nontrivial-zero}.
Hence $c_3\neq 0$.

\smallskip
\emph{Step 6: the third intersection gives a rational collision, hence an integer collision.}
From \eqref{eq:P-expansion},
\(P(\lambda)-T=\lambda^2(c_2+c_3\lambda)\).
Because $c_2\neq 0$ and $c_3\neq 0$, the value
\(\lambda_0:=-c_2/c_3\in\Q\setminus\{0\}\)
is a distinct root, so
\[F(u+\lambda_0 B,\ v-\lambda_0 A)=P(\lambda_0)=T=F(u,v).
\]
Let $m\in\Z_{>0}$ clear denominators of $\lambda_0$.
Set
\((u',v'):=\bigl(m(u+\lambda_0 B),\ m(v-\lambda_0 A)\bigr)\in\Z^2\)
and $(u_1,v_1):=(mu,mv)\in\Z^2$.
By homogeneity of degree $3$,
\[F(u',v')=m^3F(u+\lambda_0 B,v-\lambda_0 A)=m^3T=F(mu,mv)=F(u_1,v_1).\]
Since $\lambda_0\neq 0$ and $(A,B)\neq (0,0)$, we have $(u',v')\neq(u_1,v_1)$.
Thus $F$ is not injective on $\Z^2$.
\end{proof}

\subsection*{5) VERIFICATION}
\begin{itemize}
  \item Quantifiers: Theorem~\ref{thm:homog-cubic-noninj} is unconditional for every nonzero homogeneous $F\in\Z[x,y]$.
  \item Edge cases: If $F$ has a nontrivial integer zero, the proof stops early with an explicit collision.
  \item The only ``mechanical'' step is the coefficient identity \eqref{eq:c2-identity}; it is a finite polynomial identity in $a,b,c,d,u,v$ and can be checked by explicit expansion.
\end{itemize}

\subsection*{6) FINAL}
\textbf{UNRESOLVED} for general (inhomogeneous) cubics.
\begin{itemize}
  \item (i) Strongest proved partial result: every nonzero homogeneous cubic $F\in\Z[x,y]$ is non-injective on $\Z^2$ (Theorem~\ref{thm:homog-cubic-noninj}).
  \item (ii) First gap: extend from the homogeneous leading term to an arbitrary cubic polynomial.
  \item (iii) Next moves: (1) treat the surface $f(x,y)=f(x',y')$ and try to force non-diagonal integer points; (2) explore congruence obstructions forcing collisions; (3) search for known results on polynomial injections $\Z^2\to\Z$ of low degree.
  \item (iv) A minimal counterexample (if an injective cubic exists) would likely have very special lower-order terms designed to break the tangent-chord type constructions.
\end{itemize}
