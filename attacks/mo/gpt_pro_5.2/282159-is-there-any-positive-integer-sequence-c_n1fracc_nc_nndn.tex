\section*{Problem 282159: Integer solutions of $c_{n+1}=\frac{c_n(c_n+n+d)}n$}
\addcontentsline{toc}{section}{Problem 282159: Integer solutions of c_{n+1}=c_n(c_n+n+d)/n}

\subsection*{1) FORMAL RESTATEMENT}
Question: do there exist an integer $d$ and a strictly increasing sequence of positive integers $(c_n)_{n\ge 1}$ such that
\[
c_{n+1}=\frac{c_n(c_n+n+d)}{n}\quad\text{for all integers }n\ge 1?
\]
Implicit constraints: each right-hand side must be a positive integer.

\subsection*{2) QUICK LITERATURE/CONTEXT CHECK}
The MathOverflow post asks for existence; it does not contain an accepted solution.

\subsection*{3) ATTACK PLAN}
\emph{Proof-track ideas:}
(1) Derive strong congruence/valuation constraints from integrality for all $n$ and show eventual contradiction.
(2) Reinterpret recurrence as $p$-adic dynamical system and show no global integer orbit exists.

\emph{Disproof-track ideas (construction):}
(1) Try to build $(c_n)$ by forcing $n\mid c_n(c_n+d)$ inductively using CRT-like choices (but recurrence removes freedom).
(2) Search computationally for parameter pairs $(c_1,d)$ producing long integer segments to guess structure.

\subsection*{4) WORK (partial results)}

\begin{lemma}[Integrality condition simplification]
For each $n\ge 1$,
\[
c_{n+1}\in\mathbb{Z}\quad\Longleftrightarrow\quad n\mid c_n(c_n+d).
\]
\end{lemma}

\begin{proof}
We have $c_{n+1}=\frac{c_n(c_n+n+d)}n$.
Since $n\mid n\,c_n$, the divisibility $n\mid c_n(c_n+n+d)$ is equivalent to $n\mid c_n(c_n+d)$ by subtracting $n\,c_n$ from the numerator.
\end{proof}

\begin{lemma}[Shifted recurrence]
Define $C_n:=c_n+d$.
Whenever $c_{n+1}$ is defined,
\[
C_{n+1}=\frac{(c_n+n)C_n}{n}=\frac{(C_n-d+n)C_n}{n}.
\]
Equivalently,
\[
c_{n+1}+d=\frac{(c_n+n)(c_n+d)}{n}.
\]
\end{lemma}

\begin{proof}
Compute:
\[
c_{n+1}+d=\frac{c_n(c_n+n+d)}n + d
=\frac{c_n(c_n+n+d)+dn}{n}
=\frac{c_n^2+nc_n+dc_n+dn}{n}
=\frac{(c_n+n)(c_n+d)}{n}.
\]
\end{proof}

\paragraph{Computational sanity checks (small search).}
A brute-force search over parameters $d\in[-20,20]$ and $c_1\in[1,200]$ finds that all integer sequences produced by the recurrence eventually fail integrality; the longest integer initial segments in this box have length $17$ (i.e.\ $c_1,\dots,c_{17}$ defined, but $c_{18}$ fails integrality).
Examples achieving length $17$ include $(c_1,d)=(1,1)$ and $(2,-1)$, which begin
\[
(1,3,9,39,429,37323,\dots)\quad\text{and}\quad(2,4,10,40,430,37324,\dots).
\]
For $(c_1,d)=(1,1)$ the first failure occurs at $n=17$.

\subsection*{5) VERIFICATION}
The lemmas are purely algebraic and correct.
The computation is only evidence and does not prove nonexistence of infinite sequences; it is included as Phase~1 falsification attempt.

\subsection*{6) FINAL: \textbf{UNRESOLVED}}
(i) Strongest proved partial result: Lemma 1 (integrality $\Leftrightarrow n\mid c_n(c_n+d)$) and Lemma 2 (shifted form).\\
(ii) First gap: prove that for every integer $d$ and $c_1\ge 1$ the divisibility $n\mid c_n(c_n+d)$ must fail for some $n$ (or else construct explicit $(d,c_1)$ giving success for all $n$).\\
(iii) Next moves:
\begin{itemize}
\item Develop $p$-adic valuation recurrences for $v_p(c_n)$ and $v_p(c_n+d)$ along multiples of $p^k$; attempt to force a contradiction for some prime $p\nmid d$.
\item Analyze the recurrence modulo $n+1$ to obtain polynomial congruence constraints linking $c_n\bmod (n+1)$ and integrality at step $n+1$.
\item Search for invariants or monotone measures (e.g.\ growth of $\gcd(c_n,n)$ or of $\gcd(c_n,c_n+d)$) that cannot satisfy all $n$ simultaneously.
\end{itemize}
(iv) What a minimal counterexample would look like (if the claim ``no such sequence'' were false): a specific integer $d$ and $c_1$ producing an orbit where for every $n$, all prime powers dividing $n$ are absorbed into exactly one of $c_n$ or $c_n+d$ without forcing incompatible congruences at later steps.

