\section{(MO 156621) When are two $C^\ast$-algebras isomorphic as Banach spaces?}

\subsection{1) FORMAL RESTATEMENT}
Fix conventions:
\begin{itemize}
\item A Banach space isomorphism means a bounded linear bijection with bounded linear inverse.
\item A $C^\ast$-algebra is \emph{simple} if it has no nontrivial closed two-sided ideals.
\item A $C^\ast$-algebra is \emph{separable} if it is separable as a topological space in the norm topology.
\end{itemize}

Claim to test:
\begin{quote}
For all simple, separable, infinite-dimensional $C^\ast$-algebras $A,B$, there exists a Banach space isomorphism $T\colon A\to B$.
\end{quote}

\subsection{2) QUICK LITERATURE/CONTEXT CHECK}
A known operator-space result (Kirchberg) says that many non-type-I nuclear separable $C^\ast$-algebras are completely isomorphic to the CAR algebra, but there is also a sharp separation between type~I and non-type~I algebras at the Banach-space level. In particular, results of Haagerup--Rosenthal--Sukochev (as reported by Rosenthal) imply that if $B$ is type~I and $A$ is non-type~I, then $A^\ast$ cannot embed (up to Banach isomorphism) into $B^\ast$, which prevents $A$ and $B$ from being Banach-space isomorphic.

\subsection{3) ATTACK PLAN}
\paragraph{Disproof strategy.} Find two simple separable infinite-dimensional $C^\ast$-algebras, one type~I and one non-type~I, and invoke the above Banach-space invariant to show they cannot be Banach-space isomorphic.

\subsection{4) WORK (explicit counterexample)}

\subsubsection*{Step 1: choose explicit algebras}
Let
\[
A := \mathrm{CAR} \cong \bigotimes_{n=1}^\infty M_2(\mathbb C),
\]
the CAR (UHF) algebra of type $2^\infty$, and let
\[
B := \mathcal K(\ell^2),
\]
the $C^\ast$-algebra of compact operators on a separable infinite-dimensional Hilbert space.

Facts:
\begin{itemize}
\item $A$ is simple, separable, infinite-dimensional, and \emph{non-type~I} (indeed it is a Glimm algebra).
\item $B$ is simple, separable, infinite-dimensional, and \emph{type~I}.
\end{itemize}

\subsubsection*{Step 2: a Banach-space obstruction}
We use the following theorem (stated in Rosenthal's survey, based on results of Haagerup--Rosenthal--Sukochev):

\begin{quote}
(\textbf{Theorem}) If $A$ is a non-type~I $C^\ast$-algebra and $B$ is a type~I $C^\ast$-algebra, then $A^\ast$ is not Banach-space isomorphic to a subspace of $B^\ast$.
\end{quote}

\subsubsection*{Step 3: conclude $A\not\cong B$ as Banach spaces}
Assume for contradiction that there exists a Banach space isomorphism $T\colon A\to B$. Then the adjoint map
\[
T^\ast \colon B^\ast \to A^\ast
\]
is also a Banach space isomorphism. In particular, $A^\ast$ is Banach-space isomorphic to the image of $T^\ast$, which is all of $A^\ast$, hence $A^\ast$ is Banach-space isomorphic to a subspace of $B^\ast$.

This contradicts the theorem above. Therefore, no Banach space isomorphism $A\to B$ exists.

\subsection{5) VERIFICATION}
\begin{itemize}
\item The only substantive input is the quoted theorem separating type~I and non-type~I at the dual Banach space level.
\item The dual-space argument is correct: a bounded linear isomorphism has a bounded linear inverse, and taking adjoints preserves bounded invertibility.
\item Both algebras are indeed simple, separable, and infinite-dimensional.
\end{itemize}

\subsection{6) FINAL}
\textbf{COUNTEREXAMPLE/DISPROOF.}

\paragraph{Conclusion.}
The statement ``all simple, separable, infinite-dimensional $C^\ast$-algebras are isomorphic as Banach spaces'' is false. A concrete counterexample is given by the CAR algebra $A$ and the compact operators $B=\mathcal K(\ell^2)$, which cannot be Banach-space isomorphic because $A$ is non-type~I while $B$ is type~I.

