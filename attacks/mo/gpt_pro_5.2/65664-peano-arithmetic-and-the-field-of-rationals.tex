\section{MO 65664: Peano arithmetic and the field of rationals}

\subsection*{1) FORMAL RESTATEMENT}
\paragraph{Ambiguity/misstatement.}
The statement ``$(M,+,\times)$ is a model of PA'' is ambiguous because PA is normally formulated in a language containing at least $0,1,+,\times$ and induction; sometimes also $<$.
Also, for a semiring like $(\mathbb{N},+,\times)$ the phrase ``field of rationals'' can mean either:
\begin{itemize}[leftmargin=2em]
\item the fraction field of the associated \emph{integer ring} $Z^M$ (Grothendieck group of the additive monoid), or
\item the semifield of fractions of $M$ itself (pairs $(a,b)$ with $b\neq 0$, modulo $ad=bc$), which is not a field unless negatives are present.
\end{itemize}
In model theory, ``bi-interpretability'' requires specifying the languages and the interpretations.

\paragraph{Minimal corrected statement (standard).}
Let $M\models \mathrm{PA}$ in the language of arithmetic $\mathcal{L}_A=\{0,1,+,\times\}$ (equivalently with $<$; the $<$ is definable from $+$ and $\times$ in PA contexts). Let $Z^M$ be the ring of integers of $M$ (the definable extension of $M$ by adding additive inverses), and let
\[
\mathbb{Q}^M := \mathrm{Frac}(Z^M)
\]
be its fraction field, viewed as an $\mathcal{L}_{\mathrm{ring}}=\{0,1,+,\times\}$-structure.

\begin{quote}
\textbf{Theorem B (precise form).} For every model $M\models \mathrm{PA}$, the structures $M$ and $\mathbb{Q}^M$ are bi-interpretable (uniformly in $M$).
\end{quote}

\paragraph{Target strengthening (as asked).}
Replace ``$M\models\mathrm{PA}$'' by ``$M\models \mathrm{EFA}$'', where $\mathrm{EFA}$ is a standard weak arithmetic such as $I\Delta_0+\exp$ (bounded induction plus total exponentiation).

\subsection*{2) QUICK LITERATURE/CONTEXT CHECK}
The MO thread has no posted solution, but comments suggest the strengthening may be ``folklore'' and that Julia Robinson's definability of $\mathbb{Z}$ in $\mathbb{Q}$ should be the key ingredient.\footnote{MO page \url{https://mathoverflow.net/questions/65664/peano-arithmetic-and-the-field-of-rationals}.}
Separately, there are later developments on definability of $\mathbb{Z}$ in $\mathbb{Q}$ (e.g.\ universal definitions), but these do not automatically settle the bi-interpretability for nonstandard fraction fields.
For background:
\begin{itemize}[leftmargin=2em]
\item Julia Robinson proved $\mathbb{Z}$ is definable in $\mathbb{Q}$ in the first-order language of rings (classic result; not reproduced here).
\item Koenigsmann later gave a universal first-order definition of $\mathbb{Z}$ in $\mathbb{Q}$.\footnote{See summary entry: \url{https://zbmath.org/?q=an:06049998} and the associated Annals reference; also see discussions in later papers.}
\end{itemize}
I did not locate (in the limited search performed) a clean published reference that explicitly states and proves ``Theorem B holds already for EFA''.

\subsection*{3) ATTACK PLAN}
\paragraph{Problem type.}
Model theory of arithmetic; interpretability/bi-interpretability; definability of integer parts in fraction fields.

\paragraph{Likely tools.}
\begin{itemize}[leftmargin=2em]
\item \textbf{Interpretations via equivalence classes of pairs:} to interpret fraction fields in rings/semirings.
\item \textbf{Uniform definability of $\mathbb{Z}$ in $\mathbb{Q}$:} Julia Robinson/Koenigsmann-type formulas.
\item \textbf{Model-theoretic transfer:} if $\mathbb{Q}^M$ is (or is close to) an elementary extension of $\mathbb{Q}$, definability transfers.
\item \textbf{Cut/initial segment constructions:} potential counterexample route (different subrings with same fraction field).
\item \textbf{Arithmetic strength analysis:} what induction is needed to prove the number theory inside the definability argument.
\end{itemize}

\paragraph{Proof-track idea.}
Show that for any $M\models I\Delta_0+\exp$, the same first-order formula $\varphi(x)$ that defines $\mathbb{Z}$ inside $\mathbb{Q}$ defines the embedded copy of $Z^M$ inside $\mathbb{Q}^M$. Then combine with the obvious interpretation of $\mathbb{Q}^M$ in $M$ to get bi-interpretability.

\paragraph{Disproof-track idea.}
Construct two non-isomorphic models $M_1,M_2\models \mathrm{EFA}$ such that $\mathbb{Q}^{M_1}\cong \mathbb{Q}^{M_2}$ as fields. This would show $\mathbb{Q}^M$ does \emph{not} determine $M$ up to interpretation, so bi-interpretability fails.

\paragraph{Best path chosen.}
Prove the easy half (interpretation of $\mathbb{Q}^M$ in $M$) fully; isolate the exact hard point (defining $M$ in $\mathbb{Q}^M$ uniformly) and report the gap.

\subsection*{4) WORK}
\subsubsection*{Phase 1: tiny cases}
For the standard model $M=\mathbb{N}$, Theorem A asserts $\mathbb{Q}$ is bi-interpretable with $\mathbb{N}$; this is classical (using Julia Robinson's definability of $\mathbb{Z}$ in $\mathbb{Q}$ plus standard fraction-field interpretation).

\subsubsection*{Lemma 1 (Fraction-field interpretation is uniform and needs little arithmetic)}
\begin{quote}
\textbf{Claim.} Let $R$ be any commutative integral domain in the language of rings. Then its fraction field $\mathrm{Frac}(R)$ is interpretable in $R$ by the standard ``pair of elements modulo cross-multiplication'' construction.
\end{quote}

\paragraph{Proof.}
Define the interpreted universe as
\[
U := \{(a,b)\in R^2 : b\neq 0\}.
\]
Define an equivalence relation $\sim$ on $U$ by
\[
(a,b)\sim(c,d)\quad \Longleftrightarrow\quad ad=bc.
\]
Since $R$ is an integral domain, $\sim$ is an equivalence relation:
\begin{itemize}[leftmargin=2em]
\item Reflexive: $ab=ba$.
\item Symmetric: $ad=bc \Rightarrow cb=da$.
\item Transitive: if $ad=bc$ and $cf=de$ with $b,d,f\neq 0$, then multiply the first by $f$ and the second by $b$ to obtain $adf=bcf$ and $bcf=bde$, hence $adf=bde$. Since $d\neq 0$ and $R$ is a domain, cancel $d$ to get $af=be$, so $(a,b)\sim(e,f)$.
\end{itemize}
(Here ``cancel $d$'' means: from $d(af-be)=0$ infer $af-be=0$ using that $R$ has no zero divisors.)

Define addition and multiplication on equivalence classes by:
\[
[(a,b)] + [(c,d)] := [(ad+bc, bd)],\qquad
[(a,b)] \cdot [(c,d)] := [(ac, bd)].
\]
These operations are well-defined on equivalence classes by a routine verification using $ad=bc$ relations and the fact that multiplication distributes over addition. The class of $(0,1)$ is additive identity and class of $(1,1)$ is multiplicative identity. The inverse of $[(a,b)]$ (for $a\neq 0$) is $[(b,a)]$. Hence the quotient structure is a field, and it is precisely the fraction field of $R$.
All relations defining $U$, $\sim$, $+$, $\cdot$, $0$, $1$ are first-order definable in $R$.
\hfill$\square$

\paragraph{Application to the problem.}
Taking $R=Z^M$ for a model $M$ of arithmetic (PA or EFA), we obtain a uniform interpretation of $\mathbb{Q}^M=\mathrm{Frac}(Z^M)$ inside $M$ (since $Z^M$ is definable in $M$).

\subsubsection*{First hard point: interpreting $M$ in $\mathbb{Q}^M$}
To get bi-interpretability, one needs a uniform first-order definition (in the language of fields) of the embedded copy of $Z^M$ (or $M$) inside $\mathbb{Q}^M$. In the standard case $\mathbb{Q}$, this is provided by Julia Robinson's definition of $\mathbb{Z}$ in $\mathbb{Q}$. The strengthening to arbitrary (especially weak) nonstandard $M$ requires proving that the same defining formula works in $\mathbb{Q}^M$.

I do not supply a complete proof of this definability-transfer here; it is the precise missing step.

\subsection*{5) VERIFICATION}
\begin{itemize}[leftmargin=2em]
\item Lemma 1 is fully checked and requires only that the ring be an integral domain (which PA and EFA prove in every model).
\item The gap is not cosmetic: without a uniform definition of $Z^M$ inside $\mathbb{Q}^M$, the fraction field does not determine the subring uniquely, and bi-interpretability can fail in principle.
\end{itemize}

\subsection*{6) FINAL}
\begin{center}
\textbf{UNRESOLVED}
\end{center}

\paragraph{(i) Strongest fully proved partial result here.}
For any integral domain $R$ (hence for $Z^M$ in any model $M$ of PA/EFA), the fraction field $\mathrm{Frac}(R)$ is uniformly interpretable in $R$.

\paragraph{(ii) Exact first gap.}
Show that there exists a uniform first-order formula $\varphi(x)$ in the language of fields such that for every $M\models \mathrm{EFA}$ (or PA), $\varphi(\mathbb{Q}^M)$ is exactly the embedded copy of $Z^M$ (or of $M$) inside $\mathbb{Q}^M$.

\paragraph{(iii) Top 3 next moves.}
(1) Take an explicit Robinson/Koenigsmann formula for $\mathbb{Z}\subset\mathbb{Q}$ and check whether its correctness argument relativizes to $\mathbb{Q}^M$; (2) try to prove $\mathbb{Q}^M \equiv \mathbb{Q}$ for $M\models\mathrm{PA}$ or for $M\models\mathrm{EFA}$ (unlikely, but would imply transfer); (3) attempt to build two EFA-models with isomorphic fraction fields as a counterexample.

\paragraph{(iv) What a minimal counterexample would look like.}
Two non-isomorphic models $M_1,M_2\models \mathrm{EFA}$ with $\mathbb{Q}^{M_1}\cong \mathbb{Q}^{M_2}$ (as fields), or a specific $M\models \mathrm{EFA}$ such that every plausible candidate formula $\varphi$ fails to isolate $Z^M$ inside $\mathbb{Q}^M$.
