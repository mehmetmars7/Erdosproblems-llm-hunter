\section{MO 506537 --- How does an automorphism of a free module scale Haar measure?}
\label{sec:mo506537}
\noindent\textbf{MathOverflow link:} \href{https://mathoverflow.net/questions/506537}{https://mathoverflow.net/questions/506537}.

\subsection*{1) FORMAL RESTATEMENT}
Let $R$ be a second-countable locally compact Hausdorff commutative topological ring.
Let $V$ be a free $R$-module of rank $n<\infty$, endowed with the product topology via some $R$-module isomorphism $V\cong R^n$.
Let $\mu_V$ be a (left) Haar measure on the additive locally compact group $(V,+)$.

For $\alpha\in\mathrm{Aut}_R(V)\cong GL_n(R)$, define its \emph{modulus} $\Delta_V(\alpha)\in\R_{>0}$ by
\[\alpha_*\mu_V = \Delta_V(\alpha)\,\mu_V.
\]
Similarly, for $u\in R^\times$, let $m_u\colon R\to R$ be multiplication by $u$ and define
\(|u|_R:=\Delta_R(m_u)\in\R_{>0}\).

\medskip
\textbf{Target statement.} For all $\alpha\in GL_n(R)$,
\[\Delta_V(\alpha)=|\det(\alpha)|_R.
\]

\subsection*{2) QUICK LITERATURE/CONTEXT CHECK}
For $R$ a local field (or $\R,\C$) this is the standard change-of-variables/Jacobian determinant law.
The MO question asks for general locally compact commutative rings, especially rings of ad\`eles.

\subsection*{3) ATTACK PLAN}
\textbf{Proof strategies.}
\begin{itemize}
  \item Prove the identity for matrices generated by elementary and diagonal matrices, and show generation/density.
  \item Reduce to products or restricted products of cases where the result is known (fields, compact rings).
\end{itemize}
\textbf{Disproof strategies.}
\begin{itemize}
  \item Look for a locally compact ring whose $GL_n$ admits a continuous modulus character not factoring through determinant.
\end{itemize}

\subsection*{4) WORK}
\textbf{Basic sanity checks.}
\begin{itemize}
  \item If $R$ is compact, then $(V,+)$ is compact, Haar measure can be normalized to total mass $1$, and every continuous automorphism preserves it; hence $\Delta_V(\alpha)=1$. Also $|u|_R=1$ for all units $u$, so the identity holds.
\end{itemize}

\medskip
\textbf{Lemma: elementary transvections have modulus $1$ (fully proved).}
\begin{lemma}
Let $R$ be as above and identify $V\cong R^n$. Let $E\in GL_n(R)$ be an elementary transvection (add a multiple of one coordinate to another). Then $\Delta_V(E)=1$ and $|\det(E)|_R=1$.
\end{lemma}
\begin{proof}
An elementary transvection has determinant $1$, hence $|\det(E)|_R=|1|_R=1$.
As an additive-group automorphism of $R^n$, $E$ is a shear
\[(x_1,\dots,x_i,\dots,x_j,\dots,x_n)\mapsto (x_1,\dots,x_i+ax_j,\dots,x_j,\dots,x_n)
\]
for some $a\in R$.
For fixed $(x_k)_{k\neq i}$, the map is translation by $ax_j$ in the $i$th coordinate.
Translation-invariance of Haar measure on $R$ and Fubini's theorem imply that the product Haar measure on $R^n$ is invariant under such shears.
Thus $\Delta_V(E)=1$.
\end{proof}

\medskip
\textbf{Theorem: the identity holds for locally compact fields (fully proved).}
\begin{theorem}
\label{thm:field-case}
If $R$ is a locally compact field and $V\cong R^n$, then for every $\alpha\in GL_n(R)$,
\[\Delta_V(\alpha)=|\det(\alpha)|_R.
\]
\end{theorem}
\begin{proof}
Fix a Haar measure $\mu_R$ on $(R,+)$ and take $\mu_V:=\mu_R^{\otimes n}$ under $V\cong R^n$.
The map $u\mapsto |u|_R$ is a continuous homomorphism $R^\times\to\R_{>0}$.

\emph{Step 1: diagonal matrices.}
For $D=\mathrm{diag}(u_1,\dots,u_n)$, product measure gives
\[\Delta_V(D)=\prod_{i=1}^n |u_i|_R = |\det(D)|_R.
\]

\emph{Step 2: elementary matrices.}
By the lemma above, every elementary transvection has modulus $1$ and determinant $1$.

\emph{Step 3: generation over a field.}
Over a field, Gaussian elimination expresses any $A\in GL_n(R)$ as a product of elementary transvections and a diagonal matrix (row operations reduce $A$ to the identity).
Because both $\Delta_V$ and $|\det|_R$ are group homomorphisms $GL_n(R)\to\R_{>0}$ and they agree on these generators, they agree on all of $GL_n(R)$.
\end{proof}

\medskip
\textbf{Theorem: stability under finite products and restricted products (includes ad\`eles).}
\begin{theorem}
\label{thm:restricted-product}
Let $(R_i)_{i\in I}$ be a family of second-countable locally compact commutative rings.
Assume each $R_i$ contains a compact open subring $O_i\subset R_i$.
Let
\[R:=\prod_{i\in I}' R_i\]
be the restricted product with respect to the $O_i$ (so elements lie in $O_i$ for all but finitely many $i$), with its standard locally compact topology.
Let $V\cong R^n$.
Assume that for each $i$ and each $A_i\in GL_n(R_i)$,
\[\Delta_{R_i^n}(A_i)=|\det(A_i)|_{R_i}.
\]
Then for every $A=(A_i)\in GL_n(R)$,
\[\Delta_{R^n}(A)=|\det(A)|_R,\]
where $|\cdot|_R$ is the multiplicative character defined by
\(|u|_R:=\prod_i |u_i|_{R_i}\) (a finite product because $u_i\in O_i^\times$ for almost all $i$).
\end{theorem}
\begin{proof}
Choose Haar measures $\mu_i$ on $(R_i,+)$ such that $\mu_i(O_i)=1$ for all $i$.
Then the restricted product Haar measure $\mu_R$ on $R$ is characterized by the property that on basic compact opens of the form
\(\prod_{i\in S} U_i\times \prod_{i\notin S} O_i\) (finite $S\subset I$), its measure equals $\prod_{i\in S} \mu_i(U_i)$.
Similarly take product measures on $R^n$.

Let $A=(A_i)\in GL_n(R)$.
By definition of the restricted product, we have $A_i\in GL_n(O_i)$ for all but finitely many $i$.
For such $i$, the map $A_i\colon O_i^n\to O_i^n$ is a homeomorphism of a compact open subgroup, hence preserves the normalized Haar measure on $R_i^n$; equivalently $\Delta_{R_i^n}(A_i)=1$ and also $|\det(A_i)|_{R_i}=1$.
Therefore both products
\(\prod_i \Delta_{R_i^n}(A_i)\) and \(\prod_i |\det(A_i)|_{R_i}\)
are finite products.

For a basic compact-open set in $R^n$ of the form
\(U=\prod_{i\in S} U_i\times \prod_{i\notin S} O_i^n\)
(with $S$ finite), the image $A(U)$ equals
\(\prod_{i\in S\cup S_0} A_i(U_i)\times \prod_{i\notin S\cup S_0} O_i^n\)
for some finite set $S_0$ containing all indices where $A_i\notin GL_n(O_i)$.
By the defining property of $\mu_{R^n}$ and multiplicativity of measures on such rectangles,
\[
\mu_{R^n}(A(U))=\prod_{i\in S\cup S_0} \mu_{R_i^n}(A_i(U_i))
=\prod_{i\in S\cup S_0} \Delta_{R_i^n}(A_i)\,\mu_{R_i^n}(U_i).
\]
Since $\mu_{R^n}(U)=\prod_{i\in S\cup S_0} \mu_{R_i^n}(U_i)$, we obtain
\(\mu_{R^n}(A(U))=(\prod_i \Delta_{R_i^n}(A_i))\mu_{R^n}(U)\).
By uniqueness of Haar measure scaling factors, this shows
\[\Delta_{R^n}(A)=\prod_i \Delta_{R_i^n}(A_i).
\]
Applying the assumed local identity gives
\[\Delta_{R^n}(A)=\prod_i |\det(A_i)|_{R_i}=|\det(A)|_R.
\]
\end{proof}

\begin{corollary}[Ad\`eles]
Let $K$ be a global field and $\mathbb{A}_K$ its ring of ad\`eles (a restricted product of local fields with respect to their rings of integers).
Then for all $A\in GL_n(\mathbb{A}_K)$,
\[\Delta_{\mathbb{A}_K^n}(A)=|\det(A)|_{\mathbb{A}_K}
\]
where $|\cdot|_{\mathbb{A}_K}$ is the idelic norm (product of local absolute values).
\end{corollary}
\begin{proof}
Apply Theorem~\ref{thm:restricted-product} with $R_i=K_v$ ranging over all completions of $K$, and use Theorem~\ref{thm:field-case} for each local field.
\end{proof}

\subsection*{5) VERIFICATION}
\begin{itemize}
  \item The field case uses only product Haar measure, invariance under shears, and Gaussian elimination.
  \item The restricted product argument uses only the defining property of restricted product Haar measure on basic rectangles and the fact that almost all local components are integral (hence modulus $1$).
\end{itemize}

\subsection*{6) FINAL}
\textbf{UNRESOLVED} for an arbitrary second-countable locally compact commutative ring $R$.
\begin{itemize}
  \item (i) Strongest proved results here: the identity holds for locally compact fields, compact rings, finite products of such, and for restricted products with compact open subrings (in particular, for ad\`ele rings).
  \item (ii) First gap: decide whether every such ring $R$ can be reduced to these cases, or produce a genuine counterexample.
  \item (iii) Next moves: (1) consult classification/structure theorems for second-countable locally compact commutative rings; (2) analyze continuous homomorphisms $GL_n(R)\to\R_{>0}$ and whether they factor through $\det$; (3) attempt to build a counterexample using nontrivial $K_1$ phenomena.
  \item (iv) A minimal counterexample (if it exists) would be a locally compact ring $R$ where $GL_n(R)$ admits a continuous modulus character not determined by $\det$.
\end{itemize}

