\section{Problem 93601: Borel/non-AC decompositions of $\R^3$ into circles/lines}

\subsection*{1) FORMAL RESTATEMENT}
\paragraph{Literal statement.}
The attachment is an open-ended question about whether, for various partitions of $\R^3$ into geometric pieces
(circles, lines, etc.) obtained using the axiom of choice, one can prove \emph{nonexistence} of a \emph{Borel} version.

As written, it is not a single proposition; it is a bundle of existence/nonexistence questions for several different geometric families.

\paragraph{Minimal corrected statement (one concrete, checkable subproblem).}
I isolate one concrete sub-problem that is decidable directly:

\medskip
\noindent\textbf{Corrected subproblem:} 
\emph{Does there exist a Borel partition of $\R^3$ into Euclidean circles and (possibly) finitely many lines?}

\medskip
\noindent Here a \emph{Borel partition} means: there exists a standard Borel space $Y$ and a Borel map $f:\R^3\to Y$ such that each fiber $f^{-1}(y)$ is one part of the partition.
Equivalently, the induced equivalence relation $E_f=\{(x,x')\in \R^3\times\R^3: f(x)=f(x')\}$ is Borel.

\subsection*{2) QUICK LITERATURE/CONTEXT CHECK (web available)}
The attachment mentions several constructions (J\"onsson--W\"astlund, Szulkin, etc.).
I did not locate a definitive modern reference resolving the strongest ``unit circles'' Borel/non-Borel question during the allotted browsing.

\subsection*{3) ATTACK PLAN}
\begin{itemize}
\item \textbf{Proof track (for the corrected subproblem):} Give an explicit partition of $\R^3$ into circles and a line via cylindrical coordinates and prove the parameter map is Borel.
\item \textbf{Disproof track (for stronger variants, e.g.\ unit circles only):} Seek an obstruction from descriptive set theory/foliation theory; not completed here.
\end{itemize}

\subsection*{4) WORK}
\begin{theorem}[A simple Borel partition of $\R^3$ into circles plus a line]
There is a Borel map $f:\R^3\to (\R_{>0}\times\R)\,\sqcup\,\{\ast\}$ such that:
\begin{itemize}
\item for each $(r,z)\in\R_{>0}\times\R$, the fiber $f^{-1}(r,z)$ is the Euclidean circle
\[
C_{r,z}=\{(r\cos\theta,r\sin\theta,z):\theta\in[0,2\pi)\}\subset \R^3;
\]
\item the fiber $f^{-1}(\ast)$ is the $z$-axis line $L=\{(0,0,z):z\in\R\}$.
\end{itemize}
In particular, $\R^3$ admits a Borel partition into circles and one line.
\end{theorem}

\begin{proof}
Define $f:\R^3\to (\R_{>0}\times\R)\,\sqcup\,\{\ast\}$ by
\[
f(x,y,z)=
\begin{cases}
\bigl(\sqrt{x^2+y^2},\,z\bigr), & \text{if }(x,y)\neq (0,0),\\
\ast, & \text{if }(x,y)=(0,0).
\end{cases}
\]
If $(x,y)\neq(0,0)$ then $r=\sqrt{x^2+y^2}>0$, and $f^{-1}(r,z)$ is exactly the circle at height $z$ and radius $r$ around the $z$-axis.
If $(x,y)=(0,0)$ then $f^{-1}(\ast)=L$.
The fibers are pairwise disjoint and cover $\R^3$.
On $\{(x,y)\neq(0,0)\}$ the map is continuous; on $\{(x,y)=(0,0)\}$ it is constant; hence $f$ is Borel.
\end{proof}

\subsection*{5) VERIFICATION}
\begin{itemize}
\item \textbf{Boundary cases:} The only non-circle fiber is the $z$-axis; handled explicitly.
\item \textbf{Borelness check:} $\{(x,y,z):(x,y)=(0,0)\}$ is closed; the formula is continuous on its complement.
\item \textbf{Adversarial check:} Two different parameter pairs $(r,z)\neq(r',z')$ cannot share a point because $(r,z)$ is uniquely determined by $(x,y,z)$ when $r>0$.
\end{itemize}

\subsection*{6) FINAL}
\textbf{UNRESOLVED.}

\medskip
\noindent\textbf{Fail-safe details.}
\begin{enumerate}[label=(\roman*)]
\item \textbf{Strongest fully proved partial result:} Theorem above (a Borel partition of $\R^3$ into circles and one line).
\item \textbf{First gap:} Existence/nonexistence of a Borel partition of $\R^3$ into \emph{unit} circles (or the other rigid families listed in the attachment).
\item \textbf{Top 3 next moves:} 
(1) Model as a Borel equivalence relation with all classes homeomorphic to $S^1$ and apply foliation/descriptive-set-theory obstructions.
(2) Investigate whether such a relation must be non-smooth/non-hyperfinite, contradicting geometry.
(3) Attempt measurable/Borel uniformization of known choice-based constructions, or prove such uniformization impossible.
\item \textbf{Minimal counterexample shape:} A proof of nonexistence would likely extract, from any candidate Borel unit-circle partition, a contradiction with a known Borel selection theorem or with a rigidity invariant of Euclidean circle geometry.
\end{enumerate}


