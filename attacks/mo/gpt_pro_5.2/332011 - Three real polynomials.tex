\section{MO \#332011: Three real polynomials}
\MO{332011}{Three real polynomials}

\subsection*{1) FORMAL RESTATEMENT}
Let $f,g,h\in\mathbb R[x]$ and define the $3\times 3$ Wronskian polynomial
\[
W(x):=\det\begin{pmatrix}
 f(x) & g(x) & h(x)\\
 f'(x) & g'(x) & h'(x)\\
 f''(x) & g''(x) & h''(x)
\end{pmatrix} \in\mathbb R[x].
\]
Assume that all complex roots of $W$ are real.
Let $I\subset\mathbb R$ be an open interval such that $W(x)\neq 0$ for all $x\in I$.

\textbf{Claim.}
Every nontrivial real linear combination
$u(x)=af(x)+bg(x)+ch(x)$ (with $(a,b,c)\neq (0,0,0)$) has at most $2$ real zeros in $I$, counted with multiplicity.

\textbf{Edge case.}
If $f,g,h$ are linearly dependent over $\mathbb R$, then $W\equiv 0$, hence there is no interval $I$ with $W$ nonvanishing; the claim is vacuous in that case.

\subsection*{2) QUICK LITERATURE/CONTEXT CHECK}
This is a special case ($\dim=3$) of Eremenko's \emph{disconjugacy conjecture}. It is proved by Karp--Purbhoo in
\href{https://arxiv.org/abs/2309.04645}{arXiv:2309.04645}; see in particular Theorem~1.9 therein, which asserts that real-rootedness of the Wronskian of a polynomial subspace implies disconjugacy on every interval avoiding the Wronskian zeros.

\subsection*{3) ATTACK PLAN}
Use Karp--Purbhoo's Theorem~1.9 as an external input, verify its hypotheses for $V=\mathrm{span}_\mathbb R\{f,g,h\}$, and deduce the $\dim(V)=3$ bound ``$\le 2$ zeros'' on any interval $I$ avoiding zeros of $W$.

\subsection*{4) WORK}
\paragraph{Definition (disconjugate).}
Let $V$ be a $d$-dimensional real vector space of real-analytic functions on an interval $I\subset\mathbb R$.
We say $V$ is \emph{disconjugate on $I$} if every nonzero $u\in V$ has at most $d-1$ zeros on $I$, counted with multiplicities.

\paragraph{Theorem (Karp--Purbhoo, Theorem 1.9).}
Let $V\subset\mathbb R[u]$ be a finite-dimensional vector space of polynomials such that its Wronskian $\mathrm{Wr}(V)$ has only real zeros.
Then $V$ is disconjugate on every interval which avoids the zeros of $\mathrm{Wr}(V)$.

\paragraph{Proof of the MO claim.}
Let $V:=\mathrm{span}_\mathbb R\{f,g,h\}$. If $\dim(V)\le 2$, then $W\equiv 0$ and the claim is vacuous as noted.
Assume $\dim(V)=3$. Then $(f,g,h)$ is a basis of $V$, so $\mathrm{Wr}(V)$ is defined (up to a nonzero scalar) by the determinant $W$ above; hence $\mathrm{Wr}(V)$ has only real zeros.
Let $I$ be any open interval such that $W$ has no zeros on $I$; equivalently, $\mathrm{Wr}(V)$ has no zeros on $I$.
By Karp--Purbhoo Theorem~1.9, $V$ is disconjugate on $I$.
Since $d=\dim(V)=3$, every nonzero $u\in V$ has at most $d-1=2$ zeros on $I$ counted with multiplicity.
In particular, every nontrivial linear combination $u=af+bg+ch$ has at most $2$ zeros on $I$.
\qed

\subsection*{5) VERIFICATION}
\begin{itemize}[leftmargin=*]
\item The only external input is Karp--Purbhoo Theorem~1.9; its hypotheses are verified: $V$ is a real polynomial space and $\mathrm{Wr}(V)$ is real-rooted.
\item If $f,g,h$ are dependent, then $W\equiv 0$ and there is no admissible interval $I$; the statement is vacuously true.
\item Zeros are counted with multiplicity, matching the disconjugacy definition.
\end{itemize}

\subsection*{6) FINAL}
\textbf{PROOF.}

