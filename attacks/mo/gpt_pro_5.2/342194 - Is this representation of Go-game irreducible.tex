\section{MO 342194: ``Is this representation of Go-game irreducible?''}
\label{sec:mo342194}
\noindent\textbf{MathOverflow link:} \url{https://mathoverflow.net/questions/342194/is-this-representation-of-go-game-irreducible} (accessed 2026-01-16).

\subsection*{1) FORMAL RESTATEMENT}
\textbf{Definitions (as in the MO post).}
Let $G=(\ZZ^2,\sim)$ be the infinite grid graph with nearest-neighbor adjacency.
A \emph{state} is a map $\phi:\ZZ^2\to\{\mathrm{b},\mathrm{w},\mathrm{e}\}$.
The \emph{support} of $\phi$ is the set of non-empty points, assumed finite.
A connected component (in the graph sense) of monochromatic stones has a \emph{liberty} if it has an adjacent empty point.
A finite-support state is \emph{admissible} if every monochromatic connected component has at least one liberty.
Let $\mathcal{F}$ be the set of admissible finite-support states.
Let $\mathcal{H}=\ell^2(\mathcal{F})$ with standard basis $\{\ket{\phi}:\phi\in\mathcal{F}\}$.

For each $g\in\ZZ^2$, define an operator $B_g$ on basis states by ``playing black at $g$'':
\begin{itemize}[leftmargin=2em]
\item If $\phi(g)\neq \mathrm{e}$, set $B_g\phi=\phi$.
\item If $\phi(g)=\mathrm{e}$, first change $\phi(g)$ to black; then remove (set to empty) every monochromatic component (black or white) that has no liberties in the resulting state. Denote the final state by $B_g\phi$.
\end{itemize}
Define $W_g$ analogously for playing white.
Extend linearly by $B_g\ket{\phi}=\ket{B_g\phi}$ and similarly for $W_g$.
Let $\mathscr{A}$ be the von Neumann algebra on $\mathcal{H}$ generated by all $\{B_g,W_g: g\in\ZZ^2\}$.

\medskip
\noindent\textbf{Question.}
Is the representation of $\mathscr{A}$ on $\mathcal{H}$ irreducible, i.e.
\[
\mathscr{A}' = \{T\in B(\mathcal{H}): TA=AT\ \forall A\in\mathscr{A}\} = \CC\cdot I\ ?
\]

\subsection*{2) QUICK LITERATURE/CONTEXT CHECK}
The MathOverflow page has no posted answers as of 2026-01-16.
The construction resembles a C$^*$-algebra generated by partial permutations on the set of admissible positions.
No standard irreducibility theorem is immediately applicable without additional structure (amenability, transitivity, etc.).

\subsection*{3) ATTACK PLAN}
\textbf{Proof-track ideas.}
\begin{enumerate}[leftmargin=2em]
\item Show cyclicity/transitivity: if the orbit of a basis vector under the algebra is dense and the commutant acts diagonally, deduce scalar commutant.
\item Approximate matrix units: attempt to build operators sending one basis vector to another while annihilating others.
\end{enumerate}

\textbf{Disproof-track ideas.}
\begin{enumerate}[leftmargin=2em]
\item Find a nontrivial invariant of admissible states preserved by all $B_g,W_g$ (e.g. a mod-$m$ quantity), yielding a nontrivial reducing subspace.
\item Test finite analogs: on a finite graph, compute the commutant of the generated algebra. If reducible already on finite truncations, that suggests a counterexample/invariant.
\end{enumerate}

We pursued finite-graph tests (second disproof-track bullet) as sanity checks; no invariant emerged.

\subsection*{4) WORK}
\subsubsection*{4.1 Finite-graph analog and computation}
Replace $\ZZ^2$ by a finite graph $\Gamma=(V,E)$ and define admissible states on $V$ exactly as above (each monochromatic component has an adjacent empty vertex).
Define $B_v,W_v$ for $v\in V$ by the same capture-and-suicide rule.
Then $\mathcal{H}=\ell^2(\mathcal{F}(\Gamma))$ is finite-dimensional.

\medskip
\noindent\textbf{Computation (paths $P_n$).}
For the path graphs $P_n$ with $n=1,2,3$ vertices, we computed the $*$-algebra generated by $\{B_v,W_v\}$ on $\ell^2(\mathcal{F}(P_n))$.
In each case, the generated algebra is the full matrix algebra $M_{|\mathcal{F}(P_n)|}(\CC)$, hence the representation is irreducible.
Concretely:
\begin{itemize}[leftmargin=2em]
\item $P_1$: $|\mathcal{F}|=1$ (only the empty state), algebra dimension $1$.
\item $P_2$: $|\mathcal{F}|=5$, algebra dimension $25$.
\item $P_3$: $|\mathcal{F}|=15$, algebra dimension $225$.
\end{itemize}
(Here ``algebra dimension'' refers to the complex vector space dimension of the generated $*$-subalgebra, computed by iterated closure under multiplication and $*$ in a Gram--Schmidt basis.)

\subsubsection*{4.2 What this does and does not show}
These finite tests suggest that the generators can act ``richly'' even in small systems.
However, irreducibility on finite graphs does \emph{not} imply irreducibility on the infinite grid, and does not rule out nontrivial central sequences or other von Neumann algebraic obstructions.

\subsection*{5) VERIFICATION}
\begin{itemize}[leftmargin=2em]
\item The finite-graph definition matches the MO definition except that the board is finite; liberties are defined via adjacency to empty vertices, no boundary liberties.
\item The computation checks irreducibility only for $P_1,P_2,P_3$; larger finite graphs (e.g. $2\times2$ grid) were not computed here due to the rapid growth of state space and the cost of exact commutant calculations.
\item No invariant was proven; thus no disproof is claimed.
\end{itemize}

\subsection*{6) FINAL}
\begin{center}
\textbf{UNRESOLVED}
\end{center}


