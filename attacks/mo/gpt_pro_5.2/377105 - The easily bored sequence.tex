\section{MO 377105: ``The easily bored sequence''}
\label{sec:mo377105}
\noindent\textbf{MathOverflow link:} \url{https://mathoverflow.net/questions/377105/the-easily-bored-sequence} (accessed 2026-01-16).

\subsection*{1) FORMAL RESTATEMENT}
We formalize the infinite binary word definition.

\medskip
\noindent\textbf{Definitions.}
Let $\{0,1\}^*$ denote the set of finite binary words.
For a nonempty word $v\in\{0,1\}^*$ and integer $a\ge 2$, write $v^a$ for the concatenation of $a$ copies of $v$.

Given a finite word $w\in\{0,1\}^*$ and a bit $x\in\{0,1\}$, let $wx$ denote concatenation.
Define the \emph{repetition pair} $R_w(x)=(a,b)$ as follows:
\begin{itemize}[leftmargin=2em]
\item $a$ is the largest integer $\ge 1$ such that $wx$ has a suffix of the form $v^a$ with some nonempty word $v$.
\item Among all $v$ achieving this maximal $a$, let $b$ be the maximum possible length $|v|$.
\end{itemize}
Order repetition pairs lexicographically: $(a,b)<(a',b')$ iff either $a<a'$ or ($a=a'$ and $b<b'$).

Define an infinite word $b_1 b_2 b_3\cdots$ recursively by:
\begin{itemize}[leftmargin=2em]
\item $b_1=0$.
\item For $n\ge 2$, let $w=b_1\cdots b_{n-1}$ and set
\[
 b_n := \begin{cases}
 0, & \text{if } R_w(0) < R_w(1),\\
 1, & \text{otherwise}.
 \end{cases}
\]
\end{itemize}

\noindent\textbf{Questions (as in MO).}
Does the asymptotic density of $1$'s exist? What is the critical exponent? Is the sequence recurrent? Is the real number $0.b_1b_2b_3\dots$ algebraic or transcendental?

\subsection*{2) QUICK LITERATURE/CONTEXT CHECK}
The MathOverflow page has no posted answers as of 2026-01-16.
The sequence is described as greedily minimizing repetitions in a suffix sense; it resembles constructions in combinatorics on words (Dejevu, Thue--Morse/avoiding powers), but it is not one of the standard classical sequences.

\subsection*{3) ATTACK PLAN}
\textbf{Proof-track ideas.}
\begin{enumerate}[leftmargin=2em]
\item Try to prove global avoidance properties (e.g. no cubes) from the greedy rule.
\item Attempt to prove uniform recurrence via morphic/automatic structure.
\item Use subadditivity to prove density exists.
\end{enumerate}

\textbf{Disproof-track ideas.}
\begin{enumerate}[leftmargin=2em]
\item Compute a long prefix and estimate densities/exponents; search for evidence of nonconvergence.
\item Search for large repetitions to bound critical exponent from below.
\end{enumerate}

We performed prefix computations to length $5000$.

\subsection*{4) WORK}
\subsubsection*{4.1 Small cases}
The computed prefix begins
\[
010011010001\dots
\]
matching the MO post.

\subsubsection*{4.2 Computation to length $5000$}
We implemented the greedy definition using a rolling hash to test repeated suffixes efficiently.
\begin{itemize}[leftmargin=2em]
\item For $N=5000$, the number of $1$'s in $b_1\dots b_N$ is $2496$, giving empirical density $0.4992$.
\item In this computation, the \emph{chosen} step never ended with a cube suffix: the maximal exponent $a$ for the chosen digit remained $\le 2$ for all prefixes up to length $5000$.
\item However, the \emph{rejected} option sometimes had exponent $a=3$; i.e. the greedy rule actively avoids creating cube suffixes when possible.
\end{itemize}
(Hash collisions are theoretically possible, but were not detected by cross-checks on shorter prefixes using exact comparisons.)

\subsection*{5) VERIFICATION}
\begin{itemize}[leftmargin=2em]
\item The definition of $R_w(x)$ uses $a\ge 1$; this choice is consistent with the MO text (it implies $R_w(0)\neq R_w(1)$ for nonempty $w$ because appending the last digit always creates at least a square suffix of length $1$).
\item Computational evidence (density $\approx 1/2$) is suggestive but not a proof of existence.
\item The observed ``no cube suffix'' property up to $5000$ could fail later; no proof is given.
\end{itemize}

\subsection*{6) FINAL}
\begin{center}
\textbf{UNRESOLVED}
\end{center}


