\section{MO 62654: A functor mapping to both $KO^{n}(X)$ and $KO^{-n}(X)$}

\subsection*{1) Formal restatement}
The literal text is again a \emph{request for explanation}, not a proposition. 

\paragraph{Minimal corrected statement.}
Let $X$ be compact Hausdorff, let $Cl(n)$ be the real $*$-Clifford algebra with odd self-adjoint generators $e_i$ satisfying $e_i^2=1$. Let $\mathcal{C}_n(X)$ be the additive category of finite-dimensional $\mathbb{Z}/2$-graded real vector bundles $V\to X$ with a graded $*$-representation of $Cl(n)$ on each fiber. Then:
\begin{quote}
(\dagger) The assignment $V\mapsto (V,\rho,0)$ defines \emph{natural group homomorphisms} from the Grothendieck group $K_0(\mathcal{C}_n(X))$ to both $KO^{n}(X)$ (skew-adjoint model) and $KO^{-n}(X)$ (self-adjoint model), because $0$ is both self-adjoint and skew-adjoint.
\end{quote}

\subsection*{2) Quick literature/context check}
This is consistent with the Fredholm-operator models for real $K$-theory going back to Atiyah and Singer, and with the Clifford-module description of $KO$ due to Atiyah--Bott--Shapiro.

\subsection*{3) Attack plan}
\begin{itemize}[leftmargin=*]
\item \textbf{Proof track:} Check directly that $(V,\rho,0)$ is an admissible cycle in both analytic models; check that direct sum induces additivity and that homotopy/stabilization relations are respected, yielding well-defined homomorphisms out of the Grothendieck group.
\item \textbf{Disproof track:} Not applicable for (\dagger): it is a formal consequence of the definitions once the cycle models are accepted.
\end{itemize}

\subsection*{4) Work}
We work with the cycle description in the problem statement.

\begin{proposition}[Degenerate cycles from finite-dimensional Clifford bundles land in both $KO^{\pm n}$]
Let $V\to X$ be a finite-dimensional $\mathbb{Z}/2$-graded real vector bundle equipped with a fiberwise graded $*$-representation $\rho:Cl(n)\to \mathrm{End}(V_x)$. Then the triple $(V,\rho,0)$ defines:
\begin{enumerate}[label=(\alph*)]
\item an element of $KO^{n}(X)$ in the skew-adjoint-$Cl(n)$ Fredholm model, and
\item an element of $KO^{-n}(X)$ in the self-adjoint-$Cl(n)$ Fredholm model.
\end{enumerate}
Moreover, under direct sum $V\oplus W$ these assignments are additive, hence descend to group homomorphisms $K_0(\mathcal{C}_n(X))\to KO^{\pm n}(X)$.
\end{proposition}

\begin{proof}
In both models, a cycle consists of a graded real Hilbert bundle (or a Hilbert space with a representation) together with an odd $Cl(n)$-linear Fredholm operator satisfying either $F^*=F$ (self-adjoint model) or $F^*=-F$ (skew-adjoint model).

For a finite-dimensional bundle $V$, the operator $F=0$ is:
\begin{itemize}[leftmargin=*]
\item odd (it swaps even/odd degrees trivially),
\item $Cl(n)$-linear (commutes with the $Cl(n)$-action), and
\item Fredholm (on a finite-dimensional space, every linear operator is Fredholm),
\item both self-adjoint and skew-adjoint (since $0^*=0=-0$).
\end{itemize}
Thus $(V,\rho,0)$ is a valid cycle in \emph{both} definitions.

If $V\cong V'$ as $Cl(n)$-bundles, then $(V,\rho,0)$ and $(V',\rho',0)$ are isomorphic cycles, hence define the same class. Direct sum corresponds to addition in $KO^{\pm n}(X)$ by the definition of the group operation. Therefore the assignments are additive on isomorphism classes and extend uniquely to homomorphisms out of the Grothendieck group $K_0(\mathcal{C}_n(X))$.
\end{proof}

\subsection*{5) Verification}
The only nontrivial input is acceptance of the analytic cycle models for $KO^{\pm n}(X)$ (Fredholm + Clifford action), which is standard in the literature. Once those definitions are fixed, the proof is purely formal: $0$ satisfies both adjointness conditions and is Fredholm in finite dimensions; additivity follows from direct sum.

\subsection*{6) FINAL}
\textbf{PROOF.}
