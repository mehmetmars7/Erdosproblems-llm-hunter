\section{Problem 259117: field of fractions of $\QQ[G]$ for torsion-free abelian $G$}

\subsection*{1) FORMAL RESTATEMENT}
\paragraph{Literal issue.}
The prompt is a \emph{question}, not a statement with a truth value. To fit the required proof/disproof format, we convert it into a precise existence statement.

\paragraph{Conventions and definitions.}
\begin{itemize}
\item All groups are written additively, and are assumed commutative unless stated otherwise.
\item A group $G$ is \emph{torsion-free} if for all $g\in G$ and all integers $n\ge 1$, $ng=0\Rightarrow g=0$.
\item For a (commutative) group $G$, the group algebra $\QQ[G]$ is the $\QQ$-vector space with basis $\{t^g:g\in G\}$ and multiplication $t^g t^h = t^{g+h}$. (It is a commutative $\QQ$-algebra if $G$ is abelian.)
\item If $\QQ[G]$ is a domain, its fraction field is $\QQ(G)\coloneqq \mathrm{Frac}(\QQ[G])$.
\end{itemize}

\paragraph{Minimal corrected proposition (existence form).}
\begin{quote}
\textbf{(E$_{259117}$)} There exist torsion-free abelian groups $G\not\cong H$ such that $\QQ(G)\cong \QQ(H)$ as fields.
\end{quote}
The opposite (a rigidity statement) is:
\begin{quote}
\textbf{(R$_{259117}$)} For torsion-free abelian $G,H$, if $\QQ(G)\cong\QQ(H)$ as fields then $G\cong H$.
\end{quote}

\paragraph{Stress points / edge cases.}
\begin{itemize}
\item If $G$ is finitely generated torsion-free, then $G\cong \ZZ^r$; then $\QQ[G]\cong \QQ[t_1^{\pm 1},\dots,t_r^{\pm 1}]$ and $\QQ(G)\cong \QQ(t_1,\dots,t_r)$ is purely transcendental.
\item In infinite rank, $\QQ[G]$ is a large subring of a Laurent polynomial algebra on infinitely many variables; recovering $G$ from $\QQ(G)$ may depend on subtle definability questions inside a field.
\item One must be careful about the meaning of ``easy example'': the groups should be explicitly described, and the field isomorphism should be explicit or at least provable with standard invariants.
\end{itemize}

\subsection*{2) QUICK LITERATURE/CONTEXT CHECK (browsing YES)}
The original MathOverflow thread has no posted answers as of January 2026 (it displays ``You must log in to answer this question'' and shows no answers). See \href{https://mathoverflow.net/questions/259117/the-field-of-fractions-of-the-rational-group-algebra-of-a-torsion-free-abelian-g}{MO 259117}.
The question itself cites a paper of Bastos--Viswanathan asserting: $\QQ(G)$ is purely transcendental over $\QQ$ iff $G$ is free abelian (as stated in the question text).

\subsection*{3) ATTACK PLAN}
\paragraph{Proof-track ideas (toward (R$_{259117}$)).}
\begin{enumerate}[label=(P\arabic*)]
\item \emph{Transcendence degree and valuation theory.} Compute invariants of $\QQ(G)$ such as $\mathrm{trdeg}_\QQ\QQ(G)$, value groups of natural valuations, and whether these recover rank or finer invariants of $G$.
\item \emph{Recovering the monomial group.} The multiplicative subgroup $\{t^g:g\in G\}\subset \QQ(G)^\times$ is a copy of $G$. Try to characterize it intrinsically as a definable subgroup of $\QQ(G)^\times$ (e.g., by divisibility properties or valuation-theoretic properties).
\item \emph{Model-theoretic rigidity.} Try to show $G$ is interpretable in the pure field structure of $\QQ(G)$ under natural hypotheses (e.g., no primitive elements, $p$-local, rank $1$, etc.), hence isomorphism of fields implies isomorphism of groups.
\end{enumerate}

\paragraph{Disproof-track ideas (toward (E$_{259117}$)).}
\begin{enumerate}[label=(D\arabic*)]
\item \emph{Rank-$1$ subgroups of $\QQ$.} Torsion-free rank-$1$ groups are subgroups of $\QQ$. Their group algebras often look like directed unions of Laurent polynomial rings $\QQ[t^{\pm 1/n}]$. Try to find non-isomorphic such groups with isomorphic union fields.
\item \emph{Non-cancellation phenomena.} If there exist $G\not\cong H$ with $G\oplus\ZZ^r\cong H\oplus\ZZ^r$ for some $r$ (stably isomorphic) and if $\QQ(G\oplus\ZZ^r)\cong \QQ(G)(t_1,\dots,t_r)$, then perhaps $\QQ(G)$ and $\QQ(H)$ become isomorphic after adjoining finitely many transcendentals; see whether such stable behavior can be leveraged without adjoining variables.
\item \emph{Abstract field isomorphisms.} Even if $\QQ(G)$ and $\QQ(H)$ are presented as unions of towers of algebraic extensions of a rational function field, different towers may yield isomorphic abstract fields. Look for invariants (e.g., absolute Galois group, ramification in Kummer towers) that fail to distinguish.
\end{enumerate}

\subsection*{4) WORK}
We do not know a complete solution to (E$_{259117}$) or (R$_{259117}$). We record provable partial results and identify the first hard gap.

\subsubsection*{Lemma 4.1 (finitely generated case: rigidity by transcendence degree).}
\emph{Let $G,H$ be finitely generated torsion-free abelian groups. If $\QQ(G)\cong\QQ(H)$ then $G\cong H$.}

\begin{proof}
Since $G$ is finitely generated torsion-free abelian, $G\cong \ZZ^r$ for a unique $r\ge 0$ (the rank). Likewise $H\cong \ZZ^s$.
Then $\QQ[G]\cong \QQ[t_1^{\pm 1},\dots,t_r^{\pm 1}]$ and $\QQ[H]\cong \QQ[u_1^{\pm 1},\dots,u_s^{\pm 1}]$ as $\QQ$-algebras.
Taking fraction fields yields $\QQ(G)\cong \QQ(t_1,\dots,t_r)$ and $\QQ(H)\cong \QQ(u_1,\dots,u_s)$.

If $\QQ(t_1,\dots,t_r)\cong \QQ(u_1,\dots,u_s)$ as fields, then the transcendence degrees over $\QQ$ agree:
\[
r=\mathrm{trdeg}_\QQ \QQ(t_1,\dots,t_r)=\mathrm{trdeg}_\QQ \QQ(u_1,\dots,u_s)=s,
\]
since transcendence degree is a field isomorphism invariant. Hence $r=s$ and therefore $G\cong \ZZ^r\cong \ZZ^s\cong H$.
\end{proof}

\subsubsection*{Lemma 4.2 (free direct summands correspond to purely transcendental extensions).}
\emph{If $G\cong G_0\oplus \ZZ^r$ with $G_0$ torsion-free abelian, then}
\[
\QQ(G)\ \cong\ \QQ(G_0)(t_1,\dots,t_r)
\]
\emph{where $(t_1,\dots,t_r)$ are algebraically independent over $\QQ(G_0)$.}

\begin{proof}
Write $\QQ[G]\cong \QQ[G_0][\ZZ^r]\cong \QQ[G_0][t_1^{\pm1},\dots,t_r^{\pm1}]$.
Since adjoining Laurent variables and then taking fraction field gives a rational function field extension, we get
\[
\mathrm{Frac}(\QQ[G_0][t_1^{\pm1},\dots,t_r^{\pm1}])\cong \mathrm{Frac}(\QQ[G_0])(t_1,\dots,t_r)=\QQ(G_0)(t_1,\dots,t_r).
\]
\end{proof}

\subsubsection*{Rank-1 exploratory computation (directed unions).}
If $G=\ZZ[1/p]\subset \QQ$ (additively), then $\QQ[G]$ identifies with the directed union
\[
\QQ[t^{\pm 1/p^n}]\subset \QQ[t^{\pm 1/p^{n+1}}]\subset \cdots,
\]
so
\[
\QQ(G)\cong \bigcup_{n\ge 0}\QQ(t^{1/p^n}).
\]
This suggests that comparing different rank-1 groups may reduce to comparing infinite Kummer towers of rational function fields. We did not complete the classification of when two such union fields are isomorphic as abstract fields.

\subsection*{5) VERIFICATION}
\paragraph{Adversarial checks on the partial results.}
\begin{itemize}
\item Lemma 4.1 uses only standard invariants: rank classification of finitely generated torsion-free abelian groups and invariance of transcendence degree. No hidden assumptions.
\item Lemma 4.2 uses the (commutative) tensor product decomposition $\QQ[G_0\oplus \ZZ^r]\cong \QQ[G_0][\ZZ^r]$ and the identification of $\QQ[\ZZ^r]$ with Laurent polynomial rings. This is standard and can be checked on the basis $\{t^{(g_0,z)}\}$.
\item For the rank-1 union description, we verified inclusions $\QQ[t^{\pm 1/p^n}]\subset \QQ[t^{\pm 1/p^{n+1}}]$ and that the union is a domain whose fraction field is the union of the fraction fields (direct limit argument). This holds because localization commutes with directed colimits for multiplicative sets in domains.
\end{itemize}

\subsection*{6) FINAL}
\textbf{UNRESOLVED}

\paragraph{(i) Strongest fully proved partial result obtained.}
For finitely generated torsion-free abelian $G,H$, an isomorphism $\QQ(G)\cong\QQ(H)$ forces $G\cong H$ (Lemma 4.1). Also $\ZZ^r$-summands correspond to adjoining $r$ independent transcendentals (Lemma 4.2).

\paragraph{(ii) First gap.}
For infinitely generated torsion-free abelian groups, we did not find a field-theoretic invariant that \emph{provably} recovers the isomorphism type of $G$ from $\QQ(G)$, nor did we construct explicit non-isomorphic $G,H$ with $\QQ(G)\cong\QQ(H)$.

\paragraph{(iii) Top 3 next moves.}
\begin{enumerate}[label=(\alphic*)]
\item Analyze rank-1 subgroups of $\QQ$ (Baer type invariants) and compute which invariants are visible in the Kummer tower field $\bigcup_n \QQ(t^{1/p^n})$ via valuation/ramification.
\item Attempt definability: characterize the ``monomial subgroup'' $\{qt^g:q\in\QQ^\times, g\in G\}$ inside $\QQ(G)^\times$ using purely field-theoretic properties (e.g., all elements with infinitely many roots).
\item Investigate whether $\QQ(G)$ determines the divisible hull $\QQ\otimes_\ZZ G$ and whether different lattices in the same $\QQ$-vector space can yield isomorphic fraction fields.
\end{enumerate}

\paragraph{(iv) Likely shape of a minimal counterexample.}
A plausible smallest counterexample would have $G,H$ of finite rank (probably $1$ or $2$) but not finitely generated, since rank is detectable. Rank $1$ subgroups of $\QQ$ with distinct Baer type but producing isomorphic Kummer-type union fields are a natural target.

% ======================================================================
