\section{MO 379560: ``Can every 3-dimensional convex body be trapped in a tetrahedral cage?''}
\label{sec:mo379560}
\noindent\textbf{MathOverflow link:} \url{https://mathoverflow.net/questions/379560/can-every-3-dimensional-convex-body-be-trapped-in-a-tetrahedral-cage} (accessed 2026-01-16).

\subsection*{1) FORMAL RESTATEMENT}
We restate the definitions used in the MO question.

\medskip
\noindent\textbf{Definitions.}
Let $C\subset\RR^3$ be a convex body (compact, convex, with nonempty interior).
Let $T$ be a tetrahedron in $\RR^3$ and let $T^{(1)}$ denote its $1$-skeleton (the union of its $6$ edges).
The MO post defines: $C$ is \emph{trapped} by the tetrahedral cage $T^{(1)}$ if the cage is fixed and for every continuous rigid motion $f_t$ of $\RR^3$ (with $f_0=\mathrm{id}$), either
\begin{enumerate}[leftmargin=2em]
\item $f_t(C)$ intersects $T^{(1)}$ for every $t\in[0,1]$, or
\item there exists $t_0\in[0,1]$ such that $T^{(1)}$ contains a point in the interior of $f_{t_0}(C)$.
\end{enumerate}
Intuitively: you cannot move $C$ away from the cage without hitting it or engulfing an edge.

\medskip
\noindent\textbf{Question.}
Does every convex body $C\subset\RR^3$ admit a tetrahedron $T$ such that $C$ is trapped by $T^{(1)}$?

\subsection*{2) QUICK LITERATURE/CONTEXT CHECK}
The MathOverflow page has no posted answers as of 2026-01-16.
The post mentions a related paper of Oded Schramm (``How to cage an egg'', Invent. Math. 107 (1992)). Schramm's result concerns polyhedra whose edges are tangent to a given smooth strictly convex body (``midscribed'' polyhedra), for a specified combinatorial type.
This suggests that for smooth strictly convex $C$ one can find tetrahedra whose edges are tangent to $\partial C$, but tangency alone does not immediately imply the MO ``trapped'' condition.

\subsection*{3) ATTACK PLAN}
\textbf{Proof-track ideas.}
\begin{enumerate}[leftmargin=2em]
\item Use Schramm's midscribing theorem to construct a tetrahedron with all edges tangent to $\partial C$; then attempt to prove that such tangency implies trapping.
\item Try to reduce trapping to a topological obstruction in configuration space: consider the space of rigid motions of $C$ avoiding $T^{(1)}$ and show it is bounded/empty.
\end{enumerate}

\textbf{Disproof-track ideas.}
\begin{enumerate}[leftmargin=2em]
\item Search for a convex body with a ``thin'' direction that can always be threaded through any tetrahedral frame.
\item Produce a one-parameter escape motion for any chosen tetrahedron, perhaps exploiting flatness or symmetry.
\end{enumerate}

No complete proof/disproof was achieved here.

\subsection*{4) WORK}
\subsubsection*{4.1 A rigorous partial statement from Schramm-type results (tangency, not trapping)}
\textbf{Partial theorem (informal).}
For a smooth strictly convex body $C$ and a combinatorial type of convex polyhedron $P$, there exists a convex polyhedron $Q$ combinatorially equivalent to $P$ whose edges are tangent to $\partial C$.
In particular, taking $P$ to be a tetrahedron yields a tetrahedron whose $6$ edges are tangent to $\partial C$.

\textbf{Status.}
This statement is attributed to Schramm's ``How to cage an egg''; we do not reproduce its proof here.
Moreover, the relation between ``edge-tangent tetrahedron'' and the MO trapping definition is not established in this writeup.

\subsection*{5) VERIFICATION}
\begin{itemize}[leftmargin=2em]
\item We carefully separate: edge tangency (midscribing) is not shown to imply trapping.
\item No counterexample is provided.
\end{itemize}

\subsection*{6) FINAL}
\begin{center}
\textbf{UNRESOLVED}
\end{center}


