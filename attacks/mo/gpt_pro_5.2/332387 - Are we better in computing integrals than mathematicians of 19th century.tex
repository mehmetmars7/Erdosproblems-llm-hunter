\section{MO \#332387: Are we better in computing integrals than mathematicians of 19th century?}
\MO{332387}{Are we better in computing integrals than mathematicians of 19th century?}

\subsection*{1) FORMAL RESTATEMENT}
\textbf{Literal statement ambiguity:} the MO post is a qualitative question, not a theorem. ``Better at computing integrals'' could mean: (a) larger class of integrals with closed-form antiderivatives, (b) faster computation of definite integrals numerically, (c) better symbolic simplification, etc.

\textbf{Minimal corrected formal statement (one plausible formalization):}
There exists an algorithm that, given an elementary function $f(x)$, decides whether $\int f(x)\,dx$ has an elementary antiderivative and, if so, produces one.

\subsection*{2) QUICK LITERATURE/CONTEXT CHECK}
Modern symbolic integration includes the Risch algorithm (decision procedure for elementary antiderivatives) and extensive special-function/heuristic libraries in CAS.
For many definite integrals and period integrals, modern approaches include creative telescoping (producing differential/difference equations for parametric integrals), Picard--Fuchs equations for algebraic periods, and motivic/Hodge-theoretic restrictions (e.g. ``this period lies in the span of MZVs / polylogarithms'' under strong hypotheses). Numerical quadrature is also highly developed.

\subsection*{3) ATTACK PLAN}
\begin{itemize}[leftmargin=*]
\item \textbf{Proof track (for corrected statement):} present a complete proof of existence/correctness of the Risch algorithm (not feasible here).
\item \textbf{Disproof track (for literal statement):} show the literal comparative claim is not mathematically well-posed and cannot be assigned a truth value without a metric.
\end{itemize}

\subsection*{4) WORK}
Because the question is not a well-posed mathematical proposition, no gap-free proof/disproof is possible without choosing a specific metric.

\subsection*{5) VERIFICATION}
The ambiguity identification is precise: the phrase ``better in computing integrals'' is not formal. Any attempt to answer requires selecting a definition of ``better''.

\subsection*{6) FINAL}
\textbf{UNRESOLVED.}
\begin{itemize}[leftmargin=*]
\item (i) Fully proved partial result: the ambiguity analysis.
\item (ii) First gap: a formal definition of ``better'' that yields a decidable statement.
\item (iii) Next moves: (1) formalize as existence of decision procedures for classes of functions (elementary, Liouvillian, D-finite, etc.); (2) compare computational complexity/coverage of 19th-century tables vs modern CAS; (3) restrict to a benchmark family of integrals.
\item (iv) Minimal counterexample: a concrete integral family where modern CAS systematically fails but classical methods succeed (or vice versa), depending on the chosen metric.
\end{itemize}

