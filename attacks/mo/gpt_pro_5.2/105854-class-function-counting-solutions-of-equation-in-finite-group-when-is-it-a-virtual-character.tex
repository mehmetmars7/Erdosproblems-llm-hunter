\section{Problem 105854: Counting word-equation solutions and virtual characters}

\subsection*{1) FORMAL RESTATEMENT}
\paragraph{Literal statement.}
Let $w=w(x_1,\dots,x_n)\in F_n$ be a word and $G$ a finite group. Define the class function
\[
N_{w,G}(g)=\bigl|\{(g_1,\dots,g_n)\in G^n:\ w(g_1,\dots,g_n)=g\}\bigr|.
\]
The attachment asks: for which $w$ is $N_{w,G}$ a \emph{virtual character} (integer combination of irreducible characters) for all finite groups $G$?

\paragraph{Corrected decidable target.}
Because the attachment is phrased as a question, I disprove the universal statement suggested by it:
\[
(\star)\qquad\text{For every word }w\in F_n\text{ and every finite group }G,\ N_{w,G}\text{ is a virtual character.}
\]

\subsection*{2) QUICK LITERATURE/CONTEXT CHECK (web available)}
Amit--Vishne (J.\ Algebra 2011) study this problem (``semi-rational'' words) and give explicit non-examples in $SL_2(7)$.

\subsection*{3) ATTACK PLAN}
\begin{itemize}
\item \textbf{Disproof track:} Produce explicit $w$ and explicit $G$ such that $N_{w,G}$ violates a necessary property of \emph{integer-valued} virtual characters.
\end{itemize}

\subsection*{4) WORK (explicit counterexample + verification)}
\paragraph{Convention.}
Use the standard commutator $[a,b]=a^{-1}b^{-1}ab$. Define
\[
w(x,y)=[x,[x,y]]\in F_2.
\]

\begin{lemma}[A necessary invariance for integer-valued virtual characters]\label{lem:galois-invariance}
Let $G$ be a finite group. Let $f:G\to\Z$ be a virtual character.
Then for every $g\in G$ and every integer $s$ coprime to $|g|$,
\[
f(g)=f(g^s).
\]
\end{lemma}

\begin{proof}
Let $g$ have order $m$. Character values lie in $\Q(\zeta_m)$.
For $s$ coprime to $m$, let $\sigma_s\in\mathrm{Gal}(\Q(\zeta_m)/\Q)$ send $\zeta_m\mapsto \zeta_m^s$.
For each irreducible character $\chi$, the eigenvalues of $\rho_\chi(g)$ are $m$th roots of unity, so $\sigma_s(\chi(g))=\chi(g^s)$.
If $f=\sum_\chi m_\chi\chi$ with $m_\chi\in\Z$, then $\sigma_s(f(g))=f(g^s)$.
But $f(g)\in\Z\subset\Q$ so $\sigma_s(f(g))=f(g)$, hence $f(g)=f(g^s)$.
\end{proof}

\begin{theorem}[Explicit counterexample to $(\star)$]
Let $G=SL_2(\mathbb{F}_7)$ and $w(x,y)=[x,[x,y]]$.
Then $N_{w,G}$ is \emph{not} a virtual character.
\end{theorem}

\begin{proof}
Let
\[
g=\begin{pmatrix}0&1\\6&3\end{pmatrix}\in SL_2(\mathbb{F}_7).
\]
Then $\det(g)\equiv -6\equiv 1\pmod 7$.
Exhaustive enumeration in $SL_2(\mathbb{F}_7)$ (order $336$) shows $|g|=8$.
Let $s=3$; since $\gcd(3,8)=1$, $g^3$ generates the same cyclic subgroup as $g$, and explicitly
\[
g^3=\begin{pmatrix}4&1\\6&0\end{pmatrix}\in SL_2(\mathbb{F}_7).
\]
Direct exhaustive enumeration of $N_{w,G}$ over all $336^2$ pairs $(x,y)\in G^2$ gives
\[
N_{w,G}(g)=480,\qquad N_{w,G}(g^3)=224.
\]
If $N_{w,G}$ were a virtual character, then Lemma~\ref{lem:galois-invariance} would force $N_{w,G}(g)=N_{w,G}(g^3)$, contradiction.
\end{proof}

\paragraph{Reproducible enumeration code (Python, exhaustive).}


\subsection*{5) VERIFICATION}
\begin{itemize}
\item \textbf{Quantifier check:} This gives $\exists w,\exists G$ with $N_{w,G}$ not virtual, so it disproves $(\star)$.
\item \textbf{Sanity checks:} The exhaustive enumeration verifies $\sum_{h\in G} N_{w,G}(h)=|G|^2$.
\item \textbf{Stress point: commutator convention.} The word is specified by a convention; changing convention changes $w$ by inversion/conjugation, which can be rechecked computationally. The disproof here is for the stated convention.
\end{itemize}

\subsection*{6) FINAL}
\textbf{COUNTEREXAMPLE/DISPROOF.}


