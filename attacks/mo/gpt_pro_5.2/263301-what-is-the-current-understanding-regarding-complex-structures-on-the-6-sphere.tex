\section{Problem 263301: complex structures on $S^6$ and Atiyah's preprint}

\subsection*{1) FORMAL RESTATEMENT}
\paragraph{Literal issue.}
This is a multi-part \emph{status inquiry} (``what is the current understanding?'') rather than a theorem. A proof/disproof task is not well-posed.

\paragraph{Minimal corrected yes/no proposition.}
A common distilled mathematical claim in this context is:
\begin{quote}
\textbf{(A$_{263301}$)} ``$S^6$ admits \emph{no} integrable complex structure.''
\end{quote}
and its negation:
\begin{quote}
\textbf{(E$_{263301}$)} ``$S^6$ admits \emph{some} integrable complex structure.''
\end{quote}
The classical problem is to decide which of these is true.

\paragraph{Stress points.}
\begin{itemize}
\item It is known that $S^6$ admits \emph{almost} complex structures, but integrability is open.
\item Many claimed proofs in either direction have had errors; the prompt asks specifically about Atiyah's brief 2016 manuscript.
\end{itemize}

\subsection*{2) QUICK LITERATURE/CONTEXT CHECK (browsing YES)}
The MO thread \href{https://mathoverflow.net/questions/263301/what-is-the-current-understanding-regarding-complex-structures-on-the-6-sphere}{263301} contains expert comments asserting that Atiyah's proof is not accepted; in particular, it is criticized for not specifying the key differential operator and for not convincingly using integrability.

More broadly, MO discussions about earlier claimed solutions include the 2015 question about Etesi's published paper \href{https://mathoverflow.net/questions/210089/complex-structure-on-s6-gets-published-in-journ-math-phys}{210089}, which has answers and extensive discussion indicating non-acceptance by the community.

We did not find a refereed, universally accepted proof of (A$_{263301}$) or (E$_{263301}$) as of January 2026; the problem remains widely regarded as open, with many partial constraints on what a hypothetical complex structure would imply (e.g.\ nonexistence of orthogonal complex structures with respect to the round metric, algebraic dimension constraints, etc.).

\subsection*{3) ATTACK PLAN}
\paragraph{Proof-track.}
A complete proof of (A$_{263301}$) would require new geometry/topology beyond what can be reliably reconstructed here. We do not attempt to invent one.

\paragraph{Disproof-track.}
A complete disproof of (A$_{263301}$) would require an explicit integrable complex structure on $S^6$ with verified transition functions and integrability, which is likewise out of reach here.

\paragraph{What we \emph{can} do.}
We can (i) precisely flag that the prompt is not a theorem, (ii) state the minimal decision problem, and (iii) record provable partial facts (e.g.\ existence of almost complex structures; nonexistence of orthogonal complex structures) and the status consensus.

\subsection*{4) WORK}
\subsubsection*{4.1. A fully proved partial statement: almost complex vs complex}
\begin{proposition}
The existence of an almost complex structure on $S^6$ does not settle existence of a complex structure; integrability is an additional differential condition not implied by the topology alone.
\end{proposition}
\begin{proof}
An \emph{almost complex structure} on a smooth $2n$-manifold $M$ is a smooth bundle endomorphism $J:TM\to TM$ with $J^2=-\mathrm{id}$.
Such a $J$ exists on $S^6$ (classically constructed using octonions/nearly K\"ahler geometry).
A \emph{complex structure} is an atlas with holomorphic transition maps; equivalently, an almost complex structure $J$ satisfying the Newlander--Nirenberg integrability condition (vanishing Nijenhuis tensor).
It is standard that many manifolds admit almost complex structures that are not integrable; thus almost complex existence does not imply complex existence.
\end{proof}

\subsubsection*{4.2. Counterexample to ``Atiyah's argument is a settled proof'' (status claim).}
This is not a mathematical counterexample in the sense of an object falsifying a theorem, but the MO commentary provides concrete missing steps:
the claim ``Atiyah's brief paper provides a correct proof'' is contradicted by expert commentary emphasizing that the key operator is not defined and integrability is not used in a meaningful way.

\subsection*{5) VERIFICATION}
\paragraph{Scope control.}
We have \emph{not} claimed a solution to the existence problem.
We have only: (i) formalized the implicit yes/no statements, (ii) provided a correct elementary separation between almost complex and complex structures, and (iii) summarized community status reports from expert discussions.

\subsection*{6) FINAL}
\textbf{UNRESOLVED}

\paragraph{(i) Strongest fully proved partial result obtained.}
We proved the basic implication gap: almost complex $\nRightarrow$ complex; integrability is extra (Proposition 4.1).

\paragraph{(ii) First gap.}
We do not have (and did not locate) a complete accepted proof of either (A$_{263301}$) or (E$_{263301}$) as of January 2026.

\paragraph{(iii) Top 3 next moves.}
\begin{enumerate}[label=(\alphic*)]
\item Identify a single precise candidate obstruction invariant that \emph{provably} must hold for any integrable complex structure on $S^6$ (e.g.\ via KR-theory/indices) and check it.
\item For claimed constructions (e.g.\ in the literature), isolate a checkable integrability criterion and verify it in coordinates or via invariants.
\item Use known partial results (twistor theory, algebraic dimension, automorphism group constraints) to narrow possibilities and seek contradictions.
\end{enumerate}

\paragraph{(iv) Likely shape of a minimal counterexample.}
Either a rigorously defined holomorphic atlas on $S^6$ (disproving nonexistence) or a robust topological/differential invariant (e.g.\ index-theoretic) forced by integrability and violated on $S^6$.

% ======================================================================
