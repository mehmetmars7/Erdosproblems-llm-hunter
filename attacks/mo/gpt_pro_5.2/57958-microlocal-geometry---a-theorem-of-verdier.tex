\section{MO 57958: Microlocal geometry --- a theorem of Verdier}

\subsection*{1) Formal restatement}
\paragraph{Ambiguities to fix.}
One must specify the sheaf theory context (topological, complex-analytic, or \'etale), the definition of ``constructible complex'', and the precise specialization functor $\nu_0$ (Verdier specialization via deformation to the normal cone).

\paragraph{Minimal corrected statement (topological/complex-analytic setting).}
Let $E$ be a complex vector space, viewed as a complex manifold (or equivalently a real analytic manifold). Let $F\in D^b_c(E;\Bbbk)$ be a bounded derived constructible complex of $\Bbbk$-sheaves (constructible with respect to some Whitney stratification). Let $\ell:E\to\mathbb{C}$ be a complex linear map and $i_0:\{0\}\hookrightarrow E$. Let $\nu_0(F)$ denote Verdier's specialization of $F$ at $0$, an object on $T_0E\simeq E$ which is monodromic under scalar multiplication. Then:
\begin{quote}
(\*) For $\ell$ in some Zariski-open dense subset of $E^\vee$, the canonical morphism
$i_0^*\psi_\ell(F)\to i_0^*\psi_\ell(\nu_0(F))$
is an isomorphism in $D^b(\Bbbk)$.
\end{quote}

\subsection*{2) Quick literature/context check}
Verdier states such comparison results in his \emph{G\'eom\'etrie microlocale} notes (see the preview PDF around the discussion of comparing vanishing cycles with those of the specialization). The MathOverflow question asks for a precise reference and/or proof.

\subsection*{3) Attack plan}
\begin{itemize}[leftmargin=*]
\item \textbf{Proof track:} Reduce to a microlocal statement: for generic $\ell$, the vanishing/nearby cycles at $0$ depend only on microlocal data of $F$ at covector $d\ell$. Specialization $\nu_0(F)$ preserves that microlocal data, hence the comparison map is an isomorphism.
\item \textbf{Alternate proof track:} Use the deformation to the normal cone and show that (for generic $\ell$) iterated nearby cycles along $(t,\ell)$ commute; identify $\psi_\ell(\nu_0(F))$ with $\psi_t\psi_\ell$ of a two-parameter family.
\item \textbf{Disproof track:} Try to build a constructible $F$ whose nearby cycles along a generic $\ell$ detect higher-order radial variation not captured by $\nu_0(F)$. (No explicit candidate found.)
\end{itemize}

\subsection*{4) Work: partial result and precise first gap}
We can verify the statement in the easiest cases:

\begin{lemma}[Trivial case: $F$ locally constant near $0$]
If $F$ is locally constant in a neighborhood of $0\in E$, then for every linear form $\ell$ the morphism $i_0^*\psi_\ell(F)\to i_0^*\psi_\ell(\nu_0(F))$ is an isomorphism.
\end{lemma}

\begin{proof}
If $F$ is locally constant near $0$, then both $\psi_\ell(F)$ and $\phi_\ell(F)$ near $0$ identify with the restriction of $F$ to $\ell^{-1}(0)$ (the Milnor fiber carries no nontrivial monodromy for a locally constant sheaf on a neighborhood). Verdier specialization $\nu_0(F)$ agrees with $F$ on a conic neighborhood of $0$ in the tangent space, hence its nearby cycles along $\ell$ at $0$ match. Concretely, all objects involved are (canonically) $F|_{\{0\}}$.
\end{proof}

\paragraph{First real gap.}
To prove (\*), one needs a \emph{general} theorem identifying $i_0^*\psi_\ell(F)$ (or $i_0^*\phi_\ell(F)$) with a microlocal invariant at the covector $d\ell$, and then a theorem that specialization preserves that microlocal invariant for generic $\ell$.

\subsection*{5) Verification}
No counterexample was found by simple testing (constant sheaf, skyscraper sheaf, sheaf supported on a linear subspace). However, a complete proof requires importing nontrivial microlocal sheaf theory (Kashiwara--Schapira/Verdier), and I have not reconstructed a gap-free proof here.

\subsection*{6) FINAL}
\textbf{UNRESOLVED.}
(i) Strongest fully proved partial result: local-constancy near $0$ (above).\\
(ii) First gap: a complete microlocal identification showing the comparison map is an isomorphism for generic $\ell$.\\
(iii) Next moves: (1) explicitly cite/prove the relevant microlocal Morse/vanishing-cycle identification; (2) prove specialization preserves microlocal stalks away from the conormal directions; (3) assemble the genericity condition via stratified transversality on the blow-up.

