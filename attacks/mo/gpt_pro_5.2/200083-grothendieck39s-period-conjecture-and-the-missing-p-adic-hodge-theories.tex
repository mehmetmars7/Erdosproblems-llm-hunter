\section{Problem 5: ``Grothendieck's Period Conjecture and the missing $p$-adic Hodge theories''}

\subsection*{1) FORMAL RESTATEMENT}

\paragraph{Ambiguities.}
Several choices are implicit:
\begin{itemize}[nosep]
\item Which Tannakian category of ``motives'' over $\mathbb{Q}$ is being used (e.g.\ Nori, André, etc.)?
\item Which fiber functors precisely (pure motives, mixed motives, restricted by good reduction at $p$, etc.)?
\item Over which base ring/field is the isomorphism torsor considered (over $\Q_p$ vs over Fontaine period rings)?
\end{itemize}
Without fixing these, the ``scheme of isomorphisms'' is not uniquely determined.

\paragraph{Minimal corrected statement (general Tannakian torsor description).}
Let $\mathcal{T}$ be a neutral Tannakian category over a field $K$ (here $K=\Q_p$ is the intended base), and let $\omega_1,\omega_2:\mathcal{T}\to \mathrm{Vec}_K$ be two fiber functors. Then the functor
\[
A \longmapsto \mathrm{Isom}^{\otimes}(\omega_1\otimes_K A,\ \omega_2\otimes_K A)
\]
(on commutative $K$-algebras $A$) is representable by an affine $K$-scheme $P$, and $P$ is a (right) torsor under the affine group scheme $G=\mathrm{Aut}^{\otimes}(\omega_1)$.
Given one $A$-point of $P$, all other $A$-points are obtained by the (simply transitive) $G(A)$-action.

\subsection*{2) QUICK LITERATURE/CONTEXT CHECK}
The attached MathOverflow question frames this in terms of Grothendieck's period conjecture and its $p$-adic analogue, emphasizing that $p$-adic comparison yields a $B_{dR}$-point but not the generic point. The page snapshot consulted has 0 answers. Here I give a general (formal) description of points via torsor structure and an explicit computation in the simplest motivic subcategory (Tate motives).

\subsection*{3) ATTACK PLAN}
\begin{enumerate}[nosep]
\item Give a clean Tannakian torsor description of the scheme of tensor isomorphisms, which already answers ``what are other points'' in a formal sense.
\item Specialize to the Tate subcategory generated by $\Q(1)$, where the motivic Galois group is $\mathbb{G}_m$ and the torsor is $\mathbb{G}_m$, hence points are explicit units.
\item Explain why going beyond Tate motives amounts to describing $G(B_{dR})$ (highly non-explicit in general).
\end{enumerate}

\subsection*{4) WORK}

\paragraph{Lemma 4.6 (Torsor structure, formal description of points).}
Let $\mathcal{T}$ be a neutral Tannakian category over $K$, and $\omega_1,\omega_2$ fiber functors. Let $G=\mathrm{Aut}^{\otimes}(\omega_1)$. Assume $P:=\mathrm{Isom}^{\otimes}(\omega_1,\omega_2)$ is representable by an affine $K$-scheme. Then:
\begin{enumerate}[nosep]
\item For every commutative $K$-algebra $A$, the set $P(A)$ is the set of tensor isomorphisms $\omega_1\otimes A\to \omega_2\otimes A$.
\item $G$ acts on the right on $P$ by postcomposition: for $g\in G(A)$ and $\phi\in P(A)$, define $\phi\cdot g:=\phi\circ g$.
\item This action is \emph{simply transitive} on $A$-points whenever $P(A)\neq\emptyset$: if $\phi,\psi\in P(A)$ then $\psi=\phi\cdot g$ for the unique $g=\phi^{-1}\circ\psi\in G(A)$.
\end{enumerate}

\begin{proof}
(1) is the definition of the functor of points of $P$.

(2) Given $g\in G(A)=\mathrm{Aut}^{\otimes}(\omega_1\otimes A)$ and $\phi:\omega_1\otimes A\to\omega_2\otimes A$, the composite $\phi\circ g$ is again a tensor isomorphism, so defines an element of $P(A)$. Functoriality in $A$ is immediate.

(3) If $\phi,\psi\in P(A)$, define $g:=\phi^{-1}\circ \psi$. Then $g$ is an automorphism of $\omega_1\otimes A$, and since $\phi,\psi$ are tensor isomorphisms, so is $g$, hence $g\in G(A)$. Then $\phi\cdot g=\phi\circ(\phi^{-1}\circ\psi)=\psi$. Uniqueness follows because if $\phi\cdot g=\phi\cdot g'$ then composing with $\phi^{-1}$ gives $g=g'$.
\end{proof}

\paragraph{Explicit example: Tate motives.}
Consider the (hypothetical) Tannakian subcategory generated by the Tate object $\Q(1)$, whose Tannakian group is $\mathbb{G}_m$. In this case, $G=\mathbb{G}_m$ and any torsor under $G$ is (after choosing a basepoint) isomorphic to $\mathbb{G}_m$ itself. Concretely, if $A$ is a $K$-algebra, then
\[
P(A)\cong A^{\times}.
\]
Thus, once $p$-adic Hodge theory provides one $B_{dR}$-point (corresponding to Fontaine's $t\in B_{dR}^{\times}$), \emph{all} other $B_{dR}$-points in this Tate subcategory are obtained by multiplying by an arbitrary unit $u\in B_{dR}^{\times}$.

\paragraph{What is still ``missing'' in general.}
For a larger motivic category, this torsor description reduces the problem of describing $P(B_{dR})$ to describing the (typically enormous) group $G(B_{dR})$ and its action. Making this explicit is essentially as hard as describing the relevant motivic Galois group itself.

\subsection*{5) VERIFICATION}
\begin{itemize}[nosep]
\item Lemma 4.6 is a formal group-action computation; it does not depend on the particular motivic category.
\item The ``Tate motives'' example is conditional on standard expectations about the motivic Galois group of the Tate subcategory; the key algebraic point is that torsors under $\mathbb{G}_m$ have $A$-points equal to $A^\times$ after choosing a trivialization.
\item This does not solve the original question for the full category; it just explains the space of points as a torsor and gives an explicit family in a very small subcategory.
\end{itemize}

\subsection*{6) FINAL}
\textbf{UNRESOLVED}

\paragraph{(i) Strongest partial result.}
A general, rigorous description: the ``scheme of isomorphisms'' is a torsor under the motivic Galois group; once one point over a ring $A$ is known, all other $A$-points are obtained by the group action. In the Tate subcategory, this is completely explicit: points correspond to units.

\paragraph{(ii) First gap.}
Give a genuinely explicit description of $G(B_{dR})$ (or of the coordinate ring of $P$) for a nontrivial motivic category beyond Tate motives, in a way that produces concrete additional points not obtained by tautological torsor trivialization.

\paragraph{(iii) Top 3 next moves.}
\begin{enumerate}[nosep]
\item Restrict to a smaller, concrete subcategory (e.g.\ motives with good reduction at $p$ or mixed Tate motives over $\Q$) where $G$ is better understood, and compute $P(A)$ there.
\item Identify whether other Fontaine period rings ($B_{\mathrm{cris}},B_{\mathrm{st}},\dots$) provide additional natural points corresponding to more refined comparison isomorphisms.
\item Investigate whether the non-surjectivity of the Tannakian-group map corresponds to a describable quotient of $G$ whose torsor points can be parameterized explicitly.
\end{enumerate}

\paragraph{(iv) Minimal counterexample shape.}
If one hoped for a $p$-adic period conjecture analogous to the complex one, a ``counterexample'' would be a nontrivial algebraic relation in the ring of motivic periods that vanishes under the $p$-adic comparison point. The question suggests such relations exist because the comparison point is not generic.
