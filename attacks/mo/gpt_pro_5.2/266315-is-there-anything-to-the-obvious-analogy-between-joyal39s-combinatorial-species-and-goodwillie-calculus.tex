\section*{Problem 266315: Species and the left Kan extension formula}
\addcontentsline{toc}{section}{Problem 266315: Species and the left Kan extension formula}

\subsection*{1) FORMAL RESTATEMENT}
The MathOverflow post is phrased as an open-ended question about ``whether there is anything to'' an analogy, so it is \emph{not} a proposition with a definite truth value.
A concrete mathematical assertion \emph{explicitly stated in the post} is the following, which we take as the minimal well-posed statement to prove.

\begin{theorem}[Corrected statement]
Let $\mathbf{FinSet}_{\simeq}$ be the groupoid whose objects are finite sets and whose morphisms are bijections.
Let $\iota:\mathbf{FinSet}_{\simeq}\hookrightarrow \mathbf{Set}$ be the inclusion (same objects, viewed as sets).
Let $F:\mathbf{FinSet}_{\simeq}\to \mathbf{Set}$ be a (set-valued) species.
Then the left Kan extension $\mathrm{Lan}_{\iota}F:\mathbf{Set}\to \mathbf{Set}$ exists and is naturally isomorphic to the functor
\[
X\ \longmapsto\ \coprod_{n\ge 0} F([n])\times_{S_n} X^n,
\]
where $[n]=\{1,\dots,n\}$ (with $[0]=\varnothing$), $S_n=\mathrm{Aut}_{\mathbf{FinSet}_{\simeq}}([n])$,
$X^n=\mathrm{Hom}_{\mathbf{Set}}([n],X)$, and $F([n])\times_{S_n}X^n$ denotes the set of coinvariants for the diagonal action:
$(a,x)\sim(F(\sigma)(a),\,x\circ\sigma^{-1})$ for $\sigma\in S_n$.
\end{theorem}

\subsection*{2) QUICK LITERATURE/CONTEXT CHECK}
This is the standard ``power series'' formula for the left Kan extension of a species along the inclusion of finite sets into all sets; it is discussed in many introductions to Joyal species and analytic functors.

\subsection*{3) ATTACK PLAN}
Compute $\mathrm{Lan}_{\iota}F(X)$ from the definition as a colimit over the comma category $(\iota\downarrow X)$,
identify $(\iota\downarrow X)$ with a disjoint union of action groupoids $(S_n\curvearrowright X^n)$, and compute colimits over action groupoids as coinvariants.

\subsection*{4) WORK}

\begin{lemma}[Left Kan extension as a colimit over a comma category]
For every set $X$, the left Kan extension is given by
\[
(\mathrm{Lan}_{\iota}F)(X)\ \cong\ \mathrm{colim}\bigl((\iota\downarrow X)\xrightarrow{\pi}\mathbf{FinSet}_{\simeq}\xrightarrow{F}\mathbf{Set}\bigr),
\]
where $\pi$ sends an object $(I\xrightarrow{f}X)$ to $I$.
\end{lemma}

\begin{proof}
This is the defining pointwise formula for left Kan extension along $\iota$ in $\mathbf{Set}$ (which has all small colimits):
for each $X$ the value $(\mathrm{Lan}_{\iota}F)(X)$ is the colimit of the diagram $F\circ\pi$ indexed by the comma category $(\iota\downarrow X)$.
\end{proof}

\begin{lemma}[Decomposition of $(\iota\downarrow X)$ into action groupoids]
For any set $X$, there is an equivalence of groupoids
\[
(\iota\downarrow X)\ \simeq\ \coprod_{n\ge 0}\bigl(S_n\ltimes X^n\bigr),
\]
where $S_n\ltimes X^n$ is the action groupoid of the natural left action of $S_n$ on $X^n$ by permuting coordinates.
\end{lemma}

\begin{proof}
An object of $(\iota\downarrow X)$ is a pair $(I,f)$ with $I$ a finite set and $f:I\to X$ a function.
A morphism $(I,f)\to(J,g)$ is a bijection $\alpha:I\to J$ such that $g\circ\alpha=f$; hence $(\iota\downarrow X)$ is a groupoid.

Choose for each $n\ge 0$ a standard $n$-element set $[n]$.
Any finite set $I$ is (noncanonically) isomorphic to $[n]$ with $n=|I|$, so $(\iota\downarrow X)$ is equivalent to the full subgroupoid on objects of the form $([n],x)$ where $x:[n]\to X$.
Such an $x$ is exactly an element of $X^n$.

For fixed $n$, a morphism $([n],x)\to([n],y)$ is a bijection $\sigma:[n]\to[n]$ with $y\circ\sigma=x$, i.e.\ $y=x\circ\sigma^{-1}$.
Thus morphisms are in bijection with permutations $\sigma\in S_n$ sending $x$ to $y$ under the usual $S_n$-action on $X^n$.
This identifies the full subgroupoid on $n$-element objects with the action groupoid $S_n\ltimes X^n$.
Taking the disjoint union over $n$ gives the stated equivalence of groupoids.
\end{proof}

\begin{lemma}[Colimit over an action groupoid equals coinvariants]
Let $G$ be a group acting on a set $Y$ on the left.
Let $A$ be a set with a left $G$-action.
Consider the functor $D:G\ltimes Y\to\mathbf{Set}$ defined by $D(y)=A$ for all $y\in Y$ and $D(g:y\to gy)=g:A\to A$ for $g\in G$.
Then
\[
\mathrm{colim}_{G\ltimes Y} D\ \cong\ A\times_G Y := (A\times Y)/\!\sim
\]
where $(g\cdot a,y)\sim(a,g\cdot y)$ for all $g\in G$.
\end{lemma}

\begin{proof}
A colimit of a diagram in $\mathbf{Set}$ can be computed as the quotient of the disjoint union of the sets of objects by the equivalence relation generated by identifying $a\in D(y)$ with its image in $D(gy)$ under each morphism $y\to gy$.

Here $\coprod_{y\in Y}D(y)\cong \coprod_{y\in Y}A\cong A\times Y$ by sending an element $a\in D(y)$ to $(a,y)$.
For a morphism $g:y\to gy$, the relation identifies $(a,y)$ with $(g\cdot a, gy)$.
This is exactly the equivalence relation generated by $(a,g\cdot y)\sim(g\cdot a,y)$ (rename $(gy)$ as $(g\cdot y)$), which is equivalent to the usual balanced relation $(g\cdot a,y)\sim(a,g\cdot y)$.
Therefore the colimit is $(A\times Y)/\!\sim = A\times_G Y$.
\end{proof}

\begin{proof}[Proof of the theorem]
Fix a set $X$.
By the first lemma,
\[
(\mathrm{Lan}_{\iota}F)(X)\ \cong\ \mathrm{colim}_{(\iota\downarrow X)} F\circ\pi.
\]
By the second lemma, $(\iota\downarrow X)\simeq\coprod_{n\ge 0}(S_n\ltimes X^n)$.
A colimit over a disjoint union of categories is a disjoint union of colimits, hence
\[
(\mathrm{Lan}_{\iota}F)(X)\ \cong\ \coprod_{n\ge 0}\mathrm{colim}_{S_n\ltimes X^n} D_n,
\]
where $D_n$ is the restriction of $F\circ\pi$ to the $n$-th component.
On this component, the object $x\in X^n$ is sent to $F([n])$, and a morphism $\sigma:x\to \sigma\cdot x$ is sent to $F(\sigma):F([n])\to F([n])$.
Thus $D_n$ is of the form in the third lemma with $G=S_n$, $Y=X^n$, and $A=F([n])$ (with the induced action via $F$).
Therefore
\[
\mathrm{colim}_{S_n\ltimes X^n}D_n\ \cong\ F([n])\times_{S_n} X^n.
\]
Combining gives the desired formula
\[
(\mathrm{Lan}_{\iota}F)(X)\ \cong\ \coprod_{n\ge 0} F([n])\times_{S_n} X^n.
\]
Naturality in $X$ follows because all identifications were functorial: a map $f:X\to Y$ induces maps $f^n:X^n\to Y^n$ commuting with the $S_n$-actions, and hence induces maps on each coinvariant piece.
\end{proof}

\subsection*{5) VERIFICATION}
Edge case $n=0$: $X^0=\mathrm{Hom}(\varnothing,X)$ is a singleton and $S_0$ is trivial, so the $n=0$ summand is $F(\varnothing)$, as expected from the comma category.
All equivalences used are equivalences of \emph{groupoids}, so colimits are computed in $\mathbf{Set}$ with no hidden higher-categorical issues.

\subsection*{6) FINAL: \textbf{PROOF}}
The corrected statement above is proved in full in Section 4.

