\section{MO 86861: Defining $\mathbb{Z}$ in $\mathbb{Q}$}

\subsection*{1) FORMAL RESTATEMENT}

\paragraph{Ambiguity / misstatement.}
The prompt asks for an interpretation (“simplest geometric interpretation”), not a proposition. To give a proof in the required format, we formalize a precise equivalence between logical form and an algebro-geometric condition.

\paragraph{Conventions.}
We work in the first-order language of rings $\mathcal{L}_{\mathrm{ring}}=\{0,1,+,\cdot\}$.
A subset $S\subseteq\mathbb{Q}$ is \emph{definable} if there exists an $\mathcal{L}_{\mathrm{ring}}$-formula $\varphi(x)$ such that
\[
S=\{q\in\mathbb{Q}\mid \mathbb{Q}\models \varphi(q)\}.
\]
A \emph{$\Pi_2^+$-formula} (Poonen’s notation) is one of the form
\[
\forall y\ \exists z\ \Bigl( f(x,y,z)=0 \ \wedge\ g(x,y,z)\neq 0\Bigr),
\]
with $f,g\in\mathbb{Z}[x,y,z]$.

\paragraph{Corrected statement to prove (geometric translation).}
\begin{theorem}[$\Pi_2^+$ definability $\Leftrightarrow$ surjectivity on rational points of fibers]\label{thm:Pi2-geometric}
Let $S\subseteq\mathbb{Q}$ be defined by a $\Pi_2^+$-formula in one free variable $x$.
Then there exist affine varieties $V,W$ over $\mathbb{Q}$ and morphisms
\[
V \xrightarrow{\pi} W \xrightarrow{\rho} \mathbb{A}^1
\]
such that for every $t\in\mathbb{Q}$,
\[
t\in S\quad\Longleftrightarrow\quad \pi_t\colon V_t(\mathbb{Q})\to W_t(\mathbb{Q})\ \text{is surjective},
\]
where $V_t:=\rho^{-1}(t)\cap V$ and $W_t:=\rho^{-1}(t)\cap W$ are fibers over $t$, and $\pi_t$ is the restriction of $\pi$ to fibers.
Conversely, any such “fiberwise surjectivity on $\mathbb{Q}$-points” condition is definable by a $\Pi_2^+$-formula.

Moreover, a universal ($\forall$) definition of $S$ is equivalent to the complement $\mathbb{Q}\setminus S$ being \emph{diophantine} (existentially definable), i.e.\ the image of $U(\mathbb{Q})$ under a morphism $U\to \mathbb{A}^1$.
\end{theorem}

\subsection*{2) QUICK LITERATURE/CONTEXT CHECK (web available)}

Poonen explicitly restates his $\forall\exists$ definability theorem in geometric terms in the introduction of his paper, in essentially the form of Theorem \ref{thm:Pi2-geometric}.
Koenigsmann later proves a purely universal definition of $\mathbb{Z}$ in $\mathbb{Q}$.

\subsection*{3) ATTACK PLAN}

\begin{itemize}[leftmargin=2em]
\item Show how a $\Pi_2^+$ formula $\forall y\,\exists z\, (f=0\wedge g\neq 0)$ can be rewritten as a statement about existence of rational points on fibers of an explicitly constructed affine variety over $(x,y)$.
\item Convert the $g\neq 0$ condition into polynomial equations using an additional existential variable (over a field, $g\neq 0 \iff \exists u,\, gu=1$).
\item Package the resulting construction as a morphism $\pi:V\to W$ over $\mathbb{A}^1$ so that “$\forall y \exists z$” becomes surjectivity of $\pi_t$ on rational points of the fiber over $t$.
\item Do the converse: surjectivity on rational points translates back to a $\forall\exists$ sentence.
\end{itemize}

\subsection*{4) WORK (complete proof)}

\subsubsection*{Lemma: encoding $\neq 0$ by equations over $\mathbb{Q}$}

\begin{lemma}\label{lem:neq0-diophantine}
For $g\in\mathbb{Z}[x,y,z]$ and $(x,y,z)\in\mathbb{Q}^{1+m+n}$,
\[
g(x,y,z)\neq 0
\quad\Longleftrightarrow\quad
\exists u\in\mathbb{Q}\ \text{such that } g(x,y,z)\,u=1.
\]
\end{lemma}

\begin{proof}
If $g(x,y,z)\neq 0$ in the field $\mathbb{Q}$, then $u:=1/g(x,y,z)\in\mathbb{Q}$ satisfies $g(x,y,z)u=1$.
Conversely, if $g(x,y,z)u=1$, then $g(x,y,z)\neq 0$.
\end{proof}

\subsubsection*{From a $\Pi_2^+$ formula to varieties and fiberwise surjectivity}

\begin{proof}[Proof of Theorem \ref{thm:Pi2-geometric} (forward direction)]
Let $S\subseteq\mathbb{Q}$ be defined by a $\Pi_2^+$-formula:
\[
\varphi(x)\;:\;\forall y\ \exists z\ \bigl(f(x,y,z)=0 \wedge g(x,y,z)\neq 0\bigr),
\]
with $f,g\in \mathbb{Z}[x,y,z]$.

By Lemma \ref{lem:neq0-diophantine}, the condition $g(x,y,z)\neq 0$ is equivalent over $\mathbb{Q}$ to
\[
\exists u\in\mathbb{Q}\ \text{such that } g(x,y,z)u=1.
\]
Thus $\varphi(x)$ is equivalent (over $\mathbb{Q}$) to
\[
\forall y\ \exists (z,u)\ \bigl(f(x,y,z)=0 \wedge g(x,y,z)u-1=0\bigr).
\]

Now define affine varieties over $\mathbb{Q}$:
\begin{align*}
W &:= \mathbb{A}^{1+m} = \mathrm{Spec}\,\mathbb{Q}[x,y],\\
V &:= \mathrm{Spec}\,\mathbb{Q}[x,y,z,u]/(f(x,y,z),\,g(x,y,z)u-1).
\end{align*}
Let $\pi:V\to W$ be the morphism induced by the inclusion $\mathbb{Q}[x,y]\hookrightarrow \mathbb{Q}[x,y,z,u]/(\cdots)$, i.e.\ projection forgetting $(z,u)$.
Let $\rho:W\to \mathbb{A}^1$ be the projection to the $x$-coordinate.

For $t\in\mathbb{Q}$, the fiber $W_t(\mathbb{Q})$ identifies with $\{t\}\times \mathbb{Q}^m$ (all $y$ values).
The fiber $V_t(\mathbb{Q})$ consists of tuples $(t,y,z,u)$ satisfying the two equations.
The induced map $\pi_t:V_t(\mathbb{Q})\to W_t(\mathbb{Q})$ is exactly $(t,y,z,u)\mapsto (t,y)$.

Therefore:
\begin{itemize}[leftmargin=2em]
\item $\pi_t$ is surjective on $\mathbb{Q}$-points iff for every $y\in\mathbb{Q}^m$ there exists $(z,u)$ such that $(t,y,z,u)\in V(\mathbb{Q})$,
\item which holds iff $\mathbb{Q}\models \forall y\,\exists(z,u)\ (f=0\wedge gu=1)$ at $x=t$,
\item which is equivalent to $\mathbb{Q}\models \varphi(t)$.
\end{itemize}
Thus $t\in S \iff \pi_t$ is surjective, as claimed.
\end{proof}

\subsubsection*{Conversely: surjectivity on rational points gives a $\Pi_2^+$ formula}

\begin{proof}[Proof of Theorem \ref{thm:Pi2-geometric} (reverse direction)]
Conversely, suppose we are given affine varieties $V,W$ and morphisms $V\xrightarrow{\pi}W\xrightarrow{\rho}\mathbb{A}^1$ such that $S=\{t\in\mathbb{Q}: \pi_t\text{ is surjective on }\mathbb{Q}\text{-points}\}$.
Choose affine embeddings $W\hookrightarrow \mathbb{A}^{1+m}$ and $V\hookrightarrow \mathbb{A}^{1+m+n}$ over $\mathbb{Q}$ so that $\rho$ is the restriction of the projection to the first coordinate and $\pi$ is induced by forgetting the last $n$ coordinates.
Write $W$ and $V$ by polynomial equations:
\[
W:\ F_i(x,y)=0,\qquad
V:\ F_i(x,y)=0,\ G_j(x,y,z)=0,
\]
for collections of polynomials with rational coefficients (clear denominators to make them integral if desired).

Then for $t\in\mathbb{Q}$, surjectivity of $\pi_t$ means:
\[
\forall y\in\mathbb{Q}^m,\ \Bigl(\bigwedge_i F_i(t,y)=0\Bigr)\ \Rightarrow\ \exists z\in\mathbb{Q}^n,\ \bigwedge_j G_j(t,y,z)=0.
\]
Using that implication $P\Rightarrow Q$ is equivalent to $(\neg P)\vee Q$, and that over a field $\neg(P=0)$ can be expressed existentially by introducing an inverse as in Lemma \ref{lem:neq0-diophantine}, one can rewrite this into a $\Pi_2^+$ formula defining $S$.
\end{proof}

\subsubsection*{Universal formulas and diophantine complements (geometric interpretation)}
A universal definition of $S$ has the form $\forall y\,\psi(x,y)$ with quantifier-free $\psi$.
Then its complement is defined by $\exists y\,\neg\psi(x,y)$.
Over $\mathbb{Q}$, negations of atomic equations can be eliminated by Lemma \ref{lem:neq0-diophantine}, so $\mathbb{Q}\setminus S$ becomes existentially definable by polynomial equations, i.e.\ diophantine.
Geometrically, this means $\mathbb{Q}\setminus S$ is the image on $\mathbb{Q}$-points of an affine variety over $\mathbb{A}^1$ under the projection to the $x$-coordinate.

\subsection*{5) VERIFICATION}

\begin{itemize}[leftmargin=2em]
\item Quantifiers match exactly: $\forall y$ corresponds to “for every rational point in $W_t(\mathbb{Q})$” and $\exists z$ corresponds to “there exists a rational point in the fiber above it”.
\item The encoding of $\neq 0$ by $gu=1$ is valid precisely because we work over the field $\mathbb{Q}$.
\item No elimination of quantifiers in $\mathbb{Q}$ is assumed; we only used elementary logical equivalences and one field fact (Lemma \ref{lem:neq0-diophantine}).
\end{itemize}

\subsection*{6) FINAL}

\textbf{PROOF.}

\noindent\textbf{Clean theorem statement.}
Theorem \ref{thm:Pi2-geometric} gives a precise and fully proved “geometric interpretation” of $\forall\exists$ (and $\forall$) definability statements:
a $\Pi_2^+$ definition of $\mathbb{Z}$ in $\mathbb{Q}$ is equivalent to the existence of a morphism of varieties whose fibers have the property that rational points surject (Poonen explicitly formulates his result this way), and a universal definition is equivalent to the complement being an image of rational points of a variety (a diophantine set).
