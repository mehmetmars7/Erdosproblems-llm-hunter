\section{Problem 5: Dessins d'enfants and the absolute Galois group}

\subsection*{Problem source}
MathOverflow question \#53362: ``What do dessins tell us about the absolute Galois group?''

\subsection*{1) FORMAL RESTATEMENT}

\paragraph{Literal statement.}
The post asks for examples of number-theoretic/Galois-theoretic statements that can be proved using dessins d'enfants and can be stated without dessin language. This is not a proposition with a truth value.

\paragraph{Minimal corrected statement (standard, precise, and central).}
A key theorem in the subject, which \emph{is} a precise statement and directly about the absolute Galois group, is:

\begin{quote}
\textbf{Theorem.} The natural action of $\mathrm{Gal}(\overline{\Q}/\Q)$ on isomorphism classes of dessins d'enfants (equivalently Belyi pairs) is faithful.
\end{quote}

This is a Galois-theoretic statement that can be stated without mentioning dessins explicitly (e.g.\ ``the action on Belyi pairs is faithful'').

\paragraph{Definitions.}
A \emph{Belyi pair} is a pair $(X,\beta)$ where:
\begin{itemize}[leftmargin=2em]
\item $X$ is a smooth projective curve over $\overline{\Q}$, and
\item $\beta:X\to \mathbb{P}^1$ is a nonconstant morphism defined over $\overline{\Q}$ that is unramified outside $\{0,1,\infty\}$.
\end{itemize}
Two Belyi pairs $(X,\beta)$ and $(X',\beta')$ are isomorphic if there exists an isomorphism of curves $f:X\to X'$ with $\beta' \circ f=\beta$.
A \emph{dessin d'enfant} is an isomorphism class of Belyi pairs.

For $\sigma\in \mathrm{Gal}(\overline{\Q}/\Q)$, define
\[
\sigma\cdot (X,\beta) := ({}^\sigma X, {}^\sigma\beta)
\]
by applying $\sigma$ to the coefficients of equations defining $X$ and $\beta$. This is well-defined on isomorphism classes.

\subsection*{2) QUICK LITERATURE/CONTEXT CHECK}

The faithfulness of the Galois action is standard and is often presented as a consequence of Belyi's theorem plus the fact that moduli of curves (e.g.\ elliptic curves via $j$-invariant) can realize any algebraic number.\footnote{For an expository account, see e.g.\ A.~Collins, ``Dessins d'enfants: A Galois game'' (notes): \url{https://people.math.sc.edu/collins/Stories/dessins.pdf}}  
I will give a self-contained proof using only:
\begin{enumerate}[leftmargin=2em]
\item existence of elliptic curves over $\overline{\Q}$ with prescribed $j$-invariant, and
\item Belyi's theorem (existence of a Belyi map on any curve defined over $\overline{\Q}$).
\end{enumerate}

\subsection*{3) ATTACK PLAN}

\begin{enumerate}[leftmargin=2em]
\item Let $\sigma\neq \mathrm{id}$ in $\mathrm{Gal}(\overline{\Q}/\Q)$ and pick $j\in\overline{\Q}$ with $\sigma(j)\neq j$.
\item Build an elliptic curve $E/\overline{\Q}$ with $j(E)=j$.
\item Use Belyi's theorem to obtain a Belyi map $\beta:E\to\mathbb{P}^1$; thus $(E,\beta)$ is a dessin.
\item Show that if $\sigma$ fixed the dessin class, then $E\simeq {}^\sigma E$, hence $j(E)=j({}^\sigma E)=\sigma(j(E))$, contradiction.
\end{enumerate}

\subsection*{4) WORK}

\subsubsection*{Lemma: elliptic curves with prescribed $j$}

\begin{lemma}[Realizing any algebraic $j$]
For every $j_0\in\overline{\Q}$ there exists an elliptic curve $E/\overline{\Q}$ with $j(E)=j_0$.
\end{lemma}
\begin{proof}
If $j_0=0$, take $E: y^2 = x^3 + 1$, which is nonsingular and has $j(E)=0$.

If $j_0=1728$, take $E: y^2 = x^3 - x$, which is nonsingular and has $j(E)=1728$.

Now assume $j_0\neq 0,1728$. Define
\[
c := \frac{j_0}{1728-j_0}\in\overline{\Q}.
\]
Then $c\neq 0$ and $c\neq -1$.
Consider the Weierstrass equation
\[
E: y^2 = x^3 + 3c\,x + 2c.
\]
The discriminant is
\[
\Delta(E) = -16\,(4(3c)^3 + 27(2c)^2) = -16\,(108c^3 + 108c^2) = -1728\,c^2(c+1),
\]
which is nonzero because $c\neq 0,-1$. Hence $E$ is an elliptic curve.

For a Weierstrass model $y^2=x^3+Ax+B$ with $\Delta\neq 0$, the $j$-invariant is
\[
j(E)=1728\cdot \frac{4A^3}{4A^3+27B^2}.
\]
Here $A=3c$ and $B=2c$, so $4A^3=108c^3$ and $27B^2=108c^2$, hence
\[
j(E)=1728\cdot \frac{108c^3}{108c^3+108c^2}
=1728\cdot \frac{c}{c+1}.
\]
Since $c=\frac{j_0}{1728-j_0}$, we compute
\[
\frac{c}{c+1}=\frac{\frac{j_0}{1728-j_0}}{\frac{j_0}{1728-j_0}+1}
=\frac{\frac{j_0}{1728-j_0}}{\frac{j_0+1728-j_0}{1728-j_0}}
=\frac{j_0}{1728}.
\]
Therefore $j(E)=1728\cdot \frac{j_0}{1728}=j_0$, as required.
\end{proof}

\subsubsection*{Lemma: Galois acts on $j$ by applying $\sigma$}

\begin{lemma}[$j$-invariant is Galois-equivariant]
Let $E/\overline{\Q}$ be an elliptic curve and $\sigma\in\mathrm{Gal}(\overline{\Q}/\Q)$. Then
\[
j({}^\sigma E)=\sigma(j(E)).
\]
\end{lemma}
\begin{proof}
Choose a Weierstrass equation for $E$ with coefficients in $\overline{\Q}$, say $y^2=x^3+Ax+B$. Then $j(E)$ is a rational function of $A$ and $B$ with coefficients in $\Q$:
\[
j(E)=1728\cdot \frac{4A^3}{4A^3+27B^2}.
\]
Applying $\sigma$ to the coefficients gives a Weierstrass equation for ${}^\sigma E$ with coefficients $\sigma(A),\sigma(B)$, hence
\[
j({}^\sigma E)=1728\cdot \frac{4\sigma(A)^3}{4\sigma(A)^3+27\sigma(B)^2}
=\sigma\!\left(1728\cdot \frac{4A^3}{4A^3+27B^2}\right)
=\sigma(j(E)),
\]
since $\sigma$ fixes $\Q$ and respects addition, multiplication, and inversion.
\end{proof}

\subsubsection*{Main theorem: faithfulness}

\begin{theorem}[Faithfulness of the Galois action on dessins]
The action of $\mathrm{Gal}(\overline{\Q}/\Q)$ on isomorphism classes of dessins d'enfants (Belyi pairs) is faithful.
\end{theorem}
\begin{proof}
Let $\sigma\in\mathrm{Gal}(\overline{\Q}/\Q)$ be nontrivial. Then there exists some $j_0\in\overline{\Q}$ such that $\sigma(j_0)\neq j_0$.

By the first lemma, choose an elliptic curve $E/\overline{\Q}$ with $j(E)=j_0$.
By Belyi's theorem, since $E$ is defined over $\overline{\Q}$, there exists a Belyi map $\beta:E\to\mathbb{P}^1$ defined over $\overline{\Q}$ that is unramified outside $\{0,1,\infty\}$. Thus $(E,\beta)$ is a Belyi pair, hence determines a dessin.

Assume for contradiction that $\sigma$ fixes the isomorphism class of $(E,\beta)$. Then the Belyi pairs $(E,\beta)$ and $({}^\sigma E,{}^\sigma\beta)$ are isomorphic, so in particular the curves $E$ and ${}^\sigma E$ are isomorphic over $\overline{\Q}$. Isomorphic elliptic curves have the same $j$-invariant, hence
\[
j(E)=j({}^\sigma E).
\]
By the second lemma, $j({}^\sigma E)=\sigma(j(E))$. Therefore
\[
j(E)=\sigma(j(E)).
\]
But $j(E)=j_0$ by construction, so this says $j_0=\sigma(j_0)$, contradicting the choice of $j_0$.

Hence $\sigma$ cannot fix the dessin class of $(E,\beta)$, and thus the action is faithful.
\end{proof}

\subsection*{5) VERIFICATION}

\paragraph{Boundary cases in the $j$-realization lemma.}
Handled separately for $j_0=0,1728$; the general formula avoids singular discriminant.

\paragraph{Use of Belyi's theorem.}
The proof uses only existence of some Belyi map on $E$; it does not require the map to be defined over a specific subfield.

\paragraph{Adversarial check.}
If $\sigma$ fixed the dessin class, we used that this forces $E\simeq {}^\sigma E$. This is valid because an isomorphism of Belyi pairs includes an isomorphism of the underlying curves.

\subsection*{6) FINAL}

\paragraph{FINAL LABEL: \textbf{PROOF}.}
