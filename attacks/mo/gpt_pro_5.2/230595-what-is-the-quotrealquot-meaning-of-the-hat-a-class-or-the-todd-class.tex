\section{What is the ``real'' meaning of the $\hat A$ class (or the Todd class)? (MO 230595)}

\subsection*{1) FORMAL RESTATEMENT}
\paragraph{Ambiguity.} The original post asks for conceptual connections among several constructions; it is not a single theorem to prove.

\paragraph{Minimal corrected statement (precise and provable).}
We formalize the ``failure of commutativity'' description as the following standard theorem.

\begin{theorem}[Todd class as the correction between $K$-theory and cohomology Thom classes]
\label{thm:todd-thom}
Let $E\to B$ be a complex vector bundle of complex rank $n$ over a finite CW complex $B$.
Let
\begin{itemize}[leftmargin=2em]
\item $u_K(E)\in K^0(E,E\setminus B)$ be the complex $K$-theory Thom class (Bott class), so cup product with $u_K(E)$ gives the Thom isomorphism in complex $K$-theory;
\item $u_H(E)\in H^{2n}(E,E\setminus B;\QQ)$ be the (ordinary) cohomological Thom class, so cup product with $u_H(E)$ gives the Thom isomorphism in rational cohomology;
\item $\mathrm{ch}:K^0(-)\to H^{\mathrm{ev}}(-;\QQ)$ be the Chern character.
\end{itemize}
Then there exists a unique class $\mathrm{Td}(E)\in H^{\mathrm{ev}}(B;\QQ)$ (the Todd class) such that
\begin{equation}
\label{eq:todd-relation}
\mathrm{ch}\big(u_K(E)\big)=\pi^*\big(\mathrm{Td}(E)^{-1}\big)\smile u_H(E)\qquad\text{in }H^{\mathrm{ev}}(E,E\setminus B;\QQ),
\end{equation}
where $\pi:E\to B$ is the bundle projection.
Equivalently, for all $x\in K^0(B)$,
\begin{equation}
\label{eq:todd-commutativity}
\mathrm{ch}\big(x\smile u_K(E)\big)=\pi^*\big(\mathrm{ch}(x)\smile \mathrm{Td}(E)^{-1}\big)\smile u_H(E).
\end{equation}
\end{theorem}

\subsection*{2) QUICK LITERATURE/CONTEXT CHECK}
A statement equivalent to Theorem~\ref{thm:todd-thom} is widely cited as the conceptual origin of the Todd class in GRR; e.g. the nLab page ``rational Todd class is Chern character of Thom class'' states this as a proposition. \href{https://ncatlab.org/nlab/show/rational%2BTodd%2Bclass%2Bis%2BChern%2Bcharacter%2Bof%2BThom%2Bclass}{(nLab page)}.

\subsection*{3) ATTACK PLAN}
\begin{enumerate}[leftmargin=2em]
\item Prove the theorem for a complex line bundle $L$ by an explicit calculation of the restrictions of Thom classes to the zero section and the defining power series $x/(1-e^{-x})$.
\item Extend to general bundles by the splitting principle, using multiplicativity of Thom classes and the Chern character.
\end{enumerate}

\subsection*{4) WORK}

\subsubsection*{Preliminaries: Thom classes and their restriction}
Let $s:B\to E$ denote the zero section.
For ordinary cohomology: the Thom class $u_H(E)\in H^{2n}(E,E\setminus B;\QQ)$ is characterized by the property that the pullback $s^*u_H(E)$ equals the top Chern class $c_n(E)\in H^{2n}(B;\QQ)$.
For complex $K$-theory: the Thom class $u_K(E)\in K^0(E,E\setminus B)$ is characterized by $s^*u_K(E)=\lambda_{-1}(E^*)\in K^0(B)$, where $\lambda_{-1}(V)=\sum_{j=0}^{\mathrm{rk}V}(-1)^j[\Lambda^j V]$.

We will use the standard facts:
\begin{itemize}[leftmargin=2em]
\item The Chern character is a ring map $\mathrm{ch}:K^0(B)\to H^{\mathrm{ev}}(B;\QQ)$.
\item For a complex line bundle $L$ with first Chern class $x=c_1(L)\in H^2(B;\ZZ)$, one has $\mathrm{ch}(L)=e^{x}$ and $\mathrm{ch}(L^{-1})=e^{-x}$.
\item $\lambda_{-1}$ is multiplicative: $\lambda_{-1}((E\oplus F)^*)=\lambda_{-1}(E^*)\cdot\lambda_{-1}(F^*)$.
\item Thom classes are multiplicative under direct sums: $u_H(E\oplus F)=u_H(E)\smile u_H(F)$ and $u_K(E\oplus F)=u_K(E)\smile u_K(F)$ under the usual identifications.
\end{itemize}

\subsubsection*{Line bundle case}
\begin{lemma}[Theorem~\ref{thm:todd-thom} for complex line bundles]
\label{lem:line}
Let $L\to B$ be a complex line bundle with $x=c_1(L)\in H^2(B;\ZZ)$. Then
\[\mathrm{ch}\big(u_K(L)\big)=\pi^*\Big(\frac{1-e^{-x}}{x}\Big)\smile u_H(L)
\quad\text{in }H^{\mathrm{ev}}(L,L\setminus B;\QQ).
\]
Equivalently, $\mathrm{Td}(L)=\dfrac{x}{1-e^{-x}}$.
\end{lemma}

\begin{proof}
Since the relative cohomology group $H^{\mathrm{ev}}(L,L\setminus B;\QQ)$ is a free rank-one module over $H^{\mathrm{ev}}(B;\QQ)$ generated by the Thom class $u_H(L)$, there exists a unique class $\alpha\in H^{\mathrm{ev}}(B;\QQ)$ such that
\[\mathrm{ch}\big(u_K(L)\big)=\pi^*(\alpha)\smile u_H(L).
\]
Pull back along the zero section $s$ and use $s^*\pi^*=\mathrm{id}$ to get
\[\mathrm{ch}\big(s^*u_K(L)\big)=\alpha\smile s^*u_H(L).
\]
Now $s^*u_H(L)=c_1(L)=x$ by definition of the (cohomological) Thom class.
Also $s^*u_K(L)=\lambda_{-1}(L^*)=1-[L^{-1}]$ in $K^0(B)$ for a line bundle.
Applying the Chern character gives
\[\mathrm{ch}(1-[L^{-1}])=1-e^{-x}.
\]
Hence $1-e^{-x}=\alpha\,x$, i.e. $\alpha=(1-e^{-x})/x$.
This is exactly the claimed formula.
\end{proof}

\subsubsection*{General bundle via splitting principle}
\begin{proof}[Proof of Theorem~\ref{thm:todd-thom}]
By the splitting principle, there exists a map $f:B'\to B$ such that:
\begin{enumerate}[leftmargin=2em]
\item $f^*:H^*(B;\QQ)\to H^*(B';\QQ)$ is injective;
\item $f^*E\cong L_1\oplus\cdots\oplus L_n$ splits as a direct sum of complex line bundles on $B'$.
\end{enumerate}
Let $\pi':f^*E\to B'$ be the pulled-back bundle. Naturality of Thom classes and of the Chern character implies that the desired relation \eqref{eq:todd-relation} for $E$ pulls back to the analogous relation for $f^*E$.

Using multiplicativity of Thom classes and of $\lambda_{-1}$, together with Lemma~\ref{lem:line}, we compute in $H^{\mathrm{ev}}(f^*E, f^*E\setminus B';\QQ)$:
\begin{align*}
\mathrm{ch}\big(u_K(f^*E)\big)
&=\mathrm{ch}\Big(\prod_{i=1}^n u_K(L_i)\Big)
=\prod_{i=1}^n \mathrm{ch}\big(u_K(L_i)\big)\\
&=\prod_{i=1}^n \Big(\pi'^*\Big(\frac{1-e^{-x_i}}{x_i}\Big)\smile u_H(L_i)\Big)\\
&=\pi'^*\Big(\prod_{i=1}^n\frac{1-e^{-x_i}}{x_i}\Big)\smile \prod_{i=1}^n u_H(L_i)\\
&=\pi'^*\Big(\prod_{i=1}^n\frac{1-e^{-x_i}}{x_i}\Big)\smile u_H(f^*E),
\end{align*}
where $x_i=c_1(L_i)$.
Define
\[\mathrm{Td}(f^*E):=\prod_{i=1}^n\frac{x_i}{1-e^{-x_i}}\in H^{\mathrm{ev}}(B';\QQ).
\]
Then the previous identity reads
\[\mathrm{ch}\big(u_K(f^*E)\big)=\pi'^*\big(\mathrm{Td}(f^*E)^{-1}\big)\smile u_H(f^*E).
\]
Since $f^*$ is injective on rational cohomology and the above expression for $\mathrm{Td}(f^*E)$ is symmetric in the Chern roots, it comes from a unique class $\mathrm{Td}(E)\in H^{\mathrm{ev}}(B;\QQ)$ with $f^*\mathrm{Td}(E)=\mathrm{Td}(f^*E)$.
Pushing the equality back down by injectivity yields \eqref{eq:todd-relation} for $E$.
The equivalent formula \eqref{eq:todd-commutativity} follows by multiplying both sides by $\mathrm{ch}(x)$ and using that $\mathrm{ch}$ is a ring map.
\end{proof}

\subsection*{5) VERIFICATION}
\paragraph{Edge cases.}
If $n=0$ (the zero bundle), then $u_K=u_H=1$ and $\mathrm{Td}(0)=1$, so \eqref{eq:todd-relation} is tautological. For a line bundle, Lemma~\ref{lem:line} gives the explicit factor.

\paragraph{No hidden hypotheses.}
We used only: existence and naturality of Thom classes in complex $K$-theory, the ring-homomorphism property of the Chern character, multiplicativity of Thom classes/direct sums, and the splitting principle with rational injectivity.

\subsection*{6) FINAL}
\textbf{PROOF.}
