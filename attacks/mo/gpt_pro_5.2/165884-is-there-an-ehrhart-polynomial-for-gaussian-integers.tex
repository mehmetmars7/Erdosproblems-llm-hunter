\section{(MO 165884) Is there an Ehrhart polynomial for Gaussian integers?}

\subsection{1) FORMAL RESTATEMENT}
We fix:
\begin{itemize}
\item Identify $\mathbb C\cong\mathbb R^2$ with lattice $\mathbb Z[i]\cong\mathbb Z^2$.
\item Let $N\in\mathbb Z_{>0}$ and let $P\subset\mathbb C$ be a (nondegenerate) polygon with vertices in $\frac1N\mathbb Z[i]$.
\item For $z=u+vi\in\mathbb Z[i]$ define
\[
h_P(z) := \#\bigl(zP\cap \mathbb Z[i]\bigr),
\]
counting lattice points on the boundary as well.
\end{itemize}

The strongest explicit statement suggested in the excerpt is:

\begin{quote}
(\ensuremath{\dagger})\quad There exists a nonzero $D\in\mathbb Z[i]$ such that for all $z=u+vi\in\mathbb Z[i]$ with $\gcd(u,v)=1$ and $\gcd(D,z)=1$, one has
\[
h_P(u+vi) = (u^2+v^2)\operatorname{Area}(P)\ +\ a(u,v)u + b(u,v)v + c(u,v),
\]
where $a,b,c\colon \mathbb Z^2\to\mathbb Q$ are periodic in each variable modulo $N$.
\end{quote}

Stress point: for fixed congruence class $(u\bmod N, v\bmod N)$, the functions $a,b,c$ become constants, so (\ensuremath{\dagger}) asserts a \emph{single affine-linear correction} to the quadratic area term on each residue class.

\subsection{2) QUICK LITERATURE/CONTEXT CHECK}
General parametric lattice-point counting yields piecewise quasi-polynomials in parameters, but the specific global form (\ensuremath{\dagger}) is stronger.

I will disprove (\ensuremath{\dagger}) by an explicit polygon $P$ with $N=2$.

\subsection{3) ATTACK PLAN}
\paragraph{Disproof strategy.}
\begin{enumerate}
\item Choose a very simple polygon with denominator $N=2$.
\item Restrict to a single residue class $(u\bmod 2, v\bmod 2)=(1,0)$ so that periodicity forces $a,b,c$ to be constants.
\item Compute $h_P(1+2ki)$ exactly for all $k$, obtaining a formula depending on $k\bmod 2$.
\item Show that no fixed constants $a,b,c$ can match these values on infinitely many $k$; and that no finite ``exclusion'' $D$ can remove all offending $k$.
\end{enumerate}

\subsection{4) WORK (explicit counterexample and verification)}

\subsubsection*{Step 1: choose $N$ and $P$}
Let $N=2$ and let $P\subset\mathbb C$ be the right triangle
\[
P := \operatorname{conv}\Bigl\{\,0,\ \tfrac12,\ \tfrac{i}{2}\Bigr\}.
\]
Then $\operatorname{Area}(P)=\frac18$.

For $z=u+vi$, the area scales by $|z|^2=u^2+v^2$, so the leading term in (\ensuremath{\dagger}) is
\[
(u^2+v^2)\operatorname{Area}(P)=\frac{u^2+v^2}{8}.
\]

\subsubsection*{Step 2: compute $h_P(1+2ki)$ exactly}
Fix $k\in\mathbb Z_{\ge 0}$ and set $z_k:=1+2ki$.
Then $z_kP$ is the triangle with vertices
\[
A=(0,0),\quad B=\Bigl(\tfrac12,k\Bigr),\quad C=\Bigl(-k,\tfrac12\Bigr)
\]
in $\mathbb R^2$ coordinates.

\begin{lemma}\label{lem:count}
For every integer $k\ge 0$,
\[
h_P(1+2ki)= 1+\Bigl\lfloor \frac{k^2}{2}\Bigr\rfloor.
\]
Equivalently, if $k=2t$ then $h_P(1+4ti)=2t^2+1$, and if $k=2t+1$ then $h_P(1+(4t+2)i)=2t^2+2t+1$.
\end{lemma}

\begin{proof}
For $k=0$, $z_0P=P$ contains exactly the one lattice point $0$, so $h_P(1)=1$ and the formula holds.

Assume $k\ge 1$.
Since the maximal $x$-coordinate of the triangle is $1/2$, any integer lattice point $(x,y)\in z_kP\cap\mathbb Z^2$ satisfies $x\le 0$.

\medskip
\noindent\textbf{(i) The boundary line $BC$.}
A direct two-point computation shows the line through $B=(1/2,k)$ and $C=(-k,1/2)$ has equation
\[
2(2k+1)\,y \;=\; 2(2k-1)\,x \;+\; (4k^2+1).
\]
The triangle $z_kP$ lies on the side of this line containing $A=(0,0)$, hence integer points in the triangle satisfy
\begin{equation}\label{eq:BCineq}
2(2k+1)\,y \;\le\; 2(2k-1)\,x \;+\; (4k^2+1).
\end{equation}

\medskip
\noindent\textbf{(ii) The boundary line $AC$.}
The line through $A=(0,0)$ and $C=(-k,1/2)$ has equation $x=-2ky$.
The triangle lies to the right of this line (since $B$ has $x>0$), so integer points satisfy
\begin{equation}\label{eq:ACineq}
x \;\ge\; -2k y.
\end{equation}

\medskip
\noindent\textbf{(iii) Count lattice points by vertical lines $x=0,-1,\dots,-(k-1)$.}

\emph{The line $x=0$.}
Plugging $x=0$ into \eqref{eq:BCineq} gives
\[
2(2k+1)\,y \le 4k^2+1 \quad\Longleftrightarrow\quad y \le \frac{4k^2+1}{4k+2} = k - \frac{2k-1}{4k+2}.
\]
The right-hand side lies in $(k-1,k)$, so $\lfloor \frac{4k^2+1}{4k+2}\rfloor = k-1$.
Thus for $x=0$ we have integer solutions $y=0,1,\dots,k-1$, giving exactly $k$ lattice points.

\emph{The line $x=-j$ with $1\le j\le k-1$.}
From \eqref{eq:ACineq}, $-j\ge -2ky$ so $y\ge \frac{j}{2k}$.
Since $1\le j\le k-1$, we have $0<\frac{j}{2k}<1$, hence $y\ge 1$ for integer $y$.

From \eqref{eq:BCineq} with $x=-j$ we get
\[
2(2k+1)y \le 4k^2+1 - 2(2k-1)j,
\]
so
\[
y \le \frac{4k^2+1 - (4k-2)j}{4k+2}.
\]
Define
\[
Y_j := \Bigl\lfloor \frac{4k^2+1 - (4k-2)j}{4k+2}\Bigr\rfloor.
\]
Then the integer points on the line $x=-j$ inside the triangle are exactly those with $1\le y\le Y_j$; hence this line contributes $Y_j$ points (note $Y_j\ge 0$ for $j\le k-1$).

Now rewrite the fraction:
\[
\frac{4k^2+1 - (4k-2)j}{4k+2}
= (k-j) + \frac{4j-2k+1}{4k+2}.
\]
Since $1\le j\le k-1$, the last fraction lies in $(-1,1)$, so its floor is either $-1$ (if negative) or $0$ (if nonnegative). Therefore
\[
Y_j = (k-j) + \varepsilon_j
\quad\text{where}\quad
\varepsilon_j = \begin{cases}
-1,& 4j-2k+1<0,\\
0,& 4j-2k+1\ge 0.
\end{cases}
\]
The inequality $4j-2k+1<0$ is equivalent to $j < \frac{2k-1}{4}= \frac{k}{2}-\frac14$.
Thus:
\begin{itemize}
\item if $k=2t$ is even, then $\varepsilon_j=-1$ exactly for $1\le j\le t-1$ (there are $t-1$ such $j$);
\item if $k=2t+1$ is odd, then $\varepsilon_j=-1$ exactly for $1\le j\le t$ (there are $t$ such $j$).
\end{itemize}

\medskip
\noindent\textbf{(iv) Sum.}
The total number of lattice points is:
\[
h_P(1+2ki)=
\underbrace{k}_{x=0}
\;+\;
\sum_{j=1}^{k-1} Y_j
=
k + \sum_{j=1}^{k-1}(k-j) + \sum_{j=1}^{k-1}\varepsilon_j.
\]
We have $\sum_{j=1}^{k-1}(k-j)=\sum_{t=1}^{k-1} t = \frac{k(k-1)}{2}$.

For the $\varepsilon_j$ sum:
\[
\sum_{j=1}^{k-1}\varepsilon_j=
\begin{cases}
-(t-1),& k=2t,\\
-t,& k=2t+1.
\end{cases}
\]
Therefore:
\[
h_P(1+2ki)=
\begin{cases}
2t + \frac{(2t)(2t-1)}{2} -(t-1)= 2t^2+1,& k=2t,\\[4pt]
(2t+1)+\frac{(2t+1)(2t)}{2}-t = 2t^2+2t+1,& k=2t+1.
\end{cases}
\]
One checks that both cases equal $1+\lfloor k^2/2\rfloor$, completing the proof.
\end{proof}

\subsubsection*{Step 3: show the proposed quasi-polynomial form (\ensuremath{\dagger}) cannot hold}
Assume for contradiction that (\ensuremath{\dagger}) holds for this $N=2$ and this $P$, with some nonzero $D\in\mathbb Z[i]$ and some $2$-periodic coefficient functions $a,b,c$.

Consider the residue class $(u\bmod 2,v\bmod 2)=(1,0)$. For all $z=1+2ki$ (all $k$), we have $\gcd(1,2k)=1$ and $(u,v)\equiv(1,0)\pmod 2$, so periodicity forces
\[
a(1,2k)=:a_{10},\quad b(1,2k)=:b_{10},\quad c(1,2k)=:c_{10}
\]
to be constants independent of $k$.

We now show that no choice of $D$ can make the formula valid for all admissible $k$.

\begin{lemma}\label{lem:choosek}
Let $D\in\mathbb Z[i]\setminus\{0\}$ and set $M:=N(D)=|D|^2\in\mathbb Z_{>0}$. Then there exist infinitely many integers $t\ge 1$ such that both integers
\[
1+16t^2\quad\text{and}\quad 16t^2+16t+5
\]
are coprime to $M$.
In particular, for such $t$ the Gaussian integers $z=1+4ti$ and $w=1+(4t+2)i$ satisfy $\gcd(D,z)=\gcd(D,w)=1$ in $\mathbb Z[i]$.
\end{lemma}

\begin{proof}
Write $M=\prod_{r=1}^s p_r^{e_r}$ with distinct primes $p_r$.
For each $p=p_r$, consider the two polynomials over $\mathbb F_p$:
\[
f(t)=1+16t^2,\qquad g(t)=16t^2+16t+5.
\]
Each is quadratic, hence has at most $2$ roots modulo $p$; therefore the union of their root sets has at most $4$ elements in $\mathbb F_p$.
For $p\ge 5$ we can choose $t_p\in\mathbb F_p$ that is not a root of either $f$ or $g$.
For $p=2$, one checks $f(t)\equiv 1\pmod 2$ and $g(t)\equiv 1\pmod 2$, so any $t_p$ works.
For $p=3$, one may take $t_p\equiv 0$, giving $f(0)=1\not\equiv 0$ and $g(0)=5\equiv 2\not\equiv 0$ modulo $3$.

By the Chinese remainder theorem in $\mathbb Z$, there exists $t\in\mathbb Z$ such that $t\equiv t_{p_r}\pmod{p_r}$ for all $r$.
Then for each $r$, neither $f(t)$ nor $g(t)$ is divisible by $p_r$, hence $\gcd(f(t),M)=\gcd(g(t),M)=1$.
Taking $t$ in this congruence class yields infinitely many such integers.

Finally, if a Gaussian prime $\pi$ divides both $D$ and (say) $z=1+4ti$, then $N(\pi)$ divides both $N(D)=M$ and $N(z)=1+16t^2=f(t)$, contradicting $\gcd(M,f(t))=1$. Hence $\gcd(D,z)=1$. The argument for $w$ is identical.
\end{proof}

Now fix such a $t$ from Lemma~\ref{lem:choosek}, so that $z=1+4ti$ and $w=1+(4t+2)i$ are both admissible in (\ensuremath{\dagger}).

Apply (\ensuremath{\dagger}) to $z_0=1$ (always admissible since $\gcd(D,1)=1$):
\[
h_P(1)=1 = \frac{1}{8} + a_{10}\cdot 1 + b_{10}\cdot 0 + c_{10},
\]
so
\begin{equation}\label{eq:ac}
a_{10}+c_{10} = \frac78.
\end{equation}

Apply (\ensuremath{\dagger}) to $z=1+4ti$.
By Lemma~\ref{lem:count}, $h_P(z)=2t^2+1$ and the area term is $\frac{1+(4t)^2}{8}=2t^2+\frac18$, hence
\[
2t^2+1
=
\Bigl(2t^2+\frac18\Bigr) + a_{10}\cdot 1 + b_{10}\cdot (4t) + c_{10}.
\]
Using \eqref{eq:ac}, this simplifies to
\[
2t^2+1 = 2t^2+1 + 4t\,b_{10},
\]
so $b_{10}=0$ (since $t\ge 1$).

Now apply (\ensuremath{\dagger}) to $w=1+(4t+2)i$.
Lemma~\ref{lem:count} gives $h_P(w)=2t^2+2t+1$, while the area term is
\[
\frac{1+(4t+2)^2}{8} = 2t^2+2t+\frac58.
\]
Thus (\ensuremath{\dagger}) would give
\[
2t^2+2t+1 = \Bigl(2t^2+2t+\frac58\Bigr) + a_{10}\cdot 1 + b_{10}\cdot(4t+2) + c_{10}.
\]
Using $b_{10}=0$ and \eqref{eq:ac} again, the right-hand side equals
\[
2t^2+2t+\frac58 + \frac78 = 2t^2+2t+\frac{12}{8} = 2t^2+2t+\frac32,
\]
which is not equal to the integer $2t^2+2t+1$. Contradiction.

Therefore, no such $D$ and $2$-periodic $a,b,c$ can exist for this $P$ with $N=2$; i.e.\ (\ensuremath{\dagger}) is false in general.
This disproves the proposed ``quasi-polynomial mod $N$'' form.

\subsection{5) VERIFICATION}
\begin{itemize}
\item Lemma~\ref{lem:count} is a complete counting argument using explicit inequalities.
\item Lemma~\ref{lem:choosek} uses only: (a) quadratics have at most two roots mod $p$, and (b) the integer CRT.
\item The contradiction uses three admissible $z$ values in the same residue class $(u,v)\equiv(1,0)\pmod 2$, forcing constant coefficients.
\item Boundary cases: $k=0$ (i.e.\ $z=1$) is admissible for any $D$ and is used in \eqref{eq:ac}.
\end{itemize}

\subsection{6) FINAL}
\textbf{COUNTEREXAMPLE/DISPROOF.}

\paragraph{Conclusion.}
The strong global statement (\ensuremath{\dagger}) proposed as an ``Ehrhart quasi-polynomial mod $N$'' property is false for $N=2$. A counterexample is the triangle $P=\mathrm{conv}\{0,\tfrac12,\tfrac{i}{2}\}$, for which the lattice count $h_P(1+2ki)$ depends on $k\bmod 2$ in a way that cannot be represented with coefficients periodic modulo $2$ after excluding finitely many primes.
