\section{Problem 262745: is every finite $p$-group $G(\FF_p)$ for a unipotent group?}

\subsection*{1) FORMAL RESTATEMENT}
\paragraph{Literal issue.}
Again the prompt is phrased as a question. We convert it to a precise universally quantified statement.

\paragraph{Conventions/definitions.}
\begin{itemize}
\item $p$ is a prime, $n\ge 0$ an integer, and $\Gamma$ a finite group with $|\Gamma|=p^n$.
\item A \emph{(smooth affine) unipotent algebraic group} over $\FF_p$ means a smooth affine group scheme $G$ of finite type over $\FF_p$ such that over $\overline{\FF}_p$ it embeds as a closed subgroup of some $\mathrm{U}_N$ (upper unitriangular matrices), equivalently all geometric points are unipotent.
\item $\dim(G)$ means the dimension of $G$ as an algebraic variety over $\FF_p$.
\end{itemize}
(If one allows non-smooth/non-reduced group schemes, the problem changes; the question's dimension condition suggests the smooth case.)

\paragraph{Minimal corrected proposition.}
\begin{quote}
\textbf{(U$_{262745}$)} For every prime $p$, every $n\ge 0$, and every finite group $\Gamma$ of order $p^n$, there exists a smooth unipotent algebraic group $G$ over $\FF_p$ of dimension $n$ such that $G(\FF_p)\cong \Gamma$.
\end{quote}

\paragraph{Stress points.}
\begin{itemize}
\item For smooth connected unipotent $G$ over $\FF_p$, one expects $|G(\FF_p)|=p^{\dim G}$, consistent with the condition $|\Gamma|=p^n$.
\item The classification of (noncommutative) unipotent groups over finite fields is highly nontrivial; the question asks for universality, which is strong.
\item The case $p=2$ and non-abelian groups of order $16$ (e.g.\ dihedral) is a plausible minimal obstruction, as noted in the MO post.
\end{itemize}

\subsection*{2) QUICK LITERATURE/CONTEXT CHECK (browsing YES)}
The MO thread \href{https://mathoverflow.net/questions/262745/is-every-p-group-the-mathbbf-p-points-of-a-unipotent-group}{262745} has no posted answers as of January 2026 (it was edited in 2025 but remains unanswered).
An earlier related MO question \href{https://mathoverflow.net/questions/69397/p-groups-as-rational-points-of-unipotent-groups}{69397} (2011) is likewise unanswered as of January 2026.
A standard structural fact is that over a perfect field (hence over a finite field), every smooth connected unipotent group is split (i.e.\ admits a composition series with successive quotients $\Ga$); see, e.g., Conrad's notes (definition of split unipotent) where splitness is said to be automatic over perfect fields.

\subsection*{3) ATTACK PLAN}
\paragraph{Proof-track ideas.}
\begin{enumerate}[label=(P\arabic*)]
\item \emph{Induct on $n$ using central series.} Every finite $p$-group has a central series with successive quotients $\FF_p$ (as groups). Try to lift each central extension by $\FF_p$ to an extension of unipotent groups by $\Ga$.
\item \emph{Cohomological surjectivity.} The obstruction is whether the natural map from ``algebraic extensions'' to group-cohomology extensions on $\FF_p$-points is surjective.
\item \emph{Lazard/BCH for bounded class.} For $p$ large relative to nilpotency class, the BCH formula truncates and yields polynomial group laws; this may realize large classes of $p$-groups.
\end{enumerate}

\paragraph{Disproof-track ideas.}
\begin{enumerate}[label=(D\arabic*)]
\item \emph{Smallest candidate obstruction.} Test $\Gamma=D_{16}$ at $p=2,n=4$ as suggested in the MO text: attempt to show no $4$-dimensional unipotent $\FF_2$-group has $\FF_2$-points isomorphic to $D_{16}$.
\item \emph{Automorphism-scheme connectedness constraints.} If $\Gamma$ is a semidirect product $C_2\ltimes C_8$ where the two elements of $C_2$ act by automorphisms in different connected components of an algebraic automorphism group, connectedness of $\Ga$-actions may obstruct lifting.
\item \emph{Invariants of $G(\FF_p)$.} Seek group-theoretic properties forced by being $G(\FF_p)$, such as constraints on power maps, lower central series, or existence of ``algebraic'' filtrations.
\end{enumerate}

\subsection*{4) WORK}
We were not able to prove (U$_{262745}$) nor to produce a rigorous counterexample. We record the most solid structural facts and the first gap.

\subsubsection*{Lemma 4.1 (connected unipotent groups over $\FF_q$ are split and have $q^{\dim}$ points).}
Let $q=p^f$ and let $G$ be a smooth connected unipotent algebraic group over $\FF_q$ of dimension $n$.
Then $G$ is $\FF_q$-split (admits a composition series with successive quotients $\Ga$), and in particular
\[
|G(\FF_q)|=q^n.
\]

\begin{proof}
Over a perfect field (and $\FF_q$ is perfect), smooth connected unipotent groups are split in the sense of admitting a composition series with successive quotients isomorphic to $\Ga$ over the base field. (This is a standard structural theorem; see e.g.\ Conrad's discussion of $k$-split unipotent groups and that splitness is automatic over perfect fields.)

Assuming such a series
\[
G=G_0\triangleright G_1\triangleright\cdots\triangleright G_n=1,\qquad G_{i-1}/G_i\simeq \Ga,
\]
we compute $|G(\FF_q)|$ by induction on $n$.
For $n=0$ the group is trivial and has $1=q^0$ point.
For the induction step, consider
\[
1\to G_1\to G_0\to \Ga\to 1.
\]
For connected smooth algebraic groups over a finite field, Lang's theorem implies $H^1(\FF_q,G_1)=1$; this yields surjectivity of $G_0(\FF_q)\to \Ga(\FF_q)$ and exactness
\[
1\to G_1(\FF_q)\to G_0(\FF_q)\to \Ga(\FF_q)\to 1.
\]
Hence $|G_0(\FF_q)|=|G_1(\FF_q)|\cdot |\Ga(\FF_q)|$.
By induction $|G_1(\FF_q)|=q^{n-1}$ and $|\Ga(\FF_q)|=q$, giving $|G(\FF_q)|=q^n$.
\end{proof}

\subsubsection*{Consequences and reduction.}
If $\Gamma\cong G(\FF_p)$ with $G$ smooth connected unipotent and $\dim G=n$, then necessarily $|\Gamma|=p^n$ (Lemma 4.1), so the cardinality constraint is forced.

Moreover, if one allows $G$ to be non-connected, then for a finite field $\FF_p$ any smooth unipotent group has connected component $G^\circ$ unipotent and the quotient $G/G^\circ$ finite \'etale of $p$-power order, hence trivial (since finite \'etale $p$-groups over $\FF_p$ have no nontrivial connected unipotent part). Thus, in many formulations, one may reduce to the connected case; the MO discussion suggests this reduction.

\subsubsection*{First serious gap (extension-lifting problem).}
Let
\[
1\to A\to \Gamma\to \overline{\Gamma}\to 1
\]
be a central extension with $A\simeq C_p$.
Suppose inductively that $\overline{\Gamma}\cong H(\FF_p)$ for some $(n-1)$-dimensional unipotent group $H$.
To realize $\Gamma$ as $G(\FF_p)$ for a dimension-$n$ unipotent group $G$, one needs a \emph{lift} of the extension class in $H^2(\overline{\Gamma},A)$ to an \emph{algebraic group extension}
\[
1\to \Ga\to G\to H\to 1
\]
over $\FF_p$ such that on $\FF_p$-points it induces the given extension.
We were not able to prove such lifting is always possible; this is the central obstruction.

\subsection*{5) VERIFICATION}
\paragraph{Checks on Lemma 4.1.}
The proof depends on two standard inputs:
(i) splitness of smooth connected unipotent groups over perfect fields, and
(ii) Lang's theorem (or equivalently $H^1(\FF_q,U)=1$ for connected smooth unipotent $U$).
Given these, the point count $|G(\FF_q)|=q^{\dim G}$ follows by induction and exactness of rational points in extensions.

\paragraph{Quantifier and definition checks.}
The universal statement (U$_{262745}$) is very strong and depends crucially on allowing \emph{noncommutative} unipotent groups. No contradiction has been produced here.

\subsection*{6) FINAL}
\textbf{UNRESOLVED}

\paragraph{(i) Strongest fully proved partial result obtained.}
For any smooth connected unipotent $G/\FF_q$ of dimension $n$, $|G(\FF_q)|=q^n$ (Lemma 4.1). Hence any realization $\Gamma\cong G(\FF_p)$ forces $|\Gamma|=p^n$.

\paragraph{(ii) First gap.}
We cannot prove that arbitrary central extensions of $H(\FF_p)$ by $C_p$ lift to algebraic extensions of $H$ by $\Ga$ over $\FF_p$ inducing the given group extension on $\FF_p$-points.

\paragraph{(iii) Top 3 next moves.}
\begin{enumerate}[label=(\alphic*)]
\item Work out the extension-lifting map explicitly for low-dimensional unipotent groups (e.g.\ dimension $\le 4$ over $\FF_2$) and test $\Gamma=D_{16}$.
\item Develop invariants of $G(\FF_p)$ arising from the algebraic filtration by $G_i$ with quotients $\Ga$ and translate them into constraints on $\Gamma$.
\item Investigate whether disconnected/non-smooth group schemes (if allowed) trivialize the problem, and isolate the precise hypothesis that makes the statement meaningful.
\end{enumerate}

\paragraph{(iv) Likely shape of a minimal counterexample.}
If a counterexample exists, the smallest plausible one is a non-abelian group of order $16$ at $p=2$ (e.g.\ $D_{16}$ or $Q_{16}$), since these are the first $2$-groups with richer automorphism structure and nilpotency class issues that evade BCH/Lazard methods.

% ======================================================================
