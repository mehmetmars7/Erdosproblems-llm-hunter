\section{Problem 1: ``Non-linear expanders?''}

\subsection*{1) FORMAL RESTATEMENT}

\paragraph{Ambiguities / misstatements.}
\begin{itemize}[nosep]
\item The question mentions a \emph{directed} graph but defines expansion via Laplacian/adjacency eigenvalues, which is standard for \emph{undirected} (symmetric) graphs. One must choose a convention (e.g.\ symmetrize by including inverses).
\item The maps $x\mapsto x^3$ have fixed points ($0,\pm1$), hence loops if taken literally as edges. Standard expander definitions assume simple loopless graphs; one can either allow loops (harmless for many spectral formulations) or delete them (changes adjacency by a diagonal matrix supported on finitely many vertices).
\end{itemize}

\paragraph{Minimal corrected statement (spectral expander formulation).}
For each prime $p\equiv 2\pmod 3$ with $p\ge 5$, let $e_p\in\{1,\dots,p-2\}$ be the inverse of $3$ modulo $p-1$ (so $3e_p\equiv 1\pmod{p-1}$). Define permutations of the set $\F_p=\Z/p\Z$ by
\[
\tau_{\pm}(x)=x\pm 1,\qquad c(x)=x^3,\qquad c^{-1}(x)=x^{e_p}.
\]
Let $G_p$ be the (possibly loopy) undirected $4$--regular multigraph on $\F_p$ with adjacency operator
\[
(A_pf)(x)=f(x+1)+f(x-1)+f(x^3)+f(x^{e_p})\qquad (f:\F_p\to\mathbb{C}).
\]
Likewise, for $p\neq 3$, define the $6$--regular multigraph $H_p$ with adjacency operator
\[
(B_pf)(x)=f(x\pm1)+f(3x)+f(3^{-1}x)+f(x^3)+f(x^{e_p}).
\]
Question (corrected): does there exist $\varepsilon>0$ such that for all primes $p\equiv2\pmod 3$, the \emph{normalized} operators $\frac1{4}A_p$ and $\frac1{6}B_p$ have all nontrivial eigenvalues $\lambda$ satisfying $|\lambda|\le 1-\varepsilon$? Equivalently, do we have a uniform spectral gap
\[
4-\max_{i\ge 1}|\lambda_i(A_p)|\ge 4\varepsilon,\qquad
6-\max_{i\ge 1}|\lambda_i(B_p)|\ge 6\varepsilon\ ?
\]

\subsection*{2) QUICK LITERATURE/CONTEXT CHECK}
The MathOverflow question is unanswered on the page snapshot consulted (0 answers). This problem is about spectral expansion for graphs coming from explicit permutations on $\F_p$ defined by a translation and a nonlinear power map. See the attached MathOverflow page for full context.

\subsection*{3) ATTACK PLAN}

\paragraph{Proof strategies.}
\begin{enumerate}[nosep]
\item \textbf{Bourgain--Gamburd machine / sum-product:} Try to treat the random walk generated by $\tau_{\pm}$ and $c^{\pm1}$ as a walk on an algebraic group/semigroup and prove flattening + non-concentration, giving a spectral gap.
\item \textbf{Fourier-analytic operator bounds:} Write $A_p$ on the additive Fourier side. Translation diagonalizes, but the $x\mapsto x^3$ pullback creates cubic phases. One would need nontrivial exponential-sum bounds robust under iteration.
\item \textbf{Obstruction search:} Look for large sets $S\subset \F_p$ with small boundary under $x\mapsto x\pm1$ and $x\mapsto x^3$ (e.g.\ arithmetic progressions, multiplicative subgroups) or construct approximate eigenfunctions.
\end{enumerate}

\paragraph{Disproof strategies.}
\begin{enumerate}[nosep]
\item Show a family of subsets $S_p$ with $|\partial S_p|/|S_p|\to 0$ (Cheeger constant $\to0$).
\item Produce explicit eigenfunctions (or almost-eigenfunctions) with eigenvalues $\to 4$ (or $\to 6$).
\item Detect a large almost-invariant partition (nearly bipartite or nearly disconnected).
\end{enumerate}

\paragraph{Chosen path.}
I run Phase 1 computations (small primes) and prove basic structural properties (connectedness, non-bipartiteness), but I do \emph{not} obtain a proof or counterexample for uniform expansion.

\subsection*{4) WORK}

\paragraph{Lemma 4.1 (Connectedness).}
For any prime $p$, the undirected graph containing edges $x\leftrightarrow x\pm1$ is connected. Hence both $G_p$ and $H_p$ are connected.

\begin{proof}
The edges $x\leftrightarrow x+1$ form a cycle visiting all $p$ vertices. In particular the graph with just these edges is connected; adding more edges cannot disconnect it.
\end{proof}

\paragraph{Lemma 4.2 (Non-bipartite for odd primes).}
For any odd prime $p$, any graph containing all edges $x\leftrightarrow x\pm1$ is not bipartite.

\begin{proof}
The subgraph consisting of edges $x\leftrightarrow x+1$ is a cycle of length $p$. For odd $p$, this is an odd cycle, which cannot occur in a bipartite graph. Hence the whole graph is not bipartite.
\end{proof}

\paragraph{Phase 1 computation (spectral gaps for sample primes).}
I computed (in Python) the adjacency matrices for the symmetrized generating sets and extracted the spectral gap
\[
\mathrm{gap}(G_p)=4-\max_{i\ge 1}|\lambda_i(A_p)|,\qquad
\mathrm{gap}(H_p)=6-\max_{i\ge 1}|\lambda_i(B_p)|.
\]
For sample primes $p\equiv 2\pmod 3$ up to $1601$, the observed gaps are:

\begin{center}
\small
\begin{tabular}{rrrrr}
\toprule
$p$ & gap$(G_p)$ & $\max_{i\ge1}|\lambda_i(A_p)|$ & gap$(H_p)$ & $\max_{i\ge1}|\lambda_i(B_p)|$\\
\midrule
5    & 0.807418 & 3.192582 & 2.208712 & 3.791288\\
11   & 0.644042 & 3.355958 & 1.566807 & 4.433193\\
17   & 0.413432 & 3.586568 & 1.385953 & 4.614047\\
23   & 0.482544 & 3.517456 & 1.267676 & 4.732324\\
29   & 0.419206 & 3.580794 & 1.190710 & 4.809290\\
53   & 0.407850 & 3.592150 & 1.077340 & 4.922660\\
101  & 0.382379 & 3.617621 & 1.089705 & 4.910295\\
197  & 0.374481 & 3.625519 & 1.008552 & 4.991448\\
401  & 0.368379 & 3.631621 & 0.986684 & 5.013316\\
797  & 0.365527 & 3.634473 & 0.984744 & 5.015256\\
1601 & 0.361876 & 3.638124 & 0.966153 & 5.033847\\
\bottomrule
\end{tabular}
\end{center}

These computations are consistent with a \emph{positive} uniform gap (roughly $\approx 0.36$ for $G_p$ and $\approx 1.0$ for $H_p$ on this sample), but of course they do not constitute a proof.

\subsection*{5) VERIFICATION (adversarial self-check)}

\begin{itemize}[nosep]
\item \textbf{Quantifiers:} An expander claim needs a single $\varepsilon>0$ working for \emph{all} primes $p\equiv 2\pmod 3$; finite computation cannot certify this.
\item \textbf{Loop issue:} $x\mapsto x^3$ has fixed points ($0,\pm1$), creating loops if taken literally. The adjacency operators above include these. Deleting loops modifies $A_p$ by a diagonal matrix supported on $O(1)$ vertices; this can change eigenvalues by up to $1$ in operator norm, so one must fix a convention before interpreting numerical gaps as evidence.
\item \textbf{Symmetrization:} I used symmetric generating sets (including inverses) so that adjacency matrices are real symmetric; eigenvalues are well-defined and the standard expander equivalences apply.
\end{itemize}

\subsection*{6) FINAL}
\textbf{UNRESOLVED}

\paragraph{(i) Strongest fully proved partial result.}
The graphs $G_p,H_p$ (with symmetric generators) are connected for all primes $p$; for odd primes they are not bipartite (Lemmas 4.1--4.2). I also produced explicit spectral-gap data for small primes.

\paragraph{(ii) First gap.}
Prove or disprove the existence of a uniform $\varepsilon>0$ such that $\max_{i\ge1}|\lambda_i(A_p)|\le 4(1-\varepsilon)$ (and similarly for $B_p$) for all primes $p\equiv2\pmod 3$.

\paragraph{(iii) Top 3 next moves.}
\begin{enumerate}[nosep]
\item Try to adapt a flattening/expansion machine (sum-product type) to the semigroup generated by $x\mapsto x+1$ and $x\mapsto x^3$ on $\F_p$.
\item Attempt to prove a quantitative non-concentration estimate for the random walk measures under alternating translation and cubic maps.
\item Search systematically for obstructions: subsets $S_p$ with anomalously small boundary, or explicit low-degree eigenfunctions detectable via Fourier analysis and exponential-sum estimates.
\end{enumerate}

\paragraph{(iv) What a minimal counterexample would likely look like.}
A counterexample would likely manifest as a family of large subsets $S_p\subset\F_p$ with $|S_p|\to\infty$ and $|\partial S_p|/|S_p|\to 0$, or as a sequence of nontrivial eigenvalues of $\frac14A_p$ approaching $1$ (or of $\frac16B_p$ approaching $1$). No such structure was detected in small numerical tests.

