\section{MO 363950: ``Curves on potatoes''}
\label{sec:mo363950}
\noindent\textbf{MathOverflow link:} \url{https://mathoverflow.net/questions/363950/curves-on-potatoes} (accessed 2026-01-16).

\subsection*{1) FORMAL RESTATEMENT}
\textbf{Literal statement (informal).} ``Given two potatoes, show there exist congruent curves on their surfaces.''

\medskip
\noindent\textbf{Ambiguity.} ``Potato'' is not defined: it could mean any compact subset, any topological surface, any smooth embedded surface, etc.
To make the question mathematically well-posed and to allow differential-topological tools (transversality), we adopt the minimal smooth convention below.

\medskip
\noindent\textbf{Minimal corrected statement.}
Let $S_1,S_2\subset\RR^3$ be compact, connected, $C^1$ embedded surfaces without boundary (i.e. compact $2$-dimensional $C^1$ submanifolds of $\RR^3$).
Then there exist simple closed $C^1$ curves $\gamma_1\subset S_1$ and $\gamma_2\subset S_2$ and a rigid motion (Euclidean isometry) $\Phi:\RR^3\to\RR^3$ such that $\Phi(\gamma_1)=\gamma_2$.

\subsection*{2) QUICK LITERATURE/CONTEXT CHECK}
The MathOverflow page has no posted answers as of 2026-01-16.
The intended solution is a standard differential-topology argument: translate one surface until it intersects the other transversely; the intersection is a finite disjoint union of simple closed curves; translation gives congruence.

\subsection*{3) ATTACK PLAN}
\textbf{Proof track.}
\begin{enumerate}[leftmargin=2em]
\item Find a translation vector $v\in\RR^3$ so that $S_1$ and $S_2+v$ intersect transversely and nontrivially.
\item Conclude the intersection is a compact $1$-dimensional $C^1$ submanifold, hence a disjoint union of embedded circles.
\item Pick one component $C$; then $C\subset S_1$ and $C-v\subset S_2$ are congruent via translation.
\end{enumerate}

\textbf{Disproof track.}
Not pursued for the corrected smooth statement.

\subsection*{4) WORK}
\subsubsection*{Lemma 4.1 (Each $C^1$ closed embedded surface has nonconstant normal)}
Let $S\subset\RR^3$ be a connected $C^1$ embedded surface without boundary.
Then its unit normal map $\nu:S\to \mathbb{S}^2$ is not constant.

\textbf{Proof.}
If $\nu$ were constant, then every tangent plane $T_pS$ would be the same fixed plane $\nu(p)^\perp$, so $S$ would lie in an affine plane.
A connected subset of an affine plane that is a $2$-dimensional embedded submanifold is open in that plane and hence noncompact.
This contradicts compactness (for ``potatoes'' we assume compact). \qed

\subsubsection*{Lemma 4.2 (Existence of a transverse intersection point after translation)}
Let $S_1,S_2\subset\RR^3$ be compact connected $C^1$ embedded closed surfaces.
Then there exist points $p\in S_1$, $q\in S_2$ such that the normals $\nu_1(p)$ and $\nu_2(q)$ are not parallel.
Consequently, for the translation vector $v_0:=p-q$, the translated surface $S_2+v_0$ meets $S_1$ at $p$ \emph{transversely}.

\textbf{Proof.}
By Lemma 4.1, the normal map $\nu_1$ is not constant, so its image contains at least two distinct directions.
Pick any $p\in S_1$.
If all normals on $S_2$ were parallel to $\nu_1(p)$, then $\nu_2$ would be constant up to sign; since $S_2$ is connected and $\nu_2$ is continuous, it would be constant, contradicting Lemma 4.1.
Hence there exists $q\in S_2$ with $\nu_2(q)$ not parallel to $\nu_1(p)$.
Now translate $S_2$ by $v_0=p-q$ so that $q+v_0=p$. At $p$, the tangent plane of $S_2+v_0$ equals $T_qS_2$ and the tangent plane of $S_1$ equals $T_pS_1$.
Nonparallel normals mean these tangent planes are distinct codimension-$1$ subspaces, hence they intersect transversely.
Thus $S_1$ and $S_2+v_0$ meet transversely at $p$. \qed

\subsubsection*{Lemma 4.3 (Persistence of intersection under small perturbations)}
Assume $S_1$ and $S_2+v_0$ have a transverse intersection point $p$.
Then there exists an open neighborhood $U\subset\RR^3$ of $v_0$ such that for every $v\in U$, the intersection $S_1\cap (S_2+v)$ is nonempty.

\textbf{Proof.}
At a transverse intersection point, the implicit function theorem implies that the intersection set varies continuously under small $C^1$ perturbations.
More concretely, choose local charts around $p$ in which $S_1$ and $S_2+v_0$ are graphs of $C^1$ functions over transverse planes; transversality implies the graph difference has an isolated zero with invertible derivative, so a nearby zero persists for small translations $v$.
Hence in some neighborhood of $v_0$ there remains at least one intersection point. \qed

\subsubsection*{Lemma 4.4 (Generic translation yields global transversality)}
For $C^1$ embedded submanifolds $S_1,S_2\subset\RR^3$, the set of translations $v\in\RR^3$ such that $S_1$ and $S_2+v$ intersect transversely (at \emph{every} intersection point) is dense in $\RR^3$.

\textbf{Proof sketch with precise tool.}
This is a standard consequence of the parametric transversality theorem applied to the smooth map
\[
F: \RR^3\times S_2 \to \RR^3, \quad F(v,q)=q+v,
\]
and the submanifold $S_1\subset\RR^3$.
Transversality of $F_v(q):=F(v,q)$ to $S_1$ is equivalent to $S_2+v$ meeting $S_1$ transversely.
Parametric transversality asserts that the set of parameters $v$ for which $F_v$ is transverse to $S_1$ is residual (hence dense).
(We do not reproduce the proof of parametric transversality here; it is a named theorem.) \qed

\subsubsection*{Lemma 4.5 (Transverse intersection of closed surfaces is a union of circles)}
If $S_1,S_2\subset\RR^3$ are compact $C^1$ embedded surfaces without boundary and intersect transversely, then $S_1\cap S_2$ is a compact $1$-dimensional $C^1$ submanifold without boundary.
In particular it is a finite disjoint union of embedded circles.

\textbf{Proof.}
Transversality of two codimension-$1$ submanifolds in $\RR^3$ implies their intersection is a $C^1$ submanifold of codimension $2$, hence dimension $1$.
Since both are compact, the intersection is compact.
Because the surfaces have no boundary, the intersection submanifold has no boundary.
A compact $1$-manifold without boundary is a finite disjoint union of circles. \qed

\subsubsection*{Proof of the corrected statement}
By Lemma 4.2, pick $v_0$ such that $S_1$ and $S_2+v_0$ have at least one transverse intersection point.
By Lemma 4.3, there is a neighborhood $U$ of $v_0$ where $S_1\cap(S_2+v)\neq\varnothing$.
By Lemma 4.4, choose $v\in U$ such that $S_1$ and $S_2+v$ intersect transversely everywhere.
Then by Lemma 4.5, the intersection $C:=S_1\cap(S_2+v)$ is a finite disjoint union of embedded circles; choose one connected component $\gamma\subset C$.
Then $\gamma\subset S_1$ and also $\gamma\subset S_2+v$.
Define $\gamma_2:=\gamma - v \subset S_2$.
The translation $\Phi(x)=x-v$ is a rigid motion of $\RR^3$ with $\Phi(\gamma)=\gamma_2$.
Thus $\gamma\subset S_1$ and $\gamma_2\subset S_2$ are congruent simple closed curves.
\qed

\subsection*{5) VERIFICATION}
\begin{itemize}[leftmargin=2em]
\item The main external input is the parametric transversality theorem (Lemma 4.4). Its hypotheses (smoothness of the translation action) are satisfied for $C^1$ embedded submanifolds.
\item If the surfaces were merely topological or extremely non-smooth, the transversality-based argument may fail; this is why we stated a $C^1$ corrected statement.
\item Compactness is used to ensure the intersection $1$-manifold is a finite union of circles.
\end{itemize}

\subsection*{6) FINAL}
\begin{center}
\textbf{PROOF}
\end{center}


