\section{Problem 4: Singular fibers of elliptic fibrations over higher-dimensional bases}

\subsection*{1) FORMAL RESTATEMENT}
\textbf{Ambiguity/misstated form.} The attachment is a \emph{request for classification and references}, not a single yes/no theorem. It asks, informally, what singular fibers can occur for elliptic fibrations over bases of dimension $\ge 2$, beyond Kodaira's list (elliptic surfaces) and beyond Miranda/Szyd\l o's lists (certain normal-crossing collision hypotheses).

\medskip
\noindent\textbf{Minimal corrected ``statement'' to evaluate.}
A reasonable formalization of the ``hope for classification without normal crossings/minimality'' is:

\begin{quote}
(\textbf{H}) \emph{Without imposing strong hypotheses (e.g.\ relative minimality/semistability/normal-crossing discriminant), the set of possible singular fibers of an elliptic fibration over a higher-dimensional base admits a finite or otherwise manageable classification.}
\end{quote}

I will \emph{disprove} (H) in a strong sense: even over a curve base, if one drops relative minimality, there are infinitely many possible singular fibers; taking products yields the same phenomenon over higher-dimensional bases.

\subsection*{2) QUICK LITERATURE/CONTEXT CHECK (web)}
Kodaira classified singular fibers for relatively minimal elliptic surfaces (base a curve, with section, etc.). For elliptic threefolds, Miranda and later Szyd\l o studied lists of possible fibers under additional hypotheses (e.g.\ smooth total space, normal crossing discriminant, ``good'' collisions). The MO post asks what happens beyond those hypotheses. As of \date{}, the MO page shows no posted answer.

\subsection*{3) ATTACK PLAN}
\textbf{Disproof strategy.} Exhibit an explicit infinite family of elliptic fibrations with mutually non-isomorphic singular fibers, obtained by blowing up points on a smooth fiber of the trivial fibration $E\times \mathbb P^1\to \mathbb P^1$. This yields infinitely many ``non-Kodaira'' fibers (because the fibration is not relatively minimal) and shows that without minimality hypotheses, no finite classification is possible. Then take a product with an auxiliary variety to obtain the same phenomenon over higher-dimensional bases.

\subsection*{4) WORK (explicit counterexamples)}

\begin{theorem}[Infinitely many fiber types without minimality]\label{thm:infinitely-many}
Fix a smooth elliptic curve $E$ over $\C$ and an integer $N\ge 1$.
There exists a smooth projective surface $X_N$ and a proper surjective morphism
\[
\pi_N: X_N \longrightarrow \mathbb P^1
\]
such that:
\begin{enumerate}[label=(\roman*),leftmargin=2.5em]
\item For all $t\in \mathbb P^1\setminus\{0\}$, the fiber $\pi_N^{-1}(t)$ is smooth and isomorphic to $E$.
\item The special fiber $F_N:=\pi_N^{-1}(0)$ is a reduced divisor with exactly $N+1$ irreducible components: one component is isomorphic to $E$, and the other $N$ components are smooth rational curves arranged in a chain, each meeting the next transversely in one point, with the first meeting $E$ transversely in one point.
\end{enumerate}
In particular, the fibers $F_N$ are pairwise non-isomorphic as schemes for different $N$ (they have different numbers of irreducible components), so there is no finite list of possible singular fibers without imposing additional hypotheses such as relative minimality.
\end{theorem}

\begin{proof}
Start with the trivial smooth elliptic fibration
\[
\pi_0: X_0:=E\times \mathbb P^1 \longrightarrow \mathbb P^1,\qquad \pi_0(p,t)=t.
\]
Fix the point $0\in \mathbb P^1$ and a point $p_0\in E$. Let $q_0:=(p_0,0)\in X_0$.

\emph{Step 1 (first blowup).}
Let $\beta_1:X_1\to X_0$ be the blowup of $X_0$ at $q_0$.
Define $\pi_1:=\pi_0\circ \beta_1:X_1\to \mathbb P^1$.
Since $\beta_1$ is an isomorphism over $X_0\setminus\{q_0\}$, for every $t\neq 0$ the fiber $\pi_1^{-1}(t)$ identifies with $\pi_0^{-1}(t)=E\times\{t\}\cong E$, hence is smooth and isomorphic to $E$.

The special fiber $F_1=\pi_1^{-1}(0)$ is the full preimage of the fiber $E\times\{0\}$ under the blowup. As a divisor it is the union of:
\begin{itemize}[leftmargin=2em]
\item the proper transform $E_1$ of $E\times\{0\}$, which is isomorphic to $E$;
\item the exceptional divisor $C_1\cong \mathbb P^1$.
\end{itemize}
Moreover $E_1$ and $C_1$ meet transversely at one point (the point corresponding to the tangent direction of $E\times\{0\}$ at $q_0$).

\emph{Step 2 (inductive blowups).}
Assume inductively that for some $n\ge 1$ we have constructed a smooth projective surface $X_n$ with a morphism $\pi_n:X_n\to\mathbb P^1$ such that for $t\neq 0$ the fiber is $E$, and the fiber $F_n:=\pi_n^{-1}(0)$ is a chain
\[
F_n = E_n \cup C_1 \cup \cdots \cup C_n
\]
with $E_n\cong E$ and each $C_i\cong \mathbb P^1$, where the only intersections are $E_n\cap C_1$ (one transverse point) and $C_i\cap C_{i+1}$ (one transverse point).

Choose a point $q_n$ on the last component $C_n$ away from the two intersection points of $C_n$ with the rest of the fiber. In particular, $q_n$ is a smooth point of the total space $X_n$ and lies in the fiber over $0$.
Let $\beta_{n+1}:X_{n+1}\to X_n$ be the blowup of $X_n$ at $q_n$, and set $\pi_{n+1}:=\pi_n\circ \beta_{n+1}$.

As before, for $t\neq 0$ the fiber $\pi_{n+1}^{-1}(t)$ is unchanged (the center of the blowup lies over $0$), hence is still $E$.

For the special fiber, blowing up at a smooth point on the fiber adds one new exceptional component $C_{n+1}\cong\mathbb P^1$ meeting the proper transform of $C_n$ transversely at one point, and does not affect the other components. Thus $F_{n+1}$ is obtained by attaching $C_{n+1}$ to the end of the chain. This verifies the induction.

Taking $n=N$ proves existence of $(X_N,\pi_N)$ with the asserted fiber description. Non-isomorphism for different $N$ follows because the number of irreducible components of $F_N$ is $N+1$.
\end{proof}

\begin{corollary}[Higher-dimensional bases]\label{cor:product}
Let $B'$ be any smooth projective variety of positive dimension (e.g.\ $B'=\mathbb P^{m-1}$).
Then for each $N\ge 1$ the product morphism
\[
\Pi_N := \pi_N \times \mathrm{id}_{B'} : X_N\times B' \longrightarrow \mathbb P^1\times B'
\]
is an ``elliptic fibration'' over the higher-dimensional base $\mathbb P^1\times B'$ (generic fiber is $E$) whose singular fibers over $\{0\}\times B'$ contain the infinitely many types from Theorem~\ref{thm:infinitely-many}.
\end{corollary}

\begin{proof}
For $(t,b')$ with $t\neq 0$, the fiber of $\Pi_N$ is $(\pi_N^{-1}(t))\times\{b'\}\cong E$.
Over $t=0$, the fiber is $F_N\times\{b'\}$, whose scheme structure varies with $N$ as above.
\end{proof}

\paragraph{Interpretation.}
Theorem~\ref{thm:infinitely-many} and Corollary~\ref{cor:product} show that without extra hypotheses (relative minimality, or restrictions on discriminant/Weierstrass models), the set of possible singular fibers is infinite and can be made arbitrarily complicated by repeated blowups. Hence any ``Kodaira-like finite classification'' is impossible in this generality.

\subsection*{5) VERIFICATION}
\begin{itemize}[leftmargin=2em]
\item Each blowup center $q_n$ is chosen to lie only in the fiber over $0$, so fibers over $t\neq 0$ are unaffected; this is checked each step.
\item The description of the special fiber after a blowup at a smooth point of the fiber is standard: the total transform is the proper transform plus one exceptional $\mathbb P^1$ meeting transversely at one point.
\item Pairwise non-isomorphism is certified by the invariant ``number of irreducible components''.
\end{itemize}

\subsection*{6) FINAL}
\textbf{COUNTEREXAMPLE/DISPROOF.}

An explicit infinite family of elliptic fibrations $(X_N,\pi_N)$ was constructed, with singular fibers $F_N$ having $N+1$ irreducible components. This disproves the possibility of any finite Kodaira-style classification of singular fibers once relative minimality/normal-crossing-type hypotheses are dropped, and the phenomenon persists over higher-dimensional bases by taking products.
