\section{Problem 311325: Vertex coloring inherited from perfect matchings}

\subsection*{1) FORMAL RESTATEMENT}

\paragraph{Reconstructed definitions (from the attachment).}
Fix integers $n\ge 1$ and $d\ge 1$.
A \emph{bi-colored weighted graph} in the sense of the attachment consists of:
\begin{itemize}[leftmargin=2.2em]
\item a finite undirected multigraph $G=(V,E)$ with $|V|=n$,
\item an edge-weight function $w:E\to \C$,
\item for each edge $e=\{u,v\}$ an \emph{endpoint coloring} $(c_e(u),c_e(v))\in\{1,\dots,d\}^2$.
\end{itemize}
(So each edge carries two colors, one at each endpoint; they may coincide.)

A \emph{perfect matching} $M$ in $G$ is a set of edges that are pairwise vertex-disjoint and cover all vertices.
Thus perfect matchings exist only when $n$ is even.

Given a perfect matching $M$, define the \emph{inherited vertex coloring} (IVC) $\mathrm{col}_M:V\to\{1,\dots,d\}$ by
\[
\mathrm{col}_M(v):=c_e(v)\quad\text{where $e\in M$ is the unique edge incident to $v$}.
\]

Given a vertex coloring $\kappa:V\to\{1,\dots,d\}$, define its \emph{weight} by
\[
W(\kappa)\;:=\;\sum_{\substack{M\text{ perfect matching}\\ \mathrm{col}_M=\kappa}}
\ \prod_{e\in M} w(e)\ \in \C.
\]
(Thus we sum the product of edge weights over all perfect matchings that induce $\kappa$.)

Call $\kappa$ \emph{monochromatic} if all vertices have the same color.

\paragraph{The existence problem.}
For given $(n,d)$ (with $n$ even), does there exist a bi-colored weighted graph $G$ on $n$ vertices and $d$ colors such that:
\begin{enumerate}[label=(\alph*),leftmargin=2.2em]
\item for each monochromatic coloring $\kappa$ we have $W(\kappa)=1$, and
\item for each non-monochromatic coloring $\kappa$ we have $W(\kappa)=0$?
\end{enumerate}
This is the ``monochromatic with respect to IVC'' property in the attachment.

\subsection*{2) QUICK LITERATURE/CONTEXT CHECK}

The attachment connects this combinatorial condition to quantum-optical ``GHZ''-type states generated by multiphoton interference, and cites an arXiv preprint.  I did not verify external physics claims here; I focus on the combinatorial conditions.

\subsection*{3) ATTACK PLAN}

\begin{itemize}[leftmargin=2.2em]
\item \textbf{Proof (construction) track:} Provide explicit constructions for some infinite families of $(n,d)$ and verify the weight conditions exactly.
\item \textbf{Disproof track:} Derive necessary conditions (e.g.\ $n$ even) and attempt to prove nonexistence for small $(n,d)$ or with additional constraints (e.g.\ all weights positive real).
\item \textbf{Computation track:} Implement brute-force enumeration of perfect matchings for small $n$ and verify candidate constructions.
\end{itemize}

\subsection*{4) WORK}

\subsubsection*{Phase 0/1 --- Immediate necessary condition}

\begin{lemma}\label{lem:even-n}
If $n$ is odd then no such graph exists (because there are no perfect matchings).
\end{lemma}

\begin{proof}
If $n$ is odd then $G$ has no perfect matching. Hence $W(\kappa)=0$ for all colorings $\kappa$, contradicting the requirement $W(\kappa)=1$ for monochromatic $\kappa$.
\end{proof}

Henceforth assume $n$ is even.

\subsubsection*{Explicit constructions that work}

\begin{lemma}[Trivial family $n=2$]\label{lem:n2}
For $n=2$ and any $d\ge 1$, there exists a solution.
\end{lemma}

\begin{proof}
Let $V=\{1,2\}$.  For each color $c\in\{1,\dots,d\}$ include exactly one edge $e_c=\{1,2\}$ with endpoint colors $(c,c)$ and weight $w(e_c)=1$.
Then every perfect matching is a single edge; for each $c$ there is exactly one perfect matching inducing the monochromatic coloring $c$, with product weight $1$, and there are no perfect matchings inducing non-monochromatic colorings (since every edge is monochromatic).
Thus $W(\kappa)=1$ for monochromatic $\kappa$ and $W(\kappa)=0$ otherwise.
\end{proof}

\begin{lemma}[Cycle family: $(n,d)=(2m,2)$]\label{lem:cycle}
For every $m\ge 1$, there exists a solution with $n=2m$ and $d=2$.
\end{lemma}

\begin{proof}
Let $V=\{1,2,\dots,2m\}$ and consider the $2m$-cycle graph $C_{2m}$ with edges
\[
(1,2),(2,3),\dots,(2m-1,2m),(2m,1).
\]
Give weight $1$ to every edge.
Color the edges of the perfect matching $\{(1,2),(3,4),\dots,(2m-1,2m)\}$ monochromatically with color $1$ at both endpoints.
Color the remaining edges $\{(2,3),(4,5),\dots,(2m,1)\}$ monochromatically with color $2$ at both endpoints.

The cycle $C_{2m}$ has exactly two perfect matchings, namely these two alternating sets of edges.
Each of them induces a monochromatic vertex coloring (all vertices color $1$ or all vertices color $2$), and each has product of edge weights equal to $1$.
No non-monochromatic coloring occurs.
Therefore $W(\kappa)=1$ for the two monochromatic $\kappa$ and $W(\kappa)=0$ for all others.
\end{proof}

\begin{lemma}[The $K_4$ family: $(n,d)=(4,3)$]\label{lem:K4}
There exists a solution with $n=4$ and $d=3$.
\end{lemma}

\begin{proof}
Let $V=\{1,2,3,4\}$ and let $G=K_4$.
$K_4$ has exactly three perfect matchings:
\[
M_1=\{\{1,2\},\{3,4\}\},\quad
M_2=\{\{1,3\},\{2,4\}\},\quad
M_3=\{\{1,4\},\{2,3\}\}.
\]
For $r=1,2,3$, color both edges in $M_r$ monochromatically with color $r$ and assign weight $1$ to all edges.
Then each perfect matching $M_r$ induces the monochromatic coloring ``all vertices have color $r$'' and has product weight $1$.
No other perfect matchings exist, hence no other inherited colorings occur.
Thus $W(\kappa)=1$ for the three monochromatic $\kappa$ and $W(\kappa)=0$ otherwise.
\end{proof}

\subsubsection*{Phase 1 --- computational sanity check}

I implemented a direct enumeration of perfect matchings and verified:
\begin{itemize}[leftmargin=2.2em]
\item Lemma~\ref{lem:n2} works for $d=1,2,3,4,5$.
\item Lemma~\ref{lem:cycle} works for $m=1,2,3,4$.
\item Lemma~\ref{lem:K4} works exactly.
\end{itemize}
(These checks are not part of the proofs; they only guard against definition mismatches.)

\subsubsection*{What remains open in this report}

Outside the families above, I do not have a general existence theorem, nor a general obstruction.
In particular, I did \emph{not} resolve the existence question for arbitrary $(n,d)$.

\subsection*{5) VERIFICATION}

\begin{itemize}[leftmargin=2.2em]
\item In each construction, every perfect matching is explicitly described and counted, so there is no hidden cancellation.
\item All non-monochromatic inherited vertex colorings are absent because every edge used in a perfect matching is monochromatic and the graph has only the listed perfect matchings.
\item The constructions crucially use that the graph has a small, controlled set of perfect matchings.  For more complicated graphs, cancellation via complex weights may allow additional solutions, but I did not find/prove any.
\end{itemize}

\subsection*{6) FINAL}

\begin{center}
\textbf{UNRESOLVED}
\end{center}

\noindent\textbf{(i) Strongest proved partial result obtained:}
Explicit solutions exist for $(n,d)=(2,d)$ for all $d$ (Lemma~\ref{lem:n2}), for $(n,d)=(2m,2)$ for all $m$ (Lemma~\ref{lem:cycle}), and for $(n,d)=(4,3)$ (Lemma~\ref{lem:K4}). Also, $n$ must be even (Lemma~\ref{lem:even-n}).

\medskip
\noindent\textbf{(ii) First gap:}
No general criterion is proved that either constructs such graphs for all $(n,d)$ or forbids them beyond the known examples.

\medskip
\noindent\textbf{(iii) Top 3 next moves:}
\begin{enumerate}[label=(\arabic*),leftmargin=2.2em]
\item Translate the condition $W(\kappa)=\delta_{\kappa\ \mathrm{mono}}$ into algebraic constraints on a ``colored weighted adjacency tensor'' and study them as polynomial equations.
\item Search for constructions allowing cancellation (complex weights) that go beyond the ``positive weight'' regime, starting with the smallest unknown pairs $(n,d)$.
\item Prove structural obstructions (e.g.\ via counting perfect matchings, parity, or linear dependencies) that rule out large classes of graphs.
\end{enumerate}

\medskip
\noindent\textbf{(iv) What a minimal counterexample would likely look like (if nonexistence happens):}
A smallest even $n$ and smallest $d$ where any candidate graph inevitably produces a non-monochromatic inherited coloring with nonzero total weight that cannot be cancelled, perhaps because too few perfect matchings contribute to each coloring to allow destructive interference.
