\section{MO 415439 --- The $(\infty,1)$-category of all topological spaces (including ``bad'' ones)}
\label{sec:mo415439}
\noindent\textbf{MathOverflow link:} \href{https://mathoverflow.net/questions/415439}{https://mathoverflow.net/questions/415439}.

\subsection*{1) FORMAL RESTATEMENT}
The post considers the Str\o{}m model structure on the category $\mathbf{Top}$ of \emph{all} topological spaces:
\begin{itemize}
  \item weak equivalences: homotopy equivalences,
  \item cofibrations: Hurewicz cofibrations,
  \item fibrations: Hurewicz fibrations.
\end{itemize}
Let $\mathcal{C}$ be the associated simplicial localization / underlying $(\infty,1)$-category.

\medskip
\textbf{The problem is multi-part.} The MO post asks (paraphrasing) whether:
\begin{enumerate}
  \item $\mathcal{C}$ is ``nice'' (e.g. complete/cocomplete, cartesian closed),
  \item for each $T\in\mathbf{Top}$, the functor $(-)\times T$ admits a right adjoint in $\mathcal{C}$,
  \item $(-)\times T$ preserves homotopy colimits,
  \item $\mathcal{C}$ is presentable / locally presentable, etc.
\end{enumerate}

\medskip
\textbf{Stress points.}
\begin{itemize}
  \item Using \emph{all} spaces (not just compactly generated weak Hausdorff) breaks classical cartesian closedness.
  \item Weak equivalences are \emph{homotopy} equivalences (stronger than weak homotopy equivalences), so this is not the usual homotopy theory of spaces.
\end{itemize}

\subsection*{2) QUICK LITERATURE/CONTEXT CHECK}
\begin{itemize}
  \item In the ordinary 1-category $\mathbf{Top}$, not every space is exponentiable; exponentiable spaces are characterized by core-compactness in standard references.
  \item The convenient category of compactly generated weak Hausdorff spaces \emph{is} cartesian closed.
  \item As of writing, the MO question appears unanswered.
\end{itemize}

\subsection*{3) ATTACK PLAN}
\textbf{Proof track.}
\begin{itemize}
  \item Prove: for exponentiable $T$ (in a suitable sense), $(-)\times T$ has a right adjoint already on the model category level, hence in $\mathcal{C}$.
  \item Try to show: if $(-)\times T$ had a right adjoint in $\mathcal{C}$ for \emph{all} $T$, then some known 1-categorical obstruction in $\mathbf{Top}$ would disappear (unlikely).
\end{itemize}

\textbf{Disproof track.}
\begin{itemize}
  \item Find a concrete ``bad'' space $T$ such that $(-)\times T$ fails to preserve some homotopy colimit in Str\o{}m's model structure, which would rule out a right adjoint.
\end{itemize}

\subsection*{4) WORK}
\textbf{A safe lemma (``good'' $T$ are exponentiable).}
\begin{lemma}
If $T$ is exponentiable in a cartesian closed full subcategory $\mathbf{Top}_{\mathrm{cgwh}}\subset \mathbf{Top}$ (e.g. compactly generated weak Hausdorff spaces), then for $X,Y\in\mathbf{Top}_{\mathrm{cgwh}}$ there is a natural homeomorphism
\[
\mathbf{Top}_{\mathrm{cgwh}}(X\times T, Y) \cong \mathbf{Top}_{\mathrm{cgwh}}(X, Y^T),
\]
so $(-)\times T$ has a strict right adjoint on that subcategory, hence also a derived right adjoint in any homotopy localization that restricts to it.
\end{lemma}
\begin{proof}
This is standard cartesian closedness: $Y^T$ is the internal Hom object. The claimed adjunction is the defining property of $Y^T$.
\end{proof}

\medskip
\textbf{What remains difficult.}
The MO question concerns \emph{all} spaces and the localization at homotopy equivalences. It is not immediate (to this author) whether ``having an adjoint in the localization'' is equivalent to ordinary exponentiability in $\mathbf{Top}$, nor whether $(-)\times T$ preserves all homotopy colimits in the Str\o{}m model structure for arbitrary $T$.

\subsection*{5) VERIFICATION}
The lemma above is limited to a convenient subcategory and does not resolve the behavior on all of $\mathbf{Top}$.

\subsection*{6) FINAL}
\textbf{UNRESOLVED}.
\begin{itemize}
  \item (i) Partial result: exponentiable objects in a cartesian closed subcategory yield genuine right adjoints there.
  \item (ii) First gap: produce either (a) a ``bad'' $T$ for which $(-)\times T$ cannot admit a right adjoint in the localized $(\infty,1)$-category, or (b) a general theorem that it always does.
  \item (iii) Next moves: (1) analyze $(-)\times T$ on explicit homotopy pushouts; (2) compare the localization of all $\mathbf{Top}$ with that of $\mathbf{Top}_{\mathrm{cgwh}}$; (3) consult precise exponentiability criteria (core-compactness) and test whether they survive under homotopy localization.
  \item (iv) A minimal obstruction would be a space $T$ such that the functor $X\mapsto X\times T$ fails to preserve some homotopy colimit (e.g. a homotopy pushout of cofibrations) in Str\o{}m's model structure.
\end{itemize}

% -----------------------------------------------------------------------------

% -----------------------------------------------------------------------------
