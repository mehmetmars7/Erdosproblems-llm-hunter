\section{MO 55788: Two-convexity and a Lefschetz-type question}

\subsection*{1) Formal restatement}
\paragraph{Conventions.}
Fix $n\ge 3$. A \emph{$2$-plane} means an affine $2$-dimensional subspace $L\subset\mathbb{R}^n$. Components are connected components in the subspace topology.

\paragraph{Two-convexity.}
$\Omega\subset\mathbb{R}^n$ is \emph{two-convex} if for every (nondegenerate) Euclidean $3$-simplex $\Delta\subset\mathbb{R}^n$ with faces $F_1,F_2,F_3,F_4$, if $F_i\subset\Omega$ for some three distinct indices $i$, then $\Delta\subset\Omega$.

\paragraph{Target statement.}
Assume $\Omega$ is open, path-connected, and simply connected. Does it follow that for every affine $2$-plane $L\subset\mathbb{R}^n$, every connected component of $\Omega\cap L$ is simply connected?

\subsection*{2) Quick literature/context check}
The question is stated as open for $n\ge 4$ in the MathOverflow post; it reports positive results when $\partial\Omega$ is a smooth hypersurface and for $n=3$, and gives a counterexample for \emph{hyperplane} sections in $\mathbb{R}^4$ (dimension $3$ sections) but not for $2$-planes.

This problem is closely related to Gromov's and subsequent work on ``$k$-convexity'' and ``Lefschetz-type'' properties for sections; see e.g.\ the Panov--Petrunin paper (2016) and Petrunin's later problem lists for the status (still posed as a problem).

\subsection*{3) Attack plan}
\begin{itemize}[leftmargin=*]
\item \textbf{Proof track ideas.} Try to prove injectivity of $\pi_1(\Omega\cap L)\to\pi_1(\Omega)$ using a ``no compressions'' argument: any loop in $\Omega\cap L$ null-homotopic in $\Omega$ should bound a disk inside $\Omega\cap L$. Two-convexity might allow ``pushing'' fillings into $L$ by using tetrahedra.
\item \textbf{Disproof track ideas.} Attempt to modify the given $\mathbb{R}^4$ example so that a \emph{$2$-plane} section inherits a hole (e.g.\ annulus) while keeping $\Omega$ simply connected and two-convex. Candidate constructions: intersections of two-convex sets (stable under intersection) such that some plane section is the complement of a compact set in the plane.
\end{itemize}

\subsection*{4) Work: fully proved partial result}
We can prove only a very limited sufficient condition.

\begin{lemma}[Convex sets satisfy the Lefschetz plane-section property]
If $\Omega\subset\mathbb{R}^n$ is open and convex, then for every affine $2$-plane $L\subset\mathbb{R}^n$, every connected component of $\Omega\cap L$ is convex (hence simply connected).
\end{lemma}

\begin{proof}
Intersection of a convex set with an affine subspace is convex. Since $\Omega$ is open, $\Omega\cap L$ is open in $L\simeq\mathbb{R}^2$, and any connected component of a convex subset of $\mathbb{R}^2$ is itself convex (indeed convex sets are connected). Every convex open subset of $\mathbb{R}^2$ is simply connected.
\end{proof}

\subsection*{5) Verification / first gap}
The lemma above uses convexity, which is strictly stronger than two-convexity. The first genuine gap for the original problem is:

\begin{quote}
\emph{Gap:} prove (or disprove) that two-convexity $+$ $\pi_1(\Omega)=0$ implies that each component of $\Omega\cap L$ has trivial $\pi_1$ for \emph{every} affine $2$-plane $L\subset\mathbb{R}^n$ when $n\ge 4$.
\end{quote}

\subsection*{6) FINAL}
\textbf{UNRESOLVED.}
Strongest proved partial result: the convex case (above). Next moves:
\begin{enumerate}[label=(\roman*)]
\item Try to build an explicit counterexample by refining the $\mathbb{R}^4$ intersection-of-paraboloids example so that some $2$-plane section is an annulus.
\item Prove an ``incompressibility'' lemma for plane sections using a tetrahedral filling scheme implied by two-convexity.
\item Use (or develop) a version of higher convexity/``two-hull'' theory to force vanishing of $H_1$ or $\pi_1$ in plane sections.
\end{enumerate}

