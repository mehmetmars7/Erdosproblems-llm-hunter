\section{Problem 212269: Metrics on the $3$-sphere with knotted geodesics}

\subsection*{1) FORMAL RESTATEMENT}

\paragraph{Definitions.}
Let $(S^3,g)$ be a smooth Riemannian $3$-sphere.
A \emph{closed geodesic} is a smooth map $\gamma:S^1\to S^3$ that is a (unit-speed) geodesic and is periodic; equivalently, $\gamma$ is a critical point of the energy functional on the free loop space.
A closed geodesic is \emph{simple} if $\gamma$ is an embedding (no self-intersections).

A simple closed curve $\gamma(S^1)\subset S^3$ is \emph{unknotted} if it is ambient isotopic in $S^3$ to the standard circle (equivalently, if it bounds a smoothly embedded disk in $S^3$). Otherwise it is \emph{knotted}.

\paragraph{Question (explicit quantifier form).}
Does there exist a smooth Riemannian metric $g$ on $S^3$ such that
\[
\forall\ \text{simple closed geodesics }\gamma\text{ in }(S^3,g),\ \gamma(S^1)\subset S^3\ \text{is knotted}?
\]
Equivalently: does there exist $g$ on $S^3$ with \emph{no unknotted} simple closed geodesics?

\paragraph{Stress points.}
\begin{itemize}
\item The quantifier ranges over \emph{all} simple closed geodesics, potentially infinitely many.
\item The existence of \emph{some} simple closed geodesic for every metric is itself subtle in dimension $3$; the attachment cites a related MO discussion.
\item ``Knotted'' is a topological property independent of parametrization; ``geodesic'' depends on $g$.
\end{itemize}

\subsection*{2) QUICK LITERATURE/CONTEXT CHECK (web available)}

The attachment points to another MO discussion about existence of simple closed geodesics on $S^n$; that discussion contains both claims and caveats about older references. I did not locate (in the attached or quick web scan) a definitive theorem settling the present ``all simple geodesics knotted'' question.

\subsection*{3) ATTACK PLAN}

\paragraph{Proof-track strategies (construct a metric with all simple geodesics knotted).}
\begin{enumerate}[label=\textbf{P\arabic*.}]
\item \textbf{Make a prescribed knot geodesic, then try to eliminate unknots.}  One can make a chosen knot $K\subset S^3$ geodesic by modifying the metric in a tubular neighborhood. The hard part is preventing the appearance of any unknotted simple geodesic elsewhere.
\item \textbf{Use dynamical systems/contact topology.}  Closed geodesics correspond to periodic orbits of the geodesic flow on the unit tangent bundle; try to import results about Reeb flows with constrained knot types. (Nontrivial and may not apply.)
\end{enumerate}

\paragraph{Disproof-track strategies (show every metric has an unknotted simple geodesic).}
\begin{enumerate}[label=\textbf{D\arabic*.}]
\item \textbf{Min--max within the unknot class.}  Attempt a variational argument restricted to embedded unknotted loops to produce an unknotted geodesic.
\item \textbf{Topology of loop space + knot filtration.}  Attempt to show some low-dimensional homology class of the loop space must be represented by an unknotted critical loop.
\end{enumerate}

\paragraph{Chosen path.}
I can fully prove a standard but useful fact: \emph{any prescribed knot can be made a geodesic for some metric}. This addresses feasibility of ``knotted geodesics'' but does not settle the global ``all simple geodesics knotted'' quantifier. I did not obtain a full construction or a no-go theorem.

\subsection*{4) WORK}

\begin{theorem}[Any prescribed knot can be made a geodesic]\label{thm:knot-geodesic}
Let $M$ be a smooth manifold and let $\gamma:S^1\hookrightarrow M$ be a smooth embedded circle (in particular, a knot when $M=S^3$).  
Then there exists a smooth Riemannian metric $g$ on $M$ such that $\gamma$ is a (unit-speed) geodesic in $(M,g)$.
\end{theorem}

\begin{proof}
We construct $g$ in a neighborhood of $\gamma(S^1)$ and then extend it to all of $M$.

\paragraph{Step 1: tubular neighborhood and coordinates.}
Because $\gamma$ is a smooth embedding, it has a tubular neighborhood: there exists an open neighborhood $U\subset M$ of $\gamma(S^1)$ and a diffeomorphism
\[
\Phi: S^1\times D^{n-1}_\varepsilon \ \longrightarrow\ U
\]
(where $n=\dim M$ and $D^{n-1}_\varepsilon$ is the Euclidean $(n-1)$-disk of radius $\varepsilon$) such that
\[
\Phi(\theta,0)=\gamma(\theta)\quad\text{for all }\theta\in S^1.
\]

\paragraph{Step 2: define a product metric on the tubular neighborhood.}
Let $g_{S^1}$ be the standard round metric on $S^1$ of total length $L>0$ (choose any $L$; later we can rescale to unit-speed). Let $g_{D}$ be the standard Euclidean metric on $D^{n-1}_\varepsilon$.

Define a Riemannian metric $g_U$ on $U$ by pushing forward the product metric:
\[
g_U := \Phi_*(g_{S^1}\oplus g_{D}).
\]
Concretely, $g_U$ is the unique metric on $U$ for which $\Phi$ is an isometry from $(S^1\times D^{n-1}_\varepsilon,\, g_{S^1}\oplus g_D)$ onto $(U,g_U)$.

\paragraph{Step 3: verify $\gamma$ is geodesic for $g_U$.}
In the product manifold $S^1\times D^{n-1}_\varepsilon$ with metric $g_{S^1}\oplus g_D$, the curve
\[
\tilde\gamma(\theta) = (\theta,0)
\]
is a geodesic: it is the product of a geodesic on $S^1$ and a constant curve in the disk factor. More explicitly, in product coordinates, the Levi-Civita connection splits and the acceleration of $\tilde\gamma$ vanishes in both factors.

Because $\Phi$ is an isometry, it preserves geodesics. Thus $\gamma=\Phi\circ\tilde\gamma$ is a geodesic in $(U,g_U)$.

\paragraph{Step 4: extend the metric to all of $M$.}
Choose any smooth Riemannian metric $g_0$ on $M$. Since $g_U$ is defined on the open set $U$, we extend to a global smooth metric $g$ as follows.

Pick an open set $V$ with $\overline V\subset U$ containing $\gamma(S^1)$. Choose a smooth bump function $\eta:M\to[0,1]$ such that $\eta\equiv 1$ on $V$ and $\eta\equiv 0$ on $M\setminus U$. Define
\[
g := \eta\, g_U + (1-\eta)\, g_0.
\]
Because $g_U$ and $g_0$ are positive definite symmetric bilinear forms on each tangent space, and $\eta\in[0,1]$, the convex combination $g$ is positive definite everywhere; smoothness is clear.

On $V$, we have $g=g_U$. Since the geodesic equation for $\gamma$ depends only on the metric in a neighborhood of $\gamma(S^1)$, and $\gamma(S^1)\subset V$, the curve $\gamma$ is a geodesic for $g$ as well.

Finally, rescale $g$ by a constant factor if desired to make $\gamma$ unit-speed.
\end{proof}

\subsection*{5) VERIFICATION}

\paragraph{Locality check.}
The geodesic property of $\gamma$ is local along the curve: if $g$ agrees with $g_U$ on an open neighborhood of $\gamma(S^1)$, then the Levi-Civita connection along $\gamma$ is the same, hence $\gamma$ remains a geodesic.

\paragraph{No hidden assumptions.}
Tubular neighborhoods exist for embedded submanifolds; $S^1$ embedded is a standard case. Partition-of-unity/bump-function extension is standard and the convex combination of metrics remains a metric because positive definite bilinear forms form a convex cone.

\subsection*{6) FINAL}

\textbf{UNRESOLVED.}

\paragraph{Strongest fully proved partial result.}
Any given knot type $K\subset S^3$ can be made a simple closed geodesic for some smooth metric (Theorem~\ref{thm:knot-geodesic}).

\paragraph{Exact first gap.}
Either:
\begin{enumerate}[label=(\alph*)]
\item prove that \emph{every} metric on $S^3$ has at least one unknotted simple closed geodesic (disproof), or
\item construct a metric and prove that \emph{every} simple closed geodesic is knotted (proof).
\end{enumerate}

\paragraph{Top 3 next moves.}
\begin{enumerate}[label=\arabic*.]
\item Develop a min--max argument in the space of embedded \emph{unknotted} loops, showing a width is attained by an unknotted geodesic.
\item Translate the question into dynamics of the geodesic flow (a Reeb flow on a contact manifold) and investigate whether known contact-topology results force an unknotted periodic orbit in this setting.
\item Attempt to build a metric whose short geodesics are confined to a knotted tube and prove any simple geodesic must be short (or confined), ruling out unknotted ones.
\end{enumerate}

\paragraph{What a minimal counterexample would likely look like.}
If the answer is ``no'' (i.e.\ every metric admits an unknotted simple closed geodesic), the minimal obstruction would be a general theorem of the form:
\[
\forall g\ \exists\ \text{an embedded geodesic unknot in }(S^3,g),
\]
likely produced by a variational principle constrained to the unknot class.
