\section{MO \#314626: Cohomology of symmetric groups and the integers mod 12}
\MO{314626}{Cohomology of symmetric groups and the integers mod 12}

\subsection*{1) FORMAL RESTATEMENT}
Let $n\ge 4$. Let $S_n$ be the symmetric group on $\{1,\dots,n\}$. Let $A=\mathbb Z/12\mathbb Z$ be the trivial $S_n$-module. A \emph{(normalized) $3$-cocycle} is a function
\[
  c:S_n\times S_n\times S_n\to A
\]
with $c(e,\sigma,\tau)=c(\sigma,e,\tau)=c(\sigma,\tau,e)=0$ and satisfying the cocycle equation
\[
  (\delta c)(\sigma,\tau,\upsilon,\omega)
  =c(\tau,\upsilon,\omega)-c(\sigma\tau,\upsilon,\omega)+c(\sigma,\tau\upsilon,\omega)-c(\sigma,\tau,\upsilon\omega)+c(\sigma,\tau,\upsilon)=0
\]
for all $\sigma,\tau,\upsilon,\omega\in S_n$.

The MO problem asks for an \emph{explicit} $3$-cocycle $c$ whose cohomology class $[c]\in H^3(S_n,A)$ generates a direct summand isomorphic to $\mathbb Z/12\mathbb Z$ (known to exist for $n$ in the stable range).

\textbf{Stress points / edge cases:}
(i) The group $H^3(S_n,A)$ depends on $n$ for small $n$; one must specify the intended range (typically $n$ sufficiently large). (ii) ``Generates the $\mathbb Z/12$ summand'' requires fixing the decomposition and identifying which summand is meant.

\subsection*{2) QUICK LITERATURE/CONTEXT CHECK}
Homological stability results (e.g. Nakaoka) compute stable homology/cohomology of symmetric groups, and $12$-torsion appears in degree $3$ in the stable homology. The MO question suggests using the universal coefficient theorem to identify $H^3(S_n,\mathbb Z/12)$ with a group having a $\mathbb Z/12$ direct summand, and asks for an explicit cocycle representative.

\subsection*{3) ATTACK PLAN}
\begin{itemize}[leftmargin=*]
\item \textbf{Proof track (construct cocycle):} (a) compute $H^3(S_n,\mathbb Z)$ or $H_3(S_n,\mathbb Z)$ in a stable range; (b) exhibit an explicit $3$-cycle in the bar resolution generating $12$-torsion; (c) use universal coefficient to convert to a $3$-cocycle in $H^3(S_n,\mathbb Z/12)$ and verify it represents the generator.
\item \textbf{Disproof track:} check whether the asserted $\mathbb Z/12$-summand actually exists for the $n$ in question (small $n$ can fail); if not, produce a counterexample $n$.
\end{itemize}

\subsection*{4) WORK}
\paragraph{Lemma 1 (UCT yields a $\mathbb Z/12$-summand in cohomology from $\mathbb Z/12$-summand in homology).}
Let $G$ be any group and $A=\mathbb Z/12$. There is a natural short exact sequence
\[
0\to \operatorname{Ext}^1_{\mathbb Z}(H_{2}(G,\mathbb Z),A)\to H^3(G,A)\to \operatorname{Hom}_{\mathbb Z}(H_{3}(G,\mathbb Z),A)\to 0.
\]
\emph{Proof.} This is the universal coefficient theorem for cohomology of a chain complex of free abelian groups applied to a model computing $H_\ast(G,\mathbb Z)$ (e.g. the bar resolution). \qed

\paragraph{Corollary 2.}
If $H_3(S_n,\mathbb Z)$ has a direct summand isomorphic to $\mathbb Z/12$, then $H^3(S_n,\mathbb Z/12)$ has a direct summand isomorphic to $\mathbb Z/12$.
\emph{Proof.} $\operatorname{Hom}(\mathbb Z/12,\mathbb Z/12)\cong \mathbb Z/12$ embeds as a direct summand of the right term. \qed

\paragraph{What remains (gap).}
The corollary gives existence of the abstract summand but does not produce an explicit cocycle representative. Constructing one requires an explicit cycle-level model for the $12$-torsion class in $H_3(S_n,\mathbb Z)$ and then explicitly evaluating cochains.

\subsection*{5) VERIFICATION}
The UCT exact sequence is standard and its hypotheses are satisfied because a free resolution of $\mathbb Z$ over $\mathbb ZG$ yields a chain complex of free abelian groups after tensoring with $\mathbb Z$. The corollary uses only the existence of a split injection $\mathbb Z/12\hookrightarrow H_3$; if $H_3$ merely \emph{contains} an element of order $12$ without splitting, the conclusion about a direct summand needs extra argument.

\subsection*{6) FINAL}
\textbf{UNRESOLVED.}
\begin{itemize}[leftmargin=*]
\item (i) Fully proved partial result: the UCT reduction above.
\item (ii) First gap: constructing an explicit normalized $3$-cocycle $c:S_n^3\to \mathbb Z/12$ representing the $12$-torsion generator.
\item (iii) Next moves: (1) compute an explicit bar-resolution cycle representing the $12$-torsion in $H_3(S_n,\mathbb Z)$; (2) restrict to smaller subgroups detecting the class (e.g. $S_4$, $A_4$, or dihedral subgroups) and inflate; (3) use computational group cohomology (e.g. HAP/GAP) to extract a cocycle and then simplify symbolically.
\item (iv) Likely minimal counterexample structure: small $n$ where $H_3(S_n,\mathbb Z)$ has no $12$-torsion summand (so the request must specify stable $n$).
\end{itemize}

