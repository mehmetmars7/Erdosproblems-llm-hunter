\section{Problem 5: Ultraweakly dense $*$-subalgebras and MASAs in von Neumann algebras}

\subsection*{1) FORMAL RESTATEMENT}
Let $A$ be a von Neumann algebra (over $\C$), equipped with its ultraweak (weak-$*$) topology coming from a fixed predual.
Let $B\subset A$ be a $*$-subalgebra that is \emph{ultraweakly dense} in $A$ (i.e.\ $\overline{B}^{\,\sigma\text{-weak}}=A$).

A \emph{MASA} (maximal abelian $*$-subalgebra) of $B$ means a $*$-subalgebra $D\subset B$ that is abelian and maximal with respect to inclusion among abelian $*$-subalgebras of $B$.

Assumption:
\begin{quote}
For every MASA $D\subset B$, the set $D$ is ultraweakly closed in $A$.
\end{quote}

Question: Must $B=A$ (i.e.\ is $B$ automatically ultraweakly closed)?

The attachment also asks a variant for a $\mathrm{II}_1$ factor $A$:
if $B\subset A$ is ultraweakly dense and for every hyperfinite subalgebra $R\subset A$ the intersection $R\cap B$ is ultraweakly closed in $A$, must $B=A$?

\medskip
\noindent\textbf{Stress points.}
\begin{itemize}[leftmargin=2em]
\item If $B$ were assumed to be a $C^*$-algebra, then ultraweak density plus standard results can force $B=A$ (Pedersen-type results), but here $B$ is only a $*$-algebra.
\item Maximal abelian in $B$ does not automatically imply maximal abelian in $A$.
\item The problem is known to be easier when $A$ has no $\mathrm{II}_1$ summand; the $\mathrm{II}_1$ factor case is the hard regime.
\end{itemize}

\subsection*{2) QUICK LITERATURE/CONTEXT CHECK (web)}
The MO attachment mentions results of Behncke--Cuntz and Pedersen that (under additional assumptions, e.g.\ when $A$ has no $\mathrm{II}_1$ part) the answer is affirmative. The question remains about the general case and in particular $\mathrm{II}_1$ factors. As of \date{}, the MO page shows no posted answer.

\subsection*{3) ATTACK PLAN}
\textbf{Proof-track candidates.}
\begin{enumerate}[label=(P\arabic*),leftmargin=2.5em]
\item Show that the MASA-closedness hypothesis forces $B$ to be closed under enough spectral/functional calculus operations to become a $C^*$-algebra; then apply known theorems to conclude $B=A$.
\item Use masa-structure to show that for each $b\in B$, the von Neumann algebra $W^*(b)$ generated by $b$ lies in $B$, implying ultraweak closure.
\end{enumerate}
\textbf{Disproof-track candidates.}
\begin{enumerate}[label=(D\arabic*),leftmargin=2.5em]
\item Construct a proper dense $*$-subalgebra $B$ in a $\mathrm{II}_1$ factor (e.g.\ the hyperfinite $\mathrm{II}_1$ factor) engineered so that any MASA of $B$ is already ultraweakly closed in $A$.
\item Start from a natural proper dense $*$-subalgebra (like an increasing union of finite-dimensional subalgebras) and analyze why it fails the MASA-closedness property; then try to repair it.
\end{enumerate}

\subsection*{4) WORK}

\paragraph{PHASE 1: a natural dense $*$-subalgebra that fails the hypothesis.}
Let $A=B(\ell^2)$ and let $B$ be the $*$-subalgebra generated by finite-rank operators and the identity.
Then $B$ is ultraweakly dense in $B(\ell^2)$, but a typical MASA of $B$ (e.g.\ diagonal operators with eventually constant diagonal) is not ultraweakly closed in $A$ (its ultraweak closure is the full diagonal von Neumann algebra $\ell^\infty$).
Thus the hypothesis is genuinely restrictive; it rules out many standard dense $*$-subalgebras.

\paragraph{Strongest fully proved partial result obtained here.}
The MASA-closedness hypothesis is strong enough to exclude the standard ``finite rank'' and ``algebraic inductive limit'' dense subalgebras in basic examples such as $B(\ell^2)$ and the hyperfinite $\mathrm{II}_1$ factor built as an increasing union of matrix algebras.

\paragraph{First hard gap.}
I do not have either:
\begin{itemize}[leftmargin=2em]
\item a proof that the hypothesis forces $B$ to be ultraweakly closed, even in the hyperfinite $\mathrm{II}_1$ factor; nor
\item an explicit construction of a proper dense $*$-subalgebra $B$ satisfying the hypothesis.
\end{itemize}

\subsection*{5) VERIFICATION}
\begin{itemize}[leftmargin=2em]
\item The counterexample attempt in Phase 1 is only used to show the hypothesis is nontrivial; it does not address the actual question because it fails the hypothesis (as intended).
\end{itemize}

\subsection*{6) FINAL}
\textbf{UNRESOLVED.}

\noindent (i) Strongest partial: identification of standard dense $*$-subalgebras that fail the MASA-closedness condition; thus any counterexample must be more subtle.

\noindent (ii) First gap: no mechanism found to upgrade MASA-closedness to closure under functional calculus or to show $B$ must contain $W^*(b)$ for each $b\in B$.

\noindent (iii) Top 3 next moves:
\begin{enumerate}[leftmargin=2.5em]
\item In a $\mathrm{II}_1$ factor, attempt to use conditional expectations onto MASAs/hyperfinite subalgebras (when they exist) to approximate commuting elements and deduce maximality in $A$.
\item Examine whether the MASA-closedness condition forces $B$ to be ``spectrally invariant'' in $A$ (closed under holomorphic functional calculus), which would imply $B=A$.
\item Search for a candidate $B$ built from a countable dense set closed under $*$ and algebra operations but constructed to contain entire MASAs (hence ultraweakly closed on those), while still being proper.
\end{enumerate}

\noindent (iv) Likely minimal counterexample structure (if the answer is ``no''): a dense $*$-subalgebra $B$ in a $\mathrm{II}_1$ factor that is large enough to contain many complete MASAs but still misses some non-abelian ultraweak limits, preventing it from being a von Neumann algebra.
