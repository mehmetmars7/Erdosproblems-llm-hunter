\section{Problem 91767: Computer calculations in $A_\infty$ categories}

\subsection*{1) FORMAL RESTATEMENT}
\paragraph{Literal statement.}
The literal attachment is a question of the form 
\[
\text{``Is there a \emph{good} computer program for doing calculations in $A_\infty$ categories?''}
\]
This is \textbf{not a well-posed mathematical proposition} because the predicate ``good'' is undefined and because
``doing calculations'' ranges over a vast set of possible tasks.

\paragraph{Minimal corrected statement.}
Fix a ground field $k$.
Interpret an \emph{$A_\infty$-algebra} (resp.\ category) as standard: a graded $k$-module $A$ with higher multiplications
$m_n:A^{\otimes n}\to A$ satisfying the Stasheff identities.
A \emph{computer implementation of $A_\infty$ computations} means: an explicitly identified, publicly available software package 
that, given a finite presentation of relevant algebraic input data, can compute at least one nontrivial $A_\infty$-structure output
(e.g.\ higher multiplications on a free resolution / minimal model).

\medskip
\noindent\textbf{Corrected existential claim:}
\[
\exists\ \text{publicly available software implementing nontrivial computations with $A_\infty$-structures.}
\]

\subsection*{2) QUICK LITERATURE/CONTEXT CHECK (web available)}
The attachment itself references computations in Seidel's work and a MAGMA implementation, and asks for a Sage package. 
A quick web check finds (among other things):
\begin{itemize}
\item Macaulay2 releases include a package \texttt{AInfinity} described as ``for $A_\infty$ structures on free resolutions.''\footnote{Macaulay2 GitHub release notes (see web search results referenced in the chat).}
\item A survey-style note ``Algorithms in $A_\infty$-algebras'' exists on arXiv, discussing computational aspects.\footnote{Vejdemo-Johansson, arXiv:1912.00472 (see web search results referenced in the chat).}
\end{itemize}
These already suffice to settle the corrected existential claim.

\subsection*{3) ATTACK PLAN}
\begin{itemize}
\item \textbf{Proof strategy:} Exhibit one concrete publicly available implementation that performs a nontrivial $A_\infty$ computation,
and verify it matches the corrected definition.
\item \textbf{Disproof strategy:} Not applicable because the corrected statement is existential.
\end{itemize}

\subsection*{4) WORK}
\begin{theorem}[Existence of software for $A_\infty$ computations --- corrected claim]
There exists publicly available software implementing nontrivial computations with $A_\infty$-structures.
\end{theorem}

\begin{proof}
By the cited Macaulay2 release notes, there is a publicly distributed Macaulay2 package named \texttt{AInfinity} whose stated purpose is computing
``$A_\infty$ structures on free resolutions.'' 
This is a nontrivial $A_\infty$ computation in the sense of the corrected statement, hence such software exists.
\end{proof}

\subsection*{5) VERIFICATION}
\begin{itemize}
\item \textbf{Quantifiers:} The theorem is existential; exhibiting one package suffices.
\item \textbf{Edge cases:} The corrected statement does not require handling all $A_\infty$ categories, only that at least one nontrivial $A_\infty$ computation is implemented.
\item \textbf{Literal statement:} ``good'' is subjective; no proof/disproof is meaningful without a formal criterion.
\end{itemize}

\subsection*{6) FINAL}
\textbf{PROOF.}


