\section*{Problem 269387: Uniformization over finite fields (common finite \'{e}tale cover?)}
\addcontentsline{toc}{section}{Problem 269387: Uniformization over finite fields (common finite \'{e}tale cover?)}

\subsection*{1) FORMAL RESTATEMENT}
Fix a prime power $q=p^r$ and let $k=\overline{\mathbb{F}_q}$.
For smooth projective \emph{geometrically connected} curves $C_1,C_2$ over $k$ with genera $g(C_i)\ge 2$,
ask whether there exists a smooth projective geometrically connected curve $D$ over $k$ and finite \'{e}tale morphisms
$f_i:D\to C_i$ for $i=1,2$.

Equivalently: do the profinite groups $\pi_1^{\mathrm{\acute{e}t}}(C_1)$ and $\pi_1^{\mathrm{\acute{e}t}}(C_2)$ admit an isomorphic open subgroup?

\subsection*{2) QUICK LITERATURE/CONTEXT CHECK}
The MathOverflow post records several remarks: in characteristic $0$ over $\mathbb{C}$ the answer is negative by a dimension/moduli argument; in characteristic $0$ over countable fields a negative answer follows from a theorem of Mochizuki; and there is a conjecture of Bogomolov--Tschinkel about allowing \emph{dominant} (possibly ramified) maps.

No accepted answer is given in the post, and the question appears open there.

\subsection*{3) ATTACK PLAN}
\emph{Proof-track ideas:}
(1) Try to build a very large common \'{e}tale cover by controlling finite quotients of $\pi_1$ (profinite group theory + patching).
(2) Reduce to special ``universal'' curves whose \'{e}tale fundamental group contains large families of open subgroups.

\emph{Disproof-track ideas:}
(1) Find a commensurability invariant of profinite groups (stable under passing to open subgroups) that differs for two curves.
(2) Use $p$-primary invariants (e.g.\ structure of maximal pro-$p$ quotient, $p$-rank constraints, existence of certain finite simple quotients, etc.).

\emph{Best current path:} attempt invariant-based disproof, but no complete invariant obstruction is currently in hand here.

\subsection*{4) WORK (partial results only)}

\begin{proposition}[Necessary genus compatibility]
If there is a common finite \'{e}tale cover $D\to C_i$ of degrees $n_i$, then
\[
g(D)-1 = n_i\,(g(C_i)-1)\quad (i=1,2).
\]
In particular, $g(D)-1$ is a common multiple of $g(C_1)-1$ and $g(C_2)-1$.
\end{proposition}

\begin{proof}
For a finite \'{e}tale map $f:D\to C$ of degree $n$, the Riemann--Hurwitz formula with ramification term $0$ gives
$2g(D)-2 = n(2g(C)-2)$, i.e.\ $g(D)-1=n(g(C)-1)$.
Apply this to $D\to C_i$.
\end{proof}

\begin{proposition}[Always possible with ramification]
For any two smooth projective curves $C_1,C_2$ over an algebraically closed field, there exists a smooth projective curve $E$ and finite \emph{surjective} morphisms $E\to C_i$ (in general ramified).
\end{proposition}

\begin{proof}
Choose nonconstant morphisms $\phi_i:C_i\to\mathbb{P}^1$ (they exist because $C_i$ are projective curves).
Form the fiber product $C_1\times_{\mathbb{P}^1} C_2$ and take an irreducible component $Z$ dominating $\mathbb{P}^1$.
Let $E$ be the normalization of $Z$; then the projections induce finite surjective morphisms $E\to C_i$.
\end{proof}

\subsection*{5) VERIFICATION}
The partial results above are standard and fully checked (Riemann--Hurwitz; normalization of fiber products).
They do not resolve the key obstruction that the morphisms be \'{e}tale.

\subsection*{6) FINAL: \textbf{UNRESOLVED}}
(i) Strongest proved partial result here: the necessary genus condition and the existence of a common finite \emph{ramified} cover (Section 4).\\
(ii) First gap: produce (or obstruct) a common cover with \emph{no ramification}.\\
(iii) Next moves:
\begin{itemize}
\item Study commensurability invariants of $\pi_1^{\mathrm{\acute{e}t}}(C)$ in characteristic $p$ beyond the prime-to-$p$ quotient.
\item Test special families (Igusa curves, Drinfeld/Deligne--Lusztig type curves) as potential ``universal'' sources of common covers.
\item Use finiteness theorems for curves with prescribed fundamental group to constrain how large the ``commensurability class'' can be.
\end{itemize}
(iv) What a minimal counterexample would likely look like: two curves with genuinely different $p$-primary fundamental group behavior that persists under all open subgroups (e.g.\ an invariant of the pro-$p$ completion stable under passage to open subgroups).

