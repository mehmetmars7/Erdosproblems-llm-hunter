\section{Do there exist infinite-dimensional Banach spaces in which every bounded linear operator attains its norm? (MO 232291)}

\subsection*{1) FORMAL RESTATEMENT}
\paragraph{Decision problem.}
Determine whether
\[\exists\,\text{infinite-dimensional Banach }X\ \forall\,T\in\cL(X)\ \exists x\in X\ (\|x\|=1\ \wedge\ \|Tx\|=\|T\|).\]

\paragraph{Stress points.}
\begin{itemize}[leftmargin=2em]
\item Real vs complex scalars (not specified).
\item Separability, reflexivity, approximation properties (not assumed but strongly constrained if the property holds).
\item ``Every operator'' includes non-compact operators.
\end{itemize}

\subsection*{2) QUICK LITERATURE/CONTEXT CHECK}
The question is explicitly described as open in the MO post itself. A 2019 preprint about Banach spaces over $\mathrm{GF}(2)$ remarks that in the usual (real/complex) Banach-space setting the existence of an infinite-dimensional space on which every bounded operator attains its norm ``is still undetermined''; it then exhibits a positive example over $\mathrm{GF}(2)$. \href{https://arxiv.org/abs/1907.11939}{(arXiv:1907.11939)}.

More recent operator-theory literature studies existence of non-norm-attaining operators in various contexts, but the ``all operators attain'' property on a single infinite-dimensional real/complex Banach space remains an exceptional and unresolved possibility as far as the cited sources indicate.

\subsection*{3) ATTACK PLAN}
\paragraph{Proof track (existence).}
Try to construct a separable reflexive Banach space without approximation property and with strong geometric constraints forcing norm attainment for all operators.

\paragraph{Disproof track (nonexistence).}
Try to prove that every infinite-dimensional Banach space admits a bounded operator failing norm attainment, perhaps by producing a diagonal-like operator in some basis, or by structural results on $\cL(X)$.

\paragraph{Best path now.}
Since the global question appears open, we give rigorous partial results: (a) many classical spaces fail the property; (b) the property forces reflexivity of $X$.

\subsection*{4) WORK}

\subsubsection*{Phase 1: falsify the property on a standard infinite-dimensional space}
\begin{lemma}[A concrete non-norm-attaining operator on $\ell_2$]
Let $X=\ell_2$ and define $T\in\cL(\ell_2)$ by $T(e_n)=\alpha_n e_n$ where $\alpha_n=1-\frac1n$.
Then $\|T\|=1$ but $T$ does not attain its norm.
\end{lemma}

\begin{proof}
First, $\|T\|=\sup_n |\alpha_n|=1$ since $\alpha_n\nearrow 1$.
Now take any unit vector $x=(x_n)\in\ell_2$ with $\|x\|_2=1$.
Then
\[\|Tx\|_2^2=\sum_{n\ge 1}\alpha_n^2|x_n|^2.
\]
Let $\delta_n:=1-\alpha_n^2>0$. Then
\[1-\|Tx\|_2^2=\sum_{n\ge 1}(1-\alpha_n^2)|x_n|^2=\sum_{n\ge 1}\delta_n|x_n|^2.
\]
If $x\neq 0$, at least one coordinate satisfies $|x_n|^2>0$, hence the sum contains at least one strictly positive term and therefore is strictly positive. Thus $\|Tx\|_2^2<1$ for every unit vector $x$, i.e. $\|Tx\|_2<\|T\|$. So $T$ does not attain its norm.
\end{proof}

\subsubsection*{Phase 1/2: a necessary condition --- reflexivity of $X$}
\begin{lemma}[If every operator attains, then every functional attains]
\label{lem:functionals}
Let $X$ be a Banach space such that every bounded operator $T\in\cL(X)$ attains its norm.
Then every bounded linear functional $f\in X^*$ attains its norm on the unit sphere.
\end{lemma}

\begin{proof}
Fix $f\in X^*$. Consider $f$ as an operator $T_f:X\to\RR$ (or $\CC$) given by $T_f(x)=f(x)$.
Its operator norm equals $\|f\|$.
By assumption, $T_f$ attains its norm, so there exists $x\in X$ with $\|x\|=1$ and $|f(x)|=\|f\|$.
This is exactly norm attainment for $f$.
\end{proof}

\begin{theorem}[Necessary condition via James' theorem]
If $X$ is a real (or complex) Banach space such that every operator in $\cL(X)$ attains its norm, then $X$ is reflexive.
\end{theorem}

\begin{proof}
By Lemma~\ref{lem:functionals}, every functional in $X^*$ attains its norm on the unit ball.
By James' theorem (characterization of reflexivity), this implies that $X$ is reflexive.
\end{proof}

\subsection*{5) VERIFICATION}
\paragraph{Operator counterexample check.}
The diagonal operator on $\ell_2$ has norm 1. The strict inequality $\|Tx\|<1$ for all $\|x\|=1$ follows because $\sum\delta_n|x_n|^2$ is a sum of nonnegative terms with at least one strictly positive term.

\paragraph{Quantifier check for reflexivity implication.}
Lemma~\ref{lem:functionals} uses only that scalar field matches the target of $f$ and that every bounded operator attains its norm.

\subsection*{6) FINAL}
\textbf{UNRESOLVED.}

\paragraph{(i) Strongest fully proved partial result obtained here.}
We proved that many classical infinite-dimensional spaces (e.g. $\ell_2$) do \emph{not} have the property by explicit non-norm-attaining operators, and that if a space had the property then it must be reflexive (by James' theorem).

\paragraph{(ii) First gap.}
We cannot prove or refute the existence of an infinite-dimensional real/complex Banach space $X$ such that every $T\in\cL(X)$ attains its norm.

\paragraph{(iii) Top 3 next moves.}
\begin{enumerate}[leftmargin=2em]
\item Reduce the problem to separable reflexive spaces without approximation properties using known implications (Holub/Kalton-type constraints).
\item Attempt to prove a general construction of a non-norm-attaining operator on any infinite-dimensional Banach space (perhaps via basic sequences or noncompactness arguments).
\item Investigate whether recent characterizations of existence of non-norm-attaining operators in $\cL(E,F)$ can be specialized to $E=F=X$ and shown to hold for every infinite-dimensional $X$.
\end{enumerate}

\paragraph{(iv) What a minimal counterexample (a positive example) would likely look like.}
Any positive example must be an extremely rigid infinite-dimensional reflexive space, very likely lacking the approximation property and avoiding standard structures (bases, complemented copies) that allow diagonal-type constructions of non-norm-attaining operators.
