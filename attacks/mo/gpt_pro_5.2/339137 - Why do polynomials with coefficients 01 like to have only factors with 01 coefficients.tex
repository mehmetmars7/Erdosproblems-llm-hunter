\section{MO 339137: ``Why do polynomials with coefficients 0,1 like to have only factors with 0,1 coefficients?''}
\label{sec:mo339137}
\noindent\textbf{MathOverflow link:} \url{https://mathoverflow.net/questions/339137/why-do-polynomials-with-coefficients-0-1-like-to-have-only-factors-with-0-1} (accessed 2026-01-16).

\subsection*{1) FORMAL RESTATEMENT}
We extract the central conjecture.

\medskip
\noindent\textbf{Definitions.}
\begin{itemize}[leftmargin=2em]
\item A \emph{$\{0,1\}$-polynomial} is a polynomial $R(x)=\sum_{k=0}^d r_k x^k\in\RR[x]$ with each coefficient $r_k\in\{0,1\}$.
\item Let $P(x)=\sum_{i=0}^a p_i x^i$ and $Q(x)=\sum_{j=0}^b q_j x^j$ be polynomials with \emph{nonnegative coefficients}: $p_i\ge 0$, $q_j\ge 0$.
\item ``Monic'' means $p_a=1$ and $q_b=1$.
\end{itemize}

\medskip
\noindent\textbf{Conjecture (C).}
\begin{quote}
For all $a,b\in\NN$ and all monic $P,Q\in\RR[x]$ with nonnegative coefficients, if $R:=PQ$ is a $\{0,1\}$-polynomial, then every coefficient of $P$ and $Q$ lies in $\{0,1\}$.
\end{quote}

\noindent\textbf{Edge cases / stress points.}
\begin{itemize}[leftmargin=2em]
\item Constant term: $r_0=p_0q_0\in\{0,1\}$. If $r_0=1$ then $(p_0,q_0)$ could \emph{a priori} be any positive pair with product $1$ (e.g. $2$ and $1/2$).
\item The conjecture is trivially true if one factor has degree $0$ (then it equals $1$).
\item ``Nonnegative'' is essential: there are easy counterexamples with negative coefficients.
\end{itemize}

\subsection*{2) QUICK LITERATURE/CONTEXT CHECK}
From the MathOverflow post (no posted answers as of 2026-01-16), key context:
\begin{itemize}[leftmargin=2em]
\item The conjecture is motivated by extensive computer experimentation; the post reports verification by Gr\"obner basis methods up to total degree $32$.
\item A comment in the thread states that if the $\{0,1\}$-polynomial $R$ has support an arithmetic progression, then (C) follows from a result of Krasner--Ranulac (1937).
\item The analogous statement for formal power series is false.
\end{itemize}
(No attempt is made here to independently verify the 1937 reference beyond what is stated in the thread.)

\subsection*{3) ATTACK PLAN}
\textbf{Proof-track ideas.}
\begin{enumerate}[leftmargin=2em]
\item Translate to a nonnegative convolution constraint: coefficients of $R$ are $r_k=\sum_{i+j=k}p_iq_j\in\{0,1\}$. Seek rigidity from extremal/uniqueness of representations.
\item Use monotonicity at the top degrees: since $p_a=q_b=1$, the ``tails'' of $R$ reveal coefficients of $P$ and $Q$ directly (gives immediate bounds).
\item If coefficients are rational/algebraic, attempt integrality or polyhedral extreme-point arguments.
\end{enumerate}

\textbf{Disproof-track ideas.}
\begin{enumerate}[leftmargin=2em]
\item Search for small-degree counterexamples with coefficients in a small rational grid (e.g. $\{0,1/2,1\}$) by brute force enumeration.
\item Attempt to construct a ``weighted unique representation'' convolution where two or more representations sum to $1$.
\end{enumerate}

We pursue (i) basic necessary conditions + (ii) small rational-grid search. No full proof emerged.

\subsection*{4) WORK}
\subsubsection*{Lemma 4.1 (Coefficient bounds $\le 1$)}
\textbf{Claim.} If $P,Q$ are monic with nonnegative coefficients and $R=PQ$ has all coefficients $\le 1$, then $0\le p_i\le 1$ for all $i$ and $0\le q_j\le 1$ for all $j$.

\textbf{Proof.}
Write $P(x)=\sum_{i=0}^a p_i x^i$ and $Q(x)=\sum_{j=0}^b q_j x^j$ with $p_a=q_b=1$ and $p_i,q_j\ge 0$.
For each $j\in\{0,1,\dots,b\}$, the coefficient of $x^{a+j}$ in $R=PQ$ equals
\[
 r_{a+j}=\sum_{i=0}^a p_i q_{a+j-i} \ge p_a q_j = 1\cdot q_j.
\]
Since $r_{a+j}\le 1$, we obtain $q_j\le 1$. Similarly, for each $i$, consider coefficient $r_{b+i}\ge q_b p_i = p_i$, so $p_i\le 1$. Nonnegativity is already assumed. \qed

\subsubsection*{Lemma 4.2 (Zero coefficients force zero products)}
\textbf{Claim.} If $r_k=0$ for some $k$, then for every pair $(i,j)$ with $i+j=k$ we have $p_i q_j=0$.

\textbf{Proof.}
By Cauchy product, $r_k=\sum_{i+j=k} p_i q_j$ with all summands $\ge 0$. If the sum is $0$, each summand must be $0$. \qed

\subsubsection*{Corollary 4.3 (Tail zeros force factor zeros)}
Let $a=\deg P$. If $r_{a+j}=0$ then $q_j=0$. Likewise, if $r_{b+i}=0$ then $p_i=0$.

\textbf{Proof.} $r_{a+j}\ge p_a q_j=q_j$ as in Lemma 4.1. \qed

\subsubsection*{Remark 4.4 (Support sets and weighted sumset)}
Define supports $A:=\{i: p_i>0\}\subseteq\{0,\dots,a\}$ and $B:=\{j: q_j>0\}\subseteq\{0,\dots,b\}$. Lemma 4.2 implies:
\begin{itemize}[leftmargin=2em]
\item If $r_k=0$ then there is \emph{no} representation $k=i+j$ with $i\in A$, $j\in B$.
\item If $r_k=1$ then the nonnegative weights $(p_i)_{i\in A}$ and $(q_j)_{j\in B}$ satisfy $\sum_{i+j=k} p_i q_j = 1$.
\end{itemize}
If all coefficients were in $\{0,1\}$, the condition $r_k\in\{0,1\}$ would force each $k$ to have at most one representation $k=i+j$ with $i\in A$, $j\in B$.
In the weighted setting, multiple representations are allowed as long as the weighted sum equals $1$.

\subsubsection*{Computation (small rational-grid brute force)}
We performed exhaustive enumeration searches over small coefficient grids.

\begin{itemize}[leftmargin=2em]
\item Grid $\{0,1/2,1\}$, degrees $\deg P,\deg Q\le 6$:\\
No counterexample found among $\approx 1.86\times 10^6$ pairs of monic polynomials.
\item Grid $\{0,1/3,2/3,1\}$, degrees $\deg P,\deg Q\le 5$:\\
No counterexample found among $\approx 3.91\times 10^6$ pairs.
\end{itemize}
These checks do \emph{not} prove (C), but they give additional evidence that low-complexity fractional coefficients do not easily produce a $\{0,1\}$ product.

\subsection*{5) VERIFICATION}
\begin{itemize}[leftmargin=2em]
\item Lemma 4.1 uses only the monic condition and coefficientwise bounds $r_k\le 1$; it is correct.
\item Lemma 4.2 is an immediate consequence of nonnegativity.
\item Computation: the brute-force search only samples a tiny (but exhaustive within the grid) subset of possible real coefficients; failure to find a counterexample is not decisive.
\end{itemize}

\subsection*{6) FINAL}
\begin{center}
\textbf{UNRESOLVED}
\end{center}


