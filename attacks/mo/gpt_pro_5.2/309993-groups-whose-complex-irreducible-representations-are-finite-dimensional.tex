\section{Problem 309993: Groups whose complex irreducible representations are finite-dimensional}

\subsection*{1) FORMAL RESTATEMENT}

\paragraph{Literal statement extracted from the attachment.}
Let $G$ be a group. Consider the property:
\begin{quote}
(\*) every irreducible (left) complex representation of $G$ is finite-dimensional over $\C$.
\end{quote}
The question asks whether
\[
G \text{ is virtually abelian}\quad\Longleftrightarrow\quad (\*) \text{ holds}.
\]

\paragraph{Ambiguity.}
The attachment says: ``I am primarily interested in countable groups.''  This can be read either as:
\begin{itemize}[leftmargin=2.2em]
\item (Version L, \emph{literal}): the equivalence is asserted/asked for \emph{all} groups, with special interest in the countable case; or
\item (Version C, \emph{corrected}): restrict the equivalence to \emph{countable} groups.
\end{itemize}

These are genuinely different: for uncountable groups there are easy counterexamples even in the abelian case.

\paragraph{Minimal corrected statement.}
The minimal correction matching the ``primarily interested'' intent is:

\begin{quote}
(\textbf{C}) For \emph{countable} groups $G$, does $(\*)$ imply that $G$ is virtually abelian?
\end{quote}

I will \emph{disprove the literal Version L} by an explicit abelian counterexample.  I do \emph{not} resolve the corrected countable Version C.

\subsection*{2) QUICK LITERATURE/CONTEXT CHECK}

The attachment itself mentions that for countable groups the problem reduces to finitely generated groups, and cites partial results (solvable groups, locally finite groups, bounded-degree results of Isaacs--Passman, etc.).  I do not reproduce those deep results here; instead I focus on an explicit disproof of the literal equivalence as stated without the countability restriction.

\subsection*{3) ATTACK PLAN}

\begin{itemize}[leftmargin=2.2em]
\item \textbf{Disproof track (chosen):} Find a virtually abelian group $G$ that admits an \emph{explicit} infinite-dimensional irreducible complex representation.
For abelian $G$, irreducible representations correspond to simple modules over the commutative group algebra $\C[G]$.
If $\C[G]$ surjects onto a field extension $K$ of $\C$ of infinite $\C$-dimension, then $K$ is an infinite-dimensional irreducible module.
\item \textbf{Proof track (not chosen):} Attempt to prove that for countable $G$, $(\*)$ forces strong structure (residual finiteness, PI group algebra, etc.) leading to virtual abelianness.  This is known to be delicate and appears open in general.
\end{itemize}

\subsection*{4) WORK}

\subsubsection*{Phase 1 --- Constructing a counterexample}

Let $K:=\C(t)$ be the field of rational functions in one indeterminate $t$ over $\C$.
Let $G:=K^\times$ be its multiplicative group.  Then $G$ is abelian, hence certainly virtually abelian.

Define a complex representation
\[
\rho:G\longrightarrow \mathrm{GL}_{\C}(K)
\]
by letting $g\in K^\times$ act on the $\C$-vector space $K$ by multiplication:
\[
\rho(g)(x):=g\cdot x\qquad(x\in K).
\]

\begin{lemma}\label{lem:K-infdim}
As a $\C$-vector space, $K=\C(t)$ is infinite-dimensional.
\end{lemma}

\begin{proof}
The elements $1,t,t^2,t^3,\dots$ are $\C$-linearly independent in $\C(t)$ (a nontrivial linear relation would give a nonzero polynomial vanishing identically).
Thus $\dim_{\C}K=\infty$.
\end{proof}

\begin{lemma}\label{lem:K-irreducible}
The representation $\rho$ of $G=K^\times$ on the $\C$-vector space $K$ is irreducible.
\end{lemma}

\begin{proof}
Let $W\subseteq K$ be a nonzero $\C$-subspace that is $G$-stable, i.e.\ $\rho(g)W\subseteq W$ for all $g\in K^\times$.
Pick $0\neq w\in W$.  Since $w\in K^\times$, its inverse $w^{-1}\in K^\times$ exists.
By $G$-stability, $\rho(w^{-1})(w)=w^{-1}w=1\in W$.

Now for any $x\in K$, if $x=0$ then $x\in W$ because $0\in W$.
If $x\neq 0$ then $x\in K^\times$ and $x=\rho(x)(1)\in W$ because $1\in W$ and $W$ is $G$-stable.
Thus $W=K$.

Therefore the only $G$-stable subspaces are $0$ and $K$, i.e.\ $\rho$ is irreducible.
\end{proof}

\subsubsection*{Conclusion}

By Lemma~\ref{lem:K-infdim}, $\rho$ is infinite-dimensional, and by Lemma~\ref{lem:K-irreducible} it is irreducible.
Hence $G=K^\times$ is (virtually) abelian but does \emph{not} satisfy property $(\*)$.

\subsection*{5) VERIFICATION}

\begin{itemize}[leftmargin=2.2em]
\item $G=K^\times$ is a group (multiplicative group of a field) and is abelian.
\item The action $g\cdot x$ is $\C$-linear because $\C\subseteq K$ lies in the center and distributes over addition.
\item Every $g\in K^\times$ acts invertibly with inverse action $(g^{-1})\cdot x$.
\item Irreducibility is proved by the explicit ``$1\in W$'' argument; no external theorems are used.
\end{itemize}

\subsection*{6) FINAL}

\begin{center}
\textbf{COUNTEREXAMPLE/DISPROOF}
\end{center}

\noindent\textbf{Disproof of the literal equivalence (Version L).}
Let $K=\C(t)$ and $G=K^\times$. Then $G$ is abelian (hence virtually abelian), but it has the infinite-dimensional irreducible complex representation $\rho(g)(x)=gx$ on $K$.
Therefore it is \emph{false} that ``virtually abelian $\Rightarrow$ all irreducible complex representations are finite-dimensional'' if one allows arbitrary (in particular uncountable) groups.

\medskip
\noindent\textbf{Remark (status of the corrected countable version).}
This counterexample is uncountable.  The countable version (Version C) remains open here; I have not proved or disproved it in this report.
