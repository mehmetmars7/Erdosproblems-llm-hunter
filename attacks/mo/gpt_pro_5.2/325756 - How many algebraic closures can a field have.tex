\section{MO \#325756: How many algebraic closures can a field have?}
\MO{325756}{How many algebraic closures can a field have?}

\subsection*{1) FORMAL RESTATEMENT}
Work in ZF set theory (no axiom of choice). For a field $K$, an \emph{algebraic closure} is an algebraic field extension $L/K$ such that $L$ is algebraically closed. Two algebraic closures $L_1,L_2$ are considered the same if there is a $K$-isomorphism $L_1\cong_K L_2$.

Assume that $K$ has at least two non-isomorphic algebraic closures. Must $K$ then have:
\begin{enumerate}[leftmargin=*]
\item a third non-isomorphic algebraic closure?
\item infinitely many non-isomorphic algebraic closures?
\item Dedekind-infinitely many non-isomorphic algebraic closures?
\end{enumerate}

\subsection*{2) QUICK LITERATURE/CONTEXT CHECK}
In ZFC one proves existence and uniqueness up to $K$-isomorphism. In ZF, algebraic closures may fail to exist, and uniqueness can fail. The MO question states it is consistent that $\mathbb Q$ has two non-isomorphic algebraic closures.

\subsection*{3) ATTACK PLAN}
\begin{itemize}[leftmargin=*]
\item \textbf{Proof track:} attempt to show ``$\ge 2$ implies infinitely many'' by producing systematically new closures (e.g. varying cardinalities) from existing ones.
\item \textbf{Disproof track:} attempt to build a model of ZF in which a field has exactly two algebraic closures up to isomorphism.
\end{itemize}

\subsection*{4) WORK}
No purely algebraic (choice-free) construction is known here that forces a third closure from two.
What can be proved in ZFC (or weaker choice principles) is:
\begin{itemize}[leftmargin=*]
\item Existence of algebraic closures for all fields is equivalent to certain choice principles (e.g. the ultrafilter lemma suffices).
\item In ZFC the algebraic closure is unique up to $K$-isomorphism.
\end{itemize}
These do not resolve the ZF question.

\subsection*{5) VERIFICATION}
The ambiguity is set-theoretic: the number of algebraic closures can vary between models of ZF. Any definitive answer must specify the ambient axioms or show an implication provable in ZF alone.

\subsection*{6) FINAL}
\textbf{UNRESOLVED.}
\begin{itemize}[leftmargin=*]
\item (i) Fully proved partial result: none beyond standard ZFC facts; the question is inherently about ZF-consistency.
\item (ii) First gap: either a ZF-proof that ``two implies infinitely many'' or a consistency model with exactly two.
\item (iii) Next moves: (1) examine known permutation models where $\mathbb Q$ has non-isomorphic algebraic closures and see if they admit more; (2) analyze whether varying well-orderability/cardinality of closures can be forced; (3) study the extent to which ``Steinitz classification'' fails without choice.
\item (iv) Minimal counterexample structure: a model of ZF where the class of algebraic closures of $K$ splits into exactly two $K$-isomorphism types (often differing by well-orderability or cardinal).
\end{itemize}

