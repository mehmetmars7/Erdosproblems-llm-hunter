\section{MO 66897: Why do $H_4$ and $M_4$ have the same virtual Euler characteristic?}

\subsection*{1) FORMAL RESTATEMENT}
\paragraph{Terms and conventions.}
Let $\Mg{g}$ denote the moduli stack of smooth genus-$g$ curves over $\mathbb{C}$ (or over $\mathbb{Q}$; the Euler characteristic is topological). Let $H_g\subset \Mg{g}$ denote the hyperelliptic locus (stack) parameterizing hyperelliptic curves of genus $g\ge 2$.

Let $\chi(\cdot)$ denote the \emph{orbifold} Euler characteristic (also called the virtual Euler characteristic) of the corresponding orbifold/stack.

\paragraph{Concrete statement to prove.}
Assuming the two quoted formulas:
\[
\chi(\Mg{g})=\frac{\zeta(1-2g)}{2-2g},\qquad
\chi(H_g)= -\frac{(2g-1)!}{2(2g+2)!},
\]
prove the equality $\chi(\Mg{g})=\chi(H_g)$ holds for $g=4$ (and determine whether it holds for other $g$).

\subsection*{2) QUICK LITERATURE/CONTEXT CHECK}
Harer--Zagier computed $\chi(\Mg{g})$ and express it via zeta values.\footnote{J. Harer--D. Zagier, \emph{The Euler characteristic of the moduli space of curves}, Invent.\ Math.\ 85 (1986), 457--485. See e.g.\ the EUDML entry \url{https://eudml.org/doc/143377}.}
The formula for $\chi(H_g)$ follows from identifying the hyperelliptic locus with a quotient of $\Mg{0,2g+2}$ by a finite group (and using the known Euler characteristic of $\Mg{0,n}$).\footnote{E.g.\ see computations discussed in Bini's arXiv survey \url{https://arxiv.org/abs/math/0506083} and related literature on Euler characteristics of hyperelliptic moduli.}
The MO post itself contains both formulas.

\subsection*{3) ATTACK PLAN}
\paragraph{Proof strategies.}
\begin{enumerate}[leftmargin=2em]
\item Express $\chi(\Mg{g})$ using Bernoulli numbers $B_{2g}$, then set equal to $\chi(H_g)$ and solve for $g$.
\item Use explicit bounds on $|B_{2g}|$ to show the equality can only occur for small $g$, and check those $g$ directly.
\end{enumerate}

\paragraph{Disproof/construction strategies.}
Not applicable: the claimed equality for $g=4$ is an explicit numeric identity and can be checked exactly. The more interesting question is whether it holds beyond $g=4$.

\paragraph{Best path chosen.}
Do (1) and (2): reduce to an explicit equation for $B_{2g}$, then prove it has solutions $g=2,4$ only.

\subsection*{4) WORK}

\subsubsection*{Phase 1: tiny cases (computation)}
Direct computation using the stated formulas gives:
\[
\chi(\Mg{2})=\chi(H_2)=-\frac1{240},\quad
\chi(\Mg{3})=\frac1{1008}\neq -\frac1{672}=\chi(H_3),\quad
\chi(\Mg{4})=\chi(H_4)=-\frac1{1440}.
\]
So equality holds for $g=2$ and $g=4$ at least.

\subsubsection*{Lemma 1 (rewrite $\chi(\Mg{g})$ via Bernoulli numbers)}
\begin{quote}
\textbf{Claim.} For integers $g\ge 2$,
\[
\chi(\Mg{g})=\frac{B_{2g}}{4g(g-1)},
\]
where $B_{2g}$ is the $2g$th Bernoulli number (with the convention $\zeta(1-2g)=-\frac{B_{2g}}{2g}$).
\end{quote}

\paragraph{Proof.}
The standard identity between zeta values and Bernoulli numbers at negative integers is
\[
\zeta(1-2g)=-\frac{B_{2g}}{2g}\qquad (g\ge 1).
\]
Substitute into the Harer--Zagier formula:
\[
\chi(\Mg{g})=\frac{\zeta(1-2g)}{2-2g}
=\frac{-\frac{B_{2g}}{2g}}{2(1-g)}
=\frac{-\frac{B_{2g}}{2g}}{-2(g-1)}
=\frac{B_{2g}}{4g(g-1)}.
\]
\hfill$\square$

\subsubsection*{Lemma 2 (derive the hyperelliptic formula from $\Mg{0,2g+2}$)}
\begin{quote}
\textbf{Claim.} For $g\ge 2$,
\[
\chi(H_g)=-\frac{(2g-1)!}{2(2g+2)!}.
\]
\end{quote}

\paragraph{Proof.}
A hyperelliptic genus-$g$ curve is a double cover of $\mathbb{P}^1$ branched over $2g+2$ distinct points. Thus, as stacks, $H_g$ is (up to the hyperelliptic involution) the quotient of the moduli of $(2g+2)$ unordered marked points on $\mathbb{P}^1$, i.e.
\[
H_g \simeq \big[\Mg{0,2g+2}/S_{2g+2}\big] \big/ (\mathbb{Z}/2\mathbb{Z}),
\]
where the extra $\mathbb{Z}/2\mathbb{Z}$ accounts for the hyperelliptic involution (generic automorphism group of a hyperelliptic curve contains this involution).

Orbifold Euler characteristic satisfies $\chi([X/G])=\chi(X)/|G|$ for a finite group action with the standard orbifold convention. Applying this twice gives
\[
\chi(H_g)=\frac{1}{2}\cdot \frac{1}{(2g+2)!}\,\chi(\Mg{0,2g+2}).
\]
It is known that $\chi(\Mg{0,n})=(-1)^{n-3}(n-3)!$ for $n\ge 3$. Plugging $n=2g+2$ yields
\[
\chi(\Mg{0,2g+2}) = (-1)^{2g-1}(2g-1)! = -(2g-1)!.
\]
Therefore
\[
\chi(H_g)=\frac{1}{2}\cdot \frac{1}{(2g+2)!}\cdot \big(-(2g-1)!\big)
=-\frac{(2g-1)!}{2(2g+2)!}.
\]
\hfill$\square$

\subsubsection*{Theorem (exact characterization of when $\chi(\Mg{g})=\chi(H_g)$)}
\begin{quote}
\textbf{Claim.} For integers $g\ge 2$,
\[
\chi(\Mg{g})=\chi(H_g)\quad\Longleftrightarrow\quad g\in\{2,4\}.
\]
In particular, the $g=4$ coincidence occurs because $B_8=B_4=-\frac1{30}$.
\end{quote}

\paragraph{Proof.}
By Lemma 1 and Lemma 2, for $g\ge 2$ the equality $\chi(\Mg{g})=\chi(H_g)$ is equivalent to
\[
\frac{B_{2g}}{4g(g-1)} = -\frac{(2g-1)!}{2(2g+2)!}.
\]
Multiply both sides by $4g(g-1)$:
\[
B_{2g} = -2g(g-1)\frac{(2g-1)!}{(2g+2)!}.
\]
Since $(2g+2)!=(2g+2)(2g+1)(2g)(2g-1)!$, this simplifies to
\[
B_{2g} = -2g(g-1)\cdot \frac{1}{(2g+2)(2g+1)(2g)} = -\frac{g-1}{(2g+2)(2g+1)}.
\tag{$\ast$}
\]
So we need to solve $(\ast)$.

\medskip
\noindent\emph{Step 1: check small $g$.}
For $g=2$, $(\ast)$ becomes $B_4=-\frac{1}{5\cdot 6}=-\frac1{30}$, true.
For $g=3$, it becomes $B_6=-\frac{2}{7\cdot 8}=-\frac1{28}$, but $B_6=\frac1{42}$, false.
For $g=4$, it becomes $B_8=-\frac{3}{9\cdot 10}=-\frac1{30}$, true (indeed $B_8=-\frac1{30}$).
Thus $g=2,4$ are solutions.

\medskip
\noindent\emph{Step 2: exclude $g\ge 5$ by a size bound.}
For $g\ge 1$, the zeta/Bernoulli identity at positive even integers gives
\[
|B_{2g}| = \frac{2(2g)!\,\zeta(2g)}{(2\pi)^{2g}}.
\]
Since $\zeta(2g)>1$, we obtain the strict lower bound
\[
|B_{2g}| > \frac{2(2g)!}{(2\pi)^{2g}}.
\]
\noindent\emph{Label:} $(\dagger)$.
On the other hand, the right-hand side of $(\ast)$ satisfies for $g\ge 2$
\[
\left|\frac{g-1}{(2g+2)(2g+1)}\right|
< \frac{g}{(2g)(2g)}=\frac1{4g}.
\]
\noindent\emph{Label:} $(\ddagger)$.
So it suffices to show that for $g\ge 5$,
\[
\frac{2(2g)!}{(2\pi)^{2g}} \ge \frac1{4g},
\]
because then $(\dagger)$ and $(\ddagger)$ imply $|B_{2g}|>\frac1{4g}>\left|\frac{g-1}{(2g+2)(2g+1)}\right|$, contradicting $(\ast)$.

Let $L_g:=\frac{2(2g)!}{(2\pi)^{2g}}$. For $g\ge 5$,
\[
\frac{L_{g+1}}{L_g}=\frac{2(2g+2)!}{(2\pi)^{2g+2}}\cdot \frac{(2\pi)^{2g}}{2(2g)!}
=\frac{(2g+2)(2g+1)}{(2\pi)^2}.
\]
Using $\pi < 22/7$ (a classical bound), we have $\pi^2<\frac{484}{49}<10$, hence $(2\pi)^2=4\pi^2<40$. Therefore for $g\ge 5$ we get
\[
\frac{L_{g+1}}{L_g} = \frac{(2g+2)(2g+1)}{4\pi^2}
> \frac{(12)(11)}{40}=\frac{132}{40} > 1.
\]
Thus $L_g$ is strictly increasing for all $g\ge 5$, and in particular $L_g\ge L_5$ for $g\ge 5$.

It remains to bound $L_5$ from below:
\[
L_5=\frac{2\cdot 10!}{(2\pi)^{10}}.
\]
From $\pi^2<10$ we have $(2\pi)^{10} = \big((2\pi)^2\big)^5 < 40^5 = (2^3\cdot 5)^5 = 2^{15}\cdot 5^5$.
Hence
\[
L_5 > \frac{2\cdot 10!}{40^5}
= \frac{2\cdot 3628800}{102400000}
= \frac{7257600}{102400000}
= \frac{567}{8000}
> \frac{1}{20}.
\]
Since for $g\ge 5$ we have $\frac1{4g}\le \frac1{20}$, it follows that $L_g\ge L_5>\frac1{20}\ge \frac1{4g}$. This excludes any solutions to $(\ast)$ for $g\ge 5$.

Therefore the only solutions are $g=2$ and $g=4$.
\hfill$\square$

\subsection*{5) VERIFICATION}
\begin{itemize}[leftmargin=2em]
\item All manipulations are exact identities in $\mathbb{Q}$ (factorials, Bernoulli/zeta identities).
\item The only analytic input is the elementary bound $\pi<22/7$, used to deduce $\pi^2<10$ and thus a clean rational inequality.
\item Boundary $g=2$ and $g=4$ are checked explicitly; $g=3$ is checked to fail; $g\ge 5$ excluded by strict inequalities.
\end{itemize}

\subsection*{6) FINAL}
\begin{center}
\textbf{PROOF}
\end{center}
