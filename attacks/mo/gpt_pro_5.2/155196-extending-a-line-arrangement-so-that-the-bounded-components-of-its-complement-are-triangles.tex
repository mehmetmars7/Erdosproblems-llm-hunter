\section{(MO 155196) Extending a line arrangement so all bounded faces are triangles}

\subsection{1) FORMAL RESTATEMENT}
Fix conventions:
\begin{itemize}
\item A \emph{line arrangement} is a finite set $\mathcal A$ of distinct affine lines in $\mathbb R^2$.
\item The arrangement induces a planar cell decomposition; the connected components of $\mathbb R^2\setminus\bigcup_{L\in\mathcal A}L$ are called \emph{faces}. Faces are bounded or unbounded in the usual Euclidean sense.
\item A face is a \emph{triangle} if its boundary consists of exactly three line segments (edges) meeting at three vertices.
\end{itemize}

Claim to prove/disprove:
\begin{quote}
For every finite line arrangement $\mathcal A$, there exists a finite set of additional lines $\mathcal B$ such that in the arrangement $\mathcal A\cup\mathcal B$, every bounded face is a triangle.
\end{quote}

Stress points:
\begin{itemize}
\item Adding a line affects \emph{many} faces at once; unlike graph triangulation, we cannot add arbitrary diagonals, only complete lines.
\item Degenerate cases: parallel lines, multiple lines concurrent, etc.
\end{itemize}

\subsection{2) QUICK LITERATURE/CONTEXT CHECK}
The MathOverflow discussion mentions that it is easy to ``quadrilateralate away from infinity'' (make bounded faces quadrilaterals) by adding lines through a carefully chosen point, but triangulating is unclear.

I did not find a posted definitive resolution (no MO answers were visible as of Jan 14, 2026).

\subsection{3) ATTACK PLAN}
\paragraph{Proof track.}
\begin{itemize}
\item Try constructive refinement: reduce all bounded faces to at most quadrilaterals, then systematically split quadrilaterals by adding lines without reintroducing higher polygons.
\item Try invariant-based induction: define a measure of ``non-triangulation'' and show a chosen line reduces it monotonically.
\end{itemize}

\paragraph{Disproof track.}
\begin{itemize}
\item Search for a combinatorial obstruction: a line arrangement whose planar graph cannot be made bounded-triangulated by adding full lines.
\item Attempt to use projective duality and known scarcity of simplicial (all-triangle) arrangements to build an obstruction even when only bounded faces are forced to be triangles.
\end{itemize}

\subsection{4) WORK (partial results)}
\subsubsection*{Lemma 4.1 (Graph-theoretic triangulation vs line constraints)}
Every finite planar graph can be triangulated by adding edges inside faces, but in a line arrangement we can only add edges that arise as segments of \emph{entire new lines}. Therefore standard planar triangulation theorems do not directly apply.

\subsubsection*{A known easy refinement (quadrilateralation idea)}
One can often add lines through a single point to ensure no bounded face has more than four sides (a ``quadrilateralation away from infinity''). However, I did not find a verified argument that this can always be pushed to triangles without creating new non-triangular bounded faces elsewhere.

\subsection{5) VERIFICATION}
The missing part is a monotone refinement algorithm: any naive strategy ``add a diagonal line through two vertices of a non-triangular face'' can create new non-triangular bounded faces elsewhere. I did not find an invariant that guarantees termination in finitely many steps.

\subsection{6) FINAL}
\textbf{UNRESOLVED.}

\paragraph{(i) Strongest proved partial result.}
Only basic observations; no general construction to triangulate bounded faces was proved.

\paragraph{(ii) First crisp gap.}
Either exhibit:
\begin{itemize}
\item a general algorithm that, given $\mathcal A$, outputs finitely many lines $\mathcal B$ making all bounded faces triangles, with a proof of correctness; or
\item a specific line arrangement $\mathcal A$ such that no finite augmentation can make all bounded faces triangles.
\end{itemize}

\paragraph{(iii) Top 3 next moves.}
\begin{enumerate}
\item Formalize and prove the quadrilateralation claim, then analyze whether every quadrilateral can be split without global side effects.
\item Search the arrangement literature for obstructions (e.g.\ oriented matroids, pseudoline arrangements, or dual point sets) that prevent bounded-triangulations.
\item Implement computational searches for small arrangements (up to affine equivalence) to see whether any ``hard'' instances exist.
\end{enumerate}

\paragraph{(iv) Likely shape of a minimal counterexample.}
An arrangement with many parallelism/concurrency constraints whose bounded faces form a configuration forcing any added line to create at least one bounded face with $\ge4$ sides.

