\section{MO 452847 --- Regular icosahedron with a rational point on each face}
\label{sec:mo452847}
\noindent\textbf{MathOverflow link:} \href{https://mathoverflow.net/questions/452847}{https://mathoverflow.net/questions/452847}.

\subsection*{1) FORMAL RESTATEMENT}
\textbf{Literal question.}
Does there exist a regular icosahedron $P\subset\R^3$ such that for every 2-dimensional face $F$ of $P$ there exists a point $x_F\in F\cap\Q^3$?

\medskip
\textbf{Conventions / edge cases.}
\begin{itemize}
  \item ``Face'' means the closed triangular face (including edges).
  \item ``Rational point'' means a point of $\Q^3$.
  \item The icosahedron may be translated, rotated, and scaled arbitrarily.
\end{itemize}

\subsection*{2) QUICK LITERATURE/CONTEXT CHECK}
This is a Diophantine geometry / field-of-definition problem.
A standard coordinate model of the regular icosahedron has vertices in $\Q(\sqrt5)^3$.
I did not verify any definitive MO resolution as of writing.

\subsection*{3) ATTACK PLAN}
\textbf{Existence strategies.}
\begin{itemize}
  \item Try to place the icosahedron so that each face plane is defined over $\Q$ (or at least contains a rational affine line) and then check the triangle region.
  \item Search for a symmetry-respecting placement with many rational constraints.
\end{itemize}
\textbf{Nonexistence strategies.}
\begin{itemize}
  \item Find an invariant obstruction: in any placement, at least one face plane is of the form $u\cdot x=\alpha$ with $u\in\Q^3$ and irrational $\alpha$, hence contains no rational points.
\end{itemize}

\subsection*{4) WORK}
\textbf{A general criterion for rational points on a quadratic-field plane.}
Let $K=\Q(\sqrt5)$. Consider an affine plane in $\R^3$ defined by one equation
\[ n\cdot x=c, \]
with $n\in K^3\setminus\{0\}$ and $c\in K$.
Write $n=u+\sqrt5\,v$ with $u,v\in\Q^3$ and $c=\alpha+\sqrt5\,\beta$ with $\alpha,\beta\in\Q$.
Then a point $x\in\Q^3$ satisfies $n\cdot x=c$ if and only if it satisfies the two rational equations
\[u\cdot x=\alpha,\qquad v\cdot x=\beta.
\]
In particular, the $\Q^3$-points of the plane form either the empty set or an affine $\Q$-subspace.

\medskip
\textbf{A fully explicit obstruction for the \emph{standard} coordinate icosahedron (fixed orientation).}
Let $\varphi=(1+\sqrt5)/2$ and consider the usual vertex set
\[
(0,\pm 1,\pm\varphi),\quad (\pm 1,\pm\varphi,0),\quad (\pm\varphi,0,\pm 1).
\]
One triangular face is spanned by the vertices
\[
P_1=(0,1,\varphi),\quad P_2=(1,\varphi,0),\quad P_3=(\varphi,0,1).
\]
\begin{lemma}
The affine plane through $P_1,P_2,P_3$ has equation
\begin{equation}
\label{eq:faceplane}
 x+y+z=\frac{3+\sqrt5}{2}.
\end{equation}
Consequently, this face contains no rational points in $\Q^3$.
\end{lemma}
\begin{proof}
Compute two edge vectors:
\(P_2-P_1=(1,\varphi-1,-\varphi)\) and
\(P_3-P_1=(\varphi,-1,1-\varphi)\).
A normal vector is their cross product; a direct calculation gives a normal proportional to $(1,1,1)$.
Evaluating $x+y+z$ at $P_1$ gives $0+1+\varphi=(3+\sqrt5)/2$, yielding \eqref{eq:faceplane}.
If $(x,y,z)\in\Q^3$, then $x+y+z\in\Q$, which cannot equal $(3+\sqrt5)/2\notin\Q$.
So the plane --- and hence the triangular face --- contains no rational point.
\end{proof}

\begin{lemma}
\label{lem:no-translation-fixes}
In this fixed orientation, no translation of the icosahedron can make \emph{every} face contain a rational point.
\end{lemma}
\begin{proof}
The opposite face to $\triangle P_1P_2P_3$ is spanned by $-P_1,-P_2,-P_3$, and lies in the plane
\(x+y+z=-(3+\sqrt5)/2\).
Now translate the entire icosahedron by an arbitrary vector $t=(t_1,t_2,t_3)\in\R^3$.
A plane of the form $x+y+z=c$ becomes $x+y+z=c+(t_1+t_2+t_3)$.
Thus the two opposite face planes become
\[
 x+y+z=\frac{3+\sqrt5}{2}+s\quad\text{and}\quad x+y+z=-\frac{3+\sqrt5}{2}+s,
\]
where $s:=t_1+t_2+t_3\in\R$.
If both of these planes were to contain a rational point, then both right-hand sides would have to be rational numbers.
Subtracting, their difference would be $3+\sqrt5$, which is irrational.
Contradiction.
Hence no translation can make \emph{both} opposite faces contain rational points, so certainly not all $20$ faces.
\end{proof}

\subsection*{5) VERIFICATION}
\begin{itemize}
  \item The plane equation \eqref{eq:faceplane} is checked by direct normal computation and evaluation at a vertex.
  \item Lemma~\ref{lem:no-translation-fixes} uses only: opposite faces remain parallel under translation and share the same normal $(1,1,1)$ in this orientation.
\end{itemize}

\subsection*{6) FINAL}
\textbf{UNRESOLVED} for arbitrary rotations/scalings.
\begin{itemize}
  \item (i) Partial result: in the standard $\Q(\sqrt5)$ coordinate model (fixed orientation), no translation can make all faces contain rational points.
  \item (ii) First gap: allow rotations (which change which face normals are rational) and scalings.
  \item (iii) Next moves: (1) classify possible face normals up to rotation and analyze whether one can avoid any rational-normal/irrational-offset obstruction; (2) attempt constructive placements with many $\Q$-defined face planes; (3) reduce the problem to finitely many Diophantine constraints via symmetry.
  \item (iv) A minimal obstruction for full nonexistence would require showing that \emph{every} placement yields at least one rational-normal face plane with irrational offset.
\end{itemize}
