\section*{Problem 269651: Hironaka's positive characteristic resolution manuscript and LUED/GLUED}
\addcontentsline{toc}{section}{Problem 269651: Hironaka's positive characteristic resolution manuscript and LUED/GLUED}

\subsection*{1) FORMAL RESTATEMENT}
The MathOverflow post is not a single mathematical proposition; it asks for assessment of a method (``LUED/GLUED'')
and its possible algorithmic/external applications.

\emph{Minimal corrected, well-posed questions:}
\begin{enumerate}
\item[(Q1)] Does the literature provide a \emph{verified} resolution of singularities in positive characteristic in \emph{all} dimensions?
\item[(Q2)] Is LUED/GLUED currently part of standard constructive/algorithmic resolution frameworks?
\end{enumerate}
These are primarily questions of mathematical \emph{status} and \emph{adoption} rather than theorem proving.

\subsection*{2) QUICK LITERATURE/CONTEXT CHECK}
The MO post points to a Hironaka manuscript claiming a proof in characteristic $p$ and introducing LUED/GLUED.
Surveys emphasize that positive-characteristic resolution is substantially harder; as of recent surveys, the general embedded resolution problem is not presented as settled (with known results in low dimension and open problems in higher dimension).
There are also arXiv preprints claiming full resolution (e.g.\ Yi Hu 2022), but community verification is a separate matter.

\subsection*{3) ATTACK PLAN}
Since (Q1)--(Q2) are not purely formal statements, the ``proof/disproof'' track is interpreted as:
\begin{itemize}
\item Collect authoritative survey statements about the state of the problem (to support ``unresolved'' vs ``resolved'').
\item Check whether LUED/GLUED appears in constructive algorithm literature/implementations.
\end{itemize}
Any conclusion will be evidence-based rather than deductive.

\subsection*{4) WORK (evidence and rigor where possible)}
We record the following verifiable points from accessible sources:
\begin{itemize}
\item Constructive resolution in characteristic $0$ has multiple algorithmic treatments and expositions.
\item Surveys on positive characteristic resolution describe major obstructions and do not present a universally accepted all-dimensional theorem as standard background.
\item There exist arXiv claims of all-characteristic resolution; verifying correctness is nontrivial and not certified by the arXiv entry itself.
\end{itemize}

\subsection*{5) VERIFICATION}
These items do not answer whether LUED/GLUED will be fruitful; that is not a mathematical yes/no statement.
Also, ``verified'' in (Q1) is sociological (peer review/consensus), not something deducible internally.

\subsection*{6) FINAL: \textbf{UNRESOLVED}}
(i) Strongest fully supported partial result: compilation above of what can be stated rigorously from surveys and preprints.\\
(ii) First gap: a definitive, checkable criterion from the literature that certifies LUED/GLUED as yielding a correct, complete, and widely accepted all-dimensional positive-characteristic resolution algorithm.\\
(iii) Next moves:
\begin{itemize}
\item Locate a peer-reviewed publication (or major survey update) confirming/refuting the claimed all-dimensional proofs.
\item Compare LUED/GLUED steps to existing constructive invariants (coefficient ideals, maximal contact, polyhedra) and see if it yields a verifiable termination proof in positive characteristic.
\item Search for actual software implementations or complexity analyses using LUED/GLUED.
\end{itemize}
(iv) Minimal counterexample structure (if LUED/GLUED were claimed to yield full resolution): an explicit class of singularities in characteristic $p$ where the LUED/GLUED procedure fails to decrease a well-defined invariant or does not terminate.

