\section{Problem 4: ``Are there periodicity phenomena in manifold topology with odd period?''}

\subsection*{1) FORMAL RESTATEMENT}

\paragraph{Ambiguity.}
``Periodicity phenomena'' and ``interesting invariants'' are not formalized. There are many inequivalent meanings:
\begin{itemize}[nosep]
\item periodicity of a generalized cohomology theory (existence of an invertible element in some degree);
\item periodicity of surgery $L$-groups / bordism groups (isomorphisms $\Omega_n\cong \Omega_{n+m}$);
\item existence of characteristic numbers or genera defined only in certain congruence classes of $n$.
\end{itemize}

\paragraph{Minimal corrected statement (one rigorous obstruction statement).}
Let $A=\bigoplus_{n\in\Z}A_n$ be a graded-commutative ring (i.e.\ $xy=(-1)^{|x||y|}yx$ for homogeneous $x,y$), and assume $2$ is invertible in $A_0$. Then every unit of $A$ has \emph{even} degree. Consequently, any multiplicative periodicity coming from an invertible element (e.g.\ periodic generalized cohomology away from $2$) must have even period.

\subsection*{2) QUICK LITERATURE/CONTEXT CHECK}
The MathOverflow post lists many even-period phenomena (mod $2,4,8$ and powers of $2$), and asks about odd periods. With web access exhausted, I do not attempt a full literature survey; instead I prove the above general obstruction which explains why odd-period \emph{multiplicative} periodicities are hard to obtain away from $2$.

\subsection*{3) ATTACK PLAN}
\begin{enumerate}[nosep]
\item Prove the graded-commutative ``no odd-degree units when $2$ is invertible'' lemma.
\item Interpret its relevance: many manifold invariants arise from multiplicative genera/cohomology theories with coefficients in characteristic $0$ (or with $2$ inverted), so their periodicities must be even.
\item Note that this does not exclude odd-period behavior in purely $2$-torsion or non-multiplicative settings.
\end{enumerate}

\subsection*{4) WORK}

\paragraph{Lemma 4.4 (Odd-degree units force $2=0$).}
Let $A=\bigoplus_{n\in\Z}A_n$ be a graded-commutative ring. If $u\in A_d$ is a unit of degree $d$ and $d$ is odd, then $2=0$ in $A_0$.

\begin{proof}
Let $u\in A_d$ be homogeneous of odd degree $d$ and assume it is a unit. Let $v\in A_{-d}$ be its (two-sided) inverse, so $uv=vu=1$.

By graded commutativity,
\[
uv = (-1)^{|u||v|}\,vu.
\]
Here $|u||v|=d\cdot(-d)=-d^2$, and since $d$ is odd, $d^2$ is odd, hence $(-1)^{|u||v|}=(-1)^{d^2}=-1$. Therefore
\[
uv = -vu.
\]
But $uv=1$ and $vu=1$, so $1=-1$ in $A_0$, i.e.\ $2=0$ in $A_0$.
\end{proof}

\paragraph{Corollary 4.5 (No odd-degree units when $2$ is invertible).}
If $2$ is invertible in $A_0$, then every unit of $A$ has even degree.

\begin{proof}
If $u$ were a unit of odd degree, Lemma 4.4 would imply $2=0$, contradicting invertibility of $2$.
\end{proof}

\paragraph{Interpretation for ``periodicity''.}
A periodic generalized cohomology theory $E$ typically has graded-commutative coefficient ring $E_\ast$. If $2$ is inverted (or if $E_\ast$ is a $\Q$-algebra), then any periodicity generator must be in even degree by Corollary 4.5. Thus any \emph{multiplicative} periodicity in such settings has even period. This gives a structural reason odd periods are not expected in many familiar manifold-topology invariants (which often factor through such theories).

\subsection*{5) VERIFICATION}
\begin{itemize}[nosep]
\item The sign computation uses only parity: if $d$ is odd, then $d^2$ is odd, so $(-1)^{d^2}=-1$.
\item The conclusion is sharp: over $A_0$ of characteristic $2$, odd-degree units are \emph{not} excluded by this argument.
\end{itemize}

\subsection*{6) FINAL}
\textbf{UNRESOLVED}

\paragraph{(i) Strongest partial result.}
A rigorous obstruction: odd-period \emph{multiplicative} periodicity (coming from an invertible graded element) cannot occur after inverting $2$.

\paragraph{(ii) First gap.}
Translate the broad MO question into a specific formal notion of ``interesting odd-period manifold invariant'', and then either construct one or prove it cannot exist under that notion.

\paragraph{(iii) Top 3 next moves.}
\begin{enumerate}[nosep]
\item Focus on $2$-primary and torsion phenomena where Corollary 4.5 does not apply, and search for odd-degree periodicity operators there.
\item Look for invariants defined only in dimensions divisible by an odd integer (e.g.\ $3$) that do \emph{not} come from multiplicative genera (possibly unstable or non-ring-valued).
\item Examine periodicity in families of manifolds under product with a fixed odd-dimensional manifold and test whether cobordism/surgery groups exhibit such periodicity.
\end{enumerate}

\paragraph{(iv) Minimal counterexample shape.}
A positive example would likely live in a $2$-torsion context (characteristic $2$ coefficients) or in a non-multiplicative invariant not expressible via a graded-commutative coefficient ring with $2$ inverted.

