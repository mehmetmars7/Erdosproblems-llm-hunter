\section{Problem 4: Theta divisor of a number field and arithmetic reconstruction}

\subsection*{Problem source}
MathOverflow question \#47920: ``What does the theta divisor of a number field know about its arithmetic?'' (M.~Kapranov).

\subsection*{1) FORMAL RESTATEMENT}

\paragraph{Literal statement.}
The post is an open-ended question: given a number field $F$, what arithmetic invariants can be recovered from the Arakelov--Picard group $\mathrm{Pic}^{(d)}(F)$ together with a function $h^0$ (an Arakelov-theoretic theta function)?

This is not a single proposition with a truth value.

\paragraph{Ambiguity.}
``Recover'' could mean:
\begin{itemize}[leftmargin=2em]
\item determine invariants up to isomorphism of number fields;
\item determine isomorphism type of the unit group $O_F^\times$;
\item reconstruct the entire field $F$.
\end{itemize}
The post cites a remark that it ``should be possible'' to reconstruct arithmetic from $(\mathrm{Pic}^{(d)}(F),h^0)$, but does not state a theorem.

\paragraph{Minimal corrected, fully provable statement.}
A basic, completely rigorous statement one \emph{can} prove from the definition of Arakelov class groups is:

\begin{quote}
\textbf{Claim.} From the topological group $\mathrm{Pic}^{(0)}(F)$ (equivalently any torsor $\mathrm{Pic}^{(d)}(F)$) one can recover:
\begin{enumerate}[label=(\roman*),leftmargin=2em]
\item the class number $h_F = |\mathrm{Cl}(F)|$, and
\item the Dirichlet unit rank $r = r_1+r_2-1$.
\end{enumerate}
\end{quote}

This does \emph{not} reconstruct the full unit group including regulator, nor the field; it is a partial, fully proved extraction.

\subsection*{2) QUICK LITERATURE/CONTEXT CHECK}

The relevant reference is van der Geer--Schoof, ``Effectivity of Arakelov divisors and the theta divisor of a number field'' (arXiv).\footnote{\url{https://arxiv.org/abs/math/0001011}}  
The MO post quotes their remark about possible reconstruction. I will use only standard Arakelov class-group definitions and Dirichlet's unit theorem.

\subsection*{3) ATTACK PLAN}

\begin{enumerate}[leftmargin=2em]
\item Write down the definition of the Arakelov divisor class group $\mathrm{Pic}(F)$ and its degree-$0$ part.
\item Identify a natural continuous surjection $\mathrm{Pic}^{(0)}(F)\twoheadrightarrow \mathrm{Cl}(F)$ with kernel a real torus of dimension $r_1+r_2-1$ (coming from archimedean data modulo the logarithmic embedding of units).
\item Conclude that the component group recovers $\mathrm{Cl}(F)$ and the identity component dimension recovers the unit rank.
\end{enumerate}

\subsection*{4) WORK}

\subsubsection*{Definitions}

Let $F$ be a number field with signature $(r_1,r_2)$, so $[F:\Q]=r_1+2r_2$.

Let $I_F$ be the group of fractional ideals of $O_F$.

Let $F_\infty := \prod_{v\mid\infty} F_v \cong \R^{r_1}\times \C^{r_2}$.

An \emph{Arakelov divisor} may be represented as a pair
\[
D=(\mathfrak{a}, x)
\]
where $\mathfrak{a}\in I_F$ and $x=(x_v)_{v\mid\infty}\in \R^{r_1+r_2}$ (real coordinates at archimedean places). The Arakelov divisor group is
\[
\mathrm{Div}(F):= I_F \times \R^{r_1+r_2}
\]
with componentwise addition (ideals multiply, real vectors add).

Define the \emph{degree} map $\deg:\mathrm{Div}(F)\to\R$ by
\[
\deg(\mathfrak{a},x):= -\log N(\mathfrak{a}) + \sum_{v\mid\infty} n_v\, x_v,
\]
where $N(\mathfrak{a})$ is the absolute norm and $n_v=1$ for real places and $n_v=2$ for complex places. (This weighting matches the product formula.)

For $\alpha\in F^\times$, define the \emph{principal Arakelov divisor}
\[
\mathrm{div}(\alpha):=(\alpha O_F,\; (-\log|\alpha|_v)_{v\mid\infty}).
\]
The product formula implies $\deg(\mathrm{div}(\alpha))=0$.

Let $\mathrm{Prin}(F)\subset \mathrm{Div}(F)$ be the subgroup of principal divisors. Define
\[
\mathrm{Pic}(F):=\mathrm{Div}(F)/\mathrm{Prin}(F).
\]
Let $\mathrm{Pic}^{(0)}(F)$ be the degree-$0$ part, i.e.\ the image of $\ker(\deg)$ in $\mathrm{Pic}(F)$.

\subsubsection*{Exact sequence and consequences}

\begin{lemma}[Forgetting archimedean data gives ideal class]
There is a well-defined surjective homomorphism
\[
\pi:\mathrm{Pic}^{(0)}(F)\twoheadrightarrow \mathrm{Cl}(F)
\]
to the ideal class group, given on representatives by $\pi([(\mathfrak{a},x)])=[\mathfrak{a}]\in \mathrm{Cl}(F)$.
\end{lemma}
\begin{proof}
We must check:
\begin{enumerate}[leftmargin=2em]
\item \emph{Well-defined on divisor classes:} If $(\mathfrak{a},x)$ and $(\mathfrak{a}',x')$ differ by a principal divisor, then $(\mathfrak{a}',x')=(\mathfrak{a},x)+\mathrm{div}(\alpha)$ for some $\alpha\in F^\times$. On ideal components this means $\mathfrak{a}'=\mathfrak{a}\cdot \alpha O_F$, so $[\mathfrak{a}']=[\mathfrak{a}]$ in $\mathrm{Cl}(F)$. Thus $\pi$ is well-defined.
\item \emph{Homomorphism:} Clear from multiplication of ideals.
\item \emph{Surjectivity:} Given any ideal class $[\mathfrak{a}]\in\mathrm{Cl}(F)$, choose a representative ideal $\mathfrak{a}$. Choose $x\in\R^{r_1+r_2}$ so that $\deg(\mathfrak{a},x)=0$; for example, fix any $v_0\mid\infty$ and set $x_{v_0}=\frac{\log N(\mathfrak{a})}{n_{v_0}}$ and all other $x_v=0$. Then $(\mathfrak{a},x)$ represents an element of $\mathrm{Pic}^{(0)}(F)$ mapping to $[\mathfrak{a}]$.
\end{enumerate}
\end{proof}

\begin{lemma}[Kernel is a real torus of dimension $r_1+r_2-1$]
The kernel $\ker(\pi)$ is (canonically) isomorphic, as a topological group, to a real torus $(S^1)^{r_1+r_2-1}$.
\end{lemma}
\begin{proof}
An element of $\ker(\pi)$ is represented by a degree-$0$ divisor $(\mathfrak{a},x)$ whose ideal class is trivial, so $\mathfrak{a}=\alpha O_F$ for some $\alpha\in F^\times$. Then
\[
(\mathfrak{a},x) - \mathrm{div}(\alpha) = (O_F,\; x + (\log|\alpha|_v)_{v\mid\infty}),
\]
so every class in $\ker(\pi)$ has a representative of the form $(O_F,y)$ with $\deg(O_F,y)=\sum n_v y_v=0$, i.e.\ $y$ lies in the hyperplane
\[
H:=\left\{y\in\R^{r_1+r_2}:\sum_{v\mid\infty} n_v y_v=0\right\}\cong \R^{r_1+r_2-1}.
\]

Two such representatives $(O_F,y)$ and $(O_F,y')$ define the same class in $\mathrm{Pic}(F)$ iff their difference is principal:
\[
(O_F,y)-(O_F,y')=\mathrm{div}(u)
\]
for some unit $u\in O_F^\times$ (the ideal part must be trivial, hence $\alpha O_F=O_F$). This means
\[
y-y' = (-\log|u|_v)_{v\mid\infty}.
\]
Let $\Lambda\subset H$ be the image of the logarithmic embedding of units:
\[
\Lambda := \{(-\log|u|_v)_{v\mid\infty} : u\in O_F^\times\}\subset H.
\]
By Dirichlet's unit theorem, $\Lambda$ is a full-rank lattice in $H$ of rank $r_1+r_2-1$.

Therefore
\[
\ker(\pi)\ \cong\ H/\Lambda,
\]
which is a real torus of dimension $\dim H=r_1+r_2-1$ and hence is (non-canonically) isomorphic to $(S^1)^{r_1+r_2-1}$ as a topological group.
\end{proof}

\begin{corollary}[Recovering class number and unit rank]
From the topological group $\mathrm{Pic}^{(0)}(F)$ one can recover:
\begin{enumerate}[label=(\roman*),leftmargin=2em]
\item the class number $h_F$ as the number of connected components of $\mathrm{Pic}^{(0)}(F)$, and
\item the unit rank $r_1+r_2-1$ as the (real) dimension of the identity component.
\end{enumerate}
\end{corollary}
\begin{proof}
The exact sequence
\[
0\to \ker(\pi)\to \mathrm{Pic}^{(0)}(F)\xrightarrow{\pi} \mathrm{Cl}(F)\to 0
\]
shows $\mathrm{Pic}^{(0)}(F)$ is a disjoint union of $|\mathrm{Cl}(F)|$ cosets of the connected subgroup $\ker(\pi)\cong (S^1)^{r_1+r_2-1}$. Thus the number of connected components equals $|\mathrm{Cl}(F)|=h_F$, and the identity component has dimension $r_1+r_2-1$.
\end{proof}

\subsubsection*{What about recovering units? (where this stalls)}

As \emph{abstract topological groups}, all real tori of a fixed dimension $r$ are isomorphic to $(S^1)^r$. Thus $\mathrm{Pic}^{(0)}(F)$ alone cannot recover the \emph{regulator} (the covolume of the unit lattice $\Lambda$), nor the embedding of $O_F^\times$ into $\R^{r_1+r_2-1}$. The additional analytic data $h^0$ is designed to encode theta series of ideal lattices and plausibly contains metric information, but I do not have a gap-free proof that $(\mathrm{Pic}^{(d)}(F),h^0)$ determines $O_F^\times$ or $F$.

\subsection*{5) VERIFICATION}

\paragraph{Checks.}
The proofs use only standard definitions and Dirichlet's unit theorem. The only subtle point is well-definedness of $\pi$, which was checked directly.

\subsection*{6) FINAL}

\paragraph{FINAL LABEL: \textbf{UNRESOLVED}.}

\paragraph{Strongest fully proved partial result.}
From $\mathrm{Pic}^{(0)}(F)$ one can recover the class number $h_F$ and the unit rank $r_1+r_2-1$.

\paragraph{Exact first gap.}
Show that adding the theta function $h^0$ determines finer invariants such as the regulator, the unit lattice up to isometry, or even the isomorphism class of $F$.

\paragraph{Top 3 next moves.}
\begin{enumerate}[leftmargin=2em]
\item Express $h^0$ as a theta function of a lattice and attempt to recover the lattice (or its covolume) from the full theta function family; compare with known non-uniqueness phenomena for lattices sharing theta series.
\item Determine whether $(\mathrm{Pic}^{(d)}(F),h^0)$ determines the Dedekind zeta function $\zeta_F(s)$; if yes, compare with arithmetically equivalent but non-isomorphic fields as a possible obstruction to full reconstruction.
\item Work out explicitly the case of quadratic fields, where $\mathrm{Pic}^{(0)}(F)$ is a circle (rank $1$ units) or finite (imaginary quadratic), and see what $h^0$ distinguishes beyond class number.
\end{enumerate}


