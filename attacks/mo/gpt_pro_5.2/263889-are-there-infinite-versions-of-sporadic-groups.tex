\section{Problem 263889: ``infinite versions'' of sporadic groups}

\subsection*{1) FORMAL RESTATEMENT}
\paragraph{Literal issue.}
This is not a precise mathematical statement. ``Infinite version'' and ``defined in terms of fields/graphs'' are not formalized, and different reasonable formalizations lead to different answers.

\paragraph{Ambiguity pinpointed.}
The phrase ``an infinite version of a sporadic group'' could mean any of:
\begin{itemize}
\item (i) an infinite \emph{simple} group constructed by an analogous method to the sporadic finite one;
\item (ii) a functor $k\mapsto S(k)$ on fields/structures such that $S(\FF_q)$ is the sporadic group and $S(k)$ exists (possibly infinite) for other $k$;
\item (iii) an ``ambient'' infinite object (VOA, geometry, $2$-groupoid, etc.) whose automorphism group at a special point is sporadic.
\end{itemize}
These are genuinely different.

\paragraph{Minimal corrected proposition (one plausible formalization).}
A concrete yes/no version of (ii) is:
\begin{quote}
\textbf{(F$_{263889}$)} There exist a sporadic finite simple group $S$ and a construction which assigns to each field $k$ of characteristic $p$ a group $S(k)$, functorially in $k$, such that $S(\FF_q)\cong S$ for some finite $\FF_q$ and such that $S(k)$ is defined for infinite fields $k$ and yields an infinite group for some such $k$.
\end{quote}
This is only one of many possible formalizations, but it captures the ``$S(k)$ built from $L(k)$'' idea in the prompt.

\subsection*{2) QUICK LITERATURE/CONTEXT CHECK (browsing YES)}
The MO thread \href{https://mathoverflow.net/questions/263889/are-there-infinite-versions-of-sporadic-groups}{263889} has no posted answers as of January 2026.
Thus, no consensus/standard construction was presented there.
Without a precise definition, no definitive proof/disproof is possible.

\subsection*{3) ATTACK PLAN}
\paragraph{Proof-track.}
To prove (F$_{263889}$), one would need to exhibit a specific functorial construction (e.g.\ via geometries, VOAs, or generalized cohomology theories) and verify it gives the desired specialization at $\FF_q$.

\paragraph{Disproof-track.}
To disprove (F$_{263889}$) one would need a theorem stating that no such functorial extension exists for \emph{any} sporadic group, which seems far beyond current general theory and depends heavily on the exact notion of ``construction'' allowed.

\subsection*{4) WORK}
Given the ambiguity and lack of a settled theorem in the attached material, we cannot produce a gap-free proof or a counterexample in any strong formal sense.
We record one fully rigorous observation about why the literal question cannot be answered without extra structure:

\begin{proposition}
Without a formal definition of ``infinite version'' or a specified class of constructions, the question admits mutually incompatible answers by choosing different interpretations.
\end{proposition}
\begin{proof}
Under interpretation (iii), one can plausibly regard certain sporadic groups as automorphism groups of rich infinite objects (e.g.\ vertex operator algebras), so ``infinite versions'' exist in the sense of an infinite ambient structure.
Under interpretation (ii), the requirement of functorial dependence on an arbitrary field $k$ is extremely restrictive and may fail for many constructions that rely on special finite combinatorics (e.g.\ Steiner systems with fixed parameters).
Since (iii) can be ``yes'' and (ii) can be ``no'' simultaneously, the original question (without fixing meaning) cannot have a single mathematical truth value.
\end{proof}

\subsection*{5) VERIFICATION}
The above proposition is purely logical: it checks that the prompt is under-specified and therefore cannot be settled as a single theorem/disproof.

\subsection*{6) FINAL}
\textbf{UNRESOLVED}

\paragraph{(i) Strongest fully proved partial result obtained.}
We proved that the question is not well-posed without defining ``infinite version'' (Proposition 4.1).

\paragraph{(ii) First gap.}
A precise definition of the target notion (functoriality, simplicity, axiomatized geometries, etc.) is missing; without it, no proof/disproof is meaningful.

\paragraph{(iii) Top 3 next moves.}
\begin{enumerate}[label=(\alphic*)]
\item Choose one formalization (e.g.\ functorial $S(k)$ for fields $k$) and restrict the class of allowed constructions.
\item Test Mathieu groups via Steiner-system axioms: determine whether a natural first-order axiom system has infinite models and analyze their automorphism groups.
\item For the Monster, explore whether the VOA/``moonshine'' structures provide a canonical extension to other base fields or higher-categorical automorphism objects.
\end{enumerate}

\paragraph{(iv) Likely shape of a minimal counterexample.}
A sporadic group whose defining combinatorial geometry has a categorical rigidity theorem implying no infinite model with analogous axioms exists (if one adopts the ``Steiner system'' interpretation), or conversely a construction (e.g.\ in VOAs) yielding an infinite family with the sporadic as a special finite member.
