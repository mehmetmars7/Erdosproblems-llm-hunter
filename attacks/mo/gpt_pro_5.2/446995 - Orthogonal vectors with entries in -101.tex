\section{MO 446995 --- Orthogonal vectors with entries in $\{-1,0,1\}$}
\label{sec:mo446995}
\noindent\textbf{MathOverflow link:} \href{https://mathoverflow.net/questions/446995}{https://mathoverflow.net/questions/446995}.

\subsection*{1) FORMAL RESTATEMENT}
Fix $n\in\N$. Let $\mathbf{1}\in\R^n$ be the all-ones vector.

\medskip
\textbf{Claim to be proved/disproved.}
\begin{quote}
If there exist vectors $v_1,\dots,v_{n-1}\in\{-1,0,1\}^n$ such that
\[\{\mathbf{1},v_1,\dots,v_{n-1}\}\subset\R^n\]
consists of $n$ nonzero pairwise orthogonal vectors (standard dot product), then $n\in\{1,2\}$ or $4\mid n$.
\end{quote}

\medskip
\textbf{Conventions / edge cases.}
\begin{itemize}
  \item Orthogonal means $x\cdot y=\sum_{i=1}^n x_i y_i$.
  \item $n$ pairwise orthogonal nonzero vectors in $\R^n$ are linearly independent.
  \item For $n=1$ the condition is vacuous; for $n=2$ it holds with $v_1=(1,-1)$.
\end{itemize}

\subsection*{2) QUICK LITERATURE/CONTEXT CHECK}
This is a ``Hadamard-type'' necessary condition question, but with zeros allowed.
For $\pm1$-matrices (Hadamard matrices) it is classical that $n\in\{1,2\}$ or $4\mid n$.
The question asks whether allowing zeros permits new dimensions.

\subsection*{3) ATTACK PLAN}
\textbf{Proof strategies.}
\begin{itemize}
  \item Gram/determinant arguments: form the integer matrix $A$ whose columns are $\mathbf{1},v_1,\dots,v_{n-1}$.
  \item Modular restrictions from $A^\top A=\mathrm{diag}(d_1,\dots,d_n)$.
  \item Parity / 2-adic constraints.
\end{itemize}
\textbf{Disproof strategies.}
\begin{itemize}
  \item Construct an explicit example for $n\equiv 2\pmod 4$.
  \item Exhaustive/computer search for smallest counterexample $n=6,10,14,\dots$.
\end{itemize}

\subsection*{4) WORK}
Let $A\in\Z^{n\times n}$ be the matrix whose columns are
\(c_1:=\mathbf{1},c_2:=v_1,\dots,c_n:=v_{n-1}\).
Orthogonality means
\begin{equation}
\label{eq:ATA-diag}
A^\top A=\mathrm{diag}(d_1,\dots,d_n),\qquad d_j:=\|c_j\|^2\in\Z_{>0}.
\end{equation}
In particular $d_1=\|\mathbf{1}\|^2=n$.
Also, for $j\ge 2$ we have $\mathbf{1}\cdot c_j=0$, so each $c_j$ has equally many $+1$ and $-1$ entries; hence the number of nonzero entries in $c_j$ is even and therefore
\begin{equation}
\label{eq:dj-even}
 d_j \text{ is even for every } j\ge 2.
\end{equation}

\medskip
\textbf{Lemma 1: odd primes are impossible (complete).}
\begin{lemma}
If $n=p$ is an odd prime, then no such orthogonal set exists.
\end{lemma}
\begin{proof}
Assume such vectors exist.
Define the rational matrix
\(B:=A\,\mathrm{diag}(d_1^{-1},\dots,d_n^{-1})\,A^\top\in M_n(\Q)\).
Using \eqref{eq:ATA-diag},
\[
B=A(A^\top A)^{-1}A^\top=I_n.
\]
Taking the $(i,i)$ entry gives, for each row $i$,
\begin{equation}
\label{eq:rowdiag}
1=\sum_{j=1}^n \frac{A_{ij}^2}{d_j}.
\end{equation}
Since the first column is all ones, $A_{i1}=1$ and the $j=1$ term in \eqref{eq:rowdiag} equals $1/p$.
All other terms are either $0$ or $1/d_j$ with $d_j$ even.
Thus
\begin{equation}
\label{eq:denom-oddprime}
\frac{p-1}{p}=1-\frac{1}{p}=\sum_{j=2}^n \frac{A_{ij}^2}{d_j}.
\end{equation}
For $j\ge 2$, by \eqref{eq:dj-even} we have $0<d_j\le p$ and $d_j$ even, hence $p\nmid d_j$.
Therefore the right-hand side of \eqref{eq:denom-oddprime} is a rational number whose reduced denominator is not divisible by $p$ (it divides the least common multiple of the $d_j$ that appear).
But the reduced denominator of $(p-1)/p$ is exactly $p$, contradiction.
\end{proof}

\medskip
\textbf{Lemma 2: twice an odd prime is also impossible (complete).}
\begin{lemma}
\label{lem:twice-odd-prime}
If $n=2p$ with $p$ an odd prime, then no such orthogonal set exists.
\end{lemma}
\begin{proof}
Assume such vectors exist in dimension $n=2p$.
Retain the notation \eqref{eq:ATA-diag}.
For each $j$, the quantity $d_j$ is the number of nonzero entries in column $c_j$.
Let
\[
S:=\{j\in\{1,\dots,n\}: d_j=n\}
\]
be the set of \emph{full-support} columns (no zeros). For $j\in S$, the column $c_j$ lies in $\{\pm 1\}^n$.
We always have $1\in S$.

Fix a row index $i$ and apply \eqref{eq:rowdiag}.
Every full-support column contributes exactly $1/n$ to the sum (since $A_{ij}^2=1$ and $d_j=n$ for all $i$), hence
\begin{equation}
\label{eq:split-S}
1=\sum_{j\in S}\frac{1}{n}+\sum_{\substack{j\notin S\\ A_{ij}\neq 0}}\frac{1}{d_j}
=\frac{|S|}{n}+\sum_{\substack{j\notin S\\ A_{ij}\neq 0}}\frac{1}{d_j}.
\end{equation}
Rearranging gives
\begin{equation}
\label{eq:split-S-rearr}
\sum_{\substack{j\notin S\\ A_{ij}\neq 0}}\frac{1}{d_j}=\frac{n-|S|}{n}=\frac{2p-|S|}{2p}.
\end{equation}
Now, if $j\notin S$ then $d_j<n=2p$ and by \eqref{eq:dj-even} the integer $d_j$ is even. Hence $d_j\in\{2,4,\dots,2p-2\}$, and in particular
\begin{equation}
\label{eq:p-not-divide-dj}
 p\nmid d_j \quad \text{for all } j\notin S.
\end{equation}
Therefore the reduced denominator of the left-hand side of \eqref{eq:split-S-rearr} is not divisible by $p$.
Consequently, the reduced denominator of $(2p-|S|)/(2p)$ is also not divisible by $p$, which forces
\begin{equation}
\label{eq:p-divides-2p-minus-S}
 p\mid (2p-|S|).
\end{equation}
Since $2p-|S|\equiv -|S|\pmod p$, this means $p\mid |S|$.
Because $1\le |S|\le 2p$ and $p$ is prime, we get
\begin{equation}
\label{eq:S-size}
|S|\in\{p,2p\}.
\end{equation}

If $|S|=2p$, then \emph{all} columns are full-support, so $A$ is a $\{\pm 1\}$-matrix with mutually orthogonal columns.
In particular, it contains $\mathbf{1}$ and at least two further $\pm1$ columns orthogonal to $\mathbf{1}$.
Lemma~\ref{lem:threepm1} below would then force $4\mid n$, contradicting $n=2p\equiv 2\pmod 4$.
So $|S|\neq 2p$.
Thus $|S|=p$, and since $p\ge 3$ there exist \emph{at least two} distinct indices $j,k\in S\setminus\{1\}$.
Then $c_1=\mathbf{1},c_j,c_k\in\{\pm 1\}^n$ are three mutually orthogonal $\pm 1$ vectors.
Applying Lemma~\ref{lem:threepm1} again yields $4\mid n$, contradiction.
Hence no such configuration exists.
\end{proof}

\medskip
\begin{lemma}
\label{lem:threepm1}
Let $n\ge 1$ and suppose there exist three vectors $u,v,w\in\{\pm 1\}^n$ such that $u\cdot v=u\cdot w=v\cdot w=0$.
Then $4\mid n$.
\end{lemma}
\begin{proof}
Multiply coordinates so that $u=\mathbf{1}$ (replace $v_i$ by $u_i v_i$ and $w_i$ by $u_i w_i$; this preserves $\{\pm1\}$-valuedness and dot products).
Thus we have $v,w\in\{\pm1\}^n$ with
\(\mathbf{1}\cdot v=\mathbf{1}\cdot w=v\cdot w=0\).
Let
\(A=\#\{i: v_i=1,w_i=1\}\),
\(B=\#\{i: v_i=1,w_i=-1\}\),
\(C=\#\{i: v_i=-1,w_i=1\}\),
\(D=\#\{i: v_i=-1,w_i=-1\}\).
Then $n=A+B+C+D$.
The conditions become:
\begin{align*}
\mathbf{1}\cdot v=0 &\iff (A+B)-(C+D)=0 \iff A+B=C+D,\\
\mathbf{1}\cdot w=0 &\iff (A+C)-(B+D)=0 \iff A+C=B+D,\\
v\cdot w=0 &\iff (A+D)-(B+C)=0 \iff A+D=B+C.
\end{align*}
From $A+B=C+D$ and $A+C=B+D$ we get $B=C$.
Then $A+D=B+C$ gives $A+D=2B$.
But $A+B=C+D$ with $C=B$ gives $A+B=B+D$, hence $A=D$.
Thus $A=B=C=D$, so $n=4A$ is divisible by $4$.
\end{proof}

\subsection*{5) VERIFICATION}
\begin{itemize}
  \item The odd-prime lemma is purely rational-denominator bookkeeping and is gap-free.
  \item In Lemma~\ref{lem:twice-odd-prime}, the only number-theory input is: a prime $p$ cannot appear in the denominator of a rational sum unless it appears in some summand denominator.
  \item Lemma~\ref{lem:threepm1} is an elementary counting argument.
\end{itemize}

\subsection*{6) FINAL}
\textbf{UNRESOLVED} in full generality.
\begin{itemize}
  \item (i) Strongest proved results here: impossibility for $n$ an odd prime, and impossibility for $n=2p$ with $p$ an odd prime.
  \item (ii) First gap: exclude all $n\equiv 2\pmod 4$ (or construct one).
  \item (iii) Next moves: (1) extend the denominator-splitting argument to other $n$ with large prime factors; (2) develop a 2-adic obstruction controlling the multiset of norms $d_j$; (3) attempt computer search for the smallest $n\equiv 2\pmod 4$ not covered by Lemma~\ref{lem:twice-odd-prime}.
  \item (iv) A minimal counterexample (if it exists) would likely have $n\equiv 2\pmod 4$ with many columns of relatively small even norms, arranged so that primes dividing $n$ already appear among those norms.
\end{itemize}

% -----------------------------------------------------------------------------
