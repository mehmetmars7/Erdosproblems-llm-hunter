\section{Problem 298336: Homology of $\mathrm{PGL}_2(F)$}

\subsection*{1) FORMAL RESTATEMENT}

Let $F$ be a field and let $G=\mathrm{PGL}_2(F)$ be the projective linear group.
Let $T\simeq F^\times$ be the image in $G$ of the diagonal matrices $\mathrm{diag}(a,1)$.
Equivalently define a homomorphism
\[
\delta:F^\times \longrightarrow \mathrm{PGL}_2(F),\qquad
a\longmapsto [\mathrm{diag}(a,1)].
\]
The question asks whether the induced map on rational group homology
\[
\delta_\ast:H_4(F^\times;\Q)\longrightarrow H_4(\mathrm{PGL}_2(F);\Q)
\]
is surjective.

\paragraph{Identifications/definitions.}
Since $F^\times$ is abelian, there is a canonical identification
\[
H_4(F^\times;\Q)\;\cong\;\bigwedge\nolimits^4(F^\times_\Q),
\qquad F^\times_\Q:=F^\times\otimes_\Z\Q.
\]
We use standard (discrete) group homology with coefficients in $\Q$.

\paragraph{Edge cases / stress points.}
\begin{itemize}[leftmargin=2.2em]
\item If $F$ is finite then $\mathrm{PGL}_2(F)$ is finite, so $H_4(\mathrm{PGL}_2(F);\Q)=0$ and surjectivity holds trivially.
\item If $F^\times$ has small $\Q$-rank, then $H_4(F^\times;\Q)=0$, so surjectivity would force $H_4(\mathrm{PGL}_2(F);\Q)=0$.
\item The map depends on the embedding of a split torus; one can also ask what happens for other maximal tori (non-split in general).
\end{itemize}

\subsection*{2) QUICK LITERATURE/CONTEXT CHECK}

The attachment mentions Suslin's conjecture and an exact sequence in Dupont's book. I did not attempt to reconstruct all those links here; instead I focus on what can be proved directly and unconditionally from general homological algebra.

\subsection*{3) ATTACK PLAN}

\begin{itemize}[leftmargin=2.2em]
\item \textbf{Proof track:} Use Hochschild--Serre spectral sequences for the extension
$1\to \mathrm{PSL}_2(F)\to \mathrm{PGL}_2(F)\to F^\times/(F^\times)^2\to 1$
to reduce $H_4(\mathrm{PGL}_2(F);\Q)$ to coinvariants of $H_4(\mathrm{PSL}_2(F);\Q)$.
Then try to identify generators coming from $F^\times$.
\item \textbf{Disproof track:} Seek a field $F$ for which $H_4(\mathrm{PGL}_2(F);\Q)$ is known to be nonzero
but $H_4(F^\times;\Q)$ is ``too small'' to surject, e.g.\ if the map factors through something of low rank.
\end{itemize}

\subsection*{4) WORK}

\subsubsection*{Phase 1 --- Tiny cases}

\begin{lemma}\label{lem:finite-group-homology}
If $H$ is a finite group then $H_n(H;\Q)=0$ for all $n>0$.
\end{lemma}

\begin{proof}
Consider the standard bar resolution $B_\bullet\to \Z$ of $\Z$ as a $\Z[H]$-module and tensor with $\Q$.
Equivalently, compute $H_\bullet(H;\Q)$ as the homology of the chain complex $C_n=\Q[H^n]$ with the usual differential.

Define a $\Q$-linear map $s_n:C_n\to C_{n+1}$ by
\[
s_n([h_1,\dots,h_n])=\frac{1}{|H|}\sum_{g\in H}[g,h_1,\dots,h_n].
\]
A direct check of the bar differential shows $\partial s_n + s_{n-1}\partial=\mathrm{id}_{C_n}$ for $n\ge 1$.
(The only point is that averaging over $g\in H$ is legitimate because $|H|$ is invertible in $\Q$.)
Hence the complex is contractible in degrees $>0$, so $H_n(H;\Q)=0$ for $n>0$.
\end{proof}

\begin{corollary}
If $F$ is finite then $H_4(\mathrm{PGL}_2(F);\Q)=0$, hence $\delta_\ast$ is (vacuously) surjective.
\end{corollary}

\subsubsection*{Phase 2 --- A structural reduction via Hochschild--Serre}

\begin{lemma}\label{lem:torsionQ}
If $A$ is a torsion group, then $H_n(A;\Q)=0$ for all $n>0$.
\end{lemma}

\begin{proof}
Every torsion group is a directed union (direct limit) of its finite subgroups.
Group homology with coefficients in a field commutes with directed unions because it is computed from the bar complex, which is functorial and built from finite products of group elements.
Concretely, write $A=\varinjlim A_i$ with $A_i$ finite; then the bar chain complex $C_\bullet(A;\Q)$ is the direct limit of $C_\bullet(A_i;\Q)$, and homology commutes with direct limits of chain complexes of $\Q$-vector spaces.
Thus $H_n(A;\Q)=\varinjlim H_n(A_i;\Q)=0$ for $n>0$ by Lemma~\ref{lem:finite-group-homology}.
\end{proof}

\begin{lemma}\label{lem:PGL-PSL-coinv}
Let $F$ be a field and set $A:=F^\times/(F^\times)^2$.
There is a short exact sequence of groups
\[
1\longrightarrow \mathrm{PSL}_2(F)\longrightarrow \mathrm{PGL}_2(F)\longrightarrow A\longrightarrow 1.
\]
For every $n\ge 0$ there is a natural isomorphism
\[
H_n(\mathrm{PGL}_2(F);\Q)\;\cong\;H_n(\mathrm{PSL}_2(F);\Q)_{A},
\]
the coinvariants for the conjugation action of $A$ on $\mathrm{PSL}_2(F)$.
\end{lemma}

\begin{proof}
The extension is standard: $\mathrm{PGL}_2(F)=\mathrm{GL}_2(F)/F^\times$ and $\mathrm{PSL}_2(F)=\mathrm{SL}_2(F)/\{\pm I\}$, so the quotient is $F^\times/(F^\times)^2$.

Apply the Hochschild--Serre spectral sequence in homology for this extension:
\[
E^2_{p,q}=H_p\bigl(A; H_q(\mathrm{PSL}_2(F);\Q)\bigr)\;\Longrightarrow\; H_{p+q}(\mathrm{PGL}_2(F);\Q).
\]
Now $A$ is a torsion group of exponent $2$ (every class has square $1$), hence $H_p(A;M)=0$ for all $p>0$ and all $\Q[A]$-modules $M$ by Lemma~\ref{lem:torsionQ} applied to $A$ and coefficients in the $\Q$-vector space $M$.
Therefore the spectral sequence collapses at $E^2$ and yields
\[
H_n(\mathrm{PGL}_2(F);\Q)\cong E^\infty_{0,n}=E^2_{0,n}=H_0\bigl(A;H_n(\mathrm{PSL}_2(F);\Q)\bigr),
\]
which is by definition the coinvariants $H_n(\mathrm{PSL}_2(F);\Q)_A$.
\end{proof}

\paragraph{What remains.}
Lemma~\ref{lem:PGL-PSL-coinv} reduces the surjectivity question to understanding:
\begin{itemize}[leftmargin=2.2em]
\item the size/structure of $H_4(\mathrm{PSL}_2(F);\Q)$ as an $A$-module, and
\item the image of $\delta_\ast$ inside the coinvariants $H_4(\mathrm{PSL}_2(F);\Q)_A$.
\end{itemize}
I do not know an unconditional computation of $H_4(\mathrm{PSL}_2(F);\Q)$ for general fields $F$ that would let me conclude either surjectivity or a counterexample in full generality.

\subsection*{5) VERIFICATION}

\begin{itemize}[leftmargin=2.2em]
\item Lemma~\ref{lem:finite-group-homology} is standard and the contracting homotopy is explicit.
\item Lemma~\ref{lem:torsionQ} uses only the fact that torsion groups are directed unions of finite subgroups and that homology commutes with direct limits over $\Q$.
\item In Lemma~\ref{lem:PGL-PSL-coinv}, the key point is that $A$ is torsion (exponent $2$), hence $H_p(A;-)=0$ for $p>0$ over $\Q$.
\end{itemize}

\subsection*{6) FINAL}

\begin{center}
\textbf{UNRESOLVED}
\end{center}

\noindent\textbf{(i) Strongest proved partial result obtained:}
The canonical quotient $A=F^\times/(F^\times)^2$ is $2$-torsion, and
\[
H_n(\mathrm{PGL}_2(F);\Q)\cong H_n(\mathrm{PSL}_2(F);\Q)_A
\quad\text{for all }n\ge 0
\]
(Lemma~\ref{lem:PGL-PSL-coinv}). In particular, for finite fields $F$ the target $H_4(\mathrm{PGL}_2(F);\Q)$ vanishes.

\medskip
\noindent\textbf{(ii) First gap:}
Compute or estimate $H_4(\mathrm{PSL}_2(F);\Q)$ (as an $A$-module) for general fields $F$, or compute the image of $\delta_\ast$ in the coinvariants.

\medskip
\noindent\textbf{(iii) Top 3 next moves:}
\begin{enumerate}[label=(\arabic*),leftmargin=2.2em]
\item Identify known computations/relationships of $H_4(\mathrm{PSL}_2(F);\Q)$ with $K$-theory or Bloch groups and translate the map $\delta_\ast$ into that language.
\item Use an action of $\mathrm{PGL}_2(F)$ on a suitable complex (e.g.\ configurations on $\mathbb{P}^1(F)$) to build a spectral sequence whose $E^1$-page involves homology of stabilizers (tori and Borel subgroups), then track the $H_4$ generators.
\item Search for special fields $F$ (e.g.\ local fields, $\mathbb{R}$, algebraically closed fields viewed discretely) where $H_4(\mathrm{PSL}_2(F);\Q)$ is known or approachable.
\end{enumerate}

\medskip
\noindent\textbf{(iv) What a minimal counterexample would likely look like:}
A field $F$ where $H_4(\mathrm{PGL}_2(F);\Q)\neq 0$ but the image of $H_4(F^\times;\Q)\cong \bigwedge^4(F^\times_\Q)$ is constrained
(e.g.\ if it factors through a smaller quotient of $F^\times_\Q$), so that the induced map cannot surject onto the coinvariant group.
