\section{MO 76443: Enriched categories, ideals/submodules, and algebraic geometry}

\subsection*{1) FORMAL RESTATEMENT}

\paragraph{Literal vs corrected statement.}
Again, the prompt is primarily a \emph{question}. The minimal corrected statement that can be proved is an explicit categorical construction realizing the operations $I\cdot J$ and $[I,J]$ and the enrichment of $U(M)$.

\begin{definition}[The data]\label{def:IA-UM}
Let $A$ be a commutative ring and $M$ an $A$-module.
\begin{itemize}[leftmargin=2em]
\item $I(A)$ is the poset (thin category) of ideals of $A$ ordered by inclusion.
\item $U(M)$ is the poset (thin category) of $A$-submodules of $M$ ordered by inclusion.
\end{itemize}
Define:
\[
I\cdot J:=\text{the ideal generated by products }ij\ (i\in I,\ j\in J),
\]
and
\[
[I,J] := (J:I):=\{a\in A\mid aI\subseteq J\}.
\]
For submodules $U,V\subseteq M$, define
\[
[U,V]:=\{a\in A\mid aU\subseteq V\}.
\]
\end{definition}

\paragraph{Corrected statement to prove.}
\begin{theorem}[Categorical construction and enrichment]\label{thm:quantale-enrichment}
Let $A$ be a commutative ring and $M$ an $A$-module.
\begin{enumerate}[leftmargin=2em]
\item The poset $I(A)$ is a symmetric monoidal closed poset with monoidal product $I\cdot J$ and internal hom $[I,J]=(J:I)$.
\item $U(M)$ is naturally a category enriched over the monoidal poset $I(A)$ with hom-object $[U,V]=\{a\in A\mid aU\subseteq V\}$ and composition induced by ideal multiplication.
\item This structure arises functorially from the symmetric monoidal closed category $(A\text{-}\mathrm{Mod},\otimes_A,\underline{\mathrm{Hom}}_A)$ by taking subobject posets of the unit $A$ and of the object $M$.
\end{enumerate}
\end{theorem}

\subsection*{2) QUICK LITERATURE/CONTEXT CHECK (web available)}

This is standard “\emph{quantale} / \emph{residuated lattice}” structure: ideal lattices form a commutative unital quantale with multiplication given by ideal product and residual given by ideal quotient, and submodule lattices are modules over that quantale.
In enriched-category terms, this is a Lawvere-style enrichment over a monoidal poset.

\subsection*{3) ATTACK PLAN}

\begin{itemize}[leftmargin=2em]
\item Prove the closed monoidal poset property by verifying the adjunction:
\[
I\cdot J\subseteq K\quad\Longleftrightarrow\quad I\subseteq [J,K].
\]
\item Identify $I\cdot J$ as an image of a tensor product map $I\otimes_A J\to A$ induced by multiplication.
\item For enrichment, verify identity and composition axioms in the monoidal poset $I(A)$:
\[
A\subseteq [U,U],\qquad [V,W]\cdot[U,V]\subseteq [U,W].
\]
\item Explain functoriality via “subobjects of the unit” in the monoidal category of $A$-modules.
\end{itemize}

\subsection*{4) WORK (complete proofs)}

\subsubsection*{Step 1: $I(A)$ is symmetric monoidal closed}

\begin{lemma}[Ideal product as image of a tensor map]\label{lem:product-as-image}
Let $I,J\subseteq A$ be ideals. Consider the $A$-bilinear map
\[
\mu_{I,J}\colon I\otimes_A J \longrightarrow A,\qquad i\otimes j\longmapsto ij,
\]
induced by multiplication in $A$. Then
\[
\mathrm{im}(\mu_{I,J}) = I\cdot J.
\]
\end{lemma}

\begin{proof}
For any $i\in I$ and $j\in J$, $\mu_{I,J}(i\otimes j)=ij\in A$. Since $\mu_{I,J}$ is $A$-linear, its image is an $A$-submodule of $A$, hence an ideal.
Moreover, $\mathrm{im}(\mu_{I,J})$ contains every product $ij$ with $i\in I$, $j\in J$, hence it contains the ideal generated by such products, i.e.\ $I\cdot J$.
Conversely, any element of $\mathrm{im}(\mu_{I,J})$ is a finite sum of elements of the form $\mu_{I,J}(i\otimes j)=ij$, hence lies in $I\cdot J$ by definition. Thus $\mathrm{im}(\mu_{I,J})\subseteq I\cdot J$.
\end{proof}

\begin{lemma}[Ideal quotient is an ideal]\label{lem:quotient-is-ideal}
For ideals $I,J\subseteq A$, the set
\[
[I,J]=(J:I)=\{a\in A\mid aI\subseteq J\}
\]
is an ideal of $A$.
\end{lemma}

\begin{proof}
Let $a,a'\in (J:I)$ and $r\in A$. For any $x\in I$ we have $ax\in J$ and $a'x\in J$, hence $(a+a')x=ax+a'x\in J$ since $J$ is an ideal. Thus $a+a'\in (J:I)$.
Also $(ra)x=r(ax)\in J$ because $ax\in J$ and $J$ is an ideal. Hence $ra\in (J:I)$.
So $(J:I)$ is an ideal.
\end{proof}

\begin{lemma}[Closedness adjunction]\label{lem:adjunction}
For ideals $I,J,K\subseteq A$,
\[
I\cdot J\subseteq K \quad\Longleftrightarrow\quad I\subseteq [J,K].
\]
\end{lemma}

\begin{proof}
($\Rightarrow$) Assume $I\cdot J\subseteq K$ and let $a\in I$. For any $b\in J$, we have $ab\in I\cdot J\subseteq K$. Hence $aJ\subseteq K$, so $a\in [J,K]$. Thus $I\subseteq [J,K]$.

($\Leftarrow$) Assume $I\subseteq [J,K]$. Then for every $a\in I$ and $b\in J$ we have $ab\in K$. Finite sums of such products also lie in $K$, hence the ideal generated by them, namely $I\cdot J$, is contained in $K$.
\end{proof}

\begin{proposition}[$I(A)$ is symmetric monoidal closed]\label{prop:IA-closed}
The poset $I(A)$, with tensor product $I\otimes J:=I\cdot J$ and internal hom $[I,J]=(J:I)$, is a symmetric monoidal closed poset with unit object $A$.
\end{proposition}

\begin{proof}
Associativity and commutativity of ideal product are standard ring-theoretic facts and can be checked directly from generators:
\[
(I\cdot J)\cdot K = I\cdot (J\cdot K),\qquad I\cdot J=J\cdot I,\qquad A\cdot I=I.
\]
Closedness is exactly Lemma \ref{lem:adjunction}. Since $I(A)$ is a poset category, the adjunction is the closed structure.
\end{proof}

\subsubsection*{Step 2: $U(M)$ is enriched over $I(A)$}

\begin{lemma}[Hom-ideal for submodules]\label{lem:submodule-hom-ideal}
For submodules $U,V\subseteq M$, the set
\[
[U,V]:=\{a\in A\mid aU\subseteq V\}
\]
is an ideal of $A$.
\end{lemma}

\begin{proof}
The proof is identical to Lemma \ref{lem:quotient-is-ideal}: closure under addition and multiplication by $A$ follow from $V$ being a submodule.
\end{proof}

\begin{lemma}[Identity and composition inequalities]\label{lem:enriched-axioms}
For all submodules $U,V,W\subseteq M$,
\[
A\subseteq [U,U],\qquad [V,W]\cdot[U,V]\subseteq [U,W].
\]
\end{lemma}

\begin{proof}
\emph{Identity:} For any $a\in A$ and $u\in U$, we have $au\in U$ because $U$ is a submodule. Hence $aU\subseteq U$ so $a\in[U,U]$, i.e.\ $A\subseteq[U,U]$.

\emph{Composition:} Let $b\in [V,W]$ and $a\in[U,V]$. For any $u\in U$, we have $au\in V$ and then $b(au)\in W$. Hence $(ba)U\subseteq W$, so $ba\in[U,W]$.
Therefore every product $ba$ with $b\in[V,W]$ and $a\in[U,V]$ lies in $[U,W]$, and hence the ideal generated by such products, namely $[V,W]\cdot[U,V]$, is contained in $[U,W]$.
\end{proof}

\begin{proof}[Proof of Theorem \ref{thm:quantale-enrichment}]
(1) is Proposition \ref{prop:IA-closed}.

(2) Define the enriched hom-object from $U$ to $V$ to be the ideal $[U,V]$ of Lemma \ref{lem:submodule-hom-ideal}.
The enriched identity morphism $I(A)\ni A\to [U,U]$ is exactly the inequality $A\subseteq[U,U]$.
Enriched composition is a morphism in $I(A)$:
\[
[ V,W]\cdot [U,V] \longrightarrow [U,W],
\]
which in a poset-enrichment is exactly the inequality $[V,W]\cdot[U,V]\subseteq[U,W]$ established in Lemma \ref{lem:enriched-axioms}.
Associativity and unitality are automatic because we are enriching over a strict monoidal poset and composition is defined via the monoidal product inequality.

(3) In the symmetric monoidal closed category $A\text{-}\mathrm{Mod}$, ideals are precisely the subobjects of the unit object $A$, i.e.\ $I(A)\cong \mathrm{Sub}_{A\text{-}\mathrm{Mod}}(A)$.
Lemma \ref{lem:product-as-image} identifies multiplication $I\cdot J$ with the image of the composite
\[
I\otimes_A J \xrightarrow{i\otimes j} A\otimes_A A \xrightarrow{\mu} A.
\]
The residual $[I,J]$ is the right adjoint to multiplication in the subobject poset, which is exactly Lemma \ref{lem:adjunction}.
Similarly, $U(M)=\mathrm{Sub}_{A\text{-}\mathrm{Mod}}(M)$ and $[U,V]$ is the largest ideal $I$ with $I\cdot U\subseteq V$, i.e.\ the residual for the $I(A)$-action on subobjects of $M$.
\end{proof}

\subsubsection*{Tiny-case reality checks}
Let $A=\mathbb{Z}$.
\begin{itemize}[leftmargin=2em]
\item $I=(2)$ and $J=(3)$ satisfy $I\cdot J=(6)$.
\item $[(2),(6)]=\{a\in\mathbb{Z}:2a\in (6)\}=(3)$.
\end{itemize}
Let $M=\mathbb{Z}$ and $U=(2)\subseteq M$, $V=(6)\subseteq M$ as submodules. Then
\[
[U,V]=\{a\in\mathbb{Z}\mid a(2)\subseteq (6)\}=(3),
\]
agreeing with the ideal quotient.

\subsection*{5) VERIFICATION}

\begin{itemize}[leftmargin=2em]
\item All nontrivial steps reduced to explicit set inclusions.
\item Closedness is precisely the “residuation” equivalence in Lemma \ref{lem:adjunction}.
\item Enrichment over a poset only requires the identity and composition inequalities, both checked elementwise.
\end{itemize}

\subsection*{6) FINAL}

\textbf{PROOF.}

\noindent\textbf{Clean theorem statement.}
Theorem \ref{thm:quantale-enrichment} holds: $I(A)$ is a symmetric monoidal closed poset under ideal multiplication and ideal quotient, and $U(M)$ is canonically enriched over $I(A)$ with hom-ideals $[U,V]=\{a\mid aU\subseteq V\}$, arising functorially from the monoidal closed category $A\text{-}\mathrm{Mod}$.

