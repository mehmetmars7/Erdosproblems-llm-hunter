\section{MO \#316985: Existence of orthogonal basis of symmetric $n\times n$ matrices where each matrix is unitary}
\MO{316985}{Existence of orthogonal basis of symmetric $n\times n$ matrices where each matrix is unitary}

\subsection*{1) FORMAL RESTATEMENT}
Fix $n\ge 1$. Let $S_n(\mathbb C)=\{A\in M_n(\mathbb C):A^T=A\}$ with Hermitian inner product
\[
\langle A,B\rangle:=\operatorname{Tr}(AB^*)=\sum_{i,j}A_{ij}\overline{B_{ij}}.
\]
An \emph{orthogonal basis of symmetric unitaries} means a set $\mathcal B\subset S_n(\mathbb C)$ such that:
\begin{enumerate}[leftmargin=*]
\item $|\mathcal B|=\dim_\mathbb C S_n(\mathbb C)=\frac{n(n+1)}2$;
\item each $U\in\mathcal B$ is \emph{unitary} ($UU^*=I$) and \emph{symmetric} ($U^T=U$);
\item $\langle U,V\rangle=0$ for all distinct $U,V\in\mathcal B$.
\end{enumerate}
Question: does such a $\mathcal B$ exist for every $n$?

\subsection*{2) QUICK LITERATURE/CONTEXT CHECK}
The attached MO post gives an explicit construction for even $n$, and explicit examples for $n\le 11$ found computationally; it reports the odd case as open.

\subsection*{3) ATTACK PLAN}
\begin{itemize}[leftmargin=*]
\item \textbf{Proof track:} attempt a uniform construction for all $n$.
\item \textbf{Disproof track:} try to derive an obstruction for odd $n$ (e.g. combinatorial support/phase constraints, character sums, or parity arguments).
\end{itemize}
I pursue a full proof for all \emph{even} $n$ (complete), and then isolate the odd-$n$ gap.

\subsection*{4) WORK}
\paragraph{Theorem (complete solution for even $n$).}
If $n$ is even, then there exists an orthogonal basis of $S_n(\mathbb C)$ consisting of symmetric unitary matrices.

\emph{Proof.}
Let $n$ be even and set $m=n/2$. Let $\omega=e^{2\pi i/m}$.

\textbf{Step 1: diagonal part.}
For each $t\in\{0,1,\dots,n-1\}$ define the diagonal matrix
\[
D_t:=\operatorname{diag}(\zeta^{t\cdot 0},\zeta^{t\cdot 1},\dots,\zeta^{t\cdot (n-1)})\in S_n(\mathbb C),\qquad \zeta=e^{2\pi i/n}.
\]
Each $D_t$ is unitary and symmetric. Moreover, for $t\ne t'$,
\[
\langle D_t,D_{t'}\rangle=\sum_{j=0}^{n-1}\zeta^{(t-t')j}=0,
\]
while $\langle D_t,D_t\rangle=n$. Thus $\{D_t\}_{t=0}^{n-1}$ is an orthogonal set of $n$ symmetric unitaries.

\textbf{Step 2: off-diagonal part via a $1$-factorization.}
A $1$-factorization of the complete graph $K_n$ is a partition of its edge set into $n-1$ perfect matchings $F_1,\dots,F_{n-1}$, each containing $m$ disjoint edges and covering all vertices.
Fix such a factorization. For a fixed $k\in\{1,\dots,n-1\}$, write
\[
F_k=\{\{a_{k,r},b_{k,r}\}: r=0,1,\dots,m-1\}
\]
with all $a_{k,r},b_{k,r}\in\{1,\dots,n\}$ distinct and each vertex occurring in exactly one pair.

For each $\ell\in\{0,1,\dots,m-1\}$ define $U_{k,\ell}\in M_n(\mathbb C)$ by
\[
(U_{k,\ell})_{a_{k,r},b_{k,r}}=(U_{k,\ell})_{b_{k,r},a_{k,r}}:=\omega^{\ell r}
\quad\text{for }r=0,\dots,m-1,
\]
and all other entries $0$.
Then $U_{k,\ell}^T=U_{k,\ell}$.
Because each row has exactly one nonzero entry of modulus $1$, $U_{k,\ell}$ is a monomial unitary matrix, hence unitary.

\textbf{Orthogonality within one matching.}
If $\ell\ne \ell'$, then
\[
\langle U_{k,\ell},U_{k,\ell'}\rangle
=\sum_{r=0}^{m-1} \Big( (U_{k,\ell})_{a_{k,r},b_{k,r}}\overline{(U_{k,\ell'})_{a_{k,r},b_{k,r}}}
+(U_{k,\ell})_{b_{k,r},a_{k,r}}\overline{(U_{k,\ell'})_{b_{k,r},a_{k,r}}}\Big)
=2\sum_{r=0}^{m-1}\omega^{(\ell-\ell')r}=0.
\]
Also $\langle U_{k,\ell},U_{k,\ell}\rangle=2m=n$.

\textbf{Orthogonality between different matchings.}
If $k\ne k'$, then $F_k$ and $F_{k'}$ are disjoint edge sets, so the supports of $U_{k,\ell}$ and $U_{k',\ell'}$ are disjoint. Hence $\langle U_{k,\ell},U_{k',\ell'}\rangle=0$.

\textbf{Orthogonality with the diagonal part.}
Every $D_t$ is supported on the diagonal, while every $U_{k,\ell}$ has zero diagonal; thus $\langle D_t,U_{k,\ell}\rangle=0$.

\textbf{Count.}
We constructed $n$ diagonal matrices and $(n-1)\cdot m=(n-1)(n/2)=n(n-1)/2$ off-diagonal ones, totaling $n(n+1)/2=\dim S_n(\mathbb C)$.
They are pairwise orthogonal and nonzero, hence linearly independent, hence form an orthogonal basis.
\qed

\paragraph{Odd $n$ status.}
For odd $n$, the above decomposition fails because $K_n$ has no perfect matching, and empirical examples exist for small odd $n$ but no general construction is provided.

\subsection*{5) VERIFICATION}
All steps are explicit and checkable. The only nontrivial input is existence of a $1$-factorization of $K_n$ for even $n$, which is classical (round-robin tournament construction). Orthogonality computations reduce to disjoint supports and a finite geometric sum identity.

\subsection*{6) FINAL}
\textbf{UNRESOLVED.}
\begin{itemize}[leftmargin=*]
\item (i) Fully proved partial result: existence for all even $n$ (the theorem above).
\item (ii) First gap: a uniform construction (or obstruction) for odd $n$.
\item (iii) Next moves: (1) search for an odd-$n$ construction mixing diagonal/off-diagonal supports (even-$n$ separates them); (2) formulate as a weighted complex Hadamard / design problem; (3) attempt to prove nonexistence for some odd $n$ via invariant/trace constraints.
\item (iv) Likely minimal counterexample: the smallest odd $n$ where no such basis exists (numerics suggest none up to $n=11$ in the MO post).
\end{itemize}

