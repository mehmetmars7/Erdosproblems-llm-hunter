\section{(MO 122221) Discrete monoid $M$ injecting into its group completion $G$ with $BM\not\simeq BG$}

\subsection{1) FORMAL RESTATEMENT}
Fix conventions:
\begin{itemize}
\item A \emph{discrete monoid} means a (set-theoretic) monoid $(M,\cdot,1)$.
\item The \emph{group completion} $G=G(M)$ is the universal group equipped with a monoid homomorphism $\eta\colon M\to G$ such that any monoid homomorphism $M\to H$ into a group $H$ factors uniquely through $\eta$.
\item $BM$ denotes the classifying space (geometric realization of the nerve) of the one-object category with endomorphism monoid $M$; $BG$ is the usual classifying space of the group $G$.
\end{itemize}

Question interpreted as an existence/negation dichotomy:
\begin{itemize}
\item \textbf{Existence version}: $\exists$ a monoid $M$ such that the canonical map $\eta\colon M\to G(M)$ is injective and the induced map $B\eta\colon BM\to BG(M)$ is \emph{not} a homotopy equivalence.
\item \textbf{Universal version}: $\forall$ monoids $M$, if $\eta$ is injective then $B\eta$ is a homotopy equivalence.
\end{itemize}

Stress points:
\begin{itemize}
\item Injectivity $\eta$ forces cancellation (since $M$ embeds into a group).
\item The failure $BM\not\simeq BG$ is equivalent to $BM$ not being an Eilenberg--MacLane space $K(G,1)$ (since $BG$ is aspherical).
\end{itemize}

\subsection{2) QUICK LITERATURE/CONTEXT CHECK}
From the MathOverflow discussion: one standard sufficient condition is an \emph{Ore condition} / calculus of fractions, and there is a homological criterion due to Fiedorowicz: $BM\to BG$ is a homotopy equivalence if and only if certain homology groups $H_n(M,\mathbb ZG)$ vanish for all $n\ge 1$.

I did not find (as of January 14, 2026) a posted MathOverflow answer giving either a counterexample or a proof of the universal version; the page shows comments but no accepted resolution.

\subsection{3) ATTACK PLAN}
\paragraph{Proof track (try to show universal version).}
\begin{itemize}
\item Show injectivity $\Rightarrow$ cancellation. Try to upgrade cancellation to something like Ore, or otherwise show the homological groups $H_n(M,\mathbb ZG)$ vanish.
\item Use Quillen's Theorem~A / homotopy localization descriptions of $BM$ to identify $BM\simeq BG$ under suitable fraction conditions.
\end{itemize}

\paragraph{Disproof track (try to build counterexample).}
\begin{itemize}
\item Find a cancellative monoid embedded in a group (hence injective completion map) with nontrivial higher homotopy/homology of $BM$.
\item Candidate sources: finitely presented one-relator monoids, monoids with finite complete rewriting systems but nontrivial Squier complexes, or positive monoids of groups where an asphericity conjecture is known to fail.
\end{itemize}

Best path available here: I can prove several \emph{sufficient conditions} for $BM\simeq BG$ and explain the homological obstruction, but I did not reach either a complete proof of the universal version or an explicit counterexample.

\subsection{4) WORK (partial results)}

\subsubsection*{Lemma 4.1 (Injective completion map $\Rightarrow$ cancellation)}
If $\eta\colon M\to G(M)$ is injective, then $M$ is both left- and right-cancellative: $ab=ac\Rightarrow b=c$ and $ba=ca\Rightarrow b=c$.

\begin{proof}
Assume $ab=ac$ in $M$. Apply the monoid homomorphism $\eta$ to get $\eta(a)\eta(b)=\eta(a)\eta(c)$ in the group $G(M)$. Cancel $\eta(a)$ on the left to obtain $\eta(b)=\eta(c)$. Injectivity of $\eta$ gives $b=c$. The right-cancellation argument is identical. \end{proof}

\subsubsection*{Lemma 4.2 (Ore condition $\Rightarrow BM\simeq BG$)}
Assume $M$ is cancellative and satisfies a two-sided Ore condition (for all $a,b\in M$ there exist $x,y\in M$ with $ax=by$ and similarly $xa'=yb'$ on the left). Then the canonical map $BM\to BG(M)$ is a homotopy equivalence.

\begin{proof}[Outline (standard categorical input)]
Under the (two-sided) Ore condition, $M$ admits a calculus of (right and left) fractions and embeds into its group of fractions $G(M)$; moreover the category obtained by inverting all morphisms in the one-object category $M$ is equivalent to the groupoid with one object and automorphism group $G(M)$. One proves that the localization functor $M\to G(M)$ is \emph{homotopy cofinal}: for each object in the localized groupoid, the corresponding comma category is contractible. By Quillen's Theorem~A, the induced map on classifying spaces is a homotopy equivalence.
\end{proof}

\subsubsection*{The homological obstruction (Fiedorowicz criterion)}
There is a criterion (due to Fiedorowicz) expressing the failure of $BM\to BG$ to be a homotopy equivalence in terms of homology groups $H_n(M,\mathbb ZG)$; in particular, vanishing for all $n\ge 1$ is equivalent to $BM\simeq BG$.

\subsection{5) VERIFICATION}
\begin{itemize}
\item Lemma~4.1 is immediate and uses only group cancellation.
\item Lemma~4.2 depends on the precise formulation of the Ore/calc-of-fractions hypotheses and on Quillen Theorem~A. I have not reproduced the full categorical proof here; the statement is standard in localization theory of categories/monoids.
\item The key missing piece for a full resolution is either:
  \begin{itemize}
  \item a theorem ``injective $\Rightarrow$ (homological vanishing)'' (which seems unlikely in full generality), or
  \item an explicit cancellative monoid with injective group completion map but with $H_n(M,\mathbb ZG)\neq 0$ for some $n\ge 1$.
  \end{itemize}
\end{itemize}

\subsection{6) FINAL}
\textbf{UNRESOLVED.}

\paragraph{(i) Strongest proved partial result.}
Injectivity implies cancellation (Lemma~4.1). If, in addition, $M$ satisfies a two-sided Ore condition / calculus of fractions, then $BM\simeq BG$ (Lemma~4.2).

\paragraph{(ii) First crisp gap.}
Produce either (a) a proof that cancellation (or injectivity) forces the Fiedorowicz homological groups $H_n(M,\mathbb ZG)$ to vanish for $n\ge 1$, or (b) a cancellative monoid embedding in a group with $H_n(M,\mathbb ZG)\neq 0$ for some $n\ge 1$.

\paragraph{(iii) Top 3 next moves.}
\begin{enumerate}
\item Search the semigroup literature for explicit cancellative monoids of fractions-type failures with computable homology (e.g.\ using Squier complexes).
\item Test concrete candidates: positive monoids of groups with known non-asphericity of associated monoid complexes.
\item Use the Fiedorowicz criterion to compute $H_n(M,\mathbb ZG)$ for a finitely presented cancellative monoid via a projective resolution of $\mathbb Z$ as a $\mathbb ZM$-module.
\end{enumerate}

\paragraph{(iv) Likely shape of a minimal counterexample.}
A finitely presented cancellative monoid embedded in a group, but failing an Ore condition, so that a localization map is not homotopy cofinal and produces nontrivial higher homology of $BM$.

