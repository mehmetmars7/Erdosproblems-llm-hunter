\section{Problem 105512: Eigenvalues of $U\bar U$ vs.\ orthogonal ensembles}

\subsection*{1) FORMAL RESTATEMENT}
\paragraph{Literal statement.}
Interpreting the attachment as a proposition:

\medskip
\noindent\textbf{Claim (interpreting the attachment):}
Fix $N\ge 1$. Let $U$ be Haar-distributed on $U(N)$.
Let $V=U\overline{U}\in U(N)$ and let $p_{U\overline U}$ be the induced law of the multiset of eigenvalues of $V$.
Let $O$ be Haar-distributed on the component $\{Q\in O(N+1):\det(Q)=-1\}$, and remove one eigenvalue $-1$ from the spectrum of $O$.
Let $p_O$ be the induced law of the remaining $N$ eigenvalues. 
Then
\[
p_{U\overline U} = p_O.
\]

\paragraph{Edge-case conventions.}
For Haar-a.e.\ $O$ in the det$=-1$ component, the eigenvalue $-1$ has multiplicity $1$, so removing ``one $-1$'' is unambiguous almost surely.

\subsection*{2) QUICK LITERATURE/CONTEXT CHECK (web available)}
The attachment cites arXiv:1206.6687 (Appendix B.3) and Girko (1985) as sources for the identity, but asks for a conceptual proof.
I did not locate, during the allotted browsing, a short conceptual proof beyond those computations.

\subsection*{3) ATTACK PLAN}
\begin{itemize}
\item \textbf{Proof track ideas:}
  \begin{enumerate}[label=(\alph*)]
  \item Use Weyl integration on $U(N)$ to compute the pushforward measure under $U\mapsto U\bar U$ and match it to the Weyl density for $O(N+1)$ (det$=-1$ component).
  \item Prove equality of all moments $\mathbb{E}[p(\text{eigenvalues})]$ for symmetric polynomials $p$ (character expansion).
  \end{enumerate}
\item \textbf{Disproof track:} Search for a mismatch in small $N$ via exact computation/simulation.
\end{itemize}

\subsection*{4) WORK (partial results + computation)}
\begin{lemma}[Basic properties of $U\overline U$]
If $U\in U(N)$ then $V=U\overline U$ is unitary and $\det(V)=1$.
\end{lemma}
\begin{proof}
Unitarity: $V^\ast = (U\overline U)^\ast = (\overline U)^\ast U^\ast = U^T U^\ast$ and
\[
VV^\ast = U\overline U\,U^T U^\ast = U(\overline U\,U^T)U^\ast.
\]
But $\overline U\,U^T = \overline{UU^\ast} = \overline{I}=I$ (since $U^\ast=\overline U^T$), hence $VV^\ast = UIU^\ast = I$.
Similarly $V^\ast V=I$. So $V$ is unitary.

Determinant: $\det(V)=\det(U)\det(\overline U)=\det(U)\overline{\det(U)} = |\det(U)|^2 = 1$.
\end{proof}

\paragraph{Tiny case $N=1$.}
A Haar unitary $U\in U(1)$ is $e^{i\theta}$, so $U\overline U=1$ deterministically.
A Haar orthogonal matrix in $O(2)$ with determinant $-1$ has eigenvalues $\{-1,1\}$, so after removing $-1$ the remaining eigenvalue is $1$ deterministically.
Thus the claim holds for $N=1$.

\paragraph{Monte Carlo check for $N=2$ (evidence only).}
For $N=2$, the attachment states that if $U\overline U$ has eigenvalues $e^{\pm i\theta}$ then $\theta$ has density $(1+\cos\theta)/\pi$ on $[0,\pi]$, implying $\mathbb{E}[\cos\theta]=1/2$.
A simulation with 5000 samples gives
\[
\mathbb{E}[\cos\theta]\approx 0.5081\quad\text{for }U\overline U,
\qquad
\mathbb{E}[\cos\theta]\approx 0.4987\quad\text{for }O(3)\text{ det}=-1\text{ with }-1\text{ removed},
\]
consistent with $1/2$.

\subsection*{5) VERIFICATION}
The lemma and $N=1$ case are proved.
The main claim for general $N$ remains unproved here; the simulation is not a proof.

\subsection*{6) FINAL}
\textbf{UNRESOLVED.}

\medskip
\noindent\textbf{Fail-safe details.}
\begin{enumerate}[label=(\roman*)]
\item \textbf{Strongest fully proved partial result:} The lemma above and the exact verification for $N=1$.
\item \textbf{First gap:} A derivation of the full joint eigenvalue law of $U\overline U$ that yields $p_{U\overline U}=p_O$.
\item \textbf{Top 3 next moves:} (1) Weyl-integration + Jacobian pushforward; (2) moment/character identities; (3) symmetric-space/root-system identification.
\item \textbf{Likely structure of a minimal counterexample:} A mismatch in low-order eigenvalue moments for some small $N$.
\end{enumerate}


