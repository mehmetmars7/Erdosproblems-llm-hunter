\section{MO 385603: ``Spectral sequences as deformation theory''}
\label{sec:mo385603}
\noindent\textbf{MathOverflow link:} \url{https://mathoverflow.net/questions/385603/spectral-sequences-as-deformation-theory} (accessed 2026-01-16).

\subsection*{1) FORMAL RESTATEMENT}
\textbf{Literal issue.} The MO post is an invitation for references and viewpoints, not a proposition.

\medskip
\noindent\textbf{Minimal corrected statement (precise mathematical core).}
One precise claim appearing in the post is that ``filtered objects'' can be encoded as modules over a polynomial ring $k[t]$ (or its spectral analogue $S[t]$) with $t$ acting as the filtration shift.

\medskip
\noindent\textbf{Theorem (Rees/graded-module encoding).}
Let $k$ be a commutative ring.
Let $\mathrm{Seq}(k\text{-}\mathrm{Mod})$ denote the category of sequences of $k$-modules
\[
F_0 \xrightarrow{\sigma_0} F_1 \xrightarrow{\sigma_1} F_2 \xrightarrow{} \cdots
\]
(i.e. functors $\NN\to k$-Mod).
Let $k[t]$ be the graded ring with $\deg t=1$.
Let $\mathrm{GrMod}_{k[t]}$ denote the category of nonnegatively graded $k[t]$-modules $M=\bigoplus_{n\ge 0} M_n$.
Then there is an equivalence of categories
\[
\mathrm{GrMod}_{k[t]} \simeq \mathrm{Seq}(k\text{-}\mathrm{Mod})
\]
under which the action of $t$ corresponds to the structure maps $\sigma_n: F_n\to F_{n+1}$.

\subsection*{2) QUICK LITERATURE/CONTEXT CHECK}
The MO post mentions the Rees algebra viewpoint and the identification of graded $k[t]$-modules with $\Gm$-equivariant quasi-coherent sheaves on $\Aone$.
This is standard algebraic geometry: $\mathrm{QCoh}(\Aone)=k[t]$-Mod, and $\Gm$-equivariant structures correspond to gradings.

\subsection*{3) ATTACK PLAN}
Prove the equivalence directly by giving quasi-inverse functors:
\begin{enumerate}[leftmargin=2em]
\item From a graded $k[t]$-module $M$ produce a sequence $F_n=M_n$ with maps $\sigma_n$ given by multiplication by $t$.
\item From a sequence $(F_n,\sigma_n)$ build a graded $k[t]$-module $M=\bigoplus_n F_n$ with $t$ acting via the $\sigma_n$.
\item Check these constructions are functorial and mutually inverse.
\end{enumerate}

\subsection*{4) WORK}
\subsubsection*{Construction of functors}
Define a functor
\[
\Phi: \mathrm{GrMod}_{k[t]} \to \mathrm{Seq}(k\text{-}\mathrm{Mod})
\]
by sending a graded module $M=\bigoplus_{n\ge 0} M_n$ to the sequence with $F_n:=M_n$ and $\sigma_n:F_n\to F_{n+1}$ given by multiplication by $t$:
\[
\sigma_n(m)=t\cdot m \in M_{n+1}.
\]
On morphisms: a graded $k[t]$-linear map $f:M\to M'$ restricts to degreewise maps $f_n:M_n\to M'_n$ commuting with $t$, hence gives a morphism of sequences.

Conversely, define a functor
\[
\Psi: \mathrm{Seq}(k\text{-}\mathrm{Mod}) \to \mathrm{GrMod}_{k[t]}
\]
by sending $(F_n,\sigma_n)$ to the graded $k$-module $M:=\bigoplus_{n\ge 0} F_n$ with grading $M_n=F_n$, and define the $k[t]$-module structure by specifying the action of $t$ on homogeneous elements:
\[
 t\cdot x := \sigma_n(x) \in F_{n+1}=M_{n+1}\quad\text{for }x\in M_n.
\]
Extend to all of $k[t]$ by $k$-linearity and the rule $t^r\cdot x := \sigma_{n+r-1}\circ\cdots\circ\sigma_n(x)$.
On morphisms: a map of sequences $f_n:F_n\to F'_n$ commuting with $\sigma_n$ extends to a graded $k[t]$-linear map $\bigoplus f_n$.

\subsubsection*{Lemma 4.1 ($\Psi$ is well-defined)}
The above formula defines an action of the graded ring $k[t]$ on $M=\bigoplus F_n$ making $M$ a graded $k[t]$-module.

\textbf{Proof.}
We must check associativity and unitality.
Unitality: $1\in k[t]$ acts as identity by definition.
Associativity reduces to checking $(t^r)(t^s)\cdot x = t^{r+s}\cdot x$ for homogeneous $x\in F_n$.
But by definition, $t^s\cdot x$ is obtained by composing $s$ successive structure maps, landing in $F_{n+s}$; applying $t^r$ then composes a further $r$ maps, giving $r+s$ total compositions, which equals $t^{r+s}\cdot x$.
Compatibility with $k$ is clear.
Grading: $t$ has degree $1$, so $t\cdot M_n\subseteq M_{n+1}$. \qed

\subsubsection*{Lemma 4.2 ($\Phi\Psi\cong \mathrm{Id}$ and $\Psi\Phi\cong \mathrm{Id}$)}
\textbf{Claim.} For every sequence $(F_n,\sigma_n)$, $\Phi(\Psi(F_\bullet))$ equals the same sequence.
For every graded $k[t]$-module $M$, $\Psi(\Phi(M))\cong M$ canonically.

\textbf{Proof.}
If we start with $(F_n,\sigma_n)$, then $\Psi$ builds $M=\bigoplus F_n$ with $t$ acting by $\sigma_n$ on degree $n$.
Applying $\Phi$ returns the sequence of graded pieces $M_n=F_n$ and structure maps ``multiply by $t$'', which is exactly $\sigma_n$ by construction.
Thus $\Phi\Psi$ is strictly equal to the identity.

Conversely, start with graded $M=\bigoplus M_n$.
Applying $\Phi$ gives the sequence $F_n=M_n$ with $\sigma_n(m)=t\cdot m$.
Applying $\Psi$ returns the graded module $\bigoplus F_n=\bigoplus M_n$ with $t$ acting by the same rule.
This agrees with the original $k[t]$-module structure on $M$ because a graded $k[t]$-module is determined by the action of $t$ on each degree.
Therefore $\Psi\Phi\cong \mathrm{Id}$ canonically. \qed

\subsubsection*{Conclusion}
The functors $\Phi$ and $\Psi$ are quasi-inverse equivalences.
This proves the theorem.

\subsection*{5) VERIFICATION}
\begin{itemize}[leftmargin=2em]
\item No hidden finiteness assumptions: direct sums and compositions exist in $k$-Mod.
\item The equivalence is exact as stated for nonnegatively graded modules and sequences indexed by $\NN$; other indexings (e.g. $\ZZ$-filtrations) require the Laurent extension $k[t,t^{-1}]$ or bi-infinite sequences.
\end{itemize}

\subsection*{6) FINAL}
\begin{center}
\textbf{PROOF}
\end{center}

