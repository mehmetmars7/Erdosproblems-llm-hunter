\section{MO \#314630: Is there a rigid analytic geometry proof of the functional equation for the Riemann zeta function?}
\MO{314630}{Is there a rigid analytic geometry proof of the functional equation for the Riemann zeta function?}

\subsection*{1) FORMAL RESTATEMENT}
Let $\zeta(s)=\sum_{n\ge 1}n^{-s}$ for $\Re(s)>1$ and let $\xi(s)=\pi^{-s/2}\Gamma(s/2)\zeta(s)$ be the completed zeta function. The classical functional equation is
\[
\xi(s)=\xi(1-s).
\]
The MO question is: \emph{Is there an existing proof of this functional equation whose main ideas are drawn from rigid analytic geometry (non-archimedean analytic geometry), rather than from classical complex analysis / Fourier analysis / adelic harmonic analysis?}

\textbf{Ambiguity:} ``Rigid analytic geometry proof'' is not a standardly formalized phrase; minimally, one would interpret it as a proof formulated in terms of (or reducible to) geometric/cohomological duality statements in a rigid-analytic setting.

\subsection*{2) QUICK LITERATURE/CONTEXT CHECK}
Standard proofs use: Poisson summation and the Jacobi theta function; or Tate's thesis using Fourier transform on adeles; or Mellin transforms and modularity. Cohomological duality explains functional equations for zeta functions of varieties over finite fields, but an analogous cohomology theory for $\operatorname{Spec}\mathbb Z$ giving the Riemann functional equation in a rigid-analytic framework is not established in a way that yields a standalone proof.

\subsection*{3) ATTACK PLAN}
\begin{itemize}[leftmargin=*]
\item \textbf{Proof track:} locate an explicit reference giving a rigid-analytic/cohomological derivation of $\xi(s)=\xi(1-s)$; then rewrite it in a gap-free manner.
\item \textbf{Disproof track:} show that no such proof can exist under a precise definition---but this quickly becomes a meta-mathematical/terminological issue rather than a theorem.
\end{itemize}

\subsection*{4) WORK}
The query is bibliographic/conceptual rather than a precise statement with a truth value. What can be done rigorously here is to separate known mechanisms:
\begin{enumerate}[leftmargin=*]
\item \textbf{Harmonic-analysis mechanism:} the functional equation follows from self-duality of the Gaussian under Fourier transform (Poisson summation) or, adelically, from Fourier analysis on $\mathbb A_\mathbb Q$ (Tate).
\item \textbf{Geometric/cohomological mechanism (finite fields):} functional equations for Hasse--Weil zeta functions follow from Poincar\'e duality in \`etale cohomology.
\item \textbf{``Absolute'' cohomology programs:} there are proposals (e.g. Deninger/Connes--Consani/Lichtenbaum) to interpret the completed zeta function and its functional equation via a cohomology theory of an ``arithmetic'' space. These are not (as of this writing) complete replacement proofs in the rigid-analytic category.
\end{enumerate}

\subsection*{5) VERIFICATION}
Because the prompt is not a mathematically precise claim but a request for existence of a particular style of proof, a gap-free ``proof'' or ``counterexample'' is not well-defined. The correct adversarial move is to pin down a precise meaning; absent that, one can only report what is known and what is not.

\subsection*{6) FINAL}
\textbf{UNRESOLVED.}
\begin{itemize}[leftmargin=*]
\item (i) Fully proved partial result: the formal ambiguity analysis above; and the statement that the functional equation is equivalent to $\xi(s)=\xi(1-s)$ for $\xi(s)=\pi^{-s/2}\Gamma(s/2)\zeta(s)$.
\item (ii) First gap: a precise definition of ``rigid-analytic proof'' plus an existence/nonexistence theorem.
\item (iii) Next moves: (1) search for rigid-analytic formulations of Poisson summation and Fourier transform; (2) examine ``absolute cohomology'' programs for a complete derivation; (3) decide whether $p$-adic/rigid analytic geometry can naturally encode the archimedean Gamma factor.
\item (iv) Minimal counterexample structure: any rigid-analytic approach must still account for the archimedean factor $\Gamma(s/2)$, suggesting that purely non-archimedean geometry alone cannot directly yield the full functional equation without additional archimedean input.
\end{itemize}

