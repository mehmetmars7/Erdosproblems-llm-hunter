\section{MO 77770: Why do polytopes pop up in Lagrange inversion?}

\subsection*{1) FORMAL RESTATEMENT}

\paragraph{Ambiguity / misstatement.}
This is a “why/explain” prompt, not a theorem. To produce a proof, we isolate a precise mathematical claim underlying the observation.

\paragraph{Minimal corrected statement.}
Fix a commutative coefficient ring (say $\mathbb{Z}[c_1,c_2,\dots]$) and let
\[
h(z)=c_1 z + c_2 z^2 + c_3 z^3 + \cdots
\quad\text{with }c_1\neq 0,
\]
and let $H(t)=h^{-1}(t)$ be the unique formal power series satisfying $h(H(t))=t$.

Normalize by scaling: define $\widetilde h(z):=c_1^{-1}h(z)= z + \sum_{m\ge 2}\widetilde c_m z^m$ where $\widetilde c_m=c_m/c_1$.
Then $h^{-1}(t)=\widetilde h^{-1}(t/c_1)$, so it suffices to treat the case $c_1=1$.

\begin{theorem}[Coefficients of $h^{-1}$ enumerate polygon dissections / associahedron faces]\label{thm:lagrange-associahedron}
Assume $h(z)= z+\sum_{m\ge 2} c_m z^m$.
For each integer $m\ge 1$, let $\mathcal{D}_{m+1}$ be the set of dissections of a fixed convex $(m+1)$-gon by noncrossing diagonals (possibly empty set of diagonals).
Each dissection $D\in\mathcal{D}_{m+1}$ partitions the polygon into regions (sub-polygons) $R$, each with $|R|\ge 3$ sides.
Define the weight
\[
w(D):=\prod_{R\in\mathrm{Regions}(D)}\bigl(-c_{|R|-1}\bigr).
\]
Then the coefficient of $t^m$ in the compositional inverse $H(t)=h^{-1}(t)$ satisfies
\[
[t^m]\,H(t) \;=\; \sum_{D\in\mathcal{D}_{m+1}} w(D).
\]
Consequently, expanding the right-hand side as a polynomial in the variables $c_2,c_3,\dots,c_m$,
the coefficient of a monomial $\prod c_{k}^{\alpha_k}$ is exactly (up to the forced sign) the number of dissections whose region sizes have multiplicities $\alpha_k$; these are precisely the refined face numbers of the associahedron $K_{m-2}$ (dimension $m-2$).
\end{theorem}

\subsection*{2) QUICK LITERATURE/CONTEXT CHECK (web available)}

Two classical facts lie behind Theorem \ref{thm:lagrange-associahedron}:
\begin{itemize}[leftmargin=2em]
\item Lagrange inversion counts plane rooted trees (and more generally, solutions to recursive combinatorial specifications).
\item The face lattice of the associahedron is the poset of noncrossing diagonal sets (polygon dissections), and dissections are dual to plane rooted trees.
\end{itemize}

\subsection*{3) ATTACK PLAN}

\begin{enumerate}[leftmargin=2em]
\item Give an explicit bijection between dissections of an $(m+1)$-gon (with a chosen root edge) and plane rooted trees with $m$ leaves, where a region with $r$ sides corresponds to an internal vertex with $r-1$ children.
\item Write down the ordinary generating function $T(t)$ for these trees weighted by $-c_r$ according to outdegree $r$.
\item Show the recursive decomposition of trees yields the functional equation $h(T(t))=t$, so $T(t)=h^{-1}(t)$ and coefficients match.
\end{enumerate}

\subsection*{4) WORK (complete proof)}

\subsubsection*{Step 0: A canonical model for associahedron faces}
A standard combinatorial definition of the $(m-2)$-dimensional associahedron $K_{m-2}$ is:
\emph{its face poset is the poset of noncrossing sets of diagonals in a convex $(m+1)$-gon, ordered by reverse inclusion.}
Thus, counting faces of $K_{m-2}$ is equivalent to counting dissections of the $(m+1)$-gon.

Refining by the multiset of region sizes in the dissection yields the “refined face numbers” mentioned in the prompt.

\subsubsection*{Step 1: Dissections $\leftrightarrow$ plane rooted trees}

\begin{definition}[Plane rooted trees used here]\label{def:plane-rooted-tree}
A \emph{plane rooted tree} is a finite rooted tree in which at each vertex the children are given a linear order (equivalently, the tree is embedded in the plane up to isotopy preserving the root).
A \emph{leaf} is a vertex of outdegree $0$.
\end{definition}

\begin{lemma}[Bijection between polygon dissections and plane rooted trees]\label{lem:dissection-tree}
Fix an integer $m\ge 1$ and a convex polygon $P$ with vertices labeled $0,1,\dots,m$ in counterclockwise order.
Designate the edge $(0,m)$ as the \emph{root edge}.
Then there is a bijection between:
\begin{itemize}[leftmargin=2em]
\item dissections $D$ of $P$ by noncrossing diagonals, and
\item plane rooted trees $T$ with exactly $m$ leaves and with all internal vertices of outdegree $\ge 2$,
\end{itemize}
such that each region $R$ of $D$ having $|R|$ sides corresponds to an internal vertex of $T$ having outdegree $|R|-1$.
\end{lemma}

\begin{proof}
(\emph{Constructing the tree from a dissection.})
Given $D$, consider the planar graph consisting of the boundary edges of $P$ together with the diagonals in $D$.
Since diagonals are noncrossing, this is an outerplanar graph, and the regions cut out inside $P$ are exactly the regions of the dissection.

Form the (planar) dual graph $\Gamma$ of this subdivision \emph{within the disk $P$}: vertices of $\Gamma$ correspond to regions, and an edge of $\Gamma$ connects two regions that share a diagonal.
Because $P$ is simply connected and the subdivision is by noncrossing diagonals, $\Gamma$ is a tree (the dual of a dissection of a polygon is acyclic and connected).

Root $\Gamma$ at the vertex corresponding to the unique region adjacent to the root edge $(0,m)$.
Now turn $\Gamma$ into a \emph{plane rooted tree} by using the cyclic order of diagonals around each region induced by the planar embedding: walking along the boundary of the region in counterclockwise order lists the incident diagonals, and this induces a linear order on the outgoing edges from that region away from the parent region.
Leaves of the resulting rooted tree correspond to boundary edges of $P$ other than the root edge; there are exactly $m$ such edges, hence exactly $m$ leaves.

Finally, consider a region $R$ with $|R|$ sides.
Exactly one side of $R$ is either the root edge (for the root region) or the diagonal to its parent region; the remaining $|R|-1$ sides correspond to either diagonals to child regions or boundary edges (which become leaves).
Thus the corresponding vertex in the rooted tree has exactly $|R|-1$ children, i.e.\ outdegree $|R|-1$.

(\emph{Constructing the dissection from a tree.})
Conversely, given a plane rooted tree with $m$ leaves and internal outdegrees $\ge 2$, one can reconstruct a unique polygon dissection by the standard “dual graph” construction:
embed the tree in the disk, place $m+1$ polygon vertices on the boundary so that the $m$ leaves correspond in order to the $m$ boundary edges other than the root edge, and for each internal vertex of outdegree $r$ create a region with $r+1$ sides and glue regions along edges following the tree adjacency.
Because the tree is connected and acyclic, this produces a simply connected subdivision of the polygon by noncrossing diagonals.

The two constructions are inverse to each other by construction, giving a bijection with the claimed degree correspondence.
\end{proof}

\subsubsection*{Step 2: Weighted generating function and the functional equation}

\begin{definition}[Weighted tree generating function]\label{def:Tt}
Let $\mathcal{T}$ be the class of plane rooted trees in Lemma \ref{lem:dissection-tree}.
Define a weight on a tree $T\in\mathcal{T}$ by
\[
w(T):=\prod_{v\ \mathrm{internal}}\bigl(-c_{\deg^+(v)}\bigr),
\]
where $\deg^+(v)$ is the outdegree (number of children) of $v$, and leaves carry weight $1$.
Let $L(T)$ denote the number of leaves of $T$.
Define the ordinary generating series
\[
T(t):=\sum_{T\in\mathcal{T}} w(T)\, t^{L(T)}\in \mathbb{Z}[c_2,c_3,\dots][[t]].
\]
\end{definition}

\begin{lemma}[Recursive decomposition]\label{lem:tree-equation}
The series $T(t)$ satisfies
\[
T(t) \;=\; t \;+\; \sum_{r\ge 2} \bigl(-c_r\bigr)\, T(t)^r.
\]
\end{lemma}

\begin{proof}
A tree in $\mathcal{T}$ is either:
\begin{itemize}[leftmargin=2em]
\item a single leaf (contributing weight $1$ and $t^{1}$), or
\item an internal root vertex with $r\ge 2$ ordered children, each of which is again a tree in $\mathcal{T}$.
\end{itemize}
In the second case, the weight contributed by the root is $-c_r$, and the total weight is $(-c_r)$ times the product of the weights of the $r$ subtrees.
Since leaf counts add under ordered concatenation, the generating series for an ordered $r$-tuple of trees is $T(t)^r$.
Summing over $r\ge 2$ gives the claimed functional equation.
\end{proof}

\begin{lemma}[Identification with compositional inverse]\label{lem:inverse}
Let $h(z)=z+\sum_{r\ge 2} c_r z^r$. Then $T(t)=h^{-1}(t)$.
\end{lemma}

\begin{proof}
By Lemma \ref{lem:tree-equation},
\[
T(t)= t - \sum_{r\ge 2} c_r T(t)^r.
\]
Rearranging gives
\[
t = T(t) + \sum_{r\ge 2} c_r T(t)^r = h(T(t)).
\]
Since $h(0)=0$ and the coefficient of $z$ in $h(z)$ equals $1$, the functional equation $h(U(t))=t$ has a unique solution $U(t)\in t\,\mathbb{Z}[c_2,c_3,\dots][[t]]$; by definition this solution is $h^{-1}(t)$.
Thus $T(t)=h^{-1}(t)$.
\end{proof}

\subsubsection*{Step 3: Translate back to dissections and refined face numbers}

\begin{proof}[Proof of Theorem \ref{thm:lagrange-associahedron}]
Fix $m\ge 1$.
By Lemma \ref{lem:dissection-tree}, dissections of the $(m+1)$-gon with the chosen root edge are in bijection with plane rooted trees with $m$ leaves, with region sizes corresponding to outdegrees by $|R|-1$.
Under this bijection, the dissection weight $w(D)=\prod_{R}(-c_{|R|-1})$ equals the tree weight $w(T)=\prod_{v}(-c_{\deg^+(v)})$.
Hence the coefficient of $t^m$ in the generating function $T(t)$ equals the sum of weights over $\mathcal{D}_{m+1}$:
\[
[t^m]\,T(t)=\sum_{D\in\mathcal{D}_{m+1}} w(D).
\]
By Lemma \ref{lem:inverse}, $T(t)=h^{-1}(t)$, so the left-hand side is $[t^m]\,h^{-1}(t)$, proving the coefficient identity.

Finally, by definition of “refined face numbers”, fixing the multiset of region sizes in a dissection is equivalent to fixing the monomial $\prod c_{|R|-1}$ occurring in the product $w(D)$; the coefficient of that monomial is exactly the number of faces/dissections of that refinement type.
\end{proof}

\subsubsection*{Phase 1 tiny-case computations (explicit)}
Setting $c_1=1$, the inverse series begins:
\[
h^{-1}(t)= t - c_2 t^2 + (2c_2^2-c_3)t^3 + (-c_4+5c_2c_3-5c_2^3)t^4 + \cdots
\]
For $m=5$ (hexagon dissections), Theorem \ref{thm:lagrange-associahedron} yields:
\[
[t^5]\,h^{-1}(t)= -c_5 + 6c_2c_4 + 3c_3^2 -21c_2^2c_3 + 14c_2^4,
\]
corresponding to:
\begin{itemize}[leftmargin=2em]
\item $14$ triangulations ($3,3,3,3$),
\item $21$ dissections into $(3,3,4)$,
\item $6$ dissections into $(3,5)$,
\item $3$ dissections into $(4,4)$,
\item $1$ empty-diagonal dissection $(6)$.
\end{itemize}

\subsection*{5) VERIFICATION}

\begin{itemize}[leftmargin=2em]
\item The bijection Lemma \ref{lem:dissection-tree} preserves region sizes/outdegrees and counts leaves correctly (polygon has $m+1$ edges, root edge removed gives $m$ leaves).
\item The tree equation Lemma \ref{lem:tree-equation} is purely combinatorial; no analytic convergence assumptions.
\item Uniqueness of compositional inverse in formal power series ring is standard and used explicitly in Lemma \ref{lem:inverse}.
\item Signs: each region contributes a factor $(-1)$, so a dissection with $r$ regions contributes sign $(-1)^r$; this matches the alternating signs observed in inverse-series coefficients.
\end{itemize}

\subsection*{6) FINAL}

\textbf{PROOF.}

\noindent\textbf{Clean theorem statement.}
Theorem \ref{thm:lagrange-associahedron} provides an explicit, gap-free explanation:
Lagrange inversion coefficients count weighted plane rooted trees, and polygon dissections (faces of associahedra) are canonically dual to such trees; therefore refined associahedron face numbers appear as the coefficients of the inverse series.

