\section{Round 2: A variance method lower bound}

\subsection{Round-2 objective}
We pursue path \textbf{(A)} (proof direction: a stronger universal lower bound).
Round~1 established the averaging bound
\(\max_x r(x)>N/4\) (Lemma~36.1) and exhibited an interval construction with
\(\max_x r(x)=N/2\) for even \(N\) (Lemma~36.2).
The principal gap left in Round~1 is to improve the universal lower bound toward the best-known constants.
In this round we prove a strictly stronger, fully rigorous universal estimate:
\[
\max_x r(x)\ \ge\ \frac{N}{\sqrt{8-\frac{1}{N^2}}}\ >\ \frac{N}{\sqrt 8}.
\]
In particular, the minimum overlap constant satisfies \(c\ge 1/\sqrt 8\approx 0.353553\), improving the Round~1 bound \(c\ge 1/4\).

\subsection{Round-1 foundation used}
We use the following Round~1 results as black boxes (no re-proofs):
\begin{itemize}
\item The setup: a partition \(A\sqcup B=\{1,2,\dots,2N\}\) with \(|A|=|B|=N\), and
\(r(x)=\#\{(a,b)\in A\times B: a-b=x\}\).
\item The counting identity \(\sum_x r(x)=N^2\), hence Lemma~36.1:
\(\max_x r(x)\ge N^2/(4N-1)>N/4\).
\item Lemma~36.2 (interval construction) as a sanity check; it is not used in the proofs below.
\end{itemize}

\subsection{New insight / tool (Round-2)}
The new ingredient is a second-moment constraint.
Choose independent random variables
\(a\sim \mathrm{Unif}(A)\) and \(b\sim \mathrm{Unif}(B)\), and set
\(D:=a-b\).
Then
\[
\mathbb P(D=x)=\frac{r(x)}{N^2},\qquad \max_x\mathbb P(D=x)=\frac{\max_x r(x)}{N^2}.
\]
We combine:
\begin{itemize}
\item an \emph{upper} bound on \(\mathrm{Var}(D)\) coming only from the fact that \(A\) and \(B\) are complementary equal halves of \([2N]\), and
\item a \emph{lower} bound on \(\mathrm{Var}(D)\) in terms of the maximum atom \(\max_x\mathbb P(D=x)\), obtained by smoothing an integer-valued law by a continuous uniform random variable.
\end{itemize}

\subsection{Attack plan (Round-2)}
The Round~1 gap is that averaging alone forces only \(\max_x r(x)\gtrsim N/4\).
To improve this, we:
\begin{enumerate}
\item interpret \(r(x)/N^2\) as the law of \(D=a-b\);
\item show \(\mathrm{Var}(D)\le (4N^2-1)/6\);
\item show that if \(p:=\max_x\mathbb P(D=x)\) then
\(\mathrm{Var}(D)\ge \frac{1}{12p^2}-\frac{1}{12}\);
\item combine (2)--(3) and solve for \(p\), hence for \(\max_x r(x)\).
\end{enumerate}

\subsection{Work (Round-2)}
Let
\[
M:=\max_x r(x).
\]
With \(D=a-b\) as above, we have
\[
\max_x\mathbb P(D=x)=\frac{M}{N^2}.
\]

\medskip
\noindent\textbf{Lemma 36.3 (variance upper bound for \(D=a-b\)).}
For every partition \(A\sqcup B=[2N]\) with \(|A|=|B|=N\),
\[
\mathrm{Var}(D)\ \le\ \frac{4N^2-1}{6}.
\]

\noindent\emph{Proof.}
Since \(a\) and \(b\) are independent,
\(\mathrm{Var}(D)=\mathrm{Var}(a)+\mathrm{Var}(b)\).
Write
\[
S:=\sum_{i=1}^{2N} i = N(2N+1),\qquad
Q:=\sum_{i=1}^{2N} i^2=\frac{(2N)(2N+1)(4N+1)}{6}.
\]
Let
\(A_1:=\sum_{a\in A}a\) and \(A_2:=\sum_{a\in A}a^2\), so
\(B_1=S-A_1\), \(B_2=Q-A_2\).
Then
\[
\mathrm{Var}(a)=\frac{A_2}{N}-\left(\frac{A_1}{N}\right)^2,
\qquad
\mathrm{Var}(b)=\frac{B_2}{N}-\left(\frac{B_1}{N}\right)^2.
\]
Hence
\[
\mathrm{Var}(a)+\mathrm{Var}(b)
=\frac{A_2+B_2}{N}-\frac{A_1^2+B_1^2}{N^2}
=\frac{Q}{N}-\frac{A_1^2+(S-A_1)^2}{N^2}.
\]
But
\[
A_1^2+(S-A_1)^2
=2\left(A_1-\frac{S}{2}\right)^2+\frac{S^2}{2}\ \ge\ \frac{S^2}{2}.
\]
Therefore
\[
\mathrm{Var}(D)\le \frac{Q}{N}-\frac{S^2}{2N^2}.
\]
Compute
\(\frac{Q}{N}=\frac{(2N+1)(4N+1)}{3}\) and
\(\frac{S^2}{2N^2}=\frac{(2N+1)^2}{2}\), so
\[
\mathrm{Var}(D)
\le (2N+1)\left(\frac{4N+1}{3}-\frac{2N+1}{2}\right)
=(2N+1)\cdot\frac{2N-1}{6}
=\frac{4N^2-1}{6}.
\]
\qed

\medskip
\noindent\textbf{Lemma 36.4 (integer-valued variance lower bound from bounded atoms).}
Let \(X\) be an integer-valued random variable such that
\(\mathbb P(X=k)\le p\) for all \(k\in\mathbb Z\), where \(0<p\le 1\).
Then
\[
\mathrm{Var}(X)\ \ge\ \frac{1}{12p^2}-\frac{1}{12}.
\]

\noindent\emph{Proof.}
Let \(U\sim\mathrm{Unif}(-\tfrac12,\tfrac12)\) be independent of \(X\), and set \(Y:=X+U\).
For each integer \(k\), on the interval \([k-\tfrac12,k+\tfrac12)\) the density of \(Y\) equals \(\mathbb P(X=k)\), so \(Y\) has a density \(f\) satisfying
\[
0\le f(y)\le p\ \text{ for all }y\in\mathbb R,\qquad \int_{\mathbb R} f(y)\,dy=1.
\]
Let \(\mu:=\mathbb E[Y]\) and \(Z:=Y-\mu\). Then \(\mathrm{Var}(Y)=\mathbb E[Z^2]\) and the density of \(Z\) is a translate of \(f\), hence also bounded by \(p\).
Therefore, for any \(r\ge 0\),
\[
\mathbb P(|Z|\le r)\le \int_{-r}^{r} p\,dt=2rp,
\quad\text{so}\quad
\mathbb P(|Z|>r)\ge \max(0,1-2rp).
\]
Using the tail integral identity
\[
\mathbb E[Z^2]=\int_0^{\infty} \mathbb P(Z^2>t)\,dt
=\int_0^{\infty} 2r\,\mathbb P(|Z|>r)\,dr
\]
(the second equality is the substitution \(t=r^2\), \(dt=2r\,dr\)), we obtain
\[
\mathrm{Var}(Y)=\mathbb E[Z^2]
\ge \int_0^{1/(2p)} 2r(1-2rp)\,dr
=\left[r^2-\frac{4p}{3}r^3\right]_{0}^{1/(2p)}
=\frac{1}{12p^2}.
\]
Finally, since \(Y=X+U\) with \(X\perp U\),
\(
\mathrm{Var}(Y)=\mathrm{Var}(X)+\mathrm{Var}(U)=\mathrm{Var}(X)+1/12
\).
Thus
\(
\mathrm{Var}(X)+1/12\ge 1/(12p^2)
\), i.e.
\(
\mathrm{Var}(X)\ge 1/(12p^2)-1/12
\).
\qed

\medskip
\noindent\textbf{Theorem 36.5 (universal lower bound \(>N/\sqrt8\)).}
For every \(N\ge 1\) and every partition \(A\sqcup B=[2N]\) with \(|A|=|B|=N\),
\[
M=\max_x r(x)\ \ge\ \frac{N}{\sqrt{8-\frac{1}{N^2}}}\ >\ \frac{N}{\sqrt 8}.
\]
Consequently the minimum overlap constant satisfies \(c\ge 1/\sqrt 8\approx 0.353553\).

\noindent\emph{Proof.}
Apply Lemma~36.4 to \(X=D\). Here
\(p:=\max_x\mathbb P(D=x)=M/N^2\), so Lemma~36.4 gives
\[
\mathrm{Var}(D)\ \ge\ \frac{1}{12p^2}-\frac{1}{12}
=\frac{N^4}{12M^2}-\frac{1}{12}.
\]
Lemma~36.3 gives the universal upper bound
\(\mathrm{Var}(D)\le (4N^2-1)/6\).
Combining,
\[
\frac{N^4}{12M^2}-\frac{1}{12}\ \le\ \frac{4N^2-1}{6}
\quad\Rightarrow\quad
\frac{N^4}{12M^2}\ \le\ \frac{8N^2-1}{12}.
\]
Multiplying by \(12\) and rearranging yields
\(
N^4/M^2\le 8N^2-1
\), hence
\[
M^2\ \ge\ \frac{N^4}{8N^2-1}
\quad\Rightarrow\quad
M\ \ge\ \frac{N^2}{\sqrt{8N^2-1}}=\frac{N}{\sqrt{8-\frac{1}{N^2}}}.
\]
This proves the theorem.
\qed

\subsection{Adversarial verification}
\begin{itemize}
\item \emph{Distribution identity.} Each ordered pair \((a,b)\in A\times B\) is equally likely, so
\(\mathbb P(D=x)=r(x)/N^2\) exactly.
\item \emph{Lemma 36.4 hypotheses.} The only hypothesis is that \(X\) is integer-valued and its atoms are bounded by \(p\). Smoothing by \(U\) converts the discrete law to a continuous density bounded by \(p\) on \emph{all} of \(\mathbb R\), so the interval bound \(\mathbb P(|Z|\le r)\le 2rp\) is valid.
\item \emph{Tail integral identity.} The identity
\(\mathbb E[Z^2]=\int_0^{\infty} \mathbb P(Z^2>t)dt\)
holds for nonnegative random variables by Tonelli's theorem; the substitution \(t=r^2\) is elementary.
\item \emph{Small \(N\).} For \(N=1\), the bound gives \(M\ge 1/\sqrt 7\), which forces \(M\ge 1\) since \(M\in\mathbb Z_{\ge 1}\). No exceptional cases arise.
\item \emph{Consistency with Round~1 computations.} The exhaustive table for \(N\le 11\) in Round~1 has
\(M(N)/N\ge 0.429\ldots\), which is comfortably above \(1/\sqrt 8\).
\end{itemize}

\subsection{Final}
\textbf{UNRESOLVED (but strictly advanced).}
We have improved the universal lower bound from Lemma~36.1:
\(\max_x r(x) > N/4\)
to the stronger bound of Theorem~36.5:
\(\max_x r(x) > N/\sqrt 8\).
The exact optimal constant \(c\) remains open.

\subsection{Completion estimate}
\textbf{COMPLETION: 45\%}.

\subsection{References}
\begin{itemize}
\item Round~1 notes and computations contained in \texttt{36.tex}.
\item Ethan Patrick White, ``Erd\H{o}s' minimum overlap problem,'' arXiv:2201.05704.
\end{itemize}

\section{Round 3: Sharpness of the variance method and a barrier at $1/\sqrt 8$}

\subsection{Round-3 objective}
We pursue path \textbf{(A)} (proof direction), but in ``gap-closure'' mode: the Round~2 lower bound
\(c\ge 1/\sqrt 8\) is still far below the current record range.
The main goal of this round is to isolate (with proof) why the variance method of Round~2 cannot be
pushed past \(1/\sqrt 8\) \emph{without injecting genuinely new structure}.
Concretely, we prove that both ingredients of Round~2---the variance upper bound (Lemma~36.3) and the atom/variance lower bound (Lemma~36.4)---are individually sharp, and hence the constant \(1/\sqrt 8\) is an intrinsic barrier for any argument that treats \(D=a-b\) only as an arbitrary integer-valued law with bounded variance.

For context (not used in the proofs below), the current record bounds for the asymptotic constant are extremely narrow:
\(0.379005 < c < 0.380924\) (White's lower bound and AlphaEvolve's upper bound).

\subsection{Round-2 foundation used}
We use the following Round~2 statements as black boxes:
\begin{itemize}
\item Lemma~36.3: for every partition \(A\sqcup B=[2N]\) with \(|A|=|B|=N\), the random difference \(D=a-b\) satisfies
\(\mathrm{Var}(D)\le (4N^2-1)/6\).
\item Lemma~36.4: for any integer-valued random variable \(X\) with \(\max_k \mathbb P(X=k)\le p\),
\(\mathrm{Var}(X)\ge 1/(12p^2)-1/12\).
\item Theorem~36.5: combining the two yields
\(\max_x r(x)\ge N/\sqrt{8-1/N^2}>N/\sqrt 8\).
\end{itemize}

\subsection{New insight / tool (Round-3)}
The new content is a \emph{sharpness and barrier analysis}:
\begin{itemize}
\item We identify an equality case for Lemma~36.4 (discrete uniform distributions), proving that Lemma~36.4 cannot be improved under its stated hypotheses.
\item We give an explicit partition \(A\sqcup B=[2N]\) (for even \(N\)) achieving equality in Lemma~36.3, proving that the variance upper bound is also best possible.
\item We combine these to formalize a ``variance-method barrier'': no proof that only inputs ``\(D\) is integer-valued'' and ``\(\mathrm{Var}(D)\le (4N^2-1)/6\)'' can yield a universal constant better than \(1/\sqrt 8\).
\item As a byproduct, we extract a structural necessary condition: any near-extremal configuration with \(\max_x r(x)\) close to \(N/\sqrt 8\) must have its first moment nearly balanced (\(\sum_{a\in A} a\) close to \(\sum_{b\in B} b\)).
\end{itemize}

\subsection{Attack plan (Round-3)}
The gap after Round~2 is that the best-known constants are near \(0.38\), while the variance method yields only \(1/\sqrt 8\approx 0.3536\).
To understand what is missing, we:
\begin{enumerate}
\item prove sharpness of the atom/variance inequality (Lemma~36.4);
\item prove sharpness of the variance upper bound (Lemma~36.3);
\item conclude that improving Theorem~36.5 requires 
\emph{new constraints on the law of \(D\) beyond its variance} (e.g. constraints coming from the fact that \(D\) is a difference of two uniform measures supported on complementary subsets of \([2N]\)).
\end{enumerate}

\subsection{Work (Round-3)}

\medskip
\noindent\textbf{Lemma 36.6 (sharpness of Lemma~36.4).}
For each integer \(m\ge 0\), let \(X\) be uniform on \(\{-m,-m+1,\dots,m\}\).
Then \(\max_k \mathbb P(X=k)=p=1/(2m+1)\) and
\[
\mathrm{Var}(X)=\frac{1}{12p^2}-\frac{1}{12}.
\]
In particular, Lemma~36.4 is best possible as stated.

\noindent\emph{Proof.}
Here \(p=1/(2m+1)\).
By symmetry \(\mathbb E[X]=0\) and
\[
\mathrm{Var}(X)=\mathbb E[X^2]=\frac{1}{2m+1}\sum_{k=-m}^m k^2
=\frac{2}{2m+1}\sum_{k=1}^m k^2
=\frac{2}{2m+1}\cdot\frac{m(m+1)(2m+1)}{6}
=\frac{m(m+1)}{3}.
\]
On the other hand,
\[
\frac{1}{12p^2}-\frac{1}{12}=\frac{(2m+1)^2-1}{12}=\frac{4m(m+1)}{12}=\frac{m(m+1)}{3},
\]
so equality holds.
\qed

\medskip
\noindent\textbf{Lemma 36.7 (sharpness of Lemma~36.3 for even \(N\)).}
Assume \(N\) is even and set
\[
A:=\{1,2,\dots,\tfrac{N}{2}\}\ \cup\ \{\tfrac{3N}{2}+1,\tfrac{3N}{2}+2,\dots,2N\},
\qquad B:=[2N]\setminus A.
\]
Then \(|A|=|B|=N\) and for the associated random difference \(D=a-b\),
\[
\mathrm{Var}(D)=\frac{4N^2-1}{6}.
\]
Hence the variance upper bound of Lemma~36.3 is attained.

\noindent\emph{Proof.}
The set \(A\) consists of \(N/2\) pairs
\((i,2N+1-i)\) for \(1\le i\le N/2\).
Each such pair sums to \(2N+1\), hence
\[
A_1:=\sum_{a\in A} a=\frac{N}{2}(2N+1)=\frac{S}{2},
\qquad S:=\sum_{j=1}^{2N} j = N(2N+1).
\]
Therefore also \(B_1=S-A_1=S/2\), i.e. \(A_1=B_1\).
In the computation of Lemma~36.3 we had
\[
\mathrm{Var}(D)=\frac{Q}{N}-\frac{A_1^2+B_1^2}{N^2},
\qquad Q:=\sum_{j=1}^{2N} j^2.
\]
Since \(A_1=B_1=S/2\), we have \(A_1^2+B_1^2=S^2/2\), which is exactly the minimal value used in Lemma~36.3.
Thus equality holds throughout the proof of Lemma~36.3 and
\(\mathrm{Var}(D)=Q/N-S^2/(2N^2)=(4N^2-1)/6\).
\qed

\medskip
\noindent\textbf{Theorem 36.8 (variance-method barrier at \(1/\sqrt 8\)).}
Consider the class of all integer-valued random variables \(X\) with
\(\mathrm{Var}(X)\le (4N^2-1)/6\).
For such \(X\), any lower bound of the form
\(\max_k \mathbb P(X=k)\ge C/N\) valid for all \(N\) must satisfy
\(C\le 1/\sqrt 8\).
Equivalently, the constant \(1/\sqrt 8\) appearing in Theorem~36.5 is optimal for any argument that uses only the information ``\(D\) is integer-valued'' and ``\(\mathrm{Var}(D)\le (4N^2-1)/6\)''.

\noindent\emph{Proof.}
Fix \(N\ge 1\) and define
\(m:=\lfloor \sqrt{2}\,N-1\rfloor\).
Let \(X\) be uniform on \(\{-m,-m+1,\dots,m\}\).
Then by Lemma~36.6,
\[
\max_k \mathbb P(X=k)=\frac{1}{2m+1}
\quad\text{and}\quad
\mathrm{Var}(X)=\frac{m(m+1)}{3}.
\]
Using \(m\le \sqrt{2}N-1\), we have
\[
 m(m+1)\le (\sqrt{2}N-1)(\sqrt{2}N)=2N^2-\sqrt{2}N\le 2N^2-\tfrac12,
\]
hence
\[
\mathrm{Var}(X)=\frac{m(m+1)}{3}\le \frac{2N^2-\tfrac12}{3}=\frac{4N^2-1}{6}.
\]
Therefore \(X\) belongs to the stated variance-bounded class. But also
\[
\frac{1}{2m+1}=\frac{1+o(1)}{2\sqrt{2}\,N}=\frac{1+o(1)}{\sqrt{8}\,N}
\qquad (N\to\infty).
\]
Consequently, no universal inequality of the form
\(\max_k \mathbb P(X=k)\ge C/N\)
can hold with \(C>1/\sqrt 8\) on this class.
\qed

\medskip
\noindent\textbf{Corollary 36.9 (near-extremizers force near-balanced first moment).}
Let \(A\sqcup B=[2N]\), \(|A|=|B|=N\), and write
\(A_1:=\sum_{a\in A} a\), \(S:=\sum_{j=1}^{2N} j=N(2N+1)\), and
\(\delta:=A_1-S/2\).
Let \(M:=\max_x r(x)\).
Then
\[
\delta^2\ \le\ \frac{N^2}{2}\left(\frac{4N^2-1}{6}-\frac{N^4}{12M^2}+\frac{1}{12}\right).
\]
In particular, if \(M\le cN\) for some fixed \(c\), then \(|\delta|=O(N^2)\), and as \(c\downarrow 1/\sqrt 8\) the right-hand side forces \(|\delta|=o(N^2)\).

\noindent\emph{Proof.}
From the identity in the proof of Lemma~36.3 we have the \emph{exact} formula
\[
\mathrm{Var}(D)=\frac{Q}{N}-\frac{A_1^2+(S-A_1)^2}{N^2}
=\frac{4N^2-1}{6}-\frac{2\delta^2}{N^2}.
\]
(Indeed, \(A_1^2+(S-A_1)^2=2\delta^2+S^2/2\).)
On the other hand, Theorem~36.5 begins with the inequality (from Lemma~36.4)
\(
\mathrm{Var}(D)\ge N^4/(12M^2)-1/12.
\)
Combining yields
\[
\frac{4N^2-1}{6}-\frac{2\delta^2}{N^2}\ \ge\ \frac{N^4}{12M^2}-\frac{1}{12},
\]
which rearranges to the claimed bound on \(\delta^2\).
\qed

\subsection{Adversarial verification}
\begin{itemize}
\item \emph{Lemma 36.6.} The variance computation is exact and the equality check against \(1/(12p^2)-1/12\) is algebraic; no hidden regularity assumptions are used.
\item \emph{Lemma 36.7.} The construction requires \(N\) even to make the endpoints \(N/2\) and \(3N/2\) integral. The sum identity uses the explicit pairing \((i,2N+1-i)\). Equality in Lemma~36.3 is correctly characterized by \(A_1=B_1\), which holds here.
\item \emph{Theorem 36.8.} We only claim a \emph{methodological barrier} for variance-only arguments. The random variables \(X\) used in the proof need not arise as \(a-b\) for any partition; this is intentional, as the theorem is about what can be deduced from variance constraints alone.
\item \emph{Corollary 36.9.} The exact variance formula is consistent with Lemma~36.3 (it immediately implies Lemma~36.3 because \(\delta^2\ge 0\)). The rearrangement is straightforward.
\end{itemize}

\subsection{Final}
\textbf{UNRESOLVED (but strictly advanced).}
Beyond the Round~2 bound \(c\ge 1/\sqrt 8\), we have now proved that \(1/\sqrt 8\) is a genuine barrier for the entire ``variance-only'' proof strategy: improving the universal lower bound requires additional structural constraints on the representation function \(r\), not just its second moment.
We also extracted a quantitative structural condition (Corollary~36.9) showing that near-extremizers must have nearly balanced first moment.

\subsection{Completion estimate}
\textbf{COMPLETION: 55\%}.

\subsection{References}
\begin{itemize}
\item Round~2 notes in \texttt{36\_v2.tex}.
\item T. F. Bloom (ed.), Erd\H{o}s Problem \#36 discussion thread (``minimum overlap problem''), online resource. 
\item Ethan Patrick White, ``Erd\H{o}s' minimum overlap problem,'' arXiv:2201.05704.
\end{itemize}
