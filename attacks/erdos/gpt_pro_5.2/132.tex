% Erdos problem #132
% Attempt for Erdos Problem #132
% Following PROMPT_STRATEGY.MD
% Tools/Constraints:
% - Web browsing available? YES (not used)
% - Computation available? not needed

1) FORMAL RESTATEMENT

Let $A$ be a set of $n$ points in the plane. Consider the multiset of $\binom{n}{2}$ Euclidean distances determined by unordered pairs of distinct points.
For a distance value $d>0$, let $m(d)$ be the number of pairs at distance $d$.
Call $d$ a "rare distance" if $1\le m(d)\le n$.

Question 1: Must there exist at least two rare distances?
Question 2: Must the number of rare distances tend to infinity as $n\to\infty$?

2) QUICK LITERATURE/CONTEXT CHECK

(Restricted to statements explicitly present in 123-137.tex.)
- The statement is false for $n=4$: two equilateral triangles glued together yield only one rare distance.
- Hopf and Pannowitz (1934) proved there is always at least one distance that occurs between at most $n$ pairs (in fact, the diameter occurs at most $n$ times).
- It is unknown whether the number of such distances grows with $n$.

3) ATTACK PLAN

- Give an explicit counterexample for Question 1 using $n=4$ and verify multiplicities.
- Provide partial observations for Question 2 but it appears open in the text.

4) WORK

Counterexample to Question 1 (n=4).
Place points
\[
A=(0,0),\quad B=(1,0),\quad C=\left(\tfrac12,\tfrac{\sqrt3}{2}\right),\quad D=\left(\tfrac12,-\tfrac{\sqrt3}{2}\right).
\]
Then $ABC$ is an equilateral triangle of side 1 and $ABD$ is another equilateral triangle of side 1 sharing the side $AB$.
Compute the 6 pairwise distances:
- $|AB|=1$.
- $|AC|=|BC|=1$ (equilateral).
- $|AD|=|BD|=1$ (equilateral).
- $|CD|=\sqrt{ (0)^2 + (\sqrt3)^2 }=\sqrt3$.
Thus the distance multiset consists of 1 repeated 5 times and $\sqrt3$ repeated 1 time.
For $n=4$, the threshold for being rare is $\le n=4$.
So $m(\sqrt3)=1\le 4$ is rare, but $m(1)=5>4$ is not.
Hence there is only one rare distance, disproving Question 1 as stated.

5) VERIFICATION

- The coordinate placement is explicit.
- All five unit distances correspond to the five edges present among the two glued equilateral triangles.
- The remaining pair $\{C,D\}$ has distance $\sqrt3$.
No other distances occur.

6) FINAL

**UNRESOLVED**

(i) Strongest fully proved partial result:
- Question 1 is false as stated (explicit counterexample for $n=4$ above).

(ii) Exact first gap:
- Determine whether for $n\ge 5$ there must be at least two rare distances, and whether the number of rare distances must grow unboundedly with $n$ (Question 2).

(iii) Top 3 next moves:
1. Prove lower bounds on the number of distinct distance multiplicities $\le n$ using combinatorial geometry (incidence bounds) and structural properties of high-multiplicity distances.
2. Search for constructions with few rare distances for larger $n$ (extensions of glued-triangle constructions), testing whether one can keep all but O(1) distances having multiplicity $>n$.
3. Use known facts about the diameter graph (as cited in the text) as one guaranteed rare distance and attempt to find a second using convex hull / antipodal pair arguments.

(iv) What a minimal counterexample would likely look like:
- A point set where one distance value occurs very frequently (superlinear in $n$) and all other distances either do not occur or occur more than $n$ times, requiring a very rigid repeated-distance structure.


