% Erdos Problem #826

\subsection*{FORMAL RESTATEMENT}
Let $\tau(n)$ denote the number of positive divisors of $n$. The question is:

\medskip
\noindent\textbf{Problem.} Do there exist infinitely many integers $n\ge 1$ such that for all integers $k\ge 1$,
\[
\tau(n+k) \ll k,
\]
meaning: there exists an absolute constant $C>0$ (independent of $n$ and $k$) such that for the given $n$,
\[
\tau(n+k)\le Ck\qquad\text{for all }k\ge 1?
\]

\subsection*{QUICK LITERATURE/CONTEXT CHECK}
The statement says this is a stronger form of another Erd\H{o}s problem (\#248). No further results are stated in the file.

\subsection*{ATTACK PLAN}
\begin{itemize}
\item \textbf{Proof track:} Try to construct $n$ so that $n+1,n+2,\dots$ systematically avoid having too many divisors relative to their offset $k$. For small $k$ this resembles finding prime/rough patterns in short intervals, which is difficult.
\item \textbf{Disproof track:} Try to show that for every $n$, some $k$ forces $\tau(n+k)$ to be larger than $Ck$ for any fixed $C$. This would require producing very highly composite numbers in every translate, at controlled distance, which also seems difficult.
\item \textbf{What I will do:} Prove elementary lemmas that reduce the quantifier "for all $k\ge 1$" to finitely many checks for a fixed constant $C\ge 2$, and do an exact search for small $n$ for the case $C=2$ and summary counts for $C=3,4,5$ up to $50{,}000$.
\end{itemize}

\subsection*{WORK}
\paragraph{FAST REALITY CHECK (computation for fixed constants).}
Define a property $P_C(n)$: $\tau(n+k)\le Ck$ for all $k\ge 1$.
\begin{itemize}
\item For $C=2$, an exact search up to $n\le 1{,}000{,}000$ found precisely
\[
\{n\le 10^6: P_2(n)\} = \{1,2,4,6,12,36,60,72,420\}.
\]
In particular there are only 9 such $n$ up to $10^6$.
\item For $N=50{,}000$, the counts of $n\le N$ with $P_C(n)$ were:
\[
\#\{n\le 50{,}000: P_2(n)\}=9,
\quad \#\{n\le 50{,}000: P_3(n)\}=82,
\quad \#\{n\le 50{,}000: P_4(n)\}=1905,
\quad \#\{n\le 50{,}000: P_5(n)\}=2720.
\]
\end{itemize}
These computations do not prove infinitude for any $C$, but they indicate the constraint is very strong for small $C$.

\noindent\textbf{Lemma (Trivial divisor bound).}
For every integer $m\ge 1$, one has $\tau(m)\le m$.

\noindent\textbf{Proof.}
Every positive divisor of $m$ is at least $1$ and at most $m$, so there are at most $m$ distinct positive divisors.
\hfill$\square$


\noindent\textbf{Lemma (For $C\ge 2$, only finitely many $k$ need checking).}
Fix $C\ge 2$ and $n\ge 1$. If $\tau(n+k)\le Ck$ holds for every integer $k$ with $1\le k\le n$, then it holds for all integers $k\ge 1$.

\noindent\textbf{Proof.}
Assume $\tau(n+k)\le Ck$ for all $1\le k\le n$.

Let $k>n$. Then $n+k < 2k$ and by the trivial bound $\tau(m)\le m$,
\[
\tau(n+k)\le n+k < 2k \le Ck
\]
because $C\ge 2$. Thus the inequality holds automatically for all $k>n$.
\hfill$\square$


\noindent\textbf{Lemma (Divisor bound $\tau(m)\le 2\sqrt m$ and an automatic $k$-range).}
For every $m\ge 1$, one has $\tau(m)\le 2\sqrt m$. Consequently, for any $n\ge 1$, any constant $C>0$, and any integer $k$ such that
\[
k \ge \frac{2}{C^2}\bigl(1+\sqrt{1+C^2 n}\bigr),
\]
one has
\[
\tau(n+k)\le Ck.
\]

\noindent\textbf{Proof.}
For the bound $\tau(m)\le 2\sqrt m$: pair each divisor $d\le \sqrt m$ with the (distinct) divisor $m/d\ge \sqrt m$. There are at most $\lfloor \sqrt m\rfloor$ divisors $d\le \sqrt m$, hence
\[
\tau(m)\le 2\lfloor \sqrt m\rfloor \le 2\sqrt m.
\]

Now fix $n\ge 1$ and $C>0$ and assume $k\ge \frac{2}{C^2}(1+\sqrt{1+C^2 n})$. A direct rearrangement shows this is exactly the condition that the quadratic inequality
\[
C^2k^2-4k-4n\ge 0
\]
holds, i.e. that $C^2k^2\ge 4(n+k)$. Taking square roots (all terms are nonnegative) gives
\[
Ck\ge 2\sqrt{n+k}.
\]
Therefore
\[
\tau(n+k)\le 2\sqrt{n+k}\le Ck.
\]
\hfill$\square$


\subsection*{VERIFICATION}
\begin{itemize}
\item Lemma (For $C\ge 2$, only finitely many $k$ need checking): checked the strict inequality $n+k<2k$ for $k>n$; the argument needs $C\ge 2$ to conclude $2k\le Ck$.
\item Lemma (Divisor bound $\tau(m)\le 2\sqrt m$ and an automatic $k$-range): checked that the displayed condition on $k$ is equivalent to the quadratic inequality $C^2k^2-4k-4n\ge 0$ and hence to $2\sqrt{n+k}\le Ck$.
\item Computation: For $C=2$, after the finite-check lemma it suffices to check $1\le k\le n$; the search implemented exactly this and cross-checked up to $10^6$.
\end{itemize}

\subsection*{FINAL}
\textbf{UNRESOLVED.}
\begin{enumerate}
\item[(i)] \textbf{Strongest proved partial result.} For any fixed $C\ge 2$, to verify $\tau(n+k)\le Ck$ for all $k\ge 1$ it suffices to check the finite range $1\le k\le n$ (by the finite-check lemma). Computations show only 9 values of $n\le 10^6$ satisfy the inequality with $C=2$.
\item[(ii)] \textbf{First gap.} Prove or disprove that there exists an absolute constant $C$ such that $P_C(n)$ holds for infinitely many $n$.
\item[(iii)] \textbf{Top 3 next moves.}
  \begin{enumerate}
  \item For a given candidate $C$ (say $C=2$ or $3$), search for structural necessary conditions on $n$ coming from the small-$k$ constraints (e.g. $k=1$ forces $n+1$ to have at most $C$ divisors), and try to prove that these conditions can or cannot hold infinitely often.
  \item Use the $\tau(m)\le 2\sqrt m$ lemma to reduce the needed verification range further (from $k\le n$ down to $k\le O(\sqrt n)$ with better constants depending on $C$) and then combine with distributional results about $\tau$ in short intervals (if available in the allowed corpus).
  \item Extend computations to larger $N$ with optimized divisor-count sieves to see whether the set of $n$ satisfying $P_2(n)$ stabilizes (suggesting finiteness) or new examples appear.
  \end{enumerate}
\item[(iv)] \textbf{Minimal counterexample structure (if the statement were false).} For each fixed $C$, one would need that $P_C(n)$ fails for all sufficiently large $n$. A minimal obstruction would be a mechanism guaranteeing, for every large $n$, the existence of some small $k$ (say $k\le K(C)$) with $\tau(n+k)>Ck$; i.e. a uniform supply of unusually highly composite values among the first $K(C)$ successors of every $n$.
\end{enumerate}


