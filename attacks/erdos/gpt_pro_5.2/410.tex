
1) FORMAL RESTATEMENT

Let $\sigma(n)=\sum_{d\mid n} d$ be the sum-of-divisors function.
Define iterates $\sigma_1(n)=\sigma(n)$ and $\sigma_k(n)=\sigma(\sigma_{k-1}(n))$ for $k\ge 2$.

Literal question as written.
Is it true that
\[
\lim_{k\to\infty} \sigma_k(n)^{1/k}=\infty?
\]
(As written, the quantifier “for which $n$” is not explicit; the natural reading is: for every integer $n\ge 1$.)

Minimal corrected statement (standard convention).
A nontrivial version is: for every integer $n\ge 2$,
\[
\lim_{k\to\infty} \sigma_k(n)^{1/k}=\infty.
\]

2) QUICK LITERATURE/CONTEXT CHECK

The problem statement only notes that the question is discussed in Guy's collection; I do not add any external results.

3) ATTACK PLAN

Disproof-track for the literal statement:
- Check small edge cases (notably $n=1$) to see whether the limit can fail trivially.

Proof-track for the corrected statement ($n\ge 2$):
- Establish monotonic growth of $\sigma_k(n)$ and look for multiplicative lower bounds on $\sigma(n)/n$ along the orbit.

Chosen path: give a counterexample to the literal statement and briefly record basic growth facts for $n\ge 2$.

4) WORK

Counterexample to the literal statement.
Take $n=1$.
Then $\sigma(1)=1$ (the only divisor of $1$ is $1$), so the iterates satisfy
\[
\sigma_k(1)=1\quad\text{for all }k\ge 1.
\]
Therefore
\[
\sigma_k(1)^{1/k}=1^{1/k}=1\quad\text{for all }k\ge 1,
\]
and hence
\[
\lim_{k\to\infty} \sigma_k(1)^{1/k}=1\ne\infty.
\]
This disproves the literal statement “for every $n\ge 1$”.

(Additional sanity computation for $n\ge 2$.)
For example, the first few iterates for $n=2$ are
\[
2\to 3\to 4\to 7\to 8\to 15\to 24\to 60\to\cdots,
\]
which grows quickly, but this does not address the $k$th-root limit.

5) VERIFICATION

- Verified $\sigma(1)=1$ directly from the definition.
- The iteration is constant, so the limit computation is exact.

6) FINAL

**FULL SOLUTION**

**COUNTEREXAMPLE/DISPROOF**

The literal statement is false: for $n=1$, all iterates satisfy $\sigma_k(1)=1$, so $\sigma_k(1)^{1/k}=1$ for all $k$, and the limit equals $1$ rather than $\infty$.
