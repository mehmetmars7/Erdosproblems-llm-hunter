% Erdos Problem #569
% URL: https://www.erdosproblems.com/569

1) FORMAL RESTATEMENT

Let $k\ge 1$ be fixed and let $C_{2k+1}$ be the cycle on $2k+1$ vertices. Define $c_k$ to be the infimum over constants $c$ such that for every graph $H$ with $m=e(H)$ edges and no isolated vertices,
\[
R(C_{2k+1},H) \le c\,m.
\]
The problem asks for the best possible $c_k$.

2) QUICK LITERATURE/CONTEXT CHECK

The provided problem statement does not list any specific results for #569 beyond asking for the best constant. (No external results are used as black boxes.)

3) ATTACK PLAN

- Lower bounds on $c_k$: find specific $m$-edge graphs $H$ for which $R(C_{2k+1},H)$ is large compared to $m$.

- Upper bounds on $c_k$: attempt to prove a universal linear (in $m$) upper bound, possibly by structural analysis of graphs with no red $C_{2k+1}$.

4) WORK

\emph{Fast reality check.}

For $m=1$, the only graph $H$ with one edge and no isolated vertices is $K_2$. In this case $R(C_{2k+1},K_2)=2k+1$ (Lemma~569.1), forcing $c_k\ge 2k+1$.

For $k=2$ (so $C_5$) and $H=2K_2$ (two disjoint edges, $m=2$), a brute-force search gives $R(C_5,2K_2)=6$ (see the computation recorded under VERIFICATION below). For $H=3K_2$ ($m=3$), the same search gives $R(C_5,3K_2)=8$.

\medskip

\textbf{Lemma 569.1 (exact value for $H=K_2$).}
For every $\ell\ge 3$,
\[
R(C_{\ell},K_2)=\ell.
\]
In particular, $R(C_{2k+1},K_2)=2k+1$, hence any admissible constant must satisfy $c_k\ge 2k+1$.

\textbf{Proof.}
Let $\ell\ge 3$.

- Lower bound: In $K_{\ell-1}$ colour every edge red. Then there is no blue $K_2$ (no blue edges at all). Also $K_{\ell-1}$ cannot contain a cycle $C_{\ell}$ because it has fewer than $\ell$ vertices. Hence $R(C_{\ell},K_2)>\ell-1$.

- Upper bound: Consider any red/blue colouring of $K_{\ell}$. If there is a blue edge, we have a blue $K_2$. If there is no blue edge, then all edges are red and the red graph is $K_{\ell}$, which contains a (not necessarily induced) copy of $C_{\ell}$ on all $\ell$ vertices.

Thus $R(C_{\ell},K_2)\le \ell$. Combined with the lower bound, $R(C_{\ell},K_2)=\ell$.

Now set $\ell=2k+1$ to get $R(C_{2k+1},K_2)=2k+1$, and since $m=e(K_2)=1$ we obtain $c_k\ge 2k+1$.
\qed

\medskip

\textbf{Lemma 569.2 (a crude universal upper bound).}
Let $H$ be a graph with $m=e(H)$ edges and no isolated vertices. Then
\[
R(C_{2k+1},H) \le \binom{2m+2k-1}{2k}.
\]

\textbf{Proof.}
Let $n=|V(H)|$. By Lemma~568.1 (proved above), $n\le 2m$.

Since $H\subseteq K_n$, monotonicity in the second argument gives
\[
R(C_{2k+1},H) \le R(C_{2k+1},K_n).
\]
Also $C_{2k+1}\subseteq K_{2k+1}$, so monotonicity in the first argument gives
\[
R(C_{2k+1},K_n) \le R(K_{2k+1},K_n)=:R(2k+1,n).
\]
It remains to bound the clique Ramsey number $R(2k+1,n)$.

\emph{Claim:} for all integers $a,b\ge 2$,
\[
R(a,b) \le \binom{a+b-2}{a-1}.
\]
\emph{Proof of claim:} The standard recursion for clique Ramsey numbers is
\[
R(a,b) \le R(a-1,b)+R(a,b-1)\quad(a,b\ge 3),
\]
with boundary conditions $R(2,b)=b$ and $R(a,2)=a$. (To justify the recursion: in a colouring of $K_{R(a-1,b)+R(a,b-1)}$, pick a vertex $v$; if $v$ has at least $R(a-1,b)$ red neighbours then among them there is either a red $K_{a-1}$, which together with $v$ gives a red $K_a$, or a blue $K_b$; similarly for blue degree.)

Assume inductively that $R(a-1,b)\le \binom{a+b-3}{a-2}$ and $R(a,b-1)\le \binom{a+b-3}{a-1}$. Then
\[
R(a,b)\le R(a-1,b)+R(a,b-1)
\le \binom{a+b-3}{a-2}+\binom{a+b-3}{a-1} = \binom{a+b-2}{a-1},
\]
using Pascal's identity. The boundary cases match the binomial coefficients. This proves the claim.

Applying the claim with $(a,b)=(2k+1,n)$ yields
\[
R(C_{2k+1},H) \le R(2k+1,n) \le \binom{2k+1+n-2}{2k} \le \binom{2k+1+2m-2}{2k} = \binom{2m+2k-1}{2k}.
\]
\qed

\medskip

\textbf{Partial information for $k=1$.}
When $k=1$, $C_{2k+1}=C_3$ is a triangle. The file for Problem~#570 states that Sidorenko proved $R(C_3,H)\le 2m+1$ for all $m$-edge graphs $H$ without isolated vertices. Combining this with Lemma~569.1 at $k=1$ shows $c_1=3$, since $2m+1\le 3m$ for all $m\ge 1$ and equality is forced by $m=1$.

5) VERIFICATION

\emph{Brute-force small-case checks (exact).}
A direct exhaustive search over red/blue colourings (equivalently, over red graphs) gives:
\[
R(C_5,2K_2)=6,\qquad R(C_5,3K_2)=8.
\]
(Computed locally by enumerating all colourings for $N\le 7$ and using backtracking for $N=8$.)

6) FINAL

**UNRESOLVED**

(i) Strongest proved partial result here: $c_k\ge 2k+1$ (forced already by $H=K_2$), and we have the explicit (very weak) universal upper bound
\[
R(C_{2k+1},H) \le \binom{2m+2k-1}{2k}.
\]
Also, for $k=1$ we can conclude $c_1=3$ using the Sidorenko bound stated in the file for #570.

(ii) First gap (crisp): prove any bound of the form $R(C_{2k+1},H)\le C(k)\,m$ with $C(k)$ 
polynomial (or even linear) in $k$, let alone determine the optimal constant $c_k$.

(iii) Top 3 next moves:

1. Identify the extremal $H$ (as a function of $k$) that maximises $R(C_{2k+1},H)/m$; the fast checks suggest small $m$ (notably $m=1$) may dominate.

2. Attempt to prove a linear-in-$m$ bound with constant $C(k)=O(k)$ by analysing colourings with no red $C_{2k+1}$ (e.g. extracting a large bipartite subgraph, then embedding $H$ in blue by degree/matching arguments).

3. Compute (exactly or with a SAT solver) $\max_H R(C_{2k+1},H)/m$ for small $k$ and $m$ to guess the true $c_k$ and identify candidate extremisers.

(iv) Minimal counterexample structure (if linearity fails for some $k$): a sequence $H_m$ with $e(H_m)=m$ and colourings of $K_{C m}$ with no red $C_{2k+1}$ and no blue $H_m$ for arbitrarily large $m$. Any such $H_m$ must evade forcing mechanisms coming from matchings, stars, or other basic sparse subgraphs.


