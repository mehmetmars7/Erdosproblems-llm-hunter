
\bigskip
\noindent\textbf{FORMAL RESTATEMENT.}
Let $A_1,A_2\subseteq\mathbb{N}$ be disjoint asymptotic additive bases of order $2$, meaning:
for each $i\in\{1,2\}$ there exists $n_i$ such that every integer $n\ge n_i$ lies in $A_i+A_i$.
Let $A:=A_1\cup A_2$.
Question: must $A$ contain a minimal asymptotic basis of order $2$ (a subset $B\subseteq A$ which is an asymptotic basis of order $2$ and minimal under deletion of any single element)?

\medskip
\noindent\textbf{QUICK LITERATURE/CONTEXT CHECK.}
The problem statement records that H\"{a}rtter and Nathanson constructed bases containing no minimal bases, and that Erd\H{o}s--Nathanson asked the above question for unions of two disjoint bases.
(Per the project integrity constraints, I do not use or assert further literature beyond what is in the problem text.)

\medskip
\noindent\textbf{ATTACK PLAN.}
\begin{itemize}
\item Establish basic properties of $A=A_1\cup A_2$, e.g. representation counts and non-minimality of $A$ itself.
\item Try to force a minimal subbasis via a Zorn/minimization argument, potentially using the fact that $A$ has at least two independent representations for all large $n$.
\item Disproof track: attempt to combine known ``no-minimal-subbase'' examples with a disjoint-base partition.
\end{itemize}
I only obtain basic lemmas; the existence of a minimal subbasis remains unresolved.

\medskip
\noindent\textbf{WORK.}

\medskip
\noindent\textbf{Lemma 869.1 (The union $A$ is never minimal).}
Let $A_1,A_2$ be disjoint asymptotic bases of order $2$ and $A=A_1\cup A_2$.
Then $A$ itself is not a minimal asymptotic basis of order $2$.
Indeed, for every $a\in A$, the set $A\setminus\{a\}$ is still an asymptotic basis of order $2$.

\smallskip
\noindent\emph{Proof.}
If $a\in A_1$, then $A\setminus\{a\}$ still contains $A_2$.
Since $A_2$ is an asymptotic basis of order $2$, there exists $n_2$ such that for all $n\ge n_2$ we have $n\in A_2+A_2\subseteq (A\setminus\{a\})+(A\setminus\{a\})$.
Hence $A\setminus\{a\}$ is an asymptotic basis.
The same argument applies if $a\in A_2$, using $A_1$ instead.
Therefore removing any single element preserves the basis property, so $A$ is not minimal.
\qed

\medskip
\noindent\textbf{Lemma 869.2 (Uniform lower bound on representation counts).}
Let $r_X(n):=|(1_X*1_X)(n)|$ be the ordered representation function.
If $A_1,A_2$ are disjoint asymptotic bases of order $2$ and $A=A_1\cup A_2$, then there exists $n_0$ such that for all $n\ge n_0$,
\[
r_A(n)\ge 2.
\]

\smallskip
\noindent\emph{Proof.}
Since $A_i$ is an asymptotic basis of order $2$, there exists $n_i$ such that for all $n\ge n_i$ there exists at least one ordered representation $(x_i,y_i)\in A_i\times A_i$ with $x_i+y_i=n$. In particular $r_{A_i}(n)\ge 1$ for all $n\ge n_i$.
For $n\ge n_0:=\max\{n_1,n_2\}$ we have
\[
r_A(n)\ge r_{A_1}(n)+r_{A_2}(n)\ge 1+1=2,
\]
because every representation using $A_1$ or $A_2$ is also a representation using $A$, and $A_1,A_2$ are disjoint so these sets of ordered pairs are disjoint.
\qed

\medskip
\noindent\textbf{FAST REALITY CHECK (sanity examples).}
Take $A_1=\mathbb{N}$ and $A_2=\varnothing$ (not allowed since $A_2$ not a basis), so disjointness plus both being bases forces both sets to be infinite and ``cover'' the integers in different ways.
Lemma 869.1 correctly predicts that the union of two bases is extremely non-minimal: deleting any single element cannot destroy being a basis because one full base remains.
Lemma 869.2 confirms at least two representations for all large $n$ (one from each color class).

\medskip
\noindent\textbf{VERIFICATION.}
\begin{itemize}
\item Lemma 869.1: uses only the definition of asymptotic basis and the containment $A_i\subseteq A\setminus\{a\}$ when $a\notin A_i$.
\item Lemma 869.2: the inequality $r_A\ge r_{A_1}+r_{A_2}$ is valid because representations within $A_1$ and within $A_2$ are disjoint ordered pairs.
\end{itemize}

\medskip
\noindent\textbf{FINAL.} \textbf{UNRESOLVED}

(i) \emph{Strongest proved partial result.}
The union $A=A_1\cup A_2$ is never minimal (Lemma 869.1), and for all sufficiently large $n$ it has at least two ordered representations $n=a+a'$ coming separately from $A_1$ and $A_2$ (Lemma 869.2).

(ii) \emph{First gap (crisp).}
Prove (or disprove) that $A$ must contain some subset $B\subseteq A$ that is a minimal asymptotic basis of order $2$.
Equivalently, show (or refute) that there exists an inclusion-minimal asymptotic basis inside $A$.

(iii) \emph{Top 3 next moves.}
\begin{itemize}
\item Attempt a Zorn argument on the family of asymptotic bases contained in $A$, and identify precisely what chain-intersection property is needed; try to derive it from Lemma 869.2 or stronger quantitative representation bounds.
\item Explore whether one can partition a known ``no-minimal-subbase'' example into two disjoint bases (or modify it so), which would yield a counterexample.
\item Run explicit constructions of disjoint bases (e.g. via greedy algorithms) and test whether their unions contain obvious minimal subbases (computationally in a finite truncation model).
\end{itemize}

(iv) \emph{Minimal counterexample structure.}
A counterexample would consist of disjoint bases $A_1,A_2$ whose union $A$ has the property that every asymptotic basis $B\subseteq A$ can be thinned further (i.e. is not minimal). Such an $A$ would necessarily have ``redundant'' representations spread across both $A_1$ and $A_2$ so that no element becomes essential infinitely often in any subbasis.

