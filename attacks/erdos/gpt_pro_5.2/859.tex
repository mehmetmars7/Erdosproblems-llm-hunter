
\noindent\textbf{FORMAL RESTATEMENT.}

Fix $t\in\mathbb N$ with $t\ge 1$. For each $n\in\mathbb N$, let $\mathcal D(n)$ be the set of positive divisors of $n$.
Say that $n$ is \emph{$t$-representable} if there exists a finite subset $S\subseteq \mathcal D(n)$ of \emph{distinct} divisors such that
\[
\sum_{d\in S} d = t.
\]
Let
\[
S_t := \{n\in\mathbb N : n\text{ is $t$-representable}\}.
\]
If the natural density
\[
 d_t := \lim_{X\to\infty} \frac{|S_t\cap\{1,2,\dots,X\}|}{X}
\]
exists, the problem asks whether there are constants $c_1,c_2>0$ such that
\[
 d_t \sim \frac{c_1}{(\log t)^{c_2}} \qquad (t\to\infty).
\]

\medskip
\noindent\textbf{QUICK LITERATURE/CONTEXT CHECK.}

The problem statement records that Erd\H{o}s proved $d_t$ exists and that there are positive constants $c_3,c_4$ with
\[
(\log t)^{-c_3} < d_t < (\log t)^{-c_4}.
\]
We do not assume any more literature.

\medskip
\noindent\textbf{ATTACK PLAN.}

We prove two unconditional structural facts:

(1) For each fixed $t$, the membership of $n$ in $S_t$ depends only on divisibility of $n$ by integers $\le t$, hence is periodic modulo $\mathrm{lcm}(1,2,\dots,t)$. This immediately implies $d_t$ exists (and is rational).

(2) Provide elementary lower bounds on $d_t$ (e.g. $d_t\ge 1/t$ via multiples of $t$).

We then compute $d_t$ exactly for small $t$ as a sanity check.

\medskip
\noindent\textbf{WORK.}

\medskip
\noindent\underline{Lemma 859.1 (Periodicity and existence of $d_t$).}

\textbf{Lemma.}
Fix $t\ge 1$ and let $L:=\mathrm{lcm}(1,2,\dots,t)$. Then $S_t$ is a union of residue classes modulo $L$. In particular, the natural density $d_t$ exists and equals
\[
 d_t = \frac{1}{L}\,\#\{0\le r<L : r\in S_t\}.
\]

\textbf{Proof.}
Let $n\in\mathbb N$.
If $n$ is $t$-representable, then there exists a subset $S\subseteq\mathcal D(n)$ with $\sum_{d\in S} d=t$. Every $d\in S$ satisfies $d\le t$ because all summands are positive.
Thus the condition ``$n$ is $t$-representable'' depends only on which integers $d\in\{1,2,\dots,t\}$ divide $n$.

For each $d\le t$, the predicate ``$d\mid n$'' depends only on the residue class of $n$ modulo $d$, hence only on $n\bmod L$ because $d\mid L$.
Therefore the set of residues $r\bmod L$ for which the collection of divisors $\{d\le t: d\mid r\}$ admits a subset summing to $t$ is well-defined; call this set $R\subseteq\{0,1,\dots,L-1\}$.
Then $n\in S_t$ if and only if $n\bmod L\in R$, i.e. $S_t$ is a union of residue classes mod $L$.

Any union of residue classes modulo $L$ has natural density $|R|/L$ by periodicity: among $1,2,\dots,X$ each residue appears $\lfloor X/L\rfloor$ or $\lceil X/L\rceil$ times, and the proportion tends to $|R|/L$ as $X\to\infty$.
Thus $d_t$ exists and equals $|R|/L$.
\qed

\medskip
\noindent\underline{Lemma 859.2 (A trivial universal lower bound).}

\textbf{Lemma.}
For every $t\ge 1$, we have $d_t\ge 1/t$.

\textbf{Proof.}
If $n$ is a multiple of $t$, then $t\in\mathcal D(n)$, and we can represent $t$ as the sum of the single divisor $t$. Hence every multiple of $t$ lies in $S_t$.
The set of multiples of $t$ has natural density $1/t$.
Therefore $d_t\ge 1/t$.
\qed

\medskip
\noindent\underline{Lemma 859.3 (A union-bound upper estimate in terms of distinct partitions).}

\textbf{Lemma.}
Let $\mathcal P_t$ be the set of all finite subsets $S\subseteq\{1,2,\dots,t\}$ with $\sum_{d\in S} d=t$, and for $S\in\mathcal P_t$ let $m(S):=\mathrm{lcm}(S)$.
Then
\[
 d_t \le \sum_{S\in\mathcal P_t} \frac{1}{m(S)}.
\]

\textbf{Proof.}
If $n\in S_t$, then there exists $S\in\mathcal P_t$ such that every $d\in S$ divides $n$, hence $m(S)=\mathrm{lcm}(S)$ divides $n$.
Therefore
\[
S_t \subseteq \bigcup_{S\in\mathcal P_t} \{n: m(S)\mid n\}.
\]
Taking natural densities and using the union bound for densities of sets,
\[
 d_t \le \sum_{S\in\mathcal P_t} \mathrm{dens}(\{n: m(S)\mid n\}) = \sum_{S\in\mathcal P_t} \frac{1}{m(S)}.
\]
\qed

\medskip
\noindent\underline{Fast reality check (exact values for small $t$).}

Using Lemma 859.1 (period $L=\mathrm{lcm}(1,\dots,t)$) and a direct computation for $t\le 10$, we obtain the exact values:
\[
\begin{array}{c|c|c}
 t & d_t & \text{decimal}\\\hline
 1 & 1 & 1.0000\\
 2 & 1/2 & 0.5000\\
 3 & 2/3 & 0.6667\\
 4 & 1/2 & 0.5000\\
 5 & 7/15 & 0.4667\\
 6 & 7/15 & 0.4667\\
 7 & 16/35 & 0.4571\\
 8 & 3/7 & 0.4286\\
 9 & 119/315 & 0.3778\\
 10 & 2/5 & 0.4000
\end{array}
\]

\medskip
\noindent\textbf{VERIFICATION.}

\begin{itemize}
\item Lemma 859.1: the key point is that any divisor used in a sum to $t$ is automatically $\le t$, ensuring finiteness and periodicity modulo $\mathrm{lcm}(1,\dots,t)$.
\item Lemma 859.2 is valid because using the single divisor $t$ is allowed (``distinct divisors'').
\item The computed small values agree with obvious checks: $d_1=1$ and $d_2=1/2$ (only even $n$ have divisor $2$).
\end{itemize}

\medskip
\noindent\textbf{FINAL.} \textbf{UNRESOLVED.}

(i) \emph{Strongest proved partial results.}
We proved that for every fixed $t$, the set $S_t$ is periodic modulo $\mathrm{lcm}(1,\dots,t)$, hence $d_t$ exists and is rational.
We also proved the universal lower bound $d_t\ge 1/t$ and an explicit union-bound upper estimate in Lemma 859.3.
Exact values for $t\le 10$ were computed.

(ii) \emph{First gap.}
Determine the true asymptotic behaviour of $d_t$ as $t\to\infty$; in particular, prove or disprove the existence of $c_1,c_2>0$ with $d_t\sim c_1/(\log t)^{c_2}$.

(iii) \emph{Top 3 next moves.}
(1) Analyse which subsets $S\subseteq\{1,\dots,t\}$ with sum $t$ give the smallest $\mathrm{lcm}(S)$; these dominate the union-bound in Lemma 859.3.
(2) Develop a lower-bound construction beyond $1/t$ by forcing a ``binary'' divisor structure (e.g. many powers of two) with reasonably high density.
(3) Compute $d_t$ for larger $t$ (say $t=20,30,40$) using periodicity but avoiding enumerating all residues by instead enumerating the moduli $m(S)$ and using inclusion--exclusion on their lcms.

(iv) \emph{What a minimal counterexample would look like.}
If $d_t$ fails to be a power of $\log t$, then either (a) $d_t$ is much larger, requiring unexpectedly many small-lcm partitions of $t$ into distinct parts, or (b) $d_t$ is much smaller, meaning that every representation of $t$ as a sum of distinct integers forces an lcm so large that the union of multiples has unusually small density.


