% Erdos Problem #936

1) FORMAL RESTATEMENT

A positive integer \(m\) is \emph{powerful} (squarefull) if for every prime \(p\mid m\) we have \(p^2\mid m\).
The questions are:

(a) Are there only finitely many integers \(n\ge 1\) such that \(2^n-1\) is powerful?
(b) Are there only finitely many integers \(n\ge 1\) such that \(2^n+1\) is powerful?
(c) Are there only finitely many integers \(n\ge 1\) such that \(n!-1\) is powerful?
(d) Are there only finitely many integers \(n\ge 1\) such that \(n!+1\) is powerful?

Edge cases: \(2^1-1=1\) is usually excluded from the definition (it has no prime divisors), but even if included it is vacuously powerful. I will treat powerfulness as meaningful for \(m\ge 2\).

2) QUICK LITERATURE/CONTEXT CHECK

The problem statement records conditional results:
- Assuming the abc conjecture, CrowdMath shows (a) and (b) have positive answers (only finitely many \(n\)).
- Assuming abc, Cushing--Pascoe show (d) (and more generally \(|x-n!|\le k\) with powerful \(x\)) has only finitely many solutions.
I do not use any other external results.

3) ATTACK PLAN

Proof track (partial): Find a small prime (e.g. \(3\)) that divides \(2^n\pm 1\) for infinitely many \(n\) but typically only to the first power, excluding powerfulness for a large infinite set of \(n\).

Disproof track: Attempt to find infinite families of \(n\) such that \(2^n\pm 1\) or \(n!\pm 1\) are perfect squares or have all prime exponents \(\ge2\). Computations can test small \(n\).

4) WORK

Fast reality check (explicit computation).
- For \(2^n\pm 1\) with \(1\le n\le 40\), the only powerful value found is \(2^3+1=9=3^2\). No other \(2^n\pm 1\) in this range is powerful.
- For \(n!\pm 1\) with \(1\le n\le 30\), the only powerful values found are
\(4!+1=25=5^2\), \(5!+1=121=11^2\), \(7!+1=5041=71^2\). No \(n!-1\) in this range is powerful.

Lemma 936.1 (Exact 3-adic valuation for \(2^{2m}-1\)).
For every integer \(m\ge 1\),
\[v_3(4^{m}-1)=1+v_3(m).\]
Equivalently, for every even \(n\ge 2\),
\[v_3(2^{n}-1)=1+v_3(n).\]

Proof.
Write \(m=3^t u\) where \(t=v_3(m)\ge 0\) and \(3\nmid u\).
Set \(x:=4^{3^t}\). Then \(x\equiv 4^{0}\equiv 1\pmod 3\), so \(3\mid (x-1)\).
We first compute \(v_3(x-1)\).

Claim 1: If \(y\equiv 1\pmod 3\), then \(v_3(y^3-1)=v_3(y-1)+1\).
Indeed, factor \(y^3-1=(y-1)(y^2+y+1)\). Write \(y=1+3k\). Then
\[y^2+y+1=(1+3k)^2+(1+3k)+1 = 3 + 9k + 9k^2 = 3(1+3k+3k^2).
\]
The factor \(1+3k+3k^2\equiv 1\pmod 3\) is not divisible by \(3\), so \(v_3(y^2+y+1)=1\). Hence
\(v_3(y^3-1)=v_3(y-1)+1\), proving Claim 1.

Apply Claim 1 iteratively with \(y=4,4^3,4^{3^2},\dots,4^{3^t}\). Since \(4\equiv 1\pmod 3\) and \(v_3(4-1)=v_3(3)=1\), each cubing step increases the 3-adic valuation by exactly 1. Therefore
\[v_3(4^{3^t}-1)=1+t.\]
That is, \(v_3(x-1)=1+t\).

Now consider
\[4^{m}-1 = 4^{3^t u}-1 = x^{u}-1 = (x-1)(x^{u-1}+x^{u-2}+\cdots+1).
\]
Since \(x\equiv 1\pmod 3\), every term in the parenthesis is \(\equiv 1\pmod 3\), so
\(x^{u-1}+\cdots+1 \equiv u \pmod 3\). Because \(3\nmid u\), this sum is \emph{not} divisible by 3, hence has 3-adic valuation 0. Therefore
\[v_3(4^{m}-1)=v_3(x-1)=1+t=1+v_3(m),
\]
as desired. Finally, if \(n=2m\) then \(2^n-1=4^m-1\) and \(v_3(n)=v_3(m)\), giving the equivalent statement for even \(n\). QED.

Lemma 936.2 (Exact 3-adic valuation for \(2^{n}+1\) when \(n\) is odd).
For every odd integer \(n\ge 1\),
\[v_3(2^{n}+1)=1+v_3(n).\]

Proof.
For odd \(n\), we have \((-2)^n-1 = -(2^n+1)\), so \(v_3(2^n+1)=v_3((-2)^n-1)\).
Repeat the proof of Lemma 936.1 with \(4\) replaced by \(-2\): note \(-2\equiv 1\pmod 3\) and \(v_3((-2)-1)=v_3(-3)=1\). The same Claim 1 applies whenever \(y\equiv 1\pmod 3\), hence yields
\(v_3((-2)^{3^t}-1)=1+t\), and the geometric series factor \(y^{u-1}+\cdots+1\equiv u\pmod 3\) is not divisible by 3 when \(3\nmid u\). Writing \(n=3^t u\) with \(3\nmid u\) gives \(v_3((-2)^n-1)=1+t=1+v_3(n)\), hence \(v_3(2^n+1)=1+v_3(n)\). QED.

Corollary 936.3 (Infinite families of \(n\) excluded from powerfulness).
- If \(n\) is even and \(3\nmid n\), then \(3\mid (2^n-1)\) but \(9\nmid (2^n-1)\), hence \(2^n-1\) is not powerful.
- If \(n\) is odd and \(3\nmid n\), then \(3\mid (2^n+1)\) but \(9\nmid (2^n+1)\), hence \(2^n+1\) is not powerful.

Proof.
If \(3\nmid n\) then \(v_3(n)=0\). Lemma 936.1 (resp. 936.2) gives \(v_3(2^n-1)=1\) for even \(n\) (resp. \(v_3(2^n+1)=1\) for odd \(n\)). Thus 3 divides but 9 does not, so the number has a prime divisor (3) appearing to exponent 1, contradicting powerfulness. QED.

Lemma 936.4 (Prime factors of \(n!\pm 1\) are large).
For every integer \(n\ge 2\) and every prime \(p\le n\), we have
\[p\nmid (n!+1)\quad\text{and}\quad p\nmid (n!-1).\]
Equivalently, every prime divisor of \(n!\pm 1\) is \(>n\).

Proof.
Fix \(n\ge 2\) and a prime \(p\le n\). Then \(p\mid n!\), hence \(n!\equiv 0\pmod p\). Therefore \(n!+1\equiv 1\pmod p\) and \(n!-1\equiv -1\pmod p\), so neither is divisible by \(p\). QED.

5) VERIFICATION

- Lemmas 936.1 and 936.2: the key step is verifying \(v_3(y^2+y+1)=1\) when \(y\equiv 1\pmod 3\). The computation \(y=1+3k\) shows a factor of 3 and the remaining factor \(\equiv 1\pmod 3\), so no higher power of 3 divides.
- Corollary 936.3: checks the prime divisor 3 appears to exponent exactly 1.
- Lemma 936.4: uses only \(p\mid n!\Rightarrow n!\equiv 0\pmod p\), correct.

6) FINAL

UNRESOLVED
(i) Strongest proved partial result: For all even \(n\) with \(3\nmid n\), \(2^n-1\) is not powerful; for all odd \(n\) with \(3\nmid n\), \(2^n+1\) is not powerful (Corollary 936.3). Also every prime divisor of \(n!\pm 1\) exceeds \(n\) (Lemma 936.4).
(ii) First gap: Show that among the remaining indices (e.g. \(n\equiv 0\pmod 3\) for \(2^n\pm 1\)), there are only finitely many \(n\) for which all prime divisors occur with exponent \(\ge 2\).
(iii) Top 3 next moves:
  1. For \(2^n\pm 1\), analyse another small prime (e.g. 5,7,13) via explicit lifting arguments to force a prime divisor with valuation exactly 1 for infinitely many congruence classes of \(n\).
  2. For \(n!+1\), leverage Lemma 936.4 to study square divisibility by primes in \((n,2n]\) (if any divide), aiming to force a prime that divides to the first power.
  3. Extend computation (with specialised methods rather than full factorisation) to search for additional powerful instances beyond the trivial small square cases.
(iv) Minimal counterexample structure: An infinite family of powerful values would require infinitely many \(n\) such that \(2^n\pm 1\) (or \(n!\pm 1\)) avoids having any prime divisor with valuation 1; for \(2^n\pm 1\) this would in particular require all primes like 3 dividing the number to appear with exponent \(\ge2\), i.e. demanding strong congruence conditions such as \(n\) divisible by higher powers of 3 and similarly for other primes.


