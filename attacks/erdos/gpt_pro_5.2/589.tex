
\noindent\textbf{FORMAL RESTATEMENT.}

For $n\ge 1$, define $g(n)$ (called $g$ in the problem statement) to be the largest integer with the property:

For every set $P\subset\mathbb{R}^2$ of $n$ points with no four collinear, there exists a subset $Q\subseteq P$ with $|Q|\ge g(n)$ such that no three points of $Q$ are collinear.

Equivalently, $g(n)$ is the minimum, over all $n$-point sets $P$ with no four collinear, of the maximum size of a subset in general position (no three collinear).

Question: estimate the growth rate of $g(n)$.

\bigskip
\noindent\textbf{QUICK LITERATURE/CONTEXT CHECK.}

I will not use any literature beyond what is explicitly quoted in the problem statement (greedy $\gg n^{1/2}$, upper bound $o(n)$ via density Hales--Jewett, and improvements by F\"uredi and Balogh--Solymosi).

\bigskip
\noindent\textbf{ATTACK PLAN.}

\emph{Proof-track ideas.}
(1) Model collinear triples as a 3-uniform hypergraph and bound the independence number from below using probabilistic or greedy methods; ``no four collinear'' should translate to a linearity property of the hypergraph.
(2) Use combinatorial geometry incidence bounds to control the number of collinear triples and thus improve the hypergraph bound.

\emph{Disproof/upper-bound ideas.}
(1) Seek constructions with many collinear triples arranged so that every large subset contains a triple (dense 3-uniform hypergraphs realizable by point-line incidences).

Here I only re-derive the classical $\Omega(\sqrt n)$-type lower bound in a self-contained way.

\bigskip
\noindent\textbf{WORK.}

\noindent\textbf{Lemma 589.1 (hypergraph formulation; linearity from ``no four collinear'').}
Let $P$ be a set of $n$ points in $\mathbb{R}^2$ with no four collinear.
Define a 3-uniform hypergraph $\mathcal{H}(P)$ on vertex set $P$ whose hyperedges are the collinear triples of points:
\[
E(\mathcal{H}(P)):=\{\{a,b,c\}\subseteq P: a,b,c \text{ are distinct and collinear}\}.
\]
Then:

(a) A subset $Q\subseteq P$ has no three collinear if and only if $Q$ is an independent set in $\mathcal{H}(P)$ (i.e. $Q$ contains no hyperedge).

(b) $\mathcal{H}(P)$ is \emph{linear}: any two distinct hyperedges intersect in at most one vertex.
Equivalently, any unordered pair of vertices belongs to at most one hyperedge.

\noindent\emph{Proof.}
(a) By definition, a hyperedge is exactly a collinear triple in $P$.
Thus $Q$ contains a collinear triple if and only if $Q$ contains a hyperedge, i.e. is not independent.

(b) Suppose two distinct hyperedges shared two vertices $a,b$.
Then there exist distinct points $c\ne d$ such that $\{a,b,c\}$ and $\{a,b,d\}$ are hyperedges.
That means $a,b,c$ are collinear and $a,b,d$ are collinear.
But the line through $a$ and $b$ is unique, so $c$ and $d$ are also on that same line, giving four collinear points $a,b,c,d$, contradicting the hypothesis.
Therefore any two hyperedges intersect in at most one vertex.
\qed

\bigskip
\noindent\textbf{Lemma 589.2 (probabilistic lower bound on independent sets in linear 3-hypergraphs).}
Let $\mathcal{H}$ be any 3-uniform hypergraph on $n$ vertices with $m$ hyperedges.
Then for any $p\in[0,1]$ there exists an independent set of size at least
\[
 pn - p^3 m.
\]
In particular, if $\mathcal{H}$ is linear then $m\le \binom{n}{2}/3$, and choosing $p=n^{-1/2}$ yields an independent set of size at least
\[
 n^{1/2}-\frac{1}{3}n^{1/2}=\frac{2}{3}\,n^{1/2}.
\]
Consequently, for every $P\subset\mathbb{R}^2$ with $|P|=n$ and no four collinear, there exists $Q\subseteq P$ with no three collinear and
\[
|Q|\ge \frac{2}{3}\sqrt{n}.
\]

\noindent\emph{Proof.}
Select a random subset $S$ of the vertex set by including each vertex independently with probability $p$.
Then $\mathbb{E}|S|=pn$.
Let $X$ be the number of hyperedges fully contained in $S$.
Each hyperedge is contained in $S$ with probability $p^3$, hence $\mathbb{E}X=p^3 m$.

From $S$, construct $S'$ by deleting, for each hyperedge contained in $S$, one arbitrary vertex of that hyperedge (if multiple hyperedges choose the same vertex to delete, delete it once).
Then $S'$ is independent (because every hyperedge in $S$ had at least one vertex removed), and
\[
|S'|\ge |S|-X.
\]
Taking expectations,
\[
\mathbb{E}|S'|\ge \mathbb{E}|S|-\mathbb{E}X = pn - p^3 m.
\]
Therefore there exists at least one outcome of the random choice for which $|S'|\ge pn-p^3 m$, proving the first statement.

Now assume $\mathcal{H}$ is linear.
Each hyperedge contains exactly $3$ unordered vertex-pairs, and linearity implies that no unordered pair lies in more than one hyperedge.
Hence $3m\le \binom{n}{2}$, i.e. $m\le \binom{n}{2}/3$.
Choosing $p=n^{-1/2}$ gives
\[
 pn-p^3 m \ge \sqrt n - n^{-3/2}\cdot \frac{n(n-1)}{6} = \sqrt n - \frac{n-1}{6\sqrt n} \ge \frac{2}{3}\sqrt n
\]
for all $n\ge 1$ (one may check the inequality directly).

Finally, apply Lemma 589.1: for a point set $P$ with no four collinear, the hypergraph $\mathcal{H}(P)$ is linear, and an independent set is exactly a subset with no three collinear.
Thus $g(n)\ge (2/3)\sqrt n$.
\qed

\bigskip
\noindent\textbf{Fast reality check (plugging small $n$).}

The bound from Lemma 589.2 gives $g(9)\ge 2$, $g(16)\ge 2$, $g(25)\ge 3$, $g(36)\ge 4$.
These are very weak but consistent with the trivial constraints.

\bigskip
\noindent\textbf{VERIFICATION.}

\emph{Lemma 589.1.} The key point is uniqueness of the line through two distinct points.
Linearity of the hypergraph is exactly where ``no four collinear'' is used.

\emph{Lemma 589.2.} The deletion argument is valid because deleting at least one vertex from each hyperedge in $S$ guarantees no hyperedge remains fully contained.
The expectation bound is one-sided but sufficient for existence.
The linearity bound $3m\le \binom{n}{2}$ is rigorous.

\bigskip
\noindent\textbf{FINAL.}

**UNRESOLVED**

(i) Strongest proved partial result: $g(n)\ge \frac{2}{3}\sqrt n$ for all $n$, via the linear 3-uniform hypergraph formulation and a probabilistic independent-set bound (Lemmas 589.1--589.2). This recovers (up to constants) the classical greedy $\gg \sqrt n$ lower bound mentioned in the problem statement.

(ii) First gap (crisp): improve the lower bound beyond $\Theta(\sqrt n)$ (e.g. to $\Omega(\sqrt n\log n)$ as in the problem statement) using only fully justified steps, or alternately give a new explicit construction showing an upper bound $o(n)$ without importing density Hales--Jewett.

(iii) Top 3 next moves:
1. Bound the number of collinear triples $m$ in $\mathcal{H}(P)$ more sharply using planar incidence bounds under ``no four collinear'', then optimize $p$ in Lemma 589.2 to improve $|Q|$.
2. Replace the simple ``delete one per hyperedge'' argument by a refined nibble/alteration to reduce over-deletion when hyperedges overlap.
3. Computation: for small $n$ search for point sets (e.g. integer grids with perturbations) maximizing the number of 3-point lines and measure the largest general-position subset size to guess extremal behavior.

(iv) Minimal counterexample structure (to any conjectured linear lower bound): would be a point set with no four collinear but with so many 3-point lines that every subset of size $\varepsilon n$ necessarily contains a collinear triple. Combinatorially, this corresponds to a linear 3-uniform hypergraph on $n$ vertices with very small independence number, and geometrically it would require an incidence structure approaching a dense linear 3-design realizable in $\mathbb{R}^2$.

