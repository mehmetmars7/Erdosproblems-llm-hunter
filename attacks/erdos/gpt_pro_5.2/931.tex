
1) FORMAL RESTATEMENT

The text as given is ambiguous because $n_1$ appears but is not quantified, while the question asks about ``finitely many $n_2\ge n_1+k_1$''.

\medskip
\noindent\textbf{Corrected (standard) formulation.}
Fix integers $k_1\ge k_2\ge 3$.
For integers $n_1,n_2$ with $n_2\ge n_1+k_1$, define
\[
P_1(n_1):=\prod_{i=1}^{k_1}(n_1+i),\qquad P_2(n_2):=\prod_{j=1}^{k_2}(n_2+j).
\]
Write $\operatorname{rad}(N)$ for the set of primes dividing $N$ (equivalently, the squarefree kernel).
The question is:
\[
\text{Are there only finitely many pairs }(n_1,n_2)\text{ with }n_2\ge n_1+k_1\text{ such that }\operatorname{rad}(P_1(n_1))=\operatorname{rad}(P_2(n_2))?
\]
(One may also consider variants with $n_1\ge 1$; the source file notes that allowing $n_1=0$ creates additional solutions.)

2) QUICK LITERATURE/CONTEXT CHECK

From the provided problem statement: Tijdeman exhibited
\(19,20,21,22\) and \(54,55,56,57\) as an example with the same prime factors. The source also records an ``AlphaProof'' example
\(10!=1\cdot2\cdots 10\) and \(14\cdot15\cdot16\), i.e. $(n_1,k_1,n_2,k_2)=(0,10,13,3)$. Erd\H{o}s suggested that perhaps any such example must satisfy $n_2>2(n_1+k_1)$, but the AlphaProof example violates this inequality.

3) ATTACK PLAN

\begin{itemize}
\item \textbf{Clarify necessary structural constraints:} a prime appearing as a single term in one block must appear somewhere in the other block, forcing both blocks to be highly composite.
\item \textbf{Try to bound $n_2$ from below relative to $n_1$:} if the second block contained a large prime, it could not be matched by the first block when $n_2\ge n_1+k_1$.
\item \textbf{Computational scan:} search small $n_1,n_2$ for fixed $(k_1,k_2)$ to see how frequently solutions occur and what patterns they follow.
\end{itemize}

4) WORK

\textbf{Lemma 931.1 (the later block contains no primes).}
Assume $k_1\ge k_2\ge 3$ and $n_2\ge n_1+k_1$.
If $\operatorname{rad}(P_1(n_1))=\operatorname{rad}(P_2(n_2))$, then every integer in the block $\{n_2+1,n_2+2,\dots,n_2+k_2\}$ is composite.

\emph{Proof.}
Suppose, for contradiction, that $n_2+j$ is prime for some $1\le j\le k_2$.
Then $n_2+j$ divides $P_2(n_2)$, hence (by equality of prime sets) it divides $P_1(n_1)$.
But every factor of $P_1(n_1)$ is at most $n_1+k_1\le n_2$ (since $n_2\ge n_1+k_1$), so $P_1(n_1)$ is not divisible by any prime $>n_2$.
Since $n_2+j>n_2$, this contradicts divisibility. Therefore no $n_2+j$ can be prime; i.e. all of them are composite.
\hfill$\square$

\medskip
\textbf{Proposition 931.2 (verification of Tijdeman's example).}
Let $k_1=k_2=4$, $n_1=18$ and $n_2=53$. Then
\[
\operatorname{rad}\bigl(19\cdot20\cdot21\cdot22\bigr)=\operatorname{rad}\bigl(54\cdot55\cdot56\cdot57\bigr).
\]

\emph{Proof.}
Factor each block:
\[
19\cdot20\cdot21\cdot22 = 19\cdot(2^2\cdot5)\cdot(3\cdot7)\cdot(2\cdot11),
\]
so the set of prime divisors is $\{2,3,5,7,11,19\}$.
Similarly,
\[
54\cdot55\cdot56\cdot57 = (2\cdot3^3)\cdot(5\cdot11)\cdot(2^3\cdot7)\cdot(3\cdot19),
\]
so the set of prime divisors is again $\{2,3,5,7,11,19\}$.
Thus the prime-factor sets are equal.
\hfill$\square$

\medskip
\textbf{Proposition 931.3 (verification of the AlphaProof example, and Erd\H{o}s's inequality fails).}
Let $(n_1,k_1,n_2,k_2)=(0,10,13,3)$. Then
\[
\operatorname{rad}(1\cdot2\cdot\dots\cdot 10)=\operatorname{rad}(14\cdot15\cdot16)=\{2,3,5,7\}.
\]
Moreover $n_2=13\not>2(n_1+k_1)=20$.

\emph{Proof.}
We have
\[
1\cdot2\cdot\dots\cdot 10 = 10! = 2^8\cdot 3^4\cdot 5^2\cdot 7,
\]
so its prime divisors are exactly $\{2,3,5,7\}$.
Also
\[
14\cdot15\cdot16 = (2\cdot7)\cdot(3\cdot5)\cdot(2^4)=2^5\cdot3\cdot5\cdot7,
\]
whose prime divisors are again $\{2,3,5,7\}$.
Finally, $n_2=13$ while $2(n_1+k_1)=2\cdot10=20$, so $n_2>2(n_1+k_1)$ fails.
\hfill$\square$

\medskip
\textbf{FAST REALITY CHECK (local brute force for small parameters).}
Using a simple trial-division search over $0\le n_1\le 100$ and $0\le n_2\le 200$, I found many pairs $(n_1,n_2)$ with $\operatorname{rad}(P_1(n_1))=\operatorname{rad}(P_2(n_2))$ for small $(k_1,k_2)$.
For example, with $(k_1,k_2)=(3,3)$, both $(n_1,n_2)=(4,13)$ and $(4,47)$ appear (corresponding to blocks $5,6,7$ and $14,15,16$ and also $48,49,50$), all having prime set $\{2,3,5,7\}$.
This is only a sanity check and does not address finiteness.

5) VERIFICATION

\begin{itemize}
\item Lemma 931.1: the key inequality is $n_2+j>n_2\ge n_1+k_1$, hence strictly larger than every factor of $P_1(n_1)$.
\item Propositions 931.2 and 931.3: prime-factor sets were checked by explicit factorization.
\end{itemize}

6) FINAL

\textbf{UNRESOLVED}

(i) \textbf{Strongest proved partial result here.}
Any solution with $n_2\ge n_1+k_1$ forces the entire later block $\{n_2+1,\dots,n_2+k_2\}$ to consist of composite numbers (Lemma 931.1). Also, there exist nontrivial examples (Propositions 931.2 and 931.3).

(ii) \textbf{First gap (crisp).}
Prove or disprove: for fixed $k_1\ge k_2\ge 3$, there are only finitely many integer pairs $(n_1,n_2)$ with $n_2\ge n_1+k_1$ and $\operatorname{rad}(P_1(n_1))=\operatorname{rad}(P_2(n_2))$.

(iii) \textbf{Top 3 next moves.}
\begin{enumerate}
\item Classify all solutions for small $(k_1,k_2)$ computationally to detect families (if any) and to test whether the set of solutions looks finite.
\item Use Lemma 931.1 plus explicit lower bounds on the size of the largest prime factor of a product of $k_2$ consecutive integers to constrain $n_2$ (this would require an explicit theorem about primes in short intervals / large prime factors).
\item Attempt a ``largest prime'' argument: show that for large $n_2$, some $n_2+j$ has a prime factor $>n_1+k_1$, contradicting equality of prime sets.
\end{enumerate}

(iv) \textbf{What a minimal counterexample would look like.}
An infinite-family counterexample would require arbitrarily large $n_2$ such that the entire block $n_2+1,\dots,n_2+k_2$ is composite and all its prime factors lie among those dividing the earlier block $n_1+1,\dots,n_1+k_1$. In particular, the later block would need to be ``smooth'' with respect to the maximal element $n_1+k_1$.


