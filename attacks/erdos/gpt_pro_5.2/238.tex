\section*{Erd\H{o}s Problem \#238}

\begin{enumerate}[label=\arabic*)]

\item \textbf{FORMAL RESTATEMENT.}

Fix constants $c_1,c_2>0$. For $x\ge 2$, let
\[2=p_1<p_2<p_3<\cdots<p_{\pi(x)}\le x\]
be the primes up to $x$.

A ``block of $L$ consecutive primes $\le x$'' means an index $i$ such that
\[p_i, p_{i+1},\dots,p_{i+L-1}\le x\]
(and these primes are consecutive in the global sequence of primes).

The condition ``the difference between any two is $>c_2$'' for a block of consecutive primes is equivalent to requiring that every adjacent gap inside the block satisfies
\[p_{i+j+1}-p_{i+j}>c_2\qquad (0\le j\le L-2),\]
because all other pairwise differences are sums of adjacent gaps and are therefore even larger.

\textbf{Claim/conjecture.} For every $c_1,c_2>0$, for all sufficiently large $x$ there exists a block of length
\[L > c_1\log x\]
consecutive primes $\le x$ such that all adjacent gaps within the block exceed $c_2$.

Equivalently, if $M(x)$ denotes the maximum length of a consecutive-prime block $\le x$ with all internal gaps $>c_2$, the conjecture asserts that
\[\forall c_1>0\ \exists x_0:\ \forall x\ge x_0,\quad M(x)>c_1\log x,\]
i.e. $M(x)/\log x\to\infty$.

\item \textbf{QUICK LITERATURE/CONTEXT CHECK.}

A standard sieve bound (Brun/Selberg) gives for each fixed integer $d\ge 1$ an upper bound of the shape
\[\#\{p\le x: p\ \text{and}\ p+d\ \text{prime}\}\ \ll_d\ \frac{x}{(\log x)^2}.
\]
From this one can deduce that the number of prime gaps $\le c_2$ up to $x$ is $\ll x/(\log x)^2$, which already implies the conjectured phenomenon for sufficiently small $c_1$ (depending on $c_2$): there must be at least one long block with no such small gaps. This matches the partial result attributed to Erd\H{o}s on the problem page.

The full conjecture asks for blocks longer than \emph{any} constant multiple of $\log x$, i.e. a super-logarithmic block length; this is currently out of reach in this write-up.

\item \textbf{ATTACK PLAN.}

\emph{Proof strategies.}
\begin{itemize}
  \item[(P1)] Bound the number $B(x)$ of indices $i<\pi(x)$ with gap $p_{i+1}-p_i\le c_2$ by sieve methods; then use a pigeonhole/block decomposition argument to force a long run of gaps $>c_2$.
  \item[(P2)] Strengthen (P1) by proving that such ``small gaps'' cannot be too evenly distributed in $i$, forcing a block length $\omega(\log x)$. (No known unconditional method in this write-up accomplishes this strengthening.)
\end{itemize}

\emph{Disproof/construction strategies.}
\begin{itemize}
  \item[(D1)] A counterexample would require exhibiting $c_1,c_2$ and arbitrarily large $x$ for which \emph{every} block of $>c_1\log x$ consecutive primes $\le x$ contains a gap $\le c_2$.
  Establishing this would in particular require strong lower bounds and distributional control on the occurrence of small prime gaps, beyond what is known here.
\end{itemize}

\smallskip
\textbf{Best path chosen here.} I carry out (P1) completely, yielding the known ``$c_1$ sufficiently small'' version with a fully rigorous proof assuming a standard Brun/Selberg sieve upper bound for prime pairs.

\item \textbf{WORK.}

\paragraph{Lemma 238.1 (Sieve upper bound for prime pairs).}
Fix an integer $d\ge 1$. Let
\[\pi_2(x;d):=\#\{p\le x: p\ \text{and}\ p+d\ \text{are prime}\}.\]
Then there exists a constant $C(d)>0$ and an $x_0(d)$ such that for all $x\ge x_0(d)$,
\[\pi_2(x;d)\le C(d)\,\frac{x}{(\log x)^2}.\]

\begin{proof}[Justification]
This is a standard consequence of Brun's sieve / the Selberg upper-bound sieve applied to the admissible 2-tuple $(0,d)$. The constant $C(d)$ is ineffective in this write-up but depends only on $d$.
\end{proof}

\paragraph{Theorem 238.2 (Erd\H{o}s' ``small $c_1$'' version).}
Fix $c_2>0$. Then there exists a constant $c_1^\ast=c_1^\ast(c_2)>0$ such that for all sufficiently large $x$ there are more than $c_1^\ast\log x$ consecutive primes $\le x$ whose pairwise differences are all $>c_2$ (equivalently, all adjacent gaps inside the block are $>c_2$).

\begin{proof}
Let $p_1<p_2<\dots<p_m\le x$ be the primes $\le x$, where $m=\pi(x)$. For $1\le i\le m-1$ call the index $i$ \emph{bad} if the consecutive gap satisfies
\[p_{i+1}-p_i\le c_2.
\]
Let $B(x)$ be the number of bad indices.

\smallskip
\emph{Step 1: block decomposition.}
Removing the $B(x)$ bad ``edges'' partitions the ordered list $p_1,\dots,p_m$ into at most $B(x)+1$ consecutive blocks, each block having all internal adjacent gaps $>c_2$.
Let $L(x)$ be the maximum block length (number of primes in the largest such block). Then
\[m\le (B(x)+1)\,L(x),\]
so
\[L(x)\ge \frac{m}{B(x)+1}.
\]

\smallskip
\emph{Step 2: bound the number of bad indices.}
For each integer $d\in\{1,2,\dots,\lfloor c_2\rfloor\}$ let $N_d(x)$ be the number of indices $i$ with $p_{i+1}-p_i=d$.
Then
\[B(x)=\sum_{d\le c_2} N_d(x).
\]
For each $d$, the pair $(p_i,p_{i+1})$ counted by $N_d(x)$ is certainly a pair of primes $(p,p+d)$ with $p\le x$, hence
\[N_d(x)\le \pi_2(x;d).
\]
Therefore
\[
B(x)\le \sum_{d\le c_2} \pi_2(x;d).
\]
Applying Lemma~238.1 for each fixed $d$ and absorbing the finite sum into a single constant gives
\[
B(x)\le C(c_2)\,\frac{x}{(\log x)^2}
\]
for all sufficiently large $x$, for some constant $C(c_2)>0$.

\smallskip
\emph{Step 3: finish using PNT.}
By the Prime Number Theorem, for all sufficiently large $x$,
\[m=\pi(x)\ge \frac{1}{2}\frac{x}{\log x}.
\]
Hence, for large $x$,
\[
L(x)\ge \frac{m}{B(x)+1}
\ge \frac{\tfrac12 x/\log x}{2C(c_2)\,x/(\log x)^2}
=\frac{1}{4C(c_2)}\,\log x.
\]
Taking
\[c_1^\ast:=\frac{1}{5C(c_2)}
\]
and enlarging $x$ if necessary ensures $L(x)>c_1^\ast\log x$. Any block of length $L(x)$ then supplies $>c_1^\ast\log x$ consecutive primes with all internal gaps $>c_2$, hence all pairwise differences $>c_2$.
\end{proof}

\smallskip
\emph{Remark.} The open problem asks whether $L(x)/\log x\to\infty$ (equivalently: for every fixed $c_1>0$ and all large $x$, $L(x)>c_1\log x$). The above argument gives only $L(x)\gg_{c_2}\log x$.

\item \textbf{VERIFICATION.}

\begin{itemize}
  \item \emph{Equivalence of ``any two'' vs ``adjacent''.} For a block of consecutive primes, if each adjacent difference is $>c_2$, then any non-adjacent pair differs by a sum of adjacent gaps and is therefore also $>c_2$.
  \item \emph{Block counting inequality.} If $B$ edges are removed from a path on $m$ vertices, one gets at most $B+1$ connected components (blocks). The maximum block size is at least $m/(B+1)$.
  \item \emph{Use of sieve bound.} We only require an \emph{upper} bound on prime pairs at distance $d$; no conjectural asymptotic is used.
  \item \emph{Edge cases.} Small primes (e.g. the gap $1$ between $2$ and $3$) affect only finitely many indices and do not matter for large $x$.
\end{itemize}

\item \textbf{FINAL.}

\textbf{UNRESOLVED.}
\begin{itemize}
  \item[(i)] \emph{Strongest fully proved partial result obtained here.} For each fixed $c_2>0$, there exists $c_1^\ast(c_2)>0$ such that for all sufficiently large $x$ there are $>c_1^\ast\log x$ consecutive primes $\le x$ whose internal gaps are all $>c_2$.
  \item[(ii)] \emph{First gap.} To reach the full conjecture for an \emph{arbitrary} prescribed $c_1>0$, one would need to prove $L(x)/\log x\to\infty$, which is strictly stronger than the counting argument driven only by the bound $B(x)\ll x/(\log x)^2$.
  \item[(iii)] \emph{Top 3 next moves.}
    \begin{itemize}
      \item Prove that the indices of small gaps $\le c_2$ cannot be ``nearly evenly spaced'' on the scale of $\log x$; any quantitative irregularity would push $L(x)$ above a fixed multiple of $\log x$.
      \item Seek stronger upper bounds of the form $B(x)=o(x/(\log x)^2)$ (or local versions on intervals) which would immediately force $L(x)=\omega(\log x)$ by the same block argument.
      \item Investigate whether modern results on correlations of primes (or conditional Hardy--Littlewood inputs) imply that the longest run of gaps $>c_2$ is typically much larger than its average spacing.
    \end{itemize}
  \item[(iv)] \emph{What a minimal counterexample would likely look like.} Fixed $c_1,c_2>0$ and an unbounded sequence $x_j\to\infty$ such that every block of $>c_1\log x_j$ consecutive primes $\le x_j$ contains a gap $\le c_2$; i.e. the ``small-gap indices'' form a hitting set for all long blocks.
\end{itemize}

\item \textbf{COMPLETION ESTIMATE (MANDATORY).}\\
COMPLETION: 60\%.

\end{enumerate}

