\section*{Erd\H{o}s Problem \#234}

\begin{enumerate}[label=\arabic*)]

\item \textbf{FORMAL RESTATEMENT.}

Let $p_n$ denote the $n$th prime ($p_1=2,p_2=3,\dots$) and let
\[d_n := p_{n+1}-p_n \qquad (n\ge 1)\]
be the $n$th prime gap. Throughout, $\log$ denotes the natural logarithm.

For $c\ge 0$, define the set
\[A_c := \Bigl\{n\in\mathbb{N} : n\ge 2\ \text{and}\ \frac{d_n}{\log n} < c\Bigr\}.
\]
(We restrict to $n\ge 2$ so that $\log n$ is defined.)

Define the \emph{natural density} of $A_c$ (if it exists) by
\[
  f(c) := \lim_{N\to\infty} \frac{1}{N}\,\bigl|A_c\cap\{1,2,\dots,N\}\bigr|.
\]

\textbf{Claim/conjecture.} For every $c\ge 0$, the limit defining $f(c)$ exists, and the resulting function $f:[0,\infty)\to[0,1]$ is continuous.

\smallskip
\emph{Edge cases.} For $c=0$, $A_0=\varnothing$ (since $d_n>0$), so $f(0)=0$ if the above definition is used.

\item \textbf{QUICK LITERATURE/CONTEXT CHECK.}

This is an ``existence of limiting distribution'' problem for the normalized gaps
\(a_n:=d_n/\log n\). Heuristically, in Cram\'er/Hardy--Littlewood models one expects a limiting distribution (often Poisson/exponential for suitably normalized gaps), which would in particular imply the existence of $f(c)$ and its continuity.

Unconditionally, one has very strong information on \emph{extremes} of prime gaps (liminf $=0$ in the $d_n/\log p_n$ normalization, and arbitrarily large multiples of the average gap), but no unconditional theorem gives the full limiting distribution of the normalized gaps.

\item \textbf{ATTACK PLAN.}

\emph{Proof strategies.}
\begin{itemize}
  \item[(P1)] Try to show convergence of the empirical distribution functions
  \(F_N(c):=\frac1N|\{2\le n\le N: d_n/\log n<c\}|\) to a limit using deep distributional results on primes in short intervals / correlations of primes.
  \item[(P2)] Prove existence of the density for all $c$ by deriving asymptotics for counts of gaps $\le c\log n$ from (a strong form of) the Hardy--Littlewood prime tuples conjecture.
\end{itemize}

\emph{Disproof/construction strategies.}
\begin{itemize}
  \item[(D1)] Attempt to use known irregularity phenomena for primes in short intervals (\`a la Maier) to construct a $c$ for which $F_N(c)$ has distinct limit points.
  \item[(D2)] Attempt to show $f$ cannot be continuous by forcing a nontrivial point mass in any putative limiting distribution.
\end{itemize}

\smallskip
\textbf{Best path chosen here.} In this write-up, the proof track (P1)--(P2) and the disproof track (D1)--(D2) both stall at the point where one would need substantially finer distributional control of prime gaps than is established here. Accordingly, I record rigorous partial results that any full solution must be consistent with: (i) the normalized gaps have mean $1+o(1)$; (ii) this yields quantitative bounds on upper/lower densities for large $c$.

\item \textbf{WORK.}

\paragraph{Lemma 234.1 (Mean of normalized gaps).}
Let $a_n:=\dfrac{p_{n+1}-p_n}{\log n}$ for $n\ge 2$. Assuming the Prime Number Theorem, one has
\[
\lim_{N\to\infty}\frac{1}{N}\sum_{n=2}^{N} a_n = 1.
\]

\begin{proof}
Write $d_n:=p_{n+1}-p_n$. For $N\ge 2$ consider
\[
S(N):=\sum_{n=2}^{N}\frac{d_n}{\log n}.
\]
For each $n\ge 2$ we have
\[
\frac{d_n}{\log n}=\int_{p_n}^{p_{n+1}} \frac{dt}{\log n}.
\]
On the interval $[p_n,p_{n+1})$ the prime-counting function satisfies $\pi(t)=n$, hence $\log n = \log \pi(t)$ there. Summing over $n=2,\dots,N$ and concatenating the integrals gives the identity
\[
S(N)=\int_{p_2}^{p_{N+1}}\frac{dt}{\log\pi(t)}.
\]
By the Prime Number Theorem, $\pi(t)\sim t/\log t$, so for sufficiently large $t$ we have
\[
\log\pi(t)=\log t + O(\log\log t).
\]
For large $t$, the error term satisfies $|O(\log\log t)|\le \tfrac12\log t$, so the function $u\mapsto 1/(\log t+u)$ can be expanded to first order, yielding
\[
\frac{1}{\log\pi(t)} = \frac{1}{\log t} + O\!\left(\frac{\log\log t}{(\log t)^2}\right).
\]
Therefore, with $X:=p_{N+1}$,
\[
S(N)=\int_{p_2}^{X}\frac{dt}{\log t}
\;+
O\!\left(\int_{p_2}^{X}\frac{\log\log t}{(\log t)^2}\,dt\right).
\]
The main integral is the logarithmic integral $\operatorname{li}(X)$ up to an additive constant. By integration by parts,
\[
\int_{2}^{X}\frac{dt}{\log t}=\frac{X}{\log X}+O\!\left(\frac{X}{(\log X)^2}\right).
\]
For the error integral, since the integrand is eventually increasing in $t$ and bounded above by its value at $X$, one has
\[
\int_{2}^{X}\frac{\log\log t}{(\log t)^2}\,dt
\ll
\frac{X\log\log X}{(\log X)^2}.
\]
Combining these estimates gives
\[
S(N)=\frac{X}{\log X}+O\!\left(\frac{X\log\log X}{(\log X)^2}\right).
\]
Finally, by the Prime Number Theorem in the equivalent form $p_{N+1}\sim (N+1)\log(N+1)$, we have $X\sim N\log N$ and $\log X = \log N + O(\log\log N)$. Hence
\[
\frac{X}{\log X}=N\,(1+o(1)),
\qquad
\frac{X\log\log X}{(\log X)^2}=N\,o(1).
\]
Therefore $S(N)=N+o(N)$, and dividing by $N$ yields
\[
\frac{1}{N}\sum_{n=2}^{N}a_n = \frac{S(N)}{N} \to 1.
\]
\end{proof}

\paragraph{Corollary 234.2 (A Markov-type bound on large normalized gaps).}
Assume Lemma~234.1. Fix $c>0$. Then
\[
\limsup_{N\to\infty}\frac{1}{N}\bigl|\{2\le n\le N: a_n\ge c\}\bigr|\le \frac{1}{c}.
\]
In particular, for $c>1$,
\[
\liminf_{N\to\infty}\frac{1}{N}\bigl|\{2\le n\le N: a_n< c\}\bigr|\ge 1-\frac{1}{c}.
\]

\begin{proof}
For each $N$ and $c>0$,
\[
\sum_{n=2}^{N}a_n \ge \sum_{\substack{2\le n\le N\\ a_n\ge c}} a_n \ge c\,\bigl|\{2\le n\le N: a_n\ge c\}\bigr|.
\]
Divide by $N$ and take $\limsup$; Lemma~234.1 gives $\lim_{N\to\infty}\frac1N\sum_{n=2}^N a_n =1$, hence the displayed bound.
The second inequality is the complement.
\end{proof}

\smallskip
\emph{What this does not give.} These arguments control only the first moment of $a_n$ and hence give only coarse bounds on the distribution of $a_n$; they do not address existence of the density $f(c)$ for a fixed $c$, nor continuity of $f$.

\item \textbf{VERIFICATION.}

\begin{itemize}
  \item \emph{Quantifiers.} Lemma~234.1 uses only $N\to\infty$ and PNT; the conversion $S(N)=\int dt/\log\pi(t)$ is exact.
  \item \emph{Edge cases.} The definition starts at $n\ge 2$ so $\log n$ is defined. Any finite number of initial terms does not affect densities.
  \item \emph{Error control.} The step $1/\log\pi(t)=1/\log t + O(\log\log t/(\log t)^2)$ requires $\log\pi(t)=\log t + O(\log\log t)$ and that $\log t$ dominates $\log\log t$; this holds for all sufficiently large $t$.
\end{itemize}

\item \textbf{FINAL.}

\textbf{UNRESOLVED.}
\begin{itemize}
  \item[(i)] \emph{Strongest fully proved partial result obtained here.} Assuming PNT, the mean of the normalized gaps $a_n=(p_{n+1}-p_n)/\log n$ tends to $1$, and hence for every fixed $c>0$,
  \(\limsup_{N\to\infty}\frac1N|\{n\le N: a_n\ge c\}|\le 1/c\).
  \item[(ii)] \emph{First gap.} No unconditional (or even PNT-level) method in this write-up proves that the limit defining $f(c)$ exists for an arbitrary fixed $c$, nor continuity of the putative limit function.
  \item[(iii)] \emph{Top 3 next moves.}
    \begin{itemize}
      \item Target an ``asymptotic for the count of gaps $\le c\log n$'' from a strong Hardy--Littlewood prime tuples input.
      \item Try to convert known irregularity results for primes in short intervals into oscillations of $F_N(c)$ to attack existence of the density.
      \item Study whether any currently known moment information on prime gaps (even conditional) implies continuity/no-atom properties of a limiting law.
    \end{itemize}
  \item[(iv)] \emph{What a minimal counterexample would likely look like.} A value $c_0$ and two sequences $N_j\to\infty$, $M_j\to\infty$ such that
  \(F_{N_j}(c_0)\) and \(F_{M_j}(c_0)\) converge to different limits, i.e. a persistent oscillation in the frequency of ``$d_n< c_0\log n$''.
\end{itemize}

\item \textbf{COMPLETION ESTIMATE (MANDATORY).}\\
COMPLETION: 35\%.

\end{enumerate}

%%%%%%%%%%%%%%%%%%%%%%%%%%%%%%%%%%%%%%%%%%%%%%%%%%%%%%%%%%%%%%%%%%%%%%%%%%%%%%%
