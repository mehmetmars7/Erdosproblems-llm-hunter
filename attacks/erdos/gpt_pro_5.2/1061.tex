% Erdos Problem #1061

\subsection*{FORMAL RESTATEMENT}
Let $\sigma(n)=\sum_{d\mid n} d$. For $x\ge 1$, let $N(x)$ be the number of \emph{ordered} pairs $(a,b)\in\mathbb Z_{\ge 1}^2$ such that
\[
\sigma(a)+\sigma(b)=\sigma(a+b)\quad\text{and}\quad a+b\le x.
\]
(If one instead counts unordered pairs, the count differs by at most a factor $2$.)
The question asks for the growth of $N(x)$ and in particular whether $N(x)\sim c x$ for some constant $c>0$.

\subsection*{QUICK LITERATURE/CONTEXT CHECK}
The problem is posed as an open question (Guy B15). I do not use or claim any external results.

\subsection*{ATTACK PLAN}
Proof-track: look for parametric infinite families of solutions (to show $N(x)\gg x$) and then attempt to classify all solutions or at least show $N(x)=O(x)$.
Disproof-track: search computationally for superlinear growth or irregular behavior.

\subsection*{WORK}
\textbf{FAST REALITY CHECK.}
A sieve computation of $\sigma(n)$ and a direct check of all $a+b\le x$ gives the exact counts:
\begin{verbatim}
x=1000:  N(x)=1620
x=2000:  N(x)=3806
x=5000:  N(x)=12484
x=10000: N(x)=30012
x=20000: N(x)=71612
\end{verbatim}
In particular $N(x)/x$ is empirically about $1.62,1.90,2.50,3.00,3.58$ at these sample points.
The smallest solutions include $(a,b,a+b)=(1,2,3)$ and $(2,6,8)$.

\medskip
\textbf{Lemma 1061.1 (proper-divisor reformulation).}
Let $s(n):=\sigma(n)-n$ be the sum of proper divisors of $n$ (with $s(1)=0$). Then
\[
\sigma(a)+\sigma(b)=\sigma(a+b)\quad\Longleftrightarrow\quad s(a)+s(b)=s(a+b).
\]

\emph{Proof.}
Using $\sigma(n)=n+s(n)$,
\[
\sigma(a)+\sigma(b)= (a+s(a))+(b+s(b)) = (a+b) + (s(a)+s(b)).
\]
Also $\sigma(a+b)=(a+b)+s(a+b)$. Subtracting $(a+b)$ from both sides yields the equivalence. \hfill$\square$

\medskip
\textbf{Lemma 1061.2 (explicit infinite family).}
If $a\ge 1$ satisfies $\gcd(a,6)=1$, then $(a,2a)$ is a solution:
\[
\sigma(a)+\sigma(2a)=\sigma(3a).
\]
Consequently $(2a,a)$ is also a solution.

\emph{Proof.}
Assume $\gcd(a,6)=1$, so in particular $\gcd(a,2)=\gcd(a,3)=1$. By multiplicativity of $\sigma$ on coprime inputs,
\[
\sigma(2a)=\sigma(2)\sigma(a)=3\sigma(a),\qquad \sigma(3a)=\sigma(3)\sigma(a)=4\sigma(a).
\]
Therefore
\[
\sigma(a)+\sigma(2a)=\sigma(a)+3\sigma(a)=4\sigma(a)=\sigma(3a),
\]
as required. \hfill$\square$

\medskip
\textbf{Corollary 1061.3 (linear lower bound).}
There exists an absolute constant $c_0>0$ such that $N(x)\ge c_0 x - O(1)$.
In fact one may take $c_0=2/9$.

\emph{Proof.}
For each integer $a$ with $1\le a\le \lfloor x/3\rfloor$ and $\gcd(a,6)=1$, Lemma 1061.2 gives two ordered solutions $(a,2a)$ and $(2a,a)$ with sum $3a\le x$.
The set of integers $a\le y$ coprime to $6$ has size $\ge \lfloor y/3\rfloor$ (one in each block of three consecutive integers avoids both factors $2$ and $3$), so the number of such $a$ with $a\le x/3$ is $\ge \lfloor x/9\rfloor$.
Hence $N(x)\ge 2\lfloor x/9\rfloor = (2/9)x - O(1)$. \hfill$\square$

\medskip
\textbf{Lemma 1061.4 (no prime-sum solutions except $(1,2)$).}
If $a,b>1$ and $a+b$ is prime, then $\sigma(a)+\sigma(b)\ne \sigma(a+b)$. Equivalently, the only solutions with $a+b$ prime are $(a,b)=(1,2)$ and $(2,1)$.

\emph{Proof.}
If $n>1$ then $\sigma(n)\ge n+1$ because $1$ and $n$ are divisors. Thus for $a,b>1$,
\[
\sigma(a)+\sigma(b)\ge (a+1)+(b+1)=a+b+2.
\]
If $a+b=p$ is prime, then $\sigma(a+b)=\sigma(p)=p+1=a+b+1$. Therefore $\sigma(a)+\sigma(b)\ge a+b+2> a+b+1=\sigma(a+b)$, so equality is impossible.
If one of $a,b$ equals $1$, then the only way to have $a+b$ prime and satisfy the equation is the direct check $(1,2)$ (and symmetry). \hfill$\square$

\subsection*{VERIFICATION}
Lemma 1061.2 uses only the standard multiplicativity of $\sigma$ for coprime arguments and the explicit values $\sigma(2)=3$, $\sigma(3)=4$; the coprimality condition $\gcd(a,6)=1$ is exactly what is needed.
The corollary's counting bound $\#\{a\le y:\gcd(a,6)=1\}\ge \lfloor y/3\rfloor$ was checked by the simple observation that in each triple $(3t+1,3t+2,3t+3)$ at least one number is not divisible by $2$ or $3$.
The numerical counts for $N(x)$ were computed by a direct sieve for $\sigma$ and then checking all pairs with $a+b\le x$.

\subsection*{FINAL}
\textbf{UNRESOLVED}
\begin{enumerate}
\item[(i)] \textbf{Strongest proved partial result.}
There are infinitely many solutions; in fact the explicit family $(a,2a)$ with $\gcd(a,6)=1$ gives the lower bound $N(x)\ge (2/9)x-O(1)$ (Lemma 1061.2 and Corollary 1061.3). Computations show $N(20000)=71612$ ordered solutions.
\item[(ii)] \textbf{First gap (crisp).}
Prove a matching upper bound $N(x)=O(x)$ and, stronger, show that the limit $\lim_{x\to\infty} N(x)/x$ exists and is positive (or compute it).
\item[(iii)] \textbf{Top 3 next moves.}
(1) Classify all solutions with a fixed ratio $b/a$ or with $\gcd(a,b)=1$, using the multiplicativity of $\sigma$.
(2) Prove that for each fixed sum $n$ the number of solutions to $\sigma(a)+\sigma(n-a)=\sigma(n)$ is uniformly bounded, implying $N(x)=O(x)$ by summing over $n\le x$.
(3) Extend computations to larger $x$ and try to identify additional parametric families beyond $(a,2a)$ with $\gcd(a,6)=1$.
\item[(iv)] \textbf{Minimal counterexample structure.}
A counterexample to $N(x)\sim cx$ would likely involve sums $n$ admitting an unusually large number of representations, i.e. many $a$ with $\sigma(a)+\sigma(n-a)=\sigma(n)$. Such $n$ might need to be highly composite (large $\sigma(n)$) to have room for many decompositions.
\end{enumerate}


