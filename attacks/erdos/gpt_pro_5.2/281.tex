\section*{Problem 281}

\subsection*{FORMAL RESTATEMENT}

Let $(n_i)_{i\ge 1}$ be a sequence of positive integers. For any choice of residues
$a_i\in \mathbb{Z}/n_i\mathbb{Z}$, consider the set of integers avoiding \emph{all} classes
\[
E(\mathbf a)\;:=\;\{m\in \mathbb{Z}:\ \forall i\ge 1,\ m\not\equiv a_i\pmod{n_i}\}.
\]
Assume the hypothesis:
\[
\operatorname{d}(E(\mathbf a))=0\qquad\text{for every choice of residues }(a_i)_{i\ge 1}.
\]
Prove that for every $\varepsilon>0$ there exists $k=k(\varepsilon)$ such that for every residue choice
$(a_i)$ the set avoiding the \emph{first $k$} congruences,
\[
E_k(\mathbf a)\;:=\;\{m\in \mathbb{Z}:\ \forall 1\le i\le k,\ m\not\equiv a_i\pmod{n_i}\},
\]
has density $<\varepsilon$.

\subsection*{QUICK LITERATURE/CONTEXT CHECK}

This statement is known to be true; standard proofs use the profinite completion of $\mathbb{Z}$,
Haar measure, and Dini's theorem (uniform convergence of a monotone sequence of continuous functions
on a compact space).

\subsection*{ATTACK PLAN}

\begin{itemize}
\item Encode the residue choices as a point of the compact product space
$X:=\prod_{i\ge 1} (\mathbb{Z}/n_i\mathbb{Z})$.
\item For each $k$, define a continuous function $d_k:X\to[0,1]$ by
$d_k(\mathbf a)=\operatorname{d}(E_k(\mathbf a))$.
These are decreasing in $k$ and depend only on the first $k$ coordinates.
\item Show pointwise that $d_k(\mathbf a)\to 0$ for every $\mathbf a$ by embedding into the profinite integers
$\widehat{\mathbb{Z}}$ and using translation invariance + Fatou's lemma to contradict the hypothesis if the limit were $>0$.
\item Apply Dini's theorem to upgrade pointwise convergence of a decreasing sequence of continuous functions
to uniform convergence, which is exactly the desired conclusion.
\end{itemize}

\subsection*{WORK (FULL PROOF)}

\paragraph{Step 1: Work in the compact product space of residue choices.}
Let
\[
X:=\prod_{i=1}^\infty (\mathbb{Z}/n_i\mathbb{Z})
\]
with the product topology. Each factor is finite discrete, hence compact; therefore $X$ is compact by Tychonoff.
Write $\mathbf a=(a_1,a_2,\dots)\in X$.

For $k\ge 1$ define
\[
E_k(\mathbf a):=\{m\in\mathbb{Z}:\ \forall 1\le i\le k,\ m\not\equiv a_i\pmod{n_i}\}.
\]

\begin{lemma}[Each $E_k(\mathbf a)$ is periodic, and its density exists]
Let $N_k:=\operatorname{lcm}(n_1,\dots,n_k)$. Then $E_k(\mathbf a)$ is a union of residue classes modulo $N_k$.
In particular, $\operatorname{d}(E_k(\mathbf a))$ exists and equals
\[
\operatorname{d}(E_k(\mathbf a))=\frac1{N_k}\big|\{0\le r<N_k:\ \forall i\le k,\ r\not\equiv a_i\pmod{n_i}\}\big|.
\]
\end{lemma}

\begin{proof}
The congruence conditions $m\not\equiv a_i\pmod{n_i}$ for $i\le k$ depend only on $m\bmod n_i$,
hence only on $m\bmod N_k$ since each $n_i\mid N_k$.
Thus $E_k(\mathbf a)$ is a union of classes mod $N_k$, so its density is the fraction of residues mod $N_k$
satisfying the conditions.
\end{proof}

Define the function
\[
d_k:X\to[0,1],\qquad d_k(\mathbf a):=\operatorname{d}(E_k(\mathbf a)).
\]
Each $d_k$ depends only on $(a_1,\dots,a_k)$ and hence is locally constant on cylinder sets; in particular:

\begin{lemma}[$d_k$ is continuous and decreasing in $k$]
Each $d_k$ is continuous on $X$, and $d_{k+1}(\mathbf a)\le d_k(\mathbf a)$ for all $\mathbf a\in X$.
\end{lemma}

\begin{proof}
Continuity: $d_k$ depends only on the first $k$ coordinates and takes finitely many values, so it is continuous.
Monotonicity is immediate since $E_{k+1}(\mathbf a)\subseteq E_k(\mathbf a)$.
\end{proof}

\paragraph{Step 2: Embed into profinite integers and identify $d_k$ as Haar measures.}
Let $\widehat{\mathbb{Z}}$ be the profinite completion of $\mathbb{Z}$, equipped with its normalized Haar probability measure $\mu$.
For each $n\ge 1$ there is a continuous surjective projection
$\pi_n:\widehat{\mathbb{Z}}\to \mathbb{Z}/n\mathbb{Z}$.

For $\mathbf a\in X$ and $k\ge 1$, define the clopen set
\[
C_k(\mathbf a):=\{x\in\widehat{\mathbb{Z}}:\ \forall i\le k,\ \pi_{n_i}(x)\ne a_i\}.
\]
Also set $C_\infty(\mathbf a):=\bigcap_{k\ge 1} C_k(\mathbf a)$.

\begin{lemma}[Haar measure equals density for finite prefixes]
For every $\mathbf a\in X$ and every $k\ge 1$,
\[
\mu(C_k(\mathbf a))=d_k(\mathbf a).
\]
\end{lemma}

\begin{proof}
Let $N_k=\operatorname{lcm}(n_1,\dots,n_k)$. The set $C_k(\mathbf a)$ depends only on the residue of $x\bmod N_k$,
and in fact it is a union of residue classes modulo $N_k$.
Haar measure on $\widehat{\mathbb{Z}}$ assigns each residue class mod $N_k$ mass $1/N_k$.
Thus $\mu(C_k(\mathbf a))$ is exactly the fraction of residues $r\bmod N_k$ avoiding all $a_i\bmod n_i$,
which is $d_k(\mathbf a)$ by the previous lemma.
\end{proof}

Since $(C_k(\mathbf a))_{k\ge 1}$ is a decreasing sequence of measurable sets,
\[
\lim_{k\to\infty} d_k(\mathbf a)=\lim_{k\to\infty}\mu(C_k(\mathbf a))=\mu\!\left(\bigcap_{k\ge 1} C_k(\mathbf a)\right)=\mu(C_\infty(\mathbf a)).
\]
So to show $d_k(\mathbf a)\to 0$ pointwise, it suffices to show $\mu(C_\infty(\mathbf a))=0$ for all $\mathbf a$.

\paragraph{Step 3: Show $\mu(C_\infty(\mathbf a))=0$ using translation invariance + Fatou.}

\begin{lemma}[Key measure-zero claim]
For every $\mathbf a\in X$, one has $\mu(C_\infty(\mathbf a))=0$.
\end{lemma}

\begin{proof}
Assume for contradiction that $\mu(C_\infty(\mathbf a))=\delta>0$.
Let $f:=\mathbf 1_{C_\infty(\mathbf a)}$.
For each $x\in\widehat{\mathbb{Z}}$ and each $N\ge 1$ define the ergodic averages
\[
A_N(x):=\frac1N\sum_{t=0}^{N-1} f(x+t),
\]
where we view $t$ as the embedded integer in $\widehat{\mathbb{Z}}$ and use the group law on $\widehat{\mathbb{Z}}$.

Because $\mu$ is translation-invariant, for each fixed $t$ we have
$\int_{\widehat{\mathbb{Z}}} f(x+t)\,d\mu(x)=\int_{\widehat{\mathbb{Z}}} f(x)\,d\mu(x)=\delta$.
Hence for every $N$,
\[
\int_{\widehat{\mathbb{Z}}} A_N(x)\,d\mu(x)=\delta.
\]
By Fatou's lemma,
\[
\int_{\widehat{\mathbb{Z}}} \limsup_{N\to\infty} A_N(x)\,d\mu(x)\ \ge\ \limsup_{N\to\infty}\int_{\widehat{\mathbb{Z}}}A_N(x)\,d\mu(x)=\delta.
\]
Therefore there exists some $x\in\widehat{\mathbb{Z}}$ such that $\limsup_{N\to\infty}A_N(x)\ge \delta>0$.

Unwinding the definition, this says that the set
\[
S_x:=\{t\in\mathbb{N}:\ x+t\in C_\infty(\mathbf a)\}
\]
has positive \emph{upper} density (at least $\delta$).
Now define a new residue sequence $\mathbf a'=(a'_i)_{i\ge 1}$ by
\[
a'_i\equiv a_i-\pi_{n_i}(x)\pmod{n_i}\qquad (i\ge 1).
\]
Then for an integer $t$ we have
\[
x+t\in C_\infty(\mathbf a)\quad\Longleftrightarrow\quad
\forall i,\ \pi_{n_i}(x+t)\ne a_i
\quad\Longleftrightarrow\quad
\forall i,\ t\not\equiv a'_i\pmod{n_i}.
\]
Thus $S_x$ is exactly the set $E(\mathbf a')$ of integers avoiding \emph{all} residue classes $a'_i\bmod n_i$.

By the hypothesis of the problem, $\operatorname{d}(E(\mathbf a'))=0$.
In particular, $E(\mathbf a')$ has upper density $0$, contradicting that $S_x=E(\mathbf a')$ has
upper density $\ge\delta>0$.
Hence $\mu(C_\infty(\mathbf a))$ cannot be positive, so it must be $0$.
\end{proof}

Combining the previous lemmas, we have proved:

\begin{lemma}[Pointwise limit]
For every $\mathbf a\in X$, one has $d_k(\mathbf a)\downarrow 0$ as $k\to\infty$.
\end{lemma}

\paragraph{Step 4: Uniformity via Dini's theorem.}

\begin{lemma}[Dini]
Let $X$ be compact and let $f_k:X\to\mathbb{R}$ be continuous with $f_{k+1}(x)\le f_k(x)$ for all $x$.
If $f_k(x)\to 0$ for every $x$, then $f_k\to 0$ uniformly on $X$.
\end{lemma}

Apply Dini with $f_k=d_k$. Since $X$ is compact, each $d_k$ is continuous, $d_k$ decreases, and $d_k\to 0$
pointwise, we conclude $d_k\to 0$ uniformly. That is:

\begin{proposition}[Uniform smallness of finite prefixes]
For every $\varepsilon>0$ there exists $k$ such that for every $\mathbf a\in X$,
\[
d_k(\mathbf a)=\operatorname{d}(E_k(\mathbf a))<\varepsilon.
\]
\end{proposition}

This proposition is exactly the required conclusion.

\subsection*{VERIFICATION}

Edge cases:
\begin{itemize}
\item If some $n_i=1$, then $a_i\bmod 1$ is the only residue class and covers all integers; the hypothesis holds
and the conclusion is trivial for $k=1$.
\item The argument never uses any special number-theoretic structure of the $n_i$; it only uses compactness,
Haar measure, and translation invariance on $\widehat{\mathbb{Z}}$.
\end{itemize}

\subsection*{FINAL}

\textbf{FINAL: FULL SOLUTION / FULL PROOF.} The hypothesis implies $d_k(\mathbf a)\downarrow 0$ pointwise on the compact
space of residue choices; Dini's theorem upgrades this to uniform convergence, yielding the required $k(\varepsilon)$.

\subsection*{COMPLETION ESTIMATE}

\textbf{100\%} (complete proof).

% ---------------- bibliography ----------------
\begin{thebibliography}{9}

\bibitem{Simpson1986}
R.~J.~Simpson,
\emph{Exact coverings of the integers by arithmetic progressions},
Discrete Mathematics \textbf{59} (1986), 181--190.

\bibitem{Sun1991}
Z.-W.~Sun,
\emph{A theorem concerning systems of residue classes},
Acta Math. Univ. Comenian. \textbf{60} (1991), 123--131.

\bibitem{Cambie2025}
P.~Cambie,
\emph{Proving it is impossible; on Erd\H{o}s problem \#278},
preprint (2025), available online.

\end{thebibliography}

