% Solutions/attempts for Erdos Problems 887--891
% Generated following PROMPT_STRATEGY.MD.

%Erdos problem 887

\noindent 1) \textbf{FORMAL RESTATEMENT}

\noindent For an integer $n\ge 1$ and real $C>0$, define
\[
D_C(n) \;:=\; \#\Big\{d\in\mathbb{N}: d\mid n\ \text{and}\ \sqrt n< d < \sqrt n + C\,n^{1/4}\Big\}.
\]
The question is whether there exists an \emph{absolute} constant $K\in\mathbb{N}$ such that:

\noindent for every real $C>0$ there exists $N(C)\in\mathbb{N}$ with
\[
\forall n\ge N(C)\qquad D_C(n)\le K.
\]

\noindent 2) \textbf{QUICK LITERATURE/CONTEXT CHECK}

\noindent The problem statement cites Erd\H{o}s--Rosenfeld (1997) as showing that there are infinitely many $n$ with \emph{four} divisors in $(\sqrt n,\sqrt n+n^{1/4})$, and asks whether $4$ is best possible. I do not use any external results beyond what is explicitly stated in the problem text.

\noindent 3) \textbf{ATTACK PLAN}

\noindent \emph{Proof direction (bounded $D_C(n)$).}
\begin{itemize}
\item Reparameterize divisors $d$ in the short interval by complementary divisors $e=n/d$ and then by $(m,t)$ with $n=m^2-t^2$ and $t$ of size $\ll n^{1/4}$. Try to bound the number of such representations uniformly.
\item Translate the problem to counting integer lattice points on $xy=n$ inside a thin box around $(\sqrt n,\sqrt n)$. Try to show that too many lattice points force algebraic structure that cannot persist as $n\to\infty$.
\end{itemize}

\noindent \emph{Disproof direction (unbounded $D_C(n)$).}
\begin{itemize}
\item Attempt to construct $n$ with many factorizations $n=ab$ where $a,b$ are both within $O(n^{1/4})$ of $\sqrt n$ (equivalently many ``almost-square'' factorizations). This would amount to producing many solutions of $m^2-t^2=n$ with small $t$.
\end{itemize}

\noindent 4) \textbf{WORK}

\noindent \textbf{Lemma 887.1 (location of the complementary divisor).}
Let $n\ge 1$ and $C>0$. If $d\mid n$ and
\[\sqrt n< d < \sqrt n + C\,n^{1/4},\]
then the complementary divisor $e:=n/d$ satisfies
\[\sqrt n - C\,n^{1/4} < e < \sqrt n.\]
More precisely, writing $d=\sqrt n + x$ with $0<x<Cn^{1/4}$,
\[
 e = \sqrt n - x + \frac{x^2}{\sqrt n + x}
\]
and hence
\[\sqrt n - x < e < \sqrt n - x + C^2.\]

\noindent\emph{Proof.}
Since $d>\sqrt n$, we have $e=n/d<\sqrt n$.
For the lower bound, set $u:=C n^{-1/4}>0$. Then
\[
 e \;=\; \frac{n}{d}\;>\; \frac{n}{\sqrt n + Cn^{1/4}} 
 \;=\; \frac{\sqrt n}{1+u}.
\]
For any $u>0$, one has $\frac{1}{1+u} > 1-u$ because
$(1-u)(1+u)=1-u^2<1$. Therefore
\[
 e > \sqrt n(1-u)=\sqrt n - Cn^{1/4}.
\]
For the identity, write $d=\sqrt n + x$ and compute
\[
\frac{n}{\sqrt n + x}
= \frac{(\sqrt n - x)(\sqrt n + x) + x^2}{\sqrt n + x}
= \sqrt n - x + \frac{x^2}{\sqrt n + x}.
\]
Since $0<x<Cn^{1/4}$, we have $0<\frac{x^2}{\sqrt n + x} < \frac{x^2}{\sqrt n} < C^2$, giving the stated bounds.\hfill$\square$

\medskip
\noindent \textbf{Lemma 887.2 (difference-of-squares encoding with a small parameter).}
Let $n\ge 1$ and let $d>\sqrt n$ be a divisor of $n$. Put $e=n/d$ and define
\[
 m:=\frac{d+e}{2},\qquad t:=\frac{d-e}{2}.
\]
Then $m,t\in \tfrac12\mathbb{Z}$, $m>t\ge 0$, and
\[
 n = m^2 - t^2.
\]
Moreover, if $d < \sqrt n + Cn^{1/4}$ then $t < C n^{1/4}$.

\noindent\emph{Proof.}
The identities $d=m+t$ and $e=m-t$ are immediate from the definitions, so
\[
 m^2-t^2 = (m+t)(m-t)=de=n.
\]
Since $d,e\in\mathbb{Z}$, the numbers $m,t$ are either both integers or both half-integers, i.e., in $\tfrac12\mathbb{Z}$. Also $d>e>0$ implies $t>0$ and $m>t$.
For the upper bound on $t$, write $d=\sqrt n + x$ with $0<x<Cn^{1/4}$, and use the exact formula from Lemma 887.1:
\[
 d-e = (\sqrt n + x) - \Big(\sqrt n - x + \frac{x^2}{\sqrt n + x}\Big)
 = 2x - \frac{x^2}{\sqrt n + x}.
\]
Since $\frac{x^2}{\sqrt n + x}>0$, this gives $d-e<2x$, hence
\[
 t=\frac{d-e}{2} < x < Cn^{1/4}.
\]
\hfill$\square$

\medskip
\noindent \textbf{Fast reality check (computation).}
For $C=1$ I brute-scanned all integers $1\le n\le 2{,}000{,}000$ and counted divisors $d$ with $\sqrt n<d<\sqrt n+n^{1/4}$ by explicitly testing all integers $d$ in the candidate range $(\lfloor\sqrt n\rfloor,\,\lfloor \sqrt n+n^{1/4}\rfloor]$. Result:
\[
\max_{1\le n\le 2\cdot 10^6} D_1(n) = 1.
\]
So in this range no example has even two divisors in the interval (this does \emph{not} contradict the statement in the problem text that there are infinitely many $n$ with four such divisors; it only means the first examples are beyond $2\cdot 10^6$).

\noindent 5) \textbf{VERIFICATION}

\noindent Lemma 887.1: the only nontrivial inequality used is $\frac1{1+u}>1-u$ for $u>0$, verified by multiplying out. The formula for $e$ is an exact rational identity.

\noindent Lemma 887.2: the factorization $m^2-t^2=(m+t)(m-t)$ is exact. The bound $t<Cn^{1/4}$ uses the exact expression for $d-e$ and the strict inequality $x^2/(\sqrt n+x)>0$.

\noindent Computation: the divisor count method is correct because every divisor $d>\sqrt n$ with $d<\sqrt n+n^{1/4}$ lies in the scanned integer interval and is detected by the modulus test $n\bmod d=0$.

\noindent 6) \textbf{FINAL}

\noindent \textbf{UNRESOLVED}
\begin{enumerate}
\item[(i)] \textbf{Strongest proved partial result here.} Any divisor $d$ in $(\sqrt n,\sqrt n+Cn^{1/4})$ forces its complementary divisor $e=n/d$ to lie in $(\sqrt n-Cn^{1/4},\sqrt n)$ (Lemma 887.1). Equivalently, such divisors correspond to factorizations $n=de$ with both factors within $O(n^{1/4})$ of $\sqrt n$, and to representations $n=m^2-t^2$ with $t<Cn^{1/4}$ (Lemma 887.2).
\item[(ii)] \textbf{First gap (crisp).} Prove (or disprove) that the number of factor pairs $(e,d)$ with $ed=n$ and $|d-\sqrt n|<Cn^{1/4}$ is bounded by an absolute constant independent of $n$ (for each fixed $C$, eventually in $n$).
\item[(iii)] \textbf{Top 3 next moves.}
  \begin{itemize}
  \item Try to show that having $\ge 5$ such divisors forces two distinct representations $n=m^2-t^2=m'^2-t'^2$ with $t,t'\ll n^{1/4}$ that imply a nontrivial diophantine relation (e.g. a small nonzero value of $(m-m')(m+m')=(t-t')(t+t')$) and then bound the number of solutions.
  \item Computationally search for the smallest $n$ with $D_1(n)\ge 2,3,4$ and inspect the factorization pattern; attempt to reverse-engineer a parametric construction that could produce $D_1(n)\ge 5$.
  \item Attempt a ``repulsion'' lemma: if $d_1<d_2<\cdots<d_r$ are divisors in the interval, relate consecutive gaps $d_{j+1}-d_j$ to constraints on $n$ (e.g. via comparing $n/d_j$) to show $r$ must be bounded.
  \end{itemize}
\item[(iv)] \textbf{Minimal counterexample structure.} A counterexample to the existence of a universal $K$ would be an infinite sequence $n_t\to\infty$ (for some fixed $C$) with many distinct divisors $d_{t,1}<\cdots<d_{t,r_t}$ in $(\sqrt{n_t},\sqrt{n_t}+Cn_t^{1/4})$ and $r_t\to\infty$. Equivalently, many factor pairs $n_t=e_{t,j}d_{t,j}$ with both $e_{t,j},d_{t,j}$ clustered within $O(n_t^{1/4})$ of $\sqrt{n_t}$.
\end{enumerate}


