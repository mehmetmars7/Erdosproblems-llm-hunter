% Erdos Problem #901

% Erdos Problem #901

\noindent\textbf{1) FORMAL RESTATEMENT}

An \emph{$n$-uniform hypergraph} $H=(V,E)$ has vertex set $V$ and edge set
$E\subseteq\binom{V}{n}$.
A 2-coloring of $V$ is \emph{proper} if no edge is monochromatic.
A hypergraph is \emph{2-colorable} iff it admits a proper 2-coloring (this is
also called \emph{Property B}).

Let $m(n)$ be the minimum number of edges in an $n$-uniform hypergraph which is
\emph{not} 2-colorable (equivalently, which is 3-chromatic).
Estimate $m(n)$ as $n\to\infty$.

\medskip
\noindent\textbf{2) QUICK LITERATURE/CONTEXT CHECK}

The problem statement records known bounds:
\[2^n\ll m(n)\ll n^2 2^n\]
(Erd\H{o}s), and improvements of the lower bound up to roughly
$\sqrt{n/\log n}\,2^n$.
Known exact values recorded are $m(2)=3$, $m(3)=7$, $m(4)=23$.

\medskip
\noindent\textbf{3) ATTACK PLAN}

\begin{itemize}
\item Reprove the standard probabilistic-method lower bound $m(n)\ge 2^{n-1}$.
\item Give a self-contained probabilistic existence proof of an upper bound of
shape $m(n)\le C n^2 2^n$.
\item Sanity-check $n=3$ by verifying the Fano plane is not 2-colorable.
\end{itemize}

\medskip
\noindent\textbf{4) WORK}

\textbf{PHASE 1 — FAST REALITY CHECK (small cases).}

For $n=3$, the classical Fano plane has 7 vertices and 7 edges:
\[
E=\{012,034,056,135,146,236,245\}\ \text{(vertices }0,\dots,6\text{).}
\]
A brute-force check over all $2^7$ 2-colorings finds \emph{no} proper coloring,
so this 3-uniform hypergraph is not 2-colorable.
Moreover, removing any one edge makes it 2-colorable (edge-minimal within this
7-vertex structure).

\medskip
\textbf{Problem-specific lemmas (Erd\H{o}s-style exponential lower bound and a
matching existence upper bound).}

\medskip
\noindent\textbf{Lemma 901.1 (First-moment lower bound).}
If an $n$-uniform hypergraph $H$ has $m$ edges and $m<2^{n-1}$, then $H$ is
2-colorable. In particular,
\[ m(n)\ge 2^{n-1}. \]

\emph{Proof.}
Color each vertex independently red/blue with probability $1/2$.
For a fixed edge $e$ of size $n$, the probability it is monochromatic is
\[\mathbb{P}(e\text{ all red})+\mathbb{P}(e\text{ all blue})=2\cdot 2^{-n}=2^{1-n}.
\]
Let $X$ be the number of monochromatic edges. By linearity of expectation,
\[ \mathbb{E}X = m\cdot 2^{1-n}. \]
If $m<2^{n-1}$ then $\mathbb{E}X<1$. Since $X$ is a nonnegative integer-valued
random variable, there must exist a coloring with $X=0$, i.e. a proper 2-coloring.
Thus any non-2-colorable $n$-uniform hypergraph must have $m\ge 2^{n-1}$ edges.
\qed

\medskip
\noindent\textbf{Lemma 901.2 (Union bound upper bound $m(n)\ll n^2 2^n$).}
For every $n\ge 2$ there exists an $n$-uniform hypergraph with at most
$C\,n^2 2^n$ edges that is not 2-colorable, for an absolute constant $C$.
Consequently $m(n)\ll n^2 2^n$.

\emph{Proof.}
Fix $n\ge 2$ and set $v:=n^2$ vertices. Let $\mathcal{U}:=\binom{[v]}{n}$ be the
family of all $n$-subsets.
Choose $m$ edges uniformly at random \emph{without replacement} from $\mathcal{U}$.

Fix a 2-coloring of $[v]$ with red class size $r$ and blue class size $v-r$.
Let $M(r)$ be the number of monochromatic $n$-sets under this coloring:
\[ M(r)=\binom{r}{n}+\binom{v-r}{n}. \]

\emph{Step 1: a uniform lower bound on $M(r)$.}
The function $g(t)=\binom{t}{n}$ (for real $t\ge n-1$) has nonnegative second
finite differences in $t$ (discrete convexity), and one checks directly that the
sequence $\binom{t}{n}$ is convex in integer $t$.
Therefore $M(r)$ is minimized when $r$ is as close to $v/2$ as possible, hence
\[ M(r)\ge 2\binom{\lfloor v/2\rfloor}{n} \ge 2\binom{v/2 -1}{n} \quad (v\ge 2n).
\]
(For $v=n^2\ge 2n$ when $n\ge 2$.)

\emph{Step 2: convert to a monochromatic-edge fraction.}
Let
\[ \delta := \frac{2\binom{v/2 -1}{n}}{\binom{v}{n}}.
\]
Then for every coloring, at least a $\delta$ fraction of all $n$-sets are
monochromatic.
We lower bound $\delta$.
Write falling factorials $(x)_n=x(x-1)\cdots(x-n+1)$.
Then
\[
\frac{\binom{v/2 -1}{n}}{\binom{v}{n}}
=\frac{(v/2-1)_n}{(v)_n}
=\prod_{i=0}^{n-1}\frac{v/2-1-i}{v-i}.
\]
Since $v=n^2$ and $0\le i\le n-1$, we have $v-i\le v$ and
$v/2-1-i \ge v/2 - n$.
Thus each factor is at least
\[
\frac{v/2-n}{v}=\frac12-\frac{n}{v}=\frac12-\frac1n.
\]
For $n\ge 2$, $\frac12-\frac1n\ge \frac14$, so
\[
\frac{\binom{v/2 -1}{n}}{\binom{v}{n}} \ge \Bigl(\frac14\Bigr)^n = 2^{-2n}.
\]
Hence $\delta\ge 2\cdot 2^{-2n}=2^{1-2n}$.

\emph{Step 3: union bound over all colorings.}
For a fixed coloring, the probability that none of the $m$ chosen edges is
monochromatic is at most
\[
\Bigl(1-\delta\Bigr)^m \le \exp(-\delta m).
\]
There are $2^v$ colorings, so by the union bound the probability the random
hypergraph is 2-colorable is at most
\[
2^v\exp(-\delta m).
\]
Choose
\[ m:= \left\lceil \frac{(v+1)\log 2}{\delta}\right\rceil.
\]
Then $2^v\exp(-\delta m)\le e^{-\log 2}<1$, so with positive probability the
random hypergraph is \emph{not} 2-colorable.
Therefore there exists a non-2-colorable $n$-uniform hypergraph on $v=n^2$
vertices with at most $m$ edges.

Finally, using $\delta\ge 2^{1-2n}$ and $v=n^2$,
\[
 m \le \frac{(n^2+1)\log 2}{2^{1-2n}} +1
 \le C\, n^2 2^{2n}
\]
for an absolute constant $C$.
This is weaker than the stated $n^2 2^n$ bound, but it is a fully self-contained
probabilistic existence bound of the same qualitative form “polynomial times
exponential”.
\qed

\emph{Remark.} The bound obtained above is $\ll n^2 4^n$ due to a crude estimate
for $\delta$; sharpening the ratio
$\binom{v/2}{n}/\binom{v}{n}\approx 2^{-n}$ recovers the classical $\ll n^2 2^n$
scale.

\medskip
\noindent\textbf{5) VERIFICATION}

\begin{itemize}
\item Lemma~901.1: expectation computation uses only independence and the exact
probability a size-$n$ edge is monochromatic under a random 2-coloring.
\item Lemma~901.2: the convexity/minimization of
$\binom{r}{n}+\binom{v-r}{n}$ at $r\approx v/2$ is standard; the product bound for
$\delta$ is crude but valid for $v=n^2$.
\item Constants: the final bound derived is $\ll n^2 4^n$ (not $n^2 2^n$) because
of the crude $\delta\ge 2^{1-2n}$ estimate. This is honestly weaker than the
problem statement’s quoted upper bound.
\end{itemize}

\medskip
\noindent\textbf{6) FINAL}

\textbf{UNRESOLVED}

(i) \emph{Strongest proved partial result.}
A fully self-contained lower bound is $m(n)\ge 2^{n-1}$ (Lemma~901.1).
A self-contained existence upper bound via a crude union bound gives
$m(n)\ll n^2 4^n$ (Lemma~901.2).

(ii) \emph{First gap (crisp).}
Close the exponential gap: prove $m(n)=\Theta(n\,2^n)$ (as speculated in the
statement) or determine the correct polynomial factor multiplying $2^n$.

(iii) \emph{Top 3 next moves.}
\begin{enumerate}
\item Sharpen Lemma~901.2 by optimizing the vertex count $v$ and using the sharp
estimate $\binom{v/2}{n}/\binom{v}{n}\approx 2^{-n}$ to recover an explicit
$\ll n^2 2^n$ upper bound with full constants.
\item Investigate extremal/algorithmic approaches (e.g. entropy compression)
producing non-2-colorable hypergraphs with $\sim n2^n$ edges.
\item Computational: verify small $n$ values by exhaustive search (or SAT) to
better guess the correct asymptotic constant and the right polynomial factor.
\end{enumerate}

(iv) \emph{Minimal counterexample structure (if $m(n)$ is not $\Theta(n2^n)$).}
If $m(n)$ were significantly smaller than $n2^n$, one would expect a highly
structured hypergraph with very few edges but still forcing a monochromatic edge
in every 2-coloring; such a structure would likely resemble a sparse “design-like”
configuration where every balanced coloring is hit many times.


