
1) FORMAL RESTATEMENT

Fix integers $a,b,c\ge 2$ with $\gcd(a,b)=\gcd(a,c)=\gcd(b,c)=1$.
Let
\[
S(a,b,c)\;:=\;\{a^k b^\ell c^m : k,\ell,m\in \mathbb{N}_0\}\subset \mathbb{N}.
\]
(So $1\in S$.)

A finite subset $T\subset S$ is called *divisibility-antichain* if for any two distinct $x,y\in T$ neither $x\mid y$ nor $y\mid x$.

Call $S$ *$d$-complete* if there exists $N_0$ such that every integer $n\ge N_0$ can be written as
\[
 n = \sum_{t=1}^M s_t
\]
for some finite divisibility-antichain $\{s_1,\dots,s_M\}\subset S$ (in particular all $s_t$ are distinct).

**Problem.** Determine for which pairwise coprime triples $(a,b,c)$ the set $S(a,b,c)$ is $d$-complete.

Stress points / edge cases to watch:
- The antichain condition forbids using both $1$ and any other element (since $1$ divides everything).
- Divisibility structure depends strongly on prime factorizations; pairwise coprime assumption is crucial.

2) QUICK LITERATURE/CONTEXT CHECK

(Restricted to statements explicitly present in 123-137.tex.)
- Erdos and Lewin (1981) proved $S(a,b,c)$ is $d$-complete for $(a,b,c)=(3,5,7)$.
- For $(a,b,c)=(2,5,c)$ with $c$ odd and $c\not\equiv 1\pmod 4$, the sequence is $d$-complete (quoted as due to Erdos and Szekeres, 1960).
- Ma and Chen (1990) proved $d$-completeness for $(a,b,c)=(2,3,5)$.
- Erdos conjectured a stronger form for $(2,3,5)$: for every $\varepsilon>0$, all large integers can be represented as a sum of distinct non-dividing elements of $\{2^k3^\ell5^m\}$, all lying in $[b_1,(1+\varepsilon)b_1)$.

3) ATTACK PLAN

- Translate divisibility into a partial order on exponent vectors $(k,\ell,m)$.
- Try to understand antichains in that poset and whether their value-sums can cover long intervals.
- Do small computational searches for representability thresholds for specific triples (sanity checks).

4) WORK

Lemma 4.1 (Coordinatewise divisibility).
Let $(a,b,c)$ be pairwise coprime and let
$x=a^k b^\ell c^m$ and $y=a^{k'} b^{\ell'} c^{m'}$.
Then
\[
 x\mid y \quad\Longleftrightarrow\quad k\le k',\ \ell\le \ell',\ m\le m'.
\]

*Proof.* Because $a,b,c$ are pairwise coprime, every prime factor of $a$ is coprime to $b$ and $c$, etc. Thus in the prime factorization of $x$ the primes coming from $a$ occur to exponents proportional to $k$, those from $b$ proportional to $\ell$, and those from $c$ proportional to $m$, and these groups of primes are disjoint. Therefore $x\mid y$ holds exactly when the $a$-part of $x$ divides the $a$-part of $y$ (i.e. $k\le k'$), and similarly for the $b$- and $c$-parts. \qed

Lemma 4.2 (Antichains are antichains of exponent vectors).
A finite set $T\subset S(a,b,c)$ is a divisibility-antichain if and only if the corresponding set of exponent vectors
\[\{(k,\ell,m): a^k b^\ell c^m\in T\}\subset \mathbb{N}_0^3\]
is an antichain under the product order $\preceq$ defined by
$(k,\ell,m)\preceq (k',\ell',m')$ iff $k\le k'$, $\ell\le \ell'$, $m\le m'$. 

*Proof.* Immediate from Lemma 4.1. \qed

FAST REALITY CHECK (computation).
For each triple, I generated all monomials $a^k b^\ell c^m\le X$ and searched (by backtracking) for representations of each $n\le X$ as a sum of distinct monomials with the antichain condition. Results:
- For $(a,b,c)=(2,3,5)$: every integer $1\le n\le 1000$ is representable.
- For $(a,b,c)=(3,5,7)$: among $1\le n\le 1000$ exactly the following $19$ integers are *not* representable:
\[
\{2,4,6,11,13,17,18,19,20,23,29,33,37,43,51,92,100,148,185\}.
\]
In particular, all integers $186\le n\le 1000$ are representable in this sense.

5) VERIFICATION

- Lemma 4.1 uses only disjointness of prime factors of $a,b,c$; pairwise coprime is exactly what is needed.
- Lemma 4.2 is a direct reformulation.
- The computation only uses summands $\le X$, which is valid because all summands are positive; any representation of $n\le X$ uses summands $\le n\le X$.

6) FINAL

**UNRESOLVED**

(i) Strongest fully proved partial result:
- The divisibility relation in $S(a,b,c)$ is exactly coordinatewise on exponent vectors (Lemma 4.1), so the antichain condition is an antichain condition in $\mathbb{N}_0^3$ (Lemma 4.2).
- Computational evidence: $(2,3,5)$ represents all $n\le 1000$; $(3,5,7)$ represents all $186\le n\le 1000$.

(ii) Exact first gap:
- Prove (or disprove) that for every pairwise coprime triple $(a,b,c)$, the set $S(a,b,c)$ is $d$-complete (or classify precisely which triples are).

(iii) Top 3 next moves:
1. Try to build a greedy / covering argument on the exponent-poset antichains that yields long intervals of attainable sums.
2. Attempt a minimal-counterexample analysis: if a large $n$ is not representable, constrain the maximal element in any candidate antichain and derive contradictions.
3. Search for structured counterexamples where antichains are forced to be too sparse (e.g., when one base is a large prime vs a small base) and test computationally for growing gaps.

(iv) What a minimal counterexample would likely look like:
- A triple where one base is so small that divisibility-comparabilities force antichains to concentrate on a narrow "layer" of exponent vectors, potentially limiting attainable sums; or conversely where one base is huge so available summands are too lacunary.


