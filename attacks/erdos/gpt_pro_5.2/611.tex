
1) FORMAL RESTATEMENT

Let $G=(V,E)$ be a finite simple graph with $n:=|V|$, and let $\tau(G)$ be the clique transversal number (as in Problem 610).
Question A: Fix $c>0$. Suppose every maximal clique in $G$ has size at least $cn$.
Must it follow that $\tau(G)=o_c(n)$ as $n\to\infty$?

Question B: For $c>0$, define $k_c(n)$ to be the least integer such that if every maximal clique in an $n$-vertex graph has size at least $k_c(n)$, then $\tau(G)<(1-c)n$.
Estimate $k_c(n)$ as a function of $n$ (for fixed $c$).

Ambiguity note (minimal correction): the file contains the phrase "if every clique has size least $k$", which cannot literally be correct (single vertices are cliques of size $1$).
The minimal standard correction consistent with the rest of the discussion is: "if every \emph{maximal} clique has size at least $k$".

2) QUICK LITERATURE/CONTEXT CHECK

No web lookups performed. Using only what is stated in the problem file.
The file records:
- A lower bound (Erd\H{o}s--Gallai--Tuza): $k_c(n)\ge n^{c'/\log\log n}$ for some $c'>0$.
- An upper bound claim: if every (maximal) clique has size at least $k$ then $\tau(G)\le n-(kn)^{1/2}$.
- A sharp threshold result (Bollob\'{a}s--Erd\H{o}s): if every maximal clique has at least $n+3-2\sqrt n$ vertices then $\tau(G)=1$.
I do not reproduce those literature proofs here.

3) ATTACK PLAN

Proof-track ideas:
- Translate the problem into a hitting-set problem for a family of large subsets (the maximal cliques).
- Prove bounds on transversal numbers in terms of how large the complements of maximal cliques can be.
- Seek graph-structure constraints implied by large maximal cliques (minimum degree, common intersections, etc.).

Disproof/construction ideas:
- Try to construct dense graphs whose maximal cliques are all large but arranged so that any transversal must still be linear size.

4) WORK

Lemma 611.1 (very large maximal cliques force a small transversal).
Assume every maximal clique in $G$ has size at least $n-t$ for some integer $t\ge 0$.
Then
\[
\tau(G)\le t+1.
\]

Proof.
Let $C$ be any maximal clique. By hypothesis, $|V\setminus C|\le t$.
Take any subset $T\subseteq V$ with $|T|=t+1$.
If $T\cap C=\emptyset$, then $T\subseteq V\setminus C$, forcing $|V\setminus C|\ge |T|=t+1$, contradiction.
Therefore $T$ intersects every maximal clique, so $\tau(G)\le |T|=t+1$.
\qed

Lemma 611.2 (large maximal cliques imply large minimum degree).
Assume every maximal clique in $G$ has size at least $k\ge 2$.
Then the minimum degree satisfies
\[
\delta(G)\ge k-1.
\]

Proof.
Fix a vertex $v\in V$.
By Zorn's lemma on the collection of cliques containing $v$ (ordered by inclusion), $v$ is contained in some maximal clique $C$.
By hypothesis, $|C|\ge k$.
Every other vertex in $C\setminus\{v\}$ is adjacent to $v$, hence $\deg(v)\ge |C|-1\ge k-1$.
Since $v$ was arbitrary, $\delta(G)\ge k-1$.
\qed

(FAST REALITY CHECK via computation for small $n$.)
I sampled random graphs on $n=12$ vertices, computed all maximal cliques, and then computed $\tau(G)$ exactly by brute force over all vertex subsets.
With random seed $0$ and $N=500$ samples:
- For edge probability $p=0.8$:
  * Among graphs with \emph{minimum maximal clique size} $\ge 6$ (i.e. every maximal clique has at least $n/2$ vertices), there were $98$ samples; the maximum observed $\tau(G)$ was $2$.
  * For thresholds $\ge 7,8,9$, the maximum observed $\tau(G)$ was $1$.
- For $p=0.9$:
  * Among graphs with minimum maximal clique size $\ge 6$, there were $434$ samples; the maximum observed $\tau(G)$ was $2$.
  * For thresholds $\ge 7,8,9$, the maximum observed $\tau(G)$ was $1$.

One explicit sampled example (from the $p=0.8$ run) with $n=12$, minimum maximal clique size $6$, and $\tau(G)=2$ had $54$ edges:
\[
\{(0,1),(0,2),(0,3),(0,4),(0,5),(0,6),(0,7),(0,8),(0,10),(0,11),
(1,2),(1,3),(1,4),(1,5),(1,6),(1,7),(1,8),(1,10),
(2,3),(2,4),(2,5),(2,6),(2,7),(2,8),(2,9),(2,11),
(3,4),(3,5),(3,6),(3,7),(3,8),(3,10),(3,11),
(4,5),(4,6),(4,7),(4,8),(4,10),(4,11),
(5,6),(5,7),(5,10),(5,11),
(6,7),(6,8),(6,9),(6,11),
(7,8),(7,9),(7,11),
(8,9),(8,10),(8,11),
(9,11)\}.
\]

5) VERIFICATION

- Lemma 611.1 is purely set-theoretic and uses only the size lower bound on maximal cliques; checked $t=0$ gives $\tau(G)\le 1$ (true: if all maximal cliques have size $n$, the graph is complete and $\tau=1$).
- Lemma 611.2: existence of a maximal clique containing a given vertex is standard in finite graphs (can also be shown by greedy extension), so no set-theoretic issues are needed here.
- Computational check: for $n=12$, brute force over $2^{12}=4096$ vertex subsets makes $\tau(G)$ exact.

6) FINAL

\textbf{UNRESOLVED}

(i) Strongest proved partial result:
If every maximal clique has size at least $n-t$, then $\tau(G)\le t+1$ (Lemma 611.1), and if every maximal clique has size at least $k\ge2$, then $\delta(G)\ge k-1$ (Lemma 611.2). Small-$n$ computations (for $n=12$) suggest $\tau(G)$ can be as small as $1$ or $2$ under strong clique-size constraints, but this is only experimental.

(ii) First gap (crisp statement):
For fixed $c>0$, prove or disprove that there exists a function $g_c(n)=o(n)$ such that every $n$-vertex graph whose maximal cliques all have size at least $cn$ satisfies $\tau(G)\le g_c(n)$.

(iii) Top 3 next moves:
1. Prove an upper bound on the \emph{number} (or structure) of maximal cliques under the constraint $|C|\ge cn$; then combine with probabilistic/greedy hitting-set methods.
2. Search (computationally) for graphs with larger $\tau(G)$ under the constraint "all maximal cliques $\ge cn$" for moderately larger $n$ (e.g. $n\le 20$) using ILP to maximise $\tau(G)$ subject to clique constraints.
3. Attempt an explicit construction of dense graphs where maximal cliques correspond to a block design-like family of large sets with large transversal number, while still being realisable as maximal cliques of a graph.

(iv) Minimal counterexample structure (if Question A is false):
A counterexample sequence would consist of graphs $G_n$ with $n$ vertices and a fixed $c>0$ such that every maximal clique has size at least $cn$ but $\tau(G_n)\ge \varepsilon n$ for some $\varepsilon=\varepsilon(c)>0$.
Such graphs would have minimum degree $\ge cn-1$ (Lemma 611.2), yet would have many maximal cliques arranged so that any small vertex set misses at least one of them; in set-system terms, the maximal cliques would form a family of large subsets with linear transversal number, but additionally constrained by coming from a graph.
