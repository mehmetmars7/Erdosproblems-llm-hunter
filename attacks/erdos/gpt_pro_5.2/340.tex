%Erdos problem #340
\subsection*{Erd\H{o}s problem \#340}

\noindent\textbf{1) FORMAL RESTATEMENT.}

Define the greedy Sidon sequence $A=\{a_1<a_2<\cdots\}$ by $a_1=1$ and, having chosen $a_1,\dots,a_n$, let $a_{n+1}$ be the smallest integer $>a_n$ such that $\{a_1,\dots,a_n,a_{n+1}\}$ remains a Sidon set, i.e.
\[
 a_i+a_j=a_k+a_\ell\ (1\le i\le j,\ 1\le k\le \ell)\ \Rightarrow\ \{i,j\}=\{k,\ell\}.
\]
Let $D:=A-A=\{a-b: a,b\in A\}\subseteq\mathbb{Z}$.

The problem asks:
\begin{itemize}
\item Determine the order of magnitude of $|A\cap[1,N]|$ (known bounds $N^{1/3}\le |A\cap[1,N]|\le N^{1/2}$ are stated in the problem).
\item Does $D$ contain all integers? It is stated that $22\in D$, and it is not known whether $33\in D$.
\end{itemize}

\medskip
\noindent\textbf{2) QUICK LITERATURE/CONTEXT CHECK.}
We use only the facts explicitly stated above. In particular, we do not assume any deeper theorems about Sidon sets.

\medskip
\noindent\textbf{3) ATTACK PLAN.}
\begin{itemize}
\item Prove standard counting bounds for Sidon subsets of $[1,N]$ (upper bound).
\item Prove a simple growth bound for the greedy construction (a weak lower bound on $|A\cap[1,N]|$).
\item Compute initial terms and difference-set data as a reality check.
\end{itemize}

\medskip
\noindent\textbf{4) WORK.}

\medskip
\noindent\textbf{Lemma 340.1 (Sidon implies distinct differences).}
Let $S=\{s_1<\dots<s_k\}$ be a Sidon set. Then all differences $s_i-s_j$ with $i\ne j$ are distinct.

\noindent\emph{Proof.}
Suppose $s_i-s_j=s_{i'}-s_{j'}$ with $i\ne j$ and $i'\ne j'$. Rearranging gives
\[
 s_i+s_{j'} = s_{i'}+s_j.
\]
By the Sidon property, the unordered pairs must coincide: $\{i,j'\}=\{i',j\}$. If $i=i'$ then $j=j'$, and if $i=j$ then $s_i=s_j$ contradicting $i\ne j$. Thus the only way is the trivial equality with the same ordered pair, so distinctness holds.
\hfill$\square$

\medskip
\noindent\textbf{Proposition 340.2 (a classical counting upper bound).}
If $S\subseteq\{1,2,\dots,N\}$ is Sidon with $|S|=k$, then
\[
 k(k-1)\le 2(N-1)\qquad\text{and hence}\qquad k\le \sqrt{2N}+1.
\]

\noindent\emph{Proof.}
By Lemma~340.1, the $k(k-1)$ nonzero differences $s_i-s_j$ ($i\ne j$) are all distinct integers lying in $\{-(N-1),\dots,-1,1,\dots,N-1\}$, a set of size $2(N-1)$. Therefore $k(k-1)\le 2(N-1)$, implying $k\le \sqrt{2N}+1$.
\hfill$\square$

\medskip
\noindent\textbf{Proposition 340.3 (a weak growth bound for the greedy Sidon sequence).}
Let $A=\{a_n\}$ be the greedy Sidon sequence. Then
\[
 a_{n+1} \le a_n + n^3+n^2+1\qquad\text{for all }n\ge 1.
\]
In particular $a_n\le 1+\sum_{j=1}^{n-1}(j^3+j^2+1)=O(n^4)$, so
\[
|A\cap[1,N]|\ge c\,N^{1/4}\quad\text{for some constant }c>0\text{ and all large }N.
\]

\noindent\emph{Proof.}
Fix $n$ and let $S_n:=\{a_i+a_j:1\le i\le j\le n\}$ be the set of existing sums; $|S_n|=n(n+1)/2$ and $S_n\subseteq[2,2a_n]$.
A candidate integer $x>a_n$ fails to extend the Sidon property only if at least one of the following holds:
\begin{itemize}
\item[(i)] $2x\in S_n$ (collision of $x+x$ with an old sum), or
\item[(ii)] $x+a_i\in S_n$ for some $1\le i\le n$ (collision of a new sum with an old sum).
\end{itemize}
Condition (i) forbids at most $|S_n|$ values of $x$ (namely $x=s/2$ for $s\in S_n$ with $s$ even).
For (ii), fixing $i$ and $s\in S_n$ gives a forbidden value $x=s-a_i$; thus (ii) forbids at most $n|S_n|$ values of $x$.
Therefore the set of forbidden $x$ has size at most
\[
 |S_n| + n|S_n| \le (n+1)\cdot \frac{n(n+1)}{2} \le n^3+n^2.
\]
Hence among the $n^3+n^2+1$ integers $a_n+1,\dots,a_n+(n^3+n^2+1)$ there exists some $x$ that is not forbidden, so the greedy choice satisfies $a_{n+1}\le a_n+n^3+n^2+1$.
Summing the recurrence gives $a_n=O(n^4)$, which implies $|A\cap[1,N]|=\Omega(N^{1/4})$ by inversion.
\hfill$\square$

\medskip
\noindent\textbf{FAST REALITY CHECK (computed initial terms and differences).}
Using a direct implementation of the greedy definition, the first $20$ terms are
\[
1,2,4,8,13,21,31,45,66,81,97,123,148,182,204,252,290,361,401,475.
\]
For the difference set $D=A-A$, using the first $200$ terms:
\begin{itemize}
\item $22\in D$ (e.g. $204-182=22$), matching the statement.
\item Among positive integers $\le 200$, the smallest missing difference is $33$ (i.e. $33\notin (A-A)\cap[1,200]$ for the first $200$ terms).
\end{itemize}
Also, the counts of $A\cap[1,N]$ for $N\in\{10,100,10^3,10^4,10^5\}$ were:
\[
|A\cap[1,10]|=4,\ |A\cap[1,100]|=11,\ |A\cap[1,10^3]|=27,\ |A\cap[1,10^4]|=66,\ |A\cap[1,10^5]|=161.
\]

\medskip
\noindent\textbf{5) VERIFICATION.}
Proposition~340.2 is a complete argument using only Lemma~340.1 and a pigeonhole count of differences.
Proposition~340.3 is a direct forbidden-value count for the greedy extension criterion.
The computation is an exact brute-force generation of the greedy sequence for the stated prefix lengths.

\medskip
\noindent\textbf{6) FINAL.}

\noindent\textbf{UNRESOLVED}

\smallskip
\noindent (i) \textbf{Strongest fully proved partial result obtained here.}
We proved a standard Sidon upper bound $|S\cap[1,N]|\le \sqrt{2N}+1$ for any Sidon set (Proposition~340.2) and a weak lower bound for the greedy Sidon sequence $|A\cap[1,N]|=\Omega(N^{1/4})$ (Proposition~340.3). Computationally, $22\in D$ and $33$ does not occur as a difference among the first $200$ terms.

\smallskip
\noindent (ii) \textbf{Exact first gap.}
The main gap is proving the true order of growth of the greedy Sidon sequence: bridging from the weak $\Omega(N^{1/4})$ bound here to the conjectural/asked ranges (closer to $N^{1/3}$ or $N^{1/2}$), and deciding whether every integer lies in $D$.

\smallskip
\noindent (iii) \textbf{Top 3 next moves (concrete targets).}
\begin{enumerate}
\item Improve Proposition~340.3 by sharpening the forbidden-value count to $O(n^2)$, which would yield $a_n=O(n^3)$ and hence $|A\cap[1,N]|=\Omega(N^{1/3})$.
\item Prove a structural lemma about the distribution of the new sums $a_{n+1}+a_i$ relative to old sums, to control typical gap sizes in the greedy process.
\item For the difference-set question, compute $D$ to much larger depth and identify the smallest missing difference empirically; then attempt to prove that any missing difference must have a specific residue-class or carry structure.
\end{enumerate}

\smallskip
\noindent (iv) \textbf{Minimal counterexample structure.}
If $D\ne\mathbb{Z}$, a minimal counterexample would be the smallest positive integer $m$ not in $A-A$. Such an $m$ would force that no two elements of $A$ differ by $m$, imposing a strong spacing constraint on $A$ that should reflect back into the greedy construction in a detectable way (e.g. forbidding a whole translate $A+m$).


