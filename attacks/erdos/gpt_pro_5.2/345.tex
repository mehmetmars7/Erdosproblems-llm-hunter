%Erdos problem #345
\subsection*{Erd\H{o}s problem \#345}

\noindent\textbf{1) FORMAL RESTATEMENT.}
For $A\subseteq\mathbb{N}$ define
\[
P(A):=\left\{\sum_{n\in B} n:\ B\subseteq A\ \text{finite}\right\}
\]
(the set of all finite sums of \emph{distinct} elements of $A$).
A sequence $A$ is \emph{complete} if every sufficiently large integer belongs to $P(A)$.
Define the \emph{threshold of completeness}
\[
T(A):=\min\{m\in\mathbb{N}:\ \forall n\ge m,\ n\in P(A)\}.
\]
(As written, this definition is inclusive: it requires $m\in P(A)$.)

For $k\ge 1$ let $A_k:=\{n^k: n\in\mathbb{N}\}$. The problem asks whether there are infinitely many $k$ with
\[
T(A_k) > T(A_{k+1}).
\]
The statement lists claimed values for $T(A_k)$ for $k=1,2,3,4,5$.

\medskip
\noindent\textbf{2) QUICK LITERATURE/CONTEXT CHECK.}
We do not assume the listed numerical values are correct beyond what we can verify directly by computation for small $k$.

\medskip
\noindent\textbf{3) ATTACK PLAN.}
\begin{itemize}
\item Check the definition of $T(A)$ against the quoted values for $k=2,3$ via exact subset-sum computation.
\item Record any definitional ambiguity (inclusive vs exclusive threshold).
\item Provide explicit nonrepresentability certificates for the boundary values.
\end{itemize}

\medskip
\noindent\textbf{4) WORK.}

\medskip
\noindent\textbf{Proposition 345.1 (squares: $128\notin P(A_2)$, but $129\in P(A_2)$).}
Let $A_2=\{1^2,2^2,3^2,\dots\}$. Then $128\notin P(A_2)$, but
\[
129 = 8^2+6^2+4^2+3^2+2^2\in P(A_2).
\]

\noindent\emph{Proof.}
To test membership of $128$ in $P(A_2)$, it suffices to consider the finite set of squares $\le 128$:
\[
\{1,4,9,16,25,36,49,64,81,100,121\}.
\]
There are only $2^{11}=2048$ subsets, so membership is a finite check.
A subset-sum dynamic program over these squares (equivalently, exhaustive search) finds that no subset sums to $128$.
Therefore $128\notin P(A_2)$.

For $129$, the displayed representation uses distinct squares, so $129\in P(A_2)$.
\hfill$\square$

\medskip
\noindent\textbf{Proposition 345.2 (cubes: $12758\notin P(A_3)$, but $12759\in P(A_3)$).}
Let $A_3=\{1^3,2^3,3^3,\dots\}$. Then $12758\notin P(A_3)$, while
\[
12759 = 17^3+13^3+12^3+11^3+10^3+9^3+8^3+6^3+5^3+2^3\in P(A_3).
\]

\noindent\emph{Proof.}
All cubes $\le 12758$ are $1^3,2^3,\dots,23^3$ (since $23^3=12167$ and $24^3=13824>12758$), so there are $23$ candidate summands.
Membership in $P(A_3)$ for a fixed target is again a finite subset-sum problem.
A subset-sum dynamic program confirms that no subset of $\{1^3,2^3,\dots,23^3\}$ sums to $12758$, hence $12758\notin P(A_3)$.

For $12759$, the displayed representation uses distinct cubes and sums exactly to $12759$, so $12759\in P(A_3)$.
\hfill$\square$

\medskip
\noindent\textbf{FAST REALITY CHECK (inclusive vs exclusive threshold).}
By exhaustive subset-sum computation:
\begin{itemize}
\item For squares, every integer $n\ge 129$ is representable as a sum of distinct squares for all $n\le 500$, and $128$ is the largest nonrepresentable $\le 500$.
\item For cubes, $12758$ is nonrepresentable, $12759$ is representable, and for all $n$ with $12759\le n\le 20000$ the computation found $n\in P(A_3)$.
\end{itemize}
Thus, under the inclusive definition of $T(A)$ (requiring $m\in P(A)$), the thresholds suggested by these checks are
\[
T(A_2)\ge 129,\qquad T(A_3)\ge 12759.
\]
If instead one defined $T'(A)=\min\{m: \forall n>m,\ n\in P(A)\}$ (strictly greater), then the largest missing values $128$ and $12758$ would match $T'(A_2)=128$ and $T'(A_3)=12758$. The problem statement uses the inclusive form $n\ge m$.

\medskip
\noindent\textbf{5) VERIFICATION.}
Propositions~345.1--345.2 are fully certified by finite computation on the relevant finite sets of squares/cubes, together with explicit representations for the successor integers.

\medskip
\noindent\textbf{6) FINAL.}

\noindent\textbf{UNRESOLVED}

\smallskip
\noindent (i) \textbf{Strongest fully proved partial result obtained here.}
We provided explicit counterexamples to the boundary values under the inclusive definition of $T(A)$: $128\notin P(A_2)$ and $12758\notin P(A_3)$, while $129\in P(A_2)$ and $12759\in P(A_3)$ (Propositions~345.1--345.2). We also identified a likely inclusive/exclusive threshold ambiguity.

\smallskip
\noindent (ii) \textbf{Exact first gap.}
We did not prove completeness of $A_k=\{n^k\}$ for general $k$, nor did we establish or refute the existence of infinitely many $k$ with $T(A_k)>T(A_{k+1})$. The first gap is obtaining any general mechanism to compare $T(A_k)$ and $T(A_{k+1})$.

\smallskip
\noindent (iii) \textbf{Top 3 next moves (concrete targets).}
\begin{enumerate}
\item Clarify (and fix) the intended convention for $T(A)$ (inclusive vs exclusive), then verify the quoted $T(A_k)$ values for $k=4,5$ with independent computation/certificates.
\item Prove a general sufficient condition for completeness for power sequences (e.g. using additive-combinatorial covering arguments) and extract an explicit upper bound on $T(A_k)$.
\item Develop lower-bound constructions: produce explicit integers not representable as sums of distinct $k$th powers, to force $T(A_k)$ large and allow comparison with $T(A_{k+1})$.
\end{enumerate}

\smallskip
\noindent (iv) \textbf{Minimal counterexample structure.}
A counterexample to ``infinitely many $k$ with $T(A_k)>T(A_{k+1})$'' would require that from some point on, the thresholds are nonincreasing: $T(A_k)\le T(A_{k+1})$ for all large $k$. This would suggest a monotonicity phenomenon in representability by distinct powers that is not explained by current small-$k$ data.


