
\noindent\textbf{FORMAL RESTATEMENT.}
Let $\mathfrak c$ denote the initial ordinal of cardinality $2^{\aleph_0}$ (the cardinality of $\mathbb R$). Let $\beta$ be a countable ordinal and let $n$ be an integer with $2\le n<\omega$.

For an ordinal $\kappa$ and integer $r\ge 1$, write $[\kappa]^r$ for the set of $r$-element subsets of $\kappa$.
For $\kappa,\beta$ ordinals and $n\in\mathbb N$, the partition relation
\[\kappa\to(\beta,n)_2^3\]
means:

\emph{for every} coloring $c:[\kappa]^3\to\{0,1\}$, \emph{either}
\begin{itemize}
\item there exists $H\subseteq\kappa$ of order type $\beta$ such that $c$ is constantly $0$ on $[H]^3$, \emph{or}
\item there exists $K\subseteq\kappa$ with $|K|=n$ such that $c$ is constantly $1$ on $[K]^3$.
\end{itemize}

Question: is it true that $\mathfrak c\to(\beta,n)_2^3$ for every countable ordinal $\beta$ and every integer $2\le n<\omega$?

\medskip
\noindent\textbf{QUICK LITERATURE/CONTEXT CHECK.}
The problem statement records that Erd\H{o}s and Rado proved
\[\mathfrak c\to(\omega+n,4)_2^3\qquad\text{for every integer }2\le n<\omega.
\]
No additional literature claims are used here.

\medskip
\noindent\textbf{ATTACK PLAN.}
\emph{Proof track ideas.}
(1) Use monotonicity and the known Erd\H{o}s--Rado instances as base cases; attempt to extend from $\beta=\omega+n$ to general countable $\beta$ by transfinite induction on $\beta$.
(2) Try to rework the Erd\H{o}s--Rado proof to handle larger countable order types by a refined fusion/diagonalization.

\emph{Disproof track ideas.}
(1) Attempt to build an explicit $2$-coloring of $[\mathfrak c]^3$ that destroys all $0$-homogeneous sets of a given countable order type while also avoiding $1$-homogeneous $n$-sets.
(2) Check whether the statement might depend on additional axioms (e.g. CH vs. not-CH) by testing known ``pathological'' colorings.

I did not obtain a full proof or counterexample. The WORK section records basic (but problem-specific) reductions.

\medskip
\noindent\textbf{WORK.}

\medskip
\noindent\textbf{Lemma 70.1 (easy cases $n=2$ and $n=3$).}
Let $\kappa$ be an ordinal and $\beta\le\kappa$ an ordinal.
\begin{itemize}
\item[(a)] $\kappa\to(\beta,2)_2^3$ always holds.
\item[(b)] $\kappa\to(\beta,3)_2^3$ always holds.
\end{itemize}
In particular, since every countable ordinal $\beta$ satisfies $\beta<\mathfrak c$, the desired relation holds for all countable $\beta$ when $n\in\{2,3\}$.

\noindent\textbf{Proof.}
(a) Fix any coloring $c:[\kappa]^3\to\{0,1\}$. Choose any $2$-element subset $K\subseteq\kappa$. Then $[K]^3=\varnothing$, so $c$ is vacuously constantly $1$ on $[K]^3$. Thus the second alternative in the definition of $\kappa\to(\beta,2)_2^3$ always holds.

(b) Fix any coloring $c:[\kappa]^3\to\{0,1\}$. If there exists a triple $\{x,y,z\}\in[\kappa]^3$ with $c(\{x,y,z\})=1$, then letting $K=\{x,y,z\}$ gives $|K|=3$ and $c$ is constantly $1$ on $[K]^3$ (since $[K]^3$ contains exactly that one triple). Otherwise, $c$ colors every triple in $[\kappa]^3$ by $0$. In this case, since $\beta\le\kappa$, there exists $H\subseteq\kappa$ of order type $\beta$ (for example, take the first $\beta$ ordinals below $\kappa$), and then $c$ is constantly $0$ on $[H]^3$ by assumption. Hence one of the two alternatives always holds, proving $\kappa\to(\beta,3)_2^3$.
\hfill$\square$

\medskip
\noindent\textbf{Lemma 70.2 (monotonicity in parameters).}
Let $\kappa$ be an ordinal.
\begin{itemize}
\item[(a)] If $\kappa\to(\beta,n)_2^3$ holds and $\beta'\le\beta$, $n'\le n$, then $\kappa\to(\beta',n')_2^3$ holds.
\item[(b)] If $\kappa\to(\beta,n)_2^3$ holds and $\kappa'\ge \kappa$ is an ordinal, then $\kappa'\to(\beta,n)_2^3$ holds.
\end{itemize}

\noindent\textbf{Proof.}
(a) Fix a coloring $c:[\kappa]^3\to\{0,1\}$. By $\kappa\to(\beta,n)_2^3$, either there is a $0$-homogeneous set $H$ of order type $\beta$, or a $1$-homogeneous set $K$ of size $n$.

If the first alternative occurs, then $H$ contains a subset $H'\subseteq H$ of order type $\beta'$ (because $\beta'\le\beta$ and order types are inherited by appropriate subsets). Since $c$ is constantly $0$ on $[H]^3$, it is also constantly $0$ on $[H']^3$.

If the second alternative occurs, choose any subset $K'\subseteq K$ with $|K'|=n'$. Since $[K']^3\subseteq[K]^3$ and $c$ is constantly $1$ on $[K]^3$, it is also constantly $1$ on $[K']^3$.

Thus $\kappa\to(\beta',n')_2^3$ holds.

(b) Fix a coloring $c':[\kappa']^3\to\{0,1\}$. Restrict $c'$ to the subset $[\kappa]^3\subseteq[\kappa']^3$ (identifying $\kappa$ with the initial segment $\{\alpha<\kappa\}\subseteq\kappa'$). Applying $\kappa\to(\beta,n)_2^3$ to this restricted coloring yields either a $0$-homogeneous $H\subseteq\kappa$ of order type $\beta$, or a $1$-homogeneous $K\subseteq\kappa$ of size $n$. In either case, viewing $H$ or $K$ as subsets of $\kappa'$ gives the required homogeneous set for $c'$. Hence $\kappa'\to(\beta,n)_2^3$.
\hfill$\square$

\medskip
\noindent\textbf{Corollary 70.1 (what the recorded Erd\H{o}s--Rado theorem immediately implies).}
For each integer $m\ge 2$, the recorded statement $\mathfrak c\to(\omega+m,4)_2^3$ implies
\[\mathfrak c\to(\beta,4)_2^3\quad\text{for every ordinal }\beta\le \omega+m.
\]

\noindent\textbf{Proof.}
Apply Lemma 70.2(a) with $\beta'\le\omega+m$ and $n'=4$.\hfill$\square$

\medskip
\noindent\textbf{FAST REALITY CHECK (small parameter sanity).}
For this partition relation, the only genuinely nontrivial finite parameter is $n\ge 4$. Lemma 70.1 shows the statement is automatic for $n=2$ and reduces to a trivial dichotomy for $n=3$. Thus, in the main question, the first new difficulty begins at $n=4$ and countable $\beta$ beyond $\omega+\text{(finite)}$.

\medskip
\noindent\textbf{VERIFICATION.}
\begin{itemize}
\item Lemma 70.1(a): checked that $[K]^3$ is empty for $|K|=2$, so ``$1$-homogeneous'' is vacuous.
\item Lemma 70.1(b): checked the dichotomy ``some triple is colored $1$'' vs ``all triples colored $0$''; in the latter case, the existence of a subset of order type $\beta$ uses only $\beta\le\kappa$.
\item Lemma 70.2: verified both parameter reductions by subset selection and restriction to an initial segment.
\end{itemize}

\medskip
\noindent\textbf{FINAL: \textbf{UNRESOLVED}.}
\begin{itemize}
\item[(i)] \emph{Strongest proved partial result.} The relation holds for all countable $\beta$ when $n\in\{2,3\}$ (Lemma 70.1). Moreover, the recorded Erd\H{o}s--Rado theorem implies $\mathfrak c\to(\beta,4)_2^3$ for every $\beta\le \omega+m$ (for each fixed $m\ge 2$) by monotonicity (Corollary 70.1).
\item[(ii)] \emph{First gap (crisp).} Decide whether $\mathfrak c\to(\omega^2,4)_2^3$ holds (already beyond the $\omega+\text{finite}$ range stated in the problem).
\item[(iii)] \emph{Top 3 next moves.}
(1) Attempt an induction on countable ordinals $\beta$ proving $\mathfrak c\to(\beta,4)_2^3$, identifying exactly where the Erd\H{o}s--Rado proof for $\omega+n$ breaks.
(2) Try to construct an explicit $2$-coloring of $[\mathfrak c]^3$ that avoids $0$-homogeneous sets of some fixed order type (e.g. $\omega^2$) while also avoiding $1$-homogeneous $4$-sets.
(3) Explore whether additional axioms (e.g. CH) affect the statement by testing known ``canonical'' colorings on $[\omega_1]^3$ and attempting to lift them to size $\mathfrak c$.
\item[(iv)] \emph{Minimal counterexample structure.} A counterexample would be a coloring $c:[\mathfrak c]^3\to\{0,1\}$ such that (a) every subset of $\mathfrak c$ of order type $\beta$ contains a triple colored $1$ (so it is not $0$-homogeneous), and simultaneously (b) every $n$-element subset contains a triple colored $0$ (so it is not $1$-homogeneous).
\end{itemize}

