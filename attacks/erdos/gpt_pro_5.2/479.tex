
1) “FORMAL RESTATEMENT”

Fix an integer k != 1. The question is whether there exist infinitely many integers n>=2 such that

    2^n ≡ k (mod n),

i.e. n divides (2^n - k).

2) “QUICK LITERATURE/CONTEXT CHECK”

The problem statement notes that the case k=2^i (i>=1) and k=-1 is known (attributed to Graham-Lehmer-Lehmer, with a later note by Tang), while the general case is open. A quick check of the Erdos Problems page for #479 shows it is listed as OPEN (as of mid-Jan 2026). I do not rely on any external paper; I give self-contained proofs for the k=-1 family and for infinitely many k=2^i using standard number theory facts.

3) “ATTACK PLAN”

- Produce explicit infinite families of solutions for certain k (sanity check that the phenomenon is real).
- For general k, understand constraints from prime factors of n: if p|n then 2^n ≡ k (mod p), giving conditions on the order of 2 mod p.
- Try to “lift” solutions by building n with controlled prime factors (Pratt-tree / Korselt-type constructions).

4) “WORK”

Lemma 479.1 (k=2 has infinitely many solutions).
For every odd prime p, we have 2^p ≡ 2 (mod p). Hence k=2 has infinitely many solutions.

Proof.
By Fermat's little theorem, for odd prime p we have 2^{p-1} ≡ 1 (mod p). Multiply both sides by 2 to get 2^p ≡ 2 (mod p).  QED.

Lemma 479.2 (k=-1 has an explicit infinite family).
For every integer m>=1, n=3^m satisfies 2^n ≡ -1 (mod n).

Proof.
We will show the stronger statement that for every m>=1,

    2^{3^{m-1}} ≡ -1 (mod 3^m).

Then raising both sides to the third power yields 2^{3^m} ≡ (-1)^3 = -1 (mod 3^m), which is exactly 2^n ≡ -1 (mod n) with n=3^m.

To prove 2^{3^{m-1}} ≡ -1 (mod 3^m), we compute the 3-adic valuation of 2^{3^{m-1}}+1 using the special-case LTE (Lifting The Exponent) identity:

Claim: For any odd integer t>=1,
    v_3(2^t + 1) = v_3(2+1) + v_3(t).

Justification of the claim (standard LTE for p=3, a=2, b=1, t odd): since 3 divides 2+1 and t is odd, the exponent-lifting formula gives v_3(2^t+1^t)=v_3(2+1)+v_3(t).

Apply this with t = 3^{m-1}. Then v_3(t)=m-1 and v_3(2+1)=1, so

    v_3(2^{3^{m-1}} + 1) = 1 + (m-1) = m,

meaning 3^m divides 2^{3^{m-1}}+1, i.e. 2^{3^{m-1}} ≡ -1 (mod 3^m). This completes the proof.  QED.

Lemma 479.3 (infinitely many solutions for each k=2^i).
Fix i>=1 and put k=2^i. Let r be the odd part of i (so i=2^a r with r odd). Let M := r. Then for every prime p satisfying

    p ≡ 1 (mod M)  and  p does not divide i,

the integer n := i p satisfies 2^n ≡ 2^i (mod n). In particular, assuming the standard theorem that there are infinitely many primes in the arithmetic progression 1 (mod M) (Dirichlet), there are infinitely many such n.

Proof.
Let n=i p. First, modulo p we have by Fermat:

    2^{ip} = (2^p)^i ≡ 2^i (mod p),

since 2^p ≡ 2 (mod p).

Now work modulo i. We need i | (2^{ip} - 2^i) = 2^i(2^{i(p-1)} - 1). The power of 2 dividing i is at most 2^i, so the 2-part of i divides 2^i automatically. For the odd part r of i, we need

    2^{i(p-1)} ≡ 1 (mod r).

Because p ≡ 1 (mod r), we have r | (p-1), hence i(p-1) is a multiple of r. Also, since gcd(2,r)=1, Euler's theorem gives 2^{phi(r)} ≡ 1 (mod r), so certainly 2^{i(p-1)} ≡ 1 (mod r) whenever phi(r) divides i(p-1). A sufficient (simple) condition is that r divides (p-1), which we assumed; since 2^{r} ≡ 1 (mod r) is not always true, the clean way is to take p ≡ 1 (mod phi(r)) instead. Thus, replacing M by phi(r), we get:

If p ≡ 1 (mod phi(r)) and p does not divide i, then 2^{i(p-1)} ≡ 1 (mod r), hence 2^{ip} ≡ 2^i (mod i).

Finally, if gcd(i,p)=1 (i.e. p does not divide i), then congruence modulo i and modulo p implies congruence modulo ip=n. Therefore 2^n ≡ 2^i (mod n).

This gives an infinite family of solutions n=i p provided there are infinitely many primes p ≡ 1 (mod phi(r)), which is guaranteed by Dirichlet's theorem on primes in arithmetic progressions.

(Concrete small check: for i=2, phi(r)=1, so every odd prime p works and n=2p gives infinitely many solutions for k=4; for i=3, phi(r)=2, so every odd prime p works and n=3p gives infinitely many solutions for k=8.)  QED.

FAST REALITY CHECK (local computation).
I searched exhaustively for solutions with n<=5000 for a few k. The first solutions found:

- k=2: n = 2,3,5,7,11,... (all odd primes appear, consistent with Lemma 479.1).
- k=-1: n = 3,9,27,81,171,243,513,729,... (in particular 3^m works, consistent with Lemma 479.2).
- k=3: no solutions with n<=5000 (consistent with the statement that the first solution is enormous).

5) “VERIFICATION”

- Lemma 479.2 was checked numerically for m<=10 by direct modular exponentiation and matches.
- Lemma 479.3 is logically self-contained except for the infinitude of primes in a given progression; for that last step I invoked Dirichlet's theorem.

6) FINAL

UNRESOLVED

(i) Strongest proved partial result:
We have explicit infinite families for k=2 (all odd primes p) and k=-1 (all n=3^m). Moreover, for every i>=1 we can construct infinitely many solutions for k=2^i of the form n=i p with p prime in a suitable progression (using standard primes-in-progressions input).

(ii) First gap (crisp statement):
Prove or disprove: for a fixed k not of the special forms above (e.g. k=3), there are infinitely many n such that 2^n ≡ k (mod n).

(iii) Top 3 next moves (concrete):
1. Attempt to generalize the k=-1 construction: find a prime power q^m and an exponent pattern forcing 2^{q^m} ≡ k (mod q^m) via LTE.
2. Develop a lifting mechanism: from one solution n, build a larger solution n' by adjoining a new prime factor p chosen so that 2^{n'} ≡ k (mod p) and also (mod n).
3. For specific k (starting with k=3), run targeted searches for n with controlled factorization (e.g. n a product of a few primes) and look for patterns suggesting an infinite family.

(iv) Minimal counterexample structure (if the answer is “no” for some k):
A value k!=1 and a proof that the set { n>=2 : n | (2^n - k) } is finite; equivalently, all sufficiently large n fail the congruence 2^n ≡ k (mod n).

