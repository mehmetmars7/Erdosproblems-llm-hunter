% Erdos Problem #1085

\subsection*{Erd\H{o}s Problem 1085 (unit distances without separation)}

\subsubsection*{FORMAL RESTATEMENT}
For integers \(d\ge 1\) and \(n\in\mathbb{N}\), let \(f_d(n)\) be the minimum integer such that every set \(P\subset\mathbb{R}^d\) with \(|P|=n\) determines at most \(f_d(n)\) unordered pairs \(\{x,y\}\subset P\) with \(\|x-y\|=1\).
Equivalently, \(f_d(n)=\max\{\#\{\{x,y\}\subset P:\|x-y\|=1\}: P\subset\mathbb{R}^d,\ |P|=n\}\).
The problem asks for estimates of \(f_d(n)\), especially for \(d=2,3\).

\subsubsection*{QUICK LITERATURE/CONTEXT CHECK}
The problem statement records the best-known bounds in \(d=2\) and \(d=3\) and the asymptotically tight quadratic behaviour for \(d\ge 4\) (including the Lenz construction and an Erd\H{o}s-Stone based upper bound for \(d\ge 4\)). I do not add other literature.

\subsubsection*{ATTACK PLAN}
\begin{itemize}
\item \textbf{Proof track (partial).} Give a completely self-contained quadratic \emph{lower} bound construction in \(d\ge 4\) (Lenz-type: points on mutually orthogonal circles), and a general \emph{upper} bound from forbidding large equidistant cliques plus Tur\'an.
\item \textbf{Disproof track.} Not applicable (estimation problem).
\end{itemize}

\subsubsection*{WORK}

\paragraph{Fast reality check.} For \(d=4\), taking two circles in orthogonal planes of radius \(1/\sqrt2\) and placing \(m\) points on each gives \(n=2m\) points with exactly \(m^2\) cross pairs at distance \(1\). Choosing \(m=3\) equally spaced points on each circle, a brute-force check gives \(9\) cross pairs and the set of cross squared distances is exactly \(\{1\}\) (within-circle squared distances are \(\{3/2\}\)).

\begin{lemma}[Maximum size of an equidistant set]
In \(\mathbb{R}^d\), there do not exist \(d+2\) points with all pairwise distances equal to \(1\). Equivalently, any clique in a unit-distance graph in \(\mathbb{R}^d\) has size at most \(d+1\).
\end{lemma}

\begin{proof}
Suppose for contradiction that \(p_1,\dots,p_{d+2}\in\mathbb{R}^d\) satisfy \(\|p_i-p_j\|=1\) for all \(i\ne j\).
Translate so that \(p_{d+2}=0\). Then \(\|p_i\|=1\) for \(i\le d+1\), and for \(1\le i<j\le d+1\),
\[
1=\|p_i-p_j\|^2=\|p_i\|^2+\|p_j\|^2-2p_i\cdot p_j=1+1-2p_i\cdot p_j,
\]
so \(p_i\cdot p_j=1/2\).
Let \(G\) be the \((d+1)\times(d+1)\) Gram matrix \(G=(p_i\cdot p_j)_{1\le i,j\le d+1}\). Then
\(G\) has diagonal entries \(1\) and off-diagonal entries \(1/2\), so
\[
G=\frac12 I_{d+1}+\frac12 J_{d+1},
\]
where \(J_{d+1}\) is the all-ones matrix.
The eigenvalues of \(J_{d+1}\) are \(d+1\) (once) and \(0\) (multiplicity \(d\)), hence the eigenvalues of \(G\) are
\(\tfrac12+\tfrac12(d+1)=\tfrac{d+2}{2}\) (once) and \(\tfrac12\) (multiplicity \(d\)).
Thus \(\mathrm{rank}(G)=d+1\).
But \(\mathrm{rank}(G)\le d\) because \(G\) is the Gram matrix of \(d+1\) vectors in \(\mathbb{R}^d\). Contradiction.
Therefore no such \(d+2\) points exist.
\end{proof}

\begin{lemma}[A general Tur\'an-type upper bound]
For any \(d\ge 1\) and \(n\),
\[
 f_d(n)\le \left(1-\frac{1}{d+1}\right)\frac{n^2}{2}.
\]
\end{lemma}

\begin{proof}
Given \(P\subset\mathbb{R}^d\) of size \(n\), let \(G\) be its unit-distance graph.
By the previous lemma, \(G\) is \(K_{d+2}\)-free.
Tur\'an's theorem for \(K_{r}\)-free graphs states that a \(K_r\)-free graph on \(n\) vertices has at most
\(\left(1-\frac{1}{r-1}\right)\frac{n^2}{2}\) edges.
Taking \(r=d+2\) yields
\(|E(G)|\le \left(1-\frac{1}{d+1}\right)\frac{n^2}{2}\).
Since \(f_d(n)\) is the maximum possible \(|E(G)|\), the same bound holds for \(f_d(n)\).
\end{proof}

\begin{lemma}[Lenz-type construction for \(d\ge 4\)]
Let \(d\ge 4\), and set \(p:=\lfloor d/2\rfloor\ge 2\). Then for all \(n\) there exists a set \(P\subset\mathbb{R}^d\) of \(|P|=n\) points with
\[
\#\{\{x,y\}\subset P: \|x-y\|=1\}\ \ge\ \frac{p-1}{2p}n^2 - O(p).
\]
In particular, \(f_d(n)\ge \frac{p-1}{2p}n^2-O(1)\).
\end{lemma}

\begin{proof}
Embed \(p\) pairwise orthogonal 2-dimensional linear subspaces \(V_1,\dots,V_p\subset\mathbb{R}^d\) (possible since \(2p\le d\)).
In each \(V_i\), consider the circle
\(C_i:=\{x\in V_i: \|x\|=1/\sqrt2\}\), centered at the origin.
For any \(x\in C_i\) and \(y\in C_j\) with \(i\ne j\), orthogonality gives \(x\cdot y=0\), hence
\[
\|x-y\|^2=\|x\|^2+\|y\|^2-2x\cdot y = \frac12+\frac12-0=1.
\]
Thus \emph{every} pair of points chosen from two different circles is at unit distance.

Now distribute \(n\) points among the circles: choose integers \(n_1,\dots,n_p\ge 0\) with \(\sum_i n_i=n\), and pick any \(n_i\) distinct points on each \(C_i\).
The number of unit-distance pairs between different circles equals
\(\sum_{1\le i<j\le p} n_i n_j = \frac12\left(n^2-\sum_{i=1}^p n_i^2\right)\).
This is maximized when the \(n_i\) are as equal as possible, giving
\(\sum_i n_i^2\le \lceil n/p\rceil^2+(p-1)\lfloor n/p\rfloor^2\le n^2/p + O(p)\).
Therefore
\[
\sum_{i<j}n_i n_j \ge \frac12\left(n^2-\frac{n^2}{p}\right)-O(p)=\frac{p-1}{2p}n^2-O(p).
\]
Any additional unit pairs within a single circle can only increase the count, so this lower bound holds for the full unit-distance count.
\end{proof}

\subsubsection*{VERIFICATION}
\begin{itemize}
\item The equidistant-set lemma is a standard rank/Gram-matrix argument; all eigenvalues computed explicitly.
\item The Lenz construction uses only orthogonality of subspaces and radii \(1/\sqrt2\); the distance computation is exact.
\item The Tur\'an bound is not sharp for \(d\ge 4\) compared to the conjectured/recorded asymptotic \((p-1)/(2p)\), but it is rigorous and self-contained.
\end{itemize}

\subsubsection*{FINAL}
\textbf{UNRESOLVED.}
\begin{enumerate}
\item[(i)] \textbf{Strongest proved partial result here:} for \(d\ge 4\) with \(p=\lfloor d/2\rfloor\),
\(\frac{p-1}{2p}n^2-O(1)\le f_d(n)\le \left(1-\frac{1}{d+1}\right)\frac{n^2}{2}\).
\item[(ii)] \textbf{First gap:} for \(d=2,3\), close the gap between known lower and upper bounds (as recorded in the problem statement). For \(d\ge 4\), prove the sharp asymptotic upper constant \((p-1)/(2p)\) without appealing to results not already in the file.
\item[(iii)] \textbf{Top 3 next moves:}
(1) understand which complete multipartite graphs are forbidden in unit-distance graphs in \(\mathbb{R}^d\);
(2) attempt to show near-extremal graphs must be approximately \(p\)-partite, matching the Lenz construction;
(3) computational experiments in \(d=4\) to see extremal configurations for moderate \(n\) beyond the circle construction.
\item[(iv)] \textbf{Minimal counterexample structure (to the sharp quadratic constant in \(d\ge 4\)):}
a unit-distance graph on \(n\) vertices with edge density exceeding \((p-1)/(2p)+\varepsilon\), which would necessarily force a rich family of geometric constraints (many large cliques/near-cliques) contradicting Euclidean realizability.
\end{enumerate}


