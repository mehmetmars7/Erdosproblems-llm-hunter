\section*{Problem \#243 (Rational reciprocal sum and Sylvester recurrence)}

\subsection*{FORMAL RESTATEMENT}
Let $a_1<a_2<\cdots$ be a strictly increasing sequence of (nonzero) integers such that
\[
\lim_{n\to\infty}\frac{a_n}{a_{n-1}^2}=1
\quad\text{and}\quad
\sum_{n=1}^{\infty}\frac{1}{a_n}\in\mathbb{Q}.
\]
The conjectured conclusion is that there exists $n_0$ such that for all $n\ge n_0$,
\[
 a_n=a_{n-1}^2-a_{n-1}+1.
\]

\subsection*{QUICK LITERATURE/CONTEXT CHECK}
The prompt records the Erd\H{o}s--Straus (1964) result: under the same hypotheses, if the recurrence fails infinitely often, then
\[
\limsup_{n\to\infty} \frac{[a_1,\dots,a_n]}{a_{n+1}}\left(\frac{a_n^2}{a_{n+1}}-1\right)>0,
\]
where $[\cdot]$ denotes least common multiple. Sequences satisfying $a_n=a_{n-1}^2-a_{n-1}+1$ are (a shift of) Sylvester-type sequences.

\subsection*{ATTACK PLAN}
\textbf{Proof track.}
\begin{itemize}[leftmargin=2.2em]
\item Use rationality of the total sum to force strong divisibility constraints on denominators of tails.
\item Combine with the near-squaring growth to show the tail is ``as small as possible,'' potentially forcing a deterministic recurrence.
\item Compare with the exact telescoping identity satisfied by the Sylvester recurrence.
\end{itemize}
\textbf{Disproof track.}
\begin{itemize}[leftmargin=2.2em]
\item Attempt to construct a non-Sylvester sequence with $a_{n+1}=a_n^2+o(a_n^2)$ but with a carefully tuned rational reciprocal sum.
\item Show that any such perturbation would typically change the sum by an irrational amount (hard), or else violates necessary lcm/tail bounds.
\end{itemize}

\subsection*{WORK}

\begin{lemma}[Telescoping identity for the Sylvester recurrence]
Assume $a_{n+1}=a_n^2-a_n+1$ for all $n\ge 1$ and $a_1\ge 2$. Then for every $n\ge 1$,
\[
\frac{1}{a_n}=\frac{1}{a_n-1}-\frac{1}{a_{n+1}-1}.
\]
Consequently, the series converges and
\[
\sum_{n=1}^{\infty}\frac{1}{a_n}=\frac{1}{a_1-1}\in\mathbb{Q}.
\]
\end{lemma}

\begin{proof}
From $a_{n+1}-1=a_n^2-a_n=a_n(a_n-1)$ we have
\[
\frac{1}{a_n-1}-\frac{1}{a_{n+1}-1}
=\frac{1}{a_n-1}-\frac{1}{a_n(a_n-1)}
=\frac{a_n-1}{a_n(a_n-1)}
=\frac{1}{a_n}.
\]
Summing from $n=1$ to $N$ telescopes:
\[
\sum_{n=1}^{N}\frac{1}{a_n}
=\frac{1}{a_1-1}-\frac{1}{a_{N+1}-1}.
\]
Since $a_{N+1}\to\infty$, the last term tends to $0$ and the sum tends to $1/(a_1-1)$.
\end{proof}


\begin{lemma}[A general ``tail lower bound'' from rationality]
Let $a_1<a_2<\cdots$ be positive integers and suppose
\[
S:=\sum_{n=1}^{\infty}\frac{1}{a_n}=\frac{p}{q}\in\mathbb{Q}
\qquad(p,q\in\mathbb{Z},\ \gcd(p,q)=1,\ q\ge 1).
\]
Let $L_n:=\mathrm{lcm}(a_1,\dots,a_n)$ and $s_n:=\sum_{k=1}^{n}\frac{1}{a_k}$. Then the tail $t_n:=S-s_n$ satisfies
\[
 t_n>0\quad\text{and}\quad t_n\ge \frac{1}{qL_n}\qquad\text{for all }n\ge 1.
\]
\end{lemma}

\begin{proof}
Positivity: since the $a_n$ are positive and the series converges, $t_n=\sum_{k\ge n+1}1/a_k>0$.

Denominator bound: each $1/a_k$ has denominator dividing $L_n$ for $k\le n$, hence $s_n$ is a rational number with denominator dividing $L_n$. Therefore $t_n=S-s_n$ is a rational number with denominator dividing $qL_n$. Any positive rational with denominator dividing $qL_n$ is at least $1/(qL_n)$.
\end{proof}


\begin{lemma}[A crude tail upper bound under near-squaring]
Assume $a_1<a_2<\cdots$ are positive integers and
\[
\lim_{n\to\infty}\frac{a_{n+1}}{a_n^2}=1.
\]
Then for every $\varepsilon\in(0,1)$ there exists $n_0$ such that for all $n\ge n_0$,
\[
\sum_{k\ge n+1}\frac{1}{a_k}\le \frac{1}{a_{n+1}}\left(1+\frac{1}{(1-\varepsilon)a_{n+1}}+\frac{1}{(1-\varepsilon)^3 a_{n+1}^3}+\cdots\right)
\le \frac{2}{a_{n+1}}.
\]
\end{lemma}

\begin{proof}
Fix $\varepsilon\in(0,1)$. By the limit hypothesis there is $n_0$ such that for all $m\ge n_0$,
\begin{equation}
\label{eq:near-square}
 a_{m+1}\ge (1-\varepsilon)a_m^2.
\end{equation}
Iterating \eqref{eq:near-square} gives for $j\ge 1$,
\[
 a_{n+1+j}
\ge (1-\varepsilon)^{1+2+\cdots+2^{j-1}}\, a_{n+1}^{2^{j}}.
\]
(Indeed, each squaring step doubles exponents and multiplies by an extra $(1-\varepsilon)$.) Hence
\[
\frac{1}{a_{n+1+j}}
\le \frac{1}{(1-\varepsilon)^{1+2+\cdots+2^{j-1}}\, a_{n+1}^{2^{j}}}.
\]
Therefore
\[
\sum_{k\ge n+1}\frac{1}{a_k}
=\frac{1}{a_{n+1}}+\sum_{j\ge 1}\frac{1}{a_{n+1+j}}
\le \frac{1}{a_{n+1}}\left(1+\frac{1}{(1-\varepsilon)a_{n+1}}+\frac{1}{(1-\varepsilon)^3 a_{n+1}^3}+\cdots\right).
\]
For fixed $\varepsilon$, the bracketed series is dominated by
\(
1+\sum_{j\ge 1}\frac{1}{a_{n+1}^{2^j-1}}
\le 1+\sum_{m\ge 1}\frac{1}{a_{n+1}^{m}}=\frac{a_{n+1}}{a_{n+1}-1},
\)
once $a_{n+1}$ is large, and in particular is $<2$ for all $a_{n+1}\ge 2$. Thus for all sufficiently large $n$ the total tail is at most $2/a_{n+1}$.
\end{proof}


\begin{corollary}[A necessary lcm growth condition]
Under the hypotheses of Lemmas 7 and 8, writing $S=p/q$ in lowest terms, there exists $n_1$ such that for all $n\ge n_1$,
\[
 a_{n+1}\le 2q\,\mathrm{lcm}(a_1,\dots,a_n).
\]
In particular,
\[
\limsup_{n\to\infty}\frac{\mathrm{lcm}(a_1,\dots,a_n)}{a_{n+1}}\ge \frac{1}{2q}>0.
\]
\end{corollary}

\begin{proof}
Combine the lower bound $t_n\ge 1/(qL_n)$ from Lemma 7 with the upper bound $t_n\le 2/a_{n+1}$ from Lemma 8, valid for all sufficiently large $n$.
\end{proof}

\subsection*{VERIFICATION}
\begin{itemize}[leftmargin=2.2em]
\item Lemma 6 is a direct algebraic telescoping identity and is exact.
\item Lemma 7 uses only that $s_n$ has denominator dividing $L_n$ and that a positive rational with denominator $\le qL_n$ is at least $1/(qL_n)$.
\item Lemma 8: the only delicate point is the domination of the tail by a geometric series; the iterated bound shows successive reciprocals decay at least like $a_{n+1}^{-2^j}$ up to a fixed $(1-\varepsilon)$-factor, so the tail is indeed $O(1/a_{n+1})$.
\end{itemize}

\subsection*{FINAL}
\textbf{UNRESOLVED.}

\begin{itemize}[leftmargin=2.2em]
\item[(i)] \textbf{Partial results obtained:}
  \begin{itemize}[leftmargin=2.2em]
  \item Proved the exact telescoping identity for the Sylvester recurrence, yielding a rational reciprocal sum.
  \item Derived a general necessary condition for rationality under near-squaring growth: the lcm $[a_1,\dots,a_n]$ must be within a constant factor of $a_{n+1}$ infinitely often (Corollary 9).
  \end{itemize}
\item[(ii)] \textbf{Strongest obstruction / first gap:}
  Corollary 9 is far weaker than forcing the exact recurrence $a_{n+1}=a_n^2-a_n+1$. Bridging this requires a mechanism turning approximate tail/lcm constraints into an exact deterministic recurrence.
\item[(iii)] \textbf{What would likely finish:}
  Strengthening the lcm/tail comparison to an equality-type statement (as happens in Lemma 6) or leveraging the Erd\H{o}s--Straus limsup criterion to deduce eventual integrality constraints that uniquely select the Sylvester recurrence.
\item[(iv)] \textbf{Confidence assessment:}
  High confidence in Lemmas 6--8 and Corollary 9; no claim about the conjectured eventual recurrence.
\end{itemize}

\subsection*{COMPLETION ESTIMATE}
25\%.

