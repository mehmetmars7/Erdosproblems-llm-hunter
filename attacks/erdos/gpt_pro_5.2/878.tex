% Erdos Problem #878

\noindent\textbf{FORMAL RESTATEMENT.}
Let $n\ge 2$ with prime factorization
\[
 n=\prod_{i=1}^t p_i^{k_i},
\]
where the $p_i$ are distinct primes.
For each $i$, define $\ell_i\in\mathbb N$ by
\[
 p_i^{\ell_i}\le n < p_i^{\ell_i+1},
\]
i.e. $\ell_i=\lfloor\log_{p_i} n\rfloor$.
Define
\[
 f(n)=\sum_{i=1}^t p_i^{\ell_i}.
\]

Define $F(n)$ as follows. Let $a_1,\dots,a_t$ be integers with $1\le a_i\le n$, such that:
\begin{itemize}
\item $\gcd(a_i,a_j)=1$ for $i\ne j$;
\item every prime factor of each $a_i$ is among $\{p_1,\dots,p_t\}$.
\end{itemize}
(Here $a_i=1$ is allowed, since it has no prime factors.)
Then
\[
 F(n)=\max \sum_{i=1}^t a_i
\]
over all such choices.

The problem asks several asymptotic questions about $f$ and $F$, including:
\begin{itemize}
\item whether $f(n)=o(n\log\log n)$ for almost all $n$;
\item whether $F(n)\gg n\log\log n$ for almost all $n$;
\item extremal behavior of $\max_{n\le x} f(n)$ and whether $\max_{n\le x} f(n)=\max_{n\le x}F(n)$;
\item the count of $n<x$ with $f(n)=F(n)$;
\item asymptotics of $H(x)=\sum_{n<x} f(n)/n$ and upper bounds such as $H(x)\ll x\log\log\log\log x$.
\end{itemize}

\medskip
\noindent\textbf{QUICK LITERATURE/CONTEXT CHECK.}
The problem text states:
\begin{itemize}
\item $f(n)\le F(n)$ is ``trivial''.
\item Erd\H{o}s proved that $\max_{n\le x} f(n)\sim \frac{x\log x}{\log\log x}$ for a sequence of $x\to\infty$.
\item Erd\H{o}s proved bounds $x\log\log\log\log x\ll H(x)\ll x\log\log\log x$ and that $H(x)/x$ has no limit in the sense described.
\end{itemize}
I do not use any results not explicitly stated above.

\medskip
\noindent\textbf{ATTACK PLAN.}
\begin{enumerate}
\item Understand structural constraints: $F(n)$ corresponds to partitioning the prime set into disjoint ``blocks'' assigned to pairwise coprime numbers.
\item Prove basic inequalities and simple bounds for $f(n)$ in terms of $n$ and the prime factors.
\item Compute $f(n)$ and $F(n)$ for small $n$ to sanity-check questions like $f=F$ frequency and maxima.
\end{enumerate}

\medskip
\noindent\textbf{WORK.}

\noindent\textbf{Lemma 1 ($f(n)\le F(n)$).}
For every $n\ge 2$, one has $f(n)\le F(n)$.

\noindent\emph{Proof.}
For each prime divisor $p_i$ of $n$, let $b_i:=p_i^{\ell_i}$. By definition of $\ell_i$, we have $1\le b_i\le n$.
The numbers $b_i$ are pairwise coprime because they are powers of distinct primes.
Each $b_i$ has prime factors contained in $\{p_1,\dots,p_t\}$.
Thus $(a_1,\dots,a_t)=(b_1,\dots,b_t)$ is an admissible choice in the definition of $F(n)$, giving
\[
F(n)\ge \sum_{i=1}^t b_i = f(n).
\]
\hfill$\square$

\medskip
\noindent\textbf{Lemma 2 (size of each summand in $f(n)$).}
Let $p$ be a prime divisor of $n$, and let $L_p:=p^{\lfloor\log_p n\rfloor}$ (the largest power of $p$ not exceeding $n$).
Then
\[
 \frac{n}{p} < L_p \le n.
\]
Consequently, writing $\omega(n)$ for the number of distinct prime divisors of $n$,
\[
 n\sum_{p\mid n}\frac{1}{p} < f(n)\le n\,\omega(n).
\]

\noindent\emph{Proof.}
By definition, $L_p\le n < p\,L_p$. Dividing the strict inequality $n < p L_p$ by $p$ gives $n/p < L_p$.
The upper bound $L_p\le n$ is part of the definition.
Summing $n/p < L_p\le n$ over all distinct primes $p\mid n$ gives the stated bounds on $f(n)$.
\hfill$\square$

\medskip
\noindent\textbf{Lemma 3 (partition formulation for $F(n)$).}
Let $P=\{p_1,\dots,p_t\}$ be the set of distinct prime divisors of $n$.
For each nonempty subset $Q\subseteq P$, define
\[
 g(Q):=\max\{m\le n: \text{every prime factor of }m\text{ lies in }Q\}.
\]
Then
\[
F(n)=t+\max\Big\{\sum_{j=1}^s (g(Q_j)-1):\ \{Q_1,\dots,Q_s\}\text{ is a partition of }P\Big\}.
\]

\noindent\emph{Proof.}
Given any admissible $(a_1,\dots,a_t)$ for $F(n)$, delete the indices with $a_i=1$; suppose $s$ indices remain. The remaining $s$ numbers are pairwise coprime and each uses primes from $P$, hence their prime supports form a partition $Q_1,\dots,Q_s$ of $P$.
For each $j$, $a_{i_j}\le g(Q_j)$ by definition of $g(Q_j)$. Therefore
\[
\sum_{i=1}^t a_i = (t-s)\cdot 1 + \sum_{j=1}^s a_{i_j} \le (t-s)+\sum_{j=1}^s g(Q_j)= t+\sum_{j=1}^s(g(Q_j)-1).
\]
Maximizing over all admissible tuples yields
\[
F(n)\le t+\max_{\text{partitions}}\sum_{j}(g(Q_j)-1).
\]
Conversely, given a partition $Q_1,\dots,Q_s$ of $P$, choose $b_j=g(Q_j)$ for $j\le s$ and set the remaining $t-s$ numbers equal to $1$.
The $b_j$ are pairwise coprime because their prime supports are disjoint. This gives an admissible choice with sum
$t+\sum_j(g(Q_j)-1)$.
Taking the maximum over partitions yields the reverse inequality and thus equality.
\hfill$\square$

\medskip
\noindent\textbf{VERIFICATION (FAST REALITY CHECK).}
Using brute-force computation for $n\le 1000$ (with the above definitions):
\begin{itemize}
\item For $2\le n\le 1000$, the counts are:
\begin{verbatim}
# {n : f(n)=F(n)} = 596
# {n : F(n)>f(n)} = 403
\end{verbatim}
Including $n=1$ (where $f(1)=F(1)=0$ by the empty-sum convention) gives $597$ equalities among $1\le n\le 1000$.
\item The maxima on $2\le n\le 1000$ are
\begin{verbatim}
max_{n<=1000} f(n) = 2707 at n=870
max_{n<=1000} F(n) = 2707 at n=870
\end{verbatim}
with $870=2\cdot 3\cdot 5\cdot 29$ and
\[
 f(870)=2^9+3^6+5^4+29^2=512+729+625+841=2707.
\]
\item Small explicit example with strict inequality: $n=15$ gives
\[
 f(15)=9+5=14,\qquad F(15)\ge 15+1=16.
\]
\item Numerical values of $H_f(x):=\sum_{n<x} f(n)/n$ for a few small $x$:
\begin{verbatim}
H_f(10)= 8.1666666667
H_f(50)= 53.6289963452
H_f(100)=108.2303681634
H_f(1000)=1132.6889768131
\end{verbatim}
\end{itemize}

\medskip
\noindent\textbf{FINAL.} \textbf{UNRESOLVED}.

\noindent(i) \emph{Strongest proved partial result here.}
Lemma~1 ($f\le F$) and Lemma~2 (each summand in $f$ lies in $(n/p,n]$), plus Lemma~3 giving a precise partition/optimization formulation for $F(n)$.

\noindent(ii) \emph{First gap (crisp).}
Prove any of the main asymptotic statements in the problem text (e.g. $f(n)=o(n\log\log n)$ for almost all $n$, or $F(n)\gg n\log\log n$ for almost all $n$) from first principles, without importing results beyond those stated.

\noindent(iii) \emph{Top 3 next moves (concrete).}
\begin{enumerate}
\item Use Lemma~2 together with probabilistic/average estimates on $\sum_{p\mid n} 1/p$ and $\omega(n)$ to try to bound typical size of $f(n)$ (this resembles studying an ``additive function'' but with the unconventional cutoff $p^{\lfloor\log_p n\rfloor}$).
\item For $F(n)$, analyze the partition optimization (Lemma~3): understand for typical $n$ whether it is advantageous to group many primes into one block (making a number close to $n$) or to separate them into prime-power blocks.
\item Compute $\#\{n\le x: f(n)=F(n)\}$ for larger $x$ and inspect correlations with the size and distribution of prime factors of $n$.
\end{enumerate}

\noindent(iv) \emph{Minimal counterexample structure.}
A counterexample to ``$f(n)=o(n\log\log n)$ for almost all $n$'' would be a set of $n$ of positive density with unusually many distinct prime factors and/or with many small prime divisors making $\sum_{p\mid n} n/p$ large.
A counterexample to ``$F(n)\gg n\log\log n$ for almost all $n$'' would require a dense set of $n$ for which every partition of the prime set forces all $g(Q_j)$ to be much smaller than $n$, preventing the sum from reaching size $\asymp n\log\log n$.


