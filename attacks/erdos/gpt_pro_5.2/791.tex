% Erdos Problem #791

\subsection*{FORMAL RESTATEMENT}
Let $g(n)$ be the smallest integer $k$ such that there exists a set
$A\subset\{0,1,\dots,n\}$ with $|A|=k$ and
\[
\{0,1,\dots,n\} \subset A+A := \{a+a': a,a'\in A\}.
\]
Equivalently, every integer $t\in[0,n]$ can be written as $t=a+a'$ with $a,a'\in A$.
The problem asks for estimates of $g(n)$ and in particular whether $g(n)\sim 2\sqrt n$.

\subsection*{QUICK LITERATURE/CONTEXT CHECK}
From the problem statement: a simple counting gives $g(n)\ge (2n)^{1/2}$; a construction gives $g(n)\le 2\sqrt n+1$; and the claim $g(n)\sim 2\sqrt n$ is stated to be false (Mrose gives $g(n)^2\le \tfrac72 n$). Improved upper bounds with constants around $3.458\ldots$ are also quoted.

Below I reprove the elementary lower bound and a standard $2\sqrt n+O(1)$ construction, and I compute exact $g(n)$ for $n\le 25$.

\subsection*{ATTACK PLAN}
1) Reprove the standard lower bound from the fact that $A+A$ has at most $\binom{|A|+1}{2}$ unordered sums.

2) Give an explicit two-scale construction (small interval + arithmetic progression) showing $g(n)\le 2\lceil\sqrt n\rceil+1$.

3) Reality check: brute-force exact $g(n)$ for $n\le 25$.

\subsection*{WORK}
\textbf{Lemma 791.1 (Counting lower bound).}
For every $n\ge 1$,
\[
 g(n)\ge \left\lceil\frac{\sqrt{8n+9}-1}{2}\right\rceil \ge \sqrt{2n}-1.
\]

\emph{Proof.}
Let $A\subset\{0,\dots,n\}$ with $|A|=k$ and suppose $[0,n]\subset A+A$.
Consider the set of \emph{unordered} sums $a+a'$ with $a\le a'$.
There are at most
\[
\#\{\{a,a'\}: a,a'\in A\} = \binom{k+1}{2} = \frac{k(k+1)}{2}
\]
distinct such unordered pairs, hence at most $\binom{k+1}{2}$ distinct sums.
But $A+A$ must contain $n+1$ distinct integers $0,1,\dots,n$.
Therefore
\[
\frac{k(k+1)}{2} \ge n+1.
\]
Solving the quadratic inequality gives
$k\ge \frac{\sqrt{8n+9}-1}{2}$, and taking ceilings yields the first bound.
The crude inequality $\frac{\sqrt{8n+9}-1}{2}\ge \sqrt{2n}-1$ follows by expanding squares. \hfill$\square$

\medskip
\textbf{Lemma 791.2 (A constructive upper bound $g(n)\le 2\lceil\sqrt n\rceil+1$).}
Let $m:=\lceil \sqrt n\rceil$. Define
\[
A := \{0,1,2,\dots,m\}\ \cup\ \{0,m,2m,\dots,m^2\}\ \cap\ \{0,1,\dots,n\}.
\]
Then $|A|\le 2m+1$ and $[0,n]\subset A+A$. Consequently $g(n)\le 2\lceil\sqrt n\rceil+1$.

\emph{Proof.}
First, $A$ contains at most $m+1$ elements from $\{0,1,\dots,m\}$ and at most $m+1$ multiples of $m$, but $0$ is counted twice, so $|A|\le (m+1)+(m+1)-1=2m+1$.

Now fix any $t\in\{0,1,\dots,n\}$. Because $m^2\ge n\ge t$, we can write
$t=qm+r$ with integers $q\in\{0,1,\dots,m\}$ and $r\in\{0,1,\dots,m-1\}$.
Then $qm\in A$ and $r\in A$, hence $t=qm+r\in A+A$. This works for every $t\le n$.
\hfill$\square$

\medskip
\textbf{Fast reality check (exact computation for $n\le 25$).}
I brute-forced the exact minimum $g(n)$ by searching over all subsets $A\subset\{0,\dots,n\}$ for $n\le 25$.
The exact values found are:
\begin{center}
\begin{tabular}{c|cccccccccccc}
$n$ &1&2&3&4&5&6&7&8&9&10&11&12\\\hline
$g(n)$ &2&2&3&3&4&4&4&4&5&5&5&5
\end{tabular}

\medskip
\begin{tabular}{c|ccccccccccccc}
$n$ &13&14&15&16&17&18&19&20&21&22&23&24&25\\\hline
$g(n)$ &6&6&6&6&7&7&7&7&8&8&8&8&8
\end{tabular}
\end{center}
(One witness set for each $n$ was produced by the brute force; omitted here for brevity.)

\subsection*{VERIFICATION}
-- Lemma 791.1 counts unordered sums; using unordered pairs is necessary because ordered pairs can repeat the same sum.

-- Lemma 791.2: the representation $t=qm+r$ with $0\le q\le m$ and $0\le r<m$ is standard division with remainder, and the condition $t\le m^2$ ensures $q\le m$.

-- The brute-force search checked the condition $[0,n]\subset A+A$ exactly.

\subsection*{FINAL}
UNRESOLVED

(i) \textbf{Strongest proved partial result here:}
\[\sqrt{2n}-1\ \le\ g(n)\ \le\ 2\lceil\sqrt n\rceil+1.
\]
(The statement quotes better known upper constants and that $g(n)\sim 2\sqrt n$ is false.)

(ii) \textbf{First gap (crisp):} Determine the correct constant $c$ (if it exists) in
\[\lim_{n\to\infty} \frac{g(n)^2}{n}=c,
\]
or at least pin down $\liminf$ and $\limsup$ sharply.

(iii) \textbf{Top 3 next moves:}
\begin{enumerate}
\item Improve explicit constructions beyond the two-scale set in Lemma 791.2 by adding a third ``scale'' (a second step size) and optimize parameters to lower the constant in $|A|\le c\sqrt n$.
\item Prove stronger counting/entropy lower bounds that use not just $|A+A|$ but the local multiplicities of representations of sums.
\item Compute exact $g(n)$ further (e.g. up to $n=100$) and analyze the structure of extremal $A$ to guess the correct constant and suggest a construction.
\end{enumerate}

(iv) \textbf{Minimal counterexample structure to watch for:} Any attempt to prove a sharp lower constant would need to rule out bases $A$ whose sumset $A+A$ is unusually ``efficient'' (many distinct sums with minimal collisions). Extremal $A$ empirically resemble unions of a short initial segment with a sparse progression; sharpening lower bounds likely requires showing every such hybrid incurs inevitable collisions.

\bigskip

