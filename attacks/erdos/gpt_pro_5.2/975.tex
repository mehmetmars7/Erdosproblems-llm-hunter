
Let $f\in \mathbb{Z}[x]$ be an irreducible non-constant polynomial such that $f(n)\geq 1$ for all large $n\in\mathbb{N}$. Does there exist a constant $c=c(f)>0$ such that\[\sum_{n\leq X} \tau(f(n))\sim cX\log X,\]where $\tau$ is the divisor function? Van der Corput \cite{Va39} proved that\[\sum_{n\leq X} \tau(f(n))\gg_f X\log X.\]Erd\H{o}s \cite{Er52b} proved using elementary methods that\[\sum_{n\leq X} \tau(f(n))\ll_f X\log X.\]Such an asymptotic formula is known whenever $f$ is an irreducible quadratic, as proved by Hooley \cite{Ho63}. The form of $c$ depends on $f$ in a complicated fashion (see the work of McKee \cite{Mc95}, \cite{Mc97}, and \cite{Mc99} for expressions for various types of quadratic $f$). For example,\[\sum_{n\leq x}\tau(n^2+1)=\frac{3}{\pi}x\log x+O(x).\]Tao has a blog post on this topic. References [Er52b] Erd\H{o}s, P., On the sum {$\sum^x_{k=1} d(f(k))$} . J. London Math. Soc. (1952), 7--15. [Ho63] Hooley, Christopher, On the number of divisors of a quadratic polynomial . Acta Math. (1963), 97--114. [Mc95] McKee, James, On the average number of divisors of quadratic polynomials . Math. Proc. Cambridge Philos. Soc. (1995), 389--392. [Mc97] McKee, James, A note on the number of divisors of quadratic polynomials . (1997), 275--281. [Mc99] McKee, James, The average number of divisors of an irreducible quadratic polynomial . Math. Proc. Cambridge Philos. Soc. (1999), 17--22. [Va39] van der Corput, J. G., Une in\'{e}galit\'{e}{} relative au nombre des diviseurs . Nederl. Akad. Wetensch., Proc. (1939), 547--553.


\medskip
\noindent\textbf{1) FORMAL RESTATEMENT}\\
Let $f\in\mathbb{Z}[x]$ be a nonconstant irreducible polynomial.
Assume there exists $N_0$ such that $f(n)\ge 1$ for all integers $n\ge N_0$.
Let $\tau(m)=\#\{d\in\mathbb{N}: d\mid m\}$ be the divisor function.
The question is whether there exists a constant $c(f)>0$ such that, as $X\to\infty$,
\[
\sum_{1\le n\le X} \tau(f(n)) \sim c(f)\,X\log X.
\]

\medskip
\noindent\textbf{2) QUICK LITERATURE/CONTEXT CHECK}\\
The statement records: van der Corput proved a lower bound $\gg_f X\log X$; Erd\H{o}s proved an upper bound $\ll_f X\log X$ by elementary methods; and a full asymptotic is known for irreducible quadratics (Hooley and subsequent work), with an example constant $3/\pi$ for $n^2+1$.
I do not use any additional literature facts.

\medskip
\noindent\textbf{3) ATTACK PLAN}\\
\textbf{Proof track:} expand $\tau$ as a divisor sum and interchange summations:
\[
\sum_{n\le X}\tau(f(n))=\sum_{d}\#\{n\le X: d\mid f(n)\}.
\]
Then study the solution-count function $\rho(d):=\#\{a\bmod d: f(a)\equiv 0\pmod d\}$ and attempt to show
\[
\sum_{d\le f(X)} \frac{\rho(d)}{d} \sim c(f)\log X.
\]

\textbf{Disproof track:} attempt to exhibit an irreducible $f$ for which $\tau(f(n))$ has atypical average order (e.g. too many forced small prime factors or too few), preventing an $X\log X$ asymptotic.

I only prove structural identities and check the quadratic example numerically.

\medskip
\noindent\textbf{4) WORK}\\
\textbf{FAST REALITY CHECK (the example $f(n)=n^2+1$).}  Let
\[S(X):=\sum_{n\le X} \tau(n^2+1).
\]
The statement gives the asymptotic $S(X)=\frac{3}{\pi}X\log X+O(X)$.
I computed $S(X)$ exactly for several $X$ and compared to $X\log X$:
\[
\begin{array}{c|c|c|c}
X & S(X) & S(X)/(X\log X) & S(X)/\big((3/\pi)X\log X\big)\\\hline
 10^3 & 7508 & 1.0868943 & 1.1381931\\
 10^4 & 97122 & 1.0544887 & 1.1042580\\
 5\cdot 10^4 & 562870 & 1.0404465 & 1.0895530\\
 10^5 & 1191138 & 1.0346093 & 1.0834403
\end{array}
\]
The ratio appears to drift slowly downward toward $3/\pi\approx 0.9549$, consistent with a main term of size $X\log X$ plus an $O(X)$ error.

\medskip
\noindent\textbf{Lemma 975.1 (divisor-sum switching identity).}\label{lem:975-divswitch}
For every integer $X\ge 1$,
\[
\sum_{1\le n\le X} \tau(f(n))
=\sum_{d=1}^{\max_{1\le n\le X} f(n)} \#\{1\le n\le X: d\mid f(n)\}.
\]

\noindent\textbf{Proof.}
For each $n$,
\[
\tau(f(n))=\sum_{d\mid f(n)} 1 = \sum_{d=1}^{f(n)} \mathbf{1}_{d\mid f(n)}.
\]
Summing over $1\le n\le X$ gives
\[
\sum_{n\le X}\tau(f(n)) = \sum_{n\le X}\sum_{d=1}^{f(n)} \mathbf{1}_{d\mid f(n)}.
\]
Every term with $d>\max_{n\le X} f(n)$ is zero, so we may extend the inner sum to $d\le M:=\max_{n\le X} f(n)$:
\[
\sum_{n\le X}\tau(f(n)) = \sum_{n\le X}\sum_{d=1}^{M} \mathbf{1}_{d\mid f(n)}.
\]
Interchanging the finite sums yields
\[
\sum_{d=1}^{M} \sum_{n\le X} \mathbf{1}_{d\mid f(n)} = \sum_{d=1}^{M} \#\{n\le X: d\mid f(n)\},
\]
which is the desired identity. \qed

\medskip
\noindent\textbf{Lemma 975.2 (root bound modulo a prime).}\label{lem:975-rootsmodp}
Let $f\in\mathbb{Z}[x]$ have degree $g\ge 1$.
Let $p$ be a prime such that the reduction of $f$ modulo $p$ is not the zero polynomial.
Then the congruence $f(a)\equiv 0\pmod p$ has at most $g$ solutions $a\bmod p$.
Consequently,
\[
\#\{1\le n\le X: p\mid f(n)\}\le g\Big(\frac{X}{p}+1\Big).
\]

\noindent\textbf{Proof.}
Working in the field $\mathbb{F}_p$, the polynomial $\bar f\in\mathbb{F}_p[x]$ is a nonzero polynomial of degree at most $g$.  A nonzero polynomial of degree $\le g$ over a field has at most $g$ roots in that field: if it had $g+1$ distinct roots, it would be divisible by the product of $g+1$ distinct linear factors and thus have degree at least $g+1$, a contradiction.
Therefore $\bar f$ has at most $g$ roots in $\mathbb{F}_p$, i.e. at most $g$ residue classes $a\bmod p$ solve $f(a)\equiv 0\pmod p$.
Among integers $1\le n\le X$, each residue class mod $p$ occurs at most $\lceil X/p\rceil\le X/p+1$ times, giving the displayed bound. \qed

\medskip
\noindent\textbf{5) VERIFICATION}\\
Lemma~\ref{lem:975-divswitch} is a finite double-counting identity; the only care is that $f(n)$ might be negative for small $n$, but the problem assumes $f(n)\ge 1$ for all large $n$.  In the identity one may take $X$ large enough that $f(n)\ge 1$ for $1\le n\le X$, or else interpret $\tau(f(n))$ as $\tau(|f(n)|)$.

Lemma~\ref{lem:975-rootsmodp} requires that $f\bmod p$ is not identically zero; this fails for only finitely many primes dividing all coefficients of $f$.

The computation for $f(n)=n^2+1$ used trial division by all primes up to $\sqrt{n^2+1}$ for each $n\le X$; results were cross-checked by recomputation for smaller $X$.

\medskip
\noindent\textbf{6) FINAL}\\
\textbf{UNRESOLVED}

(i) \textbf{Strongest proved partial result.} The sum of divisors of values of $f$ admits an exact divisor-switching decomposition (Lemma~\ref{lem:975-divswitch}).  Also, for almost all primes $p$ the congruence $f(n)\equiv 0\pmod p$ has at most $\deg(f)$ solutions mod $p$, yielding a uniform bound on how often a fixed prime divides $f(n)$ (Lemma~\ref{lem:975-rootsmodp}).  Numerically, for $f(n)=n^2+1$ the ratio $\sum_{n\le X}\tau(f(n))/(X\log X)$ is consistent with a constant-times-$X\log X$ main term.

(ii) \textbf{First gap (crisp).} Prove (or disprove) that for every irreducible $f\in\mathbb{Z}[x]$ that is eventually positive there exists $c(f)>0$ with
\[
\sum_{n\le X}\tau(f(n)) = c(f)X\log X + o(X\log X).
\]

(iii) \textbf{Top 3 next moves.}
\begin{itemize}
\item Study $\rho(d)=\#\{a\bmod d: f(a)\equiv 0\pmod d\}$ and its average order; use Lemma~\ref{lem:975-divswitch} to reduce the main term to understanding $\sum_{d\le f(X)} \rho(d)/d$.
\item Prove good multiplicativity and uniform bounds for $\rho(p^k)$ (beyond Lemma~\ref{lem:975-rootsmodp} for $k=1$) to control contributions from prime powers.
\item For specific degrees $\ge 3$, run computations for several irreducible polynomials to see whether $\sum_{n\le X}\tau(f(n))/(X\log X)$ stabilizes and how sensitive the apparent constant is to $f$.
\end{itemize}

(iv) \textbf{Minimal counterexample structure.} A counterexample would be an irreducible $f$ for which either (a) the sum is systematically larger than any constant multiple of $X\log X$ (requiring unusually frequent large divisor counts), or (b) systematically smaller (requiring unusually few divisors).  Since irreducibility rules out an obvious factorization, any such behavior would likely come from atypical local solubility patterns $f(n)\equiv 0\pmod d$ across many moduli $d$.
