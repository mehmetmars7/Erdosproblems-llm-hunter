
\noindent 1) \textbf{FORMAL RESTATEMENT}

\noindent Let $\omega(m)$ be the number of distinct prime divisors of $m$, and let $\pi(k)$ be the number of primes $\le k$.
For a fixed integer $k\ge 1$, define
\[
S_k(n) := \sum_{i=0}^{k-1} \omega(n+i).
\]
Question (A): Is it true that
\[
\liminf_{n\to\infty} S_k(n) \le k + \pi(k) ?
\]
Question (B): Is it true that
\[
\limsup_{n\to\infty} S_k(n)\,\frac{\log\log n}{\log n} = 1?
\]

\noindent 2) \textbf{QUICK LITERATURE/CONTEXT CHECK}

\noindent The problem text states that Erd\H{o}s and Selfridge observed
\[
\liminf_{n\to\infty} S_k(n) \ge k+\pi(k)-1,
\]
and sketches a justification referencing a theorem of P\'{o}lya about $k$-smooth integers having unbounded gaps, together with the simple fact that $n(n+1)\cdots(n+k-1)$ is divisible by every prime $\le k$. The problem also states (as a known fact) that
\[
\limsup_{n\to\infty} \omega(n)\,\frac{\log\log n}{\log n} = 1.
\]
I do not use any external results beyond what is explicitly stated.

\noindent 3) \textbf{ATTACK PLAN}

\noindent \emph{For (A):}
\begin{itemize}
\item The lower bound suggests $S_k(n)$ is forced to be at least $k+\pi(k)-1$ for all large $n$ (depending on $k$). Proving an upper bound $\le k+\pi(k)$ for the liminf would require producing infinitely many $n$ where the block $[n,n+k)$ has ``as few prime factors as possible''.
\item Attempt to construct $n$ where one of the $k$ numbers is $k$-smooth (contributing many small primes at once), while the other $k-1$ numbers are each a product of a single prime $>k$ (so each contributes exactly $1$).
\end{itemize}

\noindent \emph{For (B):}
\begin{itemize}
\item Since $S_k(n)\ge \omega(n)$, the limsup is at least $1$. To show it equals $1$, one would need an upper bound preventing \emph{several} of the consecutive integers from simultaneously having near-maximal $\omega$.
\item A disproof would exhibit infinitely many $n$ for which several of $n,\dots,n+k-1$ are extremely ``smooth'' and squarefree, pushing $S_k(n)$ above the single-term extremal behavior.
\end{itemize}

\noindent 4) \textbf{WORK}

\noindent \textbf{Lemma 890.1 (Deriving the Erd\H{o}s--Selfridge lower bound from a strong ``gap'' property).}
Assume the following interpretation of the P\'{o}lya statement in the problem text:

\begin{quote}
(\*) For each fixed $k\ge 1$ there exists $N_k$ such that among any $k$ consecutive integers all exceeding $N_k$, at most one is $k$-smooth (i.e. has no prime factor $>k$).
\end{quote}

\noindent Under (\*), for every fixed $k$ we have
\[
S_k(n) \ge k+\pi(k)-1\qquad\text{for all }n\ge N_k.
\]
Consequently $\liminf_{n\to\infty} S_k(n)\ge k+\pi(k)-1$.

\noindent \emph{Proof.}
Fix $k$ and take $n\ge N_k$.
By (\*), among the integers $n,n+1,\dots,n+k-1$ there are at least $k-1$ integers that are \emph{not} $k$-smooth. For each such integer $m$, there exists a prime $p>k$ dividing $m$, and therefore $\omega(m)\ge 1$.
Summing over these $k-1$ integers yields
\[
\sum_{i=0}^{k-1} \omega(n+i) \ge k-1 + \sum_{\substack{0\le i<k\\ n+i\ \text{$k$-smooth}}} \omega(n+i).
\]
Now we lower bound the contribution coming from primes $\le k$.
For each prime $p\le k$, among the $k$ consecutive integers $n,n+1,\dots,n+k-1$ there exists an index $0\le i<p\le k$ such that $p\mid(n+i)$.
Whenever $p\mid(n+i)$, that prime contributes $1$ to $\omega(n+i)$.
Summing over all primes $p\le k$ therefore guarantees at least $\pi(k)$ prime-divisor contributions to the total sum $S_k(n)$.
Formally, write
\[
\sum_{i=0}^{k-1} \omega(n+i) = \sum_{i=0}^{k-1} \sum_{p\mid(n+i)} 1,
\]
where the inner sum is over distinct primes dividing $n+i$. For each prime $p\le k$, at least one of the inner sums contains a $1$ coming from $p$, so the whole double sum is at least $\pi(k)$. Combining this with the $k-1$ lower bound coming from the $k-1$ non-$k$-smooth integers gives
\[
S_k(n)\ge (k-1)+\pi(k)=k+\pi(k)-1.
\]
This holds for all $n\ge N_k$, hence the liminf is also at least $k+\pi(k)-1$.
\hfill$\square$

\medskip
\noindent \textbf{Lemma 890.2 (Trivial bounds for the limsup in (B)).}
Assume the fact stated in the problem text that
\[
\limsup_{n\to\infty} \omega(n)\,\frac{\log\log n}{\log n}=1.
\]
Then for every fixed $k\ge 1$,
\[
1 \le \limsup_{n\to\infty} S_k(n)\,\frac{\log\log n}{\log n} \le k.
\]

\noindent \emph{Proof.}
For the lower bound, note that $S_k(n)\ge \omega(n)$ for all $n$, hence
\[
\limsup_{n\to\infty} S_k(n)\,\frac{\log\log n}{\log n}
\ge \limsup_{n\to\infty} \omega(n)\,\frac{\log\log n}{\log n}=1.
\]

For the upper bound, for each $n$ we have the pointwise inequality
\[
S_k(n)=\sum_{i=0}^{k-1}\omega(n+i) \le k\cdot \max_{0\le i<k}\omega(n+i).
\]
Multiply by $(\log\log n)/(\log n)$ and take limsup:
\[
\limsup_{n\to\infty} S_k(n)\,\frac{\log\log n}{\log n}
\le k\cdot \limsup_{n\to\infty} \Big(\max_{0\le i<k}\omega(n+i)\Big)\frac{\log\log n}{\log n}.
\]
For each fixed $i$, the classical limsup identity for $\omega(n)$ also applies to the shifted sequence $\omega(n+i)$ (because $\log(n+i)\sim \log n$ and $\log\log(n+i)\sim \log\log n$ as $n\to\infty$).
Hence
\[
\limsup_{n\to\infty} \omega(n+i)\,\frac{\log\log n}{\log n}=1\quad\text{for each fixed }i.
\]
The limsup of the maximum of finitely many sequences equals the maximum of their limsups, so
\[
\limsup_{n\to\infty} \Big(\max_{0\le i<k}\omega(n+i)\Big)\frac{\log\log n}{\log n} = 1.
\]
Combining with the previous inequality gives the upper bound $\le k$.
\hfill$\square$

\medskip
\noindent \textbf{Fast reality check (computation).}
Using a sieve for $\omega(m)$, I computed $S_k(n)$ for $n\le 200{,}000$ and also restricted-range minima on $n\in[100{,}000,200{,}000]$.
Selected exact outputs:
\begin{itemize}
\item Over $n\in[100{,}000,200{,}000]$, the minimum values found were:
\[
\min S_2(n)=2\ \text{at }n=131071,\qquad
\min S_3(n)=4\ \text{at }n=131071.
\]
\item Over $n\le 200{,}000$, the maxima found were:
\[
\max S_1(n)=6\ \text{at }n=30030,\qquad
\max S_2(n)=10\ \text{at }n=38570,\qquad
\max S_3(n)=11\ \text{at }n=30029.
\]
\item Over $n\le 200{,}000$, the maxima of the scaled quantity $S_k(n)\,\frac{\log\log n}{\log n}$ were:
\[
\max_{n\le 200{,}000} S_1(n)\frac{\log\log n}{\log n}\approx 1.363\ \text{at }n=30030,
\]
\[
\max_{n\le 200{,}000} S_2(n)\frac{\log\log n}{\log n}\approx 2.043\ \text{at }n=30029,
\]
\[
\max_{n\le 200{,}000} S_3(n)\frac{\log\log n}{\log n}\approx 2.498\ \text{at }n=30029.
\]
\end{itemize}
These are small-$n$ phenomena; they are consistent with the fact that the asymptotic limsup for a single $\omega(n)$ equals $1$.

\noindent 5) \textbf{VERIFICATION}

\noindent Lemma 890.1 cleanly separates the elementary divisibility fact (every prime $\le k$ divides some number among $k$ consecutive integers) from the nontrivial smooth-gap input (\*).
The counting step uses only that each such prime contributes at least $1$ to some $\omega(n+i)$.

\noindent Lemma 890.2 uses only inequalities and the stated classical limsup for $\omega(n)$.
The step ``limsup of max of finitely many sequences'' is valid because for any finite family $(a^{(i)}_n)$,
\(\limsup_n \max_i a^{(i)}_n = \max_i \limsup_n a^{(i)}_n\).

\noindent Computation is a finite-range check and does not address the asymptotic questions directly.

\noindent 6) \textbf{FINAL}

\noindent \textbf{UNRESOLVED}
\begin{enumerate}
\item[(i)] \textbf{Strongest proved partial result here.} Assuming the strong gap property (\*) attributed in the problem text to P\'{o}lya, one obtains the universal lower bound $S_k(n)\ge k+\pi(k)-1$ for all sufficiently large $n$, hence $\liminf_{n\to\infty} S_k(n)\ge k+\pi(k)-1$. Also, using only the stated classical limsup for $\omega(n)$, we have the unconditional bounds $1\le \limsup S_k(n)\frac{\log\log n}{\log n}\le k$.
\item[(ii)] \textbf{First gap (crisp).} Prove either (A) $\liminf_{n\to\infty} S_k(n)\le k+\pi(k)$ for each $k$, and (B) $\limsup_{n\to\infty} S_k(n)\frac{\log\log n}{\log n}=1$ (i.e. show that having $k$ terms does not increase the extremal order beyond the single-term behavior).
\item[(iii)] \textbf{Top 3 next moves.}
  \begin{itemize}
  \item For (A), attempt to explicitly construct infinitely many $n$ where exactly one of $n,\dots,n+k-1$ collects most primes $\le k$ (a $k$-smooth number) and the other $k-1$ terms are each the product of a single prime $>k$.
  \item For (B), attempt an upper bound of the form $S_k(n)\le (1+o(1))\frac{\log n}{\log\log n}$, perhaps by showing that at most one of $n,\dots,n+k-1$ can be composed mostly of the smallest primes.
  \item Extend computations to larger $n$ for fixed $k$ to estimate the growth of $\min_{m\ge n} S_k(m)$ and to search for candidate configurations that might achieve the conjectured liminf upper bound.
  \end{itemize}
\item[(iv)] \textbf{Minimal counterexample structure.} A counterexample to (B) would require infinitely many $n$ for which at least two (or more) of the consecutive integers $n,\dots,n+k-1$ each have close to the maximal possible number of distinct prime factors (comparable to $(\log n)/(\log\log n)$). Such a configuration would force many small primes to divide multiple nearby integers, which is highly constrained by coprimality of consecutive integers.
\end{enumerate}

