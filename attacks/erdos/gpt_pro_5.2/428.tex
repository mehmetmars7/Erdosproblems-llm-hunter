
FORMAL RESTATEMENT
Does there exist a set $A\subseteq\mathbb N$ such that:
1) There are infinitely many integers $n\ge 1$ with the property that for every $a\in A$ satisfying $0<a<n$, the integer $n-a$ is prime.
2) The set $A$ has positive lower density relative to the primes, i.e.
\[
\liminf_{x\to\infty}\frac{|A\cap[1,x]|}{\pi(x)}>0.
\]
Here $\pi(x)$ is the number of primes $\le x$.

Equivalently, for infinitely many $n$ we have
\[
A\cap[1,n-1]\subseteq \{n-p: p\text{ prime},\ 1\le p\le n-1\}.
\]

Edge cases: $a=0$ is excluded by hypothesis; $n-a$ being prime includes the possibility $n-a=2$.

QUICK LITERATURE/CONTEXT CHECK
The problem statement notes that Erd\H{o}s and Graham can show the analogous statement with $\limsup$ (in place of $\liminf$) assuming the prime $k$-tuple conjecture. No other external results are used here.

ATTACK PLAN
Proof track ideas.
1) Identify unavoidable local obstructions (parity, mod $3$, etc.) forcing $A$ into certain residue classes.
2) Try to build $A$ inside a single residue class (e.g. even integers) and search for infinitely many $n$ where the reflected set $n-A$ hits only primes.

Disproof track ideas.
1) Use parity and small-modulus constraints to show that if $A$ is large enough (comparable to primes) then the constraints for $n-a$ to all be prime become mutually incompatible for large $n$.
2) Show that for any “dense relative to primes” set $A$, the set of shifts $n$ with $n-A$ all prime must be finite.

WORK
Fast reality check (finite toy models).
For a finite set $A$, the condition “$n-a$ is prime for all $a\in A$ with $a<n$” is exactly a prime-constellation condition.
For example:
- If $A=\{2\}$, the condition is that $n-2$ is prime, which holds for infinitely many $n$ (assuming infinitely many primes).
- If $A=\{2,4\}$, the condition is that $n-4$ and $n-2$ are both prime, which is equivalent to the existence of twin primes.
A brute-force check up to $n\le 200$ shows that for $A=\{2,4\}$ there are $14$ such $n$ (corresponding to twin prime pairs $\le 198$), while for $A=\{2,4,6\}$ there is exactly one such $n$, namely $n=9$ (corresponding to primes $3,5,7$).

Lemma 428.1 (parity obstruction for infinitely many $n$).
If $A$ contains both an even integer and an odd integer, then there are only finitely many $n$ satisfying condition (1). In particular, no such $A$ can satisfy the “infinitely many $n$” requirement.

Proof.
Assume $a_e\in A$ is even and $a_o\in A$ is odd.
Let $n$ be an integer with $n>\max\{a_e,a_o\}+2$.
Then both $n-a_e$ and $n-a_o$ are positive integers $>2$.
Also, $n-a_e$ has the same parity as $n$ (since $a_e$ is even), whereas $n-a_o$ has the opposite parity to $n$ (since $a_o$ is odd). Thus one of $n-a_e$ and $n-a_o$ is even.
But any even prime is equal to $2$, and we have just seen both numbers are $>2$. Therefore, at least one of $n-a_e$ and $n-a_o$ is an even integer greater than $2$, hence composite.
So for every $n>\max\{a_e,a_o\}+2$, condition (1) fails.
Thus there are at most finitely many $n$ satisfying (1), namely $n\le \max\{a_e,a_o\}+2$.
\hfill $\square$

Lemma 428.2 (eventual parity of the witnessing $n$ for infinite $A$).
Suppose $A$ is infinite (in particular, condition (2) holds) and $A$ satisfies condition (1) for infinitely many $n$.
Then either:
- $A$ is contained in the even integers and all sufficiently large witnessing $n$ must be odd, or
- $A$ is contained in the odd integers and all sufficiently large witnessing $n$ must be even.

Proof.
By Lemma 428.1, $A$ cannot contain both parities, so $A\subseteq 2\mathbb Z$ or $A\subseteq 2\mathbb Z+1$.

Assume first that $A\subseteq 2\mathbb Z$.
Since $A$ is infinite, choose two distinct elements $a_1<a_2$ of $A$.
Consider any even integer $n>a_2$.
Then $a_1,a_2<n$, so condition (1) would require both $n-a_1$ and $n-a_2$ to be prime.
But $n-a_1$ and $n-a_2$ are even (even minus even), and they are $>2$ because $n>a_2\ge a_1+2$.
An even integer $>2$ is composite, so condition (1) fails for every even $n>a_2$.
Therefore, among the infinitely many witnessing $n$, only finitely many can be even; equivalently, all sufficiently large witnessing $n$ are odd.

The case $A\subseteq 2\mathbb Z+1$ is analogous.
Choose two distinct odd elements $b_1<b_2$ of $A$.
For any odd $n>b_2$, the differences $n-b_1$ and $n-b_2$ are even integers $>2$, hence composite, so such $n$ cannot witness condition (1).
Thus all sufficiently large witnessing $n$ must be even.
\hfill $\square$

VERIFICATION
- Lemma 428.1: the bound $n>\max\{a_e,a_o\}+2$ ensures both differences exceed $2$, so “even prime” $=2$ is excluded.
- Lemma 428.2: the argument uses only the fact that an even prime is $2$ and that $A$ is fixed while witnessing $n$ vary.
- Toy computations: the finite-set checks align with known necessary constraints (e.g. $\{2,4,6\}$ forces a prime triplet and only $3,5,7$ works because of mod $3$).

FINAL
**UNRESOLVED**
(i) Strongest proved partial result: any $A$ satisfying the “infinitely many $n$” requirement must lie entirely in one parity class (Lemma 428.1), and then the witnessing $n$ must eventually have the opposite parity (Lemma 428.2).
(ii) First gap (crisp): construct (or rule out) an infinite set $A$ with $|A\cap[1,x]|\gg \pi(x)$ and infinitely many $n$ such that $\{n-a: a\in A,\ a<n\}$ are all prime.
(iii) Top 3 next moves:
  1. Local obstruction analysis beyond parity: prove that $A$ must avoid additional residue patterns mod small primes (in the spirit of admissible prime $k$-tuples) uniformly across all initial segments $A\cap[1,n)$.
  2. Constructive search: restrict to $A\subseteq 2\mathbb Z$ (or $2\mathbb Z+1$) and use computation to find large initial segments $A\cap[1,x]$ that admit some $n$ with all $n-a$ prime, then study whether these initial segments can be extended indefinitely.
  3. Sieve-theoretic barrier: attempt to show that if $|A\cap[1,x]|\ge c\pi(x)$ for fixed $c>0$, then for large $n$ the congruence conditions imposed by $\{n-a\}$ being prime force a contradiction (e.g. too many residues hit modulo a small prime).
(iv) Minimal counterexample structure: if the statement is false, then for every $A\subseteq\mathbb N$ with $|A\cap[1,x]|\gg \pi(x)$ there would exist $N(A)$ such that for all $n\ge N(A)$ some $a\in A$ with $a<n$ makes $n-a$ composite; a minimal obstruction would likely come from a fixed small modulus $q$ for which $A$ hits too many residue classes so that $n-A$ cannot avoid multiples of $q$ for all large $n$.


