\section{Erd\H{o}s Problem \#252}

\subsection*{1) ROUND-2 OBJECTIVE}
\textbf{Path (C): obstruction/correction.}
Round~1 gave complete, elementary irrationality proofs for $k=1,2$ and isolated the main obstruction for $k\ge 3$: at the ``obvious'' choice $N=p-1$ the early tail terms are too large to trap the tail in an interval of length $<1$ using only crude divisor bounds.

In this round I give a \emph{gap-free conditional resolution for all $k$} under \emph{Schinzel's hypothesis~H} (a prime-values conjecture for polynomial systems). This supplies a minimal corrected statement: assuming~H, one can carry out the factorial-tail strategy for every fixed $k$.

\subsection*{2) Round-1 FOUNDATION USED}
I will rely on the following Round~1 components (without reproving them):
\begin{itemize}
\item \textbf{Factorial-tail integrality lemma (Round~1, Lemma~\textup{\S4.1}).} If $\sum_{n\ge1} a_n/n!\in\mathbb{Q}$ with $a_n\in\mathbb{Z}$, then for every $N$ beyond the denominator, the corresponding factorial tail is an integer.
\item \textbf{Completed cases $k=1,2$.} These remain as unconditional base cases; the new work below is only needed for $k\ge 3$.
\item \textbf{Divisor power-sum bound.} For $k>1$,
\[\sigma_k(n)=n^k\sum_{d\mid n}\frac1{d^k}\le \zeta(k)\,n^k,\]
which extends Round~1's $k=2$ constant-bound argument.
\end{itemize}

\subsection*{3) NEW INSIGHT / TOOL (ROUND-2)}
The new ingredient is a modular/fractional-part control mechanism (after Schlage--Puchta):
\begin{itemize}
\item Work \emph{modulo $1$} instead of bounding the tail absolutely.
\item Choose large integers $q$ in a fixed residue class modulo $D:=(k!)^k$ such that
\[q+i-1=i\cdot r_i\quad(2\le i\le k),\]
with each $r_i$ prime, and additionally either (a) $q$ itself is prime, or (b) $q=p\cdot r$ with $p>k$ fixed prime and $r$ prime.
\item For such $q$, multiplicativity gives sharp formulas for $\sigma_k(q+i-1)$, and a polynomial-division lemma replaces
\(\frac{(q+i-1)^k}{(q+i-1)_i}\)
by an \emph{integer polynomial} in $q$ up to error $O(1/q)$.
\item Fixing $q\bmod D$ freezes the fractional parts of all terms with $i\ge2$. Comparing the ``$q$ prime'' and ``$q=p\times\text{prime}$'' situations isolates a \emph{nonzero rational number} forced to be arbitrarily close to an integer, contradicting rationality.
\end{itemize}
Schinzel's hypothesis~H is invoked only to guarantee infinitely many such $q$.

\subsection*{4) ATTACK PLAN (ROUND-2)}
\textbf{Gap after Round~1.} For $k\ge3$ the naive positivity bounds cannot force the tail into an interval of length $<1$.

\textbf{What must now be proved.} Under a hypothesis that provides infinitely many $q$ with the controlled local factorisations above, show that rationality of $\alpha_k$ forces a contradiction.

\textbf{Plan.}
\begin{enumerate}
\item Assume $\alpha_k\in\mathbb{Q}$ and use Round~1's lemma to obtain that a certain ``falling-factorial'' tail is an integer.
\item Truncate that tail after $k$ terms; the remainder is $\ll 1/q$ using $\sigma_k(n)\le\zeta(k)n^k$.
\item For special $q$, rewrite the first $k$ terms as a near-integer linear form \(\sum_{i=1}^k \sigma_{-k}(i)P_{k,i}(q)\) with integer polynomials $P_{k,i}$.
\item Pick two such integers $q_1,q_2\equiv1\pmod D$ with $q_2$ prime and $q_1=p\cdot\text{prime}$. Subtract the two near-integer relations; all $i\ge2$ contributions cancel modulo $\mathbb{Z}$, leaving
\(\big\|q_1^{k-1}/p^k\big\|\ll 1/q_1\),
impossible for large $q_1$.
\end{enumerate}

\subsection*{5) WORK (ROUND-2)}
\paragraph{Notation.}
For $x\in\mathbb{R}$ let
\[\|x\|:=\min_{m\in\mathbb{Z}}|x-m|\]
be the distance to the nearest integer.
For $m\in\mathbb{N}$ define the falling factorial
\[(x)_m:=x(x-1)\cdots(x-m+1).\]
For $k\in\mathbb{N}$ define the ``negative'' divisor sum
\[\sigma_{-k}(n):=\sum_{d\mid n}d^{-k}=\frac{\sigma_k(n)}{n^k}\in\mathbb{Q}.\]

\paragraph{Schinzel's hypothesis~H (used as a black box).}
Let $f_1,\dots,f_m\in\mathbb{Z}[t]$ be nonconstant polynomials with positive leading coefficients. Assume that for every prime $\ell$ there exists an integer $t$ such that $\ell\nmid f_1(t)\cdots f_m(t)$ (equivalently, no prime divides the product $\prod_j f_j(t)$ for all $t$). Then there are infinitely many integers $t$ for which all $f_j(t)$ are simultaneously prime.

\subsubsection*{5.1 Tail integrality in falling-factorial form (Round~1)}
\begin{lemma}[Tail integrality, falling-factorial version]\label{lem:tail-int-falling}
Let $k\ge1$ and $\alpha_k:=\sum_{n\ge1}\sigma_k(n)/n!$. If $\alpha_k=a/b\in\mathbb{Q}$ in lowest terms, then for every integer $n>b$ the tail
\[T_k(n):=\sum_{\nu\ge n}\frac{\sigma_k(\nu)}{(\nu)_{\nu-n+1}}\]
is an integer.
\end{lemma}
\noindent\emph{Justification.} This is exactly Round~1's Lemma~\textup{\S4.1} with $N=n-1$, since
\[(n-1)!/\nu!=1/(\nu(\nu-1)\cdots n)=1/(\nu)_{\nu-n+1}.\]

\subsubsection*{5.2 A uniform bound for the discarded tail}
\begin{lemma}[Discarded tail is $O(1/n)$ for fixed $k>1$]\label{lem:discarded-tail}
Fix $k\ge2$. There exists a constant $C_k>0$ such that for all integers $n\ge2$,
\[\sum_{\nu\ge n+k}\frac{\sigma_k(\nu)}{(\nu)_{\nu-n+1}}\le \frac{C_k}{n}.\]
\end{lemma}
\begin{proof}
For $k>1$ we have the elementary bound
\[\sigma_k(m)=m^k\sum_{d\mid m}\frac{1}{d^k}\le m^k\sum_{d\ge1}\frac{1}{d^k}=\zeta(k)m^k.\]
Write $\nu=n+m$ with $m\ge k$. Using
\[(n+m)_{m+1}=(n+m)\prod_{j=0}^{m-1}(n+j)\ge (n+m)\,n^m,\]
we obtain
\[\frac{\sigma_k(n+m)}{(n+m)_{m+1}}\le \zeta(k)\,\frac{(n+m)^k}{(n+m)n^m}=\zeta(k)\,\frac{(n+m)^{k-1}}{n^m}.
\]
Hence
\[\sum_{\nu\ge n+k}\frac{\sigma_k(\nu)}{(\nu)_{\nu-n+1}}
\le \zeta(k)\sum_{m\ge k}\frac{(n+m)^{k-1}}{n^m}.
\]
Use $(a+b)^{k-1}\le 2^{k-2}(a^{k-1}+b^{k-1})$ for $a,b\ge0$ to get
\[(n+m)^{k-1}\le 2^{k-2}(n^{k-1}+m^{k-1}).\]
Therefore
\begin{align*}
\sum_{m\ge k}\frac{(n+m)^{k-1}}{n^m}
&\le 2^{k-2}\Bigl(n^{k-1}\sum_{m\ge k}\frac1{n^m}+\sum_{m\ge k}\frac{m^{k-1}}{n^m}\Bigr)\\
&=2^{k-2}\Bigl(\frac{n^{k-1}}{n^k}\cdot\frac{1}{1-1/n}+\frac{1}{n^k}\sum_{j\ge0}\frac{(j+k)^{k-1}}{n^j}\Bigr)\\
&\le 2^{k-2}\Bigl(\frac{1}{n-1}+\frac{1}{n^k}\sum_{j\ge0}\frac{(j+k)^{k-1}}{2^j}\Bigr)
\qquad(n\ge2).
\end{align*}
The series $\sum_{j\ge0}(j+k)^{k-1}/2^j$ converges, so the right-hand side is $\ll 1/n$ for $n\ge2$. Absorbing constants gives the claim.
\end{proof}

\subsubsection*{5.3 Polynomial division against falling factorials}
\begin{lemma}[Integer-polynomial approximation]\label{lem:poly-division}
Fix integers $k\ge1$ and $1\le i\le k$. There exist polynomials $P_{k,i},R_{k,i}\in\mathbb{Z}[x]$ with $\deg R_{k,i}<i$ such that
\[(x+i-1)^k=P_{k,i}(x)\,(x+i-1)_i+R_{k,i}(x).\]
Consequently,
\[\frac{(x+i-1)^k}{(x+i-1)_i}=P_{k,i}(x)+O_k\!\left(\frac{1}{x}\right)\qquad(x\to+\infty).
\]
Moreover $P_{k,1}(x)=x^{k-1}$.
\end{lemma}
\begin{proof}
The falling factorial $(x+i-1)_i=(x+i-1)(x+i-2)\cdots x$ is a monic polynomial in $x$ of degree $i$ with integer coefficients. The polynomial $(x+i-1)^k$ has integer coefficients and degree $k\ge i$. Performing Euclidean division in $\mathbb{Z}[x]$ by the monic divisor $(x+i-1)_i$ gives the identity with $P_{k,i},R_{k,i}\in\mathbb{Z}[x]$ and $\deg R_{k,i}<i$.

For $x\to\infty$ the remainder term satisfies
\[\frac{R_{k,i}(x)}{(x+i-1)_i}=O\!\left(\frac{x^{i-1}}{x^i}\right)=O\!\left(\frac{1}{x}\right),\]
which yields the stated asymptotic.

For $i=1$ we have $(x)_1=x$ and $(x+0)^k/x=x^{k-1}$, so $P_{k,1}(x)=x^{k-1}$.
\end{proof}

\subsubsection*{5.4 Congruence stability of the ``$i\ge2$'' terms}
\begin{lemma}[Fixing $q\bmod (k!)^k$ fixes the $i\ge2$ fractional parts]\label{lem:cong-stable}
Fix $k\ge1$ and set $D:=(k!)^k$. Let $1\le i\le k$.
Then $\sigma_{-k}(i)=\sigma_k(i)/i^k$ has denominator dividing $i^k\mid D$.
If $q_1\equiv q_2\pmod D$ then
\[\sigma_{-k}(i)\bigl(P_{k,i}(q_1)-P_{k,i}(q_2)\bigr)\in\mathbb{Z}.
\]
In particular, for every $2\le i\le k$ the fractional part of $\sigma_{-k}(i)P_{k,i}(q)$ depends only on $q\bmod D$.
\end{lemma}
\begin{proof}
We have $\sigma_{-k}(i)=\sigma_k(i)/i^k$ with $\sigma_k(i)\in\mathbb{Z}$, so the denominator divides $i^k$.
Since $P_{k,i}\in\mathbb{Z}[x]$, from $q_1\equiv q_2\pmod{i^k}$ we get $P_{k,i}(q_1)\equiv P_{k,i}(q_2)\pmod{i^k}$ and hence
\[\frac{\sigma_k(i)}{i^k}\bigl(P_{k,i}(q_1)-P_{k,i}(q_2)\bigr)\in\mathbb{Z}.
\]
The assumption $q_1\equiv q_2\pmod D$ implies $q_1\equiv q_2\pmod{i^k}$ because $i^k\mid D$.
\end{proof}

\subsubsection*{5.5 Conditional irrationality for all $k$}
\begin{theorem}[Conditional resolution under Schinzel~H]\label{thm:schinzel-conditional}
Assume Schinzel's hypothesis~H. Then for every integer $k\ge3$ the number
\[\alpha_k=\sum_{n=1}^{\infty}\frac{\sigma_k(n)}{n!}
\]
is irrational. (Together with Round~1, this yields irrationality for all $k\ge1$.)
\end{theorem}
\begin{proof}
Fix $k\ge3$ and suppose for contradiction that $\alpha_k=a/b\in\mathbb{Q}$ in lowest terms.

\medskip
\noindent\textbf{Step 1: truncate the integer tail.}
For $n>b$, Lemma~\ref{lem:tail-int-falling} gives $T_k(n)\in\mathbb{Z}$ where
\[T_k(n)=\sum_{\nu\ge n}\frac{\sigma_k(\nu)}{(\nu)_{\nu-n+1}}.
\]
Write
\[S_k(n):=\sum_{\nu=n}^{n+k-1}\frac{\sigma_k(\nu)}{(\nu)_{\nu-n+1}}
=\sum_{i=1}^{k}\frac{\sigma_k(n+i-1)}{(n+i-1)_i}.
\]
Then
\[\|S_k(n)\|\le \sum_{\nu\ge n+k}\frac{\sigma_k(\nu)}{(\nu)_{\nu-n+1}}\le \frac{C_k}{n}
\qquad(n\ge2)
\]
by Lemma~\ref{lem:discarded-tail}.

\medskip
\noindent\textbf{Step 2: pick special $q$ using Schinzel~H.}
Fix a prime $p>k$ and set $D:=(k!)^k$.

\emph{(a) A prime $q_2\equiv1\pmod D$ with controlled neighbors.}
Consider the $k$ linear polynomials
\[f_1(t)=Dt+1,\qquad f_i(t)=\frac{D}{i}t+1\quad(2\le i\le k).
\]
They have positive leading coefficients. For any prime $\ell\le k$ we have $\ell\mid D$, hence each $f_i(t)\equiv1\pmod\ell$ and in particular $\ell\nmid\prod_i f_i(t)$ for all $t$. For $\ell>k$, there are at most $k$ forbidden classes modulo $\ell$ (one for each $f_i$), so since $\ell>k$ there exists $t$ with $\ell\nmid\prod_i f_i(t)$. Thus the local nondivisibility condition of Schinzel~H holds, and we obtain infinitely many $t$ for which all $f_i(t)$ are prime.

Fix such a $t$ and put $q_2:=f_1(t)=Dt+1$ (a prime). For $2\le i\le k$,
\[q_2+i-1=Dt+i=i\Bigl(\frac{D}{i}t+1\Bigr)=i\,f_i(t),\]
so $q_2+i-1=i\cdot r_i$ with $r_i:=f_i(t)$ prime and $r_i>i$, hence $\gcd(i,r_i)=1$.

\emph{(b) A composite $q_1=p\cdot\text{prime}$ in the same residue class, with the same neighbor control.}
Choose $a\in\{1,\dots,D\}$ such that
\[pa\equiv1\pmod D.
\]
(This is possible since $\gcd(p,D)=1$.)
Consider the $k$ linear polynomials
\[g_1(t)=Dt+a,\qquad g_i(t)=\frac{pD}{i}t+\frac{pa+i-1}{i}\quad(2\le i\le k).
\]
These have integer coefficients because $D$ is divisible by $i$ and $pa\equiv1\pmod D$ implies $pa+i-1\equiv i\pmod D$, hence $i\mid(pa+i-1)$.
As above, for primes $\ell\le k$ we have $\ell\mid D$ so $g_1(t)\equiv a\not\equiv0\pmod\ell$ (since $\gcd(a,D)=1$) and $g_i(t)\equiv (pa+i-1)/i\equiv1\pmod\ell$ for $i\ge2$. For $\ell>k$ there are at most $k$ forbidden classes, so again the local nondivisibility condition holds. Schinzel~H yields infinitely many $t$ for which all $g_i(t)$ are prime.

Fix such a $t$ with $g_1(t)>p$ and put $r:=g_1(t)$ (a prime) and $q_1:=pr$.
Then $q_1\equiv pa\equiv1\pmod D$. Moreover, for $2\le i\le k$,
\begin{align*}
q_1+i-1 &= p(Dt+a)+i-1 = pDt+(pa+i-1)\\
&= i\Bigl(\frac{pD}{i}t+\frac{pa+i-1}{i}\Bigr)= i\,g_i(t),
\end{align*}
so $q_1+i-1=i\cdot g_i(t)$ with $g_i(t)$ prime and $>i$, hence $\gcd(i,g_i(t))=1$.
Finally, since $r>p$, we have $p^2\nmid q_1$.

\medskip
\noindent\textbf{Step 3: evaluate the truncated sum at $q_2$ and $q_1$.}
For $q\in\{q_1,q_2\}$ and $2\le i\le k$, write $q+i-1=i\cdot r_i$ with $r_i$ prime and $\gcd(i,r_i)=1$.
By multiplicativity of $\sigma_k$,
\[\sigma_k(q+i-1)=\sigma_k(i)\sigma_k(r_i)=\sigma_k(i)(1+r_i^k)=\sigma_k(i)r_i^k+\sigma_k(i).
\]
Since $r_i^k=(q+i-1)^k/i^k$, this gives
\[\sigma_k(q+i-1)=(q+i-1)^k\sigma_{-k}(i)+O_k(1).
\]
For $q=q_2$ (prime) we also have
\[\sigma_k(q_2)=1+q_2^k=q_2^k\sigma_{-k}(1)+1,
\]
and for $q=q_1=pr$ with $r$ prime,
\[\sigma_k(q_1)=\sigma_k(p)\sigma_k(r)=(1+p^k)(1+r^k)=r^k\sigma_k(p)+\sigma_k(p)=q_1^k\sigma_{-k}(p)+O_k(1).
\]
Dividing by the denominators $(q+i-1)_i\asymp q^i$ and summing $i=1,\dots,k$ yields
\begin{align*}
S_k(q_2) &= \sum_{i=1}^{k}\sigma_{-k}(i)\,\frac{(q_2+i-1)^k}{(q_2+i-1)_i}+O_k\!\left(\frac{1}{q_2}\right),\\
S_k(q_1) &= \sigma_{-k}(p)\,\frac{q_1^k}{(q_1)_1}+\sum_{i=2}^{k}\sigma_{-k}(i)\,\frac{(q_1+i-1)^k}{(q_1+i-1)_i}+O_k\!\left(\frac{1}{q_1}\right).
\end{align*}
Applying Lemma~\ref{lem:poly-division} to each ratio gives integer polynomials $P_{k,i}$ with
\[\frac{(q+i-1)^k}{(q+i-1)_i}=P_{k,i}(q)+O_k\!\left(\frac{1}{q}\right),
\]
hence
\begin{align*}
S_k(q_2) &= \sum_{i=1}^{k}\sigma_{-k}(i)P_{k,i}(q_2)+O_k\!\left(\frac{1}{q_2}\right),\\
S_k(q_1) &= \sigma_{-k}(p)P_{k,1}(q_1)+\sum_{i=2}^{k}\sigma_{-k}(i)P_{k,i}(q_1)+O_k\!\left(\frac{1}{q_1}\right).
\end{align*}
Combining with Step~1 (the $\|S_k(q)\|\ll 1/q$ bound) and enlarging constants, we obtain
\begin{align}
\Bigl\|\sum_{i=1}^{k}\sigma_{-k}(i)P_{k,i}(q_2)\Bigr\| &\le \frac{A_k}{q_2},\label{eq:near-int-q2}\\
\Bigl\|\sigma_{-k}(p)P_{k,1}(q_1)+\sum_{i=2}^{k}\sigma_{-k}(i)P_{k,i}(q_1)\Bigr\| &\le \frac{A_k}{q_1},\label{eq:near-int-q1}
\end{align}
for some constant $A_k>0$ depending only on $k$.

\medskip
\noindent\textbf{Step 4: cancel the $i\ge2$ terms using $q_1\equiv q_2\equiv1\pmod D$.}
Set
\[E(q):=\sum_{i=2}^{k}\sigma_{-k}(i)P_{k,i}(q).
\]
Since $q_1\equiv q_2\pmod D$, Lemma~\ref{lem:cong-stable} implies $E(q_1)-E(q_2)\in\mathbb{Z}$, hence $\|E(q_1)\|=\|E(q_2)\|$.
Subtracting \eqref{eq:near-int-q2} from \eqref{eq:near-int-q1} and using the triangle inequality yields
\begin{equation}\label{eq:key-diff}
\bigl\|\sigma_{-k}(p)P_{k,1}(q_1)-\sigma_{-k}(1)P_{k,1}(q_2)\bigr\|\le \frac{A_k}{q_1}+\frac{A_k}{q_2}.
\end{equation}
Now $P_{k,1}(x)=x^{k-1}$ (Lemma~\ref{lem:poly-division}) and $\sigma_{-k}(1)=1$, while $q_2^{k-1}\in\mathbb{Z}$.
Therefore the left-hand side of \eqref{eq:key-diff} equals
\[\bigl\|(\sigma_{-k}(p)-1)q_1^{k-1}\bigr\|=\Bigl\|\frac{q_1^{k-1}}{p^k}\Bigr\|.
\]
Since $q_1=pr$ with $r$ prime and $p^2\nmid q_1$, we have
\[\frac{q_1^{k-1}}{p^k}=\frac{r^{k-1}}{p}\notin\mathbb{Z},
\]
so $\bigl\|q_1^{k-1}/p^k\bigr\|\ge 1/p$ (in particular $\ge p^{-k}$).

Choose $q_1$ so large that $2A_k/q_1<1/p$ and then choose $q_2>q_1$ (possible since both sequences are infinite). The right-hand side of \eqref{eq:key-diff} is then $<1/p$, contradicting the lower bound $\ge 1/p$.

This contradiction shows that $\alpha_k\notin\mathbb{Q}$.
\end{proof}

\subsection*{6) ADVERSARIAL VERIFICATION}
\begin{itemize}
\item \textbf{Quantifiers.} The contradiction uses that there are \emph{arbitrarily large} $q_1$ of the form $p\times\text{prime}$ and \emph{arbitrarily large} primes $q_2$, each satisfying the neighbor-factorisation constraints and the same congruence class $\bmod D$. This is exactly what Schinzel~H provides for the two explicit linear systems.
\item \textbf{Congruence stability.} Lemma~\ref{lem:cong-stable} uses only that $P_{k,i}\in\mathbb{Z}[x]$ and $i^k\mid D$. The choice $D=(k!)^k$ is overkill but safe.
\item \textbf{Nonzero rational isolated.} The isolated number is $q_1^{k-1}/p^k=r^{k-1}/p$. Ensuring $p^2\nmid q_1$ (equivalently $r\ne p$) makes it non-integer, giving a fixed positive lower bound on its distance to $\mathbb{Z}$.
\item \textbf{Error terms.} All $O_k(1/q)$ errors are absorbed into the constants $A_k/q$. The decisive step only needs $q_1$ large enough compared to $A_k p$.
\item \textbf{Interaction with Round~1.} Nothing here modifies the $k=1,2$ unconditional proofs; the new conditional theorem is used only for $k\ge3$.
\end{itemize}

\subsection*{7) FINAL}
\textbf{UNRESOLVED (BUT STRICTLY ADVANCED).}
Unconditionally the problem remains open for general $k$ (beyond the known cases $k\le4$). However, the above gives a complete, gap-free proof of a minimal corrected statement: \emph{assuming Schinzel's hypothesis~H, $\alpha_k$ is irrational for every $k\ge1$} (and in particular for all $k\ge3$ where Round~1's elementary method fails).

\subsection*{8) COMPLETION ESTIMATE (MANDATORY)}
COMPLETION: 60\%

\subsection*{9) REFERENCES}
\begin{thebibliography}{9}
\bibitem{SchlagePuchta11}
J.-C. Schlage-Puchta,
\emph{The irrationality of a number theoretical series},
ArXiv preprint arXiv:1105.1452 (2011).
\end{thebibliography}
