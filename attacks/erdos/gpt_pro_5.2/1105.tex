
\noindent\textbf{FORMAL RESTATEMENT.}
Let $G$ be a fixed graph.
The anti-Ramsey number $\mathrm{AR}(n,G)$ is the maximum number of colors in an edge-coloring of the complete graph $K_n$ with no \emph{rainbow} copy of $G$ (a copy in which all edges have distinct colors).
The problem asks for exact asymptotic formulas for $\mathrm{AR}(n,C_k)$ and $\mathrm{AR}(n,P_k)$.

\medskip
\noindent\textbf{QUICK LITERATURE/CONTEXT CHECK (from the problem file only).}
The text states that Erd\H{o}s--Simonovits--S\'{o}s proved $\mathrm{AR}(n,C_3)=n-1$, conjectured formulas for $C_k$ and $P_k$, and that partial proofs exist in the literature (Simonovits--S\'{o}s 1984; Yuan 2021 announced).
I do not use results beyond what is explicitly stated.

\medskip
\noindent\textbf{ATTACK PLAN.}
\begin{itemize}
\item Fully prove the base case $\mathrm{AR}(n,C_3)=n-1$ as an illustration of the method.
\item For general $k$, give only sanity checks / partial observations (no claim of solving the conjectures).
\end{itemize}

\medskip
\noindent\textbf{WORK.}

\medskip
\noindent\textbf{Claim (proved): $\mathrm{AR}(n,C_3)=n-1$ for all $n\ge 3$.}

\medskip
\noindent\textbf{Lemma 1 (lower bound construction).}
For every $n\ge 3$ there exists an edge-coloring of $K_n$ using $n-1$ colors with no rainbow triangle.

\noindent\emph{Proof.}
Label the vertices $v_1,\dots,v_n$.
Color each edge $v_iv_j$ with the smaller index: $\mathrm{col}(v_iv_j):=\min\{i,j\}$.
Then only colors $1,2,\dots,n-1$ appear.
Consider any triangle on vertices $v_i,v_j,v_k$ with $i<j<k$.
The edges $v_iv_j$ and $v_iv_k$ both receive color $i$, while $v_jv_k$ receives color $j$.
Thus the triangle uses exactly two colors and is not rainbow.
Hence $\mathrm{AR}(n,C_3)\ge n-1$. \qed

\medskip
\noindent\textbf{Lemma 2 (structure around a vertex).}
Let $K_n$ be edge-colored with no rainbow triangle.
Fix a vertex $v$ and partition the other vertices into classes by the color of the edge to $v$.
If $x$ and $y$ lie in different classes, so $\mathrm{col}(vx)\ne \mathrm{col}(vy)$, then the edge $xy$ must have color either $\mathrm{col}(vx)$ or $\mathrm{col}(vy)$.

\noindent\emph{Proof.}
Let $a:=\mathrm{col}(vx)$ and $b:=\mathrm{col}(vy)$ with $a\ne b$.
If $\mathrm{col}(xy)$ were some color $c\notin\{a,b\}$, then the triangle $vxy$ would have three pairwise distinct edge colors $a,b,c$, hence would be rainbow, contradicting the hypothesis.
Therefore $\mathrm{col}(xy)\in\{a,b\}$. \qed

\medskip
\noindent\textbf{Lemma 3 (upper bound $\le n-1$ by induction).}
Any edge-coloring of $K_n$ with no rainbow triangle uses at most $n-1$ distinct colors.

\noindent\emph{Proof.}
We argue by induction on $n$.
The case $n=3$ is direct: a coloring of $K_3$ with $3$ distinct colors is itself a rainbow triangle, so at most $2=n-1$ colors are possible.

Assume the statement for all smaller complete graphs.
Consider a rainbow-triangle-free coloring of $K_n$.
Fix a vertex $v$ and let the partition of $V(K_n)\setminus\{v\}$ into color-classes be $V_1\sqcup\cdots\sqcup V_r$, where all edges from $v$ to vertices in $V_i$ have the same color $c_i$, and the colors $c_1,\dots,c_r$ are distinct.
Then exactly $r$ distinct colors appear on edges incident to $v$.
By Lemma 2, every edge between two different parts $V_i$ and $V_j$ has color in $\{c_i,c_j\}$, so no new colors appear on inter-part edges.
Thus any color not among $\{c_1,\dots,c_r\}$ can only appear on edges \emph{inside} some part $V_i$.

For each $i$, the induced coloring on the complete graph $K_{|V_i|}$ (on vertices $V_i$) also has no rainbow triangle (any rainbow triangle inside $V_i$ would be one in $K_n$).
By the induction hypothesis, the number of colors appearing on edges with both endpoints in $V_i$ is at most $|V_i|-1$.
Therefore the total number of distinct colors in the whole coloring is at most
\[
\underbrace{r}_{\text{colors on }v\text{-edges}}\ +\ \sum_{i=1}^r \underbrace{(|V_i|-1)}_{\text{colors internal to }V_i}
\ =\ r+\Bigl(\sum_{i=1}^r |V_i|\Bigr)-r
\ =\ (n-1).
\]
This completes the induction. \qed

Lemmas 1 and 3 give $\mathrm{AR}(n,C_3)=n-1$.

\medskip
\noindent\textbf{FAST REALITY CHECK.}
For $n=3$, the formula gives $\mathrm{AR}(3,C_3)=2$, which matches the fact that using $3$ colors on the three edges makes a rainbow triangle.
For $n=4$, the formula predicts $3$, and Lemma 1 provides an explicit $3$-coloring.

\medskip
\noindent\textbf{VERIFICATION.}
\begin{itemize}
\item Lemma 2 was checked on a single triangle $vxy$ and uses only the definition of rainbow.
\item In Lemma 3, the bound counts colors by a union bound (colors outside $\{c_i\}$ occur only inside parts), so possible overlaps only strengthen the $\le$ inequality.
\item Induction base case $n=3$ is correct.
\end{itemize}

\medskip
\noindent\textbf{FINAL: \textbf{UNRESOLVED}.}
(i) \emph{Strongest proved partial result here.} The case $k=3$ is completely solved: $\mathrm{AR}(n,C_3)=n-1$ with a full proof above.

(ii) \emph{First gap (crisp).} I do not have a proof or disproof of the stated conjectural formula for $\mathrm{AR}(n,C_k)$ for any fixed $k\ge 4$, nor for the general path formula for all $n\ge k\ge 5$.

(iii) \emph{Top 3 next moves.}
1. For fixed small $k$ (e.g. $k=4,5$), attempt a complete proof by extending the vertex-partition (Gallai) structure used for triangles to forbid rainbow $C_k$.
2. Run a computer search for small $(n,k)$ to test the conjectured linear formula for $\mathrm{AR}(n,C_k)$ and locate extremal colorings.
3. Prove matching upper and lower bounds up to $O(1)$ for $\mathrm{AR}(n,P_k)$ by designing extremal colorings and stability arguments.

(iv) \emph{Minimal counterexample structure.} A minimal counterexample to the conjectured formulas would be a coloring with more colors than predicted but still avoiding a rainbow $C_k$ (or $P_k$); such a coloring would likely exhibit a recursive Gallai-type partition where inter-part edges use few colors, while many colors are concentrated inside parts without creating the forbidden rainbow structure.


