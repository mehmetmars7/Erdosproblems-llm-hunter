
1) FORMAL RESTATEMENT

Let $A\subset\mathbb R$ be a finite set with $|A|=n$.

A subset $B\subseteq A$ is called \emph{dissociated} if for all subsets $S,T\subseteq B$, the implication
\[
\sum_{b\in S} b = \sum_{b\in T} b \implies S=T
\]
holds. Equivalently, the $2^{|B|}$ subset sums $\sum_{b\in S} b$ (with $S\subseteq B$) are all distinct.

Define $f(n)$ to be the largest integer $k$ such that \emph{every} $n$-element set $A\subset\mathbb R$ contains a dissociated subset $B\subseteq A$ with $|B|\ge k$.

The question asks for estimates of $f(n)$, and in particular whether
\[
 f(n)\ge \lfloor \log_2 n\rfloor
\]
for all $n$.

2) QUICK LITERATURE/CONTEXT CHECK

The problem text states that Erd\H{o}s noted a greedy algorithm gives $f(n)\ge \lfloor \log_3 n\rfloor$. I reprove this greedy bound below.

No other external results are invoked.

3) ATTACK PLAN

Proof-track ideas:
- Reprove the greedy $\log_3 n$ lower bound carefully by controlling how many elements are “forbidden” at each step.
- Seek an upper bound on $f(n)$ by constructing a specific $n$-element set $A$ in which every dissociated subset has bounded size (e.g. $A=\{1,2,\dots,n\}$).

Disproof-track ideas:
- If $f(n)\ge \lfloor\log_2 n\rfloor$ is false, search for a small $n$ and an explicit set $A$ with no dissociated subset that large.

4) WORK

FAST REALITY CHECK (computation over integer examples):

(i) For the specific set $A=\{1,2,\dots,n\}$, brute force found the maximum dissociated subset size for $n\le 20$:
\begin{verbatim}
 n : max |B| with B dissociated and B ⊆ {1,...,n}
 1:1
 2:2
 3:2
 4:3
 5:3
 6:3
 7:4
 8:4
 9:4
10:4
11:4
12:4
13:5
14:5
15:5
16:5
17:5
18:5
19:5
20:5
\end{verbatim}
This matches $\lfloor \log_2 n\rfloor+1$ for $2\le n\le 20$.

(ii) I also brute-forced, for each $n\le 8$, the minimum (over all $n$-subsets of $\{1,2,\dots,2n\}$) of the maximum dissociated subset size. The minimizing examples were the initial segments $\{1,\dots,n\}$:
\begin{verbatim}
 n : min_A max_{B⊆A dissociated} |B|   (searched over A⊆{1,...,2n}, |A|=n)
 1:1
 2:2
 3:2
 4:3
 5:3
 6:3
 7:4
 8:4
\end{verbatim}
This is consistent with (but does not prove) a logarithmic lower bound.

Lemma 963.1 (Equivalent “$\{-1,0,1\}$-relation” formulation).
A finite set $B\subset\mathbb R$ is dissociated (subset sums all distinct) if and only if the only solution to
\[
\sum_{b\in B} \varepsilon_b\, b = 0,\qquad \varepsilon_b\in\{-1,0,1\}
\]
is the trivial one $\varepsilon_b=0$ for all $b\in B$.

Proof.
("Only if") Suppose $B$ is dissociated and $\sum_{b\in B}\varepsilon_b b=0$ with $\varepsilon_b\in\{-1,0,1\}$. Let
\[
S=\{b\in B:\varepsilon_b=1\},\qquad T=\{b\in B:\varepsilon_b=-1\}.
\]
Then the equation becomes $\sum_{b\in S} b - \sum_{b\in T} b=0$, i.e. $\sum_{b\in S} b = \sum_{b\in T} b$. Dissociativity forces $S=T$, hence $S=T=\varnothing$ and all $\varepsilon_b=0$.

("If") Conversely, assume the only $\{-1,0,1\}$-relation is trivial. If two subsets $S\ne T$ of $B$ had equal sums, then
\[
0=\sum_{b\in S}b-\sum_{b\in T}b=
\sum_{b\in B}\varepsilon_b b
\]
where $\varepsilon_b=1$ for $b\in S\setminus T$, $\varepsilon_b=-1$ for $b\in T\setminus S$, and $\varepsilon_b=0$ otherwise. This gives a nontrivial $\{-1,0,1\}$-relation, contradiction. Thus all subset sums are distinct and $B$ is dissociated. $\square$

Lemma 963.2 (Greedy lower bound $f(n)\ge \lfloor \log_3 n\rfloor$).
For every integer $n\ge 1$, every $n$-element set $A\subset\mathbb R$ contains a dissociated subset of size at least $\lfloor \log_3 n\rfloor$.

Proof.
We build a dissociated set by a greedy algorithm.

Initialize $B_0=\varnothing$.
Suppose after $t\ge 0$ steps we have built a dissociated subset $B_t\subseteq A$ with $|B_t|=t$.
Define the set of all signed sums of $B_t$ by
\[
\Sigma_t:=\Big\{\sum_{b\in B_t}\varepsilon_b b:\ \ \varepsilon_b\in\{-1,0,1\}\Big\}.
\]
Because $B_t$ is dissociated, Lemma 963.1 implies that there is no nontrivial $\{-1,0,1\}$-relation among the elements of $B_t$.
If two sign patterns $\varepsilon,\varepsilon'\in\{-1,0,1\}^{B_t}$ produced the same sum, subtracting the equalities would give such a nontrivial relation, a contradiction.
Therefore the map $\varepsilon\mapsto\sum_{b\in B_t}\varepsilon_b b$ is injective and hence $|\Sigma_t|=3^t$.

Call an element $a\in A\setminus B_t$ \emph{forbidden} if $B_t\cup\{a\}$ is not dissociated.
By Lemma 963.1, $B_t\cup\{a\}$ fails to be dissociated iff there exists a nontrivial relation
\[
\varepsilon_a a + \sum_{b\in B_t}\varepsilon_b b = 0
\]
with all coefficients in $\{-1,0,1\}$. Since $B_t$ itself is dissociated, we must have $\varepsilon_a\neq 0$, so $\varepsilon_a=\pm 1$.
Rearranging gives $a\in\Sigma_t$ or $a\in-\Sigma_t$.
Thus every forbidden $a$ lies in $\Sigma_t\cup(-\Sigma_t)$, which has size at most $2|\Sigma_t|=2\cdot 3^t$.

If $n>2\cdot 3^t$, then $A$ has more than $2\cdot 3^t$ elements, so there exists some $a\in A$ that is not forbidden.
Choosing such an $a$ and setting $B_{t+1}:=B_t\cup\{a\}$ preserves dissociativity.

Now let $t_0:=\lfloor\log_3 n\rfloor$. If $t_0=0$ we are done. If $t_0\ge 1$, then $n\ge 3^{t_0}>2\cdot 3^{t_0-1}$ (because $3>2$), and hence for every $s=0,1,\dots,t_0-1$ we have
\[
2\cdot 3^s \le 2\cdot 3^{t_0-1} < n.
\]
Therefore $n>2\cdot 3^s$ at each stage $s$, so the greedy extension step can be performed $t_0$ times, producing a dissociated subset of size $t_0=\lfloor\log_3 n\rfloor$. $\square$

Lemma 963.3 (A general upper bound from an integer model set).
Let $A=\{1,2,\dots,n\}\subset\mathbb R$. If $B\subseteq A$ is dissociated and $|B|=t$, then
\[
2^t \le 1+ \sum_{b\in B} b \le 1+ tn.
\]
In particular,
\[
 t\le \log_2(1+tn) \le \log_2(1+n^2) < 2\log_2 n + 1\quad (n\ge 2).
\]
Consequently $f(n)\le 2\log_2 n +1$ for all $n\ge 2$.

Proof.
Because $B$ is dissociated, all $2^t$ subset sums $\sum_{b\in S} b$ (with $S\subseteq B$) are distinct. Since all $b\in B$ are positive integers, these subset sums are integers in the range
\[
0 \le \sum_{b\in S}b \le \sum_{b\in B} b.
\]
A set of distinct integers inside $[0,\sum_{b\in B} b]$ can have size at most $1+\sum_{b\in B} b$. Therefore
\[
2^t \le 1+\sum_{b\in B} b.
\]
Also $\sum_{b\in B} b\le t\cdot n$ since each $b\le n$. Combining gives $2^t\le 1+tn$.

Taking $\log_2$ yields $t\le \log_2(1+tn)$. Since trivially $t\le n$, we have $1+tn\le 1+n^2$, hence $t\le \log_2(1+n^2)<2\log_2 n +1$ for $n\ge 2$.

Finally, since $A$ is an explicit $n$-element set for which every dissociated subset has size $\le 2\log_2 n+1$, the worst-case guarantee $f(n)$ cannot exceed $2\log_2 n+1$. $\square$

5) VERIFICATION

- Lemma 963.1: converting equal subset sums to a $\{-1,0,1\}$-relation and back is reversible and uses only set differences.
- Lemma 963.2: the key step is bounding forbidden elements by $\Sigma_t\cup(-\Sigma_t)$ and using injectivity to get $|\Sigma_t|=3^t$.
  I checked that if $a\in\Sigma_t$ then there is a relation $a-\sum \varepsilon_b b=0$ producing a collision in subset sums after adding $a$.
- Lemma 963.3: the inequality $2^t\le 1+\sum_{b\in B}b$ relies on all subset sums being distinct integers in a bounded interval; this holds because $A\subset\mathbb Z_{>0}$ in this lemma.

6) FINAL

**UNRESOLVED**

(i) Strongest proved partial result: the greedy guarantee $f(n)\ge \lfloor\log_3 n\rfloor$ (Lemma 963.2) and an explicit universal upper bound $f(n)\le 2\log_2 n+1$ coming from $A=\{1,\dots,n\}$ (Lemma 963.3), plus the exact characterization of dissociativity via $\{-1,0,1\}$-relations (Lemma 963.1) and small-$n$ computations.

(ii) First gap (crisp): either prove $f(n)\ge \lfloor\log_2 n\rfloor$ for all $n$ (improving the base from $3$ to $2$ in Lemma 963.2), or exhibit an explicit counterexample set $A$ (for some $n$) with no dissociated subset of size $\lfloor\log_2 n\rfloor$.

(iii) Top 3 next moves:
1. Improve the greedy counting by tightening the forbidden-set bound $|\Sigma_t\cup(-\Sigma_t)|\le 2\cdot 3^t$; any reduction toward $2^t$ could move the base from $3$ toward $2$.
2. Search computationally (over integer sets and small real perturbations) for a smallest $n$ where the lower bound $\lfloor\log_2 n\rfloor$ might fail, i.e. a set with unusually small maximum dissociated subset.
3. Prove structural lemmas: characterize sets $A$ that minimize the maximum dissociated subset size, and test whether they are essentially initial segments or dense progressions.

(iv) Minimal counterexample structure (to “$f(n)\ge \lfloor\log_2 n\rfloor$”): a smallest $n$ and an explicit $n$-element set $A\subset\mathbb R$ such that every $B\subseteq A$ of size $\lfloor\log_2 n\rfloor$ admits a nontrivial $\{-1,0,1\}$-relation; heuristically such an $A$ would have many short additive relations among its elements.


