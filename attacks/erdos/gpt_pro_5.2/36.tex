FORMAL RESTATEMENT
Let $N\ge 1$ and let $A\sqcup B=\{1,2,\dots,2N\}$ be a partition with $|A|=|B|=N$.
For an integer $x$, define
\[
 r(x):=\#\{(a,b)\in A\times B: a-b=x\}.
\]
Equivalently, $r(x)=(1_A\ast 1_{-B})(x)$ is a cross-correlation count.
The minimum overlap problem asks for the optimal constant $c>0$ such that for all sufficiently large $N$, every such partition satisfies
\[
\max_x r(x)\ \ge\ cN.
\]

QUICK LITERATURE/CONTEXT CHECK
The problem statement records the trivial bound $c\ge 1/4$, the interval example giving $c\le 1/2$, and much sharper record bounds (currently $0.379005<c<0.380924$ in the problem text).
The exact value of $c$ remains open.

ATTACK PLAN
The quantity $r(x)$ is a difference-multiplicity function between $A$ and $B$.
Lower bounds on $\max_x r(x)$ can be obtained from averaging and, more powerfully, from Fourier analysis (bounding $\|1_A\ast 1_{-B}\|_\infty$ in terms of $L^2$ and structural constraints).
Upper bounds come from explicit constructions of partitions minimizing the maximum overlap; these are typically ``balanced''/``spread'' patterns.

WORK
\textit{Fast reality check (exact small $N$).}
For $1\le N\le 11$ I brute-forced all partitions (up to swapping $A$ and $B$) and computed
\(
M(N):=\min_{A\sqcup B}\max_x r(x).
\)
The exact values found are:
\[
\begin{array}{c|ccccccccccc}
N&1&2&3&4&5&6&7&8&9&10&11\\\hline
M(N)&1&1&2&2&3&3&3&4&4&5&5\\
M(N)/N&1.000&0.500&0.667&0.500&0.600&0.500&0.429&0.500&0.444&0.500&0.455
\end{array}
\]

\medskip
Lemma 36.1 (trivial averaging lower bound).
For every $N$ and every partition $A\sqcup B=\{1,\dots,2N\}$ with $|A|=|B|=N$,
\[
\max_x r(x)\ \ge\ \frac{N^2}{4N-1}\ >\ \frac{N}{4}.
\]
In particular, the minimum overlap constant satisfies $c\ge 1/4$.

Proof.
There are $N^2$ ordered pairs $(a,b)\in A\times B$.
Each such pair contributes to exactly one difference $x=a-b$, so
\(
\sum_x r(x)=N^2.
\)
The possible differences satisfy $-(2N-1)\le x\le 2N-1$, so there are exactly $4N-1$ values of $x$.
By averaging,
\(
\max_x r(x)\ge \frac{1}{4N-1}\sum_x r(x)=\frac{N^2}{4N-1}.
\)
The strict inequality $N^2/(4N-1)>N/4$ holds because $4N>4N-1$.
\qed

\medskip
Lemma 36.2 (interval example gives $c\le 1/2$ for even $N$).
Let $N$ be even and set
\[
A:=\left\{\frac{N}{2}+1,\frac{N}{2}+2,\dots,\frac{3N}{2}\right\},
\qquad
B:=\{1,2,\dots,2N\}\setminus A.
\]
Then $|A|=|B|=N$ and
\(
\max_x r(x)=N/2.
\)
In particular, the optimal constant satisfies $c\le 1/2$.

Proof.
Write $B=B_1\cup B_2$ where $B_1=\{1,\dots,N/2\}$ and $B_2=\{3N/2+1,\dots,2N\}$.
Fix an integer $x$.
For each $a\in A$, the equation $a-b=x$ determines $b=a-x$ uniquely.
Thus
\[
 r(x)=\#\{a\in A: a-x\in B\}.
\]
If $a-x\in B_1$, then $a-x\le N/2$, so $a\le x+N/2$.
Since $a\ge N/2+1$, there are at most $N/2$ choices of $a$ in the interval $[N/2+1,\min(3N/2,x+N/2)]$.
Similarly, if $a-x\in B_2$, then $a-x\ge 3N/2+1$, so $a\ge x+3N/2+1$.
Since $a\le 3N/2$, this can only happen when $x\le -1$, and in that case $a$ must lie in a subinterval of $A$ of length at most $N/2$.
More directly: for fixed $x$, the set of $a\in A$ for which $a-x\in B_1$ is an intersection of $A$ with an interval of length $N/2$, and likewise for $B_2$.
Because $B_1$ and $B_2$ are disjoint, these two subsets of $A$ are disjoint, and each has size at most $N/2$.
Hence $r(x)\le N/2$ for every $x$.
Finally, equality $r(x)=N/2$ is attained, e.g. at $x=N/2$:
then $a-x\in B_1$ exactly when $a\in\{N+1,\dots,3N/2\}$, which has size $N/2$.
Therefore $\max_x r(x)=N/2$.
\qed

VERIFICATION
Lemma 36.1 is a straightforward counting identity.
Lemma 36.2 was checked both by the interval-counting argument and by the brute-force computation for small even $N$ (which attains $M(N)=N/2$ for $N=2,4,6,8,10$).
The table of $M(N)$ for $N\le 11$ was computed by exhaustive enumeration of partitions.

FINAL
**UNRESOLVED**
(i) Strongest proved partial result here (from WORK): $\max_x r(x)\ge N^2/(4N-1)>N/4$ for every partition (Lemma 36.1), and an explicit construction shows $\max_x r(x)=N/2$ is possible for even $N$ (Lemma 36.2).
(ii) First gap: improve the universal lower bound on $\max_x r(x)$ from $N/4$ toward the record $\approx 0.379N$, and ultimately determine the optimal constant $c$.
(iii) Top 3 next moves:
  (1) Apply Fourier analysis to $1_A$ and $1_B$ to bound $\|1_A\ast 1_{-B}\|_\infty$ from below via $L^2$ mass and constraints coming from $A\sqcup B$.
  (2) Search computationally for near-extremal partitions for moderate $N$ and fit $M(N)/N$ to guess the limiting constant.
  (3) Investigate continuous relaxations (replace indicator functions by densities on $[0,1]$) to derive a convex program lower bound, then discretize.
(iv) Minimal counterexample structure: a partition minimizing $\max_x r(x)$ must ``spread'' the difference multiset $A-B$ as evenly as possible across the $4N-1$ possible differences, suggesting $A$ and $B$ should behave like pseudorandom sets with strong local balancing properties.
