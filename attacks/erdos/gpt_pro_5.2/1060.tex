% Erdos Problem #1060

\subsection*{FORMAL RESTATEMENT}
Let $\sigma(k)=\sum_{d\mid k} d$ be the sum of divisors of $k\ge 1$. For $n\ge 1$ define
\[
f(n):=\#\{k\in\mathbb Z_{\ge 1}:\ k\sigma(k)=n\}.
\]
The problem asks whether
\[
f(n)\le n^{o(1/\log\log n)}\quad\text{as }n\to\infty,
\]
and perhaps even $f(n)\le (\log n)^{O(1)}$.

\subsection*{QUICK LITERATURE/CONTEXT CHECK}
The problem is posed as an open question (Guy B11). I do not use or claim any results not stated in the problem text.

\subsection*{ATTACK PLAN}
Proof-track: derive structural constraints on $k$ from $k\sigma(k)=n$ (e.g. $k\mid n$, size bounds, multiplicativity) and try to show the equation has few solutions.
Disproof-track: search for $n$ with unusually many representations $n=k\sigma(k)$ by enumerating $k$ and counting collisions.

\subsection*{WORK}
\textbf{FAST REALITY CHECK.}
For $n\le 10{,}000$, a brute-force check over $k\le \lfloor\sqrt n\rfloor$ finds:
\begin{verbatim}
B=10000 max f(n)=2
n=336 has solutions k=[12,14]
n=5952 has solutions k=[48,62]
\end{verbatim}
Enumerating all $k\le 10^6$ and counting values of $k\sigma(k)$ gives the following empirical maximum multiplicity:
\begin{verbatim}
max f(n) observed for k<=1,000,000: 5
achieved at n=1584858562560
with solutions k=[624960, 640080, 696384, 708660, 713232]
\end{verbatim}

\medskip
\textbf{Lemma 1060.1 (divisor and size constraints).}
If $n>1$ and $k\sigma(k)=n$, then (a) $k\mid n$, and (b) $k<\sqrt n$.

\emph{Proof.}
(a) The equation $k\sigma(k)=n$ implies $k$ divides the left-hand side, hence $k\mid n$.
(b) For every $k\ge 1$ we have $\sigma(k)\ge k$ because $k$ itself is a divisor of $k$; if $k>1$ then $\sigma(k)\ge 1+k>k$.
Thus for $n>1$ any solution has $k\ge 2$ and
\[
n=k\sigma(k)\ge k\cdot k = k^2,
\]
with strict inequality $n>k^2$ since $\sigma(k)>k$ for $k>1$. Hence $k<\sqrt n$. \hfill$\square$

\medskip
\textbf{Lemma 1060.2 (trivial divisor-function upper bound).}
For $n>1$,
\[
f(n)\le \frac{d(n)}{2},
\]
where $d(n)$ is the number of positive divisors of $n$.

\emph{Proof.}
By Lemma 1060.1(a), every solution $k$ is a divisor of $n$. By Lemma 1060.1(b), every solution satisfies $k<\sqrt n$.
Among all divisors of $n$, the number of divisors strictly less than $\sqrt n$ is at most $d(n)/2$ because divisors pair as $k\leftrightarrow n/k$, with exactly one divisor at $\sqrt n$ only when $n$ is a perfect square.
Therefore $f(n)$, which counts only divisors $k<\sqrt n$ satisfying the equation, is at most $d(n)/2$. \hfill$\square$

\subsection*{VERIFICATION}
Lemma 1060.1 uses only the basic fact that $k$ divides $k\sigma(k)$ and that $\sigma(k)\ge k$ with strict inequality for $k>1$.
Lemma 1060.2 checks the only edge case where $k=\sqrt n$ could occur (perfect squares): Lemma 1060.1(b) rules out $k=\sqrt n$ for $n>1$ because $n>k^2$ for $k>1$.
The brute-force computations explicitly restricted to $k\le \sqrt n$ (for $n\le 10^4$) and to $k\le 10^6$ (for the larger collision search), matching Lemma 1060.1(b).

\subsection*{FINAL}
\textbf{UNRESOLVED}
\begin{enumerate}
\item[(i)] \textbf{Strongest proved partial result.}
Every solution $k$ of $k\sigma(k)=n$ satisfies $k\mid n$ and $k<\sqrt n$, hence $f(n)\le d(n)/2$ for $n>1$ (Lemmas 1060.1--1060.2). Empirically, for $k\le 10^6$ the largest multiplicity observed is $f(n)=5$ at $n=1584858562560$.
\item[(ii)] \textbf{First gap (crisp).}
Prove a nontrivial asymptotic upper bound of the form $f(n)\le (\log n)^{C}$ for all sufficiently large $n$, or even $f(n)\le \exp(o(\log n/\log\log n))$.
\item[(iii)] \textbf{Top 3 next moves.}
(1) Structural: classify solutions when $k$ has restricted prime factorization (e.g. prime powers), and show that each such class contributes very few $k$ to a fixed $n$.
(2) Collision control: prove that the map $k\mapsto k\sigma(k)$ is ``almost injective'' on large ranges by showing that $k\sigma(k)$ has a rigid prime-factor pattern determined by $k$.
(3) Computation: search for $n$ with larger $f(n)$ by enumerating $k$ beyond $10^6$ using a fast divisor-sum sieve and tracking multiplicities.
\item[(iv)] \textbf{Minimal counterexample structure.}
A counterexample to polylogarithmic bounds would require some $n$ having many distinct divisors $k<\sqrt n$ with $\sigma(k)=n/k$. Such an $n$ would plausibly be very ``smooth'' (many small prime factors) to admit many candidate divisors $k$.
\end{enumerate}


