
\subsection*{Erd\H{o}s Problem \#52}
\noindent\emph{Problem statement (from the problem file).}
Let $A$ be a finite set of integers. Is it true that for every $\epsilon>0$
\[
\max(|A+A|,|AA|)\gg_\epsilon |A|^{2-\epsilon}?
\]

\paragraph{FORMAL RESTATEMENT.}
For each $\epsilon>0$, does there exist a constant $C_\epsilon>0$ such that for every finite nonempty set $A\subset\mathbb{Z}$,
\[
\max\{|A+A|,|A\cdot A|\} \ge C_\epsilon\, |A|^{2-\epsilon},
\]
where $A+A:=\{a+a':a,a'\in A\}$ and $A\cdot A:=\{aa':a,a'\in A\}$?

\paragraph{QUICK LITERATURE/CONTEXT CHECK.}
The problem file describes known partial results and records, but I do not re-derive them here (they go far beyond elementary methods).

\paragraph{ATTACK PLAN.}
\emph{Proof track:} Not realistic with only elementary tools; instead, prove basic lower bounds that illustrate the phenomenon.
\emph{Disproof track:} Try to find constructions with both $|A+A|$ and $|AA|$ very small (compared to $|A|^{2-\epsilon}$); known folklore examples suggest one of the two typically expands.
\emph{Reality check:} Exhaustively search small sets in $\{1,2,\dots,m\}$ for small $|A|$ to see minimal possible values of $\max(|A+A|,|AA|)$.

\paragraph{WORK.}
\subparagraph{FAST REALITY CHECK (small search).}
Exhaustive search over all $k$-element subsets $A\subset\{1,2,\dots,12\}$ gave the following minimal values:
\[
\begin{array}{c|ccccc}
|A|=k & 2 & 3 & 4 & 5 & 6\\\hline
\min \max(|A+A|,|AA|) & 3 & 6 & 9 & 12 & 15
\end{array}
\]
For these parameters, minimizers were initial-segment type sets (e.g. $\{1,2,3,4\}$ at $k=4$).
This only indicates that $\max(|A+A|,|AA|)$ can be as small as $\Theta(k)$ in very small ambient ranges; it does not touch the conjectured $k^{2-\epsilon}$ growth in the unrestricted setting.

\subparagraph{Lemma 1 (trivial sumset growth in $\mathbb{Z}$).}
If $A\subset\mathbb{Z}$ is finite with $|A|=k$, then
\[
|A+A|\ge 2k-1.
\]

\emph{Proof.}
List $A$ in increasing order: $a_1<a_2<\dots<a_k$.
Then the $k$ sums $a_1+a_1<a_1+a_2<\dots<a_1+a_k$ are strictly increasing.
Also the $k$ sums $a_k+a_1<a_k+a_2<\dots<a_k+a_k$ are strictly increasing.
Moreover $a_1+a_k=a_k+a_1$ is the only possible overlap between these two chains.
Therefore the union contains at least $(k)+(k)-1=2k-1$ distinct sums. \qed

\subparagraph{Lemma 2 (trivial product-set growth when $A$ contains a nonzero element).}
If $A\subset\mathbb{Z}$ is finite, nonempty, and contains some $a_0\neq 0$, then
\[
|AA|\ge |A|.
\]

\emph{Proof.}
Consider the map $\psi:A\to AA$ given by $\psi(a)=a_0a$.
If $\psi(a)=\psi(a')$, then $a_0a=a_0a'$, and since $a_0\neq 0$ we have $a=a'$.
Thus $\psi$ is injective, so $|AA|\ge |\psi(A)|=|A|$. \qed

\subparagraph{Corollary 3 (a uniform linear lower bound).}
Every finite nonempty $A\subset\mathbb{Z}$ satisfies
\[
\max(|A+A|,|AA|)\ge 2|A|-1.
\]

\emph{Proof.}
If $A=\{0\}$, then $|A+A|=1=2|A|-1$.
Otherwise $A$ has a nonzero element and Lemma~1 already yields $\max(|A+A|,|AA|)\ge |A+A|\ge 2|A|-1$. \qed

\paragraph{VERIFICATION.}
\begin{itemize}
\item Lemma~1 is verified by an explicit strictly increasing chain argument.
\item Lemma~2 handles the only obstruction $a_0=0$ separately; if $A=\{0\}$ then $|AA|=1$ and the sumset bound still gives the stated linear lower bound.
\item The finite search is explicitly bounded to subsets of $\{1,\dots,12\}$ and is not claimed to be asymptotic evidence.
\end{itemize}

\paragraph{FINAL.} \textbf{UNRESOLVED.}
\begin{enumerate}
\item[(i)] \emph{Strongest proved partial result.} Elementary methods give only linear lower bounds such as $\max(|A+A|,|AA|)\ge 2|A|-1$ (Corollary~3).
\item[(ii)] \emph{First gap.} Prove any superlinear bound $\max(|A+A|,|AA|)\ge |A|^{1+c}$ in $\mathbb{Z}$ by self-contained arguments, let alone the conjectural $|A|^{2-\epsilon}$.
\item[(iii)] \emph{Top 3 next moves.}
(1) Target a concrete intermediate exponent with a self-contained proof for integers (e.g. show $\max(|A+A|,|AA|)\ge |A|^{4/3}$ under additional hypotheses such as all elements positive and $\gcd(A)=1$).
(2) Search computationally for small sets with unusually small $\max(|A+A|,|AA|)$ and record structural patterns (near-arithmetic or near-geometric progressions) as candidate extremizers.
(3) Prove structural dichotomies: if $|A+A|$ is small then $A$ has additive structure, which should force $|AA|$ to be large (and vice versa).
\item[(iv)] \emph{Minimal counterexample structure.} A counterexample to the conjecture for some $\epsilon>0$ would be a sequence of sets $A_k\subset\mathbb{Z}$ with $|A_k|\to\infty$ such that both $|A_k+A_k|$ and $|A_kA_k|$ are $o(|A_k|^{2-\epsilon})$; heuristic candidates would need to be simultaneously close to an arithmetic progression (small sumset) and a geometric progression (small product set), which is hard to reconcile.
\end{enumerate}

