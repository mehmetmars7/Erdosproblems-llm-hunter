
\section*{Erd\H{o}s Problem \#119}

\subsection*{1) FORMAL RESTATEMENT}
Let $(z_i)_{i\ge1}$ be a sequence of complex numbers with $|z_i|=1$ for all $i$.
For $n\ge1$ define the monic polynomial
\[
 p_n(z)=\prod_{i=1}^n (z-z_i).
\]
Define
\[
 M_n := \max_{|z|=1} |p_n(z)|.
\]
Questions:
\begin{enumerate}
\item Is $\limsup_{n\to\infty} M_n = \infty$?
\item Is there $c>0$ such that $M_n>n^c$ for infinitely many $n$?
\item Is there $c>0$ such that for all sufficiently large $n$,
\[
\sum_{k\le n} M_k > n^{1+c}?
\]
\end{enumerate}

\subsection*{2) QUICK LITERATURE/CONTEXT CHECK}
From the problem text:
\begin{itemize}
\item (1) is true (Wagner), with the weaker growth $M_n>(\log n)^c$ infinitely often.
\item (2) is true (Beck): $\max_{n\le N} M_n > N^c$ for some $c>0$.
\item (3) seems open.
\item There exist sequences with $M_n\le n+1$ for all $n$ (Erd\H{o}s) and even $M_n\ll n^{1-c}$ for some $c>0$ (Linden).
\end{itemize}
I do not re-prove (1) or (2).  I give basic quantitative identities and a sanity-check computation.

\subsection*{3) ATTACK PLAN}
To force large $M_n$, one might try to prove that $p_n$ cannot stay uniformly small on the unit circle for too long, perhaps using Fourier/energy or potential theory.  For the sum question (3), one would aim for a lower bound on many $M_k$ simultaneously.
Here I only establish two ``first principles'' lemmas about averages and $L^2$ norms.

\subsection*{4) WORK}
\paragraph{Lemma 119.1 (mean of $\log|p_n|$ on the circle is $0$).}
For every $n\ge1$,
\[
\frac{1}{2\pi}\int_0^{2\pi} \log\big|p_n(e^{it})\big|\,dt = 0.
\]

\emph{Proof.}
Write $p_n(z)=\prod_{j=1}^n (z-z_j)$.  Then
\[
\log|p_n(e^{it})| = \sum_{j=1}^n \log|e^{it}-z_j|.
\]
Fix a single $z_j$ with $|z_j|=1$ and rotate so $z_j=1$ (rotation preserves the integral over $t$).  Then
\[
\frac{1}{2\pi}\int_0^{2\pi} \log|e^{it}-1|\,dt
=\frac{1}{2\pi}\int_0^{2\pi} \log\big(2|\sin(t/2)|\big)\,dt.
\]
The classical integral identity
\(
\int_0^{\pi} \log(\sin u)\,du = -\pi\log 2
\)
implies
\(
\int_0^{2\pi} \log(2|\sin(t/2)|)\,dt = 0.
\)
Therefore each summand has mean $0$, and summing over $j=1,\dots,n$ gives the claim.
\qed

\paragraph{Lemma 119.2 ($L^2$ lower bound on $M_n$).}
Let $p_n(z)=\sum_{k=0}^n a_k z^k$ (so $a_n=1$ and $a_0=(-1)^n z_1\cdots z_n$ has modulus $1$).  Then
\[
\frac{1}{2\pi}\int_0^{2\pi} |p_n(e^{it})|^2\,dt = \sum_{k=0}^n |a_k|^2 \ge 2,
\]
and hence
\[
M_n\ge \sqrt{2}.
\]

\emph{Proof.}
By Parseval's identity for trigonometric polynomials,
\[
\frac{1}{2\pi}\int_0^{2\pi} \Big|\sum_{k=0}^n a_k e^{ikt}\Big|^2 dt = \sum_{k=0}^n |a_k|^2.
\]
Since $a_n=1$ and $|a_0|=|z_1\cdots z_n|=1$, we have $\sum_{k=0}^n |a_k|^2\ge |a_0|^2+|a_n|^2=2$.
Finally, $M_n^2=\max_{|z|=1}|p_n(z)|^2\ge$ the average value of $|p_n|^2$ on the circle, so $M_n^2\ge2$.
\qed

\paragraph{FAST REALITY CHECK (numerical sampling for one explicit sequence).}
Take $z_j = e^{2\pi i\, j\varphi}$ where $\varphi=(\sqrt5-1)/2$ (an irrational).  For $1\le n\le 200$, I approximated $M_n$ by evaluating $|p_n(z)|$ on a uniform grid of $4096$ points on the unit circle.  This gives a lower bound on the true $M_n$.

The sampled values (selected) were:
\[
\begin{array}{c|c}
 n & \text{sampled }M_n\\\hline
 10 & 5.15\cdots\\
 20 & 19.50\cdots\\
 50 & 12.64\cdots\\
 100 & 15.17\cdots\\
 200 & 15.69\cdots
\end{array}
\]
and the maximum sampled value over $1\le n\le 200$ was about $134.6$ (at $n=143$).
This is consistent with unbounded growth in this example, but does not constitute a proof.

\subsection*{5) VERIFICATION}
\begin{itemize}
\item Lemma~119.1 uses the standard integral $\int_0^{\pi} \log(\sin u)du=-\pi\log2$.  The integrand has an integrable singularity; the identity is classical and can be checked by Fourier series.
\item Lemma~119.2: Parseval is exact; the step $M_n^2\ge$ average is just ``max $\ge$ mean''.
\item Numerical check is only a sanity check: grid sampling underestimates the true maximum.
\end{itemize}

\subsection*{6) FINAL}
\textbf{UNRESOLVED.}

(i) \emph{Strongest fully proved partial result obtained here.}
The exact mean-zero identity for $\log|p_n|$ on the circle (Lemma~119.1) and the uniform lower bound $M_n\ge\sqrt2$ (Lemma~119.2).

(ii) \emph{Exact first gap.}
A proof (for the open part (3)) that $\sum_{k\le n} M_k$ must grow faster than $n^{1+c}$ for some $c>0$, uniformly for all large $n$.

(iii) \emph{Top 3 next moves.}
\begin{enumerate}
\item Try to turn Lemma~119.1 into a quantitative lower bound on large deviations of $\log|p_n|$ (entropy/large deviations) implying many large $M_n$.
\item Use potential theory on the circle: relate $\log|p_n|$ to the discrete logarithmic potential of the points $z_i$.
\item Search for worst-case sequences (those with very small $M_n$) and attempt to prove that even there, partial sums $\sum M_k$ must grow superlinearly.
\end{enumerate}

(iv) \emph{Minimal counterexample structure.}
A counterexample to (3) would be a sequence $(z_i)$ for which $M_n$ is often small (e.g. $M_n\ll n^{\varepsilon}$ for all large $n$) and additionally $\sum_{k\le n} M_k$ stays $O(n^{1+c_0})$ for every fixed $c_0>0$.


