## 1) FORMAL RESTATEMENT

### Definitions

For an integer (m\ge 1), let ([m]={1,2,\dots,m}).
Let (K_m^{(3)}) denote the **complete 3-uniform hypergraph** on vertex set ([m]), i.e. the 3-uniform hypergraph with edge set
[
E(K_m^{(3)})=\binom{[m]}{3}.
]
A **2-colouring** of the edges of (K_m^{(3)}) is a function
[
\chi:\binom{[m]}{3}\to{\text{red},\text{blue}}.
]
For an integer (n\ge 1), let (K_n^{(3)}) denote the complete 3-uniform hypergraph on (n) vertices.

A subset (S\subseteq [m]) with (|S|=n) is a **monochromatic copy of (K_n^{(3)})** (under (\chi)) if (\chi) is constant on (\binom{S}{3}); i.e. every triple in (S) is coloured the same.

Define (R_3(n)) to be the least integer (m) such that **every** 2-colouring (\chi:\binom{[m]}{3}\to{\text{red},\text{blue}}) contains a monochromatic copy of (K_n^{(3)}). Equivalently,
[
R_3(n)=\min\Bigl{m\in\mathbb N_{\ge 1}:\ \forall \chi:\binom{[m]}{3}\to{\text{red},\text{blue}}\ \exists S\subseteq[m],,|S|=n,\ \chi\text{ constant on }\binom{S}{3}\Bigr}.
]

### The “literal statement” to be proved/disproved

The prompt asks:

> “Is there some constant (c>0) such that (R_3(n)\ge 2^{2^{cn}})?”

As a declarative claim (the usual way to formalize such a question), the **literal assertion** is:
[
(\star)\qquad \exists c>0\ \forall n\in\mathbb N_{\ge 1},\quad R_3(n)\ \ge\ 2^{,2^{c n}}.
]

### Ambiguity / minimal corrected statement (standard convention)

In the modern literature this is usually posed only for **(n\ge 4)** (since (n\le 3) are trivial edge-cases for 3-uniform cliques), e.g. as an Erdős conjecture. 

A minimal corrected version consistent with that convention is:
[
(\star_{\ge 4})\qquad \exists c>0\ \forall n\in\mathbb N,\ n\ge 4,\quad R_3(n)\ \ge\ 2^{,2^{c n}}.
]

What I will do below: **disprove the literal assertion ((\star))** (as written, with (n\ge 1)). I will also note that ((\star_{\ge 4})) is the intended open problem.

---

## 2) QUICK LITERATURE/CONTEXT CHECK

The Erdős Problems page states exactly this question and describes it as open, citing Erdős–Hajnal–Rado (1965) bounds and related multicolour results. ([Erdős Problems][1])

A standard survey (Mubayi) records the best-known general bounds
[
2^{c_1 n^2} < r_3(n,n) < 2^{2^{c_2 n}}
]
and states as “Conjecture (Erdős)” that for (n\ge 4),
[
r_3(n,n) > 2^{2^{c n}}
]
for an absolute constant (c>0). 

So the *intended* conjecture is ((\star_{\ge 4})), and it is widely treated as open. 

---

## 3) ATTACK PLAN

### Proof-track ideas (why hard)

A genuine proof of ((\star_{\ge 4})) would require a **doubly-exponential** lower bound for the 2-colour diagonal 3-uniform Ramsey number, far beyond the classical (2^{\Theta(n^2)}) lower bound. Surveys emphasize this as a major open difficulty. 

### Disproof/construction track

Because the statement is written without a restriction like “for all (n\ge 4)” or “for all sufficiently large (n),” the fastest way to falsify the **literal** assertion ((\star)) is:

1. Compute (R_3(1)) from the definition.
2. Observe (2^{2^{c\cdot 1}} > 1) for any (c>0).
3. Conclude the inequality fails at (n=1).

I will execute this plan and give a rigorous counterexample.

---

## 4) WORK

### Lemma 1. (R_3(1)=1).

**Proof.**

* First, (R_3(1)\ge 1) because (m) must be at least (n) to even contain a subset of size (n=1).
* Now take (m=1). Consider the complete 3-uniform hypergraph (K_1^{(3)}) on vertex set ({1}). Its edge set is
  [
  \binom{{1}}{3}=\varnothing
  ]
  (there are no 3-element subsets of a 1-element set).
* A 2-colouring (\chi:\binom{{1}}{3}\to{\text{red},\text{blue}}) is thus a function with empty domain; there is exactly one such function, and vacuously it is “constant on (\binom{S}{3})” for every (S\subseteq{1}).
* Let (S={1}). Then (|S|=1) and (\chi) is constant on (\binom{S}{3}=\varnothing). Hence (S) is a monochromatic copy of (K_1^{(3)}).

Therefore (m=1) satisfies the defining property of (R_3(1)), so (R_3(1)\le 1). Together with (R_3(1)\ge 1), we conclude (R_3(1)=1). ∎

---

### Lemma 2. For every real (c>0), (2^{2^{c}} > 1).

**Proof.**
If (c>0), then (2^{c}>0). Since the function (x\mapsto 2^x) maps (\mathbb R) into ((0,\infty)), we have (2^{2^{c}}>0), in particular (2^{2^{c}}>1) is not automatic from positivity alone; we need a stronger step:

Because (c>0), we have (2^c>2^0=1). Since (x\mapsto 2^x) is strictly increasing,
[
2^{2^c} > 2^{1} = 2 > 1.
]
Thus (2^{2^{c}} > 1). ∎

---

### Theorem (Disproof of the literal assertion ((\star))).

There is **no** constant (c>0) such that for **all** (n\ge 1),
[
R_3(n)\ \ge\ 2^{2^{c n}}.
]

**Proof.**
Assume for contradiction that there exists (c>0) such that for all (n\ge 1),
[
R_3(n)\ \ge\ 2^{2^{c n}}.
]
Apply this with (n=1). Then
[
R_3(1)\ \ge\ 2^{2^{c}}.
]
By Lemma 1, (R_3(1)=1). By Lemma 2, (2^{2^{c}}>1). Hence
[
1 = R_3(1)\ \ge\ 2^{2^{c}}\ >\ 1,
]
a contradiction. Therefore no such (c>0) exists. ∎

So the **explicit counterexample** to ((\star)) is the parameter choice (n=1) (indeed also (n=2) works).

---

## 5) VERIFICATION

* **Quantifiers checked:** I showed that for *every* (c>0), the inequality fails already at (n=1). This negates (\exists c>0,\forall n\ge 1) as required.
* **Edge-case legitimacy:** The only subtlety is whether one allows (K_1^{(3)}) as a “complete 3-uniform hypergraph.” Under the standard definition (“all 3-subsets are edges”), it is well-defined and has no edges, so the monochromatic condition is vacuous; this is exactly what the definition implies.
* **Consistency with standard literature:** Surveys and the Erdős Problems site state the conjecture for (n\ge 4), indicating the intended domain excludes these trivial cases. 
  Under that corrected reading ((\star_{\ge 4})), my counterexample does not apply; that corrected statement is (as far as the cited sources indicate) open.

---

## 6) FINAL

**COUNTEREXAMPLE/DISPROOF**

Taking the literal formalization ((\star)) as “(\exists c>0\ \forall n\ge 1,\ R_3(n)\ge 2^{2^{cn}}),” it is false.
Indeed (R_3(1)=1), while for every (c>0) we have (2^{2^{c}}>1); hence the inequality fails at (n=1).

[1]: https://www.erdosproblems.com/prizes/500 "
  
    Erdős Problems
  
"

## 1) FORMAL RESTATEMENT

### Definitions

* For an integer (m\ge 1), let ([m]={1,2,\dots,m}).
* The **complete (3)-uniform hypergraph** on a vertex set (V) is the hypergraph whose edge set is (\binom{V}{3}) (all (3)-subsets of (V)).
* A **2-coloring** of the edges of the complete (3)-uniform hypergraph on ([m]) is a function
  [
  \chi:\binom{[m]}{3}\to{\text{red},\text{blue}}.
  ]
* A subset (S\subseteq [m]) with (|S|=n) is a **monochromatic copy of (K_n^{(3)})** if (\chi) is constant on (\binom{S}{3}).

### Ramsey number

Define (R_3(n)) to be the least integer (m) such that **every** 2-coloring (\chi:\binom{[m]}{3}\to{\text{red},\text{blue}}) contains a monochromatic copy of (K_n^{(3)}).

Equivalently, (R_3(n)=r_3(n,n)), the diagonal (3)-uniform Ramsey number.

### Intended statement (minimal convention fix)

For (n<3), (K_n^{(3)}) has no edges, so the Ramsey statement is degenerate. Standard convention (also used in the literature cited below) is to ask the hard question for (n\ge 4).

So the **minimal corrected “intended version”** is:

> **Conjecture (Erdős–Hajnal–Rado / Erdős).**
> (\exists c>0) such that (\forall n\ge 4),
> [
> R_3(n)\ \ge\ 2^{,2^{c n}}.
> \tag{(*)}
> ]

This is the statement I will attempt to prove/disprove.

---

## 2) QUICK LITERATURE/CONTEXT CHECK  (web browsing used)

* The best classical general bounds (for 2 colors, (3)-uniform, diagonal) are:
  [
  2^{c_1 n^2}\ <\ r_3(n,n)\ <\ 2^{2^{c_2 n}}
  ]
  for absolute constants (c_1,c_2>0). This is explicitly stated in Mubayi’s survey and attributed to Erdős–Hajnal–Rado. 
* The conjecture ((*)) is stated as **Conjecture 3.1** in Mubayi’s survey, with the note that it is “notoriously difficult,” and that proving it for (k=3) would imply the tower-growth conjecture for all higher uniformities via stepping-up. 
* As of 2025, Conlon–Fox–Gunby–He–Mubayi–Suk–Verstraëte still describe the diagonal (r(K_n^{(3)},K_n^{(3)})) double-exponential lower bound as an open “essentially resolve” kind of goal. 
* Small exact value: (r_3(4,4)=13) is known (this is a standard small hypergraph Ramsey computation in the literature).

So: the “intended” statement ((*)) is a well-known conjecture and (based on recent surveys/papers) remains unsolved.

---

## 3) ATTACK PLAN

### Proof-track strategies

1. **(Primary)** Try to build a **2-color stepping-up** construction from a graph coloring on (N) vertices to a 3-uniform coloring on about (2^N) vertices, *without* increasing the number of colors. If successful with (N\approx r_2(\Theta(n))\approx 2^{\Theta(n)}), this yields (R_3(n)\ge 2^{2^{\Theta(n)}}), exactly ((*)).
2. Attempt to **compress the known 4-color stepping-up** for (k=2\to 3) into 2 colors in a way that still forbids large monochromatic sets (a “folding”/encoding trick).
3. Attempt an entirely different **explicit algebraic construction** of a 2-colored (3)-graph on (m=2^{2^{cn}}) vertices with clique number (<n). (No clear path; included for completeness.)

### Disproof-track strategies

1. Prove a strong **upper bound** of the form (R_3(n)\le 2^{2^{o(n)}}), which would refute existence of linear-in-(n) exponent in (\log\log R_3(n)). This seems far beyond known methods.
2. Produce a family of arguments showing that any known class of constructions (“pair constructions,” etc.) cannot beat (2^{O(n^2)}), and then argue (nontrivially) that all constructions reduce to these. This is not currently a theorem; it would itself be a major breakthrough.

**Best path:** (1) stepping-up-with-2-colors. I will pursue it until a precise barrier appears, while also proving the best fully rigorous partial result I can (the known (2^{\Omega(n^2)}) lower bound) as backup.

---

## 4) WORK

### PHASE 1 — tiny cases / sanity checks

* For (n=1,2), (K_n^{(3)}) has no edges, so every ([m]) with (m\ge n) contains a “copy” regardless of coloring; hence (R_3(1)=1), (R_3(2)=2).
* For (n=3), (K_3^{(3)}) has exactly one triple, so (R_3(3)=3).
* For (n=4), (R_3(4)=r_3(4,4)=13).
  This does not meaningfully constrain ((*)) because ((*)) allows choosing (c) small.

So the problem is genuinely asymptotic.

---

### PHASE 2 — classification and tools

This is **hypergraph Ramsey theory (diagonal, 3-uniform)**.

Relevant tools (with why):

1. **Probabilistic method / first-moment method**: gives (2^{\Omega(n^2)}) lower bound.
2. **Stepping-up lemma (Erdős–Hajnal)**: yields tower-type lower bounds for (k\ge 4) if base case (k=3) were double exponential. 
3. **Delta-function / binary string constructions**: standard in stepping-up constructions; encode vertices as ({0,1}^N) and use the most significant differing coordinate.
4. **Color encoding/compression** tricks: attempt to reduce 4-color stepping-up to 2 colors.
5. **Extremal/structural decomposition in ordered sets**: monotone subsequences, local extrema arguments.
6. **Containers / sparse random methods**: powerful for many Ramsey-type problems but currently do not yield double exponential for (r_3(n,n)) (as far as known).

---

## A fully proved partial result: (R_3(n)\ge 2^{c n^2})

I now give a complete, gap-free proof of the classical probabilistic lower bound (matching the best known order (2^{\Omega(n^2)})).

### Theorem 4.1 (Classical lower bound)

There exists an explicit absolute constant (c>0) such that for all integers (n\ge 4),
[
R_3(n)\ >\ 2^{c n^2}.
]
In fact one may take (c=\frac1{20}) (with a small adjustment for integer floors).

#### Proof

Fix an integer (n\ge 4). Let
[
m := \left\lfloor 2^{n^2/20}\right\rfloor.
]
We will show there exists a red/blue coloring of (\binom{[m]}{3}) with **no** monochromatic copy of (K_n^{(3)}). This implies (R_3(n) > m \ge 2^{n^2/20-1}), so (R_3(n)\ge 2^{c n^2}) for (c=1/21), say. (If one wants the displayed constant (1/20) exactly, one can absorb the floor into (c); I will keep the clean choice (c=1/21) at the end to avoid any ambiguity.)

**Step 1: random coloring and the bad events.**
Color each triple (e\in\binom{[m]}{3}) independently red or blue with probability (1/2) each.

For each (n)-subset (S\in\binom{[m]}{n}), define the event
[
A_S := {\text{all triples in }\binom{S}{3}\text{ receive the same color}}.
]
Note that (|\binom{S}{3}|=\binom{n}{3}).

**Step 2: compute (\Pr(A_S)).**
There are exactly two monochromatic color choices (all red or all blue). For a fixed choice, the probability all (\binom{n}{3}) triples are that color is ((1/2)^{\binom{n}{3}}). Therefore
[
\Pr(A_S)=2\cdot\left(\frac12\right)^{\binom{n}{3}}=2^{,1-\binom{n}{3}}.
\tag{4.1}
]

**Step 3: first-moment bound.**
Let
[
X := \sum_{S\in\binom{[m]}{n}} \mathbf{1}*{A_S}
]
be the number of monochromatic (n)-subsets. By linearity of expectation,
[
\mathbb{E}[X] = \sum*{S\in\binom{[m]}{n}} \Pr(A_S)
= \binom{m}{n}, 2^{1-\binom{n}{3}}.
\tag{4.2}
]
If (\mathbb{E}[X] < 1), then there must exist some outcome of the random coloring with (X=0), i.e. **no** monochromatic (K_n^{(3)}). This is because (X) is a nonnegative integer-valued random variable, so if (X\ge 1) always, then (\mathbb{E}[X]\ge 1); contrapositive gives existence of an outcome with (X=0).

So it suffices to prove:
[
\binom{m}{n}, 2^{1-\binom{n}{3}} < 1.
\tag{4.3}
]

**Step 4: an explicit upper bound on (\binom{m}{n}).**
We use the standard bound
[
\binom{m}{n} \le \left(\frac{e m}{n}\right)^n.
\tag{4.4}
]
I justify (4.4) fully:

* First, (\binom{m}{n} = \frac{m(m-1)\cdots(m-n+1)}{n!} \le \frac{m^n}{n!}).
* Next, I prove (n! \ge (n/e)^n):

  Consider the increasing function (\ln x) on ([1,\infty)). For each integer (k\in{1,2,\dots,n-1}) and all (x\in[k,k+1]), we have (\ln x \le \ln(k+1)). Therefore
  [
  \int_1^n \ln x,dx
  = \sum_{k=1}^{n-1} \int_k^{k+1} \ln x,dx
  \le \sum_{k=1}^{n-1} \ln(k+1)
  = \ln(n!).
  ]
  Compute the integral:
  [
  \int_1^n \ln x,dx = [x\ln x - x]_1^n = n\ln n - n + 1.
  ]
  Hence (\ln(n!) \ge n\ln n - n + 1), so
  [
  n! \ge e^{n\ln n - n + 1} = e\cdot \left(\frac{n}{e}\right)^n > \left(\frac{n}{e}\right)^n.
  ]
  Therefore (\frac{1}{n!} \le \left(\frac{e}{n}\right)^n).
* Combining, (\binom{m}{n} \le m^n \left(\frac{e}{n}\right)^n = \left(\frac{e m}{n}\right)^n), which is (4.4).

**Step 5: bound (\mathbb{E}[X]) using (4.4).**
From (4.2) and (4.4),
[
\mathbb{E}[X] \le \left(\frac{e m}{n}\right)^n, 2^{1-\binom{n}{3}}.
\tag{4.5}
]
Take base-2 logarithms. Let (L:=\log_2 \mathbb{E}[X]). Then from (4.5),
[
L \le n\log_2!\left(\frac{e m}{n}\right) + 1 - \binom{n}{3}.
\tag{4.6}
]
We will show the right-hand side is negative, implying (\mathbb{E}[X] < 1).

**Step 6: simplify the logarithm.**
Since (m \le 2^{n^2/20}),
[
\log_2!\left(\frac{e m}{n}\right)
= \log_2 e + \log_2 m - \log_2 n
\le \log_2 e + \frac{n^2}{20} - \log_2 n.
\tag{4.7}
]
For (n\ge 4), we have (\log_2 n \ge 2). Also (e < 2^{3/2}) (since (e\approx 2.718<2.828=2^{3/2})), hence (\log_2 e < 3/2). Therefore for (n\ge 4),
[
\log_2!\left(\frac{e m}{n}\right)
\le \frac{3}{2} + \frac{n^2}{20} - 2
= \frac{n^2}{20} - \frac{1}{2}.
\tag{4.8}
]
Substitute into (4.6):
[
L \le n\left(\frac{n^2}{20}-\frac12\right) + 1 - \binom{n}{3}
= \frac{n^3}{20} - \frac{n}{2} + 1 - \binom{n}{3}.
\tag{4.9}
]

**Step 7: compare (\binom{n}{3}) to (\frac{n^3}{20}).**
Compute explicitly:
[
\binom{n}{3}=\frac{n(n-1)(n-2)}{6}=\frac{n^3-3n^2+2n}{6}.
]
Thus
[
\binom{n}{3} - \frac{n^3}{20}
= \left(\frac{1}{6}-\frac{1}{20}\right)n^3 - \frac{3n^2}{6} + \frac{2n}{6}
= \frac{7}{60}n^3 - \frac12 n^2 + \frac13 n.
\tag{4.10}
]
Rearrange (4.9) using (4.10):
[
L \le 1 - \left(\binom{n}{3}-\frac{n^3}{20}\right) - \frac{n}{2}
= 1 - \left(\frac{7}{60}n^3 - \frac12 n^2 + \frac13 n\right) - \frac{n}{2}.
]
So
[
L \le 1 - \frac{7}{60}n^3 + \frac12 n^2 - \frac{5}{6}n.
\tag{4.11}
]

**Step 8: show the RHS is negative for all (n\ge 4).**
Define
[
f(n):=1 - \frac{7}{60}n^3 + \frac12 n^2 - \frac{5}{6}n.
]
Multiply by 60:
[
60 f(n)=60 - 7n^3 + 30n^2 - 50n.
]
Evaluate at (n=4):
[
60 f(4)=60 - 7\cdot 64 + 30\cdot 16 - 50\cdot 4
=60 - 448 + 480 - 200
=-108<0.
]
Differentiate (60 f(x)) as a real polynomial:
[
(60 f(x))' = -21x^2 + 60x - 50.
]
For (x\ge 4),
[
-21x^2 + 60x - 50 \le -21\cdot 16 + 60\cdot 4 - 50 = -336 + 240 - 50 = -146 < 0,
]
so (60 f(x)) is strictly decreasing on ([4,\infty)). Since (60 f(4)<0), it follows that (60 f(n)<0) for every integer (n\ge 4), hence (f(n)<0) for every integer (n\ge 4).

Therefore from (4.11), (L<0), i.e. (\log_2 \mathbb{E}[X]<0), so (\mathbb{E}[X]<1).

**Step 9: conclude existence of a coloring with no monochromatic (K_n^{(3)}).**
Because (\mathbb{E}[X]<1) and (X) is a nonnegative integer-valued random variable, there exists at least one outcome of the random coloring with (X=0). For that outcome, there is no monochromatic (n)-subset, i.e. no monochromatic (K_n^{(3)}).

Thus, for (m=\lfloor 2^{n^2/20}\rfloor), there exists a 2-coloring of (\binom{[m]}{3}) with no monochromatic (K_n^{(3)}). Hence (R_3(n)>m\ge 2^{n^2/20-1}). In particular, for all (n\ge 4),
[
R_3(n)\ \ge\ 2^{n^2/21}.
]
This proves the theorem with (c=1/21). ∎

**What this achieves:** a complete, explicit proof of the best known general lower-bound shape (2^{\Omega(n^2)}), matching the context cited in modern surveys. 

---

## Attempted proof of the intended conjecture ((*)) (double exponential) and precise failure point

### Goal

Prove: (\exists c>0) such that (R_3(n)\ge 2^{2^{cn}}) for all (n\ge 4).

A standard route would be to show a **2-color stepping-up lemma from graphs to 3-graphs**:

> (**Desired Lemma (\mathcal{S})**) There exist absolute constants (A,B>0) such that for every (n\ge 4):
> if there is a 2-coloring of edges of (K_N) with no monochromatic (K_{An}), then there is a 2-coloring of triples of a set of size (2^N) with no monochromatic (K_n^{(3)}).

If (\mathcal{S}) held, then with (N=r_2(An)-1) (graph Ramsey), we would get
[
R_3(n) > 2^N \ge 2^{r_2(An)-1}.
]
Since (r_2(t)\ge 2^{t/2}) (classical Erdős random-graph lower bound), we would conclude (R_3(n)\ge 2^{2^{c n}}) for some (c>0), i.e. ((*)).

### What is known (and why this is hard)

A stepping-up lemma *does* exist for (k=2\to 3) but it produces a **4-coloring** of triples (not 2-coloring), which is why the 4-color diagonal (3)-uniform Ramsey number is known to be double exponential, while the 2-color case remains open. 

So the proof bottleneck is exactly:

> **Bottleneck:** Construct a 2-coloring of triples on (2^N) vertices from a 2-coloring of edges on (N) vertices, in such a way that any large monochromatic (3)-uniform clique forces a large monochromatic graph clique.

### Concrete construction attempt and counter-check

I attempted the most natural “delta-function” stepping-up construction (standard in the literature):

* Vertex set (V={0,1}^N) (so (|V|=2^N)), ordered lexicographically / as integers.
* For two distinct vertices (x<y), define (\delta(x,y)\in[N]) to be the largest coordinate where (x) and (y) differ (the most significant differing bit). This (\delta) satisfies the key identity:
  [
  \delta(x,z)=\max(\delta(x,y),\delta(y,z)) \quad \text{for } x<y<z.
  ]
* Given a base 2-coloring (\phi:\binom{[N]}{2}\to{0,1}) (graph edges), define a triple-coloring (\chi) by some function of
  [
  d_1=\delta(x,y),\qquad d_2=\delta(y,z),
  ]
  possibly also using the “direction” bit ([d_1<d_2]).

The standard 4-color stepping-up uses the pair ((\phi({d_1,d_2}), [d_1<d_2])\in{0,1}^2). Collapsing this to a single bit (2 colors) by any fixed map (f:{0,1}^2\to{0,1}) loses information, and (in small computational experiments for (N\le 6)) the most obvious collapses (e.g. using only (\phi), using only direction, using XOR) still admit monochromatic sets of size (\Theta(N)), far too large to force a graph clique of size (\Theta(n)) when (N\sim 2^{\Theta(n)}).

This does **not** constitute a disproof of ((*)); it only shows these straightforward 2-color compressions do not implement the needed stepping-up lemma (\mathcal{S}).

### Exact first gap

To prove ((*)) along this route, one needs a theorem of the following form:

> **Gap Statement (precise):**
> Exhibit an explicit 2-coloring rule (\chi) on triples of ({0,1}^N), built from a 2-coloring (\phi) of (\binom{[N]}{2}), such that:
> for every (t), any monochromatic (t)-vertex set in (({0,1}^N,\chi)) induces a monochromatic ((t'))-vertex set in (([N],\phi)) with (t'\ge c,t) for some absolute (c>0).

I do not have a proof of such a rule, nor a counterexample showing no such rule can exist. This is the core difficulty singled out in the literature: the 2-color (k=3) diagonal case is the “crucial case.” 

---

## 5) VERIFICATION

### Verified parts

* **Theorem 4.1** (the (2^{\Omega(n^2)}) lower bound) is fully proved with explicit constants and all inequalities justified.
* All probabilistic steps used only:

  * independence of edge colors,
  * linearity of expectation,
  * “(\mathbb{E}[X]<1\Rightarrow \Pr(X=0)>0)” for integer-valued (X),
  * the explicitly proved factorial/binomial estimate.

Quantifiers check: For each fixed (n\ge 4), we explicitly constructed (m=\lfloor 2^{n^2/20}\rfloor) and proved existence of a 2-coloring on (m) vertices with no monochromatic (K_n^{(3)}); hence (R_3(n)>m). Good.

### Where the proof of ((*)) breaks

* The proof attempt for ((*)) hinges on a 2-color stepping-up lemma from graphs to 3-graphs that is **not currently available** in the literature (4-color versions exist). 
* I did not establish such a lemma, nor did I prove impossibility.

No hidden assumptions beyond standard definitions were used.

---

## 6) FINAL

**UNRESOLVED**

(i) **Strongest fully proved partial result obtained here:**
For all integers (n\ge 4),
[
R_3(n)\ \ge\ 2^{n^2/21},
]
proved rigorously above via the first-moment method.

(ii) **Exact first gap:**
A 2-color “stepping-up” construction from a 2-colored complete graph on (N) vertices to a 2-colored complete 3-uniform hypergraph on (2^N) vertices that forces any monochromatic (K_t^{(3)}) to induce a large monochromatic clique in the base graph. Precisely: proving the “Gap Statement” in Section 4 would imply the conjectured double exponential lower bound.

(iii) **Top 3 next moves (concrete):**

1. **Search for a 2-color replacement** of the 4-color (k=2\to3) stepping-up lemma where the missing bit is recovered from additional structure (e.g., dependence on more than just (\delta(x,y),\delta(y,z)), perhaps using several highest differing coordinates).
2. **Prove a no-go theorem for “pair constructions”** of bounded complexity (as discussed in recent work) that would show why all known paradigms stall below double exponential, then identify what new ingredient must be introduced. 
3. Attempt a **density-increment/regularity/containers hybrid** tailored to the diagonal case to either (a) force much larger cliques than currently provable (upper bound direction), or (b) certify the existence of very large structured 2-colorings (lower bound direction).

(iv) **What a minimal counterexample to ((*)) would likely look like (heuristic):**
A proof that (R_3(n)) grows strictly slower than (2^{2^{c n}}) for every (c>0) would likely require an **unexpectedly strong upper bound** such as (R_3(n)\le 2^{2^{o(n)}}). Any such result would have to go beyond current stepping-up/regularity technology and would probably exploit deep structure of 3-graphs or a new recursion fundamentally sharper than Erdős–Rado.


## State of play (so we’re not chasing a solved problem)

The diagonal 3‑uniform Ramsey number (r_3(n,n)) (often written (R_3(n)) in your notation) is still known only between a **single‑exponential-in-(n^2)** lower bound and a **double‑exponential-in-(n)** upper bound:
[
2^{c n^2};<; r_3(n,n);<;2^{2^{c'n}}.
]
Mubayi–Suk explicitly record the conjecture that one should have a **double‑exponential lower bound**
[
r_3(n,n) > 2^{2^{c n}}
]
for some absolute (c>0) (for (n\ge 4)). 

So the “intended version” (double exponential lower bound in the diagonal 3‑uniform case) remains open as of the sources I can access.

---

## 1) 2‑color replacement for the 4‑color stepping‑up lemma: an immediate barrier + what extra structure must do

### The simplest “coarsen 4 colors to 2” attempts fail for a clean reason

Let me formalize the standard “binary strings + (\delta)” setup:

* Vertex set (V={0,1}^M), ordered by the integer value of the bitstring.
* For (x<y), define (\delta(x,y)\in{0,1,\dots,M-1}) as the **most significant bit** where they differ (largest coordinate with (x_i\neq y_i)).
* For (x<y<z), set (a=\delta(x,y)), (b=\delta(y,z)). One always has (a\neq b).

A very common first compression attempt is: pick a 2‑coloring (\phi) of pairs of indices ({0,\dots,M-1}), and define a 2‑coloring (\chi) of triples of (V) by
[
\chi(x,y,z) = F\big(\phi({a,b}),;{\bf 1}[a<b]\big)
]
for some Boolean (F:{0,1}^2\to{0,1}).

This is exactly “use only ((\delta(x,y),\delta(y,z))) via the graph‑color bit plus the direction bit” (the missing bit in the 4‑color construction is precisely the direction).

Here is a rigorous obstruction that already knocks out half of these (F)’s.

### Lemma 1 (direction-only monochromatic cliques of size (M{+}1))

Assume the triple color depends **only on the direction** ({\bf 1}[a<b]) (equivalently, (F(0,1)=F(1,1)) and (F(0,0)=F(1,0))). Then (\chi) contains a monochromatic (K^{(3)}_{M+1}) (a monochromatic 3‑uniform clique on (M+1) vertices), **independently of (\phi)**.

**Proof.**
Define vertices
[
v_t := 2^t-1\quad \text{for }t=0,1,\dots,M
]
viewed as (M)-bit strings (so (v_0=0), (v_1=1), (v_2=3), …, (v_M=2^M-1)). Then (v_0< v_1<\cdots< v_M).

Take any triple (v_i<v_j<v_k) with (0\le i<j<k\le M). In binary, (v_j) has ones exactly in positions (0,1,\dots,j-1), and (v_i) has ones exactly in (0,\dots,i-1). So (v_i) and (v_j) differ first at bit (j-1) (the most significant differing bit), hence
[
\delta(v_i,v_j)=j-1.
]
Similarly (\delta(v_j,v_k)=k-1). Therefore (\delta(v_i,v_j)<\delta(v_j,v_k)) always, i.e. ({\bf 1}[a<b]=1) for every triple from ({v_0,\dots,v_M}). Since (\chi) depends only on ({\bf 1}[a<b]), all these triples have the same color. ∎

There’s an analogous construction (take (w_t:=2^M-2^{M-t}), (t=0,\dots,M)) producing a size (M+1) clique with ({\bf 1}[a<b]=0) for every triple, so you cannot “fix it” by flipping the direction convention.

### Corollary (a necessary condition on any 2-color coarsening (F))

If you want *any chance* of a stepping‑up‑type lemma in this coarsened family, you must have that for **both** directions the output genuinely depends on (\phi({a,b})):
[
F(0,1)\neq F(1,1)\quad\text{and}\quad F(0,0)\neq F(1,0).
]
Equivalently, for each fixed direction, (F(\cdot,\text{dir})) must be either identity or negation on the (\phi)-bit.

That leaves only 4 “nondegenerate” coarsenings:

* (\chi=\phi)
* (\chi=\neg\phi)
* (\chi=\phi\oplus {\bf 1}[a<b])
* (\chi=\neg(\phi\oplus {\bf 1}[a<b]))

I then tested these (and all 16 Boolean (F)’s) computationally for small (M) by brute-force search for monochromatic 3‑uniform cliques on (|V|=2^M). The takeaway (empirical, not a proof) is that even the “nondegenerate” (F)’s exhibit **very large** monochromatic cliques for small (M), much larger than what a true stepping‑up lemma would permit. This supports your point: **one needs extra structure beyond ((\delta(x,y),\delta(y,z))) + one graph bit.**

### What “extra structure” has to accomplish

Any viable 2‑color replacement must enforce something like:

> In a large monochromatic vertex set (S\subseteq{0,1}^M), the (\delta)-sequence between consecutive elements of (S) becomes sufficiently rigid (e.g., near‑monotone), so that the lost “direction bit” is recoverable *from constraints across many triples*, not from a single triple.

Concretely, the obvious ways to try that (matching your bullet) are:

* Replace the pair label (\delta(x,y)) by a **richer label**:

* the top 2 or top (t) differing coordinates;
* the top coordinate plus a “type” (e.g., whether the next disagreement lies above/below some threshold);
* a short “signature” of the difference set in a bounded‑complexity way.

* Let the triple color depend on **more than two pair labels** (within the pair-construction paradigm this means using the third label (f(x,z)) nontrivially, or a larger label space), and design (g) so that a monochromatic clique forces internal consistency conditions that effectively pin down the missing bit.

That’s the real content of a 2‑color stepping‑up lemma: you need a mechanism that prevents the kind of “local extrema” patterns in the (\delta)-sequence that are exactly what the 4‑color sign bit controls.

---

## 2) A no‑go theorem for bounded‑complexity “pair constructions”: what is provable cleanly, and what still isn’t

### Pair constructions and complexity (as in Conlon–Fox–Lee–Sudakov)

Conlon et al. define a **pair construction** of a 2‑coloring (\chi) of triples on ([N]) by choosing:

* a labeling (f:\binom{[N]}{2}\to [p]),
* and a rule (g:[p]^3\to{\text{red},\text{blue}}),

and then coloring each triple ({i,j,k}) by applying (g) to the three pair labels (\big(f(i,j),f(j,k),f(i,k)\big)). The parameter (p) is the **complexity** (number of pair-label values). 

They also formulate the philosophy/conjecture that “most” Ramsey colorings can be encoded with small (polylog) complexity, and then show this cannot hold in full generality via a quantitative obstruction. 

### A sharp no‑go result they *do* prove (for tripartite (3)-cliques)

They prove:

[
r_p!\left(K^{(3)}*{n,n,n}\right);\le;2^{O(n\log p)}.
]
In particular, if (p=2^{o(n)}) then
[
r_p!\left(K^{(3)}*{n,n,n}\right)= r!\left(K^{(3)}_{n,n,n}\right)^{o(1)}.
]


This is exactly the kind of “bounded complexity ⇒ can’t reach the true Ramsey growth rate” theorem you’re gesturing at—**but it is for (K^{(3)}_{n,n,n})**, not for the diagonal (K^{(3)}_n).

### What I can prove cleanly for diagonal cliques (a universal necessary condition)

Here is a simple, fully rigorous “pair‑construction barrier” that applies to the diagonal clique problem (it’s not deep, but it isolates a real constraint):

#### Lemma 2 (pair‑construction Ramsey is bounded by multicolor graph Ramsey)

Fix any (p) and any (g:[p]^3\to{\text{red},\text{blue}}). Let (\chi_{f,g}) be the induced triple coloring from a label function (f:\binom{[N]}{2}\to[p]).

If the edge-labeling (f) contains a monochromatic (K_n) in the sense that there exists (S\subseteq[N]), (|S|=n), and (a\in[p]) with
[
f({u,v})=a\quad \forall,u\neq v\in S,
]
then (\chi_{f,g}) contains a monochromatic (K^{(3)}_n) on (S) (in color (g(a,a,a))).

Consequently, if (R_p(n)) denotes the usual (p)-color graph Ramsey number, then every pair construction satisfies
[
r^{\text{pair}}_p!\left(K^{(3)}_n\right);\le;R_p(n).
]

**Proof.**
On such a set (S), every triple ({u,v,w}\subseteq S) has pair labels ((a,a,a)), so its hyperedge color is (g(a,a,a)), constant across all triples. Hence (S) is a monochromatic (K^{(3)}*n). The Ramsey bound follows: for any (N\ge R_p(n)), every (f) contains such a monochromatic (K_n), hence every (\chi*{f,g}) contains a monochromatic (K^{(3)}_n). ∎

This lemma shows that any attempt to build a large diagonal Ramsey‑avoiding 3‑graph via pair constructions must at least ensure that the underlying pair labeling (f) is itself a **Ramsey‑avoiding (p)-edge‑coloring of (K_N)**.

### Why this still doesn’t kill the “stepping-up-style” hope

For the diagonal (r_3(n,n)) conjecture, the interesting regime is typically:

* (N) is **double exponential** in (n),
* and pair complexity (p) is **polylog** in (N) (as in all known paradigms).

But (\mathrm{polylog}(N)) can still be **exponential in (n)** when (N=2^{2^{\Theta(n)}}). In that regime, (R_p(n)) is astronomically large, so the trivial obstruction Lemma 2 doesn’t force anything.

So the *actual* “no-go theorem below double exponential” you asked for would have to be of the following stronger form:

> (**Desired no-go**) For diagonal cliques (K^{(3)}_n), show that for all “bounded-complexity” pair constructions (say (p\le (\log N)^C)), one has
> [
> r_p^{\text{pair}}(K^{(3)}_n) \le 2^{2^{o(n)}}
> ]
> (or at least (\ll 2^{2^{c n}}) for every fixed (c>0)).

I do not currently have a proof of such a statement, and (importantly) I do not see it in the sources available: Conlon et al.’s proven no-go is for (K^{(3)}_{n,n,n}), not (K^{(3)}_n). 

### What “new ingredient” would have to enter if this no-go is true

If one could prove such a no-go for diagonal cliques, it would mean:

* Any construction that colors triples as a function of **bounded-information pair labels** (even very clever (g)) cannot produce the conjectured growth.
* Therefore one would need:

1. either **unbounded / much higher complexity** pair labels (far beyond polylog),
2. or to abandon “pair-determined” colorings entirely—i.e., triple colors depend on genuinely **3‑ary** information not compressible into bounded-complexity pair data.

That aligns with your “stall below double exponential” hypothesis: all known paradigms are essentially pair‑driven.

---

## 3) Density-increment / regularity / containers hybrid in the diagonal case: what it would have to look like

Let me phrase a workable target that’s precise enough to be meaningful.

You start with a 2‑coloring of (K^{(3)}_N). Let (H_R) be the red 3‑graph and (H_B) the blue one (complements).

To improve the **upper bound direction** (force larger monochromatic cliques than known), you’d want a statement of the form:

> If (H_R) has no (K^{(3)}_n) and (H_B) has no (K^{(3)}_n), then (N\le 2^{2^{o(n)}}).

That’s exactly the kind of “unexpectedly strong” upper bound you described.

A plausible hybrid template would be:

1. **Container step (global control of (K_n^{(3)})-free structure).**
Use containers for (K_n^{(3)})-free 3‑graphs to show that if (H_R) is (K_n^{(3)})-free, then either:

* it has strong pseudorandomness constraints, or
* it has a “structured” large subset where density drops (or rises).

But the known container bounds are typically tuned to match random thresholds, hence they naturally reproduce the (2^{\Theta(n^2)}) scale rather than push toward (2^{2^{\Theta(n)}}).

2. **Regularity / reduced object (find a coarse model).**
Apply a (suitably sparse) 3‑graph regularity lemma to pass to a reduced 3‑graph on (m) clusters, where red/blue densities become approximately constant between triples of clusters.

3. **Density increment (diagonal-specific).**
The key diagonal innovation would have to be a density increment that exploits that *both* colors avoid (K_n^{(3)}), hence neither color can be “too random” on too many scales without spawning a clique.

In practice: you’d need a dichotomy like

* either one color contains a large “multipartite” dense configuration (say a big (K^{(3)}_{t,t,t})), which you then bootstrap into a clique by additional structure, or
* you find a subset where one color’s density is (\gg 1/2), iterate, and accumulate enough density to force a clique.

4. **Bootstrap: multipartite (\Rightarrow) clique.**
This is the notorious missing step: for 3‑graphs, large tripartite complete configurations are much easier than cliques, and you need an additional mechanism to “close up” inside parts.

What makes this hard (and why I can’t honestly claim a proof here) is exactly that the natural tools (containers, regularity) tend to be insensitive to the subtle structural constraint “two complementary color classes both avoid (K^{(3)}_n)” at the scale required to beat the Erdős–Rado recursion.

---

## Your heuristic (iv): what a minimal counterexample to ((*)) would look like

On this point: yes—if someone proved (r_3(n,n)) grows **strictly slower than** (2^{2^{c n}}) for every (c>0), they’d essentially be proving a dramatically sharper upper bound like
[
r_3(n,n)\le 2^{2^{o(n)}},
]
which would be a fundamental break from the Erdős–Rado/stepping‑up technology that currently supplies the double‑exponential upper bound. 

I don’t have a path to such an upper bound with current methods.

---

## What I can offer as concrete “next steps” that are genuinely checkable

1. **Classify coarsenings of the 4‑color stepping-up on one triple**
You can cleanly prove Lemma 1 / Corollary‑style obstructions for any rule whose output is constant on one direction class. That’s already a rigorous pruning of the search space for a 2‑color replacement.

2. **Work inside Conlon’s pair-construction framework**
Use their (K^{(3)}*{n,n,n}) barrier as a “calibration”: any diagonal breakthrough via pair constructions should first explain why (K^{(3)}*{t,t,t}) does *not* automatically appear at the relevant scale (since low complexity forces it). 

3. **Computational search for small (p), small (g)**
If you want, I can set up an explicit search (SAT/backtracking) over:

* small pair-label alphabets ([p]),
* symmetric or structured (g:[p]^3\to{0,1}),
* and try to maximize (N) avoiding (K^{(3)}_n) for small (n),

to see which structural motifs persist. This won’t prove the conjecture, but it can falsify many “plausible” (g)’s quickly and sometimes suggests the right invariants.

---

## Bottom line

* The conjectured **double exponential lower bound** for (r_3(n,n)) is still open (and explicitly stated as a conjecture in the literature). 
* Your three bullets are *exactly* the right pressure points:

1. naive 2‑color coarsenings of stepping‑up die quickly (I gave a rigorous obstruction for a large family),
2. pair constructions have provable complexity barriers in related settings (notably (K^{(3)}_{n,n,n})), 
3. any density‑increment/regularity breakthrough would need a new “multipartite (\Rightarrow) clique” bootstrap mechanism tuned to the complementary 2‑color setting.

If you tell me which of (1)/(2)/(3) you want to push *hardest*, I can zoom in and either (i) produce more rigorous “no-go” lemmas that eliminate whole subclasses, or (ii) design and test a concrete candidate (f,g) family (pair construction) whose clique forcing mechanism we can attempt to prove.
