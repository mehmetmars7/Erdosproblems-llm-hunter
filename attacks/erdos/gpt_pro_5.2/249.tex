\section*{Erd\H{o}s Problem \#249}

\subsection*{1) FORMAL RESTATEMENT}
Let $\varphi(n)$ denote Euler's totient function. Consider the convergent series
\[
S \coloneqq \sum_{n=1}^{\infty}\frac{\varphi(n)}{2^n}.
\]
\noindent\textbf{Question.} Is $S$ irrational?

\subsection*{2) QUICK LITERATURE/CONTEXT CHECK}
\begin{itemize}[leftmargin=2em]
\item The Erd\H{o}s Problems database lists this question as open (last edited Sept 28, 2025) \cite{ErdosProblems249}.
\item OEIS entry A256936 records the decimal expansion of $S$ and gives alternative series representations \cite{OEISA256936}.
\item Pratt (2024) notes that proving irrationality for the analogous Lambert-series-type sums involving $\varphi(n)$ and $\sigma(n)$ ``seems to present difficulties'' and explicitly states these questions are open \cite{Pratt2024}.
\end{itemize}

\subsection*{3) ATTACK PLAN}
A natural attempt is to adapt Erd\H{o}s' classical ``blocks of zeros'' method used for some Lambert series (e.g. divisor-function series) to the base-$2$ expansion of $S$. The obstacle is that $\varphi(n)$ is typically of size comparable to $n$, so carries from far-out terms are harder to control than in cases where the coefficients grow very slowly.

Two other angles are:
\begin{enumerate}[leftmargin=2em]
\item Find a transformation expressing $S$ as a linear combination of more standard constants (e.g. sums of $1/(2^k-1)$ and its derivatives), then apply known irrationality/transcendence results.
\item Use exact identities for $S$ to reduce irrationality to nontrivial statements about cancellations in series with M\"obius weights.
\end{enumerate}

\subsection*{4) WORK}
\subsubsection*{4.1 Convergence and a high-precision value}
Since $0\le \varphi(n)\le n$, we have
\[
0\le \sum_{n>N}\frac{\varphi(n)}{2^n} \le \sum_{n>N}\frac{n}{2^n} = \frac{N+2}{2^N},
\]
so the series converges absolutely and very rapidly.

A direct computation (summing to $n=400$ and using the tail bound above) yields
\[
S = 1.3676308019850223507905081462130881390748919989627948529565984637621567103976692\ldots
\]
which agrees with the digits listed in OEIS A256936 \cite{OEISA256936}.

\subsubsection*{4.2 Two exact alternative series representations}
\paragraph{Representation A (M\"obius-weighted rational functions).}

\begin{proposition}
For $|x|<1$,
\[
\sum_{n=1}^{\infty} \varphi(n) x^n = \sum_{k=1}^{\infty} \mu(k)\,\frac{x^k}{(1-x^k)^2},
\]
where $\mu$ is the M\"obius function. In particular,
\[
S = \sum_{k=1}^{\infty} \mu(k)\,\frac{2^{k}}{(2^{k}-1)^2}.
\]
\end{proposition}

\begin{proof}
Using the Dirichlet-convolution identity $\varphi = \mu * \mathrm{id}$ (equivalently $\varphi(n)=\sum_{d\mid n}\mu(d)\,n/d$), write
\[
\sum_{n\ge 1}\varphi(n)x^n
=\sum_{n\ge 1}\sum_{d\mid n}\mu(d)\,\frac{n}{d}\,x^n.
\]
Set $n=dm$ and interchange sums (justified by absolute convergence for $|x|<1$):
\[
\sum_{n\ge 1}\varphi(n)x^n
=\sum_{d\ge 1}\mu(d)\sum_{m\ge 1} m\,x^{dm}.
\]
For $|y|<1$ we have $\sum_{m\ge 1} m y^m = y/(1-y)^2$. Taking $y=x^d$ gives
\[
\sum_{m\ge 1} m\,x^{dm} = \frac{x^d}{(1-x^d)^2}.
\]
Thus
\[
\sum_{n\ge 1}\varphi(n)x^n = \sum_{d\ge 1}\mu(d)\frac{x^d}{(1-x^d)^2}.
\]
Setting $x=1/2$ and simplifying $\frac{2^{-k}}{(1-2^{-k})^2}=\frac{2^{k}}{(2^{k}-1)^2}$ yields the claimed formula for $S$.
\end{proof}

\paragraph{Representation B (Lambert series with A007431).}
Define
\[
a(n)\coloneqq \sum_{d\mid n} \varphi(d)\,\mu\Bigl(\frac{n}{d}\Bigr)=(\varphi*\mu)(n).
\]
This sequence is OEIS A007431 \cite{OEISA256936,OEISA007431}. Then:

\begin{proposition}
For $|x|<1$,
\[
\sum_{n=1}^{\infty} \varphi(n) x^n 
= \sum_{m=1}^{\infty} a(m)\,\frac{x^m}{1-x^m}.
\]
In particular,
\[
S = \sum_{m=1}^{\infty} \frac{a(m)}{2^m-1}.
\]
\end{proposition}

\begin{proof}
Using the geometric series, for $|x|<1$,
\[
\sum_{m\ge 1} a(m)\frac{x^m}{1-x^m}
=\sum_{m\ge 1} a(m)\sum_{j\ge 1} x^{mj}
=\sum_{n\ge 1}\Bigl(\sum_{m\mid n} a(m)\Bigr)x^n.
\]
But $\sum_{m\mid n} a(m) = (a*1)(n) = ((\varphi*\mu)*1)(n)=\varphi(n)$ because $\mu*1=\varepsilon$ (the identity for Dirichlet convolution). Hence the coefficient of $x^n$ equals $\varphi(n)$, proving the identity.

Setting $x=1/2$ gives
\[
S=\sum_{m\ge 1} a(m)\frac{2^{-m}}{1-2^{-m}}=\sum_{m\ge 1}\frac{a(m)}{2^m-1}.
\]
\end{proof}

These exact identities are useful structural information, but none by itself resolves irrationality.

\subsection*{5) VERIFICATION}
\begin{itemize}[leftmargin=2em]
\item The tail bound $\sum_{n>N} n/2^n=(N+2)/2^N$ is standard and can be verified by differentiating $\sum_{n\ge 0} x^n=1/(1-x)$ and substituting $x=1/2$.
\item The two exact representations were numerically cross-checked: truncating both at $200$ terms gives agreement to about $\sim 10^{-58}$.
\end{itemize}

\subsection*{6) FINAL}
\textbf{UNRESOLVED.}
\begin{enumerate}[label=(\roman*),leftmargin=2.5em]
\item \textbf{Strongest fully proved partial result obtained here:} I proved two exact identities for the constant
\[S=\sum_{n\ge 1}\varphi(n)2^{-n},\]
namely $S=\sum_{k\ge 1}\mu(k)\,\frac{2^k}{(2^k-1)^2}$ and $S=\sum_{m\ge 1}\frac{a(m)}{2^m-1}$ with $a=\varphi*\mu$ (OEIS A007431).
\item \textbf{First gap preventing a full solution:} None of these identities gives a clear route to proving $S\notin\mathbb{Q}$. Standard ``block of zeros'' arguments for dyadic expansions become delicate because coefficients $\varphi(n)$ are typically of order $n$.
\item \textbf{Most promising next move:} Try to combine a sieve-based construction that forces long stretches of strong $2$-adic divisibility in $\varphi(n)$ with a method to control carries contributed by the remaining tail. Alternatively, seek a reduction from irrationality of $S$ to known irrationality results for related constants (e.g. Erd\H{o}s--Borwein type sums).
\item \textbf{Smallest plausible counterexample (if false):} A rational value $S=P/Q$ would impose strong periodicity constraints on the binary expansion of the partial sums modulo $1$. Given the apparent randomness of $\varphi(n)$, this seems unlikely, but I have no rigorous obstruction.
\end{enumerate}

\subsection*{7) COMPLETION ESTIMATE}
\textbf{Estimated likelihood of completion with additional work:} \emph{Low-to-moderate} (\(\approx 15\%\)). There is substantial evidence that this question is genuinely hard and resistant to the classical methods used for ``slower'' arithmetic functions.


% =========================================================
\begin{thebibliography}{99}

\bibitem{Romanoff1934}
N.~P.~Romanoff,
\newblock \emph{\"Uber einige S\"atze der additiven Zahlentheorie},
\newblock \emph{Mathematische Annalen} \textbf{109} (1934), 668--678.

\bibitem{WikipediaRomanov}
Wikipedia contributors,
\newblock \emph{Romanov's theorem}, Wikipedia (accessed 2026-01-17).

\bibitem{Ding2025}
Y.~Ding,
\newblock \emph{On a Romanoff type problem of Erd\H{o}s and Kalm\'ar},
\newblock arXiv:2503.22700 (2025).

\bibitem{ErdosProblems249}
T.~F.~Bloom (ed.),
\newblock \emph{Erd\H{o}s Problem \#249}, Erd\H{o}s Problems website (last edited 28 Sep 2025; accessed 2026-01-17).

\bibitem{ErdosGraham1980}
P.~Erd\H{o}s and R.~L.~Graham,
\newblock \emph{Old and new problems and results in combinatorial number theory},
\newblock Monographies de L'Enseignement Math\'ematique, Vol.~28, Universit\'e de Gen\`eve, 1980.

\bibitem{Erdos1975}
P.~Erd\H{o}s,
\newblock \emph{Some problems and results on the irrationality of the sum of infinite series},
\newblock \emph{J. Math. Sci.} \textbf{10} (1975), 1--7 (often cited as appearing in 1976).

\bibitem{Erdos1988}
P.~Erd\H{o}s,
\newblock \emph{On the irrationality of certain series: problems and results},
\newblock in \emph{New advances in transcendence theory} (Durham, 1986),
\newblock Cambridge Univ. Press, 1988, 102--109.

\bibitem{Pratt2024}
K.~Pratt,
\newblock \emph{The irrationality of a prime factor series under a prime tuples conjecture},
\newblock arXiv:2409.15185 (2024).

\bibitem{OEISA256936}
N.~J.~A.~Sloane et al.,
\newblock \emph{OEIS A256936: Decimal expansion of $\sum_{k\ge 1}\varphi(k)/2^k$},
\newblock \url{https://oeis.org/A256936} (accessed 2026-01-17).

\bibitem{OEISA007431}
N.~J.~A.~Sloane et al.,
\newblock \emph{OEIS A007431: $a(n)=\sum_{d\mid n}\varphi(d)\mu(n/d)$},
\newblock \url{https://oeis.org/A007431} (accessed 2026-01-17).

\end{thebibliography}

