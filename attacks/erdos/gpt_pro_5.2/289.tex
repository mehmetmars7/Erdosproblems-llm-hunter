\section*{Erd\H{o}s Problem 289}
\addcontentsline{toc}{section}{Erd\H{o}s Problem 289}

\subsection*{1) Formal Restatement}
\textbf{Literal statement (as in the prompt).}
Is it true that, for all sufficiently large $k$, there exist finite intervals $I_1,\dots,I_k\subset\mathbb{N}$,
\emph{distinct, not overlapping or adjacent}, with $|I_i|\ge 2$ for all $i$, such that
\[
1=\sum_{i=1}^k \sum_{n\in I_i}\frac{1}{n}\ ?
\]

\textbf{Fully explicit formulation.}
Write each $I_i$ as an integer interval $I_i=[a_i,b_i]\cap\mathbb{N}$ with $1\le a_i\le b_i$.
The constraints ``not overlapping or adjacent'' mean that for $i\ne j$,
\[
\text{either }\ b_i+1 < a_j\ \text{or}\ b_j+1 < a_i.
\]
Equivalently, after reindexing we may assume
\[
1\le a_1\le b_1 < a_2-1\le b_2 < a_3-1\le \dots < a_k-1\le b_k,
\]
with each length $b_i-a_i+1\ge 2$.
The question is whether such configurations exist for all $k$ beyond some absolute threshold.

\subsection*{2) Quick Literature / Context Check (web)}
The Erd\H{o}s Problems site lists this as \emph{open}.
It notes that Erd\H{o}s and Graham posed the question in 1980 without the ``distinct / non-overlapping / non-adjacent'' restrictions, and that in this weaker interpretation there is a simple affirmative argument (given in a public comment by V. Kovac).
The site also records an explicit representation of $2$ (not $1$) as a sum over $5$ disjoint, non-adjacent intervals found by Hickerson and Montgomery:
\[
2=\sum_{n=2}^{7}\frac1n+\Big(\frac19+\frac1{10}\Big)+\Big(\frac1{17}+\frac1{18}\Big)+\Big(\frac1{34}+\frac1{35}\Big)+\Big(\frac1{84}+\frac1{85}\Big).
\]

\subsection*{3) Attack Plan}
Two complementary approaches suggest themselves:
\begin{enumerate}[label=(\alph*)]
\item \textbf{Constructive ``gadget'' approach.}
Find a local identity that replaces one admissible interval (or a small collection of admissible intervals) by a larger collection of admissible intervals \emph{separated by gaps}, preserving the total sum. If such a gadget increases the interval count by $+1$ while maintaining all constraints, iterating would yield all sufficiently large $k$.

\item \textbf{CRT / large modulus approach.}
Encode rational pieces of $1$ using unit fractions supported on widely separated blocks of consecutive integers. The blocks must themselves be decomposable into admissible intervals (length $\ge2$) without overlaps.
This resembles Kovac's construction for the weaker variant, but the distinctness and disjointness constraints are the sticking point.
\end{enumerate}

\subsection*{4) Work}
\subsubsection*{4.1 A small impossibility check for $k=2$ (bounded search)}
As a sanity check, I searched exhaustively for two admissible intervals $I_1,I_2\subset\{1,2,\dots,300\}$,
with $|I_i|\ge 2$ and $I_1,I_2$ disjoint and non-adjacent, such that
$\sum_{n\in I_1}1/n+\sum_{n\in I_2}1/n = 1$.
No such pair exists within this search window.

This does \emph{not} prove nonexistence for $k=2$ in general; it only indicates that any such solution (if it exists) likely involves quite large endpoints.

\subsubsection*{4.2 A proved weaker variant (overlaps/repetitions allowed)}
If one drops the requirements that the intervals be distinct and non-overlapping/non-adjacent, the problem becomes much easier.
A detailed construction (using large moduli and the Chinese Remainder Theorem, plus an explicit decomposition lemma for multisets supported on four consecutive integers) is given by V. Kovac in the public discussion thread for this problem on the Erd\H{o}s Problems site.
That argument yields:

\begin{quote}
\emph{Weaker statement.} For all sufficiently large $k$ there exist intervals $I_1,\dots,I_k\subset\mathbb{N}$ (allowed to overlap and repeat) with $|I_i|\ge 2$ such that $\sum_{i=1}^k\sum_{n\in I_i}1/n=1$.
\end{quote}

I do not reproduce the full CRT-based construction here (it is several pages even in streamlined form), but the key point is that allowing overlap/repetition lets one realize rational coefficients as 
\emph{multiplicities} of unit fractions in a short consecutive block, and then split that multiset into many length-$\ge2$ sub-intervals.
This mechanism fundamentally breaks once one forbids repeats and overlaps.

\subsubsection*{4.3 Why the ``distinct + disjoint + non-adjacent'' constraints are hard}
The disjointness constraints mean the total sum is the sum over a sparse subset of $\mathbb{N}$ made of $k$ blocks.
As $k$ grows, each block must typically contribute a very small amount (since the total is fixed at $1$), hence the blocks must lie far out on the number line.
But far-out blocks have sums that are arithmetically rigid when required to be \emph{exact} rationals that fit together to $1$.
The usual flexibility of Egyptian fractions (arbitrary denominators) is much reduced by the requirement that denominators come in consecutive blocks.

\subsection*{5) Verification}
\begin{itemize}
\item The bounded search in \S\,4.1 is a factual computational claim for endpoints $\le 300$.
\item The weaker-variant statement in \S\,4.2 is explicitly labeled as such and attributed to the discussion-thread construction.
\item No unsupported claim is made about the original restricted problem.
\end{itemize}

\subsection*{6) Final}
\textbf{UNRESOLVED.}
\begin{enumerate}[label=(\roman*)]
\item \textbf{Farthest point reached:} verified computationally that no $k=2$ solution exists with endpoints $\le 300$ under the full restrictions; documented that removing the restrictions yields an affirmative theorem via a CRT/multiplicity method.
\item \textbf{Best partial results:} the weaker-variant affirmative solution (overlaps/repetitions allowed) exists in the literature/discussion; and the explicit $k=5$ representation of $2$ by Hickerson--Montgomery gives evidence that the ``disjoint blocks'' world is nonempty.
\item \textbf{Most plausible next moves:} 
find a genuine ``interval gadget'' that increases the number of \emph{disjoint, separated} intervals while preserving the total sum; or prove an unconditional infinite family of $k$ for which solutions exist (even a subsequence $k\to\infty$ would be progress).
\item \textbf{What I would try next with more time:}
search computationally (with modular pruning and rational reconstruction) for a first explicit representation of $1$ by disjoint non-adjacent intervals of length $\ge2$, even for moderate $k$ (say $k=3,4,5$), as a seed for further ``gadget'' amplifications.
\end{enumerate}

\subsection*{7) Completion Estimate}
\textbf{Estimate:} 10\%.
(The full restricted problem is open; I only established small-case evidence and recorded a solution to a weaker interpretation.)


% ============================================================
