% Erdos Problem #857

\noindent\textbf{FORMAL RESTATEMENT.}

Fix integers $n\ge 1$ and $k\ge 2$. Write $[n]=\{1,2,\dots,n\}$. A \emph{$k$-sunflower} (also called a $k$-$\Delta$-system) is a collection of sets $S_1,\dots,S_k\subseteq[n]$ such that there exists a set $C\subseteq[n]$ with
\[
S_i\cap S_j = C\quad\text{for all }1\le i<j\le k.
\]
(Equivalently: all pairwise intersections are equal as sets.)

Let $m(n,k)$ be the least integer $m$ with the property that every family $\mathcal F$ of $m$ \emph{distinct} subsets of $[n]$ contains a $k$-sunflower.

\emph{Ambiguity note.} The problem statement says ``any collection of sets $A_1,\dots,A_m$'' without explicitly forbidding repetitions. If repetitions are allowed then $m(n,k)$ changes by at most a factor of $k$ (because $k$ equal copies already form a $k$-sunflower). Since the asymptotic question in $n$ is about exponential growth, we formalise the standard distinct-family version above.

Edge cases: $k=2$ (every two sets form a $2$-sunflower) and $n$ small.

\medskip
\noindent\textbf{QUICK LITERATURE/CONTEXT CHECK.}

The problem statement itself records that for $k=3$ there is an exponential upper bound $m(n,3) \le (3/2^{2/3})^{(1+o(1))n}$ due to Naslund--Sawin, and mentions relations to the cap set problem. Per the integrity rule, in what follows we do not use any unstated literature results.

\medskip
\noindent\textbf{ATTACK PLAN.}

Two basic directions that are fully provable from scratch:

(1) \emph{Lower bounds:} build explicit sunflower-free families $\mathcal F\subseteq 2^{[n]}$.

(2) \emph{Upper bounds:} prove the classical Erd\H{o}s--Rado sunflower lemma for $r$-uniform families, then apply it to each layer $\binom{[n]}{r}$ and sum over $r$.

We also run a tiny brute-force search for small $n,k$ to sanity-check the definitions.

\medskip
\noindent\textbf{WORK.}

\medskip
\noindent\underline{Fast reality check (small $n$ by brute force).}

For distinct families $\mathcal F\subseteq 2^{[n]}$, define $F(n,k)$ to be the maximum size of a $k$-sunflower-free family; then $m(n,k)=F(n,k)+1$.

A brute-force search over all families for $n\le 4$ gives:
\[
\begin{array}{c|ccc}
 n & m(n,2) & m(n,3) & m(n,4) \\\hline
 1 & 2 & - & - \\
 2 & 2 & 4 & 5 \\
 3 & 2 & 6 & 8 \\
 4 & 2 & 9 & 13
\end{array}
\]
(where ``-'' means $k>2^n$ so the distinct-family version cannot force a $k$-sunflower).

\medskip
\noindent\underline{Lemma 857.1 (Chains are sunflower-free for $k\ge 3$).}

\textbf{Lemma.}
Let $k\ge 3$ and let $\mathcal C$ be a family of \emph{distinct} subsets of $[n]$ that is totally ordered by inclusion (i.e. a chain). Then $\mathcal C$ contains no $k$-sunflower.

\textbf{Proof.}
Assume for contradiction that distinct sets $S_1,\dots,S_k\in\mathcal C$ form a $k$-sunflower. Since $\mathcal C$ is totally ordered by inclusion, after reindexing we may assume
\[
S_1 \subset S_2 \subset \cdots \subset S_k.
\]
Because $S_1\subset S_2$, we have $S_1\cap S_2 = S_1$. Similarly, $S_1\cap S_3=S_1$.
But also $S_2\cap S_3 = S_2$ because $S_2\subset S_3$.
Since the sets are distinct, $S_2\ne S_1$, so $S_2\cap S_3 \ne S_1\cap S_2$.
Thus the pairwise intersections are not all equal, contradicting that $S_1,\dots,S_k$ is a sunflower.
\qed

\textbf{Corollary.}
For every $n\ge 1$ and $k\ge 3$, we have $m(n,k)\ge (n+1)+1 = n+2$, because the chain
$\varnothing \subset \{1\} \subset \{1,2\} \subset \cdots \subset [n]$
has size $n+1$ and is $k$-sunflower-free by the lemma.
\qed

\medskip
\noindent\underline{Lemma 857.2 (Erd\H{o}s--Rado sunflower lemma, $r$-uniform case).}

\textbf{Lemma (Sunflower lemma).}
Fix integers $r\ge 1$ and $k\ge 2$. Let $\mathcal F$ be a family of \emph{distinct} $r$-element sets (i.e. $r$-uniform). If
\[
|\mathcal F| > (k-1)^r\, r! ,
\]
then $\mathcal F$ contains a $k$-sunflower.

\textbf{Proof.}
We prove by induction on $r$.

\emph{Base case $r=1$.} Then every set in $\mathcal F$ is a singleton. If $|\mathcal F|\ge k$, pick any $k$ distinct singletons. Any two distinct singletons have intersection $\varnothing$, so all pairwise intersections are equal to $\varnothing$; hence they form a $k$-sunflower with core $\varnothing$. Since $|\mathcal F|>(k-1)^1\cdot 1!=k-1$ implies $|\mathcal F|\ge k$, the base case holds.

\emph{Induction step.} Assume $r\ge 2$ and the statement holds for $(r-1)$-uniform families.
Assume for contradiction that $\mathcal F$ is $r$-uniform, $k$-sunflower-free, and $|\mathcal F|>(k-1)^r r!$.

Choose a maximal subfamily $\{S_1,\dots,S_t\}\subseteq\mathcal F$ of pairwise disjoint sets (maximal by inclusion). Then $t\le k-1$, because if $t\ge k$ then $S_1,\dots,S_k$ are pairwise disjoint, so their pairwise intersections are all $\varnothing$, hence they form a $k$-sunflower (core $\varnothing$), contradicting sunflower-freeness.

For each $i\in\{1,\dots,t\}$ define
\[
\mathcal F_i := \{T\in\mathcal F : T\cap S_i\ne\varnothing\}.
\]
Because the disjoint family $\{S_i\}$ is maximal, every $T\in\mathcal F$ intersects at least one $S_i$, hence
\[
\mathcal F \subseteq \bigcup_{i=1}^t \mathcal F_i,
\]
and therefore
\[
|\mathcal F| \le \sum_{i=1}^t |\mathcal F_i|.
\]
Fix $i$. For each $x\in S_i$, let
\[
\mathcal F(x) := \{T\in\mathcal F : x\in T\}.
\]
Every $T\in\mathcal F_i$ contains at least one element of $S_i$, so
$\mathcal F_i \subseteq \bigcup_{x\in S_i} \mathcal F(x)$.
Hence by the union bound,
\[
|\mathcal F_i| \le \sum_{x\in S_i} |\mathcal F(x)|.
\]
Now fix an element $x$. Consider the $(r-1)$-uniform family
\[
\mathcal F_x := \{T\setminus\{x\} : T\in\mathcal F,\ x\in T\}.
\]
This has size $|\mathcal F_x|=|\mathcal F(x)|$.
If $|\mathcal F(x)|>(k-1)^{r-1}(r-1)!$, then by the induction hypothesis $\mathcal F_x$ contains a $k$-sunflower $U_1,\dots,U_k$.
Then $U_1\cup\{x\},\dots,U_k\cup\{x\}$ are distinct sets in $\mathcal F$ whose pairwise intersections are
\[(U_i\cup\{x\})\cap (U_j\cup\{x\}) = (U_i\cap U_j)\cup\{x\},
\]
which is independent of $i\ne j$ because $U_i\cap U_j$ is independent of $i\ne j$.
Thus $\mathcal F$ would contain a $k$-sunflower, contradicting sunflower-freeness.
Therefore for every element $x$ we must have
\[
|\mathcal F(x)| \le (k-1)^{r-1}(r-1)!.
\]
Returning to the bound on $|\mathcal F_i|$, we obtain
\[
|\mathcal F_i| \le \sum_{x\in S_i} (k-1)^{r-1}(r-1)! = |S_i|\,(k-1)^{r-1}(r-1)!.
\]
Since $|S_i|=r$, this gives
\[
|\mathcal F_i| \le r\,(k-1)^{r-1}(r-1)! = (k-1)^{r-1}\, r!.
\]
Finally, since $t\le k-1$,
\[
|\mathcal F| \le \sum_{i=1}^t |\mathcal F_i| \le t\,(k-1)^{r-1} r! \le (k-1)\,(k-1)^{r-1} r! = (k-1)^r r!.
\]
This contradicts the assumption $|\mathcal F|>(k-1)^r r!$.
Hence such $\mathcal F$ cannot exist; the sunflower lemma follows.
\qed

\medskip
\noindent\underline{Corollary 857.3 (A concrete upper bound for $m(n,k)$).}

\textbf{Corollary.}
For all $n\ge 1$ and $k\ge 2$,
\[
 m(n,k) \le 1 + \sum_{r=0}^n (k-1)^r r!.
\]

\textbf{Proof.}
Let $\mathcal F\subseteq 2^{[n]}$ be a family with no $k$-sunflower. For each $r$, let
$\mathcal F^{(r)}=\{S\in\mathcal F: |S|=r\}$. Then each $\mathcal F^{(r)}$ is $r$-uniform and $k$-sunflower-free, so by Lemma 857.2,
$|\mathcal F^{(r)}|\le (k-1)^r r!$.
Summing over $r=0,1,\dots,n$ gives $|\mathcal F|\le \sum_{r=0}^n (k-1)^r r!$.
Therefore the minimal $m(n,k)$ forcing a sunflower satisfies $m(n,k)\le 1+\sum_{r=0}^n (k-1)^r r!$.
\qed

\medskip
\noindent\textbf{VERIFICATION.}

\begin{itemize}
\item Lemma 857.1 uses only the chain property and distinctness; the contradiction uses the two smallest sets in the chosen subfamily, so it works for any $k\ge 3$.
\item In Lemma 857.2, the only place maximality of the disjoint family $\{S_i\}$ is used is to assert every $T\in\mathcal F$ intersects some $S_i$.
\item Endpoints: the problem statement uses the literal ``pairwise have the same intersection'', which matches our definition.
\item Small-case computation (for $n\le 4$) is consistent with the trivial fact $m(n,2)=2$.
\end{itemize}

\medskip
\noindent\textbf{FINAL.} \textbf{UNRESOLVED.}

(i) \emph{Strongest proved partial results.}
We proved the explicit lower bound $m(n,k)\ge n+2$ for all $k\ge 3$ (chain construction), and the explicit upper bound
$m(n,k)\le 1+\sum_{r=0}^n (k-1)^r r!$ (sunflower lemma by uniform layers).
We also computed exact values for $m(n,k)$ for $n\le 4$ and $k\in\{2,3,4\}$.

(ii) \emph{First gap.}
Determine the correct exponential growth rate of $m(n,k)$ as $n\to\infty$ for fixed $k\ge 3$; in particular, decide whether $m(n,k)\le c_k^n$ for some $c_k<2$ and give sharp $c_k$.

(iii) \emph{Top 3 next moves.}
(1) Strengthen the layer-by-layer bound by exploiting interactions between layers (the present upper bound is dominated by the $r=n$ term and is extremely loose).
(2) For $k=3$, attempt to reproduce a polynomial-method style argument directly in $2^{[n]}$ to recover a $c^n$ upper bound with $c<2$.
(3) Compute $F(n,k)$ exactly for $n=5,6$ (via SAT/ILP) to guess extremal structures.

(iv) \emph{What a minimal counterexample would look like.}
A near-extremal sunflower-free family would have size close to the best possible exponential base (if it exists), likely exhibiting product/tensor structure across coordinates (since tensoring preserves sunflower-freeness and stabilises exponential growth rates).


