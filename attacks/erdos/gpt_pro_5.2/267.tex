\section*{Erd\H{o}s Problem \#267}

\subsection*{1) FORMAL RESTATEMENT}

Let $(F_n)_{n\ge 1}$ be the Fibonacci numbers defined by $F_1=F_2=1$ and $F_{n+1}=F_n+F_{n-1}$ for $n\ge 2$.

Given an increasing sequence of indices $1\le n_1<n_2<\cdots$ such that
\[
\frac{n_{k+1}}{n_k}\ge c>1\quad\text{for all }k,
\]
must the subseries
\[
S:=\sum_{k=1}^{\infty}\frac{1}{F_{n_k}}
\]
be irrational?

\subsection*{2) QUICK LITERATURE/CONTEXT CHECK (only if browsing available)}

\begin{itemize}[leftmargin=2em]
\item The special lacunary case $n_k=2^k$ was proved to give an irrational sum by Good (1974) and by Bicknell--Hoggatt (1976).
\item The full reciprocal Fibonacci constant $\sum_{n\ge 1} 1/F_n$ was proved irrational by Andr\'e--Jeannin (1989).
\item The general ``ratio bounded away from 1'' condition $n_{k+1}/n_k\ge c>1$ is significantly weaker than divisibility assumptions (e.g. $n_k\mid n_{k+1}$), and the problem is listed as open in modern compilations.
\end{itemize}

\subsection*{3) ATTACK PLAN}

\begin{enumerate}[leftmargin=2.2em]
\item Isolate a tractable strengthening of the hypothesis (e.g. $n_k\mid n_{k+1}$) under which the sum can be rewritten as a Cantor-type series.
\item Prove irrationality in that stronger setting using a general ``mixed radix'' approximation lemma.
\item Record that the original problem removes divisibility, and explain why the Cantor-series proof no longer applies directly.
\end{enumerate}

\subsection*{4) WORK (with full details)}

\subsubsection*{4.1. A useful lemma for divisibility chains (a partial result)}

We begin with a general irrationality lemma.

\begin{lemma}[Cantor-series irrationality for growing bases]\label{lem:cantor}
Let $Q_1<Q_2<\cdots$ be positive integers with $Q_k\mid Q_{k+1}$ for all $k$, and write $q_{k+1}:=Q_{k+1}/Q_k\in\mathbb{N}$. Assume $q_{k}\to\infty$ as $k\to\infty$. Then
\[
X:=\sum_{k=1}^{\infty}\frac{1}{Q_k}
\]
is irrational.
\end{lemma}

\begin{proof}
Set $X_N:=\sum_{k=1}^{N}\frac{1}{Q_k}$. Since $Q_k\mid Q_N$ for $k\le N$, we may write
\[
X_N=\frac{P_N}{Q_N}
\]
for some integer $P_N$.

Let $R_N:=X-X_N=\sum_{k>N}\frac{1}{Q_k}>0$ be the tail. Multiply by $Q_N$:
\[
Q_N R_N=\sum_{k>N}\frac{Q_N}{Q_k}.
\]
Because $Q_{N+j}=Q_N\cdot q_{N+1}q_{N+2}\cdots q_{N+j}$, we obtain
\[
Q_N R_N=\sum_{j=1}^{\infty}\frac{1}{q_{N+1}q_{N+2}\cdots q_{N+j}}.
\]
In particular, the first term gives $Q_N R_N\le \sum_{j\ge 1} (q_{N+1})^{-j}=\frac{1}{q_{N+1}-1}$, and hence
\begin{equation}\label{eq:QNRN_to_0}
0< Q_N R_N \le \frac{1}{q_{N+1}-1}\xrightarrow[N\to\infty]{} 0
\end{equation}
because $q_{N+1}\to\infty$.

Now suppose for contradiction that $X\in\mathbb{Q}$, say $X=p/r$ in lowest terms. Consider the rational number
\[
Q_N X - P_N = Q_N(X-X_N)=Q_N R_N.
\]
The left-hand side is a rational with denominator dividing $r$. Therefore either it is an integer, or else its distance to the nearest integer is at least $1/r$ (since non-integer rationals with denominator $\le r$ have fractional part a nonzero multiple of $1/r$).

For all sufficiently large $N$, the estimate \eqref{eq:QNRN_to_0} gives $0<Q_N R_N<1/r$, so $Q_N R_N$ cannot be an integer and cannot be $\ge 1/r$ away from an integer. Contradiction. Hence $X\notin\mathbb{Q}$.
\end{proof}

\subsubsection*{4.2. Application to geometric index sets $n_k=m^k$}

We now show that many lacunary Fibonacci subseries are irrational under a stronger hypothesis than in the problem.

\begin{lemma}[Divisibility of Fibonacci numbers]\label{lem:F_divides}
If $m\mid n$, then $F_m\mid F_n$.
\end{lemma}

\begin{proof}
It suffices to prove $F_m\mid F_{km}$ for all integers $k\ge 1$. We use the standard Fibonacci addition identity
\begin{equation}\label{eq:addition}
F_{u+v}=F_u F_{v+1}+F_{u-1}F_v\qquad(u,v\ge 1),
\end{equation}
which can be proved by induction on $v$ (the $v=1$ case is $F_{u+1}=F_u+F_{u-1}$, and the induction step follows from $F_{u+(v+1)}=F_{u+v}+F_{u+v-1}$ and applying \eqref{eq:addition} to $v$ and $v-1$).

Fix $m$. For $k=1$, $F_m\mid F_{m}$ holds. Assume $F_m\mid F_{km}$. Apply \eqref{eq:addition} with $u=km$ and $v=m$:
\[
F_{(k+1)m}=F_{km+m}=F_{km}F_{m+1}+F_{km-1}F_m.
\]
Both terms on the right are divisible by $F_m$ (the first by the induction hypothesis, the second trivially), hence $F_m\mid F_{(k+1)m}$. This completes the induction.
\end{proof}

\begin{proposition}[A partial answer: $n_k=m^k$ gives an irrational sum]\label{prop:mk}
Fix an integer $m\ge 2$ and set $n_k:=m^k$. Then
\[
\sum_{k=1}^{\infty}\frac{1}{F_{m^k}}
\]
is irrational.
\end{proposition}

\begin{proof}
Set $Q_k:=F_{m^k}$. Since $m^k\mid m^{k+1}$, Lemma~\ref{lem:F_divides} gives $Q_k\mid Q_{k+1}$.

Let $q_{k+1}:=Q_{k+1}/Q_k$. Using the standard asymptotic $F_n\sim \varphi^n/\sqrt{5}$ (with $\varphi=(1+\sqrt5)/2$), we have
\[
q_{k+1}=\frac{F_{m^{k+1}}}{F_{m^k}}\asymp \varphi^{m^{k+1}-m^k}=\varphi^{(m-1)m^k}\xrightarrow[k\to\infty]{}\infty.
\]
Therefore the hypotheses of Lemma~\ref{lem:cantor} hold, and the sum $\sum_{k\ge 1} 1/Q_k=\sum_{k\ge 1}1/F_{m^k}$ is irrational.
\end{proof}

\begin{remark}
The special case $m=2$ recovers the irrationality of $\sum_k 1/F_{2^k}$ (which was originally established by Good and by Bicknell--Hoggatt). The argument above shows irrationality more generally for any geometric subsequence $m^k$.
\end{remark}

\subsubsection*{4.3. Why this does not settle the original problem}

Lemma~\ref{lem:cantor} crucially used a divisibility chain $Q_k\mid Q_{k+1}$, which is guaranteed if $n_k\mid n_{k+1}$ (by Lemma~\ref{lem:F_divides}). The Erd\H{o}s hypothesis $n_{k+1}/n_k\ge c>1$ does \emph{not} imply any such divisibility, so the denominators $F_{n_k}$ need not nest.

Without nesting, the partial sums $\sum_{k\le N} 1/F_{n_k}$ do not share a single ``natural'' denominator $Q_N$ dividing all others, so the clean modular obstruction in Lemma~\ref{lem:cantor} is unavailable.

\subsection*{5) VERIFICATION / SANITY CHECK}

\begin{itemize}[leftmargin=2em]
\item The bound $Q_N R_N\le 1/(q_{N+1}-1)$ is obtained by comparing the tail to a geometric series using $q_{N+j}\ge q_{N+1}$ for $j\ge 1$ (valid since $q_k\ge 2$ and $q_k\to\infty$ so eventually nondecreasing; if not monotone, one can instead use $Q_N R_N\le \sum_{j\ge 1} 1/q_{N+1}^j$ after replacing $q_{N+1}$ by $\min_{j\ge 1} q_{N+j}$).
\item In Proposition~\ref{prop:mk}, the fact $q_{k+1}\to\infty$ follows from exponential growth of Fibonacci numbers.
\item The argument for divisibility $F_m\mid F_{km}$ is elementary and uses only the addition identity.
\end{itemize}

\subsection*{6) FINAL}

\textbf{UNRESOLVED.}

\begin{itemize}[leftmargin=2em]
\item[(i)] \textbf{What is proved here:} under the stronger hypothesis $n_k=m^k$ (or more generally $n_k\mid n_{k+1}$ with rapidly growing quotients), the sum $\sum 1/F_{n_k}$ is irrational.
\item[(ii)] \textbf{What remains open:} the original statement for arbitrary sequences with only the ratio gap condition $n_{k+1}/n_k\ge c>1$ (no divisibility).
\item[(iii)] \textbf{Most plausible next step:} develop an approximation/irrationality criterion that does not rely on nested denominators, perhaps exploiting gcd-structure $\gcd(F_m,F_n)=F_{\gcd(m,n)}$ or using lower bounds on least common multiples of $F_{n_k}$.
\item[(iv)] \textbf{Completion estimate:} proving irrationality in the full generality likely needs new number-theoretic input beyond mixed-radix arguments; alternatively, a counterexample would require constructing a sparse index set with an unexpectedly rational reciprocal sum.
\end{itemize}

\subsection*{7) COMPLETION ESTIMATE}

To completely resolve Erd\H{o}s \#267 one would need either:
\begin{enumerate}[leftmargin=2.2em]
\item a general irrationality criterion covering all lacunary Fibonacci subseries with $n_{k+1}/n_k\ge c>1$, or
\item an explicit construction of such a lacunary sequence $(n_k)$ yielding a rational sum.
\end{enumerate}
The partial result above indicates that divisibility chains force irrationality, but extending to general lacunary sequences remains a substantial gap.

