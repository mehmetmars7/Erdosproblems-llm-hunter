
1) FORMAL RESTATEMENT

Fix integers $k\ge 2$ and $1\le m\le k$.
Define
\[
f_{k,m}(x):=\bigl|\{n\in\mathbb Z_{\ge 0}: n\le x\text{ and }n=a_1^k+\cdots+a_m^k\text{ for some }a_i\in\mathbb Z_{\ge 0}\}\bigr|.
\]
So $f_{k,m}(x)$ counts \emph{distinct} integers $\le x$ representable as a sum of $m$ nonnegative $k$th powers.

Questions (as written):

(Q1) Is it true that for every $\epsilon>0$,
\[
f_{k,k}(x)\gg_\epsilon x^{1-\epsilon}?\]

(Q2) Is it true that if $m<k$, then
\[
f_{k,m}(x)\gg x^{m/k}\quad\text{for all sufficiently large }x?\]

Edge cases.
$f_{k,m}(x)$ is nondecreasing in $x$, and $f_{k,m}(x)\ge 1$ for all $x\ge 0$ since $0=0^k+\cdots+0^k$.


2) QUICK LITERATURE/CONTEXT CHECK

I do not use any results not stated in the problem text.
The statement notes:

* For $k=2$, Landau proved $f_{2,2}(x)\sim c x/\sqrt{\log x}$.
* For $k>2$, it is not known whether $f_{k,k}(x)=o(x)$.


3) ATTACK PLAN

Provide elementary (weak) bounds and a clean construction giving a power-law lower bound.

(1) Trivial bound: all $k$th powers $\le x$ are representable (using one nonzero term and the rest $0$).

(2) For $m\ge 2$ and $k\ge 3$, construct many \emph{distinct} sums $a^k+b^k\le x$ using a large-$a$/small-$b$ separation argument.
This gives a provable lower bound $f_{k,m}(x)\ge c x^{3/(2k)}$ for large $x$.

Also: compute small values for sanity.


4) WORK

PHASE 1: FAST REALITY CHECK (brute force for small parameters)

Exact computed values (distinct representable integers) include:

For $k=3$:
\[
\begin{array}{c|ccc}
 & x=100 & x=500 & x=1000\\\hline
f_{3,2}(x) & 14 & 34 & 51\\
f_{3,3}(x) & 29 & 98 & 173
\end{array}
\]

For $k=4$:
\[
\begin{array}{c|ccc}
 & x=100 & x=500 & x=1000\\\hline
f_{4,2}(x) & 9 & 14 & 20\\
f_{4,3}(x) & 15 & 30 & 49\\
f_{4,4}(x) & 22 & 54 & 97
\end{array}
\]


Lemma 323.1 (trivial lower bound from single $k$th powers).

For any $k\ge 2$, any $m\ge 1$, and any $x\ge 0$,
\[
f_{k,m}(x)\ge \bigl\lfloor x^{1/k}\bigr\rfloor+1.
\]

Proof.
For each integer $0\le t\le \lfloor x^{1/k}\rfloor$, the number $t^k\le x$ is representable as
\[
t^k = t^k + 0^k+\cdots+0^k\quad\text{(with $m-1$ zeros)}.
\]
These $\lfloor x^{1/k}\rfloor+1$ values are distinct, so $f_{k,m}(x)$ is at least this large. \qed


Lemma 323.2 (a provable power-law lower bound for $m\ge 2$, $k\ge 3$).

Fix $k\ge 3$. There exists $x_0=x_0(k)$ such that for all $x\ge x_0$ and all $m\ge 2$,
\[
f_{k,m}(x)\ge \frac{1}{8}\,x^{3/(2k)}.
\]

Proof.
It suffices to prove the bound for $m=2$, since any sum of two $k$th powers is also a sum of $m$ $k$th powers by adding $(m-2)$ zeros.
So fix $m=2$.

Let
\[
M:=\left\lfloor x^{1/k}\right\rfloor,\qquad B:=\left\lfloor x^{1/(2k)}\right\rfloor.
\]
Consider integers $a$ in the range
\[
\left\lceil \frac{M}{2}\right\rceil\le a\le M.
\]
For each such $a$ and each integer $b$ with $0\le b\le B$, the sum $a^k+b^k$ satisfies
\[
a^k+b^k\le M^k+B^k\le x + x^{1/2}.
\]
To ensure $a^k+b^k\le x$ we restrict slightly: replace $M$ by
\[
M':=\left\lfloor (x-B^k)^{1/k}\right\rfloor.
\]
For $x\ge 1$, $M'\ge M-1$, and for $x$ large we still have $M'\ge M/2$.
So we take
\[
\left\lceil \frac{M}{2}\right\rceil\le a\le M'.
\]
Then $a^k\le x-B^k$ and hence $a^k+b^k\le x$ for all $0\le b\le B$.

For a fixed $a$, the values $a^k+b^k$ are strictly increasing in $b$ (since $b\mapsto b^k$ is strictly increasing on $\mathbb Z_{\ge 0}$), so for each $a$ we obtain at least $B+1$ distinct representable integers.

We now show that for $x$ large, the intervals
\[
I_a:=[a^k,\ a^k+B^k]
\]
for consecutive values of $a$ in the chosen range are disjoint. It is enough to show
\[
(a+1)^k > a^k + B^k.
\]
By the binomial theorem,
\[
(a+1)^k = a^k + k a^{k-1} + \binom{k}{2}a^{k-2}+\cdots+1 \ge a^k + k a^{k-1}.
\]
So it suffices to have $k a^{k-1} > B^k$.
For $a\ge M/2$,
\[
ka^{k-1} \ge k\left(\frac{M}{2}\right)^{k-1}.
\]
Since $M\ge x^{1/k}-1$, there exists $x_0(k)$ such that for all $x\ge x_0(k)$,
\[
k\left(\frac{M}{2}\right)^{k-1} > x^{1/2} \ge B^k.
\]
(The exponent comparison is $(k-1)/k>1/2$ for $k\ge 3$.)
Thus for $x\ge x_0(k)$, the intervals $I_a$ are pairwise disjoint over $a\in[\lceil M/2\rceil, M']\cap\mathbb Z$.

Therefore, for $x\ge x_0(k)$, all the values $a^k+b^k$ with $\lceil M/2\rceil\le a\le M'$ and $0\le b\le B$ are distinct.
Hence
\[
f_{k,2}(x)\ge (\#\{a\})\cdot (B+1).
\]
For $x$ large, $M'\ge M/2$ and $\#\{a\}\ge M/4$, while $B+1\ge B\ge \tfrac12 x^{1/(2k)}$.
Also $M\ge \tfrac12 x^{1/k}$ for $x$ large. Combining these crude inequalities yields
\[
f_{k,2}(x)\ge \frac{M}{4}\cdot \frac12 x^{1/(2k)}\ge \frac{1}{8}\,x^{1/k}\,x^{1/(2k)}=\frac{1}{8}\,x^{3/(2k)}.
\]
Finally $f_{k,m}(x)\ge f_{k,2}(x)$ for all $m\ge 2$, proving the lemma. \qed


5) VERIFICATION

-- Lemma 323.1: uses only the representation $t^k=t^k+0+\cdots+0$.

-- Lemma 323.2: key disjointness condition is $(a+1)^k-a^k\ge k a^{k-1}$ (binomial theorem), and $k a^{k-1}>B^k$ holds for all large $x$ because $a\asymp x^{1/k}$ while $B^k\asymp x^{1/2}$.

-- Computations: performed by brute force over all tuples $(a_1,\dots,a_m)$ with $a_i^k\le x$.


6) FINAL

**UNRESOLVED**

(i) Strongest fully proved partial result obtained here.

* Trivial lower bound: $f_{k,m}(x)\ge \lfloor x^{1/k}\rfloor+1$ for all $k\ge 2$, $m\ge 1$ (Lemma 323.1).
* For $k\ge 3$ and $m\ge 2$, a rigorous power-law lower bound $f_{k,m}(x)\ge \tfrac18 x^{3/(2k)}$ for all sufficiently large $x$ (Lemma 323.2).

(ii) Exact first gap.

Prove the conjectured exponent $m/k$ (or $1-\epsilon$ when $m=k$): my construction only yields exponent $3/(2k)$ (independent of $m$ once $m\ge 2$), far weaker than $m/k$ when $m$ is close to $k$.

(iii) Top 3 next moves (concrete targets).

1. Replace the ``$b$-varies'' argument in Lemma 323.2 by a method producing $\gg B^{m}$ distinct sums of $m$ $k$th powers (not just $\gg B$), ideally by controlling collisions among $b_1^k+\cdots+b_m^k$.
2. Develop an upper bound on collision multiplicity for $b_1^k+\cdots+b_m^k$ in a box, which would convert tuple-counting into distinct-sum counting.
3. Compute $f_{k,m}(x)$ for larger $x$ for small $(k,m)$ (e.g. $(3,3)$, $(4,4)$) to see whether the conjectured exponents fit data.

(iv) What a minimal counterexample would likely look like.

If $f_{k,m}(x)\not\gg x^{m/k}$ for some $m<k$, then there would be massive collisions among sums of $m$ $k$th powers, forcing the set
\[\{a_1^k+\cdots+a_m^k: 0\le a_i\le x^{1/k}\}\]
to have size $o(x^{m/k})$. Any structural explanation would need a mechanism producing many distinct $m$-tuples with the same sum (beyond the obvious permutation symmetries), at a scale polynomial in $x$.


