
\subsection*{FORMAL RESTATEMENT}
For a real number $x$, define $\|x\|:=\min_{m\in\mathbb{Z}}|x-m|$, the distance from $x$ to the nearest integer.
Given $\alpha,\beta\in\mathbb{R}$, the Littlewood conjecture asserts:
\[
\liminf_{n\to\infty} n\,\|n\alpha\|\,\|n\beta\| = 0.
\]
Question: Is this statement true for \emph{all} real pairs $(\alpha,\beta)$?

Edge cases: if $\alpha\in\mathbb{Q}$ then $\|n\alpha\|=0$ for infinitely many $n$, so the liminf is $0$ (similarly for $\beta\in\mathbb{Q}$).

\subsection*{QUICK LITERATURE/CONTEXT CHECK}
The statement identifies this as the (infamous) Littlewood conjecture. No additional background is provided in the problem file, and per the integrity rule I do not cite or rely on external results here.

\subsection*{ATTACK PLAN}
\textbf{Proof track:}
\begin{itemize}
\item Prove the conjecture in trivial/degenerate cases (e.g. rational parameters).
\item Establish unconditional bounds from classical Diophantine approximation (Dirichlet) to at least show $\liminf$ is finite and small.
\end{itemize}
\textbf{Disproof track:}
\begin{itemize}
\item Attempt to construct a pair $(\alpha,\beta)$ with a positive lower bound on $n\|n\alpha\|\|n\beta\|$; no such construction is known to me and would be highly nontrivial.
\end{itemize}
I provide two proved lemmas and numerical sanity checks, but no full proof/disproof.

\subsection*{WORK}
\paragraph{Lemma 1 (rational parameter case).}
If $\alpha\in\mathbb{Q}$ or $\beta\in\mathbb{Q}$, then
\[\liminf_{n\to\infty} n\,\|n\alpha\|\,\|n\beta\| = 0.
\]

\paragraph{Proof.}
Assume $\alpha\in\mathbb{Q}$. Then $\alpha=p/q$ with integers $p,q\ge 1$ in lowest terms.
For every multiple $n=tq$, we have $n\alpha=tp\in\mathbb{Z}$, hence $\|n\alpha\|=0$.
Therefore for infinitely many $n$ the product $n\|n\alpha\|\|n\beta\|$ equals $0$, so the liminf is $0$.
The argument with $\beta\in\mathbb{Q}$ is identical.
\qed

\paragraph{Lemma 2 (Dirichlet simultaneous approximation $\Rightarrow$ universal bound $\liminf\le 1$).}
For every pair $(\alpha,\beta)\in\mathbb{R}^2$, there exist infinitely many integers $n\ge 1$ such that
\[n\,\|n\alpha\|\,\|n\beta\| \le 1.
\]
In particular,
\[\liminf_{n\to\infty} n\,\|n\alpha\|\,\|n\beta\| \le 1.
\]

\paragraph{Proof.}
Fix an integer $N\ge 1$.
Consider the $N+1$ points in the unit square $[0,1)^2$ given by
\[P_j := (\{j\alpha\},\{j\beta\})\quad\text{for }j=0,1,\dots,N,
\]
where $\{x\}=x-\lfloor x\rfloor$ denotes the fractional part.
Partition $[0,1)^2$ into a grid of $\lceil\sqrt N\rceil\times\lceil\sqrt N\rceil$ rectangles, each of side length at most $1/\sqrt N$.
There are at most $N$ such rectangles.
By the pigeonhole principle, two distinct points $P_i,P_j$ with $0\le i<j\le N$ fall in the same rectangle.
Let $n=j-i$ (so $1\le n\le N$).
Then the coordinate differences satisfy
\[\|n\alpha\| \le |\{j\alpha\}-\{i\alpha\}| \le \frac{1}{\sqrt N},\qquad
  \|n\beta\| \le |\{j\beta\}-\{i\beta\}| \le \frac{1}{\sqrt N}.
\]
(The first inequality uses that the distance to the nearest integer is at most the distance between two fractional parts; if the difference crosses $0$, one can instead use $1-|\{j\alpha\}-\{i\alpha\}|$, which is also bounded by the rectangle size. The same holds for $\beta$.)
Therefore
\[n\,\|n\alpha\|\,\|n\beta\| \le N\cdot \frac{1}{\sqrt N}\cdot\frac{1}{\sqrt N}=1.
\]
Since this holds for each $N$, and produces some $1\le n\le N$, taking $N\to\infty$ yields infinitely many such $n$.
\qed

\paragraph{FAST REALITY CHECK (numerical search for sample pairs).}
For two sample pairs I computed
\[\min_{1\le n\le N} n\,\|n\alpha\|\,\|n\beta\|\]
for $N\in\{10,100,1000,10000\}$:
\begin{itemize}
\item $(\alpha,\beta)=(\sqrt2,\sqrt3)$: minimum values were
$0.0874886095$ at $N=10$ (attained at $n=7$),
$0.00995678225$ at $N=100$ (attained at $n=41$), and unchanged through $N=10000$.
\item $(\alpha,\beta)=(\sqrt2,\pi)$: minimum values were
$0.00622729112$ at $N=10$ (attained at $n=7$),
unchanged at $N=100$,
then $0.000660373015$ at $N=1000$ (attained at $n=113$), unchanged through $N=10000$.
\end{itemize}
These are only sanity checks (they do not prove convergence of the liminf).

\subsection*{VERIFICATION}
\begin{itemize}
\item Lemma~1: verified that $\|n\alpha\|=0$ along multiples of the denominator, forcing liminf $0$.
\item Lemma~2: verified pigeonhole step in $[0,1)^2$ and the conversion from closeness of fractional parts to bounds on $\|n\alpha\|$ and $\|n\beta\|$.
\item Numerical check: confirmed distances computed as $|x-\mathrm{round}(x)|$.
\end{itemize}

\subsection*{FINAL}
\textbf{UNRESOLVED}

(i) \textbf{Strongest proved partial result.}
\begin{itemize}
\item If $\alpha$ or $\beta$ is rational, the conjectured liminf equals $0$.
\item For all $(\alpha,\beta)$, Dirichlet's simultaneous approximation yields infinitely many $n$ with $n\|n\alpha\|\|n\beta\|\le 1$, so the liminf is always $\le 1$.
\end{itemize}

(ii) \textbf{First gap (crisp).} Improve the universal bound from Lemma~2 to show that for every pair $(\alpha,\beta)$ one can achieve
\[n\,\|n\alpha\|\,\|n\beta\| < \varepsilon\]
for arbitrarily small $\varepsilon>0$ (equivalently, show the liminf is $0$).

(iii) \textbf{Top 3 next moves.}
\begin{enumerate}
\item Strengthen the pigeonhole argument using finer partitions or geometric methods (e.g. looking for lattice points in regions defined by $\|n\alpha\|\,\|n\beta\|$) to push the bound below an absolute constant.
\item Analyze special classes of pairs $(\alpha,\beta)$ (e.g. one parameter with exceptionally good rational approximations) to force $n\|n\alpha\|\to 0$ along a subsequence, which would imply the conjecture for that pair.
\item Computationally search for sequences $n$ minimizing $n\|n\alpha\|\|n\beta\|$ for specific pairs (e.g. quadratic irrationals) to guess structure of near-minimizers.
\end{enumerate}

(iv) \textbf{Minimal counterexample structure.} A counterexample would be a pair $(\alpha,\beta)$ and a constant $c>0$ such that
\[n\,\|n\alpha\|\,\|n\beta\|\ge c\quad\text{for all }n\ge 1.
\]
Such a pair would have to simultaneously avoid very small fractional parts in both coordinates across all integer multiples, exhibiting a strong form of ``simultaneous bad approximation''.
