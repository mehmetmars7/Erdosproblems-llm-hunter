\section{Erd\H{o}s Problem \#39 (Round 2)}

\subsection*{1) ROUND-2 OBJECTIVE}
\textbf{Path (A) --- proof/construction track.}

Round~1 ended \textbf{UNRESOLVED} with only the elementary greedy lower bound
$|A\cap[1,N]|\gg N^{1/3}$ (Round~1, Lemma~39.3) and the crude interval upper bound
$|A\cap[1,N]|<\sqrt{2N}+1$ (Round~1, Lemma~39.2).

In Round~2 I push the constructive lower bound substantially past exponent $1/3$ by giving a fully explicit infinite Sidon sequence with counting function
\[
A(x)=x^{\sqrt2-1+o(1)},
\]
following and re-verifying the discrete-logarithm + deletion construction of Cilleruelo.
This improves the Round~1 lower bound exponent from $1/3\approx0.333$ to $\sqrt2-1\approx0.414$.

This does \emph{not} settle Erd\H{o}s's conjecture (which asks for $A(x)\gg_\varepsilon x^{1/2-\varepsilon}$ for \emph{every} $\varepsilon>0$). The conjecture remains open in the current literature listings\footnote{For instance, the Erd\H{o}s Problems project lists the conjectural density question among the open ``Sidon sets'' problems (last edited December~28,~2025).}. \cite{ErdosProblemsPrize,Cilleruelo2013}

\subsection*{2) ROUND-1 FOUNDATION USED}
I rely on the following Round~1 items as already established:
\begin{itemize}
\item The formal restatement/definition of an infinite Sidon set and counting function $A(x)=|A\cap[1,x]|$.
\item Round~1, Lemma~39.3: the greedy algorithm gives an explicit infinite Sidon sequence with $A(x)\gg x^{1/3}$.
\item Round~1, Lemma~39.2: $|A\cap[1,N]|\ll \sqrt N$ for any Sidon $A\subseteq[1,N]$ (upper bound context).
\end{itemize}
No Round~1 proof is repeated.

\subsection*{3) NEW INSIGHT / TOOL (ROUND-2)}
\begin{itemize}
\item \textbf{Mixed-radix encoding with carry-free addition:} represent integers in a base $b_1,b_2,\dots$ where each digit is $<b_j/2$, so that addition of two encoded numbers is digitwise (no carries).
\item \textbf{Discrete logarithm digit selection:} choose digits $x_j(p)$ so that $g_j^{x_j(p)}\equiv p\pmod{q_j}$ for a primitive root $g_j$ mod a prime $q_j$.
\item \textbf{Deletion-by-largest-element:} characterize all possible repeated sums, then remove from the sequence the largest term in each potential collision. If the removed set is thin enough, the counting function is essentially unchanged.
\end{itemize}

\subsection*{4) ATTACK PLAN (ROUND-2)}
\textbf{Gap after Round~1:} no explicit construction beating exponent $1/3$.

\textbf{Claims to prove now:}
\begin{enumerate}
\item Construct an explicit infinite sequence $A_{q,c}$ (indexed by primes) with $A_{q,c}(x)=x^{c+o(1)}$ for any fixed $0<c<1/2$.
\item Analyze all repeated sums in $A_{q,c}$ via digitwise equality (carry-free) and deduce strong congruence constraints on the corresponding primes.
\item Show:
  \begin{itemize}
  \item for $c_0:=\frac{3-\sqrt5}{2}$, the constraints are impossible, so $A_{q,c_0}$ is already Sidon;
  \item for $c:=\sqrt2-1$, remove an explicit thin subsequence of primes to destroy all repeated sums, and prove the surviving sequence is Sidon with counting function $x^{c+o(1)}$.
  \end{itemize}
\end{enumerate}

\subsection*{5) WORK (ROUND-2)}

\subsubsection*{5.1. Mixed radix and carry-free addition}
Fix integers $b_j\ge 2$.
Every $n\in\mathbb N$ has a unique expansion
\[
 n = x_1 + x_2 b_1 + x_3 b_1 b_2 + \cdots + x_k b_1\cdots b_{k-1}
\]
with digits $0\le x_j<b_j$ and $x_k\ne 0$.
(Proof: repeated Euclidean division; uniqueness by induction.)
We write this as the digit string $n=x_k\cdots x_1$.

\begin{lemma}[Carry-free digitwise sums]
\label{lem:carryfree}
If $0\le u_j,v_j\le b_j-1$ and $u_j+v_j<b_j$ for every $j$, then
\[
\sum_{j\ge1} u_j\prod_{t<j} b_t\; +\; \sum_{j\ge1} v_j\prod_{t<j} b_t
\]
has digit string $(u_j+v_j)_j$.
In particular equality of two such sums implies equality digitwise.
\end{lemma}

\begin{proof}
This is immediate from the uniqueness of mixed-radix expansions: no carry occurs in any digit, so the coefficient of $\prod_{t<j}b_t$ in the sum is exactly $u_j+v_j$.
\end{proof}

\subsubsection*{5.2. Choice of primes and the discrete-log digit map}
Choose a sequence of primes $(q_j)_{j\ge1}$ such that
\begin{equation}
\label{eq:qj-growth}
2^{2j-1}<q_j\le 2^{2j+1}\qquad(j\ge1).
\end{equation}
Existence is easy, e.g. by Bertrand's postulate applied to $2^{2j-1}$.
Set
\[
 b_j := 4q_j.
\]
For each $j$, fix a primitive root $g_j$ modulo $q_j$ (exists since $\mathbb F_{q_j}^\times$ is cyclic).

Fix a parameter $c$ with $0<c<\tfrac12$.
Partition the prime numbers into blocks
\begin{equation}
\label{eq:Pkdef}
\mathcal P = \bigcup_{k\ge2} P_k,\qquad
P_k := \Bigl\{p\text{ prime}: 2^{c(k-1)^2-3}<p\le 2^{ck^2-3}\Bigr\}.
\end{equation}

\begin{definition}[Digit assignment]
\label{def:digits}
For $p\in P_k$ define digits $x_j(p)$ by
\begin{equation}
\label{eq:digitcong}
 g_j^{x_j(p)}\equiv p\pmod{q_j},\qquad q_j+1\le x_j(p)\le 2q_j-1\qquad(1\le j\le k),
\end{equation}
and set $x_j(p):=0$ for $j>k$.
Define the integer
\begin{equation}
\label{eq:apdef}
 a_p := x_1(p) + x_2(p)b_1 + \cdots + x_k(p) b_1\cdots b_{k-1},
\end{equation}
i.e. the digit string $a_p=x_k(p)\cdots x_1(p)$ in base $(b_j)$.
Let
\[
A_{q,c}:=\{a_p: p\in\mathcal P\}.
\]
\end{definition}

\begin{lemma}[Well-defined digits]
\label{lem:digitwelldefined}
For each $p\in\mathcal P$ and each $j\ge1$, the digit $x_j(p)$ in \eqref{eq:digitcong} exists and is unique.
Moreover, for all $j$ and all $p$ we have $0\le x_j(p)\le 2q_j-1<b_j$.
\end{lemma}

\begin{proof}
Fix $j$ and a prime $p\ne q_j$.
Since $g_j$ is a primitive root mod $q_j$, there exists an exponent $e$ such that $g_j^e\equiv p\pmod{q_j}$.
Replacing $e$ by $e+t(q_j-1)$ does not change $g_j^e\bmod q_j$.
The interval $[q_j+1,2q_j-1]$ has length $q_j-1$, so it contains exactly one representative of the congruence class $e\pmod{q_j-1}$.
That representative is $x_j(p)$.
The stated inequalities are immediate.
\end{proof}

\subsubsection*{5.3. Injectivity and the counting function of $A_{q,c}$}

\begin{proposition}[Injectivity]
\label{prop:injective}
The map $p\mapsto a_p$ is injective, so $A_{q,c}$ is an infinite set.
\end{proposition}

\begin{proof}
Suppose $a_p=a_{p'}$.
By uniqueness of mixed-radix expansion, $x_j(p)=x_j(p')$ for all $j$.
Let $k$ be the largest index with $x_k(p)\ne0$; by construction $p,p'\in P_k$.
For each $1\le j\le k$ we have
$p\equiv g_j^{x_j(p)}\equiv g_j^{x_j(p')}\equiv p'\pmod{q_j}$.
Hence $p\equiv p'\pmod{Q}$ where $Q:=q_1\cdots q_k$.
From \eqref{eq:qj-growth} we get
\[
Q > \prod_{j=1}^k 2^{2j-1} = 2^{\sum_{j=1}^k(2j-1)} = 2^{k^2}.
\]
Also $|p-p'|<2^{ck^2-3}\le 2^{ck^2}$ (since $p,p'\le 2^{ck^2-3}$), and because $c<1$ we have $2^{ck^2}<2^{k^2}$ for all sufficiently large $k$.
Thus the only way to have $Q\mid(p-p')$ is $p-p'=0$, i.e. $p=p'$.
\end{proof}

\begin{proposition}[Growth of $A_{q,c}$]
\label{prop:growth}
Let $A_{q,c}(x):=|A_{q,c}\cap[1,x]|$.
For any fixed $0<c<\tfrac12$ we have
\[
A_{q,c}(x)=x^{c+o(1)}\qquad(x\to\infty).
\]
\end{proposition}

\begin{proof}
Let $B_k:=b_1\cdots b_k$.
From \eqref{eq:qj-growth} we have
\[
2^{2j+1}<b_j\le 2^{2j+3}
\quad\Rightarrow\quad
2^{\sum_{j=1}^k(2j+1)}<B_k\le 2^{\sum_{j=1}^k(2j+3)}.
\]
Since $\sum_{j=1}^k(2j+1)=k^2+2k$ and $\sum_{j=1}^k(2j+3)=k^2+4k$, we get
\begin{equation}
\label{eq:Bk-asymp}
2^{k^2+2k}<B_k\le 2^{k^2+4k}.
\end{equation}
Given $x$, choose $k$ such that $B_k<x\le B_{k+1}$.
Then \eqref{eq:Bk-asymp} implies $k^2=\log_2 x+o(\log x)$.

\emph{Lower bound.}
If $p\le 2^{ck^2-3}$ then $p\in P_\ell$ for some $\ell\le k$, and by construction $a_p< B_k<x$.
Hence
\[
A_{q,c}(x)\ge \pi(2^{ck^2-3}).
\]
By the prime number theorem, $\pi(2^{ck^2-3})=2^{ck^2+o(k^2)}=x^{c+o(1)}$.

\emph{Upper bound.}
If $p>2^{c(k+1)^2-3}$ then $p\in P_\ell$ for some $\ell\ge k+2$, and then $a_p\ge B_{\ell-1}\ge B_{k+1}\ge x$.
Thus
\[
A_{q,c}(x)\le \pi(2^{c(k+2)^2-3})=2^{c(k+2)^2+o(k^2)}=x^{c+o(1)}.
\]
Combining gives the claim.
\end{proof}

\subsubsection*{5.4. Structure of repeated sums in $A_{q,c}$}
The key point is that digitwise addition is carry-free:
by Lemma~\ref{lem:digitwelldefined}, each digit satisfies $x_j(p)\in\{0\}\cup[q_j+1,2q_j-1]$.
So for any two primes $p,p'$,
\[
0\le x_j(p)+x_j(p')\le (2q_j-1)+(2q_j-1)=4q_j-2<b_j.
\]
Thus Lemma~\ref{lem:carryfree} applies.

\begin{proposition}[Congruence constraints from a repeated sum]
\label{prop:congruences}
Assume we have a nontrivial repeated sum in $A_{q,c}$, i.e.
\begin{equation}
\label{eq:repsum}
 a_{p_1}+a_{p_2}=a_{p_1'}+a_{p_2'},\qquad \{p_1,p_2\}\ne\{p_1',p_2'\}.
\end{equation}
Reorder the four terms so that $a_{p_1}>a_{p_1'}\ge a_{p_2'}>a_{p_2}$.
Then there exist integers $k_1\ge k_2\ge2$ such that
\begin{enumerate}
\item[(i)] $p_1,p_1'\in P_{k_1}$ and $p_2,p_2'\in P_{k_2}$;
\item[(ii)] letting $Q_1:=q_1\cdots q_{k_2}$, we have
\[p_1p_2\equiv p_1'p_2'\pmod{Q_1};\]
\item[(iii)] if $k_2<k_1$ and $Q_2:=q_{k_2+1}\cdots q_{k_1}$ then
\[p_1\equiv p_1'\pmod{Q_2};\]
\item[(iv)] the indices satisfy
\[
(1-c)k_1^2<k_2^2<\frac{c}{1-c}k_1^2
\]
for all sufficiently large $k_1$.
\end{enumerate}
\end{proposition}

\begin{proof}
By Lemma~\ref{lem:carryfree}, \eqref{eq:repsum} implies digitwise equality
\begin{equation}
\label{eq:digiteq}
 x_j(p_1)+x_j(p_2)=x_j(p_1')+x_j(p_2')\qquad\text{for all }j\ge1.
\end{equation}
Define $k_1$ as the largest $j$ with $x_j(p_1)+x_j(p_2)\ge q_j+1$.
Since a digit is either $0$ or at least $q_j+1$, the condition ``$\ge q_j+1$'' means that \emph{at least one} of $p_1,p_2$ has nonzero $j$th digit.
So $k_1$ is the maximum of the block indices of $p_1,p_2$.
Similarly define $k_2$ as the largest $j$ with $x_j(p_1)+x_j(p_2)\ge 2q_j+2$.
Because $2q_j+2$ is the minimum of a sum of two nonzero digits, this means \emph{both} primes have nonzero $j$th digit, so $k_2$ is the minimum of the two block indices.
Applying the same definitions to the right side of \eqref{eq:repsum} and using \eqref{eq:digiteq} yields (i).

For (ii), fix $j\le k_2$.
Then all four primes lie in blocks $\ge j$, so by definition $g_j^{x_j(p)}\equiv p\pmod{q_j}$ for each of $p=p_1,p_2,p_1',p_2'$.
Raising $g_j$ to both sides of \eqref{eq:digiteq} gives
\[
 g_j^{x_j(p_1)+x_j(p_2)}\equiv g_j^{x_j(p_1')+x_j(p_2')}\pmod{q_j}
\quad\Rightarrow\quad
p_1p_2\equiv p_1'p_2'\pmod{q_j}.
\]
This holds for every $j\le k_2$, hence modulo $Q_1=q_1\cdots q_{k_2}$.

For (iii), assume $k_2<k_1$ and take $k_2<j\le k_1$.
Then $p_2,p_2'$ lie in $P_{k_2}$, so their $j$th digits are $0$, while $p_1,p_1'\in P_{k_1}$ so $g_j^{x_j(p_1)}\equiv p_1$ and $g_j^{x_j(p_1')}\equiv p_1'\pmod{q_j}$.
Equation \eqref{eq:digiteq} becomes $x_j(p_1)=x_j(p_1')$, so $p_1\equiv p_1'\pmod{q_j}$.
Multiplying over $j=k_2+1,\dots,k_1$ gives $p_1\equiv p_1'\pmod{Q_2}$.

For (iv), note first that from \eqref{eq:qj-growth} we have
\begin{equation}
\label{eq:Qlower}
Q_1=q_1\cdots q_{k_2}>2^{k_2^2},\qquad Q_2=q_{k_2+1}\cdots q_{k_1}>2^{k_1^2-k_2^2}.
\end{equation}
Also, since $p\in P_k$ implies $p\le 2^{ck^2-3}$, we have
\[
|p_1p_2-p_1'p_2'|\le p_1p_2+p_1'p_2'\le 2\cdot 2^{c(k_1^2+k_2^2)-6}=2^{c(k_1^2+k_2^2)-5}.
\]
If the congruence in (ii) is nontrivial then $Q_1\le |p_1p_2-p_1'p_2'|$, so by \eqref{eq:Qlower}
\[
2^{k_2^2}<Q_1\le 2^{c(k_1^2+k_2^2)-5},
\]
which rearranges to $k_2^2<\frac{c}{1-c}k_1^2+O(1)$.

Similarly, if $k_2<k_1$, the congruence in (iii) is nontrivial (since $p_1\ne p_1'$ by our ordering), so $Q_2\le |p_1-p_1'|\le p_1\le 2^{ck_1^2-3}$.
Using \eqref{eq:Qlower} again gives
\[
2^{k_1^2-k_2^2}<Q_2\le 2^{ck_1^2-3},
\]
which rearranges to $k_2^2>(1-c)k_1^2-O(1)$.
For large $k_1$ the additive constants are negligible, giving (iv).
\end{proof}

\subsubsection*{5.5. An explicit Sidon sequence with exponent $\tfrac{3-\sqrt5}{2}$}
Proposition~\ref{prop:congruences}(iv) is already enough to get a purely explicit Sidon sequence beating $1/3$ without deletions.

\begin{theorem}[Cilleruelo's ``warm-up'' Sidon sequence]
\label{thm:warmup}
Let
\[
 c_0:=\frac{3-\sqrt5}{2}\approx0.381966.
\]
Then the set $A_{q,c_0}$ is a Sidon set and
$A_{q,c_0}(x)=x^{c_0+o(1)}$.
\end{theorem}

\begin{proof}
By Proposition~\ref{prop:growth} we have the growth statement.
If a repeated sum exists, Proposition~\ref{prop:congruences}(iv) would give
\[
(1-c_0)k_1^2<k_2^2<\frac{c_0}{1-c_0}k_1^2.
\]
But $c_0$ is exactly the smaller root of $(1-c)^2=c$, so $c_0=(1-c_0)^2$ and therefore
$\frac{c_0}{1-c_0}=1-c_0$.
Hence the strict inequalities demand an integer $k_2$ lying in an empty open interval, impossible for large $k_1$.
So no repeated sums occur and $A_{q,c_0}$ is Sidon.
\end{proof}

\subsubsection*{5.6. Deletion to reach exponent $c=\sqrt2-1$}
Now take
\[
 c:=\sqrt2-1\approx0.414213.
\]
The set $A_{q,c}$ may have repeated sums, but Proposition~\ref{prop:congruences} shows they are highly structured.
We now delete the ``largest'' element from every potential collision.

\begin{definition}[Bad primes and the deleted set]
\label{def:badprimes}
For $k\ge2$, call $p\in P_k$ \emph{bad} if there exist primes $p_2,p_1',p_2'$ such that
\begin{equation}
\label{eq:bad}
 a_p+a_{p_2}=a_{p_1'}+a_{p_2'},
\qquad a_p> a_{p_1'}\ge a_{p_2'}>a_{p_2}.
\end{equation}
Let $B_k\subseteq P_k$ be the set of bad primes in block $k$.
Define the surviving prime set
\[
\mathcal P^*:=\bigcup_{k\ge2} (P_k\setminus B_k),
\qquad
A_{q,c}^*:=\{a_p: p\in\mathcal P^*\}.
\]
\end{definition}

\begin{lemma}[Deletion destroys all repeated sums]
\label{lem:delete-sidon}
$A_{q,c}^*$ is a Sidon set.
\end{lemma}

\begin{proof}
Suppose $A_{q,c}^*$ contains a repeated sum.
Choose a representation
$a_{p_1}+a_{p_2}=a_{p_1'}+a_{p_2'}$ with
$a_{p_1}\ge a_{p_1'}\ge a_{p_2'}\ge a_{p_2}$ and $\{p_1,p_2\}\ne\{p_1',p_2'\}$.
If $a_{p_1}=a_{p_1'}$ then injectivity (Proposition~\ref{prop:injective}) forces $p_1=p_1'$, and then equality of sums forces $a_{p_2}=a_{p_2'}$, contradicting nontriviality.
Hence we may assume $a_{p_1}>a_{p_1'}\ge a_{p_2'}>a_{p_2}$.
But then $p_1$ is bad by Definition~\ref{def:badprimes}, so $p_1\in B_k$ for its block $k$, contradicting $p_1\in\mathcal P^*$.
\end{proof}

It remains to show that the deletion does not destroy the density exponent.

\subsubsection*{5.7. Bounding the number of bad primes}
Fix $k_1\ge2$.
Let $p_1\in B_{k_1}$ and choose a witness collision as in \eqref{eq:bad}.
By Proposition~\ref{prop:congruences} there is some $k_2<k_1$ with
\begin{equation}
\label{eq:k2range}
 k_2^2<\frac{c}{1-c}k_1^2,
\end{equation}
and writing $Q_1=q_1\cdots q_{k_2}$, $Q_2=q_{k_2+1}\cdots q_{k_1}$ we have
\begin{equation}
\label{eq:cong-Q1Q2}
 p_1p_2\equiv p_1'p_2'\pmod{Q_1},\qquad p_1\equiv p_1'\pmod{Q_2}.
\end{equation}
Thus there exist nonzero integers $s_1,s_2$ such that
\begin{equation}
\label{eq:s1s2}
 p_1p_2-p_1'p_2'=s_1Q_1,\qquad p_1'-p_1=s_2Q_2.
\end{equation}
From $p\in P_k\Rightarrow p\le 2^{ck^2-3}$ we get
\begin{equation}
\label{eq:s1bound}
|s_1|\le \frac{|p_1p_2-p_1'p_2'|}{Q_1}\le \frac{p_1p_2+p_1'p_2'}{Q_1}\le \frac{2\cdot 2^{c(k_1^2+k_2^2)-6}}{Q_1}=\frac{2^{c(k_1^2+k_2^2)-5}}{Q_1},
\end{equation}
and similarly
\begin{equation}
\label{eq:s2bound}
|s_2|\le \frac{|p_1'-p_1|}{Q_2}\le \frac{2^{ck_1^2-3}}{Q_2}.
\end{equation}
Now rewrite
\begin{equation}
\label{eq:divides}
 p_1(p_2-p_2')=(p_1p_2-p_1'p_2')+(p_1'-p_1)p_2'=s_1Q_1+s_2p_2'Q_2.
\end{equation}
So $p_1$ divides the integer
\begin{equation}
\label{eq:sdef}
 s:=s_1Q_1+s_2p_2'Q_2.
\end{equation}

\begin{definition}[The integer sets $S_{k_2,k_1}$]
For $k_2<k_1$ define $S_{k_2,k_1}$ to be the set of all nonzero integers of the form
\eqref{eq:sdef} with parameters satisfying the bounds \eqref{eq:s1bound}, \eqref{eq:s2bound} and with $p_2'\in P_{k_2}$.
\end{definition}
Then every $p_1\in B_{k_1}$ divides some nonzero $s\in S_{k_2,k_1}$ for some $k_2$ satisfying \eqref{eq:k2range}.

\begin{lemma}[No $s\ne 0$ has two large prime divisors from $P_{k_1}$]
\label{lem:oneprime}
For fixed $k_1$ and $k_2$ as above, any $s\in S_{k_2,k_1}$ is divisible by \emph{at most one} prime $p\in P_{k_1}$, for all sufficiently large $k_1$.
\end{lemma}

\begin{proof}
If $p,p'\in P_{k_1}$ are distinct then $pp'> (2^{c(k_1-1)^2-3})^2=2^{2c(k_1-1)^2-6}$.
On the other hand, using \eqref{eq:s1bound}--\eqref{eq:s2bound} and $p_2'\le 2^{ck_2^2-3}$,
\[
|s|\le |s_1|Q_1+|s_2|p_2'Q_2\le 2^{c(k_1^2+k_2^2)-5}+2^{ck_1^2-3}2^{ck_2^2-3}
\le 2\cdot 2^{c(k_1^2+k_2^2)-5}.
\]
If $k_2$ satisfies \eqref{eq:k2range}, then for $c<1/2$ we have $k_1^2+k_2^2 \le (1+\frac{c}{1-c})k_1^2=\frac{1}{1-c}k_1^2$, hence
$|s|\le 2\cdot 2^{\frac{c}{1-c}k_1^2-5}$.
Since $c<1/2$ implies $2c>\frac{c}{1-c}$, we have
$2c(k_1-1)^2-6 > \frac{c}{1-c}k_1^2-5$ for all large $k_1$.
Thus for large $k_1$,
$pp'>2^{2c(k_1-1)^2-6}>|s|$, so $s$ cannot be divisible by two distinct primes from $P_{k_1}$.
\end{proof}

Lemma~\ref{lem:oneprime} implies the combinatorial bound
\begin{equation}
\label{eq:Bk-sumSk}
 |B_{k_1}|\le \sum_{k_2<\sqrt{\frac{c}{1-c}}\,k_1} |S_{k_2,k_1}|.
\end{equation}
We now estimate $|S_{k_2,k_1}|$.

\begin{lemma}[Size of $S_{k_2,k_1}$]
\label{lem:Sk-size}
For fixed $k_1>k_2$ we have
\[
|S_{k_2,k_1}|\le 4\Bigl(\frac{2^{c(k_1^2+k_2^2)-5}}{Q_1}\Bigr)\Bigl(\frac{2^{ck_1^2-3}}{Q_2}\Bigr)|P_{k_2}|.
\]
\end{lemma}

\begin{proof}
By definition, $s$ is determined by the triple $(s_1,s_2,p_2')$.
The number of possible $s_1$ is at most $2\cdot \frac{2^{c(k_1^2+k_2^2)-5}}{Q_1}$ (choices for $\pm s_1$), and the number of possible $s_2$ is at most $2\cdot \frac{2^{ck_1^2-3}}{Q_2}$.
Finally $p_2'$ can be any prime in $P_{k_2}$.
Multiplying gives the bound.
\end{proof}

To convert this into a bound on $|B_{k_1}|$, we use the crude estimates
\begin{equation}
\label{eq:Q12-lower}
Q_1Q_2=q_1\cdots q_{k_1}>2^{k_1^2}
\end{equation}
(from \eqref{eq:qj-growth}) and the prime number theorem in the weak form
$|P_{k}|\le \pi(2^{ck^2})=2^{ck^2+o(k^2)}$.
For the boundary exponent $c=\sqrt2-1$ we keep one more term (a $k^{-2}$ factor) to compare against $|P_{k_1}|$.

\begin{proposition}[Bad primes are a strict minority at $c=\sqrt2-1$]
\label{prop:badminority}
Let $c=\sqrt2-1$.
Then
\[
|B_{k_1}|\le \Bigl(\tfrac12+o(1)\Bigr)|P_{k_1}|\qquad(k_1\to\infty).
\]
Consequently $|P_{k_1}\setminus B_{k_1}|\gg |P_{k_1}|$.
\end{proposition}

\begin{proof}
Combine \eqref{eq:Bk-sumSk} with Lemma~\ref{lem:Sk-size} and \eqref{eq:Q12-lower}:
\[
|B_{k_1}|
\le \sum_{k_2<\alpha k_1}
4\cdot \frac{2^{c(k_1^2+k_2^2)-5}}{Q_1}\cdot \frac{2^{ck_1^2-3}}{Q_2}\,|P_{k_2}|
\le 2^{-6}\,2^{(2c-1)k_1^2}\sum_{k_2<\alpha k_1} 2^{ck_2^2}\,|P_{k_2}|,
\]
where $\alpha:=\sqrt{\frac{c}{1-c}}$.
Using the standard bound $\pi(x)\le 2x/\log x$ for $x$ large, we have
$|P_{k_2}|\le \pi(2^{ck_2^2-3})\le \frac{2\cdot 2^{ck_2^2-3}}{ck_2^2\log 2}$ for large $k_2$.
Hence
\[
2^{ck_2^2}|P_{k_2}|
\le \frac{2^{-2}}{c\log 2}\,\frac{2^{2ck_2^2}}{k_2^2}.
\]
Therefore
\begin{equation}
\label{eq:B-bound-sum}
|B_{k_1}|\le \frac{2^{-8}}{c\log 2}\,2^{(2c-1)k_1^2}
\sum_{k_2<\alpha k_1}\frac{2^{2ck_2^2}}{k_2^2}.
\end{equation}
The sum is dominated by its largest term because $2^{2ck_2^2}$ grows super-exponentially in $k_2$:
for $k_2\le \alpha k_1$ and large $k_1$, the ratio between consecutive terms is at least
$2^{2c(2k_2-1)}/4\ge 2^{ck_1}$, so the sum is at most twice the last term.
Thus
\[
\sum_{k_2<\alpha k_1}\frac{2^{2ck_2^2}}{k_2^2}
\le 2\cdot \frac{2^{2c\alpha^2 k_1^2}}{\alpha^2 k_1^2}.
\]
Insert into \eqref{eq:B-bound-sum}:
\[
|B_{k_1}|\le \frac{2^{-7}}{c\log 2}\,\frac{1}{\alpha^2}\,\frac{2^{\bigl((2c-1)+2c\alpha^2\bigr)k_1^2}}{k_1^2}.
\]
Now $\alpha^2=\frac{c}{1-c}$, so $(2c-1)+2c\alpha^2=(2c-1)+\frac{2c^2}{1-c}=\frac{3c-1}{1-c}$.
For $c=\sqrt2-1$ one checks $\frac{3c-1}{1-c}=c$ (equivalently $c^2+2c-1=0$), and also $\alpha^2=\frac{1}{\sqrt2}$.
Therefore
\begin{equation}
\label{eq:B-asymp}
|B_{k_1}|\le C\,\frac{2^{ck_1^2}}{k_1^2}
\end{equation}
for an explicit constant $C>0$ (here $C=2^{-7}\sqrt2/(c\log 2)$ works).

On the other hand, by the prime number theorem and the fact that $2^{c(k_1-1)^2}$ is exponentially smaller than $2^{ck_1^2}$,
\begin{equation}
\label{eq:Pk-asymp}
|P_{k_1}|=\pi(2^{ck_1^2-3})-\pi(2^{c(k_1-1)^2-3})=\Bigl(\frac{2^{-3}}{c\log 2}+o(1)\Bigr)\frac{2^{ck_1^2}}{k_1^2}.
\end{equation}
Comparing \eqref{eq:B-asymp} and \eqref{eq:Pk-asymp}, the ratio $|B_{k_1}|/|P_{k_1}|$ is bounded above by a fixed constant strictly smaller than $1$ (in fact much smaller), for large $k_1$.
In particular $|B_{k_1}|\le (\tfrac12+o(1))|P_{k_1}|$.
\end{proof}

\subsubsection*{5.8. Final explicit lower bound exponent $\sqrt2-1$}
\begin{theorem}[Explicit infinite Sidon set with exponent $\sqrt2-1$]
\label{thm:main}
With $c=\sqrt2-1$ and $A_{q,c}^*$ as in Definition~\ref{def:badprimes}, we have:
\begin{enumerate}
\item $A_{q,c}^*$ is a Sidon set;
\item its counting function satisfies
\[
A_{q,c}^*(x)=x^{\sqrt2-1+o(1)}.
\]
\end{enumerate}
\end{theorem}

\begin{proof}
Sidonness is Lemma~\ref{lem:delete-sidon}.
For the counting function, Proposition~\ref{prop:growth} gives $|P_k|=2^{ck^2+o(k^2)}$ and hence
$A_{q,c}(x)=x^{c+o(1)}$.
Proposition~\ref{prop:badminority} shows that a positive proportion of each block $P_k$ survives the deletion.
Since the block sizes grow like $2^{ck^2}$ and are overwhelmingly dominated by the last contributing block, removing a fixed fraction from each block does not change the exponent $c$.
Formally, for $x$ with $B_k<x\le B_{k+1}$ we have
\[
A_{q,c}^*(x)\ge \sum_{\ell\le k} |P_\ell\setminus B_\ell|
\gg \sum_{\ell\le k}|P_\ell|
\asymp |P_k|
=2^{ck^2+o(k^2)}=x^{c+o(1)}.
\]
The matching upper bound $A_{q,c}^*(x)\le A_{q,c}(x)=x^{c+o(1)}$ is immediate.
\end{proof}

\subsubsection*{5.9. What this achieves toward Erd\H{o}s's conjecture}
The original conjecture asks for $A(x)\gg_\varepsilon x^{1/2-\varepsilon}$ for \emph{all} $\varepsilon>0$.
Theorem~\ref{thm:main} gives exponent $\sqrt2-1$.
Thus it implies the conjectured bound only for those $\varepsilon$ satisfying
\[
\frac12-\varepsilon \le \sqrt2-1\quad\Longleftrightarrow\quad \varepsilon\ge \frac32-\sqrt2\approx 0.085786.
\]
The remaining gap is to improve the exponent from $\sqrt2-1$ up to $1/2-o(1)$.

\subsection*{6) ADVERSARIAL VERIFICATION}
I tried to break the new argument in the following ways:
\begin{itemize}
\item \textbf{Carry issues:} digits satisfy $x_j(p)\le 2q_j-1$ and bases are $b_j=4q_j$, so $x_j(p)+x_j(p')<4q_j=b_j$. Hence Lemma~\ref{lem:carryfree} applies globally.
\item \textbf{Uniqueness of digit choice:} the interval $[q_j+1,2q_j-1]$ has length exactly $q_j-1$, matching the period of discrete logs in $\mathbb F_{q_j}^\times$.
\item \textbf{Injectivity:} requires $Q=q_1\cdots q_k$ to exceed possible differences $|p-p'|$ inside a block. This holds because $\log_2 Q\ge k^2$ whereas $\log_2|p-p'|\le ck^2+O(1)$ and $c<1$.
\item \textbf{Repeated-sum classification:} the key step is that the thresholds $q_j+1$ and $2q_j+2$ distinguish whether 0, 1, or 2 summands have nonzero $j$th digit. This uses the digit range $\{0\}\cup[q_j+1,2q_j-1]$.
\item \textbf{Deletion correctness:} any repeated sum has a largest element; removing all primes that can be the largest element of a collision eliminates all collisions.
\item \textbf{Density after deletion:} the borderline case $c=\sqrt2-1$ is exactly where the exponent of the bad-set bound matches the block exponent. I checked that Proposition~\ref{prop:badminority} keeps a constant fraction of each block (in fact, the crude constants already make the ratio far below $1/2$).
\end{itemize}
No hidden quantifier issue was found: all ``for large $k$'' statements can be made uniform because the inequalities have polynomial/exponential separations in $k$.

\subsection*{7) FINAL}
\textbf{UNRESOLVED (BUT STRICTLY ADVANCED).}

Round~1 established only an explicit lower bound exponent $1/3$.
Round~2 constructs an explicit infinite Sidon set $A$ with
\[
|A\cap[1,x]|=x^{\sqrt2-1+o(1)}.
\]
This is a strict and substantial improvement, but still falls short of the conjectured $x^{1/2-o(1)}$.

\subsection*{8) COMPLETION ESTIMATE (MANDATORY)}
\textbf{COMPLETION: 65\%}

\subsection*{9) REFERENCES}
\begin{thebibliography}{9}
\bibitem{Cilleruelo2013}
J.~Cilleruelo,
\emph{Infinite Sidon sequences},
\texttt{arXiv:1209.0326v2} (2013).

\bibitem{ErdosProblemsPrize}
Erd\H{o}s Problems Project,
\emph{Sidon set density conjecture} (Erd\H{o}s prize page / Sidon tags),
last edited December~28,~2025.

\bibitem{Ruzsa1998}
I.~Z.~Ruzsa,
\emph{An infinite Sidon sequence},
J.~Number Theory \textbf{68} (1998), 63--71.
\end{thebibliography}
