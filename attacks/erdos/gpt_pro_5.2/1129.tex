
\subsection*{FORMAL RESTATEMENT}
Let $n\ge 2$ and let $x_1,\dots,x_n\in[-1,1]$ be pairwise distinct.
For $1\le k\le n$ define the $k$th Lagrange fundamental polynomial
\[
 l_k(x):=\frac{\prod_{i\ne k}(x-x_i)}{\prod_{i\ne k}(x_k-x_i)}.
\]
Define the Lebesgue function and Lebesgue constant
\[
 L(x):=\sum_{k=1}^n |l_k(x)|,\qquad \Lambda(x_1,\dots,x_n):=\max_{x\in[-1,1]} L(x).
\]
Problem: characterize the choice(s) of nodes $x_1,\dots,x_n$ that minimize $\Lambda(x_1,\dots,x_n)$.

\subsection*{QUICK LITERATURE/CONTEXT CHECK}
I only record context explicitly stated in the problem file.
The file states lower bounds $\Lambda\gg\log n$ (Faber) and $\Lambda>\tfrac{2}{\pi}\log n-O(1)$ (Erd\H{o}s), and that choosing $x_i$ as roots of the $n$th Chebyshev polynomial gives $\Lambda<\tfrac{2}{\pi}\log n+O(1)$.
It also states the minimising canonical choice is known only for $n\le 4$.

\subsection*{ATTACK PLAN}
\textbf{Proof track ideas.}
\begin{itemize}
\item Use structural identities of the Lagrange basis (partition of unity) and variational conditions for optimality.
\item Explore symmetry reductions for canonical node sets with $x_1=-1$, $x_n=1$.
\end{itemize}
\textbf{Disproof track ideas.}
\begin{itemize}
\item For proposed characterizations (e.g. equal interval maxima), attempt to find a node set violating the characterization yet having comparable $\Lambda$.
\end{itemize}

\subsection*{WORK}
\textbf{Lemma 1129.1 (partition of unity).}
For any distinct nodes $x_1,\dots,x_n$ and the associated Lagrange polynomials $l_k$, one has
\[
\sum_{k=1}^n l_k(x)=1\qquad\text{for all }x\in\mathbb{R}.
\]

\emph{Proof.}
Define $p(x):=\sum_{k=1}^n l_k(x)-1$.
Each $l_k$ is a polynomial of degree at most $n-1$, hence so is $p$.
For each node $x_i$ we have $l_k(x_i)=\delta_{ki}$ by construction, so
\[
\sum_{k=1}^n l_k(x_i)=1\quad\Rightarrow\quad p(x_i)=0.
\]
Thus $p$ has at least $n$ distinct roots $x_1,\dots,x_n$.
But a nonzero polynomial of degree at most $n-1$ can have at most $n-1$ roots.
Therefore $p$ must be the zero polynomial, i.e. $\sum_k l_k(x)=1$ for all $x$.
\qed

\textbf{Lemma 1129.2 (universal lower bound $\Lambda\ge 1$).}
For every choice of distinct nodes $x_1,\dots,x_n$,
\[
\forall x\in[-1,1],\quad \sum_{k=1}^n |l_k(x)|\ge 1,
\]
and hence $\Lambda(x_1,\dots,x_n)\ge 1$.

\emph{Proof.}
By Lemma 1129.1, $\sum_k l_k(x)=1$ for all $x$.
By the triangle inequality,
\[
1=\left|\sum_{k=1}^n l_k(x)\right|\le \sum_{k=1}^n |l_k(x)|.
\]
Taking the maximum over $x\in[-1,1]$ yields $\Lambda\ge 1$.
\qed

\textbf{Lemma 1129.3 (exact computation for $n=2$ and endpoints).}
If $n=2$ and $(x_1,x_2)=(-1,1)$, then $\Lambda(x_1,x_2)=1$.

\emph{Proof.}
For $x_1=-1,x_2=1$ we have
\[
 l_1(x)=\frac{x-1}{-2}=\frac{1-x}{2},\qquad l_2(x)=\frac{x+1}{2}.
\]
On $[-1,1]$ both are nonnegative and satisfy $l_1(x)+l_2(x)=1$.
Therefore $|l_1(x)|+|l_2(x)|=1$ for all $x\in[-1,1]$, so the maximum is $1$.
\qed

\textbf{FAST REALITY CHECK (numerics for small $n$).}
Using barycentric evaluation on a fine grid, I computed approximate Lebesgue constants for some named node choices.
\begin{verbatim}
n=2 endpoints
  Lambda approx 1.0000000000000002 at x≈ -0.9998

n=3 -1,0,1
  Lambda approx 1.25 at x≈ -0.5

n=4 t=0.4177 (nodes -1,-t,t,1)
  Lambda approx 1.4230953773112645 at x≈ -0.7331

Chebyshev roots n=5:  Lambda≈1.988854 ; (2/pi)log n≈1.024600
Chebyshev roots n=10: Lambda≈2.428829 ; (2/pi)log n≈1.465871
Chebyshev roots n=20: Lambda≈2.869774 ; (2/pi)log n≈1.907142
\end{verbatim}
These numbers are consistent with the statements in the problem file (logarithmic growth and the $2/\pi$ coefficient).

\subsection*{VERIFICATION}
\begin{itemize}
\item Lemma 1129.1 is the standard ``degree vs number of roots'' argument and uses only that nodes are distinct.
\item Lemma 1129.2 uses only triangle inequality and Lemma 1129.1.
\item Lemma 1129.3 is an exact computation.
\item The numerical maxima were computed over a dense grid; they certify lower bounds on $\Lambda$ and are close to known exact values for $n=2,3$.
\end{itemize}

\subsection*{FINAL}
\textbf{UNRESOLVED}

(i) \textbf{Strongest proved partial result.}
The Lagrange basis satisfies $\sum_k l_k(x)=1$ for all $x$ (Lemma 1129.1), hence $\Lambda\ge 1$ for all node sets (Lemma 1129.2), with equality achieved for $n=2$ at $\{-1,1\}$ (Lemma 1129.3).

(ii) \textbf{First gap (crisp).}
Determine, for general $n$, the node configurations $x_1,\dots,x_n\in[-1,1]$ minimizing $\Lambda(x_1,\dots,x_n)$ (beyond the known small-$n$ cases).

(iii) \textbf{Top 3 next moves.}
\begin{itemize}
\item Derive first-order optimality conditions for minimizers of $\Lambda$ (subgradient/alternation type conditions) and test whether they force the ``equal interval maxima'' property mentioned in the problem file.
\item For canonical configurations with $x_1=-1,x_n=1$, reduce the minimization to a finite-dimensional constrained optimization and attempt to prove uniqueness and symmetry for all $n$.
\item Perform rigorous computer-assisted optimization for $n=5,6$ to guess patterns and candidate algebraic constants (as done for $n=4$ in the file).
\end{itemize}

(iv) \textbf{Minimal counterexample structure.}
A counterexample to a proposed characterization would be a node set with strictly smaller $\Lambda$ than the characterized family, likely breaking symmetry or the equal-maxima condition while still matching the $\tfrac{2}{\pi}\log n+O(1)$ asymptotic scale.


