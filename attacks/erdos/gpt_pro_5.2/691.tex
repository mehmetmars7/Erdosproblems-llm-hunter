
\noindent\textbf{1) FORMAL RESTATEMENT}

Let $A\subseteq\mathbb{N}$. Define the set of multiples of $A$ by
\[
M_A:=\{n\ge 1:\ \exists a\in A\text{ with }a\mid n\}.
\]
We say $A$ is a \emph{Behrend sequence} if $M_A$ has natural density $1$, i.e.
\[
\lim_{x\to\infty}\frac{1}{x}\#(M_A\cap[1,x])=1.
\]
The problem asks for a necessary and sufficient condition on $A$ for $M_A$ to have density $1$.

\bigskip
\noindent\textbf{2) QUICK LITERATURE/CONTEXT CHECK}

The problem statement records:

- If $A$ is a set of primes (or pairwise coprime integers not containing $1$), then $M_A$ has density $1$ iff $\sum_{a\in A}1/a=\infty$.

- For certain “block sequences” built from intervals $(n_k,(1+\eta_k)n_k)$ with lacunary $n_k$, the behavior depends on the parameters and additional growth conditions.

I do not use external references beyond what is explicitly stated; below I prove some general necessary/sufficient conditions but not a full characterization.

\bigskip
\noindent\textbf{3) ATTACK PLAN}

\emph{Proof track (characterize density $1$):} attempt to rewrite $\mathbb{N}\setminus M_A$ as a sifted set and express its density via a sieve/inclusion--exclusion condition on $A$.

\emph{Disproof/obstruction track:} identify simple necessary obstructions (e.g. common gcd) and broad sufficient conditions (e.g. divergent reciprocal sum on a coprime subfamily).

We give rigorous partial results and then state the remaining gap.

\bigskip
\noindent\textbf{4) WORK}

\noindent\textbf{Lemma 691.1 (gcd obstruction).}
Assume $1\notin A$ and let $g:=\gcd(A)$ (gcd of all elements of $A$). If $g>1$, then $M_A$ has density at most $1/g$ (in particular, strictly less than $1$).

\smallskip
\noindent\emph{Proof.}
If $g=\gcd(A)>1$, then every $a\in A$ is divisible by $g$. Hence any multiple of any $a\in A$ is divisible by $g$, so
\[
M_A\subseteq \{n\ge 1: g\mid n\}.
\]
The latter set has natural density $1/g$ (exactly one residue class mod $g$), so the density of $M_A$ (if it exists) is $\le 1/g$, and its upper density is $\le 1/g$ in any case. \hfill$\square$

\medskip
\noindent\textbf{Lemma 691.2 (reduction to a primitive core).}
Define the subset
\[
A^{\min}:=\{a\in A:\ \text{there is no }b\in A\text{ with }b\mid a\text{ and }b<a\}.
\]
Then $M_A=M_{A^{\min}}$.

\smallskip
\noindent\emph{Proof.}
Clearly $M_{A^{\min}}\subseteq M_A$ because $A^{\min}\subseteq A$.
For the reverse inclusion, fix $n\in M_A$. Then $n$ is divisible by some $a\in A$. Among all elements of $A$ dividing $n$, choose one of minimal value; call it $a_0$. Such a minimal element exists because the set of positive divisors of $n$ is finite. By construction, no smaller element of $A$ divides $a_0$, hence $a_0\in A^{\min}$. Since $a_0\mid n$, we have $n\in M_{A^{\min}}$. Therefore $M_A\subseteq M_{A^{\min}}$. \hfill$\square$

\medskip
\noindent\textbf{Lemma 691.3 (pairwise-coprime divergent subfamily is sufficient).}
Assume $A$ contains an infinite subset $B\subseteq A$ such that (i) the elements of $B$ are pairwise coprime, (ii) $1\notin B$, and (iii) $\sum_{b\in B} \frac{1}{b}=\infty$. Then $M_A$ has natural density $1$.

\smallskip
\noindent\emph{Proof.}
Since $M_B\subseteq M_A$, it suffices to show $M_B$ has density $1$. Let $B_N:=\{b_1,\dots,b_N\}$ be any finite subset of $B$. Because the $b_i$ are pairwise coprime, the complement
\[
C_N:=\{n\ge 1: b_i\nmid n\ \text{for all }1\le i\le N\}
\]
is periodic modulo $\prod_{i=1}^N b_i$, and by the Chinese remainder theorem its density exists and equals
\[
\mathbf{d}(C_N)=\prod_{i=1}^N \Big(1-\frac{1}{b_i}\Big).
\]
Now let
\[
C:=\mathbb{N}\setminus M_B=\{n\ge 1: b\nmid n\ \text{for all }b\in B\}.
\]
Then $C\subseteq C_N$ for each $N$, hence the upper density satisfies
\[
\overline{\mathbf{d}}(C)\le \mathbf{d}(C_N)=\prod_{i=1}^N \Big(1-\frac{1}{b_i}\Big)
\quad\text{for all }N.
\]
Using $\log(1-t)\le -t$ for $t\in(0,1)$, we get
\[
\log\Big(\prod_{i=1}^N (1-1/b_i)\Big)=\sum_{i=1}^N \log(1-1/b_i)\le -\sum_{i=1}^N \frac{1}{b_i}.
\]
Since $\sum_{i=1}^N 1/b_i\to\infty$, the right-hand side tends to $-\infty$, so the product tends to $0$. Therefore $\overline{\mathbf{d}}(C)=0$.

Finally, for any set $S\subseteq\mathbb{N}$, $\underline{\mathbf{d}}(S)=1-\overline{\mathbf{d}}(\mathbb{N}\setminus S)$. Applying this to $S=M_B$ gives
\[
\underline{\mathbf{d}}(M_B)=1-\overline{\mathbf{d}}(C)=1.
\]
Since $\overline{\mathbf{d}}(M_B)\le 1$, we conclude $M_B$ (hence $M_A$) has natural density $1$. \hfill$\square$

\medskip
\noindent\textbf{FAST REALITY CHECK (examples).}

- If $1\in A$, then $M_A=\mathbb{N}$ and the density is $1$.

- If $A=\{2^j:j\ge 1\}$, then $M_A$ equals the even integers (every multiple of a power of $2$ is even, and every even integer is a multiple of $2$), so the density is $1/2$, not $1$.

- If $A$ is the set of all primes, then the complement $\mathbb{N}\setminus M_A=\{1\}$ has density $0$, hence $M_A$ has density $1$.

\bigskip
\noindent\textbf{5) VERIFICATION}

- Lemma 691.1 is a strict obstruction: if $\gcd(A)>1$ (and $1\notin A$), then density $1$ is impossible.

- Lemma 691.3 is consistent with the special case quoted in the problem statement (pairwise coprime $A$): it recovers “divergent reciprocal sum implies density $1$” without needing any further structure.

- All density computations used only periodicity/CRT for finite sets.

\bigskip
\noindent\textbf{6) FINAL}

\textbf{UNRESOLVED}

(i) \emph{Strongest proved partial result.} If $\gcd(A)>1$ and $1\notin A$ then $\mathbf{d}(M_A)\le 1/\gcd(A)<1$ (Lemma 691.1). If $A$ contains a pairwise coprime subfamily $B$ with $\sum_{b\in B}1/b=\infty$, then $\mathbf{d}(M_A)=1$ (Lemma 691.3). Also $M_A$ depends only on the “minimal-by-divisibility” core $A^{\min}$ (Lemma 691.2).

(ii) \emph{First gap (crisp).} Give a necessary and sufficient condition on an \emph{arbitrary} $A\subseteq\mathbb{N}$ (with no coprimality assumptions) for $\mathbf{d}(M_A)=1$.

(iii) \emph{Top 3 next moves.}
1. Prove (or refute) that one may always reduce to a \emph{primitive} set (no element divides another) and then characterize Behrend-ness by a single divergent series/condition on that primitive set.
2. Develop a sieve/inclusion--exclusion criterion for $\mathbf{d}(\mathbb{N}\setminus M_A)=0$ in terms of finite truncations $A\cap[1,y]$, aiming for a checkable “$\lim_{y\to\infty}$” condition.
3. For structured families (e.g. block sequences), compute the density of the complement by an explicit upper bound on numbers avoiding all blocks, and identify sharp thresholds.

(iv) \emph{Minimal counterexample structure.} A minimal obstruction to any simple “$\sum 1/a$ diverges” criterion would likely be a set $A$ with heavy overlaps (many shared prime factors) so that $\sum_{a\in A}1/a$ diverges but $\mathbb{N}\setminus M_A$ still has positive density; the block-sequence constructions mentioned in the statement are candidate models for such behavior.


