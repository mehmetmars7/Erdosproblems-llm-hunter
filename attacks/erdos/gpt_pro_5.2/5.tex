
\noindent\textbf{1) FORMAL RESTATEMENT.}

Let $(p_n)_{n\ge 1}$ be the increasing sequence of prime numbers. Define the normalized prime-gap sequence for $n\ge 2$ by
\[
 g_n := \frac{p_{n+1}-p_n}{\log n},
\]
(where $\log$ is the natural logarithm). Let $S$ be the set of limit points (accumulation points) of the sequence $(g_n)_{n\ge 2}$ in the extended half-line $[0,\infty]$.

The conjecture asks whether $S=[0,\infty]$, equivalently: for every $C\in[0,\infty)$ there exists an infinite sequence $n_i\to\infty$ such that $g_{n_i}\to C$.

\medskip
\noindent\textbf{2) QUICK LITERATURE/CONTEXT CHECK.}

The problem text lists several deep results about $S$ (e.g. $0\in S$, $\infty\in S$, positive measure, large finite limit points, and explicit intervals contained in $S$). I do not reprove those results here; I only prove general topological facts about limit-point sets and do a small numerical sanity check.

\medskip
\noindent\textbf{3) ATTACK PLAN.}

\begin{itemize}
\item Establish basic properties of $S$ that follow directly from definitions (closedness; equivalences with $\liminf$ and $\limsup$).
\item Use computation for moderate $n$ to see the range of observed $g_n$ and identify sample small and large normalized gaps.
\item The real conjecture requires analytic number theory input beyond this worksheet.
\end{itemize}

\medskip
\noindent\textbf{4) WORK.}

\noindent\textbf{Lemma 1 (closedness of the limit-point set).}
The set $S$ of limit points of $(g_n)$ is a closed subset of $[0,\infty]$.

\noindent\emph{Proof.}
We show that the complement $[0,\infty]\setminus S$ is open. Take $x\notin S$. By definition of ``limit point'', this means there is \emph{no} subsequence of $(g_n)$ converging to $x$. Equivalently, there exists some open neighborhood $U$ of $x$ that contains only finitely many terms of the sequence; otherwise one could choose infinitely many indices $n$ with $g_n\in U$ and then extract a convergent subsequence in the compact set $\overline{U}$.

Concretely, choose $\varepsilon>0$ such that the interval $U=(x-\varepsilon,x+\varepsilon)$ (intersected with $[0,\infty]$ if needed) contains only finitely many $g_n$. Then no point $y\in U$ can be a limit point either, because if $y$ were a limit point there would be infinitely many $g_n$ arbitrarily close to $y$, hence infinitely many $g_n\in U$, contradicting the choice of $U$. Thus $U\subseteq[0,\infty]\setminus S$.

So every $x\notin S$ has an open neighborhood contained in the complement, proving the complement is open and hence $S$ is closed. \hfill$\square$

\medskip
\noindent\textbf{Lemma 2 (limit points vs $\liminf$/$\limsup$ for this sequence).}
Let $S$ be the limit-point set of $(g_n)$. Then:
\begin{itemize}
\item $0\in S$ if and only if $\liminf_{n\to\infty} g_n = 0$.
\item $\infty\in S$ if and only if $\limsup_{n\to\infty} g_n = \infty$.
\end{itemize}

\noindent\emph{Proof.}
Because $g_n\ge 0$ for all $n$, we always have $\liminf g_n\ge 0$.

If $0\in S$, then by definition there is a subsequence $g_{n_i}\to 0$. This forces $\liminf g_n\le 0$. Combining with $\liminf g_n\ge 0$ gives $\liminf g_n=0$.

Conversely, if $\liminf g_n=0$, then for each $m\ge 1$ there exist infinitely many $n$ with $g_n<1/m$; otherwise the tail of the sequence would be bounded below by $1/m$, contradicting $\liminf=0$. Choose $n_m$ increasing such that $g_{n_m}<1/m$. Then $g_{n_m}\to 0$, so $0$ is a limit point and hence belongs to $S$.

The statement for $\infty$ is analogous: $\infty\in S$ means there is a subsequence with $g_{n_i}\to\infty$, which is equivalent to $\limsup g_n=\infty$.
\hfill$\square$

\medskip
\noindent\textbf{Lemma 3 (``everywhere dense'' is equivalent to $S=[0,\infty]$).}
If $S$ is dense in $[0,\infty]$, then $S=[0,\infty]$.

\noindent\emph{Proof.}
By Lemma 1, $S$ is closed. In a topological space, a set is dense if and only if its closure equals the whole space. Since $S$ is closed, $\overline{S}=S$. If $S$ is dense in $[0,\infty]$, then $\overline{S}=[0,\infty]$, hence $S=[0,\infty]$. \hfill$\square$

\medskip
\noindent\textbf{FAST REALITY CHECK (numerical sample).}

I computed $g_n=(p_{n+1}-p_n)/\log n$ for $2\le n\le 100000$ by sieving primes up to $1{,}396{,}739$ (enough to include $p_{100001}$). The exact extremal values found in this range were:
\begin{itemize}
\item $\min_{2\le n\le 100000} g_n = 0.17372020721773887$ occurring at $n=99984$ with prime gap $p_{n+1}-p_n=2$.
\item $\max_{2\le n\le 100000} g_n = 10.811676604023988$ occurring at $n=31545$ with prime gap $p_{n+1}-p_n=112$.
\end{itemize}
Also, the last five values in the computed range were:
\[
\begin{array}{c|c|c}
 n & p_{n+1}-p_n & g_n \\\hline
 99996 & 6 & 0.5211551889986991\\
 99997 & 20 & 1.7371824543734329\\
 99998 & 16 & 1.3897447563484642\\
 99999 & 20 & 1.7371794365154385\\
 100000 & 12 & 1.0423067565678044
\end{array}
\]

\medskip
\noindent\textbf{5) VERIFICATION.}

\begin{itemize}
\item Lemma 1: the proof only uses the definition of limit point and basic topology of $\mathbb{R}$ (extended at $\infty$). In particular, it shows the complement is open by exhibiting an open neighborhood with only finitely many sequence terms.
\item Lemma 2: checked both directions using standard equivalences between $\liminf$ and existence of arbitrarily small values along a subsequence, similarly for $\limsup$.
\item Lemma 3: checked that it uses only ``closed + dense implies whole space''.
\item Computation: verified prime gaps and $\log n$ values were computed with natural logarithms; extrema are reported with their indices.
\end{itemize}

\medskip
\noindent\textbf{6) FINAL.}

\textbf{UNRESOLVED}

(i) \emph{Strongest proved partial result here.} The limit-point set $S$ of $g_n=(p_{n+1}-p_n)/\log n$ is closed (Lemma 1). Moreover $0\in S\iff \liminf g_n=0$ and $\infty\in S\iff \limsup g_n=\infty$ (Lemma 2). A numerical sanity check for $n\le 100000$ found values of $g_n$ ranging from about $0.17$ to about $10.81$.

(ii) \emph{First gap (crisp).} Show that every finite $C\ge 0$ is a limit point of $(g_n)$ (equivalently, that $S$ is dense, hence equal to $[0,\infty]$ by Lemma 3).

(iii) \emph{Top 3 next moves (concrete).}
\begin{enumerate}
\item Try to prove that $S$ contains a nontrivial interval $[0,c]$ for some explicit $c$ (or enlarge such an interval if available), then attempt to propagate to larger $C$ by combining constructions of small gaps and large gaps in controlled patterns.
\item Computational exploration: for increasing cutoffs $n\le N$, approximate the empirical coverage of intervals by observed values of $g_n$ and search for apparent gaps in the observed set of values.
\item Formulate and attempt intermediate statements of the form: ``for every $C$ and every $\varepsilon>0$, the set of $n$ with $|g_n-C|<\varepsilon$ has positive lower density''; even sparse infinitude would already give $C\in S$.
\end{enumerate}

(iv) \emph{Minimal counterexample structure.} A counterexample to $S=[0,\infty]$ would be a nonempty open interval $(a,b)\subseteq[0,\infty)$ with $(a,b)\cap S=\varnothing$; equivalently, there would exist $\varepsilon>0$ such that only finitely many $n$ satisfy $|g_n-c|<\varepsilon$ for some $c\in(a,b)$. By Lemma 1 this is the only possible way for $S$ to fail being all of $[0,\infty]$.

\bigskip

