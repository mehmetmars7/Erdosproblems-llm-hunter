% Erdos Problem #604
% Solutions/partial progress file (auto-generated)

1) FORMAL RESTATEMENT

Let $A\subset\mathbb R^2$ be a set of $n\ge 2$ distinct points.
For $x\in A$, define the \emph{pinned distance set}
\[
D(x):=\{\,\|x-y\|: y\in A,\ y\neq x\,\}
\]
and let $d(x):=|D(x)|$.
The question asks whether one must have
\[
\exists x\in A\ \text{with}\ d(x)\gg n^{1-o(1)}
\]
(and perhaps even $d(x)\gg n/\sqrt{\log n}$).
Here $f(n)\gg g(n)$ means there exists an absolute $c>0$ and $N$ such that for all $n\ge N$, $f(n)\ge c g(n)$; and $n^{1-o(1)}$ means: for every $\varepsilon>0$, $d(x)\ge c(\varepsilon)n^{1-\varepsilon}$ for all large $n$.

2) QUICK LITERATURE/CONTEXT CHECK

No web lookups performed. Using only what is stated in the problem file.
- The integer grid example suggests the bound $\gg n/\sqrt{\log n}$ would be best possible.
- The best bound recorded in the file is $\gg n^{c-o(1)}$ with $c\approx 0.864137$, attributed there to Katz--Tardos.
- Erd\H{o}s conjectured (as recorded) that $\sum_{x\in A} d(x)\gg n^2/\sqrt{\log n}$.

3) ATTACK PLAN

Proof-track ideas:
- Relate small $d(x)$ to many repeated distances from $x$, i.e. many isosceles triangles with apex $x$.
- Express the total number of isosceles triangles via incidences between points and perpendicular bisectors.
- Seek an upper bound on these incidences; any nontrivial upper bound on isosceles triangles would force a lower bound on some $d(x)$.

Disproof/construction ideas:
- Try structured point sets (integer grids, Cartesian products, points on circles) and compute $d(x)$ to see how small it can be.

4) WORK

(FAST REALITY CHECK via computation: integer grids)
I computed $d(x)$ for $m\times m$ integer grids $A=\{0,1,\dots,m-1\}^2$ (so $n=m^2$), using squared distances.
Exact outputs:
- $m=10$ ($n=100$): $\min_x d(x)=19$, $\max_x d(x)=50$, average $\approx 37.24$.
- $m=30$ ($n=900$): $\min_x d(x)=119$, $\max_x d(x)=381$, average $\approx 266.73$; the maximum value $381$ occurs at exactly $4$ grid points.
(These are consistent with the heuristic $\max_x d(x)$ being on the order of $n/\sqrt{\log n}$ for grids.)

Lemma 604.1 (small pinned-distance sets force many isosceles triangles).
Let $A\subset\mathbb R^2$ have $|A|=n\ge 2$ and fix $x\in A$.
For each distance value $r\in D(x)$, let
\[
t_r := |\{y\in A\setminus\{x\}: \|x-y\|=r\}|.
\]
Then $\sum_{r\in D(x)} t_r = n-1$.
If $d(x)=|D(x)|\le D$, then the number $T_x$ of \emph{unordered} isosceles triangles with apex $x$,
\[
T_x := |\{\{y,z\}\subseteq A\setminus\{x\}: y\neq z,\ \|x-y\|=\|x-z\|\}|,
\]
satisfies
\[
T_x \ge d(x)\binom{\frac{n-1}{d(x)}}{2} \ge \binom{\left\lceil\frac{n-1}{D}\right\rceil}{2}.
\]
Consequently, if $d(x)\le D$ for all $x\in A$, the total number $T:=\sum_{x\in A} T_x$ satisfies
\[
T \ge n\binom{\left\lceil\frac{n-1}{D}\right\rceil}{2}.
\]

Proof.
For fixed $x$, we have
\[
T_x = \sum_{r\in D(x)} \binom{t_r}{2}.
\]
The function $t\mapsto \binom{t}{2}=\frac{t(t-1)}{2}$ is convex for real $t\ge 0$.
By Jensen's inequality applied to the multiset $(t_r)_{r\in D(x)}$ (or by the discrete smoothing argument: moving one unit of mass from a smaller $t_r$ to a larger one increases the sum), the sum $\sum_r \binom{t_r}{2}$ is minimised, subject to $\sum_r t_r=n-1$, when the $t_r$ are as equal as possible.
In particular,
\[
\sum_{r\in D(x)} \binom{t_r}{2} \ge d(x)\binom{\frac{n-1}{d(x)}}{2}.
\]
If $d(x)\le D$ then $(n-1)/d(x)\ge (n-1)/D$, so at least one $t_r\ge \lceil (n-1)/D\rceil$ by pigeonhole, which yields
$T_x=\sum_r\binom{t_r}{2}\ge \binom{\lceil (n-1)/D\rceil}{2}$.
Summing over $x$ gives the claimed lower bound on $T$.
\qed

Lemma 604.2 (bisector incidence identity for isosceles triangles).
Let $A\subset\mathbb R^2$ be finite.
For two distinct points $y\neq z$ in $\mathbb R^2$, let $B(y,z)$ denote the perpendicular bisector line of the segment $yz$.
Let $T^{\mathrm{ord}}$ be the number of \emph{ordered} isosceles triangles
\[
T^{\mathrm{ord}} := |\{(x,y,z)\in A^3: y\neq z,\ \|x-y\|=\|x-z\|\}|.
\]
Then
\[
T^{\mathrm{ord}} = \sum_{(y,z)\in A^2,\ y\neq z} |A\cap B(y,z)|.
\]

Proof.
Fix an ordered pair $(y,z)$ with $y\neq z$.
A point $x\in\mathbb R^2$ satisfies $\|x-y\|=\|x-z\|$ if and only if $x$ lies on the perpendicular bisector $B(y,z)$ (this is the standard characterisation of bisectors in Euclidean geometry).
Therefore, for fixed $(y,z)$, the number of points $x\in A$ forming an ordered isosceles triangle $(x,y,z)$ is exactly $|A\cap B(y,z)|$.
Summing over all ordered pairs $(y,z)$ with $y\neq z$ yields the identity.
\qed

5) VERIFICATION

- Lemma 604.1: checked extremes. If $d(x)=n-1$, then $t_r=1$ for all $r$ and $T_x=0$; the bound gives $d\binom{(n-1)/d}{2}=(n-1)\binom{1}{2}=0$.
If $d(x)=1$, then all $n-1$ points lie on one circle around $x$, so $T_x=\binom{n-1}{2}$; the bound gives $1\cdot\binom{n-1}{2}$.
- Lemma 604.2: no degeneracy issues when $y\neq z$; the bisector line is well-defined.
- Computation sanity: in $2\times 2$ grid ($n=4$), each point has distances $1$ and $\sqrt2$, so $d(x)=2$ for all $x$, matching output.

6) FINAL

\textbf{UNRESOLVED}

(i) Strongest proved partial result:
A reduction: if every point has $d(x)\le D$, then the configuration must determine at least $n\binom{\lceil (n-1)/D\rceil}{2}$ isosceles triangles (Lemma 604.1), and the total number of ordered isosceles triangles equals a perpendicular-bisector incidence sum (Lemma 604.2).

(ii) First gap (crisp statement):
Prove a nontrivial upper bound on
$\sum_{(y,z)\in A^2,\ y\neq z} |A\cap B(y,z)|$
valid for all $n$-point sets $A\subset\mathbb R^2$ that is strong enough to force $\max_x d(x)\gg n^{1-o(1)}$ (or $\gg n/\sqrt{\log n}$).

(iii) Top 3 next moves:
1. For candidate extremal sets (integer grids, Cartesian products, points on circles), compute/estimate both $\max_x d(x)$ and the bisector-incidence sum to see what upper bounds are plausible.
2. Target a lemma of the form: most bisector lines contain $O(1)$ points of $A$, with only few exceptional lines; then sum and combine with Lemma 604.1.
3. Try to bound the number of pairs $(y,z)$ that share the same bisector line, and/or exploit that many pairs forcing the same bisector suggests strong additive structure.

(iv) Minimal counterexample structure (if the strongest conjecture $\gg n/\sqrt{\log n}$ is false):
A counterexample would be a sequence of sets $A_n$ with $|A_n|=n$ such that \emph{every} $x\in A_n$ has $d(x)=o(n/\sqrt{\log n})$.
By Lemma 604.1, such a set would generate \emph{superquadratically} many isosceles triangles, hence would have many points lying on many shared perpendicular bisectors (strong global structure, plausibly lattice-like).


