\section{Erd\H{o}s Problem \#200: Longest arithmetic progression of primes up to $N$}

\subsection*{1) Formal restatement (quantifiers, definitions, edge cases)}
For $N\ge 2$, let $L(N)$ be the maximum integer $k$ such that there exist integers $a\ge 2$ and $d\ge 1$ with
\[
a,\, a+d,\, a+2d,\,\dots,\, a+(k-1)d \ \text{all prime and}\ a+(k-1)d\le N.
\]
(Equivalently, $L(N)$ is the length of the longest arithmetic progression contained in the set of primes $\le N$.)

\paragraph{Problem statement.}
Is it true that
\[
L(N)=o(\log N)\qquad (N\to\infty)?
\]

\paragraph{Edge cases.}
$L(N)$ is nondecreasing in $N$ and $L(N)\ge 1$ always.
For $k\ge 3$, any $k$-term progression of odd primes has even common difference $d$.

\subsection*{2) Quick literature/context check}
\begin{itemize}[leftmargin=2em]
\item The primes contain arbitrarily long arithmetic progressions (Green--Tao theorem) \cite{GreenTao08}, so $L(N)\to\infty$, but existing proofs give no usable quantitative growth rate for $L(N)$.
\item A standard application of the prime number theorem (via the Chebyshev function $\vartheta(x)=\sum_{p\le x}\log p\sim x$) yields the upper bound
\[
L(N)\le (1+o(1))\log N.
\]
A complete proof is included below.
\item Heuristics based on treating primes as a random set of density $\approx 1/\log N$ (and refined Hardy--Littlewood type constants) predict that the expected number of $k$-term prime progressions up to $N$ is about
\[
\asymp \frac{N^2}{(k-1)(\log N)^k},
\]
suggesting that the maximal $k$ satisfies $k\approx 2\log N/\log\log N$ (hence $o(\log N)$). A statement of this heuristic count appears, for instance, in The Prime Pages discussion of prime arithmetic progressions \cite{PrimePagesAP}.
\end{itemize}

\subsection*{3) Attack plan}
\begin{enumerate}[leftmargin=2em]
\item Prove the known PNT-based upper bound $L(N)\le (1+o(1))\log N$ carefully, since it is explicitly mentioned in the prompt.
\item For the main open question $L(N)=o(\log N)$, outline why ``current technology'' (sieve bounds, distribution in residue classes, Green--Tao transference) does not presently deliver such a sharpening.
\item Provide a quick computational sanity check for moderate $N$.
\end{enumerate}

\subsection*{4) Work}
\paragraph{Theorem 4.1 (PNT upper bound).}
Let $p_0, p_1,\dots,p_{k-1}$ be a $k$-term arithmetic progression of primes with $p_{k-1}\le N$.
Then
\[
k\le (1+o(1))\log N\qquad (N\to\infty).
\]

\begin{proof}
Write the progression as $p_j=a+jd$ with $d\ge 1$.

\emph{Step 1: small primes up to $k/2$ divide $d$.}
Let $q$ be a prime with $q\le k/2$.
Suppose for contradiction that $q\nmid d$.
Then $d$ is invertible modulo $q$, so the congruence
\[
a+jd\equiv 0\pmod q
\]
has exactly one solution class $j\equiv j_0\pmod q$.
Among indices $j\in\\{0,1,\dots,k-1\\}$ there are at least $\lceil k/q\rceil\ge 2$ solutions (since $k\ge 2q$).
Hence at least two distinct terms $a+jd$ are divisible by $q$.
At most one of them can equal $q$; the other is a multiple of $q$ exceeding $q$, hence composite.
This contradicts that all terms are prime.
Therefore $q\mid d$ for every prime $q\le k/2$.

Let
\[
P(x):=\prod_{q\le x\atop q\text{ prime}} q
\]
denote the primorial.
We have shown $P(k/2)\mid d$, so $d\ge P(k/2)$.

\emph{Step 2: almost all primes up to $k$ divide $d$.}
For large $k$, $P(k/2)$ is enormous, in particular $P(k/2)>k$, so $d>k$.
Now let $q$ be a prime with $k/2<q\le k$.
If $q\nmid d$, then as above there is at least one index $j\in\\{0,1,\dots,k-1\\}$ such that $a+jd\equiv 0\pmod q$ (since $k\ge q$).
That term is a prime divisible by $q$, so it must equal $q$.
But $q\le k<d$, so the only term of the progression that could possibly equal $q$ is the \emph{first} term ($j=0$); all others satisfy $a+jd\ge a+d>d>k\ge q$.
Thus, for primes $q$ with $k/2<q\le k$, the only possible exception to ``$q\mid d$'' is $q=a$ (and this can happen only if $a\le k$).
In particular, the set of primes $q\le k$ not dividing $d$ has size at most $1$, so
\[
\log d\\ \ge\\ \sum_{q\le k\atop q\text{ prime}}\log q\\ -\\ \log k\\ =\\ \vartheta(k)-\log k,
\]
where $\vartheta(k):=\sum_{q\le k,\\ q\text{ prime}}\log q$ is Chebyshev's function.

\emph{Step 3: apply the prime number theorem.}
The prime number theorem implies $\vartheta(k)\sim k$ as $k\to\infty$.
Hence
\[
\log d\ge (1+o(1))k.
\]
Since $p_{k-1}=a+(k-1)d\le N$, we have $N\ge (k-1)d$, and therefore
\[
\log N\ge \log d+\log(k-1)\ge (1+o(1))k.
\]
Rearranging gives $k\le (1+o(1))\log N$.
\end{proof}

\paragraph{Computational check (moderate $N$).}
A brute-force search over all prime pairs $(p,q)$ with $p<q\le N$ (testing the progression with step $d=q-p$) gives:
\[
L(10^3)=7,\quad L(5\cdot 10^3)=10,\quad L(10^4)=10,\quad L(2\cdot 10^5)=10,
\]
with a longest example for these ranges being
\[
199,\\ 409,\\ 619,\dots,\\ 2089\qquad (\text{length }10,\\ d=210).
\]
This is only a sanity check; it is not evidence for asymptotic behaviour.

\subsection*{5) Verification / consistency checks}
\begin{itemize}[leftmargin=2em]
\item The key modular argument is checked in two regimes: $q\le k/2$ forces $\ge2$ multiples, hence $q\mid d$; $q\in(k/2,k]$ gives at most one multiple, forcing a near-total divisibility of $d$ when $d>k$.
\item The only analytic input is $\vartheta(x)\sim x$, a standard equivalent form of the prime number theorem.
\end{itemize}

\subsection*{6) Final}
\paragraph{\textbf{UNRESOLVED.}}
\begin{enumerate}[leftmargin=2em]
\item \textbf{Strongest proved partial result included here.}
Theorem~4.1 proves the known upper bound $L(N)\le (1+o(1))\log N$ from the prime number theorem.
Also $L(N)\to\infty$ by Green--Tao \cite{GreenTao08}.
\item \textbf{First precise obstacle.}
To improve the bound to $o(\log N)$ one would need a quantitative upper bound excluding $k$-term prime progressions for $k$ as large as $c\log N$, uniformly in the common difference.
Current sieve/analytic tools (and the transference methods in Green--Tao) do not supply such a uniform ``union bound'' at this scale.
\item \textbf{Most plausible next lemma.}
A plausible route would be a sharp upper bound for the number of $k$-term prime progressions up to $N$ of the form
\[
\#\\{(a,d):\\ a,a+d,\dots,a+(k-1)d\le N\text{ all prime}\\}\\ \le\\ N^2\exp(-c k\log\log N)
\]
for $k$ up to (say) $\varepsilon\log N$, which would imply (by the first moment method) that $L(N)=o(\log N)$.
Proving such bounds seems to require substantially stronger uniformity estimates for primes in progressions than are currently available.
\end{enumerate}

\subsection*{7) Completion estimate}
\[
\textbf{Completion: }40\%.
\]
(We can prove the standard PNT upper bound cleanly and explain the heuristic, but the main asymptotic sharpening remains open.)

% ======================================================================

