% Erdos Problem #811

\noindent\textbf{1) FORMAL RESTATEMENT.}

Let $m\ge 1$ and $n\ge 2$.
An edge-coloring of $K_n$ using exactly $m$ colors is called \emph{balanced} if every vertex is incident to exactly $\lfloor n/m\rfloor$ edges of each color.

\textbf{Ambiguity / existence condition.} If $n\equiv 1\pmod m$ then $\lfloor n/m\rfloor=(n-1)/m=:q$ is an integer and the vertex-degree condition becomes ``each vertex has degree $q$ in each color class''. However this forces each color class to have exactly $nq/2$ edges, so a necessary condition for existence is that $nq$ is even (equivalently $m\mid \binom{n}{2}$). If $nq$ is odd, no balanced coloring exists and statements about ``every balanced coloring'' are vacuous.

\smallskip
Now fix a finite simple graph $G$ and let $m:=e(G)$.
Question: For which graphs $G$ is it true that for all sufficiently large $n$ with $m\mid (n-1)$ and for which balanced $m$-colorings of $K_n$ exist, \emph{every} balanced $m$-edge-coloring of $K_n$ contains a rainbow copy of $G$ (a copy of $G$ whose $m$ edges all have distinct colors)?

\medskip
\noindent\textbf{2) QUICK LITERATURE/CONTEXT CHECK.}

The problem text reports that Erd\H{o}s--Tuza studied this and obtained bounds for a quantitative variant, and that recent work shows infinitely many graphs fail the property; in particular it says $K_4$ lacks the property. I do not use results beyond what is written in the problem text.

\medskip
\noindent\textbf{3) ATTACK PLAN.}

\emph{Proof strategies for positive cases.}
\begin{itemize}
\item Translate a rainbow-copy search into a system of distinct representatives (SDR) problem using the regularity of each color class; apply Hall-type arguments for graphs $G$ with bounded structure (stars, matchings, trees).
\item Use greedy embedding: order the edges of $G$ and embed them one-by-one, using that each vertex has many available edges in each color.
\end{itemize}

\emph{Disproof strategies / negative cases.}
\begin{itemize}
\item Construct balanced colorings from structured decompositions of $K_n$ into regular spanning subgraphs that avoid certain rainbow patterns (e.g. avoiding rainbow cliques).
\item Look for parity/degree obstructions and exploit high symmetry (e.g. group-based colorings) to forbid rainbow embeddings.
\end{itemize}

I do not classify all $G$; the work below proves two basic structural lemmas and one unconditional positive result for stars when $n$ is large compared to $m$.

\medskip
\noindent\textbf{4) WORK.}

\textbf{Lemma 811.1 (structure of balanced colorings).}
Assume $m\mid (n-1)$ and set $q:=(n-1)/m$. If $\chi$ is a balanced edge-coloring of $K_n$ with $m$ colors, then for each color $c$ the color class $H_c$ (the spanning subgraph of edges colored $c$) is $q$-regular and has exactly $\frac{nq}{2}$ edges. In particular, a necessary condition for existence of a balanced coloring is that $nq$ is even.

\emph{Proof.}
By balance, every vertex of $K_n$ is incident to exactly $q$ edges of color $c$, so $H_c$ is $q$-regular on the full vertex set.
Summing degrees in $H_c$ gives $\sum_{v} d_{H_c}(v)=nq$, which equals $2|E(H_c)|$ by the handshake lemma, hence $|E(H_c)|=nq/2$. Therefore $nq$ must be even for $|E(H_c)|$ to be an integer.
\qed

\textbf{Lemma 811.2 (rainbow stars for large $n$).}
Let $G$ be the star $K_{1,m}$, so $e(G)=m$.
Assume $m\mid (n-1)$ and let $q=(n-1)/m$.
If $q\ge m$ and $\chi$ is any balanced edge-coloring of $K_n$ with $m$ colors, then $K_n$ contains a rainbow copy of $K_{1,m}$.

\emph{Proof.}
Fix any vertex $v$ of $K_n$ as a candidate center.
For each color $c\in[m]$, let $N_c(v)$ be the set of neighbors $u$ such that edge $vu$ has color $c$.
By balance, $|N_c(v)|=q$ for every color $c$.

We seek distinct neighbors $u_c\in N_c(v)$ for all $c\in[m]$; then edges $vu_c$ form a rainbow $K_{1,m}$.
This is a system of distinct representatives (SDR) problem for the family of sets $\{N_c(v)\}_{c\in[m]}$.
By Hall's marriage theorem it suffices to check: for every subset $S\subseteq [m]$,
\[
\left|\bigcup_{c\in S} N_c(v)\right|\ge |S|.
\]
But for any nonempty $S$, the union size is at least the maximum of the individual sizes, so
\[
\left|\bigcup_{c\in S} N_c(v)\right|\ge \max_{c\in S}|N_c(v)|=q.
\]
If $q\ge m$ then for all $S\subseteq [m]$ we have $q\ge m\ge |S|$, hence Hall's condition holds.
Therefore an SDR exists, giving distinct neighbors $u_c$ and thus a rainbow $K_{1,m}$ centered at $v$.
\qed

\textbf{FAST REALITY CHECK (existence/parity).}
For $m=6$ and $n=7$ (the smallest $n\equiv 1\pmod 6$), we have $q=(n-1)/m=1$ and thus $nq=7$ is odd.
By Lemma 811.1, no balanced $6$-coloring of $K_7$ exists.
More generally, when $m$ is even and $q$ is odd, balanced colorings cannot exist because each color class would have $nq/2$ edges with $n$ odd.

\medskip
\noindent\textbf{5) VERIFICATION.}

\begin{itemize}
\item Lemma 811.1: the only nontrivial check is integrality; degree sum $nq$ must be even.
\item Lemma 811.2: verified Hall's condition carefully. The worst-case union size is $q$ (when all $N_c(v)$ coincide), and the assumed condition $q\ge m$ exactly forces $q\ge |S|$ for every $S\subseteq[m]$.
\item The star embedding indeed gives a rainbow copy because each edge uses a different color by construction.
\end{itemize}

\medskip
\noindent\textbf{6) FINAL.} \textbf{UNRESOLVED.}

\begin{enumerate}
\item[(i)] \emph{Strongest proved partial result here:} balanced $m$-colorings (when they exist) decompose $K_n$ into $m$ spanning $q$-regular graphs (Lemma 811.1). For the special case $G=K_{1,m}$, if $n\ge m^2+1$ (so $q=(n-1)/m\ge m$), then every balanced coloring contains a rainbow copy of $G$ (Lemma 811.2).
\item[(ii)] \emph{First gap (crisp):} determine for which non-star graphs $G$ (in particular cliques such as $K_4$ or cycles such as $C_6$) every balanced $e(G)$-coloring of $K_n$ (for large admissible $n$) must contain a rainbow copy of $G$.
\item[(iii)] \emph{Top 3 next moves:}
  \begin{enumerate}
  \item Extend the SDR/Hall approach from stars to matchings and small trees: prove that for fixed $G$ a rainbow embedding exists whenever $q$ exceeds an explicit function of $|V(G)|$ and $\Delta(G)$.
  \item Attempt a general ``rainbow blow-up lemma'' style greedy embedding: order edges of $G$ and keep track of forbidden colors/vertices, using the $q$-regularity of each color class.
  \item For specific targets ($C_6$, $K_4$), search for explicit balanced colorings on the smallest admissible $n$ using SAT/ILP to see the shape of obstructions.
  \end{enumerate}
\item[(iv)] \emph{What a minimal counterexample would likely look like:} a balanced coloring in which the $m$ regular color classes are highly structured (e.g. arising from a group action or a design) so that any attempt to choose $m$ edges with distinct colors forces unwanted vertex collisions, preventing a rainbow copy of the target $G$.
\end{enumerate}

