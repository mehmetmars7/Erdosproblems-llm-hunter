% Erdős Problem #89
% URL: https://www.erdosproblems.com/89

\noindent\textbf{FORMAL RESTATEMENT.}
Let $A\subset\mathbb{R}^2$ be a set of $n\ge 2$ distinct points. Let
\[
D(A):=\bigl|\{\,|x-y|:\ x,y\in A,\ x\ne y\,\}\bigr|
\]
be the number of distinct Euclidean distances determined by unordered pairs from $A$.
Question: does there exist an absolute constant $c>0$ such that for all sufficiently large $n$ and all such $A$,
\[
D(A)\ge c\,\frac{n}{\sqrt{\log n}}\ ?
\]

\medskip
\noindent\textbf{QUICK LITERATURE/CONTEXT CHECK.}
The problem statement notes that a $\sqrt n\times \sqrt n$ integer grid would make the bound $\gg n/\sqrt{\log n}$ best possible, and that Guth--Katz proved the lower bound $\gg n/\log n$. I do not use external results beyond elementary geometry/graph counting.

\medskip
\noindent\textbf{ATTACK PLAN.}
\begin{itemize}
\item \emph{Proof track:} prove a weaker lower bound for $D(A)$ by bounding how many pairs can realise the \emph{same} distance, and then pigeonholing over distances.
\item \emph{Disproof track:} look for constructions with $o(n/\sqrt{\log n})$ distinct distances; the known grid heuristic suggests this is unlikely.
\end{itemize}
We carry out the weaker proof track to obtain $D(A)=\Omega(\sqrt n)$.

\medskip
\noindent\textbf{WORK.}

\smallskip
\noindent\textbf{Lemma 89.1 (few pairs at any fixed distance).}
Fix $r>0$. Let $A\subset\mathbb{R}^2$ with $|A|=n$.
Let $U_r(A)$ be the number of unordered pairs $\{x,y\}\subset A$ with $|x-y|=r$.
Then
\[
U_r(A)\le \frac{\sqrt 2}{2}\,n^{3/2}+\frac14 n.
\]
\textit{Proof.}
Consider the bipartite graph $G=(L\sqcup R,E)$ where $L$ is a copy of $A$ (viewed as \emph{circle centers}) and $R$ is another copy of $A$ (viewed as \emph{points}). Put an edge between $\ell\in L$ (center $x\in A$) and $r\in R$ (point $y\in A$) iff $|x-y|=r$. Then $|L|=|R|=n$ and $|E|$ counts \emph{directed} distance-$r$ pairs, so
\[
|E|=2U_r(A).
\]

\emph{Claim 1:} $G$ is $K_{2,3}$-free.
Indeed, take two distinct centers $x_1\ne x_2$ (vertices in $L$). The set of points $y\in A$ that are adjacent to both $x_1$ and $x_2$ are exactly the intersection points of the two circles of radius $r$ centered at $x_1$ and $x_2$. Two distinct circles in the plane intersect in at most two points: subtracting their equations yields a line, which meets a circle in at most two points. Hence $x_1,x_2$ have at most two common neighbors in $R$, so there cannot be three points adjacent to both, i.e. no $K_{2,3}$.

\emph{Claim 2 (a concrete Kővári--Sós--Turán bound for $K_{2,3}$-free graphs):}
If a bipartite graph has parts of size $n$ and is $K_{2,3}$-free, then
\[
|E|\le \sqrt 2\,n^{3/2}+\frac12 n.
\]
\emph{Proof of Claim 2.}
Let the degrees of vertices on the right side be $d_1,\dots,d_n$ so that $\sum_{i=1}^n d_i=|E|$.
For a right vertex with degree $d_i$, the number of unordered pairs of its left neighbors is $\binom{d_i}{2}$.
Summing over $R$ counts, for each unordered left pair $\{u,v\}$, the number of common right neighbors.
Since the graph is $K_{2,3}$-free, any left pair has at most $2$ common right neighbors, so
\[
\sum_{i=1}^n \binom{d_i}{2} \le 2\binom{n}{2}=n(n-1).
\]
Rewrite the left-hand side as
\[
\sum_{i=1}^n \binom{d_i}{2}=
\frac12\sum_{i=1}^n(d_i^2-d_i)
=
\frac12\Bigl(\sum_{i=1}^n d_i^2\Bigr)-\frac12|E|.
\]
By Cauchy--Schwarz,
\[
\sum_{i=1}^n d_i^2 \ge \frac1n\Bigl(\sum_{i=1}^n d_i\Bigr)^2 = \frac{|E|^2}{n}.
\]
Therefore
\[
\frac12\Bigl(\frac{|E|^2}{n}\Bigr)-\frac12|E|
\le n(n-1)
\le n^2,
\]
which implies $|E|^2 - n|E| -2n^3\le 0$.
Solving this quadratic inequality in $|E|$ gives
\[
|E|\le \frac{n+\sqrt{n^2+8n^3}}{2}\le \frac{n+\sqrt{8}\,n^{3/2}}{2}=\sqrt 2\,n^{3/2}+\frac12 n.
\]
This proves Claim 2.

Combining Claim 1 and Claim 2 yields $|E|\le \sqrt 2\,n^{3/2}+\tfrac12 n$, hence
\[
U_r(A)=\frac{|E|}{2}\le \frac{\sqrt 2}{2}\,n^{3/2}+\frac14 n.
\]
\hfill $\square$

\smallskip
\noindent\textbf{Lemma 89.2 (a weak distinct-distance lower bound).}
For every finite $A\subset\mathbb{R}^2$ with $|A|=n\ge 2$,
\[
D(A)\ge \frac{n-1}{\sqrt 2\,\sqrt n+\tfrac12}\,\sqrt n \ge \frac{1}{2\sqrt 2}\,\sqrt n\quad\text{for }n\ge 4.
\]
In particular $D(A)=\Omega(\sqrt n)$.
\textit{Proof.}
Let the distinct distance values determined by $A$ be $r_1,\dots,r_m$ where $m=D(A)$.
Then every unordered pair of distinct points contributes to exactly one $U_{r_i}(A)$, so
\[
\binom{n}{2} = \sum_{i=1}^m U_{r_i}(A).
\]
By Lemma 89.1, each $U_{r_i}(A)\le \tfrac{\sqrt2}{2}n^{3/2}+\tfrac14 n$. Hence
\[
\binom{n}{2}\le m\Bigl(\frac{\sqrt2}{2}n^{3/2}+\frac14 n\Bigr),
\]
so
\[
D(A)=m\ge \frac{n(n-1)/2}{(\sqrt2/2)n^{3/2}+n/4}
=\frac{n-1}{\sqrt2\,\sqrt n+1/2}\,\sqrt n.
\]
For $n\ge 4$, the denominator satisfies $\sqrt2\,\sqrt n+1/2\le 2\sqrt2\,\sqrt n$, giving $m\ge \tfrac1{2\sqrt2}\sqrt n$.
\hfill $\square$

\smallskip
\noindent\textbf{FAST REALITY CHECK (integer grids).}
For an $N\times N$ integer grid ($n=N^2$ points), the distinct squared distances are exactly the distinct values of $a^2+b^2$ with $0\le a,b\le N-1$ not both zero.
A quick computation gave:
\begin{verbatim}
N=2  n=4     D=2
N=3  n=9     D=5
N=4  n=16    D=9
N=5  n=25    D=14
N=10 n=100   D=50
N=20 n=400   D=179
N=50 n=2500  D=992
N=100 n=10000 D=3663
\end{verbatim}
These values are consistent with the heuristic $D\approx n/\sqrt{\log n}$ for the grid, but no asymptotic estimate is proved here.

\medskip
\noindent\textbf{VERIFICATION.}
\begin{itemize}
\item \emph{Circle intersection:} two radius-$r$ circles are either disjoint, tangent, or meet in exactly two points; never three.
\item \emph{$K_{2,3}$-free implication:} a $K_{2,3}$ in the incidence graph would force two circles to share three points, contradicting the previous fact.
\item \emph{Algebra in Claim 2:} checked the quadratic inequality and the step $n(n-1)\le n^2$.
\item \emph{Distinct-distance pigeonhole:} the decomposition $\binom{n}{2}=\sum U_{r_i}$ holds because each pair has exactly one distance.
\end{itemize}

\medskip
\noindent\textbf{FINAL.} \textbf{UNRESOLVED}
\begin{enumerate}
\item[(i)] \textbf{Strongest proved partial result.}
For every $A\subset\mathbb{R}^2$ with $|A|=n$, one has $D(A)=\Omega(\sqrt n)$ (Lemma 89.2), via the concrete bound $U_r(A)=O(n^{3/2})$ for pairs at any fixed distance $r$ (Lemma 89.1).
\item[(ii)] \textbf{First gap (crisp).}
Improve the exponent from $1/2$ to $1$ up to logarithmic factors: prove that $D(A)\ge c\,n/\sqrt{\log n}$ for all sufficiently large $n$.
\item[(iii)] \textbf{Top 3 next moves.}
\begin{itemize}
\item Strengthen Lemma 89.1 to an upper bound $U_r(A)=O(n^{4/3})$ (or better) using incidence geometry beyond the $K_{2,3}$ extremal graph argument.
\item Prove a \emph{multi-distance energy inequality}: bound $\sum_r U_r(A)^2$ (or related additive energy) to control distance multiplicities more sharply.
\item Compute/experiment with candidate near-extremal sets (integer grids, ``perturbed'' grids, circular arcs) to identify structural features that force many repeated distances.
\end{itemize}
\item[(iv)] \textbf{Likely structure of a minimal counterexample.}
A hypothetical set $A$ with unusually small $D(A)$ must have many repeated distances, i.e. large $U_r(A)$ for many radii $r$. This suggests a highly structured configuration with many integer-like constraints (e.g. many points in approximate lattice position), rather than a generic or random configuration where almost all distances are distinct.
\end{enumerate}


