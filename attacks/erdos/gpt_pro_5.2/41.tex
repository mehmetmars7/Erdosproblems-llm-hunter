
\subsection*{Erd\H{o}s Problem \#41}

\subsection*{FORMAL RESTATEMENT}
Let $A\subseteq\mathbb N$ be infinite.  Assume that for all $a,b,c,a',b',c'\in A$,
\[
 a+b+c=a'+b'+c'\ \Longrightarrow\ \{a,b,c\}=\{a',b',c'\}\ \text{(as multisets)}.
\]
Equivalently, all triple sums $a_i+a_j+a_k$ with indices $i\le j\le k$ are distinct.  (This is the standard $B_3$ condition.)

Is it true that
\[
\liminf_{N\to\infty}\frac{|A\cap[1,N]|}{N^{1/3}}=0?
\]

\subsection*{QUICK LITERATURE/CONTEXT CHECK}
The statement quotes results for even $h$ and mentions the known $h=2$ analogue for Sidon sets. I do not use any results beyond what is explicitly stated; the work below is elementary.

\subsection*{ATTACK PLAN}
\textbf{Proof track:} Try to adapt the $h=2$ liminf argument (for Sidon sets) to $h=3$ using a density increment or covering argument on sumsets.

\textbf{Disproof track:} Attempt to construct an explicit infinite $B_3$ sequence with $|A\cap[1,N]|\gg N^{1/3}$ for all large $N$ (i.e. with a positive liminf), which would refute the claim.

Here I give two structural lemmas: a universal $O(N^{1/3})$ upper bound for finite $B_3$ sets in $[1,N]$, and a lemma showing an infinite $B_3$ set must also be Sidon ($B_2$).

\subsection*{WORK}
\paragraph{Fast reality check (local computation: greedy $B_3$ sequence).}
I implemented the greedy $B_3$ construction (start with $1$ and repeatedly add the smallest integer preserving distinct triple sums).  The first 20 terms are
\[
1,2,5,14,33,72,125,219,376,573,745,1209,1557,2442,3098,4048,5298,6704,7839,10987.
\]
For the first 27 terms (max $=37769$), sample counts were $|A\cap[1,10^4]|=19$ and $|A\cap[1,5\cdot 10^4]|=27$.

\paragraph{Lemma 41.1 (finite $B_3$ size bound by counting sums).}
Let $A\subseteq[1,N]$ be a $B_3$ set and write $|A|=m$. Then
\[
\binom{m+2}{3}\ \le\ 3N-2,
\qquad\text{hence}\qquad m\ \le\ (18N)^{1/3}+2.
\]

\emph{Proof.}
Because $A$ is $B_3$, all sums $a_i+a_j+a_k$ with $i\le j\le k$ are distinct. The number of such (multiset) triples equals the number of weakly increasing triples of indices from $\{1,\dots,m\}$, which is
\[
\#\{(i,j,k):1\le i\le j\le k\le m\}=\binom{m+2}{3}.
\]
Each triple sum lies in the interval $[3,3N]$, which contains $3N-2$ integers. Therefore
\[
\binom{m+2}{3}\le 3N-2.
\]
Using $\binom{m+2}{3}=\frac{m(m+1)(m+2)}{6}\ge \frac{m^3}{6}$ gives $m^3\le 18N$, hence $m\le (18N)^{1/3}$. Adding $2$ absorbs the crude inequalities for small $m$. \qed

\paragraph{Lemma 41.2 (infinite $B_3$ implies Sidon).}
Let $A\subseteq\mathbb N$ be infinite and satisfy the $B_3$ property. Then $A$ is Sidon ($B_2$): if $a+b=c+d$ with $a,b,c,d\in A$, then $\{a,b\}=\{c,d\}$.

\emph{Proof.}
Assume for contradiction that there exist $a,b,c,d\in A$ with $a+b=c+d$ but $\{a,b\}\ne\{c,d\}$.

Because $A$ is infinite, we can choose $e\in A$ distinct from $a,b,c,d$.
Then
\[
 e+a+b\ =\ e+c+d.
\]
The two triples $(e,a,b)$ and $(e,c,d)$ are not the same multiset of three elements, because $e$ appears in both and $\{a,b\}\ne\{c,d\}$. This contradicts the $B_3$ property.  Therefore no such nontrivial equality of pair sums can occur, and $A$ is Sidon. \qed

\subsection*{VERIFICATION}
\begin{itemize}
\item Lemma 41.1: verified that the counting of weakly increasing triples is $\binom{m+2}{3}$ and that the sum range $[3,3N]$ contains $3N-2$ integers.
\item Lemma 41.2: the only subtlety is ensuring the extra element $e$ exists and is distinct from $a,b,c,d$; this uses infinitude of $A$.
\item Computation: the greedy $B_3$ script explicitly rechecked the $B_3$ property for the computed initial segment.
\end{itemize}

\subsection*{FINAL}
\textbf{UNRESOLVED}

(i) \emph{Strongest proved partial result.}
Any finite $B_3$ set in $[1,N]$ has size at most $(18N)^{1/3}+2$ (Lemma 41.1).  Moreover, any infinite $B_3$ set is automatically Sidon (Lemma 41.2).

(ii) \emph{First gap (crisp statement).}
I cannot prove or disprove the target liminf statement:
\[
\liminf_{N\to\infty}\frac{|A\cap[1,N]|}{N^{1/3}}=0
\]
for every infinite $B_3$ set $A$.

(iii) \emph{Top 3 next moves (concrete).}
\begin{enumerate}
\item Attempt to adapt the known $B_2$ liminf argument to $B_3$ by analyzing overlaps of $A$ with shifted copies and using the $B_3$ constraint to control additive energy of $A$.
\item Use Lemma 41.2 to reduce the problem to properties of Sidon sets with additional constraints, and try to show that a Sidon set with large $N^{1/3}$-density forces a collision among triple sums.
\item Computation: search for large finite $B_3$ subsets of $[1,N]$ for moderate $N$ (via backtracking) to see whether extremal examples suggest a block structure that could be extended to an infinite set with positive liminf.
\end{enumerate}

(iv) \emph{What a minimal counterexample would likely look like.}
A counterexample would be an infinite $B_3$ set with $|A\cap[1,N]|\ge c N^{1/3}$ for all sufficiently large $N$ for some fixed $c>0$. By Lemma 41.2 it would also be Sidon, so it would need to be both $B_2$ and $B_3$-rigid while maintaining near-extremal $N^{1/3}$ growth.


