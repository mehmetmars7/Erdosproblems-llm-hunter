### 1) ROUND-2 OBJECTIVE

**(A) Proof-direction (gap-closure):**
Close Round-1 gap (1) by proving a **strictly stronger universal lower bound** on the growth of any additive complement (A) to the squares, namely the recorded best lower bound
[
\liminf_{N\to\infty}\frac{A(N)}{\sqrt N}\ \ge\ \frac{4}{\pi}.
]
This directly strengthens Round-1 Lemma 33.1 and tightens the admissible range for the sought optimal (,c=\inf_A \limsup A(N)/\sqrt N).

---

### 2) ROUND-1 FOUNDATION USED

I will rely on the following Round-1 items exactly as given:

1. **Definition / setup:** ( \mathcal S={n^2:n\ge 0}), (A\subseteq \mathbb N) is an additive complement if (A+\mathcal S) contains all sufficiently large integers; (A(N)=|A\cap{1,\dots,N}|).
2. **Lemma 33.1 (union bound lower bound):** the method “covering interval by translates (a+\mathcal S)” and bounding coverage by summing per-translate contributions.

I will **not** re-prove Lemma 33.1; I only use its *strategy* (covering + union bound) but with a new weighting that yields the constant (4/\pi).

---

### 3) NEW INSIGHT / TOOL (ROUND-2)

**Weighted union bound with a square-root singular weight + endpoint truncation.**

Instead of counting each covered integer with weight (1) (Round-1), assign a weight
[
w_{N}(n)\ :=\ \frac{1}{\sqrt{N-n}}
]
to integers (n\le N), but **truncate away the last (K) integers** near (N), so that (N-n\ge K) on the region we care about.

Then along any translate (a+m^2\le N-K), the weight becomes
[
w_{N}(a+m^2)=\frac{1}{\sqrt{(N-a)-m^2}},
]
and the sum over (m) is controlled by a quarter-circle integral giving (\pi/2).
This is the structural source of the constant (4/\pi).

---

### 4) ATTACK PLAN (ROUND-2)

**Gap after Round-1:** Lemma 33.1 only yields (\liminf A(N)/\sqrt N\ge 1). Round-1 explicitly notes that stronger bounds require capturing unavoidable “overlap/structure” beyond the crude count.

**What must be proved now:** a rigorous inequality implying
[
A(N)\ \ge\ \Big(\frac{4}{\pi}-o(1)\Big)\sqrt N.
]

**Plan:**

1. Fix large (N), choose a truncation parameter (K=K(N)\to\infty) but (K=o(N)) (concretely (K=\lceil N^{1/2}\rceil)).
2. Put weights (w_N(n)=1/\sqrt{N-n}) on the interval ([N_0,,N-K]).
3. Use that ([N_0,N-K]\subseteq A+\mathcal S) so the weighted sum of this interval is bounded by the sum of weighted contributions of translates (a+\mathcal S).
4. Show each translate contributes (\le \pi/2 + 1/\sqrt K) to this weighted sum (uniformly, by an integral comparison).
5. Compute the total weight of ([N_0,N-K]) as (\sim 2\sqrt N).
6. Combine to get (A(N)\ge (4/\pi-o(1))\sqrt N).

This overcomes the Round-1 obstacle because the weight is engineered so that **each translate has essentially constant total weight (\pi/2)** once we avoid the dangerous endpoint where (N-a-m^2) could be tiny.

---

### 5) WORK (ROUND-2)

#### 5.1. Setup

Assume (A+\mathcal S) contains every integer (n\ge N_0).
Fix (N) so large that (N-K\ge N_0), where later we choose (K=\lceil N^{1/2}\rceil).

Define the weighted sum of the target interval
[
W(N,K)\ :=\ \sum_{n=N_0}^{N-K}\frac{1}{\sqrt{N-n}}
\ =\ \sum_{j=K}^{N-N_0}\frac{1}{\sqrt{j}}
\quad (j=N-n).
]

Because ([N_0,N-K]\subseteq A+\mathcal S), we have the set cover inclusion
[
{N_0,N_0+1,\dots,N-K}\ \subseteq\ \bigcup_{a\in A\cap[1,N]} (a+\mathcal S).
]
Apply the weighted union bound: for any nonnegative weight function (w),
[
\sum_{x\in \cup_i E_i} w(x)\ \le\ \sum_i \sum_{x\in E_i} w(x).
]
Here (w(x)=1/\sqrt{N-x}) on ([N_0,N-K]) and (0) outside.

Thus
[
W(N,K)\ \le\ \sum_{a\in A\cap[1,N]}\ \sum_{\substack{m\ge 0\ a+m^2\in[N_0,N-K]}}
\frac{1}{\sqrt{N-(a+m^2)}}.
\tag{1}
]

Note that if (a>N-K), then (a+m^2\ge a>N-K), so such (a) contributes **nothing** to the right-hand side of (1). Hence we may restrict (a\le N-K):
[
W(N,K)\ \le\ \sum_{a\in A\cap[1,N-K]} T_{N,K}(a),
]
where
[
T_{N,K}(a)\ :=\ \sum_{\substack{m\ge 0\ a+m^2\le N-K}}\frac{1}{\sqrt{N-a-m^2}}.
]

Let (R:=N-a). Then (R\ge K) whenever (a\le N-K), and
[
T_{N,K}(a)\ =\ \sum_{m^2\le R-K}\frac{1}{\sqrt{R-m^2}}.
\tag{2}
]

So the key is to bound the “per-translate weight” (2) uniformly.

---

#### 5.2. Key analytic lemma: per-translate weight (\le \pi/2 + 1/\sqrt K)

**Lemma 5.1 (Truncated square-root sum bound).**
Let (R\ge K\ge 1), and put (M:=\big\lfloor\sqrt{R-K}\big\rfloor). Then
[
\sum_{m=0}^{M}\frac{1}{\sqrt{R-m^2}}
\ \le\ \frac{\pi}{2}+\frac{1}{\sqrt K}.
\tag{3}
]

**Proof.**
Consider (f(t)=\dfrac{1}{\sqrt{R-t^2}}) for (0\le t<\sqrt R). This is increasing on ([0,\sqrt R)).

Split the sum into the first (M) terms and the last term:
[
\sum_{m=0}^{M}\frac{1}{\sqrt{R-m^2}}
====================================

\sum_{m=0}^{M-1} f(m) + f(M).
]

Because (f) is increasing, for each integer (m\le M-1),
[
\int_{m}^{m+1} f(t),dt \ \ge\ f(m)\cdot 1,
]
so summing gives
[
\sum_{m=0}^{M-1} f(m)\ \le\ \int_{0}^{M} f(t),dt.
]
Also, by the definition of (M), we have (M^2\le R-K), hence
[
R-M^2\ \ge\ K
\quad\Longrightarrow\quad
f(M)=\frac{1}{\sqrt{R-M^2}}\ \le\ \frac{1}{\sqrt K}.
]

Therefore
[
\sum_{m=0}^{M}\frac{1}{\sqrt{R-m^2}}
\le \int_0^M \frac{dt}{\sqrt{R-t^2}}+\frac{1}{\sqrt K}
\le \int_0^{\sqrt{R-K}} \frac{dt}{\sqrt{R-t^2}}+\frac{1}{\sqrt K}.
]

Finally, compute the integral:
[
\int_0^{\sqrt{R-K}} \frac{dt}{\sqrt{R-t^2}}
=\arcsin!\Big(\frac{t}{\sqrt R}\Big)\Big|_{0}^{\sqrt{R-K}}
=\arcsin!\Big(\sqrt{1-\frac{K}{R}}\Big)
\le \frac{\pi}{2}.
]
Combine to obtain (3). (\square)

---

#### 5.3. Apply the lemma to the cover

By (2) and Lemma 5.1, for each (a\in A\cap[1,N-K]) (so (R=N-a\ge K)),
[
T_{N,K}(a)\ \le\ \frac{\pi}{2}+\frac{1}{\sqrt K}.
]
Hence
[
W(N,K)
\le
\sum_{a\in A\cap[1,N-K]} T_{N,K}(a)
\le
|A\cap[1,N-K]|\Big(\frac{\pi}{2}+\frac{1}{\sqrt K}\Big)
\le
A(N)\Big(\frac{\pi}{2}+\frac{1}{\sqrt K}\Big).
\tag{4}
]

So
[
A(N)\ \ge\ \frac{W(N,K)}{\frac{\pi}{2}+\frac{1}{\sqrt K}}.
\tag{5}
]

---

#### 5.4. Estimate (W(N,K)) and take limits

Recall
[
W(N,K)=\sum_{j=K}^{N-N_0}\frac{1}{\sqrt{j}}.
]

Use the integral bound (valid for decreasing positive functions):
[
\sum_{j=K}^{L}\frac{1}{\sqrt{j}}
\ \ge\ \int_{K}^{L+1}\frac{dx}{\sqrt{x}}
\ =\ 2\big(\sqrt{L+1}-\sqrt{K}\big).
]
Taking (L=N-N_0) gives
[
W(N,K)\ \ge\ 2\big(\sqrt{N-N_0+1}-\sqrt{K}\big).
\tag{6}
]

Now pick
[
K:=\big\lceil N^{1/2}\big\rceil.
]
Then (\sqrt{K}\sim N^{1/4}) and (\frac{1}{\sqrt K}\sim N^{-1/4}), hence
[
\frac{W(N,K)}{\sqrt N}
\ \ge
\frac{2(\sqrt{N-N_0+1}-\sqrt K)}{\sqrt N}
=2\Big(\sqrt{1-\frac{N_0-1}{N}}-\frac{\sqrt K}{\sqrt N}\Big)
\to 2(1-0)=2.
\tag{7}
]
Also,
[
\frac{\pi}{2}+\frac{1}{\sqrt K}\ \to\ \frac{\pi}{2}.
\tag{8}
]

Divide (5) by (\sqrt N) and take (\liminf_{N\to\infty}):
[
\liminf_{N\to\infty}\frac{A(N)}{\sqrt N}
\ \ge
\liminf_{N\to\infty}
\frac{\frac{W(N,K)}{\sqrt N}}{\frac{\pi}{2}+\frac{1}{\sqrt K}}
\ \ge
\frac{2}{\pi/2}
===============

\frac{4}{\pi}.
]

This proves:

**Theorem 33.3 (Moser-type lower bound).**
If (A+\mathcal S) contains all integers (n\ge N_0), then
[
\boxed{\ \liminf_{N\to\infty}\frac{A(N)}{\sqrt N}\ \ge\ \frac{4}{\pi}\ }.
]

In particular, for the original optimization target (c_*=\inf_{A}\limsup A(N)/\sqrt N),
[
c_*\ \ge\ \frac{4}{\pi}.
]

---

### 6) ADVERSARIAL VERIFICATION

**(i) Endpoint / division-by-zero issues.**
The weight (w_N(n)=1/\sqrt{N-n}) is only used for (n\le N-K), so (N-n\ge K\ge 1): no singularity. Good.

**(ii) Missing contributions from (a>N-K).**
If (a>N-K), then (a+m^2\ge a>N-K) so it does not hit the weighted interval ([N_0,N-K]). Hence discarding them is legitimate.

**(iii) Discretization gap near (\sqrt{R-K}).**
The only non-integral remainder is handled by isolating the last term (f(M)) and bounding it by (1/\sqrt K) using (R-M^2\ge K). This is the critical step: it prevents the “(R) near a square” blow-up that would have destroyed the (\pi/2) constant.

**(iv) Quantifiers / asymptotics.**
We must ensure (N-K\ge N_0) so that every integer in the weighted interval is indeed guaranteed representable. With (K=\lceil N^{1/2}\rceil), this holds for all sufficiently large (N), so taking (\liminf) is valid.

**(v) Could a different choice of (K) improve the constant?**
The leading constant comes from the ratio (2 / (\pi/2)=4/\pi). Any (K\to\infty) with (K=o(N)) yields the same limit, because (a) removing the last (K) terms removes only (O(\sqrt K)=o(\sqrt N)) weight, and (b) the per-translate error term becomes (o(1)). So the constant (4/\pi) is robust under this scheme.

**(vi) Consistency with Round-1 Lemma 33.1.**
Lemma 33.1 gave (\liminf\ge 1). The new argument refines the same union-bound idea with weights, giving (\liminf\ge 4/\pi>1). No contradiction; it strictly strengthens.

---

### 7) FINAL (EXACTLY ONE)

**UNRESOLVED (BUT STRICTLY ADVANCED).**
We still do not determine the optimal constant (c) for (\inf_A \limsup A(N)/\sqrt N), but we have **rigorously closed** Round-1 gap (1) by proving the sharpest lower bound recorded in Round-1:
[
\liminf_{N\to\infty}\frac{A(N)}{\sqrt N}\ \ge\ \frac{4}{\pi},
]
hence any candidate (c) must satisfy (c\ge 4/\pi).

---

### 8) COMPLETION ESTIMATE (MANDATORY)

**COMPLETION: 40%**

---

### 9) REFERENCES

No external references used; only elementary calculus ((\int_0^{\sqrt{R}} \frac{dt}{\sqrt{R-t^2}}=\pi/2)) and weighted union bound / monotone integral comparison.
