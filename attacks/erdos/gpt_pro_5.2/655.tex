\section*{PROBLEM 1 (\#655): Distinct distances with bounded concentric multiplicity}

\subsection*{FORMAL RESTATEMENT}
Fix $n\in\mathbb{N}$ and distinct points $x_1,\dots,x_n\in\mathbb{R}^2$.

\begin{itemize}[leftmargin=*]
\item Convention: a \emph{circle} with center $x_i$ means the set $\{y\in\mathbb{R}^2: \|y-x_i\|=\rho\}$ for some radius $\rho>0$ (not the closed disk).
\item Hypothesis (``no centered circle contains three other points''): for every $i\in\{1,\dots,n\}$ and every radius $\rho>0$, the circle centered at $x_i$ of radius $\rho$ contains at most two points of $\{x_1,\dots,x_n\}\setminus\{x_i\}$. Equivalently,
\[
\forall i\ \forall \rho>0:\quad \bigl|\{j\neq i: \|x_i-x_j\|=\rho\}\bigr|\le 2.
\]
\item Let the \emph{distance set}
\[
D(x_1,\dots,x_n):=\{\|x_i-x_j\|: 1\le i<j\le n\},
\]
and write $|D|$ for its cardinality.
\end{itemize}

Question (as written): Does there exist an absolute constant $c>0$ and an $n_0\in\mathbb{N}$ such that for all $n\ge n_0$ and all point sets satisfying the hypothesis,
\[
|D(x_1,\dots,x_n)|\ \ge\ (1+c)\frac{n}{2}\ ?
\]

\subsection*{QUICK LITERATURE/CONTEXT CHECK}
A quick scan of the online discussion of Erd\H{o}s problem \#655 reports that the conjectured bound is false, and specifically notes an observation (attributed there to Zach Hunter) that $n$ equally spaced points on a circle give a counterexample.

\subsection*{ATTACK PLAN}
\begin{enumerate}[leftmargin=*]
\item \textbf{Disproof strategy:} take the vertex set of a regular $n$-gon on a circle. Check the hypothesis by computing, for a fixed vertex, how many other vertices occur at the same distance. Then compute the number of distinct interpoint distances.
\item \textbf{Verification strategy:} handle odd/even $n$ carefully (the ``opposite vertex'' case when $n$ is even) and confirm that the chord lengths are all distinct for step sizes $1,\dots,\lfloor n/2\rfloor$.
\end{enumerate}

\subsection*{WORK}
\begin{theorem}[Counterexample]
The statement in Problem \#655 is false. In fact, for every $n\ge 3$ there exist points $x_1,\dots,x_n\in\mathbb{R}^2$ satisfying the hypothesis but with
\[
|D(x_1,\dots,x_n)|=\left\lfloor\frac{n}{2}\right\rfloor.
\]
Consequently, for every $c>0$ and all sufficiently large $n$, one has
$|D(x_1,\dots,x_n)|<(1+c)\frac{n}{2}$.
\end{theorem}

\begin{proof}
Fix $n\ge 3$ and set $R=1$. Define points on the unit circle by
\[
 x_k := \bigl(\cos(2\pi k/n),\,\sin(2\pi k/n)\bigr)\in\mathbb{R}^2,\qquad k=0,1,\dots,n-1.
\]
Relabel them as $x_1,\dots,x_n$ if desired.

\medskip
\noindent\textbf{Step 1: Verify the hypothesis.}
Fix a vertex $x_i$. For each integer $t$ with $1\le t\le n-1$, the two vertices $x_{i+t}$ and $x_{i-t}$ (indices modulo $n$) have the same distance to $x_i$, and that distance depends only on $t$.
Indeed, the central angle between $x_i$ and $x_{i+t}$ is $2\pi t/n$, so the chord length is
\[
 d_t := \|x_i-x_{i+t}\| = 2\sin\Bigl(\frac{\pi t}{n}\Bigr).
\]
For a fixed $t$ with $1\le t< n/2$, there are \emph{exactly two} vertices at distance $d_t$ from $x_i$ (namely $x_{i\pm t}$).
If $n$ is even and $t=n/2$, there is \emph{exactly one} vertex at that distance (the opposite vertex).
Hence for every radius $\rho>0$, the circle centered at $x_i$ of radius $\rho$ contains at most two other points of the configuration. This is exactly the hypothesis.

\medskip
\noindent\textbf{Step 2: Compute the number of distinct distances.}
Every interpoint distance in this configuration is some $d_t$ with $1\le t\le \lfloor n/2\rfloor$ (because $d_t=d_{n-t}$).
We claim these values are all distinct:
if $1\le s<t\le\lfloor n/2\rfloor$, then
\[ 0<\frac{\pi s}{n}<\frac{\pi t}{n}\le\frac{\pi}{2}, \]
and $\sin$ is strictly increasing on $(0,\pi/2]$, so
$\sin(\pi s/n)<\sin(\pi t/n)$ and therefore $d_s<d_t$.
Thus
\[
D(x_1,\dots,x_n)=\{d_t: 1\le t\le \lfloor n/2\rfloor\}
\quad\text{and}\quad
|D(x_1,\dots,x_n)|=\left\lfloor\frac{n}{2}\right\rfloor.
\]

\medskip
\noindent\textbf{Step 3: Contradict the claimed lower bound.}
Let $c>0$ be arbitrary. For all $n>\frac{2}{c}$,
\[
\left\lfloor\frac{n}{2}\right\rfloor\le \frac{n}{2} < (1+c)\frac{n}{2},
\]
so the required inequality fails for this $n$ and this configuration.
Since this works for arbitrarily large $n$, no such absolute constant $c>0$ can exist.
\end{proof}

\medskip
\noindent\textbf{(Tiny-case computation check.)}
For the regular $n$-gon construction, one gets $|D|=\lfloor n/2\rfloor$ for small $n$ as well:
\[
\begin{array}{c|cccccccc}
 n & 3 & 4 & 5 & 6 & 7 & 8 & 9 & 10\\\hline
 |D| & 1 & 2 & 2 & 3 & 3 & 4 & 4 & 5
\end{array}
\]

\subsection*{VERIFICATION}
\begin{itemize}[leftmargin=*]
\item \textbf{Quantifiers:} the counterexample is constructed for \emph{every} $n\ge 3$, so it defeats any universal $c>0$ and any $n_0$.
\item \textbf{Hypothesis check:} for each center $x_i$ and each radius $\rho$, the points at distance $\rho$ correspond to a step size $t$; there are at most two such points (or one if $n$ is even and $t=n/2$).
\item \textbf{Distinctness of distances:} proved via strict monotonicity of $\sin$ on $(0,\pi/2]$.
\item \textbf{Edge cases:} for $n=1,2$ the hypothesis is vacuous and the conclusion is irrelevant; the statement asks for $n$ sufficiently large anyway.
\end{itemize}

\subsection*{FINAL}
\textbf{LABEL: FULL SOLUTION}\\
\textbf{SUBLABEL: COUNTEREXAMPLE/DISPROOF}\\
A regular $n$-gon (equally spaced points on a circle) satisfies the hypothesis but determines exactly $\lfloor n/2\rfloor$ distinct distances, contradicting any lower bound of the form $(1+c)\,n/2$ with fixed $c>0$.

\subsection*{COMPLETION ESTIMATE}
COMPLETION: 100\%.


%%%%%%%%%%%%%%%%%%%%%%%%%%%%%%%%%%%%%%%%%%%%%%%%%%%%%%%%%%%%%%%%%%%%%%%%%%%%%%%
