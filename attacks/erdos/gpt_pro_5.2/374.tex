\section*{Problem 374. Growth of $D_k$ for products of factorials that are squares}

\subsection*{FORMAL RESTATEMENT}
For a finite set $A\subset \mathbb{N}$ let
\[
  m(A)\;:=\;\prod_{a\in A} a!,\qquad M(A)\;:=\;\max A.
\]
For $k\ge 1$ define
\[
  F_k\;:=\;\bigl\{n\in\mathbb{N}:\exists A\subset\mathbb{N}\text{ with }|A|\le k,\ M(A)=n,\ m(A)\text{ a perfect square}\bigr\},
\]
with the convention $F_0:=\varnothing$, and define
\[
  D_k:=F_k\setminus F_{k-1}.
\]
Equivalently, for $m\in\mathbb{N}$ define $F(m)$ as the minimal $k\ge 2$ (if it exists) for which there are
$a_1<\cdots<a_k=m$ with $\prod_{i=1}^k a_i!$ a square; then $m\in D_k$ iff $F(m)=k$.

The question asks for the order of growth of
\[
D_k(n):=|D_k\cap\{1,2,\dots,n\}|
\]
for $3\le k\le 6$. In particular, is $D_6(n)\gg n$?

\subsection*{QUICK LITERATURE / CONTEXT CHECK}
The problem is recorded as open on the Erd\H{o}s problems database, together with several basic facts (no primes in any $D_k$; $D_2=\{a^2:a>1\}$; $D_6$ has least element $527$; and $D_k=\varnothing$ for $k>6$).\footnote{See the summary on the Erd\H{o}s problems page.}

Erd\H{o}s--Graham (1976) introduced/analysed the sets $F_k,D_k$ and proved, among other things:
\begin{itemize}
\item $D_k=\varnothing$ for $k>6$ (so every $n\in\mathbb{N}$ that ever occurs occurs with $k\le 6$);
\item $D_3(n)=o(D_4(n))$ (in particular $D_3(n)=o(n)$);
\item $D_4$ has positive density, using the explicit inclusion ``non-squarefree $\subset F_4$''.
\end{itemize}
Dujella--Najman--Saradha--Shorey (2014) give a clean list of standard constructions producing many elements of $D_3$, and record the best known upper bounds for $D_3(n)$.

\subsection*{ATTACK PLAN}
\begin{enumerate}
\item Use the explicit constructions of Erd\H{o}s--Graham to give unconditional \emph{lower bounds} for $D_k(n)$.
\item Combine these with known \emph{upper bounds} (mainly for $D_3(n)$) to bracket the possible growth.
\item For $k\in\{5,6\}$, use the semiprime families $2p$ and $13p$ together with Erd\H{o}s--Graham's ``almost all $pq$ are not in $F_4$'' and ``almost all $13p$ are not in $F_5$'' to get explicit $\gg n/\log n$ lower bounds.
\item Explain why the remaining gaps (e.g. whether $D_6(n)\gg n$) look genuinely open: proving linear growth would require control on how often $n$ belongs to $F_5$ (or $F_4$) among squarefree integers.
\end{enumerate}

\subsection*{WORK}
\subsubsection*{1. Two algebraic identities that power many inclusions}
\begin{lemma}[Square-factor and two-factor identities]
\label{lem:factor-identities}
Let $x,y\in\mathbb{N}$.
\begin{enumerate}
\item If $x\ge 1$ then
\[
  x!\,(x-1)!\;=\;\frac{(x!)^2}{x}.
\]
\item If $x,y\ge 1$ then
\[
  x!\,(x-1)!\,y!\,(y-1)!\;=\;\frac{(x!\,y!)^2}{xy}.
\]
\item More generally, for any $x_1,\dots,x_t\ge 1$,
\[
  \prod_{i=1}^t \bigl(x_i!\,(x_i-1)!\bigr)
  
  =\;\frac{\bigl(\prod_{i=1}^t x_i!\bigr)^2}{\prod_{i=1}^t x_i}.
\]
\end{enumerate}
\end{lemma}
\begin{proof}
(1) is immediate from $(x-1)!=x!/x$. Multiplying (1) for $x$ and $y$ yields (2). Statement (3) is the same calculation iterated $t$ times.
\end{proof}

\subsubsection*{2. A uniform $\Omega(\sqrt{n})$ lower bound for $D_3(n)$}
We recall the standard ``$D_3$ construction'' (Dujella--Najman--Saradha--Shorey, property (iv)).

\begin{lemma}[An explicit infinite family in $D_3$]
\label{lem:D3-family}
Fix any integer $b>1$ and let $Q(b!)$ be the squarefree part of $b!$, i.e. write $b!=Q(b!)\,s^2$ with $Q(b!)$ squarefree. For each integer $a>1$, set
\[
  n\;:=\;a^2\,Q(b!).
\]
Then $n\in D_3$.
\end{lemma}
\begin{proof}
Let $A:=\{n,\ n-1,\ b\}$. Then $|A|=3$ and $M(A)=n$. Using Lemma \ref{lem:factor-identities}(1),
\[
  m(A)=n!\,(n-1)!\,b!\;=\;\frac{(n!)^2}{n}\,b!
  
  =\;\frac{(n!)^2}{a^2Q(b!)}\,Q(b!)s^2
  
  =\;\Bigl(\frac{n!\,s}{a}\Bigr)^2,
\]
so $m(A)$ is a perfect square and hence $n\in F_3$.

Also $n$ is not a perfect square because $Q(b!)$ is squarefree and $Q(b!)>1$ for $b>1$, hence $n\notin D_2=F_2$. Therefore $n\in D_3=F_3\setminus F_2$.
\end{proof}

Taking $b=2$ gives $Q(2!)=2$ and hence the explicit infinite subsequence $\{2a^2:a\ge 2\}\subset D_3$. Counting these yields:
\begin{corollary}[A quantitative lower bound for $D_3(n)$]
\label{cor:D3-lower}
There is an absolute constant $c>0$ such that for all $n$,
\[
  D_3(n)\ge c\,\sqrt{n}.
\]
In particular $D_3$ is infinite.
\end{corollary}
\begin{proof}
From Lemma \ref{lem:D3-family} with $b=2$, every integer of the form $2a^2$ with $a\ge 2$ lies in $D_3$. The count of such integers $\le n$ is $\lfloor\sqrt{n/2}\rfloor-1\ge c\sqrt{n}$ for a suitable absolute $c$ and all $n$.
\end{proof}

\subsubsection*{3. Known upper bounds for $D_3(n)$}
Erd\H{o}s--Graham proved $D_3(n)=o(D_4(n))$ (in particular $D_3(n)=o(n)$). Dujella--Najman--Saradha--Shorey record a sharper bound due to Saradha--Shorey:
\[
  D_3(X)=O\!\left(\frac{X\,\log_3 X}{\log X\,\log_2 X}\right)
\]
(where $\log_j$ denotes iterated logarithms). We will treat this as a black-box literature input.

\subsubsection*{4. $D_4(n)$ has linear growth (positive density)}
Erd\H{o}s--Graham observed that if $n$ has a nontrivial square factor then $n\in F_4$. The proof is a one-line application of Lemma \ref{lem:factor-identities}(2).

\begin{lemma}[Non-squarefree integers lie in $F_4$]
\label{lem:non-squarefree-in-F4}
If $n=m^2 r$ with $m>1$, then $n\in F_4$.
\end{lemma}
\begin{proof}
Let $A:=\{n,n-1,r,r-1\}$. Then $|A|\le 4$ and $M(A)=n$. By Lemma \ref{lem:factor-identities}(2),
\[
  m(A)=n!\,(n-1)!\,r!\,(r-1)!
  
  =\;\frac{(n!\,r!)^2}{nr}.
\]
Since $nr=(m^2r)r=(mr)^2$ is a perfect square, the right-hand side is a perfect square.
\end{proof}

Let $\mathsf{NSF}(n)$ be the set of integers $\le n$ that are \emph{not squarefree}. It is classical that $|\mathsf{NSF}(n)|=(1-6/\pi^2)n+o(n)$. Combining this with Lemma \ref{lem:non-squarefree-in-F4} and the fact that $D_3(n)=o(n)$ gives a linear lower bound for $D_4(n)$.

\begin{proposition}[$D_4$ has positive density]
\label{prop:D4-density}
We have
\[
  D_4(n)\ge \Bigl(1-\frac{6}{\pi^2}\Bigr)n - o(n)
\]
and in particular $D_4(n)\gg n$.
\end{proposition}
\begin{proof}
By Lemma \ref{lem:non-squarefree-in-F4}, every non-squarefree $\le n$ belongs to $F_4$. Among these, the squares belong to $D_2$ and contribute $o(n)$. Also $D_3(n)=o(n)$. Hence all but $o(n)$ of the non-squarefree integers $\le n$ are in $F_4\setminus F_3=D_4$.
\end{proof}

\subsubsection*{5. A $\gg n/\log n$ lower bound for $D_5(n)$}
Erd\H{o}s--Graham prove (i) if $p\in\{2,3,5,7,11\}$ is a proper divisor of $n$ then $n\in F_5$, and (ii) for any fixed prime $q$, almost all semiprimes of the form $pq$ (with $p$ prime) do \emph{not} lie in $F_4$. They also show that if $a_1\in D_3$ and $a_1=2P$ with $P$ an odd prime, then $a_1\in\{6,10\}$.

From these inputs one can extract a clean lower bound using the family $n=2p$.

\begin{proposition}[Many semiprimes are in $D_5$]
\label{prop:D5-lower}
Let $S(X):=\{2p\le X: p\text{ prime},\ p>5\}$. Then, as $X\to\infty$, all but $o(|S(X)|)$ elements of $S(X)$ lie in $D_5$. Consequently
\[
  D_5(X)\ge (1-o(1))\pi(X/2)\ \ \text{and hence}\ \ D_5(X)\gg \frac{X}{\log X}.
\]
\end{proposition}
\begin{proof}
Fix $n=2p$ with $p$ an odd prime.

\emph{Step 1: $n\in F_5$.} Since $2$ is a proper divisor of $n$, Erd\H{o}s--Graham's Fact~7 implies $n\in F_5$.

\emph{Step 2: $n\notin F_3$ for $p>5$.} If $n\in F_3$, then $n\in D_3$ (because $n$ is not a square). But Erd\H{o}s--Graham show that if a number of the form $2P$ with odd prime $P$ lies in $D_3$, then it must be $6$ or $10$. Thus for $p>5$ we have $2p\notin F_3$.

\emph{Step 3: for almost all primes $p$, $2p\notin F_4$.} Erd\H{o}s--Graham state that for any fixed prime $q$, almost all numbers of the form $Pq$ (with $P$ prime) do not belong to $F_4$. Applying this with $q=2$ gives that for all but $o(\pi(X))$ primes $p\le X$, the number $2p$ is not in $F_4$.

Now combine: for such primes $p>5$ with $2p\notin F_4$, we have
\[
  2p\in F_5\setminus F_4 = D_5.
\]
Counting these primes gives the claimed bound.
\end{proof}

\subsubsection*{6. A $\gg n/\log n$ lower bound for $D_6(n)$}
Erd\H{o}s--Graham also prove:
\begin{itemize}
\item If $n=ab$ with $a,b>1$ and $a\ne b$ then $n\in F_6$ (by Lemma \ref{lem:factor-identities}(3) with $t=3$ and $(x_1,x_2,x_3)=(n,a,b)$).
\item (Their Theorem~3) For almost all primes $p$, the number $13p$ does \emph{not} lie in $F_5$.
\end{itemize}
Together with the ``almost all $pq\notin F_4$'' statement, this yields a semiprime lower bound for $D_6$.

\begin{proposition}[Many semiprimes are in $D_6$]
\label{prop:D6-lower}
Let $T(X):=\{13p\le X: p\text{ prime},\ p\ne 13\}$. Then, as $X\to\infty$, all but $o(|T(X)|)$ elements of $T(X)$ lie in $D_6$. Consequently
\[
  D_6(X)\ge (1-o(1))\pi(X/13)\ \ \text{and hence}\ \ D_6(X)\gg \frac{X}{\log X}.
\]
\end{proposition}
\begin{proof}
Fix $n=13p$ with $p$ prime and $p\ne 13$.

\emph{Step 1: $n\in F_6$.} Write $n=ab$ with $a=13$, $b=p$, so $a,b>1$ and $a\ne b$. Erd\H{o}s--Graham's construction gives $n\in F_6$.

\emph{Step 2: $n\notin F_4$ for almost all primes $p$.} Apply the Erd\H{o}s--Graham statement ``for any fixed prime $q$, almost all $Pq$ are not in $F_4$'' with $q=13$.

\emph{Step 3: $n\notin F_5$ for almost all primes $p$.} Erd\H{o}s--Graham's Theorem~3 states that for almost all primes $p$, $13p\notin F_5$.

For primes $p$ satisfying both Step 2 and Step 3, we have $13p\notin F_5$, but $13p\in F_6$, hence $13p\in D_6=F_6\setminus F_5$.

Counting such primes gives the result.
\end{proof}

\subsubsection*{7. What this does (and does not) say about the original question}
Collecting the proven bounds:
\begin{itemize}
\item $D_3(n)$ satisfies \(c\sqrt{n}\le D_3(n)=o(n)\), and in fact one has the sharper upper bound
$D_3(n)=O\bigl(n\log_3 n/(\log n\log_2 n)\bigr)$.
\item $D_4(n)\gg n$ (indeed, $D_4(n)\ge (1-6/\pi^2)n-o(n)$).
\item $D_5(n)\gg n/\log n$ and $D_6(n)\gg n/\log n$.
\end{itemize}

The question ``is $D_6(n)\gg n$?'' remains open: Proposition \ref{prop:D6-lower} only gives $\gg n/\log n$, and no mechanism above forces a positive proportion of integers to avoid $F_5$.

\subsection*{VERIFICATION}
All identities reduce to Lemma \ref{lem:factor-identities}, whose proof is just $(x-1)!=x!/x$. The only nontrivial inputs are the cited literature facts from Erd\H{o}s--Graham (1976) and Dujella--Najman--Saradha--Shorey (2014); the deductions of Propositions \ref{prop:D5-lower} and \ref{prop:D6-lower} from those facts are straightforward.

\subsection*{FINAL}
\textbf{LABEL: UNRESOLVED.} \\
\textbf{SUB-LABEL: N/A.} \\
We obtained unconditional lower bounds
\(D_4(n)\gg n\),\(D_5(n)\gg n/\log n\),\(D_6(n)\gg n/\log n\),
and the bracket \(c\sqrt{n}\le D_3(n)\le n\log_3 n/(\log n\log_2 n)\),
but the precise order of growth (e.g. whether $D_6(n)\gg n$) is not resolved.

\subsection*{COMPLETION ESTIMATE}
To reach a full solution one would need, for each $k\in\{3,4,5,6\}$, matching upper and lower bounds (ideally asymptotics) for $D_k(n)$. The main missing ingredient is structural control of how often a squarefree integer lies in $F_4$ or $F_5$ (especially among semiprimes), beyond the current ``almost all $pq\notin F_4$'' type statements.

%%%%%%%%%%%%%%%%%%%%%%%%%%%%%%%%%%%%%%%%%%%%%%%%%%%%%%%%%%%%%%%%%%%%%%%%%%%%%%
