% Erdos Problem #332
% URL: https://www.erdosproblems.com/332

1) FORMAL RESTATEMENT

Let $A\subseteq\mathbb N$.
Define
\[
D(A):=\{d\in\mathbb Z:\ d\text{ occurs infinitely often as }a_1-a_2\text{ with }a_1,a_2\in A\}.
\]
Equivalently,
\[
D(A)=\{d\in\mathbb Z:\ |A\cap(A+d)|=\infty\},
\]
where $A+d:=\{a+d: a\in A\}$.

A set $S\subseteq\mathbb Z$ has \emph{bounded gaps} (is \emph{syndetic}) if there exists an integer $L\ge 1$ such that every interval of $L$ consecutive integers meets $S$.

Question: what conditions on $A$ ensure that $D(A)$ has bounded gaps?
The statement says that a sufficient condition is that $A$ has positive density.


2) QUICK LITERATURE/CONTEXT CHECK

I do not use any external results beyond the problem statement.
I will prove from scratch the classical sufficient condition:

If $A$ has positive upper Banach density (in particular, positive asymptotic density), then $D(A)$ is syndetic.


3) ATTACK PLAN

Prove a sufficient condition (positive density) by a translate-packing argument:

(1) Using positive upper Banach density $\delta>0$, build a finite maximal set of shifts $X$ such that the translates $A+x$ have pairwise finite intersections.

(2) Maximality forces every translate $A+t$ to have infinite intersection with some $A+x$ from the family, which implies $t-x\in D(A)$.

(3) Conclude $\mathbb Z = X + D(A)$, hence $D(A)$ has bounded gaps.

Also provide simple examples as sanity checks.


4) WORK

PHASE 1: FAST REALITY CHECK (sanity examples)

* If $A=2\mathbb N$ (even numbers), then $D(A)$ contains all even integers (each difference occurs infinitely often), hence has bounded gaps (gap size $2$).

* If $A=\{2^n: n\ge 0\}$, then every nonzero difference $2^i-2^j$ occurs at most once (because the binary expansion is unique), so $D(A)=\{0\}$ (if we allow $a_1=a_2$). In particular, $D(A)\cap\mathbb N$ has unbounded gaps.

So density-type hypotheses are necessary in some form.


Definitions for the proof.
The \emph{upper Banach density} of $A\subseteq\mathbb Z$ is
\[
d^*(A):=\limsup_{N\to\infty}\ \max_{M\in\mathbb Z}\ \frac{|A\cap\{M+1,\dots,M+N\}|}{N}.
\]
If $A\subseteq\mathbb N$ has positive upper asymptotic density, then viewing it as a subset of $\mathbb Z$ gives $d^*(A)>0$.
Also $d^*(A+t)=d^*(A)$ for every integer shift $t$ (by translation invariance of the definition).


Lemma 332.1 (finite packing by almost-disjoint translates).

Assume $d^*(A)=\delta>0$.
There exists a finite set of integers $X=\{x_1,\dots,x_m\}$ such that:

(a) For all $i\neq j$, the intersection $(A+x_i)\cap(A+x_j)$ is finite.

(b) $X$ is maximal with respect to property (a): if $t\in\mathbb Z\setminus X$, then there exists $x\in X$ such that $(A+t)\cap(A+x)$ is infinite.

Moreover, necessarily $m\le 1/\delta$.

Proof.
We build $X$ greedily.
Start with $X=\{0\}$. While there exists an integer $t$ such that $(A+t)\cap(A+x)$ is finite for every $x\in X$, add such a $t$ to $X$.
This process either stops after finitely many steps or continues indefinitely.

We show it cannot continue beyond $\lfloor 1/\delta\rfloor+1$ steps.
Suppose we have constructed $m$ shifts $x_1,\dots,x_m$ with pairwise finite intersections.
For each $i$, the set $A+x_i$ has upper Banach density $\delta$.
Because the pairwise intersections are finite, we may delete from each $A+x_i$ a finite set so that the resulting sets become pairwise disjoint; deleting finitely many points does not change upper Banach density.
Call these disjoint sets $B_i\subseteq A+x_i$ with $d^*(B_i)=\delta$.
Then for any interval $I$ of length $N$,
\[
\bigl|\bigcup_{i=1}^m B_i\cap I\bigr| = \sum_{i=1}^m |B_i\cap I|.
\]
Taking a limsup over $N$ and a max over shifts in the definition of $d^*$ yields
\[
d^*\Bigl(\bigcup_{i=1}^m B_i\Bigr) \ge \sum_{i=1}^m d^*(B_i)=m\delta.
\]
But $\bigcup_i B_i\subseteq\mathbb Z$, so its upper Banach density is at most $1$.
Therefore $m\delta\le 1$, i.e. $m\le 1/\delta$.
So the greedy process must stop after finitely many steps, producing a finite maximal family $X$ satisfying (a).
The stopping condition is exactly maximality (b).
\qed


Lemma 332.2 (maximality forces $\mathbb Z=X+D(A)$).

Let $A\subseteq\mathbb Z$ with $d^*(A)>0$, and let $X$ be as in Lemma 332.1.
Then
\[
\mathbb Z = X + D(A):=\{x+d: x\in X,\ d\in D(A)\}.
\]

Proof.
Let $t\in\mathbb Z$ be arbitrary.
By maximality of $X$ (Lemma 332.1(b)), there exists $x\in X$ such that $(A+t)\cap(A+x)$ is infinite.
Thus there are infinitely many integers $y$ with $y\in A+t$ and $y\in A+x$.
Write $y=a_1+t=a_2+x$ with $a_1,a_2\in A$.
Then
\[
a_2-a_1 = t-x.
\]
Because this happens for infinitely many such $y$, the difference $t-x$ occurs infinitely often as a difference of two elements of $A$.
Hence $t-x\in D(A)$.
Therefore $t=x+(t-x)\in X+D(A)$.
Since $t$ was arbitrary, $\mathbb Z\subseteq X+D(A)$, and the reverse inclusion is trivial. \qed


Proposition 332.3 (positive density $\Rightarrow$ bounded gaps in $D(A)$).

If $A\subseteq\mathbb N$ has positive (upper) density, then $D(A)$ has bounded gaps.

Proof.
Positive upper density implies positive upper Banach density: if $\overline d(A)>0$ then $d^*(A)\ge \overline d(A)>0$.
Apply Lemma 332.1 to obtain a finite set $X$ and Lemma 332.2 to obtain $\mathbb Z=X+D(A)$.
Let
\[
L:=\max X - \min X + 1.
\]
This is a finite positive integer because $X$ is finite.
Fix any integer $m$ and consider the interval $[m,m+L-1]$.
Let $t:=m+\max X$.
By $\mathbb Z=X+D(A)$ there exist $x\in X$ and $d\in D(A)$ with $t=x+d$, so $d=t-x$.
Because $x\in X$, we have $\min X\le x\le \max X$, hence
\[
m = t-\max X \le d = t-x \le t-\min X = m+\max X-\min X = m+L-1.
\]
So $d\in D(A)\cap[m,m+L-1]$.
Thus every interval of length $L$ meets $D(A)$, proving bounded gaps. \qed


5) VERIFICATION

-- Lemma 332.1: the only nontrivial point is bounding $m$. For pairwise finite intersections, deleting finitely many points makes the sets disjoint; upper Banach density is unchanged under finite modifications.
The inequality $d^*(\cup B_i)\ge\sum d^*(B_i)$ holds for disjoint sets because counts add on each interval.

-- Lemma 332.2: ``infinite intersection'' is exactly what is needed for membership in $D(A)$.

-- Proposition 332.3: the bounded-gap constant depends on the constructed finite set $X$ but is finite.


6) FINAL

**UNRESOLVED**

(i) Strongest fully proved partial result obtained here.

* A complete proof that positive density of $A$ implies $D(A)$ has bounded gaps (Proposition 332.3).

(ii) Exact first gap.

Find weaker hypotheses on $A$ (strictly weaker than positive density) that still force $D(A)$ to be syndetic, or determine sharp thresholds/characterizations.

(iii) Top 3 next moves (concrete targets).

1. Identify density-free conditions implying $d^*(A)>0$ on long intervals (e.g. piecewise syndeticity) and test whether they suffice for bounded gaps in $D(A)$.
2. Investigate conditions ensuring $D(A)$ has positive density (stronger than bounded gaps), and whether this can be characterized by additive energy of $A$.
3. Construct sparse sets $A$ (density $0$) where $D(A)$ is nevertheless syndetic, to map the boundary of the phenomenon.

(iv) What a minimal counterexample would likely look like.

A minimal counterexample to ``$D(A)$ syndetic'' under a proposed weaker condition would be a sparse $A$ whose translates almost never overlap infinitely often, so that $A\cap(A+d)$ is finite for all but a very sparse set of $d$. The powers-of-two example illustrates how extreme sparsity can force $D(A)$ to be tiny.

