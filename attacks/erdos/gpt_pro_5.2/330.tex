% Erdos Problem #330
% URL: https://www.erdosproblems.com/330

1) FORMAL RESTATEMENT

A set $A\subseteq\mathbb N$ is an \emph{additive basis of order $2$} if there exists $N_0$ such that every $n\ge N_0$ can be written as $n=a+a'$ with $a,a'\in A$.

It is \emph{minimal} (as an order-2 basis) if no proper subset of $A$ is an order-2 basis; equivalently, for every $a\in A$, the set $A\setminus\{a\}$ is not an order-2 basis.

Density convention.
For $S\subseteq\mathbb N$ define the upper asymptotic density
\[
\overline d(S):=\limsup_{N\to\infty}\frac{|S\cap\{1,\dots,N\}|}{N}.
\]

For $a\in A$, define the ``essential set''
\[
E_a:=\{n\in\mathbb N: n\text{ cannot be represented as }b+c\text{ with }b,c\in A\setminus\{a\}\}.
\]
So $E_a$ is the set of integers whose representations as a sum of two elements of $A$ necessarily use the element $a$.

Problem: Does there exist a \emph{minimal} order-2 basis $A$ with positive density (say $\overline d(A)>0$) such that for every $a\in A$,
\[
\overline d(E_a)>0?
\]


2) QUICK LITERATURE/CONTEXT CHECK

No external results are used; the statement attributes the question to Erd\H{o}s and Nathanson.


3) ATTACK PLAN

I cannot construct such a set.
I will instead:

(1) Prove a clean equivalence between minimality and the infinitude of each $E_a$.

(2) Give a concrete positive-density basis example $A=2\mathbb N\cup\{1\}$ and compute $E_a$ for different $a\in A$ to show why the requirement ``$\overline d(E_a)>0$ for all $a$'' is strong.


4) WORK

PHASE 1: FAST REALITY CHECK / SANITY EXAMPLE

Let
\[
A:=2\mathbb N\cup\{1\}=\{1,2,4,6,8,\dots\}.
\]
Then $\overline d(A)=1/2$.
Every sufficiently large even $n$ is a sum of two even numbers from $A$, and every sufficiently large odd $n$ is $1+$even.
So $A$ is an order-2 basis of positive density.

However, as shown in Lemma 330.2, not every element is ``density-essential'': $E_1$ has density $1/2$, but $E_{2t}$ has density $0$ for each even element $2t$.


Lemma 330.1 (minimality $\Leftrightarrow$ every $E_a$ is infinite).

Let $A$ be an order-2 basis. Then $A$ is minimal if and only if $E_a$ is infinite for every $a\in A$.

Proof.
($\Rightarrow$) If $A$ is minimal, then for each $a\in A$ the set $A\setminus\{a\}$ is not an order-2 basis.
So for each $a$ there are infinitely many integers that are not representable as a sum of two elements of $A\setminus\{a\}$.
By definition, those integers form $E_a$. Thus $E_a$ is infinite.

($\Leftarrow$) Conversely, if $E_a$ is infinite for every $a\in A$, then removing any $a$ destroys the basis property (since infinitely many integers fail to be representable). Hence no proper subset obtained by deleting one element is a basis, so $A$ is minimal.
\qed


Lemma 330.2 (in the example $A=2\mathbb N\cup\{1\}$, only $1$ is density-essential).

Let $A=2\mathbb N\cup\{1\}$.

(a) $E_1$ is exactly the set of odd integers, so $\overline d(E_1)=1/2$.

(b) For any even element $2t\in A$, the set $E_{2t}$ is finite (hence $\overline d(E_{2t})=0$).

Proof.
(a) If we remove $1$, the remaining set is $2\mathbb N$. Sums of two even numbers are even, so no odd integer is representable.
Thus every odd integer lies in $E_1$.
Conversely, every sufficiently large even integer is representable as a sum of two even numbers, so even integers are not in $E_1$ (beyond a finite initial segment that is irrelevant for density).
Hence $E_1$ is the odd integers and has upper density $1/2$.

(b) Fix $t\ge 1$ and remove the element $2t$.
We show that every sufficiently large even integer $n$ is still representable as a sum of two even elements of $A\setminus\{2t\}$.
Indeed, for any even $n\ge 4t+6$, we have $n-2\ge 4t+4$ and $n-2\neq 2t$.
Then
\[
n = 2 + (n-2),
\]
where both summands are even and belong to $A\setminus\{2t\}$ (since $2\neq 2t$ and $n-2\neq 2t$ by the inequality above).
Therefore, all sufficiently large even $n$ are not in $E_{2t}$.
Also, all sufficiently large odd $n$ are representable as $1+$even without needing the removed even element.
So $E_{2t}$ is contained in a finite initial segment, hence is finite and has density $0$.
\qed


5) VERIFICATION

-- Lemma 330.1: uses only definitions; ``not a basis'' means infinitely many failures, which is exactly infinitude of $E_a$.

-- Lemma 330.2(b): the representation $n=2+(n-2)$ avoids $2t$ for all $n\ge 4t+6$; the inequality ensures $n-2\ge 2t+?$, so it cannot equal $2t$.


6) FINAL

**UNRESOLVED**

(i) Strongest fully proved partial result obtained here.

* Minimality is equivalent to each $E_a$ being infinite (Lemma 330.1).
* Example analysis: a positive-density basis can have some elements with $\overline d(E_a)=0$ (Lemma 330.2), showing the extra requirement ``$\overline d(E_a)>0$ for all $a$'' is genuinely stronger than minimality.

(ii) Exact first gap.

Construct (or rule out) a minimal order-2 basis $A$ with positive density such that every element is essential for a positive-density set of integers.

(iii) Top 3 next moves (concrete targets).

1. Attempt a construction by partitioning $\mathbb N$ into disjoint positive-density sets $E_{a_i}$ of decreasing densities and forcing each $a_i$ to be necessary for representing $E_{a_i}$.
2. Prove an impossibility theorem: show that in any positive-density basis, there must exist some $a\in A$ with $\overline d(E_a)=0$.
3. Study known constructions of minimal bases (if available) and compute/estimate $\overline d(E_a)$ for typical elements.

(iv) What a minimal counterexample would likely look like.

If such an $A$ does not exist, then every minimal basis of positive density would necessarily contain at least one ``dispensable in density'' element $a$ with $\overline d(E_a)=0$.
A minimal counterexample (to a positive answer) would likely be a very rigid minimal basis where each element controls a positive-density residue-class-type obstruction.


