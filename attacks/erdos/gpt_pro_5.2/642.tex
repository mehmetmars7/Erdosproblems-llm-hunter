\section*{Problem 642}

\subsection*{1) FORMAL RESTATEMENT}
For an integer $n\ge 1$, let $f(n)$ be the maximum number of edges in an $n$-vertex (simple, undirected) graph $G$ with the property:

\begin{quote}
For every cycle $C$ in $G$ with vertex set $S:=V(C)$, the number of \emph{diagonals} of $C$ present in $G$ (i.e., edges of $G[S]$ joining two nonconsecutive vertices of $C$) is strictly less than $|S|$.
\end{quote}
Equivalently, for every cycle $C$ of length $\ell$,
\[
  |E(G[S])| = \ell + \#\{\text{diagonals of $C$ present in }G\} < 2\ell.
\]
Question. Is it true that $f(n)\ll n$, i.e. $f(n)=O(n)$?

\subsection*{2) QUICK LITERATURE/CONTEXT CHECK}
Letting a ``diagonal'' mean a chord present in the graph, the best published general upper bounds currently are
\[
  f(n)\ll n^{3/2}\quad\text{(Chen--Erd\H{o}s--Staton, 1996)}
\qquad\text{and}\qquad
  f(n)\ll n(\log n)^8\quad\text{(Dragani\'{c}--Methuku--Munh\'{a} Correia--Sudakov, 2024)}.
\]
The linear bound remains open.

\subsection*{3) ATTACK PLAN}
Provide:
\begin{enumerate}[label=\arabic*.]
\item A clean linear lower bound construction, to pin down $f(n)$ from below.
\item A structural sufficient condition (in terms of arboricity) guaranteeing the cycle/diagonal property.
\item A summary of what would be needed to close the remaining polylogarithmic gap in the known upper bound.
\end{enumerate}

\subsection*{4) WORK}
\paragraph{4.1. A sufficient condition via arboricity.}
Recall that the \emph{arboricity} $\mathrm{arb}(G)$ is the least $a$ such that $E(G)$ can be partitioned into $a$ forests.
Equivalently (Nash--Williams),
\[\mathrm{arb}(G)=\max_{H\subseteq G,\,|V(H)|\ge 2}\left\lceil\frac{|E(H)|}{|V(H)|-1}\right\rceil.\]

\begin{lemma}[Arboricity $\le 2$ implies the diagonal condition]
If $\mathrm{arb}(G)\le 2$, then $G$ has the property that every cycle has fewer diagonals than vertices.
\end{lemma}
\begin{proof}
Assume $\mathrm{arb}(G)\le 2$.
Let $C$ be any cycle in $G$ with vertex set $S$ and $|S|=\ell$.
The induced subgraph $G[S]$ is a subgraph of $G$, hence also has arboricity at most $2$.
Therefore $G[S]$ can be written as the union of two forests on $\ell$ vertices, so
\[
  |E(G[S])|\le 2(\ell-1)=2\ell-2.
\]
Since $C$ contributes exactly $\ell$ edges inside $G[S]$, the number of diagonals of $C$ present in $G$ is
\[
  |E(G[S])| - \ell \le (2\ell-2)-\ell = \ell-2 < \ell,
\]
as required.
\end{proof}

\paragraph{4.2. A linear lower bound: $f(n)\ge 2n-2$ for $n\ge 4$.}
We explicitly construct an $n$-vertex graph with $2n-2$ edges satisfying the diagonal condition.

\begin{construction}
Fix $n\ge 4$ and vertex set $\{1,2,\dots,n\}$.
Let $T_1$ and $T_2$ be the following spanning trees:
\begin{align*}
E(T_1)&:=\{\,1i: 3\le i\le n\,\}\ \cup\ \{23\},\\
E(T_2)&:=\{\,21\,\}\ \cup\ \{\,2i:4\le i\le n\,\}\ \cup\ \{34\}.
\end{align*}
Then $|E(T_1)|=|E(T_2)|=n-1$ and $E(T_1)\cap E(T_2)=\varnothing$.
Define $G:=T_1\cup T_2$.
\end{construction}

\begin{lemma}
For $n\ge 4$, the graph $G$ above has $|E(G)|=2n-2$ and satisfies the diagonal condition.
\end{lemma}
\begin{proof}
By construction $|E(G)|=|E(T_1)|+|E(T_2)|=2n-2$.
Moreover $G$ is the union of two forests, hence $\mathrm{arb}(G)\le 2$.
By the previous lemma, $G$ satisfies the diagonal condition.
\end{proof}

Consequently,
\[
  f(n)\ge 2n-2\qquad(n\ge 4).
\]

\paragraph{4.3. Status summary.}
Combining the explicit lower bound with the best current upper bounds yields
\[
  2n-2\ \le\ f(n)\ \ll\ n(\log n)^8.
\]
Closing the gap requires either:
(i) improving the upper bound to $O(n)$ by showing that constant average degree already forces a cycle with at least as many chords as vertices, or
(ii) constructing graphs with $\omega(n)$ edges in which every cycle has $<\ell$ diagonals.

\subsection*{5) VERIFICATION / SANITY CHECKS}
\begin{itemize}
\item In a graph of arboricity $\le 2$, every induced subgraph on $\ell$ vertices has at most $2\ell-2$ edges; for cycle vertex sets this forces $\#\text{diagonals}\le \ell-2$.
\item The construction produces two edge-disjoint spanning trees for every $n\ge 4$; their union has $2n-2$ edges and arboricity at most $2$.
\end{itemize}

\subsection*{6) FINAL}
\noindent\textbf{UNRESOLVED.}

\smallskip
\noindent\textbf{Best partial results/progress achieved here:}
\begin{itemize}
\item Gave a simple sufficient condition: arboricity $\le 2$ implies the diagonal property.
\item Built an explicit extremal-style construction with $2n-2$ edges (for $n\ge 4$), proving $f(n)\ge 2n-2$.
\item Summarized the current best known upper bounds and the remaining gap.
\end{itemize}

\noindent\textbf{Concrete next moves that look most promising:}
\begin{enumerate}[label=\arabic*.]
\item Try to prove a constant-average-degree forcing result: show that if $|E(G)|\ge Cn$ then some cycle has at least $|C|$ diagonals.
\item Alternatively, attempt explicit superlinear constructions satisfying the diagonal condition (e.g. iterated 2-arboric expansions) to refute linearity.
\item Investigate whether random-walk/expander methods used for the $n(\log n)^8$ upper bound can be strengthened to remove the polylog factor.
\end{enumerate}

\subsection*{7) COMPLETION ESTIMATE}
\noindent\textbf{COMPLETION: 40\%}

