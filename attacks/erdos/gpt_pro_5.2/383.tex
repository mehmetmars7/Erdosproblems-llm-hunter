\section*{Problem \#383}

\subsection*{1) FORMAL RESTATEMENT}

Fix an integer $k\ge 0$. For a prime $p$, define
\[
F_k(p)\ :=\ \prod_{i=0}^{k}(p^2+i).
\]
Let $P(n)$ denote the largest prime factor of $n$ (with $P(1):=1$).

\noindent\textbf{Question.} Is it true that for every fixed $k$ there exist infinitely many primes $p$ such that
\[
P\bigl(F_k(p)\bigr) = p\ ? 
\]

Equivalently (for $p>k$ so that $p\nmid p^2+i$ for $1\le i\le k$), the condition is:
\[
\max_{1\le i\le k} P(p^2+i)\ \le\ p.
\]
Indeed, the factor $p^2$ ensures that $p$ divides $F_k(p)$, and the requirement that no prime $>p$ divides any of the other factors forces $p$ to be the largest prime divisor.

\subsection*{2) QUICK LITERATURE/CONTEXT CHECK}

The Erd\H{o}s Problems forum thread for \#383 lists the problem as open and notes that a positive answer would imply the second part of Problem \#382.\footnote{\url{https://www.erdosproblems.com/forum/thread/383} (accessed 2026-01-17).}
It also gives a heuristic: the ``probability'' that an integer $n$ has no prime divisor $\ge n^{1/2}$ is $1-\log 2>0$, suggesting the answer should be yes.\footnote{Same source as above.}
A comment in the thread mentions a conditional approach for the first nontrivial case using conjectures about Newman--Shanks--Williams primes.\footnote{Same source as above; comment by user ``Dogmachine''.}

\subsection*{3) ATTACK PLAN}

\begin{enumerate}[leftmargin=2.5em]
\item Re-express the condition as a simultaneous smoothness requirement: $p^2+i$ must be $p$-smooth for $1\le i\le k$.
\item Produce computational evidence for small $k$ by searching primes $p$ up to various cutoffs.
\item Heuristically, model the events ``$p^2+i$ is $p$-smooth'' as roughly independent with probability $\approx 1-\log 2$; then the expected count up to $X$ is $\asymp \frac{X}{\log X}(1-\log 2)^k$, which diverges for fixed $k$.
\item The hard step is making any of this unconditional: it is a problem about smooth values of quadratic polynomials at prime arguments, well beyond current sieve technology in general.
\end{enumerate}

\subsection*{4) WORK}

\paragraph{Computational evidence.}
For each fixed $k$, we searched primes $p$ up to a cutoff $P_{\max}$ and checked whether $P(p^2+i)\le p$ holds for all $1\le i\le k$.
The table below records counts and the first few solutions found.

\medskip
\begin{center}
\begin{tabular}{r|r|l}
$k$ & search range & primes $p$ found (first few; count)\\\hline
1 & $p\le 50000$ & $7,41,43,47,73,\dots$ (count $1265$)\\
2 & $p\le 50000$ & $41,157,211,313,421,\dots$ (count $311$)\\
3 & $p\le 50000$ & $443,599,1229,1277,1301,\dots$ (count $87$)\\
4 & $p\le 50000$ & $8663,15361,16223,23873,\dots$ (count $11$)\\
5 & $p\le 50000$ & $15361,43777,44531,45131$ (count $4$)\\
6 & $p\le 200000$ & $125441,143881$ (count $2$)\\
7 & $p\le 200000$ & none found (count $0$)\\
\end{tabular}
\end{center}

\noindent These computations support (but do not prove) the conjecture that for each fixed $k$ there may be infinitely many such primes $p$.

\paragraph{Heuristic.}
Write $n=p^2+i\approx p^2$. The condition $P(n)\le p$ is essentially $P(n)\le n^{1/2}$.
The Dickman--de Bruijn heuristic for smooth numbers suggests
\[
\mathbb{P}(P(n)\le n^{1/2}) \approx \rho(2)=1-\log 2.
\]
Assuming (very optimistically) that the $k$ conditions for $i=1,\dots,k$ behave like independent random events of probability $1-\log 2$, one predicts that among primes $p\le X$ the expected number of solutions is
\[
\sim \frac{X}{\log X}\,(1-\log 2)^k,
\]
which tends to infinity with $X$ for each fixed $k$.

\subsection*{5) VERIFICATION}

\begin{itemize}[leftmargin=2em]
\item The equivalence ``$P(F_k(p))=p$ iff $P(p^2+i)\le p$ for all $1\le i\le k$'' holds for all primes $p>k$; for $p\le k$ one must treat separately because $p$ can divide some $p^2+i$.
The computational search handled the definition directly by checking prime factors of each $p^2+i$.
\item The computations above are finite and cannot distinguish ``infinitely many'' from ``finitely many but many''; they serve only as evidence.
\end{itemize}

\subsection*{6) FINAL}

\textbf{UNRESOLVED.}

\begin{enumerate}[leftmargin=2.5em]
\item[(i)] \textbf{Strongest partial result obtained.}
Reformulated the problem as requiring simultaneous $p$-smoothness of the $k$ consecutive integers $p^2+1,\dots,p^2+k$.
Provided explicit computational evidence: solutions exist for $k=1,2,3,4,5$ with many primes below $50000$, and for $k=6$ at least two primes below $200000$.
\item[(ii)] \textbf{First hard gap.}
No known unconditional method currently proves that infinitely many primes $p$ make even a single quadratic value $p^2+1$ have all prime factors $\le p$ (i.e.\ be $p$-smooth), let alone $k$ simultaneous values.
\item[(iii)] \textbf{Top 3 next moves.}
\begin{enumerate}[leftmargin=2.5em]
\item Attempt a lower-bound sieve for primes $p$ for which $p^2+i$ has no prime factor in $(p, p^{1+\varepsilon}]$ and then control larger prime factors via additional arguments (a difficult ``parity problem'' barrier is expected).
\item Seek conditional proofs under strong hypotheses on prime values/smooth values of polynomials (e.g.\ variants of Dickson/Schinzel, or very strong distributional conjectures).
\item Explore connections to known families where $p^2+i$ factors in a controlled way (e.g.\ algebraic identities or Pell-type parametrisations) that might force $p^2+i$ to be smooth for infinitely many $p$.
\end{enumerate}
\item[(iv)] \textbf{Why it might be true.}
Heuristics based on smooth numbers (and the constant $1-\log 2$) suggest a positive density of $p^2+i$ should have all prime factors $\le p$, and for fixed $k$ the expected number of primes $p\le X$ satisfying all $k$ constraints should still grow like $\frac{X}{\log X}(1-\log 2)^k$.
\end{enumerate}

\subsection*{7) COMPLETION ESTIMATE}
Estimated probability of a full resolution from these notes alone: \textbf{15\%}.

