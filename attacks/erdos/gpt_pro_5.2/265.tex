\section*{Erd\H{o}s Problem \#265}

\subsection*{1) FORMAL RESTATEMENT}

Find (or bound) the maximal possible growth rate of a sequence of integers $(a_n)_{n\ge 1}$ with $a_n\to\infty$ such that both series
\[
S_0:=\sum_{n=1}^{\infty}\frac{1}{a_n},
\qquad
S_1:=\sum_{n=1}^{\infty}\frac{1}{a_n-1}
\]
converge and are rational numbers (with the implicit requirement $a_n\neq 1$ so that $a_n-1\neq 0$).

\subsection*{2) QUICK LITERATURE/CONTEXT CHECK (only if browsing available)}

\begin{itemize}[leftmargin=2em]
\item (Cantor) The choice $a_n=\binom{n}{2}$ (for $n\ge 3$) gives both sums rational by telescoping.
\item (Erd\H{o}s folklore/criterion) A classical irrationality criterion states that if $a_n$ grows faster than doubly exponential in the sense $a_n^{1/2^n}\to\infty$, then $\sum 1/a_n$ is irrational; hence any sequence with $S_0\in\mathbb{Q}$ must satisfy $\limsup a_n^{1/2^n}<\infty$.
\item (Kova\v{c}--Tao) There exist sequences with \emph{doubly exponential} growth (i.e. $a_n^{1/\beta^n}\to\infty$ for some $\beta>1$) for which $\sum 1/(a_n+j)\in\mathbb{Q}$ simultaneously for $j=0,1,\dots,d-1$. In particular, the condition $a_n^{1/n}\to\infty$ is achievable, but it remains open whether one can achieve $\limsup a_n^{1/2^n}>1$ while keeping both sums rational.
\end{itemize}

\subsection*{3) ATTACK PLAN}

\begin{enumerate}[leftmargin=2.2em]
\item Verify Cantor's example explicitly by writing both series in partial fractions and telescoping.
\item Record the known upper bound (no faster than doubly exponential) via the classical irrationality criterion for $\sum 1/a_n$.
\item Record the known lower bound (doubly exponential growth is achievable) via the Kova\v{c}--Tao existence theorem.
\item Identify the remaining gap: determining the optimal doubly-exponential exponent (e.g. whether $\limsup a_n^{1/2^n}>1$ is possible).
\end{enumerate}

\subsection*{4) WORK (with full details)}

\subsubsection*{4.1. Cantor's polynomial-growth example}

Let
\[
 a_n:=\binom{n}{2}=\frac{n(n-1)}{2},\qquad n\ge 3.
\]
Then $a_n\to\infty$ and $a_n-1\neq 0$ for $n\ge 3$.

\begin{proposition}[First sum is rational]
\[\sum_{n=3}^{\infty}\frac{1}{\binom{n}{2}}\in\mathbb{Q}.
\]
\end{proposition}

\begin{proof}
For $n\ge 3$,
\[
\frac{1}{\binom{n}{2}}=\frac{2}{n(n-1)}=2\Bigl(\frac{1}{n-1}-\frac{1}{n}\Bigr).
\]
Hence the partial sums telescope:
\[
\sum_{n=3}^{N}\frac{1}{\binom{n}{2}}
=2\sum_{n=3}^{N}\Bigl(\frac{1}{n-1}-\frac{1}{n}\Bigr)
=2\Bigl(\frac{1}{2}-\frac{1}{N}\Bigr)
=1-\frac{2}{N}.
\]
Letting $N\to\infty$ gives $\sum_{n=3}^{\infty}\frac{1}{\binom{n}{2}}=1$.
\end{proof}

\begin{proposition}[Second sum is rational]
\[\sum_{n=3}^{\infty}\frac{1}{\binom{n}{2}-1}\in\mathbb{Q}.
\]
\end{proposition}

\begin{proof}
For $n\ge 3$,
\[
\binom{n}{2}-1=\frac{n(n-1)-2}{2}=\frac{(n-2)(n+1)}{2},
\]
so
\[
\frac{1}{\binom{n}{2}-1}=\frac{2}{(n-2)(n+1)}.
\]
We decompose into partial fractions:
\[
\frac{2}{(n-2)(n+1)}=\frac{2}{3}\Bigl(\frac{1}{n-2}-\frac{1}{n+1}\Bigr)
\qquad\text{(since }\frac{1}{n-2}-\frac{1}{n+1}=\frac{3}{(n-2)(n+1)}\text{).}
\]
Therefore
\[
\sum_{n=3}^{N}\frac{1}{\binom{n}{2}-1}
=\frac{2}{3}\sum_{n=3}^{N}\Bigl(\frac{1}{n-2}-\frac{1}{n+1}\Bigr).
\]
The sum telescopes, leaving only the first three positive terms:
\[
\sum_{n=3}^{N}\Bigl(\frac{1}{n-2}-\frac{1}{n+1}\Bigr)
=\Bigl(1+\frac12+\frac13\Bigr)-\Bigl(\frac{1}{N-1}+\frac{1}{N}+\frac{1}{N+1}\Bigr).
\]
Letting $N\to\infty$ yields
\[
\sum_{n=3}^{\infty}\frac{1}{\binom{n}{2}-1}
=\frac{2}{3}\Bigl(1+\frac12+\frac13\Bigr)
=\frac{2}{3}\cdot\frac{11}{6}
=\frac{11}{9}.
\]
\end{proof}

Thus polynomial growth is possible.

\subsubsection*{4.2. Known bounds on how fast $a_n$ can grow}

\begin{proposition}[No faster than doubly exponential (known criterion)]
Assume $(a_n)$ is strictly increasing and $\sum_{n\ge 1} 1/a_n\in\mathbb{Q}$. Then it cannot happen that $a_n^{1/2^n}\to\infty$.
\end{proposition}

\begin{proof}
A known irrationality criterion (attributed in the literature to Erd\H{o}s--Straus and related works) states:
\begin{quote}
If $(a_n)$ is a strictly increasing sequence of positive integers for which $\lim_{n\to\infty} a_n^{1/2^n}=\infty$, then $\sum_{n\ge 1} 1/a_n$ is irrational.
\end{quote}
Applying this contrapositive to the hypothesis $\sum 1/a_n\in\mathbb{Q}$ proves the claim.
\end{proof}

So $a_n$ cannot grow faster than doubly exponential.

\begin{proposition}[Doubly exponential growth is achievable (known construction)]
There exists $\beta>1$ and a strictly increasing integer sequence $(a_n)$ such that
\[
\lim_{n\to\infty} a_n^{1/\beta^n}=\infty,
\qquad
\sum_{n\ge 1}\frac{1}{a_n}\in\mathbb{Q},
\qquad
\sum_{n\ge 1}\frac{1}{a_n-1}\in\mathbb{Q}.
\]
\end{proposition}

\begin{proof}
This follows from the higher-dimensional existence theorem of Kova\v{c}--Tao on simultaneous rationality of the sums $\sum 1/(a_n+j)$ for consecutive shifts $j=0,1,\dots,d-1$ (take $d=2$ and then shift indices so that $a_n-1$ corresponds to $a_n+j$).
\end{proof}

\subsection*{5) VERIFICATION / SANITY CHECK}

\begin{itemize}[leftmargin=2em]
\item The Cantor example starts at $n=3$ to avoid the term $\binom{2}{2}-1=0$.
\item Both series telescope to explicit rationals $1$ and $11/9$.
\item The ``no faster than doubly exponential'' bound is consistent with the existence of shifted Sylvester-type examples having $a_n\asymp C^{2^n}$ and rational reciprocal sums.
\end{itemize}

\subsection*{6) FINAL}

\textbf{UNRESOLVED.}

\begin{itemize}[leftmargin=2em]
\item[(i)] \textbf{What is proved here:} explicit polynomial-growth examples exist (Cantor's $a_n=\binom{n}{2}$ with rational sums $1$ and $11/9$). Also, by known criteria, growth faster than doubly exponential is impossible if $\sum 1/a_n$ is rational.
\item[(ii)] \textbf{What remains open:} the optimal \emph{doubly exponential exponent}. In particular, it is unknown whether one can arrange both sums rational with $\limsup a_n^{1/2^n}>1$.
\item[(iii)] \textbf{Most plausible next step:} refine the existing doubly-exponential constructions to push the exponent toward $2$, or prove an obstruction showing $a_n^{1/2^n}\to 1$ must hold.
\item[(iv)] \textbf{Completion estimate:} closing the gap likely requires either a new rigidity principle for simultaneous Ahmes sums, or a substantially sharper construction; the remaining step is not a routine optimization of known arguments.
\end{itemize}

\subsection*{7) COMPLETION ESTIMATE}

To reach a full resolution one needs either:
\begin{enumerate}[leftmargin=2.2em]
\item an explicit construction achieving $\limsup a_n^{1/2^n}>1$ with both sums rational, or
\item a proof that any such sequence must satisfy $a_n^{1/2^n}\to 1$ (or at least $\limsup\le 1$).
\end{enumerate}

%%%%%%%%%%%%%%%%%%%%%%%%%%%%%%%%%%%%%%%%%%%%%%%%%%%%%%%%%%%%%%%%%%%%%%%%%%%%%%%
