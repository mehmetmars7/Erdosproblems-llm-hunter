
\subsection*{Erd\H{o}s Problem 688}

\paragraph{1) FORMAL RESTATEMENT.}
Fix an integer $n\ge 2$. For a real parameter $\epsilon\ge 0$, let
\[
\mathcal P_\epsilon(n):=\{\,p:\ p\text{ prime},\ n^\epsilon<p\le n\,\}.
\]
We say $\epsilon$ is \emph{admissible} if there exist residue classes $a_p\pmod p$ for every prime $p\in\mathcal P_\epsilon(n)$ such that
every integer $m\in[1,n]$ satisfies at least one congruence $m\equiv a_p\pmod p$ with $p\in\mathcal P_\epsilon(n)$.
Define
\[
\epsilon_n:=\sup\{\epsilon\ge 0:\ \epsilon\text{ is admissible}\}.
\]
(The strict inequality $n^\epsilon<p$ means the supremum need not be attained; the problem asks for estimates as $n\to\infty$.)

Question: estimate $\epsilon_n$; in particular, is $\epsilon_n=o(1)$?

\paragraph{2) QUICK LITERATURE/CONTEXT CHECK.}
The problem statement records Erd\H{o}s's lower bound
\[
\epsilon_n \gg \frac{\log\log\log n}{\log\log n}.
\]
No other external results are used or claimed here.

\paragraph{3) ATTACK PLAN.}
\begin{itemize}
\item Reformulate admissibility in terms of a shift $t$ modulo the product of the primes in $\mathcal P_\epsilon(n)$ (Lemma~688.1).
\item Derive simple necessary conditions from counting/coverage constraints (Lemma~688.2).
\item Use small-$n$ computations to sanity check the definition and see when admissibility first appears.
\end{itemize}

\paragraph{4) WORK.}
\medskip
\noindent\textbf{Lemma 688.1 (shift reformulation for a fixed prime set).}
Fix $n\ge 2$ and a set of primes $S\subseteq\{p:\ p\le n\}$, and let $P_S:=\prod_{p\in S}p$.
Choosing residue classes $\{a_p\pmod p\}_{p\in S}$ is equivalent (by CRT) to choosing a residue class $t\pmod{P_S}$ such that
$t\equiv -a_p\pmod p$ for all $p\in S$.
With this $t$, an integer $m\in[1,n]$ is covered (i.e.\ $m\equiv a_p\pmod p$ for some $p\in S$) if and only if $m+t$ is divisible by some
prime in $S$, equivalently
\[
\gcd(m+t,P_S)>1.
\]
In particular, $\epsilon$ is admissible (for the set $S=\mathcal P_\epsilon(n)$) if and only if there exists $t$ such that
\[
\gcd(t+1,P_S)>1,\ \gcd(t+2,P_S)>1,\ \dots,\ \gcd(t+n,P_S)>1.
\]

\emph{Proof.}
Identical to Lemma~687.1, but with the prime set $S$ in place of $\{p\le x\}$. \hfill$\square$

\medskip
\noindent\textbf{Lemma 688.2 (simple counting necessary condition).}
Let $S=\mathcal P_\epsilon(n)$ and suppose $\epsilon$ is admissible, witnessed by residue classes $\{a_p\pmod p\}_{p\in S}$.
Then
\[
\sum_{p\in S}\left\lceil\frac{n}{p}\right\rceil \ge n.
\]

\emph{Proof.}
For each $p\in S$, the chosen residue class $a_p\pmod p$ hits at most $\lceil n/p\rceil$ integers in $[1,n]$, because among any $p$ consecutive integers
there is at most one representative of a fixed residue class, and the interval $[1,n]$ has length $n$.
Let
\[
C_p:=\#\{\,m\in[1,n]:\ m\equiv a_p\pmod p\,\}.
\]
Then $C_p\le \lceil n/p\rceil$ for each $p$.
If every $m\in[1,n]$ is covered by at least one of the residue classes, then counting incidences gives
\[
n \le \sum_{p\in S} C_p \le \sum_{p\in S}\left\lceil\frac{n}{p}\right\rceil,
\]
which is the stated inequality. \hfill$\square$

\medskip
\noindent\textbf{FAST REALITY CHECK (local computation for small $n$).}
For each $n\le 20$, I tested admissibility for prime sets of the form $S=\{p:\ p_{\min}\le p\le n\}$ (all primes in an upper tail),
using the shift formulation (Lemma~688.1) and scanning residues modulo $P_S$ to find the maximum run of consecutive integers with gcd$>1$ with $P_S$.
Results:
\begin{itemize}
\item For $n=4,6,10$, \emph{no} choice of residue classes for \emph{all} primes $p\le n$ covers $[1,n]$ (verified by brute force over all residue choices).
Equivalently, even with $S=\{p\le n\}$, the maximum run length is $<n$.
\item For $n=3,5,7,8,9,11,\dots,20$, covering is possible, but in all feasible cases found, the smallest prime in $S$ must be $2$
(i.e.\ excluding $p=2$ makes the maximum run $<n$). For example:
\begin{verbatim}
n=17: S={3,5,7,11,13,17} gives maxrun=12 < 17  (so p_min=3 fails)
n=17: S={2,3,5,7,11,13,17} gives maxrun=25 >=17 (so p_min=2 works)
\end{verbatim}
\end{itemize}
These computations are only for small $n$ and do not address asymptotics.

\paragraph{5) VERIFICATION.}
\begin{itemize}
\item Lemma~688.1 is an exact CRT equivalence.
\item Lemma~688.2 is a necessary condition only (it ignores overlaps between the classes), so it cannot by itself prove admissibility.
\item The computed ``no'' instances ($n=4,6,10$) were verified by exhaustive search over all residue choices for all primes $\le n$.
\end{itemize}

\paragraph{FINAL.}
\noindent\textbf{UNRESOLVED}

\smallskip
\noindent(i) \textbf{Strongest proved partial result.}
Admissibility for $\epsilon$ is equivalent to the existence of a shift $t$ such that $t+1,\dots,t+n$ are all divisible by at least one prime in
$\mathcal P_\epsilon(n)$ (Lemma~688.1). Any admissible $\epsilon$ must satisfy the counting constraint
$\sum_{n^\epsilon<p\le n}\lceil n/p\rceil\ge n$ (Lemma~688.2).

\smallskip
\noindent(ii) \textbf{First gap (crisp statement).}
Prove any asymptotic upper bound on $\epsilon_n$ (e.g.\ $\epsilon_n=o(1)$) \emph{without} invoking literature beyond what is stated in the problem file.
Equivalently: show that if one excludes all primes $\le n^\epsilon$ for fixed $\epsilon>0$, then for $n$ large, some $m\in[1,n]$ avoids all chosen
classes modulo primes $>n^\epsilon$.

\smallskip
\noindent(iii) \textbf{Top 3 next moves.}
\begin{enumerate}
\item Use Lemma~688.2 together with explicit estimates for $\sum_{p>n^\epsilon}1/p$ (if allowed) to obtain a quantitative upper bound on admissible $\epsilon$.
\item In the shift model, study runs of length $n$ of integers all having a prime factor $>n^\epsilon$ and attempt to upper-bound their length using
sieve methods on $y$-rough numbers.
\item Extend computation to moderately larger $n$ (where $P_S$ remains tractable) to guess the growth of $\epsilon_n$ and identify extremal shifts $t$.
\end{enumerate}

\smallskip
\noindent(iv) \textbf{What a minimal counterexample would likely look like.}
If $\epsilon_n$ does not tend to $0$, there would exist a fixed $\epsilon_0>0$ and infinitely many $n$ with a shift $t$ such that every one of
$t+1,\dots,t+n$ has a prime divisor in $(n^{\epsilon_0},n]$; this would force unusually long intervals consisting entirely of integers with a relatively large
prime factor.


