
\noindent\textbf{FORMAL RESTATEMENT.}\\
Let $f:\mathbb{C}\to\mathbb{C}$ be an entire function and write $f^{(n)}$ for its $n$-th complex derivative ($n\in\mathbb{N}$).  For an increasing infinite sequence $n_1<n_2<\cdots$ of nonnegative integers, define
\[
Z_{(n_k)}(f)\;:=\;\big\{z\in\mathbb{C}:\exists k\ge 1\ \ f^{(n_k)}(z)=0\big\}\;=\;\bigcup_{k\ge1} \{z: f^{(n_k)}(z)=0\}.
\]
The \emph{literal statement} asks whether there exists a nonzero entire $f$ such that for \emph{every} infinite increasing sequence $(n_k)$, the set $Z_{(n_k)}(f)$ is dense in $\mathbb{C}$.

\medskip
\noindent\emph{Ambiguity/misstated intent.}  As noted in the problem text, if polynomials are allowed then the literal statement is trivial (any nonzero polynomial works). A minimal correction consistent with the surrounding discussion is:

\smallskip
\noindent\emph{Corrected statement (transcendental version).}  Does there exist a \emph{transcendental} entire $f$ (i.e. not a polynomial) with the same property?

\medskip
\noindent\textbf{QUICK LITERATURE/CONTEXT CHECK.}\\
The problem text reports that Erd\H{o}s claimed an affirmative solution in the early 1980s but gives no verifiable reference, and that the polynomial case is trivial. I will not use any literature beyond what is explicitly in the problem text.

\medskip
\noindent\textbf{ATTACK PLAN.}\\
\emph{Proof track (literal statement).}  (1) Prove a useful logical reformulation of the quantification over all infinite subsequences. (2) Exhibit an explicit nonzero entire function with the property (a polynomial).

\smallskip
\noindent\emph{Transcendental track.}  Use the reformulation to extract necessary structural consequences for any transcendental solution; attempt either a construction (e.g. by designing Taylor coefficients so that high derivatives have widely spread zeros) or an obstruction (show that every transcendental entire has an open disc avoiding zeros for infinitely many derivatives). I do not complete this track.

\medskip
\noindent\textbf{WORK.}\\
\emph{Phase 1: fast reality check (tiny examples).}
\begin{itemize}
\item $f(z)=e^z$: then $f^{(n)}(z)=e^z$ for all $n$, so $Z_{(n_k)}(f)=\varnothing$ for every sequence. Fails.
\item $f(z)=\sin z$: then $f^{(2m)}=\pm\sin z$ and $f^{(2m+1)}=\pm\cos z$. If we take the even subsequence $n_k=2k$, then $Z_{(n_k)}(f)=\pi\mathbb{Z}$, which is not dense. Fails.
\item Nonzero polynomial: succeeds (proved below).
\end{itemize}

\medskip
\noindent\textbf{Lemma 906.1 (cofinite reformulation).}  \emph{Fix an entire $f$. The following are equivalent:}
\begin{enumerate}
\item[(A)] \emph{For every infinite increasing sequence $(n_k)$, the set $Z_{(n_k)}(f)$ is dense in $\mathbb{C}$.}
\item[(B)] \emph{For every nonempty open set $U\subset\mathbb{C}$, the set}
\[A_U(f):=\{n\in\mathbb{N}: \exists z\in U\ \ f^{(n)}(z)=0\}\]
\emph{is cofinite (i.e. $\mathbb{N}\setminus A_U(f)$ is finite).}
\end{enumerate}

\noindent\textbf{Proof.}
(A)$\Rightarrow$(B): Suppose (B) fails for some nonempty open $U$. Then $B_U:=\mathbb{N}\setminus A_U(f)$ is infinite, so we can list it as a strictly increasing sequence $n_1<n_2<\cdots$. By definition of $B_U$, for every $k$ we have $f^{(n_k)}(z)\ne 0$ for all $z\in U$, hence $Z_{(n_k)}(f)\cap U=\varnothing$. Therefore $Z_{(n_k)}(f)$ cannot be dense, contradicting (A).

(B)$\Rightarrow$(A): Let $(n_k)$ be any infinite increasing sequence and let $U\subset\mathbb{C}$ be any nonempty open set. By (B), $A_U(f)$ is cofinite, so there exists $k$ with $n_k\in A_U(f)$. By definition of $A_U(f)$, there exists $z\in U$ with $f^{(n_k)}(z)=0$, i.e. $z\in Z_{(n_k)}(f)\cap U$. Since every nonempty open $U$ meets $Z_{(n_k)}(f)$, the set $Z_{(n_k)}(f)$ is dense. \hfill$\square$

\medskip
\noindent\textbf{Lemma 906.2 (polynomials solve the literal statement).}  \emph{If $f$ is a nonzero polynomial, then $f$ satisfies condition (A) of Lemma~906.1.}

\noindent\textbf{Proof.} Let $f$ be a nonzero polynomial of degree $d\ge 0$. For every $n>d$, the derivative $f^{(n)}$ is identically $0$, hence its zero set is all of $\mathbb{C}$. Given any infinite increasing sequence $(n_k)$, choose $k$ with $n_k>d$ (possible since $n_k\to\infty$). Then $\{z: f^{(n_k)}(z)=0\}=\mathbb{C}$, so $Z_{(n_k)}(f)=\mathbb{C}$ is dense. \hfill$\square$

\medskip
\noindent\textbf{Lemma 906.3 (accumulation of derivative zeros at every point).}  \emph{Assume $f$ satisfies (A) in Lemma~906.1. Then for every $z_0\in\mathbb{C}$ there exists a sequence $z_n\to z_0$ with $f^{(n)}(z_n)=0$ for all sufficiently large $n$.}

\noindent\textbf{Proof.} Fix $z_0$. For each integer $m\ge 1$, let $U_m:=\{z:|z-z_0|<1/m\}$. By Lemma~906.1(B), for each $m$ there exists $N_m$ such that for all $n\ge N_m$ there exists at least one zero of $f^{(n)}$ in $U_m$.

Define an increasing sequence $M(n)$ by
\[
M(n):=\max\{m\ge 1: N_m\le n\}
\]
(which is well-defined for all $n\ge N_1$ and satisfies $M(n)\to\infty$ as $n\to\infty$ because $N_m$ is finite for each $m$). For each $n\ge N_1$, choose $z_n\in U_{M(n)}$ such that $f^{(n)}(z_n)=0$; this is possible by the definition of $N_{M(n)}\le n$.

Then $|z_n-z_0|<1/M(n)\to 0$, hence $z_n\to z_0$ along $n\to\infty$. \hfill$\square$

\medskip
\noindent\textbf{VERIFICATION.}\\
\begin{itemize}
\item \emph{Quantifiers.} Lemma~906.1 explicitly checks both directions and uses only the fact that any infinite subset of $\mathbb{N}$ admits a strictly increasing enumeration.
\item \emph{Edge case $n_1=0$ vs $n_1\ge 1$.} Lemma~906.2 works either way because $n_k\to\infty$ and eventually exceeds the degree.
\item \emph{Nonzero constant polynomials.} If $f\equiv c\ne 0$, then for all $n\ge 1$ we have $f^{(n)}\equiv 0$, so the property still holds.
\item \emph{Lemma~906.3 does not force $f$ to be a polynomial.} The conclusion only gives zeros $z_n$ of $f^{(n)}$ approaching $z_0$, not that $f^{(n)}(z_0)=0$ for all large $n$.
\end{itemize}

\medskip
\noindent\textbf{FINAL.}\\
\textbf{UNRESOLVED} (for the \emph{corrected transcendental version}; the \emph{literal} statement is solved by Lemma~906.2).

(i) Strongest proved partial result: Lemma~906.1 gives an exact reformulation: the property holds iff for every nonempty open $U$, all but finitely many derivatives $f^{(n)}$ have a zero in $U$. Lemma~906.3 shows this forces zeros of $f^{(n)}$ to accumulate at \emph{every} point as $n\to\infty$.

(ii) First gap (crisp): Construct a \emph{transcendental} entire $f$ satisfying Lemma~906.1(B), or prove that no transcendental entire $f$ can satisfy Lemma~906.1(B).

(iii) Top 3 next moves:
\begin{enumerate}
\item Try a constructive diagonalization on Taylor coefficients: choose $a_m$ so that for each rational disc $U_j$ there is $N(j)$ such that for all $n\ge N(j)$, the polynomial truncation of $f^{(n)}$ on $U_j$ has a zero in $U_j$ and the tail is small enough for Rouch\'e to preserve that zero.
\item Try to prove an obstruction: show that for any transcendental entire $f$ there exists some disc $U$ and an infinite set $S\subset\mathbb{N}$ such that $f^{(n)}$ is zero-free on $U$ for all $n\in S$.
\item Test candidate explicit series (e.g. lacunary or rapidly decaying coefficients) computationally: for a proposed $f$, numerically locate zeros of $f^{(n)}$ for growing $n$ and check whether some fixed disc is avoided infinitely often.
\end{enumerate}

(iv) Minimal counterexample structure (to existence): a transcendental entire $f$ for which there exists at least one nonempty open disc $U$ such that the set $\{n: f^{(n)}\text{ has no zeros in }U\}$ is infinite (equivalently, violates Lemma~906.1(B) for that $U$).


