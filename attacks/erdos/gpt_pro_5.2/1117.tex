
1) FORMAL RESTATEMENT

Let \(f:\mathbb{C}\to\mathbb{C}\) be an entire function that is not a monomial (i.e., not of the form \(c z^m\) with \(c\ne 0\), \(m\in\mathbb{Z}_{\ge 0}\)). For each \(r>0\), define
\[
M(r):=\max_{|z|=r} |f(z)|
\]
and let
\[
\nu(r):=\#\{z\in\mathbb{C}: |z|=r\text{ and }|f(z)|=M(r)\}.
\]
(As noted in the problem statement, \(\nu(r)\) is finite for non-monomial \(f\); we reprove this below.)

Question 1: Can \(\limsup_{r\to\infty} \nu(r)=\infty\) occur?

Question 2: Can \(\liminf_{r\to\infty} \nu(r)=\infty\) occur?

2) QUICK LITERATURE/CONTEXT CHECK

I did not browse the web; I only record what is stated in the problem text:

- Herzog--Piranian (1968) showed Question 1 has answer ``yes''.
- Question 2 is still open; Gl\"ucksam--Pardo-Sim\'on (2024) give an ``approximate'' affirmative answer.

3) ATTACK PLAN

Proof track (for Question 2):

- Try to modify the Herzog--Piranian construction to control \(\nu(r)\) from below for \emph{all sufficiently large} radii \(r\), not just along a sparse sequence.

Disproof track (for Question 2):

- Try to prove that every non-monomial entire function admits infinitely many radii with \(\nu(r)\) bounded (e.g. \(\nu(r)\le C\) for infinitely many \(r\)), using growth/regularity theorems for entire functions.

Below I only establish basic finiteness and a polynomial bound; these do not resolve Question 2.

4) WORK

\textbf{Lemma 1117.1 (Real-analyticity of the boundary modulus).}
Let \(f\) be entire and fix \(r>0\). The function
\[
\theta\mapsto f(r e^{i\theta})
\]
is complex-analytic as a function of \(e^{i\theta}\), and in particular the real-valued function
\[
 g_r(\theta):=|f(r e^{i\theta})|^2
\]
is a \(2\pi\)-periodic real-analytic function of \(\theta\).

\emph{Proof.}
Write the power series \(f(z)=\sum_{n=0}^\infty c_n z^n\), which converges uniformly on \(|z|\le r\) since \(f\) is entire. Then
\[
 f(r e^{i\theta}) = \sum_{n=0}^\infty c_n r^n e^{i n\theta},
\]
with uniform convergence in \(\theta\). This is a uniformly convergent Fourier series with only nonnegative frequencies, hence defines a real-analytic function of \(\theta\) (indeed it extends to a complex-analytic function of \(\theta\) in a horizontal strip). Taking modulus squared preserves real-analyticity, so \(g_r\) is real-analytic and \(2\pi\)-periodic. \qed

\textbf{Lemma 1117.2 (Constant modulus on a circle forces a monomial).}
Let \(f\) be entire. Suppose there exists \(r>0\) and a constant \(C>0\) such that
\[
|f(z)|=C\quad\text{for all }|z|=r.
\]
Then \(f\) is a monomial: \(f(z)=c z^m\) for some integer \(m\ge 0\) and constant \(c\ne 0\).

\emph{Proof.}
Define \(g(z):=f(r z)/C\). Then \(g\) is entire and satisfies \(|g(z)|=1\) for all \(|z|=1\).
Because \(g\) is analytic on a neighborhood of the closed unit disk and not identically zero, its zero set inside \(|z|\le 1\) is finite (zeros of a nonzero analytic function are isolated, and a compact set contains only finitely many isolated points).
Let \(a_1,\dots,a_s\) be the zeros of \(g\) in \(|z|<1\), listed with multiplicity.
Define the finite Blaschke product
\[
B(z)=\prod_{j=1}^s \frac{z-a_j}{1-\overline{a_j}z}.
\]
This \(B\) is analytic on a neighborhood of \(|z|\le 1\), satisfies \(|B(z)|=1\) for \(|z|=1\), and has exactly the same zeros (with multiplicity) in \(|z|<1\) as \(g\).

Now \(h(z):=g(z)/B(z)\) is analytic and nonvanishing on a neighborhood of \(|z|\le 1\), so both \(h\) and \(1/h\) are analytic on \(|z|\le 1\). On \(|z|=1\) we have \(|h(z)|=|g(z)|/|B(z)|=1\). By the maximum modulus principle applied to \(h\) and to \(1/h\) on the unit disk, we obtain
\(|h(z)|\le 1\) and \(|h(z)|\ge 1\) for \(|z|<1\). Hence \(|h(z)|=1\) on the open unit disk. An analytic function with constant modulus on a domain is constant (e.g. by the open mapping theorem), so \(h\equiv e^{i\phi}\) for some real \(\phi\).
Thus, on \(|z|\le 1\),
\[
 g(z)=e^{i\phi} B(z).
\]
Since both sides are analytic on a neighborhood of \(|z|\le 1\), the identity theorem extends this equality to that neighborhood.

Finally, because \(g\) is entire, the right-hand side must be entire. A Blaschke factor \(\frac{z-a}{1-\overline{a}z}\) with \(a\ne 0\) has a pole at \(z=1/\overline{a}\), which lies outside the unit disk. Therefore, for \(e^{i\phi}B(z)\) to be entire, every \(a_j\) must equal \(0\). Hence \(B(z)=z^s\), and \(g(z)=e^{i\phi} z^s\) for all \(z\). Undoing the definition of \(g\), we get
\(f(z)=C e^{i\phi} (z/r)^s\), i.e. \(f(z)=c z^m\) with \(m=s\). \qed

\textbf{Lemma 1117.3 (Finiteness of \(\nu(r)\) for non-monomial entire functions).}
If \(f\) is entire and not a monomial, then for every \(r>0\) the number \(\nu(r)\) is finite.

\emph{Proof.}
Fix \(r\). By Lemma 1117.1, \(g_r(\theta)=|f(r e^{i\theta})|^2\) is a real-analytic periodic function. If \(g_r\) were constant, then \(|f|\) would be constant on \(|z|=r\), forcing \(f\) to be a monomial by Lemma 1117.2, contrary to hypothesis. Hence \(g_r\) is non-constant.
A non-constant real-analytic function on a compact interval has only finitely many critical points: its derivative \(g_r'(\theta)\) is real-analytic and not identically zero, so its zeros are isolated, hence finite in \([0,2\pi]\).
Every maximum point of \(g_r\) is a critical point, so there are finitely many maxima, hence finitely many points \(z\) with \(|z|=r\) attaining \(M(r)\). Therefore \(\nu(r)<\infty\). \qed

\textbf{Lemma 1117.4 (Polynomial bound).}
If \(f\) is a polynomial of degree \(n\) which is not a monomial, then \(\nu(r)\le 2n\) for every \(r>0\).

\emph{Proof.}
Write \(f(z)=\sum_{j=0}^n c_j z^j\). Then
\(f(r e^{i\theta})=\sum_{j=0}^n c_j r^j e^{ij\theta}\) is a trigonometric polynomial of degree \(n\), and
\(g_r(\theta)=|f(r e^{i\theta})|^2\) is a real trigonometric polynomial with frequencies in \([-n,n]\). Its derivative \(g_r'(\theta)\) is a real trigonometric polynomial of degree at most \(n\). If \(g_r'\equiv 0\), then \(g_r\) is constant, forcing \(f\) to have constant modulus on \(|z|=r\), hence to be a monomial by Lemma 1117.2 (applied to the entire function \(f\)), contradicting the hypothesis. Therefore \(g_r'\) is a nonzero trigonometric polynomial of degree at most \(n\), so it has at most \(2n\) zeros in \([0,2\pi)\).
The maximum points of \(g_r\) lie among the zeros of \(g_r'\), so there are at most \(2n\) such points. Hence \(\nu(r)\le 2n\). \qed

\textbf{FAST REALITY CHECK (a simple explicit example).}
Let \(f(z)=1+z+z^2\). For any \(r>0\) and \(|z|=r\), the triangle inequality gives
\(|f(z)|\le 1+|z|+|z|^2 = 1+r+r^2\), with equality only when \(z\) has argument \(0\) (since all coefficients are positive real). Thus \(M(r)=1+r+r^2\) and \(\nu(r)=1\) for all \(r>0\).
A numerical sampling check at \(r\in\{0.5,1,2\}\) (dense grid of angles) agrees with \(\nu(r)=1\).

5) VERIFICATION

- Lemma 1117.2: the key steps are (a) finitely many zeros in \(|z|\le 1\), (b) constructing the finite Blaschke product with the same interior zeros, and (c) applying maximum modulus to \(h\) and \(1/h\). Each step was justified.
- Lemma 1117.3 depends only on real-analyticity plus the non-constantness ensured by Lemma 1117.2.
- Lemma 1117.4 uses the fact that a nonzero trigonometric polynomial of degree \(n\) has at most \(2n\) zeros on a period; this is standard and follows from writing it as a polynomial in \(e^{i\theta}\) of degree at most \(n\) and applying the fundamental theorem of algebra.

6) FINAL

**UNRESOLVED**

(i) Strongest proved partial result: \(\nu(r)\) is finite for each \(r\) when \(f\) is not a monomial, and for polynomials of degree \(n\) one has the uniform bound \(\nu(r)\le 2n\) for all \(r\) (Lemmas 1117.3--1117.4).

(ii) First gap (crisp): Either construct an explicit entire non-monomial \(f\) such that \(\nu(r)\to\infty\) along all sufficiently large radii (i.e. \(\liminf_{r\to\infty}\nu(r)=\infty\)), or prove that every such \(f\) has infinitely many radii \(r\) with \(\nu(r)\) bounded.

(iii) Top 3 next moves:
1. Study (and attempt to re-derive) the Herzog--Piranian construction that makes \(\limsup\nu(r)=\infty\), and isolate exactly where the radii with large \(\nu(r)\) occur; try to densify that set of radii.
2. Attempt a ``stability'' lemma: if \(\nu(r)\) is large at some radius, can one force it to remain large on an interval of radii \([r,r+\Delta]\)? Proving such persistence would move from limsup to liminf.
3. Numerically explore candidate entire functions with many comparable dominant terms on \(|z|=r\) (e.g. lacunary power series with carefully tuned coefficients) and track \(\nu(r)\) as \(r\) varies, looking for patterns that might suggest a constructive approach to liminf.

(iv) Minimal counterexample structure (if liminf cannot be infinite): one would expect a theorem producing infinitely many radii where the maximum modulus is attained essentially by a single ``dominant direction'' (few maxima), perhaps tied to regularity of the maximum modulus function \(M(r)\) or to Wiman--Valiron type concentration; a minimal counterexample to the affirmative liminf conjecture would be an entire function whose boundary modulus does not admit such concentration at infinitely many radii.


