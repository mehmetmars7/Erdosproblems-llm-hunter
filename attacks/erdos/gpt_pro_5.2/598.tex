
Let $m$ be an infinite cardinal and $\kappa$ be the successor cardinal of $2^{\aleph_0}$. Can one colour the countable subsets of $m$ using $\kappa$ many colours so that every $X\subseteq m$ with $\lvert X\rvert=\kappa$ contains subsets of all possible colours?

\noindent\textbf{1) FORMAL RESTATEMENT}\par
Let $m$ be an infinite cardinal, viewed as an initial ordinal, and let
\[
\kappa := (2^{\aleph_0})^+,
\]
the successor cardinal of the continuum.
Let $[m]^{\aleph_0}$ denote the set of all countable subsets of $m$.

The question asks whether there exists a function (a colouring)
\[
c : [m]^{\aleph_0} \to \kappa
\]
such that for every subset $X\subseteq m$ with $|X|=\kappa$ we have
\[
 c\big([X]^{\aleph_0}\big) = \kappa,
\]
i.e., for every colour $\xi<\kappa$ there exists a countable $A\subseteq X$ with $c(A)=\xi$.

Edge case: if $m<\kappa$ then there is no $X\subseteq m$ of size $\kappa$, so the requirement is vacuous.

\noindent\textbf{2) QUICK LITERATURE/CONTEXT CHECK}\par
No additional results are stated in the problem text. To respect the integrity constraint, I do not import external set-theoretic results here.

\noindent\textbf{3) ATTACK PLAN}\par
\begin{itemize}
\item Proof-track idea: attempt an explicit construction of $c$ from a well-ordering of $[m]^{\aleph_0}$, ensuring each colour class ``hits'' every $X$ of size $\kappa$.
\item Disproof-track idea: given an arbitrary colouring $c$, try to diagonalize and build an $X$ of size $\kappa$ that omits some colour, by ensuring $[X]^{\aleph_0}$ avoids a specified colour class.
\end{itemize}
I did not reach a construction or a diagonal counterexample; the WORK gives necessary conditions and sanity checks.

\noindent\textbf{4) WORK}\par
\textbf{Fast reality check (vacuity).} If $m<\kappa$, then the requirement ``for every $X\subseteq m$ with $|X|=\kappa$'' has no instances, so any function $c:[m]^{\aleph_0}\to\kappa$ works.

\medskip
\noindent\textbf{Lemma 598.1 (vacuous case).}\\
If $m<\kappa$, then the answer to the literal question is ``yes'' (trivially).

\noindent\emph{Proof.} If $m<\kappa$ then there is no subset $X\subseteq m$ with $|X|=\kappa$, so the stated requirement on $X$ is vacuously satisfied by any colouring. \qed

\medskip
\noindent\textbf{Lemma 598.2 (each colour class must be $\kappa$-hitting, and large when $m=\kappa$).}\\
Assume $m=\kappa$ and let $c:[\kappa]^{\aleph_0}\to\kappa$ satisfy the requirement. For each colour $\xi<\kappa$ let
\[
F_{\xi}:=\{A\in[\kappa]^{\aleph_0}: c(A)=\xi\}.
\]
Then:
\begin{enumerate}
\item For every $X\subseteq\kappa$ with $|X|=\kappa$, there exists $A\in F_{\xi}$ with $A\subseteq X$ (i.e., $F_{\xi}$ hits every $\kappa$-subset).
\item In particular, $|F_{\xi}|\ge\kappa$ for every $\xi<\kappa$.
\end{enumerate}

\noindent\emph{Proof.}
(1) is just a restatement of the defining requirement applied to a fixed colour $\xi$.

(2) Suppose for contradiction that $|F_{\xi}|<\kappa$ for some $\xi$. Since each $A\in F_{\xi}$ is countable, the union
\[
U := \bigcup F_{\xi}
\]
has cardinality $|U|<\kappa$ because $\kappa$ is a successor cardinal and hence regular: a union of $<\kappa$ many countable sets has size $<\kappa$.
Therefore $X:=\kappa\setminus U$ still has $|X|=\kappa$.
But then no $A\in F_{\xi}$ can satisfy $A\subseteq X$ (every such $A$ is contained in $U$), contradicting (1).
Hence $|F_{\xi}|\ge\kappa$. \qed

\noindent\textbf{5) VERIFICATION}\par
\begin{itemize}
\item Lemma 598.2 uses regularity of $\kappa=(2^{\aleph_0})^+$, which is standard for successor cardinals: if $\lambda<\kappa$, then $\lambda\cdot\aleph_0<\kappa$.
\item The vacuous case $m<\kappa$ is a genuine ambiguity/edge case in the literal wording; in most nontrivial readings one assumes $m\ge\kappa$.
\end{itemize}

\noindent\textbf{6) FINAL}\par
\textbf{UNRESOLVED}

(i) \emph{Strongest proved partial result.} If $m<\kappa$ the requirement is vacuous (Lemma 598.1). If $m=\kappa$ and such a colouring exists, then each colour class must be $\kappa$-hitting and has size at least $\kappa$ (Lemma 598.2).

(ii) \emph{First gap (crisp).} For the nonvacuous case $m\ge\kappa$, either construct a colouring $c:[m]^{\aleph_0}\to\kappa$ with $c([X]^{\aleph_0})=\kappa$ for every $X\subseteq m$ of size $\kappa$, or construct (from an arbitrary $c$) an $X$ of size $\kappa$ and a colour $\xi$ such that no countable subset of $X$ has colour $\xi$.

(iii) \emph{Top 3 next moves.}
\begin{enumerate}
\item Diagonal attempt: given $c$, try to build $X={\langle x_{\xi}:\xi<\kappa\rangle}$ recursively so that for a fixed colour $\xi_*$ every candidate $A\in F_{\xi_*}$ is blocked by omitting at least one element of $A$ from $X$.
\item Constructive attempt: explicitly partition $[m]^{\aleph_0}$ into $\kappa$ many families each of which hits every $X$ of size $\kappa$; Lemma 598.2 shows each family must have size at least $\kappa$.
\item Specialize to $m=\kappa$ first and attempt to exploit the well-ordering of $[\kappa]^{\aleph_0}$ to define $c$ by a transfinite recursion guaranteeing the hitting property for each colour.
\end{enumerate}

(iv) \emph{What a minimal counterexample would look like.} A counterexample (to existence) would be a proof that for every colouring $c:[m]^{\aleph_0}\to\kappa$ there exists some $X\subseteq m$ with $|X|=\kappa$ and some colour $\xi<\kappa$ such that $F_{\xi}$ fails to hit $X$ (i.e., no $A\subseteq X$ has colour $\xi$).


