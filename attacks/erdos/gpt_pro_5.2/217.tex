\section*{Erd\H{o}s Problem \#217}

\subsection*{1. Formal Restatement}
Let $n\in\mathbb{N}$ and let $S\subset\mathbb{R}^2$ be a set of $n$ points in \emph{general position}, meaning:
\begin{enumerate}
  \item no three points of $S$ are collinear;
  \item no four points of $S$ are concyclic.
\end{enumerate}
Let
\[D(S):=\{d(p,q): p,q\in S,\ p\neq q\}\]
be the set of \emph{distinct} pairwise distances determined by $S$.

\noindent\textbf{Problem \#217.}
For which $n$ does there exist $S\subset\mathbb{R}^2$ in general position with
\begin{enumerate}
  \item $|D(S)|=n-1$ (exactly $n-1$ distinct distances), and
  \item there is an ordering $\delta_1,\dots,\delta_{n-1}$ of the elements of $D(S)$ such that the distance $\delta_i$ occurs \emph{exactly $i$ times} among the $\frac{n(n-1)}{2}$ unordered pairs of points in $S$.
\end{enumerate}
(Note that the total number of pairs is $\frac{n(n-1)}{2}=\sum_{i=1}^{n-1} i$, so these multiplicities exhaust all pairs.)

\subsection*{2. Quick Literature/Context Check}
The prompt states:
\begin{itemize}
  \item There is an explicit example for $n=4$ (``an isosceles triangle with the point in the centre'').
  \item For $n=5$ there is a construction due to Pomerance, and Pal\'asti proved existence for all $n\le 8$.
  \item Erd\H{o}s believed nonexistence for all sufficiently large $n$; one route to prove this would be via strong lower bounds on the number of distinct distances determined by $n$ points in general position.
\end{itemize}
This strongly suggests the classification of all $n$ is open.

\subsection*{3. Attack Plan}
\textbf{Proof approach (construct for many $n$).}
Try to build crescent configurations inductively:
\begin{itemize}
  \item Start from a known small configuration.
  \item Add a point so that it introduces exactly one new distance and increments multiplicities in a controlled way.
\end{itemize}
The challenge is simultaneously preserving general position and achieving the exact multiplicity pattern $1,2,\dots,n-1$.

\textbf{Disproof approach (show impossible for large $n$).}
Use lower bounds on the number of distinct distances in general position. If one could prove that any $n$ points in general position determine at least $n$ distinct distances for all sufficiently large $n$, then crescent configurations (which have only $n-1$ distinct distances) would be impossible for large $n$.

\subsection*{4. Work}
I can fully verify the claimed example for $n=4$ (and note the trivial $n=3$ case), but I cannot determine existence/nonexistence for all $n$.

\medskip
\noindent\textbf{Proposition 4.1 ($n=3$ sanity check).}
For $n=3$, an isosceles triangle with distinct side lengths gives $|D(S)|=2=n-1$ and multiplicities $1$ and $2$.

\smallskip
\noindent\emph{Proof.}
Let $S=\{A,B,C\}$ with $AB=AC\neq BC$. Then the distinct distances are $\{AB,BC\}$; the distance $AB$ occurs twice (pairs $\{A,B\},\{A,C\}$) and $BC$ occurs once.
General-position constraints are vacuous for $n=3$ beyond ``not collinear'', which is satisfied by any nondegenerate triangle.
\hfill$\square$

\medskip
\noindent\textbf{Proposition 4.2 (Explicit crescent configuration for $n=4$).}
Let
\[B=(-1,0),\qquad C=(1,0),\qquad A=(0,2).
\]
Let $O$ be the circumcenter of $\triangle ABC$; for these coordinates one computes
\[O=\Bigl(0,\frac{3}{4}\Bigr).
\]
Set $S:=\{A,B,C,O\}$.
Then $S$ is in general position and determines $3$ distinct distances whose multiplicities are $1,2,3$.

\smallskip
\noindent\emph{Proof.}
\underline{Step 1: compute the three distance values and multiplicities.}
First,
\[BC=2.
\]
Next,
\[AB^2=(0+1)^2+(2-0)^2=1+4=5,\quad AC^2=(0-1)^2+(2-0)^2=5,
\]
so $AB=AC=\sqrt{5}$.
Finally, since $O$ is the circumcenter, the three distances $OA,OB,OC$ are equal to the circumradius $R$.
Because $A$ and $O$ lie on the $y$-axis,
\[R=OA=\left|2-\frac{3}{4}\right|=\frac{5}{4},
\]
and one may check similarly that $OB=OC=5/4$.
Thus the multiset of distances among the $6$ pairs consists of:
\begin{itemize}
  \item $2$ occurring once (pair $\{B,C\}$),
  \item $\sqrt{5}$ occurring twice (pairs $\{A,B\},\{A,C\}$),
  \item $5/4$ occurring three times (pairs $\{O,A\},\{O,B\},\{O,C\}$).
\end{itemize}
Hence there are $3=n-1$ distinct distances with multiplicities $1,2,3$.

\underline{Step 2: general position.}
No three of $A,B,C$ are collinear since they form a nondegenerate triangle.
The point $O=(0,3/4)$ is not on the line $BC$ (which has equation $y=0$), so $B,C,O$ are not collinear.
Also $A$ is not on the line through $B$ and $O$ (slope $3/4$) nor on the line through $C$ and $O$ (slope $-3/4$), so no triple involving $O$ is collinear.

For concyclicity: the points $A,B,C$ determine a unique circle, namely their circumcircle, whose center is $O$ and whose radius is $R=5/4$.
Since $d(O,O)=0\neq R$, the point $O$ is not on this circle. Therefore $A,B,C,O$ are not concyclic.
\hfill$\square$

\medskip
\noindent\textbf{Sticking point for \#217.}
Beyond small $n$, achieving the exact multiplicity pattern appears to require delicate global control of the distance multiset; no general inductive step is provided here.

\subsection*{5. Verification}
\begin{itemize}
  \item The $n=4$ configuration was checked explicitly: all pairwise distances computed, multiplicities matched, and the general-position conditions verified.
  \item Edge cases: for $n<4$, the ``no four concyclic'' condition is vacuous; for $n=3$ the required multiplicities $1,2$ are achievable as shown.
\end{itemize}

\subsection*{6. FINAL}
\textbf{UNRESOLVED.}
\begin{enumerate}
  \item[(i)] \textbf{Strongest fully proved partial result obtained here:}
  An explicit crescent configuration exists for $n=4$ (Proposition 4.2), and the multiplicity condition is trivially achievable for $n=3$ (Proposition 4.1).
  \item[(ii)] \textbf{First gap blocking completion:}
  No method is given to either construct crescent configurations for all $n$ (or all $n\le N$) or to prove a lower bound on distinct distances in general position that would rule them out for large $n$.
  \item[(iii)] \textbf{Most promising next moves:}
  \begin{itemize}
    \item Develop or use stronger lower bounds for the number of distinct distances determined by $n$ points under the ``no three collinear/no four concyclic'' constraints, aiming to prove $|D(S)|\ge n$ for large $n$.
    \item Search for an inductive ``add one point'' mechanism that adjusts distance multiplicities in the required way while maintaining general position.
    \item Translate the multiplicity constraints into constraints on distance graphs (edge-colored complete graphs) and use extremal graph theory to force forbidden geometric patterns.
  \end{itemize}
  \item[(iv)] \textbf{Conjectured final answer and minimal counterexample shape:}
  I conjecture the answer is: \textbf{such configurations exist only for bounded $n$} (in line with Erd\H{o}s's belief).
  A minimal counterexample to this conjecture would be an explicit crescent configuration with $n=9$ in general position (since existence is known up to $n=8$ per the prompt).
\end{enumerate}

\subsection*{7. Completion Estimate}
\textbf{20\%}. Small examples are verified, but there is no progress toward the main classification ``for which $n$'' beyond what is already indicated in the prompt.

