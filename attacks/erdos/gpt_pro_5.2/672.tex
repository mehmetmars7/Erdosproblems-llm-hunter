
1) FORMAL RESTATEMENT

Let $k\ge 4$ be an integer. Let $n,d\in\mathbb Z_{\ge 1}$ with $\gcd(n,d)=1$. Consider the arithmetic progression
\[
 n,\ n+d,\ \dots,\ n+(k-1)d
\]
of length $k$, and its product
\[
 P(n,d,k):=\prod_{i=0}^{k-1}(n+id)\in\mathbb Z_{\ge 1}.
\]
Question: Can $P(n,d,k)$ be a perfect power, i.e. can there exist integers $y\ge 1$ and $\ell\ge 2$ such that
\[
 P(n,d,k)=y^{\ell}?
\]

2) QUICK LITERATURE/CONTEXT CHECK

As stated in the problem text:
- Erd\H{o}s--Selfridge (1975) proved that the product of consecutive integers ($d=1$) is never a perfect power.
- There are infinitely many $(n,d)$ with $\gcd(n,d)=1$ such that $n(n+d)(n+2d)$ is a square (Pell-equation phenomenon), so $k=3$ admits squares.
- Results in the text rule out many cases, including impossibility for $4\le k\le 34$ (as quoted).

I do not use any other external results here.

3) ATTACK PLAN

Proof-oriented ideas:
- Use gcd structure of terms in a primitive arithmetic progression ($\gcd(n,d)=1$) to control how primes can appear across different factors.
- Use the constraint that primes $p>k-1$ can divide at most one term, forcing their exponents to be multiples of $\ell$ within a single term.

Disproof-oriented ideas:
- Search for explicit counterexamples for small $k$ by brute force.

4) WORK

FAST REALITY CHECK (brute force)

I searched for solutions with $k\in\{4,5,6\}$, $1\le n\le 200$, $1\le d\le 50$, $\gcd(n,d)=1$, checking whether $\prod_{i=0}^{k-1}(n+id)$ is a perfect power (exponent $\ge2$). No solutions were found.

I extended the search for $k=4$ to $1\le n\le 1000$, $1\le d\le 200$ (still $\gcd(n,d)=1$), again finding no solutions.

Lemma 672.1 (primes dividing $d$ divide no term)

Assume $\gcd(n,d)=1$. Then for every integer $i\ge 0$,
\[
\gcd(n+id,\ d)=1.
\]
In particular, if a prime $p$ divides $d$, then $p$ divides none of the terms $n+id$.

Proof.
Using the Euclidean algorithm,
\[
\gcd(n+id,d)=\gcd(n,d),
\]
because $\gcd(n+id,d)=\gcd(n+id-id,d)=\gcd(n,d)$. Since $\gcd(n,d)=1$, the claim follows.


Lemma 672.2 (large primes cannot hit two terms)

Assume $\gcd(n,d)=1$ and let $p$ be a prime. Suppose $p$ divides both $n+id$ and $n+jd$ for some integers $0\le i<j\le k-1$. Then $p\mid (j-i)$. In particular, if $p>k-1$ then $p$ divides at most one of the $k$ terms.

Proof.
If $p\mid(n+id)$ and $p\mid(n+jd)$, then $p$ divides their difference
\[
(n+jd)-(n+id)=(j-i)d.
\]
By Lemma 672.1, $\gcd(n+id,d)=1$, so no prime divisor of $n+id$ can divide $d$. In particular, $p\nmid d$. Since $p\mid (j-i)d$ and $p\nmid d$, we conclude $p\mid (j-i)$.

If $p>k-1$, then $1\le j-i\le k-1<p$, so $p\mid (j-i)$ is impossible unless $j-i=0$, contradicting $i<j$. Hence $p$ cannot divide two distinct terms.


Lemma 672.3 (prime-power exponent is localized for $p>k-1$)

Assume $\gcd(n,d)=1$ and suppose
\[
\prod_{i=0}^{k-1}(n+id)=y^{\ell}
\]
for some integers $y\ge 1$ and $\ell\ge 2$.
Let $p$ be a prime with $p>k-1$. Then there exists at most one index $i\in\{0,\dots,k-1\}$ with $p\mid(n+id)$, and if such $i$ exists then
\[
 v_p(n+id)\equiv 0\pmod{\ell}.
\]

Proof.
By Lemma 672.2, $p$ divides at most one term, so there is at most one such $i$.
If $p$ divides $n+id$, then the exponent of $p$ in the product is exactly $v_p(n+id)$ (since no other factor contributes $p$). But the product equals $y^{\ell}$, so every prime exponent in it is a multiple of $\ell$. Hence $v_p(n+id)\equiv 0\pmod{\ell}$.


5) VERIFICATION

- Lemma 672.1: uses only $\gcd(n,d)=1$ and the identity $\gcd(n+id,d)=\gcd(n,d)$.
- Lemma 672.2: the step "$p\nmid d$" uses Lemma 672.1 correctly.
- Lemma 672.3: uses unique contribution of $p$ to the valuation when $p>k-1$.

Sanity check: if $d=1$ (consecutive integers) then Lemma 672.2 reduces to the familiar fact that a prime $p>k-1$ can divide at most one of $n,n+1,\dots,n+k-1$.

6) FINAL

\textbf{UNRESOLVED}

(i) Strongest proved partial result:

In a primitive progression ($\gcd(n,d)=1$), any prime $p>k-1$ divides at most one term, so in any hypothetical perfect-power identity $\prod_{i=0}^{k-1}(n+id)=y^{\ell}$, the $p$-adic valuation of that term must be a multiple of $\ell$.

(ii) First gap (crisp):

I cannot turn the localization of large prime powers (Lemma 672.3) into a contradiction for general $k\ge 4$.

(iii) Top 3 next moves:

1. Combine Lemma 672.3 with a lower bound guaranteeing existence of a prime $p>k-1$ dividing exactly one term to exponent $1$ (or to an exponent not divisible by $\ell$). This would require a "large prime factor" result for at least one term in the progression.
2. For fixed $k$, push computation further (larger $n,d$) and factor products to look for patterns in near-misses (e.g. products that are almost squares).
3. Investigate constraints coming from small primes $p\le k-1$ that can divide multiple terms; one would like to show they cannot supply the needed exponents to make all terms combine into an $\ell$th power.

(iv) Minimal counterexample structure:

A minimal counterexample would be a triple $(n,d,k)$ with $\gcd(n,d)=1$, $k\ge 4$, such that the product is an $\ell$th power with $\ell\ge 2$, and with all primes $p>k-1$ appearing only with exponents divisible by $\ell$ inside individual terms. One expects such a counterexample (if it exists) would force each term to be unusually "powerful" with respect to its large prime factors.


