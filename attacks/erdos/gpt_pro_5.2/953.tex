% Erdos Problem #953
% URL: https://www.erdosproblems.com/953

\section*{Erd\H{o}s Problem \#953}

\subsection*{FORMAL RESTATEMENT}
Fix $r>0$ and let $D_r:=\{x\in\mathbb R^2:|x|<r\}$.
Let $A\subset D_r$ be Lebesgue measurable and assume that for all distinct $a,b\in A$,
\[
|a-b|\notin \mathbb Z.
\]
Let $m(\cdot)$ denote planar Lebesgue measure.
Question: how large can $m(A)$ be as a function of $r$?

\subsection*{QUICK LITERATURE/CONTEXT CHECK}
The extracted text states that the trivial upper bound is $O(r)$ and mentions an unpublished remark attributed to S\'ark\"{o}zi and a lower bound adaptation giving $\gg r^{0.26}$.
I will only prove the trivial $O(r)$ bound (with an explicit constant) and record the trivial small-$r$ lower bound.

\subsection*{ATTACK PLAN}
\begin{itemize}
\item Reduce to a 1D statement: for each horizontal line, the intersection of $A$ with that line has no integer distances, hence its 1D measure is $\le 1$.
\item Integrate the 1D bound over $y\in[-r,r]$ by Fubini to get $m(A)\le 2r$.
\item Sanity-check small $r$ (e.g. $r<1/2$) where the whole disk works.
\end{itemize}

\subsection*{WORK}
\noindent\textbf{Fast reality check.}
If $0<r<\tfrac12$, then taking $A=D_r$ gives $|a-b|<1$ for all $a,b\in A$, hence no integer distances. So $m(A)$ can be as large as $\pi r^2$ for small $r$.

\medskip
\noindent\textbf{Lemma 953.1 (1D measure bound for avoiding integer distances).}
Let $B\subset\mathbb R$ be Lebesgue measurable and suppose that for all distinct $x,y\in B$,
\[|x-y|\notin\mathbb Z.\]
Then the 1D Lebesgue measure $m_1(B)\le 1$.

\medskip
\noindent\emph{Proof.}
For each integer $k\in\mathbb Z$, define
\[
B_k := (B\cap [k,k+1)) - k \ \subseteq [0,1).
\]
Then $m_1(B_k)=m_1(B\cap [k,k+1))$ by translation invariance.
We claim the sets $\{B_k\}_{k\in\mathbb Z}$ are pairwise disjoint.
Indeed, if $t\in B_k\cap B_\ell$, then there exist $x\in B\cap[k,k+1)$ and $y\in B\cap[\ell,\ell+1)$ with $t=x-k=y-\ell$.
Thus $x-y=k-\ell\in\mathbb Z$.
If $k\ne \ell$ then $x\ne y$ and this contradicts the hypothesis.
Hence $B_k\cap B_\ell=\emptyset$ for $k\ne\ell$.

Since all $B_k\subseteq[0,1)$ are disjoint and $\mathbb Z$ is countable,
\[
\sum_{k\in\mathbb Z} m_1(B_k)= m_1\Bigl(\bigsqcup_{k\in\mathbb Z} B_k\Bigr)\le m_1([0,1))=1.
\]
But
\[
\sum_{k\in\mathbb Z} m_1(B_k)=\sum_{k\in\mathbb Z} m_1(B\cap[k,k+1))=m_1(B).
\]
Therefore $m_1(B)\le1$.
\hfill $\square$

\medskip
\noindent\textbf{Lemma 953.2 (trivial planar upper bound $m(A)\le 2r$).}
Let $A\subset D_r$ be measurable with no integer distances. Then
\[
m(A)\le 2r.
\]

\medskip
\noindent\emph{Proof.}
For each $y\in\mathbb R$, define the horizontal section
\[
A_y:=\{x\in\mathbb R:(x,y)\in A\}.
\]
By Fubini's theorem,
\[
m(A)=\int_{\mathbb R} m_1(A_y)\,dy.
\]
If $(x_1,y)$ and $(x_2,y)$ are distinct points of $A$ on the same horizontal line, their Euclidean distance is $|x_1-x_2|$.
Since $A$ has no integer distances, it follows that for each fixed $y$, the measurable set $A_y\subset\mathbb R$ satisfies the 1D hypothesis of Lemma 953.1.
Hence $m_1(A_y)\le1$ for all $y$.

Also, since $A\subset D_r$, we have $A_y=\emptyset$ unless $|y|<r$.
Therefore
\[
m(A)=\int_{-r}^{r} m_1(A_y)\,dy\le \int_{-r}^{r} 1\,dy=2r.
\]
\hfill $\square$

\subsection*{VERIFICATION}
\begin{itemize}
\item Lemma 953.1: checked disjointness of $B_k$ uses only the fact that equal fractional parts in different integer blocks would give an integer difference.
\item Lemma 953.2: checked that distance along horizontal line equals $|x_1-x_2|$ exactly, so integer distances in 2D would be forced by integer differences in the 1D sections.
\item Edge case: if $r<1/2$, the lower bound $m(A)\ge \pi r^2$ from $A=D_r$ is consistent with the upper bound $2r$.
\end{itemize}

\subsection*{FINAL}
**UNRESOLVED**

(i) \textbf{Strongest proved partial result.}
We proved an explicit ``trivial'' upper bound $m(A)\le 2r$ for any measurable $A\subset D_r$ avoiding integer distances (Lemma 953.2). For $r<1/2$, one can achieve $m(A)=\pi r^2$ by taking $A=D_r$.

(ii) \textbf{First gap.}
Improve the growth-rate bounds: either produce constructions with $m(A)\gg r^{\alpha}$ for larger $\alpha$ (or even $\gg r$), or prove a better upper bound $o(r)$ or $r^{1-\varepsilon}$.

(iii) \textbf{Top 3 next moves.}
\begin{enumerate}
\item Explore lower-bound constructions by placing many small discs with radii shrinking to control distances between different components, and optimize the total area under the ``avoid all integers'' constraint.
\item Attempt Fourier-analytic bounds: interpret the forbidden distances as constraints on the autocorrelation of $1_A$ and exploit positivity/oscillation of Bessel functions for integer radii.
\item Investigate whether one can strengthen Lemma 953.1 by considering multiple directions (not just horizontal) to improve the constant or power of $r$.
\end{enumerate}

(iv) \textbf{What a minimal counterexample would likely look like.}
If one hopes for an upper bound $o(r)$, a near-extremal set would likely be ``essentially 1-dimensional'': concentrated near a curve or union of thin strips so that typical pairwise distances avoid integers. If one hopes for a linear lower bound $\gg r$, a construction would need to arrange many components while controlling the dense set of inter-component distances so that none hit an integer.

