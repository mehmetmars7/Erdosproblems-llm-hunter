\section*{Problem 389. A divisibility problem for two consecutive blocks}

\subsection*{Formal restatement}
For each integer $n\ge 1$, does there exist an integer $k\ge 1$ such that
\begin{equation}
\label{eq:P389-main}
\prod_{i=0}^{k-1}(n+i)\;\Big|\;\prod_{i=0}^{k-1}(n+k+i)
\qquad\text{in }\mathbb Z?
\end{equation}
Equivalently,
\[
 n(n+1)\cdots(n+k-1) \mid (n+k)(n+k+1)\cdots(n+2k-1).
\]
Examples given: for $n=2$ one can take $k=5$, and for $n=3$ one can take $k=4$.

\subsection*{Quick literature/context check (browsing available)}
The Erd\H{o}s Problems site (Problem \#389) records the question and notes that the smallest such $k$ (if it exists) grows quickly; computed values for $n$ up to $18$ are tabulated in OEIS A375071 (with an index shift).
No general proof (or counterexample) is indicated there.

\subsection*{Attack plan}
\begin{enumerate}[leftmargin=2em]
\item Rewrite \eqref{eq:P389-main} into equivalent forms (factorials / binomial coefficients) to expose structure.
\item Use $p$-adic valuations (Legendre/Kummer) to express the divisibility condition prime-by-prime.
\item Check whether a simple constructive choice of $k$ (e.g. depending on $n$ via lcm/factorials/primorials) forces the valuation inequalities.
\end{enumerate}

\subsection*{Work}
\paragraph{Rewriting via factorials.}
Let
\[
L(n,k):=\prod_{i=0}^{k-1}(n+i)=\frac{(n+k-1)!}{(n-1)!},
\qquad
R(n,k):=\prod_{i=0}^{k-1}(n+k+i)=\frac{(n+2k-1)!}{(n+k-1)!}.
\]
Then \eqref{eq:P389-main} is the condition $L(n,k)\mid R(n,k)$, i.e.
\begin{equation}
\label{eq:P389-ratio}
\frac{R(n,k)}{L(n,k)}
=\frac{(n-1)!(n+2k-1)!}{\bigl((n+k-1)!\bigr)^2}\in\mathbb Z.
\end{equation}

\paragraph{Binomial-coefficient form.}
Using
\[
\binom{n+k-1}{k}=\frac{(n+k-1)!}{(n-1)!\,k!}
\quad\text{and}\quad
\binom{n+2k-1}{k}=\frac{(n+2k-1)!}{(n+k-1)!\,k!},
\]
we have
\[
L(n,k)=k!\binom{n+k-1}{k},\qquad R(n,k)=k!\binom{n+2k-1}{k}.
\]
Therefore
\begin{equation}
\label{eq:P389-binomial}
L(n,k)\mid R(n,k)
\iff
\binom{n+k-1}{k}\;\Big|\;\binom{n+2k-1}{k}.
\end{equation}
This binomial formulation is often more convenient for $p$-adic valuation work.

\paragraph{Prime-by-prime formulation.}
Let $v_p(\cdot)$ be the $p$-adic valuation. Then \eqref{eq:P389-main} is equivalent to
\begin{equation}
\label{eq:P389-valuation}
\forall p\ \text{prime}:\quad
v_p\bigl(R(n,k)\bigr)\ge v_p\bigl(L(n,k)\bigr).
\end{equation}
Equivalently, using \eqref{eq:P389-ratio},
\[
\forall p:\quad
v_p((n+2k-1)!)+v_p((n-1)!)\ge 2v_p((n+k-1)!).
\]
By Legendre's formula $v_p(N!)=\sum_{j\ge 1}\lfloor N/p^j\rfloor$, this becomes an explicit (but complicated) family of inequalities in $k$.

\paragraph{What I can verify unconditionally.}
\begin{itemize}[leftmargin=2em]
\item The examples: $2\cdot 3\cdot 4\cdot 5\cdot 6=720$ divides $7\cdot 8\cdot 9\cdot 10\cdot 11=55440$ (quotient $77$), and $3\cdot 4\cdot 5\cdot 6=360$ divides $7\cdot 8\cdot 9\cdot 10=5040$ (quotient $14$).
\item Computation (separate) shows the smallest $k$ for $n=4$ is already large ($k=207$), consistent with OEIS A375071 (after the index shift).
\end{itemize}

\paragraph{Why the natural ``one-line'' constructions fail.}
A tempting sufficient condition for \eqref{eq:P389-main} would be a bijection between the left block and right block such that each left term divides its matched right term. However, because the right block is only a shift by $k$, divisibility of the form $(n+i)\mid(n+k+j)$ forces rigid congruences and fails for most terms. Also, choices like $k=\mathrm{lcm}(1,\dots,n)$ or $k=n!$ do not empirically satisfy \eqref{eq:P389-main} even for small $n$.

\subsection*{Verification}
\begin{itemize}[leftmargin=2em]
\item \eqref{eq:P389-ratio} follows by substituting the factorial expressions for $L(n,k)$ and $R(n,k)$.
\item \eqref{eq:P389-binomial} follows from $L=k!\binom{n+k-1}{k}$ and $R=k!\binom{n+2k-1}{k}$.
\item \eqref{eq:P389-valuation} is the standard characterization of divisibility via $p$-adic valuations.
\end{itemize}

\subsection*{Final}
\begin{quote}
\textbf{UNRESOLVED.}
\begin{enumerate}[leftmargin=2.2em]
\item[(i)] \textbf{Strongest proved partial result:}
The divisibility condition is exactly the binomial divisibility
$\binom{n+k-1}{k}\mid\binom{n+2k-1}{k}$, equivalently the integrality of the ratio in \eqref{eq:P389-ratio}, and it reduces to the family of $p$-adic inequalities \eqref{eq:P389-valuation}.
\item[(ii)] \textbf{First gap / obstruction:}
I did not find a construction of $k$ (as a function of $n$) that forces \eqref{eq:P389-valuation} simultaneously for all primes, nor a monotonicity argument that guarantees existence for sufficiently large $k$.
\item[(iii)] \textbf{Top 3 next moves:}
(1) Use Kummer's theorem (carries in base $p$) to reinterpret $v_p\bigl(\binom{n+k-1}{k}\bigr)$ and $v_p\bigl(\binom{n+2k-1}{k}\bigr)$ and attempt to engineer $k$ so that carries only increase.
(2) Search for a group action / combinatorial partition that would make \eqref{eq:P389-binomial} obvious for a suitable $k$.
(3) Try to build $k$ inductively by forcing the valuation inequalities for primes up to a growing bound, using Chinese-remainder-type constraints.
\item[(iv)] \textbf{What a minimal counterexample would look like:}
An $n$ such that for every $k\ge 1$ there is some prime $p$ with
$v_p\bigl(\binom{n+2k-1}{k}\bigr)<v_p\bigl(\binom{n+k-1}{k}\bigr)$.
Empirically, if counterexamples exist they likely occur at modest $n$ but require very large search ranges in $k$.
\end{enumerate}
\end{quote}

\subsection*{Completion estimate}
A full solution would be either an explicit constructive $k(n)$ with a proof of \eqref{eq:P389-valuation} for all primes $p$, or a proof that some $n$ admits no such $k$. Both directions appear to require new structural insight into the ratio of the two binomial coefficients in \eqref{eq:P389-binomial}.

%============================================================
