
\noindent\textbf{1) FORMAL RESTATEMENT.}

Question: Does there exist $\varepsilon>0$ such that for all sufficiently large integers $n$ there exist
\begin{itemize}
\item a simple graph $G$ on $n$ vertices with at least $\varepsilon n^2$ edges, and
\item an edge-coloring $\chi:E(G)\to [n]$
\end{itemize}
with the property that every cycle $C_4\subseteq G$ has its four edges colored with four distinct colors (i.e. each $C_4$ is rainbow)?

\medskip
\noindent\textbf{2) QUICK LITERATURE/CONTEXT CHECK.}

No additional results are stated in the problem text beyond that it is related to [809]. I do not use any external literature.

\medskip
\noindent\textbf{3) ATTACK PLAN.}

\emph{Existence strategies.}
\begin{itemize}
\item View the coloring as a decomposition of $E(G)$ into $n$ color classes with the constraint that no $C_4$ contains two edges from the same class; attempt algebraic constructions (finite fields/geometries) where 4-cycles correspond to algebraic relations that force distinct colors.
\item Try to build $G$ as a dense bipartite-like graph and define colors via a function of endpoints (e.g. slopes/differences) designed so that a 4-cycle cannot repeat a color.
\end{itemize}

\emph{Nonexistence strategies.}
\begin{itemize}
\item Derive upper bounds on $|E(G)|$ from the rainbow-$C_4$ constraint by reducing to extremal bounds for $C_4$-free graphs on unions of color classes.
\item Reinterpret the condition as a restriction on induced matchings (or other sparse structures) in each color class; attempt to prove that $n$ such classes cannot cover $\Theta(n^2)$ edges.
\end{itemize}

I did not resolve the main question; below are fully proved structural lemmas and tiny-$n$ computations.

\medskip
\noindent\textbf{4) WORK.}

\textbf{Lemma 810.1 (two colors cannot contain a $C_4$).}
Let $G$ and $\chi:E(G)\to [n]$ satisfy the rainbow-$C_4$ property. Fix two distinct colors $a\ne b$ and let $H$ be the subgraph of $G$ with edge set
\[
E(H)=\{e\in E(G):\chi(e)\in\{a,b\}\}.
\]
Then $H$ is $C_4$-free.

\emph{Proof.}
If $H$ contained a copy of $C_4$, then that 4-cycle would be a $C_4$ in $G$ whose four edges are colored using only colors $a$ and $b$. By the pigeonhole principle, among 4 edges and 2 colors, some color would repeat on at least two edges, contradicting the rainbow-$C_4$ requirement.
\qed

\textbf{Lemma 810.2 (a self-contained $C_4$-free extremal bound).}
Let $H$ be a $C_4$-free graph on $N$ vertices with $m$ edges. Then
\[
 m\;\le\; \frac12 N^{3/2}+\frac14 N.
\]

\emph{Proof.}
For each vertex $v$, let $d(v)$ be its degree. Count the number $P$ of (unordered) length-$2$ paths $x-v-y$ with distinct endpoints $x\ne y$ (i.e. choose a center $v$ and two distinct neighbors).
On one hand,
\[
P\;=\;\sum_{v\in V(H)} \binom{d(v)}{2}
\;=\;\frac12\sum_v \bigl(d(v)^2-d(v)\bigr)
\;=\;\frac12\Bigl(\sum_v d(v)^2\Bigr)-m.
\]
On the other hand, since $H$ is $C_4$-free, any unordered pair of vertices $\{x,y\}$ has \emph{at most one} common neighbor: if $x$ and $y$ had two distinct common neighbors $v\ne w$, then $x-v-y-w-x$ would be a 4-cycle. Therefore, each unordered pair $\{x,y\}$ can be the endpoints of at most one length-$2$ path, so
\[
P\;\le\;\binom{N}{2}=\frac{N(N-1)}{2}.
\]
Combine these bounds:
\[
\frac12\Bigl(\sum_v d(v)^2\Bigr)-m\;\le\;\frac{N(N-1)}{2}.
\]
By Cauchy--Schwarz,
\[
\sum_v d(v)^2\;\ge\;\frac{\bigl(\sum_v d(v)\bigr)^2}{N}
\;=\;\frac{(2m)^2}{N}=\frac{4m^2}{N}.
\]
Substitute into the inequality to get
\[
\frac12\cdot \frac{4m^2}{N}-m\;\le\;\frac{N(N-1)}{2}
\quad\Rightarrow\quad
\frac{2m^2}{N}\;\le\;m+\frac{N(N-1)}{2}\;\le\;m+\frac{N^2}{2}.
\]
Multiply by $N$:
\[
2m^2\;\le\;mN+\frac{N^3}{2}.
\]
Treat this as a quadratic inequality in $m$:
\[
2m^2-mN-\frac{N^3}{2}\le 0.
\]
The positive root is
\[
 m\;\le\;\frac{N+\sqrt{N^2+4N^3}}{4}
\;=\;\frac{N+N\sqrt{1+4N}}{4}.
\]
Using $\sqrt{1+4N}\le 1+2\sqrt{N}$ (since $(1+2\sqrt{N})^2=1+4N+4\sqrt{N}\ge 1+4N$), we get
\[
 m\le \frac{N+N(1+2\sqrt{N})}{4}=\frac12 N^{3/2}+\frac12\cdot \frac{N}{2}=\frac12 N^{3/2}+\frac14 N.
\]
\qed

\textbf{Consequence (very weak edge upper bound).}
Let $G$ satisfy the rainbow-$C_4$ property with $n$ colors on $n$ vertices, and let $e_i$ be the number of edges of color $i$. Then for every pair $i\ne j$,
\[
 e_i+e_j\le \frac12 n^{3/2}+\frac14 n
\]
by Lemma 810.1 and Lemma 810.2.
Summing over all $\binom{n}{2}$ pairs and using that each $e_i$ appears in exactly $n-1$ pairs gives
\[
(n-1)|E(G)|=\sum_{i\ne j,\,i<j} (e_i+e_j)\le \binom{n}{2}\left(\frac12 n^{3/2}+\frac14 n\right),
\]
hence
\[
|E(G)|\le \frac{n}{2}\left(\frac12 n^{3/2}+\frac14 n\right)=\frac14 n^{5/2}+\frac18 n^2.
\]
This is far from excluding $\varepsilon n^2$, but it is a clean consequence of the constraint.

\textbf{FAST REALITY CHECK (exact search for $n=4,5$).}
I brute-forced all graphs on $n=4$ and $n=5$ vertices and all edge-colorings with $n$ colors to maximize $|E(G)|$ under the rainbow-$C_4$ condition. The maximum edge counts found were:
\begin{itemize}
\item $n=4$: maximum $|E(G)|=5$. One example uses edges
\[
\{01,02,03,12,13\}
\]
with coloring $01\mapsto 0$, $02\mapsto 0$, $03\mapsto 1$, $12\mapsto 2$, $13\mapsto 3$ (here $ij$ denotes edge $\{i,j\}$).
\item $n=5$: maximum $|E(G)|=7$. One example uses edges
\[
\{01,02,03,04,12,13,24\}
\]
with coloring $01\mapsto 1$, $02\mapsto 1$, $03\mapsto 2$, $04\mapsto 2$, $12\mapsto 0$, $13\mapsto 3$, $24\mapsto 3$.
\end{itemize}
In both cases a direct check verified that every present $C_4$ has four distinct colors.

\medskip
\noindent\textbf{5) VERIFICATION.}

\begin{itemize}
\item Lemma 810.1: checked that any $C_4$ inside the union of two colors would necessarily repeat one of those colors.
\item Lemma 810.2: verified the key $C_4$-free implication: two distinct common neighbors create a 4-cycle. The Cauchy--Schwarz step is standard and explicitly written.
\item Computations: exhaustive search for $n=4,5$ checked all subgraphs and all colorings by backtracking and verified the rainbow-$C_4$ property by enumerating all 4-cycles on 4-vertex subsets.
\end{itemize}

\medskip
\noindent\textbf{6) FINAL.} \textbf{UNRESOLVED.}

\begin{enumerate}
\item[(i)] \emph{Strongest proved partial result here:} in any such coloring, the union of any two color classes is $C_4$-free (Lemma 810.1), and any $C_4$-free graph on $n$ vertices has at most $\frac12 n^{3/2}+\frac14 n$ edges (Lemma 810.2). Consequently, any $n$-color rainbow-$C_4$ coloring has $|E(G)|\le \frac14 n^{5/2}+\frac18 n^2$.
\item[(ii)] \emph{First gap (crisp):} either construct, for some fixed $\varepsilon>0$, graphs on all large $n$ with $\varepsilon n^2$ edges that admit an $n$-edge-coloring in which every $C_4$ is rainbow, or prove that for every such coloring one has $|E(G)|=o(n^2)$.
\item[(iii)] \emph{Top 3 next moves:}
  \begin{enumerate}
  \item Strengthen Lemma 810.1 from ``two colors'' to a structural statement about each color class (e.g. show each class must be an induced matching or has bounded ``$C_4$-participation''), then use extremal bounds for induced matchings.
  \item Attempt an explicit algebraic construction on $\mathbb{F}_q$-based vertex sets with colors given by a bilinear form or slope, and prove that a 4-cycle forces four distinct such invariants.
  \item Compute the maximum possible edge density for moderate $n$ (say $n\le 10$) via SAT/ILP to guess whether the optimum grows like $\Theta(n^2)$ or $o(n^2)$.
  \end{enumerate}
\item[(iv)] \emph{What a minimal counterexample would likely look like:} if the answer is ``no'' (no constant $\varepsilon$), then any attempted dense construction would necessarily force some color class to contain two edges that can be completed to a 4-cycle using edges of other colors; thus a minimal counterexample would resemble a graph whose edges can be partitioned into $n$ classes each avoiding pairwise ``$C_4$-completions''---a structure closely related to large unions of induced matchings.
\end{enumerate}


