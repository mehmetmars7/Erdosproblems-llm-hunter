
Determine which countable ordinals $\beta$ have the property that, if $\alpha=\omega^{^\beta}$, then in any red/blue colouring of the edges of $K_\alpha$ there is either a red $K_\alpha$ or a blue $K_3$. This property holds for $\beta=2$ and not for $3\leq \beta <\omega$ (Specker \cite{Sp57}) and for $\beta=\omega$ (Chang \cite{Ch72}). The first open case is $\beta=\omega^2$ (see [591] ). Galvin and Larson \cite{GaLa74} have shown that if $\beta\geq 3$ has this property then $\beta$ must be 'additively indecomposable', so that in particular $\beta=\omega^\gamma$ for some $\gamma<\omega_1$. Galvin and Larson conjecture that every $\beta\geq 3$ of this form has this property. See also [590] . References [Ch72] Chang, C. C., A partition theorem for the complete graph on {$\omega\sp{\omega }$} . J. Combinatorial Theory Ser. A (1972), 396-452. [GaLa74] Galvin, Fred and Larson, Jean, Pinning countable ordinals . Fund. Math. (1974/75), 357-361. [Sp57] Specker, Ernst, Teilmengen von Mengen mit Relationen . Comment. Math. Helv. (1957), 302-314.

\bigskip
\noindent\textbf{FORMAL RESTATEMENT}

Let $\beta$ be a countable ordinal and set $\alpha:=\omega^\beta$ (interpreting the problem's $\omega^{^\beta}$ as $\omega^\beta$).
For a red/blue colouring $c:[\alpha]^2\to\{\text{red},\text{blue}\}$, ask whether there must exist either
\begin{itemize}
\item a red $K_\alpha$, i.e. $X\subseteq\alpha$ with order type $\alpha$ and $c(\{x,y\})=\text{red}$ for all distinct $x,y\in X$, or
\item a blue $K_3$, i.e. $x<y<z<\alpha$ with all three edges blue.
\end{itemize}

Problem: determine exactly which countable ordinals $\beta$ satisfy $\omega^\beta\to(\omega^\beta,3)^2$.

\bigskip
\noindent\textbf{QUICK LITERATURE/CONTEXT CHECK}

I will not import external results beyond what the problem text explicitly states.
The text states: the property holds for $\beta=2$ and fails for all finite $\beta\ge 3$ (Specker), and holds for $\beta=\omega$ (Chang).
It also states a necessary condition for $\beta\ge 3$ proved by Galvin--Larson: $\beta$ must be additively indecomposable (hence of the form $\omega^\gamma$).
The first open case is stated to be $\beta=\omega^2$.

\bigskip
\noindent\textbf{ATTACK PLAN}

\emph{Proof-track ideas.}
\begin{itemize}
\item Attempt to generalize Chang's $\beta=\omega$ argument to $\beta=\omega^2$ by iterating a suitable ``pinning'' or canonical decomposition along the tower $\omega^{\omega\cdot n}$.
\item Use the additively indecomposable structure of $\beta$ (i.e. $\beta=\omega^\gamma$) to get strong tail-isomorphism properties of $\alpha=\omega^\beta$ and try to build a red copy by transfinite recursion under no-blue-triangle.
\end{itemize}

\emph{Disproof-track ideas.}
\begin{itemize}
\item Try to build explicit triangle-free-blue colourings on $\omega^\beta$ when $\beta$ is not additively indecomposable (e.g. $\beta=\gamma+\delta$ with $\delta>0$), by splitting into blocks and coloring cross-block edges in a controlled way.
\item Search for the smallest additively indecomposable open case ($\beta=\omega^2$) and attempt a Specker-type construction adapted to the normal form of ordinals $<\omega^{\omega^2}$.
\end{itemize}

I did not resolve the classification. I record two self-contained lemmas about the ordinal notions mentioned in the statement.

\bigskip
\noindent\textbf{WORK}

\medskip
\noindent\textbf{FAST REALITY CHECK (stated base cases)}

\begin{itemize}
\item $\beta=1$ gives $\alpha=\omega$; then $\omega\to(\omega,3)^2$ holds by the constructive argument of Lemma~\ref{lem:omega_case} above.
\item $\beta=2$ ($\alpha=\omega^2$) is stated in the problem text to satisfy the property (Specker).
\item Finite $\beta\ge 3$ are stated in the text to fail (Specker).
\item $\beta=\omega$ ($\alpha=\omega^\omega$) is stated in the text to satisfy the property (Chang).
\item The first open case is $\beta=\omega^2$.
\end{itemize}

\medskip
\noindent\textbf{Lemma 1 (additively indecomposable ordinals are exactly powers of $\omega$).}\label{lem:add_indec}
Let $\beta$ be a nonzero countable ordinal.
Call $\beta$ \emph{additively indecomposable} if for all ordinals $\gamma,\delta<\beta$ one has $\gamma+\delta<\beta$.
Then $\beta$ is additively indecomposable if and only if $\beta=\omega^\eta$ for some ordinal $\eta$.

\emph{Proof.}
($\Rightarrow$)
Write $\beta$ in Cantor normal form:
\[
\beta=\omega^{\eta_0}c_0+\omega^{\eta_1}c_1+\cdots+\omega^{\eta_m}c_m,
\]
where $m\ge 0$, $\eta_0>\eta_1>\cdots>\eta_m$ are ordinals and $c_i\in\mathbb{N}$ with $c_i\ge 1$.
If either $m\ge 1$ or $c_0\ge 2$, we will produce $\gamma,\delta<\beta$ with $\gamma+\delta\ge\beta$, contradicting additively indecomposable.

If $m\ge 1$ (so there is a nonzero ``tail''), set
\[
\gamma:=\omega^{\eta_0}c_0 \quad\text{and}\quad \delta:=\omega^{\eta_1}c_1+\cdots+\omega^{\eta_m}c_m.
\]
Then $\gamma<\beta$ (because $\delta>0$) and also $\delta<\beta$ (because $\delta<\omega^{\eta_0}$ while $\beta\ge\omega^{\eta_0}$).
But by definition of ordinal addition for Cantor normal form with strictly decreasing exponents,
\[
\gamma+\delta=\omega^{\eta_0}c_0+\omega^{\eta_1}c_1+\cdots+\omega^{\eta_m}c_m=\beta,
\]
which contradicts $\gamma+\delta<\beta$.

If instead $m=0$ but $c_0\ge 2$, set
\[
\gamma:=\omega^{\eta_0}(c_0-1)\quad\text{and}\quad\delta:=\omega^{\eta_0}.
\]
Then $\gamma<\beta$ and $\delta<\beta$.
Moreover ordinal addition here agrees with multiplication by a natural number:
\[
\gamma+\delta=\omega^{\eta_0}(c_0-1)+\omega^{\eta_0}=\omega^{\eta_0}c_0=\beta,
\]
again contradicting additively indecomposable.

Therefore we must have $m=0$ and $c_0=1$, i.e. $\beta=\omega^{\eta_0}$.

($\Leftarrow$)
Now assume $\beta=\omega^\eta$.
Let $\gamma,\delta<\omega^\eta$.
Write $\gamma$ and $\delta$ in Cantor normal form. In particular, the largest exponent appearing in either is some ordinal $<\eta$.
Let $\theta<\eta$ be the maximum of those leading exponents.
Then $\gamma<\omega^{\theta+1}$ and $\delta<\omega^{\theta+1}$.
We claim that if $a,b<\omega^{\theta+1}$ then $a+b<\omega^{\theta+1}$.
Indeed $\omega^{\theta+1}=\omega^{\theta}\cdot\omega$, so for each $a<\omega^{\theta+1}$ there exist $m<\omega$ and $r<\omega^{\theta}$ with $a=\omega^{\theta}m+r$; similarly $b=\omega^{\theta}n+s$ with $n<\omega$ and $s<\omega^{\theta}$.
Then, since $r<\omega^{\theta}$ and $n>0$ implies $r+\omega^{\theta}n=\omega^{\theta}n$ (and if $n=0$ the statement is trivial), we compute
\[
a+b=(\omega^{\theta}m+r)+(\omega^{\theta}n+s)=\omega^{\theta}m+(r+\omega^{\theta}n)+s=\omega^{\theta}(m+n)+s.
\]
Because $m+n<\omega$ and $s<\omega^{\theta}$, we have $\omega^{\theta}(m+n)+s<\omega^{\theta}(m+n+1)\le\omega^{\theta}\cdot\omega=\omega^{\theta+1}$.
Applying this with $a=\gamma$ and $b=\delta$ gives $\gamma+\delta<\omega^{\theta+1}$.
Finally, because $\theta+1\le\eta$ we have $\omega^{\theta+1}\le\omega^\eta$ and hence
$\gamma+\delta<\omega^{\theta+1}\le\omega^\eta=\beta$.
Thus $\beta$ is additively indecomposable.\qed

\medskip
\noindent\textbf{Lemma 2 (tail self-similarity of $\omega^\beta$ for $\beta>0$).}\label{lem:tail_iso}
Let $\beta>0$ be an ordinal and set $\alpha:=\omega^\beta$.
Then for every $\xi<\alpha$, the tail interval $[\xi,\alpha):=\{\eta:\xi\le \eta<\alpha\}$ is order-isomorphic to $\alpha$.
Equivalently, $\xi+\alpha=\alpha$ for all $\xi<\alpha$.

\emph{Proof.}
Since $\beta>0$, $\alpha=\omega^\beta$ is a (nonzero) power of $\omega$, hence by Lemma~\ref{lem:add_indec} it is additively indecomposable as an ordinal.
In particular for any $\xi<\alpha$ we have $\xi+\alpha=\alpha$.
Consider the map $f:\alpha\to [\xi,\alpha)$ given by $f(\eta)=\xi+\eta$.
Ordinal addition is strictly increasing in the second argument, so $f$ is strictly increasing.
Its image is $[\xi,\xi+\alpha)=[\xi,\alpha)$ because $\xi+\alpha=\alpha$.
Thus $f$ is an order-isomorphism from $\alpha$ onto the tail $[\xi,\alpha)$.\qed

\bigskip
\noindent\textbf{VERIFICATION}

\begin{itemize}
\item Lemma~\ref{lem:add_indec}: the forward direction uses only Cantor normal form and explicit choices of $\gamma,\delta$ witnessing decomposability. The backward direction reduces the sum to a smaller power $\omega^{\theta+1}$ using the leading exponent. All steps are ordinal-arithmetic checks.
\item Lemma~\ref{lem:tail_iso}: verifies that the tail has the same order type via the explicit order-isomorphism $\eta\mapsto \xi+\eta$.
\end{itemize}

\bigskip
\noindent\textbf{FINAL}

\textbf{UNRESOLVED}

(i) \emph{Strongest proved partial result.}
The ordinal constraint ``$\beta$ additively indecomposable'' appearing in the problem text is equivalent to ``$\beta=\omega^\eta$'' (Lemma~\ref{lem:add_indec}), and for $\alpha=\omega^\beta$ (with $\beta>0$) every tail $[\xi,\alpha)$ is order-isomorphic to $\alpha$ (Lemma~\ref{lem:tail_iso}).
These structural facts are consistent with the conjectural ``self-similarity'' needed to build red copies of $K_\alpha$.

(ii) \emph{First gap (crisp).}
Decide whether $\omega^{\omega^2}\to(\omega^{\omega^2},3)^2$ holds (the first open case $\beta=\omega^2$), i.e. either construct a triangle-free-blue colouring of $[\omega^{\omega^2}]^2$ with no red $K_{\omega^{\omega^2}}$, or give a proof that none exists.

(iii) \emph{Top 3 next moves.}
\begin{enumerate}
\item Using Lemma~\ref{lem:tail_iso}, attempt a transfinite recursion that builds an increasing red set of type $\omega^{\omega^2}$ from red sets of type $\omega^{\omega\cdot m}$, under the hypothesis ``no blue triangle''.
\item Isolate a ``one-step'' strengthening: prove that for some $m$, any triangle-free-blue colouring on $\omega^{\omega\cdot(m+1)}$ contains a red $\omega^{\omega\cdot(m+1)}$ provided it contains cofinally many red $\omega^{\omega\cdot m}$ blocks.
\item Try to explicitly generalize Specker-type counterexamples from $\omega^n$ ($n\ge 3$) to exponents with Cantor normal form having at least two terms (i.e. $\beta$ not additively indecomposable), to re-prove the Galvin--Larson necessity direction by direct construction.
\end{enumerate}

(iv) \emph{Minimal counterexample structure.}
A minimal counterexample at $\beta=\omega^2$ would be a colouring on $\omega^{\omega^2}$ with no blue triangle and no red subset of order type $\omega^{\omega^2}$.
By restriction to smaller initial segments, such a colouring would still need to contain red copies of the largest known true cases (e.g. a red $\omega^\omega$), while carefully arranging blue edges between blocks to block extension to order type $\omega^{\omega^2}$.


