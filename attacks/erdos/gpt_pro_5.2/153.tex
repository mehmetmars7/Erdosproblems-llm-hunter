
\section*{Erdos problem \#153}

\subsection*{1) FORMAL RESTATEMENT}
A finite set $A\subset\mathbb Z$ is a \emph{Sidon set} if
\[
 a+b=c+d\ (a\le b,\ c\le d,\ a,b,c,d\in A)\implies \{a,b\}=\{c,d\}.
\]
Equivalently, the sums $a_i+a_j$ with $i\le j$ are all distinct.
Let
\[
A+A:=\{a+a':a,a'\in A\}=\{s_1<s_2<\cdots<s_t\}
\]
(sorted increasingly, with no repetitions). The question asks whether for Sidon sets $A$,
\[
\frac{1}{t}\sum_{i=1}^{t-1}(s_{i+1}-s_i)^2\xrightarrow[|A|\to\infty]{}\infty.
\]

\subsection*{2) QUICK LITERATURE/CONTEXT CHECK}
The problem file does not cite any known partial results for this question. I therefore do not assume any.

\subsection*{3) ATTACK PLAN}
Let $d_i:=s_{i+1}-s_i$ be the consecutive gaps in $A+A$. Then $\sum d_i=s_t-s_1$.
A first step is to relate $\sum d_i^2$ to $(\sum d_i)^2$ (Cauchy--Schwarz).
To go beyond a constant lower bound, one would need a genuinely Sidon-specific obstruction to $A+A$ being ``almost evenly spaced.''

\subsection*{4) WORK}
\paragraph{Lemma 153.1 (size and span of $A+A$ for Sidon sets).}
Let $A=\{a_1<\cdots<a_n\}$ be Sidon. Then $|A+A|=t=\binom{n+1}{2}$ and $s_1=2a_1$, $s_t=2a_n$.

\textit{Proof.}
Because $A$ is Sidon, all sums $a_i+a_j$ with $1\le i\le j\le n$ are distinct. There are exactly $\binom{n+1}{2}$ such pairs, so $t=\binom{n+1}{2}$.
The smallest sum is $a_1+a_1=2a_1$, and the largest is $a_n+a_n=2a_n$, hence $s_1=2a_1$ and $s_t=2a_n$. \qed

\paragraph{Lemma 153.2 (Cauchy--Schwarz lower bound for mean squared gaps).}
With $d_i:=s_{i+1}-s_i$ for $i=1,\dots,t-1$, we have
\[
\frac{1}{t}\sum_{i=1}^{t-1} d_i^2\ \ge\ \frac{(s_t-s_1)^2}{t(t-1)}.
\]

\textit{Proof.}
By Cauchy--Schwarz,
\[
\sum_{i=1}^{t-1} d_i^2\ \ge\ \frac{\left(\sum_{i=1}^{t-1} d_i\right)^2}{t-1}.
\]
But $\sum_{i=1}^{t-1} d_i = s_t-s_1$. Divide by $t$ to get the stated inequality. \qed

\paragraph{Lemma 153.3 (diameter lower bound from distinct differences).}
If $A=\{a_1<\cdots<a_n\}$ is Sidon, then its diameter satisfies
\[
\operatorname{diam}(A):=a_n-a_1\ \ge\ \binom{n}{2}.
\]

\textit{Proof.}
For $1\le i<j\le n$, the differences $a_j-a_i$ are positive integers in $\{1,2,\dots, a_n-a_1\}$. We claim they are all distinct.
Indeed, if $a_j-a_i=a_{j'}-a_{i'}$ with $i<j$ and $i'<j'$, then
$a_j+a_{i'}=a_{j'}+a_i$, and Sidon-ness forces $\{a_j,a_{i'}\}=\{a_{j'},a_i\}$.
Because $a_j>a_i$ and $a_{j'}>a_{i'}$, the only possibility is $(j,i)=(j',i')$, so the differences are distinct.
Thus there are $\binom{n}{2}$ distinct positive integers among $\{1,\dots,a_n-a_1\}$, forcing $a_n-a_1\ge \binom{n}{2}$. \qed

\paragraph{Consequence (a uniform constant lower bound).}
Combining Lemmas 153.1--153.3 with Lemma 153.2 gives
\[
\frac{1}{t}\sum_{i=1}^{t-1} d_i^2\ \ge\ \frac{(2\operatorname{diam}(A))^2}{t(t-1)}
\ \ge\ \frac{4\binom{n}{2}^2}{\binom{n+1}{2}(\binom{n+1}{2}-1)}.
\]
The right-hand side tends to a positive constant ($\approx 1$) as $n\to\infty$, but this does \emph{not} prove divergence.

\paragraph{FAST REALITY CHECK (computed small cases).}
I computed $\frac{1}{t}\sum_{i=1}^{t-1} (s_{i+1}-s_i)^2$ for two families.

(1) \emph{Greedy Sidon sets starting at $0$} (lexicographically smallest Sidon set):
\[
\begin{array}{c|l|c}
 n & A & \frac{1}{t}\sum d_i^2 \\\hline
 2 & \{0,1\} & 0.333333\\
 3 & \{0,1,3\} & 0.833333\\
 4 & \{0,1,3,8\} & 2.25\\
 5 & \{0,1,3,8,10\} & 2.48\\
 6 & \{0,1,3,8,10,18\} & 3.746031\\
 7 & \{0,1,3,8,10,18,23\} & 4.132653\\
 8 & \{0,1,3,8,10,18,23,25\} & 6.351852\\
 9 & \{0,1,3,8,10,18,23,25,31\} & 21.416667\\
 10& \{0,1,3,8,10,18,23,25,31,33\} & 18.954545\\
 11& \{0,1,3,8,10,18,23,25,31,33,44\} & 41.363636\\
 12& \{0,1,3,8,10,18,23,25,31,33,44,46\} & 39.119048
\end{array}
\]

(2) \emph{Best values among Sidon sets inside a small box} (brute force over $A\subseteq\{0,1,\dots,M\}$ for small $(n,M)$):
\[
\begin{array}{c|c|l}
 n & \min \frac{1}{t}\sum d_i^2 & \text{argmin example }A \\\hline
 4 & 1.8 & (0,1,4,6)\\
 5 & 2.8 & (0,1,4,9,11)\\
 6 & 3.5238095 & (0,1,4,10,15,17)
\end{array}
\]
These data are compatible with slow growth but are far from decisive.

\subsection*{5) VERIFICATION}
\begin{itemize}
\item Lemma 153.1: uses only the Sidon definition on unordered pairs $(i\le j)$.
\item Lemma 153.2: standard Cauchy--Schwarz; the identity $\sum d_i=s_t-s_1$ is exact.
\item Lemma 153.3: the reduction ``equal differences $\Rightarrow$ equal sums'' is checked explicitly.
\item Computations: verified by a brute-force script (not included here) using direct enumeration of sums.
\end{itemize}

\subsection*{6) FINAL}
\textbf{UNRESOLVED}

\begin{enumerate}
\item[(i)] \textbf{Strongest fully proved partial result obtained here.}
A universal lower bound:
\[
\frac{1}{t}\sum_{i=1}^{t-1}(s_{i+1}-s_i)^2\ \ge\ \frac{(s_t-s_1)^2}{t(t-1)}\ =\ \frac{4(a_n-a_1)^2}{\binom{n+1}{2}(\binom{n+1}{2}-1)}\ \ge\ \frac{4\binom{n}{2}^2}{\binom{n+1}{2}(\binom{n+1}{2}-1)},
\]
which stays bounded away from $0$ as $n=|A|\to\infty$.

\item[(ii)] \textbf{Exact first gap.}
Upgrade the above \emph{constant} lower bound to a bound that grows with $n$, i.e. prove
\[
\frac{1}{t}\sum_{i=1}^{t-1}(s_{i+1}-s_i)^2\ \ge\ \omega(1)
\quad\text{for every Sidon set }A\text{ with }|A|=n\to\infty.
\]

\item[(iii)] \textbf{Top 3 next moves (concrete targets).}
\begin{enumerate}
\item Prove a Sidon-specific anti-regularity statement for $A+A$: e.g. show that $A+A$ must contain a gap of size at least $g(n)\to\infty$.
\item Relate the gap-squared sum to an additive-combinatorial invariant (e.g. additive energy of $A+A$) and show Sidon-ness forces that invariant to be small/large enough to imply divergence.
\item Computational search for extremal families: for each $n\le 12$ (or higher), search for Sidon sets minimizing the functional to guess the true growth rate.
\end{enumerate}

\item[(iv)] \textbf{Minimal counterexample structure (if the divergence claim is false).}
A counterexample would be an infinite family of Sidon sets $A_n$ with $|A_n|\to\infty$ and
\[
\sup_n\ \frac{1}{|A_n+A_n|}\sum_{i}(s_{i+1}-s_i)^2\ <\ \infty.
\]
Such a family would force $A_n+A_n$ to have ``almost bounded second moment of gaps'' despite having $\Theta(n^2)$ distinct sums inside an interval of length $2\operatorname{diam}(A_n)=\Omega(n^2)$.
\end{enumerate}


