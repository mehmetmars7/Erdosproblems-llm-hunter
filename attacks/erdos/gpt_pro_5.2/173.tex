
PROBLEM

In any $2$-colouring of $\mathbb{R}^2$, for all but at most one triangle $T$, there is a monochromatic congruent copy of $T$.

OUTPUT

## 1) FORMAL RESTATEMENT

A **2-coloring** is a map $c:\mathbb R^2\to\{\text{red},\text{blue}\}$.

Two triangles $T,T'$ in $\mathbb R^2$ are **congruent** if there is an isometry (composition of translation, rotation, reflection) mapping $T$ to $T'$.

Given a triangle $T$, say that $T$ is **Ramsey for the coloring $c$** if there exists a monochromatic congruent copy of $T$, i.e. an isometric image $T'$ of $T$ whose three vertices all have the same color.

The statement to prove can be phrased as:

> For every 2-coloring $c$ of $\mathbb R^2$, the set of congruence classes of triangles $T$ that are *not* Ramsey for $c$ has size at most $1$.

The problem text also notes there are colorings (“alternating strips”) where one equilateral triangle must be excluded, and that Shader proved the statement for a fixed right triangle.

---

## 2) QUICK LITERATURE/CONTEXT CHECK

I will not use or claim any external results beyond what is explicitly written in the problem text.

The problem text reports:

* An alternating-strip coloring can force exclusion of one equilateral triangle.
* Shader (1976) proved the statement for a single right-angled triangle.

---

## 3) ATTACK PLAN

I do not currently have a route to the full theorem. I instead analyze the simplest cited obstruction mechanism (alternating strip colorings) and prove a couple of precise, elementary geometric facts about when a triangle can be made monochromatic in such a coloring.

---

## 4) WORK

### Lemma 4.1 (monochromatic copies inside a single strip)

Consider the horizontal strip coloring
\[
c(x,y)=\begin{cases}
\text{red},& y\in [2k,2k+1)\text{ for some }k\in\mathbb Z,\\
\text{blue},& y\in [2k+1,2k+2)\text{ for some }k\in\mathbb Z.
\end{cases}
\]

If a triangle $T$ has a congruent copy whose three vertices have $y$-coordinates contained in an interval of length $<1$, then $T$ has a monochromatic copy (in fact, entirely red or entirely blue) in this coloring.

**Proof.**
Let $T'$ be a congruent copy of $T$ whose vertex $y$-coordinates are all in some interval $(\alpha,\alpha+1)$ of length $<1$.
Choose an integer $k$ so that $(\alpha,\alpha+1)$ lies entirely inside a single color strip, either inside $[2k,2k+1)$ or inside $[2k+1,2k+2)$.
(That is always possible because the strip boundaries occur at integers, and any open interval of length $<1$ can be shifted into the interior of one half of a period-2 partition.)
Then all three vertices of $T'$ lie in that strip and hence share the same color.
$\square$

---

### Lemma 4.2 (minimal vertical span of an equilateral triangle)

Let $E_s$ be an equilateral triangle of side length $s>0$.
Among all congruent copies of $E_s$ in the plane, the minimal possible difference between the maximum and minimum $y$-coordinates of its vertices equals its altitude
\[
h=\frac{\sqrt3}{2}s.
\]

**Proof.**
Let $E_1$ be a unit equilateral triangle with vertices $A,B,C$.
The three edge directions are separated by $60^\circ$ angles.
Fix any placement of $E_1$ and consider the vertical unit direction $u=(0,1)$.
For any edge of length $1$ making angle $\phi$ with the horizontal, its vertical projection length is $|\sin\phi|$.

Among the three edge directions of an equilateral triangle, for any given direction $u$, there is an edge whose direction makes angle at most $30^\circ$ from the horizontal-orthogonal direction; equivalently, among the three angles between $u$ and the three edge directions, the minimum is at most $30^\circ$.
Therefore for at least one edge $e$ of the triangle, the vertical projection of that edge has length at least $\cos(30^\circ)=\sqrt3/2$.

But the vertical span (max $y$ minus min $y$ among vertices) is at least the vertical projection length of *some* edge, since that projection is the absolute $y$-difference between its endpoints. Hence the vertical span is always at least $\sqrt3/2$ for the unit triangle.

On the other hand, if one side of the equilateral triangle is placed horizontally, then the third vertex lies exactly $\sqrt3/2$ units above that side, so the vertical span equals $\sqrt3/2$.
Thus the minimum possible vertical span for $E_1$ is $\sqrt3/2$.
Scaling by factor $s$ multiplies all vertical spans by $s$, so the minimum possible vertical span for $E_s$ is $(\sqrt3/2)s$.
$\square$

---

### Corollary 4.3 (small equilateral triangles always appear in strip coloring)

In the strip coloring of Lemma 4.1, every equilateral triangle of side length $s<\frac{2}{\sqrt3}$ has a monochromatic congruent copy.

**Proof.**
If $s<2/\sqrt3$, then its altitude $(\sqrt3/2)s<1$.
By Lemma 4.2 there is a congruent copy with vertical span $<1$, and then Lemma 4.1 yields a monochromatic copy.
$\square$

---

## 5) VERIFICATION

* Lemma 4.1 is purely geometric and uses only the definition of the strip coloring.
* Lemma 4.2 uses the fact that among three directions separated by $60^\circ$, one is within $30^\circ$ of any chosen direction; this implies a projection lower bound. The inequality is tight by the horizontal-side placement.
* Corollary 4.3 shows that strip colorings cannot forbid all equilateral triangles; only sufficiently “large” ones could potentially be obstructed.

---

## 6) FINAL

**UNRESOLVED**

(i) **Strongest fully proved partial result obtained here.**

For the standard horizontal alternating-strip 2-coloring, I proved:

* any triangle that can be placed with vertex $y$-coordinates in an interval of length $<1$ has a monochromatic copy (Lemma 4.1);
* the minimal possible vertical span of an equilateral triangle of side length $s$ is $(\sqrt3/2)s$ (Lemma 4.2);
* consequently, every equilateral triangle with side length $s<2/\sqrt3$ has a monochromatic copy in this strip coloring (Corollary 4.3).

(ii) **Exact first gap.**

Prove (or disprove) the full statement: in an arbitrary 2-coloring of $\mathbb R^2$, at most one congruence class of triangle fails to have a monochromatic copy. I do not have a mechanism to control arbitrary colorings.

(iii) **Top 3 next moves (concrete targets).**

1. For strip colorings (and their rotations), determine exactly which triangle shapes/sizes are forbidden and whether exactly one equilateral triangle can be avoided.
2. Prove stability/robustness: show that if two triangle shapes are forbidden then one can derive a contradiction via a geometric configuration forcing both colors.
3. Attempt to extend Shader’s right-triangle argument to general acute/obtuse triangles via decomposition into right triangles or via affine/rotational averaging.

(iv) **What a minimal counterexample would likely look like.**

A counterexample would be a 2-coloring $c$ and two noncongruent triangles $T_1,T_2$ such that neither has a monochromatic copy. Any such coloring would have to avoid two distinct distance/angle patterns simultaneously; one expects it to be highly structured (e.g., periodic stripes or tilings) rather than random.


