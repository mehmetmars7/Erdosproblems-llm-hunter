% Erdős Problem #12
% URL: https://www.erdosproblems.com/12

1) \textbf{FORMAL RESTATEMENT}

Let $A\subseteq \mathbb{N}$ be infinite. Assume $A$ satisfies the following property:
\begin{quote}
(\*) There do not exist \emph{distinct} $a,b,c\in A$ such that $b>a$, $c>a$, and $a\mid (b+c)$.
\end{quote}
Questions:
\begin{enumerate}
\item[(Q1)] Does there exist such an $A$ with
\[\liminf_{N\to\infty} \frac{|A\cap\{1,\dots,N\}|}{N^{1/2}}>0?\]
\item[(Q2)] Does there exist an absolute constant $c>0$ such that for every such $A$ there are infinitely many $N$ with
\[|A\cap\{1,\dots,N\}|<N^{1-c}?\]
\item[(Q3)] Is it true that for every such $A$ one has
\[\sum_{n\in A}\frac{1}{n}<\infty?\]
\end{enumerate}

Stress points:
\begin{itemize}
\item The condition only forbids divisibility by an element $a$ that is \emph{smaller} than the other two elements.
\item For finite truncations, the property can hold vacuously for large elements because there may not be two larger elements to pair with a given $a$.
\end{itemize}

2) \textbf{QUICK LITERATURE/CONTEXT CHECK}

From the problem statement: Erd\H{o}s--S\'{a}rk\"{o}zy proved such $A$ must have density $0$; they also construct very large examples (size $>N/f(N)$ infinitely often). The statement also records the explicit example $A=\{p^2: p\equiv 3\pmod 4\}$ giving
\[\liminf \frac{|A\cap\{1,\dots,N\}|}{N^{1/2}}\log N>0,
\]
and a stronger construction by Elsholtz--Planitzer, as well as upper bounds in special cases (pairwise coprime elements).

I do not use these external results; I only prove elementary structural lemmas and verify one of the stated constructions has property (\*).

3) \textbf{ATTACK PLAN}

\emph{Toward Q1 (constructive):}
\begin{itemize}
\item Find structured infinite sets where divisibility $a\mid(b+c)$ can be ruled out by congruence obstructions (e.g. make $b+c$ always a nonresidue modulo primes dividing $a$).
\item Try to push size up to $\gg \sqrt{N}$ by choosing $A$ of ``quadratic size'' like squares of primes.
\end{itemize}

\emph{Toward Q2/Q3 (upper bounds):}
\begin{itemize}
\item Use the residue obstruction: for each $a\in A$, elements above $a$ cannot occupy both residue classes $r$ and $-r$ modulo $a$. Try to aggregate these constraints over many $a$.
\item For Q3, attempt to show that too many elements in dyadic ranges create forbidden triples.
\end{itemize}

4) \textbf{WORK}

\textbf{Lemma 12.1 (residue-class obstruction for a fixed divisor in $A$).}
Let $A$ satisfy (\*), and fix $a\in A$.
Then among the set
\[A_{>a}:=\{x\in A: x>a\},\]
there do not exist two distinct elements $b,c\in A_{>a}$ with
\[b\equiv -c\pmod a.
\]
In particular, the set of residue classes modulo $a$ occupied by elements of $A_{>a}$ has size at most $\lceil a/2\rceil$.

\textbf{Proof.}
If $b\equiv -c\pmod a$, then $b+c\equiv 0\pmod a$, hence $a\mid (b+c)$.
If also $b>a$ and $c>a$ and $a,b,c$ are distinct (as assumed), this contradicts property (\*).
Thus no such pair $b,c$ can exist.

For the ``in particular'' statement: residue classes modulo $a$ come in pairs $\{r,-r\}$ (with the possibility $r\equiv -r\pmod a$ only when $2r\equiv 0\pmod a$). Choosing at most one class from each pair yields at most $\lceil a/2\rceil$ classes.
\qed

\textbf{Corollary 12.1a.} If $A$ is infinite and satisfies (\*), then $1\notin A$.

\textbf{Proof.}
If $1\in A$, then for any two distinct $b,c\in A$ with $b,c>1$ one has $1\mid (b+c)$, violating (\*) with $a=1$.
An infinite $A$ has at least two elements $>1$, contradiction.
\qed

\textbf{Lemma 12.2 (verification of the prime-square example mentioned in the problem text).}
Let
\[A:=\{p^2: p\text{ prime and } p\equiv 3\pmod 4\}.
\]
Then $A$ satisfies property (\*).

\textbf{Proof.}
Take distinct $a,b,c\in A$ with $b>a$ and $c>a$. Write
\[a=p^2,\qquad b=q^2,\qquad c=r^2,
\]
where $p,q,r$ are distinct primes all congruent to $3\pmod 4$.
Assume for contradiction that $p^2\mid (q^2+r^2)$.
Then in particular $p\mid (q^2+r^2)$, so in $\mathbb{Z}/p\mathbb{Z}$ we have
\[q^2\equiv -r^2\pmod p.
\]

\emph{Case 1: $p\nmid r$.}
Then $r$ is invertible modulo $p$, so
\[(q r^{-1})^2\equiv -1\pmod p.
\]
Thus $-1$ is a quadratic residue modulo $p$.
But if $p\equiv 3\pmod 4$, then $-1$ is \emph{not} a quadratic residue modulo $p$: indeed, if $x^2\equiv -1\pmod p$ then raising both sides to the $(p-1)/2$ power gives
\[x^{p-1}\equiv (-1)^{(p-1)/2}\equiv -1\pmod p
\]
because $(p-1)/2$ is odd when $p\equiv 3\pmod 4$. This contradicts Fermat's little theorem $x^{p-1}\equiv 1\pmod p$ for $x\not\equiv 0\pmod p$.
So Case 1 is impossible.

\emph{Case 2: $p\mid r$.}
Since $r$ is prime, this forces $r=p$.
Then $c=r^2=p^2=a$, contradicting that $a,c$ are distinct.

Both cases lead to contradiction; hence $p^2\nmid (q^2+r^2)$. Therefore no forbidden triple exists and $A$ satisfies (\*).
\qed

\textbf{FAST REALITY CHECK / COMPUTATION (finite version sanity).}
For small $N$ I brute-forced the maximum size of a subset $A\subseteq\{1,\dots,N\}$ satisfying the finite analogue of (\*) (no distinct $a,b,c\in A$ with $b,c>a$ and $a\mid(b+c)$).
For $N=1$ through $18$, the maximum sizes found were:
\[
\begin{array}{c|cccccccccccccccccc}
N & 1&2&3&4&5&6&7&8&9&10&11&12&13&14&15&16&17&18\\\hline
\max |A| & 1&2&2&3&3&3&4&4&4&5&5&5&6&6&6&6&7&7
\end{array}
\]
(Example maximizers exist; note that this finite problem can be satisfied ``vacuously'' by choosing $A$ concentrated near $N$, so these numbers do not directly inform the infinite-density questions.)

5) \textbf{VERIFICATION}

\begin{itemize}
\item Lemma~12.1 is a direct translation of the forbidden configuration into congruence language, so there is no hidden assumption.
\item Lemma~12.2: checked both possibilities $p\mid r$ and $p\nmid r$. The key number theory fact used is Fermat's little theorem, which is standard and suffices to show $x^2\equiv -1\pmod p$ has no solution when $p\equiv 3\pmod 4$.
\item Computation: brute force over all subsets for $N\le 18$ is exact.
\end{itemize}

6) \textbf{FINAL}

\textbf{UNRESOLVED}

(i) Strongest proved partial result: Lemma~12.1 gives a strong per-element residue obstruction; Corollary~12.1a rules out $1\in A$ for infinite sets. Lemma~12.2 rigorously verifies that the infinite set $\{p^2: p\equiv 3\ (\mathrm{mod}\ 4)\}$ satisfies property (\*).

(ii) First gap (crisp): Determine whether there exists an infinite $A$ satisfying (\*) with
\[\liminf_{N\to\infty} \frac{|A\cap[1,N]|}{\sqrt{N}}>0.
\]
Equivalently, construct such a set with $|A\cap[1,N]|\ge c\sqrt{N}$ for all large $N$, or prove this is impossible.

(iii) Top 3 next moves:
\begin{enumerate}
\item Try to amplify the construction in Lemma~12.2 (prime squares) by allowing more integers while preserving the ``$-1$ nonresidue'' obstruction, e.g. numbers supported on primes $\equiv 3\pmod 4$ with controlled exponents.
\item Use Lemma~12.1 in an aggregated way: pick many small $a\in A$ and derive a global upper bound on $|A\cap[1,N]|$ by counting residue classes simultaneously (a kind of ``entropy'' or graph container approach).
\item For Q3, attempt to show that if $\sum_{n\in A}1/n=\infty$ then one can find infinitely many scales where $A$ is too dense in an interval, forcing a forbidden triple.
\end{enumerate}

(iv) Minimal counterexample structure: For a negative answer to Q3, a ``minimal'' counterexample would be an infinite set $A$ satisfying (\*) and having divergent reciprocal sum. Such an $A$ would necessarily place many elements in low ranges (to make $\sum 1/n$ diverge), yet Lemma~12.1 suggests that each small $a$ severely restricts the residue distribution of larger elements. A plausible minimal counterexample would therefore have a highly structured residue pattern mod many small $a$ (akin to a Chinese-remainder style construction), balancing divergence of $\sum 1/n$ against the per-modulus obstructions.


