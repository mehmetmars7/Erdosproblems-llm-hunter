\subsection*{Erd\H{o}s problem \#342}

\noindent\textbf{1) FORMAL RESTATEMENT.}
Define the Ulam sequence $A=\{a_1<a_2<\cdots\}$ by $a_1=1$, $a_2=2$, and for $n\ge 2$, let $a_{n+1}$ be the smallest integer $>a_n$ that can be expressed as $a_i+a_j$ in \emph{exactly one} way with $1\le i<j\le n$.

The questions in the statement include:
\begin{itemize}
\item Does $A$ have positive density? what is $|A\cap[1,x]|$?
\item Are the successive differences $a_{n+1}-a_n$ eventually periodic?
\item Are there infinitely many twin pairs $a, a+2\in A$?
\end{itemize}

\medskip
\noindent\textbf{2) QUICK LITERATURE/CONTEXT CHECK.}
The statement gives initial terms
\[
1,2,3,4,6,8,11,13,16,18,26,28,36,38,47,48,\dots
\]
and reports that numerical evidence suggests about $7\%$ of integers lie in $A$ and that no periodicity is known.

\medskip
\noindent\textbf{3) ATTACK PLAN.}
\begin{itemize}
\item Prove basic structural facts about the greedy/unique-representation construction (infinity, permanence of uniqueness).
\item Compute a moderately long prefix to check twin pairs and density heuristics.
\end{itemize}

\medskip
\noindent\textbf{4) WORK.}

\medskip
\noindent\textbf{Lemma 342.1 (the Ulam sequence is infinite).}
For every $n\ge 2$, the integer $a_n+a_{n-1}$ has a unique representation as a sum $a_i+a_j$ with $1\le i<j\le n$. In particular, $a_{n+1}$ exists for all $n$.

\noindent\emph{Proof.}
Consider $s:=a_n+a_{n-1}$. It is representable as $a_{n-1}+a_n$.
We claim there is no other representation using indices $\le n$.
Any sum $a_i+a_j$ with $j\le n-1$ is at most $a_{n-1}+a_{n-2}<a_{n-1}+a_n=s$ because $a_n>a_{n-2}$.
Thus any representation of $s$ must involve $a_n$.
But if $a_n+a_i=s$, then $a_i=a_{n-1}$, giving the same pair.
Hence $s$ has exactly one representation among pairs from $\{a_1,\dots,a_n\}$.
Since $s>a_n$, the greedy rule has at least one eligible candidate above $a_n$, so $a_{n+1}$ is well-defined.
\hfill$\square$

\medskip
\noindent\textbf{Lemma 342.2 (uniqueness is permanent).}
For each $m\ge 3$, the representation of $a_m$ as $a_i+a_j$ with $1\le i<j<m$ is unique not only at the time $a_m$ is chosen, but forever: there is no other pair $(i',j')$ with $1\le i'<j'<m$ such that $a_m=a_{i'}+a_{j'}$.

\noindent\emph{Proof.}
By definition of the construction, when $a_m$ is chosen it has exactly one representation as $a_i+a_j$ with $i<j<m$.
If there were another representation $a_m=a_{i'}+a_{j'}$ with $i'<j'<m$, it would already exist at the time $a_m$ is selected (since it uses only earlier terms), contradicting the selection rule.
No later terms can create new representations of $a_m$ because all later terms are positive and larger than $a_m$.
\hfill$\square$

\medskip
\noindent\textbf{FAST REALITY CHECK (computed prefix).}
A brute-force generation of the first $200$ Ulam numbers produced:
\begin{itemize}
\item The first $30$ terms match the statement:
\[1,2,3,4,6,8,11,13,16,18,26,28,36,38,47,48,53,57,62,69,72,77,82,87,97,99,102,106,114,126.\]
\item Among the first $200$ terms, there are $54$ twin pairs $a,a+2\in A$ (counted by scanning the set).
\item The $200$th term is $a_{200}=1792$, so $|A\cap[1,1792]|/1792\approx 0.1116$ at this small scale.
\item No perfect period $\le 50$ was detected in the last $200$ successive differences among the first $500$ terms (a finite check; not a proof).
\end{itemize}

\medskip
\noindent\textbf{5) VERIFICATION.}
Lemmas~342.1--342.2 are proved by direct inequalities using the strict increase of $a_n$.
The computation is an exact implementation maintaining counts of representations by distinct pairs.

\medskip
\noindent\textbf{6) FINAL.}

\noindent\textbf{UNRESOLVED}

\smallskip
\noindent (i) \textbf{Strongest fully proved partial result obtained here.}
We proved that the Ulam sequence is well-defined and infinite (Lemma~342.1) and that each selected term has a permanently unique representation by earlier terms (Lemma~342.2). Computations up to $200$ terms show many twin pairs in that range.

\smallskip
\noindent (ii) \textbf{Exact first gap.}
We did not prove any asymptotic statement about density, growth of $a_n$, periodicity of differences, or infinitude of twin pairs. The first gap is controlling representation counts globally as $n$ grows.

\smallskip
\noindent (iii) \textbf{Top 3 next moves (concrete targets).}
\begin{enumerate}
\item Prove any nontrivial bound on $|A\cap[1,x]|$ (even $x^c$ upper/lower bounds) by estimating how many integers have exactly one representation by current pairs.
\item Develop a ``local limit'' heuristic/lemma: show that representation counts around scale $a_n$ behave like a random model, then try to make at least one implication rigorous (e.g. infinitely many twins or positive density).
\item Search for and prove modular constraints (if any) on $A$ that could force or preclude eventual periodicity of differences.
\end{enumerate}

\smallskip
\noindent (iv) \textbf{Minimal counterexample structure.}
A counterexample to positive density would be evidence that $|A\cap[1,x]|/x\to 0$, meaning that as $x$ grows, integers with exactly one representation as $a_i+a_j$ become sparse. A counterexample to ``infinitely many twins'' would mean there exists $X$ such that no twin pair occurs above $X$, forcing strong constraints on which even gaps can appear.


