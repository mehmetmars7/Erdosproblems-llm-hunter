
\subsection*{FORMAL RESTATEMENT}
Let $p>q\ge 2$ be coprime integers.
Let
\[\mathcal{S} := \{p^k q^\ell : k,\ell\in\mathbb{Z}_{\ge 0}\}.\]
A \emph{representation} of a positive integer $n$ is a finite set of \emph{distinct} elements $\{s_1,\dots,s_m\}\subseteq\mathcal{S}$ such that
\[n=s_1+\cdots+s_m\]
and no $s_i$ divides $s_j$ for $i\ne j$.
(Equivalently, the exponent pairs $(k_i,\ell_i)$ form an antichain in the product order: $(k_i,\ell_i)\not\le (k_j,\ell_j)$ and $(k_j,\ell_j)\not\le (k_i,\ell_i)$ for $i\ne j$.)
We call $n$ \emph{representable} if it has such a representation.

The problem asks (for $\{p,q\}\ne\{2,3\}$):
\begin{itemize}
\item What can be said about the asymptotic density of \emph{non-representable} numbers?
\item Are there infinitely many \emph{coprime non-representable} numbers?
(Here the natural interpretation is: infinitely many non-representable $n$ with $\gcd(n,pq)=1$.)
\end{itemize}

It also asks (in the special case $\{p,q\}=\{2,3\}$) about the fastest-growing function $f(n)\to\infty$ such that every sufficiently large $n$ has a representation by $2^k3^\ell$ with all summands $>f(n)$.

\subsection*{QUICK LITERATURE/CONTEXT CHECK}
The problem file states (Erd\H{o}s--Lewin) that there are finitely many non-representable numbers if and only if $\{p,q\}=\{2,3\}$, and it includes a short induction proof that every $n$ is representable for $\{2,3\}$.
It also reports several results of Yu--Chen and Yang--Zhao about density and about the function $f(n)$.
I will not use any results not explicitly stated in the problem file.

\textbf{Note.} My small-scale computations below suggest that for some pairs (e.g. $(p,q)=(11,2)$) non-representable numbers are extremely frequent up to $50{,}000$, which would be inconsistent with ``density zero of non-representables''. This could indicate that the density statement in the problem text is about representables (or has hypotheses swapped), or that the true asymptotic regime occurs far beyond the tested range. I do not rely on that contextual sentence.

\subsection*{ATTACK PLAN}
\emph{Proof-track strategies.}
\begin{itemize}
\item For $\{2,3\}$, formalize and verify the given induction, including the non-divisibility check.
\item For general $(p,q)$, exploit scaling closure: if $n$ is representable then so are $pn$ and $qn$; attempt to combine this with a ``basis'' of representable residues.
\item Look for modular obstructions (e.g. modulo $p^a$ or $q^b$) to produce infinite non-representables or to bound density.
\end{itemize}

\emph{Disproof/Construction strategies.}
\begin{itemize}
\item Computationally enumerate representable sums for given $(p,q)$ and look for patterns in the non-representable set.
\item Try to force nonrepresentability by considering numbers in short intervals where all available antichain sums are too sparse.
\end{itemize}

\subsection*{WORK}
\paragraph{FAST REALITY CHECK (computation).}
For fixed $(p,q)$ and a bound $N$, I enumerated all antichains of the poset of exponent pairs $(k,\ell)$ with $p^k q^\ell\le N$, and recorded the set of achievable sums $\le N$.
The following are exact counts for representability in $\{0,1,\dots,N\}$.

\begin{itemize}
\item $(p,q)=(3,2)$ (i.e. $\{2,3\}$): for $N=500$, every integer $0\le n\le 500$ was representable (non-representables: $0$).

\item $(p,q)=(5,2)$: the fraction of non-representables decreased but stayed substantial:
\[
\begin{array}{c|c|c}
N & \#\{1\le n\le N: n \text{ non-representable}\} & \text{density}\\hline
2000 & 785 & 0.3925\\
50000 & 15638 & 0.31275\\
200000 & 52995 & 0.264975
\end{array}
\]
For $N=200$, the first $30$ non-representables were
\[3,6,11,12,15,17,19,22,23,24,30,31,34,38,44,46,47,48,49,55,59,60,62,63,68,71,73,75,76,79.\]

\item $(p,q)=(11,2)$: representables were very sparse in the tested range; for $N=50000$ the non-representable density was about $0.89948$.
For $N=200$, the first $30$ non-representables were
\[3,5,6,7,9,10,12,14,17,18,20,21,23,24,25,28,29,31,33,34,35,36,37,39,40,41,42,45,46,47.\]
\end{itemize}

These computations illustrate that representability behavior varies strongly with $(p,q)$.

\paragraph{Lemma 1110.1 (the $\{2,3\}$ case: every $n$ is representable).}
Let $p=3$ and $q=2$.
Then every integer $n\ge 1$ is representable as a sum of distinct numbers of the form $2^a3^b$, with no summand dividing another.
Moreover, one can ensure the stronger property:
\begin{quote}
If $n$ is even, then $n$ has a representation in which every summand is even.
\end{quote}

\paragraph{Proof.}
We prove the stronger property by induction on $n\ge 1$.

\emph{Base cases.}
For $n=1$, take the single summand $1=2^0 3^0$.
For $n=2$, take the single summand $2=2^1 3^0$ (even summand).

\emph{Inductive step.}
Assume the claim holds for all positive integers $<n$.

\underline{Case 1: $n$ even.}
Write $n=2m$ with $m<n$.
By the induction hypothesis applied to $m$, there is a representation
\[m=\sum_{i=1}^t 2^{a_i}3^{b_i}
\]
with distinct summands and no divisibility among them.
Multiplying by $2$ gives
\[n=2m=\sum_{i=1}^t 2^{a_i+1}3^{b_i}.
\]
All summands are even.
If $2^{a_i}3^{b_i}$ does not divide $2^{a_j}3^{b_j}$ then $(a_i,b_i)$ is not coordinatewise $\le (a_j,b_j)$.
Adding $1$ to every $a_i$ preserves all coordinatewise comparisons, so none of the new summands divides another.
Distinctness is also preserved.
Thus this is a valid representation of $n$ with all summands even.

\underline{Case 2: $n$ odd.}
Let $3^k$ be the largest power of $3$ such that $3^k\le n$.
Define $m:=n-3^k$.
Then $m\ge 0$ and $m$ is even (odd minus odd).
If $m=0$, then $n=3^k$ is representable with a single summand.
If $m>0$, then by Case 1 applied to $m$ (since $m<n$), there exists a representation
\[m=\sum_{i=1}^t 2^{a_i}3^{b_i}
\]
in which every summand is even, hence each $a_i\ge 1$.
Then
\[n=3^k + \sum_{i=1}^t 2^{a_i}3^{b_i}
\]
is a sum of allowed terms.
We must check the non-divisibility condition.
Since $3^k$ is odd, it cannot divide any even summand $2^{a_i}3^{b_i}$.
Conversely, an even summand cannot divide $3^k$ because any divisor of $3^k$ is odd.
Thus no divisibility relation involves $3^k$.
Among the even summands, the induction hypothesis already guarantees no divisibility.
Finally, $3^k$ is distinct from the even summands.
Hence this is a valid representation of $n$.

This completes the induction. \hfill $\square$

\paragraph{Lemma 1110.2 (scaling closure).}
Fix coprime integers $p>q\ge 2$.
If $n$ is representable, then $pn$ and $qn$ are representable.

\paragraph{Proof.}
Suppose $n=\sum_{i=1}^m p^{k_i}q^{\ell_i}$ is a representation with no divisibility among the summands.
Then
\[pn=\sum_{i=1}^m p^{k_i+1}q^{\ell_i}\]
uses summands still of the form $p^k q^\ell$.
For two indices $i\ne j$, we have
\[p^{k_i}q^{\ell_i}\mid p^{k_j}q^{\ell_j} \iff (k_i\le k_j\ \text{and}\ \ell_i\le \ell_j).
\]
Adding $1$ to both $k_i$ and $k_j$ preserves the truth of $k_i\le k_j$, so the divisibility relations among summands are unchanged.
Hence no new divisibility relation is created, and the summands remain distinct.
Thus $pn$ is representable.
The same argument with $q$ in place of $p$ shows $qn$ is representable. \hfill $\square$

\subsection*{VERIFICATION}
\begin{itemize}
\item In Lemma 1110.1 (odd case), the key non-divisibility check is parity: the representation of $m$ uses only even terms, so no even term can divide the odd term $3^k$, and $3^k$ cannot divide any even term.
\item Lemma 1110.2 uses only the coprimeness of $p$ and $q$ to interpret divisibility as coordinatewise comparison of exponent pairs, which is valid because $p$ and $q$ have disjoint prime factors.
\item The computations enumerate all antichains among terms $\le N$ and sum them, so the representability check within $[0,N]$ is exhaustive.
\end{itemize}

\subsection*{FINAL}
\textbf{UNRESOLVED}

(i) \textbf{Strongest proved partial result.}
Lemma 1110.1 gives a complete proof that for $\{p,q\}=\{2,3\}$ every integer is representable (with the stronger parity property).
Lemma 1110.2 shows representability is closed under multiplication by $p$ and by $q$.

(ii) \textbf{First gap.}
For $\{p,q\}\ne\{2,3\}$, determine whether the set of non-representable integers has positive density, zero density, or fails to have a density; and determine whether infinitely many $n$ with $\gcd(n,pq)=1$ are non-representable.

(iii) \textbf{Top 3 next moves.}
1. For fixed $(p,q)$, attempt to find an explicit modular obstruction: a modulus $M$ such that the set of achievable sums modulo $M$ misses a residue class, and then lift to an infinite family.
2. Use Lemma 1110.2 to reduce the problem to ``primitive'' nonrepresentables (those not divisible by $p$ or $q$) and study their distribution.
3. Extend computation to larger $N$ for representative parameter regimes (e.g. $q=2$ with various $p$) and attempt to fit the growth of the counting function of nonrepresentables; in particular, test whether the nonrepresentable density appears to converge.

(iv) \textbf{What a minimal counterexample would likely look like.}
A minimal non-representable integer (for fixed $(p,q)$) should be coprime to $pq$ (otherwise Lemma 1110.2 forces entire multiplicative chains of nonrepresentables), and it should avoid being close to a single large term $p^k q^\ell$ in a way that would allow a greedy subtraction while maintaining the antichain condition. Structurally, it would likely lie in a residue class that is hard to hit by antichain sums modulo some $M$.


