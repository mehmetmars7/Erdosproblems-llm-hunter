
Is there an infinite graph $G$ which contains no $K_4$ and is not the union of countably many triangle-free graphs? A problem of Erd\H{o}s and Hajnal. Folkman \cite{Fo70} and Ne\v{s}et\v{r}il and R\"{o}dl \cite{NeRo75} have proved that for every $n\geq 1$ there is a graph $G$ which contains no $K_4$ and is not the union of $n$ triangle-free graphs. See also [582] and [596] . References [Fo70] Folkman, Jon, Graphs with monochromatic complete subgraphs in every edge coloring . SIAM J. Appl. Math. (1970), 19-24. [NeRo75] Ne\u set\u ril, Jaroslav and R\"odl, Vojt\v ech, Type theory of partition properties of graphs . (1975), 405-412.

\bigskip
\noindent\textbf{FORMAL RESTATEMENT}

A (simple, undirected) graph is $G=(V,E)$ with $E\subseteq [V]^2$.
A graph is \emph{$K_4$-free} if it has no 4-vertex clique.
A graph is \emph{triangle-free} if it has no 3-vertex clique.

Say that $G$ is a \emph{countable union of triangle-free graphs} if there exist edge-sets $E_0,E_1,\dots\subseteq E$ such that $E=\bigcup_{n<\omega} E_n$ and each graph $(V,E_n)$ is triangle-free.

Question: Does there exist an infinite (necessarily uncountable, see Lemma~\ref{lem:countable_trivial}) $K_4$-free graph $G$ such that $E(G)$ is \emph{not} a countable union of triangle-free graphs?
Equivalently: does there exist an infinite $K_4$-free graph $G$ such that every countable edge-colouring of $E(G)$ yields a monochromatic triangle?

\bigskip
\noindent\textbf{QUICK LITERATURE/CONTEXT CHECK}

I will not import external results beyond what is explicitly stated in the problem text.
The text states that for each finite $n$ there is a (necessarily finite or infinite) $K_4$-free graph not decomposable into $n$ triangle-free graphs (Folkman; Ne\v{s}et\v{r}il--R\"{o}dl).
The question asks for the countably infinite version ($n=\aleph_0$), and is presented as open.

\bigskip
\noindent\textbf{ATTACK PLAN}

\emph{Proof-track (existence) ideas.}
\begin{itemize}
\item Try to build an uncountable $K_4$-free graph $G$ with strong edge-Ramsey property $G\to(K_3)_{\aleph_0}$ (every countable edge-colouring yields monochromatic triangle), e.g. via transfinite recursion ensuring that each new vertex ``codes'' many constraints.
\item Use known finite $n$-Folkman graphs as building blocks and attempt a compactness/limit argument to force monochromatic triangles for countably many colours.
\end{itemize}

\emph{Disproof-track (nonexistence) ideas.}
\begin{itemize}
\item Attempt to prove that every $K_4$-free graph admits a countable edge-colouring with no monochromatic triangle, perhaps by decomposing edges into countably many triangle-free subgraphs using structural properties of $K_4$-free graphs.
\item Try to show any $K_4$-free graph has ``countable triangle arboricity'' (a countable decomposition avoiding monochromatic triangles) by a rank decomposition of triangles.
\end{itemize}

I did not resolve existence/nonexistence. I record two equivalences/necessary conditions that constrain any counterexample.

\bigskip
\noindent\textbf{WORK}

\medskip
\noindent\textbf{FAST REALITY CHECK}

\begin{itemize}
\item For $n=1$: a graph is not the union of $1$ triangle-free graph exactly if it contains a triangle. There are infinite $K_4$-free graphs with triangles (e.g. a disjoint union of infinitely many triangles), so the finite-$n$ analogue is trivial for $n=1$.
\item For countably many colours, if $G$ is countable then one can colour edges injectively with $\omega$ colours, so there is no monochromatic triangle. Thus any counterexample must be uncountable (Lemma~\ref{lem:countable_trivial}).
\end{itemize}

\medskip
\noindent\textbf{Lemma 1 (countable union vs countable edge-colouring).}\label{lem:countable_union_colouring}
For a graph $G=(V,E)$, the following are equivalent:
\begin{enumerate}
\item $E$ is a countable union of triangle-free edge-sets, i.e. $E=\bigcup_{n<\omega} E_n$ with each $(V,E_n)$ triangle-free.
\item There exists a colouring $\kappa:E\to\omega$ such that for each $n<\omega$, the colour class $E_n:=\kappa^{-1}(\{n\})$ is triangle-free.
Equivalently, $\kappa$ has no monochromatic triangle.
\end{enumerate}

\emph{Proof.}
(1)$\Rightarrow$(2):
Given $E=\bigcup_{n<\omega} E_n$ with each $(V,E_n)$ triangle-free, define a colouring $\kappa:E\to\omega$ by
\[
\kappa(e):=\min\{n\in\omega: e\in E_n\}.
\]
Then $\kappa$ is well-defined since the $E_n$ cover $E$.
Fix $n$ and consider the colour class $\kappa^{-1}(\{n\})$.
By definition, $\kappa^{-1}(\{n\})\subseteq E_n$.
Any subgraph of a triangle-free graph is triangle-free, so $(V,\kappa^{-1}(\{n\}))$ is triangle-free.
Hence there is no monochromatic triangle.

(2)$\Rightarrow$(1):
Given $\kappa:E\to\omega$, let $E_n:=\kappa^{-1}(\{n\})$.
Then $E=\bigcup_{n<\omega}E_n$ and each $(V,E_n)$ is triangle-free because otherwise a triangle in $(V,E_n)$ would be a monochromatic triangle of colour $n$.
\qed

\medskip
\noindent\textbf{Lemma 2 (countable graphs are never counterexamples).}\label{lem:countable_trivial}
If a graph $G=(V,E)$ has countably many edges (in particular if $V$ is countable), then $G$ is a countable union of triangle-free graphs.

\emph{Proof.}
Enumerate the edges as $E=\{e_0,e_1,e_2,\dots\}$.
For each $n$, let $E_n:=\{e_n\}$.
Then each $(V,E_n)$ has at most one edge and hence is triangle-free.
Moreover $E=\bigcup_{n<\omega} E_n$.
So $G$ is a countable union of triangle-free graphs.\qed

\bigskip
\noindent\textbf{VERIFICATION}

\begin{itemize}
\item Lemma~\ref{lem:countable_union_colouring} uses the ``minimal index'' trick to pass from a cover to a genuine colouring; the triangle-free property is preserved under taking subsets.
\item Lemma~\ref{lem:countable_trivial} shows that any genuine counterexample must be uncountable (at least uncountably many edges). This matches the intuition that the question is only nontrivial in the uncountable regime.
\end{itemize}

\bigskip
\noindent\textbf{FINAL}

\textbf{UNRESOLVED}

(i) \emph{Strongest proved partial result.}
The problem is equivalent to asking whether there exists a $K_4$-free graph $G$ such that every countable edge-colouring of $E(G)$ has a monochromatic triangle (Lemma~\ref{lem:countable_union_colouring}). Any such $G$ must be uncountable (Lemma~\ref{lem:countable_trivial}).

(ii) \emph{First gap (crisp).}
Either construct an uncountable $K_4$-free graph $G$ with the Ramsey property $G\to(K_3)_{\aleph_0}$, or prove that every $K_4$-free graph admits a countable edge-colouring with no monochromatic triangle.

(iii) \emph{Top 3 next moves.}
\begin{enumerate}
\item Try to prove a decomposition theorem for $K_4$-free graphs: can one cover $E(G)$ by countably many triangle-free subgraphs via a transfinite rank on edges/triangles?
\item Construct candidate counterexamples from known finite $n$-Folkman graphs by transfinite ``amalgamation'' ensuring $K_4$-freeness is preserved while forcing monochromatic triangles for any countable colouring.
\item Run small-model computations for finite approximations: search for graphs $G$ with $\omega$-like decomposition obstructions to guide an infinite construction.
\end{enumerate}

(iv) \emph{Minimal counterexample structure.}
Any counterexample must be uncountable and $K_4$-free, yet so edge-Ramsey for triangles that countably many colours cannot destroy all monochromatic triangles.
Since $K_4$-free graphs can still have many triangles, a counterexample would likely have triangles ``spread'' so that any countable partition of edges captures an entire triangle in some part.


