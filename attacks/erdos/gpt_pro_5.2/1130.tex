% Erdos Problem #1130

\subsection*{FORMAL RESTATEMENT}
Let $n\ge 1$ and let $x_1<\dots<x_n$ be distinct points in $[-1,1]$.
Define the Lagrange polynomials $l_k$ as in Problem 1129.
Set $x_0=-1$ and $x_{n+1}=1$ and define
\[
\Upsilon(x_1,\dots,x_n):=\min_{0\le i\le n}\ \max_{x\in[x_i,x_{i+1}]}\ \sum_{k=1}^n |l_k(x)|.
\]
Questions:
\begin{enumerate}
\item Is it true that for all node sets, $\Upsilon(x_1,\dots,x_n)\ll \log n$?
\item Describe node sets $x_1,\dots,x_n$ that maximize $\Upsilon(x_1,\dots,x_n)$.
\end{enumerate}

\subsection*{QUICK LITERATURE/CONTEXT CHECK}
I only record context explicitly stated in the problem file.
The file states that Erd\H{o}s proved $\Upsilon(x_1,\dots,x_n)<\sqrt{n}$, and suggests a conjectured characterization of maximizers via equalizing certain interval maxima.

\subsection*{ATTACK PLAN}
\textbf{Proof track ideas.}
\begin{itemize}
\item Relate $\Upsilon$ to the Lebesgue constant $\Lambda$ from Problem 1129 via simple inequalities (Lemma 1130.1).
\item Investigate configurations where all interval-wise maxima are large; maximizing $\Upsilon$ means even the \emph{smallest} interval maximum is large.
\end{itemize}
\textbf{Disproof track ideas.}
\begin{itemize}
\item Try to construct node sets with $\Upsilon$ growing faster than $\log n$ (e.g. by clustering points so that $\sum |l_k(x)|$ is large on every subinterval).
\end{itemize}

\subsection*{WORK}
\textbf{Lemma 1130.1 (comparison with Lebesgue constant).}
Let
\[
\Lambda(x_1,\dots,x_n):=\max_{x\in[-1,1]}\sum_{k=1}^n|l_k(x)|
\]
be the Lebesgue constant from Problem 1129.
Then
\[
1\le \Upsilon(x_1,\dots,x_n)\le \Lambda(x_1,\dots,x_n).
\]

\emph{Proof.}
For the lower bound, Lemma 1129.1 gives $\sum_k l_k(x)=1$ for all $x$, hence by the triangle inequality $\sum_k|l_k(x)|\ge 1$ for all $x\in[-1,1]$.
Therefore each interval maximum $\max_{x\in[x_i,x_{i+1}]}\sum_k|l_k(x)|$ is at least $1$, so the minimum over $i$ is at least $1$.

For the upper bound, for each fixed $i$ we have
\[
\max_{x\in[x_i,x_{i+1}]}\sum_k|l_k(x)|\le \max_{x\in[-1,1]}\sum_k|l_k(x)|=\Lambda.
\]
Taking the minimum over $i$ preserves the inequality, giving $\Upsilon\le\Lambda$.
\qed

\textbf{Lemma 1130.2 (endpoint inclusion forces $\Upsilon=1$).}
If $x_1=-1$ and $x_n=1$, then $\Upsilon(x_1,\dots,x_n)=1$.

\emph{Proof.}
If $x_1=-1$, then the interval $[x_0,x_1]=[-1,-1]$ is a single point.
At $x=-1$ we have $l_1(-1)=1$ and $l_k(-1)=0$ for $k\ne 1$, hence $\sum_k|l_k(-1)|=1$.
Thus
\[
\max_{x\in[x_0,x_1]}\sum_k|l_k(x)| = 1.
\]
Similarly, if $x_n=1$, then on the degenerate interval $[x_n,x_{n+1}]=[1,1]$ we have $\sum_k|l_k(1)|=1$.
Therefore the minimum over $i$ of these interval maxima is at most $1$.
Combined with the universal lower bound $\Upsilon\ge 1$ from Lemma 1130.1, we get $\Upsilon=1$.
\qed

\textbf{FAST REALITY CHECK (numerics for small examples).}
The value of $\Upsilon$ depends strongly on whether endpoints are included as nodes.
\begin{verbatim}
Nodes [-1,0,1] (n=3): interval maxima [1.0, 1.25, 1.25, 1.0]
  => Upsilon = min(...) = 1.0

Nodes [-0.5,0,0.5] (n=3): interval maxima [7.0, 1.25, 1.25, 7.0]
  => Upsilon = 1.25

Nodes [-0.9,-0.3,0.2] (n=3): interval maxima [1.8, 1.327273, 1.189394, 11.133333]
  => Upsilon ≈ 1.189394
\end{verbatim}

\subsection*{VERIFICATION}
\begin{itemize}
\item Lemma 1130.1 uses only Lemma 1129.1 and monotonicity of max/min with respect to domain inclusion.
\item Lemma 1130.2 carefully checks the degenerate intervals when endpoints are among the nodes.
\item Numerical interval maxima are grid-based and should be read as approximations.
\end{itemize}

\subsection*{FINAL}
\textbf{UNRESOLVED}

(i) \textbf{Strongest proved partial result.}
Always $1\le \Upsilon\le\Lambda$ (Lemma 1130.1), and if the node set includes both endpoints $\{-1,1\}$ then $\Upsilon=1$ exactly (Lemma 1130.2).

(ii) \textbf{First gap (crisp).}
Prove or disprove the conjectured upper bound $\Upsilon(x_1,\dots,x_n)\ll \log n$ uniformly over all node sets.

(iii) \textbf{Top 3 next moves.}
\begin{itemize}
\item Search for constructions where every interval $[x_i,x_{i+1}]$ contains points where $\sum_k|l_k(x)|$ is large, so that the \emph{minimum} of the interval maxima grows.
\item Attempt to relate $\Upsilon$ to more tractable functionals (e.g. $L^2$ norms) using inequalities like Cauchy--Schwarz on each subinterval.
\item For candidate maximizers, attempt to prove a necessary equalization condition: at optimality, all interval maxima should coincide.
\end{itemize}

(iv) \textbf{Minimal counterexample structure.}
A counterexample to $\Upsilon\ll\log n$ would be a family of node sets $x^{(n)}_1<\dots<x^{(n)}_n$ such that on \emph{every} subinterval $[x^{(n)}_i,x^{(n)}_{i+1}]$ the quantity $\sum_k|l_k(x)|$ attains values much larger than $\log n$, forcing their minimum to grow faster than $\log n$.


