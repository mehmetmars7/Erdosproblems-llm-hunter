\section{Round 2: Continuation and Gap-Closure Mode (Erd\H{o}s--Tur\'an, order 2)}

\subsection{1) ROUND-2 OBJECTIVE}
\textbf{Path (A).} Continue the Round~1 investigation in the difficult ``thin'' regime by adding a \emph{generating-function/Fourier} inequality that constrains a hypothetical bounded representation function. The aim is strict progress (not a full resolution): obtain a quantitative necessary condition for bounded \(r_A\), and isolate a rigorously proved obstruction to several common proof strategies.

\subsection{2) ROUND-1 FOUNDATION USED}
I use the following Round~1 results \emph{as given}.
\begin{enumerate}
\item \textbf{Lemma 28.1.} If \(A+A\) contains every integer \(\ge N_0\), then with \(S_N:=A\cap[1,N]\) and \(m_N=|S_N|\) one has
\[
\frac{m_N(m_N+1)}{2}\ge N-N_0+1,\quad\text{hence }m_N\gtrsim\sqrt N.
\]
\item \textbf{Lemma 28.2.} For any finite \(S\subseteq\{1,\dots,N\}\),
\[
\max_{2\le n\le 2N} r_S(n)\ge \frac{|S|^2}{2N-1}.
\]
\item \textbf{Corollary 28.3.} If \(|A\cap[1,N]|/\sqrt N\) is unbounded along a sequence, then \(\limsup_{n\to\infty} r_A(n)=\infty\).
\item Truncation: for \(n\le N\), every representation \(n=a+b\) uses \(a,b\le N\), hence \(r_A(n)=r_{A\cap[1,N]}(n)\) for \(n\le N\).
\end{enumerate}

\subsection{3) NEW INSIGHT / TOOL (ROUND-2)}
The new tool is a generating-function inequality (a S\'andor-type estimate) that links \(\limsup r_A\) to \(\liminf r_A\).

\begin{theorem}[S\'andor-type bound]
Let \(R(n)=r_A(n)\). If \(\limsup_{n\to\infty} R(n)=K<\infty\), then
\[
\liminf_{n\to\infty} R(n)\le K-2\sqrt K+1=(\sqrt K-1)^2.
\]
\end{theorem}

In addition, two rigorously checked contextual facts (used as consequences/obstructions, not as black boxes inside the proof above) are:
\begin{enumerate}
\item For an asymptotic basis \(A\), one has \(r_A(n)\ge 8\) for infinitely many \(n\) (hence \(\limsup r_A(n)\ge 8\)).
\item There exists an asymptotic basis \(A\subset\mathbb N\) such that \(\{n: r_A(n)=2\}\) has natural density \(1\). Thus large values of \(r_A(n)\) cannot be forced on a positive-density set in general.
\end{enumerate}

\subsection{4) ATTACK PLAN (ROUND-2)}
\textbf{Round~1 gap.} Round~1 proves Erd\H{o}s--Tur\'an under the extra hypothesis that \(|A\cap[1,N]|/\sqrt N\) is unbounded; it leaves open the ``thin'' regime \(|A\cap[1,N]|\asymp\sqrt N\), where bounded representation could conceivably occur.

\textbf{Round~2 plan.} Assume \(\limsup r_A(n)=K<\infty\) and try to contradict eventual lower bounds on \(r_A(n)\) by analytic control of the generating function \(F(z)=\sum_{a\in A} z^a\), noting that
\[
F(z)^2=\sum_{n\ge 0} r_A(n)z^n.
\]
Compare \(F(z)^2\) to a tuned geometric series \(\sum C z^n = C/(1-z)\), integrate on \(|z|=r\) with \(r=1-1/N\), and combine a \(\log N\) upper bound for \(\int |1-re^{2\pi i\alpha}|^{-1}\,d\alpha\) with a \(\sqrt N\) lower bound for \(F(r^2)\) forced by the growth constraints coming from the basis property.

\subsection{5) WORK (ROUND-2)}
\subsubsection{5.1 Setup and notation}
Let \(A=\{a_1<a_2<\cdots\}\subseteq\mathbb N_0\). Define the ordered representation function
\[
R(n):=r_A(n)=\#\{(i,j): a_i+a_j=n\}.
\]
Define the generating function (for \(|z|<1\))
\[
F(z):=\sum_{k\ge 1} z^{a_k}.
\]
Then
\[
F(z)^2=\sum_{n\ge 0} R(n)z^n.
\]
Assume \(\limsup_{n\to\infty} R(n)=K<\infty\).

\subsubsection{5.2 A growth dichotomy lemma}
\begin{lemma}\label{lem:growth-dichotomy}
If \(a_n>\frac{n(n+1)}{2}\) for infinitely many \(n\), then \(R(m)=0\) for infinitely many \(m\). In particular \(\liminf_{m\to\infty} R(m)=0\).
\end{lemma}
\begin{proof}
Fix such an \(n\). For any \(m\le a_n\), any representation \(m=a_i+a_j\) must use \(a_i,a_j\le m\le a_n\), hence \(i,j\le n\). Thus every represented \(m\in[0,a_n]\) arises from an \emph{unordered} pair \(\{i,j\}\) with \(1\le i\le j\le n\). There are exactly \(n(n+1)/2\) such pairs, hence at most \(n(n+1)/2\) distinct sums \(a_i+a_j\) with \(i\le j\le n\). Since \([0,a_n]\) contains \(a_n+1>n(n+1)/2\) integers, at least one \(m\in[0,a_n]\) is not representable; i.e. \(R(m)=0\). Repeating for infinitely many such \(n\) yields infinitely many such \(m\).
\end{proof}

Hence, in proving the main inequality, we may restrict to the complementary case:
\begin{equation}\label{eq:growth-cond}
\exists n_1\ \forall n\ge n_1:\quad a_n\le \frac{n(n+1)}{2}.
\end{equation}

\subsubsection{5.3 S\'andor-type inequality: bounded \texorpdfstring{\(\limsup\)}{limsup} forces small \texorpdfstring{\(\liminf\)}{liminf}}
\begin{theorem}\label{thm:sandor}
If \(\limsup_{n\to\infty} R(n)=K\), then
\[
\liminf_{n\to\infty} R(n)\le K-2\sqrt K+1.
\]
\end{theorem}
\begin{proof}
If Lemma~\ref{lem:growth-dichotomy} applies then \(\liminf R(n)=0\) and the claim holds since \((\sqrt K-1)^2\ge 0\). Assume therefore \eqref{eq:growth-cond}.

Assume for contradiction that
\[
\liminf_{n\to\infty}R(n)\ >\ K-2\sqrt K+1.
\]
Then there exist \(\varepsilon>0\) and \(n_2\) such that for all \(n\ge n_2\),
\begin{equation}\label{eq:R-window}
K-2\sqrt K+1+\varepsilon\ \le\ R(n)\ \le\ K.
\end{equation}
Define
\begin{equation}\label{eq:C-def}
C:=K-\sqrt K+\varepsilon.
\end{equation}
Consider \(f(x):=(x-C)^2/x\), continuous on \((0,\infty)\). On the compact interval
\[
I:=[K-2\sqrt K+1+\varepsilon,\ K],
\]
we have \(f(x)<1\) for all \(x\in I\), hence by compactness there exists \(\delta>0\) with
\begin{equation}\label{eq:delta-ineq}
(x-C)^2\le (1-\delta)^2 x\qquad (x\in I).
\end{equation}
Applying \eqref{eq:delta-ineq} to \eqref{eq:R-window} gives, for all \(n\ge n_2\),
\begin{equation}\label{eq:R-C-square}
(R(n)-C)^2\le (1-\delta)^2 R(n).
\end{equation}

Let \(r:=1-1/N\) with \(N\in\mathbb N\) large, and write \(z:=re^{2\pi i\alpha}\) for \(\alpha\in[0,1]\). Define
\[
I_N:=\int_0^1\left|F(z)^2-\sum_{n\ge 0} C z^n\right|\,d\alpha.
\]
Since \(\sum_{n\ge 0} C z^n=C/(1-z)\), this is \(\int_0^1\left|F(z)^2-\frac{C}{1-z}\right|\,d\alpha\).

\medskip\noindent\textbf{Upper bound.}
By Cauchy--Schwarz and Parseval,
\[
I_N\le \left(\int_0^1\left|\sum_{n\ge 0}(R(n)-C)z^n\right|^2\,d\alpha\right)^{1/2}
=\left(\sum_{n\ge 0}(R(n)-C)^2 r^{2n}\right)^{1/2}.
\]
Split at \(n_2\). The initial part \(n<n_2\) is bounded by a constant \(c_1\). For \(n\ge n_2\), use \eqref{eq:R-C-square}:
\[
\sum_{n\ge n_2}(R(n)-C)^2 r^{2n}
\le (1-\delta)^2\sum_{n\ge 0}R(n)r^{2n}
=(1-\delta)^2 F(r^2)^2.
\]
Hence
\begin{equation}\label{eq:IN-upper}
I_N\le c_2+(1-\delta)F(r^2)
\end{equation}
for a constant \(c_2\).

\medskip\noindent\textbf{Lower bound.}
By the triangle inequality,
\[
I_N\ge \int_0^1|F(z)^2|\,d\alpha-\int_0^1\left|\sum_{n\ge 0}Cz^n\right|\,d\alpha.
\]
Since \(|F(z)^2|=|F(z)|^2\), Parseval gives
\[
\int_0^1|F(z)^2|\,d\alpha=\int_0^1|F(z)|^2\,d\alpha=\sum_{k\ge 1} r^{2a_k}=F(r^2).
\]
Also
\[
\int_0^1\left|\sum_{n\ge 0}Cz^n\right|\,d\alpha
=C\int_0^1\frac{1}{|1-z|}\,d\alpha.
\]
For \(0\le \alpha\le 1/2\),
\[
|1-z|^2=(1-r)^2+2r(1-\cos 2\pi\alpha)=(1-r)^2+2r\sin^2(\pi\alpha),
\]
and since \(1-r=1/N\) and \(\sin(\pi\alpha)\ge 2\alpha\) on \([0,1/2]\), we get \(|1-z|\ge \max\{1/N,\alpha\}\). Therefore
\[
\int_0^1\frac{1}{|1-z|}\,d\alpha
=2\int_0^{1/2}\frac{1}{|1-z|}\,d\alpha
\le 2\left(\int_0^{1/N}N\,d\alpha+\int_{1/N}^{1/2}\frac{d\alpha}{\alpha}\right)
\le c_3\log N
\]
for some constant \(c_3\). Hence
\begin{equation}\label{eq:IN-lower}
I_N\ge F(r^2)-c_3\log N.
\end{equation}

Combining \eqref{eq:IN-upper} and \eqref{eq:IN-lower} gives
\[
F(r^2)-c_3\log N\le c_2+(1-\delta)F(r^2),
\]
so
\begin{equation}\label{eq:deltaFr2}
\delta\,F(r^2)\le c_2+c_3\log N.
\end{equation}

\medskip\noindent\textbf{Lower bound on \(F(r^2)\).}
Let \(M:=\lfloor\sqrt N\rfloor\). For \(n\) with \(n_1\le n\le M\), the growth condition \eqref{eq:growth-cond} implies
\[
2a_n\le n(n+1)\le N,
\]
hence \(r^{2a_n}\ge r^N=(1-1/N)^N\). For all large \(N\), \((1-1/N)^N\ge 1/4\). Therefore
\[
F(r^2)=\sum_{k\ge 1} r^{2a_k}\ge \sum_{n=n_1}^{M} r^{2a_n}\ge (M-n_1)\cdot (1-1/N)^N\ge c_4\sqrt N
\]
for some constant \(c_4>0\). Substituting into \eqref{eq:deltaFr2} yields
\[
\delta c_4\sqrt N\le c_2+c_3\log N,
\]
which is false for all sufficiently large \(N\). This contradiction shows that \eqref{eq:R-window} cannot hold for any \(\varepsilon>0\). Hence
\(
\liminf_{n\to\infty}R(n)\le K-2\sqrt K+1
\), proving the theorem.
\end{proof}

\subsubsection{5.4 Consequences for the Erd\H{o}s--Tur\'an setting}
If \(A+A\) is cofinite (i.e. \(r_A(n)\ge 1\) for all \(n\ge N_0\)), then
\[
\liminf_{n\to\infty} r_A(n)\ge 1.
\]
If (contrary to Erd\H{o}s--Tur\'an) \(\limsup r_A(n)=K<\infty\), Theorem~\ref{thm:sandor} forces
\[
1\le (\sqrt K-1)^2,
\]
hence \(K\ge 4\).

Independently (as a known global constraint for asymptotic bases of order \(2\)), one has \(r_A(n)\ge 8\) for infinitely many \(n\), so \(\limsup r_A(n)\ge 8\). Moreover, there exists an asymptotic basis with \(r_A(n)=2\) on a set of density \(1\), showing that any approach attempting to force ``many'' large values of \(r_A\) (e.g. positive density) cannot work in full generality.

\subsection{6) ADVERSARIAL VERIFICATION}
\begin{enumerate}
\item \textbf{Ordered vs unordered:} throughout, \(R(n)\) counts \emph{ordered} pairs \((i,j)\), matching Round~1.
\item \textbf{Use of compactness in \eqref{eq:delta-ineq}:} since \(f\) is continuous and \(f(x)<1\) pointwise on compact \(I\), one has \(\max_I f=1-\eta\) for some \(\eta>0\); setting \((1-\delta)^2:=1-\eta\) is valid.
\item \textbf{Integral estimate:} the bound \(\int_0^1 |1-re^{2\pi i\alpha}|^{-1}d\alpha\ll\log N\) is obtained directly via \(|1-z|\ge \max\{1/N,\alpha\}\) on \([0,1/2]\), so no hidden external analytic input is used.
\item \textbf{Lower bound on \(F(r^2)\):} for \(n\le \sqrt N\), \(2a_n\le n(n+1)\le N\), hence \(r^{2a_n}\ge r^N\). The constant lower bound on \((1-1/N)^N\) for large \(N\) is standard.
\item \textbf{Interaction with Round~1:} the new theorem targets the Round~1 gap (bounded \(\limsup\) scenario) and is consistent with the existence of bases where \(r_A(n)=2\) for density \(1\).
\end{enumerate}

\subsection{7) FINAL (exactly one)}
\textbf{UNRESOLVED (BUT STRICTLY ADVANCED).} Round~2 adds a fully detailed generating-function argument yielding Theorem~\ref{thm:sandor}, a quantitative constraint on bounded \(\limsup r_A\), and isolates a rigorously proved obstruction to ``positive-density large values'' strategies (via existence of bases with \(r_A(n)=2\) on density~\(1\)).

\subsection{8) COMPLETION ESTIMATE (mandatory)}
\noindent\textbf{COMPLETION: 40\%}

\subsection{9) REFERENCES}
\begin{enumerate}
\item Cs. S\'andor, \emph{A note on a conjecture of Erd\H{o}s--Tur\'an}, Integers \textbf{8} (2008).
\item Y.-G. Chen, \emph{On the Erd\H{o}s--Tur\'an conjecture}, C. R. Acad. Sci. Paris \textbf{350} (2012).
\item Y. Ding and L. Zhao, \emph{A new upper bound on Ruzsa's number on the Erd\H{o}s--Tur\'an conjecture}, arXiv:2307.12311.
\end{enumerate}
