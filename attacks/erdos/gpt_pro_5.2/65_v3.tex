\section{Erd\H{o}s--Hajnal reciprocal cycle-length sum (Round 3)}

\subsection{1) ROUND-3 OBJECTIVE}

\textbf{Path chosen: (A) proof-oriented gap closure.}
Round~2 left the global minimiser question open but produced
(i) exact minimisers for several small parameter pairs and
(ii) a structural mechanism ruling out dense balanced bipartite competitors.

In this round we add a new \emph{uniform} method for the complete-bipartite family
$K_{2,n-2}$: we prove an extremal bound for graphs with no $3$- or $4$-cycles and
use it to settle several new exact parameter pairs.

\subsection{2) ROUND-2 FOUNDATION USED}

We rely on the following Round~2 results without re-proving them.

\begin{itemize}
  \item \textbf{(R2 Lemma 1)} For $2\le a\le b$, the complete bipartite graph $K_{a,b}$ has distinct cycle lengths
  \[\{4,6,8,\dots,2a\}\qquad\text{and hence}\qquad S(K_{a,b})=\sum_{j=2}^a\frac1{2j}=\tfrac12(H_a-1).\]
  \item \textbf{(R2 Proposition 8--12)} For $(n,m)=(8,12)$, every $8$-vertex $12$-edge graph $G$ satisfies $S(G)\ge 1/4$,
  with equality achieved by $K_{2,6}$.
  \item \textbf{(R2 Theorem ES)} (External tool used in Round~2.) Dense balanced bipartite graphs are bipancyclic above the
  Entringer--Schmeichel threshold, yielding a lower bound on $S(G)$ for that competitor class.
\end{itemize}

\subsection{3) NEW INSIGHT / TOOL (ROUND 3)}

The new tool is a sharp \emph{$2$-path counting inequality} for graphs with girth at least $5$:

\begin{quote}
If $G$ has no $C_3$ and no $C_4$, then its number of edges satisfies
\[|E(G)|\le \frac{n\sqrt{n-1}}{2}.\]
\end{quote}

This immediately forces the presence of a triangle or a $4$-cycle whenever
$m>\frac{n\sqrt{n-1}}{2}$.  For the parameter line $m=2n-4$ (which corresponds to $K_{2,n-2}$),
this yields new \emph{exact minimisation results} for all $n\le 12$, extending Round~2's solved case $(8,12)$
(and subsuming its ad hoc proof).

\subsection{4) ATTACK PLAN (ROUND 3)}

Round~2 proved exact minimisation for $(n,m)=(8,12)$ but had no general method.
The remaining gap for this family is:

\begin{quote}
For which $n$ does every $n$-vertex graph with $m=2n-4$ edges necessarily contain a triangle or a $4$-cycle?
\end{quote}

If this is forced, then automatically $S(G)\ge 1/4$ (since $1/3>1/4$), and the complete bipartite graph
$K_{2,n-2}$ achieves equality because it has only $4$-cycles.

We prove a general extremal bound for $(C_3,C_4)$-free graphs and compare it to $2n-4$.

\subsection{5) WORK (ROUND 3)}

\subsubsection*{5.1. Notation}
For a graph $G$, let
\[\mathcal C(G):=\{\ell\ge 3:\ G\text{ contains a (simple) cycle of length }\ell\},\qquad
S(G):=\sum_{\ell\in\mathcal C(G)}\frac1\ell.\]
The sum is over \emph{distinct} cycle lengths.

\subsubsection*{5.2. A $2$-path bound for $(C_3,C_4)$-free graphs}

\begin{lemma}[Counting $2$-paths]\label{lem:2paths}
Let $G$ be a graph on $n$ vertices with $m$ edges. If $G$ is triangle-free and $C_4$-free, then
\begin{equation}\label{eq:2path-ineq}
\sum_{v\in V(G)}\binom{d(v)}{2}\ \le\ \binom{n}{2}-m.
\end{equation}
\end{lemma}

\begin{proof}
A (directed) $2$-path is an ordered triple $(x,v,y)$ with distinct vertices and edges $xv,vy\in E(G)$.
For each middle vertex $v$, there are $\binom{d(v)}{2}$ unordered endpoint pairs $\{x,y\}$ giving such a $2$-path,
so the total number of (unordered-endpoint) $2$-paths is $\sum_v \binom{d(v)}{2}$.

Because $G$ is triangle-free, the endpoints $x,y$ of any $2$-path $x-v-y$ are \emph{nonadjacent}.
Thus each $2$-path determines a non-edge $\{x,y\}\notin E(G)$.

Because $G$ is $C_4$-free, no unordered pair $\{x,y\}$ can be the endpoints of two distinct $2$-paths:
if $x-v-y$ and $x-v'-y$ were distinct $2$-paths with the same endpoints, then
$x-v-y-v'-x$ would be a $4$-cycle.

Hence the map ``$2$-path $\mapsto$ its endpoint pair'' injects into the set of non-edges,
which has size $\binom{n}{2}-m$. This proves \eqref{eq:2path-ineq}.
\end{proof}

\begin{lemma}[Edge upper bound for girth $\ge 5$]\label{lem:edges-girth5}
Let $G$ be a graph on $n$ vertices with $m$ edges. If $G$ has no $C_3$ and no $C_4$, then
\[m\le \frac{n\sqrt{n-1}}{2}.
\]
In particular, if $m>\frac{n\sqrt{n-1}}{2}$ then $G$ contains a triangle or a $4$-cycle.
\end{lemma}

\begin{proof}
Start from Lemma~\ref{lem:2paths}.
Using $\binom{d}{2}=\frac12(d^2-d)$ and $\sum_v d(v)=2m$, we have
\[
\sum_v\binom{d(v)}{2}=\frac12\sum_v d(v)^2-m.
\]
By Cauchy--Schwarz,
\[\sum_v d(v)^2\ \ge\ \frac1n\Bigl(\sum_v d(v)\Bigr)^2\ =\ \frac{(2m)^2}{n}.
\]
Therefore
\[
\sum_v\binom{d(v)}{2}\ \ge\ \frac12\cdot\frac{4m^2}{n}-m\ =\ \frac{2m^2}{n}-m.
\]
Combine with Lemma~\ref{lem:2paths}:
\[
\frac{2m^2}{n}-m\ \le\ \binom{n}{2}-m.
\]
Cancel $-m$ and rearrange:
\[
\frac{2m^2}{n}\ \le\ \frac{n(n-1)}{2}
\qquad\Longrightarrow\qquad
m^2\ \le\ \frac{n^2(n-1)}{4}
\qquad\Longrightarrow\qquad
m\ \le\ \frac{n\sqrt{n-1}}{2}.
\]
The final ``in particular'' follows by contrapositive.
\end{proof}

\subsubsection*{5.3. New exact minimisers for the $K_{2,n-2}$ line}

\begin{proposition}[Exact minimisation for $(n,m)=(n,2n-4)$ up to $n=12$]\label{prop:2n-4}
Let $7\le n\le 12$ and let $G$ be a graph on $n$ vertices with $m=2n-4$ edges.
Then
\[S(G)\ge \frac14.
\]
Moreover $S(K_{2,n-2})=1/4$, so $\min S(G)=1/4$ for these parameter pairs, and a complete bipartite graph is a minimiser.
\end{proposition}

\begin{proof}
First, by (R2 Lemma~1) the complete bipartite graph $K_{2,n-2}$ has only $4$-cycles, hence
$S(K_{2,n-2})=1/4$.

Now let $G$ be any $n$-vertex graph with $m=2n-4$ edges.
If $G$ contains a triangle $C_3$, then $S(G)\ge 1/3>1/4$.
So we may assume $G$ is triangle-free.

If $G$ contains a $4$-cycle, then $S(G)\ge 1/4$ and we are done.
Thus the only way to have $S(G)<1/4$ would be for $G$ to contain no $C_3$ and no $C_4$.

But for $7\le n\le 12$ we have the strict inequalities
\[
2n-4\ >\ \frac{n\sqrt{n-1}}{2}
\qquad\text{(checked below),}
\]
so Lemma~\ref{lem:edges-girth5} forbids the existence of a $(C_3,C_4)$-free graph with $m=2n-4$.
Therefore every such $G$ has a $C_3$ or $C_4$, and in either case $S(G)\ge 1/4$.

For completeness, the numerical comparisons are:
\[
\begin{array}{c|c|c}
 n & 2n-4 & \tfrac{n\sqrt{n-1}}{2}\\\hline
 7 & 10 & \approx 8.57\\
 8 & 12 & \approx 10.58\\
 9 & 14 & \approx 12.73\\
 10 & 16 & 15\\
 11 & 18 & \approx 17.39\\
 12 & 20 & \approx 19.90
\end{array}
\]
This finishes the proof.
\end{proof}

\begin{remark}[Where the method breaks]
For $n=13$ one has $2n-4=22<\tfrac{13\sqrt{12}}{2}\approx 22.52$, so Lemma~\ref{lem:edges-girth5} no longer
forces a $C_3$ or $C_4$ on the line $m=2n-4$. Thus if the complete-bipartite minimiser conjecture were to fail
first for the family $K_{2,n-2}$, $n=13$ is the first place this counting argument leaves room.
\end{remark}

\subsection{6) ADVERSARIAL VERIFICATION}

\begin{itemize}
\item \textbf{Does Lemma~\ref{lem:2paths} overcount $2$-paths?}
We count unordered endpoint pairs at each middle vertex $v$, giving exactly $\binom{d(v)}{2}$.
Different middle vertices produce distinct $2$-paths. This is the standard counting identity.

\item \textbf{Is ``two $2$-paths with the same endpoints give a $C_4$'' always valid?}
Yes: if $x-v-y$ and $x-v'-y$ with $v\neq v'$, then the edges $xv,vy,yv',v'x$ form a $4$-cycle on $\{x,v,y,v'\}$.
No further edges are needed.

\item \textbf{Triangle-free necessity in Lemma~\ref{lem:2paths}.}
If $xy$ were an edge, a $2$-path $x-v-y$ would create a triangle. Thus in a triangle-free graph,
endpoints of any $2$-path are automatically nonadjacent, which is exactly what allows the injection into non-edges.

\item \textbf{Sharpness / boundary issues.}
Lemma~\ref{lem:edges-girth5} is tight up to rounding for certain $n$ (it is the usual ``Reiman-type'' bound).
For the present application we only need strict separation $2n-4>\tfrac{n\sqrt{n-1}}{2}$,
which holds for $n\le 12$ and fails at $n=13$, consistent with the numerical table.

\item \textbf{Cycle lengths in $K_{2,n-2}$.}
Any simple cycle in a bipartite graph alternates between parts. With a part of size $2$, a simple cycle can use
at most $2$ vertices from that part, hence must have length exactly $4$. So $\mathcal C(K_{2,n-2})=\{4\}$ and
$S(K_{2,n-2})=1/4$.
\end{itemize}

\subsection{7) FINAL (EXACTLY ONE)}

\textbf{UNRESOLVED (BUT STRICTLY ADVANCED).}

Round~3 adds a new extremal lemma for $(C_3,C_4)$-free graphs and uses it to prove an explicit new block
of exact minimiser instances: for every $7\le n\le 12$ with $m=2n-4$, one has $\min S(G)=1/4$, achieved by
$K_{2,n-2}$.  This both extends and conceptually simplifies the Round~2 solved case $(8,12)$.
The global ``complete bipartite minimises $S$'' question remains open.

\subsection{8) COMPLETION ESTIMATE (MANDATORY)}

\textbf{COMPLETION: 55\%}

\subsection{9) REFERENCES}

\begin{enumerate}
\item A.~Gy\'arf\'as, J.~Koml\'os, and E.~Szemer\'edi,
\emph{On the distribution of cycle lengths in graphs},
J.\ Graph Theory \textbf{8} (1984), 441--462.

\item H.~Liu and R.~Montgomery,
\emph{A solution to Erd\H{o}s and Hajnal's odd cycle problem},
J.\ Amer.\ Math.\ Soc.\ \textbf{36} (2023), 1191--1234.

\item R.~C.~Entringer and E.~F.~Schmeichel,
\emph{Edge conditions and cycle structure in bipartite graphs},
Ars Combinatoria \textbf{26} (1988), 229--232.
\end{enumerate}
