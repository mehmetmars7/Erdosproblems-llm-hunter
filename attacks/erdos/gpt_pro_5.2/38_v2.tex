\section{Erd\H{o}s Problem \#38 --- Round 2 (continuation \& gap-closure)}

\subsection*{1) ROUND-2 OBJECTIVE}
\textbf{Path (C): obstruction/correction.}
Round~1 isolated one explicit obstruction (a gcd obstruction) and one adversarial family of sets $A$ (periodic block sets) but left two concrete gaps:
(i) the gcd obstruction was built on a set $A=d\mathbb N$ whose Schnirelmann density is actually $0$ under the stated convention $\mathbb N=\{1,2,\dots\}$,
(ii) the periodic-block calculation was only carried out for ``small'' shifts $b\le P-L$.

In Round~2 I (a) correct the gcd obstruction with a density-$1/d$ set that \emph{does} contain $1$, and (b) strengthen the periodic-block calculation to an \emph{all-shifts} formula (for $\alpha\le 1/2$) on a single period. This yields new necessary structural conditions on any candidate $B$, in particular that $1\in B$ and that $B$ cannot have unbounded multiplicative gaps.

\subsection*{2) ROUND-1 FOUNDATION USED}
I rely on the following Round~1 components (without re-proving them):
\begin{itemize}
\item \textbf{Round~1 formal restatement} of the problem and notation.
\item \textbf{Round~1 Lemma 38.2} (periodic block sets): for $A_{P,L}=\bigcup_{k\ge 0}\{kP+1,\dots,kP+L\}$ one has $d_s(A_{P,L})=L/P$.
\item \textbf{Round~1 Lemma 38.2 (finite-$N$ upper bound)} in the regime $N=KP$ and $1\le b\le P-L$, giving
\[
|(A_{P,L}\cup(A_{P,L}+b))\cap[1,N]|\le \left(\frac LP+\frac bP\right)N.
\]
\end{itemize}
I will explicitly flag where Round~1 contained an error and supply a replacement argument.

\subsection*{3) NEW INSIGHT / TOOL (ROUND-2)}
\begin{itemize}
\item \textbf{Schnirelmann sensitivity at $1$.} Under $\mathbb N=\{1,2,\dots\}$, any $A$ with $1\notin A$ has $d_s(A)=0$. This invalidates the Round~1 use of $A=d\mathbb N$ as a positive-density example.
\item \textbf{All-shifts union formula for block sets when $\alpha\le \tfrac12$.}
For $A_{P,L}$ with $L\le P/2$ and for every shift $0\le b\le P-1$, one can compute $|(A_{P,L}\cup(A_{P,L}+b))\cap[1,P]|$ exactly and deduce the sharp upper bound
\[
\frac{|(A_{P,L}\cup(A_{P,L}+b))\cap[1,P]|}{P}\le \alpha+\frac{\min(b,P-b)}{P},\qquad \alpha=L/P\le \tfrac12.
\]
\item \textbf{New structural consequences for $B$.}
Specialising to $\alpha=1/2$ forces $1\in B$ (from $N=2$) and forces $B$ to intersect the ``middle'' interval $[f(1/2)P,\,P-f(1/2)P]$ for every even $P$, implying bounded multiplicative gaps $b_{n+1}/b_n\ll 1/f(1/2)$.
\end{itemize}

\subsection*{4) ATTACK PLAN (ROUND-2)}
\begin{enumerate}
\item \textbf{Fix the Round~1 gcd obstruction.} Prove the elementary lemma ``$1\notin A\Rightarrow d_s(A)=0$'' and replace $A=d\mathbb N$ by $A=\{1\}\cup d\mathbb N$, which has $d_s(A)=1/d$.
\item \textbf{Close the wrap-around gap for periodic blocks.} Compute the union size on a single period $[1,P]$ for \emph{all} shifts $0\le b<P$ when $L\le P/2$.
\item \textbf{Extract new necessary conditions on $B$ from $\alpha=1/2$.} Apply the problem's property to $A_{P,P/2}$ at $N=P$ to force existence of a shift $b\in B\cap[1,P]$ with $\min(b,P-b)\ge f(1/2)P$. This yields:
(i) $1\in B$ (take $P=2$),
(ii) bounded multiplicative gaps (take $P$ just below $b_{n+1}$).
\end{enumerate}
These steps strictly strengthen the Round~1 obstruction analysis without assuming anything about additive bases beyond the problem statement.

\subsection*{5) WORK (ROUND-2)}

\subsubsection*{5.1 A basic correction: sets missing $1$ have $d_s=0$}
\begin{lemma}[Sensitivity at $1$]
\label{lem:missing1}
If $A\subseteq\mathbb N$ and $1\notin A$, then $d_s(A)=0$.
\end{lemma}
\begin{proof}
By definition,
\[
 d_s(A)=\inf_{N\ge 1}\frac{|A\cap\{1,\dots,N\}|}{N}\le \frac{|A\cap\{1\}|}{1}=0.
\]
\end{proof}

\subsubsection*{5.2 Corrected gcd obstruction (fixing Round~1 Lemma 38.1)}
Round~1 argued with $A=d\mathbb N$, but Lemma~\ref{lem:missing1} shows $d_s(d\mathbb N)=0$ under the present convention.
The obstruction still holds after a minimal repair.

\begin{lemma}[Corrected gcd obstruction]
\label{lem:gcdcorrect}
Let $d\ge 2$. Suppose $B\subseteq d\mathbb N$.
Then the desired property cannot hold (for any choice of $f$) because it fails already for $\alpha=1/d$.
In particular, any candidate $B$ must satisfy $\gcd(B)=1$.
\end{lemma}
\begin{proof}
Define
\[
A:=\{1\}\cup d\mathbb N=\{1,d,2d,3d,\dots\}.
\]
\emph{Claim 1: $d_s(A)=1/d$.}
For $N\ge 1$ we have $|A\cap[1,N]|=1+\lfloor N/d\rfloor$.
Using $\lfloor N/d\rfloor\ge N/d-1$ gives
\[
\frac{|A\cap[1,N]|}{N}=\frac{1+\lfloor N/d\rfloor}{N}\ge \frac{1+(N/d-1)}{N}=\frac1d.
\]
On the other hand for multiples $N=kd$ one has
\[
\frac{|A\cap[1,kd]|}{kd}=\frac{1+k}{kd}=\frac1d+\frac{1}{kd},
\]
which tends to $1/d$ as $k\to\infty$.
Hence $d_s(A)=1/d$.

Now fix any $b\in B$. By assumption $b\equiv 0\pmod d$.
Then $d\mathbb N+b\subseteq d\mathbb N$, so
\[
A+b=(\{1\}+b)\cup(d\mathbb N+b)\subseteq \{1+b\}\cup d\mathbb N.
\]
Therefore
\[
A\cup(A+b)=A\cup\{1+b\}.
\]
Consequently for every $N$ we have
\[
|(A\cup(A+b))\cap[1,N]|\le |A\cap[1,N]|+1.
\]
Choose $N$ large so that $\frac{2}{N}<f(1/d)$.
Since $\lfloor N/d\rfloor\le N/d$, we have
\[
\frac{|A\cap[1,N]|}{N}=\frac{1+\lfloor N/d\rfloor}{N}\le \frac{1+N/d}{N}=\frac1d+\frac1N.
\]
Hence for every $b\in B$,
\[
\frac{|(A\cup(A+b))\cap[1,N]|}{N}\le \frac{|A\cap[1,N]|}{N}+\frac1N\le \frac1d+\frac{2}{N}<\frac1d+f(1/d),
\]
contradicting the required inequality at $\alpha=1/d$.
This proves the failure and hence forces $\gcd(B)=1$.
\end{proof}

\subsubsection*{5.3 All-shifts union formula for periodic block sets when $\alpha\le 1/2$}
We now strengthen the Round~1 periodic-block computation by removing the restriction $b\le P-L$ when $L\le P/2$ and by working at a single period ($N=P$).

\begin{lemma}[One-period union formula for $\alpha\le 1/2$]
\label{lem:allshifts}
Let integers $P\ge 2$ and $L$ satisfy $1\le L\le P/2$ and define the periodic block set
\[
A_{P,L}:=\bigcup_{k\ge 0}\{kP+1,\dots,kP+L\}.
\]
Fix $b\in\{0,1,\dots,P-1\}$. Then
\[
|(A_{P,L}\cup(A_{P,L}+b))\cap[1,P]|=L+\min\{L,\,b,\,P-b\}.
\]
In particular, writing $\alpha=L/P\le 1/2$, for every $1\le b\le P-1$ we have the sharp upper bound
\[
\frac{|(A_{P,L}\cup(A_{P,L}+b))\cap[1,P]|}{P}\le \alpha+\frac{\min(b,P-b)}{P}.
\]
\end{lemma}
\begin{proof}
On the first period we have $A_{P,L}\cap[1,P]=\{1,2,\dots,L\}$.
For $0\le b\le P-1$,
\[
(A_{P,L}+b)\cap[1,P]=\{a+b\le P: a\in A_{P,L}\}.
\]
Since $b\ge 0$, no element of $A_{P,L}$ from periods $\ge 1$ can shift \emph{into} $[1,P]$.
Thus only $a\in A_{P,L}\cap[1,P-b]=\{1,\dots,\min(L,P-b)\}$ contribute, and hence
\[
(A_{P,L}+b)\cap[1,P]=\{b+1,\dots, b+\min(L,P-b)\}.
\]
Set $m:=\min(L,P-b)$; then $(A_{P,L}+b)\cap[1,P]$ is the interval $[b+1,b+m]$.
We consider three cases.

\medskip
\noindent\emph{Case 1: $0\le b<L$.}
Because $L\le P/2$, we have $P-b> P/2\ge L$, so $m=L$.
The overlap of $[1,L]$ and $[b+1,b+L]$ is $[b+1,L]$, of size $L-b$.
Therefore
\[
|(A_{P,L}\cup(A_{P,L}+b))\cap[1,P]|=L+L-(L-b)=L+b.
\]
Since here $b<L$ and $P-b\ge L$, we have $\min\{L,b,P-b\}=b$, giving the claimed formula.

\medskip
\noindent\emph{Case 2: $L\le b\le P-L$.}
Then $P-b\ge L$, hence $m=L$, and the shifted interval $[b+1,b+L]$ lies entirely to the right of $[1,L]$.
So the two intervals are disjoint and
\[
|(A_{P,L}\cup(A_{P,L}+b))\cap[1,P]|=L+L=2L.
\]
Here $\min\{L,b,P-b\}=L$, again matching the formula.

\medskip
\noindent\emph{Case 3: $P-L<b\le P-1$.}
Then $P-b<L$, so $m=P-b$.
Also $b> P-L\ge P/2\ge L$, so $b\ge L$ and the two intervals $[1,L]$ and $[b+1,P]$ are disjoint.
Hence
\[
|(A_{P,L}\cup(A_{P,L}+b))\cap[1,P]|=L+(P-b)=L+\min\{L,b,P-b\},
\]
since in this regime $\min\{L,b,P-b\}=P-b$.

This exhausts all $b\in\{0,\dots,P-1\}$.
The inequality follows immediately by dividing by $P$ and using $\min\{L,b,P-b\}\le \min\{b,P-b\}$.
\end{proof}

\subsubsection*{5.4 Immediate forcing of $1\in B$}
\begin{proposition}
\label{prop:1inB}
If a set $B$ satisfies the problem property for some function $f:(0,1)\to(0,\infty)$, then $1\in B$.
\end{proposition}
\begin{proof}
Take $\alpha=1/2$ and let $A$ be the odd numbers: $A=\{1,3,5,\dots\}$.
Then $d_s(A)=1/2$ and for $N=2$ we have $A\cap[1,2]=\{1\}$.
If $b\ge 2$ then $(A+b)\cap[1,2]=\emptyset$, so
$|(A\cup(A+b))\cap[1,2]|=1$.
But the required inequality at $\alpha=1/2$ is
\[
|(A\cup(A+b))\cap[1,2]|\ge (1/2+f(1/2))\cdot 2 = 1+2f(1/2) > 1,
\]
forcing $|(A\cup(A+b))\cap[1,2]|\ge 2$, hence $b=1$.
Thus $1\in B$.
\end{proof}

\subsubsection*{5.5 Bounded multiplicative gaps forced by $\alpha=1/2$}
Set $f_0:=f(1/2)>0$.

\begin{proposition}[Middle-interval condition at every even scale]
\label{prop:middleinterval}
Assume $B$ satisfies the problem property with function $f$.
Then for every even integer $P\ge 2$ there exists $b\in B\cap\{1,\dots,P-1\}$ such that
\[
\min\{b,\,P-b\}\ge f_0 P.
\]
Equivalently,
\[
B\cap [\,f_0 P,\,P-f_0 P\,]\neq\emptyset\qquad\text{for every even }P\ge 2.
\]
\end{proposition}
\begin{proof}
Fix even $P\ge 2$ and consider the block set $A_{P,P/2}$.
By Round~1 Lemma~38.2, $d_s(A_{P,P/2})=1/2$.
Apply the problem property to this $A$ and $N=P$.
Since shifting by $b\ge P$ produces no elements in $[1,P]$, the witnessing $b\in B$ must satisfy $1\le b\le P-1$.
Lemma~\ref{lem:allshifts} (with $L=P/2$) gives
\[
\frac{|(A\cup(A+b))\cap[1,P]|}{P}\le \frac12+\frac{\min(b,P-b)}{P}.
\]
The desired lower bound is $\frac12+f_0$.
Therefore $\min(b,P-b)\ge f_0P$, as required.
\end{proof}

\begin{corollary}[No unbounded multiplicative gaps]
\label{cor:boundedratio}
Let $B=\{b_1<b_2<\cdots\}$. If $B$ satisfies the problem property, then
\[
 b_{n+1}\le \frac{1}{f_0}\,b_n+2\qquad\text{for all }n\ge 1.
\]
In particular $\limsup_{n\to\infty} b_{n+1}/b_n\le 1/f_0<\infty$, so $B$ cannot be lacunary.
\end{corollary}
\begin{proof}
Fix $n\ge 1$.
If $b_{n+1}\le 2$ then $b_{n+1}\le 2\le b_n/f_0+2$ (since $b_n\ge 1$), so the claim is trivial.
Thus we may assume $b_{n+1}\ge 3$, so the even integer $P$ defined below satisfies $P\ge 2$.
Let $P$ be the largest even integer strictly smaller than $b_{n+1}$, i.e.
$P=b_{n+1}-1$ if $b_{n+1}$ is odd and $P=b_{n+1}-2$ if $b_{n+1}$ is even.
Then $P\ge b_{n+1}-2$ and $P<b_{n+1}$.
Hence $B\cap[1,P]=\{b_1,\dots,b_n\}$.
By Proposition~\ref{prop:middleinterval} there exists $b\in B\cap[1,P]$ with $b\ge f_0P$.
Since $b\le b_n$ we get $b_n\ge f_0P\ge f_0(b_{n+1}-2)$.
Rearranging yields $b_{n+1}\le b_n/f_0+2$.

The ratio statement follows immediately.
\end{proof}

\subsubsection*{5.6 A small generalisation for rational $\alpha\le 1/2$}
For later use it is convenient to record the same ``middle interval'' forcing for any rational density $\alpha\le 1/2$.

\begin{proposition}[Middle-interval condition for rational $\alpha\le 1/2$]
\label{prop:rationalalpha}
Fix $\alpha\in(0,1/2]$ and write $f_\alpha:=f(\alpha)>0$.
Assume $\alpha=L_0/P_0$ in lowest terms.
Then for every multiple $P=mP_0$ (with $m\in\mathbb N$) there exists $b\in B\cap\{1,\dots,P-1\}$ such that
\[
\min\{b,\,P-b\}\ge f_\alpha P.
\]
Equivalently, $B\cap[f_\alpha P,\,P-f_\alpha P]\neq\emptyset$ for every such $P$.
\end{proposition}
\begin{proof}
Take $A=A_{P,L}$ with $L=\alpha P=mL_0$ so that $d_s(A)=\alpha$ by Round~1 Lemma~38.2.
Apply the problem property at $N=P$.
As before, the witnessing $b$ must satisfy $1\le b\le P-1$.
Lemma~\ref{lem:allshifts} gives
\[
\frac{|(A\cup(A+b))\cap[1,P]|}{P}\le \alpha+\frac{\min(b,P-b)}{P}.
\]
Comparing with the required lower bound $\alpha+f_\alpha$ forces $\min(b,P-b)\ge f_\alpha P$.
\end{proof}

\subsection*{6) ADVERSARIAL VERIFICATION}
\begin{itemize}
\item \textbf{Check the corrected gcd obstruction.}
The only delicate point is that $d\mathbb N+b\subseteq d\mathbb N$ (not equality), but this suffices because it implies $A\cup(A+b)=A\cup\{1+b\}$ and hence the union count increases by at most $1$.
The computed Schnirelmann density $d_s(\{1\}\cup d\mathbb N)=1/d$ is robust because the ratio is \emph{bounded below} by $1/d$ for all $N$ and approaches $1/d$ along $N=kd$.
\item \textbf{Quantifiers / ``for every $N$'' issues.}
All deductions from $\alpha=1/2$ used the problem requirement at specific even values $N=P$ (not asymptotic $N\to\infty$), so there is no hidden passage to a subsequence.
\item \textbf{Boundary cases in Lemma~\ref{lem:allshifts}.}
The casework uses $L\le P/2$ to ensure in Case~1 that $P-b\ge L$ and in Case~3 that $b\ge L$; these are exactly where wrap-around could otherwise appear.
The endpoints $b=0$ and $b=P-1$ are covered: the formula gives union size $L$ and $L+1$ respectively, matching the direct interval picture.
\item \textbf{Could Proposition~\ref{prop:middleinterval} fail because the witness $b$ might be $\ge P$?}
No: if $b\ge P$ then $(A_{P,P/2}+b)\cap[1,P]=\emptyset$, so the union count is $P/2$, which cannot reach $(1/2+f_0)P$ since $f_0>0$.
\item \textbf{Internal consistency with Round~1.}
Round~1 Lemma~38.2 remains correct (it crucially uses sets containing $1$), and Lemma~\ref{lem:allshifts} is a strict strengthening on the special case $N=P$ and $\alpha\le 1/2$.
The only Round~1 piece that needed correction was the use of $A=d\mathbb N$ as a positive-$d_s$ example.
\end{itemize}

\subsection*{7) FINAL}
\textbf{UNRESOLVED (BUT STRICTLY ADVANCED).}
We have not resolved whether such a non-basis $B$ exists.
However, Round~2 makes two substantive advances beyond Round~1:
\begin{enumerate}
\item The gcd obstruction is repaired in a way compatible with the stated Schnirelmann density convention (Lemma~\ref{lem:gcdcorrect}).
\item New necessary structural conditions are proved for any candidate $B$:
\begin{itemize}
\item $1\in B$ (Proposition~\ref{prop:1inB});
\item $B$ must hit middle intervals at every even scale (Proposition~\ref{prop:middleinterval}), implying bounded multiplicative gaps (Corollary~\ref{cor:boundedratio}).
\end{itemize}
\end{enumerate}
These conditions rule out broad classes of natural non-bases (e.g. lacunary sequences) and tighten the search space for a potential counterexample.

\subsection*{8) COMPLETION ESTIMATE (MANDATORY)}
\textbf{COMPLETION: 35\%}.

\subsection*{9) REFERENCES}
No external references were used beyond the problem statement and Round~1 results.
