
1) FORMAL RESTATEMENT

Fix an integer \(r\ge 2\). An integer \(n\ge 1\) is \emph{\(r\)-powerful} if for every prime \(p\mid n\) one has \(p^r\mid n\).

The problem asks:
(a) For \(r\ge 4\): does there exist a solution to
\[a_1+a_2+\cdots + a_{r-2}=c\]
where each \(a_i\) is \(r\)-powerful, the \(a_i\) are ``coprime'' (ambiguous; see below), and \(c\) is also \(r\)-powerful?
(b) If such solutions exist for \(r\ge 4\), are there only finitely many?
(c) For \(r=3\): are there infinitely many triples of coprime \(3\)-powerful numbers \((a,b,c)\) with \(a+b=c\)?

Ambiguity: For more than two integers, ``coprime'' may mean either (i) pairwise coprime, i.e. \(\gcd(a_i,a_j)=1\) for all \(i\ne j\), or (ii) collectively coprime, i.e. \(\gcd(a_1,\dots,a_{r-2})=1\). The example displayed in the problem statement for \(r=5\) satisfies (ii) but not (i).

2) QUICK LITERATURE/CONTEXT CHECK

The statement itself records:
- For (c), the answer is yes: Nitaj proved infinitely many coprime triples \(a,b,c\) of 3-powerful numbers with \(a+b=c\); Cohn and Walsh gave other constructions.
- Cambie found examples for \(r=5,7,8\) for question (a) (existence), and one explicit identity is displayed.
I do not use any other external results.

3) ATTACK PLAN

- Verify explicitly the provided sample identities (Nitaj example for \(r=3\) and Cambie example for \(r=5\)), including the relevant coprimality notion.
- Prove basic gcd properties that any coprime additive triple must satisfy.

4) WORK

Fast reality check (explicit computation).
Both displayed identities in the problem statement were checked exactly by integer arithmetic:
- \(2^3\cdot 3^5\cdot 73^3+271^3=919^3\).
- \(3^7\cdot 61^5=2^8 3^{10}5^7 +2^{12}23^6 +11^5 13^5\).

Lemma 939.1 (Coprime addends force coprime sum).
If integers \(a,b\ge 1\) satisfy \(\gcd(a,b)=1\) and \(a+b=c\), then \(\gcd(a,c)=\gcd(b,c)=1\).

Proof.
We have \(\gcd(a,c)=\gcd(a,a+b)=\gcd(a,b)=1\) by the standard gcd identity \(\gcd(x,x+y)=\gcd(x,y)\). Similarly, \(\gcd(b,c)=\gcd(b,a+b)=\gcd(b,a)=1\). QED.

Proposition 939.2 (Verification of the Nitaj example for \(r=3\)).
Let
\[a:=2^3\cdot 3^5\cdot 73^3,\qquad b:=271^3,\qquad c:=919^3.
\]
Then \(a+b=c\), the numbers \(a,b,c\) are 3-powerful, and \(\gcd(a,b)=\gcd(a,c)=\gcd(b,c)=1\).

Proof.
First, direct computation shows \(a+b=c\) (checked exactly).
Next, the prime factorisations are
\[a=2^3\cdot 3^5\cdot 73^3,\qquad b=271^3,\qquad c=919^3.
\]
Every prime exponent in \(a,b,c\) is at least 3, hence each is 3-powerful by definition.
Finally, \(\gcd(a,b)=1\) because the primes dividing \(a\) are \(2,3,73\) while the only prime dividing \(b\) is 271. By Lemma 939.1, this implies \(\gcd(a,c)=\gcd(b,c)=1\) as well, and the computation also confirms all pairwise gcds are 1. QED.

Proposition 939.3 (Verification and coprimality analysis of the \(r=5\) example).
Define
\[A:=2^8 3^{10}5^7,\qquad B:=2^{12}23^6,\qquad C:=11^5 13^5,\qquad L:=3^7 61^5.
\]
Then \(A+B+C=L\), each of \(A,B,C,L\) is 5-powerful, and \(\gcd(A,B,C)=1\) but \(\gcd(A,B)=2^8\ne 1\).

Proof.
The identity \(A+B+C=L\) holds by direct computation.
Each of \(A,B,C,L\) is 5-powerful because every prime appears with exponent \(\ge 5\) in its factorisation:
\(A\) has exponents \(8,10,7\); \(B\) has exponents \(12,6\); \(C\) has exponents \(5,5\); \(L\) has exponents \(7,5\).
For gcds: \(A\) and \(B\) share the prime 2, with \(\gcd(A,B)=2^8\) since \(A\) has 2-adic exponent 8 and \(B\) has exponent 12. Also \(\gcd(A,C)=\gcd(B,C)=1\) because they have disjoint prime supports. Thus \(\gcd(A,B,C)=1\) but the summands are not pairwise coprime. QED.

5) VERIFICATION

- Proposition 939.2: the equality and factorizations were checked explicitly; the gcd claim matches Lemma 939.1.
- Proposition 939.3: confirms the ambiguity in ``coprime'' for multiple summands: the displayed example is not pairwise coprime.

6) FINAL

UNRESOLVED
(i) Strongest proved partial result: The explicit Nitaj example provides a concrete coprime (indeed pairwise coprime) triple of 3-powerful numbers with \(a+b=c\) (Proposition 939.2). The displayed \(r=5\) identity is verified and shows existence if ``coprime'' is interpreted as \(\gcd(A,B,C)=1\) (Proposition 939.3).
(ii) First gap: For \(r\ge 4\) and the intended coprimality notion (especially pairwise coprime), determine whether any solutions to \(a_1+\cdots+a_{r-2}=c\) with all terms \(r\)-powerful exist.
(iii) Top 3 next moves:
  1. Resolve the coprimality convention (pairwise vs collective) and, for the pairwise version, search for computational solutions for small \(r\ge4\) within bounded ranges.
  2. Derive congruence obstructions modulo small primes/powers for sums of pairwise-coprime \(r\)-powerful numbers, aiming to rule out existence for certain \(r\).
  3. If solutions exist, attempt to parameterise them via elliptic-curve or higher-genus Diophantine methods suggested by the known \(r=3\) constructions (without assuming any literature facts beyond the statement).
(iv) Minimal counterexample structure: A smallest solution for the pairwise-coprime \(r\ge4\) existence question would likely have each \(a_i\) supported on disjoint sets of primes with exponents exactly \(r\) or just above, since sharing primes immediately breaks pairwise coprimality and large exponents rapidly increase size.


