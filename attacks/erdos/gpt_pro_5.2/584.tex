
\noindent\textbf{FORMAL RESTATEMENT.}

The statement in the problem uses asymptotic notation ($\gg$) and does not fully specify dependencies when $\delta=\delta(n)$.
The minimal conventional restatement consistent with the text is:

For each $n\ge 1$ and each real $\delta\in(0,1/2]$, let $G$ be a simple graph on $n$ vertices with at least $\delta n^2$ edges.
Is there an absolute constant $c>0$ such that one can always find subgraphs $H_1,H_2\subseteq G$ satisfying:

(1) $e(H_1)\ge c\,\delta^3 n^2$, and for every two edges $e,f\in E(H_1)$ there exists a (simple) cycle in $H_1$ of length at most $6$ containing both $e$ and $f$; moreover if $e$ and $f$ share a vertex then there is a $4$-cycle in $H_1$ containing both.

(2) $e(H_2)\ge c\,\delta^2 n^2$, and for every two edges $e,f\in E(H_2)$ there exists a (simple) cycle in $H_2$ of length at most $8$ containing both $e$ and $f$.

Here $e(H)$ denotes the number of edges.

\bigskip
\noindent\textbf{QUICK LITERATURE/CONTEXT CHECK.}

I will not use any literature beyond what is explicitly quoted in the problem statement (Duke--Erd\H{o}s, Duke--Erd\H{o}s--R\"{o}dl, Fox--Sudakov).

\bigskip
\noindent\textbf{ATTACK PLAN.}

\emph{Proof-track ideas.}
(1) Prune to a subgraph of linear minimum degree $\Omega(\delta n)$, then use common-neighbor arguments to force many short cycles and attempt to extract a large ``cycle-connected'' edge set.
(2) Use a dependent-random-choice style selection of a vertex subset with large common neighborhoods to build many $4$- and $6$-cycles.

\emph{Disproof-track ideas.}
(1) Try sparse pseudorandom graphs with few short cycles, or bipartite graphs with large girth, as potential obstructions when $\delta=n^{-c}$.
(2) For small $n$, brute force search for graphs where no subgraph of the required size has the stated cycle connectivity.

In this write-up I only obtain trivial regimes (very dense graphs) and structural lemmas.

\bigskip
\noindent\textbf{WORK.}

\noindent\textbf{Lemma 584.1 (dense subgraph with linear minimum degree).}
Let $G$ be a graph on $n$ vertices with $e(G)\ge \delta n^2$.
Then $G$ contains a subgraph $H\subseteq G$ with
\[
 e(H)\ge \frac{\delta n^2}{2}
 \qquad\text{and}\qquad
 \delta(H)\ge \frac{\delta n}{2},
\]
where $\delta(H)$ denotes the minimum degree of $H$.

\noindent\emph{Proof.}
Start with $H:=G$.
While there exists a vertex $v$ of current degree $<\delta n/2$, delete $v$ and all incident edges from $H$.
This process terminates after deleting some set of vertices.
When a vertex is deleted, fewer than $\delta n/2$ edges are removed.
If $t$ vertices are deleted in total, then fewer than $t\cdot(\delta n/2)$ edges are deleted.
Since $t\le n$, fewer than $(\delta n^2)/2$ edges are deleted overall.
Thus the final graph $H$ has
\[
 e(H)\ge e(G)-\frac{\delta n^2}{2}\ge \frac{\delta n^2}{2}.
\]
By construction, every remaining vertex has degree at least $\delta n/2$ in $H$, so $\delta(H)\ge \delta n/2$.
\qed

\bigskip
\noindent\textbf{Lemma 584.2 (adjacent edges in a $4$-cycle under degree $>n/2$).}
Let $H$ be a graph on $n$ vertices with minimum degree $\delta(H)>\frac n2$.
Then for any two edges $uv$ and $vw$ sharing a vertex $v$, there exists a $4$-cycle in $H$ containing both edges.

\noindent\emph{Proof.}
Fix adjacent edges $uv$ and $vw$.
We seek a vertex $x\notin\{u,v,w\}$ adjacent to both $u$ and $w$.
Since $\deg(u)>n/2$ and $\deg(w)>n/2$, we have
\[
|N(u)|+|N(w)|>n,
\]
so by the pigeonhole principle $N(u)\cap N(w)\neq\varnothing$.
Also, $v\in N(u)\cap N(w)$, but we need a vertex distinct from $v$.
Because degrees are integers, $\deg(u)\ge\lfloor n/2\rfloor+1$ and $\deg(w)\ge\lfloor n/2\rfloor+1$, hence
\[\deg(u)+\deg(w)\ge 2\bigl(\lfloor n/2\rfloor+1\bigr)\ge n+1.\]
Using $N(u)\cup N(w)\subseteq V\setminus\{u,w\}$ (which has size $n-2$), we get
\[
|N(u)\cap N(w)|\ge |N(u)|+|N(w)|-(n-2)\ge 3,
\]
so there exists $x\in N(u)\cap N(w)$ with $x\ne v$.
Then $u-v-w-x-u$ is a $4$-cycle containing both $uv$ and $vw$.
\qed

\bigskip
\noindent\textbf{Lemma 584.3 (any two edges in a $\le 6$-cycle under degree $\ge\lceil 2n/3\rceil$, $n\ge 10$).}
Let $H$ be a graph on $n\ge 10$ vertices with minimum degree $\delta(H)\ge \lceil 2n/3\rceil$.
Then every pair of edges of $H$ is contained in a simple cycle of length at most $6$.
Moreover, if the two edges share a vertex then they are contained in a $4$-cycle.

\noindent\emph{Proof.}
If the two edges share a vertex, the $4$-cycle claim follows from Lemma 584.2 because $\lceil 2n/3\rceil>n/2$ for all $n\ge 2$.
So assume the edges are disjoint: $e_1=ab$ and $e_2=cd$ with $a,b,c,d$ all distinct.

We will build a $6$-cycle $a-b-x-c-d-y-a$.
It suffices to find vertices $x,y$ such that
\[
 x\in N(b)\cap N(c)\setminus\{a,b,c,d\},\qquad
 y\in N(a)\cap N(d)\setminus\{a,b,c,d,x\},
\]
and such that $x\ne y$.

First note that for any two vertices $u,v$ we have
\[
|N(u)\cap N(v)|\ge \deg(u)+\deg(v)-(n-2),
\]
because $N(u)\cup N(v)\subseteq V\setminus\{u,v\}$ has size at most $n-2$.
Applying this with $u=b,v=c$ and using $\deg(b),\deg(c)\ge\lceil 2n/3\rceil$ gives
\[
|N(b)\cap N(c)|\ge 2\lceil 2n/3\rceil-(n-2).
\]
For $n\ge 10$, the right-hand side is at least $6$. Indeed, write $n=3q+r$ with $r\in\{0,1,2\}$.
Then $\lceil 2n/3\rceil=2q$ if $r=0$, $2q+1$ if $r=1$, and $2q+2$ if $r=2$, so
\[
2\lceil 2n/3\rceil-(n-2)=\begin{cases}
q+2,&r=0,\\
q+3,&r=1,\\
q+4,&r=2.
\end{cases}
\]
When $n\ge 10$, we have $q\ge 3$, and in each case the displayed value is $\ge 6$.
Therefore there exists some $x\in N(b)\cap N(c)$ avoiding the at most four forbidden vertices $\{a,b,c,d\}$.
So we can choose $x\in N(b)\cap N(c)\setminus\{a,b,c,d\}$.

Similarly,
\[
|N(a)\cap N(d)|\ge 2\lceil 2n/3\rceil-(n-2)\ge 6.
\]
We must avoid the forbidden set $\{a,b,c,d,x\}$ of size $5$, so we can choose
\[y\in N(a)\cap N(d)\setminus\{a,b,c,d,x\}.
\]

With these choices, the vertices $a,b,x,c,d,y$ are all distinct and all required adjacencies hold:
$ab, bx, xc, cd, dy, ya$ are edges of $H$.
Therefore $a-b-x-c-d-y-a$ is a simple cycle of length $6$ containing both $ab$ and $cd$.
\qed

\bigskip
\noindent\textbf{Consequences (very dense regime).}
If $G$ has minimum degree at least $\lceil 2n/3\rceil$ (and in particular for all large $n\ge 10$), then taking $H_1=H_2=G$ satisfies the cycle-connectivity conditions (with cycle length $\le 6$), and $e(G)=\delta n^2$ automatically dominates $\delta^3 n^2$ and $\delta^2 n^2$ when $\delta$ is a fixed constant.
This does not address the hard regime $\delta=n^{-c}$.

\bigskip
\noindent\textbf{Fast reality check (sanity script for Lemma 584.3).}

For each $n\in\{6,7,8,9,10\}$ I generated random graphs and searched for one with minimum degree at least $\lceil 2n/3\rceil$.
For each such graph found, I exhaustively checked that every pair of edges lies on a cycle of length at most $6$.
A witness graph was found for each $n$ and passed the check.

\bigskip
\noindent\textbf{VERIFICATION.}

\emph{Checking Lemma 584.3 details.}
The key step is choosing $x\in N(b)\cap N(c)$ avoiding four forbidden vertices and then choosing $y\in N(a)\cap N(d)$ avoiding five forbidden vertices.
The explicit lower bound $|N(u)\cap N(v)|\ge 2\lceil 2n/3\rceil-(n-2)$ is evaluated by a $n\bmod 3$ computation to guarantee at least $6$ common neighbors for $n\ge 10$, giving the required slack.
The accompanying script provides an additional sanity check for small $n$.

\emph{Quantifiers.}
The main open difficulty in the original problem is when $\delta$ depends on $n$ and is small; the lemmas above only address the ``high minimum degree'' and ``linear minimum degree'' preprocessing.

\bigskip
\noindent\textbf{FINAL.}

**UNRESOLVED**

(i) Strongest proved partial result: any graph with $e(G)\ge\delta n^2$ contains a subgraph with $\ge \delta n^2/2$ edges and minimum degree $\ge \delta n/2$ (Lemma 584.1). In the very dense regime $\delta(H)\ge\lceil 2n/3\rceil$, the whole graph is already ``$C_{\le 6}$-edge-connected'' in the sense that every two edges lie on a cycle of length $\le 6$ and adjacent edges lie on a $4$-cycle (Lemmas 584.2--584.3).

(ii) First gap (crisp): starting from a graph of density $\delta$ (especially $\delta=n^{-c}$), extract a subgraph $H_1$ with $\Omega(\delta^3 n^2)$ edges in which \emph{every} pair of edges lies in a common cycle of length at most $6$ (and adjacent pairs in a $4$-cycle), and similarly extract $H_2$ with $\Omega(\delta^2 n^2)$ edges with cycle length bound $8$.

(iii) Top 3 next moves:
1. From Lemma 584.1, try to find a large vertex subset $U$ such that every pair of vertices in $U$ has many common neighbors (dependent random choice), then define $H_1$ by keeping edges inside $U$ (or between $U$ and a common-neighborhood set) to force many short cycles.
2. Prove a robust local-to-global statement: if every edge lies in many $4$-cycles and $6$-cycles, then there is a large subgraph where every \emph{pair} of edges is cycle-connected.
3. For potential obstructions, compute extremal examples for small $n$ at densities $\delta\approx n^{-c}$ and look for graphs where all short-cycle-rich subgraphs are too small.

(iv) Minimal counterexample structure: likely a moderately sparse graph (e.g. bipartite-like or pseudorandom) where common neighborhoods are small, so $4$- and $6$-cycles are scarce and any attempt to enforce ``every pair of edges on a short cycle'' forces a drastic edge loss. In the hard regime $\delta=n^{-c}$, one expects any counterexample to have large girth or at least few short cycles relative to its edge count.


