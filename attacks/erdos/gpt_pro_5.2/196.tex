\section*{Erd\H{o}s problem 196}

\subsection*{1) FORMAL RESTATEMENT}
A permutation of $\mathbb{N}$ is a bijection $\pi:\mathbb{N}\to\mathbb{N}$ viewed as a sequence $(\pi(1),\pi(2),\dots)$.
A \emph{monotone $k$-term arithmetic progression in $\pi$} means there exist indices $i_1<\dots<i_k$ such that
$\pi(i_1),\dots,\pi(i_k)$ form an arithmetic progression and are monotone in that index order.
Question: Must every permutation of $\mathbb{N}$ contain a monotone 4-term arithmetic progression?

\subsection*{2) QUICK LITERATURE/CONTEXT CHECK}
Problem text states: DEGS77 proved every permutation contains a monotone 3-term AP. The 4-term case is asked and is treated as open in the statement.

\subsection*{3) ATTACK PLAN}
Provide two rigorous lemmas:
(1) a clean equivalence formulation in terms of the position function,
(2) a compactness/extension lemma explaining why large finite avoiders do not automatically yield infinite avoiders.
Then give small computational evidence (finite avoiders exist for large $n$ but may fail to extend).

\subsection*{4) WORK}

\paragraph{Lemma 4.1 (Position-function formulation).}
Let $\mathrm{pos}(x)$ be the unique index with $\pi(\mathrm{pos}(x))=x$.
Then $\pi$ contains a monotone $k$-AP iff there exist $a,d\ge 1$ such that the sequence
$a,a+d,\dots,a+(k-1)d$ has positions $\mathrm{pos}(a),\dots,\mathrm{pos}(a+(k-1)d)$ that are monotone (increasing or decreasing).
\textit{Proof.}
If $\pi(i_1),\dots,\pi(i_k)$ are values of a $k$-AP in monotone order along increasing indices, then those values can be written as $a+td$ and their indices are
exactly $\mathrm{pos}(a+td)$ which are monotone.
Conversely if the positions of a value-AP are monotone, reading $\pi$ at those indices yields a monotone AP subsequence. \hfill$\square$

\paragraph{Lemma 4.2 (K\H{o}nig-tree extension principle).}
Fix $k\ge 3$. Let $\mathcal{T}$ be the rooted tree whose vertices at depth $n$ are permutations of $\{1,\dots,n\}$ that avoid monotone $k$-APs,
with an edge from $\sigma\in S_n$ to $\tau\in S_{n+1}$ if $\tau$ restricts to $\sigma$ on $\{1,\dots,n\}$.
If every node has at least one child (i.e. every avoiding permutation extends), then there exists an infinite permutation of $\mathbb{N}$ avoiding monotone $k$-APs.
\textit{Proof.}
$\mathcal{T}$ is finitely branching (each node has at most $n+1$ children). If every depth has at least one node and every node has a child,
then by K\H{o}nig’s lemma there is an infinite root-to-infinity path, i.e. a coherent sequence of extensions $\sigma_1\prec\sigma_2\prec\cdots$.
This defines an infinite permutation of $\mathbb{N}$ whose every finite restriction avoids monotone $k$-APs; hence the whole permutation avoids. \hfill$\square$

\subsection*{FAST REALITY CHECK (computed)}
Heuristic local-search found explicit finite permutations avoiding monotone 4-APs for $n=20,25,30$.
Example for $n=20$ (verified by direct check):
\[
(11,2,15,13,18,1,17,6,14,12,8,7,3,10,5,16,20,9,19,4)
\]
contains no monotone 4-term AP among values in $\{1,\dots,20\}$ (checked by brute enumeration of all 4-APs in $[20]$).

\subsection*{6) FINAL}
\textbf{UNRESOLVED}

(i) Strongest proved partial results: clean reformulation via $\mathrm{pos}$ (Lemma 4.1) and an extension/compactness criterion (Lemma 4.2).
Finite avoiders exist for sizeable $n$ (computed examples), so any infinite proof must use nontrivial extension obstructions.

(ii) First gap: prove that \emph{every} infinite permutation must contain a monotone 4-AP, or construct one avoiding them.

(iii) Top 3 next moves:
1. Prove that the extension tree $\mathcal{T}$ for $k=4$ has only finite height (no infinite branch).
2. Try to show a density/regularity property of position function forces monotone 4-AP.
3. Search systematically for partial permutations that cannot be extended beyond some length while avoiding monotone 4-AP.

(iv) Minimal counterexample structure (if 4-AP is unavoidable): existence of some finite $N$ such that \emph{every} permutation of $\{1,\dots,N\}$
already contains a monotone 4-AP, forcing it inside any infinite permutation via restriction.

