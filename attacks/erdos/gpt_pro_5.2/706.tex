
1) \textbf{FORMAL RESTATEMENT}

Fix $r\in\mathbb{N}$.
For a finite point set $P\subset\mathbb{R}^2$ and a set of distances $A\subset(0,\infty)$ with $|A|=r$, define a graph $G(P,A)$ with vertex set $P$ and an edge between distinct $x,y\in P$ iff $\|x-y\|_2\in A$.
Define $L(r)$ to be the minimal integer such that for \emph{all} such $(P,A)$,
\[
\chi(G(P,A))\le L(r).
\]
The problem asks for good bounds/estimates for $L(r)$ as $r\to\infty$, in particular whether $L(r)\le r^{O(1)}$.

2) \textbf{QUICK LITERATURE/CONTEXT CHECK}

I will only use facts stated in the problem text.
For $r=1$ (unit distance graphs), the text states $5\le L(1)\le 7$.

3) \textbf{ATTACK PLAN}

- \emph{Lower bounds:} build explicit finite configurations forcing many colors (e.g. cliques).
- \emph{Upper bounds:} produce an explicit coloring of the plane (or of all such graphs) depending only on $r$.
A simple product-coloring argument will give an exponential upper bound $7^r$.

4) \textbf{WORK}

\textbf{Lemma 706.1 (Clique lower bound $L(r)\ge r+1$).}
For every $r\ge 1$, $L(r)\ge r+1$.

\textit{Proof.}
Let $P=\{0,1,2,\dots,r\}\subset\mathbb{R}$ viewed as a subset of $\mathbb{R}^2$ on the $x$-axis.
Let $A=\{1,2,\dots,r\}\subset(0,\infty)$, so $|A|=r$.
For any two distinct points $i,j\in P$ we have $\|i-j\|_2=|i-j|\in\{1,2,\dots,r\}=A$, so every pair is an edge.
Thus $G(P,A)$ is the complete graph $K_{r+1}$, whose chromatic number is $r+1$.
Therefore by definition of $L(r)$, we must have $L(r)\ge r+1$.
\hfill$\square$

\textbf{Lemma 706.2 (A $7$-coloring for a single forbidden distance).}
Assume $L(1)\le 7$ (as stated in the problem text). Then for every $d>0$, the infinite distance graph on $\mathbb{R}^2$ with edges between points at distance exactly $d$ is $7$-colorable.

\textit{Proof.}
Let $d>0$.
Consider the scaling map $S:\mathbb{R}^2\to\mathbb{R}^2$ given by $S(x)=x/d$.
If two points $x,y$ satisfy $\|x-y\|_2=d$, then $\|S(x)-S(y)\|_2=1$.
By the assumption $L(1)\le 7$, there exists a proper $7$-coloring $c:\mathbb{R}^2\to\{1,\dots,7\}$ such that no unit-distance pair has the same color.
Then $c\circ S$ is a $7$-coloring of $\mathbb{R}^2$ such that if $\|x-y\|_2=d$, then $c(S(x))\ne c(S(y))$, i.e. $(c\circ S)(x)\ne (c\circ S)(y)$.
\hfill$\square$

\textbf{Proposition 706.3 (Exponential upper bound $L(r)\le 7^r$).}
Assume $L(1)\le 7$.
Then for every $r\ge 1$,
\[
L(r)\le 7^r.
\]

\textit{Proof.}
Let $A=\{d_1,\dots,d_r\}\subset(0,\infty)$ and let $P\subset\mathbb{R}^2$ be finite.
For each $j\in\{1,\dots,r\}$, by Lemma 706.2 there exists a $7$-coloring
\[
 c_j:\mathbb{R}^2\to\{1,\dots,7\}
\]
that is proper for the distance-$d_j$ graph, i.e. $\|x-y\|_2=d_j\Rightarrow c_j(x)\ne c_j(y)$.
Define a combined coloring
\[
 C:\mathbb{R}^2\to\{1,\dots,7\}^r,\qquad C(x)=(c_1(x),\dots,c_r(x)).
\]
This uses at most $7^r$ colors.
If $x,y\in P$ are adjacent in $G(P,A)$, then $\|x-y\|_2\in A$, so $\|x-y\|_2=d_j$ for some $j$.
For that $j$, $c_j(x)\ne c_j(y)$, hence $C(x)\ne C(y)$.
Therefore $C$ is a proper coloring of $G(P,A)$ using at most $7^r$ colors.
Since $(P,A)$ were arbitrary, $L(r)\le 7^r$.
\hfill$\square$

\textbf{FAST REALITY CHECK (small $r$).}
The bounds above yield:
- $r=1$: $2\le L(1)\le 7$ from Lemma 706.1 and the stated fact (the text also states $L(1)\ge 5$).
- $r=2$: $3\le L(2)\le 49$.
- $r=3$: $4\le L(3)\le 343$.

5) \textbf{VERIFICATION}

- Lemma 706.1: the arithmetic progression construction makes all pairwise distances fall into $A$, so it is indeed a clique.
- Lemma 706.2: scaling preserves the adjacency relation up to rescaling of forbidden distance; the pullback of a proper coloring is proper.
- Proposition 706.3: if an edge has length $d_j$, the $j$-th coordinate of $C$ differs, guaranteeing a proper coloring.

6) \textbf{FINAL}

\textbf{UNRESOLVED}

(i) \textbf{Strongest proved partial result.}
Using only the stated bound $L(1)\le 7$, I proved the general bounds
\[
 r+1\le L(r)\le 7^r.
\]

(ii) \textbf{First gap (crisp).}
Improve $L(r)\le 7^r$ to a polynomial bound $L(r)\le r^{O(1)}$ (or construct superpolynomial lower bounds).

(iii) \textbf{Top 3 next moves.}
1. Seek better single-distance colorings that compose more efficiently than direct products (e.g. using a common refinement for all $d\in A$).
2. Look for lower bounds beyond $r+1$, e.g. configurations in the plane with $\Omega(r^2)$ points realizing only $r$ distinct distances but forcing high chromatic number.
3. Computational exploration for small $r$ to guess growth of $L(r)$: search for point sets with many-distance edge sets forcing large $\chi$.

(iv) \textbf{Minimal counterexample structure.}
A minimal counterexample to $L(r)\le r^{O(1)}$ would be a distance set $A$ of size $r$ such that the plane distance graph $G(\mathbb{R}^2,A)$ contains finite induced subgraphs with chromatic number growing faster than any polynomial in $r$.


