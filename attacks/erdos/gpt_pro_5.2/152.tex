\section{Erd\H{o}s Problem \#152: Isolated points in the sumset of a Sidon set}

\subsection*{1) Formal restatement (quantifiers, definitions, edge cases)}
\paragraph{Definition (Sidon set).}
A finite set $A\subset \mathbb{N}$ is a \emph{Sidon set} (a $B_2$ set) if
\[
a+b=c+d \ \text{with}\ a,b,c,d\in A \ \Longrightarrow\ \{a,b\}=\{c,d\}.
\]
Equivalently, the map $\{a,b\}\mapsto a+b$ from unordered pairs in $A$ is injective.

\paragraph{Definition (sumset and isolated sums).}
Write
\[
A+A := \left\{a+b:\ a,b\in A\right\}\subset \mathbb{Z}.
\]
An element $s\in A+A$ is called \emph{$\pm 1$--isolated} if
\[
s-1\notin A+A \quad\text{and}\quad s+1\notin A+A.
\]
Let
\[
I(A):=\left\lvert \left\{s\in A+A:\ s\pm 1\notin A+A\right\}\right\rvert
\]
denote the number of isolated sums.

\paragraph{Problem statement (one clean formulation).}
Is it true that
\[
\forall M\in \mathbb{N}\ \exists n_0(M)\ \forall\text{ finite Sidon }A\subset \mathbb{N}\ (\,\left\lvert A\right\rvert\ge n_0(M)\ \Rightarrow\ I(A)\ge M\,)?
\]
Equivalently, does $I(A)\to \infty$ along \emph{every} sequence of finite Sidon sets with $\left\lvert A\right\rvert\to\infty$?

\paragraph{Edge cases.}
\begin{itemize}[leftmargin=2em]
\item If $\left\lvert A\right\rvert=1$, then $A+A$ is a singleton and that unique sum is isolated.
\item If $\gcd(A)\ge 2$ (e.g.\ $A\subset 2\mathbb{Z}$), then $A+A\subset 2\mathbb{Z}$ and \emph{every} element of $A+A$ is isolated (since neighbors have opposite parity). This makes the problem nontrivial mainly in the ``$\gcd(A)=1$'' regime.
\item Translating $A$ by an integer $t$ replaces $A+A$ by $(A+A)+2t$, which preserves adjacency; hence $I(A)$ is translation invariant.
\end{itemize}

\subsection*{2) Quick literature/context check (high-level, only what's needed)}
This question originates in work of Erd\H{o}s--S\'ark\"ozy--S\'os on the structure of sumsets of Sidon sets, where they study the decomposition of $A+A$ into intervals of consecutive integers and related gap statistics \cite{ESS94,ESS95}. A convenient summary of results and open questions appears in O'Bryant's Sidon set bibliography \cite{OBryant04}; among other things it records that \cite{ESS94} proves a \emph{quadratic} lower bound on the number of ``interval starts'' in $A+A$ (defined below), but the number of \emph{intervals of length one} (i.e.\ isolated points) is left as an open problem \cite{OBryant04}.

\subsection*{3) Attack plan}
Let $S:=A+A$.
\begin{enumerate}[leftmargin=2em]
\item Relate isolated points in $S$ to the interval decomposition of $S$ into maximal runs of consecutive integers.
\item Use known lower bounds for the number of runs (intervals) in $S$ from \cite{ESS94} to show that $S$ is ``highly fragmented'' (average run length bounded).
\item The key missing step is to upgrade ``many runs'' to ``many singleton runs''. One way this could happen is if one can show that run lengths cannot be almost always equal to $2$, or that the number of runs is $>(1/2+\varepsilon)\left\lvert S\right\rvert$ for large $\left\lvert A\right\rvert$.
\item Alternatively, attempt a counterexample construction in which $A+A$ is (asymptotically) a disjoint union of runs of length $\ge 2$ only.
\end{enumerate}

\subsection*{4) Work (partial results and computations)}
Throughout let $n:=\left\lvert A\right\rvert$.

\paragraph{Lemma 4.1 (size of the sumset).}
If $A$ is Sidon then
\[
\left\lvert A+A\right\rvert=\frac{n(n+1)}{2}.
\]
\begin{proof}
Each unordered pair $\{a,b\}$ with $a=b$ or $a<b$ produces a sum $a+b$. The Sidon property says these sums are distinct for distinct unordered pairs, hence the number of sums equals the number of unordered pairs with repetition:
\(
\binom{n}{2}+n=\frac{n(n+1)}{2}.
\)
\end{proof}

\paragraph{Runs, endpoints, and isolated points.}
Write the elements of $S$ in increasing order $s_1<s_2<\cdots<s_t$ where $t=\left\lvert S\right\rvert$.
A \emph{run} (maximal interval) in $S$ is a set of the form $\{m,m+1,\dots,m+\ell-1\}\subset S$ such that $m-1\notin S$ and $m+\ell\notin S$.
Let $B(S)$ be the number of runs, and let $I(S)$ be the number of runs of length $1$.
Then $I(S)=I(A)$ by definition.

Define the \emph{left-endpoints} set
\[
L(S):=\left\{s\in S:\ s-1\notin S\right\}.
\]
Then $\left\lvert L(S)\right\rvert=B(S)$, since each run has exactly one left endpoint.

Similarly define \emph{right endpoints}
\[
R(S):=\left\{s\in S:\ s+1\notin S\right\},
\]
and $\left\lvert R(S)\right\rvert=B(S)$ as well.

\paragraph{Lemma 4.2 (a deterministic inequality: isolated points from many runs).}
For any finite $S\subset \mathbb{Z}$,
\[
I(S)\ \ge\ 2B(S)-\left\lvert S\right\rvert.
\]
\begin{proof}
Write $S$ as a disjoint union of its runs, with lengths $\ell_1,\dots,\ell_{B(S)}$.
Then $\left\lvert S\right\rvert=\sum_i \ell_i$.
If $I(S)$ of these runs have length $1$ and the remaining $B(S)-I(S)$ have length at least $2$, then
\[
\left\lvert S\right\rvert\ge I(S)\cdot 1 + (B(S)-I(S))\cdot 2 = 2B(S)-I(S).
\]
Rearrange.
\end{proof}

\paragraph{A quadratic lower bound for the number of runs.}
Erd\H{o}s--S\'ark\"ozy--S\'os define, for $d\ge 1$,
\[
B(S,d):=\left\{s\in S:\ s-d\notin S\right\}.
\]
In particular $B(S,1)=L(S)$ and $\left\lvert B(S,1)\right\rvert=B(S)$.
According to \cite[summary of \cite{ESS94}]{OBryant04}, there is an absolute constant $c_1>0$ such that for every finite Sidon set $A$ and every $d\ge 1$,
\[
\left\lvert B(A+A,d)\right\rvert\ \ge\ c_1\, n^2.
\]
For $d=1$ this yields
\[
B(S)=\left\lvert L(S)\right\rvert=\left\lvert B(S,1)\right\rvert\ \ge\ c_1 n^2.
\]
Combining with Lemma~4.1 gives that the \emph{average} run length is bounded:
\[
\frac{\left\lvert S\right\rvert}{B(S)} \le \frac{\tfrac12 n(n+1)}{c_1 n^2} \le \frac{1}{2c_1}\left(1+\frac1n\right).
\]
So $S=A+A$ is forced to have \emph{many} runs (components)---indeed $\Omega(n^2)$ of them---but this alone does not force $I(S)\to\infty$, since it is still consistent with (say) all runs having length $2$.

\paragraph{Trivial ``easy'' family where the conclusion holds strongly.}
If $\gcd(A)\ge 2$ then every $s\in A+A$ is isolated, so $I(A)=\left\lvert A+A\right\rvert=\tfrac12 n(n+1)$.
\begin{proof}
If $\gcd(A)=g\ge 2$ then $A\subset g\mathbb{Z}$, hence $A+A\subset g\mathbb{Z}$.
But $s\pm 1\notin g\mathbb{Z}$ so $s\pm 1\notin A+A$ for all $s\in A+A$.
\end{proof}

\paragraph{Small computational sanity check.}
A quick backtracking search (not exhaustive over all diameters, but over sets with maximum element $\le 50$ after translation) found no Sidon set of size $4\le n\le 9$ with $I(A)=0$.
This is \emph{evidence} (not a proof) that ``zero isolated sums'' may be impossible.

Pseudocode sketch:
\begin{quote}\small
\begin{verbatim}
backtrack(A, Sums):
  if |A|=n: compute sumset S and I(A); record
  else:
    for x from last(A)+1 to Max:
      if all x+a not in Sums and 2x not in Sums:
        recurse with A \cup {x} and updated Sums
\end{verbatim}
\end{quote}

\subsection*{5) Verification / consistency checks}
\begin{itemize}[leftmargin=2em]
\item Lemma~4.1 is standard and follows immediately from injectivity of pair-sum map.
\item Lemma~4.2 is a one-line inequality from the run decomposition.
\item The implication ``many run starts $\Rightarrow$ bounded average run length'' is immediate from $\left\lvert S\right\rvert=\sum \ell_i$.
\item The ``$\gcd(A)\ge 2$'' family is checked directly.
\end{itemize}

\subsection*{6) Final}
\paragraph{\textbf{UNRESOLVED.}}
\begin{enumerate}[leftmargin=2em]
\item \textbf{Strongest proved partial result included here.}
Let $A$ be a finite Sidon set of size $n$ and $S=A+A$.
Then $\left\lvert S\right\rvert=\tfrac12n(n+1)$ (Lemma~4.1), and known results of Erd\H{o}s--S\'ark\"ozy--S\'os imply that the number of runs $B(S)$ satisfies $B(S)\ge c_1 n^2$ for an absolute $c_1>0$ \cite{ESS94,OBryant04}. Hence the average run length in $S$ is bounded by an absolute constant.
\item \textbf{First precise obstacle.}
A lower bound on the \emph{number of runs} does not control the distribution of run lengths; it is consistent with $S$ being a disjoint union of runs of length exactly $2$, which would force $I(A)=0$.
\item \textbf{Most plausible next lemma.}
Show a strict inequality of the form $B(S) > (1/2+\varepsilon)\left\lvert S\right\rvert$ for all sufficiently large Sidon sets (or along all large Sidon sets), which by Lemma~4.2 would give $I(A)\ge 2B(S)-\left\lvert S\right\rvert\ge 2\varepsilon\left\lvert S\right\rvert\gg n^2$.
Any mechanism forcing an asymptotic average run length $<2$ would already yield $I(A)\to\infty$.
\end{enumerate}

\subsection*{7) Completion estimate}
\[
\textbf{Completion: }35\%.
\]
(We have rigorous structural partial results and a clear obstruction, but no route currently to force singleton runs.)

% ======================================================================

