% Erdos Problem #467
% URL: https://www.erdosproblems.com/467

\noindent\textbf{1) FORMAL RESTATEMENT.}

Fix a (large) real $x$; for concreteness interpret $x$ as a large integer and quantify over integers $n$ with $1\le n<x$. For each prime $p\le x$ choose a residue class $a_p\in\{0,1,\dots,p-1\}$. Partition the primes $\{p\le x\}$ into a disjoint union $A\sqcup B$ with $A,B\ne\emptyset$. The desired property is:
\[
\forall n<x\ \exists p\in A,\ q\in B\ \text{s.t.}\ n\equiv a_p\pmod p\ \text{and}\ n\equiv a_q\pmod q.
\]

\medskip
\noindent\textbf{2) QUICK LITERATURE/CONTEXT CHECK.}

The statement notes that \cite{ErGr80} is missing quantifiers and that the above is an interpretation. I will work only with the explicitly stated formulation.

\medskip
\noindent\textbf{3) ATTACK PLAN.}

\begin{itemize}
\item First, rewrite the condition in a way that separates the roles of $A$ and $B$.
\item Prove a necessary counting condition on any single collection of prime-residue classes that covers all $n<x$.
\item Perform a small-$x$ exhaustive search to sanity-check the statement and to see where obstructions appear.
\end{itemize}

\medskip
\noindent\textbf{4) WORK.}

\noindent\textbf{Lemma 467.1 (equivalence to two separate coverings).}
The condition
\[
\forall n<x\ \exists p\in A\ \text{s.t.}\ n\equiv a_p\pmod p\quad\text{and}\quad \exists q\in B\ \text{s.t.}\ n\equiv a_q\pmod q
\]
is equivalent to the pair of covering statements:
\[
\forall n<x:\ n\in \bigcup_{p\in A}\{m: m\equiv a_p\ (\mathrm{mod}\ p)\}\quad\text{and}\quad
\forall n<x:\ n\in \bigcup_{q\in B}\{m: m\equiv a_q\ (\mathrm{mod}\ q)\}.
\]

\textit{Proof.}
The original condition says: for each $n<x$ there is at least one congruence from the $A$-family that $n$ satisfies, and at least one congruence from the $B$-family that $n$ satisfies. This is exactly the statement that $n$ lies in the union of the $A$-progressions and in the union of the $B$-progressions. \hfill$\square$

\medskip
\noindent\textbf{Lemma 467.2 (counting necessary condition for a single cover).}
Let $P$ be a set of primes and let $a_p$ be a residue class modulo $p$ for each $p\in P$. Suppose that for every integer $n$ with $1\le n<x$ there exists some $p\in P$ with $n\equiv a_p\pmod p$. Then
\[
\sum_{p\in P}\Big\lceil\frac{x-1}{p}\Big\rceil\ \ge\ x-1.
\]
In particular,
\[
\sum_{p\in P}\frac{1}{p}\ \ge\ 1-\frac{|P|}{x-1}.
\]

\textit{Proof.}
For each $p\in P$, the arithmetic progression $\{n: 1\le n<x,\ n\equiv a_p\ (\mathrm{mod}\ p)\}$ contains at most $\lceil (x-1)/p\rceil$ integers. The union of these sets covers the $x-1$ integers $1,2,\dots,x-1$, hence by the union bound
\[
 x-1\le \sum_{p\in P}\Big|\{n:1\le n<x,\ n\equiv a_p\ (\mathrm{mod}\ p)\}\Big|\le \sum_{p\in P}\Big\lceil\frac{x-1}{p}\Big\rceil.
\]
Since $\lceil (x-1)/p\rceil\le (x-1)/p+1$, rearranging yields $\sum_{p\in P} 1/p\ge 1-|P|/(x-1)$. \hfill$\square$

\medskip
\noindent\textbf{Corollary 467.3 (a necessary condition on $x$).}
If the conclusion of the problem holds for a given $x$, then necessarily
\[
\sum_{p\le x}\frac1p\ \ge\ 2 - O\Big(\frac{\pi(x)}{x}\Big),
\]
since by Lemma~467.1 each of the two sets $A$ and $B$ must separately satisfy Lemma~467.2.

\textit{Proof.}
Apply Lemma~467.2 to $P=A$ and $P=B$ and add the resulting inequalities. \hfill$\square$

\medskip
\noindent\textbf{FAST REALITY CHECK (small $x$ exhaustive search).}

I wrote an exact backtracking search for the interpreted statement (covering all integers $1\le n<x$). For each $x\le 20$ it searched over all partitions $A\sqcup B$ of the primes $\le x$ and, for each side, searched over all residue choices to see whether that side covers all $n<x$.

The script returned \emph{no} solutions for $5\le x\le 20$:
\begin{verbatim}
x=5: none
x=6: none
x=7: none
x=8: none
x=9: none
x=10: none
x=11: none
x=12: none
x=13: none
x=14: none
x=15: none
x=16: none
x=17: none
x=18: none
x=19: none
x=20: none
\end{verbatim}
This is consistent with Corollary~467.3: numerically,
\begin{verbatim}
x=10:  sum_{p<=x} 1/p = 1.176190476190476
x=20:  sum_{p<=x} 1/p = 1.4554777523817768
x=30:  sum_{p<=x} 1/p = 1.5334387718720317
x=100: sum_{p<=x} 1/p = 1.8028172010488706
x=1000:sum_{p<=x} 1/p = 2.198080127175088
\end{verbatim}
so for $x\le 100$ the harmonic-sum necessary condition ``total $\ge 2$'' is not even met.

\medskip
\noindent\textbf{5) VERIFICATION.}

\begin{itemize}
\item Lemma~467.1 is a direct rewriting; no hidden assumptions.
\item Lemma~467.2 uses only the trivial union bound and the fact that one residue class mod $p$ hits at most $\lceil (x-1)/p\rceil$ integers in $[1,x)$.
\item The computational check is exhaustive for each tested $x$ because it tries all partitions and all residues (via backtracking) for each side.
\end{itemize}

\medskip
\noindent\textbf{6) FINAL.} \textbf{UNRESOLVED}

(i) \textbf{Strongest proved partial result.}
The problem is equivalent to constructing \emph{two disjoint prime-modulus covering systems} of the finite interval $\{1,2,\dots,x-1\}$ (Lemma~467.1). Any such cover by primes $P$ must satisfy the counting condition $\sum_{p\in P} \lceil(x-1)/p\rceil\ge x-1$ (Lemma~467.2), forcing in particular $\sum_{p\in P}1/p\gtrsim 1$. Exhaustive search finds no examples for $x\le 20$.

(ii) \textbf{First gap (crisp).}
Construct, for all sufficiently large $x$, explicit residues $a_p$ and a partition $A\sqcup B$ such that \emph{each} of $A$ and $B$ individually covers all integers $1\le n<x$ by the single residue class assigned to each prime.

(iii) \textbf{Top 3 next moves.}
\begin{enumerate}
\item Attempt an explicit constructive cover for one side using small primes (to cover most residues) plus a controlled use of larger primes to mop up remaining uncovered $n$; then repeat disjointly for the other side.
\item Reformulate as a finite set-cover problem: each prime $p$ contributes one translate of the partition of $[1,x)$ into $p$ residue classes; seek two disjoint families whose unions both cover. Try a greedy algorithm and attempt to prove it terminates when $\sum_{p\le x}1/p$ is large enough.
\item Push computation with optimized search/ILP for moderate $x$ (say $x$ in the hundreds) to locate the smallest $x$ where a solution exists (if any), guided by the harmonic-sum constraint.
\end{enumerate}

(iv) \textbf{Minimal counterexample structure.}
A minimal failing $x$ would be one for which \emph{every} assignment of residues and every partition $A\sqcup B$ leaves some $n<x$ uncovered by $A$ or uncovered by $B$. By Lemma~467.1, it is enough to understand obstructions to a \emph{single} prime-residue cover of $[1,x)$ using a given subset of primes.


