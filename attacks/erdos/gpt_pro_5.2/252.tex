\section*{Erd\H{o}s Problem \#252}

\subsection*{1) FORMAL RESTATEMENT}
Fix an integer $k\ge 1$ and define the divisor power-sum function
\[
\sigma_k(n) := \sum_{d\mid n} d^k\qquad (n\in\mathbb{N}).
\]
Define the real number
\[
\alpha_k := \sum_{n=1}^{\infty} \frac{\sigma_k(n)}{n!}.
\]
The question is whether $\alpha_k\notin\mathbb{Q}$ for every integer $k\ge 1$.

\subsection*{2) QUICK LITERATURE/CONTEXT CHECK}
The irrationality of $\alpha_k$ is known unconditionally for $1\le k\le 4$: Erd\H{o}s observed that $k=1,2$ are ``reasonably straightforward'' \cite{Er52}; $k=3$ was proved by Schlage-Puchta \cite{ScPu06} and independently by Friedlander--Luca--Stoiciu \cite{FLC07}; $k=4$ was proved by Pratt \cite{Pr22}. For general $k$, irrationality is known conditionally (Schinzel's conjecture or the prime tuples conjecture) \cite{ScPu06,FLC07}. The problem is listed as open for general $k$ in \cite{ErdosProblems252}.

\subsection*{3) ATTACK PLAN}
\textbf{Track A (prove irrationality for each $k$):}
Use the standard ``factorial series'' method: if $\alpha_k=a/b\in\mathbb{Q}$, then for all $N\ge b$ the quantity $N!\alpha_k$ is an integer, so the tail
\[
T_{k}(N):=\sum_{n>N} \frac{\sigma_k(n)}{(N+1)(N+2)\cdots n}
\]
must be an integer. One seeks infinitely many $N$ for which $T_k(N)$ lies strictly between two consecutive integers.

\textbf{Track B (disprove):}
Try to exploit possible cancellation or known closed forms for related factorial series. No such mechanism is known for $\sigma_k(n)$.

\textbf{Track C (bridge to known proofs):}
For $k\ge 3$, known proofs use deeper input (sieve + primes with controlled local factorisation). Reproducing them fully is beyond what is attempted here; instead, I give complete proofs for $k=1,2$ and explain where the naive approach fails for $k\ge 3$.

\subsection*{4) WORK}
\subsubsection*{4.1 A basic lemma for factorial series}
\begin{lemma}\label{lem:factorial_rational_tail}
Let $(a_n)_{n\ge 1}$ be integers and let $A:=\sum_{n\ge1} a_n/n!$ (assumed convergent). If $A=a/b\in\mathbb{Q}$ in lowest terms, then for every integer $N\ge b$ the tail
\[
\sum_{n>N} a_n\frac{N!}{n!} \,=\, \sum_{n>N} \frac{a_n}{(N+1)(N+2)\cdots n}
\]
is an integer.
\end{lemma}
\begin{proof}
Fix $N\ge b$. Since $b\mid N!$, we have $N!A = N!a/b\in\mathbb{Z}$.
Also
\[
N!A = \sum_{n=1}^{N} a_n\frac{N!}{n!} + \sum_{n>N} a_n\frac{N!}{n!}.
\]
For $n\le N$ the ratio $N!/n!$ is an integer, hence the first sum is an integer. Therefore the second sum is an integer as well.
\end{proof}

\subsubsection*{4.2 The case $k=1$ (complete proof)}
Write $\sigma_1(n)=\sigma(n)$. Define
\[
\alpha_1 = \sum_{n\ge1} \frac{\sigma(n)}{n!}.
\]
\begin{theorem}\label{thm:k1}
$\alpha_1$ is irrational.
\end{theorem}
\begin{proof}
Assume for contradiction that $\alpha_1=a/b\in\mathbb{Q}$ in lowest terms.
Choose a prime $p>\max\{b,100\}$ and set $N:=p-1$.
Then $N\ge b$, so by Lemma \ref{lem:factorial_rational_tail} the tail
\[
T:=\sum_{n\ge p} \frac{\sigma(n)}{p(p+1)\cdots n}
\]
must be an integer.

The first term ($n=p$) satisfies $\sigma(p)=p+1$, so
\[
\frac{\sigma(p)}{p} = \frac{p+1}{p} = 1 + \frac{1}{p}.
\]
Write
\[
T = \left(1+\frac{1}{p}\right) + R,\qquad
R:=\sum_{m\ge1} \frac{\sigma(p+m)}{p(p+1)\cdots (p+m)}.
\]
Clearly $R>0$, hence $T>1$. It remains to show $T<2$.

We use two elementary bounds:
\begin{enumerate}
\item For every $n\ge1$, the divisor-counting function $\tau(n)$ satisfies $\tau(n)\le 2\sqrt{n}$ (divisors come in pairs $d$ and $n/d$, with at most $\sqrt{n}$ divisors $\le\sqrt{n}$).
Hence
\begin{equation}\label{eq:sigma_bound_k1}
\sigma(n)=\sum_{d\mid n} d \le n\tau(n) \le 2n^{3/2}.
\end{equation}
\item For $m\ge 1$,
\begin{equation}\label{eq:denom_bound}
 p(p+1)\cdots(p+m) = (p+m)\prod_{j=0}^{m-1} (p+j) \ge (p+m)\,p^m.
\end{equation}
\end{enumerate}
Combining \eqref{eq:sigma_bound_k1}--\eqref{eq:denom_bound}, for every $m\ge1$,
\[
\frac{\sigma(p+m)}{p(p+1)\cdots(p+m)}
\le \frac{2(p+m)^{3/2}}{(p+m)p^m} = \frac{2\sqrt{p+m}}{p^m}.
\]
Therefore
\[
R\le \sum_{m\ge1} \frac{2\sqrt{p+m}}{p^m}.
\]
Using $\sqrt{p+m}\le \sqrt{p}+\sqrt{m}$ and then $\sqrt{m}\le m$, we obtain
\[
R\le 2\sqrt{p}\sum_{m\ge1}\frac{1}{p^m} + 2\sum_{m\ge1}\frac{m}{p^m}
= 2\sqrt{p}\cdot \frac{1}{p-1} + 2\cdot \frac{p}{(p-1)^2}.
\]
Since $p>100$, we have $p-1>p/2$, hence
\[
2\sqrt{p}\cdot\frac{1}{p-1} < 2\sqrt{p}\cdot\frac{2}{p} = \frac{4}{\sqrt{p}} \le \frac{4}{10}=0.4,
\]
and
\[
2\cdot\frac{p}{(p-1)^2} < 2\cdot \frac{p}{(p/2)^2} = \frac{8}{p} < 0.08.
\]
So $R<0.48$. Consequently,
\[
T = 1 + \frac{1}{p} + R < 1 + 0.01 + 0.48 < 2.
\]
Thus $1<T<2$, so $T$ is not an integer, contradicting Lemma \ref{lem:factorial_rational_tail}. This contradiction shows $\alpha_1\notin\mathbb{Q}$.
\end{proof}

\subsubsection*{4.3 The case $k=2$ (complete proof)}
Define
\[
\alpha_2 = \sum_{n\ge1} \frac{\sigma_2(n)}{n!}.
\]
\begin{theorem}\label{thm:k2}
$\alpha_2$ is irrational.
\end{theorem}
\begin{proof}
Assume for contradiction that $\alpha_2=a/b\in\mathbb{Q}$ in lowest terms.
Choose a prime $p>\max\{b,17\}$ and set $N:=p-1$.
Then $N\ge b$, so by Lemma \ref{lem:factorial_rational_tail} the tail
\[
T:=\sum_{n\ge p} \frac{\sigma_2(n)}{p(p+1)\cdots n}
\]
must be an integer.

For the first term ($n=p$), since $p$ is prime,
\[
\sigma_2(p)=1+p^2,\qquad \frac{\sigma_2(p)}{p} = p + \frac{1}{p}.
\]
Write
\[
T = \left(p+\frac{1}{p}\right) + R,\qquad
R:=\sum_{m\ge1} \frac{\sigma_2(p+m)}{p(p+1)\cdots (p+m)}.
\]
\emph{Lower bound.}
For $m=1$, the divisors $1$ and $p+1$ contribute, so
$\sigma_2(p+1)\ge 1+(p+1)^2$, hence
\[
R\ge \frac{1+(p+1)^2}{p(p+1)} = \frac{p+1}{p}+\frac{1}{p(p+1)} > 1.
\]
Therefore $T>p+1$.

\emph{Upper bound.}
We first bound the constant
\[
C:=\sum_{d=1}^{\infty}\frac{1}{d^2}.
\]
Since $x\mapsto 1/x^2$ is decreasing on $(0,\infty)$, for every integer $d\ge 4$,
\[
\frac{1}{d^2}\le \int_{d-1}^{d} \frac{dx}{x^2} = \frac{1}{d-1}-\frac{1}{d}.
\]
Summing from $d=4$ to $\infty$ gives $\sum_{d\ge 4} 1/d^2 \le 1/3$. Hence
\begin{equation}\label{eq:C_bound}
C = 1+\frac{1}{4}+\frac{1}{9}+\sum_{d\ge4}\frac{1}{d^2} \le 1+\frac{1}{4}+\frac{1}{9}+\frac{1}{3} = \frac{61}{36}.
\end{equation}
Now for any $n\ge1$,
\[
\sigma_2(n)=\sum_{d\mid n} d^2 = n^2\sum_{e\mid n} \frac{1}{e^2} \le n^2\sum_{e\ge1}\frac{1}{e^2} = C n^2 \le \frac{61}{36} n^2.
\]
Therefore, for every $m\ge1$,
\begin{equation}\label{eq:um_def}
\frac{\sigma_2(p+m)}{p(p+1)\cdots(p+m)} \le \frac{61}{36}\cdot \frac{(p+m)^2}{p(p+1)\cdots(p+m)}.
\end{equation}
We bound $R$ by splitting off $m=1$:
\[
R \le \frac{61}{36}\cdot\frac{(p+1)^2}{p(p+1)} \, +\, \frac{61}{36}\sum_{m\ge2}\frac{(p+m)^2}{p(p+1)\cdots(p+m)}.
\]
The first piece is
\[
\frac{61}{36}\cdot\frac{(p+1)^2}{p(p+1)} = \frac{61}{36}\cdot\frac{p+1}{p}.
\]
For the remainder, define for $m\ge2$,
\[
u_m := \frac{(p+m)^2}{p(p+1)\cdots(p+m)}.
\]
Then for $m\ge2$,
\[
\frac{u_{m+1}}{u_m}
= \frac{(p+m+1)^2}{(p+m)^2}\cdot\frac{1}{p+m+1}
= \frac{p+m+1}{(p+m)^2}
\le \frac{2}{p+m}
\le \frac{2}{p+2}.
\]
Let $r:=2/(p+2)<1$. Then $u_m\le u_2\,r^{m-2}$ for all $m\ge2$, hence
\[
\sum_{m\ge2} u_m \le u_2\sum_{j\ge0} r^j = \frac{u_2}{1-r}.
\]
Compute
\[
u_2 = \frac{(p+2)^2}{p(p+1)(p+2)} = \frac{p+2}{p(p+1)},
\qquad
\frac{1}{1-r} = \frac{1}{1-2/(p+2)} = \frac{p+2}{p}.
\]
Therefore
\[
\sum_{m\ge2}u_m \le \frac{p+2}{p(p+1)}\cdot\frac{p+2}{p} = \frac{(p+2)^2}{p^2(p+1)}.
\]
Putting the bounds together,
\[
R \le \frac{61}{36}\cdot\frac{p+1}{p} + \frac{61}{36}\cdot\frac{(p+2)^2}{p^2(p+1)}.
\]
Hence
\begin{equation}\label{eq:1overp_plus_R}
\frac{1}{p}+R \le \frac{1}{p} + \frac{61}{36}\cdot\frac{p+1}{p} + \frac{61}{36}\cdot\frac{(p+2)^2}{p^2(p+1)}.
\end{equation}
Using $p\ge 17$, we have $(p+1)/p\le 18/17$ and $(p+2)^2/(p^2(p+1))\le (p+2)^2/p^3 \le (19/17)^2\cdot(1/17) = 361/4913$. Substituting into \eqref{eq:1overp_plus_R} gives
\[
\frac{1}{p}+R \le \frac{1}{17} + \frac{61}{36}\cdot\frac{18}{17} + \frac{61}{36}\cdot\frac{361}{4913}
< 2.
\]
Consequently $T = p + (1/p+R) < p+2$.

We have shown $p+1 < T < p+2$, so $T$ is not an integer, contradicting Lemma \ref{lem:factorial_rational_tail}. This contradiction implies $\alpha_2\notin\mathbb{Q}$.
\end{proof}

\subsubsection*{4.4 Why the naive bounds break for $k\ge 3$}
For $k\ge 3$, the first tail term at $N=p-1$ is
\[
\frac{\sigma_k(p)}{p} = \frac{1+p^k}{p} = p^{k-1} + \frac{1}{p},
\]
but the next term (at $n=p+1$) is typically of size about $(p+1)^{k-1}/p\asymp p^{k-2}$, which grows without bound. Thus bounding the entire tail in an interval of length $<1$ requires nontrivial cancellation or congruence structure beyond the elementary estimates used above. This is where the sieve-based arguments in \cite{ScPu06,FLC07,Pr22} enter.

\subsection*{5) VERIFICATION (adversarial check)}
\begin{itemize}
\item Lemma \ref{lem:factorial_rational_tail} is standard; the only needed fact is $b\mid N!$ for $N\ge b$.
\item In Theorem \ref{thm:k1}, the bounds $\tau(n)\le 2\sqrt{n}$ and $\sigma(n)\le n\tau(n)$ are elementary and hold for all $n$.
\item In Theorem \ref{thm:k2}, the inequality $\sigma_2(n)=n^2\sum_{e\mid n} 1/e^2 \le n^2\sum_{e\ge1}1/e^2$ is exact, and the bound $\sum_{e\ge1}1/e^2\le 61/36$ is justified by a telescoping integral comparison.
\item The geometric-series domination for $\sum_{m\ge2}u_m$ uses the uniform ratio bound $u_{m+1}/u_m \le 2/(p+2)<1$.
\end{itemize}

\subsection*{6) FINAL}
\textbf{UNRESOLVED (for general $k$).}
\begin{enumerate}
\item[(i)] \textbf{Most advanced partial result proved here.}
Complete, self-contained irrationality proofs are given for $k=1$ and $k=2$ (Theorems \ref{thm:k1} and \ref{thm:k2}).
\item[(ii)] \textbf{Strongest obstruction / missing lemma.}
For $k\ge 3$, at the natural choice $N=p-1$ the second tail term (near $n=p+1$) is too large to keep the tail in an interval of length $<1$ using only crude divisor bounds; controlling this requires producing primes with special multiplicative structure in nearby integers, which is a genuinely sieve-theoretic input.
\item[(iii)] \textbf{Next concrete steps.}
To extend beyond $k=2$ by elementary means, one would need a systematic way to select $N$ so that the early tail terms have nearly-integer sum and the remaining tail is $<1$; this is exactly what the known arguments for $k=3,4$ accomplish via deep distribution results for primes and almost-primes.
\item[(iv)] \textbf{Counterexample search description.}
A counterexample would be an explicit $k$ and rational value of $\alpha_k$. Since $\sigma_k(n)\in\mathbb{Z}$, one can compute high-precision approximations of $\alpha_k$ and attempt rational reconstruction for bounded denominators; this cannot certify irrationality, but can falsify specific rational hypotheses.
\end{enumerate}

\subsection*{7) COMPLETION ESTIMATE}
$\mathbf{97\%}$ confidence in the correctness of the complete proofs for $k=1,2$ and the cited status ``known for $k\le 4$; open for general $k$''.


