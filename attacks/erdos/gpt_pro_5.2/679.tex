% Erdos Problem #679
\item
\textbf{FORMAL RESTATEMENT.}
Fix $\varepsilon>0$. Does there exist an infinite set of integers $n$ such that for all integers $k$ with $k<n$ and $k$ sufficiently large (i.e. $k\ge K(\varepsilon)$), one has
\[
\omega(n-k)\;<\;(1+\varepsilon)\,\frac{\log k}{\log\log k}
\qquad\text{where }\omega(m)=\#\{p:\,p\text{ prime},\ p\mid m\}\,?
\]

\textbf{QUICK LITERATURE/CONTEXT CHECK.}
The problem statement itself notes (and I will treat as given) that a stronger estimate with an additive $+O(1)$ term,
\[\omega(n-k) < \frac{\log k}{\log\log k} + O(1),\]
would be false for every large $n$ (by the argument attributed in the statement to \texttt{DottedCalculator}). I do not assume any further external results beyond what is explicitly written in the problem text.

\textbf{ATTACK PLAN.}
A natural way to \emph{force} large values of $\omega(n-k)$ is to make $n-k$ divisible by a primorial $Q:=\prod_{p\le y}p$, because then $\omega(n-k)\ge\omega(Q)=\pi(y)$. One then compares $\pi(y)$ to $\log k/\log\log k$ for $k$ of size $Q$.
To \emph{prove} the original question, one would need to show there are infinitely many $n$ for which \emph{every} large $k<n$ avoids making $n-k$ have too many distinct prime factors.

\textbf{WORK.}
\textbf{Lemma 1 (Primorial divisor appears with $k$ of comparable size).}
Let $y\ge2$ and $Q:=\prod_{p\le y}p$. If $n\ge 2Q$, then there exists an integer $k$ with
\[Q\le k <2Q\quad\text{and}\quad Q\mid (n-k).
\]
In particular $\omega(n-k)\ge \omega(Q)=\pi(y)$.

\emph{Proof.}
Write $n\equiv r\pmod Q$ with $0\le r<Q$.
Set $k:=Q+r$. Then $Q\le k<2Q$.
Also $n-k\equiv r-(Q+r)\equiv -Q\equiv 0\pmod Q$, so $Q\mid(n-k)$.
Since $Q$ is squarefree with prime factors exactly the primes $\le y$, we have $\omega(Q)=\pi(y)$, and hence $\omega(n-k)\ge\omega(Q)=\pi(y)$.
\hfill$\square$

\textbf{Lemma 2 (Elementary lower bound for $\pi(y)$ in terms of $Q$).}
With $Q=\prod_{p\le y}p$ as above,
\[\pi(y)\ge \frac{\log Q}{\log y}.
\]

\emph{Proof.}
Taking logs,
\[\log Q=\sum_{p\le y}\log p\le \sum_{p\le y}\log y=\pi(y)\log y.
\]
Rearranging gives $\pi(y)\ge \log Q/\log y$.
\hfill$\square$

\textbf{Lemma 3 (Relating $\log k/\log\log k$ for $k\in[Q,2Q)$ to $Q$).}
If $Q\ge 3$ and $Q\le k<2Q$, then
\[
\frac{\log Q}{\log\log(2Q)}\le \frac{\log k}{\log\log k}\le \frac{\log(2Q)}{\log\log Q}.
\]

\emph{Proof.}
Since $Q\le k<2Q$, we have $\log Q\le \log k\le \log(2Q)$.
Also $\log\log Q\le \log\log k\le \log\log(2Q)$ because $\log\log$ is increasing for $Q\ge 3$.
Combine these monotonicity bounds to obtain the displayed inequalities.
\hfill$\square$

\textbf{What these lemmas give (and what they do not).}
Lemma~1 shows that for \emph{every} $n$ large enough relative to $Q$ there exists some $k\asymp Q$ with
\[\omega(n-k)\ge\pi(y)\ge \frac{\log Q}{\log y}.
\]
To contradict the desired inequality for a given $\varepsilon>0$, one would like to arrange
\[\frac{\log Q}{\log y}\;\gtrsim\;(1+\varepsilon)\frac{\log Q}{\log\log Q},
\]
i.e. roughly $\log y\lesssim \frac{1}{1+\varepsilon}\log\log Q$.
But $y$ and $Q=\prod_{p\le y}p$ are linked by the distribution of primes (since $\log Q=\sum_{p\le y}\log p$). Without additional input on how $\log Q$ grows with $y$, the above comparison cannot be completed.
This is exactly the kind of second-order relationship alluded to in the statement (the disproof of the stronger $+O(1)$ version depends on sharper prime distribution information).

\textbf{VERIFICATION (FAST REALITY CHECK).}
I ran a local brute-force sanity check for moderate $n$, computing
\[D(n):=\max_{3\le k<n}\Bigl(\omega(n-k)-\frac{\log k}{\log\log k}\Bigr),\]
using a sieve for $\omega(\cdot)$. The following maxima were found:
\begin{verbatim}
n=5000:    max diff ~ 1.861 at k=170,  n-k=4830 with omega=5, rhs=3.139
n=50000:   max diff ~ 2.270 at k=20,   n-k=49980 with omega=5, rhs=2.730
n=100000:  max diff ~ 2.526 at k=670,  n-k=99330 with omega=6, rhs=3.474
n=100000 with k>=1000: max diff ~ 2.354 at k=1330, n-k=98670 with omega=6, rhs=3.646
\end{verbatim}
These data are only a sanity check; they do not address the quantifier structure in the problem (which asks for $\omega(n-k)$ to be small for \emph{all} large $k<n$ for infinitely many $n$).

\textbf{FINAL.}
\textbf{UNRESOLVED.}

(i) \textbf{Strongest proved partial result.}
For any $y$ and its primorial $Q=\prod_{p\le y}p$, every $n\ge 2Q$ admits some $k\in[Q,2Q)$ with $\omega(n-k)\ge\pi(y)\ge \log Q/\log y$ (Lemmas~1--2). This gives a systematic way to build \emph{some} large $k$ with many prime factors in $n-k$.

(ii) \textbf{First gap (crisp).}
One needs a quantitative relationship between $y$ and $Q=\prod_{p\le y}p$ strong enough to compare $\pi(y)$ to $\log Q/\log\log Q$ at the scale $k\asymp Q$. With only the elementary inequalities in Lemma~2 and monotonicity, I cannot decide whether the factor $(1+\varepsilon)$ makes the statement true or false.

(iii) \textbf{Top 3 next moves.}
(1) Prove an explicit bound of the form $\log Q\ge c_1 y$ (or a two-sided comparison) using only tools allowed by the problem context, and then translate it into a comparison between $\pi(y)$ and $\log Q/\log\log Q$.
(2) Numerically, for each $n$ up to a larger cutoff, compute
\[\max_{k\ge K} \frac{\omega(n-k)\,\log\log k}{\log k}
\]
for a fixed moderate $K$, to see whether values exceed $1+\varepsilon$ densely or sparsely.
(3) Try to adapt the construction behind the stated disproof of the $+O(1)$ version to see whether it can also defeat the multiplicative $(1+\varepsilon)$ buffer (or conversely show that it inherently only gives a smaller additive improvement).

(iv) \textbf{Minimal counterexample structure.}
A counterexample $n$ to the \emph{desired} property (for a given $\varepsilon$) would exhibit some large $k<n$ (necessarily with $k\gtrsim K(\varepsilon)$) such that $n-k$ is divisible by a highly squarefree integer (e.g. a primorial-like product) with about $(1+\varepsilon)\log k/\log\log k$ distinct prime factors.


