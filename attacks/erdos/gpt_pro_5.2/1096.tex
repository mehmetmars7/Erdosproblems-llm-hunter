
\medskip
\noindent\textbf{FORMAL RESTATEMENT}

\smallskip
\noindent
Fix a real number $q>1$. Define the set
\[
X(q):=\Big\{\sum_{i\in S} q^i\ :\ S\subseteq \{0,1,2,\dots\}\text{ finite}\Big\}\subset\mathbb{R}_{\ge 0}.
\]
Let $\{x_k\}_{k\ge 1}$ be the strictly increasing sequence of \emph{distinct} elements of $X(q)$, so $x_1=0<x_2<\cdots$.

\smallskip
\noindent
\textbf{Question.}
Is it true that there exists $\epsilon>0$ such that for every $q$ with $1<q<1+\epsilon$ we have
\[
 x_{k+1}-x_k\to 0\quad\text{as }k\to\infty?
\]

\medskip
\noindent\textbf{QUICK LITERATURE/CONTEXT CHECK}

\smallskip
\noindent
The problem text mentions a conjectured threshold near $q_0\approx 1.3247$ (real root of $x^3=x+1$) and states (i) a result that Pisot--Vijayaraghavan numbers cannot have the gap-$\to0$ property, and (ii) a uniform bound $x_{k+1}-x_k\le 1$ for all $k$ when $1<q\le 2$. Per the integrity rule for this task, I do not assert any additional literature beyond what is written in the problem file.

\medskip
\noindent\textbf{ATTACK PLAN}

\smallskip
\noindent
\begin{itemize}
\item Prove from first principles the uniform gap bound $x_{k+1}-x_k\le 1$ for $1<q\le 2$ (a strong partial result).
\item Use the recursive structure $X_{m+1}=X_m\cup(X_m+q^m)$ to understand how gaps behave as $m$ grows.
\item Compute large initial segments numerically for $q$ close to $1$ to see whether typical gaps appear to shrink.
\end{itemize}

\medskip
\noindent\textbf{WORK}

\smallskip
\noindent\textbf{Lemma 1096.1 (finite-stage decomposition).}
For $m\ge 0$, define the finite subset
\[
X_m(q):=\Big\{\sum_{i=0}^{m-1} \varepsilon_i q^i\ :\ \varepsilon_i\in\{0,1\}\Big\}.
\]
Then
\[
X_{m+1}(q)=X_m(q)\ \cup\ \bigl(X_m(q)+q^m\bigr),
\]
and the maximum element of $X_m(q)$ is
\[
\max X_m(q)=\sum_{i=0}^{m-1} q^i=\frac{q^m-1}{q-1}.
\]

\smallskip
\noindent\emph{Proof.}
A subset $S\subseteq\{0,1,\dots,m\}$ either does not contain $m$ or does contain $m$. If it does not contain $m$, its sum lies in $X_m(q)$. If it does contain $m$, then writing $S=S'\cup\{m\}$ with $S'\subseteq\{0,\dots,m-1\}$, we have
\[
\sum_{i\in S}q^i=q^m+\sum_{i\in S'}q^i\in X_m(q)+q^m.
\]
This proves the union decomposition.

For the maximum, since all terms $q^i$ are positive, the maximum is attained by taking all $\varepsilon_i=1$, giving the geometric sum $\sum_{i=0}^{m-1}q^i=(q^m-1)/(q-1)$. \qed

\smallskip
\noindent\textbf{Lemma 1096.2 (uniform gap bound for $1<q\le 2$).}
Assume $1<q\le 2$. For every $m\ge 1$, if we list the \emph{distinct} elements of $X_m(q)$ in increasing order, the gap between any two consecutive elements is at most $1$. Consequently, for the full ordered sequence $0=x_1<x_2<\cdots$ of $X(q)$, we have
\[
 x_{k+1}-x_k\le 1\quad\text{for all }k\ge 1.
\]

\smallskip
\noindent\emph{Proof.}
We prove by induction on $m$ that the maximum gap between consecutive elements in the sorted list of $X_m(q)$ is at most $1$.

\emph{Base case $m=1$.} We have $X_1(q)=\{0,1\}$, so the only gap is $1$.

\emph{Induction step.} Assume the claim for $X_m(q)$, and consider $X_{m+1}(q)$. By Lemma 1096.1,
\[
X_{m+1}(q)=A\cup (A+q^m),\quad\text{where }A:=X_m(q).
\]
Sort $A$ increasingly as $a_1<\cdots<a_r$ (distinct values). Then $A+q^m$ sorts as $(a_1+q^m)<\cdots<(a_r+q^m)$.

Gaps \emph{within} the $A$ block are at most $1$ by the induction hypothesis. The same holds within the shifted block $A+q^m$ (since shifting does not change gaps).

It remains to control the gap between the top of $A$ and the bottom of $A+q^m$. The smallest element of $A+q^m$ is $q^m$ (shift of $0\in A$), and the largest element of $A$ is $\max A=(q^m-1)/(q-1)$ by Lemma 1096.1. Therefore the separation between the two blocks is
\[
\Delta_m:=q^m-\frac{q^m-1}{q-1}.
\]
If $\Delta_m\le 0$, then the two blocks overlap as intervals, so there is no gap created between them.
If $\Delta_m>0$, then the union creates a gap of size exactly $\Delta_m$ between the last element of $A$ and the first element of $A+q^m$.

We now show $\Delta_m\le 1$ for all $m$ whenever $1<q\le 2$.
Compute
\[
\Delta_m-1
=\left(q^m-\frac{q^m-1}{q-1}\right)-1
=\frac{(q-1)q^m-(q^m-1)-(q-1)}{q-1}
=\frac{(q-2)(q^m-1)}{q-1}.
\]
Since $1<q\le 2$, we have $q-2\le 0$, and also $q^m-1\ge 0$ and $q-1>0$. Hence $(q-2)(q^m-1)/(q-1)\le 0$, so $\Delta_m-1\le 0$, i.e. $\Delta_m\le 1$.

Therefore every possible new ``inter-block'' gap is $\le 1$, and the within-block gaps are $\le 1$ by induction. This completes the induction for all $m$.

Finally, to pass from $X_m(q)$ to $X(q)$: given any bound $B>0$, if $q^m>B$ then every element of $X(q)$ that is $\le B$ uses only exponents $<m$ (since any term $q^i$ with $i\ge m$ is already $>B$). Thus the initial segment of the increasing sequence $(x_k)$ up to $B$ coincides with the sorted list of $X_m(q)$ up to $B$, and the gap bound $\le 1$ carries over to all $k$. \qed

\smallskip
\noindent\textbf{FAST REALITY CHECK (numerics for $q$ near $1$).}

\smallskip
\noindent
Because $X(q)$ is infinite, I approximated initial segments by taking $X_m(q)$ for moderate $m$, sorting, and examining the first $K$ consecutive gaps. For $q=11/10$ and $m=18$, I also computed all $2^{18}=262144$ sums exactly as rationals (scaled integers), obtaining an exact minimum gap.

\smallskip
\noindent
\emph{Exact rational check:} for $q=11/10$ and $m=18$, among all distinct elements of $X_{18}(q)$,
\[
\min\{y'-y:\ y<y',\ y,y'\in X_{18}(11/10)\}=1.8372458061\times 10^{-7}.
\]

\smallskip
\noindent
\emph{Floating-point exploratory check (first $K$ gaps within $X_m$):}
\begin{itemize}
\item For $q=1.1$ with $m=18$, among the first $K=50{,}000$ sorted values from $X_{18}(q)$ (after removing the trivial initial gap $1$), the smallest observed gap was $\approx 4.28\times 10^{-7}$ and the largest observed gap was $0.1$.
\item For $q=1.001$ with $m=20$, among the first $K=50{,}000$ sorted values from $X_{20}(q)$ (after removing the trivial initial gap $1$), the smallest observed gap was $\approx 1.20\times 10^{-13}$.
\end{itemize}
These computations provide evidence that very small gaps occur already at modest scales when $q$ is close to $1$, but they do not by themselves prove $x_{k+1}-x_k\to 0$.

\medskip
\noindent\textbf{VERIFICATION}

\smallskip
\noindent
\begin{itemize}
\item Lemma 1096.1 is purely combinatorial: every subset either contains $m$ or not.
\item Lemma 1096.2 checks the only possible new gap created when passing from $X_m$ to $X_{m+1}$, and bounds it by an explicit algebraic calculation. The sign case $\Delta_m\le 0$ was handled separately.
\item The ``finite-stage implies initial-segment'' argument at the end of Lemma 1096.2 is checked: if $x\le B$ and $q^m>B$, then no term $q^i$ with $i\ge m$ can appear in a representation of $x$.
\item The exact rational computation for $q=11/10$ avoids floating-point artifacts.
\end{itemize}

\medskip
\noindent\textbf{UNRESOLVED}

\smallskip
\noindent
(i) \emph{Strongest proved partial result here.}
For every $q$ with $1<q\le 2$, we proved $x_{k+1}-x_k\le 1$ for all $k$ (Lemma 1096.2). We also established the recursive decomposition $X_{m+1}=X_m\cup(X_m+q^m)$ (Lemma 1096.1).

\smallskip
\noindent
(ii) \emph{First gap (crisp).}
Prove (or disprove) that there exists $\epsilon>0$ such that for every $q\in(1,1+\epsilon)$ and every $\delta>0$ there exists $K$ with
\[
 x_{k+1}-x_k<\delta\quad\text{for all }k\ge K.
\]

\smallskip
\noindent
(iii) \emph{Top 3 next moves.}
\begin{enumerate}
\item Quantify how much overlap occurs between $X_m$ and $X_m+q^m$ for $q$ close to $1$ and large $m$, and convert overlap into explicit small gaps between consecutive sorted sums.
\item Replace floating computations by exact arithmetic for carefully chosen rational $q$ and larger $m$ to search for persistent large gaps (or to measure how the maximum gap in the tail behaves).
\item Try to relate the gap-$\to0$ property to the existence of many near-collisions of the form $\sum \varepsilon_i q^i\approx \sum \varepsilon'_i q^i$ (small values of polynomials with coefficients in $\{-1,0,1\}$ evaluated at $q$).
\end{enumerate}

\smallskip
\noindent
(iv) \emph{Minimal counterexample structure.}
A counterexample for small $\epsilon$ would consist of some $q\in(1,1+\epsilon)$ and a $\delta>0$ such that infinitely many consecutive gaps satisfy $x_{k+1}-x_k\ge \delta$. This would mean infinitely many real intervals of length $\delta$ contain no elements of $X(q)$, forcing strong ``no-approximation'' constraints on differences of two subset sums $\sum \varepsilon_i q^i-\sum \varepsilon'_i q^i$.

