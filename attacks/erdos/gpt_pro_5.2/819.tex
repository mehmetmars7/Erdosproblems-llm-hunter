\subsection*{Erdos Problem \#819}

\paragraph{FORMAL RESTATEMENT.}
For each positive integer $N$, let
\[
m := \lfloor N^{1/2}\rfloor.
\]
Define $f(N)$ to be the maximum of
\[
|(A+A)\cap [1,N]|
\]
over all subsets $A\subseteq\{1,2,\dots,N\}$ with $|A|=m$, where $A+A:=\{a+b:\ a,b\in A\}$ and $[1,N]=\{1,2,\dots,N\}$.
The problem asks to estimate $f(N)$.

\paragraph{QUICK LITERATURE/CONTEXT CHECK.}
The statement records that Erd\H{o}s and Freud proved
\[
\left(\frac{3}{8}-o(1)\right)N \le f(N) \le \left(\frac{1}{2}+o(1)\right)N.
\]
I do not use any results not already stated in the problem text.

\paragraph{ATTACK PLAN.}
\emph{Upper bounds:} use counting of unordered pairs to control how many distinct sums can appear.
\emph{Lower bounds:} construct explicit $A$ producing many sums in $[1,N]$; as a baseline, compute small $N$ exactly to see what is achievable.

\paragraph{WORK.}
\textbf{Lemma 819.1 (unconditional counting upper bound).}
Let $m=\lfloor \sqrt N\rfloor$. For every $A\subseteq[1,N]$ with $|A|=m$,
\[
|(A+A)\cap[1,N]| \le \binom{m+1}{2} = \frac{m(m+1)}{2} \le \frac{N}{2} + O(\sqrt N).
\]

\emph{Proof.}
Every sum $s\in A+A$ can be represented as $s=a+b$ with $a,b\in A$.
If we ignore order, each unordered pair $\{a,b\}$ with $a,b\in A$ produces at most one sum.
There are exactly $\binom{m}{2}$ unordered pairs with $a\ne b$ and $m$ diagonal pairs $\{a,a\}$, so in total there are $\binom{m}{2}+m=\binom{m+1}{2}$ unordered pairs.
Distinct sums are at most the number of unordered pairs, hence $|A+A|\le \binom{m+1}{2}$, and a fortiori $|(A+A)\cap[1,N]|\le \binom{m+1}{2}$.
Since $m^2\le N < (m+1)^2$, we have $m^2\le N$ and $m\le \sqrt N$, giving $\binom{m+1}{2}\le \frac{1}{2}N + O(\sqrt N)$.
\hfill $\square$

\textbf{Lemma 819.2 (baseline lower bound).}
For $m=\lfloor\sqrt N\rfloor$, taking $A=\{1,2,\dots,m\}$ gives
\[
|(A+A)\cap[1,N]| \ge 2m-1 = 2\lfloor\sqrt N\rfloor-1.
\]

\emph{Proof.}
If $A=\{1,2,\dots,m\}$ then
\[
A+A = \{2,3,\dots,2m\}.
\]
All these sums are $\le 2m\le 2\sqrt N\le N$ for $N\ge4$; in any case, $(A+A)\cap[1,N]$ contains $\{2,3,\dots,2m\}$.
Thus $|(A+A)\cap[1,N]|\ge 2m-1$.
\hfill $\square$

\textbf{FAST REALITY CHECK (exact values for $N\le 30$).}
I exhaustively computed $f(N)$ for $1\le N\le 30$ by enumerating all subsets $A\subseteq[1,N]$ of size $m=\lfloor\sqrt N\rfloor$.
The maximal values found were:
\[
\begin{array}{c|c|c}
N & m & f(N)\\\hline
1 & 1 & 0\\
2\le N\le 3 & 1 & 1\\
4\le N\le 8 & 2 & 3\\
9\le N\le 15 & 3 & 6\\
16\le N\le 24 & 4 & 10\\
25\le N\le 30 & 5 & 15
\end{array}
\]
For example, at $N=16$ the set $A=\{1,2,4,8\}$ achieves $f(16)=10$.

\paragraph{VERIFICATION.}
Lemma 819.1: the only step is the injection ``distinct sums $\le$ unordered pairs''; this is valid because each unordered pair produces a single integer sum, though multiple pairs may collide.
Lemma 819.2: sums of the initial segment form the full interval $[2,2m]$; checking the intersection with $[1,N]$ is immediate.
The computed $f(N)$ values are finite enumerations and can be rechecked.

\paragraph{FINAL.} \textbf{UNRESOLVED.}
\begin{enumerate}
\item[(i)] Strongest proved partial results here: the unconditional upper bound $f(N)\le \binom{\lfloor\sqrt N\rfloor+1}{2} = \frac{1}{2}N+O(\sqrt N)$ (Lemma 819.1), matching the scale of the upper bound stated in the problem, and the explicit lower bound $f(N)\ge 2\lfloor\sqrt N\rfloor-1$ (Lemma 819.2). Exact values for $N\le 30$ were computed.
\item[(ii)] First gap: produce (or rigorously derive) a linear-in-$N$ lower bound $f(N)\ge cN$ with an explicit constant $c>0$ using only arguments developed here (i.e. without importing the stated Erd\H{o}s--Freud result).
\item[(iii)] Top 3 next moves: (1) attempt a probabilistic construction of a size-$\sqrt N$ set $A$ for which many $s\le N$ have at least one representation $s=a+b$; (2) search computationally for maximizers for larger $N$ (e.g. $N\le 60$) to guess structure; (3) prove structural lemmas relating large $|(A+A)\cap[1,N]|$ to quasi-Sidon behaviour (few collisions) to approach the $\sim \tfrac12 N$ barrier.
\item[(iv)] A minimal counterexample to any claimed linear lower bound would be a sequence $N_j\to\infty$ such that every $A\subseteq[1,N_j]$ with $|A|=\lfloor\sqrt{N_j}\rfloor$ satisfies $|(A+A)\cap[1,N_j]|=o(N_j)$; the small-$N$ data already contradict this behaviour at low scales, suggesting any counterexample would require subtle additive structure.
\end{enumerate}


