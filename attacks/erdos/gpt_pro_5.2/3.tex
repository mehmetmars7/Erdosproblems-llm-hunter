
\noindent\textbf{1) FORMAL RESTATEMENT.}

Let $A\subseteq\mathbb{N}$. Assume
\[
\sum_{n\in A} \frac{1}{n} = \infty.
\]
Question: must $A$ contain arithmetic progressions of every length? Concretely, for each integer $k\ge 3$, must there exist integers $a\ge 1$ and $d\ge 1$ such that
\[
\{a,a+d,a+2d,\dots,a+(k-1)d\}\subseteq A?
\]
(Here ``non-trivial'' means $d\ne 0$.)

\medskip
\noindent\textbf{2) QUICK LITERATURE/CONTEXT CHECK.}

The problem text explains that this is closely connected to upper bounds on $r_k(N)$, the maximum size of a subset of $\{1,\dots,N\}$ containing no non-trivial $k$-term arithmetic progression. It notes that a bound like
\[
r_k(N)\ll_k \frac{N}{(\log N)(\log\log N)^2}
\]
would suffice, and that even the case $k=3$ is nontrivial but is proved by Bloom--Sisask. I do not reprove those deep results here; I only establish elementary implications from the divergence hypothesis.

\medskip
\noindent\textbf{3) ATTACK PLAN.}

\begin{itemize}
\item Use divergence of $\sum_{n\in A}1/n$ to force $A$ to be relatively large inside infinitely many initial intervals $[1,N]$.
\item Compare $|A\cap[1,N]|$ to $r_k(N)$: if $|A\cap[1,N]|>r_k(N)$ for infinitely many $N$, then $A$ must contain a $k$-term AP.
\item Sanity-check finite versions computationally: maximize the reciprocal sum over $k$-AP-free subsets of $[1,N]$ for small $N$.
\end{itemize}

\medskip
\noindent\textbf{4) WORK.}

\noindent\textbf{Lemma 1 (dyadic decomposition upper bound).}
Let $A\subseteq\mathbb{N}$. For each integer $j\ge 0$ define the dyadic block
\[B_j := A\cap(2^j,2^{j+1}].\]
Then
\[
\sum_{n\in A\cap[2,\infty)} \frac{1}{n}\;\le\;\sum_{j=0}^{\infty} \frac{|B_j|}{2^j}.
\]

\noindent\emph{Proof.}
Fix $j\ge 0$. If $n\in(2^j,2^{j+1}]$ then $n>2^j$ and hence $1/n\le 1/2^j$. Therefore
\[
\sum_{n\in B_j}\frac{1}{n}\le \sum_{n\in B_j}\frac{1}{2^j} = \frac{|B_j|}{2^j}.
\]
Summing this inequality over $j\ge 0$ yields the claim. \hfill$\square$

\medskip
\noindent\textbf{Lemma 2 (a concrete density lower bound forced by divergence).}
Let $A\subseteq\mathbb{N}$. Suppose that for some integer $J_0\ge 2$ and some constant $C>0$ we have
\[
|A\cap[1,N]|\le \frac{C\,N}{(\log N)(\log\log N)^2}\qquad\text{for all }N\ge 2^{J_0}.
\]
Then the reciprocal series converges:
\[
\sum_{n\in A}\frac{1}{n}<\infty.
\]
Equivalently, if $\sum_{n\in A}1/n=\infty$, then for every $C>0$ and every $J_0$ there exists some $N\ge 2^{J_0}$ with
\[
|A\cap[1,N]|> \frac{C\,N}{(\log N)(\log\log N)^2}.
\]

\noindent\emph{Proof.}
Assume the displayed upper bound on $|A\cap[1,N]|$ holds for all $N\ge 2^{J_0}$. For each integer $j\ge J_0$ consider the dyadic block $B_j=A\cap(2^j,2^{j+1}]$. Then
\[
|B_j|\le |A\cap[1,2^{j+1}]|\le \frac{C\,2^{j+1}}{(\log 2^{j+1})(\log\log 2^{j+1})^2}.
\]
Now apply Lemma 1:
\[
\sum_{n\in A\cap[2,\infty)}\frac{1}{n}
\le \sum_{j=0}^{J_0-1}\frac{|B_j|}{2^j} + \sum_{j=J_0}^{\infty}\frac{|B_j|}{2^j}.
\]
The first sum is finite because it has finitely many terms. For the tail, substitute the bound on $|B_j|$:
\[
\frac{|B_j|}{2^j}
\le \frac{C\,2^{j+1}}{2^j\,(\log 2^{j+1})(\log\log 2^{j+1})^2}
= \frac{2C}{(\log 2^{j+1})(\log\log 2^{j+1})^2}.
\]
Using $\log 2^{j+1}=(j+1)\log 2$ and $\log\log 2^{j+1}=\log((j+1)\log 2)$, the tail is dominated by a constant multiple of
\[
\sum_{j=J_0}^{\infty} \frac{1}{(j+1)\,\big(\log(j+1)\big)^2},
\]
which converges (for example by the integral test, since $\int^{\infty} \frac{dx}{x(\log x)^2}<\infty$). Thus $\sum_{n\in A\cap[2,\infty)} 1/n$ converges, and adding the (possible) $n=1$ term shows $\sum_{n\in A}1/n<\infty$.

Taking the contrapositive gives the stated consequence when the reciprocal series diverges. \hfill$\square$

\medskip
\noindent\textbf{Lemma 3 (link to $r_k(N)$ by definition).}
Fix $k\ge 3$ and $N\ge 1$. If $|A\cap\{1,\dots,N\}|>r_k(N)$, then $A$ contains a non-trivial $k$-term arithmetic progression whose terms all lie in $\{1,\dots,N\}$.

\noindent\emph{Proof.}
By definition, $r_k(N)$ is the \emph{maximum} possible size of a subset of $\{1,\dots,N\}$ that contains no non-trivial $k$-term arithmetic progression. The set $A\cap\{1,\dots,N\}$ is a subset of $\{1,\dots,N\}$ of size strictly larger than this maximum, so it cannot be $k$-AP-free. Therefore it contains a non-trivial $k$-term arithmetic progression. \hfill$\square$

\medskip
\noindent\textbf{FAST REALITY CHECK (finite $k=3$ toy model).}

For $N\le 30$, I computed (by exact backtracking) a 3-term-AP-free subset $B\subseteq\{1,\dots,N\}$ maximizing the reciprocal weight $\sum_{n\in B}1/n$. The exact optimum for $N=30$ was
\[
\max_{\substack{B\subseteq[30]\ \text{no 3-AP}}}\ \sum_{n\in B}\frac{1}{n}
\;=\;2.370142337534\dots
\]
achieved by
\[
B=\{1,2,4,5,10,11,13,23,26,27,30\},
\]
and the script verified that $B$ contains no 3-term arithmetic progression.

\medskip
\noindent\textbf{5) VERIFICATION.}

\begin{itemize}
\item Lemma 1: each $n\in(2^j,2^{j+1}]$ satisfies $1/n\le 1/2^j$; summing blockwise is valid because the blocks partition $A\cap[2,\infty)$.
\item Lemma 2: the only analytic input is the convergence of $\sum 1/(j(\log j)^2)$, which follows from the standard integral test.
\item Lemma 3: this is an immediate consequence of the definition of $r_k(N)$; no hidden monotonicity is assumed.
\item Computation: the maximizing set for $N=30$ was checked directly for absence of 3-term APs.
\end{itemize}

\medskip
\noindent\textbf{6) FINAL.}

\textbf{UNRESOLVED}

(i) \emph{Strongest proved partial result here.} If $\sum_{n\in A}1/n=\infty$, then $A$ cannot satisfy the uniform upper bound
\[|A\cap[1,N]|\le \frac{C\,N}{(\log N)(\log\log N)^2}\quad\text{for all large }N\]
for any fixed $C>0$; otherwise the reciprocal series would converge (Lemma 2). Moreover, for any fixed $k$, whenever $|A\cap[1,N]|>r_k(N)$ holds, $A$ contains a $k$-term AP (Lemma 3).

(ii) \emph{First gap (crisp).} To conclude ``arbitrarily long'' APs from Lemmas 2--3 one needs sufficiently strong upper bounds on $r_k(N)$ for all $k$ (e.g. of order $N/((\log N)(\log\log N)^2)$), or a different mechanism that turns harmonic divergence directly into long progressions.

(iii) \emph{Top 3 next moves (concrete).}
\begin{enumerate}
\item Prove a quantitative statement: find an explicit function $f(N)\to\infty$ such that divergence of $\sum_{n\in A}1/n$ forces $|A\cap[1,N]|\ge N/(\log N\,f(N))$ for infinitely many $N$. Then target bounds of the form $r_k(N)\ll N/(\log N\,f(N))$.
\item Extend the finite ``maximum reciprocal sum among 3-AP-free sets'' computation to larger $N$ (and to $k=4$ if feasible) to guess the correct growth rate; a bounded maximum would heuristically support the Erd\H{o}s conjecture for that $k$.
\item Explore whether existing density-increment frameworks for Szemer\'edi-type theorems can be adapted to the reciprocal weight $\sum_{n\in A\cap[1,N]}1/n$ (rather than cardinality), by proving lemmas that replace uniform density by weighted density.
\end{enumerate}

(iv) \emph{Minimal counterexample structure.} A counterexample would be a set $A$ with $\sum_{n\in A}1/n=\infty$ but with some fixed $k$ such that $A\cap[1,N]$ is $k$-AP-free for every $N$. Necessarily $|A\cap[1,N]|\le r_k(N)$ for all $N$, and Lemma 2 shows that this counterexample would have to violate bounds of the shape $r_k(N)\ll N/((\log N)(\log\log N)^2)$ infinitely often.

\bigskip

