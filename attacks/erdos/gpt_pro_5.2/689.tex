
\subsection*{Erd\H{o}s Problem 689}

\paragraph{1) FORMAL RESTATEMENT.}
Let $n$ be a positive integer and let $\mathcal P(n):=\{p:\ p\text{ prime},\ 2\le p\le n\}$.
Choose residue classes $a_p\pmod p$ for every $p\in\mathcal P(n)$.
For an integer $m$, let
\[
c(m):=\#\{\,p\in\mathcal P(n):\ m\equiv a_p\pmod p\,\}
\]
be the number of chosen congruences satisfied by $m$.
The problem asks: for all sufficiently large $n$, does there exist a choice of $\{a_p\}_{p\le n}$ such that
\[
c(m)\ge 2 \quad\text{for every integer }m\in[1,n]\,?
\]
More generally, for fixed $r\ge 2$, can one choose $\{a_p\}$ so that $c(m)\ge r$ for all $m\in[1,n]$ when $n$ is large enough depending on $r$?

\paragraph{2) QUICK LITERATURE/CONTEXT CHECK.}
No external results are used or claimed here beyond what is in the problem statement.
The statement notes a related question appears in Green's open problems list (with $2$ replaced by $10$).

\paragraph{3) ATTACK PLAN.}
\begin{itemize}
\item Reformulate via a shift $t$ modulo the primorial $P(n):=\prod_{p\le n}p$ (Lemma~689.1):
the condition $c(m)\ge 2$ becomes: each $t+m$ is divisible by at least two distinct primes $\le n$.
\item Derive necessary conditions from double-counting incidences (Lemma~689.2).
\item Use small-$n$ computation to sanity check feasibility.
\end{itemize}

\paragraph{4) WORK.}
\medskip
\noindent\textbf{Lemma 689.1 (shift reformulation of the ``at least two congruences'' condition).}
Let $P(n):=\prod_{p\le n}p$.
For any choice of residue classes $\{a_p\pmod p\}_{p\le n}$ there exists a unique residue class $t\pmod{P(n)}$ such that
$t\equiv -a_p\pmod p$ for all primes $p\le n$.
For this $t$ and any integer $m$,
\[
m\equiv a_p\pmod p\quad\Longleftrightarrow\quad p\mid (t+m).
\]
Therefore, for $m\in[1,n]$,
\[
c(m)\ge 2\quad\Longleftrightarrow\quad \#\{\,p\le n:\ p\mid (t+m)\,\}\ge 2,
\]
i.e.\ $t+m$ is divisible by at least two distinct primes $\le n$.

\emph{Proof.}
As in Lemma~687.1, CRT gives unique $t\pmod{P(n)}$ with $t\equiv -a_p\pmod p$.
Then $m\equiv a_p\pmod p$ is equivalent to $m+t\equiv 0\pmod p$, i.e.\ $p\mid (m+t)$.
Counting how many primes $p\le n$ satisfy this is exactly $c(m)$.
Requiring $c(m)\ge 2$ is therefore equivalent to requiring at least two distinct primes $p\le n$ divide $m+t$.
\hfill$\square$

\medskip
\noindent\textbf{Lemma 689.2 (double-counting necessary condition).}
If there exist residue classes $\{a_p\pmod p\}_{p\le n}$ such that every $m\in[1,n]$ satisfies at least two congruences, then
\[
\sum_{\substack{\text{prime }p\\ p\le n}}\left\lceil\frac{n}{p}\right\rceil \ge 2n.
\]

\emph{Proof.}
For each prime $p\le n$, let
\[
C_p:=\#\{\,m\in[1,n]:\ m\equiv a_p\pmod p\,\}.
\]
Then $C_p\le \lceil n/p\rceil$ for each $p$.
If each $m\in[1,n]$ lies in at least two of these residue classes, then counting incidences gives
\[
\sum_{p\le n} C_p \ge 2n.
\]
Combining with $C_p\le \lceil n/p\rceil$ yields the claimed necessary inequality. \hfill$\square$

\medskip
\noindent\textbf{FAST REALITY CHECK (local computation for small $n$).}
Using the shift formulation (Lemma~689.1), for small $n$ I computed the maximal length of a run of consecutive residues modulo $P(n)$ for which
each integer has at least two distinct prime divisors from $\{p\le n\}$.
For $n\le 20$ the maximum run length is far smaller than $n$:
\begin{verbatim}
n=11: P=2310     maxrun(with >=2 primes)<=3   example shift t=19
n=13: P=30030    maxrun(with >=2 primes)<=4   example shift t=2714
n=17: P=510510   maxrun(with >=2 primes)<=5   example shift t=7577
n=19: P=9699690  maxrun(with >=2 primes)<=6   example shift t=58119
\end{verbatim}
In particular, for all $n\le 20$ tested, there is no choice of residue classes achieving $c(m)\ge 2$ for every $m\in[1,n]$.

\paragraph{5) VERIFICATION.}
\begin{itemize}
\item Lemma~689.1 is an exact CRT equivalence.
\item Lemma~689.2 is necessary but not sufficient; overlaps can only reduce coverage, so it cannot prove existence.
\item The computations test feasibility only for small $n$ and cannot settle the large-$n$ question.
\end{itemize}

\paragraph{FINAL.}
\noindent\textbf{UNRESOLVED}

\smallskip
\noindent(i) \textbf{Strongest proved partial result.}
The problem is equivalent to finding (for each large $n$) a shift $t$ such that each of $t+1,\dots,t+n$ is divisible by at least two distinct primes $\le n$
(Lemma~689.1). Any such configuration must satisfy the incidence-counting constraint
$\sum_{p\le n}\lceil n/p\rceil\ge 2n$ (Lemma~689.2).

\smallskip
\noindent(ii) \textbf{First gap (crisp statement).}
Construct, for all sufficiently large $n$, a shift $t$ such that every integer in $[t+1,t+n]$ has at least two distinct prime divisors $\le n$,
or else prove that this is impossible for infinitely many $n$.

\smallskip
\noindent(iii) \textbf{Top 3 next moves.}
\begin{enumerate}
\item Try an explicit CRT construction assigning to each position $1\le j\le n$ a pair of (possibly repeating) primes $p_j,q_j\le n$
and solve $t\equiv -j\pmod{p_j}$ and $t\equiv -j\pmod{q_j}$ consistently, managing collisions when primes repeat.
\item In probabilistic terms, estimate the density of integers divisible by at least two primes $\le n$ in residue classes modulo $P(n)$,
and attempt to show there exists a length-$n$ interval consisting entirely of such integers.
\item Compute the maximal run length for larger $n$ as far as tractable (larger primorials quickly become huge) to detect growth trends.
\end{enumerate}

\smallskip
\noindent(iv) \textbf{What a minimal counterexample would likely look like.}
If the statement is false, one would expect infinitely many $n$ such that \emph{every} block of $n$ consecutive residues modulo $P(n)$ contains
some integer having at most one prime divisor $\le n$ (i.e.\ either coprime to $P(n)$ or divisible by exactly one prime $\le n$). Such an $n$ would
force a strong obstruction to packing length-$n$ intervals with ``$\ge 2$ small prime factors'' integers.

\subsection*{Erd\H{o}s Problem 689}

\paragraph{1) FORMAL RESTATEMENT.}

\textbf{Conventions.}
\begin{itemize}
\item $\mathbb N:=\{1,2,3,\dots\}$.
\item ``Prime'' means a rational prime $p\ge 2$.
\item For integers $x,y$ and $p\in\mathbb N$, $x\equiv y\pmod p$ means $p\mid(x-y)$.
\item A ``residue class $a_p \pmod p$'' means choosing an integer $a_p$ and considering its class mod $p$; equivalently choose $a_p\in\{0,1,\dots,p-1\}$.
\end{itemize}

\textbf{Data.} Fix $n\in\mathbb N$. Let
\[
\mathcal P(n):=\{p:\ p\text{ prime and }2\le p\le n\}.
\]
Choose residues $(a_p)_{p\in\mathcal P(n)}$ with each $a_p\in\mathbb Z/p\mathbb Z$.
For each integer $m\in\{1,2,\dots,n\}$, define the ``coverage count''
\[
c(m):=\#\{p\in\mathcal P(n):\ m\equiv a_p\pmod p\}.
\]

\textbf{Asymptotic version (standard ``sufficiently large'' reading).}
For a fixed integer $r\ge 2$, define the statement $\mathbf{E}(r)$:

\medskip
\noindent $\mathbf{E}(r)$: $\exists N_r\in\mathbb N$ such that $\forall n\in\mathbb N$ with $n\ge N_r$, $\exists$ a choice of residues $(a_p)_{p\in\mathcal P(n)}$ with
\[
\forall m\in\{1,2,\dots,n\},\qquad c(m)\ge r.
\]
\medskip

Erd\H{o}s Problem \#689 asks about $\mathbf{E}(2)$. It also explicitly asks the analogous question for general fixed $r$.

\textbf{Literal-source ambiguity.}
The original wording (in Erd\H{o}s' source) does not explicitly fix the quantifier structure in $n$. In particular, it does not clearly distinguish between:
\begin{itemize}
\item[(A)] $\forall$ sufficiently large $n$, $\exists$ residues \dots\ (the $\mathbf{E}(r)$ form),
\item[(B)] $\exists$ arbitrarily large $n$ such that $\exists$ residues \dots,
\item[(C)] $\exists n$ such that $\exists$ residues \dots.
\end{itemize}
Here I treat (A) as the minimal corrected statement consistent with modern conventions and common formalizations, while also noting counterexamples to the over-strong reading ``for all $n$''.

\textbf{Stress points / edge cases.}
\begin{itemize}
\item If $|\mathcal P(n)|<r$ then $c(m)\le |\mathcal P(n)|<r$ for all $m$, so the property is impossible (e.g.\ $n=2$ for $r=2$).
\item The integer $m=1$ has no prime divisors, so any construction must cover $1$ using \emph{nonzero} residue classes (e.g.\ primes $p$ with $a_p\equiv 1\pmod p$, etc.).
\item Large primes $p$ near $n$ have residue classes intersecting $[1,n]$ in at most one element; small primes cover many elements. Any construction must balance these regimes.
\end{itemize}

\paragraph{2) QUICK LITERATURE/CONTEXT CHECK.}
\begin{itemize}
\item \textbf{Original source (Erd\H{o}s 1979).} Erd\H{o}s asks: ``Are there residues $c_p$ for every prime $p$ with $2\le p\le n$ so that every positive integer $x\le n$ satisfies at least $2$ (or at least $r$) of the congruences $x\equiv c_p\pmod p$?'' The wording does not pin down the quantifier structure in $n$ (existence for some $n$ vs all large $n$), hence the ambiguity above.
\item \textbf{Current status / formalization on erdosproblems.com.} The problem is listed as \textbf{OPEN} and phrased as ``Let $n$ be sufficiently large\dots'' and also notes the $r$-fold generalization.
\item \textbf{Community discussion / heuristic approach.} Forum discussions contain heuristic numerology and possible strategy sketches (e.g.\ Tao; Sawhney), emphasizing links to large gaps between primes technology and (potentially) linear equations in primes methods to control residue-class hitting of prime/semiprime survivors. These are not complete proofs.
\end{itemize}

\paragraph{3) ATTACK PLAN.}

\textbf{Problem type.} Sieve/covering congruences / probabilistic combinatorics with strong number-theoretic distribution input. Closely aligned with constructions for long strings of composites (large prime gaps) but with a \emph{double-hit} requirement.

\textbf{Likely tools (landscape).}
\begin{enumerate}
\item \textbf{CRT/primorial reduction:} translate residue choices to a single shift modulo $P(n)=\prod_{p\le n}p$.
\item \textbf{Double counting / incidence bounds:} necessary conditions like $\sum_{p\le n}\lceil n/p\rceil\ge rn$.
\item \textbf{Sieve theory:} classify numbers in $[1,n]$ by roughness after sieving with small primes; estimate survivors.
\item \textbf{Probabilistic method:} random choice of residue classes for a range of primes; estimate coverage.
\item \textbf{Hypergraph covering / R\"odl nibble:} interpret residue classes as hyperedges; seek near-disjoint efficient covering of survivors.
\item \textbf{Large gaps between primes technology:} constructions of long prime-free intervals via congruences; here needs ``two hits''.
\item \textbf{Linear equations in primes theory:} to justify that certain structured progressions contain needed prime patterns with correct pseudorandomness.
\item \textbf{Local obstructions:} for $r\ge 3$, potential sieve-based lower bounds on unavoidable survivors that exceed remaining covering capacity.
\end{enumerate}

\textbf{Dual-track strategies.}
\begin{itemize}
\item \textbf{(A) Proof track (best guess for $r=2$).}
Reduce to a shift problem; implement a two-stage sieve/cover: (i) give almost everyone one hit; (ii) use remaining primes to give survivors a second hit, requiring strong distribution input.
\item \textbf{(B) Disproof track.}
Try to show for infinitely many $n$ any choice of residues leaves some $m\le n$ with $c(m)\le 1$. For $r\ge 3$, attempt to formalize a sieve-capacity obstruction.
\end{itemize}

\paragraph{4) WORK.}

\subsubsection*{PHASE 0 --- HYGIENE}

\textbf{Lemma 4.1 (CRT shift equivalence).}
Let $n\in\mathbb N$ and let $(a_p)_{p\in\mathcal P(n)}$ be residue classes $a_p\in\mathbb Z/p\mathbb Z$. Define the primorial
\[
P(n):=\prod_{p\in\mathcal P(n)} p.
\]
Then there exists a unique residue class $t\in\mathbb Z/P(n)\mathbb Z$ such that
\[
t\equiv -a_p \pmod p\quad\text{for all }p\in\mathcal P(n).
\]
Moreover, for every integer $m$,
\[
m\equiv a_p\pmod p \quad\Longleftrightarrow\quad p\mid(t+m).
\]
Consequently for $m\in\{1,\dots,n\}$,
\[
c(m)=\#\{p\in\mathcal P(n): p\mid(t+m)\},
\]
and the condition $c(m)\ge r$ for all $m\in\{1,\dots,n\}$ is equivalent to:
\[
\forall m\in\{1,\dots,n\},\quad t+m\text{ is divisible by at least $r$ distinct primes }\le n.
\]

\emph{Proof.}
The moduli $p\in\mathcal P(n)$ are pairwise coprime, so by CRT the system
$t\equiv -a_p\pmod p$ for all $p\in\mathcal P(n)$ has a unique solution modulo $P(n)$.
Fix such $t$.
For a given prime $p\le n$,
\[
m\equiv a_p\pmod p
\iff m-a_p\equiv 0\pmod p
\iff m+t\equiv a_p+t\equiv 0\pmod p
\iff p\mid(t+m),
\]
using $t\equiv -a_p\pmod p$.
Counting such primes gives the claimed formula for $c(m)$, and the equivalence for $c(m)\ge r$ follows.
\hfill$\square$

\subsubsection*{PHASE 1 --- FAST REALITY CHECK}

\textbf{Tiny $n$ by hand (disproof of the ``for all $n$'' reading).}
If one (incorrectly) interprets the claim as ``for every $n$ there exists a choice\dots'', it is false:
\begin{itemize}
\item $n=2$: $\mathcal P(2)=\{2\}$, hence $c(m)\le 1$ for all $m$, so $c(m)\ge 2$ is impossible.
\item $n=3$: $\mathcal P(3)=\{2,3\}$. If $c(m)\ge 2$ for $m=1,2,3$, then each $m$ satisfies both congruences. Hence $1,2,3$ would all lie in the same residue class modulo $6$, impossible.
\end{itemize}

\textbf{Computation for small $n\le 20$.}
Using Lemma~4.1, the property depends only on $t\bmod P(n)$.
A brute-force scan for $n\le 20$ shows the maximum cyclic run length (mod $P(n)$) of consecutive residues $x$ such that $x$ has at least two distinct prime divisors from $\mathcal P(n)$ is:
\[
\begin{array}{c|c}
n & \text{max run length}\\
\hline
2 & 0\\
3,4,5,6 & 1\\
7,8,9,10 & 2\\
11,12 & 3\\
13,14,15,16 & 4\\
17,18 & 5\\
19,20 & 6
\end{array}
\]
In particular, for every $n\le 20$ tested, the max run length is $<n$, so no solution exists for $n\le 20$ in the $r=2$ problem.

\subsubsection*{PHASE 2 --- LANDSCAPE: Necessary conditions and easy partial results}

\textbf{Lemma 4.2 (incidence bound).}
Fix $n\in\mathbb N$ and $r\in\mathbb N$.
Suppose there exist residues $(a_p)_{p\in\mathcal P(n)}$ such that $c(m)\ge r$ for all $m\in\{1,\dots,n\}$.
Then
\[
\sum_{p\in\mathcal P(n)} \left\lceil \frac{n}{p}\right\rceil \ \ge\ r n.
\]

\emph{Proof.}
For each prime $p\le n$, let
\[
C_p:=\#\{m\in\{1,\dots,n\}:\ m\equiv a_p\pmod p\}.
\]
Then $C_p\le \lceil n/p\rceil$.
Counting incidences,
\[
\sum_{m=1}^n c(m)
=\sum_{m=1}^n \#\{p\le n:\ m\equiv a_p\pmod p\}
=\sum_{p\le n} \#\{m\le n:\ m\equiv a_p\pmod p\}
=\sum_{p\le n} C_p.
\]
If $c(m)\ge r$ for each $m$, then $\sum_{m=1}^n c(m)\ge rn$, so
\[
rn \le \sum_{p\le n} C_p \le \sum_{p\le n}\left\lceil\frac{n}{p}\right\rceil.
\]
\hfill$\square$

\textbf{Corollary 4.3 (trivial upper bound on uniform coverage).}
For any choice of residues $(a_p)$,
\[
\min_{1\le m\le n} c(m)
\le \frac{1}{n}\sum_{m=1}^n c(m)
= \frac{1}{n}\sum_{p\le n} C_p
\le \sum_{p\le n}\left(\frac{1}{p}+\frac{1}{n}\right)
= \sum_{p\le n}\frac{1}{p} + \frac{\pi(n)}{n}.
\]
In particular, $\min_{m\le n} c(m)$ cannot exceed $(1+o(1))\log\log n$ as $n\to\infty$ (using known asymptotics for $\sum_{p\le n}1/p$ and $\pi(n)$).

\textbf{Lemma 4.4 (all-zero choice).}
Fix $n\ge 2$. Choose $a_p\equiv 0\pmod p$ for every prime $p\le n$.
Then for each integer $m$ with $2\le m\le n$,
\[
c(m)=\omega(m),
\]
where $\omega(m)$ is the number of \emph{distinct} prime factors of $m$.
In particular,
\[
c(m)\ge 2 \quad\text{for all }m\in\{2,\dots,n\}\text{ that are not prime powers.}
\]
Moreover $c(1)=0$.

\emph{Proof.}
For $m\le n$, the condition $m\equiv 0\pmod p$ is equivalent to $p\mid m$.
Because $m\le n$, every prime divisor of $m$ lies in $\mathcal P(n)$.
Thus $c(m)$ counts exactly the distinct prime divisors of $m$, i.e.\ $\omega(m)$.
Also $\omega(m)=1$ iff $m$ is a power of a single prime, and $\omega(1)=0$.
\hfill$\square$

\subsubsection*{PHASE 3 --- DUAL-TRACK SOLVE (attempts and first crisp gaps)}

\textbf{Proof-track outline (why it plausibly reduces to deep input).}
By Lemma~4.1, for $r=2$ one seeks $t$ such that each $t+m$ ($1\le m\le n$) has at least two distinct prime divisors $\le n$.
A natural strategy is:
\begin{enumerate}
\item \emph{Stage I:} choose many residues to give almost every $m$ at least one hit (a first small prime divisor).
\item \emph{Stage II:} use remaining primes to give the sparse set of one-hit survivors a second hit.
\end{enumerate}
A fully rigorous implementation appears to require strong distribution input (e.g.\ pseudorandomness of primes/semiprimes in many residue classes simultaneously) to ensure enough survivors are hit efficiently.

\textbf{Disproof-track attempt (obstruction for $r\ge 3$).}
Heuristic sieve-capacity arguments suggest $r\ge 3$ may fail, because after a certain sieving stage too many semiprimes may survive relative to the ability of remaining primes (one congruence class each) to supply the additional hits. Turning this into a rigorous unconditional disproof would require strong \emph{worst-case} upper bounds on how many survivors a residue class can hit, summed over the remaining primes.

\paragraph{5) VERIFICATION.}

\begin{itemize}
\item Lemma~4.1 is a direct CRT equivalence; all implications are explicit.
\item Lemma~4.2 is pure double counting and holds for all $n,r$.
\item Corollary~4.3 uses only $\lceil n/p\rceil\le n/p+1$ and $\min\le \text{average}$.
\item Lemma~4.4 is correct because $m\le n$ forces all prime divisors of $m$ to lie in $\mathcal P(n)$.
\item The computation certifies failure for $n\le 20$ but does not settle asymptotics.
\end{itemize}

\paragraph{6) FINAL.}

\noindent\textbf{UNRESOLVED}

\smallskip
\noindent\textbf{(i) Strongest fully proved partial results obtained here.}
\begin{enumerate}
\item Exact equivalence (Lemma~4.1): the problem is equivalent to finding a shift $t$ such that each of $t+1,\dots,t+n$ has at least $r$ distinct prime divisors $\le n$.
\item Necessary incidence bound (Lemma~4.2): any solution must satisfy $\sum_{p\le n}\lceil n/p\rceil\ge rn$.
\item Near-solution (Lemma~4.4): choosing $a_p\equiv 0$ gives $c(m)=\omega(m)$ for $m\ge 2$, so all $m\in[2,n]$ except prime powers already satisfy $c(m)\ge 2$.
\item Computation: for every $n\le 20$, no choice of residues achieves $c(m)\ge 2$ for all $m\in[1,n]$.
\end{enumerate}

\smallskip
\noindent\textbf{(ii) Exact first gap (single crisp statement).}
\begin{quote}
Prove that for all sufficiently large $n$ there exists a choice of residues $(a_p)_{p\le n}$ such that every $m\in[1,n]$ lies in at least two of the selected classes; equivalently (Lemma~4.1), prove that for all sufficiently large $n$ there exists $t$ such that every integer in $[t+1,t+n]$ has at least two distinct prime factors $\le n$.
\end{quote}

\smallskip
\noindent\textbf{(iii) Top 3 next moves (specific, technical).}
\begin{enumerate}
\item Formulate a precise hypergraph covering lemma for survivors after a first sieving stage: vertices = primes in a range; hyperedges = residue classes intersected with the survivor set; then apply a nibble/covering argument, contingent on pseudorandomness estimates.
\item Prove a robust pseudorandomness-of-survivors statement: after removing multiples of small primes, show remaining primes/semiprimes are equidistributed enough across many moduli to support probabilistic covering.
\item For $r\ge 3$, try to formalize the capacity obstruction into an unconditional inequality: a worst-case upper bound on how many semiprimes (with specified roughness) a single residue class mod $p$ can hit, summed over available $p$.
\end{enumerate}

\smallskip
\noindent\textbf{(iv) What a minimal counterexample would likely look like.}
If $\mathbf{E}(2)$ is false, one expects infinitely many $n$ such that every length-$n$ block modulo $P(n)$ contains some residue with at most one prime divisor $\le n$; equivalently, no shift $t$ can make all $t+1,\dots,t+n$ have two small prime factors. Such an obstruction would manifest as an unavoidable occurrence, in every block, of either:
\begin{itemize}
\item a residue coprime to $P(n)$ (no small prime factors), or
\item a residue divisible by exactly one prime $\le n$ (a ``one-hit'' number relative to $\mathcal P(n)$),
\end{itemize}
with this inevitability persisting despite the freedom to choose $t$.

\paragraph{7) COMPLETION ESTIMATE (MANDATORY).}
\noindent COMPLETION: 40\%
