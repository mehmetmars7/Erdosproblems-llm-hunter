% Erdos Problem #689
% Solution attempt generated from 685-689.tex

\subsection*{Erd\H{o}s Problem 689}

\paragraph{1) FORMAL RESTATEMENT.}
Let $n$ be a positive integer and let $\mathcal P(n):=\{p:\ p\text{ prime},\ 2\le p\le n\}$.
Choose residue classes $a_p\pmod p$ for every $p\in\mathcal P(n)$.
For an integer $m$, let
\[
c(m):=\#\{\,p\in\mathcal P(n):\ m\equiv a_p\pmod p\,\}
\]
be the number of chosen congruences satisfied by $m$.
The problem asks: for all sufficiently large $n$, does there exist a choice of $\{a_p\}_{p\le n}$ such that
\[
c(m)\ge 2 \quad\text{for every integer }m\in[1,n]\,?
\]
More generally, for fixed $r\ge 2$, can one choose $\{a_p\}$ so that $c(m)\ge r$ for all $m\in[1,n]$ when $n$ is large enough depending on $r$?

\paragraph{2) QUICK LITERATURE/CONTEXT CHECK.}
No external results are used or claimed here beyond what is in the problem statement.
The statement notes a related question appears in Green's open problems list (with $2$ replaced by $10$).

\paragraph{3) ATTACK PLAN.}
\begin{itemize}
\item Reformulate via a shift $t$ modulo the primorial $P(n):=\prod_{p\le n}p$ (Lemma~689.1):
the condition $c(m)\ge 2$ becomes: each $t+m$ is divisible by at least two distinct primes $\le n$.
\item Derive necessary conditions from double-counting incidences (Lemma~689.2).
\item Use small-$n$ computation to sanity check feasibility.
\end{itemize}

\paragraph{4) WORK.}
\medskip
\noindent\textbf{Lemma 689.1 (shift reformulation of the ``at least two congruences'' condition).}
Let $P(n):=\prod_{p\le n}p$.
For any choice of residue classes $\{a_p\pmod p\}_{p\le n}$ there exists a unique residue class $t\pmod{P(n)}$ such that
$t\equiv -a_p\pmod p$ for all primes $p\le n$.
For this $t$ and any integer $m$,
\[
m\equiv a_p\pmod p\quad\Longleftrightarrow\quad p\mid (t+m).
\]
Therefore, for $m\in[1,n]$,
\[
c(m)\ge 2\quad\Longleftrightarrow\quad \#\{\,p\le n:\ p\mid (t+m)\,\}\ge 2,
\]
i.e.\ $t+m$ is divisible by at least two distinct primes $\le n$.

\emph{Proof.}
As in Lemma~687.1, CRT gives unique $t\pmod{P(n)}$ with $t\equiv -a_p\pmod p$.
Then $m\equiv a_p\pmod p$ is equivalent to $m+t\equiv 0\pmod p$, i.e.\ $p\mid (m+t)$.
Counting how many primes $p\le n$ satisfy this is exactly $c(m)$.
Requiring $c(m)\ge 2$ is therefore equivalent to requiring at least two distinct primes $p\le n$ divide $m+t$.
\hfill$\square$

\medskip
\noindent\textbf{Lemma 689.2 (double-counting necessary condition).}
If there exist residue classes $\{a_p\pmod p\}_{p\le n}$ such that every $m\in[1,n]$ satisfies at least two congruences, then
\[
\sum_{\substack{\text{prime }p\\ p\le n}}\left\lceil\frac{n}{p}\right\rceil \ge 2n.
\]

\emph{Proof.}
For each prime $p\le n$, let
\[
C_p:=\#\{\,m\in[1,n]:\ m\equiv a_p\pmod p\,\}.
\]
Then $C_p\le \lceil n/p\rceil$ for each $p$.
If each $m\in[1,n]$ lies in at least two of these residue classes, then counting incidences gives
\[
\sum_{p\le n} C_p \ge 2n.
\]
Combining with $C_p\le \lceil n/p\rceil$ yields the claimed necessary inequality. \hfill$\square$

\medskip
\noindent\textbf{FAST REALITY CHECK (local computation for small $n$).}
Using the shift formulation (Lemma~689.1), for small $n$ I computed the maximal length of a run of consecutive residues modulo $P(n)$ for which
each integer has at least two distinct prime divisors from $\{p\le n\}$.
For $n\le 20$ the maximum run length is far smaller than $n$:
\begin{verbatim}
n=11: P=2310     maxrun(with >=2 primes)<=3   example shift t=19
n=13: P=30030    maxrun(with >=2 primes)<=4   example shift t=2714
n=17: P=510510   maxrun(with >=2 primes)<=5   example shift t=7577
n=19: P=9699690  maxrun(with >=2 primes)<=6   example shift t=58119
\end{verbatim}
In particular, for all $n\le 20$ tested, there is no choice of residue classes achieving $c(m)\ge 2$ for every $m\in[1,n]$.

\paragraph{5) VERIFICATION.}
\begin{itemize}
\item Lemma~689.1 is an exact CRT equivalence.
\item Lemma~689.2 is necessary but not sufficient; overlaps can only reduce coverage, so it cannot prove existence.
\item The computations test feasibility only for small $n$ and cannot settle the large-$n$ question.
\end{itemize}

\paragraph{FINAL.}
\noindent\textbf{UNRESOLVED}

\smallskip
\noindent(i) \textbf{Strongest proved partial result.}
The problem is equivalent to finding (for each large $n$) a shift $t$ such that each of $t+1,\dots,t+n$ is divisible by at least two distinct primes $\le n$
(Lemma~689.1). Any such configuration must satisfy the incidence-counting constraint
$\sum_{p\le n}\lceil n/p\rceil\ge 2n$ (Lemma~689.2).

\smallskip
\noindent(ii) \textbf{First gap (crisp statement).}
Construct, for all sufficiently large $n$, a shift $t$ such that every integer in $[t+1,t+n]$ has at least two distinct prime divisors $\le n$,
or else prove that this is impossible for infinitely many $n$.

\smallskip
\noindent(iii) \textbf{Top 3 next moves.}
\begin{enumerate}
\item Try an explicit CRT construction assigning to each position $1\le j\le n$ a pair of (possibly repeating) primes $p_j,q_j\le n$
and solve $t\equiv -j\pmod{p_j}$ and $t\equiv -j\pmod{q_j}$ consistently, managing collisions when primes repeat.
\item In probabilistic terms, estimate the density of integers divisible by at least two primes $\le n$ in residue classes modulo $P(n)$,
and attempt to show there exists a length-$n$ interval consisting entirely of such integers.
\item Compute the maximal run length for larger $n$ as far as tractable (larger primorials quickly become huge) to detect growth trends.
\end{enumerate}

\smallskip
\noindent(iv) \textbf{What a minimal counterexample would likely look like.}
If the statement is false, one would expect infinitely many $n$ such that \emph{every} block of $n$ consecutive residues modulo $P(n)$ contains
some integer having at most one prime divisor $\le n$ (i.e.\ either coprime to $P(n)$ or divisible by exactly one prime $\le n$). Such an $n$ would
force a strong obstruction to packing length-$n$ intervals with ``$\ge 2$ small prime factors'' integers.
