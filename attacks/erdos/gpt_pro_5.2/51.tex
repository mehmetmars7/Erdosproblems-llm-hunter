
\subsection*{Erd\H{o}s Problem \#51}
\noindent\emph{Problem statement (from the problem file).}
Is there an infinite set $A\subset\mathbb{N}$ such that for every $a\in A$ there is an integer $n$ with $\varphi(n)=a$, and yet if $n_a$ is the smallest such integer then $n_a/a\to\infty$ as $a\to\infty$?

\paragraph{FORMAL RESTATEMENT.}
Does there exist an infinite set $\mathcal{A}\subset\mathbb{N}$ such that
\begin{itemize}
\item for every $a\in\mathcal{A}$, the equation $\varphi(n)=a$ has at least one solution $n\in\mathbb{N}$;
\item writing $n_a:=\min\{n\in\mathbb{N}:\varphi(n)=a\}$, one has $\lim_{\substack{a\to\infty\\ a\in\mathcal{A}}} \frac{n_a}{a}=\infty$.
\end{itemize}

\paragraph{QUICK LITERATURE/CONTEXT CHECK.}
The problem file mentions Carmichael's question about uniqueness of solutions to $\varphi(n)=t$ and that Erd\H{o}s proved: if there exists such a $t$ with a unique preimage then there are infinitely many.
I do not use any further results.

\paragraph{ATTACK PLAN.}
\emph{Construction track:} Search for a family of totients $a$ that force the smallest preimage to have many small prime factors, making $n_a/a$ large.
\emph{Disproof track:} Try to show that for every totient value $a$ there is always some preimage $n$ with $n\ll a(\log\log a)$ or similar, which would preclude divergence.
\emph{Reality check:} Compute $n_a$ for all $a$ realized by $\varphi(n)$ up to a cutoff and inspect ratios.

\paragraph{WORK.}
\subparagraph{FAST REALITY CHECK (numerics).}
Scanning all $n\le 300000$ and recording, for each totient value $a=\varphi(n)\le 20000$, the minimal preimage $n_a$, we found:
\begin{itemize}
\item among the $4486$ values $a\le 20000$ that occur as $\varphi(n)$ with some $n\le 300000$, the maximum observed ratio $n_a/a$ was about $2.035$ (attained at $a=5888$, $n_a=11985$).
\item for values of the form $a=p-1$ with $p$ prime, one has $n_a=p$ and $n_a/a=1+1/a$, quickly tending to $1$.
\end{itemize}
This finite search does not show large ratios, but it is far from the asymptotic regime.

\subparagraph{Lemma 1 (basic ratio identity).}
For every $n\ge 1$,
\[
\frac{n}{\varphi(n)}=\prod_{p\mid n}\frac{1}{1-1/p}.
\]

\emph{Proof.}
This is immediate from Lemma~1 of Problem~\#50: $\varphi(n)/n=\prod_{p\mid n}(1-1/p)$, hence invert. \qed

\subparagraph{Lemma 2 (prime-shifted totients have minimal preimage equal to the prime).}
If $p$ is prime and $a:=p-1$, then the smallest $n$ with $\varphi(n)=a$ is $n_a=p$.

\emph{Proof.}
We have $\varphi(p)=p-1=a$, so $n_a\le p$.
On the other hand, for every integer $n>1$, one has $\varphi(n)\le n-1$ with equality only when $n$ is prime.
Thus if $\varphi(n)=p-1$, then $n\ge p$.
Combining gives $n_a=p$. \qed

\subparagraph{Lemma 3 (a general lower bound).}
For any $a\ge 1$ that occurs as $a=\varphi(n)$ for some $n$, one has $n_a\ge a+1$.

\emph{Proof.}
If $a=1$ then $n_a=1$ and the claim holds.
For $a\ge 2$, any $n$ with $\varphi(n)=a$ satisfies $n>1$ and hence $\varphi(n)\le n-1$.
Thus $a\le n-1$, i.e. $n\ge a+1$.
Taking the minimum over solutions gives $n_a\ge a+1$. \qed

\paragraph{VERIFICATION.}
\begin{itemize}
\item Lemma~2 uses only the elementary inequality $\varphi(n)\le n-1$ for $n>1$.
\item Lemma~1 is the standard prime-factor identity; it shows how large ratios require many small prime factors.
\item The numerical scan is explicitly bounded and cannot resolve asymptotics.
\end{itemize}

\paragraph{FINAL.} \textbf{UNRESOLVED.}
\begin{enumerate}
\item[(i)] \emph{Strongest proved partial result.} The ratio $n/\varphi(n)$ is the multiplicative product over primes dividing $n$ (Lemma~1), and for $a=p-1$ one has $n_a=p$ so $n_a/a\to 1$ along this infinite family (Lemma~2). Any totient value $a$ satisfies $n_a\ge a+1$ (Lemma~3).
\item[(ii)] \emph{First gap.} Construct an explicit infinite set $\mathcal{A}$ of totient values for which the minimal preimage $n_a$ necessarily has $n_a/a\to\infty$, or prove that such a set cannot exist.
\item[(iii)] \emph{Top 3 next moves.}
(1) Seek a family of totient values $a$ whose preimages must include many small primes (forcing large $n/\varphi(n)$), and then prove that no smaller $n$ can realize the same $a$.
(2) Experiment computationally with larger cutoffs to see whether $\max_{a\le X} n_a/a$ grows, and if so for which structural $a$ (highly composite, smooth, etc.).
(3) Prove structural lemmas about the minimal preimage: e.g. constraints on its prime factorization or on the largest prime factor in terms of $a$.
\item[(iv)] \emph{Minimal counterexample structure.} If the answer is ``no'', one would expect a universal function $g(a)=o(\infty)$ such that for every totient value $a$ there exists some $n\le g(a)\,a$ with $\varphi(n)=a$; a counterexample to this would be a sequence of totients $a_k$ whose all preimages satisfy $n/a_k\to\infty$.
\end{enumerate}

