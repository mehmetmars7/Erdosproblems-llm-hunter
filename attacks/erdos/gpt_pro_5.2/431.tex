
Are there two infinite sets $A$ and $B$ such that $A+B$ agrees with the set of prime numbers up to finitely many exceptions? A problem of Ostmann, sometimes known as the 'inverse Goldbach problem'. The answer is surely no. The best result in this direction is due to Elsholtz and Harper \cite{ElHa15}, who showed that if $A,B$ are such sets then for all large $x$ we must have\[\frac{x^{1/2}}{\log x\log\log x} \ll \lvert A \cap [1,x]\rvert \ll x^{1/2}\log\log x\]and similarly for $B$. Elsholtz \cite{El01} has proved there are no sets $A,B,C$ (all of size at least $2$) such that $A+B+C$ agrees with the set of prime numbers up to finitely many exceptions. Granville \cite{Gr90} proved, conditional on the prime $k$-tuples conjecture, that there are infinite sets $B$ and $C$ such that\[\{ \tfrac{b+c}{2}: b\in B, c\in C\}\]is a subset of the primes. Tao and Ziegler \cite{TaZi23} gave an unconditional proof that there are infinite sets $B=\{b_1<\cdots\}$ and $C=\{c_1<\cdots\}$ such that\[\{ b_i+c_j : b_i\in B, c_j\in C, i<j\}\]is a subset of the primes. See also [429] and [432] . References [El01] Elsholtz, Christian, The inverse Goldbach problem . Mathematika (2001), 151-158. [ElHa15] Elsholtz, Christian and Harper, Adam J., Additive decompositions of sets with restricted prime factors . Trans. Amer. Math. Soc. (2015), 7403-7427. [Gr90] Granville, Andrew, A note on sums of primes . Canad. Math. Bull. (1990), 452--454. [TaZi23] Tao, Terence and Ziegler, Tamar, Infinite partial sumsets in the primes . J. Anal. Math. (2023), 375--389.


\medskip
\noindent\textbf{FORMAL RESTATEMENT}

Let $\mathbb N:=\{1,2,3,\dots\}$. For sets $A,B\subseteq\mathbb N$ define the (binary) sumset
\[
A+B := \{a+b : a\in A,\ b\in B\}\subseteq \mathbb N.
\]
Let $\mathcal P\subseteq\mathbb N$ be the set of prime numbers. The phrase ``$A+B$ agrees with the primes up to finitely many exceptions'' means
\[
\lvert (A+B)\,\triangle\,\mathcal P\rvert <\infty
\quad\text{equivalently,}\quad
\exists N_0\ \forall n\ge N_0:\ \bigl(n\in A+B \iff n\in\mathcal P\bigr).
\]
Question: do there exist \emph{infinite} $A,B\subseteq\mathbb N$ with this property?

Stress points:
(i) parity (all primes except $2$ are odd);
(ii) congruence obstructions mod small primes $q$ (primes contain at most one multiple of $q$);
(iii) ``infinite'' means we cannot use singleton tricks.

\medskip
\noindent\textbf{QUICK LITERATURE/CONTEXT CHECK}

I am constrained to not use any external literature beyond what is explicitly stated in the problem text. The problem text reports:
- Elsholtz--Harper give size constraints $\lvert A\cap[1,x]\rvert,\lvert B\cap[1,x]\rvert\asymp x^{1/2}$ up to polylog factors if such $A,B$ exist.
- Elsholtz proves there are no $A,B,C$ of size at least $2$ with $A+B+C$ agreeing with primes up to finitely many exceptions.
- Granville and Tao--Ziegler give constructions of infinite partial sumsets contained in the primes (different from full $A+B$).

\medskip
\noindent\textbf{ATTACK PLAN}

Proof track (show nonexistence):
1) Push modular obstructions: for each small prime $q$, large sums in $A+B$ cannot hit $0\pmod q$; force $A$ and $B$ into sparse residue patterns.
2) Try to combine many moduli to force $A,B$ into near-arithmetic progressions, then show $A+B$ cannot model primes.

Disproof/construction track (try to build $A,B$):
1) Attempt to build $A,B$ with opposite parity and with residue supports avoiding $0$ modulo many small primes, while still having enough sums to hit almost all primes.
(Heuristically this seems extremely hard; I did not find such a construction.)

I pursue the modular-obstruction direction to obtain necessary conditions (partial progress).

\medskip
\noindent\textbf{WORK}

\noindent\emph{FAST REALITY CHECK (finite truncations).}
I brute-forced a small finite analogue: search for finite $A\subseteq\{2,4,\dots,30\}$ and $B\subseteq\{1,3,\dots,29\}$ with $2\le |A|,|B|\le 5$ such that
\[(A+B)\cap[1,30] = \mathcal P\cap[1,30].\]
Result: no such pairs were found.
\begin{verbatim}
N=30, M=30, sizes 2..5, parity constrained: num_solutions 0
elapsed_seconds 24.225643157958984
\end{verbatim}
This does not prove anything about infinite sets, but it sanity-checks that the constraint is already tight in small ranges.

\medskip
\noindent\textbf{Lemma 431.1 (eventual parity separation).}
Assume $A,B\subseteq\mathbb N$ are infinite and there exists $N_0$ such that for all $n\ge N_0$,
\[n\in A+B \iff n\in\mathcal P.\]
Then there exist $\varepsilon_A,\varepsilon_B\in\{0,1\}$ with $\varepsilon_A+\varepsilon_B\equiv 1\pmod 2$ and finite subsets $F_A\subseteq A$, $F_B\subseteq B$ such that every $a\in A\setminus F_A$ satisfies $a\equiv \varepsilon_A\pmod 2$ and every $b\in B\setminus F_B$ satisfies $b\equiv \varepsilon_B\pmod 2$.

\noindent\emph{Proof.}
Suppose first that $A$ contains infinitely many even integers and $B$ contains infinitely many even integers. Choose an even $b_0\in B$ and an infinite increasing sequence of even $a_1<a_2<\cdots$ in $A$. Then all sums $a_i+b_0$ are even and strictly increasing, hence infinitely many of them exceed $N_0$. For such large $i$ we have $a_i+b_0\in A+B$, hence (by the agreement hypothesis) $a_i+b_0$ must be prime. But every even integer greater than $2$ is composite, a contradiction. Therefore, at least one of the sets $A,B$ contains only finitely many even integers.

Similarly, suppose $A$ contains infinitely many odd integers and $B$ contains infinitely many odd integers. Choose an odd $b_1\in B$ and an infinite increasing sequence of odd $a'_1<a'_2<\cdots$ in $A$. Then $a'_i+b_1$ are even, strictly increasing, and eventually exceed $N_0$, giving the same contradiction. Thus, at least one of $A,B$ contains only finitely many odd integers.

Because $A$ is infinite, it cannot have only finitely many evens \emph{and} only finitely many odds. Hence exactly one of these parity classes occurs infinitely often in $A$; call that parity $\varepsilon_A$. Likewise $B$ has an eventual parity $\varepsilon_B$. Finally, if $\varepsilon_A=\varepsilon_B$ then for all sufficiently large $a\in A$ and $b\in B$ we have $a+b$ even, so $A+B$ contains infinitely many even integers exceeding $N_0$, contradicting agreement with primes. Therefore $\varepsilon_A\ne\varepsilon_B$, i.e. $\varepsilon_A+\varepsilon_B\equiv 1\pmod 2$. Taking $F_A$ and $F_B$ to be the finite sets of exceptional parity elements completes the proof. \hfill$\Box$

\medskip
\noindent\textbf{Lemma 431.2 (residue-support obstruction modulo a prime).}
Let $q$ be a fixed prime. Let $A,B\subseteq\mathbb N$ be infinite. Define
\[
R_A(q):=\{r\in\mathbb Z/q\mathbb Z : \text{infinitely many }a\in A\text{ satisfy }a\equiv r\pmod q\},
\]
and define $R_B(q)$ analogously. If $A+B$ contains only finitely many multiples of $q$, then
\[
R_A(q)\cap\bigl(-R_B(q)\bigr)=\varnothing,
\quad\text{equivalently}\quad 0\notin R_A(q)+R_B(q)
\quad\text{and hence}\quad |R_A(q)|+|R_B(q)|\le q.
\]

\noindent\emph{Proof.}
Assume for contradiction that there exist residues $r\in R_A(q)$ and $s\in R_B(q)$ with $r+s\equiv 0\pmod q$. By definition of $R_A(q)$ and $R_B(q)$, there exist infinitely many $a\in A$ with $a\equiv r\pmod q$ and infinitely many $b\in B$ with $b\equiv s\pmod q$. Fix one such $b=b_*$.

Consider the infinite set $\{a+b_*: a\in A,\ a\equiv r\pmod q\}\subseteq A+B$. All these sums are distinct (because $a$ are distinct) and each satisfies
\[(a+b_*)\equiv r+s\equiv 0\pmod q,
\]
so they are multiples of $q$. Hence $A+B$ contains infinitely many multiples of $q$, contradicting the hypothesis that $A+B$ contains only finitely many multiples of $q$. Therefore $R_A(q)\cap(-R_B(q))=\varnothing$.

The inequality $|R_A(q)|+|R_B(q)|\le q$ follows because $-R_B(q)$ has the same size as $R_B(q)$ and both subsets lie in a set of size $q$ with empty intersection. \hfill$\Box$

\medskip
\noindent\emph{Application to Problem 431.}
If $A+B$ agrees with primes beyond some $N_0$, then for any fixed prime $q$ we have: all sufficiently large multiples of $q$ are composite and hence cannot lie in $A+B$; therefore $A+B$ contains only finitely many multiples of $q$, and Lemma 431.2 applies.

\medskip
\noindent\textbf{VERIFICATION}

- Lemma 431.1: The only prime of even parity is $2$, but it is a single exception. The argument only uses that there are infinitely many even sums if both parities persist, producing infinitely many composite even numbers $>2$; this is correct.
- Lemma 431.2: The proof uses a fixed $b_*$ and infinitely many distinct $a$ in a residue class, ensuring infinitely many distinct sums. No step assumes $A$ or $B$ has positive density.
- Edge case: if $q\ge N_0$, the set $A+B$ may contain the prime $q$ itself, but that is still only one multiple of $q$; ``finitely many'' is unaffected.

\medskip
\noindent\textbf{FINAL}

**UNRESOLVED**

(i) Strongest proved partial result: If $A+B$ agrees with primes beyond $N_0$, then (a) $A$ and $B$ are eventually of opposite parity (Lemma 431.1), and (b) for every prime $q$, the sets of residue classes occurring infinitely often in $A$ and $B$ satisfy $0\notin R_A(q)+R_B(q)$ (Lemma 431.2).

(ii) First gap (crisp): Prove that there do not exist infinite $A,B\subseteq\mathbb N$ with $\lvert (A+B)\triangle\mathcal P\rvert<\infty$.

(iii) Top 3 next moves:
1. Strengthen Lemma 431.2 to show that for many small primes $q$, one of $R_A(q),R_B(q)$ must be very small (ideally a singleton), and quantify how these constraints interact across many moduli.
2. Combine the modular restrictions with additive-combinatorial structure theorems (e.g. small-doubling type statements for $A,B$ implied by $A+B$ being ``thin'') to force rigid structure, then contradict the irregular distribution of primes.
3. Computation: search for finite sets $A,B$ with $|A|,|B|\ge 2$ such that $(A+B)\cap[1,N]$ matches primes for larger $N$ and examine forced residue patterns; attempt to extrapolate an obstruction.

(iv) What a minimal counterexample would likely look like: $A$ and $B$ would have opposite parity, and for each small prime $q$ the infinite residue supports $R_A(q)$ and $R_B(q)$ would avoid additive inverses. Moreover, according to the problem text, such a counterexample would have to satisfy $|A\cap[1,x]|,|B\cap[1,x]|$ on the order of $x^{1/2}$ up to polylog factors.


