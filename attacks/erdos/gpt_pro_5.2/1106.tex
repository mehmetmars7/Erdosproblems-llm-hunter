
\noindent\textbf{FORMAL RESTATEMENT.}
Let $p(n)$ denote the partition function of $n$ (number of integer partitions of $n$), with $p(0)=1$ and $p(n)\in\mathbb N$ for $n\ge 1$.
For $n\ge 1$ define
\[P(n):=\prod_{k=1}^n p(k),\qquad F(n):=\omega(P(n)),
\]
where $\omega(N)$ is the number of distinct prime divisors of a positive integer $N$.
Questions:
\begin{itemize}
\item Does $F(n)\to\infty$ as $n\to\infty$?
\item Is $F(n)>n$ for all sufficiently large $n$?
\end{itemize}

\medskip
\noindent\textbf{QUICK LITERATURE/CONTEXT CHECK (from the problem file only).}
The text reports:
(i) Schinzel observed (Oberwolfach problem book) that $F(n)\to\infty$ follows from the asymptotic for $p(n)$ and a theorem of Tijdeman; details in Erd\H{o}s--Ivi\'{c} (1990).
(ii) Schinzel--Wirsing proved $F(n)\gg \log n$.
(iii) Ono proved that every prime divides $p(n)$ for some $n\ge 1$ (indeed for a positive density of $n$ for each fixed prime).
I do not reprove any of these external statements.

\medskip
\noindent\textbf{ATTACK PLAN.}
\begin{itemize}
\item Prove basic structural facts about $F(n)$ (monotonicity) and give a clean conditional implication from (iii) to $F(n)\to\infty$.
\item Provide computational sanity checks for $F(n)$ for moderate $n$.
\end{itemize}

\medskip
\noindent\textbf{WORK.}

\medskip
\noindent\textbf{Lemma 1 (monotonicity).}
The function $F(n)$ is nondecreasing in $n$.

\noindent\emph{Proof.}
We have $P(n+1)=P(n)\,p(n+1)$.
Thus the set of prime divisors of $P(n+1)$ is the union of the set of prime divisors of $P(n)$ with those of $p(n+1)$.
In particular it contains the prime divisors of $P(n)$, so $\omega(P(n+1))\ge \omega(P(n))$, i.e. $F(n+1)\ge F(n)$. \qed

\medskip
\noindent\textbf{Lemma 2 (conditional: Ono-type statement $\Rightarrow F(n)\to\infty$).}
Assume the following statement:
\begin{quote}
For every prime $q$ there exists an integer $m(q)\ge 1$ such that $q\mid p(m(q))$.
\end{quote}
Then $F(n)\to\infty$.

\noindent\emph{Proof.}
Fix $y\ge 2$ and let $\mathcal P(y)$ be the set of primes $q\le y$.
For each $q\in\mathcal P(y)$ choose $m(q)$ with $q\mid p(m(q))$.
Let
\[N(y):=\max\{m(q): q\le y,\ q\text{ prime}\}.
\]
Then for every $n\ge N(y)$ and every prime $q\le y$ we have $q\mid P(n)$ because $q\mid p(m(q))$ and $p(m(q))$ is a factor of $P(n)$.
Therefore $F(n)=\omega(P(n))\ge |\mathcal P(y)|=\pi(y)$ for all $n\ge N(y)$.
Since $\pi(y)\to\infty$ as $y\to\infty$, it follows that $F(n)\to\infty$ as $n\to\infty$. \qed

\medskip
\noindent\textbf{FAST REALITY CHECK (computation of $F(n)$).}
I computed $p(k)$ for $1\le k\le n$, factored each $p(k)$, and tracked the union of prime factors.
The exact values obtained include:
\[
F(10)=5,\qquad F(50)=34,\qquad F(100)=96,\qquad F(200)=242,\qquad F(300)=412.
\]
In this computation, the first $n$ for which $F(n)>n$ was
\[n=116\quad\text{with}\quad F(116)=117.
\]
These data strongly suggest $F(n)\to\infty$ and support the possibility that $F(n)>n$ eventually, but they are not a proof.

\medskip
\noindent\textbf{VERIFICATION.}
\begin{itemize}
\item Lemma 1 is a direct set-inclusion argument on prime divisors.
\item Lemma 2 is purely logical and does not use any properties of $p(n)$ beyond the assumption.
\item Computation: $F(n)$ was computed as the size of the union of $\mathrm{primefactors}(p(k))$ for $k\le n$.
\end{itemize}

\medskip
\noindent\textbf{FINAL: \textbf{UNRESOLVED}.}
(i) \emph{Strongest proved partial result here.} $F(n)$ is nondecreasing (Lemma 1). Moreover, if one assumes the statement ``every prime divides some partition number'' (which the problem text attributes to Ono), then $F(n)\to\infty$ follows immediately (Lemma 2).

(ii) \emph{First gap (crisp).} I do not have a self-contained proof (without importing the cited deep results) that infinitely many distinct primes divide the set $\{p(1),p(2),\dots\}$, equivalently that $F(n)\to\infty$ unconditionally.

(iii) \emph{Top 3 next moves.}
1. Use (and explicitly write out) the asymptotic formula for $p(n)$ together with a suitable ``many small prime factors'' theorem (as suggested in the problem text) to force new primes to appear in $\prod_{k\le n}p(k)$.
2. Attempt to show that for each fixed prime $q$, the congruence $p(n)\equiv 0\pmod q$ holds for infinitely many $n$ by elementary modular-form arguments (this is essentially what the text attributes to Ono).
3. Extend the computation to larger $n$ (with certified integer factorizations) to test the conjecture $F(n)>n$ and estimate growth rate.

(iv) \emph{Minimal counterexample structure.} A counterexample to $F(n)\to\infty$ would require that all $p(1),p(2),\dots$ are composed of primes from a fixed finite set $S$; equivalently, $p(n)$ would be an $S$-smooth integer for all $n$, which seems incompatible with the rapid growth and varied congruence behavior of $p(n)$, but proving this incompatibility is the main obstacle.
