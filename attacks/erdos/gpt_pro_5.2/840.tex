
Let $f(N)$ be the size of the largest quasi-Sidon subset $A\subset\{1,\ldots,N\}$, where we say that $A$ is quasi-Sidon if\[\lvert A+A\rvert=(1+o(1))\binom{\lvert A\rvert}{2}.\]How does $f(N)$ grow? Considered by Erd\H{o}s and Freud \cite{ErFr91}, who proved\[\left(\frac{2}{\sqrt{3}}+o(1)\right)N^{1/2} \leq f(N) \leq \left(2+o(1)\right)N^{1/2}.\](Although both bounds were already given by Erd\H{o}s in \cite{Er81h}.) Note that $2/\sqrt{3}=1.15\cdots$. The lower bound is taking a genuine Sidon set $B\subset [1,N/3]$ of size $\sim N^{1/2}/\sqrt{3}$ and taking the union with $\{N-b : b\in B\}$. The upper bound was improved by Pikhurko \cite{Pi06} to\[f(N) \leq \left(\left(\frac{1}{4}+\frac{1}{(\pi+2)^2}\right)^{-1/2}+o(1)\right)N^{1/2}\](the constant here is $=1.863\cdots$). The analogous question with $A-A$ in place of $A+A$ is simpler, and there the maximal size is $\sim N^{1/2}$, as proved by Cilleruelo . See also [30] , [819] , and [864] . References [Er81h] Erd\H{o}s, P., Some problems and results on additive and multiplicative number theory . Analytic number theory (Philadelphia, Pa., 1980) (1981), 171-182. [ErFr91] Erd\H{o}s, P. and Freud, R., On sums of a {S}idon-sequence . J. Number Theory (1991), 196--205. [Pi06] Pikhurko, Oleg, Dense edge-magic graphs and thin additive bases . Discrete Math. (2006), 2097--2107.

\medskip
\noindent\textbf{1) FORMAL RESTATEMENT}

For $A\subseteq\{1,2,\dots,N\}$ define the sumset
\[
A+A:=\{a+a': a,a'\in A\} \subseteq \{2,3,\dots,2N\}.
\]
Write $m:=|A|$.  In asymptotic statements we consider sequences $A=A(N)$ with $m=m(N)\to\infty$.

Call $A$ \emph{quasi-Sidon} if
\[
|A+A| = (1+o(1))\binom{m}{2},
\]
i.e. there exist errors $\varepsilon_N\to 0$ with
\[
|A+A| = (1+\varepsilon_N)\binom{m}{2}.
\]
Define
\[
f(N):=\max\{|A|: A\subseteq\{1,\dots,N\}\text{ is quasi-Sidon}\}.
\]
The question is: determine the growth rate of $f(N)$ (especially the leading constant in $f(N)\sim c\sqrt N$ if it exists).

\medskip
\noindent\textbf{2) QUICK LITERATURE/CONTEXT CHECK}

I will not browse or use outside facts beyond the problem text.
The problem text itself records that Erd\H{o}s and Freud proved
\[
\left(\frac{2}{\sqrt{3}}+o(1)\right)\sqrt N \le f(N) \le (2+o(1))\sqrt N,
\]
with a later improved upper constant $1.863\dots$.
Below I (a) re-derive the easy universal upper bound $2\sqrt N+O(1)$ directly, and (b) fully verify the \emph{mechanism} behind the stated lower bound (the ``reflection'' construction from a Sidon set).

\medskip
\noindent\textbf{3) ATTACK PLAN}

\begin{itemize}
\item Prove the universal upper bound from $|A+A|\le 2N-1$.
\item Prove that a genuine Sidon set is automatically quasi-Sidon.
\item Prove the reflection lemma: if $B\subset[1,N/3]$ is Sidon then $A:=B\cup\{N-b:b\in B\}$ is quasi-Sidon and satisfies an explicit sumset count.
\item Do a small-$N$ brute-force computation as a sanity check (exact Sidon maxima for $N\le 30$).
\end{itemize}

\medskip
\noindent\textbf{4) WORK}

\textbf{Fast reality check (exact Sidon maxima for small $N$).}
As a proxy for quasi-Sidon, I computed the maximum size of a genuine Sidon subset $B\subseteq\{1,\dots,N\}$ (all sums $b_i+b_j$ with $i\le j$ distinct) for $5\le N\le 30$.
The exact maxima obtained were:
\begin{center}
\begin{tabular}{c|cccccccccccccccc}
$N$ &5&6&7&8&9&10&11&12&13&14&15&16&17&18&19&20\\\hline
$\max|B|$ &3&3&4&4&4&4&4&5&5&5&5&5&5&6&6&6
\end{tabular}
\end{center}
\begin{center}
\begin{tabular}{c|cccccccccc}
$N$ &21&22&23&24&25&26&27&28&29&30\\\hline
$\max|B|$ &6&6&6&6&6&7&7&7&7&7
\end{tabular}
\end{center}
This is consistent with sizes on the order of $\sqrt N$.

\medskip
\noindent\textbf{Lemma 4.1 (Universal upper bound).}
If $A\subseteq\{1,\dots,N\}$ is quasi-Sidon with $|A|=m\to\infty$, then
\[
m\le (2+o(1))\sqrt N.
\]
In particular $f(N)\le (2+o(1))\sqrt N$.

\noindent\emph{Proof.}
Since $A\subseteq\{1,\dots,N\}$ we have $2\le a+a'\le 2N$, hence $|A+A|\le 2N-1$.
If $A$ is quasi-Sidon then $|A+A|=(1+o(1))\binom{m}{2}$, so for large $N$,
\[
(1+o(1))\binom{m}{2} \le 2N.
\]
Using $\binom{m}{2}=\frac{m^2-m}{2}\sim \frac{m^2}{2}$ as $m\to\infty$, this implies
$(1+o(1))\frac{m^2}{2}\le 2N$, i.e. $m^2\le (4+o(1))N$, hence $m\le (2+o(1))\sqrt N$.
\hfill$\Box$

\medskip
\noindent\textbf{Lemma 4.2 (Sidon $\Rightarrow$ quasi-Sidon).}
Let $A\subseteq\{1,\dots,N\}$ have $|A|=m\to\infty$.
Assume $A$ is a genuine Sidon set in the sense that all sums $a+a'$ with $a\le a'$ are distinct.
Then
\[
|A+A| = \binom{m}{2}+m = \left(1+O\left(\frac{1}{m}\right)\right)\binom{m}{2},
\]
so $A$ is quasi-Sidon.

\noindent\emph{Proof.}
Distinctness of sums for $a\le a'$ means there is exactly one sum for each unordered pair $\{a,a'\}$, with $a\ne a'$ contributing $\binom{m}{2}$ sums and $a=a'$ contributing $m$ diagonal sums $2a$.
Thus $|A+A|=\binom{m}{2}+m$.
Dividing by $\binom{m}{2}$ gives $1+\frac{m}{\binom{m}{2}}=1+\frac{2}{m-1}=1+O(1/m)=1+o(1)$.
\hfill$\Box$

\medskip
\noindent\textbf{Lemma 4.3 (Reflection construction from a Sidon set).}
Let $N\ge 1$ and let $B\subseteq\{1,\dots,\lfloor N/3\rfloor\}$ be a genuine Sidon set with $|B|=t\to\infty$.
Define
\[
C:=\{N-b: b\in B\}\subseteq\{\lceil 2N/3\rceil,\dots,N-1\},
\qquad A:=B\cup C.
\]
Then $|A|=2t$ and
\[
|A+A| = 2t^2+1.
\]
In particular
\[
|A+A| = \left(1+O\left(\frac{1}{t}\right)\right)\binom{|A|}{2},
\]
so $A$ is quasi-Sidon.

\noindenti\emph{Proof.}
We decompose $A+A$ into three disjoint ranges:
\begin{itemize}
\item $B+B \subseteq [2,2\lfloor N/3\rfloor] \subseteq [2,2N/3]$;
\item $C+C = (N-B)+(N-B)=2N-(B+B) \subseteq [4N/3,2N-2]$;
\item $B+C = B+(N-B)=N+(B-B) \subseteq [\lceil 2N/3\rceil+1, \lfloor 4N/3\rfloor-1]$.
\end{itemize}
These intervals are pairwise disjoint, so
\[
|A+A| = |B+B| + |C+C| + |B+C|.
\]
Because $B$ is Sidon, all sums $b+b'$ with $b\le b'$ are distinct, hence
\[
|B+B|=\frac{t(t+1)}{2}.
\]
The map $s\mapsto 2N-s$ is a bijection between $B+B$ and $C+C$, so $|C+C|=|B+B|=t(t+1)/2$.

It remains to compute $|B+C|$.
Note that $B+C = \{N + (b-b'): b,b'\in B\}$, so $|B+C| = |B-B|$.
We claim that for a Sidon set, ordered differences are unique except for the zero difference:
if
\[
b_1-b_2=b_3-b_4
\]
with $b_i\in B$ and $b_1\ne b_2$, then $(b_1,b_2)=(b_3,b_4)$.
Indeed, the difference equality rearranges to
\[
b_1+b_4 = b_3+b_2.
\]
By the Sidon property (uniqueness of sums as unordered pairs), the multiset $\{b_1,b_4\}$ must equal $\{b_3,b_2\}$.
If $b_1\ne b_2$ then necessarily $b_1=b_3$ and $b_4=b_2$, which forces $b_2=b_4$ and then also $b_2=b_4$; in particular $(b_1,b_2)=(b_3,b_4)$.
Thus every nonzero ordered difference $b-b'$ occurs exactly once, while the zero difference occurs for every pair $(b,b)$.
Hence
\[
|B-B| = 1 + (t^2-t) = t^2-t+1.
\]
Therefore
\[
|A+A| = \frac{t(t+1)}{2}+\frac{t(t+1)}{2}+(t^2-t+1) = t(t+1)+t^2-t+1 = 2t^2+1.
\]
Finally, $\binom{|A|}{2}=\binom{2t}{2}=2t^2-t$, and
\[
\frac{|A+A|}{\binom{|A|}{2}} = \frac{2t^2+1}{2t^2-t} = 1 + \frac{t+1}{2t^2-t} = 1+O\left(\frac{1}{t}\right)=1+o(1),
\]
so $A$ is quasi-Sidon.
\hfill$\Box$

\medskip
\noindent\textbf{Comment (how the stated lower bound arises).}
Lemma 4.3 shows that 
\emph{if} one has a Sidon set $B\subseteq[1,N/3]$ of size $t\sim \sqrt{N/3}$, then the reflected union produces a quasi-Sidon set of size $|A|=2t\sim \frac{2}{\sqrt3}\sqrt N$.
The existence of such $B$ (and hence the stated lower bound) is part of what the problem text attributes to Erd\H{o}s and Freud.

\medskip
\noindent\textbf{5) VERIFICATION}

\begin{itemize}
\item Lemma 4.1 uses only the trivial range $A+A\subseteq\{2,\dots,2N\}$.
\item Lemma 4.2 correctly counts diagonal sums $2a$ separately from off-diagonal sums.
\item In Lemma 4.3 the disjointness of the three sum ranges is checked via the bounds $B\subseteq[1,N/3]$ and $C\subseteq[2N/3,N]$.
\item The difference-uniqueness claim used in Lemma 4.3 was derived directly from the Sidon sum-uniqueness property.
\end{itemize}

\medskip
\noindent\textbf{6) FINAL}

\textbf{**UNRESOLVED**}

(i) \emph{Strongest proved partial result.} A direct proof of the universal upper bound $f(N)\le (2+o(1))\sqrt N$ (Lemma 4.1), and a complete verification that the ``reflection'' construction from a Sidon set $B\subset[1,N/3]$ produces a quasi-Sidon set $A$ with an exact sumset count $|A+A|=2|B|^2+1$ (Lemma 4.3).

(ii) \emph{First gap (crisp).} Determine the optimal constant $c$ (or even narrow it to a single value) such that $f(N)=(c+o(1))\sqrt N$.

(iii) \emph{Top 3 next moves.}
\begin{enumerate}
\item Identify and analyze candidate extremal constructions for quasi-Sidon sets (e.g., reflected Sidon-type constructions), and compute their achieved constant.
\item Prove a structural theorem: if $|A+A|=(1+o(1))\binom{|A|}{2}$ then $A$ must resemble (in a precise sense) a Sidon set or a reflected Sidon set; turn this into a sharper upper bound on $|A|$.
\item Compute exact/near-exact values of $f(N)$ for moderate $N$ (via ILP / SAT / backtracking) under a concrete finite-$N$ proxy for ``$o(1)$'', to conjecture the limiting constant.
\end{enumerate}

(iv) \emph{What a minimal counterexample would likely look like.} Any counterexample to a conjectured constant $c$ would be a family $A\subseteq[1,N]$ with $|A|\approx c\sqrt N$ in which almost all pair sums are distinct, but whose sumset still fits inside $[2,2N]$. Such a set must have very few additive collisions, so it should be close to a Sidon set (possibly after a reflection/translation), and the obstruction to improving the constant should come from a delicate packing of nearly distinct sums into the fixed interval.

