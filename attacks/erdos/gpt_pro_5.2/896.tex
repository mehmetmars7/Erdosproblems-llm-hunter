

\noindent\textbf{1) FORMAL RESTATEMENT}

Fix $N\ge 1$.
For subsets $A,B\subseteq\{1,2,\dots,N\}$ define, for each integer $m\ge 1$,
\[
 r_{A,B}(m):=\#\{(a,b)\in A\times B: ab=m\}.
\]
Define
\[
 F(A,B):=\#\{m\in\mathbb{N}: r_{A,B}(m)=1\}.
\]
Question: estimate
\[ M(N):=\max_{A,B\subseteq\{1,\dots,N\}} F(A,B)\]
as $N\to\infty$.

\medskip
\noindent\textbf{2) QUICK LITERATURE/CONTEXT CHECK}

The problem statement records bounds (attributed there to van Doorn):
\[
(1+o(1))\frac{N^2}{\log N}\le M(N)\ll \frac{N^2}{(\log N)^\delta(\log\log N)^{3/2}}
\quad\text{with }\delta\approx 0.086.
\]

\medskip
\noindent\textbf{3) ATTACK PLAN}

\begin{itemize}
\item Give an explicit construction of $A,B$ with many uniquely represented
products and prove uniqueness directly.
\item Give simple general identities/inequalities involving $F(A,B)$ (e.g.
characterizing when $F=|A||B|$).
\item Do a brute-force reality check for small $N$.
\end{itemize}

\medskip
\noindent\textbf{4) WORK}

\textbf{PHASE 1 — FAST REALITY CHECK (brute force for small $N$).}

I brute-forced $M(N)$ for $2\le N\le 10$ by enumerating all pairs of subsets
$(A,B)$ (using bitmasks).
The exact maxima found were:
\[
\begin{array}{c|ccccccccc}
N&2&3&4&5&6&7&8&9&10\\\hline
M(N)&2&4&7&10&13&18&22&29&32
\end{array}
\]
(Examples of maximizers exist; one example for $N=10$ is
$A=\{1,2,3,4,5,6,7,8\}$ and $B=\{1,7,8,9,10\}$, which yields $F(A,B)=32$.)

\medskip
\textbf{Problem-specific lemmas (injectivity and a clean lower-bound construction).}

\medskip
\noindent\textbf{Lemma 896.1 (Injectivity criterion).}
For any finite sets $A,B$ of positive integers,
\[ F(A,B)=|A||B| \quad\Longleftrightarrow\quad \text{the map }A\times B\to\mathbb{N},\ (a,b)\mapsto ab\text{ is injective}. \]

\emph{Proof.}
We always have $F(A,B)\le |A||B|$ since each $m$ with $r_{A,B}(m)=1$ accounts for
exactly one pair $(a,b)\in A\times B$.

If the map is injective, then every product $ab$ occurs for exactly one pair, so
$r_{A,B}(m)\in\{0,1\}$ for all $m$ and $F(A,B)=|A||B|$.

Conversely, if $F(A,B)=|A||B|$, then every pair $(a,b)\in A\times B$ must produce
an $m=ab$ with $r_{A,B}(m)=1$ (otherwise we would have strictly fewer than
$|A||B|$ values with exactly one representation). Thus distinct pairs cannot have
the same product, i.e. the map is injective.
\qed

\medskip
\noindent\textbf{Lemma 896.2 (Large-prime construction gives many unique products).}
Let
\[A:=\{1,2,\dots,\lfloor N/2\rfloor\},\qquad B:=\{p\ \text{prime}: N/2<p\le N\}.
\]
Then every $m$ of the form $m=ap$ with $a\in A$ and $p\in B$ has exactly one
representation $m=ab$ with $a\in A$, $b\in B$.
In particular
\[
F(A,B)=|A||B|=\lfloor N/2\rfloor\bigl(\pi(N)-\pi(N/2)\bigr).
\]

\emph{Proof.}
Fix $a\in A$ and $p\in B$ and let $m=ap$.
We show that if $m=a'p'$ with $a'\in A$ and $p'\in B$, then necessarily
$p'=p$ and $a'=a$.

Since $p\mid m$ and $p'$ is prime, either $p'=p$ or $p\mid a'$.
But $p> N/2$ and $a'\le \lfloor N/2\rfloor$, so $p\nmid a'$.
Hence $p'=p$, and then $a'=m/p=a$.
Therefore $r_{A,B}(m)=1$ for every $m$ in the image of $A\times B$.
Moreover the same argument shows the map $(a,p)\mapsto ap$ is injective, so by
Lemma~896.1 we have $F(A,B)=|A||B|$.
\qed

\medskip
\noindent\textbf{5) VERIFICATION}

\begin{itemize}
\item Lemma~896.1: the only subtle direction is that $F=|A||B|$ forces every pair
$(a,b)$ to map to a uniquely represented product. This follows because the total
number of pairs is $|A||B|$, and a value $m$ with $r(m)=1$ accounts for exactly one
pair.
\item Lemma~896.2: the key inequality is $p>N/2\ge a'$, so $p\nmid a'$. This is
correct.
\end{itemize}

\medskip
\noindent\textbf{6) FINAL}

\textbf{UNRESOLVED}

(i) \emph{Strongest proved partial result.}
The explicit choice $A=\{1,\dots,\lfloor N/2\rfloor\}$ and
$B=\{p\text{ prime}: N/2<p\le N\}$ yields
$F(A,B)=\lfloor N/2\rfloor(\pi(N)-\pi(N/2))$ (Lemma~896.2), i.e. on the order of
$N\cdot(\#\text{primes in }(N/2,N])$.
Exact brute-force maxima for $N\le 10$ are
$M(2)=2, M(3)=4,\dots,M(10)=32$.

(ii) \emph{First gap (crisp).}
Prove an asymptotic formula (or a tight order of magnitude) for
$M(N)=\max_{A,B\subseteq[N]}F(A,B)$ as $N\to\infty$.
In particular, determine whether $M(N)\sim cN^2/\log N$ for some constant $c>0$.

(iii) \emph{Top 3 next moves.}
\begin{enumerate}
\item Improve the lower bound construction by allowing $B$ to contain integers
with a guaranteed “unique large prime factor” beyond $N/2$, while enlarging $A$.
\item Develop an upper bound mechanism: bound the number of uniquely represented
products by controlling collisions $ab=a'b'$ via divisor statistics or energy
methods.
\item Computational: for moderate $N$ (say $N\le 18$) use ILP/SAT to find exact
$M(N)$ and examine the structure of extremizers, to guess the right asymptotic
constant and shape.
\end{enumerate}

(iv) \emph{Minimal counterexample structure (if the conjectured order is wrong).}
If $M(N)$ were substantially larger than $N^2/\log N$, extremizers would need to
avoid the ubiquitous multiplicative collisions in $[1,N]^2$; this would likely
force $A$ and $B$ to concentrate on numbers with a “signature” prime factor in a
narrow range, so that products carry an essentially unique large prime.


