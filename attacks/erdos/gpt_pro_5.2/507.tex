
1) FORMAL RESTATEMENT

Let $D\subset\mathbb R^2$ be the closed unit disk (radius $1$).
For an integer $n\ge 3$, define
\[
\alpha(n):=\sup_{P\subset D,\ |P|=n}\ \ \min_{\{x,y,z\}\subset P}\ \mathrm{Area}(\triangle xyz),
\]
where $\mathrm{Area}(\triangle xyz)$ denotes the Euclidean area of the triangle with vertices $x,y,z$.
Equivalently, $\alpha(n)$ is the smallest real number such that \emph{every} set of $n$ points in $D$ contains a triangle of area at most $\alpha(n)$.
The problem asks for good asymptotic estimates of $\alpha(n)$ as $n\to\infty$.

2) QUICK LITERATURE/CONTEXT CHECK

The problem statement records the bounds
\[\frac{\log n}{n^2}\ll \alpha(n) \ll \frac{1}{n^{7/6+o(1)}}.
\]
It also notes the easy upper bound $\alpha(n)\ll 1/n$ and that Erd\H{o}s observed $\alpha(n)\gg 1/n^2$.
Below I prove the easy upper bound and a clean lower-bound construction for infinitely many $n$ (namely primes).

3) ATTACK PLAN

\emph{Proof track:} prove a universal upper bound by partitioning the disk into $O(n)$ small-area convex regions so that one region must contain three points.

\emph{Construction track:} build $n$ points with no three collinear on a rational grid of mesh $1/n$ so that all triangle areas are integer multiples of $1/(2n^2)$, hence bounded below by $\asymp 1/n^2$.

4) WORK

Lemma 507.1 (Trivial upper bound $\alpha(n)=O(1/n)$).
For every $n\ge 3$,
\[
\alpha(n)\le \frac{2\pi}{n-1}.
\]

\emph{Proof.}
Let $m:=\left\lfloor\frac{n-1}{2}\right\rfloor$.
Then $2m\le n-1$, so $n>2m$.
Partition the unit disk $D$ into $m$ closed circular sectors of equal area, each sector having area $\pi/m$.
Each sector is convex.

If every sector contained at most $2$ of the $n$ points, then the total number of points would be at most $2m<n$, a contradiction.
Hence some sector contains at least $3$ points of the set.
Any triangle formed by three points inside a convex region lies inside that region, so its area is at most the area of the region.
Therefore there exists a triangle of area at most $\pi/m$.

Finally, since $m=\lfloor (n-1)/2\rfloor \ge (n-3)/2$ for $n\ge 3$,
\[
\frac{\pi}{m}\le \frac{\pi}{(n-3)/2}=\frac{2\pi}{n-3}\le \frac{2\pi}{n-1},
\]
where the last inequality holds for $n\ge 3$.
Thus $\alpha(n)\le 2\pi/(n-1)$.
\qed

Lemma 507.2 (Lower bound $\alpha(p)\ge 1/(2p^2)$ for primes $p$).
Let $p$ be an odd prime. Then
\[
\alpha(p)\ge \frac{1}{2p^2}.
\]

\emph{Proof.}
\underline{Step 1: Construct $p$ points in the unit disk.}
For each $t\in\{0,1,\dots,p-1\}$ define the point
\[
P_t:=\left(\frac{t}{p}-\frac12,\ \frac{t^2\bmod p}{p}-\frac12\right)\in\mathbb R^2.
\]
Then $P_t\in[-\tfrac12,\tfrac12]^2$, which is contained in the unit disk (since its farthest corner has distance $\sqrt{(1/2)^2+(1/2)^2}=\sqrt2/2<1$ from the origin).
So $\{P_t\}_{t=0}^{p-1}\subset D$.

\underline{Step 2: No three points are collinear.}
Suppose, for contradiction, that three distinct indices $a,b,c\in\{0,\dots,p-1\}$ give collinear points $P_a,P_b,P_c$ in $\mathbb R^2$.
Since translation by $(-1/2,-1/2)$ does not affect collinearity, the integer grid points
\[(a, a^2\bmod p),\ (b, b^2\bmod p),\ (c, c^2\bmod p)\in\{0,\dots,p-1\}^2
\]
are collinear in $\mathbb R^2$.
Collinearity of three integer points is equivalent to the vanishing of the $2\times 2$ determinant
\[
\det\begin{pmatrix}
 b-a & (b^2\bmod p)-(a^2\bmod p) \\[2pt]
 c-a & (c^2\bmod p)-(a^2\bmod p)
\end{pmatrix}=0.
\]
Reducing this integer determinant modulo $p$ shows that the three points are collinear in the affine plane over $\mathbb F_p$.
Thus there exist $m,k\in\mathbb F_p$ such that
\[x^2 \equiv mx+k\pmod p\quad\text{for }x\in\{a,b,c\}.
\]
Equivalently, the quadratic polynomial $f(x)=x^2-mx-k\in\mathbb F_p[x]$ has three distinct roots $a,b,c$.
But a nonzero polynomial of degree $2$ over a field has at most $2$ roots.
This contradiction shows no three of the $P_t$ are collinear.

\underline{Step 3: Quantize triangle areas.}
Take any three distinct indices $a,b,c$. The points $P_a,P_b,P_c$ have coordinates that are integer multiples of $1/p$.
The (signed) twice-area of the triangle is the absolute value of a determinant of the form
\[
2\,\mathrm{Area}(\triangle P_aP_bP_c)=\left|\det\begin{pmatrix}
 x_b-x_a & y_b-y_a\\
 x_c-x_a & y_c-y_a
\end{pmatrix}\right|,
\]
where each difference $x_b-x_a, y_b-y_a$ is a multiple of $1/p$.
Hence $2\,\mathrm{Area}(\triangle P_aP_bP_c)$ is an integer multiple of $1/p^2$, and thus the area is an integer multiple of $1/(2p^2)$.
Because the three points are not collinear, the area is nonzero, so
\[\mathrm{Area}(\triangle P_aP_bP_c)\ge \frac{1}{2p^2}.
\]
Thus the minimum triangle area among these $p$ points is at least $1/(2p^2)$.
By definition of $\alpha(p)$ as the supremum over $p$-point sets of their minimum triangle area, we conclude $\alpha(p)\ge 1/(2p^2)$.
\qed

FAST REALITY CHECK (computed for small primes).
For the point set $\{P_t\}_{t=0}^{p-1}$ above, I computed the exact minimum area among all $\binom{p}{3}$ triangles:
\[
 p=3:\ \min=1/18\approx 0.0555556;\qquad
 p=5:\ \min=1/50=0.02;\qquad
 p=7:\ \min=1/98\approx 0.0102041;\qquad
 p=11:\ \min=1/242\approx 0.00413223.
\]
In each case this equals $1/(2p^2)$, confirming the lower bound is tight for this construction.

5) VERIFICATION

\emph{Lemma 507.1:} The only geometric input is that a triangle contained in a region has area at most the region's area; convexity ensures containment. The pigeonhole step uses $n>2m$.

\emph{Lemma 507.2:} The no-three-collinear step reduces collinearity over $\mathbb R$ to a determinant identity over $\mathbb Z$ and then modulo $p$, which is valid because the coordinates are integers. The area quantization uses that determinants of rational differences have denominator dividing $p^2$.

6) FINAL
UNRESOLVED

(i) \emph{Strongest proved partial result here:} A universal upper bound $\alpha(n)\le 2\pi/(n-1)$ (Lemma 507.1) and an explicit construction showing $\alpha(p)\ge 1/(2p^2)$ for every prime $p$ (Lemma 507.2), with computed small-prime verification.

(ii) \emph{First gap (crisp):} Extend the lower bound from primes to all $n$ with a constant-factor loss (i.e. prove $\alpha(n)\ge c/n^2$ for all large $n$) without appealing to unproved number-theoretic input inside this writeup.

(iii) \emph{Top 3 next moves:}
1. Prove a prime-gap statement sufficient to pass from Lemma 507.2 to all $n$ (e.g. use Bertrand's postulate, with a full proof, to get a prime $p\in[n,2n]$ and then take an $n$-point subset).
2. Improve the trivial upper bound $O(1/n)$ by a more geometric pigeonhole (e.g. exploiting triangulations/convex hull area) to see if one can push below exponent $1$.
3. For small $n$ (say $n\le 8$), numerically optimize point sets to estimate $\alpha(n)$ and see how the extremizers look (grid-like, boundary-like, etc.).

(iv) \emph{Minimal counterexample structure (to an $\Theta(1/n^2)$ lower bound):} a family of $n$-point sets in the disk with minimum triangle area $o(1/n^2)$ would necessarily contain many near-collinear triples at scales much smaller than $1/n$; understanding how such near-collinearities can be forced or avoided seems central.


