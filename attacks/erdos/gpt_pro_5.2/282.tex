\section*{Erd\H{o}s Problem \#282}

\subsection*{1) FORMAL RESTATEMENT}
Let $A\subseteq \mathbb{N}$ be infinite, and let $x\in(0,1)\cap\mathbb{Q}$. Consider the greedy procedure that repeatedly subtracts the \emph{largest} unit fraction from the allowed set $\{1/n : n\in A\}$ that does not exceed the current remainder.

\medskip
\noindent\textbf{(Ambiguity and minimal correction.)}
As written in many informal descriptions, one chooses at each step the minimal $n\in A$ with $n\ge 1/x$ and replaces $x\leftarrow x-1/n$. For $A=\mathbb{N}$ this automatically yields \emph{distinct} denominators (indeed a strictly increasing sequence), but for sparse $A$ one can obtain repeats. Since the goal is an \emph{Egyptian fraction} (distinct denominators), the standard correction (used e.g. in the ``odd greedy expansion'' literature) is:

\begin{quote}
At step $j$, choose the least $n_j\in A\setminus\{n_1,\dots,n_{j-1}\}$ such that $\frac1{n_j}\le x_{j-1}$ (equivalently $n_j\ge 1/x_{j-1}$), and set $x_j=x_{j-1}-\frac1{n_j}$.
\end{quote}
We call this the \emph{distinct-denominator greedy algorithm} for $(x,A)$. It \emph{terminates} if $x_t=0$ for some finite $t$, yielding
\[
 x=\sum_{j=1}^t \frac1{n_j},\qquad n_1<\cdots<n_t\in A.
\]

\medskip
\noindent\textbf{Main question (Stein).}
With $A=2\mathbb{N}-1$ (the odd positive integers) and $x=a/b$ in lowest terms with $b$ odd: \emph{must the distinct-denominator greedy algorithm terminate?}

\medskip
\noindent\textbf{Further questions included in the prompt.}
\begin{enumerate}[label=(\alph*)]
\item Characterise pairs $(x,A)$ for which termination holds.
\item (Congruence classes) Fix $d\ge 1$ and $a\bmod d$. When $x=m/n$ admits a representation as a sum of distinct unit fractions with denominators $\equiv a\pmod d$ (Graham's condition), must the corresponding restricted greedy algorithm terminate?
\item (Squares) For $A=\{n^2:n\in\mathbb{N}\}$, Graham characterised exactly which $x$ admit a representation with distinct square denominators. Must the restricted greedy algorithm terminate for such $x$?
\end{enumerate}

\subsection*{2) QUICK LITERATURE/CONTEXT CHECK}
The problem for odd denominators is attributed to Stein and is widely stated as open (often called ``Stein's conjecture'' or the ``odd greedy expansion problem''). It is known (Breusch--Stewart) that every rational with odd denominator admits \emph{some} Egyptian fraction expansion with all denominators odd, but it is unknown whether the \emph{odd greedy} choice always reaches such an expansion.

The problem page also records:
\begin{itemize}
\item Fibonacci (1202) proved termination for $A=\mathbb{N}$ (the classical greedy/Sylvester algorithm).
\item Graham gave a necessary-and-sufficient condition for existence of an expansion with denominators in a fixed residue class $\bmod d$.
\item Graham also gave a necessary-and-sufficient condition for existence of an expansion with distinct square denominators, namely $x\in [0,\pi^2/6-1)\cup [1,\pi^2/6)$.
\end{itemize}
The question here is about \emph{termination of the greedy algorithm} in these settings.

\subsection*{3) ATTACK PLAN}
\begin{enumerate}[label=\textbf{P\arabic*.},leftmargin=*]
\item \textbf{Invariant/obstruction search.} Identify arithmetic invariants (e.g. parity, $p$-adic valuations, congruence constraints) that any finite $A$-Egyptian expansion forces on $x$. Use them to rule out termination for certain $(x,A)$ and to separate ``existence of some expansion'' from ``greedy termination''.
\item \textbf{Termination proofs for subclasses.} Prove termination for special families (e.g. $x=1/(2k+1)$, or bounded numerator, or specific congruence classes) by finding a monotone measure that decreases under the restricted greedy step.
\item \textbf{Search for counterexamples (falsification).} For the square-denominator and odd-denominator greedy variants, attempt to find a rational $x$ satisfying the known existence criteria but for which the greedy algorithm provably loops or generates an infinite series.
\item \textbf{Understand known ``long expansion'' phenomena.} For odd greedy expansions, huge denominators and long expansions occur (e.g. $3/179$). Try to extract a structural description of when the numerator sequence behaves regularly (consecutive integers), which may hint at non-termination mechanisms.
\end{enumerate}

\subsection*{4) WORK}
\subsubsection*{4.1. Classical case $A=\mathbb{N}$: termination proof (Fibonacci/Sylvester)}
Let $x=a/b\in(0,1)$ in lowest terms. Set $n=\lceil b/a\rceil$ and define
\[
 x' = \frac{a}{b}-\frac1n = \frac{an-b}{bn}.
\]
Since $n-1< b/a\le n$, we have $a(n-1)<b\le an$, hence $0\le an-b < a$. If $an=b$ then $x'=0$ and we terminate. Otherwise, in lowest terms $x'=a'/b'$ has numerator $a'\le an-b < a$. Thus the numerator strictly decreases at each step, so termination occurs in at most $a$ steps.

\subsubsection*{4.2. A necessary obstruction for $A$ odd}
\begin{lemma}
If $n_1,\dots,n_t$ are \emph{odd} positive integers then $\sum_{j=1}^t \frac1{n_j}$, when reduced to lowest terms, has odd denominator.
\end{lemma}
\begin{proof}
Let $D=\prod_{j=1}^t n_j$ (odd). Then
\[
\sum_{j=1}^t \frac1{n_j}=\frac{\sum_{j=1}^t D/n_j}{D}.
\]
Any common divisor of numerator and denominator divides $D$, hence is odd, so the reduced denominator remains odd.
\end{proof}
\begin{corollary}
If $x=a/b$ in lowest terms has $b$ even, then $x$ cannot be written as a finite sum of distinct unit fractions with odd denominators. In particular, the odd-denominator greedy algorithm cannot terminate on such $x$.
\end{corollary}

\subsubsection*{4.3. Small computational sanity check for odd greedy termination}
Implementing the \emph{distinct-denominator} odd greedy algorithm (choose the least unused odd $n\ge 1/x$ each step), an exhaustive check over all reduced fractions $a/b\in(0,1)$ with odd $b\le 51$ found termination for every such fraction within at most $16$ steps. Some expansions have enormous final denominators even for modest $b$.

\medskip
\noindent\emph{Warning:} This is only evidence. There are known examples (e.g. $3/179$) where the odd greedy expansion has many terms and astronomically large denominators.

\subsubsection*{4.4. Congruence classes and squares}
The prompt cites existence theorems (Graham) for Egyptian fraction representations with denominators restricted to a residue class $\bmod d$, and for representations with distinct square denominators. In both cases the open-ended question is: \emph{does the associated restricted greedy algorithm always terminate whenever a representation exists?}

A basic necessary condition for greedy termination is of course \emph{existence} of some expansion in the allowed denominator set; beyond that, there is no known general monotone quantity analogous to the numerator-decrease argument in \S4.1.

\subsection*{5) VERIFICATION}
\begin{itemize}
\item The proof in \S4.1 is standard and fully explicit: the numerator decreases strictly.
\item The parity obstruction in \S4.2 is immediate from using an odd common denominator.
\item The computational check in \S4.3 was implemented with exact integer arithmetic and a ``no-repeated-denominators'' rule; it is exhaustive for odd denominators $\le 51$ but does not address larger denominators (where known blow-ups occur).
\end{itemize}

\subsection*{6) FINAL}
\textbf{UNRESOLVED.}
\begin{enumerate}[label=(\roman*),leftmargin=*]
\item \textbf{Furthest point reached:} proved basic necessary obstructions (even denominators impossible for odd $A$), recalled/verified classical termination for $A=\mathbb{N}$, and performed a complete finite check for all reduced $a/b$ with odd $b\le 51$ showing termination in these cases.
\item \textbf{Key gap:} no global invariant/monotone measure is found that forces termination (or yields a forced infinite trajectory) for $A=2\mathbb{N}-1$ with $b$ odd, nor for the congruence-class or square-denominator greedy variants.
\item \textbf{Most promising next step:} search for an explicit non-terminating example by exploiting ``consecutive numerator'' phenomena (as in long odd greedy expansions), or develop a Lyapunov-type function controlling the growth of denominators for restricted greedy steps.
\item \textbf{Explicit missing lemma that would close it:} either (a) a proof that every odd-denominator rational eventually reaches a unit fraction under the odd greedy update, or (b) an explicit $x$ with odd denominator for which the odd greedy remainder sequence never becomes $0$.
\end{enumerate}

\subsection*{7) COMPLETION ESTIMATE}
\noindent\textbf{COMPLETION: 20\%}

%%%%%%%%%%%%%%%%%%%%%%%%%%%%%%%%%%%%%%%%%%%%%%%%%%%%%%%%%%%%%%%%%%%%%%%%%%%%%%
