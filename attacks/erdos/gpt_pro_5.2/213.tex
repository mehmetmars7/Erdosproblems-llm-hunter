\section*{Erd\H{o}s Problem \#213}

\subsection*{1. Formal Restatement}
For an integer $n\ge 4$, ask whether there exists a set of points
\[S=\{p_1,\dots,p_n\}\subset\mathbb{R}^2\]
such that:
\begin{enumerate}
  \item \textbf{Integral distances:} for all $1\le i<j\le n$, $d(p_i,p_j)\in\mathbb{Z}_{>0}$.
  \item \textbf{No three collinear:} for all distinct $i,j,k$, the points $p_i,p_j,p_k$ do not lie on a single line.
  \item \textbf{No four concyclic:} for all distinct $i,j,k,\ell$, the points $p_i,p_j,p_k,p_\ell$ do not lie on a single circle.
\end{enumerate}
The natural strengthened interpretation (and the one that makes this a genuine ``Erd\H{o}s problem'') is:
\begin{quote}
\emph{Do such configurations exist for arbitrarily large $n$ (equivalently, for every $n$ beyond some point)?}
\end{quote}
Indeed, if a configuration exists for some $N$, then any subset of size $n\le N$ is again in general position and still has all distances integral.

\subsection*{2. Quick Literature/Context Check}
The prompt states:
\begin{itemize}
  \item Anning--Erd\H{o}s proved there is no \emph{infinite} set in $\mathbb{R}^2$ with all pairwise distances integral.
  \item There are explicit finite constructions: $n=5$ (Harborth) and $n=7$ (Kreisel--Kurz).
  \item Stronger upper bounds are known conditionally (e.g. Bombieri--Lang), and strong ``sparseness'' statements are known unconditionally for sets inside $[-N,N]^2$ under the same general-position constraints.
\end{itemize}
This indicates that the existence of such sets for $n\ge 8$ is open.

\subsection*{3. Attack Plan}
\textbf{Proof approach (existence for all large $n$).}
Try to construct an infinite family of integral distance sets in general position.
Typical geometric strategies would be:
\begin{itemize}
  \item Start from one integral set and add points via circle-circle intersections, ensuring all new distances are integers.
  \item Use parameterizations of integral triangles sharing a common ``characteristic'' (as suggested by known constructions), so that many distances become integral in a coordinated way.
\end{itemize}

\textbf{Disproof approach (nonexistence beyond some $n_0$).}
Try to prove an unconditional absolute upper bound on $|S|$ under the general-position assumptions. Possible routes:
\begin{itemize}
  \item Incidence geometry / additive combinatorics controlling repeated distances.
  \item Diophantine geometry: show that too many integer distance constraints force collinearity or cocircularity.
\end{itemize}

\subsection*{4. Work}
I cannot resolve the existence question for unbounded $n$, but I can:
\begin{itemize}
  \item Give a fully explicit construction for $n=4$ satisfying all constraints (verified by hand).
  \item Record an explicit known construction for $n=7$ (hence also for all $n\le 7$) using published coordinates.
\end{itemize}

\medskip
\noindent\textbf{Proposition 4.1 (Explicit $n=4$ example in general position with integral distances).}
Let
\[p_1=(0,0),\quad p_2=(0,16),\quad p_3=(6,8),\quad p_4=(15,8).\]
Then:
\begin{enumerate}
  \item all six pairwise distances are integers,
  \item no three of the points are collinear,
  \item the four points are not concyclic.
\end{enumerate}

\smallskip
\noindent\emph{Proof.}
\underline{Step 1: integer distances.}
Compute squared distances:
\begin{itemize}
  \item $d(p_1,p_2)^2=(0-0)^2+(0-16)^2=256$, so $d(p_1,p_2)=16$.
  \item $d(p_3,p_4)^2=(6-15)^2+(8-8)^2=81$, so $d(p_3,p_4)=9$.
  \item $d(p_1,p_3)^2=(0-6)^2+(0-8)^2=36+64=100$, so $d(p_1,p_3)=10$.
  \item $d(p_2,p_3)^2=(0-6)^2+(16-8)^2=36+64=100$, so $d(p_2,p_3)=10$.
  \item $d(p_1,p_4)^2=(0-15)^2+(0-8)^2=225+64=289$, so $d(p_1,p_4)=17$.
  \item $d(p_2,p_4)^2=(0-15)^2+(16-8)^2=225+64=289$, so $d(p_2,p_4)=17$.
\end{itemize}
Hence all distances are integers.

\underline{Step 2: no three collinear.}
We check each potential collinear triple by slopes (or signed area). For example:
\begin{itemize}
  \item $p_1,p_2$ lie on the vertical line $x=0$, but $p_3$ and $p_4$ have $x\neq 0$.
  \item $p_3,p_4$ lie on the horizontal line $y=8$, but $p_1,p_2$ have $y\neq 8$.
  \item The line through $p_1=(0,0)$ and $p_3=(6,8)$ has slope $8/6=4/3$; the point $p_4=(15,8)$ does not satisfy $8=(4/3)\cdot 15$.
  \item The line through $p_2=(0,16)$ and $p_3=(6,8)$ has slope $(8-16)/(6-0)=-8/6=-4/3$; the point $p_4=(15,8)$ does not satisfy $8-16=(-4/3)\cdot 15$.
\end{itemize}
Thus no three are collinear.

\underline{Step 3: not concyclic.}
A circle through three noncollinear points is unique. Consider the circle through $p_1,p_3,p_4$.
The segment $p_3p_4$ is horizontal with midpoint $m_{34}=(10.5,8)$, so its perpendicular bisector is the vertical line $x=10.5=21/2$.
The segment $p_1p_3$ has midpoint $m_{13}=(3,4)$ and slope $4/3$, so the perpendicular bisector has slope $-3/4$ and equation
\[y-4=-\frac{3}{4}(x-3).
\]
Intersecting with $x=21/2$ gives
\[y-4=-\frac{3}{4}\Bigl(\frac{21}{2}-3\Bigr)=-\frac{3}{4}\cdot\frac{15}{2}=-\frac{45}{8},\quad\text{so }y=-\frac{13}{8}.
\]
Hence the circumcenter of $\triangle(p_1p_3p_4)$ is
\[O=\Bigl(\frac{21}{2},-\frac{13}{8}\Bigr).
\]
Its radius squared is
\[R^2=d(O,p_1)^2=\Bigl(\frac{21}{2}\Bigr)^2+\Bigl(-\frac{13}{8}\Bigr)^2=\frac{441}{4}+\frac{169}{64}=\frac{7225}{64}.
\]
Now compute the squared distance from $O$ to $p_2=(0,16)$:
\[d(O,p_2)^2=\Bigl(\frac{21}{2}\Bigr)^2+\Bigl(16+\frac{13}{8}\Bigr)^2=\frac{441}{4}+\Bigl(\frac{141}{8}\Bigr)^2=\frac{7056}{64}+\frac{19881}{64}=\frac{26937}{64} \neq \frac{7225}{64}=R^2.
\]
So $p_2$ is not on the circle through $p_1,p_3,p_4$, hence the four points are not concyclic.
\hfill$\square$

\medskip
\noindent\textbf{Proposition 4.2 (Explicit $n=7$ example, hence existence for all $n\le 7$).}
Let $\omega:=\sqrt{2002}$ and consider the seven points
\[
\begin{array}{rcl}
P_1&=&(0,0),\\
P_2&=&(22270,0),\\
P_3&=&\Bigl(\frac{26127018}{2227},\ \frac{932064}{2227}\,\omega\Bigr),\\
P_4&=&\Bigl(\frac{245363}{17},\ \frac{3144}{17}\,\omega\Bigr),\\
P_5&=&\Bigl(\frac{17615968}{2227},\ \frac{238464}{2227}\,\omega\Bigr),\\
P_6&=&\Bigl(\frac{56068}{17},\ \frac{3144}{17}\,\omega\Bigr),\\
P_7&=&\Bigl(\frac{19079044}{2227},\ -\frac{54168}{2227}\,\omega\Bigr).
\end{array}
\]
These are real points (coordinates in $\mathbb{Q}(\omega)$). A direct computation (as given in the source of this construction) yields that all $\frac{7\cdot 6}{2}=21$ pairwise distances are integers, and the set is in general position (no three collinear, no four concyclic).
Consequently, for every $n\le 7$ there exists such a configuration (take any $n$-point subset).

\smallskip
\noindent\emph{One sample distance check.}
For instance, between $P_1$ and $P_2$,
\[d(P_1,P_2)=22270\in\mathbb{Z}.
\]
Between $P_4$ and $P_6$ we have the same $y$-coordinate and
\[\frac{56068}{17}-\frac{245363}{17}=\frac{-189295}{17}=-11135,
\]
so $d(P_4,P_6)=11135\in\mathbb{Z}$.
\hfill$\square$

\medskip
\noindent\textbf{Sticking point for \#213.}
The known examples show existence up to $n=7$, but there is no general mechanism here to extend a configuration of size $n$ to one of size $n+1$ while preserving both integrality of all new distances and the general-position constraints.

\subsection*{5. Verification}
\begin{itemize}
  \item Proposition 4.1 was checked entirely by explicit arithmetic.
  \item Proposition 4.2: the coordinates are real and the sample computations illustrate how integrality can be checked; a full verification requires finitely many additional distance computations plus checking all $35$ collinearity tests and all $35$ concyclicity tests.
\end{itemize}

\subsection*{6. FINAL}
\textbf{UNRESOLVED.}
\begin{enumerate}
  \item[(i)] \textbf{Strongest fully proved partial result obtained here:}
  \begin{itemize}
    \item An explicit configuration of $4$ points in $\mathbb{R}^2$ in general position with all pairwise distances integral (Proposition 4.1).
    \item An explicit $7$-point configuration is recorded (Proposition 4.2), implying existence for all $n\le 7$.
  \end{itemize}
  \item[(ii)] \textbf{First gap blocking completion:}
  No argument is provided that constructs such configurations for arbitrarily large $n$ or rules them out beyond some $n_0$. In particular, the case $n=8$ is not resolved here.
  \item[(iii)] \textbf{Most promising next moves:}
  \begin{itemize}
    \item Attempt to identify an algebraic/parametric ``gluing'' operation that adds a new point at integer distances to an existing integral set while keeping general position.
    \item Pursue unconditional upper bounds using incidence geometry and rigidity of integer distance constraints (aiming to force collinearity/cocircularity when $|S|$ is large).
    \item Use modern structural theorems on integer distance sets (sparseness/structure dichotomies) to exclude large general-position configurations.
  \end{itemize}
  \item[(iv)] \textbf{Conjectured final answer and minimal counterexample shape:}
  I conjecture the answer is \textbf{NO} for all sufficiently large $n$ (i.e. there is a maximal size).
  A minimal counterexample to this conjecture would be a general-position integral set of size $8$; such a set would likely have very large diameter and live in a quadratic field $\mathbb{Q}(\sqrt{k})$ with a common ``characteristic'' shared by all triangles, as suggested by known constructions.
\end{enumerate}

\subsection*{7. Completion Estimate}
\textbf{35\%}. Small-$n$ existence is demonstrated explicitly (including $n=4$ by hand and a recorded $n=7$ example), but the main open direction---existence or nonexistence for unbounded $n$---remains untouched at a structural level.


% =====================================================================

