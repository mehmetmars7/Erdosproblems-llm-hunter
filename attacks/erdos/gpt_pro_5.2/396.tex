
1) FORMAL RESTATEMENT

For integers $n\ge 1$ define the central binomial coefficient
\[
\binom{2n}{n}=\frac{(2n)!}{(n!)^2}.
\]

Fix an integer $k\ge 0$.

Question.
Does there exist an integer $n\ge k+1$ such that
\[
\prod_{i=0}^{k} (n-i) \mid \binom{2n}{n}?
\]
(We restrict to $n\ge k+1$ so the product is a nonzero integer.)

Edge cases.
For $k=0$, the question is whether there exists $n$ with $n\mid\binom{2n}{n}$.

2) QUICK LITERATURE/CONTEXT CHECK

I do not claim any results beyond what the problem statement itself records.

The statement records:
- $n+1\mid\binom{2n}{n}$ for all $n$ (Catalan integrality).
- Pomerance proved: for each $k\ge 0$, infinitely many $n$ satisfy $(n-k)\mid\binom{2n}{n}$, but this set has upper density $<1/3$.
- Pomerance also proved: for each fixed $k\ge 1$, the set of $n$ with $\prod_{1\le i\le k}(n+i)\mid\binom{2n}{n}$ has density $1$.

3) ATTACK PLAN

Proof-track (existential construction):
- Rewrite the divisibility condition prime-by-prime using $p$-adic valuations (Legendre formula).
- Try to choose $n$ with unusually large $v_p\!\left(\binom{2n}{n}\right)$ for small primes and controlled $v_p(n-i)$.

Disproof-track:
- Try to find a specific small $k$ (e.g. $k=2$) where computation suggests impossibility, then attempt to prove impossibility via a $p$-adic obstruction.

Chosen path: derive exact $p$-adic criterion and compute small cases for $k=0,1,2$.

4) WORK

PHASE 1 — FAST REALITY CHECK (computed)

(1) Solutions to $n\mid\binom{2n}{n}$ for $n\le 200$.
By direct computation, the integers $n\le 200$ such that $n\mid\binom{2n}{n}$ are
\[
1,2,6,15,20,28,42,45,66,77,88,91,104,110,126,140,153,156,170,187,190.
\]

(2) Solutions to $n(n-1)\mid\binom{2n}{n}$ for $n\le 200$.
By direct computation, the only $n\le 200$ satisfying $n(n-1)\mid\binom{2n}{n}$ is $n=2$.

(3) For $k=2$, i.e. $n(n-1)(n-2)\mid\binom{2n}{n}$, there is no solution for $n\le 200$.

Lemma 1 (Catalan divisibility: $(n+1)\mid\binom{2n}{n}$).
For every integer $n\ge 0$,
\[
(n+1)\mid \binom{2n}{n}.
\]

Proof.
Consider the adjacent binomial coefficient
\[
\binom{2n}{n+1}=\frac{(2n)!}{(n+1)!(n-1)!}.
\]
A direct ratio computation gives
\[
\binom{2n}{n+1}=\binom{2n}{n}\cdot \frac{n}{n+1}.
\]
Rearranging,
\[
\binom{2n}{n}-\binom{2n}{n+1}=\binom{2n}{n}\left(1-\frac{n}{n+1}\right)=\frac{1}{n+1}\binom{2n}{n}.
\]
The left-hand side is an integer difference of integers, hence is an integer. Therefore $\binom{2n}{n}/(n+1)$ is an integer, i.e. $(n+1)\mid\binom{2n}{n}$. \qed

Lemma 2 ($p$-adic valuation formula and divisibility criterion).
Let $p$ be prime and $n\ge 1$. Then
\[
 v_p\!\left(\binom{2n}{n}\right)=\sum_{j\ge 1}\Bigl(\bigl\lfloor\tfrac{2n}{p^j}\bigr\rfloor-2\bigl\lfloor\tfrac{n}{p^j}\bigr\rfloor\Bigr).
\]
Moreover, for fixed $k\ge 0$ and $n\ge k+1$,
\[
\prod_{i=0}^{k}(n-i)\mid\binom{2n}{n}
\]
holds if and only if for every prime $p$,
\[
\sum_{i=0}^{k} v_p(n-i)\le v_p\!\left(\binom{2n}{n}\right).
\]

Proof.
Legendre's formula for factorial valuations states
\[
 v_p(m!)=\sum_{j\ge 1}\Bigl\lfloor\frac{m}{p^j}\Bigr\rfloor.
\]
Applying this to $(2n)!$ and $n!$ gives
\[
 v_p\!\left(\binom{2n}{n}\right)=v_p((2n)!)-2v_p(n!)
 =\sum_{j\ge 1}\Bigl(\bigl\lfloor\tfrac{2n}{p^j}\bigr\rfloor-2\bigl\lfloor\tfrac{n}{p^j}\bigr\rfloor\Bigr).
\]

For the divisibility criterion, note that for any nonzero integer $M$, $M\mid\binom{2n}{n}$ is equivalent to $v_p(M)\le v_p\!\left(\binom{2n}{n}\right)$ for all primes $p$.
Here $M=\prod_{i=0}^k (n-i)$, and by additivity of $v_p$ on products,
\[
 v_p(M)=\sum_{i=0}^{k} v_p(n-i).
\]
This gives the equivalence. \qed

Lemma 3 (prime obstruction for $k=0$ at $n=p$).
If $p$ is an odd prime, then
\[
 p\nmid \binom{2p}{p}.
\]

Proof.
Work modulo $p$.
First, for $1\le j\le p-1$, the binomial coefficient $\binom{p}{j}$ is divisible by $p$, because
\[
\binom{p}{j}=\frac{p(p-1)\cdots(p-j+1)}{j!}
\]
has a factor of $p$ in the numerator and $j!$ is not divisible by $p$.
Hence in the binomial expansion
\[
(1+x)^p=\sum_{j=0}^p \binom{p}{j}x^j,
\]
all intermediate coefficients vanish mod $p$, so
\[
(1+x)^p\equiv 1+x^p\pmod p.
\]
Squaring both sides yields
\[
(1+x)^{2p}=\bigl((1+x)^p\bigr)^2\equiv (1+x^p)^2\equiv 1+2x^p+x^{2p}\pmod p.
\]
On the other hand,
\[
(1+x)^{2p}=\sum_{j=0}^{2p}\binom{2p}{j}x^j.
\]
Comparing the coefficient of $x^p$ on both sides modulo $p$ gives
\[
\binom{2p}{p}\equiv 2 \pmod p.
\]
Since $p$ is odd, $2\not\equiv 0\pmod p$, so $p\nmid \binom{2p}{p}$. \qed

5) VERIFICATION

- Lemma 1: checked numerically for $n\le 200$ as part of the computation.
- Lemma 2: The valuation identity is a direct consequence of Legendre's formula; the second statement is a standard equivalence between divisibility and prime-by-prime valuations.
- Lemma 3: For $p=3$, the lemma predicts $3\nmid\binom{6}{3}=20$, which holds.

6) FINAL

**UNRESOLVED**

(i) Strongest fully proved partial result obtained here.
- For all $n$, $(n+1)\mid\binom{2n}{n}$ (Lemma 1).
- Exact prime-by-prime criterion (Lemma 2).
- For odd primes $p$, $p\nmid\binom{2p}{p}$, so $n=p$ can never solve the $k=0$ case (Lemma 3).
- Computation: for $k=2$ there is no solution with $n\le 200$; for $k=1$ only $n=2$ works up to $200$.

(ii) Exact first gap.
Even the first genuinely new case $k=2$ (existence of some $n$ with $n(n-1)(n-2)\mid\binom{2n}{n}$) is not resolved by the arguments here.

(iii) Top 3 next moves (concrete targets).
1. Use Lemma 2 to search for $n$ where $v_p\!\left(\binom{2n}{n}\right)$ is unusually large for small primes $p$ while the product $\prod_{i=0}^k(n-i)$ has controlled $p$-adic size.
2. Try to leverage carry-counting interpretations of $v_p\!\left(\binom{2n}{n}\right)$ (equivalently, the sum formula in Lemma 2) to design $n$ with large valuations.
3. Extend computation to larger $n$ for small $k$ using a valuation-based check (avoiding huge integers) to search for a first example or to build evidence for a $k$-counterexample.

(iv) What a minimal counterexample would likely look like.
A counterexample would be a smallest $k\ge 2$ such that for every $n\ge k+1$ there exists a prime $p$ with
\[
\sum_{i=0}^{k} v_p(n-i) > v_p\!\left(\binom{2n}{n}\right).
\]
Such a $p$ would likely be a small prime that divides many of the consecutive factors $n,n-1,\dots,n-k$ with a larger total exponent than the central binomial coefficient can supply.


