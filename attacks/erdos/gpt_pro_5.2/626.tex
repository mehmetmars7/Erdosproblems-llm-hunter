\section*{Problem 626}

\subsection*{1) FORMAL RESTATEMENT}
Throughout, graphs are finite, simple, undirected.  The \emph{girth} $\operatorname{girth}(G)$ is the length of the shortest cycle in $G$, with $\operatorname{girth}(G)=\infty$ if $G$ is acyclic.  The chromatic number is $\chi(G)$.
We take $\log$ to be the natural logarithm.

Fix an integer $k\ge 4$.  For each $n\in\mathbb{N}$ define
\[
g_k(n)\ :=\ \max\Bigl\{\, m\in\mathbb{N}\ :\ \exists\text{ graph }G\text{ with }|V(G)|=n,\ \chi(G)=k,\ \operatorname{girth}(G)>m\,\Bigr\}.
\]
Equivalently, $g_k(n)=\max\{\operatorname{girth}(G)-1:\ |V(G)|=n,\ \chi(G)=k\}$, since $k\ge4$ forces $\operatorname{girth}(G)<\infty$.

\medskip
\noindent\textbf{Question A.} Does the limit
\[
\lim_{n\to\infty}\frac{g_k(n)}{\log n}
\]
exist (as a finite real number), and if so what is its value?

\medskip
Fix an integer $m\ge 3$.  For each $n\in\mathbb{N}$ define
\[
h^{(m)}(n)\ :=\ \max\Bigl\{\,\chi(G)\ :\ \exists\text{ graph }G\text{ with }|V(G)|=n,\ \operatorname{girth}(G)>m\,\Bigr\}.
\]

\medskip
\noindent\textbf{Question B.} Does the limit
\[
\lim_{n\to\infty}\frac{\log h^{(m)}(n)}{\log n}
\]
exist, and if so what is its value?

\subsection*{2) QUICK LITERATURE/CONTEXT CHECK}
Web browsing available: YES. Computation available: YES.

The Erd\H{o}s Problems site lists this problem as open, with the following known bounds (for fixed $k\ge4$ and all sufficiently large $n$):
\[
\frac{1}{4\log k}\,\log n\ \le\ g_k(n)\ \le\ \frac{2}{\log(k-2)}\,\log n + 1,
\]
with the lower bound attributed to Kostochka and the upper bound to Erd\H{o}s.

For $h^{(m)}(n)$, Erd\H{o}s proved a general lower bound
$\displaystyle \lim_{n\to\infty}\frac{\log h^{(m)}(n)}{\log n}\gg \frac{1}{m}$,
and for \emph{odd} $m$ an upper bound
$\displaystyle \lim_{n\to\infty}\frac{\log h^{(m)}(n)}{\log n}\le \frac{2}{m+1}$,
and conjectured this upper bound is sharp for odd $m$.  For even $m$ Erd\H{o}s suggested the limit should lie in $[2/(m+2),\,2/m]$ but could not prove this even for $m=4$.

\subsection*{3) ATTACK PLAN}
I attempted the following ``meta'' approaches, aiming either to (i) force the limit to exist by a general regularity principle, or (ii) exhibit mechanisms that could make the limit fail to exist.

\begin{enumerate}[label=\textbf{(\Alph*)},leftmargin=3.2em]
\item \textbf{Invert the extremal functions.}
Define the inverse threshold function
$N_k(M):=\min\{n:\exists\,G\text{ on }n\text{ vertices with }\chi(G)=k\text{ and }\operatorname{girth}(G)>M\}$.
If $\log N_k(M)$ grows linearly in $M$ with an asymptotic slope, then $g_k(n)$ is asymptotically that slope's reciprocal times $\log n$.
This reduces existence of $\lim g_k(n)/\log n$ to existence of $\lim \log N_k(M)/M$.

Similarly, define $N^{(m)}(t):=\min\{n:\exists\,G\text{ on }n\text{ vertices with }\operatorname{girth}(G)>m\text{ and }\chi(G)\ge t\}$.
If $N^{(m)}(t)$ is regularly varying (power law), then $\log h^{(m)}(n)/\log n$ has a limit, and vice versa.
So one can try to prove ``regular variation'' for these inverse functions.
\item \textbf{Try to prove submultiplicativity/supermultiplicativity for $N_k(M)$.}
If one could build from two graphs (with chromatic number $k$ and large girth) a new graph whose vertex count is roughly the product while girth adds, one would get a submultiplicativity statement
$N_k(M_1+M_2)\ \le\ C\,N_k(M_1)N_k(M_2)$
for some constant $C$, and hence by Fekete's lemma a limit for $\log N_k(M)/M$.
I do not know an operation that simultaneously preserves \emph{exact} chromatic number $k$ and yields additive girth in such a clean way.
\item \textbf{Oscillation search.}
To show non-existence, one would want two infinite sequences $n_j,n'_j\to\infty$ such that $g_k(n_j)\sim c_1\log n_j$ but $g_k(n'_j)\sim c_2\log n'_j$ with $c_1\ne c_2$.  Since $g_k(n)$ is a \emph{maximum} over graphs, oscillation would have to come from the availability of much better constructions only at certain sizes $n$, in a way that cannot be ``filled in'' for intermediate $n$ (e.g. by adding isolated vertices).  I did not find a rigorous mechanism producing two different constants.
\end{enumerate}

\subsection*{4) WORK}
No complete solution (existence or non-existence of the limits) was found.  I record rigorous structural facts that reduce the limit question to an inverse-growth question, plus small-$n$ sanity checks.

\subsubsection*{4.1. Basic monotonicity}
\begin{lemma}
For fixed $k\ge4$, the function $n\mapsto g_k(n)$ is nondecreasing.
For fixed $m\ge3$, the function $n\mapsto h^{(m)}(n)$ is nondecreasing.
\end{lemma}
\begin{proof}
If $G$ is a witness for $g_k(n)$ (resp.\ $h^{(m)}(n)$), then adding isolated vertices increases $n$ without changing $\chi(G)$ or introducing cycles.  Hence the extremal value cannot decrease when $n$ increases.
\end{proof}

\subsubsection*{4.2. Inverse function viewpoint for $g_k(n)$}
Define
\[
N_k(M)\ :=\ \min\Bigl\{\,n\in\mathbb{N}\ :\ \exists\text{ graph }G\text{ on }n\text{ vertices with }\chi(G)=k,\ \operatorname{girth}(G)>M\,\Bigr\}.
\]
Then by definition,
\begin{equation}\label{eq:inv-rel}
g_k(n)\ =\ \max\{\,M\in\mathbb{N}: N_k(M)\le n\,\}.
\end{equation}

\begin{proposition}\label{prop:limit-transfer-gk}
Assume the limit
\[
L_k\ :=\ \lim_{M\to\infty}\frac{\log N_k(M)}{M}
\]
exists and satisfies $0<L_k<\infty$.  Then the limit
\[
\lim_{n\to\infty}\frac{g_k(n)}{\log n}
\]
exists and equals $1/L_k$.
\end{proposition}
\begin{proof}
Let $n\to\infty$, and set $M:=g_k(n)$.  By \eqref{eq:inv-rel} we have
\[
N_k(M)\ \le\ n\ <\ N_k(M+1).
\]
Taking logarithms,
\[
\log N_k(M)\ \le\ \log n\ <\ \log N_k(M+1).
\]
Divide through by $M$ (for large $n$ one has $M\to\infty$ because $g_k(n)\gg\log n$ from known bounds, but even without that, one can restrict to $n$ for which $M\ge1$):
\[
\frac{\log N_k(M)}{M}\ \le\ \frac{\log n}{M}\ <\ \frac{\log N_k(M+1)}{M}.
\]
By the assumed limit for $\log N_k(\cdot)/(\cdot)$,
\[
\frac{\log N_k(M)}{M}\to L_k,\qquad
\frac{\log N_k(M+1)}{M}=\frac{M+1}{M}\cdot\frac{\log N_k(M+1)}{M+1}\to 1\cdot L_k=L_k.
\]
Hence $\log n/M\to L_k$, i.e.\ $M/\log n\to 1/L_k$, proving the claim.
\end{proof}

\subsubsection*{4.3. Inverse function viewpoint for $h^{(m)}(n)$}
Fix $m\ge3$ and define
\[
N^{(m)}(t)\ :=\ \min\Bigl\{\,n\in\mathbb{N}\ :\ \exists\text{ graph }G\text{ on }n\text{ vertices with }\operatorname{girth}(G)>m,\ \chi(G)\ge t\,\Bigr\}.
\]
Then
\[
h^{(m)}(n)\ =\ \max\{\,t\in\mathbb{N}: N^{(m)}(t)\le n\,\}.
\]

\begin{proposition}\label{prop:limit-transfer-hm}
Assume the limit
\[
A_m\ :=\ \lim_{t\to\infty}\frac{\log N^{(m)}(t)}{\log t}
\]
exists and satisfies $0<A_m<\infty$.  Then the limit
\[
\lim_{n\to\infty}\frac{\log h^{(m)}(n)}{\log n}
\]
exists and equals $1/A_m$.
\end{proposition}
\begin{proof}
Let $n\to\infty$ and set $t:=h^{(m)}(n)$.  Then $N^{(m)}(t)\le n< N^{(m)}(t+1)$, so
\[
\log N^{(m)}(t)\ \le\ \log n\ <\ \log N^{(m)}(t+1).
\]
Divide by $\log t$ (for $t\ge2$):
\[
\frac{\log N^{(m)}(t)}{\log t}\ \le\ \frac{\log n}{\log t}\ <\ \frac{\log N^{(m)}(t+1)}{\log t}.
\]
By hypothesis, $\frac{\log N^{(m)}(t)}{\log t}\to A_m$.  Also
\[
\frac{\log N^{(m)}(t+1)}{\log t}
=\frac{\log N^{(m)}(t+1)}{\log(t+1)}\cdot\frac{\log(t+1)}{\log t}\ \to\ A_m\cdot 1=A_m.
\]
Hence $\log n/\log t\to A_m$, i.e.\ $\log t/\log n\to 1/A_m$, proving the claim.
\end{proof}

\subsubsection*{4.4. Small-$n$ sanity checks (exhaustive search for $n\le 6$)}
I exhaustively enumerated all labeled graphs on $n\le 6$ vertices and computed $\chi(G)$, $\omega(G)$, and $\operatorname{girth}(G)$ by brute force.

\begin{itemize}
\item For $k=4$ one finds $g_4(4)=g_4(5)=g_4(6)=2$, i.e.\ every $4$-chromatic graph on at most $6$ vertices contains a triangle (so $\operatorname{girth}=3$).
\item For $m=3$ (triangle-free graphs), one finds $h^{(3)}(5)=h^{(3)}(6)=3$, witnessed by $C_5$ (and $C_5$ plus an isolated vertex).
\end{itemize}
These checks are consistent with the definitions and with the general phenomenon that forcing larger girth rapidly forces larger $n$ for fixed chromatic number.

\subsection*{5) VERIFICATION}
Propositions \ref{prop:limit-transfer-gk} and \ref{prop:limit-transfer-hm} were checked carefully: they use only the extremal definitions plus monotonicity of the inverse thresholds and basic limit algebra.  The exhaustive $n\le 6$ computations serve as a consistency check of the definitions (not as evidence for asymptotics).

\subsection*{6) FINAL}
\textbf{UNRESOLVED.}

\begin{enumerate}[label=\textbf{(\roman*)},leftmargin=2.6em]
\item The question asks for existence (and value) of two asymptotic limits.  Current known bounds leave a constant-factor gap for $g_k(n)/\log n$ and an exponent gap for $\log h^{(m)}(n)/\log n$ (especially for even $m$).
\item I proved that existence of $\lim g_k(n)/\log n$ would follow from existence of $\lim \log N_k(M)/M$ for the inverse threshold $N_k(M)$, and similarly for $h^{(m)}$ via $N^{(m)}(t)$.
\item I did not find (a) a construction proving the limits exist, nor (b) a mechanism forcing non-existence by oscillation.
\item Any full resolution likely requires either new ``product'' constructions/subadditivity for $N_k(M)$ or sharp matching upper/lower constructions for $h^{(m)}(n)$.
\end{enumerate}

\subsection*{7) COMPLETION ESTIMATE}
\textbf{COMPLETION: 25\%.}

