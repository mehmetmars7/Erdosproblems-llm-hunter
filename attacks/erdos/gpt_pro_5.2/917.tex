
\noindent\textbf{FORMAL RESTATEMENT.}\\
Fix an integer $k\ge 4$. Call a graph $G$ \emph{$k$-critical (edge-critical)} if $\chi(G)=k$ and for every edge $e\in E(G)$ we have $\chi(G-e)<k$. For $n\ge 1$ define
\[
 f_k(n):=\max\{|E(G)|: |V(G)|=n\ \text{and $G$ is $k$-critical}\}.
\]
The questions ask:
\begin{enumerate}
\item[(Q1)] Is it true that $f_k(n)\gg_k n^2$?
\item[(Q2)] Is it true that $f_6(n)\sim n^2/4$?
\item[(Q3)] More generally, for $k\ge 6$, is it true that
\[ f_k(n) \sim \frac12\Big(1-\frac{1}{\lfloor k/3\rfloor}\Big)n^2\ ?\]
\end{enumerate}

\medskip
\noindent\textbf{QUICK LITERATURE/CONTEXT CHECK.}\\
The problem text states several known results (Dirac's construction and bounds of Toft/Stiebitz/Luo--Ma--Yang). I will not import those papers; instead I will fully verify the explicit Dirac-type construction given in the problem text and extract concrete lower bounds from it.

\medskip
\noindent\textbf{ATTACK PLAN.}\\
\emph{Construction track.} Verify that the join of odd cycles gives $k$-critical graphs with quadratic edge count for $k$ divisible by $3$ (Dirac/Erd\H{o}s construction). Generalize to $k=3t$.

\smallskip
\noindent\emph{Asymptotic track.} The conjectured exact asymptotics (Q2--Q3) are not resolved here; we instead record the explicit lower bound constructions.

\medskip
\noindent\textbf{WORK.}\\
\emph{Phase 1: fast reality check (computation on the explicit construction).}  Let $m=2$ so the odd cycles are $C_5$. I constructed the graph obtained by taking two disjoint copies of $C_5$ and adding all edges between them (the Dirac construction). A brute-force backtracking coloring check gives $\chi=6$ and, for every edge $e$, $\chi(G-e)\le 5$ (so the graph is $6$-critical). For $m=1$ the construction is $K_6$, which is also $6$-critical.

\medskip
\noindent\textbf{Lemma 917.1 (Dirac construction for $k=6$ is edge-critical and dense).}  Fix $m\ge 1$ and let $A,B$ be disjoint vertex sets with $|A|=|B|=2m+1$. Let $G$ be the graph obtained by:
\begin{itemize}
\item adding edges of a cycle $C_{2m+1}$ on $A$,
\item adding edges of a cycle $C_{2m+1}$ on $B$,
\item adding all edges between $A$ and $B$ (i.e. making $(A,B)$ complete bipartite).
\end{itemize}
Then $|V(G)|=4m+2$,
\[
|E(G)|=(2m+1)^2+2(2m+1)=4m^2+8m+3,
\]
and $G$ is $6$-critical.

\noindent\textbf{Proof.}
\emph{Step 1: edge count.} Each cycle contributes $2m+1$ edges, so internal edges total $2(2m+1)$. The complete bipartite part contributes $(2m+1)(2m+1)=(2m+1)^2$ edges. Summing gives $|E(G)|=(2m+1)^2+2(2m+1)=4m^2+8m+3$.

\emph{Step 2: $\chi(G)=6$.} Each odd cycle $C_{2m+1}$ has chromatic number $3$, so $\chi(G)\ge 3$ on $A$ and on $B$. Because every vertex in $A$ is adjacent to every vertex in $B$, no color used on $A$ can be reused on $B$. Therefore any proper coloring uses at least $3+3=6$ colors, so $\chi(G)\ge 6$.

Conversely, a proper $6$-coloring exists: color the cycle on $A$ with colors $1,2,3$ and color the cycle on $B$ with colors $4,5,6$. Cross edges only connect vertices in different parts, so there is no color conflict. Thus $\chi(G)\le 6$, hence $\chi(G)=6$.

\emph{Step 3: deleting any edge reduces the chromatic number.}

\smallskip
\noindent\underline{Case 3a: $e$ is an edge within $A$ (or within $B$).}  Removing an edge from an odd cycle breaks it into a path, which is bipartite and thus $2$-colorable. Concretely, $C_{2m+1}-e$ is a path on $2m+1$ vertices, hence has $\chi=2$. The other part is still an odd cycle with $\chi=3$. As above, cross edges force disjoint color sets between the parts, so
\[
\chi(G-e) \le 2+3=5<6.
\]

\smallskip
\noindent\underline{Case 3b: $e$ is a cross edge between $A$ and $B$.}  Write $e=uv$ with $u\in A$ and $v\in B$. I will exhibit a proper $5$-coloring of $G-e$.

First, 3-color the cycle on $A$ with colors $1,2,3$ in such a way that $u$ is the \emph{only} vertex of color $1$. This is possible: label the vertices of $A$ cyclically as $u=a_0,a_1,\dots,a_{2m}$ and assign colors
\[
\mathrm{col}(a_0)=1,\qquad \mathrm{col}(a_i)=\begin{cases}2 & i\text{ odd},\\ 3 & i\text{ even and }i\ge 2.
\end{cases}
\]
One checks that adjacent vertices receive different colors and that color $1$ appears exactly once.

Second, 3-color the cycle on $B$ with colors $1,4,5$ in such a way that $v$ is the only vertex of color $1$ (use the same pattern with the palette $\{1,4,5\}$).

In $G-e$, the only missing cross adjacency is between $u$ and $v$. Every other pair $(a\in A, b\in B)$ remains adjacent. Since color $1$ appears only on $u$ in $A$ and only on $v$ in $B$, the only potential cross-part color conflict for color $1$ would be between $u$ and $v$, but that edge was deleted. All other colors are disjoint between parts. Hence this is a proper coloring of $G-e$ using the five colors $\{1,2,3,4,5\}$, so $\chi(G-e)\le 5<6$.

Since every edge falls into one of these cases, deleting any edge reduces the chromatic number below $6$, i.e. $G$ is $6$-critical. \hfill$\square$

\medskip
\noindent\textbf{Lemma 917.2 (generalisation to $k=3t$ by joining $t$ odd cycles).}  Fix integers $t\ge 2$ and $m\ge 1$. Let $G$ be the join of $t$ disjoint copies of the odd cycle $C_{2m+1}$ (i.e. take $t$ disjoint cycles and add all edges between any two distinct cycles). Then:
\begin{enumerate}
\item[(a)] $\chi(G)=3t$ and $G$ is $(3t)$-critical.
\item[(b)] Writing $n:=|V(G)|=t(2m+1)$, we have
\[
|E(G)|=\frac12\Big(1-\frac{1}{t}\Big)n^2 + n.
\]
\end{enumerate}

\noindent\textbf{Proof.}
\emph{Step 1: chromatic number.} Each cycle has chromatic number $3$. Since every vertex in one cycle is adjacent to every vertex in another, the color sets used on distinct cycles must be disjoint. Hence $\chi(G)\ge 3t$. Conversely, 3-color each cycle and use disjoint color palettes for different cycles; this gives a proper $3t$-coloring. Thus $\chi(G)=3t$.

\emph{Step 2: edge count.} There are $t$ cycles, each contributing $2m+1$ internal edges, for a total of $t(2m+1)=n$ internal edges. For cross edges: between each pair of distinct cycles (there are $\binom{t}{2}$ such pairs) we have a complete bipartite graph on $(2m+1,2m+1)$, contributing $(2m+1)^2$ edges per pair. Thus cross edges total $\binom{t}{2}(2m+1)^2$. Using $n=t(2m+1)$ gives
\[
\binom{t}{2}(2m+1)^2 = \frac{t(t-1)}{2}\cdot \frac{n^2}{t^2}=\frac12\Big(1-\frac1t\Big)n^2,
\]
so $|E(G)|=n+\frac12(1-1/t)n^2$.

\emph{Step 3: edge-criticality.} Let $e$ be an arbitrary edge.

\smallskip
\noindent\underline{Case 3a: $e$ is within one cycle.} Removing $e$ makes that cycle a path (2-colorable) while all other cycles remain 3-colorable. Because join edges remain between all different parts, a proper coloring may be taken by using 2 colors on the broken cycle and 3 fresh colors on each of the other $t-1$ cycles, for a total of $2+3(t-1)=3t-1$ colors. Hence $\chi(G-e)\le 3t-1<3t$.

\smallskip
\noindent\underline{Case 3b: $e$ is between two cycles.} Suppose $e=uv$ with $u$ in cycle $A$ and $v$ in cycle $B$, where $A\ne B$. Color every cycle other than $A$ and $B$ with its own disjoint palette of $3$ colors.

Now color $A$ with colors $1,2,3$ so that $u$ is the unique vertex of color $1$ (the same explicit pattern as in Lemma~917.1). Color $B$ with colors $1,4,5$ so that $v$ is the unique vertex of color $1$. Since the only missing adjacency between the parts $A$ and $B$ is the deleted edge $uv$, and color $1$ appears only at $u$ in $A$ and only at $v$ in $B$, there is no conflict. All other cycles use disjoint palettes, so there are no cross-cycle conflicts.

Counting colors: the pair $A,B$ uses $\{1,2,3,4,5\}$ (5 colors), and each of the remaining $t-2$ cycles uses 3 fresh colors, for a total of $5+3(t-2)=3t-1$ colors. Thus $\chi(G-e)\le 3t-1<3t$.

This holds for every edge $e$, so $G$ is $(3t)$-critical. \hfill$\square$

\medskip
\noindent\textbf{VERIFICATION.}\\
\begin{itemize}
\item Lemma~917.1 and Lemma~917.2 give explicit colorings after edge deletion; no appeal to external theorems is needed.
\item The ``unique color'' 3-coloring pattern of an odd cycle is checked by direct adjacency along the cyclic order.
\item Computation for $m=1,2,3$ (two-cycle join) confirmed $\chi=6$ and edge-criticality for small sizes.
\end{itemize}

\medskip
\noindent\textbf{FINAL.}\\
\textbf{UNRESOLVED.}

(i) Strongest proved partial result: explicit quadratic lower bounds for infinitely many $n$ via Lemma~917.2. In particular, when $k=3t$ and $n=t(2m+1)$,
\[
 f_k(n)\ge \frac12\Big(1-\frac{1}{t}\Big)n^2+n.
\]
For $k=6$ this recovers Dirac's construction (Lemma~917.1) with $|E|=4m^2+8m+3$ on $n=4m+2$ vertices, i.e. $|E|=n^2/4+n+\tfrac12$.

(ii) First gap (crisp): Determine whether, for each fixed $k\ge 4$, there is a constant $c_k>0$ such that $f_k(n)\ge c_k n^2$ for \emph{all} sufficiently large $n$, and determine the true asymptotic constant(s) in (Q2)--(Q3) (even for $k=6$).

(iii) Top 3 next moves:
\begin{enumerate}
\item Extend the explicit join construction to \emph{all} large $n$ (not only those representable as a sum of $t$ odd cycle lengths) while preserving $k$-criticality, e.g. via a controlled criticality-preserving composition operation.
\item For $k=6$, attempt to prove an upper bound $f_6(n)\le (1/4+o(1))n^2$ or else construct $6$-critical graphs with density strictly exceeding $1/4$.
\item For general $k$, study whether the ``join of color-critical pieces'' paradigm is asymptotically optimal, and if not, identify a concrete local gadget that increases edge density while preserving edge-criticality.
\end{enumerate}

(iv) Minimal counterexample structure (to the conjectured asymptotic for a given $k$): a sequence of $k$-critical graphs on $n$ vertices whose edge density exceeds the conjectured constant by a fixed $\varepsilon>0$ infinitely often (to disprove), or conversely an argument showing that any $k$-critical graph with near-extremal edge count must resemble a complete multipartite join of a bounded number of almost-3-chromatic blocks (to prove).

