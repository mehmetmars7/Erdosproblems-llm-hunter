% Erdos solutions generated from 1014-1021.tex

% Erdos Problem #1014

\noindent\textbf{FORMAL RESTATEMENT.}
Fix an integer $k\ge 3$. For each integer $\ell\ge 1$, let $R(k,\ell)$ be the least integer $n$ such that every (simple, undirected) graph $G$ on $n$ vertices contains either a clique $K_k$ or an independent set of size $\ell$. The problem asks to prove that
\[
\lim_{\ell\to\infty} \frac{R(k,\ell+1)}{R(k,\ell)} = 1.
\]
Equivalently, one wants $R(k,\ell+1)-R(k,\ell)=o(R(k,\ell))$ as $\ell\to\infty$.

\medskip
\noindent\textbf{QUICK LITERATURE/CONTEXT CHECK.}
The problem statement itself says this limit is open even for $k=3$.

\medskip
\noindent\textbf{ATTACK PLAN.}
\emph{Proof track:} (1) Use the standard Ramsey recursion to relate $R(k,\ell+1)$ and $R(k,\ell)$; try to show the increment is much smaller than $R(k,\ell)$. (2) Seek a comparison of $R(k-1,\ell)$ with $R(k,\ell)$ for fixed $k$.

\emph{Disproof track:} Try to force a multiplicative jump $R(k,\ell+1)\ge (1+\varepsilon)R(k,\ell)$ for infinitely many $\ell$ by constructing near-extremal $K_k$-free graphs whose independence number changes abruptly with $\ell$.

\medskip
\noindent\textbf{WORK.}

\noindent\textbf{Lemma 1 (Ramsey recursion).}
For integers $k,\ell\ge 2$,
\[
R(k,\ell) \le R(k-1,\ell) + R(k,\ell-1).
\]
\textit{Proof.}
Let $n:=R(k-1,\ell)+R(k,\ell-1)$ and let $G$ be any graph on $n$ vertices.
Pick a vertex $v\in V(G)$ and let $N(v)$ be its neighborhood and $\overline N(v)$ its non-neighborhood (excluding $v$).
Then $|N(v)|+|\overline N(v)|=n-1$.
If $|N(v)|\ge R(k-1,\ell)$ then the induced subgraph $G[N(v)]$ contains either a $K_{k-1}$, in which case $G$ contains a $K_k$ using $v$, or an independent set of size $\ell$, which is also independent in $G$.
If instead $|N(v)|<R(k-1,\ell)$, then $|\overline N(v)|\ge n-1-(R(k-1,\ell)-1)=R(k,\ell-1)$.
In that case, the induced subgraph on $\overline N(v)$ contains either a $K_k$ (hence $G$ does), or an independent set of size $\ell-1$; adjoining $v$ gives an independent set of size $\ell$ in $G$.
Thus every graph on $n$ vertices contains a $K_k$ or an independent $\ell$-set, so $R(k,\ell)\le n$.
\hfill$\square$

\medskip
\noindent\textbf{Lemma 2 (a conditional reduction to lower parameters).}
Fix $k\ge 3$. Suppose that as $\ell\to\infty$ both
\[
\frac{R(k-1,\ell+1)}{R(k-1,\ell)}\to 1
\qquad\text{and}\qquad
\frac{R(k-1,\ell)}{R(k,\ell)}\to 0.
\]
Then $\displaystyle \frac{R(k,\ell+1)}{R(k,\ell)}\to 1$.

\textit{Proof.}
Monotonicity of Ramsey numbers in the second argument gives $R(k,\ell+1)\ge R(k,\ell)$, hence
\[\liminf_{\ell\to\infty}\frac{R(k,\ell+1)}{R(k,\ell)}\ge 1.\]
Applying Lemma 1 with $(k,\ell+1)$ gives
\[
R(k,\ell+1) \le R(k,\ell) + R(k-1,\ell+1).
\]
Divide by $R(k,\ell)$:
\[
\frac{R(k,\ell+1)}{R(k,\ell)} \le 1 + \frac{R(k-1,\ell+1)}{R(k,\ell)}
= 1 + \frac{R(k-1,\ell+1)}{R(k-1,\ell)}\cdot\frac{R(k-1,\ell)}{R(k,\ell)}.
\]
By the two hypotheses, the product tends to $1\cdot 0=0$, so the right-hand side tends to $1$. Therefore
\[\limsup_{\ell\to\infty}\frac{R(k,\ell+1)}{R(k,\ell)}\le 1.\]
Together with the $\liminf\ge 1$ this implies the desired limit $=1$.
\hfill$\square$

\medskip
\noindent\textbf{FAST REALITY CHECK.}
A brute-force check for $(k,\ell)=(3,3)$ confirms $R(3,3)=6$.
(Enumeration of all graphs on $n\le 6$ vertices shows that $n=5$ is not forced, while $n=6$ is forced.)

\medskip
\noindent\textbf{VERIFICATION.}
Lemma 1 uses only the partition of vertices into neighbors/non-neighbors of a fixed vertex and the defining property of Ramsey numbers; all quantifiers are explicit and no extra hypotheses (connectivity, etc.) are used.
Lemma 2 is a purely conditional implication; the only inputs are Lemma 1 and monotonicity $R(k,\ell+1)\ge R(k,\ell)$.

\medskip
\noindent\textbf{FINAL.}
\textbf{UNRESOLVED}

(i) Strongest proved partial result: Lemma 1 (Ramsey recursion) and Lemma 2 (a clean conditional reduction showing the ratio limit for parameter $k$ would follow from a ratio limit for $k-1$ and from $R(k-1,\ell)=o(R(k,\ell))$).

(ii) First gap (crisp): prove either $R(k-1,\ell)=o(R(k,\ell))$ for fixed $k$ or otherwise show directly that $R(k,\ell+1)-R(k,\ell)=o(R(k,\ell))$.

(iii) Top 3 next moves:
1. Prove quantitative separation $R(k,\ell)\ge \ell^{1+\varepsilon_k}\,R(k-1,\ell)$ (any fixed $\varepsilon_k>0$ would imply $R(k-1,\ell)=o(R(k,\ell))$).
2. Search for a subadditivity/smoothness inequality like $R(k,\ell+t)\le (1+o(1))R(k,\ell)$ for $t=o(\ell)$ via refined recursion.
3. Computationally search (small $\ell$) for whether $R(3,\ell+1)/R(3,\ell)$ ever exhibits large jumps, to guide potential counterexample heuristics.

(iv) Minimal counterexample structure: a fixed $k\ge 3$ and an $\varepsilon>0$ such that infinitely many $\ell$ satisfy $R(k,\ell+1)\ge (1+\varepsilon)R(k,\ell)$; this would require near-extremal $K_k$-free graphs on about $R(k,\ell)$ vertices whose independence number is tightly concentrated around $\ell$.


