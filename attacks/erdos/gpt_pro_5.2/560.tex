% Erdos Problem #560

1) FORMAL RESTATEMENT

For a graph $G$, the (2-colour) size Ramsey number $\hat R(G)$ is the least integer $m$ such that there exists a (finite simple) graph $H$ with $m$ edges satisfying:
for every red/blue colouring of $E(H)$, the graph $H$ contains a monochromatic copy of $G$.
We write this property as $H\to (G)_2$.

The problem asks to determine $\hat R(K_{n,n})$ for the complete bipartite graph with equal parts of size $n$.

2) QUICK LITERATURE/CONTEXT CHECK

The problem statement itself records known bounds for $n\ge 6$:
\[
\frac{1}{60}n^2 2^n < \hat R(K_{n,n}) < \frac{3}{2}n^3 2^n.
\]
No external improvements are assumed here.

3) ATTACK PLAN

Provide fully proved baseline bounds:
- Trivial lower bound from the fact $H$ must contain an uncoloured copy of $K_{n,n}$.
- Trivial upper bound using the usual Ramsey number $R(K_{n,n};2)$ by taking $H=K_{R(K_{n,n};2)}$.

Also sanity-check the smallest values (e.g. $n=1,2$).

4) WORK

PHASE 1 — FAST REALITY CHECK

- $n=1$: $K_{1,1}$ is a single edge, so $\hat R(K_{1,1})=1$ (take $H$ to be one edge).
- $n=2$: $K_{2,2}\cong C_4$. Since $R(C_4;2)=6$ (Problem 555), taking $H=K_6$ gives $\hat R(K_{2,2})\le \binom{6}{2}=15$.

Lemma 560.1 (Trivial edge-count lower bound).
For every graph $G$, $\hat R(G)\ge |E(G)|$.
In particular,
\[
\hat R(K_{n,n})\ge n^2.
\]

Proof.
Let $H$ be any graph with $H\to (G)_2$.
Consider the monochromatic colouring where every edge of $H$ is coloured red.
Then $H$ must contain a red copy of $G$, i.e. $H$ must contain $G$ as an (uncoloured) subgraph.
Therefore $|E(H)|\ge |E(G)|$.
Taking the minimum over all such $H$ gives $\hat R(G)\ge |E(G)|$.
For $G=K_{n,n}$, $|E(G)|=n^2$.
\qed

Lemma 560.2 (Upper bound via the usual Ramsey number).
For every graph $G$,
\[
\hat R(G)\le \binom{R(G;2)}{2}.
\]
In particular,
\[
\hat R(K_{n,n})\le \binom{R(K_{n,n};2)}{2}.
\]

Proof.
Let $R:=R(G;2)$, so by definition every red/blue colouring of $E(K_R)$ contains a monochromatic copy of $G$.
Take $H:=K_R$, which has $\binom{R}{2}$ edges.
Then $H\to (G)_2$, so by the definition of size Ramsey number,
\[
\hat R(G)\le |E(H)|=\binom{R}{2}.
\]
Applying this with $G=K_{n,n}$ gives the stated bound.
\qed

5) VERIFICATION

- Lemma 560.1: the all-red colouring is a legitimate 2-colouring; the definition of $\hat R(G)$ requires a monochromatic $G$ in every such colouring, forcing $G\subseteq H$.

- Lemma 560.2: the definition of $R(G;2)$ is exactly that $K_{R}$ arrows $G$ in 2 colours, so taking $H=K_R$ is valid.

- Reality check $n=1$ is immediate; for $n=2$ the computed $R(C_4;2)=6$ implies the stated upper bound 15.

6) FINAL

**UNRESOLVED**

(i) Strongest proved partial result:
- Always $n^2\le \hat R(K_{n,n})\le \binom{R(K_{n,n};2)}{2}$ (Lemmas 560.1–560.2).
- Explicit small case: $\hat R(K_{2,2})\le 15$ via $K_6$.

(ii) First gap (crisp):
Determine the correct order of magnitude (and ideally asymptotic constant factors) of $\hat R(K_{n,n})$; in particular, explain/prove the exponential-in-$n$ growth $\\sim n^{\\Theta(1)}2^n$ suggested by the bounds in the problem statement.

(iii) Top 3 next moves:
1. Attempt to reproduce the lower bound construction (or a simpler version) showing $\hat R(K_{n,n})\ge c n^2 2^n$ by probabilistic methods on sparse host graphs.
2. Attempt to build an explicit host graph $H$ with $O(n^3 2^n)$ edges that forces monochromatic $K_{n,n}$ under any 2-colouring, possibly using expansion or containers.
3. Compute $\hat R(K_{n,n})$ for $n=2,3$ via exhaustive search/SAT to guess the exact small values and identify extremal host graph shapes.

(iv) Minimal counterexample structure:
A minimal host graph $H$ beating the $n^2 2^n$ scale would likely be sparse but highly structured (e.g. bounded degree with good expansion) so that any red/blue colouring creates a large monochromatic biclique. Conversely, a minimal colouring avoiding $K_{n,n}$ on a candidate host graph would likely be pseudorandom with each colour class near the $K_{n,n}$-free extremal threshold.

