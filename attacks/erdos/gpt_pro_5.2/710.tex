% Erdos Problem #710

1) \textbf{FORMAL RESTATEMENT}

For each $n\ge 1$, let $f(n)$ be the minimal real number such that there exist \emph{distinct integers}
\[
a_1,a_2,\dots,a_n\in (n,\,n+f(n))
\]
with the divisibility constraints
\[
 k\mid a_k\qquad\text{for all }1\le k\le n.
\]
(There is a mild ambiguity about whether $f(n)$ is required to be an integer and about the open interval endpoints; in the computations below I interpret $f(n)$ as the minimal \emph{integer} such that the open interval $(n,n+f(n))$ contains a feasible set.)

2) \textbf{QUICK LITERATURE/CONTEXT CHECK}

Using only what is stated in the problem text: Erd\H{o}s--Pomerance proved
\[
(2/\sqrt{e}+o(1))n\left(\frac{\log n}{\log\log n}\right)^{1/2}\le f(n)\le (1.7398\cdots+o(1))n(\log n)^{1/2}.
\]

3) \textbf{ATTACK PLAN}

- Prove elementary lower bounds from the need to include a multiple of $n$.
- Prove elementary upper bounds by explicit constructions (very non-optimal).
- Compute small-$n$ exact values (in the integer/open-interval interpretation) as a sanity check.

4) \textbf{WORK}

\textbf{Lemma 710.1 (Basic lower bound $f(n)>n$; hence $f(n)\ge n+1$ if $f(n)$ is integral).}
For any $n\ge 1$, any feasible $a_n$ must satisfy $a_n\ge 2n$.
Consequently, if the feasible interval is $(n,n+f(n))$, then necessarily $n+f(n)>2n$, i.e. $f(n)>n$.

\textit{Proof.}
Because $n\mid a_n$ and $a_n>n$ (since $a_n\in(n,n+f(n))$), the smallest possible value of $a_n$ is the least multiple of $n$ strictly exceeding $n$, namely $2n$.
Thus any feasible solution requires $(n,n+f(n))$ to contain at least one multiple of $n$, hence must extend beyond $2n$.
Formally, if $n+f(n)\le 2n$ then $(n,n+f(n))\subseteq(n,2n]$ contains no multiple of $n$, contradiction.
Therefore $f(n)>n$.
\hfill$\square$

\textbf{Lemma 710.2 (Crude explicit upper bound $f(n)\le n(n+1)$).}
For every $n\ge 1$, there exist distinct integers $a_1,\dots,a_n$ with $k\mid a_k$ and
\[
a_k\in (n,\,n+n(n+1))\quad\text{for all }k,
\]
so in particular $f(n)\le n(n+1)$.

\textit{Proof.}
Set
\[
 a_k := k(n+1),\qquad 1\le k\le n.
\]
Then $k\mid a_k$ is immediate.
The values are distinct because $k\mapsto k(n+1)$ is injective.
Moreover, since $k\ge 1$ and $n\ge 1$, we have $a_k\ge n+1>n$.
Also $a_k\le n(n+1)$, hence $a_k < n + n(n+1)$.
Thus $a_k\in(n,\,n+n(n+1))$ for every $k$.
Therefore a feasible set exists in that interval, proving $f(n)\le n(n+1)$.
\hfill$\square$

\textbf{FAST REALITY CHECK (exact small $n$ in the integer/open-interval model).}
I computed (by Hopcroft--Karp bipartite matching) the minimal \emph{integer} $f(n)$ such that there exist distinct integers $a_k\in(n,n+f(n))$ with $k\mid a_k$.
Results for $1\le n\le 20$:
\begin{verbatim}
 n :  1  2  3  4  5  6  7  8  9  10 11 12 13 14 15 16 17 18 19 20
 f :  2  3  4  6  6  9  9 11 13 15 15 17 16 19 20 25 24 27 26 29
\end{verbatim}

5) \textbf{VERIFICATION}

- Lemma 710.1: checks the first available multiple of $n$ above $n$ is $2n$; correct.
- Lemma 710.2: divisibility and distinctness are immediate; interval containment checked by bounding $a_k$ between $n+1$ and $n(n+1)$.
- Computation: The matching formulation is: left nodes $k=1,\dots,n$, right nodes integers $x\in\{n+1,\dots,n+f(n)-1\}$ (since the interval is open), with edges when $k\mid x$. A perfect matching gives the desired distinct multiples.

6) \textbf{FINAL}

\textbf{UNRESOLVED}

(i) \textbf{Strongest proved partial result.}
I proved elementary bounds $f(n)>n$ (Lemma 710.1) and $f(n)\le n(n+1)$ (Lemma 710.2), and computed exact small values for $n\le 20$ in the integer/open-interval interpretation.

(ii) \textbf{First gap (crisp).}
Determine the true asymptotic growth of $f(n)$ (beyond the stated bounds in the problem text), i.e. find constants/functions $g(n)$ with $f(n)\sim g(n)$.

(iii) \textbf{Top 3 next moves.}
1. Try to sharpen the constructive upper bound by a greedy or matching-based algorithm that keeps $a_k$ near $2n$ while avoiding collisions.
2. For lower bounds, analyze the scarcity of numbers with large prime factors in short intervals: large $k$ force use of relatively rare multiples.
3. Extend computation to larger $n$ to guess the correct asymptotic scale (e.g. whether $f(n)/\bigl(n\sqrt{\log n}\bigr)$ stabilizes).

(iv) \textbf{Minimal counterexample structure.}
A minimal counterexample to a proposed upper bound $f(n)\le U(n)$ would be an $n$ such that the bipartite graph between $\{1,\dots,n\}$ and $(n,n+U(n))\cap\mathbb{Z}$ (edges given by divisibility) has no matching covering the left side; by Hall's theorem, there must exist a subset $S\subseteq\{1,\dots,n\}$ for which the set of multiples of elements of $S$ in the interval has size $<|S|$.
