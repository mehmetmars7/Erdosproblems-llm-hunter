\section*{Erd\H{o}s Problem 288}
\addcontentsline{toc}{section}{Erd\H{o}s Problem 288}

\subsection*{1) Formal Restatement}
\textbf{Literal statement (as in the prompt).}
Is it true that there are only finitely many pairs of intervals $I_1,I_2$ such that
\[
\sum_{n_1\in I_1}\frac{1}{n_1}+\sum_{n_2\in I_2}\frac{1}{n_2}\in\mathbb{N}?
\]

\textbf{Ambiguities to resolve.}
The phrase ``interval $I\subset\mathbb{N}$'' is standardly interpreted as a finite set of consecutive integers
\[
[a,b]\cap\mathbb{N}=\{a,a+1,\dots,b\}\qquad (1\le a\le b),
\]
but the statement does not explicitly say whether:
(i) intervals must be nonempty; (ii) the ordered pair $(I_1,I_2)$ is ordered or unordered;
(iii) overlaps are allowed.
None of these materially affects the problem's difficulty; below I take 
\emph{nonempty finite} integer intervals and treat $(I_1,I_2)$ as an \emph{ordered} pair.

\textbf{Minimal ``corrected'' statement I will work with.}
Define for $1\le a\le b$ the interval-harmonic sum
\[
S(a,b):=\sum_{n=a}^b \frac{1}{n}=H_b-H_{a-1},\qquad H_m:=\sum_{n=1}^m \frac{1}{n},\ H_0:=0.
\]
Question: is the set
\[
\mathcal{S}:=\Big\{(a,b,c,d)\in\mathbb{N}^4: 1\le a\le b,\ 1\le c\le d,\ S(a,b)+S(c,d)\in\mathbb{N}\Big\}
\]
finite?

\subsection*{2) Quick Literature / Context Check (web)}
The Erd\H{o}s Problems site lists this problem as \emph{open} and gives the example
\[
\frac{1}{3}+\frac{1}{4}+\frac{1}{5}+\frac{1}{6}+\frac{1}{20}=1.
\]
It also notes that the problem remains open even if $|I_2|=1$ (i.e. $I_2$ is a singleton), and records a comment that Richard K. Guy asserted finiteness without supplying a proof.

(References for the above are in the external prompt source; I will not attempt a full bibliography here, but see \S\,2 of Problem~291 below for the Erdosproblems citation pattern.)

\subsection*{3) Attack Plan}
A plausible line of attack is to exploit 
\emph{large primes that occur with exponent one} in the least common multiple of denominators.
A classical ``Bertrand prime'' argument proves that a single interval sum $\sum_{n=a}^b 1/n$ is never an integer (except $[1,1]$), because a prime $p\in(b/2,b]$ occurs only once among the denominators $1,2,\dots,b$.

For two-interval sums, one hopes to show that such ``large primes'' must appear in \emph{both} intervals (or in controlled ways), forcing strong constraints on the endpoints and (ideally) bounding them.

Concretely:
\begin{enumerate}[label=(\alph*)]
\item Prove a clean lemma about non-integrality of one interval sum using Bertrand's postulate.
\item Use the same modular idea to derive necessary conditions for 
$S(a,b)+S(c,d)\in\mathbb{Z}$.
\item Do a small brute-force search for solutions with small endpoints to sanity-check (and to look for counterexamples/patterns).
\end{enumerate}

\subsection*{4) Work}
\subsubsection*{4.1 A useful lemma: a single interval sum is never an integer}
\begin{theorem}
\label{thm:single-interval-not-integer}
Let $1\le a\le b$ be integers. If $(a,b)\neq(1,1)$ then
\[
\sum_{n=a}^b \frac{1}{n}\notin\mathbb{Z}.
\]
\end{theorem}
\begin{proof}
If $b=1$, then necessarily $(a,b)=(1,1)$ and the sum equals $1$.
Assume $b\ge 2$.

\emph{Case 1: $a>\frac b2$.}
Then $b-a+1\le b-\lfloor b/2\rfloor < a$, so
\[
\sum_{n=a}^b \frac{1}{n} < \frac{b-a+1}{a} < 1.
\]
This sum is positive, hence it cannot be an integer.

\emph{Case 2: $a\le \frac b2$.}
By Bertrand's postulate (and direct check for $b=2,3$), there exists a prime $p$ with
\[
\frac b2 < p \le b.
\]
Since $a\le b/2 < p\le b$, we have $p\in\{a,a+1,\dots,b\}$.
Let $L:=\mathrm{lcm}(a,a+1,\dots,b)$.
Because $p>b/2$, we have $p^2>b$, hence no integer in $[a,b]$ is divisible by $p^2$.
Therefore $p$ divides $L$ to exponent exactly $1$.
Write
\[
\sum_{n=a}^b \frac{1}{n} = \frac{A}{L},\qquad A:=\sum_{n=a}^b \frac{L}{n}\in\mathbb{Z}.
\]
For $n\neq p$ in $[a,b]$, we have $p\mid L$ and $p\nmid n$, hence $p\mid (L/n)$.
For $n=p$, the term is $L/p$, which is \emph{not} divisible by $p$ because $p$ appears only once in $L$.
Consequently
\[
A \equiv \frac{L}{p}\not\equiv 0 \pmod p.
\]
Thus $p\nmid A$, so after reducing $A/L$ to lowest terms, a factor $p$ remains in the denominator; in particular the fraction is not an integer.
\end{proof}

\textbf{Remarks.}
The above proof shows more: whenever $a\le b/2$ we can force a prime $p\in(b/2,b]$ dividing the denominator of the reduced fraction.
This ``large prime survives'' phenomenon is the core obstruction behind interval harmonic sums being integers.

\subsubsection*{4.2 A necessary condition template for two intervals}
Suppose $S(a,b)+S(c,d)\in\mathbb{Z}$.
Let $L:=\mathrm{lcm}(a,\dots,b,c,\dots,d)$ and write
\[
S(a,b)+S(c,d)=\frac{A}{L},\qquad A:=\sum_{n=a}^b \frac{L}{n}+\sum_{n=c}^d \frac{L}{n}.
\]
If $p$ is a prime such that:
(i) $p$ divides $L$ to exponent $1$; and
(ii) among $\{a,\dots,b,c,\dots,d\}$ there is exactly one multiple of $p$ (namely $p$ itself),
then the same congruence argument as in Theorem~\ref{thm:single-interval-not-integer} gives
$A\not\equiv 0\pmod p$, hence $A/L\notin\mathbb{Z}$.

Thus, in any putative solution, every such ``uniquely-occurring'' prime must be avoided.
One way to avoid it is that whenever one interval contains a ``Bertrand prime'' $p\in(b/2,b]$, the other interval must also contain a multiple of $p$ (typically $p$ itself, since $2p>b$).
This yields soft endpoint comparability constraints (e.g. if $a\le b/2$ and $p\in(b/2,b]$ lies in $[a,b]$, then necessarily $d\ge p > b/2$, hence $b<2d$).
These constraints are far from enough to conclude finiteness.

\subsubsection*{4.3 Small computational sanity check}
I did a brute-force search (exact rational arithmetic) for all intervals with endpoints up to $300$.
Treating $(I_1,I_2)$ as an \emph{ordered} pair of nonempty intervals, the only solutions to
$\sum_{n\in I_1}1/n+\sum_{n\in I_2}1/n\in\mathbb{Z}$ with endpoints $\le 300$ are the following $11$:
\begin{align*}
&([1,1],[1,1])\mapsto 2,\\
&([1,2],[1,2])\mapsto 3,\\
&([1,2],[2,2])\mapsto 2,\quad ([2,2],[1,2])\mapsto 2,\\
&([2,2],[2,2])\mapsto 1,\\
&([1,3],[6,6])\mapsto 2,\quad ([6,6],[1,3])\mapsto 2,\\
&([2,3],[6,6])\mapsto 1,\quad ([6,6],[2,3])\mapsto 1,\\
&([3,6],[20,20])\mapsto 1,\quad ([20,20],[3,6])\mapsto 1.
\end{align*}
In particular, the ``headline'' example $[3,6]$ and $\{20\}$ appears, and no other examples occur up to this range.

\textbf{Caveat.}
This is only evidence and does not resolve the problem: the question is whether \emph{infinitely many} solutions exist.

\subsection*{5) Verification}
\begin{itemize}
\item The proof of Theorem~\ref{thm:single-interval-not-integer} is complete: it cleanly separates the ``short near-$b$'' case ($a>b/2$) from the ``contains a Bertrand prime'' case ($a\le b/2$) and uses an explicit modulo-$p$ congruence.
\item The ``necessary condition template'' in \S\,4.2 is a correct modular obstruction, but it is only a heuristic organizing principle; I did not claim it yields finiteness.
\item The computational list in \S\,4.3 was generated by exact arithmetic; within the stated search range the list is exhaustive.
\end{itemize}

\subsection*{6) Final}
\textbf{UNRESOLVED.}
\begin{enumerate}[label=(\roman*)]
\item \textbf{Farthest point reached:} proved a clean unconditional lemma that a \emph{single} interval harmonic sum is never an integer except $[1,1]$, and derived a modular ``uniquely occurring prime'' obstruction for two-interval sums; additionally, exhaustive computation up to endpoints $\le 300$ finds only $11$ ordered-pair solutions.
\item \textbf{Best partial results:} Theorem~\ref{thm:single-interval-not-integer} (non-integrality of one interval sum), and the modular obstruction in \S\,4.2.
\item \textbf{Most plausible next moves:}
try to turn the ``Bertrand prime must be shared'' heuristic into a \emph{uniform} bound on endpoints, perhaps by tracking several primes in dyadic ranges and forcing incompatible constraints on both intervals.
\item \textbf{What I would try next with more time:}
formalize a sieve-style argument over primes $p\in (x/2,x]$ that appear exactly once across the union of two intervals, attempting to show that for large endpoints such primes are unavoidable.
\end{enumerate}

\subsection*{7) Completion Estimate}
\textbf{Estimate:} 20\%.
(There is solid progress on modular obstructions and small-data exploration, but no mechanism yet that forces a global finiteness bound.)


% ============================================================
