\subsection*{Erd\H{o}s problem \#338}

\noindent\textbf{1) FORMAL RESTATEMENT.}

A set $A\subseteq\mathbb{N}$ has \emph{restricted order} $r$ if every sufficiently large integer can be expressed as a sum of at most $r$ \emph{distinct} elements of $A$.

The problem asks:
\begin{quote}
If $A$ is a basis of (unrestricted) order $r$, must $A$ have a restricted order (possibly depending on $r$)?
In particular, must every basis of order $2$ have a restricted order?
\end{quote}
The statement also records a construction of Bateman for $h\ge 3$ that is a basis of order $h$ but has no restricted order.

\medskip
\noindent\textbf{2) QUICK LITERATURE/CONTEXT CHECK.}
The statement reports:
\begin{itemize}
\item Bateman constructed, for each $h\ge 3$, a basis of order $h$ with no restricted order.
\item It also states that Kelly [Ke84] proved every basis of order $2$ has restricted order.
\item It asks further questions about sets of exact order $2$ and deletions $A\setminus\{a\}$.
\end{itemize}
Only the Bateman construction is proved here.

\medskip
\noindent\textbf{3) ATTACK PLAN.}
\begin{itemize}
\item Prove Bateman's example works: basis of order $h$ but no restricted order.
\item Do a small modular sanity check for $h=3$.
\item Record what remains open/unstudied here (notably the order-$2$ and exact-order-$2$ refinements).
\end{itemize}

\medskip
\noindent\textbf{4) WORK.}

Fix an integer $h\ge 3$ and define
\[
A:=\{1\}\cup\{x\in\mathbb{N}: h\mid x\}.
\]

\medskip
\noindent\textbf{Proposition 338.1 ($A$ is a basis of order $h$).}
Every integer $n\ge 1$ can be expressed as a sum of at most $h$ (not necessarily distinct) elements of $A$.

\noindent\emph{Proof.}
Write $n=qh+r$ with $0\le r\le h-1$. Then $qh\in A$ (it is a multiple of $h$), and $1\in A$. Hence
\[
 n = (qh) + \underbrace{1+\cdots+1}_{r\text{ times}}.
\]
This uses $1+r\le h$ summands (or just $r$ summands if $q=0$ and we interpret $qh=0$ as absent). Thus $A$ is a basis of order $h$.
\hfill$\square$

\medskip
\noindent\textbf{Proposition 338.2 ($A$ has no restricted order).}
No matter what $r\in\mathbb{N}$ is chosen, there are infinitely many integers that cannot be written as a sum of at most $r$ \emph{distinct} elements of $A$.

\noindent\emph{Proof.}
A sum of distinct elements of $A$ uses at most one copy of $1$ (because $1$ appears only once in $A$) and any number of distinct multiples of $h$.
Therefore every such sum is congruent to either $0$ or $1$ modulo $h$.
In particular, no integer congruent to $2\pmod h$ can be represented in this way, and there are infinitely many such integers. Hence $A$ has no restricted order.
\hfill$\square$

\medskip
\noindent\textbf{FAST REALITY CHECK (the case $h=3$).}
For $h=3$, $A=\{1,3,6,9,\dots\}$. Any sum of distinct elements is congruent to $0$ or $1$ modulo $3$, so $2,5,8,11,\dots$ are never representable. This matches Proposition~338.2.

\medskip
\noindent\textbf{5) VERIFICATION.}
Both propositions are proved by explicit modular reasoning.

\medskip
\noindent\textbf{6) FINAL.}

\noindent\textbf{UNRESOLVED}

\smallskip
\noindent (i) \textbf{Strongest fully proved partial result obtained here.}
We gave a complete proof that for every $h\ge 3$ there exists a basis of order $h$ with no restricted order (Propositions~338.1--338.2), answering the general question in the negative for $r\ge 3$.

\smallskip
\noindent (ii) \textbf{Exact first gap.}
The remaining questions in the statement concern order $2$ and exact order $2$ (including whether $A\setminus\{a\}$ must remain a basis of order $2$). No such results are proved here.

\smallskip
\noindent (iii) \textbf{Top 3 next moves (concrete targets).}
\begin{enumerate}
\item Try to prove directly (without external literature) that every basis of order $2$ has some finite restricted order; or find an explicit counterexample if possible.
\item Analyze the stronger hypothesis ``exact order $2$'' and determine whether it forces robustness under deletions $A\setminus\{a\}$.
\item Search for structural invariants (e.g. local densities in residue classes) that prevent restricted representations even when unrestricted representations exist.
\end{enumerate}

\smallskip
\noindent (iv) \textbf{Minimal counterexample structure.}
A counterexample to ``every basis of order $2$ has restricted order'' would be a set $A$ such that every large $n$ is a sum of two elements of $A$ (with repetition allowed), yet for each fixed $r$ infinitely many integers cannot be represented as sums of $\le r$ distinct elements. Any such $A$ would need to force heavy reuse of certain small elements (as in Bateman's example), but without increasing the order beyond $2$.


