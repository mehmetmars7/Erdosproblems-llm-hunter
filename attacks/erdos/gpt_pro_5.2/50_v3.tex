\section{Problem 50: derivatives of the totient distribution function (Round 4)}

\subsection{1) ROUND-4 OBJECTIVE}

\textbf{Path (C) (obstruction/correction).}
Round~3 established a strong obstruction on a dense countable set: for every
\(t\in\mathcal E:=\{\varphi(n)/n:n\ge 1\}\) the left derivative satisfies
\(G'_-(t)=+\infty\), hence no such \(t\) can have a finite positive derivative.
The remaining gap is the possibility of a point \(x\notin\mathcal E\) where the
(two-sided) derivative exists and is \(>0\).

In this round we add a new, conceptually different obstruction that applies to
\emph{almost every point with respect to the distribution measure itself}:
since \(dG\) is purely singular (Round~1), the \emph{upper symmetric density}
\(\limsup_{h\to0}\frac{G(x+h)-G(x-h)}{2h}\) is \(+\infty\) for \(dG\)-almost every
\(x\). Consequently, the set of points where \(G'(x)\) could possibly exist as a
finite real number is a \(dG\)-null set (in addition to being Lebesgue-null by
Round~1).

\subsection{2) ROUND-3 FOUNDATION USED}

We use the following Round~(4--1)=Round~3 and Round~1 results as black boxes:

\begin{itemize}
\item (Round~1 Lemma~1) The limit distribution function
\[
G(u):=\lim_{X\to\infty}\frac1X\#\Bigl\{n\le X:\frac{\varphi(n)}n\le u\Bigr\}
\]
exists, is continuous and strictly increasing on \([0,1]\), and the associated
probability measure \(\mu:=dG\) is \emph{purely singular} with respect to Lebesgue
measure \(\lambda\); in particular \(G'(x)=0\) for \(\lambda\)-a.e.\ \(x\).

\item (Round~3 Corollary~\ref{cor:infinite-left}) For every \(t\in\mathcal E\),
the left derivative exists in the extended sense and equals \(+\infty\):
\[
G'_-(t)=\lim_{h\to0^+}\frac{G(t)-G(t-h)}{h}=+\infty.
\]

\item (Round~3 Lemma) \(\mathcal E\) is countable and dense in \([0,1]\).
\end{itemize}

\subsection{3) NEW INSIGHT / TOOL (ROUND-4)}

A general differentiation principle for \emph{singular measures on the line}:

\begin{quote}
If \(\mu\perp\lambda\) is a finite Borel measure on \(\mathbb R\), then the upper
Lebesgue density \(\limsup_{r\to0}\mu((x-r,x+r))/(2r)\) is \(+\infty\) for
\(\mu\)-almost every \(x\).
\end{quote}

Applied to \(\mu=dG\), this yields an \emph{uncountable, dense} set of points
(outside \(\mathcal E\)) where the upper symmetric Dini derivative of \(G\) is infinite.
This strictly strengthens Round~3, which treated only the countable dense set \(\mathcal E\).

\subsection{4) ATTACK PLAN (ROUND-4)}

\begin{enumerate}
\item Encode \(G\) as a singular probability measure \(\mu=dG\) on \([0,1]\).
\item Prove the measure-theoretic lemma: for singular \(\mu\), the upper symmetric
density is infinite \(\mu\)-a.e. (using a Besicovitch covering argument).
\item Deduce: the set of points where \(G'(x)\) can be finite is \(\mu\)-null.
Combine with Round~3 to sharpen the candidate set for a finite positive derivative:
\[
\{x:\, G'(x)\ \text{exists and }0<G'(x)<\infty\}\subseteq
([0,1]\setminus\mathcal E)\cap N,
\]
where \(N\) is \(\mu\)-null (and also \(\lambda\)-null by Round~1).
\end{enumerate}

\subsection{5) WORK (ROUND-4)}

\subsubsection*{5.1. Densities of a singular measure}

Let \(\mu\) be a finite Borel measure on \(\mathbb R\). Define its \emph{upper symmetric
Lebesgue density} at \(x\) by
\begin{equation}\label{eq:upper-density}
\Theta^*(\mu,x):=\limsup_{r\to0^+}\frac{\mu((x-r,x+r))}{2r}\in[0,+\infty].
\end{equation}

\begin{lemma}[Singular measures have infinite upper density \(\mu\)-a.e.]\label{lem:singular-density}
If \(\mu\perp\lambda\) (Lebesgue measure), then \(\Theta^*(\mu,x)=+\infty\) for \(\mu\)-almost every \(x\).
Equivalently, for every \(M>0\),
\[
\mu\bigl(\{x:\Theta^*(\mu,x)\le M\}\bigr)=0.
\]
\end{lemma}

\begin{proof}
Fix \(M>0\). Let
\[
A_M:=\{x:\Theta^*(\mu,x)\le M\}.
\]
Since \(\mu\perp\lambda\), there exists a Borel set \(S\subset\mathbb R\) with
\(\lambda(S)=0\) and \(\mu(\mathbb R\setminus S)=0\). Hence \(\mu(A_M)=\mu(A_M\cap S)\),
so it suffices to show \(\mu(A_M\cap S)=0\).

Fix \(\eta>0\). Because \(\lambda(S)=0\), choose an open set \(O\supset S\) with
\(\lambda(O)<\eta\).

For each \(x\in A_M\cap S\), the condition \(\Theta^*(\mu,x)\le M\) implies that for
\(\varepsilon=1\) there exists \(r_x>0\) such that
\[
\frac{\mu((x-r,x+r))}{2r}\le M+1\qquad(0<r<r_x).
\]
Replace \(r_x\) by a smaller radius if needed so that also \((x-r_x,x+r_x)\subset O\).
Set \(I_x:=(x-r_x,x+r_x)\). Then
\begin{equation}\label{eq:Ix-bound}
\mu(I_x)\le 2(M+1)r_x=(M+1)|I_x|.
\end{equation}

We now invoke the (one-dimensional) \emph{Besicovitch covering theorem} for intervals:
from the family \(\{I_x:x\in A_M\cap S\}\) one can extract a countable subfamily
\(\{I_j\}_{j\ge1}\) that still covers \(A_M\cap S\) and has \emph{overlap bounded by \(2\)},
i.e.\ every point of \(\mathbb R\) belongs to at most \(2\) of the intervals \(I_j\).

Using \eqref{eq:Ix-bound} and the overlap bound,
\[
\mu(A_M\cap S)\le \sum_{j\ge1}\mu(I_j)\le (M+1)\sum_{j\ge1}|I_j|
\le (M+1)\cdot 2\,\lambda\Bigl(\bigcup_{j\ge1} I_j\Bigr).
\]
Since every \(I_j\subset O\), we have \(\lambda(\cup_j I_j)\le\lambda(O)<\eta\), hence
\[
\mu(A_M\cap S)\le 2(M+1)\eta.
\]
Letting \(\eta\to0\) gives \(\mu(A_M\cap S)=0\), and thus \(\mu(A_M)=0\).
\end{proof}

\subsubsection*{5.2. Application to the totient distribution}

Let \(G\) be the distribution function from Round~1 and set \(\mu:=dG\), so that
\(\mu((a,b])=G(b)-G(a)\). For \(x\in(0,1)\),
\begin{equation}\label{eq:symm-derivative}
\Theta^*(\mu,x)=\limsup_{h\to0^+}\frac{\mu((x-h,x+h))}{2h}
=\limsup_{h\to0^+}\frac{G(x+h)-G(x-h)}{2h}.
\end{equation}

\begin{theorem}[Upper symmetric Dini derivative is infinite \(\mu\)-a.e.]\label{thm:round4-main}
For the totient distribution function \(G\) and \(\mu=dG\),
\[
\limsup_{h\to0^+}\frac{G(x+h)-G(x-h)}{2h}=+\infty
\qquad\text{for }\mu\text{-almost every }x\in(0,1).
\]
In particular, the set of points where the (two-sided) derivative \(G'(x)\) exists as a finite real number
has \(\mu\)-measure \(0\).
\end{theorem}

\begin{proof}
By Round~1, \(\mu\perp\lambda\). Lemma~\ref{lem:singular-density} therefore yields
\(\Theta^*(\mu,x)=+\infty\) for \(\mu\)-almost every \(x\). Using \eqref{eq:symm-derivative}
gives the displayed limsup statement.

If \(G'(x)\) exists and is finite, then the symmetric difference quotients converge to \(G'(x)\),
so the limsup in \eqref{eq:symm-derivative} is finite. Hence differentiability can occur only on a \(\mu\)-null set.
\end{proof}

\subsubsection*{5.3. A strengthened reduction of Erd\H{o}s's question}

\begin{corollary}[Candidate points for a finite positive derivative are doubly exceptional]\label{cor:round4-reduction}
If there exists \(x\in(0,1)\) such that \(G'(x)\) exists and \(0<G'(x)<\infty\), then:
\begin{enumerate}
\item \(x\notin\mathcal E\) (Round~3).
\item \(x\) belongs to a \(\mu\)-null set \(N\subset(0,1)\) (Theorem~\ref{thm:round4-main}).
\item \(x\) belongs to a \(\lambda\)-null set (since \(G'(x)=0\) for \(\lambda\)-a.e.\ \(x\), Round~1).
\end{enumerate}
Moreover, the set
\[
B:=\Bigl\{x\in(0,1): \limsup_{h\to0^+}\frac{G(x+h)-G(x-h)}{2h}=+\infty\Bigr\}
\]
has \(\mu(B)=1\) and is dense in \((0,1)\); hence \(B\setminus\mathcal E\) is also dense and uncountable.
\end{corollary}

\begin{proof}
Items (1) and (3) are immediate from the cited Round~3 and Round~1 statements.
For (2), if \(G'(x)\) exists and is finite (in particular, positive), then the
symmetric quotient has finite limsup, hence \(\Theta^*(\mu,x)<\infty\). Theorem~\ref{thm:round4-main}
shows this can only happen on a \(\mu\)-null set.

For density: \(\mu(B)=1\) by Theorem~\ref{thm:round4-main}. Since \(G\) is strictly increasing
(Round~1), every open interval \(I\subset(0,1)\) satisfies \(\mu(I)=G(\sup I)-G(\inf I)>0\).
Thus \(\mu(I\cap B)=\mu(I)>0\), so \(I\cap B\neq\emptyset\). Hence \(B\) is dense.
Finally, \(\mathcal E\) is countable (Round~3), so removing it from dense \(B\) leaves
a dense uncountable set.
\end{proof}

\subsection{6) ADVERSARIAL VERIFICATION}

\begin{itemize}
\item \textbf{Does Lemma~\ref{lem:singular-density} really use only singularity?}
Yes: the only arithmetic input is the black-box fact \(\mu\perp\lambda\) (Round~1).
The rest is a covering/overlap estimate.

\item \textbf{Dependence on the Besicovitch covering theorem.}
We used only the one-dimensional statement for intervals with centers in the set
being covered; in \(\mathbb R\) the overlap constant can be taken \(N=2\).
All hypotheses are satisfied because for each \(x\in A_M\cap S\) we constructed an
interval \(I_x\ni x\) and can take it arbitrarily small.

\item \textbf{Endpoints.}
The density definition \eqref{eq:upper-density} is written for interior points.
This suffices for Erd\H{o}s's question \(x\in(0,1)\). One may ignore boundary
issues by restricting to \([\delta,1-\delta]\) and letting \(\delta\to0\).

\item \textbf{Is \(\mu(\mathcal E)=0\) justified?}
Yes: \(G\) is continuous (Round~1), hence \(\mu\) has no point masses.
A countable set has \(\mu\)-measure \(0\).

\item \textbf{No hidden use of Round~3's deep analytic inputs.}
Round~4's new theorem does not depend on Tenenbaum--Toulmonde.
It only strengthens the global obstruction by exploiting pure singularity.
\end{itemize}

\subsection{7) FINAL (EXACTLY ONE)}

\textbf{UNRESOLVED (BUT STRICTLY ADVANCED).}

Beyond the dense countable obstruction \(\mathcal E\) from Round~3, we proved a
new global obstruction: because \(\mu=dG\) is purely singular, the symmetric upper
Dini derivative of \(G\) is infinite at \(\mu\)-almost every point.
Thus any point \(x\) with a finite positive derivative (if it exists at all) must
lie outside \(\mathcal E\) \emph{and} in a \(\mu\)-null exceptional set.

\subsection{8) COMPLETION ESTIMATE (MANDATORY)}

\textbf{COMPLETION: 78\%}

\subsection{9) REFERENCES}

\begin{enumerate}
\item I. J. Schoenberg, \emph{On the distribution of the values of an arithmetical function},
Trans. Amer. Math. Soc. \textbf{41} (1937), 57--70. (Existence/continuity/strict monotonicity of the limit distribution.)

\item P. Erd\H{o}s, \emph{On the distribution function of additive functions},
Ann. of Math. (2) \textbf{47} (1946), 1--20. (Pure singularity for the totient distribution function.)

\item G. Tenenbaum and V. Toulmonde, \emph{Sur le comportement local de la r\'epartition de l'indicatrice d'Euler},
Funct. Approx. Comment. Math. \textbf{41} (2009), 89--111. (Used in Round~3 for the local modulus on \(\mathcal E\).)

\item P. Mattila, \emph{Geometry of Sets and Measures in Euclidean Spaces}, Cambridge Studies in Advanced Mathematics 44,
Cambridge Univ. Press, 1995. (Reference for the Besicovitch covering theorem used in Lemma~\ref{lem:singular-density}.)
\end{enumerate}
