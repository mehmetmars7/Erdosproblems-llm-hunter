\section*{Problem 377. Is $\displaystyle f(n)=\sum_{p\le n}1_{p\nmid \binom{2n}{n}}\,\frac1p$ bounded?}

\subsection*{FORMAL RESTATEMENT}
Define for $n\ge 1$
\[
  f(n)\;:=\;\sum_{\substack{p\le n\\ p\text{ prime}}} \mathbf{1}_{p\nmid \binom{2n}{n}}\,\frac{1}{p}.
\]
Is there an absolute constant $C$ such that $f(n)\le C$ for all $n$?

\subsection*{QUICK LITERATURE / CONTEXT CHECK}
Erd\H{o}s--Graham--Ruzsa--Straus (1975) introduced $f(n)$ and proved:
\begin{itemize}
\item $\displaystyle \lim_{x\to\infty}\frac1x\sum_{n\le x} f(n)=\sum_{k\ge 2}\frac{\log k}{2^k}=:\gamma_0$ and $\lim_{x\to\infty}\frac1x\sum_{n\le x} f(n)^2=\gamma_0^2$, hence for almost all $n$, $f(n)=\gamma_0+o(1)$.
\item For all sufficiently large $n$, $f(n)\le c\log\log n$ for some absolute constant $c<1$.
\item They remark explicitly that they cannot decide whether $f(n)$ is unbounded.
\end{itemize}
The Erd\H{o}s problems database (Problem~377) records the same status.

Numerically, $\gamma_0\approx 0.5078339229$.

\subsection*{ATTACK PLAN}
To prove boundedness one would need a mechanism showing that for every $n$ there are ``enough'' primes $p\le n$ that \emph{do} divide $\binom{2n}{n}$, forcing the complement (weighted by $1/p$) to have bounded harmonic mass.

To disprove boundedness one would need to construct infinitely many $n$ for which many small primes simultaneously satisfy $p\nmid\binom{2n}{n}$, so that $\sum 1/p$ over those primes grows.

Kummer's theorem again translates $p\nmid\binom{2n}{n}$ into the digit condition ``all base-$p$ digits of $n$ are $\le (p-1)/2$'', making the problem a question about how often $n$ can have small digits in many bases at once.

\subsection*{WORK}
\subsubsection*{1. A simple implication: bounded $f(n)$ would bound the smallest odd prime factor}
Let $g(n)$ be the smallest odd prime divisor of $\binom{2n}{n}$ (if it exists). If $g(n)>y$ then no odd prime $p\le y$ divides $\binom{2n}{n}$, hence
\[
  f(n)\ge \sum_{3\le p\le y} \frac1p.
\]
Since $\sum_{p\le y}1/p\to\infty$ with $y$, it follows that
\[
  \sup_n f(n)<\infty \ \Longrightarrow\ \sup_n g(n)<\infty.
\]
EGRS (1975) note that even the boundedness of $g(n)$ is unknown.

\subsubsection*{2. Relation to Problem 376 and ``avoiding many primes''}
If for some finite set of primes $\mathcal{P}$ we can find $n$ with $p\nmid\binom{2n}{n}$ for all $p\in\mathcal{P}$, then
\[
  f(n)\ge \sum_{p\in\mathcal{P}}\frac1p.
\]
Thus unboundedness of $f(n)$ would follow from the ability to simultaneously avoid an increasing collection of small primes.

However, even avoiding $\{3,5,7\}$ (Problem~376) is open; and avoiding any two odd primes is known to be possible infinitely often (EGRS), which only yields a bounded lower bound.

\subsubsection*{3. Computation for small $n$ (evidence only)}
Using Kummer's digit criterion to test $p\nmid\binom{2n}{n}$, a brute-force search over $1\le n\le 20000$ gives:
\begin{itemize}
\item the maximum value of $f(n)$ in this range is
\[\max_{1\le n\le 20000} f(n) \approx 1.1792429058\quad\text{attained at }n=3250,\]
\item other large values occur at $n=3160$ ($\approx 1.1155$) and at $n=756$ ($\approx 1.0770$), both of which are also solutions to Problem~376 (coprime to $105$).
\end{itemize}
This does not decide boundedness but is consistent with the possibility that $f(n)$ has a modest global supremum.

\subsection*{VERIFICATION}
The logical implications in \S1--\S2 are direct. The numerical values in \S3 were computed exactly from the digit condition (not by evaluating $\binom{2n}{n}$).

\subsection*{FINAL}
\textbf{LABEL: UNRESOLVED.} \\
\textbf{SUB-LABEL: N/A.} \\
The existence of an absolute constant $C$ with $f(n)\le C$ for all $n$ remains open. Known results show $f(n)$ is typically close to the constant $\gamma_0\approx 0.5078$ and always satisfies $f(n)\le c\log\log n$ for some $c<1$ for large $n$, but no boundedness/unboundedness proof is known.

\subsection*{COMPLETION ESTIMATE}
A full solution would require either:
\begin{itemize}
\item a construction of integers $n$ for which $p\nmid\binom{2n}{n}$ for enough small primes $p$ that $\sum 1/p$ grows without bound (proving unboundedness), or
\item a uniform argument forcing the set of avoided primes to have bounded harmonic weight for every $n$ (proving boundedness).
\end{itemize}
Both directions appear to hinge on deep control of simultaneous small-digit expansions (or carries) in many bases.

%%%%%%%%%%%%%%%%%%%%%%%%%%%%%%%%%%%%%%%%%%%%%%%%%%%%%%%%%%%%%%%%%%%%%%%%%%%%%%
\section*{Bibliography (informal)}
\begin{thebibliography}{9}
\bibitem{EGRS75}
P.~Erd\H{o}s, R.~L.~Graham, I.~Z.~Ruzsa, and E.~G.~Straus,
\emph{On the prime factors of $\binom{2n}{n}$}, Math. Comp. 29 (1975), 83--92.

\bibitem{EG76}
P.~Erd\H{o}s and R.~L.~Graham,
\emph{On products of factorials}, Bull. Inst. Math. Acad. Sinica 4 (1976), 337--355.

\bibitem{DNSS14}
A.~Dujella, F.~Najman, N.~Saradha, and T.~N.~Shorey,
\emph{Products of three factorials}, Publ. Math. Debrecen 85 (2014), 3--13.

\bibitem{BlCr25}
T.~F.~Bloom and E.~Croot,
\emph{Integers with small digits in multiple bases}, arXiv:2509.02835 (2025).

\bibitem{Yu22}
H.~Yu,
\emph{Fractal projections with an application in number theory}, Proc. Lond. Math. Soc. 125 (2022), 1060--1109.
\end{thebibliography}

