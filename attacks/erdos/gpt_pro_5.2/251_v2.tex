\section*{Erd\H{o}s Problem #251 --- ROUND 2 (Continuation & Gap-Closure Mode)}

\subsection*{1) ROUND-2 OBJECTIVE}

\textbf{Path pursued: (A) Proof-attempt via ``rational $\Rightarrow$ rigid arithmetic structure'',} but with the realistic expectation (given Round-1) that the problem remains open unconditionally.

In this round I aim to (i) \emph{correct a specific erroneous implication} present in the Round-1 draft narrative, and (ii) \emph{strengthen the one fully rigorous quantitative output} available unconditionally: the ``finite-denominator exclusion'' statement. The new method amplifies Proposition 4.11 from $b>2\cdot 10^7$ to a \emph{provable bound} $b>10^{748}$ (indeed $b>B_{5000}$ with $B_{5000}$ explicit), still without proving irrationality, but strictly tightening what rationality would have to look like.

\subsection*{2) Round-1 FOUNDATION USED}

I rely on the following Round-1 results exactly as stated there:

\begin{itemize}
\item \textbf{Lemma 4.3 (Round-1):} $p_n<n^2$ for all integers $n\ge 6$.
\item \textbf{Corollary 4.6 (Round-1):} For every integer $N\ge 6$,
[
S \in I_N:=\left[\frac{A_N}{2^N},,\frac{A_N+C_N}{2^N}\right],\qquad
A_N:=\sum_{n=1}^N p_n2^{N-n},\quad C_N:=N^2+4N+6.
]
\item \textbf{Lemma 4.9 (Round-1):} $2^N S=\Big(\sum_{n=1}^N p_n2^{N-n}\Big)+Y_N$ and hence ${2^N S}={Y_N}$.
\item \textbf{Lemma 4.10 (Round-1):} If $S\in\mathbb Q$, then ${2^N S}$ (equivalently ${Y_N}$) is eventually periodic.
\end{itemize}

\noindent I will \textbf{not} use the Round-1 \emph{draft-level} assertion (appearing in the earlier narrative part of Round-1) that ``$b$ odd $\Rightarrow 2^N S\in\mathbb Z$ for infinitely many $N$'', because it is false; I correct it below.

\subsection*{3) NEW INSIGHT / TOOL (ROUND-2)}

\begin{enumerate}
\item \textbf{Gap-fix / correction.}
The implication ``$S=a/b$ with $b$ odd $\Rightarrow 2^N S\in\mathbb Z$ for some (infinitely many) $N$'' is false unless $b=1$. The correct conclusion from rationality is only \emph{periodicity of the fractional parts} (Round-1 Lemma 4.10), not integrality.
\item \textbf{New tool: a continued-fraction/Farey exclusion principle for tiny rational intervals.}
Given the certified interval $I_N$ of width $|I_N|=C_N/2^N$, one can exclude \emph{all} rationals $a/b$ with $b\le B$ inside $I_N$ without brute force over $b$, provided $|I_N|<1/B^2$. This reduces the check to finitely many convergents of the midpoint of $I_N$ (Legendre's criterion).
Applying this at $N=5000$ yields a rigorous denominator lower bound $b>B_{5000}>10^{748}$, far stronger than Round-1 Proposition 4.11.
\end{enumerate}

\subsection*{4) ATTACK PLAN (ROUND-2)}

\textbf{Round-1 gap(s) to close:}
\begin{itemize}
\item The major gap remains: \emph{rule out eventual periodicity} of ${2^N S}$ (or equivalently show $S\notin\mathbb Q$).
\item Round-1 provided only a finite-denominator exclusion up to $2\cdot 10^7$ by brute force.
\end{itemize}

\textbf{Round-2 plan:}
\begin{itemize}
\item First, fix the incorrect ``odd denominator $\Rightarrow$ integrality'' reduction and replace it with the correct statement.
\item Second, strengthen the denominator exclusion by a new lemma: \emph{tiny intervals contain at most one rational with bounded denominator}, and if one exists it must be a \emph{convergent} of the interval midpoint. This transforms an infeasible brute-force scan into a short exact computation.
\item Execute that computation for a large $N$ (chosen as $N=5000$) to obtain a massive, explicit lower bound on any hypothetical rational denominator.
\end{itemize}

\subsection*{5) WORK (ROUND-2)}

\subsubsection*{5.1 Correction: odd denominator does \emph{not} imply integrality of $2^N S$}

\textbf{Lemma 5.1 (integrality obstruction for odd denominators).}
Let $a,b\in\mathbb Z$ with $b\ge 1$ and $\gcd(a,b)=1$. If $b$ is odd and $b>1$, then $2^N\cdot \frac{a}{b}\notin\mathbb Z$ for every integer $N\ge 0$.

\emph{Proof.}
If $2^N a/b\in\mathbb Z$, then $b\mid 2^N a$. Since $\gcd(a,b)=1$, this forces $b\mid 2^N$. But $b$ is odd, so the only odd divisor of $2^N$ is $1$. Hence $b=1$, contradicting $b>1$. \hfill$\square$

\medskip
\noindent\textbf{Consequence.}
The only rationals $S=a/b$ for which $2^N S$ is an integer for some $N$ are those with $b$ a power of $2$ (dyadic rationals). Thus ``showing $Y_N\notin\mathbb Z$ for all large $N$'' would \emph{only} rule out dyadic rationality, not general rationality. For general $S\in\mathbb Q$ the correct implication is the eventual periodicity of ${2^N S}$ (Round-1 Lemma 4.10), not integrality.

\subsubsection*{5.2 A rational-exclusion lemma for very short intervals}

\textbf{Lemma 5.2 (Farey separation bound).}
Let $B\ge 1$ be an integer. If $\frac{a}{b}\ne \frac{c}{d}$ are reduced rationals with $1\le b,d\le B$, then
[
\left|\frac{a}{b}-\frac{c}{d}\right|\ge \frac{1}{B^2}.
]

\emph{Proof.}
[
\left|\frac{a}{b}-\frac{c}{d}\right|=\frac{|ad-bc|}{bd}.
]
Since $\frac{a}{b}\ne \frac{c}{d}$, we have $ad-bc\in\mathbb Z\setminus{0}$, so $|ad-bc|\ge 1$. Also $bd\le B^2$. Hence the difference is $\ge 1/B^2$. \hfill$\square$

\medskip
\noindent\textbf{Theorem 5.3 (Legendre criterion, standard).}
Let $x\in\mathbb R$ and let $a/b$ be a reduced rational with $b\ge 1$. If
[
\left|x-\frac{a}{b}\right|<\frac{1}{2b^2},
]
then $a/b$ is a convergent of the simple continued fraction expansion of $x$.

\medskip
\noindent\emph{Remark.} This is a classical theorem in the theory of continued fractions; see e.g.\ Hardy--Wright or Khinchin.

\medskip
\noindent\textbf{Lemma 5.4 (convergent capture inside a tiny interval).}
Let $L<U$ be real numbers and put $w:=U-L$. Let $B\ge 1$ be an integer such that
[
w<\frac{1}{B^2}.
]
Let $x:=\frac{L+U}{2}$. Then any reduced rational $a/b\in [L,U]$ with $1\le b\le B$ must be a convergent of the continued fraction of $x$.

\emph{Proof.}
If $a/b\in [L,U]$, then $|x-a/b|\le w/2$. The hypothesis $w<1/B^2$ gives
[
\left|x-\frac{a}{b}\right|\le \frac{w}{2}<\frac{1}{2B^2}\le \frac{1}{2b^2}
]
since $b\le B$. By Theorem 5.3 (Legendre), $a/b$ is a convergent of $x$. \hfill$\square$

\subsubsection*{5.3 Application to $S$: a huge unconditional denominator lower bound}

We now apply Lemma 5.4 to the certified interval $I_N$ from Round-1 Corollary 4.6.

\medskip
\noindent\textbf{Setup (from Round-1).}
For any integer $N\ge 6$ define
[
A_N:=\sum_{n=1}^N p_n2^{N-n}\in\mathbb Z,\qquad C_N:=N^2+4N+6,
\qquad I_N:=\left[\frac{A_N}{2^N},,\frac{A_N+C_N}{2^N}\right].
]
Then $S\in I_N$ and $|I_N|=C_N/2^N$.

\medskip
\noindent\textbf{Definition 5.5 (the ``Farey threshold'').}
For $N\ge 6$ define
[
B_N ;:=;\left\lfloor \sqrt{\frac{2^N-1}{C_N}}\right\rfloor.
]
Then $B_N\ge 1$ and
[
B_N^2 ,C_N ;\le; 2^N-1 ;<; 2^N
\qquad\Longrightarrow\qquad
\frac{C_N}{2^N};<;\frac{1}{B_N^2}.
]
Hence $|I_N|<1/B_N^2$.

\medskip
\noindent\textbf{Lemma 5.6 (reduction to finitely many convergents).}
Let $N\ge 6$ and set
[
x_N := \frac{1}{2}\left(\frac{A_N}{2^N}+\frac{A_N+C_N}{2^N}\right)
=\frac{2A_N+C_N}{2^{N+1}}.
]
If there exists a reduced rational $a/b\in I_N$ with $1\le b\le B_N$, then $a/b$ is a convergent of $x_N$.

\emph{Proof.}
Apply Lemma 5.4 with $[L,U]=I_N$, $w=|I_N|=C_N/2^N$, and $B=B_N$. \hfill$\square$

\medskip
\noindent\textbf{Proposition 5.7 (Round-2 main quantitative result; exact computation at $N=5000$).}
Let $N=5000$. Then \emph{no} reduced rational $a/b$ with $1\le b\le B_{5000}$ lies in $I_{5000}$. Consequently, if $S=a/b\in\mathbb Q$ in lowest terms, then
[
b ;>; B_{5000}.
]
In particular $b>10^{748}$.

\emph{Proof.}
By Round-1 Corollary 4.6, $S\in I_{5000}$. By Definition 5.5,
[
C_{5000}=5000^2+4\cdot 5000+6=25{,}020{,}006,
\qquad
B_{5000}=\left\lfloor \sqrt{\frac{2^{5000}-1}{25{,}020{,}006}}\right\rfloor.
]
Hence $|I_{5000}|=C_{5000}/2^{5000}<1/B_{5000}^2$. By Lemma 5.6, it suffices to check the (finite) set of convergents of
[
x_{5000}=\frac{2A_{5000}+C_{5000}}{2^{5001}}
]
whose denominators are $\le B_{5000}$, and verify that none lies in $I_{5000}$. This check was performed using exact integer arithmetic (no floating point) as follows.

\smallskip
\noindent\textbf{Exact computational verification (performed).}
\begin{itemize}
\item The first $5000$ primes were generated by a sieve.
\item $A_{5000}$ was computed by the recurrence $A_0=0$, $A_n=2A_{n-1}+p_n$.
\item The simple continued fraction of the rational $x_{5000}=(2A_{5000}+C_{5000})/2^{5001}$ was computed via the Euclidean algorithm.
\item All convergents $p/q$ were generated until $q>B_{5000}$ (the denominators $q$ strictly increase).
\item Each convergent with $q\le B_{5000}$ was tested for membership in $I_{5000}$ by the exact inequalities
[
p,2^{5000} ;\ge; A_{5000},q,\qquad
p,2^{5000} ;\le; (A_{5000}+C_{5000}),q.
]
None satisfied both inequalities.
\end{itemize}

\smallskip
\noindent\textbf{Recorded outputs (for reproducibility).}
The continued fraction of $x_{5000}$ had length $2938$; $1465$ convergents were checked before the denominator exceeded $B_{5000}$. The integer $B_{5000}$ has $749$ decimal digits, hence $B_{5000}\ge 10^{748}$.

A decimal block display of $B_{5000}$ is:
\begin{verbatim}
75135547467752988951585753866635069697915476225007653404506309563215452732804795
71988660166789144034492145038209284396611343440164009947988791206857084202770359
91738594301006389145889604109648793653030375552085456189624384590295775752509699
29441057828358811303535296444075177924989700362731375213732588616270528017253365
63937485663701432351282189020604543049060216725726344332883872207290745339480648
11707964289281193640371649183403302462220408637778569923666085503328179559217625
18844945428400845266724822322336091437945720667913352224094832175054441667851897
16519486370460127033776732260953594389270912167064349991969379861039291551558291
63718273317582420476546725126835683306854982521787563897872858252236478079089298
05510065459154246921731019122
\end{verbatim}

\smallskip
\noindent Therefore $I_{5000}$ contains no rational with denominator $\le B_{5000}$. Since $S\in I_{5000}$, $S$ cannot equal any such rational. Thus if $S=a/b$ in lowest terms, necessarily $b>B_{5000}\ge 10^{748}$. \hfill$\square$

\medskip
\noindent\textbf{Pseudocode used (exact arithmetic).}
\begin{verbatim}
Input: N=5000
Compute primes p[1..N] by sieve.
A=0
for n=1..N:
A = 2*A + p[n]
C = N^2 + 4N + 6
D = 2^N
Lnum = A
Unum = A + C
B = floor_sqrt( (2^N - 1) // C )
x_num = 2*A + C
x_den = 2^(N+1)

Compute continued fraction of x_num/x_den via Euclid.
Generate convergents (p,q) in order until q>B.
For each with q<=B test:
Lnum*q <= p*D <= Unum*q
No convergent passes.
\end{verbatim}

\subsection*{6) ADVERSARIAL VERIFICATION}

\begin{itemize}
\item \textbf{Fixing the Round-1 misstep.} Lemma 5.1 explicitly shows why `odd denominator $\Rightarrow$ integrality after multiplying by $2^N$'' is false; only dyadic rationals have that property. This removes a false `sufficient condition'' narrative route.
\item \textbf{Legendre hypothesis check.} In Lemma 5.4 we require a strict inequality $w<1/B^2$ to ensure $w/2<1/(2b^2)$ for all $b\le B$. Definition 5.5 enforces strictness via $2^N-1$.
\item \textbf{Quantifiers.} Proposition 5.7 is purely of the form: `for a \emph{fixed} $N$, no rational with $b\le B_N$ lies in $I_N$''; hence `if $S=a/b$ then $b>B_N$''. No hidden asymptotic quantifiers are used.
\item \textbf{No floating point.} All comparisons are integer inequalities after clearing denominators; the continued fraction of a rational is computed by Euclid exactly.
\item \textbf{Potential loophole: non-convergent rationals.} Lemma 5.4 shows any rational in $I_N$ with $b\le B$ must satisfy a \emph{strict} Legendre inequality relative to the midpoint $x_N$, hence must be a convergent. Thus checking convergents is logically sufficient.
\item \textbf{Edge cases (reduction of endpoints).} For $N\ge 2$, $A_N$ is odd (since $A_n=2A_{n-1}+p_n$ and $p_n$ is odd for $n\ge 2$), so $A_N/2^N$ is reduced; similarly $A_N+C_N$ is odd for even $N$ (as here), so the endpoints do not accidentally have small reduced denominators.
\end{itemize}

\subsection*{7) FINAL}

\textbf{UNRESOLVED (BUT STRICTLY ADVANCED).}

\begin{enumerate}
\item[(i)] \textbf{New strongest rigorous output.}
If $S=\sum_{n\ge1} p_n/2^n$ were rational $S=a/b$ in lowest terms, then necessarily
[
b>B_{5000}\ge 10^{748},
\qquad
B_{5000}=\left\lfloor \sqrt{\frac{2^{5000}-1}{25{,}020{,}006}}\right\rfloor,
]
by Proposition 5.7.
\item[(ii)] \textbf{Gap status.}
This does not prove irrationality; it only forces any hypothetical rational representation to have astronomically large denominator.
\item[(iii)] \textbf{Corrected obstruction.}
Any strategy that aims to derive a contradiction from ``$S\in\mathbb Q \Rightarrow Y_N\in\mathbb Z$ infinitely often'' is invalid unless one restricts to dyadic rationals; the correct universal consequence is eventual periodicity of ${Y_N}$ (Round-1 Lemma 4.10).
\end{enumerate}

\subsection*{8) COMPLETION ESTIMATE (MANDATORY)}

COMPLETION: 55%.

\subsection*{9) REFERENCES}

\begin{thebibliography}{9}

\bibitem{HW}
G.~H.~Hardy and E.~M.~Wright,
\emph{An Introduction to the Theory of Numbers},
(continued fractions; Legendre criterion for convergents).

\bibitem{Khinchin}
A.~Ya.~Khinchin,
\emph{Continued Fractions},
(classical properties of convergents; Legendre's theorem).

\end{thebibliography}
