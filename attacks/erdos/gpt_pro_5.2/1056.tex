% Erdos Problem #1056

\paragraph{FORMAL RESTATEMENT.}
Fix an integer $k\ge 2$.
Does there exist a prime $p$ and \emph{consecutive} nonempty integer intervals $I_1,\dots,I_k$ such that
\[
\prod_{n\in I_i} n \equiv 1 \pmod p\qquad\text{for all }1\le i\le k?
\]
Here “consecutive intervals” means there exist integers $a_1\le b_1 < a_2\le b_2 < \cdots < a_k\le b_k$ with $a_{i+1}=b_i+1$ and $I_i=\{a_i,a_i+1,\dots,b_i\}$.

\paragraph{QUICK LITERATURE/CONTEXT CHECK.}
I did not use external sources. I only use the examples stated in the problem text and perform a local computation search for small $k$.

\paragraph{ATTACK PLAN.}
\emph{Proof-track:} reduce the condition to factorial congruences in $\mathbb{F}_p$ and try to force repeated factorial values; relate to Wilson's theorem and possible cycle structure of $t!\bmod p$.
\emph{Disproof-track:} search for an obstruction (e.g. show that for each fixed $p$ the maximum $k$ is bounded).
Best path here: obtain equivalent formulations (lemmas) and do computational exploration; no general proof is found.

\paragraph{WORK.}
\textbf{Lemma 1 (localization away from multiples of $p$).}
If $(p;I_1,\dots,I_k)$ is a solution, then translating all intervals by a multiple of $p$ (replacing every $n$ by $n+tp$ for a fixed integer $t$) yields another solution.
Moreover, in any solution, none of the intervals may contain a multiple of $p$.

\emph{Proof.}
For any integer $t$, $n+tp\equiv n\pmod p$, so
$\prod_{n\in I_i}(n+tp)\equiv \prod_{n\in I_i}n\equiv 1\pmod p$.
Consecutiveness is preserved under translation.
If some interval contained a multiple of $p$, then its product would be $0\pmod p$, contradicting the requirement that it be $1\pmod p$.
\hfill$\square$

\textbf{Lemma 2 (factorial-chain formulation).}
Assume $I_1,\dots,I_k$ lie inside a block of consecutive integers with no multiples of $p$.
After translation (Lemma 1), we may assume
$I_1\cup\cdots\cup I_k \subseteq \{1,2,\dots,p-1\}$.
Write $I_i=[a_i,b_i]$ with $1\le a_i\le b_i\le p-1$ and $a_{i+1}=b_i+1$.
Define $a_0:=a_1-1$ and $a_i:=b_i$ for $1\le i\le k$.
Then the condition $\prod_{n\in I_i} n \equiv 1\pmod p$ for all $i$ is equivalent to
\[
a_{i}! \equiv a_{i-1}! \pmod p\qquad\text{for all }1\le i\le k,
\]
where $0!=1$.

\emph{Proof.}
For $1\le a\le b\le p-1$, all numbers $1,2,\dots,p-1$ are invertible modulo $p$, so
\[
\prod_{n=a}^{b} n \equiv \frac{b!}{(a-1)!}\pmod p.
\]
Thus $\prod_{n=a}^{b}n\equiv 1$ iff $b!\equiv (a-1)!$.
Applying this to $I_i=[a_i,b_i]$ gives $b_i!\equiv (a_i-1)!$.
Since $a_i-1=b_{i-1}$, this is exactly $a_i!\equiv a_{i-1}!$ in the notation above.
\hfill$\square$

\textbf{FAST REALITY CHECK (computation).}
Using Lemma 2, I searched primes $p$ for increasing chains $a_0<a_1<\cdots<a_k\le p-1$ with $a_i!\equiv a_{i-1}!\pmod p$.
Within the searched range, the smallest prime $p$ found for each $k\in\{2,\dots,8\}$ (together with one solution as consecutive intervals in $[1,p-1]$) was:
\begin{verbatim}
k=2:  p=5,   intervals: [1,1], [2,3]
k=3:  p=17,  intervals: [1,1], [2,5], [6,11]
k=4:  p=17,  intervals: [1,1], [2,5], [6,11], [12,15]
k=5:  p=23,  intervals: [1,1], [2,4], [5,8], [9,11], [12,21]
k=6:  p=71,  intervals: [8,9], [10,19], [20,51], [52,61], [62,63], [64,70]
k=7:  p=599, intervals: [29,50], [51,122], [123,183], [184,250], [251,289],
               [290,500], [501,539]
k=8:  p=599, same first 7 intervals plus [540,555]
\end{verbatim}
(Each list is consecutive and lies in $[1,p-1]$, hence can be translated to an integer solution by Lemma 1.)

\paragraph{VERIFICATION.}
For $k=2$, $p=5$ gives $[1,1]$ product $1$, and $[2,3]$ product $6\equiv 1\pmod 5$.
For $k=3$, $p=17$ and the reported intervals satisfy the factorial equalities $1!\equiv 0!$, $5!\equiv 1!$, and $11!\equiv 5!$ (verified by the program).

\paragraph{FINAL: UNRESOLVED.}
(i) \emph{Strongest proved partial result here.} The problem is equivalent to finding long chains of repeated factorial values modulo $p$ (Lemma 2), and explicit solutions exist at least for $2\le k\le 8$ with primes and intervals as above.
(ii) \emph{First gap.} Prove that for every $k$ there exists some prime $p$ with a factorial repetition chain of length $k$, or else exhibit some $k$ and prove that no such prime $p$ exists.
(iii) \emph{Top 3 next moves.} (1) Study the directed graph on $\{0,1,\dots,p-1\}$ with edges $a\to b$ iff $a<b$ and $b!\equiv a!\pmod p$; show that its longest path length can be made arbitrarily large by choosing $p$. (2) Use Wilson's theorem $(p-1)!\equiv -1\pmod p$ to relate factorial collisions near $p-1$ to multiplicative inverses and possibly force repetitions. (3) Extend computation to larger $k$ to guess the growth of the minimal prime $p(k)$ and look for constructive patterns in the interval endpoints.
(iv) \emph{What a minimal counterexample would look like.} A counterexample would be a specific $k$ such that for every prime $p$, the factorial sequence $0!,1!,\dots,(p-1)!$ has fewer than $k+1$ indices with equal factorial values along an increasing chain, i.e. the graph described in (iii) has maximum path length $<k$.


