% Solutions file generated from /mnt/data/1083-1087.tex

% Erdos Problem #1083

\subsection*{Erd\H{o}s Problem 1083 (distinct distances in \(\mathbb{R}^d\), \(d\ge 3\))}

\subsubsection*{FORMAL RESTATEMENT}
Fix an integer \(d\ge 3\). For a finite set \(P\subset \mathbb{R}^d\) of size \(|P|=n\), let
\[
\Delta(P):=\{\,\|x-y\|: x,y\in P,\ x\ne y\,\}
\]
be the set of distinct (Euclidean) distances determined by \(P\). Define
\[
f_d(n):=\min\{\,|\Delta(P)|: P\subset\mathbb{R}^d,\ |P|=n\,\}.
\]
The problem asks for asymptotics of \(f_d(n)\) as \(n\to\infty\) with \(d\) fixed, and in particular whether
\(f_d(n)=n^{\frac{2}{d}-o(1)}\).

\subsubsection*{QUICK LITERATURE/CONTEXT CHECK}
The problem statement itself records bounds of Erd\H{o}s,
\(n^{1/d}\ll_d f_d(n)\ll_d n^{2/d}\), and later improvements for \(d=3\) and general \(d\ge 4\).
In what follows I do \emph{not} use any additional literature results beyond what is already stated in the problem file.

\subsubsection*{ATTACK PLAN}
\begin{itemize}
\item \textbf{Proof track (partial).} Establish the classical lower bound \(f_d(n)\gg_d n^{1/d}\) via a polynomial method argument, and the lattice upper bound \(f_d(n)\ll_d n^{2/d}\) via an explicit grid construction.
\item \textbf{Disproof track.} Not applicable here (the question is to estimate/decide an asymptotic, not a single yes/no statement with fixed parameters).
\end{itemize}

\subsubsection*{WORK}

\paragraph{Fast reality check.}
\(f_d(n)=1\) is possible when \(n\le d+1\) (take the vertices of a regular simplex). For the \(d\)-dimensional integer grid \([0,k-1]^d\) (so \(n=k^d\)), a small computation gives (for squared distances):
\(d=3,k=2: |\Delta|=3\); \(d=3,k=3: |\Delta|=9\);
\(d=4,k=2: |\Delta|=4\). (Computed exactly by brute force.)

\begin{lemma}[Polynomial-method lower bound]
Let \(P\subset\mathbb{R}^d\) have \(|P|=n\), and let \(|\Delta(P)|=m\). Then
\[
 n\le \binom{d+2m}{d}.
\]
In particular, for fixed \(d\) there is a constant \(c_d>0\) such that
\(m\ge c_d\, n^{1/d}\).
\end{lemma}

\begin{proof}
Let \(\{\delta_1,\dots,\delta_m\}\) be the set of \emph{squared} distances determined by \(P\):
\(\delta_i\in \{\|x-y\|^2: x\ne y\in P\}\), all distinct.
Fix an ordering \(P=\{p_1,\dots,p_n\}\subset\mathbb{R}^d\).
For each \(i\in\{1,\dots,n\}\) define a polynomial in \(d\) variables
\[
F_i(x):=\prod_{t=1}^m \big(\|x-p_i\|^2-\delta_t\big).
\]
Each factor \(\|x-p_i\|^2-\delta_t\) is a polynomial of total degree \(2\) in \(x\), hence \(F_i\) has total degree \(2m\).
Now evaluate at the points \(p_j\):
\begin{itemize}
\item If \(j\ne i\), then \(\|p_j-p_i\|^2\) equals one of the \(\delta_t\), so one factor vanishes and hence \(F_i(p_j)=0\).
\item If \(j=i\), then \(F_i(p_i)=\prod_{t=1}^m(0-\delta_t)\ne 0\), since all \(\delta_t>0\) (distinct points).
\end{itemize}
Therefore the \(n\times n\) matrix \(M=(F_i(p_j))_{i,j}\) is diagonal with nonzero diagonal entries, so it is invertible.
In particular, \(F_1,\dots,F_n\) are linearly independent in the vector space of real polynomials in \(d\) variables of total degree at most \(2m\).
The dimension of that space is \(\binom{d+2m}{d}\) (number of monomials of total degree \(\le 2m\) in \(d\) variables).
Hence \(n\le \binom{d+2m}{d}\).

For the asymptotic consequence: for fixed \(d\),
\(\binom{d+2m}{d}\le C_d (2m)^d\) for a constant \(C_d\) (e.g. \(C_d=1/d!\) times a factor depending only on \(d\)). Thus
\(n\le C'_d m^d\), so \(m\ge c_d n^{1/d}\).
\end{proof}

\begin{lemma}[Grid/lattice upper bound]
For each integer \(k\ge 2\), let \(P=[0,k-1]^d\cap \mathbb{Z}^d\) so \(|P|=k^d\). Then
\[
|\Delta(P)|\le d(k-1)^2.
\]
Consequently, for all \(n\) there exists a set of \(n\) points in \(\mathbb{R}^d\) determining at most \(C_d n^{2/d}\) distinct distances.
\end{lemma}

\begin{proof}
For \(x,y\in P\), the squared distance is
\(\|x-y\|^2=\sum_{j=1}^d (x_j-y_j)^2\).
Each coordinate difference \(x_j-y_j\) is an integer in \([-(k-1),k-1]\), so \((x_j-y_j)^2\in\{0,1,\dots,(k-1)^2\}\). Hence
\(\|x-y\|^2\in\{0,1,\dots,d(k-1)^2\}\).
Excluding \(0\) (which corresponds to \(x=y\)) shows there are at most \(d(k-1)^2\) possible nonzero squared distances, and therefore at most that many distances.

Now given general \(n\), choose \(k=\lceil n^{1/d}\rceil\) and take any subset of \(P\) of size \(n\). This does not increase the number of distinct distances, and yields
\(|\Delta|\ll_d k^2\ll_d n^{2/d}\).
\end{proof}

\subsubsection*{VERIFICATION}
\begin{itemize}
\item The polynomial argument uses only that the set of squared distances has size \(m\) and that all \(\delta_t>0\); this holds for any set of distinct points.
\item Degree count: \(\|x-p_i\|^2\) is a quadratic polynomial; the product over \(m\) values has degree \(2m\).
\item Grid bound counts \emph{possible} squared distance values, giving an upper bound (not necessarily tight), which is all that is needed.
\end{itemize}

\subsubsection*{FINAL}
\textbf{UNRESOLVED.}
\begin{enumerate}
\item[(i)] \textbf{Strongest proved partial result here:} for fixed \(d\ge 3\),
\(c_d n^{1/d}\le f_d(n)\le C_d n^{2/d}\), with fully proved lemmas above.
\item[(ii)] \textbf{First gap:} improve the lower bound exponent to \(2/d-o(1)\) (or disprove it) for any fixed \(d\ge 3\).
\item[(iii)] \textbf{Top 3 next moves:}
(1) sharpen the polynomial method by using a lower-degree basis tailored to the realized distance set;
(2) attempt incidence bounds between points and spheres to upper bound multiplicities of distances and force more distinct values;
(3) computational search in \(d=3\) for extremal configurations for modest \(n\) to guess extremizers beyond grids.
\item[(iv)] \textbf{Minimal counterexample structure (if the conjectured \(n^{2/d-o(1)}\) fails):}
families of \(n\)-point sets in fixed \(\mathbb{R}^d\) with unusually many repeated distances, plausibly coming from highly structured algebraic varieties (unions of spheres/cylinders/cones) rather than full-dimensional lattices.
\end{enumerate}


