
\noindent\textbf{FORMAL RESTATEMENT.}
Let $R(3,k)$ denote the classical Ramsey number: the smallest integer $n$ such that every graph on $n$ vertices contains either a triangle $K_3$ or an independent set of size $k$.
The problem asks to show
\[
R(3,k+1)-R(3,k)\to\infty\quad (k\to\infty),
\]
and also to prove or disprove the stronger claim
\[
R(3,k+1)-R(3,k)=o(k).
\]

\noindent\textbf{QUICK LITERATURE/CONTEXT CHECK.}
No external results are used here. I provide only elementary bounds and sanity checks.

\noindent\textbf{ATTACK PLAN.}
1) Reprove the standard recursion upper bound for $R(3,k)$ and an elementary construction-based lower bound.
2) Use these to understand what they imply (and do not imply) about first differences.
3) Verify the constructions in small cases computationally (triangle-freeness and independence number).

\noindent\textbf{WORK.}
\textbf{Fast reality check (cycle construction verification).}
For each $k$ from $3$ to $8$, I verified by a short script that the cycle graph $C_{2k-2}$ is triangle-free and has independence number $k-1$.
This supports the lower bound Lemma 544.2 below.

\medskip
\textbf{Lemma 544.1 (standard recursion upper bound).}
For $k\ge 2$,
\[
R(3,k)\le R(3,k-1)+k.
\]
In particular, $R(3,k)\le \frac{k(k+1)}{2}$ for all $k\ge 2$.

\textbf{Proof.}
Let $n=R(3,k-1)+k$ and let $G$ be any graph on $n$ vertices.
Pick a vertex $v$.
Let $N(v)$ be its neighbor set and $\overline N(v)$ its non-neighbor set (excluding $v$ itself).
Then $|N(v)|+|\overline N(v)|=n-1=R(3,k-1)+k-1$.
If $|N(v)|\ge R(2,k)=k$, then $N(v)$ contains an independent set of size $k$ in the complement graph; equivalently, either $N(v)$ contains an edge (which together with $v$ forms a triangle) or $N(v)$ contains an independent set of size $k$ in $G$.
More directly for $R(3,k)$, we use the standard Ramsey argument:
If $|N(v)|\ge R(2,k)=k$, then either $N(v)$ contains an edge, forming a triangle with $v$, or else $N(v)$ itself is an independent set of size $k$.
So in this case $G$ contains a triangle or an independent $k$-set.

Otherwise $|N(v)|\le k-1$, hence $|\overline N(v)|\ge R(3,k-1)$.
By definition of $R(3,k-1)$, the induced subgraph on $\overline N(v)$ contains either a triangle (which is also a triangle in $G$) or an independent set of size $k-1$.
In the latter case, adding $v$ (which has no edges to $\overline N(v)$) yields an independent set of size $k$.
Thus every graph on $n$ vertices has a triangle or an independent $k$-set, proving $R(3,k)\le R(3,k-1)+k$.
Summing this inequality from $2$ up to $k$ and using $R(3,2)=3$ gives $R(3,k)\le k(k+1)/2$. \qed

\medskip
\textbf{Lemma 544.2 (elementary lower bound via cycles).}
For $k\ge 2$,
\[
R(3,k)\ge 2k-1.
\]

\textbf{Proof.}
Consider the cycle graph $C_{2k-2}$ on $2k-2$ vertices.
A cycle has no triangles, so $C_{2k-2}$ is triangle-free.
Its independence number is $k-1$: in an even cycle, the maximum independent set consists of every other vertex, giving size $(2k-2)/2=k-1$, and no independent set can be larger because any independent set contains at most one vertex from each edge-pair.
Therefore there exists a triangle-free graph on $2k-2$ vertices with no independent set of size $k$.
By the definition of $R(3,k)$, this implies $R(3,k)>2k-2$, hence $R(3,k)\ge 2k-1$. \qed

\noindent\textbf{VERIFICATION.}
The proof of Lemma 544.1 is the standard vertex-neighborhood recursion specialized to $(3,k)$.
Lemma 544.2 was computationally checked for $k=3,\dots,8$ by computing independence numbers of $C_{2k-2}$.

\noindent\textbf{FINAL.}\;\textbf{UNRESOLVED.}
\begin{itemize}
\item[(i)] \textbf{Strongest proved partial result.} Elementary bounds:
\[2k-1\le R(3,k)\le \frac{k(k+1)}{2}.
\]
These do not resolve the growth of the first difference $R(3,k+1)-R(3,k)$.
\item[(ii)] \textbf{First gap (crisp statement).} Prove a lower bound on $R(3,k+1)-R(3,k)$ that tends to infinity with $k$, i.e. find an explicit function $g(k)\to\infty$ such that $R(3,k+1)\ge R(3,k)+g(k)$ for all sufficiently large $k$.
\item[(iii)] \textbf{Top 3 next moves.}
  \begin{itemize}
  \item Study extremal triangle-free graphs with bounded independence number (Ramsey graphs): relate $R(3,k)-1$ to the maximum $n$ for which there exists triangle-free $G$ on $n$ vertices with $\alpha(G)\le k-1$. Any method that forces $n$ to increase by $\omega(1)$ when $k$ increases by $1$ would prove the first claim.
  \item Attempt to prove an approximate convexity/smoothness property for $R(3,k)$ (e.g. $R(3,k+1)+R(3,k-1)\ge 2R(3,k)+\omega(1)$), which would imply increasing differences.
  \item Computationally: for moderate $k$ (where exact Ramsey numbers are known by computation), examine the empirical behavior of differences to guess the correct scale for the two questions.
  \end{itemize}
\item[(iv)] \textbf{Minimal counterexample structure to look for.} If $R(3,k+1)-R(3,k)$ failed to grow, one would need infinitely many $k$ for which there exists a triangle-free graph on $R(3,k)-1$ vertices with independence $k-1$ that cannot be extended (by adding many vertices) while keeping triangle-free and keeping independence $\le k$. Understanding such ``rigid'' extremal Ramsey graphs is a plausible route to either proving or disproving divergence of the difference.
\end{itemize}

