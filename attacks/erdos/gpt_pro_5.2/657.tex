
Is it true that if $A\subset \mathbb{R}^2$ is a set of $n$ points such that every subset of $3$ points determines $3$ distinct distances (i.e. $A$ has no isosceles triangles) then $A$ must determine at least $f(n)n$ distinct distances, for some $f(n)\to \infty$? In \cite{Er73} Erd\H{o}s attributes this problem (more generally in $\mathbb{R}^k$) to himself and Davies. In \cite{Er97e} he does not mention Davis, but says this problem was investigated by himself, F\"{u}redi, Ruzsa, and Pach. In \cite{Er73} Erd\H{o}s says it is not even known in $\mathbb{R}$ whether $f(n)\to \infty$. Sarosh Adenwalla has observed that this is equivalent to minimising the number of distinct differences in a set $A\subset \mathbb{R}$ of size $n$ without three-term arithmetic progressions. Dumitrescu \cite{Du08} proved that, in these terms,\[(\log n)^c \leq f(n) \leq 2^{O(\sqrt{\log n})}\]for some constant $c>0$. Hunter observed in the comments that a result of Ruzsa coupled with standard tools of additive combinatorics (with details given by Alfaiz and Tang) allow recent progress on the size of subsets without three-term arithmetic progression (see \cite{BlSi23} which improves slightly on the bounds due to Kelley and Meka \cite{KeMe23}) yield\[2^{c(\log n)^{1/9}}\leq f(n)\]for some constant $c>0$. Straus has observed that if $2^k\geq n$ then there exist $n$ points in $\mathbb{R}^k$ which contain no isosceles triangle and determine at most $n-1$ distances. See also [135] . References [BlSi23] T. F. Bloom and O. Sisask, An improvement to the Kelley-Meka bounds on three-term arithmetic progressions . arXiv:2309.02353 (2023). [Du08] Dumitrescu, Adrian, On distinct distances and {$\lambda$}-free point sets . Discrete Math. (2008), 6533--6538. [Er73] Erd\H{o}s, P., Problems and results on combinatorial number theory . A survey of combinatorial theory (Proc. Internat. Sympos., Colorado State Univ., Fort Collins, Colo., 1971) (1973), 117-138. [Er97e] Erd\H{o}s, Paul, Some of my favourite unsolved problems . Math. Japon. (1997), 527-537. [KeMe23] Kelley, Z. and Meka, R., Strong Bounds for 3-Progressions . arXiv:2302.05537 (2023).


1) FORMAL RESTATEMENT
Let $n\ge 1$ and let $A\subset\mathbb{R}^2$ with $|A|=n$.
Assume that for every triple of distinct points $a,b,c\in A$ the three pairwise distances
$|a-b|,|a-c|,|b-c|$ are all distinct (equivalently, $A$ contains no isosceles triangle).
Let
\[
D(A):=\bigl|\bigl\{|x-y|\ :\ x,y\in A,\ x\neq y\bigr\}\bigr|
\]
be the number of distinct (nonzero) distances determined by $A$.
Question: does there exist a function $f:\mathbb{N}\to\mathbb{R}_{>0}$ with $f(n)\to\infty$
such that every such $A$ satisfies $D(A)\ge f(n)\,n$?

Edge cases: for $n\le 2$ the hypothesis is vacuous and $D(A)\in\{0,1\}$.

2) QUICK LITERATURE/CONTEXT CHECK
The problem statement records nontrivial lower bounds coming from additive-combinatorial reformulations
(Dumitrescu; later improvements in the comments on erdosproblems.com). I do not rely on any external
results beyond what is explicitly written in the problem statement.

3) ATTACK PLAN
Proof-track ideas:
- Use the ``no isosceles'' constraint to bound multiplicities of repeated distances, then apply
  double counting to force many distinct distances.
- Translate to 1D difference-set questions (as in the statement) and try to import bounds for
  3-term-AP-free sets.

Disproof-track ideas:
- Attempt to construct point sets with no isosceles triangles but only $O(n)$ distinct distances,
  e.g. by arranging repeated distances along a large matching of equal-length segments.

At present I only reach linear lower bounds from basic multiplicity control.

4) WORK

Lemma 1 (distinct distances from a fixed apex).
If $A\subset\mathbb{R}^2$ has no isosceles triangle and $p\in A$, then the $n-1$ distances
$\{|p-q|: q\in A\setminus\{p\}\}$ are all distinct. In particular $D(A)\ge n-1$ for $n\ge 2$.

Proof.
Fix $p\in A$. Suppose $q,r\in A\setminus\{p\}$ are distinct and satisfy $|p-q|=|p-r|$.
Then the triple $\{p,q,r\}$ forms an isosceles triangle with apex $p$, contradicting the hypothesis.
Thus the $n-1$ distances from $p$ to the other points are pairwise distinct.
Since $D(A)$ counts distinct distances appearing among all pairs, it is at least the number of distinct
distances appearing among the pairs $\{p,q\}$ with $q\neq p$, namely $n-1$. $\square$

Lemma 2 (fixed-length edges form a matching).
Let $A\subset\mathbb{R}^2$ have no isosceles triangle, and fix a distance value $d>0$.
Consider the graph on vertex set $A$ whose edges are the unordered pairs $\{x,y\}$ with $|x-y|=d$.
Then this graph has maximum degree $1$; equivalently, the edges of length $d$ form a matching.
In particular, the multiplicity of the distance value $d$ among all $\binom n2$ pairs is at most
$\lfloor n/2\rfloor$.

Proof.
If a vertex $x\in A$ were incident to two distinct edges $\{x,y\}$ and $\{x,z\}$ of length $d$,
then $|x-y|=|x-z|=d$ with $y\neq z$, so $\{x,y,z\}$ would be an isosceles triangle with apex $x$,
contradiction. Hence every vertex has degree at most $1$, so the edges form a matching and there are at
most $\lfloor n/2\rfloor$ such edges. $\square$

Proposition 3 (double counting).
If $A$ has no isosceles triangle and $n\ge 2$, then
\[
D(A)\ \ge\ \frac{\binom n2}{\lfloor n/2\rfloor}\ \ge\ n-1.
\]

Proof.
Write $\binom n2=\sum_{d} m_d$, where the sum runs over distinct distance values $d$ and $m_d$
is the number of pairs at distance $d$. By Lemma 2, $m_d\le \lfloor n/2\rfloor$ for every $d$, so
$\binom n2\le D(A)\,\lfloor n/2\rfloor$, giving the first inequality. The second inequality is a direct
check: if $n=2m$ then $\binom n2/(n/2)=2m-1=n-1$; if $n=2m+1$ then $\binom n2/\lfloor n/2\rfloor=2m+1=n$.
$\square$

FAST REALITY CHECK (brute force on the integer grid $[-2,2]^2$).
I searched all $n$-subsets of $[-2,2]^2$ and minimized the number of distinct squared distances subject
to the no-isosceles constraint. Results (distinct distances, then one witness set):
- $n=3$: $D=3$, e.g. $\{(-2,-2), (-2,-1), (-2,1)\}$.
- $n=4$: $D=3$, e.g. $\{(-2,-2), (-2,-1), (0,-2), (0,-1)\}$.
- $n=5$: $D=6$, e.g. $\{(-2,-1), (-1,1), (0,2), (1,-2), (2,0)\}$.
- $n=6$: $D=7$, e.g. $\{(-2,-2), (-2,-1), (-2,1), (-2,2), (0,-2), (0,2)\}$.
These are only sanity checks inside a small search window, not proofs of the true minima.

5) VERIFICATION
- Lemma 1: the only step is that $|p-q|=|p-r|$ produces an isosceles triangle with apex $p$, which is
  exactly the forbidden configuration.
- Lemma 2: I checked that forbidding isosceles triangles rules out two equal-length edges sharing a
  common endpoint; the conclusion that the equal-length edges form a matching follows.
- Proposition 3: the decomposition $\binom n2=\sum_d m_d$ and the inequality $m_d\le\lfloor n/2\rfloor$
  are correct, so the bound is valid.

6) FINAL: **UNRESOLVED**
(i) Strongest proved partial result: $D(A)\ge n-1$ for every $n\ge 2$, and moreover each fixed distance
value occurs at most $\lfloor n/2\rfloor$ times (Lemma 2), hence $D(A)\ge \binom n2/\lfloor n/2\rfloor$.
(ii) First gap (crisp): prove any superlinear bound $D(A)\ge n\cdot g(n)$ with $g(n)\to\infty$ (even
$g(n)=\log\log n$) for all no-isosceles $A\subset\mathbb{R}^2$.
(iii) Top 3 next moves (concrete):
  1. Prove an improved multiplicity bound $m_d=o(n)$ for many $d$, e.g. show equal-length matchings
     cannot cover a positive fraction of all pairs once higher-order constraints are used.
  2. Work in the 1D reformulation (distinct differences without 3-term AP) and push the strongest known
     lower bounds for $|A-A|$ in that setting back to planar configurations.
  3. Computation: search for structured large examples with small $D(A)$ to guess extremizers and then
     attempt a stability theorem around those extremizers.
(iv) Minimal counterexample structure (if the conjecture is true): a family $A_n$ with $D(A_n)=O(n)$
would need distance multiplicities close to the matching upper bound $\lfloor n/2\rfloor$ for many
distance values, i.e. most pairs partitioned into near-perfect matchings by distance classes.

