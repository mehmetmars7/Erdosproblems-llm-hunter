% Erdos Problem #324
% URL: https://www.erdosproblems.com/324

1) FORMAL RESTATEMENT

We seek a polynomial $f(x)\in\mathbb Z[x]$ such that for all pairs of nonnegative integers $a<b$ and $c<d$,
\[
\bigl(\{a,b\}\neq\{c,d\}\bigr)\ \Longrightarrow\ f(a)+f(b)\neq f(c)+f(d).
\]
Equivalently, the map
\[
\{a,b\}\mapsto f(a)+f(b)
\]
from 2-element subsets of $\mathbb Z_{\ge 0}$ to $\mathbb Z$ is injective.

Question: does such an integer-coefficient polynomial exist?

Edge cases.
If one allowed $a=b$ then $f(a)+f(a)$ would collide trivially under permutations; the problem explicitly requires $a<b$.


2) QUICK LITERATURE/CONTEXT CHECK

I do not use external literature.
The statement itself suggests:

* ``Probably $f(x)=x^5$ should work.''
* The Lander--Parkin--Selfridge conjecture would imply $f(x)=x^n$ works for all $n\ge 5$.


3) ATTACK PLAN

I cannot resolve existence. I will instead:

(1) Prove that degree-1 polynomials cannot work.

(2) Give explicit counterexamples showing that the monomials $x^2$, $x^3$, and $x^4$ do not work.

(3) Do a computation-based sanity check: search for collisions for $x^d$ for $d=5,6$ up to a moderate range.


4) WORK

PHASE 1: FAST REALITY CHECK (collision search for $f(x)=x^d$)

For $f(x)=x^d$ and bounds $0\le a<b\le M$, I searched for distinct pairs with equal sums $a^d+b^d=c^d+d^d$.
The first collisions found were:

* $d=2$, already within $M=20$: $0^2+5^2=3^2+4^2=25$.
* $d=3$, already within $M=20$: $1^3+12^3=9^3+10^3=1729$.
* $d=4$, no collision up to $M=100$, but a collision appears by $M=200$:
  $59^4+158^4=133^4+134^4=635318657$.

For $d=5$ and $d=6$, no collision was found for $M\le 200$.
(This does not prove injectivity for any $d\ge 5$; it is only a sanity check.)


Lemma 324.1 (no linear polynomial works).

No polynomial of degree $\le 1$ has the required property.

Proof.
Let $f(x)=ux+v$ with $u,v\in\mathbb Z$.
If $u=0$, then $f$ is constant and $f(a)+f(b)$ is constant, so injectivity fails.
Assume $u\neq 0$.
Then
\[
f(a)+f(b)=u(a+b)+2v.
\]
Thus $f(a)+f(b)$ depends only on $a+b$.
Choose distinct pairs with the same sum, e.g.
\[
(a,b)=(0,3),\qquad (c,d)=(1,2),
\]
which satisfy $0<3$, $1<2$, and $a+b=c+d=3$.
Then $f(0)+f(3)=f(1)+f(2)$, so the required distinctness fails.
\qed


Lemma 324.2 ($f(x)=x^2$ fails).

For $f(x)=x^2$ there exist distinct pairs $a<b$ and $c<d$ with $f(a)+f(b)=f(c)+f(d)$.

Proof.
Take $(a,b)=(0,5)$ and $(c,d)=(3,4)$.
Then
\[
0^2+5^2=25=9+16=3^2+4^2.
\]
The pairs are distinct and satisfy $0<5$, $3<4$. So $x^2$ fails. \qed


Lemma 324.3 ($f(x)=x^3$ fails).

For $f(x)=x^3$ there exist distinct pairs $a<b$ and $c<d$ with $f(a)+f(b)=f(c)+f(d)$.

Proof.
Take $(a,b)=(1,12)$ and $(c,d)=(9,10)$.
Compute
\[
1^3+12^3 = 1+1728 = 1729,
\]
and
\[
9^3+10^3 = 729+1000 = 1729.
\]
The pairs are distinct and satisfy $1<12$, $9<10$. So $x^3$ fails. \qed


Lemma 324.4 ($f(x)=x^4$ fails).

For $f(x)=x^4$ there exist distinct pairs $a<b$ and $c<d$ with $f(a)+f(b)=f(c)+f(d)$.

Proof.
Take $(a,b)=(59,158)$ and $(c,d)=(133,134)$.
Direct computation gives
\[
59^4+158^4 = 635318657 = 133^4+134^4.
\]
The pairs are distinct and satisfy $59<158$, $133<134$. So $x^4$ fails. \qed


5) VERIFICATION

-- Lemma 324.1: the collision uses only additivity of a linear function.

-- Lemmas 324.2--324.4: each is verified by explicit arithmetic equality; order constraints $a<b$ are satisfied.

-- Computation: the collision search was an exhaustive check over pairs $0\le a<b\le M$ for each $M$ reported.


6) FINAL

**UNRESOLVED**

(i) Strongest fully proved partial result obtained here.

* No linear polynomial works (Lemma 324.1).
* The monomials $x^2$, $x^3$, and $x^4$ fail by explicit equal-sum identities (Lemmas 324.2--324.4).
* Computation finds no collisions for $x^5$ or $x^6$ on the range $0\le a<b\le 200$.

(ii) Exact first gap.

Construct (or rule out) an integer-coefficient polynomial $f$ such that $f(a)+f(b)$ is injective on pairs $a<b$.
Even proving this for the specific candidate $f(x)=x^5$ is beyond what I achieved.

(iii) Top 3 next moves (concrete targets).

1. Attempt a proof for $f(x)=x^5$ by showing the Diophantine equation $a^5+b^5=c^5+d^5$ has no nontrivial solutions in nonnegative integers with $a<b$, $c<d$.
2. Search computationally for the smallest collision for $x^5$ (if any) by extending the brute-force range significantly and/or using modular sieving.
3. Explore mixed-degree polynomials such as $f(x)=x^5+Cx^2$ to break any potential modular obstructions.

(iv) What a minimal counterexample would likely look like.

If no such polynomial exists, then for every $f\in\mathbb Z[x]$ there must be distinct pairs $a<b$, $c<d$ with $f(a)+f(b)=f(c)+f(d)$.
A minimal counterexample to existence would likely arise from a structured parametric family of solutions to the Diophantine equation
\[f(a)+f(b)=f(c)+f(d),\]
analogous to known parametric families for equal sums of like powers.


