\section{Erd\H{o}s Problem \#41 (Round 3)}

\subsection*{Problem}
An infinite set $A\subset\mathbb N$ is a \emph{$B_3$-set} if all sums
\[
 a_1+a_2+a_3\qquad (a_1\le a_2\le a_3,\ a_i\in A)
\]
are distinct.  The Erd\H{o}s question is whether every infinite $B_3$-set satisfies
\[
\liminf_{N\to\infty}\frac{|A\cap[1,N]|}{N^{1/3}}=0.
\]

\subsection*{1) ROUND-3 OBJECTIVE}
\textbf{Path (C): obstruction/correction + strict progress.}
Round~2 claimed a full proof by invoking ``Helm obstructions'' (H1)--(H2) as if they were unconditional.
In this round we (i) identify the precise missing hypothesis in Helm's Theorem~1
(the proof only works for \emph{monotonically distributed} sequences),
(ii) record the strongest correct consequence obtainable from Helm (pseudo-cubes and monotone distribution are ruled out), and
(iii) add a new structural lemma (a local version of Lemma~41.1) and a quantitative ``early-gap \,$\Rightarrow$\,small ratio'' criterion.
This isolates a concrete remaining obstruction: to prove Erd\H{o}s' liminf statement it would suffice to force Helm's long gap blocks to begin sufficiently early.

\subsection*{2) Round-2 foundation used}
We rely on the following vetted Round~1 facts.
\begin{itemize}
\item \textbf{Lemma 41.1 (Round~1).} If $A\subseteq[1,N]$ is $B_3$ and $m:=|A|$, then $\binom{m+2}{3}\le 3N-2$, hence $m\le (18N)^{1/3}+2$.
\item \textbf{Lemma 41.2 (Round~1).} Every infinite $B_3$-set is Sidon ($B_2$).
\end{itemize}

\subsection*{3) New insight / tool (Round 3)}
\begin{enumerate}
\item \textbf{Correct reading of Helm (1996).} Helm's key non-$B_3$ criterion (his Theorem~1) assumes the sequence is \emph{monotonically distributed}.  This hypothesis was omitted in Round~2, so the Round~2 ``full proof'' does not go through.
\item \textbf{Helm's Theorem~3 (gap blocks).} Under the (still hypothetical) assumption $\liminf A(n)n^{-1/3}>0$, Helm proves the existence of long consecutive blocks of unusually sparse length-$N$ intervals (``$\omega$-gaps'').
\item \textbf{Strengthening Lemma~41.1 to a \emph{local} interval bound.} Any translate of a finite $B_3$ set is $B_3$, hence the $O(N^{1/3})$ bound holds in every interval of length $N$.
\item \textbf{New quantitative criterion.} If Helm-type gap blocks occur \emph{early enough} relative to their length, then the Erd\H{o}s liminf conclusion follows.
\end{enumerate}

\subsection*{4) Attack plan (Round 3)}
\begin{itemize}
\item \textbf{Gap after Round~2.} Round~2 treated Helm's Theorem~1 as an unconditional obstruction (H2).  In fact Helm assumes \emph{monotone distribution}.  There is no proof (and it is not true in general) that every $B_3$ set is monotonically distributed.
\item \textbf{What we can still do.} We state the correct Helm theorems and derive rigorous partial consequences.  We then combine them with the local interval bound to obtain a clean ``if the gap block starts early, then the normalized counting function dips to~0'' lemma.
\item \textbf{Why this is progress.} It isolates one sharply formulated missing step: show that Helm's gap blocks (which exist \emph{at all large scales} under $\liminf>0$) can be taken to start within $o(\log^{1/6}N)$ blocks for infinitely many scales.  That condition would settle Erd\H{o}s Problem~41.
\end{itemize}

\subsection*{5) Work (Round 3)}

\paragraph{5.1. Local $O(N^{1/3})$ bound on every interval.}
\begin{lemma}[Local version of Lemma~41.1]
\label{lem:local}
Let $A\subseteq\mathbb N$ be a $B_3$ set.  For any integers $M\ge 0$ and $N\ge 1$,
\[
|A\cap (M,M+N]|\le (18N)^{1/3}+2.
\]
\end{lemma}
\begin{proof}
Let $B:=\{a-M: a\in A\cap(M,M+N]\}$.  Then $B\subseteq[1,N]$ and $B$ is $B_3$ (translation preserves triple-sum distinctness).  Apply Lemma~41.1 (Round~1) to $B$.
\end{proof}

\paragraph{5.2. What Helm actually proves (and what he does not).}
We record the parts of Helm's paper that are directly relevant.

\begin{definition}[Helm: monotonically distributed]
Let $A$ be an increasing sequence of naturals and $A(x):=|A\cap[1,x]|$.
For $N\in\mathbb N$ and $l\in\{1,\dots,N\}$ set
\[
I_l^{(N)}:=((l-1)N,\,lN],\qquad A_l^{(N)}:=|A\cap I_l^{(N)}|.
\]
We call $A$ \emph{monotonically distributed} if there exist functions $\psi:\mathbb N\to\mathbb N$ with $\psi(N)\to\infty$ and a sequence $N_r\to\infty$ such that for all $r$ and all $1\le l<\psi(N_r)$,
\[
A_l^{(N_r)}\ge A_{l+1}^{(N_r)}.
\]
\end{definition}

\begin{theorem}[Helm, Theorem~1 (precise)]
\label{thm:helm1}
If $A$ is \emph{monotonically distributed} and
\[
0<\liminf_{n\to\infty}\frac{A(n)}{n^{1/3}}<\limsup_{n\to\infty}\frac{A(n)}{n^{1/3}}<\infty,
\]
then $A$ is \emph{not} a $B_3$-sequence.
\end{theorem}

\begin{theorem}[Helm, Theorem~2 (pseudo-cubes)]
\label{thm:helm2}
If $A(n)\sim \alpha n^{1/3}$ for some $\alpha>0$ (``pseudo-cube''), then $A$ is not a $B_3$-sequence.
\end{theorem}

\begin{remark}[Correction of the Round~2 ``closing step'']
Round~2 asserted two unconditional obstructions:
\begin{quote}
(H1) $A(n)\sim\alpha n^{1/3}$ implies not $B_3$;\quad
(H2) $0<\liminf A(n)n^{-1/3}<\limsup A(n)n^{-1/3}<\infty$ implies not $B_3$.
\end{quote}
(H1) matches Helm's Theorem~2.
However (H2) is \emph{missing the hypothesis ``$A$ is monotonically distributed''} from Helm's Theorem~1.
Thus the Round~2 deduction ``$\liminf>0$ forces a contradiction'' is not valid.
\end{remark}

\paragraph{5.3. Correct partial consequence of Helm + Lemma 41.1.}
\begin{corollary}[Corrected Round~2 conclusion]
\label{cor:monotone-case}
Let $A\subset\mathbb N$ be an infinite $B_3$-set and $A(n)=|A\cap[1,n]|$.
\begin{enumerate}
\item If $A$ is monotonically distributed, then necessarily
\[
\liminf_{n\to\infty}\frac{A(n)}{n^{1/3}}=0.
\]
\item In particular, no $B_3$-sequence can be pseudo-cube ($A(n)\sim \alpha n^{1/3}$).
\end{enumerate}
\end{corollary}
\begin{proof}
By Lemma~41.1 (Round~1), the normalized ratio $A(n)/n^{1/3}$ is bounded above.
If $A$ were monotonically distributed and had $\liminf>0$, then either $\liminf<\limsup$ and Helm Theorem~\ref{thm:helm1} contradicts $B_3$, or $\liminf=\limsup>0$ and Helm Theorem~\ref{thm:helm2} contradicts $B_3$.
\end{proof}

\paragraph{5.4. Helm's gap-block theorem (what any counterexample must look like).}
\begin{definition}[$\omega$-gap (Helm)]
Given $\omega:\mathbb N\to(0,\infty)$ and a scale $N$, we call $I_l^{(N)}$ an \emph{$\omega$-gap} if
\[
A_l^{(N)}<\frac{N^{1/3}}{\omega(N)\,l^{2/3}}.
\]
\end{definition}

\begin{theorem}[Helm, Theorem~3]
\label{thm:helm3}
Assume $A$ is a $B_3$-sequence with $\liminf_{n\to\infty}A(n)n^{-1/3}>0$.
Fix $\delta>0$ and define $\omega(n):=(\log n)^{1/6-\delta}$.
Then for any sufficiently large $N$ and any $\alpha$ with $0<\alpha\le 1/2$, there exist at least
\[
\kappa_N:=\lfloor \log^{1/6}N\rfloor
\]
\emph{consecutive} $\omega$-gaps of $A$ below $N^{1+\alpha}$.
Equivalently, there exists $l_0<N^\alpha$ such that for all $l_0<l\le l_0+\kappa_N$,
\[
A_l^{(N)}<\frac{N^{1/3}}{l^{2/3}(\log N)^{1/6-\delta}}.
\]
\end{theorem}

\begin{remark}[Interpretation]
Theorem~\ref{thm:helm3} shows that any hypothetical $B_3$ counterexample to Erd\H{o}s' liminf claim must be extremely \emph{non-uniform} at every large scale: for every large $N$ it contains a block of $\asymp\log^{1/6}N$ consecutive length-$N$ intervals each having an extra $(\log N)^{1/6-\delta}$ factor of sparsity beyond the ``natural'' $l^{-2/3}$ decay.
\end{remark}

\paragraph{5.5. New lemma: early gap blocks force small normalized density.}
Theorem~\ref{thm:helm3} does not control where the gap block begins.
The next lemma shows that \emph{if} the gap block begins early enough (relative to its length), then Erd\H{o}s' liminf conclusion follows.

\begin{lemma}[Early consecutive gaps $\Rightarrow$ small ratio]
\label{lem:early-gap}
Let $A\subseteq\mathbb N$ be a $B_3$-set and write $A(x)=|A\cap[1,x]|$.
Fix $N\ge 1$ and integers $l_0\ge 0$ and $K\ge 1$.
Assume that for each $l=l_0+1,\dots,l_0+K$ we have
\begin{equation}
\label{eq:gap-hyp}
A_l^{(N)}\le \frac{N^{1/3}}{\omega(N)\,l^{2/3}}
\end{equation}
for some parameter $\omega(N)\ge 1$.
Then, with $X:=(l_0+K)N$, we have
\begin{equation}
\label{eq:ratio-bound}
\frac{A(X)}{X^{1/3}}
\le
(18)^{1/3}\Big(\frac{l_0}{l_0+K}\Big)^{1/3}
+\frac{C}{\omega(N)}\Big(\frac{K}{l_0+K}\Big)^{1/3},
\end{equation}
where $C>0$ is an absolute constant.
In particular, if $\omega(N)\to\infty$ along a sequence of $N$ and $l_0=o(K)$ along that same sequence, then
\[
\liminf_{x\to\infty}\frac{A(x)}{x^{1/3}}=0.
\]
\end{lemma}

\begin{proof}
Write $X=(l_0+K)N$.  By monotonicity of $A(\cdot)$,
\[
A(X)=A(l_0N)+\sum_{l=l_0+1}^{l_0+K}A_l^{(N)}.
\]
First, Lemma~41.1 (Round~1) applied to the initial segment $[1,l_0N]$ gives
$A(l_0N)\le (18l_0N)^{1/3}+2$.
Second, the gap hypothesis \eqref{eq:gap-hyp} gives
\[
\sum_{l=l_0+1}^{l_0+K}A_l^{(N)}
\le \frac{N^{1/3}}{\omega(N)}\sum_{l=l_0+1}^{l_0+K}\frac{1}{l^{2/3}}.
\]
The sum satisfies $\sum_{l=l_0+1}^{l_0+K}l^{-2/3}\ll \int_{l_0}^{l_0+K}t^{-2/3}\,dt\ll (l_0+K)^{1/3}-l_0^{1/3}\ll K^{1/3}$, hence
\[
\sum_{l=l_0+1}^{l_0+K}A_l^{(N)}\ll \frac{N^{1/3}K^{1/3}}{\omega(N)}.
\]
Divide by $X^{1/3}=N^{1/3}(l_0+K)^{1/3}$ and absorb the additive ``$+2$'' into the constant to obtain \eqref{eq:ratio-bound}.
If $l_0=o(K)$ and $\omega(N)\to\infty$, the right-hand side tends to $0$.
\end{proof}

\begin{remark}[A concrete reduction]
Combine Helm's Theorem~\ref{thm:helm3} with Lemma~\ref{lem:early-gap}.
If one could strengthen Helm's conclusion to ensure that the block of $\kappa_N\asymp\log^{1/6}N$ consecutive $\omega$-gaps may be chosen with start index $l_0=o(\kappa_N)$ infinitely often, then Erd\H{o}s' liminf conjecture would follow.
\end{remark}

\subsection*{6) Adversarial verification}
\begin{itemize}
\item \textbf{Translation invariance (Lemma~\ref{lem:local}).} Triple-sum distinctness is preserved under $a\mapsto a-M$ since $a+b+c=a'+b'+c'$ iff $(a-M)+(b-M)+(c-M)=(a'-M)+(b'-M)+(c'-M)$. Hence the local bound is rigorous.
\item \textbf{The Round~2 flaw.} Helm Theorem~1 explicitly assumes ``monotonically distributed''.  Dropping this hypothesis changes the statement and is not justified by any argument in Round~2.
\item \textbf{Lemma~\ref{lem:early-gap} constants.} The proof only uses (i) Lemma~41.1 (Round~1) at $l_0N$, and (ii) the integral comparison $\sum_{l=l_0+1}^{l_0+K}l^{-2/3}\ll K^{1/3}$.  No hidden hypotheses.
\item \textbf{Edge cases.} If $l_0=0$, then the first term in \eqref{eq:ratio-bound} vanishes and we get $A(KN)/(KN)^{1/3}\ll 1/\omega(N)$, consistent with the formula.
\end{itemize}

\subsection*{7) FINAL}
\textbf{UNRESOLVED (BUT STRICTLY ADVANCED).}
\begin{itemize}
\item The Round~2 ``full proof'' is not correct: it relies on an unconditional form of Helm's Theorem~1 that omits the needed \emph{monotonic distribution} hypothesis.
\item What \emph{is} proved: Erd\H{o}s' liminf statement holds for the important regularity classes covered by Helm
(monotonically distributed sequences and pseudo-cubes), and any putative counterexample with $\liminf>0$ must exhibit Helm's long consecutive gap blocks at every scale.
\item New progress: Lemma~\ref{lem:local} (local interval bound) and Lemma~\ref{lem:early-gap} (early gap blocks force the normalized counting function to dip to $0$) isolate a concrete missing step that would settle the conjecture.
\end{itemize}

\subsection*{8) Completion estimate}
\textbf{COMPLETION: 65\%}

\subsection*{9) References}
\begin{itemize}
\item M. Helm, \emph{On the distribution of $B_3$-sequences}, J. Number Theory 58 (1996), 124--129.
\item (For current-status bookkeeping) Erd\H{o}s Problems website entry for Problem \#41.
\end{itemize}
