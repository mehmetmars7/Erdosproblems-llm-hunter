\section*{Erd\H{o}s Problem \#311}

\subsection*{FORMAL RESTATEMENT}
For each $N\in\mathbb N$, define
\[
\delta(N):=\min\left\{\left|1-\sum_{n\in A}\frac1n\right|:\;
A\subseteq\{1,2,\dots,N\},\;
\forall S\subseteq A,\ \sum_{n\in S}\frac1n\neq 1
\right\}.
\]
Equivalently, we look for a subset $A$ whose reciprocal sum is as close as possible to $1$,
subject to the constraint that \emph{no} subsubset $S\subseteq A$ has reciprocal sum exactly $1$.

The question asks for the asymptotic behavior of $\delta(N)$ as $N\to\infty$; in particular,
whether
\[
\delta(N)=e^{-(c+o(1))N}
\quad\text{for some constant }c\in(0,1).
\]

\textbf{Stress points / ambiguity checks.}
\begin{itemize}
\item The constraint is on \emph{all} subsubsets $S\subseteq A$, including $S=A$; thus any $A$ with $\sum_{n\in A}1/n=1$ is forbidden.
\item Without the constraint, the minimal nonzero distance to $1$ is at least $1/\mathrm{lcm}(1,\dots,N)$ by a denominator argument (proved below).
\item The statement ``$\delta(N)=e^{-(c+o(1))N}$'' should be read as a conjectural asymptotic, not as already-established.
\end{itemize}

\subsection*{QUICK LITERATURE/CONTEXT CHECK}
A quick web check indicates this problem is currently listed as open; known general bounds include
the trivial lower bound $1/\mathrm{lcm}(1,\dots,N)\approx e^{-N}$ and a much weaker (larger) upper
bound of the form $\exp\!\big(-cN/(\log N\,\log\log N)^3\big)$ due to work of Tang (as described on
the Erd\H{o}s Problems site). (See sources cited in the accompanying chat message.)

\subsection*{ATTACK PLAN}
\begin{enumerate}
\item Prove unconditional basic bounds: a clean arithmetic lower bound on $\delta(N)$.
\item Compute small $N$ exactly to calibrate growth and to test whether minimizers violate the ``no subsubset sums to $1$'' constraint.
\item (Unfinished) Seek constructions giving exponentially small $\delta(N)$, or obstructions showing $\delta(N)$ cannot be that small.
\end{enumerate}

\subsection*{WORK}
\textbf{Lemma 1 (Denominator lower bound).}
Let $L_N:=\mathrm{lcm}(1,2,\dots,N)$. For every subset $A\subseteq\{1,\dots,N\}$ we have
\[
\sum_{n\in A}\frac1n = \frac{m}{L_N}
\]
for some integer $m$. Consequently, for any $A$ with $\sum_{n\in A}\frac1n\neq 1$,
\[
\left|1-\sum_{n\in A}\frac1n\right|\ge \frac{1}{L_N},
\]
and hence
\[
\delta(N)\ge \frac{1}{L_N}.
\]

\emph{Proof.}
Each $1/n$ equals $(L_N/n)/L_N$ with $L_N/n\in\mathbb Z$. Summing over $n\in A$ gives an integer
numerator over $L_N$. If the sum is not exactly $1=L_N/L_N$, then the numerator differs from $L_N$
by a nonzero integer, hence by at least $1$ in absolute value. Dividing by $L_N$ yields the bound. \qed

\medskip
\textbf{Remark.} Since $\log L_N = (1+o(1))N$ (a classical estimate equivalent to the prime number theorem),
Lemma 1 gives $\delta(N)\ge \exp(-(1+o(1))N)$.

\medskip
\textbf{Small-$N$ computation.}
Using an exact meet-in-the-middle search on the subset sums of $\{1,2,\dots,N\}$ (scaled by $L_N$),
one finds the following values of $\delta(N)$ for $1\le N\le 25$, together with one witness set $A$.
All the listed witnesses $A$ are ``$1$-avoiding'' in the required sense (no subsubset sums to $1$).

\begin{center}
\small
\begin{tabular}{r|l|l}
$N$ & $\delta(N)$ & one witness $A\subseteq\{1,\dots,N\}$ \\
\hline
1 & $1$ & $\{\}$ \\
2 & $\frac{1}{2}$ & $\{2\}$ \\
3 & $\frac{1}{6}$ & $\{2, 3\}$ \\
4 & $\frac{1}{12}$ & $\{2, 3, 4\}$ \\
5 & $\frac{1}{30}$ & $\{2, 3, 5\}$ \\
6 & $\frac{1}{30}$ & $\{2, 3, 5\}$ \\
7 & $\frac{1}{105}$ & $\{2, 5, 6, 7\}$ \\
8 & $\frac{1}{120}$ & $\{2, 5, 6, 8\}$ \\
9 & $\frac{1}{252}$ & $\{2, 4, 7, 9\}$ \\
10 & $\frac{1}{360}$ & $\{2, 6, 8, 9, 10\}$ \\
11 & $\frac{1}{2310}$ & $\{2, 6, 7, 10, 11\}$ \\
12 & $\frac{1}{2310}$ & $\{3, 4, 7, 10, 11, 12\}$ \\
13 & $\frac{1}{2310}$ & $\{3, 4, 7, 10, 11, 12\}$ \\
14 & $\frac{1}{2772}$ & $\{2, 7, 9, 11, 12, 14\}$ \\
15 & $\frac{1}{2772}$ & $\{2, 7, 9, 11, 12, 14\}$ \\
16 & $\frac{1}{6160}$ & $\{2, 8, 11, 12, 14, 15, 16\}$ \\
17 & $\frac{1}{30940}$ & $\{2, 7, 12, 13, 14, 15, 17\}$ \\
18 & $\frac{1}{30940}$ & $\{2, 7, 12, 13, 14, 15, 17\}$ \\
19 & $\frac{5}{298452}$ & $\{2, 7, 11, 12, 14, 17, 19\}$ \\
20 & $\frac{5}{298452}$ & $\{2, 7, 11, 12, 14, 17, 19\}$ \\
21 & $\frac{11}{1244880}$ & $\{2, 12, 13, 14, 16, 18, 19, 20, 21\}$ \\
22 & $\frac{23}{17907120}$ & $\{4, 7, 9, 11, 14, 15, 16, 17, 19, 21, 22\}$ \\
23 & $\frac{1}{1105104}$ & $\{2, 7, 11, 12, 13, 16, 23\}$ \\
24 & $\frac{1}{1105104}$ & $\{3, 4, 7, 11, 13, 16, 23\}$ \\
25 & $\frac{1}{1593900}$ & $\{2, 9, 11, 14, 20, 21, 22, 23, 25\}$ \\
\end{tabular}
\normalsize
\end{center}

\medskip
These values match the list recorded in the problem discussion thread (at least up to $N=25$),
and illustrate that $\delta(N)$ can be \emph{much} smaller than $1/N$ even for modest $N$.

\subsection*{VERIFICATION (adversarial proof check; stress-test edge cases)}
\begin{itemize}
\item Lemma 1 is purely arithmetic and exact: the key point is that every reciprocal sum over $\{1,\dots,N\}$ is a rational with denominator dividing $L_N$.
\item The computational table: meet-in-the-middle is exact because it operates on integers $L_N/n$ and checks exact equality/inequality to $L_N$; no floating point is needed.
\item The ``$1$-avoiding'' condition for each witness was additionally checked by a second meet-in-the-middle subset-sum test within $A$.
\end{itemize}

\subsection*{FINAL}
\textbf{UNRESOLVED.}

\begin{itemize}
\item \textbf{What is proved here.} The clean unconditional lower bound
$\delta(N)\ge 1/\mathrm{lcm}(1,\dots,N)=\exp(-(1+o(1))N)$, and exact values for $N\le 25$.
\item \textbf{Key obstacle.} No argument here gives a matching exponential \emph{upper} bound
$\delta(N)\le \exp(-(c+o(1))N)$, nor a disproof of such an upper bound.
\item \textbf{Most promising next step.} Develop explicit constructions of $1$-avoiding sets $A$
with reciprocal sum extremely close to $1$ (e.g. via structured Egyptian-fraction approximations),
or prove anti-concentration results preventing too-close approximation under the $1$-avoiding constraint.
\item \textbf{Deliverables if continuing.} Either (i) a construction giving $\delta(N)\le e^{-(c+o(1))N}$,
or (ii) a lower bound $\delta(N)\ge e^{-o(N)}$ (or similar) ruling out the conjectured exponential scale.
\end{itemize}

\subsection*{COMPLETION ESTIMATE}
COMPLETION: 45\%

%%%%%%%%%%%%%%%%%%%%%%%%%%%%%%%%%%%%%%%%%%%%%%%%%%%%%%%%%%%%%%%%%%%%%%%%%%%%%%%
