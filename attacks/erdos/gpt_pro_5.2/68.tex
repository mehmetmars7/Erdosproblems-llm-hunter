% Erd\H{o}s Problem #68

\noindent\textbf{FORMAL RESTATEMENT.}
Define the real number
\[
S:=\sum_{n=2}^{\infty}\frac{1}{n!-1}.
\]
(Each term is well-defined since $n!-1\neq 0$ for $n\ge 2$.)
Question: is $S$ irrational?

\medskip
\noindent\textbf{QUICK LITERATURE/CONTEXT CHECK.}
The problem statement notes:
(i) the decimal expansion is OEIS A331373;
(ii) Weisenberg observed the identity $\sum_{n\ge2}\frac{1}{n!-1}=\sum_{k\ge1}\sum_{n\ge2}\frac{1}{(n!)^k}$;
(iii) Erd\H{o}s noted (in the cited source) that $\sum \frac{1}{n!+t}$ should be transcendental for each integer $t$.
No additional literature claims are used here.

\medskip
\noindent\textbf{ATTACK PLAN.}
\emph{Proof track ideas.}
(1) Try to adapt the classical irrationality proof of $e=\sum_{n\ge0}1/n!$ (multiply by $N!$ and analyze a tiny tail), but denominators $n!-1$ do not divide $N!$.
(2) Use the double-series expansion to view $S$ as a sum of rapidly convergent "factorial-power" series and attempt a Cantor-series/Engel-series style irrationality criterion.

\emph{Disproof track ideas.}
(1) Attempt to detect a rational relation by computing many digits / continued fraction and searching for a very small denominator; this cannot prove rationality, but may guide structure.

I did not reach an irrationality proof or a rationality counterexample; below are problem-specific lemmas and computations.

\medskip
\noindent\textbf{WORK.}

\medskip
\noindent\textbf{Lemma 68.1 (geometric-series expansion).}
For every integer $n\ge 2$,
\[
\frac{1}{n!-1}=\sum_{k=1}^{\infty}\frac{1}{(n!)^k}.
\]
Consequently,
\[
S=\sum_{n=2}^{\infty}\frac{1}{n!-1}=\sum_{n=2}^{\infty}\sum_{k=1}^{\infty}\frac{1}{(n!)^k}=\sum_{k=1}^{\infty}\sum_{n=2}^{\infty}\frac{1}{(n!)^k},
\]
with absolute convergence justifying the interchange of sums.

\noindent\textbf{Proof.}
For $n\ge 2$ we have $0<1/n!\le 1/2$. Then
\[
\frac{1}{n!-1}=\frac{1}{n!\,(1-1/n!)}=\frac{1}{n!}\cdot\sum_{k=0}^{\infty}\Big(\frac{1}{n!}\Big)^k
=\sum_{k=1}^{\infty}\frac{1}{(n!)^k},
\]
where we used the geometric-series identity $1/(1-x)=\sum_{k\ge0}x^k$ valid for $\lvert x\rvert<1$.

For the double sum, note that
\[
\sum_{n=2}^{\infty}\sum_{k=1}^{\infty}\Big\lvert\frac{1}{(n!)^k}\Big\rvert
=\sum_{n=2}^{\infty}\sum_{k=1}^{\infty}\frac{1}{(n!)^k}
=\sum_{n=2}^{\infty}\frac{1}{n!-1}
<\sum_{n=2}^{\infty}\frac{2}{n!}<\infty,
\]
so the series is absolutely convergent and Fubini/Tonelli applies to swap summations.
This proves the lemma.\hfill$\square$

\medskip
\noindent\textbf{Lemma 68.2 (explicit tail bound).}
For every integer $N\ge 2$,
\[
0<\sum_{n=N+1}^{\infty}\frac{1}{n!-1}
<\frac{4}{(N+1)!}.
\]

\noindent\textbf{Proof.}
For $n\ge 2$ we have $n!-1\ge \frac12 n!$ (since $n!\ge 2$), hence
\[\frac{1}{n!-1}\le \frac{2}{n!}.\]
Therefore
\[
\sum_{n=N+1}^{\infty}\frac{1}{n!-1}
\le 2\sum_{n=N+1}^{\infty}\frac{1}{n!}
=\frac{2}{(N+1)!}\sum_{j=0}^{\infty}\frac{1}{(N+2)(N+3)\cdots (N+1+j)}.
\]
For each $j\ge 0$, the denominator $(N+2)(N+3)\cdots(N+1+j)$ is at least $(N+2)^j$, so
\[
\sum_{j=0}^{\infty}\frac{1}{(N+2)(N+3)\cdots (N+1+j)}
\le \sum_{j=0}^{\infty}\frac{1}{(N+2)^j}=\frac{1}{1-1/(N+2)}=\frac{N+2}{N+1}.
\]
Thus
\[
\sum_{n=N+1}^{\infty}\frac{1}{n!-1}
\le \frac{2}{(N+1)!}\cdot\frac{N+2}{N+1}
<\frac{4}{(N+1)!}
\]
for all $N\ge 2$. Positivity is clear termwise.\hfill$\square$

\medskip
\noindent\textbf{FAST REALITY CHECK (computation).}
Using 80-digit arithmetic, I evaluated partial sums and a high-precision estimate:
\begin{itemize}
\item Partial sum through $n=24$:
\[
\sum_{n=2}^{24}\frac{1}{n!-1}
\approx 1.25349875569995347164336087086146605027663723112268710688\dots
\]
\item A direct high-precision summation (mpmath) gives
\[
S\approx 1.25349875569995347164336093790579894036923220833201341706\dots
\]
\item The difference between the $n\le 24$ partial sum and the high-precision estimate is about
\[6.7\times 10^{-26},\]
consistent with Lemma 68.2 (for $N=24$, the tail bound $4/25!\approx 1.34\times 10^{-25}$).
\item Continued fraction initial segment (for sanity checking only):
\[[1, 3, 1, 17, 8, 1, 4, 3, 2, 2, 2, 1, \dots].\]
\end{itemize}

\medskip
\noindent\textbf{VERIFICATION.}
\begin{itemize}
\item Lemma 68.1: checked algebraic identity $1/(n!-1)=(1/n!)/(1-1/n!)$ and geometric series for $\lvert 1/n!\rvert<1$.
\item Lemma 68.2: verified each inequality and the comparison of the factorial tail by a geometric series with ratio $1/(N+2)$.
\item Computation: the tail estimate and the observed numerical difference match in order of magnitude.
\end{itemize}

\medskip
\noindent\textbf{FINAL: \textbf{UNRESOLVED}.}
\begin{itemize}
\item[(i)] \emph{Strongest proved partial result.} The identity $\frac{1}{n!-1}=\sum_{k\ge1}1/(n!)^k$ holds termwise and yields the absolutely convergent double-series representation of $S$ (Lemma 68.1). The remainder after $N$ terms satisfies an explicit bound $<4/(N+1)!$ (Lemma 68.2).
\item[(ii)] \emph{First gap (crisp).} Prove that no rational $p/q$ equals $S$, i.e.
\[S\neq p/q\quad\text{for all integers }p,q\ (q\ge1).\]
A concrete route would be: for each $q$, show there exists $N$ such that $\big\lVert q\sum_{n=2}^{N}\frac{1}{n!-1}\big\rVert$ is bounded away from the tail error, forcing $qS$ to have nonzero distance to the integers.
\item[(iii)] \emph{Top 3 next moves.}
(1) Multiply $S$ by a structured integer (e.g. a product of $(n!-1)$ up to some cutoff) and try to force a contradiction via congruences/modular reductions.
(2) Exploit Lemma 68.1 to express $S$ in a "factorial-Cantor" expansion and seek a general irrationality criterion for expansions with bases $n!$.
(3) Extend computation: compute many continued-fraction convergents and check whether any convergent denominator interacts unusually with the factorial-minus-one structure (guiding a modular argument).
\item[(iv)] \emph{Minimal counterexample structure.} If $S$ were rational, say $S=p/q$, then the extremely small tails (Lemma 68.2) would force the partial sums to be extraordinarily close to $p/q$ with error $\ll 1/(N!)$. Any disproof strategy would likely need to show that such "too-good" approximations cannot persist because the denominators $n!-1$ are not compatible with a fixed $q$ modulo many integers.
\end{itemize}


