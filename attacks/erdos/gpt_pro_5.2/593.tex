
Characterize those finite 3-uniform hypergraphs which appear in every 3-uniform hypergraph of chromatic number $>\aleph_0$. Similar problems were investigated by Erd\H{o}s, Galvin, and Hajnal \cite{EGH75}. Erd\H{o}s claims that for graphs the problem is completely solved: a graph of chromatic number $\geq \aleph_1$ must contain all finite bipartite graphs but need not contain any fixed odd cycle. References [EGH75] Erd\H{o}s, P. and Galvin, F. and Hajnal, A., On set-systems having large chromatic number and not containing prescribed subsystems . (1975), 425--513.

\bigskip
\noindent\textbf{FORMAL RESTATEMENT}

A \,\emph{3-uniform hypergraph} is a pair $H=(V,E)$ where $V$ is a set of vertices and $E\subseteq [V]^3$ is a set of 3-element subsets (hyperedges).
A \,\emph{proper vertex-colouring} of $H$ with a set of colours $\Gamma$ is a map $f:V\to\Gamma$ such that no hyperedge $e\in E$ is monochromatic, i.e. $f$ is not constant on $e$.
The \,\emph{chromatic number} $\chi(H)$ is the least cardinal $\kappa$ for which there is a proper colouring with $\kappa$ colours.

A finite 3-uniform hypergraph $F$ \emph{appears in} $H$ if there is an injective map $\varphi:V(F)\to V(H)$ such that for every edge $\{a,b,c\}\in E(F)$, the image $\{\varphi(a),\varphi(b),\varphi(c)\}$ lies in $E(H)$ (i.e. $F$ is isomorphic to a subhypergraph of $H$).

Problem: characterize all finite 3-uniform hypergraphs $F$ such that for every 3-uniform hypergraph $H$ with $\chi(H)>\aleph_0$, $F$ appears in $H$.

\bigskip
\noindent\textbf{QUICK LITERATURE/CONTEXT CHECK}

I will not import external results beyond what is explicitly stated in the problem text.
The text gives an analogy for graphs: uncountably chromatic graphs contain every finite bipartite graph but can avoid any fixed odd cycle.
The hypergraph analogue is posed as an open-ended characterization problem.

\bigskip
\noindent\textbf{ATTACK PLAN}

\emph{Proof-track ideas (unavoidability).}
\begin{itemize}
\item Prove general ``compactness'' consequences of $\chi(H)>\aleph_0$, e.g. that $H$ contains finite subhypergraphs of arbitrarily large finite chromatic number, and then try to infer that certain finite configurations must embed.
\item Identify a candidate class of ``unavoidable'' finite 3-graphs (hypergraph analogue of bipartite graphs), perhaps those with a certain 2-colourability/acyclicity property, and try to embed them via a recursive construction.
\end{itemize}

\emph{Disproof-track ideas (avoidability).}
\begin{itemize}
\item For a given finite $F$, attempt to construct a 3-uniform hypergraph $H$ with $\chi(H)>\aleph_0$ but $F$-free, by imposing intersection restrictions (e.g. linearity) or large-girth type constraints while keeping uncountable chromatic number.
\end{itemize}

I did not reach a characterization. I record two rigorous lemmas forced by the hypothesis $\chi(H)>\aleph_0$.

\bigskip
\noindent\textbf{WORK}

\medskip
\noindent\textbf{FAST REALITY CHECK}

\begin{itemize}
\item If $\chi(H)>\aleph_0$, then in particular $\chi(H)>2$ and $\chi(H)>k$ for every finite $k$. Thus $H$ cannot be ``matching-like'' (disjoint edges), and must contain finite subconfigurations that force more than $k$ colours.
\end{itemize}

\medskip
\noindent\textbf{Lemma 1 (finite-$k$ compactness for hypergraph colourings).}\label{lem:hypergraph_compactness}
Let $H=(V,E)$ be a (not necessarily finite) hypergraph and let $k\in\mathbb{N}$.
If every finite subhypergraph of $H$ is $k$-colourable, then $H$ is $k$-colourable.

\emph{Proof.}
Let $[k]:=\{1,2,\dots,k\}$ and consider the product space $X:=[k]^V$ of all functions $f:V\to[k]$, equipped with the product topology where $[k]$ has the discrete topology.
Since $[k]$ is finite and discrete, it is compact, and therefore by Tychonoff's theorem $X$ is compact.

For each hyperedge $e=\{v_1,\dots,v_r\}\in E$ (here $r=3$ in our application), define
\[
C_e:=\{f\in X: \text{$f$ is \emph{not} constant on $e$}\}.
\]
Then $C_e$ is open (indeed clopen) in $X$: it is the complement of the closed set where $f(v_1)=\cdots=f(v_r)$.
A function $f\in X$ is a proper $k$-colouring of $H$ exactly when $f\in\bigcap_{e\in E} C_e$.

We claim the family $\{C_e:e\in E\}$ has the finite intersection property.
Indeed, take finitely many edges $e_1,\dots,e_m$.
Let $V_0\subseteq V$ be the finite set of vertices appearing in these edges.
By assumption, the finite subhypergraph induced on $V_0$ is $k$-colourable, so there exists $g:V_0\to[k]$ which is not constant on any $e_i$.
Extend $g$ arbitrarily to a function $f:V\to[k]$.
Then $f\in C_{e_i}$ for each $i$, so $f\in\bigcap_{i=1}^m C_{e_i}$.
Thus every finite intersection is nonempty.

By compactness of $X$, the intersection $\bigcap_{e\in E} C_e$ is nonempty.
Any $f$ in this intersection is a proper $k$-colouring of $H$.\qed

\medskip
\noindent\textbf{Corollary 1.}\label{cor:finite_obstructions}
If a hypergraph $H$ is not $k$-colourable for some finite $k$, then $H$ has a finite subhypergraph which is not $k$-colourable.
In particular, if $\chi(H)>\aleph_0$ then for every finite $k$ there exists a finite subhypergraph $F_k\subseteq H$ with $\chi(F_k)>k$.

\emph{Proof.}
The first statement is the contrapositive of Lemma~\ref{lem:hypergraph_compactness}.
For the second, if $\chi(H)>\aleph_0$ then $H$ is not $k$-colourable for any finite $k$, so apply the first statement to each $k$.\qed

\medskip
\noindent\textbf{Lemma 2 (a forced finite configuration: intersecting edges).}\label{lem:intersecting_edges}
Let $H=(V,E)$ be a 3-uniform hypergraph.
If the edges of $H$ are pairwise disjoint (i.e. $e\cap e'=\emptyset$ for all distinct $e,e'\in E$), then $\chi(H)\le 2$.
Consequently, if $\chi(H)>2$ (and hence if $\chi(H)>\aleph_0$), then $H$ contains two edges with nonempty intersection.

\emph{Proof.}
Assume edges are pairwise disjoint.
Define a 2-colouring $f:V\to\{\text{red},\text{blue}\}$ as follows.
For each edge $e=\{a,b,c\}\in E$, set $f(a)=\text{red}$ and $f(b)=f(c)=\text{blue}$.
This is consistent because distinct edges are disjoint, so no vertex receives two different prescriptions.
For vertices not contained in any edge, assign (say) $\text{red}$.
Then every edge contains both colours, so no edge is monochromatic.
Thus $f$ is a proper 2-colouring and $\chi(H)\le 2$.
The final sentence is the contrapositive.\qed

\bigskip
\noindent\textbf{VERIFICATION}

\begin{itemize}
\item Lemma~\ref{lem:hypergraph_compactness} uses only finite-$k$ compactness; it does \emph{not} extend to $k=\aleph_0$, so it is compatible with the existence of hypergraphs of chromatic number $\aleph_0$ whose finite subhypergraphs are all 2-colourable.
\item Lemma~\ref{lem:intersecting_edges} is checked on the boundary case $E=\emptyset$ (then $\chi(H)=1\le 2$).
\end{itemize}

\bigskip
\noindent\textbf{FINAL}

\textbf{UNRESOLVED}

(i) \emph{Strongest proved partial result.}
If $\chi(H)>\aleph_0$, then for every finite $k$ the hypergraph contains a finite subhypergraph of chromatic number $>k$ (Corollary~\ref{cor:finite_obstructions}); and in particular $H$ must contain two intersecting edges (Lemma~\ref{lem:intersecting_edges}).

(ii) \emph{First gap (crisp).}
Find a necessary-and-sufficient intrinsic property of a finite 3-uniform hypergraph $F$ guaranteeing that \emph{every} 3-uniform hypergraph $H$ with $\chi(H)>\aleph_0$ contains an isomorphic copy of $F$.
Even the candidate criterion ``$F$ is 2-colourable'' (hypergraph analogue of bipartite) is not proved here.

(iii) \emph{Top 3 next moves.}
\begin{enumerate}
\item Test the conjectural analogue ``unavoidable iff 2-colourable'' by attempting: (a) an embedding lemma for every finite 2-colourable 3-graph into any $H$ with $\chi(H)>\aleph_0$, and (b) for each non-2-colourable $F$ construct an $F$-free $H$ with $\chi(H)>\aleph_0$.
\item Build explicit candidate $H$ with $\chi(H)=\aleph_1$ and strong intersection restrictions (e.g. linear 3-graphs) and check which small $F$ embed.
\item Perform small-configuration searches inside known high-chromatic finite 3-graphs to guess the ``unavoidable'' class and then attempt to lift to the uncountable setting.
\end{enumerate}

(iv) \emph{Minimal counterexample structure.}
To disprove a candidate ``unavoidable'' finite $F$, one would need a 3-uniform hypergraph $H$ with $\chi(H)>\aleph_0$ but $F\not\subseteq H$.
Any such $H$ must still contain finite subhypergraphs with arbitrarily large finite chromatic number, so $H$ cannot be too sparse globally; the obstruction must instead come from forbidding a specific local intersection pattern encoded by $F$.


