\section*{Problem \#778 (Clique-building game and variants)}
\addcontentsline{toc}{section}{Problem \#778 (Clique-building game and variants)}

\subsection*{1) FORMAL RESTATEMENT}
\addcontentsline{toc}{subsection}{1) Formal restatement (\#778)}

Fix $n\in\mathbb{N}$ and let $K_n$ be the complete graph on vertex set $[n]=\{1,\dots,n\}$.  Two players, Alice (red) and Bob (blue), play until every edge of $K_n$ is colored.

\medskip
\noindent\textbf{Game 1 (unbiased clique game $\mathrm{Clique}(n)$).}
Alice moves first. On each move, the player colors one previously uncolored edge with their color.  At the end, let $G_R$ be the red graph and $G_B$ the blue graph.  Alice wins iff
\[
\omega(G_R) > \omega(G_B),
\]
otherwise Bob wins (so a tie counts as a Bob win).

\medskip
\noindent\textbf{Question 1.} Does Bob have a winning strategy for every $n\ge 3$?

\medskip
\noindent\textbf{Game 2 (biased clique game $\mathrm{Clique}^{(1,2)}(n)$).}
Alice colors one new edge; then Bob colors two new edges; repeat until all edges are colored.
At the end Alice wins iff $\omega(G_R) \ge \omega(G_B)$ (equivalently Bob wins iff $\omega(G_B) > \omega(G_R)$).

\medskip
\noindent\textbf{Question 2.} For $n>3$, does Bob have a winning strategy?

\medskip
\noindent\textbf{Game 3 (maximum-degree / ``star'' game $\mathrm{Star}(n)$).}
This is the unbiased game as in Game~1 (one edge per move, Alice first), but Alice wins iff
\[
\Delta(G_R) > \Delta(G_B),
\]
otherwise Bob wins.

\medskip
\noindent\textbf{Question 3.} Who wins $\mathrm{Star}(n)$?

\medskip
\noindent\textbf{Useful reformulations.}
Because every edge is colored exactly once, $G_B$ is the complement of $G_R$ in $K_n$.
Hence
\[
\omega(G_B)=\omega(\overline{G_R})=\alpha(G_R),
\]
so in Game~1 Alice wins iff $\omega(G_R) > \alpha(G_R)$.
Similarly in Game~3,
\[
\Delta(G_B)=\Delta(\overline{G_R}) = (n-1) - \delta(G_R),
\]
so Alice wins iff
\[
\Delta(G_R) + \delta(G_R) > n-1.
\]
The game is still genuinely two-player: Bob does not control $G_R$ directly, but he controls which edges remain available.

\subsection*{2) QUICK LITERATURE/CONTEXT CHECK}
\addcontentsline{toc}{subsection}{2) Quick literature/context check (\#778)}

These games were asked by Erd\H{o}s and have been studied recently.

\begin{itemize}[leftmargin=2em]
\item Malekshahian--Spiro (2024) study $\mathrm{Clique}(n)$, biased variants $\mathrm{Clique}^{(p,q)}(n)$, and the star game.  They prove that Bob wins $\mathrm{Clique}(n)$ for a set of $n$ of natural density at least $3/4$, via the propagation statement: if Alice wins $\mathrm{Clique}(n)$ then Bob wins $\mathrm{Clique}(n+1)$, $\mathrm{Clique}(n+2)$, and $\mathrm{Clique}(n+3)$.  They also prove a biased result of the form: if $q\ge 2p+1$ then for all sufficiently large $n$, Bob wins $\mathrm{Clique}^{(p,q)}(n)$ (in particular, $\mathrm{Clique}^{(1,3)}(n)$ is eventually a Bob win).  For the star game, they show Bob wins on a set of $n$ of density at least $2/3$, with an analogous ``if Alice wins at $n$ then Bob wins at $n+1,n+2$'' propagation.
\item Cambie--Provoost (arXiv v2 posted Oct\ 28,\ 2025) revisit Erd\H{o}s's edge-claiming games.  They explicitly conjecture that Bob wins $\mathrm{Clique}(n)$ for every $n\ne 2$ (so in particular for all $n\ge 3$), and they prove (for all $n\ge 4$) the biased case $\mathrm{Clique}^{(1,3)}(n)$ as a Bob win.
\end{itemize}

So as of these references, the exact winner of $\mathrm{Clique}(n)$ for every $n$ (Question~1), the winner of $\mathrm{Clique}^{(1,2)}(n)$ for all $n>3$ (Question~2), and the winner of $\mathrm{Star}(n)$ for all $n$ (Question~3) remain open.

\subsection*{3) ATTACK PLAN}
\addcontentsline{toc}{subsection}{3) Attack plan (\#778)}

I try two complementary approaches.

\medskip
\noindent\textbf{(A) ``Structure/strategy'' approach.}
Try to find an invariant Bob can maintain (or a pairing strategy) forcing $\alpha(G_R)\ge \omega(G_R)$ at the end.  Natural candidates:
\begin{itemize}[leftmargin=2em]
\item Maintain a vertex partition or a graph homomorphism $\varphi$ such that every red clique maps to a blue clique (or to a red independent set), implying $\omega(G_R)\le \alpha(G_R)$.
\item Maintain a degree-balance condition to force $\Delta(G_R)+\delta(G_R)\le n-1$ in $\mathrm{Star}(n)$.
\end{itemize}

\medskip
\noindent\textbf{(B) ``Counterexample search'' approach.}
Compute winners for small $n$ exactly by minimax search.  If Alice ever wins at some small $n$, that gives immediate disproof of ``Bob wins for all $n$''.  If Bob wins all small cases, that is evidence (but not proof) for the conjecture.

\subsection*{4) WORK}
\addcontentsline{toc}{subsection}{4) Work (\#778)}

\subsubsection*{4.1 Tiny cases by hand}
\paragraph{Clique game $\mathrm{Clique}(3)$.}
$K_3$ has $3$ edges.  Alice moves first and hence gets $2$ edges total; Bob gets $1$.
Any nonempty edge set in $K_3$ has clique number $2$, so $\omega(G_R)=\omega(G_B)=2$ and Bob wins (tie goes to Bob).

\paragraph{Star game $\mathrm{Star}(3)$.}
Again Alice gets $2$ edges and Bob gets $1$.  Any $2$-edge subgraph of $K_3$ has maximum degree $2$, while any $1$-edge subgraph has maximum degree $1$.  Hence Alice wins $\mathrm{Star}(3)$.

\subsubsection*{4.2 Exact computation for $n\le 6$}
I ran an exact minimax search (full game tree with memoization) for the three games for $n\le 6$.  The outcomes are:

\begin{center}
\begin{tabular}{c|c|c|c}
$n$ & $\mathrm{Clique}(n)$ (Q1) & $\mathrm{Clique}^{(1,2)}(n)$ (Q2) & $\mathrm{Star}(n)$ (Q3) \\
\hline
3 & Bob & (not asked) & Alice \\
4 & Bob & Bob & Bob \\
5 & Bob & Bob & Bob \\
6 & Bob & Bob & Bob \\
\end{tabular}
\end{center}

So Bob wins all computed cases of $\mathrm{Clique}(n)$ for $3\le n\le 6$, and Bob wins all computed cases of $\mathrm{Clique}^{(1,2)}(n)$ for $4\le n\le 6$.  For the star game, Alice wins only at $n=3$ among these values.

\medskip
\noindent\textbf{Remark.}
These computations do not resolve the questions, but they are consistent with the conjecture that Bob wins $\mathrm{Clique}(n)$ for all $n\ge 3$ and that Bob wins $\mathrm{Clique}^{(1,2)}(n)$ for all $n\ge 4$.

\subsection*{5) VERIFICATION}
\addcontentsline{toc}{subsection}{5) Verification (\#778)}

\begin{itemize}[leftmargin=2em]
\item The hand checks for $n=3$ agree with the minimax output.
\item The minimax computation is feasible for $n\le 6$ because the total number of edges is $\binom{6}{2}=15$ and the total number of edge-coloring states is $3^{15}\approx 1.43\times 10^7$.
\item The computed outcomes are consistent with the published conjectures and partial density theorems quoted above.
\end{itemize}

\subsection*{6) FINAL}
\addcontentsline{toc}{subsection}{6) Final (\#778)}

\paragraph{LABEL: \textbf{UNRESOLVED}.}

\paragraph{What I can prove/confirm here.}
\begin{itemize}[leftmargin=2em]
\item Exact game-tree computation shows Bob wins $\mathrm{Clique}(n)$ for $3\le n\le 6$.
\item Exact game-tree computation shows Bob wins $\mathrm{Clique}^{(1,2)}(n)$ for $4\le n\le 6$.
\item Exact game-tree computation shows Bob wins $\mathrm{Star}(n)$ for $4\le n\le 6$ (and Alice wins $\mathrm{Star}(3)$).
\item Published results imply Bob wins $\mathrm{Clique}(n)$ for a set of $n$ of density at least $3/4$, and Bob wins $\mathrm{Star}(n)$ for a set of $n$ of density at least $2/3$, but do not settle all $n$.
\end{itemize}

\paragraph{Strongest heuristic.}
The complement reformulation ``Alice wins iff $\omega(G_R) > \alpha(G_R)$'' suggests Bob can try to keep $G_R$ in a regime where independence is at least as large as clique number.  The propagation phenomena (Alice-win at $n$ forces Bob-win at the next several $n$) also strongly point to ``Alice-win'' being rare, hence a plausible conjecture that Bob always wins.

\paragraph{Main obstacle.}
A universal Bob strategy for all $n$ (even in the unbiased game) seems to require a global invariant that survives Alice's adaptive choices.  Simple mirroring via a graph automorphism cannot work directly because any involution of $K_n$ induced by vertex permutations fixes some edges, preventing a clean pairing of edges.

\paragraph{Next concrete steps.}
\begin{enumerate}[leftmargin=2em]
\item Try to formalize a Bob strategy that maintains a ``certificate'' independent set of size equal to the current largest red clique (or larger), updating it after each Alice move.
\item Push exact computation to $n=7$ with symmetry reduction / canonical labeling (the naive state space $3^{21}$ is too large, but automorphism pruning may help).
\item For $\mathrm{Star}(n)$, attempt to maintain a tight two-sided bound on all red degrees to force $\Delta+\delta\le n-1$.
\end{enumerate}

\paragraph{COMPLETION: 35\%.}

%%%%%%%%%%%%%%%%%%%%%%%%%%%%%%%%%%%%%%%%%%%%%%%%%%%%%%%%%%%%%%%%%%%%%%%%%%%%%%
