\section*{Problem 769}

\subsection*{1. Formal restatement}

\begin{definition}[Cube decompositions and $c(n)$]
Fix $n\ge 2$.
Let $Q_n=[0,1]^n$ be the unit $n$-cube.
A \emph{decomposition of $Q_n$ into $k$ (smaller) $n$-cubes} means a partition of $Q_n$ into $k$ closed sets with disjoint interiors, each of which is an $n$-dimensional cube similar to $Q_n$ (i.e. a homothetic copy; equivalently: a translate of a scaled copy).

Define $c(n)$ to be the smallest integer such that for every integer $k\ge c(n)$, the cube $Q_n$ admits a decomposition into $k$ smaller $n$-cubes.
\end{definition}

\paragraph{Question (Erd\H{o}s).}
Give good bounds for $c(n)$; in particular, is it true that $c(n)\gg n^n$?

\subsection*{2. Quick literature/context check (web-browsing was available)}

The ErdosProblems project page for \#769 summarizes the classical results and cites work of Hudelson (1998) and Connor--Marmorino (2018).
Connor--Marmorino prove (among other results) the bounds
\[
 c(n)\ge 2^{n+1}-1\ \ (n\ge 3),
\qquad
 c(n)\le e^2 n^n\ \ (n+1\text{ composite}),
\qquad
 c(n)\le 1.8\,n^{n+1}\ \ (n+1\text{ prime}).
\]
No resolution of the growth question ``$c(n)\gg n^n$?'' was found in this quick check.

\subsection*{3. Strategy}

To \emph{prove} $c(n)\gg n^n$ one would need a mechanism preventing decompositions for some $k$ of size $\ll n^n$.
Known lower bounds are only exponential in $n$.

To \emph{disprove} $c(n)\gg n^n$ it would suffice to construct decompositions for all $k\ge C^n$ for some fixed $C$ (much smaller than $n$ for large $n$).
Hudelson proved an upper bound $c(n)<6^n$ under the number-theoretic condition $\gcd(2^n-1,3^n-1)=1$; if that condition holds for infinitely many $n$, then $c(n)/n^n\to 0$ along that subsequence.
However, whether $\gcd(2^n-1,3^n-1)=1$ infinitely often is itself a difficult open problem (see also Problem \#770 / \#820).

What we can do completely here is settle the planar case $n=2$ (which is classical): $c(2)=6$.

\subsection*{4. Work}

\subsubsection*{4.1 The case $n=2$: a complete determination $c(2)=6$}

\begin{theorem}
For the unit square $Q_2=[0,1]^2$, one has $c(2)=6$.
Equivalently: for every integer $k\ge 6$ the unit square can be tiled by $k$ smaller squares, and there is no tiling by $5$ squares.
\end{theorem}

\begin{proof}
We prove (i) impossibility for $k=5$ and (ii) existence for every $k\ge 6$.

\smallskip
\noindent\emph{(i) No tiling by 5 squares.}
Assume for contradiction that $[0,1]^2$ is tiled by exactly $5$ squares.
Each of the four corners of the unit square must belong to some tile.
A single tile cannot contain two distinct corners of the unit square unless it is the whole unit square (side length $1$), which is impossible here.
Hence four \emph{distinct} tiles cover the four corners.
Let the fifth tile be the unique tile that is not a corner tile.

A non-corner tile cannot touch two adjacent sides of the unit square: if a square touches both the bottom edge $y=0$ and the left edge $x=0$, its bottom-left corner must be $(0,0)$, i.e. it would be a corner tile.
Similarly it cannot touch two opposite sides (that would force side length $1$).
Therefore the fifth tile touches at most one side of the boundary.
By symmetry, assume it touches the bottom edge but not a corner.

Let the bottom-left corner tile have side length $a$ and the top-left corner tile have side length $d$.
Since no non-corner tile touches the left edge, these two tiles partition the entire left edge, so
\begin{equation}
 a+d=1. \tag{1}
\end{equation}
Similarly, if the bottom-right and top-right corner tiles have side lengths $b$ and $c$, then
\begin{equation}
 b+c=1. \tag{2}
\end{equation}
Since the fifth tile does not touch the top edge, the two top corner tiles partition the entire top edge, so
\begin{equation}
 d+c=1. \tag{3}
\end{equation}
From (1) and (3) we get $a=c$, and then from (2) we get $b=1-c=1-a$.
Now look again at the bottom edge: it is covered by the bottom-left corner tile (length $a$), the fifth tile (some positive length $e>0$), and the bottom-right corner tile (length $b$), so
\begin{equation}
 a+e+b=1. \tag{4}
\end{equation}
Substituting $b=1-a$ into (4) gives $a+e+(1-a)=1$, hence $e=0$, contradiction.
So no tiling by $5$ squares exists.

\smallskip
\noindent\emph{(ii) Tilings for every $k\ge 6$.}
We exhibit explicit tilings for $k\in\{6,7,8\}$ and then show how to increase the number of tiles by $3$.

\begin{itemize}[leftmargin=2em]
\item A $6$-tiling: one square of side $2/3$ and five squares of side $1/3$.
Concretely, take the square $[0,2/3]\times[0,2/3]$ and tile the remaining region by five $1/3$-squares:
\begin{align*}
&[0,1/3]\times[2/3,1],\quad [1/3,2/3]\times[2/3,1],\\
&[2/3,1]\times[0,1/3],\quad [2/3,1]\times[1/3,2/3],\quad [2/3,1]\times[2/3,1].
\end{align*}
\item A $7$-tiling: start with the $2\times 2$ grid of four squares of side $1/2$ and subdivide one of them into four squares of side $1/4$.
For instance, subdivide the top-left $[0,1/2]\times[1/2,1]$ into a $2\times 2$ grid.
\item An $8$-tiling: one square of side $3/4$ and seven squares of side $1/4$.
Concretely, take the square $[0,3/4]\times[0,3/4]$ and tile the remaining region by seven $1/4$-squares:
four along the top strip $[0,1]\times[3/4,1]$ and three along the right strip $[3/4,1]\times[0,3/4]$.
\end{itemize}

Now observe the following operation.
Given any tiling by $k$ squares, choose any tile that is a square of side length $s$ and subdivide it into four congruent squares of side length $s/2$ (a $2\times 2$ grid).
This replaces one tile by four tiles and hence increases the number of tiles from $k$ to $k+3$.

Therefore, starting from the base tilings for $6,7,8$, we obtain tilings for
\[
6+3t,\qquad 7+3t,\qquad 8+3t\qquad (t\ge 0).
\]
Every integer $k\ge 6$ is congruent to exactly one of $6,7,8$ modulo $3$, so every $k\ge 6$ is achievable.
Combining with (i), we conclude $c(2)=6$.
\end{proof}

\subsubsection*{4.2 Bounds in higher dimension (quoted)}

For $n\ge 3$, the determination of $c(n)$ is open.
The best bounds located in the quick literature check are those quoted above from Connor--Marmorino (2018).
In particular, the lower bound is exponential in $n$, while the upper bound is about $n^n$ (up to constants and an extra $n$ factor when $n+1$ is prime).
This leaves the growth question $c(n)\gg n^n$ wide open.

\subsection*{5. Obstacles/checks}

The threshold property in the definition of $c(n)$ (``all $k\ge c(n)$'') is delicate:
it is not enough to construct many values of $k$ (like perfect powers $m^n$), one needs a method that reaches \emph{every} large $k$.
Current constructions are number-theoretic and seem closely tied to gcd-type constraints such as $\gcd(2^n-1,3^n-1)$.

\subsection*{6. Conclusion}

\textbf{UNRESOLVED (for general $n$).}
We determined the planar value $c(2)=6$ with a complete proof.
For $n\ge 3$, the best known bounds (in the quick literature check) are
\[
2^{n+1}-1\le c(n)\le e^2 n^n\quad (n+1\text{ composite}),
\qquad
2^{n+1}-1\le c(n)\le 1.8\,n^{n+1}\quad (n+1\text{ prime}),
\]
and whether $c(n)\gg n^n$ remains open.

\subsection*{7. If UNRESOLVED: what we have and what is missing}

\begin{itemize}[leftmargin=2em]
\item \textbf{Strongest proved statement here:} $c(2)=6$.
\item \textbf{First genuine gap toward ``$c(n)\gg n^n$'':} producing any lower bound that is super-exponential in $n$ (or even of order $n^n$), or alternatively producing a construction forcing $c(n)$ to be $o(n^n)$.
\item \textbf{Notable conditional route to disprove $c(n)\gg n^n$:} if $\gcd(2^n-1,3^n-1)=1$ for infinitely many $n$, then (by Hudelson's result) $c(n)<6^n$ along an infinite subsequence, hence $c(n)/n^n\to 0$ on that subsequence.
\end{itemize}

%%%%%%%%%%%%%%%%%%%%%%%%%%%%%%%%%%%%%%%%%%%%%%%%%%%%%%%%%%%%%%%%%%%%%%%%%%%%%%
