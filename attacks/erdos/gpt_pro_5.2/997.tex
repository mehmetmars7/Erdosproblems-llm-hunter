% Erdos Problem #997

1) FORMAL RESTATEMENT
A sequence $(x_m)_{m\ge 1}$ in $(0,1)$ is \emph{well-distributed} (in the sense of Hlawka and Petersen) if for every $\epsilon>0$ there exists $k_0(\epsilon)$ such that for every integer $k\ge k_0$, every integer $n\ge 0$, and every interval $I\subseteq [0,1]$,
\[
\bigl|\#\{n<m\le n+k : x_m\in I\} - k\,|I|\bigr| < \epsilon k.
\]
Let $p_n$ be the $n$-th prime and fix $\alpha\in(0,1)$.  Consider the sequence
\[
 x_n := \{\alpha p_n\} \in [0,1).
\]
Question: Is it true that for every $\alpha\in(0,1)$ the sequence $(\{\alpha p_n\})_{n\ge 1}$ is \emph{not} well-distributed?

2) QUICK LITERATURE/CONTEXT CHECK
The problem statement notes: Erd\H{o}s proved non-well-distribution for lacunary $(n_k)$ and a.e. $\alpha$, and it records that there exists at least one irrational $\alpha$ with $(\{\alpha p_n\})$ not well-distributed.  I do not use results beyond the definition.

3) ATTACK PLAN
Prove necessary conditions for well-distribution, then show these fail at least for some classes of $\alpha$ (e.g. rational $\alpha$).  For irrational $\alpha$, the problem appears to require deep information about primes in short intervals/arithmetic progressions to force irregularity in every block.

4) WORK

FAST REALITY CHECK (easy obstruction for rational $\alpha$).
If $\alpha=a/q$ is rational in lowest terms, then $\{\alpha p_n\}\in\{0,1/q,2/q,\dots,(q-1)/q\}$ for all $n$, a finite set.  A well-distributed sequence must be dense (Lemma 2 below), so this immediately rules out all rational $\alpha$.

Lemma 1 (Well-distributed $\Rightarrow$ uniformly distributed).
If $(x_m)$ is well-distributed, then it is uniformly distributed in the usual sense: for every interval $I\subseteq[0,1]$,
\[
\lim_{N\to\infty}\frac{1}{N}\#\{1\le m\le N: x_m\in I\} = |I|.
\]

Proof.
Fix an interval $I$ and $\epsilon>0$.  Choose $k_0(\epsilon)$ from the definition.  For $N\ge k_0$, write $N = qk + r$ with $k\ge k_0$ and $0\le r<k$.  Partition $\{1,\dots,N\}$ into $q$ consecutive blocks of length $k$ plus a remainder of length $r$.

For each full block, the well-distribution inequality gives
\[
\bigl|\#(\text{block}\cap I) - k|I|\bigr|<\epsilon k.
\]
Summing over the $q$ blocks gives
\[
\bigl|\#\{1\le m\le qk: x_m\in I\} - qk|I|\bigr| < q\epsilon k=\epsilon qk.
\]
The remainder contributes at most $r\le k$ points, so
\[
\bigl|\#\{1\le m\le N: x_m\in I\} - N|I|\bigr|
\le \epsilon qk + k + r|I|.
\]
Divide by $N=qk+r\ge qk$ to get
\[
\left|\frac{1}{N}\#\{1\le m\le N: x_m\in I\} - |I|\right|
\le \epsilon + \frac{k+r|I|}{N} \le \epsilon + \frac{2k}{N}.
\]
Letting $N\to\infty$ (hence $k/N\to 0$ for fixed $k$) and then $\epsilon\to 0$ yields the claim. \qed

Lemma 2 (Well-distributed $\Rightarrow$ dense and no long gaps).
If $(x_m)$ is well-distributed, then for every interval $I\subseteq[0,1]$ with $|I|>0$ there exists $k_0$ such that every block of length $k\ge k_0$ contains at least one term in $I$.
In particular, the set $\{x_m:m\ge 1\}$ is dense in $[0,1]$.

Proof.
Fix an interval $I$ with $|I|>0$.  Apply the definition with $\epsilon:=|I|/2$ to obtain $k_0$.  For any $k\ge k_0$ and any $n\ge 0$,
\[
\#\{n<m\le n+k: x_m\in I\} \ge k|I| - \epsilon k = \frac{|I|}{2}k.
\]
Since the right-hand side is $\ge 1$ for all $k\ge 2/|I|$, it follows that for all sufficiently large $k$ every such block contains at least one element of $I$.  This implies density because every nonempty open interval eventually must contain some $x_m$. \qed

Proposition 3 (Rational $\alpha$ gives non-well-distribution for primes).
If $\alpha=a/q\in(0,1)$ is rational, then $(\{\alpha p_n\})$ is not well-distributed.

Proof.
As noted in the FAST REALITY CHECK, the values lie in the finite set $\{0,1/q,\dots,(q-1)/q\}$.  Choose an interval $I\subseteq[0,1]$ of length $|I|>0$ that avoids this finite set (e.g. a small open interval between two consecutive multiples of $1/q$).  Then $\#\{n<m\le n+k: x_m\in I\}=0$ for all $n,k$, but $k|I|$ grows linearly in $k$, contradicting the definition for any $\epsilon<|I|$.  Hence the sequence is not well-distributed. \qed

5) VERIFICATION
- Lemma 1 uses only the block definition of well-distribution and basic counting; no hidden number theory.
- Lemma 2 uses the well-distribution inequality with $\epsilon=|I|/2$ to force a positive lower bound.
- Proposition 3 is immediate once Lemma 2 is established.

6) FINAL
**UNRESOLVED**
(i) Strongest proved partial result: For every rational $\alpha\in(0,1)$, the prime sequence $(\{\alpha p_n\})$ is not well-distributed (Proposition 3).  More generally, any well-distributed sequence must be dense (Lemma 2) and uniformly distributed (Lemma 1).
(ii) First gap: Handle irrational $\alpha$.  Prove that for every irrational $\alpha$ there exist intervals $I$ and arbitrarily large blocks of consecutive primes indices for which the count in $I$ deviates by $\gg k$ from $k|I|$.
(iii) Top 3 next moves:
  1. Reduce well-distribution failure to a concrete prime-in-arithmetic-progressions-in-short-index-intervals statement and try to prove that statement (even with weak parameters).
  2. For fixed irrational $\alpha$, search computationally for large $k$ and $n$ where the block discrepancy for $x_m=\{\alpha p_m\}$ is unusually large, to guess a mechanism (e.g. clustering near rationals).
  3. Attempt a ``Diophantine approximation + primes in progressions'' strategy: approximate $\alpha$ by $a/q$ and compare $\{\alpha p_n\}$ to $\{a p_n/q\}$; prove that short blocks of primes have biased residue distribution mod $q$ strongly enough to break well-distribution.
(iv) Minimal counterexample structure (if the statement is false): a counterexample would be an irrational $\alpha$ such that $(\{\alpha p_n\})$ is well-distributed, hence dense and uniformly distributed with uniformly small discrepancy on every consecutive block.  Such an $\alpha$ would force extremely strong ``local'' equidistribution of primes through the map $p\mapsto \{\alpha p\}$, far beyond global equidistribution.


