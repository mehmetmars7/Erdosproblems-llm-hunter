

FORMAL RESTATEMENT

For a fixed integer $a\in\mathbb{Z}$, determine whether there exist infinitely many integers $n\in\mathbb{N}$ such that
\[
\varphi(n)\mid (n+a),
\]
where $\varphi(n)$ is Euler's totient function (the number of integers $1\leq m\leq n$ with $\gcd(m,n)=1$).
Equivalently: for fixed $a$, does the congruence $n\equiv -a\pmod{\varphi(n)}$ hold for infinitely many $n\ge1$?
Edge cases: $n=1$ has $\varphi(1)=1$ and hence always satisfies $\varphi(1)\mid (1+a)$.

QUICK LITERATURE/CONTEXT CHECK

The problem statement attributes the question to a conjecture of Graham.
It notes Lehmer's conjecture: $\varphi(n)\mid (n-1)$ iff $n$ is prime, and the exercise that $\varphi(n)\mid n$ iff $n=2^\alpha 3^\beta$.
No other results are assumed below.

ATTACK PLAN

Proof track:
(1) Prove the classification $\varphi(n)\mid n \iff n=2^\alpha 3^\beta$ from scratch.
(2) Use multiplicativity of $\varphi$ to build infinite families of $n$ for some $a$ by forcing $n+a$ to be a multiple of $\varphi(n)$.

Disproof track:
Search for an $a$ where the congruence appears to have only finitely many solutions; try to prove finiteness via congruence obstructions.
(At present, this seems hard; we only obtain partial families.)

WORK

Lemma 1 (classification of $\varphi(n)\mid n$).
For $n\in\mathbb{N}$, one has $\varphi(n)\mid n$ if and only if $n=2^\alpha 3^\beta$ for some integers $\alpha,\beta\ge0$.

Proof.
Write the prime factorization $n=\prod_{i=1}^t p_i^{e_i}$ with distinct primes $p_i$ and exponents $e_i\ge1$.
Then
\[
\varphi(n)=n\prod_{i=1}^t\Bigl(1-\frac1{p_i}\Bigr)=n\prod_{i=1}^t\frac{p_i-1}{p_i}.
\]
Thus
\[
\frac{n}{\varphi(n)}=\prod_{i=1}^t\frac{p_i}{p_i-1}.
\]
The condition $\varphi(n)\mid n$ is equivalent to $\frac{n}{\varphi(n)}\in\mathbb{Z}$.

Let $S$ be the set of odd primes dividing $n$.
For each odd prime $p\in S$, the integer $p-1$ is even, so the denominator of the product above contributes at least one factor $2$.
Therefore the denominator of $\prod_{p\mid n} \frac{p}{p-1}$ is divisible by $2^{|S|}$.
The numerator of this product is $\prod_{p\mid n} p$, whose $2$-adic valuation is exactly $1$ if $2\mid n$ and $0$ if $2\nmid n$.
Hence, for the whole product to be an integer, one must have $|S|\le 1$ (otherwise the denominator has a factor $2^2$ while the numerator has at most one factor $2$).

If $|S|=0$, then $n$ is a power of $2$ (or $n=1$), so $n=2^\alpha 3^\beta$ with $\beta=0$.
If $|S|=1$, write $S=\{p\}$ with $p$ an odd prime.
Then $2\mid n$, because if $2\nmid n$ then $\frac{n}{\varphi(n)}=\frac{p}{p-1}$ which is not an integer for $p>2$.
So the set of primes dividing $n$ is exactly $\{2,p\}$ and
\[
\frac{n}{\varphi(n)}=\frac{2}{1}\cdot \frac{p}{p-1} = \frac{2p}{p-1}.
\]
If this is an integer, then $p-1\mid 2p$.
Since $\gcd(p-1,p)=1$, we have $p-1\mid 2$.
Thus $p-1\in\{1,2\}$, giving $p\in\{2,3\}$.
Because $p$ is odd, $p=3$.
So the only possible odd prime divisor is $3$.
Therefore $n$ has no prime divisors other than $2$ and $3$, i.e.
$n=2^\alpha 3^\beta$.

Conversely, if $n=2^\alpha 3^\beta$, then $\varphi$ is multiplicative on coprime factors and
\[
\varphi(2^\alpha)=\begin{cases}1,&\alpha=0,\\2^{\alpha-1},&\alpha\ge1,\end{cases}
\qquad
\varphi(3^\beta)=\begin{cases}1,&\beta=0,\\2\cdot 3^{\beta-1},&\beta\ge1.\end{cases}
\]
Hence $\varphi(2^\alpha 3^\beta)$ is a product of powers of $2$ and $3$ and divides $2^\alpha 3^\beta$.
This proves the equivalence. $\square$

Lemma 2 (infinite family for many negative shifts).
Let $k\in\mathbb{N}$ satisfy $\varphi(k)\mid k$ (equivalently $k=2^\alpha 3^\beta$ by Lemma~1), and let $a=-k$.
Then there exist infinitely many $n\in\mathbb{N}$ such that $\varphi(n)\mid (n+a)$.

Proof.
Let $k$ be fixed with $\varphi(k)\mid k$ and choose any prime $p\ge5$.
Then $\gcd(k,p)=1$, so multiplicativity gives
\[
\varphi(kp)=\varphi(k)\varphi(p)=\varphi(k)(p-1).
\]
Also
\[
(kp)+a = kp-k = k(p-1).
\]
Because $\varphi(k)\mid k$, we may write $k=\varphi(k)\,t$ for some integer $t$.
Then
\[
kp-k = \varphi(k)\,t\,(p-1),
\]
which is divisible by $\varphi(k)(p-1)=\varphi(kp)$.
Thus $\varphi(kp)\mid (kp-k)$.
Since there are infinitely many primes $p\ge5$, this gives infinitely many $n=kp$ solving $\varphi(n)\mid (n-k)$.
$\square$

Fast reality check (computation; exact search).
Using a brute-force check of $\varphi(n)\mid (n+a)$ for $1\le n\le 200$ and $-10\le a\le 10$, we found the following counts of solutions (and the first few solutions in each case):
\[
\begin{array}{r|l}
 a & \#\{n\le 200: \varphi(n)\mid n+a\}\ \text{ and sample solutions}\\\hline
 -10 & 6\ \ (1,2,4,6,10,50)\\
 -9 & 8\ \ (1,2,3,5,9,21,27,105)\\
 -8 & 16\ \ (1,2,4,6,8,12,14,16,24,40,\dots)\\
 -7 & 6\ \ (1,2,3,7,15,49)\\
 -6 & 15\ \ (1,2,4,6,10,18,30,42,\dots)\\
 -5 & 6\ \ (1,2,3,5,25,165)\\
 -4 & 19\ \ (1,2,4,6,8,12,20,28,\dots)\\
 -3 & 5\ \ (1,2,3,9,195)\\
 -2 & 27\ \ (1,2,4,6,10,14,22,26,\dots)\\
 -1 & 47\ \ (1,2,3,5,7,11,13,17,\dots)\\
 0 & 21\ \ (1,2,4,6,8,12,16,18,\dots)\\
 1 & 4\ \ (1,2,3,15)\\
 2 & 7\ \ (1,2,4,6,10,30,70)\\
 3 & 7\ \ (1,2,3,5,9,21,45)\\
 4 & 10\ \ (1,2,4,6,8,12,14,20,\dots)\\
 5 & 5\ \ (1,2,3,7,75)\\
 6 & 8\ \ (1,2,4,6,10,18,42,90)\\
 7 & 5\ \ (1,2,3,5,33)\\
 8 & 12\ \ (1,2,4,6,8,12,16,22,\dots)\\
 9 & 10\ \ (1,2,3,9,11,15,27,\dots)\\
 10 & 11\ \ (1,2,4,6,10,14,26,\dots)
\end{array}
\]
These data are only sanity checks (they do not address infinitude), but they confirm the explicit infinite families from Lemma~2 when $a=-k$ with $k=2^\alpha 3^\beta$ (e.g. $a=-6$ has many solutions $n=6p$).

VERIFICATION

(1) Lemma~1 uses only the exact product formula $\varphi(n)=n\prod_{p\mid n}(1-1/p)$ and a 2-adic valuation argument. The key step is: if $n$ has at least two odd prime divisors, then the denominator contributes $2^2$ while the numerator contributes at most one factor of $2$. This is correct because the numerator is $\prod_{p\mid n} p$ (each prime once).
(2) Lemma~2 checks divisibility by explicit factorization and multiplicativity. The only external input is the infinitude of primes (Euclid), which is standard.
(3) Edge cases: $k=1$ gives $a=-1$ and $n=p$ prime yields $\varphi(p)=p-1\mid p-1$, consistent.

FINAL

**UNRESOLVED**

(i) Strongest proved partial result.
We proved that $\varphi(n)\mid n$ iff $n=2^\alpha 3^\beta$ (Lemma~1), and consequently for every $a=-2^\alpha 3^\beta$ (including $a=0$ and $a=-1$) there are infinitely many solutions $n$ (Lemma~2, via $n=kp$ with $p$ prime and $k=-a$).

(ii) First gap (crisp statement).
For a fixed integer $a$ not of the form $a=-2^\alpha 3^\beta$, prove or disprove: there exist infinitely many $n\in\mathbb{N}$ such that $\varphi(n)\mid (n+a)$.

(iii) Top 3 next moves.
1. Try to construct infinite families for positive $a$ (e.g. $a=1$) by forcing $n+a$ to factor as $\varphi(n)\,t$ with controlled prime factors of $n$.
2. Search systematically for parametric families $n=f(t)$ where $\varphi(n)$ has an explicit form and the congruence $n\equiv -a\pmod{\varphi(n)}$ becomes a manageable congruence in $t$.
3. Computation: for each fixed $a$ (say $|a|\le 20$), search for solutions with restricted prime patterns (e.g. $n=pq$, $n=2^e p$, $n=2^e 3^f p$) to guess candidate infinite families.

(iv) Minimal counterexample structure.
A potential counterexample to Graham's conjecture would be an integer $a$ for which all solutions $n$ have bounded size or belong to a very sparse set; computationally, $a=1$ looks comparatively sparse for $n\le 200$ (only $n=1,2,3,15$), suggesting that a minimal counterexample might be a small positive $a$ with strong congruence obstructions.

