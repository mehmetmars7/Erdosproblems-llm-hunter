% Erdos Problem #973
% URL: https://www.erdosproblems.com/973

Does there exist a constant $C>1$ such that, for every $n\geq 2$, there exists a sequence $z_i\in \mathbb{C}$ with $z_1=1$ and $\lvert z_i\rvert \geq 1$ for all $1\leq i\leq n$ with\[\max_{2\leq k\leq n+1}\left\lvert \sum_{1\leq i\leq n}z_i^k\right\rvert < C^{-n}?\] This is Problem 7.3 in \cite{Ha74}, where it is attributed to Erd\H{o}s. Erd\H{o}s proved (as described on p.35 of \cite{Tu84b}) that such a sequence does exist with $\lvert z_i\rvert\leq 1$. Indeed, Erd\H{o}s' construction gives a value of $C\approx 1.32$. In \cite{Er92f} (a different) Erd\H{o}s refines this analysis, proving that if\[M_2=\min_{z_j} \max_{2\leq k\leq n+1} \left\lvert \sum_{1\leq j\leq n}z_j^k\right\rvert,\]where the minimum is take over all $z_j\in \mathbb{C}$ with $\max \lvert z_j\rvert=1$, then\[(1.746)^{-n} < M_2 < (1.745)^{-n}.\]Tang notes in the comments that Theorem 6.1 of \cite{Tu84b} implies that, if $\lvert z_i\rvert \geq 1$ for all $i$, then\[\max_{2\leq k\leq n+1}\left\lvert \sum_{1\leq i\leq n}z_i^k\right\rvert \geq (2e)^{-(1+o(1))n}.\]See also [519] . References [Er92f] Erd\H{o}s, L., On some problems of {P}. {T}ur\'an concerning power sums of complex numbers . Acta Math. Hungar. (1992), 11--24. [Ha74] Hayman, W. K., Research problems in function theory: new problems . (1974), 155--180. [Tu84b] Tur\'an, Paul, On a new method of analysis and its applications . (1984), xvi+584.

%Erdos problem 973

\medskip
\noindent\textbf{1) FORMAL RESTATEMENT}\\
Question: Does there exist an absolute constant $C>1$ such that for every integer $n\ge 2$ there exist complex numbers $z_1,\dots,z_n$ with
\[
 z_1=1,\qquad |z_i|\ge 1\ (1\le i\le n),
\]
and such that
\[
 \max_{2\le k\le n+1}\Big|\sum_{i=1}^n z_i^k\Big| < C^{-n}?
\]

\medskip
\noindent\textbf{2) QUICK LITERATURE/CONTEXT CHECK}\\
The statement records that Erd\H{o}s produced sequences under the opposite constraint $|z_i|\le 1$ (and more generally $\max|z_i|=1$) with exponentially small power sums, while Tur\'an's theorem gives an exponential lower bound when $|z_i|\ge 1$.  I do not use any other external input.

\medskip
\noindent\textbf{3) ATTACK PLAN}\\
\textbf{Proof track:} attempt an explicit construction meeting $|z_i|\ge 1$ that forces the first $n$ power sums to be exponentially small; investigate whether one can restrict attention to $|z_i|=1$ without loss, or whether varying radii helps.

\textbf{Disproof track:} attempt to prove a universal lower bound of the form $\max_{2\le k\le n+1}|\sum z_i^k|\ge c^{-n}$ for some $c<1$ that is too large to allow any fixed $C>1$; this seems incompatible with the stated Tur\'an lower bound (which is also exponential), so a disproof would require sharpening.

I only manage a complete analysis for the smallest case $n=2$.

\medskip
\noindent\textbf{4) WORK}\\
\textbf{FAST REALITY CHECK (the case $n=2$).}  For $n=2$ and $z_1=1$, writing $z_2=re^{i\theta}$, I performed a coarse grid search over $r\in[1,4]$ and $\theta$ on a $2000$-point mesh.  The minimal value found for
\[
\max\big(|1+z_2^2|,\ |1+z_2^3|\big)
\]
was $0.6180339887\dots$ attained at $r=1$ and $\theta=2\pi/5$.
The next proposition proves this is the true optimum.

\medskip
\noindent\textbf{Lemma 973.1 (monotonicity in the radius for a single term).}\label{lem:973-radius}
Fix an integer $k\ge 1$ and an angle $\theta\in\mathbb{R}$.  For $r\ge 1$, define
\[
F_{k,\theta}(r):=\big|1+r^k e^{ik\theta}\big|.
\]
Then $F_{k,\theta}(r)$ is nondecreasing for $r\ge 1$.

\noindent\textbf{Proof.}
Compute the squared modulus:
\[
F_{k,\theta}(r)^2 = \big(1+r^k e^{ik\theta}\big)\big(1+r^k e^{-ik\theta}\big)=1+r^{2k}+2r^k\cos(k\theta).
\]
Differentiate for $r>0$:
\[
\frac{d}{dr}F_{k,\theta}(r)^2 = 2k r^{2k-1} + 2k r^{k-1}\cos(k\theta)
=2k r^{k-1}\big(r^k+\cos(k\theta)\big).
\]
For $r\ge 1$, we have $r^k\ge 1$ and $\cos(k\theta)\ge -1$, hence $r^k+\cos(k\theta)\ge 0$, so the derivative is nonnegative.
Thus $F_{k,\theta}(r)^2$ is nondecreasing for $r\ge 1$, and therefore so is $F_{k,\theta}(r)=\sqrt{F_{k,\theta}(r)^2}$. \qed

\medskip
\noindent\textbf{Proposition 973.2 (exact minimax for $n=2$).}\label{prop:973-n2}
Let $n=2$, with $z_1=1$ and $z_2\in\mathbb{C}$ satisfying $|z_2|\ge 1$.
Define
\[
M_2^{(\ge 1)}:=\inf_{|z_2|\ge 1}\ \max\big(|1+z_2^2|,\ |1+z_2^3|\big).
\]
Then
\[
M_2^{(\ge 1)}=\frac{\sqrt5-1}{2},
\]
and the infimum is attained, for instance, at $z_2=e^{2\pi i/5}$ (and by symmetry at $e^{-2\pi i/5}$).

\noindent\textbf{Proof.}
Write $z_2=re^{i\theta}$ with $r\ge 1$.
By Lemma~\ref{lem:973-radius}, for each fixed $\theta$ the quantities $|1+z_2^2|=|1+r^2 e^{2i\theta}|$ and $|1+z_2^3|=|1+r^3 e^{3i\theta}|$ are both nondecreasing in $r$ for $r\ge 1$.
Therefore their maximum is minimized at the smallest allowed radius $r=1$.
Hence
\[
M_2^{(\ge 1)} = \inf_{\theta\in\mathbb{R}} \max\big(|1+e^{2i\theta}|,\ |1+e^{3i\theta}|\big).
\]
Using $|1+e^{i\varphi}|=2|\cos(\varphi/2)|$, we rewrite
\[
|1+e^{2i\theta}|=2|\cos\theta|,\qquad |1+e^{3i\theta}|=2|\cos(3\theta/2)|.
\]
Let
\[
m(\theta):=\max\big(|\cos\theta|,\ |\cos(3\theta/2)|\big),\qquad\text{so }M_2^{(\ge 1)}=2\inf_{\theta} m(\theta).
\]
The function $m$ is $2\pi$-periodic and even, so it suffices to minimize over $\theta\in[0,\pi]$.
At any minimizer $\theta_*$, one must have $|\cos\theta_*|=|\cos(3\theta_*/2)|$; otherwise if (say) $|\cos\theta_*|>|\cos(3\theta_*/2)|$ then by continuity one can perturb $\theta_*$ slightly to decrease $|\cos\theta|$ while keeping $|\cos(3\theta/2)|$ strictly below the previous maximum, reducing $m$, contradicting minimality.
Thus we solve the equation
\[
|\cos\theta|=|\cos(3\theta/2)|.
\]
There are two cases.

\emph{Case 1: $\cos\theta=\cos(3\theta/2)$.}
Then either $\theta\equiv 3\theta/2\pmod{2\pi}$ (giving $\theta\equiv 0$) or $\theta\equiv -3\theta/2\pmod{2\pi}$ (giving $5\theta/2\equiv 0\pmod{2\pi}$, hence $\theta\in\{0,4\pi/5\}$ within $[0,\pi]$).
At $\theta=0$, $m(0)=1$.  At $\theta=4\pi/5$, $|\cos\theta|=|\cos(4\pi/5)|=|\cos(\pi-\pi/5)|=\cos(\pi/5)=\frac{\sqrt5+1}{4}\approx 0.809$.
So this case yields $2m\ge 2\cdot \frac{\sqrt5+1}{4}=\frac{\sqrt5+1}{2}\approx 1.618$.

\emph{Case 2: $\cos\theta=-\cos(3\theta/2)$.}
Using $\cos a+\cos b=2\cos\frac{a+b}{2}\cos\frac{a-b}{2}$, the equation becomes
\[
\cos\theta+\cos(3\theta/2)=0
\iff 2\cos\Big(\frac{5\theta}{4}\Big)\cos\Big(\frac{\theta}{4}\Big)=0.
\]
Within $\theta\in[0,\pi]$, the factor $\cos(\theta/4)$ is positive, so we must have $\cos(5\theta/4)=0$, i.e.
\[
\frac{5\theta}{4}=\frac{\pi}{2}+\pi m \quad\Longrightarrow\quad \theta=\frac{2\pi}{5}(1+2m).
\]
The unique solution in $[0,\pi]$ is $\theta=2\pi/5$.
At this angle,
\[
|\cos\theta|=\cos(2\pi/5)=\cos(72^\circ)=\frac{\sqrt5-1}{4}.
\]
Hence $m(2\pi/5)=\frac{\sqrt5-1}{4}$ and the corresponding objective value is
\[
2m(2\pi/5)=\frac{\sqrt5-1}{2}.
\]
Comparing with Case~1 (and boundary values), this is the global minimum.
Thus $M_2^{(\ge 1)}=(\sqrt5-1)/2$, achieved at $z_2=e^{i\theta}=e^{2\pi i/5}$. \qed

\medskip
\noindent\textbf{5) VERIFICATION}\\
Lemma~\ref{lem:973-radius} was checked by explicit differentiation; the key inequality $r^k+\cos(k\theta)\ge r^k-1\ge 0$ holds for $r\ge 1$.
In Proposition~\ref{prop:973-n2}, the reduction to $r=1$ is valid because both competing terms are separately nondecreasing in $r$ for fixed $\theta$.
The trigonometric minimization checks all solutions to the equal-max condition and compares values.

\medskip
\noindent\textbf{6) FINAL}\\
\textbf{UNRESOLVED}

(i) \textbf{Strongest proved partial result.} For $n=2$ the exact minimax value under the constraint $|z_2|\ge 1$ is $M_2^{(\ge 1)}=(\sqrt5-1)/2$ (Proposition~\ref{prop:973-n2}), achieved on the unit circle.  Also, for each individual term $|1+r^k e^{ik\theta}|$ is minimized at the smallest allowed radius $r=1$ (Lemma~\ref{lem:973-radius}).

(ii) \textbf{First gap (crisp).} For general $n$, construct (or rule out) sequences $z_1=1$, $|z_i|\ge 1$ such that
\[
\max_{2\le k\le n+1}\Big|\sum_{i=1}^n z_i^k\Big|\le C^{-n}
\]
for some absolute constant $C>1$ independent of $n$.

(iii) \textbf{Top 3 next moves.}
\begin{itemize}
\item Determine whether one may assume $|z_i|=1$ without loss for the minimax problem under $|z_i|\ge 1$; Lemma~\ref{lem:973-radius} shows this for $n=2$ but not for coupled sums.
\item Attempt an explicit construction: choose $z_i$ as (possibly repeated) points on or outside the unit circle designed to make the first $n$ power sums small simultaneously (e.g. via polynomial/Newton-identity control of power sums), while keeping $|z_i|\ge 1$.
\item Compute numerical optima for small $n$ with constrained optimization (e.g. restricting to $|z_i|=1$ and varying phases) to guess the correct exponential base for the $|z_i|\ge 1$ constraint.
\end{itemize}

(iv) \textbf{Minimal counterexample structure.} A negative answer would mean: for every $C>1$ there are infinitely many $n$ such that for all choices with $z_1=1$ and $|z_i|\ge 1$,
\[
\max_{2\le k\le n+1}\Big|\sum_{i=1}^n z_i^k\Big|\ge C^{-n}.
\]
In other words the minimax quantity for the $|z_i|\ge 1$ constraint would decay at best subexponentially (or with base $\le 1$), contradicting the heuristic suggested by the stated Tur\'an-type exponential lower bound; any such counterexample would likely hinge on a coupled constraint preventing simultaneous cancellation across the $n$ different exponents $k=2,\dots,n+1$.


