\section{Problem 20 (Sunflowers): Round 2 --- continuation \& gap-closure}

\subsection{1) ROUND-2 OBJECTIVE}
\textbf{Path (A) --- full proof in a nontrivial special slice.}

Round 1 fully treated the case $(n,k)=(2,3)$ by translating to graphs and obtaining the exact value $f(2,3)=7$ (Lemma~20.3). In this round I close the entire \emph{$n=2$ slice} by determining \emph{the exact value of $f(2,k)$ for every $k\ge 2$}. This is the most promising gap-closure direction because (i) it strictly extends Round~1's exact computation, (ii) it produces an infinite family of exact values and extremal constructions, and (iii) it rules out any counterexample strategy based on $2$-uniform families.

\subsection{2) ROUND-1 FOUNDATION USED}
I rely only on the following Round~1 items:
\begin{itemize}
\item The definition of $k$-sunflower and $f(n,k)$ (Round~1 \S1).
\item The graph translation idea from Lemma~20.3 for $n=2$ (2-sets as edges; a $3$-sunflower is either a $K_{1,3}$ or a matching of size $3$).
\end{itemize}
No other Round~1 lemma is reused.

\subsection{3) NEW INSIGHT / TOOL (ROUND-2)}
\textbf{New tool: exact extremal theorem for graphs with bounded matching and bounded maximum degree.}

I upgrade the Round~1 translation from $(n,k)=(2,3)$ to general $(n,k)=(2,k)$, and then invoke the sharp extremal theorem of Abbott--Hanson--Sauer (1972), stated as Theorem~3.2 in \cite{Kupavskii2025}. This yields an \emph{exact formula} (and extremal structure) for the maximum size of a $k$-sunflower-free $2$-uniform family.

\subsection{4) ATTACK PLAN (ROUND-2)}
\textbf{Round~1 gap addressed:} Round~1 computed $f(2,3)$ exactly but did not treat $f(2,k)$ for general $k$.

\textbf{Claims to prove now:}
\begin{enumerate}
\item (General translation) A $k$-sunflower of $2$-sets is \emph{equivalent} to either a star $K_{1,k}$ or a matching of size $k$.
\item (Exact extremal value) The maximum number of edges in a graph with \emph{no} $K_{1,k}$ and \emph{no} matching of size $k$ is
\[
\operatorname{ex}_\Delta(k):=
\begin{cases}
 k(k-1), & k\ \text{odd},\\
 \Big\lfloor \dfrac{(k-1)(2k-1)}{2}\Big\rfloor, & k\ \text{even},
\end{cases}
\]
with explicit extremal constructions.
\end{enumerate}

\textbf{Why this closes the gap:} Once (1) holds, the $n=2$ sunflower problem becomes a graph extremal problem. Theorem~3.2 of \cite{Kupavskii2025} solves that extremal problem exactly, yielding $f(2,k)=\operatorname{ex}_\Delta(k)+1$.

\subsection{5) WORK (ROUND-2)}

\paragraph{Lemma 20.4 (Sunflowers of 2-sets are stars or matchings).}
Fix $k\ge 3$. Let $\mathcal F$ be a family of $2$-element sets, and let $G$ be the simple graph with edge set $E(G)=\mathcal F$.
Then $\mathcal F$ contains a $k$-sunflower if and only if $G$ contains either
\begin{enumerate}
\item a matching of size $k$ (i.e., $k$ pairwise-disjoint edges), or
\item a star $K_{1,k}$ (i.e., a vertex of degree at least $k$).
\end{enumerate}

\emph{Proof.}
If $\{e_1,\dots,e_k\}\subseteq E(G)$ is a $k$-sunflower, then each $e_i$ has size $2$, so the common pairwise intersection (the core) has size either $0$ or $1$.
\begin{itemize}
\item If the core is empty, then $e_i\cap e_j=\varnothing$ for $i\neq j$, so the edges are pairwise disjoint: a matching of size $k$.
\item If the core has size $1$, then there is a vertex $v$ with $v\in e_i$ for every $i$, and since $e_i\cap e_j=\{v\}$, the other endpoints are all distinct. Thus $\{e_1,\dots,e_k\}$ is exactly the edge set of a $K_{1,k}$ centered at $v$.
\end{itemize}
Conversely, a matching of size $k$ is a $k$-sunflower with core $\varnothing$, and a $K_{1,k}$ is a $k$-sunflower with core equal to its center vertex.
\hfill$\square$

\paragraph{Definition.}
Let
\[
M(k):=\max\{|E(G)|:\ \Delta(G)\le k-1\ \text{and}\ \nu(G)\le k-1\},
\]
where $\Delta(G)$ is maximum degree and $\nu(G)$ is maximum matching size.
By Lemma~20.4,
\begin{equation}
 f(2,k)=M(k)+1.
 \tag{5.1}
\end{equation}

\paragraph{Theorem 20.5 (Exact value of $f(2,k)$ for all $k$).}
For every integer $k\ge 2$,
\[
 f(2,k)=
 \begin{cases}
  k(k-1)+1, & k\ \text{odd},\\[4pt]
  \Big\lfloor \dfrac{(k-1)(2k-1)}{2}\Big\rfloor + 1, & k\ \text{even}.
 \end{cases}
\]
Moreover, the lower bounds are attained by explicit $k$-sunflower-free graphs described below.

\emph{Proof.}
We split into lower bounds (constructions) and the matching upper bound.

\smallskip
\noindent\textbf{Step 1: Lower bounds (explicit constructions).}

\smallskip
\noindent\emph{Case 1: $k$ odd.}
Let $G$ be the disjoint union of two cliques $K_k\cup K_k$.
Then $\Delta(G)=k-1$, and
\[
|E(G)|=2\binom{k}{2}=k(k-1).
\]
Each $K_k$ (with $k$ odd) has matching number $(k-1)/2$, hence
\[
\nu(G)=2\cdot\frac{k-1}{2}=k-1,
\]
so $G$ has no matching of size $k$.
Also $\Delta(G)=k-1$ rules out $K_{1,k}$.
By Lemma~20.4, this graph is $k$-sunflower-free, so
\begin{equation}
M(k)\ge k(k-1)\qquad (k\ \text{odd}).
\tag{5.2}
\end{equation}

\smallskip
\noindent\emph{Case 2: $k$ even.}
Set $N:=2k-1$ and label vertices by $\mathbb Z/N\mathbb Z$.
Let $H$ be the circulant graph on $\mathbb Z/N\mathbb Z$ with step set
\[ S:=\{\pm 1,\pm 2,\dots,\pm (k/2-1)\},\]
so every vertex has degree $k-2$ in $H$.
Consider the cycle $C$ whose edges are the ``step $k/2$'' edges
\[ \{\{x,x+k/2\}: x\in\mathbb Z/N\mathbb Z\}. \]
Since $\gcd(k/2,N)=1$, these edges form a single cycle of length $N$.
Let $M$ be a maximum matching in this cycle $C$, so $|M|=\lfloor N/2\rfloor = k-1$ and exactly one vertex (say $0$) is uncovered.
Define $G:=H\cup M$.
Then in $G$, every vertex covered by $M$ has degree $(k-2)+1=k-1$, while the uncovered vertex has degree $k-2$. Hence $\Delta(G)=k-1$.
The number of edges is
\[
|E(G)|=|E(H)|+|M|=\frac{N(k-2)}{2}+(k-1)=\frac{k(2k-3)}{2}=\Big\lfloor\frac{(k-1)(2k-1)}{2}\Big\rfloor.
\]
Because $|V(G)|=2k-1<2k$, $G$ cannot contain a matching of size $k$.
Also $\Delta(G)=k-1$ rules out $K_{1,k}$.
Thus $G$ is $k$-sunflower-free, and
\begin{equation}
M(k)\ge \Big\lfloor\frac{(k-1)(2k-1)}{2}\Big\rfloor\qquad (k\ \text{even}).
\tag{5.3}
\end{equation}

\smallskip
\noindent\textbf{Step 2: Matching upper bounds (Abbott--Hanson--Sauer).}

We now invoke the following sharp extremal result.

\smallskip
\noindent\textbf{Theorem 20.6 (Abbott--Hanson--Sauer, 1972; as stated in \cite[Thm.~3.2]{Kupavskii2025}).}
Let $s\ge 2$ and let $G$ be a graph with \emph{no} matching of size $s$ and \emph{no} vertex of degree $s$.
Then
\[
|E(G)|\le
\begin{cases}
 s(s-1), & s\ \text{odd},\\
 (s-1)^2+\dfrac{s-2}{2}, & s\ \text{even}.
\end{cases}
\]

\smallskip
\noindent\emph{Applying Theorem~20.6 with $s=k$:}
The conditions ``no vertex of degree $k$'' and ``no matching of size $k$'' are exactly
\[
\Delta(G)\le k-1\qquad\text{and}\qquad \nu(G)\le k-1.
\]
Hence for every $k\ge 2$,
\begin{equation}
M(k)\le
\begin{cases}
 k(k-1), & k\ \text{odd},\\
 (k-1)^2+\dfrac{k-2}{2}=\Big\lfloor\dfrac{(k-1)(2k-1)}{2}\Big\rfloor, & k\ \text{even}.
\end{cases}
\tag{5.4}
\end{equation}

\smallskip
\noindent\textbf{Step 3: Conclude $M(k)$ and hence $f(2,k)$.}
Combining the lower bounds (5.2)--(5.3) with the upper bounds (5.4) shows equality holds in both parity cases, i.e., $M(k)$ equals the stated quantity.
Finally, (5.1) gives the displayed formula for $f(2,k)$.
\hfill$\square$

\subsection{6) ADVERSARIAL VERIFICATION}
\textbf{(i) Core-size check for Lemma~20.4.}
For $2$-sets, the core of a sunflower can only have size $0$ or $1$; there is no hidden third case. The proof explicitly splits by $|C|\in\{0,1\}$.

\textbf{(ii) Boundary cases.}
\begin{itemize}
\item $k=2$: any two distinct edges form a $2$-sunflower with core their intersection, so $f(2,2)=2$. Theorem~20.5 gives $\lfloor(1\cdot 3)/2\rfloor+1=2$.
\item $k=3$: Theorem~20.5 gives $3\cdot 2+1=7$, matching Round~1 Lemma~20.3.
\item $k=4$: Theorem~20.5 gives $\lfloor 3\cdot 7/2\rfloor+1=11$; the explicit construction in Step~1 (cycle-plus-matching augmentation) recovers the $10$-edge example on $7$ vertices.
\end{itemize}

\textbf{(iii) Construction correctness for even $k$.}
The only subtle point is that the ``step $k/2$'' edges form a single odd cycle, so a maximum matching has size $k-1$ and leaves exactly one uncovered vertex. The gcd computation
\[\gcd(k/2,2k-1)=1\]
is valid because any common divisor divides both $k$ and $2k-1$, hence divides $1$.
Also, the added matching edges are not already present in $H$ because the step $k/2$ exceeds the maximum step $(k/2-1)$ used in $H$.

\textbf{(iv) Dependence on external theorem.}
The only externally imported fact is Theorem~20.6, used solely to upper-bound $M(k)$. All hypotheses are verified explicitly for $s=k$.

\subsection{7) FINAL (EXACTLY ONE)}
\textbf{UNRESOLVED (BUT STRICTLY ADVANCED).}

The original sunflower conjecture question (existence of $c_k$ with $f(n,k)\le c_k^n$ for all $n$) remains open.
However, Round~2 strictly advances the investigation by closing an infinite family of exact cases:
\emph{the entire $n=2$ slice is solved exactly}, yielding an explicit formula for $f(2,k)$ for every $k\ge 2$.

\subsection{8) COMPLETION ESTIMATE}
\noindent\textbf{COMPLETION: 30\%}

\subsection{9) REFERENCES}
\begin{thebibliography}{9}
\bibitem{Kupavskii2025}
A.~Kupavskii,
\emph{Erd\H{o}s's forbidden sunflower problem},
arXiv:2511.17142 (2025). (Theorem~3.2 is used here.)
\end{thebibliography}
