\section*{Problem 761}

\subsection*{1) FORMAL RESTATEMENT}
For a graph $G$:
\begin{itemize}
\item The \emph{chromatic number} $\chi(G)$ is the minimum $m$ such that $V(G)$ can be partitioned into $m$ independent sets.
\item The \emph{cochromatic number} $\zeta(G)$ is the minimum $m$ such that $V(G)$ can be partitioned into $m$ parts,
where each part induces either a clique or an independent set.
\item For an \emph{orientation} $D$ of $G$ (replace each undirected edge by one of the two possible directions),
let $\chi_d(D)$ be the minimum number of colours needed to colour $V(D)$ so that each colour class induces an
\emph{acyclic} digraph (equivalently: contains no directed cycle). This is the standard \emph{dichromatic number} of $D$.
\item Define the (undirected) \emph{dichromatic number} of $G$ as
\[
\delta(G):=\max\{\chi_d(D): D\ \text{is an orientation of }G\}.
\]
\end{itemize}

The questions are:
\begin{enumerate}
\item[(Q1)] Does $\chi(G)$ unbounded force $\delta(G)$ unbounded? Equivalently: for every $k$ does there exist $f(k)$
so that $\chi(G)\ge f(k)$ implies $\delta(G)\ge k$?
\item[(Q2)] Does $\zeta(G)$ large force the existence of a (not necessarily induced) subgraph $H\subseteq G$ with large $\delta(H)$?
\end{enumerate}

\subsection*{2) QUICK LITERATURE/CONTEXT CHECK}
The first question is widely known as the Erd\H{o}s--Neumann-Lara conjecture. A key reference is Mohar--Wu,
\emph{Dichromatic number and fractional chromatic number} (arXiv:1510.05982), which states the conjecture and
notes that even the case $k=3$ remains open.
They prove a \emph{fractional} version: large fractional chromatic number forces large fractional dichromatic number.

The second question is attributed to Erd\H{o}s--Gimbel in the same source list.
A known result of Alon--Krivelevich--Sudakov (1997) shows that large $\chi(G)$ forces the presence of a subgraph
with large $\zeta(\cdot)$ (in the order $\gg \chi/\log\chi$); hence a positive answer to (Q2) would imply a positive answer to (Q1).

\subsection*{3) ATTACK PLAN}
For (Q1) one can try to:
\begin{itemize}
\item prove quantitative lower bounds on $\delta(G)$ in terms of graph parameters that are forced by large $\chi(G)$
(e.g. density, expansion, fractional chromatic number), or
\item construct a counterexample family with $\chi(G)\to\infty$ but with every orientation $k$-dicolourable for fixed $k$.
\end{itemize}
Mohar--Wu suggest that the fractional version may be a stepping stone: if one could control $\chi_f$ from below by $\chi$
within a large class of graphs, then one would obtain partial confirmations.

For (Q2), one may try to show: if $\zeta(G)$ is large, then $G$ contains a subgraph $H$ with large $\chi_f(H)$,
then appeal to the fractional dichromatic lower bounds.

\subsection*{4) WORK}
\paragraph{Basic observations.}
\begin{itemize}
\item Always $1\le \delta(G)\le \chi(G)$.
Indeed, for any orientation $D$ a proper $\chi(G)$-colouring makes every colour class an independent set, hence acyclic.
\item If $G$ is a forest then $\delta(G)=1$ (every orientation is acyclic).
If $G$ contains a cycle, then $\delta(G)\ge 2$ (orient that cycle cyclically).
Hence the threshold for guaranteeing $\delta(G)\ge 2$ is $\chi(G)\ge 3$ (since $\chi\le 2$ graphs can be forests).
\item For a complete graph $K_n$, $\delta(K_n)$ equals the maximum dichromatic number of an $n$-vertex tournament.
Known results on tournaments show this grows unboundedly with $n$ (on the order $\Theta(n/\log n)$ up to constants),
so $\chi(K_n)=n$ does force $\delta(K_n)\to\infty$.
\end{itemize}

\paragraph{Fractional evidence (Mohar--Wu).}
Let $\chi_f(G)$ denote the fractional chromatic number of $G$.
Mohar--Wu prove that if $\chi_f(G)\ge t$, then the fractional dichromatic number of $G$ satisfies
\[
\delta_f(G)\ \gtrsim\ \frac{t}{\log t}.
\]
Since $\delta(G)\ge \delta_f(G)$, this yields a (logarithmically weakened) implication
``large $\chi_f$ $\Rightarrow$ large $\delta$''.
This does not settle (Q1) because $\chi_f$ can be much smaller than $\chi$ in some graphs.

\paragraph{Link between (Q2) and (Q1).}
If (Q2) had a positive answer of the form
\[\zeta(G)\ge g(k)\ \Rightarrow\ \exists H\subseteq G\ \text{with }\delta(H)\ge k,\]
then combining with the Alon--Krivelevich--Sudakov theorem that large $\chi(G)$ forces a subgraph with
$\zeta$ on the order $\chi/\log\chi$ would imply (Q1).

\subsection*{5) OBSTRUCTIONS/CAVEATS}
\begin{itemize}
\item The key open difficulty in (Q1) is already at $k=3$: it is unknown whether there exist graphs with arbitrarily
large chromatic number whose every orientation is $2$-dicolourable.
\item For (Q2), even translating ``large $\zeta$'' into ``large $\chi_f$ on some subgraph'' is nontrivial; $\zeta$ mixes cliques and independent sets.
\end{itemize}

\subsection*{6) FINAL}
\textbf{UNRESOLVED.}
\begin{itemize}
\item[(i)] \emph{Where I got stuck:} I did not prove a general implication $\chi(G)\to\infty\Rightarrow \delta(G)\to\infty$ nor construct a counterexample.
\item[(ii)] \emph{Partial progress:} I recorded standard reductions and verified the easy threshold for $\delta\ge 2$,
and explained how fractional results (Mohar--Wu) and the AKS subgraph theorem connect to the questions.
\item[(iii)] \emph{What seems next:} Either (a) build a family with bounded $\delta$ but unbounded $\chi$ by controlling all orientations,
or (b) strengthen Mohar--Wu-type arguments to integral dichromatic number (or show that large $\chi$ forces large $\chi_f$
inside a subgraph with structure forcing large $\delta$).
\item[(iv)] \emph{Best guess:} The conjecture is true but difficult; available evidence is consistent with at least polylogarithmic growth.
\end{itemize}

\subsection*{7) COMPLETION ESTIMATE}
$15\%$.

