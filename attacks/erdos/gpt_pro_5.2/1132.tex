% Erdos Problem #1132

\subsection*{FORMAL RESTATEMENT}
Let $(x_k)_{k\ge 1}$ be an infinite sequence of real numbers in $[-1,1]$ such that for each $n$ the initial segment $x_1,\dots,x_n$ consists of distinct points (otherwise the Lagrange denominators vanish).
For each $n\ge 1$, define $l_k^{(n)}(x)$ (for $1\le k\le n$) to be the Lagrange fundamental polynomial associated to the nodes $x_1,\dots,x_n$:
\[
 l_k^{(n)}(x)=\frac{\prod_{1\le i\le n,\ i\ne k}(x-x_i)}{\prod_{1\le i\le n,\ i\ne k}(x_k-x_i)}.
\]
Define
\[
L_n(x):=\sum_{k=1}^n \bigl|l_k^{(n)}(x)\bigr|.
\]
Questions:
\begin{enumerate}
\item Must there exist $x\in(-1,1)$ such that $L_n(x) > \tfrac{2}{\pi}\log n - O(1)$ for infinitely many $n$?
\item Is it true that for almost all $x\in(-1,1)$,
\[\limsup_{n\to\infty} \frac{L_n(x)}{\log n}\ge \frac{2}{\pi}?\]
\end{enumerate}

\subsection*{QUICK LITERATURE/CONTEXT CHECK}
I only record context explicitly stated in the problem file.
The file states that for each fixed set of nodes $x_1,\dots,x_n$ one has
\[
\max_{x\in[-1,1]} L_n(x) > \frac{2}{\pi}\log n - O(1)
\]
(Erd\H{o}s), and that Bernstein implies the set of $x$ with $\limsup L_n(x)/\log n\ge 2/\pi$ is everywhere dense.

\subsection*{ATTACK PLAN}
\textbf{Proof track ideas.}
\begin{itemize}
\item Use compactness of $[-1,1]$ and continuity of $L_n$ to attempt a diagonal argument extracting a single $x$ witnessing large values infinitely often; the main obstacle is that $L_n$ changes with $n$.
\item For special classes of sequences $(x_k)$, attempt to show $L_n(x)$ blows up (e.g. for clustered nodes) or has controlled growth (e.g. for well-separated nodes).
\end{itemize}
\textbf{Disproof track ideas.}
\begin{itemize}
\item Try to build a ``good'' infinite sequence $x_k$ such that for each fixed $x$ the growth of $L_n(x)$ is smaller than $(2/\pi)\log n$ along all large $n$.
\end{itemize}

\subsection*{WORK}
\textbf{Lemma 1132.1 (universal pointwise lower bound).}
For any $n\ge 1$, any distinct nodes $x_1,\dots,x_n$, and any $x\in[-1,1]$,
\[
L_n(x)=\sum_{k=1}^n |l_k^{(n)}(x)|\ge 1.
\]

\emph{Proof.}
For fixed $n$, the Lagrange polynomials satisfy the partition of unity
\[
\sum_{k=1}^n l_k^{(n)}(x)=1
\]
for all $x$, by the same argument as Lemma 1129.1.
Applying the triangle inequality yields
\[
1=\left|\sum_{k=1}^n l_k^{(n)}(x)\right|\le \sum_{k=1}^n |l_k^{(n)}(x)|=L_n(x).
\]
\qed

\textbf{Lemma 1132.2 (values at nodes).}
For each $n\ge 1$ and each $1\le j\le n$, one has
\[
L_n(x_j)=1.
\]

\emph{Proof.}
By definition of the Lagrange basis, $l_k^{(n)}(x_j)=\delta_{kj}$.
Therefore
\[
L_n(x_j)=\sum_{k=1}^n |\delta_{kj}|=1.
\qedhere
\]

\textbf{Lemma 1132.3 (continuity and attainment of maxima).}
For fixed $n$ and fixed distinct nodes $x_1,\dots,x_n$, the function $x\mapsto L_n(x)$ is continuous on $[-1,1]$ and hence attains a maximum on $[-1,1]$.

\emph{Proof.}
Each $l_k^{(n)}(x)$ is a polynomial, hence continuous. The absolute value of a continuous function is continuous, and finite sums of continuous functions are continuous. Thus $L_n$ is continuous on the compact set $[-1,1]$, so it attains its maximum by the extreme value theorem.
\qed

\textbf{FAST REALITY CHECK (a concrete infinite sequence).}
Take the explicit (clustered) sequence $x_k=1-2^{-k}$.
For this sequence I computed the exact values of $L_n(0)$ for $2\le n\le 10$.
\begin{verbatim}
Sequence x_k = 1-2^{-k}
  n=2:  L_n(0) = 5
  n=3:  L_n(0) = 29
  n=4:  L_n(0) = 269
  n=5:  L_n(0) = 4589
  n=6:  L_n(0) = 151469
  n=7:  L_n(0) = 9845549
  n=8:  L_n(0) = 1270075949
  n=9:  L_n(0) = 326409519149
  n=10: L_n(0) = 167448083323949
\end{verbatim}
This shows that for some sequences, $L_n(x)$ can grow extremely quickly at a fixed interior point $x=0$.

\subsection*{VERIFICATION}
\begin{itemize}
\item Lemmas 1132.1--1132.2 rely only on the basic Lagrange interpolation identities and are correct whenever the nodes are distinct.
\item The exact $L_n(0)$ values for $x_k=1-2^{-k}$ were computed using rational arithmetic (Fractions), so the integers listed are exact.
\item The example does not address the problem's lower-bound order $(2/\pi)\log n$ in a meaningful way (growth here is much faster), but it sanity-checks that no pointwise upper bound of order $\log n$ can hold without additional assumptions on $(x_k)$.
\end{itemize}

\subsection*{FINAL}
\textbf{UNRESOLVED}

(i) \textbf{Strongest proved partial result.}
For each $n$ and node set, $L_n(x)\ge 1$ for all $x$ and $L_n(x_j)=1$ at each node (Lemmas 1132.1--1132.2). For a concrete clustered sequence $x_k=1-2^{-k}$, one can have $L_n(0)$ extremely large even for moderate $n$ (exact values listed above).

(ii) \textbf{First gap (crisp).}
Prove or disprove that for every infinite sequence $(x_k)$ there exists a \emph{single} point $x\in(-1,1)$ such that $L_n(x)\ge \tfrac{2}{\pi}\log n - O(1)$ for infinitely many $n$.

(iii) \textbf{Top 3 next moves.}
\begin{itemize}
\item Attempt a compactness/diagonal lemma: if $\max_{x}L_n(x)$ is large for each $n$, prove existence of a point $x$ where $L_n(x)$ is frequently large by controlling the modulus of continuity of $L_n$.
\item Look for an ``energy'' inequality relating averages of $L_n(x)$ over $x$ (or $\int L_n(x)\,dx$) to $\max_x L_n(x)$, to transfer large maxima into a large limsup set.
\item Explore candidate counterexamples: sequences $(x_k)$ designed so that the locations of maxima of $L_n$ move around rapidly with $n$, potentially preventing a fixed $x$ from seeing large values infinitely often.
\end{itemize}

(iv) \textbf{Minimal counterexample structure.}
A counterexample would be a sequence $(x_k)$ such that for every fixed $x\in(-1,1)$, $L_n(x)$ stays below $(2/\pi-\delta)\log n$ (up to $O(1)$) for all sufficiently large $n$, while still allowing $\max_x L_n(x)$ to be large for each $n$.


