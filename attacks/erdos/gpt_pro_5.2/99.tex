\section*{Erd\H{o}s Problem \#99}

\begin{enumerate}[leftmargin=*,label=\arabic*.]

\item \textbf{Formal Restatement.}

For a finite set $A\subset\mathbb{R}^2$, define
\[
\delta(A):=\min\{\|x-y\|:x,y\in A,\ x\ne y\}\quad\text{(minimum pairwise distance)},
\]
\[
\operatorname{diam}(A):=\max\{\|x-y\|:x,y\in A\}\quad\text{(diameter)}.
\]
For each $n\ge 2$, let
\[
D(n):=\inf\{\operatorname{diam}(A): A\subset\mathbb{R}^2,\ |A|=n,\ \delta(A)\ge 1\}.
\]
By scaling, this is equivalent to restricting to $\delta(A)=1$.
A set $A$ is \emph{diameter-minimising} if $|A|=n$, $\delta(A)\ge 1$, and $\operatorname{diam}(A)=D(n)$.

\medskip
\noindent\textbf{Question (\#99).} For all sufficiently large $n$, must every diameter-minimising $A$ contain three points that form an equilateral triangle of side length $1$?

\item \textbf{Quick literature/context check.}

The Erd\H{o}s Problems page lists this as open (last edited date not shown on the page extract above, but accessed 2026-01-17).\cite{ErdosProblems99}
It notes that asymptotically the minimal diameter is achieved by points arranged like a triangular (hexagonal) lattice intersected with a disk (``Thue's theorem'' on densest circle packing underlies this asymptotic picture).\cite{ErdosProblems99}
It also notes the statement is false for $n=4$ (a diameter-minimising configuration can be a unit square, which has no unit equilateral triangle) and that small $n$ are studied by Bezdek--Fodor (1999).\cite{ErdosProblems99,BeFo99}

\item \textbf{Attack plan.}

\textbf{Proof track.}
Show that any diameter-minimiser must be ``close enough'' to the triangular lattice that it forces at least one exact unit equilateral triangle. A possible route is to study the contact graph of unit distances (tangencies of radius-$1/2$ disks), show many contacts are forced in an extremiser, and then prove that sufficiently dense unit-circle packings necessarily contain a triangular face (a $3$-cycle) which corresponds to a unit equilateral triangle.

\textbf{Disproof track.}
Construct, for arbitrarily large $n$, a diameter-minimising configuration whose unit-distance graph is triangle-free (equivalently, has no triple of pairwise unit distances), hence contains no unit equilateral triangle. The difficulty is that being diameter-minimising is a global extremal condition, so one must match the true optimum $D(n)$.

\textbf{Phase 1 small cases.}
Check $n=3$ (equilateral triangle is optimal) and $n=4$ (square configuration) to understand behaviour at the first nontrivial sizes.

\item \textbf{Work.}

I did not resolve the ``sufficiently large $n$'' question. I provide a fully proved sharp statement for $n=4$ (consistent with the remark in the problem statement) as a concrete verified base case.

\subsubsection*{A proved base case: $n=4$.}

\begin{proposition}\label{prop:D4}
Let $A\subset\mathbb{R}^2$ be a set of $4$ points with $\delta(A)\ge 1$. Then $\operatorname{diam}(A)\ge \sqrt{2}$. Moreover, equality is achieved (for example) by the vertices of a unit square.
Consequently, there exists a diameter-minimising $4$-point set with $\delta(A)=1$ that contains no unit equilateral triangle.
\end{proposition}

\begin{proof}
Let $A=\{p_1,p_2,p_3,p_4\}$ with all pairwise distances $\ge 1$.
Let $\operatorname{diam}(A)=D$.
Consider the convex hull $\operatorname{conv}(A)$.

\smallskip
\noindent\emph{Case 1: $\operatorname{conv}(A)$ is a quadrilateral.}
Then all four points are vertices of a convex quadrilateral. In any convex quadrilateral, at least one interior angle is $\ge 90^\circ$ (otherwise the sum of four angles would be $<4\cdot 90^\circ=360^\circ$).
Assume without loss of generality that the angle at vertex $p_2$ in the quadrilateral $p_1p_2p_3p_4$ is $\angle p_1p_2p_3\ge 90^\circ$.
In the triangle $p_1p_2p_3$, by the law of cosines,
\[
\|p_1-p_3\|^2
=\|p_1-p_2\|^2+\|p_2-p_3\|^2-2\|p_1-p_2\|\,\|p_2-p_3\|\cos(\angle p_1p_2p_3)
\ge \|p_1-p_2\|^2+\|p_2-p_3\|^2,
\]
because $\cos(\angle p_1p_2p_3)\le 0$.
Since $\|p_1-p_2\|\ge 1$ and $\|p_2-p_3\|\ge 1$, we obtain
\[
\|p_1-p_3\|^2 \ge 1^2+1^2 =2,
\]
so $\|p_1-p_3\|\ge \sqrt{2}$.
Therefore $D\ge \sqrt{2}$.

\smallskip
\noindent\emph{Case 2: $\operatorname{conv}(A)$ is a triangle.}
Then three of the points, say $p_1,p_2,p_3$, are the triangle's vertices and the remaining point $p_4$ lies strictly inside that triangle.
Because $p_4$ is inside, the rays from $p_4$ to $p_1,p_2,p_3$ partition the full angle $2\pi$ around $p_4$ into three angles whose sum is $2\pi$.
Hence at least one of these angles is $\ge 2\pi/3$; without loss of generality,
\[
\angle p_1p_4p_2 \ge \frac{2\pi}{3}.
\]
Using the law of cosines in triangle $p_1p_4p_2$ and the distance bounds $\|p_1-p_4\|\ge 1$, $\|p_2-p_4\|\ge 1$, we get
\[
\|p_1-p_2\|^2
=\|p_1-p_4\|^2+\|p_2-p_4\|^2-2\|p_1-p_4\|\,\|p_2-p_4\|\cos(\angle p_1p_4p_2)
\ge 1+1-2\cdot 1\cdot 1\cos\Bigl(\frac{2\pi}{3}\Bigr)=3.
\]
Thus $\|p_1-p_2\|\ge \sqrt{3}$, and in particular $D\ge \sqrt{3}>\sqrt{2}$.

\smallskip
\noindent In both cases, $\operatorname{diam}(A)=D\ge \sqrt{2}$.
Equality is achieved by the vertices of a unit square, which satisfy $\delta(A)=1$ and $\operatorname{diam}(A)=\sqrt{2}$.
A unit square contains no equilateral triangle of side $1$ (any triple of vertices has side lengths $\{1,1,\sqrt{2}\}$).
\end{proof}

\subsubsection*{Remarks toward large $n$ (not a solution).}

The conjecture ``large $n$ forces a unit equilateral triangle in every diameter-minimiser'' can be interpreted as a rigidity statement about optimal finite circle packings: asymptotically optimal packings resemble the triangular lattice, which has abundant unit equilateral triangles. The challenge is to bridge from an asymptotic density statement to the existence of an \emph{exact} unit equilateral triangle in the finite extremiser.

\item \textbf{Verification.}

Proposition~\ref{prop:D4} is self-contained and checks both combinatorial hull cases.
It also cleanly separates the two geometric possibilities for four points.
The conclusion about absence of a unit equilateral triangle in a unit square is immediate by inspection of distances.

\item \textbf{FINAL: UNRESOLVED.}

\begin{itemize}[leftmargin=*]
\item[(i)] \textbf{Strongest fully proved partial result:}
Proposition~\ref{prop:D4}: $D(4)=\sqrt{2}$ and there exists a diameter-minimising configuration (unit square) with no unit equilateral triangle.
\item[(ii)] \textbf{Most plausible route forward:}
Relate diameter-minimisation to forced structure in the unit-distance (tangency) graph of the associated circle packing; prove that sufficiently optimal finite packings must contain a triangular face (a $3$-cycle) in the contact graph, which would be a unit equilateral triangle.
\item[(iii)] \textbf{First place the proof attempt breaks:}
Current arguments from asymptotic optimality (triangular lattice density) do not directly imply the existence of an \emph{exact} unit equilateral triangle in the finite extremiser, because one may imagine perturbations that preserve diameter while breaking exact unit-triangle equalities.
\item[(iv)] \textbf{What a counterexample would have to look like:}
An infinite family of diameter-minimising sets $A_n$ with $\delta(A_n)=1$ and no triple of points forming an equilateral triangle of side $1$.
\end{itemize}

\item \textbf{Completion estimate.}

\textbf{COMPLETION: 20\%.}

\end{enumerate}

\begin{thebibliography}{99}

\bibitem{Er86}
P.~Erd\H{o}s,
\newblock On some metric and combinatorial geometric problems,
\newblock \emph{Discrete Mathematics} \textbf{60} (1986), 147--153.

\bibitem{FiRe92}
P.~C. Fishburn and J.~A. Reeds,
\newblock Unit distances between vertices of a convex polygon,
\newblock \emph{Computational Geometry} \textbf{2} (1992), no.~2, 81--91.

\bibitem{EFPR93}
P.~Erd\H{o}s, Z.~F\"uredi, J.~Pach, and I.~Z. Ruzsa,
\newblock The grid revisited,
\newblock \emph{Discrete Mathematics} \textbf{111} (1993), 189--196.

\bibitem{GK15}
L.~Guth and N.~H. Katz,
\newblock On the Erd\H{o}s distinct distances problem in the plane,
\newblock \emph{Annals of Mathematics} \textbf{181} (2015), 155--190.

\bibitem{BeFo99}
A.~Bezdek and F.~Fodor,
\newblock Minimal diameter of certain sets in the plane,
\newblock \emph{Journal of Combinatorial Theory, Series A} \textbf{85} (1999), no.~1, 105--111.

\end{thebibliography}

