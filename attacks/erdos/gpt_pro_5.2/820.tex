\subsection*{Erdos Problem \#820}

\paragraph{FORMAL RESTATEMENT.}
For each integer $n\ge1$, define $H(n)$ to be the smallest integer $\ell\ge2$ such that there exists an integer $k$ with $2\le k<\ell$ and
\[
\gcd(k^n-1,\ \ell^n-1)=1.
\]
(Here $\gcd$ is the greatest common divisor.)
Equivalently, $H(n)$ is the least $\ell$ such that some smaller base $k$ produces multiplicatively ``coprime'' repunit-like numbers $k^n-1$ and $\ell^n-1$.
The problem asks, among other things, whether $H(n)=3$ for infinitely many $n$, i.e. whether
\[
\gcd(2^n-1,\ 3^n-1)=1
\]
holds for infinitely many $n$, and for estimates on the growth of $H(n)$.

\paragraph{QUICK LITERATURE/CONTEXT CHECK.}
The statement records: (a) Erd\H{o}s proved there exists $c>0$ and infinitely many $n$ with $H(n) > \exp(n^{c/(\log\log n)^2})$; (b) a sketch (in comments) of a stronger lower bound $H(n) > \exp(n^{c/\log\log n})$ for infinitely many $n$; and (c) the exact values $H(n)$ for $1\le n\le 10$.
I do not use any results not already stated in the problem text.

\paragraph{ATTACK PLAN.}
\emph{Elementary track:} prove basic upper and lower bounds that hold for every $n$, and compute small values to sanity-check.
\emph{Deeper track:} understand common prime divisors of $k^n-1$ and $\ell^n-1$ via multiplicative orders modulo primes; without external results this track remains largely open.

\paragraph{WORK.}
\textbf{Lemma 820.1 (trivial lower bound).}
For every $n\ge1$, $H(n)\ge 3$.

\emph{Proof.}
If $\ell=2$, the only possible choice is $k=1$ (since $k<\ell$), but then $k^n-1=0$ and
\[
\gcd(0,2^n-1)=2^n-1>1.
\]
So no $\ell=2$ works, hence $H(n)\ge3$.
\hfill $\square$

\textbf{Lemma 820.2 (a general exponential upper bound via radicals).}
For every $n\ge2$, let
\[
\ell := \operatorname{rad}(2^n-1) := \prod_{p\mid (2^n-1)} p
\]
(the product of the distinct prime divisors of $2^n-1$). Then with $k=2$ we have
\[
\gcd(2^n-1,\ \ell^n-1)=1.
\]
Consequently,
\[
H(n) \le \operatorname{rad}(2^n-1) \le 2^n-1.
\]

\emph{Proof.}
Let $p$ be any prime divisor of $2^n-1$. By definition of $\ell$, we have $p\mid\ell$, hence $\ell\equiv 0\pmod p$ and so
\[
\ell^n-1 \equiv -1 \not\equiv 0 \pmod p.
\]
Thus no prime $p\mid (2^n-1)$ divides $\ell^n-1$.
Every common divisor of $2^n-1$ and $\ell^n-1$ must in particular be composed of primes dividing $2^n-1$, but we have just shown none of those primes divide $\ell^n-1$.
Therefore $\gcd(2^n-1,\ell^n-1)=1$.
Since $n\ge2$ implies $2^n-1\ge3$, we have $\ell\ge3$ and indeed $k=2<\ell$.
Finally, $\operatorname{rad}(m)\le m$ for every integer $m\ge1$, giving $\ell\le 2^n-1$.
\hfill $\square$

\textbf{FAST REALITY CHECK (exact $H(n)$ for $n\le 10$ and gcd test for $k=2,\ell=3$).}
By direct brute-force search over $\ell$ and $k<\ell$, I confirmed the sequence in the problem statement:
\[
H(1),\dots,H(10)=3,3,3,6,3,18,3,6,3,12.
\]
Also, for $1\le n\le 100$, the set of $n$ such that $\gcd(2^n-1,3^n-1)=1$ is
\[
\{1,2,3,5,7,9,13,14,15,17,19,21,25,26,27,29,31,34,37,38,39,41,45,47,49,51,53,57,59,61,62,63,65,67,71,73,74,79,81,85,87,89,91,93,94,97,98\}.
\]

\paragraph{VERIFICATION.}
Lemma 820.2: the argument checks each prime divisor $p$ of $2^n-1$ individually; choosing $\ell$ divisible by all such primes forces $\ell^n-1\equiv -1\pmod p$, so these primes cannot divide $\ell^n-1$. Since any common divisor must use primes from $2^n-1$, the gcd is $1$.
The computed sequences are finite checks.

\paragraph{FINAL.} \textbf{UNRESOLVED.}
\begin{enumerate}
\item[(i)] Strongest proved partial results here: the universal bounds $3\le H(n)\le \operatorname{rad}(2^n-1)\le 2^n-1$ for $n\ge2$ (Lemmas 820.1--820.2), and verification of the exact values $H(n)$ for $1\le n\le 10$ plus the list of $n\le100$ for which $\gcd(2^n-1,3^n-1)=1$.
\item[(ii)] First gap: decide whether $\gcd(2^n-1,3^n-1)=1$ for infinitely many $n$ (equivalently, whether $H(n)=3$ infinitely often).
\item[(iii)] Top 3 next moves: (1) study primes $p$ for which $2^n\equiv 3^n\equiv 1\pmod p$, i.e. orders of $2$ and $3$ modulo $p$ both divide $n$; (2) compute $H(n)$ and $\gcd(2^n-1,3^n-1)$ for much larger $n$ to look for patterns in the exceptional set; (3) attempt to bound from below the number of distinct prime factors shared by $k^n-1$ and $\ell^n-1$ for small $\ell$ by using order considerations.
\item[(iv)] A minimal counterexample to ``$H(n)=3$ infinitely often'' would be a situation where for all sufficiently large $n$ there exists a prime $p$ with $2^n\equiv 3^n\equiv 1\pmod p$; equivalently, the multiplicative orders of $2$ and $3$ modulo $p$ both divide $n$, forcing a persistent shared prime divisor of $2^n-1$ and $3^n-1$.
\end{enumerate}
