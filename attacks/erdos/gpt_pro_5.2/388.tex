\section*{Problem 388. Equal products of two disjoint blocks of consecutive integers}

\subsection*{Formal restatement}
For integers $m\ge 0$ and $k\ge 1$ define the \emph{rising factorial block}
\[
P_k(m) \;:=\; \prod_{i=1}^{k}(m+i) \;=\; \frac{(m+k)!}{m!}.
\]
The problem asks about integer solutions to
\begin{equation}
\label{eq:P388-main}
P_{k_1}(m_1)=P_{k_2}(m_2)
\end{equation}
under the constraints
\[
 k_1,k_2>3,\qquad m_1,m_2\in\mathbb Z_{\ge 0},\qquad m_1+k_1\le m_2.
\]
The wording ``Are there only finitely many solutions?'' is ambiguous:
\begin{itemize}[leftmargin=2em]
\item[(A)] \emph{Fixed lengths:} for each fixed pair $(k_1,k_2)$ with $k_1,k_2>3$, are there only finitely many pairs $(m_1,m_2)$ satisfying \eqref{eq:P388-main} and $m_1+k_1\le m_2$?
\item[(B)] \emph{All parameters:} are there only finitely many quadruples $(m_1,k_1,m_2,k_2)$ with $k_1,k_2>3$ and $m_1+k_1\le m_2$ satisfying \eqref{eq:P388-main}?
\end{itemize}
The statement ``classify all solutions'' suggests (B), while the appearance of fixed $a,b$ in the generalization suggests (A) as a natural first step.

A more general variant mentioned is
\begin{equation}
\label{eq:P388-ab}
 a\,P_{k_1}(m_1)=b\,P_{k_2}(m_2),
\end{equation}
with fixed nonzero integers $a,b$ and $k_1>2$.

\subsection*{Quick literature/context check (browsing available)}
A classical source of examples and discussion is the MathOverflow question ``Equal products of consecutive integers''; several small identities are known, e.g.
$2\cdot 3\cdot 4\cdot 5\cdot 6 = 8\cdot 9\cdot 10$ and
$19\cdot 20\cdot 21\cdot 22 = 55\cdot 56\cdot 57$.
The smallest known example with \emph{both} block lengths $>3$ and with disjointness $m_1+k_1\le m_2$ is
\begin{equation}
\label{eq:P388-known}
8\cdot 9\cdot 10\cdot 11\cdot 12\cdot 13\cdot 14
\;=\;
63\cdot 64\cdot 65\cdot 66
\;=\;17297280.
\end{equation}
This example also appears in OEIS A163263 (numbers with multiple non-overlapping consecutive-product representations).
The Erd\H{o}s Problems site (Problem \#388) records these examples and highlights the ``smoothness'' obstruction below.

\subsection*{Attack plan}
\begin{enumerate}[leftmargin=2em]
\item Prove structural necessary conditions for any solution (e.g. compare lengths; prime-factor ``smoothness'' constraints).
\item Do a bounded brute-force search to verify the smallest solutions and gain confidence about the landscape.
\item Attempt to leverage general finiteness theorems for polynomial Diophantine equations of the form $f(x)=g(y)$ (e.g. Siegel/Bilu--Tichy) to prove finiteness for fixed $(k_1,k_2)$; identify the precise obstruction to converting this into a full classification across \emph{all} lengths.
\end{enumerate}

\subsection*{Work}
\paragraph{Lemma 1 (the longer block must be the earlier one).}
Assume $m_1+k_1\le m_2$ and \eqref{eq:P388-main} holds with $k_1,k_2\ge 1$. Then $k_1>k_2$.

\begin{proof}
Because $m_1+k_1\le m_2$, every factor on the right is at least $m_2+1\ge m_1+k_1+1$, while every factor on the left is at most $m_1+k_1$. Hence
\[
P_{k_2}(m_2)=\prod_{j=1}^{k_2}(m_2+j) \;\ge\; (m_1+k_1+1)^{k_2}.
\]
Also
\[
P_{k_1}(m_1)=\prod_{i=1}^{k_1}(m_1+i) \;\le\; (m_1+k_1)^{k_1}.
\]
If $k_1\le k_2$ then
\[(m_1+k_1+1)^{k_2} \;\ge\; (m_1+k_1+1)^{k_1} \;>\; (m_1+k_1)^{k_1} \;\ge\; P_{k_1}(m_1),\]
contradicting $P_{k_1}(m_1)=P_{k_2}(m_2)$. Therefore $k_1>k_2$.
\end{proof}

\paragraph{Lemma 2 (``smoothness'' obstruction).}
Let $A:=m_1+k_1$. If \eqref{eq:P388-main} holds, then every prime factor of each integer in the right block
$\{m_2+1,\dots,m_2+k_2\}$ is $\le A$. Equivalently, the entire right block consists of $A$-smooth integers.

\begin{proof}
All prime factors of $P_{k_1}(m_1)=\prod_{i=1}^{k_1}(m_1+i)$ are $\le m_1+k_1=A$, since each factor $m_1+i\le A$. Equality \eqref{eq:P388-main} forces the same set of primes (with the same multiplicities) to divide $P_{k_2}(m_2)=\prod_{j=1}^{k_2}(m_2+j)$, hence every prime dividing any $m_2+j$ must be $\le A$.
\end{proof}

\paragraph{A computational check (small search).}
A brute-force search over
\[
4\le k_1,k_2\le 10,\quad 0\le m_1\le 500,\quad m_1+k_1\le m_2\le 500
\]
finds exactly one solution with both lengths $>3$, namely \eqref{eq:P388-known} (i.e. $m_1=7,k_1=7,m_2=62,k_2=4$).
This does not prove uniqueness, but it supports the conjecture recorded on the Erd\H{o}s Problems site that \eqref{eq:P388-known} might be the only solution under the stated constraints.

\paragraph{Where a full solution would likely come from.}
Lemmas 1--2 reduce any putative classification to understanding long runs of consecutive $A$-smooth integers {
\em above} $A$ (since $m_2+1>A$).
One plausible route is to combine:
\begin{itemize}[leftmargin=2em]
\item quantitative bounds on the maximal length of a run of $A$-smooth integers in $[x,x+L]$ with $x$ much larger than $A$,
\item together with the size constraint imposed by \eqref{eq:P388-main}, which forces the growth rates of $P_{k_1}(m_1)$ and $P_{k_2}(m_2)$ to match.
\end{itemize}
I did not find a way to complete this program in a gap-free way.

\subsection*{Verification}
\begin{itemize}[leftmargin=2em]
\item Lemma 1 uses only monotone inequalities and the disjointness condition $m_1+k_1\le m_2$; the strict inequality step $(A+1)^{k_1}>(A)^{k_1}$ is immediate.
\item Lemma 2 is immediate from prime-factor containment.
\item The identity \eqref{eq:P388-known} was verified by direct multiplication (in a separate computation).
\end{itemize}

\subsection*{Final}
\begin{quote}
\textbf{UNRESOLVED.}
\begin{enumerate}[leftmargin=2.2em]
\item[(i)] \textbf{Strongest proved partial result:}
Any solution of \eqref{eq:P388-main} with $m_1+k_1\le m_2$ must satisfy $k_1>k_2$ and the entire right block $\{m_2+1,\dots,m_2+k_2\}$ is $(m_1+k_1)$-smooth (Lemmas 1--2).
In addition, a bounded brute-force search (specified above) finds only the solution \eqref{eq:P388-known} with $k_1,k_2>3$ in that range.
\item[(ii)] \textbf{First gap / obstruction:}
Turning the smoothness obstruction into a global finiteness or uniqueness theorem requires nontrivial control of long runs of $A$-smooth numbers above $A$ that is uniform in the parameters and compatible with the size constraint from \eqref{eq:P388-main}.
\item[(iii)] \textbf{Top 3 next moves:}
(1) Translate \eqref{eq:P388-main} into a statement about smoothness of a short interval and try to apply strongest available bounds on smooth numbers in short intervals.
(2) For fixed $(k_1,k_2)$, treat $P_{k_1}(x)=P_{k_2}(y)$ as an integral-point problem on an algebraic curve and attempt an effective determination of all integral points (in small degrees this reduces to elliptic/hyperelliptic methods).
(3) Expand the computational search using modular sieving informed by Lemma 2 (restricting to candidate intervals whose elements are all $A$-smooth).
\item[(iv)] \textbf{What a minimal counterexample (to ``only finitely many solutions'') would look like:}
It would require an infinite family of pairs of disjoint blocks with $k_1>k_2\ge 4$ in which the later block forms unusually long runs of $A$-smooth numbers relative to $A=m_1+k_1$, while simultaneously matching the magnitude constraint $\log P_{k_1}(m_1)\approx \log P_{k_2}(m_2)$.
\end{enumerate}
\end{quote}

\subsection*{Completion estimate}
What remains is either (a) an unconditional finiteness/uniqueness theorem for \eqref{eq:P388-main} under $k_1,k_2>3,\ m_1+k_1\le m_2$, or (b) an explicit infinite family (or even a second distinct example). Any such proof likely needs deep input on smooth numbers or an effective resolution of integral points on families of curves.

%============================================================
