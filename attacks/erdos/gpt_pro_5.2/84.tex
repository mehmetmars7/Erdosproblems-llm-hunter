## 1) FORMAL RESTATEMENT

**Conventions.**
All graphs are **finite, simple, undirected** graphs on a **labelled** vertex set (V(G)={1,\dots,n}).
A **cycle** means a **simple cycle** (no repeated vertices) of length (\ell\ge 3).

**Cycle set.** For a graph (G) on (n) vertices define
[
S(G);:=;{\ell\in{3,4,\dots,n}:\text{ (G) contains a cycle of length (\ell)}}.
]
A set (A\subseteq{3,\dots,n}) is called a **cycle set on (n) vertices** if there exists an (n)-vertex graph (G) with (S(G)=A).

**Counting function.** Let
[
f(n);:=;\bigl|{S(G):; G \text{ is a graph on }n\text{ vertices}}\bigr|
]
(the number of distinct cycle sets realized by (n)-vertex graphs).

**Claims to prove.**

1. (f(n)=o(2^n)) as (n\to\infty), i.e.
   [
   \lim_{n\to\infty}\frac{f(n)}{2^n}=0.
   ]
2. (\displaystyle \lim_{n\to\infty}\frac{f(n)}{2^{n/2}}=\infty), i.e.
   [
   \forall M>0;\exists N;\forall n\ge N:\quad f(n)>M\cdot 2^{n/2}.
   ]

**Edge cases.** For (n\le 2), ({3,\dots,n}=\emptyset), so the only cycle set is (\emptyset) and (f(1)=f(2)=1). For asymptotics we only care about (n\to\infty).

---

## 2) QUICK LITERATURE/CONTEXT CHECK

Web check confirms the current landscape is as stated in the prompt:

* Verstraëte (2004) proved the Erdős–Faudree “vanishing fraction” conjecture (f(n)=o(2^n)) with an explicit upper bound (f(n)\le 2^{,n-n^{1/10}}). ([arXiv][1])
* Nenadov (arXiv v2, Sept 22, 2025) improved this to
  [
  f(n)\le 2^{,n-\Omega!\Bigl(\frac{\sqrt n}{\log^{3/2}n}\Bigr)},
  ]
  which is of the advertised form (2^{n-n^{1/2-o(1)}}). ([arXiv][1])
* The best general lower bound remains the Faudree construction giving (f(n)\ge 2^{n/2}) for even (n), via (S(G)\cap{n/2+1,\dots,n}=A). ([arXiv][1])
* Nenadov explicitly notes that improving (2^{n/2}) even to (2^{(1+c)n/2}) for any constant (c>0) “would already be interesting,” i.e., the lower-bound strengthening is open. ([arXiv][1])

So: **Claim (1)** is settled in the literature; **claim (2)** is (still) a conjectural strengthening.

---

## 3) ATTACK PLAN

### Proof track for (1): (f(n)=o(2^n))

Use Nenadov’s explicit upper bound
[
f(n)\le 2^{,n-\Omega(\sqrt n/\log^{3/2}n)}.
]
Then divide by (2^n) and show the exponent (\Omega(\sqrt n/\log^{3/2}n)\to\infty), forcing the ratio to (0).

### Proof track for (2): (f(n)/2^{n/2}\to\infty)

Try to build (\omega(1)) *many disjoint families* of size (2^{n/2}) each (or one family of size (2^{n/2+\omega(1)})) by:

* combining independent gadgets whose cycle lengths lie in disjoint length ranges, or
* constructing a parametric family where the cycle set allows decoding (n/2) “bits” plus an additional (\omega(1)) parameter.

I attempted several such gadget/gluing ideas (disjoint unions with disjoint length spectra; parity separation; “shift” gadgets), but each ran into the same obstruction: cycle lengths in different components inevitably overlap in ({3,\dots,n}), and ensuring overlap-free decoding seems to require **high girth** in the bit-encoding component, which collapses the entropy.

### Disproof track for (2)

A disproof would require an upper bound (f(n)\le C\cdot 2^{n/2}) along infinitely many (n) (or bounded limsup of the ratio). Nothing in the current literature suggests such a strong upper bound, and Nenadov’s work improves only the (o(2^n)) direction.

**Conclusion of plan:** I can give a fully rigorous proof of (1) and the standard (2^{n/2}) lower bound; for (2) I can’t complete a gap-free argument (it appears open).

---

## 4) WORK

### Phase 1: small-(n) reality check (computed by brute force)

By exhaustive enumeration of all labelled graphs for (n\le 6), the values are:
[
f(1)=1,; f(2)=1,; f(3)=2,; f(4)=4,; f(5)=6,; f(6)=11.
]
This is consistent with (f(n)) growing much more slowly than (2^n) but faster than (2^{n/2}) at these tiny (n) (not asymptotically meaningful).

---

### Theorem 4.1: (f(n)=o(2^n))

**Theorem (Nenadov 2025, quoted precisely).**
There exists an absolute constant (c>0) such that for all sufficiently large (n),
[
f(n);\le;2^{,n-c,\frac{\sqrt n}{\log^{3/2}n}}.
]
Equivalently, (f(n)\le 2^{n-\Omega(\sqrt n/\log^{3/2}n)}). ([arXiv][1])

**Proposition 4.2.** The above bound implies (f(n)=o(2^n)).

**Proof.** For sufficiently large (n), by Nenadov’s theorem,
[
\frac{f(n)}{2^n};\le;\frac{2^{,n-c\frac{\sqrt n}{\log^{3/2}n}}}{2^n}
;=;2^{-c\frac{\sqrt n}{\log^{3/2}n}}.
]
Let (g(n)=c\frac{\sqrt n}{\log^{3/2}n}). Since (\sqrt n\to\infty) and (\log^{3/2}n=o(\sqrt n)), we have (g(n)\to\infty). Therefore (2^{-g(n)}\to 0), hence
[
\lim_{n\to\infty}\frac{f(n)}{2^n}=0,
]
which is exactly (f(n)=o(2^n)). ∎

This completes claim (1).

---

### Lemma 4.3 (Faudree lower bound): (f(n)\ge 2^{\lfloor n/2\rfloor})

I now give a **complete** proof of the standard construction (also described in Nenadov’s introduction). ([arXiv][1])

#### Construction (even (n))

Let (n=2m) be even. For each subset
[
A\subseteq {m+1,m+2,\dots,2m},
]
define a graph (G_A) on vertex set ({1,2,\dots,2m}) with edge set
[
E(G_A)={{i,i+1}:;1\le i\le 2m-1};\cup;{{1,a}:;a\in A}.
]
So (G_A) consists of the Hamilton path (1-2-\cdots-2m) plus chords from vertex (1) to the vertices in (A).

#### Claim 4.4: Every cycle in (G_A) contains vertex (1).

**Proof.** Remove vertex (1) from (G_A). The remaining graph is the path
[
2-3-\cdots-2m,
]
which is a tree and hence contains no cycle. Therefore any cycle of (G_A) must use vertex (1). ∎

#### Claim 4.5: Every cycle length in (G_A) is either

* (Type I) (a) for some (a\in A), or
* (Type II) (b-a+2) for some distinct (a,b\in A) with (a<b).

**Proof.** By Claim 4.4, any cycle (C) contains vertex (1). Since (C) is a simple cycle, vertex (1) has degree exactly (2) in (C), so (C) uses exactly two edges incident to (1). The only neighbors of (1) in (G_A) are vertex (2) (from the Hamilton path) and the vertices (a\in A) (from the added chords).

* If (C) uses edges ({1,2}) and ({1,a}) with (a\in A), then the rest of (C) must be the unique path in the Hamilton path from (2) to (a), namely (2-3-\cdots-a). Hence (C) is
  [
  1-2-3-\cdots-a-1
  ]
  and has length (a) (Type I).
* If (C) uses edges ({1,a}) and ({1,b}) with (a,b\in A), (a<b), then the rest of (C) must be the unique Hamilton-path segment from (a) to (b), namely (a-(a+1)-\cdots-b). Thus
  [
  C:\quad 1-a-(a+1)-\cdots-b-1
  ]
  and its length is ((b-a+1)+1=b-a+2) (Type II).

No other choices of the two edges at vertex (1) are possible, so the classification is complete. ∎

#### Claim 4.6: For all (A\subseteq{m+1,\dots,2m}),

[
S(G_A)\cap{m+1,m+2,\dots,2m};=;A.
]

**Proof.**

* **((\supseteq))** Let (a\in A). By Claim 4.5 (Type I), (G_A) contains the cycle (1-2-\cdots-a-1) of length (a). Since (a\in{m+1,\dots,2m}), we have (a\in S(G_A)\cap{m+1,\dots,2m}). Hence (A\subseteq S(G_A)\cap{m+1,\dots,2m}).

* **((\subseteq))** Let (\ell\in S(G_A)\cap{m+1,\dots,2m}). Then (G_A) has a cycle of length (\ell). By Claim 4.5, either:

  * (\ell=a) for some (a\in A), in which case (\ell\in A); or
  * (\ell=b-a+2) for some (a<b) in (A).

  In the second case, since (a\ge m+1) and (b\le 2m),
  [
  \ell=b-a+2;\le;2m-(m+1)+2;=;m+1.
  ]
  So a Type II cycle has length **at most (m+1)**. Because (\ell\ge m+1) (by assumption (\ell\in{m+1,\dots,2m})), the only way the Type II case can occur is (\ell=m+1).

  If (\ell=m+1) is realized as Type II, then (b-a+2=m+1), i.e. (b-a=m-1). With (a,b\in{m+1,\dots,2m}), the maximum possible difference is (2m-(m+1)=m-1), achieved **only** by ((a,b)=(m+1,2m)). In particular, this forces (a=m+1\in A). Hence (\ell=m+1\in A) also in this subcase.

  Therefore in all cases (\ell\in A), proving (S(G_A)\cap{m+1,\dots,2m}\subseteq A).

Combining the two inclusions gives the desired equality. ∎

#### Consequence 4.7: injectivity and counting

If (A\neq A'), then
[
S(G_A)\cap{m+1,\dots,2m}=A\neq A'=S(G_{A'})\cap{m+1,\dots,2m},
]
so (S(G_A)\neq S(G_{A'})). Hence the map (A\mapsto S(G_A)) is injective, and
[
f(2m);\ge;2^{m};=;2^{n/2}.
]

#### Odd (n)

For (n=2m+1), take the above (G_A) on vertices ({1,\dots,2m}) and add an isolated vertex (2m+1). This does not create any new cycles, so the cycle set remains (S(G_A)). Thus (f(2m+1)\ge 2^m=2^{\lfloor n/2\rfloor}).

So for all (n),
[
f(n);\ge;2^{\lfloor n/2\rfloor}.
]

This is the standard lower bound. ([arXiv][1])

---

### Status of claim (2)

The above only gives
[
\frac{f(n)}{2^{n/2}};\ge;\frac{2^{\lfloor n/2\rfloor}}{2^{n/2}}
;=;\begin{cases}
1,&n\text{ even},[2pt]
2^{-1/2},&n\text{ odd},
\end{cases}
]
which is bounded and **does not** imply divergence to (\infty).

Moreover, Nenadov explicitly indicates that even improving the lower bound beyond (2^{n/2}) by a fixed exponential factor is open. ([arXiv][1])
I do not currently have a gap-free construction proving (\frac{f(n)}{2^{n/2}}\to\infty).

---

## 5) VERIFICATION

### Check of claim (1)

* Quantifiers: Nenadov provides an inequality for all sufficiently large (n), which is enough for the limit definition of (o(2^n)).
* The decay exponent (c\sqrt n/\log^{3/2}n\to\infty) is correct since (\log^{3/2}n=o(\sqrt n)).
* No hidden assumptions: only uses the quoted bound. ([arXiv][1])

### Check of Faudree construction

* Removing vertex (1) leaves a path, so cycles must include (1): correct.
* In a simple cycle, the degree of vertex (1) inside the cycle is (2), so exactly two incident edges at (1) are used: correct.
* Enumerated possibilities for those two edges exhaust all: neighbors of (1) are exactly (2) and (A).
* Computation of Type II max length: (b-a+2\le 2m-(m+1)+2=m+1): correct.
* Uniqueness of achieving length (m+1) as Type II: difference (m-1) forces ((a,b)=(m+1,2m)): correct.
* Injectivity via intersection with ({m+1,\dots,2m}): correct.

### Boundary cases

* For (n=2m) with (m=1) (i.e. (n=2)), the construction is vacuous; but the asymptotic claims begin at large (n). For (n=4) ((m=2)), the construction still works and yields (2^2=4) distinct intersections, consistent with brute force (f(4)=4).

---

## 6) FINAL

**UNRESOLVED**

(i) **Strongest fully proved partial result obtained here.**

* Using Nenadov’s Theorem 1.1, we have a complete deduction that
  [
  f(n)\le 2^{,n-\Omega(\sqrt n/\log^{3/2}n)}\quad\Longrightarrow\quad f(n)=o(2^n).
  ]
  ([arXiv][1])
* The Faudree construction is proved in full detail, yielding
  [
  f(n)\ge 2^{\lfloor n/2\rfloor}\quad\text{for all }n.
  ]
  ([arXiv][1])

(ii) **Exact first gap.**
To prove (\displaystyle \frac{f(n)}{2^{n/2}}\to\infty), it suffices to show **any** explicit function (\omega(n)\to\infty) such that
[
f(n);\ge;2^{n/2},\omega(n)
\quad\text{for all sufficiently large }n.
]
Equivalently, one needs a construction producing at least (2^{n/2+\log_2\omega(n)}) **distinct** cycle sets on (n) vertices. I do not have a rigorous injection/construction achieving (\omega(n)\to\infty).

(iii) **Top 3 next moves (concrete targets).**

1. **Disjoint-length-range gluing lemma:** Find a family (\mathcal G) of graphs on (n_1) vertices with many distinct cycle sets and with **forbidden short lengths** (girth (>k)), so that disjoint union with a (k)-vertex gadget yields a product lower bound (f(n_1+k)\gtrsim f(n_1)\cdot f(k)). The hard part is maintaining (f(n_1)\approx 2^{n_1/2}) while enforcing girth (>k).
2. **Cycle-length “translation” gadget:** Build a graph operation on a fixed (n) vertices that maps a base cycle-set encoding (A) to a family (A\mapsto S_{A,t}) with an extra parameter (t\in{1,\dots,\omega(n)}) and prove ((A,t)\mapsto S_{A,t}) injective.
3. **Probabilistic lower bound on distinct cycle sets:** Define a random sparse family of graphs with (\exp(\Theta(n))) possible edge choices but prove the induced distribution on cycle sets has entropy (>n/2+\omega(1)), implying at least (2^{n/2+\omega(1)}) distinct realized cycle sets.

(iv) **What a minimal counterexample to the divergence claim would look like.**
If (\frac{f(n)}{2^{n/2}}\not\to\infty), then there exists (M<\infty) and infinitely many (n) with (f(n)\le M\cdot 2^{n/2}). Any structural proof of such a bound would have to show that every realizable cycle set (A\subseteq{3,\dots,n}) is determined by only (\le n/2+O(1)) bits of information (strong global constraints among possible cycle lengths). No such constraint is currently known; indeed the literature still treats even improving the (2^{n/2}) lower bound as open. ([arXiv][1])

[1]: https://arxiv.org/pdf/2501.09904 "Improved bound on the number of cycle sets"
