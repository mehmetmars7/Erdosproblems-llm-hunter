\section*{Erd\H{o}s Problem \#257}

\subsection*{1) FORMAL RESTATEMENT}
Let $A\subseteq\mathbb{N}:=\{1,2,3,\dots\}$ be an infinite set. Define
\[
S(A):=\sum_{n\in A}\frac{1}{2^n-1}.
\]
Since $0<\frac{1}{2^n-1}\le 2^{-(n-1)}$ for $n\ge 2$, the sum converges absolutely for every $A$.

The problem asks whether the following statement is true:
\[
(\dagger)\qquad \forall\text{ infinite }A\subseteq\mathbb{N}:\ S(A)\notin\mathbb{Q}.
\]

\subsection*{2) QUICK LITERATURE/CONTEXT CHECK}
(From the standard summary and cited papers.)
\begin{itemize}[leftmargin=2.2em]
\item For $A=\mathbb{N}$, one has
\[S(\mathbb{N})=\sum_{n\ge1}\frac{1}{2^n-1}=\sum_{m\ge1}\frac{d(m)}{2^m},\]
where $d(m)$ is the divisor-counting function; Erd\H{o}s (1948) proved this is irrational \cite{Er48}.
\item Tao--Ter\"{a}v\"{a}inen (2025) proved irrationality when $A$ is the set of primes, and more generally when $A$ is the set of prime powers \cite{TaTe25}.
\item Erd\H{o}s (1968) proved $S(A)$ is irrational whenever $(a,b)=1$ for all distinct $a,b\in A$ and $\sum_{n\in A} \frac{1}{n}<\infty$ \cite{Er68d}.
\item Erd\H{o}s also speculated a far-reaching ``bounded perturbation'' generalisation $\sum_{n\in A}1/(2^n-t_n)$; this was disproved by Kova\v{c}--Tao (2024), who give bounded $t_n$ for which such a sum can be rational \cite{KoTa24}.
\item As of late 2025/early 2026 the general statement $(\dagger)$ remains listed as open on the Erd\H{o}s Problems website \cite{ErP257}.
\end{itemize}

\subsection*{3) ATTACK PLAN}
Two natural routes:
\begin{enumerate}[leftmargin=2.2em]
\item \textbf{Lambert-series / base-2 expansion route.} Rewrite $S(A)$ as a Lambert-type series
$\sum f_A(m)/2^m$ where $f_A(m)$ counts divisors of $m$ lying in $A$; then try to show the base-$2$ expansion cannot be eventually periodic.
\item \textbf{Construction route (for a counterexample).} Try to choose an infinite $A$ whose induced coefficients $f_A(m)$ lead to an eventually periodic base-$2$ expansion after carrying.
\end{enumerate}
I did not manage to complete either route for arbitrary infinite $A$.

\subsection*{4) WORK}
\paragraph{Rewriting as a Lambert-type series.}
For every $n\ge1$,
\[
\frac{1}{2^n-1}=\frac{2^{-n}}{1-2^{-n}}=\sum_{k=1}^{\infty}2^{-kn}
\]
(geometric series). Hence, by Tonelli's theorem (all terms are nonnegative),
\[
S(A)=\sum_{n\in A}\sum_{k\ge1}2^{-kn}
=\sum_{m\ge1}\Bigl(\#\{(n,k):n\in A,\ k\ge1,\ kn=m\}\Bigr)\,2^{-m}.
\]
If we define
\[
 f_A(m):=\#\{d\in A: d\mid m\},
\]
then the map $(n,k)\mapsto m=kn$ identifies the above count with $f_A(m)$, giving the identity
\begin{equation}\label{eq:lambert}
S(A)=\sum_{m\ge1}\frac{f_A(m)}{2^m}.
\end{equation}
In particular, when $A=\mathbb{N}$ we have $f_A(m)=d(m)$.

\paragraph{Tiny numerical experiments (illustrative only).}
Below are high-precision numerical values (computed from long partial sums):
\begin{itemize}[leftmargin=2.2em]
\item $S(\mathbb{N})\approx 1.60669515241529176378\dots$ (the Erd\H{o}s--Borwein constant).
\item For $A=2\mathbb{N}$ (even indices), $S(A)=\sum_{k\ge1}\frac{1}{2^{2k}-1}\approx 0.42109768603342377729\dots$.
\item For $A=\{2^k:k\ge0\}$, $S(A)=\sum_{k\ge0}\frac{1}{2^{2^k}-1}\approx 1.40393682788217832057\dots$.
\end{itemize}
These computations do not distinguish rational/irrational and are included only as sanity checks.

\paragraph{A simple fully proved special case (conditional on $S(\mathbb{N})$).}
If $A$ is cofinite (i.e. $\mathbb{N}\setminus A$ is finite), then
\[
S(A)=S(\mathbb{N})-\sum_{n\in\mathbb{N}\setminus A}\frac{1}{2^n-1}.
\]
The second term is a \emph{finite} sum of rationals and hence rational. Therefore:
\begin{quote}
If $S(\mathbb{N})$ is irrational (known by Erd\H{o}s \cite{Er48}), then $S(A)$ is irrational for every cofinite $A$.
\end{quote}
This does not address sparse $A$.

\subsection*{5) VERIFICATION}
\begin{itemize}[leftmargin=2.2em]
\item Identity \eqref{eq:lambert} is correct: each $2^{-m}$ coefficient counts divisors $d\in A$ of $m$.
\item The problem remains that $f_A(m)$ can be irregular and may be large, and carrying in base $2$ complicates direct periodicity arguments.
\end{itemize}

\subsection*{6) FINAL}
\noindent \textbf{UNRESOLVED}

\medskip
\noindent (i) \textbf{Strongest fully proved partial result obtained here.}
The Lambert-series identity \eqref{eq:lambert} holds for every $A\subseteq\mathbb{N}$ (proved above).
Also, if $A$ is cofinite then $S(A)$ is irrational \emph{assuming} Erd\H{o}s' theorem that $S(\mathbb{N})$ is irrational \cite{Er48}.

\smallskip
\noindent (ii) \textbf{First gap.}
A proof of $(\dagger)$ for arbitrary infinite $A$ would need a method to rule out eventual periodicity (in base $2$) of the sum \eqref{eq:lambert} \emph{after carries}, using only that $A$ is infinite. I did not find such an argument.

\smallskip
\noindent (iii) \textbf{Top 3 next moves.}
\begin{enumerate}[leftmargin=2.2em]
\item Try to extend Erd\H{o}s' 1968 method \cite{Er68d} to drop the pairwise-coprime hypothesis, perhaps by controlling overlaps in divisor structure of $A$.
\item Seek a structural dichotomy for infinite $A$ (e.g. ``contains a dense multiplicative substructure'' vs ``very sparse'') and prove irrationality in each regime.
\item Attempt an explicit counterexample search in the style of Kova\v{c}--Tao \cite{KoTa24}, but with the rigid choice $t_n\equiv1$; i.e. try to design $A$ so that the base-$2$ expansion of \eqref{eq:lambert} becomes eventually periodic.
\end{enumerate}

\smallskip
\noindent (iv) \textbf{Minimal counterexample structure.}
A counterexample would be an infinite $A$ such that $S(A)\in\mathbb{Q}$. Any such $A$ must be infinite (by hypothesis) and cannot be cofinite (since $S(\mathbb{N})$ is irrational). It would likely need to be extremely sparse and carefully engineered so that the induced coefficients $f_A(m)$ in \eqref{eq:lambert} produce eventual periodicity after carrying.

\subsection*{7) COMPLETION ESTIMATE (MANDATORY)}
\noindent COMPLETION: 35\%.

%%%%%%%%%%%%%%%%%%%%%%%%%%%%%%%%%%%%%%%%%%%%%%%%%%%%%%%%%%%%%%%%%%%%%%%%%%%%%%%
