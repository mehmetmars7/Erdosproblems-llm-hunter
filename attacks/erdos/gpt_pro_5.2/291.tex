\section*{Erd\H{o}s Problem 291}
\addcontentsline{toc}{section}{Erd\H{o}s Problem 291}

\subsection*{1) Formal Restatement}
Let $n\ge 1$.
Define
\[
L_n := \mathrm{lcm}(1,2,\dots,n),
\qquad
H_n := \sum_{k=1}^n \frac{1}{k},
\qquad
a_n := L_n\,H_n \in \mathbb{Z}.
\]
So $H_n=a_n/L_n$, but this need not be reduced.
The problem asks:
\begin{quote}
Do both events
\quad (i)\ \gcd(a_n,L_n)=1\ \ and\ \ (ii)\ \gcd(a_n,L_n)>1
occur for infinitely many $n$?
\end{quote}
Equivalently: is it true that the reduced denominator of $H_n$ equals $L_n$ infinitely often, and is smaller than $L_n$ infinitely often?

\subsection*{2) Quick Literature / Context Check (web)}
The Erd\H{o}s Problems site lists this as \emph{open} (last edited 12 Jan 2026).
It records:
\begin{itemize}
\item The second question is ``trivially yes'': e.g. any $n$ beginning with digit $2$ in base $3$ satisfies $3\mid\gcd(a_n,L_n)$.
\item A necessary and sufficient criterion for a prime $p\le n$ to divide $\gcd(a_n,L_n)$ in terms of the \emph{leading base-$p$ digit} of $n$.
\item A heuristic (citing Shiu) that the number of $n\le x$ with $\gcd(a_n,L_n)=1$ should be $\asymp x/\log x$, hence infinite but density zero.
\item A conditional result of Wu--Yan (assuming linear independence of $1/\log p$ over $\mathbb{Q}$ for finite sets of primes, implied by Schanuel's conjecture) that the set where $\gcd(a_n,L_n)>1$ has upper density $1$.
\end{itemize}

\subsection*{3) Attack Plan}
Use the prime-by-prime criterion to:
\begin{enumerate}[label=(\alph*)]
\item Prove rigorously the ``leading digit'' criterion.
\item Deduce the easy infinite family for $\gcd(a_n,L_n)>1$.
\item Explore what would be needed to force $\gcd(a_n,L_n)=1$ infinitely often (namely, avoidance of certain ``bad'' leading digits for every prime $p\le n$).
\end{enumerate}

\subsection*{4) Work}
\subsubsection*{4.1 Leading-digit criterion for prime divisibility}
For a prime $p$ and integer $n\ge 1$, write the base-$p$ expansion
\[
 n = k p^m + r,\qquad 1\le k\le p-1,\quad 0\le r < p^m,
\]
where $p^m\le n < p^{m+1}$. The digit $k$ is the \emph{leading digit} of $n$ in base $p$.

\begin{proposition}[Criterion]
\label{prop:criterion}
Let $n\ge 1$ and let $p\le n$ be a prime.
Let $k$ be the leading digit of $n$ in base $p$.
Then
\[
 p\mid \gcd(a_n,L_n)
\quad\Longleftrightarrow\quad
p\mid \mathrm{num}(H_k),
\]
where $H_k=\sum_{j=1}^k 1/j$ is written in lowest terms and $\mathrm{num}(H_k)$ denotes its numerator.
\end{proposition}
\begin{proof}
We start from the integer identity
\[
 a_n = \sum_{t=1}^n \frac{L_n}{t}.
\]
Fix a prime $p\le n$.
Since $p\mid L_n$, the congruence class of $a_n$ modulo $p$ is controlled by those $t$ for which $p\mid t$.
Indeed, if $p\nmid t$, then $L_n/t\equiv 0\pmod p$.
Thus
\[
 a_n \equiv \sum_{1\le t\le n\,:\, p\mid t} \frac{L_n}{t}\pmod p.
\]
Write $t=p u$. Then $1\le u\le \lfloor n/p\rfloor$ and
\[
\frac{L_n}{t}=\frac{L_n}{p u} = \Big(\frac{L_n}{p}\Big)\frac{1}{u}.
\]
Reducing modulo $p$, we may factor out $L_n/p$ because $L_n/p$ is an integer.
Hence
\[
 a_n \equiv \Big(\frac{L_n}{p}\Big) \sum_{u=1}^{\lfloor n/p\rfloor}\frac{1}{u}\pmod p.
\]
Now use the base-$p$ representation $n=kp^m+r$ with $0\le r < p^m$.
Then
\[
\left\lfloor \frac{n}{p}\right\rfloor = k p^{m-1} + \left\lfloor \frac{r}{p}\right\rfloor.
\]
Iterating this observation $m$ times shows that the residue of the harmonic sum
$\sum_{u=1}^{\lfloor n/p\rfloor} 1/u$ modulo $p$ agrees with
$\sum_{u=1}^{k} 1/u$ modulo $p$:
indeed, in the finite field $\mathbb{F}_p$, the map $u\mapsto u+p^j$ is a bijection on nonzero residues, and one partitions $\{1,2,\dots,\lfloor n/p\rfloor\}$ into $p^{m-1}$ blocks each congruent to $\{1,2,\dots,k\}$ mod $p$ (plus a tail), yielding
\[
\sum_{u=1}^{\lfloor n/p\rfloor}\frac{1}{u}\equiv \sum_{u=1}^{k}\frac{1}{u}\pmod p.
\]
Therefore
\[
 a_n \equiv \Big(\frac{L_n}{p}\Big) H_k \pmod p.
\]
Since $k<p$, the denominator of $H_k$ is not divisible by $p$, so $H_k$ is well-defined in $\mathbb{F}_p$.
Also, $L_n/p$ is not divisible by $p$ if and only if $p^2\nmid L_n$; but regardless, the congruence shows that $a_n\equiv 0\pmod p$ holds exactly when $H_k\equiv 0\pmod p$ in $\mathbb{F}_p$, i.e. when the numerator of $H_k$ (in lowest terms) is divisible by $p$.

Finally, $p\mid\gcd(a_n,L_n)$ iff $p\mid a_n$ (since always $p\mid L_n$ for $p\le n$), proving the claim.
\end{proof}

\subsubsection*{4.2 Infinitely many $n$ with $\gcd(a_n,L_n)>1$}
\begin{corollary}
There are infinitely many $n$ with $\gcd(a_n,L_n)>1$.
\end{corollary}
\begin{proof}
Take $p=3$.
If $n$ begins with digit $2$ in base $3$, then the leading digit $k$ is $2$.
But
\[
H_2=1+\frac12=\frac{3}{2}
\]
has numerator divisible by $3$.
By Proposition~\ref{prop:criterion}, $3\mid\gcd(a_n,L_n)$.
There are infinitely many such $n$ (e.g. $n=2\cdot 3^m$ for $m\ge 0$).
\end{proof}

\subsubsection*{4.3 What about $\gcd(a_n,L_n)=1$? (data + obstacles)}
Using Proposition~\ref{prop:criterion} as an algorithm (test each prime $p\le n$ via the leading digit of $n$ in base $p$), one can compute small values without manipulating the enormous integer $L_n$.
For $n\le 300$, the values of $n$ with $\gcd(a_n,L_n)=1$ occur in the following consecutive ranges:
\[
[1,5],\ [9,17],\ [27,32],\ [49,53],\ [88,99],\ [125,155],\ [243,271],\ [289,293].
\]
This supports the heuristic that such $n$ exist but are sparse.

However, proving infinitude appears difficult: one must show that for infinitely many $n$, \emph{no} prime $p\le n$ has a ``bad'' leading digit $k$ with $p\mid \mathrm{num}(H_k)$.
Even for $p=3$, avoiding divisibility by $3$ already forces the leading base-$3$ digit of $n$ to be $1$ rather than $2$, i.e.
$n\in[3^m,2\cdot 3^m-1]$.
Other primes impose additional, interacting constraints.

\subsection*{5) Verification}
\begin{itemize}
\item Proposition~\ref{prop:criterion} is the key claim; its proof reduces $a_n$ mod $p$ to a harmonic sum determined by the leading base-$p$ digit. This matches the criterion stated in the Erd\H{o}s Problems discussion.
\item The ``infinitely many $n$ with gcd$>1$'' corollary is rigorous and uses only $p=3$ and $H_2=3/2$.
\item The numerical ranges for gcd$=1$ up to $300$ were computed using the criterion and modular inverses (no floating-point).
\end{itemize}

\subsection*{6) Final}
\textbf{UNRESOLVED.}
\begin{enumerate}[label=(\roman*)]
\item \textbf{Farthest point reached:} proved the prime-divisibility criterion (Proposition~\ref{prop:criterion}) and gave a complete proof that $\gcd(a_n,L_n)>1$ occurs infinitely often.
\item \textbf{Best partial results:} the criterion reduces the conjecture ``$\gcd(a_n,L_n)=1$ infinitely often'' to an explicit avoidance problem over primes; computation up to $n\le 300$ exhibits many gcd$=1$ values.
\item \textbf{Most plausible next moves:} find a constructive infinite family of $n$ for which all leading-digit tests avoid ``bad'' digits, possibly by restricting $n$ to carefully chosen intervals (dyadic/triadic) and using distribution results on primes.
\item \textbf{What I would try next with more time:} attempt to prove that for infinitely many $m$ there exists an $n$ in $[3^m,2\cdot 3^m)$ that simultaneously avoids the finite set of ``bad digit'' constraints coming from primes $p\le 2\cdot 3^m$, perhaps via a probabilistic method plus a union bound strengthened by structural sparsity of bad digits.
\end{enumerate}

\subsection*{7) Completion Estimate}
\textbf{Estimate:} 35\%.
(The ``$>1$ infinitely often'' half is proved; the difficult half is the infinitude of gcd$=1$.)

