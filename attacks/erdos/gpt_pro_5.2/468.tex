
\noindent\textbf{1) FORMAL RESTATEMENT.}

For $n\ge 2$, list the positive divisors of $n$ that are $>1$ in increasing order:
\[
1<d_1<d_2<\cdots<d_t=n.
\]
Define
\[
D_n:=\{d_1,\ d_1+d_2,\ \dots,\ d_1+\cdots+d_t\}.
\]
Question 1: determine/estimate
\[
\bigl|D_n\setminus \bigcup_{m<n} D_m\bigr|.
\]
Question 2: for an integer $N$, define
\[
f(N):=\min\{n\ge 2: N\in D_n\}
\]
(if such $n$ exists; otherwise interpret $f(N)=\infty$). Is it true that $f(N)=o(N)$ as $N\to\infty$ (perhaps for almost all $N$)?

\medskip
\noindent\textbf{2) QUICK LITERATURE/CONTEXT CHECK.}

No external results are cited in the problem statement beyond the definitions. I will rely only on direct arguments and explicit computations.

\medskip
\noindent\textbf{3) ATTACK PLAN.}

\begin{itemize}
\item Establish basic identities: $|D_n|$ and the maximal element of $D_n$.
\item Find explicit obstructions (values $N$ that never lie in any $D_n$) and explicit constructions (families of $N$ that do).
\item Compute $D_n$ and $f(N)$ for small ranges to see patterns and plausible conjectures.
\end{itemize}

\medskip
\noindent\textbf{4) WORK.}

\noindent\textbf{Lemma 468.1 (size and endpoint).}
For $n\ge 2$ with $t=\tau(n)-1$ divisors $>1$, the set $D_n$ has size
\[
|D_n|=t=\tau(n)-1,
\]
and its largest element is
\[
\max D_n = d_1+\cdots+d_t = \bigl(\sigma(n)-1\bigr),
\]
where $\tau(n)$ is the number of positive divisors of $n$ and $\sigma(n)$ is the sum of positive divisors of $n$.

\textit{Proof.}
Each partial sum $s_j:=d_1+\cdots+d_j$ is strictly increasing in $j$ because the $d_i$ are positive, hence the $t$ partial sums are all distinct and $|D_n|=t$. The final partial sum is the sum of all divisors of $n$ except $1$, i.e. $\sigma(n)-1$. \hfill$\square$

\medskip
\noindent\textbf{Lemma 468.2 (the integer $4$ never occurs).}
One has $4\notin \bigcup_{n\ge 2} D_n$. Equivalently, $f(4)=\infty$.

\textit{Proof.}
Suppose for contradiction that $4\in D_n$ for some $n\ge 2$. The first divisor $d_1$ satisfies $d_1\ge 2$. If $d_1\ge 3$, then $d_1\ne 4$ (since $d_1$ must divide $n$ and any divisor $\ge 4$ forces $2$ to divide $n$, contradicting minimality of $d_1$), and also $d_1>4$ is impossible. Thus necessarily $d_1=2$. But then the next divisor $d_2$ satisfies $d_2\ge 3$, so the second partial sum is $d_1+d_2\ge 2+3=5$, and all further partial sums are even larger. Therefore no partial sum equals $4$, contradiction. \hfill$\square$

\medskip
\noindent\textbf{Lemma 468.3 (two-prime construction).}
Let $p<q$ be primes and set $n=pq$. Then
\[
D_{pq}=\{p,\ p+q,\ p+q+pq\}.
\]
In particular, any integer of the form $N=p+q$ with primes $p<q$ lies in some $D_n$ (namely $n=pq$).

\textit{Proof.}
The divisors of $pq$ are $1,p,q,pq$, hence the divisors $>1$ are $p<q<pq$. The partial sums are exactly $p$, $p+q$, and $p+q+pq$. \hfill$\square$

\medskip
\noindent\textbf{FAST REALITY CHECK (computations).}

I computed $D_n$ for $n\le 50$ and tracked the number of \emph{new} elements
\[\nu(n):=|D_n\setminus \bigcup_{m<n} D_m|.\]
The output (format $(n,\nu(n),|D_n|,\max D_n)$) was:
\begin{verbatim}
[(2, 1, 1, 2), (3, 1, 1, 3), (4, 1, 2, 6), (5, 1, 1, 5),
 (6, 1, 3, 11), (7, 1, 1, 7), (8, 1, 3, 14), (9, 1, 2, 12),
 (10, 1, 3, 17), (11, 0, 1, 11), (12, 3, 5, 27), (13, 1, 1, 13),
 (14, 1, 3, 23), (15, 1, 3, 23), (16, 1, 4, 30), (17, 0, 1, 17),
 (18, 2, 5, 38), (19, 1, 1, 19), (20, 2, 5, 41), (21, 2, 3, 31),
 (22, 1, 3, 35), (23, 0, 1, 23), (24, 1, 7, 59), (25, 0, 2, 30),
 (26, 0, 3, 41), (27, 1, 3, 39), (28, 1, 5, 55), (29, 1, 1, 29),
 (30, 3, 7, 71), (31, 0, 1, 31), (32, 1, 5, 62), (33, 1, 3, 47),
 (34, 1, 3, 53), (35, 0, 3, 47), (36, 4, 8, 90), (37, 1, 1, 37),
 (38, 0, 3, 59), (39, 0, 3, 55), (40, 2, 7, 89), (41, 0, 1, 41),
 (42, 3, 7, 95), (43, 1, 1, 43), (44, 1, 5, 83), (45, 1, 5, 77),
 (46, 1, 3, 71), (47, 0, 1, 47), (48, 3, 9, 123), (49, 1, 2, 56),
 (50, 2, 5, 92)]
\end{verbatim}
So $\nu(n)$ can be $0$ (e.g. $n=11,17,23$), and can be larger (e.g. $\nu(36)=4$ for $n=36$).

For the inverse function $f(N)$, I brute-forced $D_n$ for $n\le 20000$ and recorded the first occurrence of each $N\le 1000$. The only missing value in $[2,1000]$ was $N=4$, consistent with Lemma~468.2.

Some summary statistics from that computation:
\begin{verbatim}
missing [4]
max f(N) among N<=1000: (981, 5072)
max ratio f(N)/N among N<=1000: 5.895238095238096 achieved at (630, 3714)
\end{verbatim}
and the ten largest ratios $f(N)/N$ for $N\le 1000$ were
\begin{verbatim}
5.895 630 3714
5.334 628 3350
5.170 981 5072
5.075 796 4040
4.973 934 4645
4.973 916 4555
4.966 738 3665
4.966 732 3635
4.965 724 3595
4.955 670 3320
\end{verbatim}
(format: ratio, $N$, $f(N)$).

\medskip
\noindent\textbf{5) VERIFICATION.}

\begin{itemize}
\item Lemma~468.2 depends only on the ordering of divisors $>1$ and the fact that if $n$ is even then $2$ is the smallest divisor $>1$.
\item Lemma~468.3 is an explicit divisor computation for semiprimes.
\item Computations were done by exact divisor enumeration and exact set difference/union tracking.
\end{itemize}

\medskip
\noindent\textbf{6) FINAL.} \textbf{UNRESOLVED}

(i) \textbf{Strongest proved partial result.}
We have exact structural identities $|D_n|=\tau(n)-1$ and $\max D_n=\sigma(n)-1$ (Lemma~468.1), and a proven obstruction $4\notin \bigcup_n D_n$ (Lemma~468.2). We also have a simple semiprime construction that forces a prescribed partial sum $p+q$ (Lemma~468.3). Computation up to $N\le 1000$ (searching $n\le 20000$) found that every $N\in[2,1000]\setminus\{4\}$ occurs in some $D_n$.

(ii) \textbf{First gap (crisp).}
Prove (or refute) that $f(N)=o(N)$ as $N\to\infty$ (perhaps for almost all $N$), even under the caveat that $f(4)=\infty$. A weaker crisp target is to prove an explicit bound $f(N)\le N^{1-\varepsilon}$ for some fixed $\varepsilon>0$ and all sufficiently large admissible $N$.

(iii) \textbf{Top 3 next moves.}
\begin{enumerate}
\item Constructive approach: given $N$, build an $n$ whose initial divisors $d_1<d_2<\cdots$ have a controlled pattern so that some initial partial sum equals $N$; then optimize this construction to make $n$ as small as possible.
\item Study ``coverage'' of $\bigcup_{n\le X} D_n$: estimate how large a set of integers can be produced as partial sums up to height $X$, and whether the complement is small.
\item Computation: push the search range (both in $n$ and in $N$) and look for $N$ with unusually large ratios $f(N)/N$; attempt to guess the true growth of $\sup_{N\le X} f(N)/N$.
\end{enumerate}

(iv) \textbf{Minimal counterexample structure.}
A counterexample to $f(N)=o(N)$ would be a sequence $N_j\to\infty$ such that any $n$ with $N_j\in D_n$ must satisfy $n\ge cN_j$ for some fixed $c>0$ (or even $n\ge N_j/\phi(N_j)$ with $\phi(N_j)\to\infty$ failing). Structurally this would mean: whenever the increasing list of divisors of $n$ is summed from the bottom, the partial sums avoid $N_j$ unless $n$ is large.


