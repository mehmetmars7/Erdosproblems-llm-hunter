
\noindent\textbf{FORMAL RESTATEMENT.}
Fix an integer $k\ge 3$.
Let $G_k$ be the bipartite graph with parts
\[
Y:=\{y_1,\dots,y_k\},\qquad Z:=\{z_{ij}: 1\le i<j\le k\},\quad |Z|=\binom{k}{2},
\]
where each $z_{ij}$ is adjacent to $y_i$ and $y_j$ and has no other neighbors.
Let $\mathrm{ex}(n,G_k)$ be the Tur\'an number: the maximum number of edges in an $n$-vertex graph containing no subgraph isomorphic to $G_k$.
The question asks whether for every $k\ge 3$ there exists $c_k>0$ such that
\[
\mathrm{ex}(n,G_k) \ll n^{3/2-c_k}
\qquad\text{as } n\to\infty.
\]

\medskip
\noindent\textbf{QUICK LITERATURE/CONTEXT CHECK.}
The problem statement notes that for $k=3$ one has $G_3=C_6$ and that a strong bound $\mathrm{ex}(n,C_6)\ll n^{7/6}$ is known (per the extracted text). It also says it is not even known in general whether $\mathrm{ex}(n,G_k)=o(n^{3/2})$.

\medskip
\noindent\textbf{ATTACK PLAN.}
\emph{Upper-bound track:} Use that $G_k$ is a subgraph of the complete bipartite graph $K_{k,\binom{k}{2}}$, so any $G_k$-free graph is also $K_{k,\binom{k}{2}}$-free; then apply a K\H{o}v\'ari--S\'os--Tur\'an type counting argument.
\emph{Sharpening track (for small $k$):} For $k=3$, prove a (weaker than the best-known) power saving for $C_6$ via counting length-3 paths.

\medskip
\noindent\textbf{WORK.}

\noindent\textbf{Lemma 1 (K\H{o}v\'ari--S\'os--Tur\'an bound for $K_{s,t}$, with proof).}
Let $s\ge 2$ and $t\ge 2$ be integers.
If $G$ is a bipartite graph with parts $(A,B)$, $|A|=m$, $|B|=n$, and $G$ contains no copy of $K_{s,t}$ with the $s$ vertices in $A$ and the $t$ vertices in $B$, then
\[
|E(G)| \le (t-1)^{1/s}\, n\, m^{1-1/s} + (s-1)n.
\]

\textit{Proof.}
For each vertex $b\in B$, let $d(b)$ be its degree into $A$.
Count the number of pairs $(S,b)$ where $S\subseteq N(b)$ and $|S|=s$.
On one hand,
\[
\sum_{b\in B} \binom{d(b)}{s}
\]
counts these pairs.
On the other hand, for a fixed $s$-subset $S\subseteq A$, the number of $b\in B$ adjacent to all vertices of $S$ is the common neighborhood size $|N(S)|$.
The $K_{s,t}$-free assumption implies $|N(S)|\le t-1$ for every $S$.
Therefore
\[
\sum_{b\in B} \binom{d(b)}{s} = \sum_{S\in\binom{A}{s}} |N(S)| \le (t-1)\binom{m}{s}.
\]
Now use the convexity bound: for $x\ge s-1$, one has
\[
\binom{x}{s} = \frac{x(x-1)\cdots(x-s+1)}{s!} \ge \frac{(x-s+1)^s}{s!}.
\]
Applying this with $x=d(b)$ gives
\[
\sum_{b\in B}\frac{(d(b)-s+1)^s}{s!} \le \sum_{b\in B} \binom{d(b)}{s} \le (t-1)\binom{m}{s} \le (t-1)\frac{m^s}{s!}.
\]
Multiply by $s!$ and apply that $\sum (d(b)-s+1)^s \ge n\bigl(\frac{\sum(d(b)-s+1)}{n}\bigr)^s$ by convexity of $u\mapsto u^s$ on $\mathbb{R}_{\ge 0}$:
\[
\Bigl(\sum_{b\in B}(d(b)-s+1)\Bigr)^s \le n^{s-1}\sum_{b\in B}(d(b)-s+1)^s \le n^{s-1}(t-1)m^s.
\]
Taking $s$th roots yields
\[
\sum_{b\in B}(d(b)-s+1) \le (t-1)^{1/s} n^{1-1/s} m.
\]
Since $\sum_{b\in B} d(b)=|E(G)|$, this becomes
\[
|E(G)| - (s-1)n \le (t-1)^{1/s} n^{1-1/s} m.
\]
Rewriting gives the claimed inequality.
\hfill$\square$

\medskip
\noindent\textbf{Corollary 1 (a general but weak bound for $\mathrm{ex}(n,G_k)$).}
For fixed $k\ge 3$,
\[
\mathrm{ex}(n,G_k) \ll n^{2-1/k}.
\]

\textit{Proof.}
Because $G_k$ is a subgraph of $K_{k,\binom{k}{2}}$, any graph that contains $K_{k,\binom{k}{2}}$ also contains $G_k$.
Hence every $G_k$-free graph is $K_{k,\binom{k}{2}}$-free, so
$\mathrm{ex}(n,G_k)\le \mathrm{ex}(n,K_{k,\binom{k}{2}})$.
It is standard (and easy to check) that extremal graphs for $K_{k,t}$ can be taken bipartite: deleting edges inside the parts cannot create a $K_{k,t}$.
Applying Lemma 1 with $s=k$ and $t=\binom{k}{2}$ in the balanced case $m\asymp n\asymp n/2$ gives $\mathrm{ex}(n,K_{k,t})=O(n^{2-1/k})$, hence the claim.
\hfill$\square$

\medskip
\noindent\textbf{Lemma 2 (a complete but weak bound for $C_6=G_3$).}
Every $C_6$-free graph is $K_{3,3}$-free. Consequently,
\[
\mathrm{ex}(n,C_6)\le \mathrm{ex}(n,K_{3,3}) \ll n^{5/3}.
\]

\textit{Proof.}
First, any copy of $K_{3,3}$ contains a $6$-cycle: if the bipartition is
$\{a_1,a_2,a_3\}\cup\{b_1,b_2,b_3\}$ with all cross-edges present, then
\[
a_1-b_1-a_2-b_2-a_3-b_3-a_1
\]
traverses distinct vertices and is a (simple) cycle of length $6$.
Therefore, if a graph contains $K_{3,3}$ then it contains $C_6$, so a $C_6$-free graph is necessarily $K_{3,3}$-free. This gives
$\mathrm{ex}(n,C_6)\le \mathrm{ex}(n,K_{3,3})$.

To bound $\mathrm{ex}(n,K_{3,3})$, it is enough to consider bipartite graphs: given any $n$-vertex graph, take a maximum cut and keep only the crossing edges; this produces a bipartite subgraph with at least half the edges, and deleting edges cannot create a forbidden $K_{3,3}$.
Now apply Lemma~1 with $s=t=3$ to a bipartite graph with $m\asymp n\asymp n/2$ vertices on each side. Lemma~1 yields
\[
|E(G)| \le 2^{1/3}\,|B|\,|A|^{2/3} + 2|B| = O((|A|+|B|)^{5/3}),
\]
which implies $\mathrm{ex}(n,K_{3,3})\ll n^{5/3}$ and hence the displayed bound for $\mathrm{ex}(n,C_6)$.
\hfill$\square$

\medskip
\noindent\textbf{FAST REALITY CHECK.}
Exact enumeration for $n\le 7$ gives:
\[
\mathrm{ex}(3,C_6)=3,\ \mathrm{ex}(4,C_6)=6,\ \mathrm{ex}(5,C_6)=10,\ \mathrm{ex}(6,C_6)=11,\ \mathrm{ex}(7,C_6)=13.
\]

\medskip
\noindent\textbf{VERIFICATION.}
Lemma 1 and Corollary 1 are complete.
Lemma 2 is complete.

\medskip
\noindent\textbf{FINAL.}
\textbf{UNRESOLVED}

(i) Strongest proved partial result: a complete proof of the K\H{o}v\'ari--S\'os--Tur\'an bound (Lemma 1), the resulting general estimate $\mathrm{ex}(n,G_k)=O(n^{2-1/k})$, and the elementary reduction $\mathrm{ex}(n,C_6)\le\mathrm{ex}(n,K_{3,3})=O(n^{5/3})$ (Lemma 2).

(ii) First gap (crisp): prove any exponent strictly below $3/2$ for $\mathrm{ex}(n,G_k)$ (even $\mathrm{ex}(n,G_k)=o(n^{3/2})$) for a fixed $k\ge 4$. (For $k=3$ the extracted problem text states a much stronger bound $\mathrm{ex}(n,C_6)\ll n^{7/6}$, but I have not reproduced that proof here.)

(iii) Top 3 next moves:
1. For $k=3$, complete a clean $O(n^{4/3})$ proof by a careful count of 3-paths and an application of Cauchy--Schwarz to codegrees (using both bipartition sides).
2. For general $k$, reinterpret $G_k$-containment in terms of selecting $k$ vertices with many pairwise-distinct common neighbors, and attempt to bound such configurations via codegree energy.
3. Compute $\mathrm{ex}(n,G_4)$ for small $n$ by brute force / SAT (very small $n$) to detect plausible growth rates and extremal constructions.

(iv) Minimal counterexample structure: if the conjectured power saving fails for some fixed $k$, there should exist $G_k$-free graphs with $\Omega(n^{3/2})$ edges. Such graphs would have average degree $\Omega(n^{1/2})$ while avoiding a configuration where $k$ vertices have pairwise codegrees that can be chosen disjointly across all pairs.
