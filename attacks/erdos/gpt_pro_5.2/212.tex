\section*{Erd\H{o}s Problem \#212}

\subsection*{1. Formal Restatement}
Let $d(\cdot,\cdot)$ denote the Euclidean distance on $\mathbb{R}^2$,
\[d((x_1,y_1),(x_2,y_2)) := \sqrt{(x_1-x_2)^2+(y_1-y_2)^2}.\]
A set $S\subseteq\mathbb{R}^2$ is a \emph{rational distance set} if
\[\forall p\neq q\in S,\quad d(p,q)\in\mathbb{Q}.\]
A set $S\subseteq\mathbb{R}^2$ is \emph{dense} if its closure is all of $\mathbb{R}^2$, equivalently
\[\forall x\in\mathbb{R}^2\ \forall \varepsilon>0\ \exists s\in S\text{ with } d(x,s)<\varepsilon.\]

\noindent\textbf{Problem \#212 (existence form).}
Does there exist a dense set $S\subseteq\mathbb{R}^2$ such that $S$ is a rational distance set?

\subsection*{2. Quick Literature/Context Check}
The prompt itself indicates that this is the classic \emph{Erd\H{o}s--Ulam problem}: whether a dense rational distance set exists in the plane.
It also indicates conditional negative results (e.g. assuming Bombieri--Lang), and unconditional structural results for rational distance sets lying on algebraic curves.
As stated in the prompt, the problem is widely believed to be open.

\subsection*{3. Attack Plan}
\textbf{Proof approach (show existence).}
Try to build a countable dense set $S$ by an explicit inductive construction:
\begin{itemize}
  \item Start with a finite rational-distance configuration $S_0$.
  \item At stage $k$, add a new point $p_k$ very close to a prescribed target point in a countable dense subset of $\mathbb{R}^2$ while forcing all new distances to land in $\mathbb{Q}$.
  \item The key subproblem: given an existing finite rational-distance set $F$ and a target location $x$, find a point $p$ arbitrarily close to $x$ such that $d(p,f)\in\mathbb{Q}$ for every $f\in F$.
\end{itemize}
This ``local extension'' step is the main obstacle; it becomes a Diophantine approximation problem on intersections of many circles.

\textbf{Disproof approach (show nonexistence).}
Try to show that any rational distance set $S$ must be contained in a ``thin'' subset of the plane (e.g. a countable union of curves of empty interior), or must fail to intersect some open ball.
A natural route is:
\begin{itemize}
  \item Prove strong structural constraints on large finite rational-distance sets.
  \item Upgrade these constraints to infinite sets via compactness/limiting arguments.
\end{itemize}
The prompt suggests that deep Diophantine geometry methods might be relevant here.

\subsection*{4. Work}
I cannot resolve existence/nonexistence in $\mathbb{R}^2$, but I can fully prove some basic structural facts and give explicit dense rational-distance sets on 1-dimensional subsets (lines/circles). These serve as sanity checks and clarify what is and is not ruled out by elementary arguments.

\medskip
\noindent\textbf{Lemma 4.1 (Countability).}
\emph{Every rational distance set $S\subseteq\mathbb{R}^2$ is at most countable.}

\smallskip
\noindent\emph{Proof.}
If $|S|\le 1$ there is nothing to prove. Otherwise pick distinct points $A,B\in S$.
For each $P\in S$ define
\[\Phi(P):=\bigl(d(P,A),\ d(P,B)\bigr)\in \mathbb{Q}_{\ge 0}^2.
\]
Fix any pair $(r,s)\in\mathbb{Q}_{\ge 0}^2$. The fiber
\[\Phi^{-1}(r,s)=\{P\in\mathbb{R}^2: d(P,A)=r\text{ and } d(P,B)=s\}
\]
is the intersection of the two circles
\[C_A(r):=\{P: d(P,A)=r\},\qquad C_B(s):=\{P: d(P,B)=s\}.
\]
Because $A\neq B$, the circles $C_A(r)$ and $C_B(s)$ cannot coincide (distinct centers), and two distinct circles in the Euclidean plane intersect in at most two points. Hence $|\Phi^{-1}(r,s)|\le 2$ for every $(r,s)$.
Since $\mathbb{Q}_{\ge 0}^2$ is countable and $S=\bigcup_{(r,s)\in\Phi(S)}\Phi^{-1}(r,s)$, it follows that $S$ is a countable union of finite sets, hence countable.
\hfill$\square$

\medskip
\noindent\textbf{Remark.}
Countability alone does \emph{not} rule out density, since there exist countable dense subsets of $\mathbb{R}^2$ (e.g. $\mathbb{Q}^2$). So Lemma 4.1 does not resolve \#212.

\medskip
\noindent\textbf{Proposition 4.2 (Dense rational-distance set on a line).}
The set $L:=\mathbb{Q}\times\{0\}\subset\mathbb{R}^2$ is dense in the $x$-axis and satisfies $d(p,q)\in\mathbb{Q}$ for all $p,q\in L$.

\smallskip
\noindent\emph{Proof.}
Density in the $x$-axis is standard: for any real $x$ and $\varepsilon>0$ there exists $q\in\mathbb{Q}$ with $|q-x|<\varepsilon$.
For $p=(q_1,0), q=(q_2,0)$ we have $d(p,q)=|q_1-q_2|\in\mathbb{Q}$.
\hfill$\square$

\medskip
\noindent\textbf{Proposition 4.3 (Dense rational-distance set on the unit circle).}
There exists a countable set $S\subseteq\{(x,y):x^2+y^2=1\}$ that is dense on the unit circle and satisfies $d(p,q)\in\mathbb{Q}$ for all distinct $p,q\in S$.

\smallskip
\noindent\emph{Construction.}
For $t\in\mathbb{Q}$ define
\[a(t):=\frac{1-t^2}{1+t^2},\qquad b(t):=\frac{2t}{1+t^2}.
\]
Note that $a(t),b(t)\in\mathbb{Q}$ and
\[a(t)^2+b(t)^2=\frac{(1-t^2)^2+(2t)^2}{(1+t^2)^2}=\frac{(1+t^2)^2}{(1+t^2)^2}=1.
\]
Now define
\[p(t):=\bigl(a(t)^2-b(t)^2,\ 2a(t)b(t)\bigr)\in\mathbb{R}^2.
\]
Let $S:=\{p(t): t\in\mathbb{Q}\}$.

\smallskip
\noindent\emph{Proof that $S$ lies on the unit circle.}
Let $a=a(t)$ and $b=b(t)$.
Then
\[(a^2-b^2)^2+(2ab)^2=(a^2+b^2)^2=1,
\]
so $p(t)$ lies on $x^2+y^2=1$.

\smallskip
\noindent\emph{Proof that pairwise distances are rational.}
Take $t,u\in\mathbb{Q}$ and write $a_t=a(t)$, $b_t=b(t)$, $a_u=a(u)$, $b_u=b(u)$.
Consider the angles $\theta,\varphi$ with
\[\cos(\theta/2)=a_t,\ \sin(\theta/2)=b_t,\qquad \cos(\varphi/2)=a_u,\ \sin(\varphi/2)=b_u.
\]
Then $p(t)=(\cos\theta,\sin\theta)$ and $p(u)=(\cos\varphi,\sin\varphi)$, and the chord length formula gives
\[d\bigl(p(t),p(u)\bigr)=2\left|\sin\Bigl(\frac{\theta-\varphi}{2}\Bigr)\right|.
\]
Using the sine difference identity,
\[\sin\Bigl(\frac{\theta-\varphi}{2}\Bigr)=\sin(\theta/2)\cos(\varphi/2)-\cos(\theta/2)\sin(\varphi/2)=b_t a_u-a_t b_u\in\mathbb{Q},
\]
so $d(p(t),p(u))\in\mathbb{Q}$.

\smallskip
\noindent\emph{Proof that $S$ is dense on the unit circle.}
The map $t\mapsto (a(t),b(t))$ parametrizes the unit circle minus the point $(-1,0)$, and it is continuous as a function of real $t$.
Since $\mathbb{Q}$ is dense in $\mathbb{R}$ and the parametrization is continuous, the image $\{(a(t),b(t)):t\in\mathbb{Q}\}$ is dense in the circle.
The map $(a,b)\mapsto (a^2-b^2,2ab)$ is continuous and is exactly the angle-doubling map on the circle; the image of a dense subset under a continuous map is dense.
Hence $S=\{p(t):t\in\mathbb{Q}\}$ is dense on the unit circle.
\hfill$\square$

\medskip
\noindent\textbf{Sticking point for \#212.}
The constructions above are dense on 1-dimensional sets (a line or a circle). Extending these to a set dense in all of $\mathbb{R}^2$ requires controlling rationality of distances \emph{between} points lying near many different directions/locations simultaneously. Lemma 4.1 shows we must build a countable dense set, but gives no mechanism to enforce the rationality constraints across the entire set.

\subsection*{5. Verification}
\begin{itemize}
  \item Lemma 4.1: the only geometric input is that two distinct circles intersect in $\le 2$ points; this is standard and covers all cases because distinct centers prevent coincident circles.
  \item Proposition 4.3: all algebraic identities were explicitly expanded; the distance computation reduces to a rational expression via a trigonometric identity in which the relevant sine/cosine values are rational by construction.
  \item Density: used only continuity and density of $\mathbb{Q}$ in $\mathbb{R}$.
\end{itemize}

\subsection*{6. FINAL}
\textbf{UNRESOLVED.}
\begin{enumerate}
  \item[(i)] \textbf{Strongest fully proved partial result obtained here:}
  \begin{itemize}
    \item Every rational distance set in $\mathbb{R}^2$ is at most countable (Lemma 4.1).
    \item There exist countable dense rational-distance sets on a line (Proposition 4.2) and on the unit circle (Proposition 4.3).
  \end{itemize}
  \item[(ii)] \textbf{First gap blocking completion:}
  There is no known method (in this write-up) to perform the ``local extension step'': given a finite rational-distance set $F\subset\mathbb{R}^2$ and a target point $x$, construct a point $p$ arbitrarily close to $x$ with $d(p,f)\in\mathbb{Q}$ for all $f\in F$. Without such a step, one cannot inductively build a dense planar set.
  \item[(iii)] \textbf{Most promising next moves:}
  \begin{itemize}
    \item Develop an extension lemma for finite rational-distance sets using arithmetic geometry of the configuration space cut out by multiple circle equations.
    \item Prove an unconditional ``finite union of curves'' (or other thin-structure) theorem for infinite rational distance sets, strong enough to contradict density.
    \item Explore constructions based on group laws on conics/elliptic curves that might allow adding points while controlling many distances at once.
  \end{itemize}
  \item[(iv)] \textbf{Conjectured final answer and minimal counterexample shape:}
  I conjecture the answer is \textbf{NO} (no dense rational-distance set in $\mathbb{R}^2$).
  A hypothetical minimal counterexample would necessarily be a \emph{countable} dense set with highly rigid algebraic structure, likely organized by an underlying algebraic group action that forces all interpoint distances into $\mathbb{Q}$.
\end{enumerate}

\subsection*{7. Completion Estimate}
\textbf{15\%}. The work here establishes basic necessary structure (countability) and gives dense examples on 1-dimensional sets, but it does not substantially bridge the core difficulty of enforcing rationality of \emph{all} distances in a set dense in \emph{two} dimensions.


% =====================================================================

