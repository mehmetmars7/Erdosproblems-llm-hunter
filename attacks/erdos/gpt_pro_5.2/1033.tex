% Erdos Problem #1033
% URL: https://www.erdosproblems.com/1033

Let $h(n)$ be such that every graph on $n$ vertices with $>n^2/4$ many edges contains a triangle whose vertices have degrees summing to at least $h(n)$. Estimate $h(n)$. In particular, is it true that\[h(n)\geq (2(\sqrt{3}-1)-o(1))n?\] Erd\H{o}s and Laskar \cite{ErLa85} proved\[2(\sqrt{3}-1)n \geq h(n) \geq (1+c)n\]for some $c>0$. The lower bound was improved to $\frac{21}{16}n$ by Fan \cite{Fa88}. References [ErLa85] Erd\H{o}s, Paul and Laskar, Renu, A note on the size of a chordal subgraph . Congr. Numer. (1985), 81--86. [Fa88] Fan, Genghua, Degree sum for a triangle in a graph . J. Graph Theory (1988), 249--263.

%Erdos problem #1033
\subsection*{Erd\H{o}s Problem \#1033 --- Solution Attempt}

\textbf{FORMAL RESTATEMENT.}
For each integer $n\ge 3$, define $h(n)$ to be the largest integer with the property:
for every simple graph $G$ on $n$ vertices with $e(G) > n^2/4$, there exists a triangle $abc$ in $G$ such that
\[
\deg(a)+\deg(b)+\deg(c) \ge h(n).
\]
The problem asks for asymptotic estimates for $h(n)$, and in particular whether
\[
 h(n) \ge (2(\sqrt{3}-1)-o(1))n
\]
as $n\to\infty$.

\textbf{QUICK LITERATURE/CONTEXT CHECK.}
The problem statement itself records the bounds
\[
(1+c)n \le h(n) \le 2(\sqrt{3}-1)n
\]
for some absolute $c>0$, and that the lower bound was improved to $(21/16)n$.
I do not use external results beyond what is explicitly in the problem file.

\textbf{ATTACK PLAN.}
\emph{Proof track:} derive general lower bounds on the maximum triangle degree-sum in a graph with $>n^2/4$ edges.
Start from averaging arguments on degrees and edges, then force a triangle via common-neighbour counting.

\emph{Disproof/upper-bound track:} construct graphs with just over $n^2/4$ edges in which every triangle has small degree-sum.
A natural idea: take a complete bipartite graph $K_{a,b}$ (triangle-free, $ab$ edges) with $ab$ close to $n^2/4$, and then add a sparse graph inside the smaller part to create triangles while controlling degrees.

\textbf{WORK.}

\textbf{Lemma 1 (Edge-sum averaging forces a ``large'' edge, hence a ``large'' triangle).}
Let $G$ be a graph on $n$ vertices with $m=e(G)$ edges.
Then there exists an edge $uv\in E(G)$ with
\[
\deg(u)+\deg(v) \ge \frac{4m}{n}.
\]
If moreover $m>n^2/4$, then $\deg(u)+\deg(v) > n$ and hence $u,v$ have a common neighbour $w$, so $uvw$ is a triangle with
\[
\deg(u)+\deg(v)+\deg(w) \ge \frac{4m}{n} + 2.
\]

\emph{Proof.}
For each edge $uv$, define its endpoint-degree-sum $s(uv):=\deg(u)+\deg(v)$.
Sum over all edges:
\[
\sum_{uv\in E(G)} s(uv) = \sum_{uv\in E(G)} (\deg(u)+\deg(v)).
\]
Each vertex $x$ contributes $\deg(x)$ to $s(uv)$ for each incident edge $uv$, hence contributes $\deg(x)^2$ in total.
Therefore
\[
\sum_{uv\in E(G)} s(uv) = \sum_{x\in V(G)} \deg(x)^2.
\]
By Cauchy--Schwarz,
\[
\sum_{x\in V} \deg(x)^2 \ge \frac{\left(\sum_x \deg(x)\right)^2}{n} = \frac{(2m)^2}{n} = \frac{4m^2}{n}.
\]
Divide by $m$ to get the average value of $s(uv)$ over edges:
\[
\frac{1}{m}\sum_{uv\in E} s(uv) \ge \frac{4m}{n}.
\]
Hence some edge $uv$ satisfies $s(uv)\ge 4m/n$.

If $m>n^2/4$, then $4m/n>n$.
Now $|N(u)\cup N(v)|\le n-2$, while $|N(u)|+|N(v)|=\deg(u)+\deg(v)>n$.
By inclusion--exclusion,
\[
|N(u)\cap N(v)| = |N(u)|+|N(v)|-|N(u)\cup N(v)| \ge (\deg(u)+\deg(v))-(n-2) > 2,
\]
so in particular $N(u)\cap N(v)\neq\emptyset$.
Pick $w\in N(u)\cap N(v)$; then $uvw$ is a triangle.
Finally, since $w$ is adjacent to both $u$ and $v$, we have $\deg(w)\ge 2$, giving
\[
\deg(u)+\deg(v)+\deg(w) \ge \frac{4m}{n} + 2.
\]
\qed

\medskip
\textbf{Consequence.} Since $m>n^2/4$ implies $4m/n\ge n+\frac{4}{n}$, Lemma 1 yields the universal bound
\[
 h(n) \ge n+2 \qquad (n\ge 3).
\]
(This is very weak compared to the conjectured constant $\approx 1.464\,n$.)

\medskip
\textbf{Lemma 2 (An explicit construction giving the upper bound constant $2(\sqrt{3}-1)$ up to $o(n)$).}
For infinitely many $n$ (and in fact for all sufficiently large $n$ after rounding), there exists a graph $G$ on $n$ vertices with $e(G)>n^2/4$ such that every triangle $T$ in $G$ satisfies
\[
\sum_{v\in T} \deg(v) \le \left(2(\sqrt{3}-1)+o(1)\right)n.
\]
In particular, this shows
\[
 h(n) \le 2(\sqrt{3}-1)n + o(n).
\]

\emph{Construction and proof.}
Fix $n$ and choose an integer $a$ close to $a\approx n/\sqrt{3}$, and let $b:=n-a$.
Let $A$ and $B$ be disjoint vertex sets with $|A|=a$ and $|B|=b$.
Start with the complete bipartite graph between $A$ and $B$, i.e. add all $ab$ edges between $A$ and $B$.
This base graph is triangle-free.
Now add a graph $H$ on the vertex set $A$ only, with maximum degree at most $\Delta$ (to be chosen), and no edges inside $B$.
Let $G$ be the resulting graph.

\emph{Triangle structure.}
Any triangle in $G$ must use exactly two vertices from $A$ and one from $B$, because:
there are no edges inside $B$, and any edge inside $A$ comes from $H$.
Thus every triangle consists of an $H$-edge $xy$ in $A$ together with any vertex $z\in B$ (since all $A$--$B$ edges are present).

\emph{Degree control.}
Every vertex in $B$ has degree exactly $a$.
Every vertex $x\in A$ has degree $\deg_G(x)=b+\deg_H(x)\le b+\Delta$.
Therefore any triangle $xyz$ with $x,y\in A$ and $z\in B$ satisfies
\[
\deg_G(x)+\deg_G(y)+\deg_G(z) \le (b+\Delta)+(b+\Delta)+a = n + b + 2\Delta. \tag{5}
\]

\emph{Edge count.}
The number of edges is
\[
 e(G)=ab + e(H).
\]
If $H$ has maximum degree at most $\Delta$, then $e(H)\le a\Delta/2$.
Conversely, for even $\Delta$ and $a$ large, there exists an explicit $\Delta$-regular graph on $A$ (e.g. a circulant graph connecting each vertex to its $\Delta/2$ nearest neighbours on each side on a cycle), which has exactly $a\Delta/2$ edges.
So we can achieve $e(H)=\lfloor a\Delta/2\rfloor$ with maximum degree $\le\Delta$.

We choose parameters so that $e(G)>n^2/4$ while making the right-hand side of (5) as small as possible.
Write $x:=b/n$ and $y:=\Delta/n$.
Then $a/n=1-x$ and (ignoring lower-order rounding errors)
\[
\frac{e(G)}{n^2} \approx x(1-x) + \frac{(1-x)y}{2}.
\]
To have $e(G)>n^2/4$, it suffices to enforce
\[
 x(1-x) + \frac{(1-x)y}{2} > \frac{1}{4}. \tag{6}
\]
Meanwhile, (5) gives the triangle degree-sum bound
\[
\frac{1}{n}(n+b+2\Delta) = 1 + x + 2y. \tag{7}
\]
We now optimize (7) subject to (6).
For fixed $x\in(0,1)$, the smallest $y$ making (6) hold is
\[
 y = \frac{1}{2(1-x)} - 2x.
\]
Substituting into (7) gives
\[
 f(x) := 1 + x + 2\left(\frac{1}{2(1-x)} - 2x\right) = 1 - 3x + \frac{1}{1-x}.
\]
Differentiate:
\[
 f'(x) = -3 + \frac{1}{(1-x)^2}.
\]
Setting $f'(x)=0$ gives $(1-x)^2=1/3$, i.e. $x=1-1/\sqrt{3}$.
At this value,
\[
 f(x)=1-3\Bigl(1-\frac{1}{\sqrt{3}}\Bigr) + \sqrt{3} = -2 + \sqrt{3} + \frac{3}{\sqrt{3}} = 2\sqrt{3}-2 = 2(\sqrt{3}-1).
\]
Thus the optimal constant for this construction is exactly $2(\sqrt{3}-1)$.
Choosing integers $a,b,\Delta$ with
\[
\frac{b}{n}=1-\frac{1}{\sqrt{3}}+o(1),\qquad \frac{\Delta}{n}=\left(\frac{\sqrt{3}}{2}-2+\frac{2}{\sqrt{3}}\right)+o(1),
\]
and taking $H$ to be a (nearly) $\Delta$-regular circulant graph on $A$ yields $e(G)=n^2/4+\Omega(n)$ and (5) gives
\[
\max_{\text{triangles }T} \sum_{v\in T}\deg(v) \le \left(2(\sqrt{3}-1)+o(1)\right)n.
\]
This proves the lemma.
\qed

\medskip
\textbf{FAST REALITY CHECK (small cases / computation).}
By exhaustive enumeration of all labelled graphs on $n\le 7$ vertices, the exact values found are:
\[
 h(3)=6,\quad h(4)=8,\quad h(5)=9,\quad h(6)=10,\quad h(7)=12.
\]
(Here $h(n)$ is computed exactly from the definition: among graphs with $e(G)>n^2/4$, minimize the maximum degree-sum over all triangles.)

\textbf{VERIFICATION.}
\begin{itemize}
\item Lemma 1: the averaging identity $\sum_{uv\in E}(\deg u+\deg v)=\sum_v \deg(v)^2$ is checked by double-counting contributions of each vertex to its incident edges.
The triangle existence step uses inclusion--exclusion and the bound $|N(u)\cup N(v)|\le n-2$.
\item Lemma 2: every triangle must use two vertices from $A$ and one from $B$ because there are no edges inside $B$.
Degree bounds follow from the construction. The optimization is a one-variable calculus problem.
Rounding issues only contribute $o(n)$ to the degree-sum and $O(n)$ to edge counts, so the asymptotic constant is unchanged.
\end{itemize}

\textbf{FINAL.} \textbf{UNRESOLVED.}

(i) \emph{Strongest proved partial result.} I proved a very weak universal lower bound $h(n)\ge n+2$ (Lemma 1) and gave a fully explicit construction showing $h(n)\le 2(\sqrt{3}-1)n+o(n)$ (Lemma 2).
I also computed exact $h(n)$ for $n\le 7$.

(ii) \emph{First gap (crisp).} Prove or disprove the conjectured asymptotic lower bound
\[h(n)\ge (2(\sqrt{3}-1)-o(1))n.\]
Equivalently: show that for every $\varepsilon>0$ and all large $n$, every graph with $e(G)>n^2/4$ has a triangle with degree-sum at least $(2(\sqrt{3}-1)-\varepsilon)n$.

(iii) \emph{Top 3 next moves.}
1. Strengthen Lemma 1 by leveraging not just one large edge but correlations among high-degree vertices in graphs with $>n^2/4$ edges; target a lemma that forces a triangle whose vertices all lie in the high-degree set.
2. Attempt a stability-type argument near Mantel's theorem: if $e(G)>n^2/4$, then $G$ is close to bipartite; quantify how internal edges force triangles, and how degrees distribute between parts.
3. Extend the brute force computation to $n=8,9$ with pruning/ILP to guess the extremal structure and the correct next lower-bound constant.

(iv) \emph{What a minimal counterexample would likely look like.} A graph with $e(G)=n^2/4+o(n^2)$ that is ``almost'' complete bipartite, with all extra edges placed inside a smaller part in a bounded-degree way, so that every triangle has two vertices in that part (bounded additional degree) and one vertex in the other part (degree about the smaller part size). Such a structure is exactly what Lemma 2 exploits.

