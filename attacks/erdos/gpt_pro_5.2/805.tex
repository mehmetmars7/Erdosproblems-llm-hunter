# Erdős Problem #805 — Research Analysis

## 1) FORMAL RESTATEMENT

### Problem statement (precise form)

Let (n\ge 1) be an integer. Define
[
m_n(r)
:=\min\Bigl{m:\ \exists\text{ a graph }G\text{ on }n\text{ vertices such that }
\forall S\subseteq V(G),\ |S|=m,\
\omega(G[S])\ge r\ \text{and}\ \alpha(G[S])\ge r
\Bigr}.
]

Here:

* graphs are finite, simple, undirected;
* (G[S]) denotes the induced subgraph on (S);
* (\omega(\cdot)) is clique number, (\alpha(\cdot)) independence number;
* logarithms are natural.

Erdős–Hajnal ask:

> (**EH-805**)
> Is it true that
> [
> m_n(\lceil\log n\rceil)\le (\log n)^3
> \quad\text{for all sufficiently large }n?
> ]

Equivalently: does there exist, for large (n), a graph on (n) vertices such that **every** induced subgraph on ((\log n)^3) vertices contains both

* a clique of size at least (\log n), and
* an independent set of size at least (\log n)?

Erdős and Hajnal conjectured the answer is **no**.

---

## 2) QUICK LITERATURE / CONTEXT CHECK

Known results (chronological):

* **Alon–Sudakov (2007)** proved a non-existence result when
  [
  m \le c,\frac{(\log n)^3}{\log\log n},
  ]
  showing that graphs with the required property **cannot** exist at that scale.

* **Alon–Bucić–Sudakov (2021)** constructed graphs showing
  [
  m_n(\log n)\ \le\ 2^{2^{(\log\log n)^{1/2+o(1)}}}.
  ]

Thus the current gap is
[
\frac{(\log n)^3}{\log\log n}
\ \lesssim\
m_n(\log n)
\ \lesssim\
2^{2^{(\log\log n)^{1/2+o(1)}}},
]
and the specific threshold (m=(\log n)^3) remains **open**.

---

## 3) ATTACK PLAN

Two possible routes:

### (A) Disproof route (Erdős–Hajnal conjecture true)

Strengthen the Alon–Sudakov lower-bound mechanism to eliminate the remaining (\log\log n) gap, showing that ((\log n)^3) is still too small.

### (B) Construction route (Erdős–Hajnal conjecture false)

Improve the ABS21 construction dramatically, compressing the current double-exponential bound down to polylogarithmic scale.

The only route where a **fully rigorous result** can currently be completed is (A), up to the known ((\log n)^3/\log\log n) barrier.

---

## 4) WORK (Complete Rigorous Partial Result)

### Key auxiliary parameter

For integers (n>s>t\ge2), define (f(n,s,t)) to be the **minimum possible independence number** of an (n)-vertex graph (H) satisfying:

> every induced subgraph of (H) on (s) vertices contains an independent set of size at least (t).

---

### Lemma 1 (Alon–Sudakov reduction lemma)

Let (n) be large and (n/2 > s > t \ge 2).
If
[
f(\lfloor n/2\rfloor,s,t) > \frac{s}{t-1},
]
then **no** graph (G) on (n) vertices has the property that every induced subgraph on (s) vertices contains **both**

* a clique of size at least (t), and
* an independent set of size at least (t).

#### Proof

Assume such a graph (G) exists.

Let (F := f(\lfloor n/2\rfloor,s,t)).
Every induced subgraph of (G) on (\lfloor n/2\rfloor) vertices must contain an independent set of size at least (F).

Iteratively remove such independent sets (I_1,I_2,\dots).
After at most (t-1) steps, the union
[
U := I_1\cup\cdots\cup I_{t-1}
]
satisfies
[
|U| \ge \min{\lfloor n/2\rfloor,(t-1)F} > s.
]

But (G[U]) is ((t-1))-colorable, hence contains **no** clique of size (t).
Choosing any (s)-subset of (U) contradicts the defining property of (G). ∎

---

### Theorem 2 (Alon–Sudakov lower bound)

There exists an absolute constant (c_0>0) such that for all sufficiently large (n),
[
m_n(\lceil\log n\rceil)
\ \ge
c_0,\frac{(\log n)^3}{\log\log n}.
]

#### Proof

Let
[
t=\lceil\log n\rceil,
\qquad
s=\left\lfloor c_0\frac{(\log n)^3}{\log\log n}\right\rfloor,
\qquad
n'=\lfloor n/2\rfloor.
]

Alon–Sudakov proved that for (2t\le s<n'/2),
[
f(n',s,t)
\ \ge
C,t\frac{\log(n'/s)}{\log(s/t)}
]
for an absolute constant (C>0).

Elementary estimates give:
[
\log(n'/s)\ge \tfrac12\log n,
\qquad
\log(s/t)\le 3\log\log n,
]
hence
[
f(n',s,t)
\ \ge
\frac{C}{6}\frac{(\log n)^2}{\log\log n}.
]

Meanwhile,
[
\frac{s}{t-1}
\le
2c_0\frac{(\log n)^2}{\log\log n}.
]

Choosing (c_0<C/12) ensures
[
f(n',s,t)>\frac{s}{t-1},
]
so Lemma 1 applies, proving non-existence at this scale. ∎

---

## 5) VERIFICATION

* All inheritance steps are valid for induced subgraphs.
* Parameter constraints (2t\le s\ll n) are satisfied.
* Floors/ceilings affect constants only.
* The argument explicitly shows **where the (\log\log n) loss occurs**.

No hidden assumptions are used.

---

## 6) FINAL: **UNRESOLVED**

### Summary

* It is **rigorously proved** that
  [
  m_n(\log n)\ \ge\ c,\frac{(\log n)^3}{\log\log n}.
  ]
* It remains **open** whether this lower bound can be pushed to ((\log n)^3), or whether a construction exists at that scale.
* The precise obstacle is that current bounds on the auxiliary function (f(n,s,t)) lose a factor of (\log\log n), and no method is known to recover it using both the clique and independence requirements simultaneously.

### Minimal next gap

> Prove or refute:
> [
> f(n,\log^3 n,\log n)\ \gg\ (\log n)^2
> ]
> under the **two-sided** Erdős–Hajnal hypothesis.

This is exactly the missing step separating the conjecture from resolution.

If you want, next we can:

* isolate the obstruction combinatorially (why (\log\log n) appears),
* attempt a strengthening of the Alon–Sudakov inequality,
* or analyze why the ABS21 construction fundamentally stalls far above polylogarithmic scale.
