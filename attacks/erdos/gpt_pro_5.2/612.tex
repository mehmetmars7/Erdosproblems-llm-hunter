% Erdos Problem #612
% URL: https://www.erdosproblems.com/612

FORMAL RESTATEMENT
Let $r\in\mathbb{Z}_{\ge 1}$. Let $G$ be a finite, simple, connected graph with
\begin{itemize}
\item $n:=|V(G)|$ vertices,
\item minimum degree $\delta(G)=d$,
\item diameter $D:=\mathrm{diam}(G)$.
\end{itemize}
The original Erd\H{o}s--Pach--Pollack--Tuza (EPPT) question, as stated in the
problem file, asks for asymptotic upper bounds of the form
\[
D \le c\,\frac{n}{d} + O(1)
\]
under clique-forbidding hypotheses and certain divisibility conditions on $d$:
\begin{itemize}
\item If $G$ contains no $K_{2r}$ and $(r-1)(3r+2)\mid d$, conjecturally
\[
D \le \frac{2(r-1)(3r+2)}{2r^2-1}\,\frac{n}{d}+O(1).
\]
\item If $G$ contains no $K_{2r+1}$ and $(3r-1)\mid d$, conjecturally
\[
D \le \frac{3r-1}{r}\,\frac{n}{d}+O(1).
\]
\end{itemize}
The same problem file also records that (at least) the even-clique-free part of
this conjecture is false for every fixed $r\ge 2$ (via constructions of
Czabarka--Szentmikl\H{o}ssy--Sz\H{o}nyi, and also Cambie--Jooken).
It then proposes the following amended conjecture:

\medskip
\noindent\textbf{Amended conjecture (as in the problem file).}
If $G$ contains no $K_{k+1}$, then
\[
D \le \Bigl(3-\frac{2}{k}\Bigr)\frac{n}{d} + O(1).
\]

QUICK LITERATURE/CONTEXT CHECK
I am \emph{not} using any external sources here; I only use what is explicitly
written in the problem file.

The problem file states:
\begin{itemize}
\item EPPT gave constructions showing sharpness of the original proposed
constants and proved the case $2r+1=3$.
\item A general unconditional bound is known for all connected graphs:
$D\le 3\,\frac{n}{d+1}+O(1)$.
\item For $r\ge2$ the original even-clique-free conjecture is false (CSS21), and
there is also a specific $K_4$-free counterexample at minimum degree $16$ (CaJo25).
\item The amended bound $\bigl(3-2/k\bigr)\frac{n}{d}+O(1)$ is proposed for
$K_{k+1}$-free graphs, and is known under the stronger assumption
``$k$-colorable'' for $k=3,4$.
\end{itemize}
I do not assume or claim anything beyond these statements.

ATTACK PLAN
Two standard approaches to diameter bounds with minimum degree are:
\begin{itemize}
\item \textbf{Packing neighborhoods along a diameter geodesic:}
choose a shortest path of length $D$ and pack disjoint closed neighborhoods
$N[v]$ of vertices spaced $3$ apart. This yields the universal constant $3$.
\item \textbf{Expansion from forbidden cliques:}
if $G$ is $K_{k+1}$-free, then neighborhoods $N(v)$ are $K_k$-free.
Tur\'an-type extremal bounds force neighborhoods to be ``not too dense'', which
can force at least one neighbor to have many neighbors outside $N[v]$.
This gives local two-step expansion, a potential ingredient toward improving the
constant below $3$.
\end{itemize}
I carry out these two ingredients rigorously below.

WORK
\textbf{Lemma 1 (unconditional diameter bound via disjoint closed neighborhoods).}
Let $G$ be connected with $n$ vertices and minimum degree $d\ge1$.
Then
\[
D \le \frac{3n}{d+1}-1.
\]

\emph{Proof.}
Let $v_0,v_1,\dots,v_D$ be a shortest path of length $D$ between two vertices at
distance $D$ (so $\mathrm{dist}(v_0,v_D)=D$). For each integer $i\ge0$ with
$3i\le D$, consider the closed neighborhood
\[
N[v_{3i}] := \{v_{3i}\}\cup N(v_{3i}).
\]
Because $\delta(G)=d$, each $N[v_{3i}]$ has size at least $d+1$.

We claim these sets are pairwise disjoint.
Take $i<j$ with $3i,3j\le D$.
If there were a vertex $x\in N[v_{3i}]\cap N[v_{3j}]$, then
$x$ is adjacent to $v_{3i}$ or equals it, and also adjacent to $v_{3j}$ or equals it.
In any case there is a path of length at most $2$ from $v_{3i}$ to $v_{3j}$ going
through $x$, so
$\mathrm{dist}(v_{3i},v_{3j})\le 2$.
But along the chosen path, $\mathrm{dist}(v_{3i},v_{3j})=3(j-i)\ge 3$, a contradiction.
Thus $N[v_{3i}]\cap N[v_{3j}]=\varnothing$.

Let $t:=\lfloor D/3\rfloor$. Then we have $t+1$ disjoint sets
$N[v_0],N[v_3],\dots,N[v_{3t}]$, each of size at least $d+1$.
Hence
\[
 n \ge (t+1)(d+1) = \bigl(\lfloor D/3\rfloor+1\bigr)(d+1).
\]
Rearranging gives
$\lfloor D/3\rfloor \le \frac{n}{d+1}-1$, so
$D \le 3\bigl(\frac{n}{d+1}-1\bigr)+2 = \frac{3n}{d+1}-1$.
\hfill $\square$

\medskip
\textbf{Lemma 2 (a forced second neighborhood in $K_{k+1}$-free graphs).}
Let $k\ge2$ and let $G$ be $K_{k+1}$-free with minimum degree $d$.
Let $v$ be a vertex of degree exactly $d$.
Then there exists a neighbor $w\in N(v)$ such that
\[
|N(w)\setminus N[v]| \ge \Bigl\lceil \frac{d}{k-1}\Bigr\rceil - 1.
\]
In particular, the ball of radius $2$ around $v$ satisfies
\[
|B_2(v)|=|\{x:\mathrm{dist}(v,x)\le2\}| \ge 1+d+\Bigl\lceil\frac{d}{k-1}\Bigr\rceil-1
= d+\Bigl\lceil\frac{d}{k-1}\Bigr\rceil.
\]

\emph{Proof.}
Because $G$ is $K_{k+1}$-free, the induced subgraph $H:=G[N(v)]$ on the
neighborhood $N(v)$ is $K_k$-free: if $H$ contained a $K_k$ then adding $v$ would
create a $K_{k+1}$ in $G$.
Here $|V(H)|=|N(v)|=d$.

We first use the following standard extremal estimate for $K_k$-free graphs.

\emph{Sublemma (Tur\'an upper bound, in the form needed here).}
If $H$ is $K_k$-free on $d$ vertices ($k\ge2$), then
\[
 e(H) \le \Bigl(1-\frac{1}{k-1}\Bigr)\frac{d^2}{2}.
\]

\emph{Proof of sublemma.}
We sketch a self-contained argument.
Among all $K_k$-free graphs on $d$ labeled vertices, choose one with the maximum
number of edges; call it $H$.

\underline{Step 1: $H$ must be complete multipartite.}
Suppose $x,y$ are non-adjacent vertices in $H$.
If $N_H(x)\neq N_H(y)$, assume $\deg_H(x)\ge \deg_H(y)$.
Construct a new graph $H'$ by \emph{symmetrizing} $y$ toward $x$:
remove all edges incident to $y$ and then connect $y$ to exactly the neighbors
of $x$ (so $N_{H'}(y):=N_H(x)$), leaving all other adjacencies unchanged.
Then $e(H')\ge e(H)$ because $y$'s degree becomes $\deg_H(x)\ge \deg_H(y)$.
Also, $H'$ is still $K_k$-free: any clique of size $k$ that uses $y$ in $H'$
corresponds to a clique of size $k$ using $x$ in $H$ (replace $y$ by $x$), since
$y$ and $x$ have the same neighbors in $H'$.
Thus $H'$ is also an extremal $K_k$-free graph with at least as many edges.
Iterating this symmetrization operation, we may assume that in our extremal
choice, every pair of non-adjacent vertices have identical neighborhoods.
This property implies that the non-adjacency relation is an equivalence relation:
if $u\not\sim v$ and $v\not\sim w$, then $N(u)=N(v)=N(w)$ so $u\not\sim w$.
Hence the vertex set partitions into independent ``parts'' in which vertices are
pairwise non-adjacent, and between any two distinct parts the graph is complete.
That is, $H$ is complete multipartite.

\underline{Step 2: $H$ has at most $k-1$ parts.}
If $H$ had $k$ nonempty parts, picking one vertex from each part would give a
$K_k$ (because edges between distinct parts are complete), contradicting
$K_k$-freeness.
So $H$ is complete $t$-partite for some $t\le k-1$.

\underline{Step 3: bound the number of edges.}
Let the part sizes be $s_1,\dots,s_t$ with $\sum_i s_i=d$.
A complete $t$-partite graph has
\[
 e(H)=\sum_{1\le i<j\le t} s_i s_j = \frac{1}{2}\Bigl(\bigl(\sum_i s_i\bigr)^2 - \sum_i s_i^2\Bigr)
=\frac{1}{2}\bigl(d^2-\sum_i s_i^2\bigr).
\]
By Cauchy--Schwarz, $\sum_i s_i^2\ge \frac{1}{t}(\sum_i s_i)^2=\frac{d^2}{t}$,
so
\[
 e(H)\le \frac{1}{2}\Bigl(d^2-\frac{d^2}{t}\Bigr)=\Bigl(1-\frac{1}{t}\Bigr)\frac{d^2}{2}
\le \Bigl(1-\frac{1}{k-1}\Bigr)\frac{d^2}{2},
\]
since $t\le k-1$ implies $1-1/t\le 1-1/(k-1)$.
This proves the sublemma.
\hfill $\blacksquare$

Returning to the main lemma, the sublemma gives
\[
\frac{2e(H)}{|V(H)|} \le \Bigl(1-\frac{1}{k-1}\Bigr)d.
\]
So the average degree in $H$ is at most $\bigl(1-\frac{1}{k-1}\bigr)d$.
Therefore there exists a vertex $w\in V(H)=N(v)$ with
\[
\deg_H(w) \le \Bigl(1-\frac{1}{k-1}\Bigr)d.
\]
Now compare $w$'s degree in $G$ to its degree inside $N(v)$.
Vertex $w$ is adjacent to $v$, and it has $\deg_H(w)$ neighbors inside $N(v)$.
All its remaining neighbors lie outside $N[v]=\{v\}\cup N(v)$.
Hence
\[
|N(w)\setminus N[v]| = \deg_G(w) - 1 - \deg_H(w)
\ge d - 1 - \Bigl(1-\frac{1}{k-1}\Bigr)d
= \frac{d}{k-1}-1.
\]
Since the left-hand side is an integer, it is at least
$\lceil \frac{d}{k-1}\rceil-1$.
\hfill $\square$

VERIFICATION
\textbf{Boundary checks.}
Lemma 1 is meaningful for $d\ge1$; for $d=n-1$ it gives $D\le 3n/n-1=2$, which is
true since such a graph is complete.
Lemma 2 assumes $k\ge2$; for $k=2$ (triangle-free) it yields a neighbor $w$ with
$|N(w)\setminus N[v]|\ge d-1$, i.e., a minimum-degree vertex $v$ has a neighbor
whose neighbors are mostly outside $N[v]$; this matches the intuition that
triangle-freeness forbids edges inside $N(v)$.

\textbf{FAST REALITY CHECK (computation, small $n$).}
I exhaustively checked all connected graphs on $n=6$ vertices with minimum degree
$d\ge1$ and computed their diameters.
There were $0$ violations of the explicit inequality
$D\le 3n/(d+1)-1$.
(For reference, the largest observed value of $D(d+1)/n$ in this universe was
$5\cdot 2/6=1.666\ldots$, attained by a path graph with $d=1$ and $D=5$.)

FINAL
**UNRESOLVED**
(i) Strongest proved partial result here: the universal bound
$D\le \frac{3n}{d+1}-1$ (Lemma 1), and a genuine clique-free ``two-step
expansion'' statement for $K_{k+1}$-free graphs (Lemma 2), namely
$|B_2(v)|\ge d+\lceil d/(k-1)\rceil$ for some minimum-degree vertex $v$.

(ii) First gap (crisp): turn clique-freeness into a \emph{global} improvement of
the constant $3$ in Lemma 1; concretely, prove (or disprove) the amended
conjecture that every $K_{k+1}$-free graph satisfies
$D\le (3-2/k)\,\frac{n}{d}+O(1)$.

(iii) Top 3 next moves:
1. Strengthen Lemma 2 from a single ball $B_2(v)$ to an inequality for arbitrary
sets $S$ (expansion of $S$) that controls overlap, e.g. a lower bound on
$|N(S)\setminus S|$ in terms of $|S|,d,k$.
2. Apply such an expansion bound to BFS layers (from both ends of a diameter
geodesic) to upper bound the number of layers needed to cover $n$ vertices,
seeking a coefficient $<3$.
3. For potential disproof of the amended constant, attempt an explicit
construction where vertices are arranged in a long chain of dense
$K_{k+1}$-free ``clusters'' with limited inter-cluster connections so that the
minimum degree comes mostly from intra-cluster edges while diameter remains
linear in the number of clusters.

(iv) Minimal counterexample structure:
A smallest counterexample to the amended conjecture at fixed $k$ would likely be
a $K_{k+1}$-free, roughly $d$-regular graph with large $d$ and diameter
$D\approx c\,\frac{n}{d}$ for some $c>(3-2/k)$, and whose BFS layers from an
endpoint expand much more slowly than in a typical expander; structurally, such
examples are expected to concentrate many edges inside layers/clusters (wasting
minimum degree on short-range edges) while keeping few edges that advance the
distance from an endpoint.


