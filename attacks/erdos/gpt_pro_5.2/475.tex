
1) “FORMAL RESTATEMENT”

Let p be a prime and let A be any finite subset of F_p \ {0}. Write |A|=t. The question is whether there always exists an ordering (a_1,...,a_t) of the elements of A such that the partial sums

    s_m := a_1 + a_2 + ... + a_m   (mod p)

are pairwise distinct for all 1<=m<=t.

Equivalently: the walk starting at 0 and taking steps a_1,...,a_t visits t distinct vertices of the cyclic group F_p.

2) “QUICK LITERATURE/CONTEXT CHECK”

The problem statement lists several partial results (t<=12; p-3<=t<=p-1; and various growing ranges). A quick check of the Erdos Problems page for #475 shows it is listed as OPEN (as of mid-Jan 2026). I do not use any external results beyond the problem statement.

3) “ATTACK PLAN”

- Prove the conjecture for special structured sets A (e.g. geometric progressions), to understand mechanisms.
- Search for counterexamples by brute force for small primes p and all subsets A.
- For a general proof, one expects a mix of additive combinatorics and an “absorption”/rearrangement argument to fix collisions of partial sums.

4) “WORK”

Proposition 475.1 (full set A = F_p^* has a valid ordering).
Let g be a primitive root modulo p (so g has multiplicative order p-1). Consider the ordering

    a_1=1, a_2=g, a_3=g^2, ..., a_{p-1}=g^{p-2}.

Then the partial sums s_m are all distinct (for 1<=m<=p-1).

Proof.
For m>=1, we have a geometric series identity in F_p:

    s_m = 1 + g + g^2 + ... + g^{m-1} = (g^m - 1)/(g - 1),

where division by (g-1) is valid because g!=1.
Suppose s_m = s_n for some 1<=m,n<=p-1. Multiply by (g-1): g^m - 1 = g^n - 1, hence g^m = g^n. Since g has order p-1 and 1<=m,n<=p-1, this forces m=n. Thus the partial sums are pairwise distinct.  QED.

Proposition 475.2 (geometric progression sets).
Let A = {1, g, g^2, ..., g^{t-1}} be a geometric progression in F_p^* with g!=1, and assume t <= ord(g) (so the listed powers are distinct). Then the same ordering a_j = g^{j-1} has distinct partial sums.

Proof.
The same computation gives s_m = (g^m - 1)/(g - 1). If s_m=s_n then g^m=g^n, and since 0<=m-1,n-1<=t-1 < ord(g), we get m=n.  QED.

FAST REALITY CHECK (local computation).
I brute-forced the conjecture for small primes by checking every subset A of F_p^* and testing whether some permutation gives distinct partial sums.

- For p in {3,5,7,11,13}, every subset A ⊆ F_p^* admits a valid ordering (no counterexamples).
- For p=17, I checked all subsets of sizes up to 9 (inclusive) and found no counterexample in that range.
(The full p=17 check over all 2^16 subsets is heavier; the computation above was cut off by time limits before completing size 10 and above.)

5) “VERIFICATION”

- Proposition 475.1 produces an explicit valid ordering for the extremal case t=p-1 (matching the partial result mentioned in the problem statement).
- The proof uses only that g has order p-1 (primitive root) and g!=1.
- In Proposition 475.2, the hypothesis t<=ord(g) is needed so that g^m=g^n implies m=n for exponents in {1,...,t}.

6) FINAL

UNRESOLVED

(i) Strongest proved partial result:
The conjecture holds when A is the full multiplicative group F_p^* (explicit ordering via powers of a primitive root), and more generally for any geometric progression A={1,g,...,g^{t-1}} with t<=ord(g).

(ii) First gap (crisp statement):
Prove (or find a counterexample to) the statement: every subset A ⊆ F_p^* has some ordering with distinct partial sums.

(iii) Top 3 next moves (concrete):
1. Try to extend Proposition 475.2 from geometric progressions to sets A with small multiplicative doubling (A·A small), via Freiman-type structure plus a rearrangement/absorption step.
2. Attempt a probabilistic construction: show that a random ordering has few collisions of partial sums and then repair collisions locally.
3. Systematically search for counterexamples in the “middle range” of sizes, say t around p/2, for moderately large primes (e.g. p~200-500), using backtracking with pruning.

(iv) Minimal counterexample structure (if the answer is “no”):
A prime p and a subset A ⊆ F_p^* such that for every permutation (a_1,...,a_t), two distinct partial sums coincide, i.e. there exist 1<=m<n<=t with a_1+...+a_m ≡ a_1+...+a_n (mod p).


