\section*{Problem 776: many sizes in an antichain with multiplicities}

\subsection*{1) Formal restatement}
Let \(\mathcal{F}\subseteq 2^{[n]}\) be an antichain.
Fix \(r\in\mathbb{N}\) and assume that for every size \(k\) occurring in \(\mathcal{F}\), there are at least \(r\) sets in \(\mathcal{F}\) of size \(k\).
How large can \(|\{|A|:A\in\mathcal{F}\}|\) be?
For \(r=1\) the maximum is \(n-2\).
For \(r>1\) it is stated that \(n-2\) is impossible but \(n-3\) is achievable for large \(n\).
Question: determine/estimate how large \(n\) must be (as a function of \(r\)) to guarantee an example with \(n-3\) distinct sizes.

\subsection*{2) Quick literature/context check (no more than 8 lines)}
\begin{itemize}[leftmargin=*]
\item For \(r=1\), there are explicit constructions with \(n-2\) distinct sizes.
\item For \(r>1\), one loses at least one additional size (so \(n-2\) is impossible).
\item The quantitative threshold \(n(r)\) for achieving \(n-3\) sizes is the open part.
\end{itemize}

\subsection*{3) Attack plan}
\begin{itemize}[leftmargin=*]
\item Prove \(n-2\) distinct sizes are impossible when \(r>1\) by showing sizes \(1\) and \(n-1\) cannot coexist.
\item Give an explicit construction achieving sizes \(2,3,\dots,n-2\) (i.e.\ \(n-3\) sizes) once \(n\) is sufficiently large in terms of \(r\); this yields an explicit upper bound \(n(r)\le Cr+O(1)\).
\end{itemize}

\subsection*{4) Work}

\paragraph{Impossibility of \(n-2\) sizes for \(r>1\).}
In any antichain with more than one set, sizes \(0\) and \(n\) are absent (since \(\varnothing\subseteq A\subseteq [n]\) for all \(A\)).
Thus achieving \(n-2\) sizes would force all but one of the sizes \(1,2,\dots,n-1\) to occur, in particular both sizes \(1\) and \(n-1\).

If \(r>1\), the presence of size \(1\) means \(\mathcal{F}\) contains two distinct singletons \(\{a\}\) and \(\{b\}\).
Any \((n-1)\)-set omits exactly one element, hence contains at least one of \(a,b\).
So any \((n-1)\)-set \(X\in\mathcal{F}\) would contain \(\{a\}\) or \(\{b\}\), contradicting antichain-ness.
Therefore sizes \(1\) and \(n-1\) cannot both occur, and \(n-2\) distinct sizes are impossible for \(r>1\).

\paragraph{A linear-in-\(r\) construction achieving \(n-3\) sizes.}
We construct an antichain \(\mathcal{F}\subseteq 2^{[n]}\) with at least \(r\) sets of each size \(2,3,\dots,n-2\), provided \(n\ge 8r+10\).
Set \(t=\lfloor n/2\rfloor\).

\emph{Step 1: build an intersecting antichain \(\mathcal{G}\) with sizes \(2,3,\dots,t\).}
Use a common element \(1\).
Let
\[
A=\{2,3,\dots,r+1\},\qquad C=\{r+2,\dots,r+t-1\},
\]
and let \(B=\{r+t,\dots,n\}\) be the remaining elements.
Then \(|C|=t-2\) and \(|B|=b=n-r-t+1\).

Define \(r\) size-2 sets:
\[
G_{2,i}=\{1,i+1\}\qquad(i=1,\dots,r).
\]
For each \(s\in\{3,\dots,t\}\), let \(F_s=\{r+2,\dots,r+s-2\}\subseteq C\) (so \(|F_s|=s-3\)).
Assign to each pair \((s,i)\) (with \(3\le s\le t\) and \(1\le i\le r\)) a \emph{distinct} label \(P_{s,i}\in\binom{B}{2}\),
and set
\[
G_{s,i}=\{1\}\cup P_{s,i}\cup F_s.
\]
Let \(\mathcal{G}=\{G_{s,i}:2\le s\le t,\ 1\le i\le r\}\).

\emph{Step 2: \(\mathcal{G}\) is an intersecting antichain.}
All sets contain \(1\), so \(\mathcal{G}\) is intersecting.
No size-2 set \(G_{2,i}\) is contained in any \(G_{s,i'}\) with \(s\ge 3\), because \(i+1\in A\) and all \(G_{s,i'}\) (for \(s\ge 3\)) use only \(\{1\}\cup B\cup C\), avoiding \(A\setminus\{1\}\).
If \(s,s'\ge 3\) and \(G_{s,i}\subseteq G_{s',i'}\), then their intersections with \(B\) satisfy
\(P_{s,i}\subseteq P_{s',i'}\), forcing \(P_{s,i}=P_{s',i'}\), contradicting distinctness.
Hence \(\mathcal{G}\) is an antichain.

\emph{Step 3: add complements.}
Let
\[
\mathcal{F}=\mathcal{G}\ \cup\ \{[n]\setminus G: G\in\mathcal{G}\}.
\]
Because \(\mathcal{G}\) is intersecting, no \(G\in\mathcal{G}\) is contained in the complement of another member, so \(\mathcal{F}\) is still an antichain.
The sizes appearing in \(\mathcal{F}\) are exactly \(2,3,\dots,n-2\), i.e.\ \(n-3\) distinct sizes, and each such size occurs at least \(r\) times (from \(\mathcal{G}\) or from complements).

\emph{Step 4: enough labels.}
We need \(\binom{b}{2}\ge r(t-2)\) to assign distinct pairs \(P_{s,i}\).
For \(n\ge 8r+10\) one checks that \(b=n-r-t+1\ge 3r+6\) and \(\binom{b}{2}\) dominates \(r(t-2)\), so the labeling is feasible.
Thus \(n(r)\le 8r+10\) is an explicit sufficient bound.

\subsection*{5) Verification}
\begin{itemize}[leftmargin=*]
\item The \(n-2\) impossibility uses only \(r>1\): two singletons force containment into any \((n-1)\)-set.
\item The construction is checked by separating (i) size-2 sets using reserved elements, and (ii) larger sets distinguished by unique 2-element labels in \(B\).
\item Intersecting \(\mathcal{G}\) guarantees no cross-containment when adding complements.
\end{itemize}

\subsection*{6) Final (exactly one label and one sub-label)}
\noindent\textbf{LABEL: UNRESOLVED.}\\
\textbf{SUBLABEL: Proved \(n-2\) sizes impossible for \(r>1\); gave an explicit construction showing \(n(r)\le 8r+10\) suffices for \(n-3\) sizes.}

\subsection*{7) Completion estimate}
Determining the true threshold \(n(r)\) would require either improved constructions (reducing the constant \(8\)) or nontrivial lower bounds showing \(n\) must grow at least linearly (or faster) in \(r\).

