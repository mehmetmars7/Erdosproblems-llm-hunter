
1) FORMAL RESTATEMENT

Fix an integer $N\ge 1$.

For a set $A\subseteq\{1,2,\dots,N\}$ define
\[
\sigma_A: \mathcal P(A)\to\mathbb Q,\qquad \sigma_A(S):=\sum_{n\in S}\frac1n,
\]
where $\mathcal P(A)$ is the power set.

We say that $A$ is \emph{reciprocal-subset-sum distinct} if $\sigma_A$ is injective, i.e.
\[
S\neq T\ \Longrightarrow\ \sum_{n\in S}\frac1n\neq\sum_{n\in T}\frac1n.
\]
Let
\[
R(N):=\max\bigl\{|A|:\ A\subseteq\{1,\dots,N\}\text{ and $A$ is reciprocal-subset-sum distinct}\bigr\}.
\]

Problem: determine the order of magnitude of $R(N)$ as $N\to\infty$.

Edge cases.
$A=\emptyset$ always works, so $R(N)\ge 0$. For $N=1$, $A=\{1\}$ works so $R(1)=1$.


2) QUICK LITERATURE/CONTEXT CHECK

I do not use any external results beyond what is explicitly stated in the problem text.
The statement records bounds of Bleicher and Erd\H{o}s:
\[
\frac{N}{\log N}\prod_{i=3}^k\log_i N\ \le\ R(N)\ \le\ \frac{1}{\log 2}\,\log_r N\left(\frac{N}{\log N}\prod_{i=3}^r\log_i N\right),
\]
valid for any $k\ge 4$ with $\log_k N\ge k$ and any $r\ge 1$ with $\log_{2r}N\ge 1$.
(In these bounds $\log_i$ denotes the $i$-fold iterated logarithm.)
I do not re-prove these bounds here.


3) ATTACK PLAN

Give fully proved (but much weaker) structural statements:

(1) A clean sufficient condition on $A$ (a ``superincreasing'' condition for reciprocals) that guarantees injectivity.

(2) An explicit construction $A\subseteq\{1,\dots,N\}$ giving a rigorous lower bound $R(N)\ge \lfloor\log_2 N\rfloor+1$.

(3) A general denominator-counting upper bound of the form $2^{|A|}\le \mathrm{lcm}(A)\,H_N+1$, where $H_N$ is the harmonic number.

Also, do a brute-force reality check for small $N$.


4) WORK

PHASE 1: FAST REALITY CHECK (exact brute force for $N\le 20$)

I exhaustively searched all subsets $A\subseteq\{1,\dots,N\}$ for $N\le 20$ and computed $R(N)$ exactly.
The values found are:
\[
\begin{array}{c|cccccccccc}
N & 1&2&3&4&5&6&7&8&9&10\\\hline
R(N) & 1&2&3&4&5&5&6&7&8&9
\end{array}
\qquad
\begin{array}{c|cccccccccc}
N & 11&12&13&14&15&16&17&18&19&20\\\hline
R(N) & 10&10&11&12&12&13&14&14&15&15
\end{array}
\]
One optimal set for $N=20$ is
\[
A=\{1,2,3,4,5,7,8,9,10,11,13,14,16,17,19\},\qquad |A|=15.
\]

(These values are only a sanity check and do not indicate the true asymptotic size.)


Lemma 321.1 (superincreasing reciprocals $\Rightarrow$ all subset sums distinct).

Let $A=\{a_1<a_2<\cdots<a_m\}$ be a set of positive integers. Suppose that for every $1\le i\le m$ we have
\[
\frac{1}{a_i} > \sum_{j=i+1}^m \frac{1}{a_j}.
\]
Then the subset sums $\sum_{n\in S} \frac{1}{n}$ for $S\subseteq A$ are all distinct.

Proof.
Assume for contradiction that there exist distinct subsets $S,T\subseteq A$ with
\[\sum_{n\in S}\frac1n=\sum_{n\in T}\frac1n.\]
Let $i$ be the smallest index such that $a_i\in S\triangle T$ (symmetric difference).
Without loss of generality, $a_i\in S\setminus T$.
Then
\[
0 = \sum_{n\in S}\frac1n-\sum_{n\in T}\frac1n
= \frac{1}{a_i} + \sum_{j>i,\ a_j\in S\setminus T}\frac{1}{a_j} - \sum_{j>i,\ a_j\in T\setminus S}\frac{1}{a_j}.
\]
Rearranging gives
\[
\frac{1}{a_i}
= \sum_{j>i,\ a_j\in T\setminus S}\frac{1}{a_j} - \sum_{j>i,\ a_j\in S\setminus T}\frac{1}{a_j}
\le \sum_{j>i}\frac{1}{a_j}.
\]
This contradicts the assumed strict inequality $\frac1{a_i} > \sum_{j>i}\frac1{a_j}$.
Therefore no such $S,T$ exist, and all subset sums are distinct. \qed


Lemma 321.2 (explicit construction: powers of $2$).

Let $m\ge 1$ and $A_m:=\{1,2,4,\dots,2^{m-1}\}$. Then $A_m$ is reciprocal-subset-sum distinct.
Consequently, for every $N\ge 1$,
\[
R(N)\ge \bigl\lfloor \log_2 N\bigr\rfloor+1.
\]

Proof.
Write $A_m=\{a_1<\cdots<a_m\}$ with $a_i=2^{i-1}$. For $1\le i\le m$,
\[
\sum_{j=i+1}^m \frac{1}{a_j} = \sum_{t=1}^{m-i} \frac{1}{2^{i-1+t}} = \frac{1}{2^{i}}\sum_{t=0}^{m-i-1}\frac{1}{2^{t}}
=\frac{1}{2^{i}}\cdot\frac{1-(1/2)^{m-i}}{1-1/2}
=\frac{1}{2^{i-1}}\left(1-\frac{1}{2^{m-i}}\right)
<\frac{1}{2^{i-1}}=\frac{1}{a_i}.
\]
So $A_m$ satisfies the hypothesis of Lemma 321.1 and is reciprocal-subset-sum distinct.

If $2^{m-1}\le N$, then $A_m\subseteq\{1,\dots,N\}$, hence $R(N)\ge m$.
The largest such $m$ is $m=\lfloor\log_2 N\rfloor+1$. \qed


Lemma 321.3 (a denominator-counting upper bound).

Let $A\subseteq\{1,\dots,N\}$ be reciprocal-subset-sum distinct. Let $L:=\mathrm{lcm}(A)$ and let $H_N:=\sum_{n=1}^N \frac{1}{n}$.
Then
\[
2^{|A|}\le L\,H_N+1.
\]

Proof.
For any subset $S\subseteq A$, write
\[
\sigma_A(S)=\sum_{n\in S}\frac1n.
\]
Multiplying by $L$ gives
\[
L\sigma_A(S)=\sum_{n\in S}\frac{L}{n}.
\]
Because $L$ is a common multiple of all $n\in A$, each term $\frac{L}{n}$ is an integer, hence $L\sigma_A(S)$ is an integer.
Also $\sigma_A(S)\ge 0$ and
\[
\sigma_A(S)\le \sum_{n\in A}\frac1n\le \sum_{n=1}^N\frac1n=H_N.
\]
Therefore
\[
0\le L\sigma_A(S)\le L H_N.
\]
Thus $L\sigma_A(S)$ takes values in the integer set $\{0,1,2,\dots,\lfloor L H_N\rfloor\}$, which has size at most $L H_N+1$.
Since $A$ is reciprocal-subset-sum distinct, $\sigma_A$ is injective on $\mathcal P(A)$, so the $2^{|A|}$ values of $\sigma_A(S)$ are all distinct, and hence the $2^{|A|}$ integers $L\sigma_A(S)$ are all distinct.
So
\[
2^{|A|} \le L H_N+1.
\]
\qed


5) VERIFICATION

-- Lemma 321.1: the choice of the minimal index $i$ where $S$ and $T$ differ is valid since $A$ is finite and ordered.
The strict inequality hypothesis is used exactly once to contradict $\frac{1}{a_i}\le\sum_{j>i}1/a_j$.

-- Lemma 321.2: the tail-sum computation is a finite geometric series; the strict inequality holds because $1-2^{-(m-i)}<1$.

-- Lemma 321.3: all subset sums are rationals with denominator dividing $L$; multiplying by $L$ indeed gives integers.
The range bound uses $A\subseteq\{1,\dots,N\}$.

-- Small-$N$ computation: checked internally by exhaustive enumeration; the pattern $R(6)=5$ is consistent with the collision
$\frac12+\frac13+\frac16=1$ (so one cannot simultaneously include $\{1,2,3,6\}$).


6) FINAL

**UNRESOLVED**

(i) Strongest fully proved partial result obtained here.

* A clean sufficient condition (Lemma 321.1) and a concrete construction (Lemma 321.2) give
\[
R(N)\ge \lfloor\log_2 N\rfloor+1.
\]
* A general denominator-counting inequality (Lemma 321.3) gives
\[
2^{R(N)}\le \mathrm{lcm}(1,2,\dots,N)\,H_N+1
\]
(and hence an explicit, albeit weak, upper bound $R(N)=O(N\log N)$ using $\mathrm{lcm}(1,\dots,N)\le N!$).

(ii) Exact first gap.

Bridge the gap between the elementary bound $R(N)\ge c\log N$ proved here and the much larger bounds recorded in the problem statement (on the scale $N/\log N$ times iterated logs). In particular, I did not produce a construction achieving $R(N)\gg N/\log N$.

(iii) Top 3 next moves (concrete targets).

1. Reproduce (from scratch) the Bleicher--Erd\H{o}s lower-bound construction: give an explicit family $A\subseteq\{1,\dots,N\}$ of size $\gg N/\log N$ whose reciprocal subset sums are distinct.
2. Strengthen Lemma 321.3 by inserting a sharper estimate for the number of attainable numerators $L\sigma_A(S)$ (not just the crude interval bound), e.g. via spacing/Diophantine constraints.
3. Improve the small-$N$ search to $N\approx 40$--$50$ and look for a stable pattern (e.g. whether ``exclude exactly multiples of 6'' continues), which might suggest a constructive schema.

(iv) What a minimal counterexample would likely look like.

Any counterexample to a conjectured lower bound of the form $R(N)\gg N/\log N$ would have to show that every reciprocal-subset-sum-distinct set $A\subseteq\{1,\dots,N\}$ is forced to omit a positive fraction of integers in a structured way (e.g. by unavoidable Egyptian-fraction-type identities creating collisions among subset sums). The brute-force optimum sets up to $N=20$ already omit specific composite numbers (e.g. $6,12,15,18,20$), suggesting that ``local'' rational identities constrain membership.


