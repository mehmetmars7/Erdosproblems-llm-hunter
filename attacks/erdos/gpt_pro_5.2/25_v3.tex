\section{Round-4 Objective}
We pursue \textbf{(A) a proof strategy} by extending the Round~3 ``nested moduli'' theorem to a strictly larger structural class.
Round~3 established existence of \(\delta_{\log}(A)\) when \(n_1\mid n_2\mid n_3\mid\cdots\).
The mechanism was that, at stage \(k\), the relevant period equals \(n_k\), so the new constraint removes exactly one residue class inside one period, yielding a linear discrepancy bound and summable harmonic errors.

In this round we generalize this mechanism to cases where the stage period \(\mathrm{lcm}(n_1,\dots,n_k)\) may exceed \(n_k\), but only by a \emph{uniformly bounded} factor.
This rules out a broad class of potential counterexamples (including those with huge gaps \(n_{k+1}/n_k\) so that the Round~1/2 ``no single interval dominates'' hypothesis (H2) can fail).

\section{Round-3 Foundation Used}
We rely on the following Round~3 (and earlier) ingredients.
\begin{itemize}
\item \textbf{Stage sets and activation identity} (Round~1):
\(A^{(k)}:=\{n: n\not\equiv a_i\!\!\pmod{n_i}\ \forall i\le k\}\) and
\(\mathbf 1_A(n)=\mathbf 1_{A^{(k)}}(n)\) for \(n_k\le n<n_{k+1}\).
\item \textbf{Monotone convergence of stage densities} (Round~2): \(A^{(k+1)}\subseteq A^{(k)}\), hence \(\delta_k:=d(A^{(k)})\) is decreasing and converges to \(\delta:=\lim_k\delta_k\).
\item \textbf{Discrepancy method blueprint} (Round~3): define a discrepancy parameter for the residue set of a periodic stage-set, prove a linear-in-\(k\) discrepancy recursion under a ``delete \(O(1)\) points per period'' structure, and use Abel summation to convert discrepancy bounds into harmonic-sum error bounds.
\end{itemize}

\section{New Insight / Tool (Round-4)}
\subsection*{(i) A bounded-lcm structural hypothesis that preserves the Round~3 mechanism}
Let
\begin{equation}
L_k := \mathrm{lcm}(n_1,\dots,n_k).
\tag{4.1}
\end{equation}
We introduce the hypothesis:
\begin{equation}
\exists C\ge 1\ \forall k\ge 1,\qquad \frac{L_k}{n_k}\le C.
\tag{LC}
\end{equation}
Since \(n_k\mid L_k\), this is equivalent to writing
\(L_k=d_k n_k\) with integers \(1\le d_k\le C\).
Under (LC), the \(k\)-th excluded residue class \(a_k\pmod{n_k}\) occupies exactly \(d_k\le C\) residues in one period \(\bmod L_k\).
Thus each new stage deletes only \(O_C(1)\) residues per period (not necessarily just one), and we can re-run the Round~3 discrepancy recursion.

\subsection*{(ii) Harmonic sampling from an \emph{arbitrary} starting point}
Round~3 used a harmonic averaging lemma starting at a period boundary.
Here the activation interval starts at \(n_k\), which need not be a multiple of \(L_k\).
We prove a shifted version: for any periodic set with discrepancy \(\Delta\), the harmonic sum over \([X,Y)\) differs from \(\delta\log(Y/X)\) by \(O(\Delta/X)\).

\subsection*{(iii) (LC) forces geometric growth}
A crucial bonus: (LC) implies a uniform growth factor
\(n_{k+1}\ge (1+1/C)n_k\).
Hence \(\sum_k k/n_k\) converges, so the accumulated harmonic errors are bounded.

\section{Attack Plan (Round-4)}
The remaining obstacle after Round~3 was that, in general, the stage period is \(L_k\), which may be much larger than \(n_k\), preventing the ``one deletion per period'' discrepancy recursion.
Under (LC) we recover this deletion structure with \(\le C\) deletions per period.

Concretely we will:
\begin{enumerate}
\item Work modulo \(L_k\), defining the allowed residue set \(R_k\subset \{0,\dots,L_k-1\}\).
\item Prove a recursion: \(R_k\) is obtained from the lifted set for stage \(k-1\) by deleting at most \(d_k\le C\) residues.
\item Deduce a discrepancy recursion \(\Delta_k\le \Delta_{k-1}+O(C)\), hence \(\Delta_k=O_C(k)\).
\item Prove the shifted harmonic averaging lemma with error \(O(\Delta_k/n_k)\).
\item Sum the errors across stages (using geometric growth) and conclude
\(S(x)/\log x\to \delta\), i.e. \(\delta_{\log}(A)\) exists.
\end{enumerate}

\section{Work (Round-4)}
\subsection{Setup and residue sets modulo \texorpdfstring{$L_k$}{Lk}}
Assume \(n_1\ge 1\) and \(n_1<n_2<\cdots\), with residue classes \(a_i\pmod{n_i}\), and define \(A\) as in Round~1.
If \(n_1=1\) then \(A=\varnothing\) (since every \(n\ge 1\) satisfies \(n\equiv a_1\pmod 1\)), hence \(\delta_{\log}(A)=0\).
So we henceforth assume
\begin{equation}
 n_1\ge 2.
\tag{4.2}
\end{equation}

For each \(k\ge 1\), define \(L_k\) by \eqref{4.1} and the residue set
\begin{equation}
R_k := \bigl\{r\in\{0,1,\dots,L_k-1\}: \forall i\le k,\ r\not\equiv a_i\pmod{n_i}\bigr\}.
\tag{4.3}
\end{equation}
Then \(A^{(k)}\) is periodic modulo \(L_k\) with
\begin{equation}
\mathbf 1_{A^{(k)}}(n)=\mathbf 1_{R_k}(n\bmod L_k).
\tag{4.4}
\end{equation}
Let
\begin{equation}
\delta_k := \frac{|R_k|}{L_k}=d(A^{(k)}).
\tag{4.5}
\end{equation}
By Round~2, \((\delta_k)\) decreases and converges to \(\delta\in[0,1]\).

\subsection{Discrepancy and its recursion under (LC)}
For a set \(R\subseteq\{0,\dots,q-1\}\) of density \(\delta=|R|/q\), define its (prefix) discrepancy
\begin{equation}
\Delta(R;q) := \max_{0\le t\le q}\Bigl|\,|R\cap\{0,1,\dots,t-1\}|-\delta t\,\Bigr|.
\tag{4.6}
\end{equation}
For stage \(k\) we write
\begin{equation}
\Delta_k := \Delta(R_k;L_k).
\tag{4.7}
\end{equation}

\begin{lemma}[Lift-and-delete structure under (LC)]
Assume (LC), so \(L_k=d_k n_k\) with \(1\le d_k\le C\).
Let
\begin{equation}
\widetilde R_{k-1} := \{r\in\{0,\dots,L_k-1\}: r\bmod L_{k-1}\in R_{k-1}\}
\tag{4.8}
\end{equation}
be the lift of \(R_{k-1}\) to modulus \(L_k\).
Then
\begin{equation}
R_k = \widetilde R_{k-1}\setminus E_k,
\tag{4.9}
\end{equation}
where
\begin{equation}
E_k \subseteq \{r\in\{0,\dots,L_k-1\}: r\equiv a_k\pmod{n_k}\}
\quad\text{and}\quad |E_k|\le d_k\le C.
\tag{4.10}
\end{equation}
Moreover, \(\widetilde R_{k-1}\) is a disjoint union of \(L_k/L_{k-1}\) translates of \(R_{k-1}\) by multiples of \(L_{k-1}\), so
\begin{equation}
\Delta(\widetilde R_{k-1};L_k)=\Delta(R_{k-1};L_{k-1})=\Delta_{k-1}.
\tag{4.11}
\end{equation}
\end{lemma}

\begin{proof}
Because \(L_{k-1}\mid L_k\), membership in the first \(k-1\) congruence conditions depends only on \(r\bmod L_{k-1}\), which gives \eqref{4.8}.
Imposing the additional condition \(r\not\equiv a_k\pmod{n_k}\) removes from \(\widetilde R_{k-1}\) those residues congruent to \(a_k\pmod{n_k}\); this yields \eqref{4.9}--\eqref{4.10}.
Since \(n_k\mid L_k\), the congruence \(r\equiv a_k\pmod{n_k}\) has exactly \(L_k/n_k=d_k\) solutions modulo \(L_k\), so \(|E_k|\le d_k\le C\).

Finally, \(\widetilde R_{k-1}\) is obtained by repeating the pattern of \(R_{k-1}\) in consecutive blocks of length \(L_{k-1}\), hence its prefix discrepancy equals that of \(R_{k-1}\), giving \eqref{4.11}.
\end{proof}

\begin{lemma}[Linear discrepancy bound under (LC)]
Assume (LC). Then
\begin{equation}
\Delta_k \le \Delta_{k-1}+2C\qquad(k\ge 2).
\tag{4.12}
\end{equation}
In particular, \(\Delta_k\ll_C k\).
\end{lemma}

\begin{proof}
Work in modulus \(L_k\).
By Lemma~\eqref{4.9}, \(R_k\) is obtained from \(\widetilde R_{k-1}\) by deleting at most \(C\) points.
Hence for every prefix length \(t\) we have
\begin{equation}
\bigl|\,|R_k\cap[0,t)|-|\widetilde R_{k-1}\cap[0,t)|\,\bigr|\le C.
\tag{4.13}
\end{equation}
Also, deleting at most \(C\) points changes the density by at most \(C/L_k\):
\begin{equation}
|\delta_k-\delta(\widetilde R_{k-1})|\le \frac{C}{L_k}.
\tag{4.14}
\end{equation}
Therefore, for every \(t\in[0,L_k]\),
\begin{align}
\Bigl|\,|R_k\cap[0,t)|-\delta_k t\,\Bigr|
&\le \Bigl|\,|\widetilde R_{k-1}\cap[0,t)|-\delta(\widetilde R_{k-1})t\,\Bigr| \\
&\quad + \bigl|\,|R_k\cap[0,t)|-|\widetilde R_{k-1}\cap[0,t)|\,\bigr| \\
&\quad + |\delta_k-\delta(\widetilde R_{k-1})|\,t \\
&\le \Delta(\widetilde R_{k-1};L_k)+C+\frac{C}{L_k}\,L_k.
\end{align}
Using \eqref{4.11} gives
\(\Delta_k\le \Delta_{k-1}+2C\), which is \eqref{4.12}.
Iterating yields \(\Delta_k\ll_C k\).
\end{proof}

\subsection{(LC) forces geometric growth}
\begin{lemma}[Uniform growth factor]
Assume (LC). Then for every \(k\ge 1\),
\begin{equation}
\frac{n_{k+1}}{n_k}\ge 1+\frac{1}{C}.
\tag{4.15}
\end{equation}
Consequently \(n_k\gg_C (1+1/C)^k\) and
\begin{equation}
\sum_{k\ge 1}\frac{k}{n_k}<\infty.
\tag{4.16}
\end{equation}
\end{lemma}

\begin{proof}
Since \(L_{k+1}\ge \mathrm{lcm}(n_k,n_{k+1})=\dfrac{n_k n_{k+1}}{\gcd(n_k,n_{k+1})}\) and (LC) gives \(L_{k+1}\le C n_{k+1}\), we obtain
\begin{equation}
\frac{n_k}{\gcd(n_k,n_{k+1})}\le C,
\qquad\text{so}\qquad \gcd(n_k,n_{k+1})\ge \frac{n_k}{C}.
\tag{4.17}
\end{equation}
Write \(n_k=d u\) and \(n_{k+1}=d v\) with \(d=\gcd(n_k,n_{k+1})\) and \(\gcd(u,v)=1\).
Then \eqref{4.17} gives \(u\le C\).
Since \(n_{k+1}>n_k\), we have \(v>u\), hence \(v\ge u+1\).
Therefore
\[\frac{n_{k+1}}{n_k}=\frac{v}{u}\ge \frac{u+1}{u}\ge 1+\frac{1}{C},\]
proving \eqref{4.15}.
Geometric growth follows by iteration.
Finally, \eqref{4.16} holds because \(\sum k r^{-k}\) converges for every \(r>1\).
\end{proof}

\subsection{Shifted harmonic averaging for periodic sets}
\begin{lemma}[Shifted harmonic averaging via discrepancy]
Let \(B\subseteq\mathbb N\) be periodic with period \(q\ge 1\), and let \(\delta\) be its (natural) density.
Let \(R\subseteq\{0,\dots,q-1\}\) be the corresponding residue set (so \(n\in B\iff n\bmod q\in R\)), and let \(\Delta:=\Delta(R;q)\).
Then for all real \(1\le X<Y\),
\begin{equation}
\sum_{X\le n<Y}\frac{\mathbf 1_B(n)}{n}
=\delta\sum_{X\le n<Y}\frac{1}{n}+O\!\left(\frac{\Delta}{X}\right).
\tag{4.18}
\end{equation}
Equivalently,
\begin{equation}
\sum_{X\le n<Y}\frac{\mathbf 1_B(n)}{n}
=\delta\log\frac{Y}{X}+O\!\left(\frac{\Delta}{X}\right)+O\!\left(\frac{1}{X}\right).
\tag{4.19}
\end{equation}
The implied constants are absolute.
\end{lemma}

\begin{proof}
Define the centered periodic function
\(g(n):=\mathbf 1_B(n)-\delta\).
Let
\(G(t):=\sum_{0\le n<t} g(n)\).
By definition of discrepancy, for \(0\le t\le q\) we have \(|G(t)|\le \Delta\).
Also \(G(t+q)=G(t)\) because \(\sum_{0\le n<q}g(n)=|R|-\delta q=0\).
Thus \(|G(t)|\le \Delta\) for all integers \(t\ge 0\).

For integers \(m\ge X\), define partial sums from \(X\):
\(H(m):=\sum_{X\le n\le m} g(n)\).
Then
\(H(m)=G(m+1)-G(X)\), so
\begin{equation}
|H(m)|\le 2\Delta\qquad(m\ge X).
\tag{4.20}
\end{equation}

Apply Abel summation:
\begin{align}
\sum_{X\le n<Y}\frac{g(n)}{n}
&=\frac{H(\lfloor Y\rfloor-1)}{\lfloor Y\rfloor}
+\sum_{m=\lceil X\rceil}^{\lfloor Y\rfloor-2} H(m)\left(\frac{1}{m}-\frac{1}{m+1}\right).
\end{align}
Using \eqref{4.20} and \(\sum_{m\ge X}(1/m-1/(m+1))=1/X\) gives
\[\left|\sum_{X\le n<Y}\frac{g(n)}{n}\right|\ll \frac{\Delta}{X}.
\]
Since \(\mathbf 1_B=\delta+g\), this is \eqref{4.18}. The conversion to \eqref{4.19} uses
\(\sum_{X\le n<Y}1/n=\log(Y/X)+O(1/X)\).
\end{proof}

\subsection{Existence of logarithmic density under bounded lcm-factor}
\begin{theorem}[Bounded lcm-factor case]
Assume \(n_1\ge 2\) and that (LC) holds, i.e.
\(\mathrm{lcm}(n_1,\dots,n_k)\le C n_k\) for all \(k\).
Then the logarithmic density \(\delta_{\log}(A)\) exists, and
\begin{equation}
\delta_{\log}(A)=\delta:=\lim_{k\to\infty}\delta_k,
\tag{4.21}
\end{equation}
where \(\delta_k=d(A^{(k)})\).
\end{theorem}

\begin{proof}
Let
\(S(x):=\sum_{\substack{1\le n\le x\\ n\in A}}\frac{1}{n}\).
Fix large \(x\) and let \(K=K(x)\) be the unique index with \(n_K\le x<n_{K+1}\).
By the activation identity (Round~1), for \(n_k\le n<n_{k+1}\) we have
\(\mathbf 1_A(n)=\mathbf 1_{A^{(k)}}(n)\).
Hence
\begin{equation}
S(x)=\sum_{n< n_1}\frac{1}{n}
+\sum_{k=1}^{K-1}\sum_{n=n_k}^{n_{k+1}-1}\frac{\mathbf 1_{A^{(k)}}(n)}{n}
+\sum_{n=n_K}^{\lfloor x\rfloor}\frac{\mathbf 1_{A^{(K)}}(n)}{n}.
\tag{4.22}
\end{equation}

We approximate each inner sum by \(\delta_k\log(n_{k+1}/n_k)\).
Apply Lemma~\eqref{4.18} to the periodic set \(A^{(k)}\), whose period is \(L_k\) and whose residue set is \(R_k\).
With \(X=n_k\) and \(Y=n_{k+1}\) we obtain
\begin{equation}
\sum_{n=n_k}^{n_{k+1}-1}\frac{\mathbf 1_{A^{(k)}}(n)}{n}
=\delta_k\sum_{n=n_k}^{n_{k+1}-1}\frac{1}{n}
+O\!\left(\frac{\Delta_k}{n_k}\right).
\tag{4.23}
\end{equation}
Similarly, for the final partial block,
\begin{equation}
\sum_{n=n_K}^{\lfloor x\rfloor}\frac{\mathbf 1_{A^{(K)}}(n)}{n}
=\delta_K\sum_{n=n_K}^{\lfloor x\rfloor}\frac{1}{n}
+O\!\left(\frac{\Delta_K}{n_K}\right).
\tag{4.24}
\end{equation}
By Lemma~\eqref{4.12}, \(\Delta_k\ll_C k\). By Lemma~\eqref{4.16}, \(\sum k/n_k<\infty\). Therefore
\begin{equation}
\sum_{k\ge 1}\frac{\Delta_k}{n_k}<\infty,
\tag{4.25}
\end{equation}
so the total contribution of the error terms in \eqref{4.23}--\eqref{4.24} is bounded uniformly in \(x\).
Using \(\sum_{n=u}^{v}1/n=\log(v/u)+O(1/u)\) and absorbing the bounded error, \eqref{4.22} becomes
\begin{equation}
S(x)=O(1)
+\sum_{k=1}^{K-1}\delta_k\log\frac{n_{k+1}}{n_k}
+\delta_K\log\frac{x}{n_K}.
\tag{4.26}
\end{equation}
Divide by \(\log x\):
\begin{equation}
\frac{S(x)}{\log x}
=\sum_{k=1}^{K-1}\frac{\log(n_{k+1}/n_k)}{\log x}\,\delta_k
+\frac{\log(x/n_K)}{\log x}\,\delta_K
+o(1).
\tag{4.27}
\end{equation}
The weights in \eqref{4.27} are nonnegative. Their total mass is
\[
W(x):=\sum_{k=1}^{K-1}\frac{\log(n_{k+1}/n_k)}{\log x}+\frac{\log(x/n_K)}{\log x}
=\frac{\log(x/n_1)}{\log x}
=1+O\!\left(\frac{1}{\log x}\right).
\]
Define normalized weights \(\widetilde w_k:=w_k/W(x)\), so that \(\sum_{k=1}^{K}\widetilde w_k=1\).
Then \eqref{4.27} may be rewritten as
\[
\frac{S(x)}{\log x}=W(x)\sum_{k=1}^{K}\widetilde w_k\,\delta_k+o(1).
\]
Since \(W(x)\to 1\), it suffices to show \(\sum_{k=1}^{K}\widetilde w_k\delta_k\to\delta\).
Fix \(\varepsilon>0\) and choose \(K_0\) with \(\delta_{K_0}-\delta<\varepsilon\).
Let \(\alpha(x):=\sum_{k< K_0}\widetilde w_k\). Then
\[
\alpha(x)\le \frac{1}{W(x)}\,\frac{\log(n_{K_0}/n_1)}{\log x}\longrightarrow 0\qquad(x\to\infty).
\]
For \(k\ge K_0\) we have \(\delta\le \delta_k\le \delta_{K_0}\), hence
\[
\delta\,(1-\alpha(x))\le \sum_{k=1}^{K}\widetilde w_k\delta_k
\le \alpha(x)+\delta_{K_0}(1-\alpha(x))=\delta_{K_0}+\alpha(x)(1-\delta_{K_0}).
\]
Taking \(x\to\infty\) and then \(\varepsilon\to 0\) gives \(\sum_{k=1}^{K}\widetilde w_k\delta_k\to\delta\), and therefore \(S(x)/\log x\to\delta\), proving \eqref{4.21}.

\end{proof}

\subsection{Examples and scope}
\paragraph{(LC) is strictly more general than the nested case.}
Round~3 corresponds to \(C=1\) (since then \(L_k=n_k\) for all \(k\)).
For \(C>1\) there are many non-nested examples.
For instance, fix an odd integer sequence \(M_j\uparrow\infty\) with \(M_{j+1}\gg M_j\) (e.g. \(M_j=10^{10^j}\)) and define
\[n_{2j-1}:=3M_j,\qquad n_{2j}:=6M_j.\]
Then \(n_{2j-1}\nmid n_{2j-2}\) (so the moduli are not nested), while
\(\mathrm{lcm}(n_1,\dots,n_{2j-1})=6M_j=2n_{2j-1}\) and
\(\mathrm{lcm}(n_1,\dots,n_{2j})=6M_j=n_{2j}\), so (LC) holds with \(C=2\).
The ratios \(n_{2j-1}/n_{2j-2}\) can be made enormous, so the Round~1/2 hypothesis (H2) can fail badly even though \(\delta_{\log}(A)\) exists by Theorem~\eqref{4.21}.

\section{Adversarial Verification}
We stress-test the new argument.
\begin{itemize}
\item \textbf{Edge case \(n_1=1\).} Handled: then \(A=\varnothing\) and \(\delta_{\log}(A)=0\).
\item \textbf{Correctness of the lift step.} The lift \(\widetilde R_{k-1}\) in \eqref{4.8} uses only that \(L_{k-1}\mid L_k\), which always holds by definition of lcm.
\item \textbf{Size of the deleted set.} Because \(n_k\mid L_k\), the congruence class \(a_k\pmod{n_k}\) has exactly \(L_k/n_k=d_k\) representatives in \([0,L_k)\). Under (LC), \(d_k\le C\), so deletion is indeed \(O_C(1)\).
\item \textbf{Discrepancy recursion.} The only place where ``deleting \(\le C\) points'' is used is \eqref{4.13} and the density-shift bound \eqref{4.14}. Both are sharp and unconditional.
\item \textbf{Shifted harmonic averaging lemma.} The periodicity of the centered prefix sum \(G(t)\) is checked explicitly; no hidden assumption about starting at a period boundary is used.
\item \textbf{Summability of errors.} Theorem~\eqref{4.21} uses \(\sum \Delta_k/n_k<\infty\). This is guaranteed because \(\Delta_k\ll_C k\) and (LC) forces geometric growth of \(n_k\). The growth lemma \eqref{4.15} is derived from a clean gcd argument.
\item \textbf{Quantifier check.} The conclusion is a full limit as \(x\to\infty\), not only along endpoints \(x=n_K\). This is ensured because we control the final partial interval \([n_K,x]\) via \eqref{4.24}.
\end{itemize}

\section{Final}
\textbf{UNRESOLVED (BUT STRICTLY ADVANCED).}

Beyond Round~3 (nested moduli), we have proved a strictly stronger unconditional existence theorem:
\begin{quote}
If the stage lcm \(\mathrm{lcm}(n_1,\dots,n_k)\) is always within a bounded factor of the current modulus \(n_k\), then \(\delta_{\log}(A)\) exists.
\end{quote}
Equivalently, we rule out counterexamples in the regime where each new constraint deletes only \(O(1)\) residues in one stage period.

The general open case still hinges on controlling the accumulated bias terms when \(L_k/n_k\) is unbounded.

\section{Completion Estimate}
\textbf{COMPLETION: 78\%}.

\section{References}
\begin{itemize}
\item Erdos problems website, Problem 25 (accessed Jan.~2026): \texttt{https://www.erdosproblems.com/25}.
\item Erdos problems website, Problem 486 (accessed Jan.~2026): \texttt{https://www.erdosproblems.com/486}.
\end{itemize}
