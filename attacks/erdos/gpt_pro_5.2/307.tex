\section*{Erd\H{o}s Problem \#307}

\subsection*{FORMAL RESTATEMENT}
Find (or rule out) finite sets of primes $P,Q$ such that
\[
\left(\sum_{p\in P}\frac{1}{p}\right)\left(\sum_{q\in Q}\frac{1}{q}\right)=1.
\]
Here $P,Q$ are \emph{sets} (no repetition), and all primes are positive.

\textbf{Stress points / ambiguity checks.}
\begin{itemize}
\item If a prime appears in both $P$ and $Q$, then the $p$-adic valuation of the product is $-2$, so equality to $1$ is impossible; hence any solution must have $P\cap Q=\varnothing$ (proved below).
\item The equation is \emph{verifiable}: one explicit example would settle it affirmatively.
\item It is helpful to repackage the equation in an integral form using products of primes (also done below).
\end{itemize}

\subsection*{QUICK LITERATURE/CONTEXT CHECK}
A quick web check (Erd\H{o}s Problems site; MathOverflow) indicates that the existence of such prime sets is currently listed as open; weakened coprime versions allowing $1\in P\cup Q$ have explicit examples, but no example is known with all denominators prime. (See the sources cited in the accompanying chat message.)

\subsection*{ATTACK PLAN}
\begin{enumerate}
\item Derive \emph{structural necessities}: disjointness; an exact integer system relating the products $\prod P$ and $\prod Q$; size lower bounds.
\item Recast the problem as a search for a 2-cycle of a multiplicative arithmetic map $g$ on squarefree integers. This gives a crisp computational/structural target.
\item (Unfinished) Attempt constructions/obstructions for 2-cycles of $g$.
\end{enumerate}

\subsection*{WORK}
Let
\[
A:=\sum_{p\in P}\frac1p,\qquad B:=\sum_{q\in Q}\frac1q,
\]
so the desired identity is $AB=1$.

\medskip
\textbf{Lemma 1 (Disjointness).} If $AB=1$ with $P,Q$ sets of primes, then $P\cap Q=\varnothing$.

\emph{Proof.}
Assume for contradiction that some prime $r\in P\cap Q$. Write
\[
A=\frac{1}{r}+u,\qquad B=\frac{1}{r}+v
\]
where $u,v$ are sums of reciprocals of primes $\neq r$. Since $r\nmid$ denominators in $u$, we have $u\in\mathbb Z_r$ (the $r$-adic integers), and similarly $v\in\mathbb Z_r$.
Then
\[
A=\frac{1+ru}{r},\quad B=\frac{1+rv}{r},
\]
and $1+ru\equiv 1\pmod r$, so $v_r(1+ru)=0$, hence $v_r(A)=-1$; similarly $v_r(B)=-1$. Therefore
\[
v_r(AB)=v_r(A)+v_r(B)=-2\neq 0=v_r(1),
\]
contradiction. \qed

\medskip
\textbf{Lemma 2 (Reduced-form denominators).}
For any finite set $P$ of primes, if we set $D_P:=\prod_{p\in P}p$ and
\[
N_P:=\sum_{p\in P}\frac{D_P}{p}\in\mathbb Z,
\]
then
\[
\sum_{p\in P}\frac1p=\frac{N_P}{D_P}\quad\text{and}\quad \gcd(N_P,D_P)=1.
\]

\emph{Proof.}
The displayed fraction identity is just common-denominator addition. For $\gcd$,
fix $r\in P$. Modulo $r$, every term $D_P/p$ with $p\neq r$ is divisible by $r$,
while the term $D_P/r$ is \emph{not} divisible by $r$. Hence $N_P\not\equiv 0\pmod r$.
As this holds for every $r\in P$, no prime divisor of $D_P$ divides $N_P$, so $\gcd(N_P,D_P)=1$. \qed

\medskip
\textbf{Corollary 3 (Integer ``amicable'' form).}
Assume $AB=1$ with $P\cap Q=\varnothing$.
Let $D:=D_P=\prod_{p\in P}p$ and $E:=D_Q=\prod_{q\in Q}q$.
Then necessarily
\[
N_P=E\qquad\text{and}\qquad N_Q=D,
\]
i.e.
\begin{equation}\label{eq:amicable}
E=\sum_{p\in P}\frac{D}{p},\qquad
D=\sum_{q\in Q}\frac{E}{q}.
\end{equation}

\emph{Proof.}
By Lemma 2 we may write
\[
A=\frac{N_P}{D},\qquad B=\frac{N_Q}{E}
\]
with $\gcd(N_P,D)=\gcd(N_Q,E)=1$. The equation $AB=1$ becomes
\[
N_PN_Q=DE.
\]
Since $P$ and $Q$ are disjoint, $\gcd(D,E)=1$; also $\gcd(N_P,D)=1$ implies every prime
dividing $D$ must divide $N_Q$, hence $D\mid N_Q$. Similarly $E\mid N_P$.
Write $N_Q=D\cdot t$ and $N_P=E\cdot s$ with $s,t\in\mathbb Z_{>0}$.
Then $N_PN_Q=DE\cdot st=DE$ forces $st=1$, hence $s=t=1$, i.e. $N_P=E$ and $N_Q=D$.
Substituting back yields \eqref{eq:amicable}. \qed

\medskip
\textbf{Reformulation via an arithmetic map.}
For squarefree $n$ define
\[
g(n):=\sum_{p\mid n}\frac{n}{p}.
\]
If $n=\prod_{i=1}^k p_i$ then $g(n)=\sum_{i=1}^k \prod_{j\neq i} p_j$.
By Lemma 2, $\gcd(n,g(n))=1$ for squarefree $n$.

If \eqref{eq:amicable} holds, then with $D=\prod_{p\in P}p$ and $E=\prod_{q\in Q}q$
we have $E=g(D)$ and $D=g(E)$. Thus Problem \#307 is equivalent to asking whether
\emph{$g$ has a 2-cycle} on squarefree integers: does there exist squarefree $D\neq 1$
such that $E:=g(D)$ is squarefree and $g(E)=D$?

\medskip
\textbf{A simple size obstruction.}
From $AB=1$ we have $A+B\ge 2$ (AM--GM), hence
\[
\sum_{r\in P\cup Q}\frac1r = A+B \ge 2.
\]
Since the reciprocal sum is maximized by taking the smallest primes, one obtains a
lower bound on $|P\cup Q|$.
A direct computation gives
\[
\sum_{i=1}^{58}\frac{1}{p_i}\approx 1.9987400431<2
\quad\text{and}\quad
\sum_{i=1}^{59}\frac{1}{p_i}\approx 2.0023501514>2,
\]
where $p_i$ denotes the $i$th prime. Therefore any solution must have
\[
|P\cup Q|\ge 59.
\]

\subsection*{VERIFICATION (adversarial proof check; stress-test edge cases)}
\begin{itemize}
\item Lemma 1: The key check is $u,v\in\mathbb Z_r$; this holds because each term $1/p$ with $p\neq r$ is an $r$-adic integer. Also $1+ru\equiv 1\pmod r$ ensures $v_r(A)=-1$ with no cancellation.
\item Lemma 2: The congruence argument mod $r$ is robust because exactly one term survives mod $r$.
\item Corollary 3: The divisibility deductions use coprimality $\gcd(N_P,D)=1$ and $\gcd(D,E)=1$; both are proved/obvious from Lemma 2 and disjointness.
\item The ``$|P\cup Q|\ge 59$'' bound uses numerical computation of prime reciprocal sums; it is not logically necessary for Corollary 3, but it is a correct necessary condition given the displayed computed inequalities.
\end{itemize}

\subsection*{FINAL}
\textbf{UNRESOLVED.}

\begin{itemize}
\item \textbf{What is proved here.} Any solution must have $P\cap Q=\varnothing$, and with
$D=\prod_{p\in P}p$, $E=\prod_{q\in Q}q$ it must satisfy the \emph{exact} integer system
\[
E=\sum_{p\in P}\frac{D}{p},\qquad D=\sum_{q\in Q}\frac{E}{q},
\]
equivalently $E=g(D)$ and $D=g(E)$ for $g(n)=\sum_{p\mid n}n/p$ on squarefree $n$.
Also necessarily $|P\cup Q|\ge 59$.
\item \textbf{Key obstacle.} No method here constructs a 2-cycle of $g$ (and none is ruled out).
\item \textbf{Most promising next step.} Use the 2-cycle formulation to guide a targeted search:
seek squarefree $D$ with many prime factors such that $g(D)$ is squarefree and $g(g(D))=D$;
try to incorporate congruence constraints and size heuristics to prune.
\item \textbf{Deliverables if continuing.} Either an explicit $(P,Q)$ (settling the problem affirmatively),
or a structural obstruction proving $g$ has no 2-cycles on squarefree integers.
\end{itemize}

\subsection*{COMPLETION ESTIMATE}
COMPLETION: 35\%

%%%%%%%%%%%%%%%%%%%%%%%%%%%%%%%%%%%%%%%%%%%%%%%%%%%%%%%%%%%%%%%%%%%%%%%%%%%%%%%
