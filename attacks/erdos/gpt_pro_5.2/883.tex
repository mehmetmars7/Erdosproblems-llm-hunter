
For $A\subseteq \{1,\ldots,n\}$ let $G(A)$ be the graph with vertex set $A$, where two integers are joined by an edge if they are coprime. Is it true that if\[\lvert A\rvert >\lfloor\tfrac{n}{2}\rfloor+\lfloor\tfrac{n}{3}\rfloor-\lfloor\tfrac{n}{6}\rfloor\]then $G(A)$ contains all odd cycles of length $\leq \frac{n}{3}+1$? Is it true that, for every $\ell\geq 1$, if $n$ is sufficiently large and\[\lvert A\rvert >\lfloor\tfrac{n}{2}\rfloor+\lfloor\tfrac{n}{3}\rfloor-\lfloor\tfrac{n}{6}\rfloor\]then $G(A)$ must contain a complete $(1,\ell,\ell)$ triparite graph on $2\ell+1$ vertices? A problem of Erd\H{o}s and S\'{a}rk\H{o}zy \cite{ErSa97}, who prove that if\[\lvert A\rvert >\lfloor\tfrac{n}{2}\rfloor+\lfloor\tfrac{n}{3}\rfloor-\lfloor\tfrac{n}{6}\rfloor\]then $G(A)$ contains all odd cycles of length $\leq cn$ for some constant $c>0$. This threshold is the best possible, since one could take $A$ to be the set of $m\leq n$ which are divisible by either $2$ or $3$, in which case $G(A)$ contains no triangles. The second question was solved by S\'{a}rk\"{o}zy \cite{Sa99} who proved that, for large $n$, if $\lvert A\rvert$ exceeds the given threshold then $G(A)$ contains a complete $(1,\ell,\ell)$ triparite graph with\[\ell \gg \frac{\log n}{\log\log n}.\]

1) FORMAL RESTATEMENT

Let $n\in\mathbb N$ and let $A\subseteq\{1,2,\dots,n\}$. Define a simple graph $G(A)$ with vertex set $A$ and an (undirected) edge $\{x,y\}$ for distinct $x,y\in A$ iff $\gcd(x,y)=1$.
Define
\[T(n):=\Big\lfloor\frac n2\Big\rfloor+\Big\lfloor\frac n3\Big\rfloor-\Big\lfloor\frac n6\Big\rfloor.
\]

Question 1: Is it true that if $|A|>T(n)$ then $G(A)$ contains, for every odd integer $L$ with $3\le L\le \lfloor n/3+1\rfloor$, a (simple) cycle of length $L$ as a subgraph?

Question 2: For each fixed $\ell\ge 1$, is it true that for all sufficiently large $n$, every $A\subseteq\{1,\dots,n\}$ with $|A|>T(n)$ has $G(A)$ containing a complete tripartite subgraph with parts of sizes $(1,\ell,\ell)$ (denoted $K_{1,\ell,\ell}$), i.e. $2\ell+1$ vertices with all edges between different parts present?

2) QUICK LITERATURE/CONTEXT CHECK

The problem statement itself records:
- Erd\H{o}s--S\'{a}rk\H{o}zy (1997) proved that if $|A|>T(n)$ then $G(A)$ contains all odd cycles of length $\le c n$ for some absolute $c>0$.
- The extremal example $E:=\{m\le n: 2\mid m\text{ or }3\mid m\}$ has size $T(n)$ and yields no triangles.
- S\'{a}rk\"{o}zy (1999) proved that for large $n$ one gets a $K_{1,\ell,\ell}$ with $\ell\gg \log n/\log\log n$, which implies Question 2 is YES for each fixed $\ell$.

Per the integrity rules, I do not import any additional literature beyond what is explicitly written above.

3) ATTACK PLAN

- Prove basic structural facts about the threshold $T(n)$ and about the extremal construction $E$.
- Use computation to sanity-check Question 1 for small $n$ (exhaustively) and to sanity-check Question 2 for small $\ell$.
- Since the deep results cited are not proved here, treat the main asymptotic questions as unresolved, but record what can be rigorously deduced from the statements given.

4) WORK

Lemma 883.1 (The threshold counts integers divisible by $2$ or $3$).
For every $n\in\mathbb N$,
\[T(n)=\big|\{m\in\{1,\dots,n\}: 2\mid m\text{ or }3\mid m\}\big|.
\]

Proof.
Let $E_2:=\{m\le n: 2\mid m\}$ and $E_3:=\{m\le n: 3\mid m\}$. Then $|E_2|=\lfloor n/2\rfloor$ and $|E_3|=\lfloor n/3\rfloor$.
Their intersection is $E_2\cap E_3=\{m\le n: 6\mid m\}$ which has size $\lfloor n/6\rfloor$.
By inclusion--exclusion,
\[|E_2\cup E_3|=|E_2|+|E_3|-|E_2\cap E_3|=\Big\lfloor\frac n2\Big\rfloor+\Big\lfloor\frac n3\Big\rfloor-\Big\lfloor\frac n6\Big\rfloor=T(n).\]
\qed

Lemma 883.2 (The extremal set $E$ yields no odd cycles).
Let $E:=\{m\le n:2\mid m\text{ or }3\mid m\}$. Then $G(E)$ is bipartite, hence contains no odd cycle (in particular, no triangle).

Proof.
Partition $E$ into
\[X:=\{m\le n:2\mid m\}\quad\text{and}\quad Y:=\{m\le n: m\text{ odd and }3\mid m\}.
\]
Every vertex in $Y$ is divisible by $3$ and odd; every vertex in $X$ is divisible by $2$.

- There are no edges within $X$ because any two distinct even integers have gcd at least $2$.
- There are no edges within $Y$ because any two distinct multiples of $3$ have gcd at least $3$.

So every edge of $G(E)$ must go between $X$ and $Y$ (vertices divisible by $6$ are in $X$ and are isolated, since they share gcd $\ge2$ with all of $X$ and gcd $\ge3$ with all of $Y$).
Thus $G(E)$ is bipartite with bipartition $(X,Y)$ and has no odd cycles. \qed

Lemma 883.3 (Downward closure of $K_{1,\ell,\ell}$).
If a graph contains $K_{1,m,m}$ as a subgraph with $m\ge \ell$, then it contains $K_{1,\ell,\ell}$ as a subgraph.

Proof.
A $K_{1,m,m}$ consists of three disjoint vertex parts $\{v\}\cup B\cup C$ with $|B|=|C|=m$, and all edges between different parts present.
Choose any subsets $B'\subseteq B$ and $C'\subseteq C$ with $|B'|=|C'|=\ell$.
Then the induced subgraph on $\{v\}\cup B'\cup C'$ is exactly $K_{1,\ell,\ell}$. \qed

FAST REALITY CHECK (computation for Question 1).
An exhaustive brute-force check was run for $6\le n\le 22$:
for each $n$, for every $A\subseteq\{1,\dots,n\}$ with $|A|>T(n)$, the graph $G(A)$ contains all odd cycle lengths $L\le \lfloor n/3+1\rfloor$.
In particular:
- for $12\le n\le 17$ (where one must have $3$- and $5$-cycles), all such $G(A)$ contained both a triangle and a $5$-cycle;
- for $18\le n\le 22$ (where one must have $3$-, $5$-, and $7$-cycles), all such $G(A)$ contained $3$-, $5$-, and $7$-cycles.
No counterexample was found in this range.

FAST REALITY CHECK (computation for Question 2 at $\ell=2$).
Exhaustively for $6\le n\le 22$:
- For $n=6,7,8$, there exist sets $A$ with $|A|>T(n)$ but $G(A)$ does NOT contain $K_{1,2,2}$. One explicit example is $A=\{1,2,3,4,6\}$ for $n=6$.
- For every $n$ with $9\le n\le 22$ and every $A$ with $|A|>T(n)$, the graph $G(A)$ DOES contain $K_{1,2,2}$.

5) VERIFICATION

- Lemma 883.2: checked that vertices divisible by $6$ are indeed isolated inside $E$, so they do not create odd cycles across the bipartition.
- Computational checks were exhaustive over all subsets $A\subseteq\{1,\dots,n\}$ with $|A|>T(n)$ in the stated ranges, not randomized.
- For Question 2, the counterexample at $n=6$ is consistent with the “for sufficiently large $n$” quantifier in the question.

6) FINAL

**UNRESOLVED**

(i) Strongest proved partial result.
- The threshold $T(n)$ equals the size of the extremal set $E$ of integers divisible by $2$ or $3$ (Lemma 883.1).
- The extremal set $E$ yields a bipartite coprime graph, hence no odd cycles (Lemma 883.2), explaining sharpness.
- Exhaustive computation verified Question 1 for all $6\le n\le 22$.
- For Question 2, Lemma 883.3 shows that the (stated) existence of $K_{1,m,m}$ with $m\gg \log n/\log\log n$ implies an affirmative answer for each fixed $\ell$.

(ii) First gap (crisp).
Prove or disprove Question 1 in full:
\[\forall n\ \forall A\subseteq\{1,\dots,n\}\ (|A|>T(n)\Rightarrow G(A)\text{ contains a cycle of every odd length }L\le \lfloor n/3+1\rfloor).\]

(iii) Top 3 next moves.
1. Try to push the constant-$c$ result (odd cycles up to $c n$) toward the explicit target $n/3+1$ by a quantitative strengthening of the underlying counting/expansion argument.
2. Attempt to characterize (near-)extremal $A$ with $|A|=T(n)+1$ and determine forced structure of the additional element(s) not divisible by $2$ or $3$.
3. For potential disproof, attempt to construct large $A$ by taking $E$ and adding numbers all sharing a fixed large prime factor (to reduce coprime edges), and test by computation for larger $n$.

(iv) Minimal counterexample structure.
A minimal counterexample to Question 1 would likely have $|A|=T(n)+1$ (or close) and resemble $E$ with a small number of added elements not divisible by $2$ or $3$, chosen so that coprimality edges among the added elements and between added elements and $E$ are sparse enough to forbid some odd cycle length. Any such construction must still avoid creating short odd cycles (triangles, then $5$-cycles, etc.).


