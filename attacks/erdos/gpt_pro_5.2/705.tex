% Erdos Problem #705

1) \textbf{FORMAL RESTATEMENT}

Does there exist an integer $k\ge 3$ such that the following holds?

For every \emph{finite} set $P\subset \mathbb{R}^2$, let $G(P)$ be the graph with vertex set $P$ and an (undirected, simple) edge $\{x,y\}$ between distinct points $x,y\in P$ iff $\|x-y\|_2=1$.
If the girth of $G(P)$ is at least $k$ (i.e. $G(P)$ has no cycle of length $<k$; with the convention that an acyclic graph has girth $\infty$), must it follow that the chromatic number satisfies $\chi(G(P))\le 3$?

Edge cases/conventions:
- Graphs are finite, simple, undirected.
- Girth$(G)=\infty$ if $G$ has no cycles.

2) \textbf{QUICK LITERATURE/CONTEXT CHECK}

I will only use facts explicitly stated in the problem text.
The text notes:
- There are unit distance graphs with $\chi=4$ and girth $3$ (e.g. Moser spindle).
- de Grey constructed a unit distance graph with $\chi=5$.
- O'Donnell constructed a unit distance graph with $\chi=4$ and girth $4$.
- Wormald constructed a unit distance graph with $\chi=4$ and girth $5$.
Hence, if such a $k$ exists, it must satisfy $k\ge 6$.

3) \textbf{ATTACK PLAN}

\emph{Disproof strategy:} construct, for each $k$, a finite unit distance graph in $\mathbb{R}^2$ with girth $\ge k$ but $\chi\ge 4$.
This would show no such universal $k$ exists.

\emph{Proof strategy:} show that large girth forces enough geometric/graph sparsity (e.g. bounded degeneracy) to imply $3$-colorability for unit distance graphs.
A plausible approach would be to derive a structural constraint on unit distance graphs of large girth beyond purely abstract graph theory.

I do not see a route to complete either track here, so I record rigorous partial facts.

4) \textbf{WORK}

\textbf{Lemma 705.1 (No $K_4$ in a planar unit-distance representation).}
If $P\subset\mathbb{R}^2$ is any set of points, then the unit distance graph $G(P)$ has clique number at most $3$. Equivalently, there do not exist four distinct points in $\mathbb{R}^2$ with all pairwise distances equal to $1$.

\textit{Proof.}
Assume for contradiction that there exist distinct points $x_1,x_2,x_3,x_4\in\mathbb{R}^2$ with $\|x_i-x_j\|_2=1$ for all $i\ne j$.
Then $x_1,x_2,x_3$ form an equilateral triangle of side length $1$.
By applying a rigid motion of the plane, we may assume
\[
 x_1=(0,0),\qquad x_2=(1,0),\qquad x_3=\left(\tfrac12,\tfrac{\sqrt3}{2}\right).
\]
The condition $\|x_4-x_1\|_2=\|x_4-x_2\|_2=1$ means that $x_4$ lies in the intersection of the two circles of radius $1$ centered at $x_1$ and $x_2$.
Solving
\[
\|x_4-(0,0)\|_2^2=1,\qquad \|x_4-(1,0)\|_2^2=1
\]
shows this intersection consists of the two points
\[
\left(\tfrac12,\tfrac{\sqrt3}{2}\right)=x_3\quad\text{and}\quad\left(\tfrac12,-\tfrac{\sqrt3}{2}\right).
\]
Since $x_4$ is distinct from $x_3$, we must have $x_4=(\tfrac12,-\tfrac{\sqrt3}{2})$.
But then
\[
\|x_4-x_3\|_2^2=\left(\tfrac12-\tfrac12\right)^2+\left(-\tfrac{\sqrt3}{2}-\tfrac{\sqrt3}{2}\right)^2 = ( -\sqrt3 )^2 = 3,
\]
so $\|x_4-x_3\|_2=\sqrt3\ne 1$, contradicting the assumed pairwise unit distances.
Therefore no such 4-tuple exists, and $\omega(G(P))\le 3$.
\hfill$\square$

\textbf{Lemma 705.2 (Moore-type lower bound from girth).}
Let $G$ be a finite simple graph of minimum degree $\delta\ge 2$ and girth $g\ge 3$.
Write $g=2r+1$ (odd) or $g=2r$ (even), where $r\ge 1$.
Then
\[
|V(G)|\ge
\begin{cases}
1+\delta\sum_{i=0}^{r-1}(\delta-1)^i,& g=2r+1,\\
2\sum_{i=0}^{r-1}(\delta-1)^i,& g=2r.
\end{cases}
\]

\textit{Proof.}
Fix a vertex $v\in V(G)$.
Consider a breadth-first search (BFS) exploration of $G$ starting from $v$.
Let $L_i$ be the set of vertices at graph distance exactly $i$ from $v$.

\underline{Claim 1:} If $g\ge 2r+1$ (odd girth), then the induced subgraph on $\bigcup_{i=0}^r L_i$ is a tree (i.e. has no cycles), and in particular each vertex in $L_i$ ($1\le i\le r$) has at most one neighbor in $L_{i-1}$.

Indeed, if some vertex in $L_i$ had two distinct neighbors in $L_{i-1}$, those two paths back to $v$ together with the two edges into the vertex would create a cycle of length at most $2i\le 2r$, contradicting girth $\ge 2r+1$.
Similarly, if there were an edge within $L_i$ for some $i\le r$, combining it with shortest paths to $v$ would create a cycle of length at most $2i+1\le 2r+1$; for $i\le r-1$ this is at most $2r-1$, contradicting the girth bound. (For $i=r$, this would create a cycle of length $2r+1=g$, which is allowed; so for the odd case we only need tree structure up to level $r$ and use the argument below that only counts new vertices, not edges within $L_r$.)

Now we lower bound $|L_i|$ for $i\le r$.
We have $|L_0|=1$.
Since $\deg(v)\ge\delta$, we have $|L_1|\ge\delta$.
For $1\le i\le r-1$, each vertex in $L_i$ has at least $\delta$ neighbors total, at most one in $L_{i-1}$ by Claim 1, and no neighbors in $\bigcup_{j\le i-1}L_j$ other than possibly $L_{i-1}$ (else a shorter path to $v$ exists).
Therefore each vertex in $L_i$ contributes at least $\delta-1$ neighbors in $L_{i+1}$, and these neighbors must be distinct (otherwise two vertices in $L_i$ sharing a neighbor in $L_{i+1}$ would create a cycle of length at most $2(i+1)\le 2r$, again contradicting girth $\ge 2r+1$).
Hence $|L_{i+1}|\ge (\delta-1)|L_i|$ for $i=1,\dots,r-1$.
By induction, $|L_i|\ge \delta(\delta-1)^{i-1}$ for $1\le i\le r$.
Summing gives
\[
|V(G)|\ge \sum_{i=0}^r |L_i| \ge 1+\sum_{i=1}^r \delta(\delta-1)^{i-1} = 1+\delta\sum_{j=0}^{r-1}(\delta-1)^j,
\]
which is the claimed bound in the odd case.

\underline{Claim 2:} If $g\ge 2r$ (even girth), then performing BFS simultaneously from the two endpoints of an edge yields a similar bound.

Pick an edge $uv\in E(G)$. Explore BFS layers outward from the set $\{u,v\}$.
Define $L_0=\{u,v\}$ and $L_i$ = vertices at distance exactly $i$ from $\{u,v\}$.
A standard girth argument (as above) shows that up to distance $r-1$, these layers form two disjoint trees rooted at $u$ and $v$, so there are no identifications that would create a cycle of length $<2r$.
Then $|L_0|=2$, $|L_1|\ge 2(\delta-1)$, and for $1\le i\le r-2$ we again have $|L_{i+1}|\ge (\delta-1)|L_i|$.
Thus for $0\le i\le r-1$, $|L_i|\ge 2(\delta-1)^i$.
Summing yields
\[
|V(G)|\ge \sum_{i=0}^{r-1} |L_i| \ge 2\sum_{i=0}^{r-1}(\delta-1)^i,
\]
the even case bound.
\hfill$\square$

\textbf{Lemma 705.3 (Acyclic case implies $2$-colorable).}
If $G$ is a graph with girth $\infty$ (i.e. a forest), then $\chi(G)\le 2$.

\textit{Proof.}
Every forest is bipartite: each connected component is a tree, and fixing a root, color vertices by parity of distance from the root. Adjacent vertices differ in distance parity, so edges are properly colored. Thus two colors suffice.
\hfill$\square$

5) \textbf{VERIFICATION}

- Lemma 705.1 used an explicit coordinate model for an equilateral triangle and checked the only candidates for the fourth point; the distance computation gives $\sqrt3$, so the contradiction is quantitative.
- Lemma 705.2: The only subtle point is avoiding short cycles via vertex identifications in BFS layers. The proof explicitly argues that a collision in the BFS tree creates a cycle of length $\le 2r$ (odd case) or $\le 2r-1$ (even case), contradicting the stated girth lower bounds.
- Lemma 705.3 is standard and fully verified.

6) \textbf{FINAL}

\textbf{UNRESOLVED}

(i) \textbf{Strongest proved partial result.}
From the examples stated in the problem text, any such $k$ (if it exists) must satisfy $k\ge 6$ (since there is a unit distance graph with $\chi=4$ and girth $5$). Independently, I proved structural facts: unit distance graphs in $\mathbb{R}^2$ have clique number $\le 3$ (Lemma 705.1), and any graph of minimum degree $\delta\ge 2$ and girth $g$ must have many vertices (Lemma 705.2).

(ii) \textbf{First gap (crisp).}
Decide whether there exists a finite unit distance graph $G\subset\mathbb{R}^2$ with girth $\ge 6$ and $\chi(G)\ge 4$.

(iii) \textbf{Top 3 next moves.}
1. Try to build a unit-distance gadget that forces $4$ colors while forbidding cycles up to length $5$, then compose it to raise girth.
2. Try a computational search for small $(\chi=4,\,\text{girth}\ge 6)$ unit distance graphs using constrained coordinate templates (e.g. enforce distances via algebraic coordinates).
3. Seek a geometric sparsity theorem: prove that a unit distance graph in $\mathbb{R}^2$ with girth $\ge k$ must be $3$-degenerate for some absolute $k$.

(iv) \textbf{Minimal counterexample structure.}
A minimal counterexample (if it exists for some $k$) can be assumed $4$-critical (removing any vertex drops chromatic number) and to have girth $\ge k$. Such graphs have minimum degree at least $3$ (for $4$-criticality), so Lemma 705.2 would force a rapidly growing vertex count as $k$ increases.


