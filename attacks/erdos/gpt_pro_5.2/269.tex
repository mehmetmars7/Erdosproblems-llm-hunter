\section*{Problem 269}

\subsection*{Problem statement (as given)}
Let $\mathcal{P}$ be a finite set of primes and let $a_1<a_2<\cdots$ be the increasing sequence of positive integers all of whose prime factors belong to $\mathcal{P}$.
Is
\[
\sum_{n=1}^\infty \frac{1}{[a_1,\ldots,a_n]}
\]
irrational?  (Here $[\cdot]$ denotes the least common multiple, $\operatorname{lcm}$.)

% Source for statement/status: citeturn1view0

\subsection*{1. Formal restatement}

Fix a finite nonempty set of primes $\mathcal{P}$ with $|\mathcal{P}|\ge 2$.  Let
\[
S(\mathcal{P}):=\{m\in\mathbb{N}:\text{ every prime divisor of }m\text{ lies in }\mathcal{P}\}
\]
(the $\mathcal{P}$--smooth positive integers).  Enumerate $S(\mathcal{P})$ in increasing order:
\[
a_1<a_2<a_3<\cdots,
\qquad a_1=1.
\]
For each $n\ge 1$ define
\[
L_n := \operatorname{lcm}(a_1,a_2,\dots,a_n).
\]
Question: is the real number
\[
S(\mathcal{P}):=\sum_{n=1}^\infty \frac{1}{L_n}
\]
irrational for every finite $\mathcal{P}$ with $|\mathcal{P}|\ge 2$?

Edge cases / sanity checks.
\begin{itemize}[itemsep=2pt]
\item If $|\mathcal{P}|=1$, say $\mathcal{P}=\{p\}$, then $a_n=p^{n-1}$, hence $L_n=p^{n-1}$ and $\sum_{n\ge 1}1/L_n=\sum_{k\ge 0}p^{-k}=p/(p-1)\in\mathbb{Q}$.  So the hypothesis $|\mathcal{P}|\ge 2$ is necessary.
\item $L_n$ is nondecreasing and $L_n\to\infty$, so the series converges (proved below).
\end{itemize}

\subsection*{2. Quick literature/context check}

The Erd\H{o}s Problems Project lists this as open for finite $|\mathcal{P}|\ge 2$, and notes that Erd\H{o}s remarked that if duplicate summands are removed, then the resulting series can be proved irrational. % citeturn1view0

The discussion thread contains further comments, including a proof sketch for the case when $\mathcal{P}$ is infinite (then the sum is always irrational). % citeturn17view0

\subsection*{3. Attack plan}

I followed two complementary approaches.

\begin{enumerate}[label=\textbf{(\alph*)},itemsep=2pt]
\item \textbf{Structural rewrite.}  Since $L_n\mid L_{n+1}$, partial sums have denominator $L_n$ exactly:
\[
\sum_{j=1}^n \frac1{L_j}=\frac{M_n}{L_n},\qquad M_{n+1}=M_n\cdot\frac{L_{n+1}}{L_n}+1.
\]
This suggests Cantor/Engel-series style irrationality criteria based on bounding tails versus $1/L_n$.
\item \textbf{Remove duplicates.}  Let $d_1<d_2<\cdots$ be the strictly increasing sequence of \emph{distinct} values taken by $(L_n)$, and consider the simpler series $\sum_i 1/d_i$.  Here each ratio $d_{i+1}/d_i$ is a prime in $\mathcal{P}$ (or a prime-power step), so ratios $\ge 3$ occur infinitely often when $|\mathcal{P}|\ge 2$.  This yields a clean irrationality proof (done below).
\item \textbf{Computation.}  Numerically approximate $S(\mathcal{P})$ for small $\mathcal{P}$ to look for rational patterns (none observed).
\end{enumerate}

\subsection*{4. Work}

\subsubsection*{4.1 Convergence (complete)}

\begin{lemma}[Absolute convergence]\label{lem:conv269}
For any finite set of primes $\mathcal{P}$, the series $\sum_{n\ge 1}1/L_n$ converges.
\end{lemma}

\begin{proof}
Since $a_n\le L_n$ for each $n$, we have $0<1/L_n\le 1/a_n$.  Therefore
\[
\sum_{n=1}^\infty \frac1{L_n}\le \sum_{n=1}^\infty \frac1{a_n}=\sum_{m\in S(\mathcal{P})}\frac1m.
\]
The last sum factorizes as a finite Euler product:
\[
\sum_{m\in S(\mathcal{P})}\frac1m
=\prod_{p\in\mathcal{P}}\left(\sum_{e=0}^\infty p^{-e}\right)
=\prod_{p\in\mathcal{P}}\frac{1}{1-1/p}<\infty.
\]
Hence $\sum 1/L_n$ converges by comparison.
\end{proof}

\subsubsection*{4.2 The one-prime case (complete)}

As already noted in \S1, if $\mathcal{P}=\{p\}$ then $a_n=p^{n-1}$ and $L_n=p^{n-1}$, so
\[
\sum_{n=1}^\infty \frac1{L_n}=\sum_{k=0}^\infty p^{-k}=\frac{p}{p-1}\in\mathbb{Q}.
\]

\subsubsection*{4.3 Irrationality after removing duplicate summands (complete)}

Define the \emph{distinct-denominator} subsequence:
let $1\le n_1<n_2<\cdots$ be the indices for which $L_{n_{i+1}}>L_{n_i}$ and set
\[
d_i:=L_{n_i}.
\]
Then $d_1<d_2<\cdots$ and $d_i\mid d_{i+1}$ for all $i$, and the ``duplicates removed'' series is
\[
S^\ast(\mathcal{P}):=\sum_{i=1}^\infty \frac1{d_i}.
\]

\begin{lemma}[A general irrationality criterion]\label{lem:cantor269}
Let $d_1<d_2<\cdots$ be positive integers with $d_i\mid d_{i+1}$ for all $i$, and suppose that
\[
r_i:=\frac{d_{i+1}}{d_i}\ge 2 \ \text{ for all }i,\qquad\text{and } r_i\ge 3 \text{ for infinitely many }i.
\]
Then $\sum_{i\ge 1} 1/d_i$ is irrational.
\end{lemma}

\begin{proof}
Let $x:=\sum_{i\ge 1}1/d_i$. Suppose for contradiction that $x=A/B$ in lowest terms.

Choose an index $k$ such that $B\mid d_k$ and $r_k\ge 3$.  (Such $k$ exists because $d_k\to\infty$ and thus eventually contains every prime power dividing $B$, and by hypothesis there are infinitely many $k$ with $r_k\ge 3$.)

Let $s_k:=\sum_{i=1}^k 1/d_i$. Since $d_i\mid d_k$, we can write $s_k=m/d_k$ for some integer $m$.

The tail is bounded using $r_i\ge 2$ for all $i\ge k$:
\[
0<x-s_k=\sum_{i>k}\frac1{d_i}
\le \sum_{j\ge 1}\frac1{d_{k+1}\,2^{j-1}}
=\frac{2}{d_{k+1}}
=\frac{2}{r_k d_k}
\le \frac{2}{3d_k}
<\frac{1}{d_k}.
\]
On the other hand, since $B\mid d_k$, both $x$ and $s_k$ are rationals with denominator dividing $d_k$, hence $x-s_k$ is a \emph{nonzero} rational with denominator dividing $d_k$, so
\[
x-s_k\ge \frac1{d_k},
\]
a contradiction.  Therefore $x$ is irrational.
\end{proof}

\begin{proposition}[Duplicates removed $\implies$ irrational for $|\mathcal{P}|\ge 2$]\label{prop:dup269}
If $\mathcal{P}$ is a finite set of primes with $|\mathcal{P}|\ge 2$, then $S^\ast(\mathcal{P})$ is irrational.
\end{proposition}

\begin{proof}
Because $|\mathcal{P}|\ge 2$, $\mathcal{P}$ contains some prime $p\ge 3$.  For each $k\ge 1$, the $\mathcal{P}$--smooth number $p^k$ occurs among the $a_n$, so eventually $L_n$ is multiplied by $p$ when passing from $\operatorname{lcm}(\dots,p^{k-1})$ to $\operatorname{lcm}(\dots,p^k)$.  Therefore, among the distinct values $d_i$, the ratio $d_{i+1}/d_i$ equals $p$ infinitely often, in particular $\ge 3$ infinitely often.  Also each step multiplies $d_i$ by at least $2$, so $r_i\ge 2$ always.  Lemma~\ref{lem:cantor269} applies.
\end{proof}

\subsubsection*{4.4 Numerical evidence for the original series}

Using exact partial sums $S_N=M_N/L_N$ (with $M_{n+1}=M_n\cdot(L_{n+1}/L_n)+1$), I computed high-precision approximations for the original series for several small $\mathcal{P}$:
\[
\begin{array}{rcl}
\mathcal{P}=\{2,3\}: & S(\mathcal{P})\approx& 1.8541655198989682312628678975872517001576\ldots\\
\mathcal{P}=\{2,5\}: & S(\mathcal{P})\approx& 1.4651446526821165569780824808414807518065\ldots\\
\mathcal{P}=\{3,5\}: & S(\mathcal{P})\approx& 1.2843912422550503447214007946752098555857\ldots\\
\mathcal{P}=\{2,3,5\}: & S(\mathcal{P})\approx& 1.9012010608647925442436935551811169862973\ldots
\end{array}
\]
No obvious rational pattern emerged, but this is not decisive.

\subsection*{5. Verification}

\begin{itemize}[itemsep=2pt]
\item Lemma~\ref{lem:conv269} is straightforward comparison to a finite Euler product; the only subtlety is $a_n\le L_n$, which holds since each $a_n$ divides $L_n$.
\item Lemma~\ref{lem:cantor269} is a standard ``tail $<1/d_k$'' irrationality argument; the key is ensuring a step with ratio $\ge 3$ \emph{after} the denominator contains $B$.
\item Proposition~\ref{prop:dup269} uses the fact that powers $p^k$ occur in the $\mathcal{P}$-smooth list, hence the lcm must pick up the factor $p$ infinitely often; this is correct.
\item The numerical computations are only sanity checks.
\end{itemize}

\subsection*{6. Final}

\textbf{UNRESOLVED.}

\begin{enumerate}[label=\textbf{(\roman*)},itemsep=4pt]
\item \textbf{Farthest-reaching partial results proved here.}
\begin{itemize}[itemsep=2pt]
\item The series converges for all finite $\mathcal{P}$ (Lemma~\ref{lem:conv269}).
\item If $|\mathcal{P}|=1$, the sum is rational: $p/(p-1)$.
\item If duplicate denominators are removed (i.e.\ sum over distinct $L_n$ values once each), the resulting series is irrational for every finite $\mathcal{P}$ with $|\mathcal{P}|\ge 2$ (Proposition~\ref{prop:dup269}).
\end{itemize}

\item \textbf{Precise obstacle.}
The original series has repeated denominators: writing it as $\sum_i c_i/d_i$ where $d_i$ are distinct and $c_i\ge 1$ are multiplicities, the tail bound in Lemma~\ref{lem:cantor269} can fail because $c_i$ may be large compared to $d_{i+1}/d_i$ (which is bounded when $\mathcal{P}$ is finite).  Controlling the multiplicities $c_i$ sharply enough to force ``tail $<1/d_i$'' at a stage where the (hypothetical) rational denominator divides $d_i$ is exactly what is missing.

\item \textbf{Most promising next steps.}
\begin{itemize}[itemsep=2pt]
\item Understand growth/structure of the multiplicities $c_i$ (how many $\mathcal{P}$-smooth numbers share the same lcm level) and whether $c_i$ is frequently $< d_{i+1}/d_i$ or otherwise bounded on an infinite subsequence.
\item Seek a more robust irrationality criterion for $\sum c_i/d_i$ with $d_i\mid d_{i+1}$ that tolerates occasional large digits $c_i$ (a ``Cantor series'' approach).
\item Exploit $p$-adic information at a prime $p\in\mathcal{P}$ where the $p$-adic exponent in $L_n$ jumps, hoping to isolate a non-cancellable $p$-adic contribution.
\end{itemize}

\item \textbf{Belief about truth value.}
Based on the ``duplicates removed'' result and numerical experiments, I would \emph{guess} the original sum is irrational for all finite $|\mathcal{P}|\ge 2$, but I do not see a rigorous route from the arguments above to the full statement.
\end{enumerate}

\subsection*{7. Completion estimate}
\[
\textbf{30\%}
\]
(I proved convergence and the ``duplicates removed'' irrationality, but not the original statement.)

%%%%%%%%%%%%%%%%%%%%%%%%%%%%%%%%%%%%%%%%%%%%%%%%%%%%%%%%%%%%%%%%%%%%%%%%%%%%%%%
