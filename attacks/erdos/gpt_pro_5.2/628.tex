% Erdos Problem #628
% URL: https://www.erdosproblems.com/628

FORMAL RESTATEMENT
Let $k\ge3$ and let $G$ be a graph with
\begin{itemize}
\item chromatic number $\chi(G)=k$, and
\item no clique of size $k$ (i.e., $G$ contains no subgraph isomorphic to $K_k$).
\end{itemize}
Let $a,b\ge2$ be integers with $a+b=k+1$.
The Erd\H{o}s--Lov\'asz--Tihany conjecture asks whether there exist two
vertex-disjoint subgraphs $G_1,G_2\subseteq G$ such that
\[
\chi(G_1)\ge a\quad\text{and}\quad \chi(G_2)\ge b.
\]

(Here ``subgraph'' may be taken as induced on its vertex set, since adding edges
cannot decrease chromatic number.)

QUICK LITERATURE/CONTEXT CHECK
I do not use external sources.
The problem file states the conjecture was made in 1981 and is now known in
several special cases (including when $\min(a,b)=2$, when $\alpha(G)=2$, and when
$G$ is claw-free).
I do not assume any of these results in the work below.

ATTACK PLAN
First record reductions that make the problem more tractable:
\begin{itemize}
\item Replace ``two disjoint subgraphs'' by a partition of the vertex set into two
parts whose induced subgraphs have the required chromatic numbers.
\item Reduce to $k$-critical graphs (minimal graphs of chromatic number $k$).
\end{itemize}
Then prove at least one nontrivial base case (here: $k=3$).

WORK
\textbf{Lemma 1 (equivalent partition formulation).}
Let $G$ be a graph and let $a,b\ge2$.
Then $G$ contains vertex-disjoint subgraphs $G_1,G_2$ with
$\chi(G_1)\ge a$ and $\chi(G_2)\ge b$
if and only if there exists a vertex subset $S\subseteq V(G)$ such that
\[
\chi\bigl(G[S]\bigr)\ge a\quad\text{and}\quad \chi\bigl(G[V(G)\setminus S]\bigr)\ge b.
\]

\emph{Proof.}
($\Rightarrow$) Suppose $G_1,G_2$ are vertex-disjoint subgraphs with vertex sets
$S:=V(G_1)$ and $T:=V(G_2)$.
Then $T\subseteq V(G)\setminus S$, so
$G[V(G)\setminus S]$ contains $G_2$ as a subgraph.
Chromatic number is monotone under taking supergraphs on the same vertex set
(adding edges cannot reduce $\chi$), and also monotone under taking induced
supergraphs (adding vertices can only increase or keep $\chi$).
Thus
\[
\chi(G[V\setminus S]) \ge \chi(G[T]) \ge \chi(G_2) \ge b,
\]
and by definition $\chi(G[S])\ge \chi(G_1)\ge a$.

($\Leftarrow$) Conversely, if such an $S$ exists, take $G_1:=G[S]$ and
$G_2:=G[V\setminus S]$. These are vertex-disjoint induced subgraphs with the
required chromatic numbers.
\hfill $\square$

\medskip
\textbf{Lemma 2 (reduction to $k$-critical graphs).}
Fix $k,a,b$ with $a+b=k+1$.
If every $k$-critical graph $H$ with no $K_k$ is $(a,b)$-splittable (in the sense
of Lemma 1), then every $k$-chromatic graph $G$ with no $K_k$ is
$(a,b)$-splittable.

\emph{Proof.}
Given $G$ with $\chi(G)=k$, there exists an induced subgraph $H\subseteq G$ that is
\emph{$k$-critical}: $\chi(H)=k$ and removing any vertex drops the chromatic number.
(Indeed, take an induced subgraph with $\chi=k$ having the minimum number of
vertices; it must be $k$-critical.)
If $G$ contains no $K_k$, then neither does $H$.
By assumption, $H$ is $(a,b)$-splittable, so there is a subset
$S\subseteq V(H)$ with $\chi(H[S])\ge a$ and $\chi(H[V(H)\setminus S])\ge b$.
Since $H[S]$ and $H[V(H)\setminus S]$ are induced subgraphs of $G$ on disjoint
vertex sets, the same subset $S$ witnesses that $G$ is $(a,b)$-splittable.
\hfill $\square$

\medskip
\textbf{Lemma 3 (the conjecture holds for $k=3$).}
Let $G$ be a graph with $\chi(G)=3$ and containing no $K_3$.
Then $G$ is $(2,2)$-splittable, i.e., it contains two vertex-disjoint subgraphs
of chromatic number at least $2$.

\emph{Proof.}
Here $k=3$ forces $a=b=2$.
A graph has chromatic number at least $2$ iff it contains at least one edge.
So it suffices to show that $G$ contains two vertex-disjoint edges, i.e., a
matching of size $2$.

Since $\chi(G)=3$, $G$ contains an odd cycle $C$ (otherwise it would be bipartite
and 2-colorable). Because $G$ is $K_3$-free, the odd cycle $C$ has length at
least $5$ (triangles are excluded).
Any cycle of length at least $5$ contains two vertex-disjoint edges, for example
on vertices $(v_1,v_2,v_3,v_4,v_5,\dots)$ the edges $v_1v_2$ and $v_3v_4$ are
disjoint.
These two edges are subgraphs of $G$ with chromatic number $2$ and are
vertex-disjoint, proving $(2,2)$-splittability.
\hfill $\square$

VERIFICATION
\textbf{FAST REALITY CHECK.}
The smallest $K_3$-free $3$-chromatic graphs are odd cycles $C_{2\ell+1}$
($\ell\ge2$). For $C_5$, the lemma produces two disjoint edges immediately.

FINAL
**UNRESOLVED**
(i) Strongest proved partial results in this write-up: an equivalent partition
formulation (Lemma 1), a reduction to $k$-critical graphs (Lemma 2), and a full
proof of the conjecture in the first nontrivial case $k=3$ (Lemma 3).

(ii) First gap (crisp): establish the conjecture for $k=4$ (i.e., every
$4$-chromatic $K_4$-free graph has a partition into induced subgraphs of
chromatic numbers at least $3$ and $2$), and more generally for arbitrary
$k$-critical $K_k$-free graphs.

(iii) Top 3 next moves:
1. Work inside $k$-critical graphs (by Lemma 2): exploit standard critical-graph
properties (e.g., minimum degree $\ge k-1$) to force the existence of large
chromatic induced subgraphs on disjoint vertex sets.
2. Try to prove the $\min(a,b)=2$ case directly from Lemma 1: find an induced
$(k-1)$-chromatic subgraph whose complement still contains an edge.
3. Perform small-$n$ computational searches for $k=4,5$ to look for the smallest
potential counterexamples and to guess structural invariants that certify
splittability.

(iv) Minimal counterexample structure:
A minimal counterexample for given $(k,a,b)$ would be a $k$-critical,
$K_k$-free graph $G$ such that for every vertex subset $S$ either
$\chi(G[S])\le a-1$ or $\chi(G[V\setminus S])\le b-1$.
Such a graph would have to be highly constrained: $k$-critical forces
$\delta(G)\ge k-1$ and strong connectivity-like behavior, while $K_k$-freeness
prevents the usual ``obvious'' obstructions coming from large cliques.


