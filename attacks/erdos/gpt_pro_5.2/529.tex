% Erdos Problem #529
% URL: https://www.erdosproblems.com/529

Let $d_k(n)$ be the expected distance from the origin after taking $n$ random steps from the origin in $\mathbb{Z}^k$ (conditional on no self intersections) - that is, a self-avoiding walk . Is it true that\[\lim_{n\to \infty}\frac{d_2(n)}{n^{1/2}}= \infty?\]Is it true that\[d_k(n)\ll n^{1/2}\]for $k\geq 3$? Slade \cite{Sl87} proved that, for $k$ sufficiently large, $d_k(n)\sim Dn^{1/2}$ for some constant $D>0$ (independent of $k$). Hara and Slade (\cite{HaSl91} and \cite{HaSl92}) proved this for all $k\geq 5$. For $k=2$ Duminil-Copin and Hammond \cite{DuHa13} have proved that $d_2(n)=o(n)$. It is now conjectured that $d_k(n)\ll n^{1/2}$ is false for $k=3$ and $k=4$, and more precisely (see for example Section 1.4 of \cite{MaSl93}) that $d_2(n)\sim Dn^{3/4}$, $d_3(n)\sim n^{\nu}$ where $\nu\approx 0.59$, and $d_4(n)\sim D(\log n)^{1/8}n^{1/2}$. Madras and Slade \cite{MaSl93} have a monograph on the topic of self-avoiding walks. See also [528] . References [DuHa13] Duminil-Copin, Hugo and Hammond, Alan, Self-avoiding walk is sub-ballistic . Comm. Math. Phys. (2013), 401--423. [HaSl91] Hara, Takashi and Slade, Gordon, Critical behaviour of self-avoiding walk in five or more dimensions . Bull. Amer. Math. Soc. (N.S.) (1991), 417--423. [HaSl92] Hara, Takashi and Slade, Gordon, Self-avoiding walk in five or more dimensions. {I}. {T}he critical behaviour . Comm. Math. Phys. (1992), 101--136. [MaSl93] Madras, Neal and Slade, Gordon, The self-avoiding walk . (1993), xiv+425. [Sl87] Slade, Gordon, The diffusion of self-avoiding random walk in high dimensions . Comm. Math. Phys. (1987), 661--683.

%Erdos problem 529

\subsection*{FORMAL RESTATEMENT}
Fix an integer $k\ge 2$ and $n\ge 1$. Let $\mathsf{SAW}_{k}(n)$ be the set of all length-$n$ nearest-neighbour walks
\[
\omega=(\omega_0,\omega_1,\dots,\omega_n),\qquad \omega_0=0\in\mathbb Z^k,\qquad \|\omega_{t+1}-\omega_t\|_1=1,
\]
that are \emph{self-avoiding}: $\omega_i\neq \omega_j$ for all $0\le i<j\le n$.
Let $\mathbb P_{k,n}$ be the uniform probability measure on $\mathsf{SAW}_{k}(n)$, and let $X_n(\omega):=\omega_n$ be the endpoint.
Let $\|\cdot\|$ denote the Euclidean norm on $\mathbb R^k$, and define
\[
d_k(n):=\mathbb E_{k,n}\big[\|X_n\|\big].
\]
The questions are:

(1) Is it true that $\displaystyle \lim_{n\to\infty}\frac{d_2(n)}{\sqrt n}=+\infty$?

(2) Is it true that for each fixed $k\ge 3$ there is a constant $C_k$ such that $d_k(n)\le C_k\sqrt n$ for all $n\ge 1$?

\emph{Ambiguities in the literal statement.} The phrase ``expected distance'' is not explicit about the metric; I fix Euclidean norm because it is standard in the self-avoiding walk literature and matches the scaling statements in the problem text. Also ``taking $n$ random steps ... conditional on no self intersections'' is interpreted as conditioning simple random walk on self-avoidance; Lemma~\ref{lem:529-uniform} shows this is equivalent to the uniform measure on $\mathsf{SAW}_k(n)$.

\subsection*{QUICK LITERATURE/CONTEXT CHECK}
The problem statement itself records (i) diffusive scaling $d_k(n)\sim D\sqrt n$ for all $k\ge 5$ (\cite{Sl87,HaSl91,HaSl92}), (ii) sub-ballisticity in $k=2$ ($d_2(n)=o(n)$, \cite{DuHa13}), and (iii) conjectured exponents in $k=2,3,4$ (see \cite{MaSl93}). Per the integrity rule, I do not use any additional literature facts beyond what is explicitly written there.

\subsection*{ATTACK PLAN}
\emph{Proof-track ideas.}
(i) Compare $d_k(n)$ to simple random walk via an ``entropic repulsion'' argument (showing superdiffusivity in $k=2$) or via transience/weak self-interaction in $k\ge 3$ (showing diffusive behavior).
(ii) Use counting/probabilistic bounds on the endpoint distribution of self-avoiding walks, e.g. lower bounds on the fraction of walks reaching radius $\ge R$.

\emph{Disproof-track ideas.}
(i) For (1) try to show the conditional endpoint distribution stays at radius $O(\sqrt n)$ with positive probability for infinitely many $n$.
(ii) For (2) try to build a mechanism forcing typical radius $\gg \sqrt n$ in $k=3$ or $4$.

I do not see a route to a complete proof/disproof here; I therefore focus on rigorous basic lemmas and exact small-$n$ computations.

\subsection*{WORK}
\begin{lemma}[Conditioning equals uniform measure]\label{lem:529-uniform}
Let $(S_t)_{t=0}^n$ be simple random walk on $\mathbb Z^k$ started at $0$, i.e. $S_{t+1}-S_t$ is uniformly distributed on the $2k$ coordinate unit vectors $\{\pm e_i\}_{i=1}^k$ and independent over $t$. Let $E_n$ be the event that $(S_0,\dots,S_n)$ has no self-intersections. Then the conditional law of $(S_0,\dots,S_n)$ given $E_n$ is the uniform law on $\mathsf{SAW}_k(n)$.
\end{lemma}
\begin{proof}
Every specific nearest-neighbour walk $\omega$ of length $n$ has unconditioned probability $(2k)^{-n}$ under simple random walk, because each of the $n$ steps has probability $(2k)^{-1}$ and steps are independent. In particular, all $\omega\in\mathsf{SAW}_k(n)$ have the same probability $(2k)^{-n}$. Therefore, for each $\omega\in\mathsf{SAW}_k(n)$,
\[
\mathbb P\big((S_0,\dots,S_n)=\omega\mid E_n\big)
=\frac{\mathbb P((S_0,\dots,S_n)=\omega)}{\mathbb P(E_n)}
=\frac{(2k)^{-n}}{|\mathsf{SAW}_k(n)|(2k)^{-n}}
=\frac1{|\mathsf{SAW}_k(n)|}.
\]
Thus the conditional distribution is uniform on $\mathsf{SAW}_k(n)$.
\end{proof}

\begin{lemma}[Crude counting and endpoint bounds]\label{lem:529-count}
For each $k\ge 2$ and $n\ge 1$,
\[
|\mathsf{SAW}_k(n)| \le 2k\,(2k-1)^{n-1}.
\]
Moreover, $|\mathsf{SAW}_k(n)|\ge 2k$ and hence
\[
d_k(n)\ge n\cdot \frac{2k}{|\mathsf{SAW}_k(n)|}\ge n\,(2k-1)^{-(n-1)}.
\]
Also, $d_k(n)\le n$ for all $k,n$.
\end{lemma}
\begin{proof}
For the upper bound: the first step has exactly $2k$ choices. For each subsequent step, there are at most $2k-1$ choices, because the walk cannot immediately return to the previous vertex (that move would repeat $\omega_{t-1}$ at time $t+1$). Self-avoidance can only further restrict choices, so the total number of walks is at most $2k(2k-1)^{n-1}$.

There are exactly $2k$ ``straight-line'' self-avoiding walks that take the same step direction at every time (choose $\pm e_i$ once and repeat); each has endpoint norm $\|X_n\|=n$. Under the uniform measure, the expected distance is at least $n$ times the probability of landing on one of these $2k$ walks:
\[
d_k(n)\ge n\cdot \mathbb P(\text{walk is straight}) = n\cdot \frac{2k}{|\mathsf{SAW}_k(n)|}.
\]
Combining with the upper bound on $|\mathsf{SAW}_k(n)|$ yields $d_k(n)\ge n(2k-1)^{-(n-1)}$.

Finally, every nearest-neighbour walk satisfies $\|X_n\|\le \sum_{t=0}^{n-1}\|\omega_{t+1}-\omega_t\| = n$ because each step has Euclidean length $1$, hence $d_k(n)\le n$.
\end{proof}

\paragraph{FAST REALITY CHECK (exact small-$n$ computation).}
I exhaustively enumerated all self-avoiding walks of length $n$ in $\mathbb Z^2$ for $n\le 12$ and in $\mathbb Z^3$ for $n\le 10$ (depth-first search with a visited-set constraint), and computed $d_k(n)$ as the uniform average of the Euclidean norm of the endpoint. The tables include the exact counts $c_n^{(k)}:=|\mathsf{SAW}_k(n)|$ as a sanity check.

\medskip
\noindent\textbf{Dimension $k=2$ (exact for $1\le n\le 12$):}
\[
\begin{tabular}{cccc}
 \hline
$n$ & $c^{(2)}_n$ & $d_2(n)$ & $d_2(n)/\sqrt n$\\ \hline
1 & 4 & 1.000000 & 1.000000\\
2 & 12 & 1.609476 & 1.138071\\
3 & 36 & 2.046268 & 1.181413\\
4 & 100 & 2.557026 & 1.278513\\
5 & 284 & 2.951205 & 1.319819\\
6 & 780 & 3.390529 & 1.384178\\
7 & 2172 & 3.747689 & 1.416493\\
8 & 5916 & 4.149886 & 1.467206\\
9 & 16268 & 4.487147 & 1.495716\\
10 & 44100 & 4.861013 & 1.537187\\
11 & 120292 & 5.184254 & 1.563112\\
12 & 324932 & 5.537103 & 1.598424\\
\hline
\end{tabular}
\]

\medskip
\noindent\textbf{Dimension $k=3$ (exact for $1\le n\le 10$):}
\[
\begin{tabular}{cccc}
 \hline
$n$ & $c^{(3)}_n$ & $d_3(n)$ & $d_3(n)/\sqrt n$\\ \hline
1 & 6 & 1.000000 & 1.000000\\
2 & 30 & 1.531371 & 1.082843\\
3 & 150 & 1.907569 & 1.101335\\
4 & 726 & 2.275768 & 1.137884\\
5 & 3534 & 2.577424 & 1.152659\\
6 & 16926 & 2.884472 & 1.177581\\
7 & 81390 & 3.149166 & 1.190273\\
8 & 387966 & 3.417774 & 1.208366\\
9 & 1853886 & 3.657142 & 1.219047\\
10 & 8809878 & 3.899908 & 1.233259\\
\hline
\end{tabular}
\]

In these ranges, the ratio $d_k(n)/\sqrt n$ increases with $n$ (from $1$ at $n=1$ up to about $1.598$ at $n=12$ for $k=2$, and up to about $1.233$ at $n=10$ for $k=3$). This is consistent with superdiffusive behavior in $k=2$ but is far too small a range to be decisive.

\subsection*{VERIFICATION}
\begin{itemize}
\item Quantifiers: $d_k(n)$ is defined for all $n\ge 1$ and $k\ge 2$; the conjectures are asymptotic in $n$ at fixed $k$.
\item Lemma~\ref{lem:529-uniform}: verified that all length-$n$ nearest-neighbour paths have probability $(2k)^{-n}$, so conditioning yields uniformity on $\mathsf{SAW}_k(n)$.
\item Lemma~\ref{lem:529-count}: the step-count bound uses only the ``no immediate backtrack'' restriction, which is implied by self-avoidance.
\item Computation: the enumerated counts in $k=2$ match standard initial values ($4,12,36,100,\dots$), serving as an internal consistency check; in $k=3$ the values are similarly internally consistent.
\end{itemize}

\subsection*{FINAL}
\textbf{**UNRESOLVED**}

(i) \emph{Strongest proved partial result.} The conditioning description is exactly the uniform law on $\mathsf{SAW}_k(n)$ (Lemma~\ref{lem:529-uniform}); the trivial bounds $n(2k-1)^{-(n-1)}\le d_k(n)\le n$ hold for all $k,n$ (Lemma~\ref{lem:529-count}); and exact values of $d_k(n)$ were computed for the $(k,n)$ ranges above.

(ii) \emph{First gap (crisp statement).} Prove either: (a) for $k=2$, there exists a function $\phi(n)\to\infty$ with $d_2(n)\ge \phi(n)\sqrt n$ for all large $n$, or (b) for fixed $k\ge 3$, there is $C_k$ with $d_k(n)\le C_k\sqrt n$ for all $n$.

(iii) \emph{Top 3 next moves.}
1. Prove a nontrivial lower bound on $\mathbb P_{2,n}(\|X_n\|\ge R)$ for $R\gg \sqrt n$ that holds uniformly in large $n$ (e.g. show polynomial, not exponentially small, weight on radii $\ge n^{1/2+\varepsilon}$).
2. Develop a decomposition/renewal structure for SAWs in $k=3,4$ that yields a second-moment bound on $\|X_n\|$ of order $n$ (which would imply $d_k(n)=O(\sqrt n)$ by Cauchy--Schwarz).
3. Computationally: extend exact enumeration to larger $n$ in $k=2$ (transfer-matrix / symmetry reduction) to observe whether $d_2(n)/\sqrt n$ continues to grow and at what apparent rate.

(iv) \emph{Minimal counterexample structure.} A counterexample to (1) would require $d_2(n)\le C\sqrt n$ along an infinite sequence of $n$; heuristically this would mean that, under the uniform measure on $n$-step planar SAWs, a positive fraction of mass stays within $O(\sqrt n)$ radius. A counterexample to (2) for some $k\ge 3$ would require that typical SAWs in $\mathbb Z^k$ have endpoint distance $\gg \sqrt n$.


