% Erdos problem #125
% Attempt for Erdos Problem #125
% Following PROMPT_STRATEGY.MD
% Tools/Constraints:
% - Web browsing available? YES (not used; I restrict to what is stated in 123-137.tex)
% - Computation available (Python)? YES (used for small-case checks)

1) FORMAL RESTATEMENT

Define
\[
A:=\Big\{\sum_{i\ge 0} \varepsilon_i 3^i : \varepsilon_i\in\{0,1\}\text{ finitely supported}\Big\},\qquad
B:=\Big\{\sum_{i\ge 0} \varepsilon_i 4^i : \varepsilon_i\in\{0,1\}\text{ finitely supported}\Big\},
\]
and let
\[
C:=A+B:=\{a+b: a\in A,\ b\in B\} \subset \mathbb{N}_0.
\]
For $X\ge 1$, write
\[
C(X):=|C\cap [0,X]|,\qquad \delta(X):=\frac{C(X)}{X+1}.
\]
The (natural) density $d(C)$, if it exists, is $\lim_{X\to\infty} \delta(X)$.
The lower density is $\underline d(C)=\liminf_{X\to\infty} \delta(X)$.

**Question.** Is $\underline d(C)>0$? (Equivalently, does $C$ have positive density in the usual informal sense?)

**Ambiguity in the stated generalization.** The text writes a condition
$\sum_{i=1}^k \log_{n_k}(2)>1$ and then gives a counterexample $(n_1,n_2,n_3)=(3,9,81)$.
But $\sum_{i=1}^3 \log_{81}(2)<1$, so that counterexample would not contradict the literal condition.
The standard-looking correction is $\sum_{i=1}^k \log_{n_i}(2)>1$.

2) QUICK LITERATURE/CONTEXT CHECK

(Restricted to statements explicitly present in 123-137.tex.)
- The problem asks whether $A+B$ has positive density.
- Hasler and Melfi show (as stated) that the lower density of $C$ is at most $1015/1458\approx 0.69616$.
- Hasler and Melfi also show (as stated) that $\limsup_{x\to\infty} \log|C\cap[1,x]|/\log x \ge 0.9777$.

3) ATTACK PLAN

- Understand the growth rates of $A$ and $B$ and how they interact in the sumset.
- Try simple combinatorial covering heuristics (block decompositions by digit-length) and look for obstructions.
- Do computations of $\delta(X)$ for moderate $X$ to get a feel for possible limits and fluctuations.

4) WORK

Lemma 4.1 (Exact counts on digit-length blocks).
For each integer $m\ge 0$,
\[
|A\cap [0,3^m)| = 2^m.
\]
Similarly, for each $n\ge 0$,
\[
|B\cap [0,4^n)| = 2^n.
\]

*Proof.* Every $a\in A\cap [0,3^m)$ has a base-3 expansion using only digits 0/1 in positions $0,\dots,m-1$.
There are exactly $2^m$ choices of those digits, and they produce distinct integers in $[0,3^m)$.
The base-4 statement is identical. \qed

Lemma 4.2 (Power-law growth of $A$ and $B$).
Let $\alpha:=\log_3 2$ and $\beta:=\log_4 2=1/2$.
Then there are absolute constants $c_1,c_2>0$ such that for all $X\ge 1$,
\[
 c_1 X^\alpha \le |A\cap [0,X]| \le c_2 X^\alpha,
\qquad
 c_1 X^\beta \le |B\cap [0,X]| \le c_2 X^\beta.
\]

*Proof.* Choose $m$ with $3^m\le X<3^{m+1}$. Then by Lemma 4.1,
$2^m\le |A\cap[0,X]|\le 2^{m+1}$.
Since $m=\lfloor \log_3 X\rfloor$, this gives $|A\cap[0,X]|=\Theta(2^{\log_3 X})=\Theta(X^{\log_3 2})$.
The same argument with base 4 gives the $B$ bound. \qed

FAST REALITY CHECK (computation).
I computed $\delta(X)=|C\cap[0,X]|/(X+1)$ for several $X$ by generating $A\cap[0,X]$ and $B\cap[0,X]$ and forming all sums $\le X$.
Results:
- $X=100$: $\delta(X)\approx 0.9802$.
- $X=1000$: $\delta(X)\approx 0.8571$.
- $X=10000$: $\delta(X)\approx 0.9267$.
- $X=100000$: $\delta(X)\approx 0.8616$.
- $X=1000000$: $\delta(X)\approx 0.8736$.
These values suggest large fluctuations but do not settle whether $\underline d(C)>0$.

5) VERIFICATION

- Lemmas 4.1 and 4.2 are exact/elementary and rely only on the digit definitions.
- The density computations are consistent with the fact that $A$ and $B$ are sparse individually (zero density) but their sumset may be much denser.
- The cited upper bound on lower density (Hasler--Melfi) indicates that even if $\underline d(C)>0$, the density cannot be arbitrarily close to 1 (as stated in the problem file).

6) FINAL

**UNRESOLVED**

(i) Strongest fully proved partial result:
- Exact digit-block counts for $A$ and $B$ (Lemma 4.1) and their power-law growth (Lemma 4.2).
- Numerical evidence up to $X=10^6$ shows $\delta(X)$ around $0.86$--$0.93$ at the sampled points, but with substantial fluctuation.

(ii) Exact first gap:
- Prove (or disprove) that $\underline d(C)>0$.

(iii) Top 3 next moves:
1. Work at matched scales $3^m\approx 4^n$: analyze $A\cap[0,3^m)$ and $B\cap[0,4^n)$ as "digit-cubes" and attempt to show $A_m+B_n$ covers a positive proportion of an interval.
2. Study $C$ modulo $3^m$ and $4^n$ and use CRT-type decompositions to transfer modular covering to interval covering.
3. Try to construct (or rule out) infinitely many long gaps in $C$ using digit constraints and carry arguments in mixed bases.

(iv) What a minimal counterexample would likely look like:
- A proof of $\underline d(C)=0$ would likely produce a sequence of scales $X_j\to\infty$ where $C\cap[0,X_j]$ misses a positive proportion of integers, coming from systematic digit-carry obstructions between base 3 and base 4 expansions.


