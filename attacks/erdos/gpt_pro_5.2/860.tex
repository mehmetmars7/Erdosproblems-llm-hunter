% Erdos Problem #860

\noindent\textbf{FORMAL RESTATEMENT.}

Let $p_1=2<p_2=3<p_3=5<\cdots$ be the primes in increasing order, and let $\pi(n)=|\{p\le n : p\text{ prime}\}|$.

Define $h(n)$ to be the least positive integer $L$ such that for every integer $m\ge 1$ there exist \emph{distinct integers}
\[
a_1,a_2,\dots,a_{\pi(n)}\in\mathbb Z
\]
with
\[
 m < a_i < m+L\quad\text{and}\quad p_i\mid a_i\quad\text{for each }i=1,2,\dots,\pi(n).
\]
(Thus the available integers are exactly $m+1, m+2,\dots,m+L-1$.)

The problem asks for estimates of $h(n)$ as $n\to\infty$.

\medskip
\noindent\textbf{QUICK LITERATURE/CONTEXT CHECK.}

The problem file records the following known results: Erd\H{o}s--Pomerance proved $h(n)\ll n^{3/2}/\sqrt{\log n}$; Erd\H{o}s--Selfridge proved $h(n)>(3-o(1))n$; Ruzsa proved $h(n)/n\to\infty$. We do not assume any additional literature.

\medskip
\noindent\textbf{ATTACK PLAN.}

We prove a clean elementary lower bound using gaps between multiples of the largest prime $\le n$, and a clean elementary upper bound by a greedy ``choose an unused multiple'' procedure.
We also compute exact $h(n)$ for $n\le 10$ to sanity-check.

\medskip
\noindent\textbf{WORK.}

\medskip
\noindent\underline{Lemma 860.1 (Lower bound from gaps between multiples).}

\textbf{Lemma.}
Let $q$ be the largest prime $\le n$ (so $q=p_{\pi(n)}$). Then $h(n)\ge q+1$.

\textbf{Proof.}
Fix $m$ to be a multiple of $q$, say $m=q\ell$.
The integers strictly between $m$ and $m+q$ are $m+1,m+2,\dots,m+q-1$; none of these are divisible by $q$ because the next multiple of $q$ after $m$ is $m+q$, which is not included in the open interval $(m,m+q)$.

Therefore, if $L\le q$, then the interval $(m,m+L)$ contains no integer divisible by $q$, so it is impossible to choose $a_{\pi(n)}$ with $q\mid a_{\pi(n)}$.
Thus any $L$ that works for all $m$ must satisfy $L\ge q+1$.
\qed

\medskip
\noindent\underline{Lemma 860.2 (A greedy upper bound).}

\textbf{Lemma.}
For every $n\ge 2$,
\[
 h(n) \le n\,\pi(n) + 1.
\]

\textbf{Proof.}
Let $L:=n\pi(n)+1$. Fix an arbitrary integer $m\ge 1$.
We construct the required distinct integers $a_1,\dots,a_{\pi(n)}$ inductively.

Suppose $1\le i\le \pi(n)$ and we have already chosen distinct integers $a_1,\dots,a_{i-1}$ with
\[
 m<a_j<m+L\quad\text{and}\quad p_j\mid a_j\quad\text{for }1\le j<i.
\]
Consider the first $i$ multiples of $p_i$ that are strictly larger than $m$.
These are
\[
 b_r := p_i\Bigl(\Bigl\lfloor \frac{m}{p_i}\Bigr\rfloor + r\Bigr)\qquad (r=1,2,\dots,i).
\]
Each $b_r$ is divisible by $p_i$ and satisfies $m < b_r \le m + r p_i \le m + i p_i$.
Since $p_i\le n$ and $i\le \pi(n)$, we have $i p_i \le \pi(n) n = L-1$, hence $b_r < m+L$.
So all $b_1,\dots,b_i$ lie in the allowed interval $(m,m+L)$.

Among the $i$ candidates $b_1,\dots,b_i$, at most $i-1$ can coincide with the previously chosen $a_1,\dots,a_{i-1}$, because the $a_j$ are distinct.
Therefore there exists at least one $r\in\{1,\dots,i\}$ such that $b_r\notin\{a_1,\dots,a_{i-1}\}$.
Choose $a_i:=b_r$ for such an $r$.
Then $a_i$ is divisible by $p_i$, lies in $(m,m+L)$, and is distinct from $a_1,\dots,a_{i-1}$.

Proceeding inductively constructs distinct $a_1,\dots,a_{\pi(n)}$ in $(m,m+L)$ with $p_i\mid a_i$ for each $i$.
Since $m$ was arbitrary, $L$ works for all $m$, hence $h(n)\le L = n\pi(n)+1$.
\qed

\medskip
\noindent\underline{Fast reality check (exact $h(n)$ for $n\le 10$).}

For $n\le 10$ (so the relevant primes are $\{2\}$, $\{2,3\}$, $\{2,3,5\}$, or $\{2,3,5,7\}$), the pattern of divisibility in intervals is periodic modulo $P=\prod_{p\le n} p$.
Brute-force checking all $m\bmod P$ and solving the matching problem for each $m$ gives:
\[
\begin{array}{c|ccccccccc}
 n & 2&3&4&5&6&7&8&9&10\\\hline
 h(n) & 3&5&5&7&7&11&11&11&11
\end{array}
\]

\medskip
\noindent\textbf{VERIFICATION.}

\begin{itemize}
\item Lemma 860.1 uses only the fact that multiples of $q$ occur exactly every $q$ integers, and the open-interval endpoint exclusion.
\item Lemma 860.2 carefully accounts for the open interval: we ensured $b_r\le m+(L-1)$, hence $b_r<m+L$.
\item For $n=2$, Lemma 860.1 gives $h(2)\ge 3$ and Lemma 860.2 gives $h(2)\le 3$; the computed value is $h(2)=3$.
\end{itemize}

\medskip
\noindent\textbf{FINAL.} \textbf{UNRESOLVED.}

(i) \emph{Strongest proved partial results.}
We proved the explicit lower bound $h(n)\ge p_{\pi(n)}+1$ and the explicit upper bound $h(n)\le n\pi(n)+1$.
We also computed exact values of $h(n)$ for $n\le 10$.

(ii) \emph{First gap.}
Sharpen the growth rate of $h(n)$: even determining the correct order of magnitude between $n\pi(n)$ and the much smaller $n^{3/2}/\sqrt{\log n}$ recorded in the problem file is open here.

(iii) \emph{Top 3 next moves.}
(1) Replace the greedy argument by a Hall-type expansion argument for the bipartite graph between primes $p\le n$ and integers in $(m,m+L)$, to reduce $L$.
(2) Search computationally for the worst-case $m$ for moderate $n$ (say $n\le 30$) and attempt to guess the asymptotic constant factors.
(3) Construct explicit ``obstruction intervals'' forcing large $h(n)$ by packing many primes' multiples into the same integers (high overlap), to strengthen lower bounds.

(iv) \emph{What a minimal counterexample would look like.}
A minimal obstruction to a smaller $h(n)$ would be an interval where the set of available multiples of the primes $\le n$ has large overlaps, so that any attempt to pick distinct representatives forces leaving some prime unmatched; such intervals likely have many integers with unusually many small prime factors.


