%Erdos problem #159
\section*{Erdos problem \#159}

\subsection*{1) FORMAL RESTATEMENT}
Let $R(C_4,K_n)$ be the Ramsey number: the least $N$ such that every graph $G$ on $N$ vertices contains a copy of $C_4$ or an independent set of size $n$.
Equivalently, $R(C_4,K_n)$ is the least $N$ such that every $C_4$-free graph on $N$ vertices has independence number at least $n$.
The problem asks to show $R(C_4,K_n)$ is ``on the order of'' $n^2$.

\subsection*{2) QUICK LITERATURE/CONTEXT CHECK}
The problem file gives no cited bounds here. I therefore provide only elementary bounds proved below.

\subsection*{3) ATTACK PLAN}
Upper bound: bound the maximum number of edges in a $C_4$-free graph by counting common neighbors; then use a general independence number lower bound.
Lower bound: exhibit explicit $C_4$-free graphs with small independence number (I only obtain a linear lower bound here).

\subsection*{4) WORK}
\paragraph{Lemma 159.1 (edge bound for $C_4$-free graphs via common neighbors).}
Let $G$ be a $C_4$-free graph on $N$ vertices with degree sequence $(d(v))_{v\in V}$. Then
\[
\sum_{v\in V} \binom{d(v)}{2}\ \le\ \binom{N}{2}.
\]
Consequently, if $\bar d=\frac{1}{N}\sum_v d(v)$ is the average degree, then
\[
\bar d(\bar d-1)\le N-1\quad\text{and hence}\quad \bar d\le \frac{1+\sqrt{4N-3}}{2}<\sqrt N+1.
\]
In particular,
\[
|E(G)|=\frac{N\bar d}{2}\ <\ \frac{N^{3/2}}{2}+\frac{N}{2}.
\]

\textit{Proof.}
For any unordered pair of distinct vertices $\{x,y\}$, let $c(x,y)$ be the number of common neighbors of $x$ and $y$.
If $c(x,y)\ge 2$, pick two distinct common neighbors $u\neq v$; then $x-u-y-v-x$ is a $4$-cycle, contradicting that $G$ is $C_4$-free.
Thus $c(x,y)\le 1$ for all $\{x,y\}$.

Now count (unordered) length-$2$ paths by their middle vertex: vertex $v$ contributes $\binom{d(v)}{2}$ unordered pairs of distinct neighbors, i.e. $\binom{d(v)}{2}$ unordered endpoint pairs $\{x,y\}$ with $v$ a common neighbor.
Summing over $v$ gives $\sum_v \binom{d(v)}{2} = \sum_{\{x,y\}} c(x,y)\le \binom{N}{2}$.

By convexity (Jensen), $\sum_v \binom{d(v)}{2}\ge N\binom{\bar d}{2}=\frac{N\bar d(\bar d-1)}{2}$, so
$\frac{N\bar d(\bar d-1)}{2}\le \frac{N(N-1)}{2}$, i.e. $\bar d(\bar d-1)\le N-1$. Solving the quadratic gives the stated bound on $\bar d$ and hence on $|E(G)|=N\bar d/2$. \qed

\paragraph{Lemma 159.2 (a simple independence lower bound from edge count).}
For any graph $G$ on $N$ vertices with $e=|E(G)|$,
\[
\alpha(G)\ \ge\ \frac{N^2}{2e+N}.
\]

\textit{Proof.}
Let $d_1,\dots,d_N$ be the degrees. A standard bound (a one-line form of the Caro--Wei inequality) is
\[
\alpha(G)\ge \sum_{i=1}^N \frac{1}{d_i+1}.
\]
Using Cauchy--Schwarz,
\[
\sum_{i=1}^N \frac{1}{d_i+1}\ \ge\ \frac{\bigl(\sum_{i=1}^N 1\bigr)^2}{\sum_{i=1}^N (d_i+1)}
\ =\ \frac{N^2}{\sum_i d_i + N}\ =\ \frac{N^2}{2e+N}.
\]
\qed

\paragraph{Corollary 159.3 (an $O(n^2)$ upper bound on $R(C_4,K_n)$).}
There is an absolute constant $C$ such that $R(C_4,K_n)\le C n^2$ for all $n$.
For instance, $C=16$ works.

\textit{Proof.}
Let $G$ be any $C_4$-free graph on $N$ vertices. By Lemma 159.1, $e<\frac12 N^{3/2}+\frac12 N$.
Then Lemma 159.2 gives
\[
\alpha(G)\ge \frac{N^2}{2e+N}\ >\ \frac{N^2}{N^{3/2}+2N}\ =\ \frac{\sqrt N}{1+2/\sqrt N}.
\]
If $N\ge 16 n^2$ then $\sqrt N\ge 4n$ and $1+2/\sqrt N\le 1+1/(2n)\le 2$, hence $\alpha(G)\ge 2n>n$.
So every $C_4$-free graph on $N\ge 16n^2$ vertices has an independent set of size $n$, i.e. $R(C_4,K_n)\le 16n^2$. \qed

\paragraph{Lemma 159.4 (a linear lower bound construction).}
For every $n\ge 2$,
\[
R(C_4,K_n)\ge 3(n-1)+1.
\]

\textit{Proof.}
Let $G$ be the disjoint union of $n-1$ triangles ($K_3$'s). Then $G$ is $C_4$-free (it has no cycle of length $4$).
Any independent set can contain at most one vertex from each triangle, so $\alpha(G)=n-1$.
Thus on $N=3(n-1)$ vertices there exists a $C_4$-free graph with no independent set of size $n$, so $R(C_4,K_n)>3(n-1)$, i.e. $R(C_4,K_n)\ge 3(n-1)+1$. \qed

\paragraph{FAST REALITY CHECK (computed exact value for $n=3$).}
By brute force over all graphs on $N\le 7$ vertices, I found:
\[
R(C_4,K_3)=7.
\]
(Equivalently: there exists a $C_4$-free graph on $6$ vertices with independence number $2$, but none on $7$ vertices.)

\subsection*{5) VERIFICATION}
\begin{itemize}
\item Lemma 159.1: the ``two common neighbors imply a $C_4$'' implication is checked explicitly.
\item Lemma 159.2: uses only Caro--Wei and Cauchy--Schwarz.
\item Corollary 159.3: constants tracked; not optimized.
\item Computation of $R(C_4,K_3)$: verified by exhaustive enumeration.
\end{itemize}

\subsection*{6) FINAL}
\textbf{UNRESOLVED}

\begin{enumerate}
\item[(i)] \textbf{Strongest fully proved partial result obtained here.}
I proved an explicit quadratic upper bound $R(C_4,K_n)\le 16n^2$ (Corollary 159.3) and a linear lower bound $R(C_4,K_n)\ge 3n-2$ (Lemma 159.4).

\item[(ii)] \textbf{Exact first gap.}
Provide a matching quadratic lower bound: construct, for infinitely many $n$, a $C_4$-free graph on $\Omega(n^2)$ vertices whose independence number is $<n$.

\item[(iii)] \textbf{Top 3 next moves (concrete targets).}
\begin{enumerate}
\item Find/construct explicit families of $C_4$-free graphs with edge density $\Theta(N^{3/2})$ and prove their independence number is $O(\sqrt N)$.
\item Improve the linear lower bound using better $C_4$-free building blocks with smaller independence ratio than disjoint triangles.
\item Extend computations to larger $n$ for small $K_n$ (e.g. compute $R(C_4,K_4)$ if feasible via SAT/ILP) to guess the correct constant/factor.
\end{enumerate}

\item[(iv)] \textbf{Minimal counterexample structure (if the claimed order $n^2$ is false).}
If $R(C_4,K_n)$ were $o(n^2)$, then every $C_4$-free graph on $N=o(n^2)$ vertices would already force an independent set of size $n$. That would require a stronger universal lower bound $\alpha(G)\gg \sqrt N$ for $C_4$-free graphs than what is obtained from the general edge bound alone.
\end{enumerate}


