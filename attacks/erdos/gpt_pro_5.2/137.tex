% Erdos problem #137
% Attempt for Erdos Problem #137
% Following PROMPT_STRATEGY.MD
% Tools/Constraints:
% - Web browsing available? YES (not used)
% - Computation available (Python)? YES (used for small-case checks)

1) FORMAL RESTATEMENT

A positive integer $N$ is **powerful** if for every prime $p\mid N$ we have $p^2\mid N$ (equivalently, $v_p(N)\ge 2$ for all primes dividing $N$).

Fix an integer $k\ge 3$. For $m\ge 1$ define the product of $k$ consecutive integers
\[
N_{m,k}:= m(m+1)\cdots (m+k-1).
\]

**Question.** Can $N_{m,k}$ be powerful for some $m$ when $k\ge 3$? (The case $k=2$ has the example $8\cdot 9=72$.)

2) QUICK LITERATURE/CONTEXT CHECK

(Restricted to statements explicitly present in 123-137.tex.)
- The problem is open for all $k\ge 3$ as stated.

3) ATTACK PLAN

- Prove elementary constraints on prime exponents in products of consecutive integers.
- Use computation to search for small examples (if any exist).

4) WORK

Lemma 4.1 (Large primes occur in at most one factor).
Let $p$ be a prime and $k\ge 1$.
If $p>k$, then among the $k$ consecutive integers $m,m+1,\dots,m+k-1$, at most one is divisible by $p$.

*Proof.* If $p$ divides both $m+i$ and $m+j$ with $0\le i<j\le k-1$, then $p$ divides their difference $(m+j)-(m+i)=j-i$.
But $1\le j-i\le k-1<p$, impossible. \qed

Lemma 4.2 (A sufficient condition for non-powerfulness).
If there exists a prime $p>k$ such that for some $0\le i\le k-1$ we have $p\mid (m+i)$ but $p^2\nmid (m+i)$, then $N_{m,k}$ is not powerful.
In particular, if the interval $[m,m+k-1]$ contains a prime $p>k$, then $N_{m,k}$ is not powerful.

*Proof.* By Lemma 4.1, such a prime $p>k$ divides exactly one factor among the $k$ consecutive integers.
Therefore $v_p(N_{m,k})=v_p(m+i)=1$, which violates powerfulness.
If $p$ itself lies in $[m,m+k-1]$ and $p>k$, then $p$ divides the product exactly once because $p^2\nmid p$. \qed

FAST REALITY CHECK (computation).
A brute-force search found no powerful products in the tested ranges:
- For $k=3$, no $m\le 50000$ has $m(m+1)(m+2)$ powerful.
- For $k=4$, no $m\le 5000$ has $m(m+1)(m+2)(m+3)$ powerful.
- For $k=5$, no $m\le 5000$ has $m\cdots(m+4)$ powerful.

5) VERIFICATION

- Lemma 4.1 is a standard "difference less than $p$" argument.
- Lemma 4.2 correctly translates the exponent condition $v_p(N)\ge 2$.
- The computation is a finite search; it provides evidence but not a proof.

6) FINAL

**UNRESOLVED**

(i) Strongest fully proved partial result:
- If a prime $p>k$ appears to exponent exactly 1 in some factor of the interval, then the product is not powerful (Lemmas 4.1--4.2).
- No examples were found for $k=3$ up to $m=50000$.

(ii) Exact first gap:
- Either construct an explicit example $(m,k)$ with $k\ge 3$ for which $N_{m,k}$ is powerful, or prove such examples cannot exist.

(iii) Top 3 next moves:
1. Use results on primes in short intervals to guarantee existence of a prime $p>k$ in $[m,m+k-1]$ infinitely often (or always for large $m$), which would rule out powerfulness.
2. Analyze squarefull constraints: if $p>k$ divides some $m+i$, then $p^2$ must divide it, forcing very rigid factorization patterns across the interval.
3. For fixed small $k$ (e.g. $k=3$), attempt a Diophantine descent using the near-coprimality of consecutive integers and valuation constraints.

(iv) What a minimal counterexample would likely look like:
- A short interval of length $k$ consisting entirely of integers each having all its "large" prime factors squared, arranged so that every prime dividing the product appears at least twice overall.

