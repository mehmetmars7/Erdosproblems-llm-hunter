
1) FORMAL RESTATEMENT

For an integer $d\ge 2$ and $k\in\mathbb{N}_0$, define
\[
P(d,k) := \Big\{\sum_{i\ge k} \varepsilon_i d^i : \varepsilon_i\in\{0,1\}\text{ and only finitely many }\varepsilon_i\neq 0\Big\}.
\]
So $P(d,0)$ is the set of nonnegative integers with base-$d$ digits in $\{0,1\}$.

Given integers $3\le d_1<\cdots<d_r$, consider representations of integers of the form
\[
 n = \sum_{j=1}^r c_j a_j, \qquad c_j\in\{0,1\},\ a_j\in P(d_j,k).
\]

**Ambiguity in the statement.** The problem text writes the condition
$\sum_{1\le i\le r} 1/(d_r-1)\ge 1$, which simplifies to $r/(d_r-1)\ge 1$ and conflicts with the later sentence "Pomerance observed that the condition $\sum 1/(d_i-1)\ge 1$ is necessary".

**Minimal correction consistent with the surrounding text:** replace the condition by
\[
\sum_{i=1}^r \frac{1}{d_i-1}\ \ge\ 1.
\]

**Question 1 (k=0).** Under $\sum_{i=1}^r 1/(d_i-1)\ge 1$, must every sufficiently large integer be representable as above with $k=0$?

**Question 2 (k\ge 1).** Under $\gcd(d_1,\dots,d_r)=1$ and $\sum_{i=1}^r 1/(d_i-1)\ge 1$, does there exist $k\ge 1$ such that every sufficiently large integer is representable with this fixed $k$?

2) QUICK LITERATURE/CONTEXT CHECK

- The first question has a simple positive solution (credited in the text to Aristotle and to a note of V. Alexeev).
- The second question is conjectured to have a positive answer (BEGL96), and is known for $(d_1,d_2,d_3)=(3,4,7)$ (BEGL96).
- Pomerance observed the condition $\sum 1/(d_i-1)\ge 1$ is necessary for both questions (Tao sketched explanation).

3) ATTACK PLAN

For Question 1:
- Reinterpret the allowed representations as subset sums of a multiset of powers.
- Apply the classical "complete sequence" criterion (coverage by subset sums) and show the key inequality follows from $\sum 1/(d_i-1)\ge 1$.

For Question 2:
- Prove easy necessary conditions (e.g. $\gcd(d_1,\dots,d_r)=1$).
- Do computational checks for the known example $(3,4,7)$.

4) WORK

Reformulation.
For each $d_i$, an element of $P(d_i,0)$ is a subset sum of the power multiset $\{d_i^0,d_i^1,d_i^2,\dots\}$.
Thus choosing $a_i\in P(d_i,0)$ for each $i$ and summing is equivalent to taking a subset sum of the *disjoint union* multiset
\[
\mathcal{M}:=\bigsqcup_{i=1}^r \{d_i^j: j\in\mathbb{N}_0\},
\]
where the copies are labeled by $i$ (so, for example, $1=d_i^0$ occurs with multiplicity $r$).
Therefore Question 1 asks whether $\mathcal{M}$ is a complete sequence (up to finitely many exceptions).

Lemma 4.1 (Subset-sum coverage criterion).
Let $w_1\le w_2\le\cdots$ be a nondecreasing sequence of positive integers (allowing repetition), and define
$S_t:=\sum_{j=1}^t w_j$.
If
\[
 w_{t+1} \le S_t+1\quad\text{for all }t\ge 1,
\]
then for every $t$ the set of subset sums of $\{w_1,\dots,w_t\}$ contains the entire interval $[0,S_t]\cap\mathbb{Z}$.
In particular, if the inequality holds for all $t$ and $S_t\to\infty$, then every nonnegative integer is representable as a subset sum of the multiset $\{w_j\}$.

*Proof.* Induct on $t$.
For $t=1$, subset sums of $\{w_1\}$ are $\{0,w_1\}$, which contains $[0,w_1]$ iff $w_1=1$; in our application $w_1=1$ holds because each progression contributes $1$.
Assume subset sums of $\{w_1,\dots,w_t\}$ cover all integers $0\le m\le S_t$.
If $w_{t+1}\le S_t+1$, then adding $w_{t+1}$ to those subset sums gives all integers in the interval $[w_{t+1},w_{t+1}+S_t]$.
Since $w_{t+1}\le S_t+1$, we have $w_{t+1}\le S_t+1$ so the two intervals $[0,S_t]$ and $[w_{t+1},w_{t+1}+S_t]$ overlap or touch; hence together they cover $[0,S_t+w_{t+1}] = [0,S_{t+1}]$.
This completes the induction. \qed

Lemma 4.2 (Geometric progression partial sums).
Fix $d\ge 2$. For any integer $t\ge 1$,
\[
1+d+d^2+\cdots+d^{t-1} = \frac{d^t-1}{d-1}.
\]
In particular, for $x=d^t$ we have
\[
\sum_{\substack{j\ge 0\\ d^j<x}} d^j = \frac{x-1}{d-1}.
\]

*Proof.* Standard geometric series identity. \qed

Proposition 4.3 (Question 1: sufficient condition).
Assume $\sum_{i=1}^r \frac{1}{d_i-1}\ge 1$.
Then every sufficiently large integer is representable as a sum of distinct elements of $\mathcal{M}$ (equivalently in the form $\sum_{i=1}^r c_i a_i$ with $a_i\in P(d_i,0)$).

*Proof.* List the multiset $\mathcal{M}$ in nondecreasing order as $w_1\le w_2\le\cdots$.
Fix a term $w_{t+1}=x$ in this list. By construction, $x=d_i^k$ for some $i,k\ge 0$, and all terms $<x$ from the $i$-th progression contribute exactly
$(x-1)/(d_i-1)$ to the total sum of terms $<x$ (Lemma 4.2 when $x$ is itself a power).
Summing over all progressions,
\[
S_t = \sum_{j\le t} w_j \ \ge\ \sum_{i=1}^r \frac{x-1}{d_i-1} \ -\ O(1),
\]
where $O(1)$ accounts for the fact that in progressions where $x$ is not exactly a power, the sum of terms $<x$ differs from $(x-1)/(d_i-1)$ by at most a constant depending only on $d_i$.
In particular there is a constant $C_0$ (depending only on $d_1,\dots,d_r$) such that
\[
S_t \ge (x-1)\sum_{i=1}^r \frac{1}{d_i-1} - C_0.
\]
If $\sum_{i=1}^r 1/(d_i-1)\ge 1$, then for all sufficiently large $x$ we get $S_t\ge x-1$, i.e.
$w_{t+1}=x\le S_t+1$.
Thus the hypothesis of Lemma 4.1 holds for all sufficiently large indices $t$.
Applying Lemma 4.1 from that index onward shows that all integers above some threshold are representable as subset sums of $\mathcal{M}$.
Translating back to the original notation gives the desired representation with $k=0$. \qed

Proposition 4.4 (Necessity of $\sum 1/(d_i-1)\ge 1$ for Question 1).
If $\sum_{i=1}^r 1/(d_i-1) <1$, then infinitely many integers are not representable as subset sums of $\mathcal{M}$, hence Question 1 fails.

*Proof.* Let $x$ be a term of $\mathcal{M}$ that is a large power of the largest base $d_r$ (so $x\to\infty$ along such powers).
As above, the sum of all terms of $\mathcal{M}$ that are $<x$ is asymptotic to $x\sum 1/(d_i-1)$.
More precisely, there exists a constant $C_1$ such that
\[
\sum_{\substack{w\in\mathcal{M}\\ w<x}} w \le (x-1)\sum_{i=1}^r \frac{1}{d_i-1} + C_1.
\]
If $\sum 1/(d_i-1) <1$, then for all sufficiently large $x$ the right-hand side is $<x-1$.
But any subset sum using only terms $<x$ is at most the sum of all such terms, hence is $<x-1$.
Any subset sum using at least one term $\ge x$ is $\ge x$.
Therefore the integer $x-1$ is not representable for all sufficiently large such $x$.
Since there are infinitely many choices of $x$, infinitely many integers are missed. \qed

Proposition 4.5 (A necessary condition for Question 2: gcd).
Fix $k\ge 1$.
If $g:=\gcd(d_1,\dots,d_r)>1$, then no representation with $a_i\in P(d_i,k)$ can produce integers not divisible by $g^k$. In particular, not all sufficiently large integers are representable.

*Proof.* Each $a_i\in P(d_i,k)$ is a sum of powers $d_i^j$ with $j\ge k$, hence is divisible by $d_i^k$.
If $g\mid d_i$ for all $i$, then $g^k\mid d_i^k\mid a_i$ for all $i$, so $g^k$ divides the whole sum. \qed

FAST REALITY CHECK (computation).
For $(d_1,d_2,d_3)=(3,4,7)$ (where $\sum 1/(d_i-1)=1$):
- For $k=0$, all integers $0\le n\le 500$ are representable.
- For $k=1$, among $0\le n\le 5000$ there are $37$ missing values; the largest missing value is $581$.

5) VERIFICATION

- Lemma 4.1 is standard and the induction is explicit.
- Proposition 4.3 uses only the geometric-series lower bound and the subset-sum criterion.
- Proposition 4.4 explicitly produces infinitely many gaps when $\sum 1/(d_i-1)<1$.
- Proposition 4.5 checks a necessary modular obstruction for $k\ge 1$.
- The computational checks are consistent with the stated (3,4,7) theorem for some $k\ge 1$.

6) FINAL

**UNRESOLVED**

(i) Strongest fully proved partial result:
- For the corrected statement of Question 1, $\sum_{i=1}^r 1/(d_i-1)\ge 1$ is sufficient (Proposition 4.3) and also necessary (Proposition 4.4).
- For Question 2, $\gcd(d_1,\dots,d_r)=1$ is necessary for any fixed $k\ge 1$ (Proposition 4.5).

(ii) Exact first gap:
- Prove or disprove Question 2: assuming $\gcd(d_1,\dots,d_r)=1$ and $\sum 1/(d_i-1)\ge 1$, does there exist some $k\ge 1$ for which all sufficiently large integers are representable?

(iii) Top 3 next moves:
1. For fixed $k$, study the representable residue classes modulo products of $d_i^k$ and attempt a covering/CRT argument.
2. Try to adapt the complete-sequence method to the truncated multiset of powers $\{d_i^j:j\ge k\}$, quantifying the loss from removing the small powers.
3. In the spirit of the known $(3,4,7)$ result, search for special triples where a finite "local" argument plus a recursive scaling covers all large integers.

(iv) What a minimal counterexample would likely look like:
- A set of bases with $\gcd=1$ and $\sum 1/(d_i-1)\ge 1$ but for which, after removing the first $k$ digits, there remains an unavoidable modular obstruction creating infinitely many missing residue classes.


