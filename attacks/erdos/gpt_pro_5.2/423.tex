% Erdos Problem #423
% URL: https://www.erdosproblems.com/423

FORMAL RESTATEMENT
Let $(a_k)_{k\ge 1}$ be a sequence of positive integers defined recursively as follows.
\[
 a_1:=1,\qquad a_2:=2.
\]
For each $k\ge 3$, let $a_k$ be the least integer $m$ such that
(i) $m>a_{k-1}$, and
(ii) there exist integers $1\le i<j\le k-1$ with
\[
 m=\sum_{t=i}^j a_t.
\]
(Thus $m$ is the sum of at least two consecutive earlier terms of the sequence.)

The problem asks for the asymptotic behaviour of $a_k$ as $k\to\infty$ (e.g. growth rate, density of the set $\{a_k\}$ in $\mathbb N$, and/or the typical size of the gap $a_k-k$).

Edge conventions: $\mathbb N=\{1,2,3,\dots\}$. “Consecutive” means consecutive indices. Sums must involve at least two terms, i.e. $j-i+1\ge 2$.

QUICK LITERATURE/CONTEXT CHECK
The problem statement attributes the question to Hofstadter/Ulam and identifies the sequence as OEIS A005243. No external results are used here beyond what is explicitly in the problem statement.

ATTACK PLAN
Proof track ideas.
1) Control the set of consecutive sums produced by the initial segment $(a_1,\dots,a_{k})$; if one can show these sums cover a long interval above $a_k$, then the greedy rule would force $a_{k+1}=a_k+1$ for long stretches, suggesting $a_k\sim k$.
2) Establish deterministic upper bounds on $a_k$ in terms of earlier terms (e.g. a Fibonacci-type inequality), then try to sharpen to a linear bound.

Disproof/construction ideas.
1) Try to show there are infinitely many “forbidden” integers that can never occur (or occur only after being skipped), forcing $a_k\ge (1+\delta)k$ along a subsequence.
2) Search computationally for persistent congruence obstructions or structured gaps.

WORK
Fast reality check (explicit computation).
Using a minimal Python implementation of the greedy definition (maintaining all consecutive-sum values incrementally), the first terms are
\[
1,2,3,5,6,8,10,11,14,16,17,18,19,21,22,24,25,29,30,32,\dots
\]
For selected indices $m$ the computed values $a_m$ and the “missing count” $a_m-m$ are:
\[
\begin{array}{c|c|c}
 m & a_m & a_m-m\\\hline
10&16&6\\
20&32&12\\
50&78&28\\
100&146&46\\
200&267&67\\
500&612&112\\
1000&1149&149\\
2000&2189&189\\
5000&5261&261
\end{array}
\]
(These data suggest $a_m/m\downarrow 1$ and $a_m-m$ grows slowly, but this is only empirical.)

Lemma 423.1 (two-term upper bound).
For every $k\ge 3$,
\[
 a_k\le a_{k-1}+a_{k-2}.
\]
Proof.
Fix $k\ge 3$. By definition, $a_k$ is the least integer $>a_{k-1}$ that is a sum of at least two consecutive earlier terms among $a_1,\dots,a_{k-1}$.
The integer $s:=a_{k-2}+a_{k-1}$ is the sum of two consecutive earlier terms (indices $k-2$ and $k-1$). Also $s>a_{k-1}$ because $a_{k-2}\ge 1$.
Therefore $s$ is an admissible candidate in the minimisation defining $a_k$. Since $a_k$ is the least admissible candidate, we must have $a_k\le s=a_{k-2}+a_{k-1}$.
\hfill $\square$

Lemma 423.2 (Fibonacci domination).
Define $(F_k)_{k\ge 1}$ by $F_1=1$, $F_2=2$, and $F_k=F_{k-1}+F_{k-2}$ for $k\ge 3$.
Then for all $k\ge 1$,
\[
 a_k\le F_k.
\]
Proof.
We argue by induction on $k$.
For $k=1,2$ we have $a_1=1=F_1$ and $a_2=2=F_2$.
Assume $k\ge 3$ and that $a_{k-1}\le F_{k-1}$ and $a_{k-2}\le F_{k-2}$. By Lemma 423.1,
\[
 a_k\le a_{k-1}+a_{k-2}\le F_{k-1}+F_{k-2}=F_k.
\]
This completes the induction.
\hfill $\square$

VERIFICATION
- Quantifiers: Lemma 423.1 requires $k\ge 3$ so that $a_{k-2}$ exists and the block $(a_{k-2},a_{k-1})$ has length $2$.
- Strictness: $a_{k-2}\ge 1$ guarantees $a_{k-2}+a_{k-1}>a_{k-1}$, so the candidate sum is indeed admissible.
- Lemma 423.2 uses only Lemma 423.1 and the exact same initial conditions.
- Computations: the greedy construction was implemented exactly as in the definition (candidate must be a consecutive sum of at least two earlier terms). The listed values match the sequence prefix stated in the problem statement.

FINAL
**UNRESOLVED**
(i) Strongest proved partial result: the greedy sequence satisfies the deterministic upper bound $a_k\le F_k$ where $F_k$ is the Fibonacci-type sequence with $F_1=1,F_2=2$ (Lemma 423.2). Trivially $a_k\ge k$ since $(a_k)$ is strictly increasing in $\mathbb N$.
(ii) First gap (crisp): prove any nontrivial asymptotic beyond exponential domination, e.g. show $a_k=O(k)$ or even $\lim_{k\to\infty}a_k/k=1$.
(iii) Top 3 next moves:
  1. Prove an interval-covering lemma: show that the set of consecutive sums from $(a_1,\dots,a_k)$ contains an interval $[a_k+1,a_k+L_k]$ with $L_k\to\infty$.
  2. Quantify gap control: attempt to prove a bound of the form $a_{k+1}-a_k\le C\,k^\alpha$ for some $\alpha<1$ by analysing new suffix sums created when $a_{k+1}$ is added.
  3. Computation-guided structure: search for an explicit description of the “missing set” $\{1,2,\dots,a_k\}\setminus\{a_1,\dots,a_k\}$ (e.g. via residues or greedy invariants) and try to prove it has density $0$.
(iv) Minimal counterexample structure: if one conjectures $a_k\sim k$, a minimal counterexample would require a subsequence $k_j$ and a fixed $\delta>0$ with $a_{k_j}\ge (1+\delta)k_j$, equivalently that up to height $a_{k_j}$ there are at least $\delta k_j$ integers never selected by the greedy rule; such a failure would likely manifest as long stretches with no new consecutive-sum values near the current maximum.


