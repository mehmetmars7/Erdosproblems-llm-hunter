% Erdos solutions for problems 1107--1111
% Generated from 1107-1111.tex

% Erdos Problem #1107
% URL: https://www.erdosproblems.com/1107

\subsection*{FORMAL RESTATEMENT}
Fix an integer $r\ge 2$. A positive integer $n$ is called \emph{$r$-powerful} if for every prime $p$ dividing $n$ we have $p^r\mid n$.
Equivalently, writing $n=\prod_{p} p^{e_p}$ with $e_p\in\mathbb{Z}_{\ge 0}$, we require that for every prime $p$,
\[e_p\in\{0\}\cup\{r,r+1,r+2,\dots\}.\]
(We regard $1$ as $r$-powerful since it has no prime divisors.)

The problem asks whether the following is true:

\emph{For each fixed $r\ge 2$ there exists an integer $N_r$ such that every integer $n\ge N_r$ can be written as}
\[n=a_1+\cdots+a_m\]
\emph{with $m\le r+1$ and each $a_i$ a positive $r$-powerful integer.}

\subsection*{QUICK LITERATURE/CONTEXT CHECK}
The problem file states that the claim is true for $r=2$ (squarefull numbers): Heath--Brown proved that every sufficiently large integer is the sum of three squarefull numbers.
No general result for $r\ge 3$ is stated in the problem file, so I treat the general case as open here and I do not use any external literature beyond what is explicitly stated in the problem text.

\subsection*{ATTACK PLAN}
\emph{Proof-track strategies.}
\begin{itemize}
\item Try a local-to-global approach: understand the set of residues of $r$-powerful numbers modulo $M$, and attempt to show that the $(r+1)$-fold sumset covers all residues for sufficiently large $M$, then lift to integers.
\item Exploit that $r$th powers are $r$-powerful: attempt to reduce the problem to a Waring-type statement plus a bounded number of extra $r$-powerful ``adjustment'' terms.
\item Attempt an interval covering argument: show that sums of $r$-powerful numbers form long consecutive intervals once the scale is large enough.
\end{itemize}

\emph{Disproof-track strategies.}
\begin{itemize}
\item Search for congruence obstructions: find a modulus $M$ such that $\mathcal{P}_r+\cdots+\mathcal{P}_r$ (with $r+1$ summands) misses some residue class, where $\mathcal{P}_r$ denotes the set of $r$-powerful numbers.
\item Try to build an infinite family of integers forced to have small $p$-adic valuations after subtracting any bounded number of $r$-powerful numbers.
\item Computationally hunt for a growing sequence of counterexamples requiring $>r+1$ terms.
\end{itemize}

\subsection*{WORK}
\paragraph{FAST REALITY CHECK (computation).}
I brute-forced representations $n=a_1+\cdots+a_{r+1}$ with each $a_i$ $r$-powerful (repetitions allowed) for small $r$ and ranges of $n$.

\begin{itemize}
\item For $r=2$ and $1\le n\le 100000$, the integers \emph{not} representable as a sum of $\le 3$ squarefull numbers were exactly
\[\{7,15,23,87,111,119\},\]
and every $n\ge 120$ up to $100000$ was representable.
Moreover, among $1\le n\le 20000$, the counts of minimal number of squarefull summands were:
1 term: $267$; 2 terms: $11571$; 3 terms: $8156$; non-representable: $6$.

\item For $r=3$ and $1\le n\le 200000$, the integers \emph{not} representable as a sum of $\le 4$ $3$-powerful numbers were exactly the following $45$ numbers (all $\le 2039$):
\[\begin{aligned}
&5,6,7,12,13,14,15,20,21,22,23,31,38,39,46,47,53,58,69,77,79,85,95,101,103,111,\\
&175,196,212,228,231,247,327,444,458,490,606,662,860,975,1167,1470,1821,1967,2039.
\end{aligned}\]
Every integer $n\ge 2040$ up to $200000$ was representable as a sum of $\le 4$ $3$-powerful numbers.
Among $1\le n\le 50000$, the counts of minimal number of $3$-powerful summands were:
1 term: $97$; 2 terms: $3601$; 3 terms: $34484$; 4 terms: $11773$; non-representable: $45$.
\end{itemize}

These computations are evidence only; they do not constitute a proof that the exceptional set is finite for any $r\ge 3$.

\paragraph{Lemma 1107.1 (radical characterization).}
Let $r\ge 2$ and let $\operatorname{rad}(n)=\prod_{p\mid n} p$ be the radical of $n$. Then a positive integer $n$ is $r$-powerful if and only if
\[\operatorname{rad}(n)^r\mid n.\]

\paragraph{Proof.}
Assume $n$ is $r$-powerful. For every prime $p\mid n$ we have $p^r\mid n$. Multiplying these divisibilities over all primes dividing $n$ gives
\[\prod_{p\mid n} p^r\mid n,\]
i.e. $\operatorname{rad}(n)^r\mid n$.

Conversely, assume $\operatorname{rad}(n)^r\mid n$. Let $p$ be a prime dividing $n$. Then $p\mid \operatorname{rad}(n)$, hence $p^r\mid \operatorname{rad}(n)^r$, hence $p^r\mid n$. Since this holds for every prime $p\mid n$, $n$ is $r$-powerful by definition. \hfill $\square$

\paragraph{Lemma 1107.2 (unique factorization $n=m^r s$ with $s\mid m^{r-1}$).}
Fix $r\ge 2$. A positive integer $n$ is $r$-powerful if and only if there exist integers $m\ge 1$ and $s\ge 1$ such that
\[n=m^r s \quad\text{and}\quad s\mid m^{r-1}.\]
Moreover, if such a representation exists then the pair $(m,s)$ is uniquely determined by $n$.

\paragraph{Proof.}
Write the prime factorization $n=\prod_p p^{e_p}$.

\emph{($\Rightarrow$)} Assume $n$ is $r$-powerful. Thus if $e_p>0$ then $e_p\ge r$.
For each prime $p$, write $e_p=r f_p + g_p$ where $f_p=\lfloor e_p/r\rfloor\in\mathbb{Z}_{\ge 0}$ and $g_p\in\{0,1,\dots,r-1\}$.
Define
\[m:=\prod_p p^{f_p},\qquad s:=\prod_p p^{g_p}.
\]
Then $n=m^r s$ by construction.
We show $s\mid m^{r-1}$. Fix a prime $p$. The exponent of $p$ in $s$ is $g_p\le r-1$.
If $g_p>0$ then $e_p>0$, hence $e_p\ge r$, so $f_p\ge 1$ and the exponent of $p$ in $m^{r-1}$ is $(r-1)f_p\ge r-1\ge g_p$.
If $g_p=0$ the claim is trivial for this prime.
Therefore the $p$-adic exponent of $s$ is at most the $p$-adic exponent of $m^{r-1}$ for every $p$, hence $s\mid m^{r-1}$.

\emph{($\Leftarrow$)} Assume $n=m^r s$ with $s\mid m^{r-1}$.
Let $p$ be any prime dividing $n$. Then $p$ divides $m^r s$, so $p$ divides $m$ or $s$.
If $p\mid s$, the condition $s\mid m^{r-1}$ implies $p\mid m$ as well.
In all cases, $p\mid m$, so write $m=p^a u$ with $a\ge 1$. Then $m^r$ is divisible by $p^{ra}$, hence $n=m^r s$ is divisible by $p^r$.
Thus every prime divisor $p$ of $n$ satisfies $p^r\mid n$, so $n$ is $r$-powerful.

\emph{Uniqueness.} The decomposition $e_p=r f_p+g_p$ with $0\le g_p<r$ is unique for each prime $p$, so the exponent sequences $(f_p)_p$ and $(g_p)_p$ are uniquely determined. Hence $m$ and $s$ are uniquely determined. \hfill $\square$

\paragraph{Corollary 1107.3 (counting bound).}
Let $P_r(x)$ be the number of $r$-powerful integers $n$ with $1\le n\le x$. Then for all $x\ge 1$,
\[P_r(x)\le \sum_{m\le x^{1/r}} \tau\!\left(m^{r-1}\right),\]
where $\tau(n)$ is the number of positive divisors of $n$.

\paragraph{Proof.}
By Lemma 1107.2, each $r$-powerful $n$ corresponds to a unique pair $(m,s)$ with $n=m^r s$ and $s\mid m^{r-1}$.
If $n\le x$ then $m^r\le n\le x$, so $m\le x^{1/r}$.
For a fixed $m$, there are at most $\tau(m^{r-1})$ choices for $s$. Summing over $m\le x^{1/r}$ gives the bound. \hfill $\square$

\subsection*{VERIFICATION}
\begin{itemize}
\item Edge cases: $n=1$ is $r$-powerful and Lemma 1107.1 holds since $\operatorname{rad}(1)=1$.
The decomposition in Lemma 1107.2 gives $m=s=1$ for $n=1$, consistent with uniqueness.
\item In Lemma 1107.2($\Rightarrow$), the step ``$g_p>0\Rightarrow f_p\ge 1$'' uses that $g_p>0\Rightarrow e_p>0$ and $n$ is $r$-powerful, hence $e_p\ge r$.
\item Computations: for a given bound $N$, every summand in a representation of $n\le N$ is at most $n$, hence at most $N$. Thus restricting the candidate $r$-powerful summands to $\le N$ does not miss any representations of numbers $\le N$.
\end{itemize}

\subsection*{FINAL}
\textbf{UNRESOLVED}

(i) \textbf{Strongest proved partial result.}
Lemmas 1107.1--1107.2 give a complete structural description of $r$-powerful numbers, including the unique representation $n=m^r s$ with $s\mid m^{r-1}$, and Corollary 1107.3 gives an explicit (though weak) counting upper bound.

(ii) \textbf{First gap.}
Show (or refute) that for each $r\ge 2$ there exists $N_r$ such that every integer $n\ge N_r$ is the sum of at most $r+1$ $r$-powerful integers.

(iii) \textbf{Top 3 next moves.}
1. For fixed $r$, analyze the set of residues of $r$-powerful numbers modulo $M$ and test whether their $(r+1)$-fold sumset equals $\mathbb{Z}/M\mathbb{Z}$ for all sufficiently large $M$.
2. Prove an interval-covering statement: find $\theta>0$ such that for large $X$ the set $(\mathcal{P}_r+\cdots+\mathcal{P}_r)\cap [1,X]$ contains an interval of length $X^\theta$.
3. Extend computation to $r=4,5$ and search for stabilization of the maximal nonrepresentable integer, to guide conjectures about the minimal obstruction.

(iv) \textbf{What a minimal counterexample would likely look like.}
If the statement fails for some $r$, a minimal counterexample should likely be forced by a local obstruction: a residue class modulo some prime power $p^k$ that cannot be hit by a sum of $r+1$ $r$-powerful residues, reflecting the constraint that each prime dividing an $r$-powerful summand appears with exponent at least $r$.


