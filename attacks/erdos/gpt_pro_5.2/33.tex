FORMAL RESTATEMENT
Let $\mathcal{S}:=\{n^2: n\ge 0\}$ be the set of (nonnegative) perfect squares.
A set $A\subseteq\mathbb{N}$ is an \emph{additive complement to the squares} if there exists $N_0$ such that every integer $n\ge N_0$ can be written as
\[
 n = s + a\quad\text{with }s\in\mathcal{S},\ a\in A.
\]
Write $A(N):=|A\cap\{1,2,\dots,N\}|$.
The problem asks for the smallest constant $c$ such that there exists such an $A$ with
\[
\limsup_{N\to\infty} \frac{A(N)}{\sqrt{N}} = c.
\]

QUICK LITERATURE/CONTEXT CHECK
The problem statement reports that Erdős observed there exist complements with finite $\limsup$ (and $>1$), and that Moser proved a universal lower bound $\liminf A(N)/\sqrt{N}>1.06$.
It further records the best-known lower bound $\liminf A(N)/\sqrt{N}\ge 4/\pi\approx1.273$.
The optimal constant $c$ in the $\limsup$ remains open in the statement.

ATTACK PLAN
A lower bound comes from counting: each translate $a+\mathcal{S}$ contributes at most $\asymp\sqrt{N}$ numbers up to $N$.
Improving the constant requires understanding overlaps between different translates.
An upper bound requires explicit constructions of $A$ with controlled size; since the gap between consecutive squares near $x$ is about $2\sqrt{x}$, one expects to use multiple scales of squares to reduce the size of $A$ below the naive ``all residues up to $2\sqrt{x}$'' strategy.

WORK
Lemma 33.1 (trivial lower bound $A(N)\gtrsim \sqrt{N}$).
Assume $A+\mathcal{S}$ contains every integer $n\ge N_0$.
Then for every $N\ge N_0$,
\[
A(N)\ \ge\ \frac{N-N_0+1}{\lfloor\sqrt{N}\rfloor+1}.
\]
In particular,
\(
\liminf_{N\to\infty} A(N)/\sqrt{N}\ge 1.
\)

Proof.
Fix $N\ge N_0$ and consider the interval $[N_0,N]$.
For each $n\in[N_0,N]$, choose one representation $n=a+s$ with $a\in A$ and $s\in\mathcal{S}$.
Necessarily $a\le n\le N$, so the representing $a$ lies in $A\cap[1,N]$.
Thus the union of translates
\(
\textstyle\bigcup_{a\in A\cap[1,N]} (a+\mathcal{S})
\)
covers the $N-N_0+1$ integers in $[N_0,N]$.
For a fixed $a\le N$, the translate $a+\mathcal{S}$ intersects $[N_0,N]$ in at most the number of squares $s\le N-a$, which is at most $\lfloor\sqrt{N-a}\rfloor+1\le \lfloor\sqrt{N}\rfloor+1$.
Therefore the total number of integers covered by the union is at most
\(
A(N)\,(\lfloor\sqrt{N}\rfloor+1).
\)
Since this union covers $N-N_0+1$ integers, we must have
$A(N)(\lfloor\sqrt{N}\rfloor+1)\ge N-N_0+1$.
Dividing yields the claimed inequality.
\qed

\medskip
Lemma 33.2 (finite-interval upper bound with constant $2$).
For each $N\ge 1$, the set
\[
A_N:=\{0,1,2,\dots,2\lfloor\sqrt{N}\rfloor\}
\]
has size $|A_N|=2\lfloor\sqrt{N}\rfloor+1$ and satisfies
\(
\{0,1,2,\dots,N\}\subseteq A_N+\mathcal{S}.
\)

Proof.
Fix $x\in\{0,1,\dots,N\}$ and let $m:=\lfloor\sqrt{x}\rfloor$.
Then $m^2\le x<(m+1)^2$, so
\[
0\le x-m^2 \le (m+1)^2-1-m^2 = 2m.
\]
Since $m\le\lfloor\sqrt{N}\rfloor$, we have $2m\le 2\lfloor\sqrt{N}\rfloor$ and therefore $a:=x-m^2\in A_N$.
Thus $x=m^2+a\in \mathcal{S}+A_N$.
\qed

\medskip
\textit{Fast reality check (finite set-cover analog).}
As a toy model, fix $M$ and ask for the smallest $A\subseteq\{0,1,\dots,M\}$ such that every $n\in\{0,1,\dots,M\}$ can be written $n=a+s$ with $s$ a square.
This is a finite set-cover instance.
I solved it exactly for $M\le 100$; the minima $k(M)$ and ratios $k(M)/\sqrt{M}$ are:
\[
\begin{array}{c|cccccccccc}
M&10&20&30&40&50&60&70&80&90&100\\\hline
k(M)&4&6&8&9&11&12&13&15&15&17\\
\text{ratio }k(M)/\sqrt{M}&1.265&1.342&1.461&1.423&1.556&1.549&1.554&1.677&1.581&1.700
\end{array}
\]
(Each $k(M)$ above comes with an explicit optimal $A$; e.g. for $M=100$ one optimal set is
$\{0,2,4,6,7,8,9,10,11,12,13,14,15,16,18,20,82\}$.)

VERIFICATION
Lemma 33.1 is a direct union bound on the number of covered integers in $[N_0,N]$.
Lemma 33.2 is checked by taking the nearest square below $x$ and bounding the residual by the square gap $2m+1$.
The finite set-cover values were checked by direct verification of coverage.

FINAL
**UNRESOLVED**
(i) Strongest proved partial result here (from WORK): any additive complement must satisfy $\liminf A(N)/\sqrt{N}\ge 1$ (Lemma 33.1). The problem statement records much stronger known lower bounds (e.g. $4/\pi$) and the existence of some complement with finite $\limsup$.
(ii) First gap: determine (or sharply bound) the infimum possible value of $\limsup A(N)/\sqrt{N}$.
(iii) Top 3 next moves:
  (1) Improve Lemma 33.1 by controlling overlaps between translates $a+\mathcal{S}$ (e.g. via second-moment / energy estimates for the representation function of squares).
  (2) Construct explicit multi-scale sets $A$ whose shifts by squares cover all large integers while keeping $A(N)$ close to $c\sqrt{N}$, and track the achieved constant numerically.
  (3) Push the finite set-cover computations to larger $M$ (with better algorithms) to guess the limiting constant and the ``shape'' of near-optimal $A$.
(iv) Minimal counterexample structure: if the true optimal $c$ were close to $1$, one would need a complement $A$ in which most translates $a+\mathcal{S}$ overlap heavily, so that the union covers $[1,N]$ using far fewer than $\sqrt{N}$ shifts; any such construction would likely exploit nontrivial arithmetic structure of the square set (e.g. congruence restrictions) across many scales.


