\section*{Problem 627}

\subsection*{1) FORMAL RESTATEMENT}
Let $G$ range over all graphs on vertex set $[n]=\{1,\dots,n\}$.  Let $\chi(G)$ denote the chromatic number and $\omega(G)$ the clique number.  Define
\[
f(n)\ :=\ \max\left\{ \frac{\chi(G)}{\omega(G)}:\ |V(G)|=n\right\}.
\]
(So $f(n)$ is the maximum possible ratio between chromatic number and clique number among $n$-vertex graphs.)
We take $\log$ to be the natural logarithm unless explicitly written $\log_2$.

\medskip
\noindent\textbf{Question.} Does the limit
\[
\lim_{n\to\infty}\frac{f(n)}{n/(\log n)^2}
\]
exist (as a finite real number), and if so what is its value?

\subsection*{2) QUICK LITERATURE/CONTEXT CHECK}
Web browsing available: YES. Computation available: YES.

The Erd\H{o}s Problems site lists this question as open.  It records that Erd\H{o}s proved
\[
f(n)\asymp \frac{n}{(\log n)^2},
\]
and that if the above limit exists, it must lie in $(\log 2)^2\cdot[1/4,1]$ (with $\log$ natural).

A very recent note (Araujo--Filipe, arXiv:2512.16062v1, Dec.\ 2025) connects the existence/value of the limit for $f(n)$ to the long-open problem of whether $\lim_{k\to\infty}\frac{\log R(k,k)}{k}$ exists for diagonal Ramsey numbers $R(k,k)$, under an additional conjecture (their Conjecture~1.1).  The note also reports the first asymptotic improvement (since 1967) on the best known constant in an upper bound of the form $f(n)\le (C+o(1))\frac{n}{(\log n)^2}$ (with all logarithms base $2$ in that paper).

\subsection*{3) ATTACK PLAN}
Since the question is a limit-existence problem and the best-known information ties $f(n)$ to Ramsey theory, my approach was:

\begin{enumerate}[label=\textbf{(\Alph*)},leftmargin=3.2em]
\item \textbf{Re-derive basic bounds with explicit constants.}
As a sanity check and as a baseline, prove (with full details) a lower bound
$f(n)\ge c\,n/(\log n)^2$ with an explicit $c>0$ via the probabilistic method (random graphs).
\item \textbf{Relate $f(n)$ to Ramsey numbers.}
Show how constructions of graphs with simultaneously small $\alpha(G)$ and $\omega(G)$ (Ramsey graphs) force lower bounds on $f(n)$, and how upper bounds on $f(n)$ would follow from uniform control of such Ramsey phenomena.
\item \textbf{Assess whether the limit might fail to exist.}
Try to see whether known oscillations/unknowns in Ramsey asymptotics could plausibly imply oscillations of $f(n)(\log n)^2/n$.
I did not find a rigorous oscillation construction.
\end{enumerate}

\subsection*{4) WORK}
I did not resolve the limit existence question.  I provide a complete, self-contained proof of a classical Erd\H{o}s-type \emph{lower bound with constant} via random graphs, plus a precise Ramsey-based inequality that explains why the limit is entangled with diagonal Ramsey asymptotics.

\subsubsection*{4.1. A probabilistic lower bound: $f(n)\ge \frac{1}{4}\frac{n}{(\log_2 n)^2}$}
\begin{theorem}\label{thm:random-lower}
For all sufficiently large $n$,
\[
f(n)\ \ge\ \frac{1}{4}\cdot \frac{n}{(\log_2 n)^2}.
\]
Equivalently, in natural logarithms,
\[
f(n)\ \ge\ \frac{(\log 2)^2}{4}\cdot \frac{n}{(\log n)^2}.
\]
\end{theorem}

\begin{proof}
Consider the binomial random graph $G\sim G(n,1/2)$ on vertex set $[n]$.

Fix an integer $r\ge 3$.  The probability that $G$ contains a clique on a specific $r$-set of vertices is $2^{-\binom{r}{2}}$, since all $\binom{r}{2}$ edges must be present independently with probability $1/2$.
By the union bound,
\[
\mathbb{P}(\omega(G)\ge r)\ \le\ \binom{n}{r}\,2^{-\binom{r}{2}}.
\]
The complement $\overline{G}$ is also distributed as $G(n,1/2)$, and $\alpha(G)=\omega(\overline{G})$, so
\[
\mathbb{P}(\alpha(G)\ge r)\ \le\ \binom{n}{r}\,2^{-\binom{r}{2}}.
\]
Therefore,
\[
\mathbb{P}\bigl(\max\{\omega(G),\alpha(G)\}\ge r\bigr)
\ \le\ 2\binom{n}{r}\,2^{-\binom{r}{2}}.
\]

Now set $r:=\left\lceil 2\log_2 n\right\rceil$.
We estimate the RHS crudely using $\binom{n}{r}\le n^r$:
\[
2\binom{n}{r}\,2^{-\binom{r}{2}}
\ \le\ 2 n^r\,2^{-r(r-1)/2}
\ =\ 2\cdot 2^{r\log_2 n - \frac{r(r-1)}{2}}.
\]
Write $r=2\log_2 n + \delta$ with $\delta\in[0,1]$.
Then
\[
r\log_2 n - \frac{r(r-1)}{2}
= \left(2\log_2 n+\delta\right)\log_2 n - \frac{(2\log_2 n+\delta)(2\log_2 n+\delta-1)}{2}.
\]
Expanding, the leading $2(\log_2 n)^2$ terms cancel, and what remains is $-\Theta(\log_2 n)$.
In particular, the exponent tends to $-\infty$ as $n\to\infty$, so the RHS tends to $0$.
Hence for all sufficiently large $n$, with positive probability we have simultaneously
\[
\omega(G)\ \le\ r,\qquad \alpha(G)\ \le\ r.
\]
Fix such a graph $G$.

Any proper coloring partitions $V(G)$ into $\chi(G)$ independent sets, each of size at most $\alpha(G)\le r$, so $\chi(G)\ge n/r$.
Therefore
\[
\frac{\chi(G)}{\omega(G)}
\ \ge\ \frac{n/r}{r}
\ =\ \frac{n}{r^2}
\ \ge\ \frac{n}{(2\log_2 n + 1)^2}.
\]
For large $n$, $(2\log_2 n + 1)^2\le 4(\log_2 n)^2\,(1+o(1))$, hence
\[
\frac{\chi(G)}{\omega(G)}\ \ge\ \frac{1}{4}\cdot \frac{n}{(\log_2 n)^2}
\]
for all sufficiently large $n$.  Taking the maximum over all graphs on $n$ vertices yields the stated bound on $f(n)$.
Finally, use $\log_2 n = (\log n)/(\log 2)$ to convert to natural logs.
\end{proof}

\subsubsection*{4.2. A Ramsey-number lower bound (exact inequality)}
Let $R(s,t)$ denote the Ramsey number, i.e.\ the least $N$ such that every graph on $N$ vertices has either a clique of size $s$ or an independent set of size $t$.

\begin{proposition}\label{prop:ramsey-lower}
Let $k(n)$ be the unique integer such that
\[
R(k(n),k(n))\ \le\ n\ <\ R(k(n)+1,k(n)+1).
\]
Then there exists a graph $G$ on $n$ vertices with $\omega(G)\le k(n)$ and $\alpha(G)\le k(n)$, and hence
\[
f(n)\ \ge\ \frac{n}{k(n)^2}.
\]
\end{proposition}
\begin{proof}
By definition of Ramsey numbers, the strict inequality $n< R(k(n)+1,k(n)+1)$ means that there exists a graph $G$ on $n$ vertices with \emph{no} clique and \emph{no} independent set of size $k(n)+1$.  Hence $\omega(G)\le k(n)$ and $\alpha(G)\le k(n)$.
As in the proof of Theorem~\ref{thm:random-lower}, $\chi(G)\ge n/\alpha(G)\ge n/k(n)$, so
\[
\frac{\chi(G)}{\omega(G)}\ \ge\ \frac{n/k(n)}{k(n)}=\frac{n}{k(n)^2}.
\]
Maximizing over all $G$ gives $f(n)\ge n/k(n)^2$.
\end{proof}

This proposition shows that fine asymptotics for $f(n)$ are controlled by the growth of $R(k,k)$.

\subsubsection*{4.3. Small-$n$ sanity check (exhaustive search for $n\le 6$)}
I exhaustively enumerated all labeled graphs on $n\le 6$ vertices and computed $\chi(G)$ and $\omega(G)$ exactly.  The maximum ratio $\chi/\omega$ is:
\[
f(2)=f(3)=f(4)=1,\qquad f(5)=f(6)=\frac{3}{2}.
\]
A witness for $f(5)=3/2$ is the 5-cycle $C_5$ (which has $\chi=3$ and $\omega=2$); for $n=6$ a witness is $C_5$ plus an isolated vertex.

\subsection*{5) VERIFICATION}
Theorem~\ref{thm:random-lower} was checked line-by-line: it uses only union bounds, $\binom{n}{r}\le n^r$, and the standard inequality $\chi(G)\ge n/\alpha(G)$.  Proposition~\ref{prop:ramsey-lower} follows directly from the definition of Ramsey numbers.

\subsection*{6) FINAL}
\textbf{UNRESOLVED.}

\begin{enumerate}[label=\textbf{(\roman*)},leftmargin=2.6em]
\item The limit-existence problem for $f(n)/(n/(\log n)^2)$ remains open (per current literature summaries), and recent work connects it to deep open questions in Ramsey theory.
\item I provided a complete probabilistic lower bound with explicit constant $1/4$ (in base-$2$ logarithms), and a precise inequality linking $f(n)$ to diagonal Ramsey numbers.
\item These results do not determine whether $f(n)(\log n)^2/n$ converges; they do, however, explain why the constant is tied to the asymptotics of $R(k,k)$.
\item Any full resolution likely requires either (a) proving convergence of suitable Ramsey-number quotients or (b) producing an explicit oscillation phenomenon.
\end{enumerate}

\subsection*{7) COMPLETION ESTIMATE}
\textbf{COMPLETION: 35\%.}

