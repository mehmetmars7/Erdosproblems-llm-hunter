
1) FORMAL RESTATEMENT

A set $A\subseteq\mathbb N$ is a \emph{Sidon set} (also called a $B_2[1]$ set) if every integer has at most one representation
\[
n=a+b\quad\text{with }a,b\in A\text{ and }a\le b.
\]
Equivalently, all sums $a+b$ with $a\le b$ are distinct.

The problem asks how large
\[
\limsup_{N\to\infty}\frac{|A\cap\{1,\dots,N\}|}{\sqrt{N}}
\]
can be, over all infinite Sidon sets $A\subseteq\mathbb N$.


2) QUICK LITERATURE/CONTEXT CHECK

I do not add external results.
The statement says:

* Erd\H{o}s constructed a Sidon set with limsup $\ge 1/2$.
* Kr\"{u}ckeberg constructed one with limsup $\ge 1/\sqrt{2}$.
* Erd\H{o}s--Tur\'an proved the limsup is always $\le 1$.


3) ATTACK PLAN

I do not attempt the full limsup optimization.
Instead I prove from scratch the classical easy upper bound for finite Sidon sets in $\{1,\dots,N\}$ using distinct differences.
I also compute exact maxima for small $N$ to sanity-check scaling.


4) WORK

PHASE 1: FAST REALITY CHECK (exact maximum Sidon size for small $N$)

By exhaustive search, the maximum size of a Sidon subset of $\{1,\dots,N\}$ for several $N$ is:
\[
\begin{array}{c|ccccc}
N & 10 & 20 & 30 & 40 & 50\\\hline
\max |A| & 4 & 6 & 7 & 8 & 9
\end{array}
\]
The ratios $(\max|A|)/\sqrt{N}$ for these $N$ are approximately $1.265,1.342,1.278,1.265,1.273$.
(These are small-$N$ artifacts; the asymptotic limsup concerns infinite sets.)


Lemma 329.1 (Sidon $\Rightarrow$ distinct positive differences).

Let $A$ be a Sidon set, and let $a_1<a_2<\cdots<a_m$ be distinct elements of $A$.
Then all positive differences $a_j-a_i$ with $1\le i<j\le m$ are distinct.

Proof.
Suppose for contradiction that $a_j-a_i=a_{j'}-a_{i'}$ with $i<j$ and $i'<j'$ and the ordered pairs $(i,j)\neq(i',j')$.
Rearranging gives
\[
a_j+a_{i'}=a_{j'}+a_i.
\]
Because $i<j$ and $i'<j'$, we have $a_i<a_j$ and $a_{i'}<a_{j'}$.
Thus $a_i\le a_{i'}$ or $a_{i'}\le a_i$; in either case, the equality above gives two (possibly reordered) representations of the same integer as a sum of two elements of $A$.
More formally, consider the two unordered pairs $\{a_j,a_{i'}\}$ and $\{a_{j'},a_i\}$.
If they are different, then we have a collision of sums among distinct pairs, contradicting the Sidon property.
If they are the same, then $a_j=a_{j'}$ and $a_{i'}=a_i$, which forces $(i,j)=(i',j')$.
So no two distinct pairs can yield the same difference.
\qed


Lemma 329.2 (finite Sidon upper bound in $\{1,\dots,N\}$).

Let $S\subseteq\{1,\dots,N\}$ be a Sidon set of size $m$.
Then
\[
\binom{m}{2} \le N-1,
\]
and consequently
\[
m \le \frac{1+\sqrt{1+8(N-1)}}{2} \le \sqrt{2N}+1.
\]

Proof.
List $S=\{s_1<\cdots<s_m\}$.
By Lemma 329.1, the $\binom{m}{2}$ positive differences $s_j-s_i$ (for $i<j$) are all distinct.
Each difference is an integer between $1$ and $N-1$, since $1\le s_i<s_j\le N$.
There are only $N-1$ possible values in $\{1,\dots,N-1\}$, so
\[
\binom{m}{2}\le N-1.
\]
Solving the quadratic inequality $m(m-1)/2\le N-1$ gives
\[
m\le \frac{1+\sqrt{1+8(N-1)}}{2}.
\]
Finally $\sqrt{1+8(N-1)}\le \sqrt{8N}$ for $N\ge 1$, giving $m\le \sqrt{2N}+1$.
\qed


5) VERIFICATION

-- Lemma 329.1: the implication ``equal differences $\Rightarrow$ equal sums'' is correct by rearrangement. The Sidon property applies to sums with $a\le b$; equality of unordered pairs covers this.

-- Lemma 329.2: the difference range is indeed within $[1,N-1]$.

-- Computation: brute-force search over subsets for $N\le 50$.


6) FINAL

**UNRESOLVED**

(i) Strongest fully proved partial result obtained here.

* A standard finite upper bound: any Sidon subset of $\{1,\dots,N\}$ has size at most $\sqrt{2N}+1$ (Lemma 329.2).

(ii) Exact first gap.

Determine the optimal possible value of $\limsup_{N\to\infty}|A\cap[1,N]|/\sqrt{N}$ over infinite Sidon sets $A$.
The lemmas here only reprove an easy universal upper bound with constant $\sqrt{2}$ (on finite initial segments) and do not address the sharp constant in the limsup problem.

(iii) Top 3 next moves (concrete targets).

1. Produce an explicit infinite Sidon construction with a provable limsup constant (at least $1/2$ as stated, ideally approaching $1$).
2. Relate the limsup constant to embeddings of finite Sidon sets into perfect difference sets (as referenced in the statement).
3. Study the extension problem: when can a finite Sidon set in $[1,N]$ be extended to a larger Sidon set without losing the Sidon property?

(iv) What a minimal counterexample would likely look like.

If the true supremum limsup constant were $<1$, then there would exist an $\varepsilon>0$ such that every Sidon set satisfies $|A\cap[1,N]|\le (1-\varepsilon)\sqrt{N}$ infinitely often.
A minimal obstruction would likely come from a structural limitation on how often dense finite Sidon sets can occur along an infinite set (e.g. a rigidity preventing them from appearing at infinitely many scales).


