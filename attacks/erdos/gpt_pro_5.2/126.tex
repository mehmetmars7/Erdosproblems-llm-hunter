
1) FORMAL RESTATEMENT

For a finite set $A\subset \mathbb{N}$ with $|A|=n$, define
\[
P(A):=\prod_{\substack{a,b\in A\\ a\neq b}} (a+b).
\]
Let $\omega(m)$ denote the number of *distinct* prime factors of $m$.
Define
\[
f(n):=\min_{\substack{A\subset\mathbb{N}\\ |A|=n}} \omega(P(A)).
\]
(Equivalently, $f(n)$ is the maximal integer such that every $n$-element set $A$ makes $P(A)$ have at least $f(n)$ distinct prime factors.)

Given the bounds quoted in the text, one knows $\log n \ll f(n) \ll n/\log n$.

**Question.** Is it true that
\[
\lim_{n\to\infty} \frac{f(n)}{\log n} = \infty\ ?
\]

2) QUICK LITERATURE/CONTEXT CHECK

(Restricted to statements explicitly present in 123-137.tex.)
- Erdos and Turan (1934) proved $\log n\ll f(n)\ll n/\log n$.
- The problem asks whether $f(n)/\log n\to\infty$.

3) ATTACK PLAN

- Prove structural properties of $f(n)$ (monotonicity, small-$n$ exact values).
- Attempt to force many primes by considering congruence constraints on sums $a+b$.
- Use small computational experiments to guess behavior.

4) WORK

Lemma 4.1 (Monotonicity).
The function $f(n)$ is nondecreasing: $f(n+1)\ge f(n)$ for all $n\ge 1$.

*Proof.* Let $A$ be any set of size $n+1$ and let $B\subset A$ be any $n$-element subset.
Then $P(B)$ divides $P(A)$ because $P(B)$ is the subproduct of $P(A)$ over pairs inside $B$.
Therefore $\omega(P(A))\ge \omega(P(B))$.
Since this holds for every $B$ of size $n$, and by definition every $n$-element set satisfies $\omega(P(B))\ge f(n)$, we obtain $\omega(P(A))\ge f(n)$ for every $A$ of size $n+1$.
Taking the minimum over $A$ shows $f(n+1)\ge f(n)$. \qed

Lemma 4.2 (Exact value $f(3)=2$).
For every set $A=\{a,b,c\}$ of three distinct positive integers,
$\omega(P(A))\ge 2$.
Moreover, equality occurs (e.g. for $A=\{1,3,5\}$), so $f(3)=2$.

*Proof.* Write the three pairwise sums as
\[x=a+b,\quad y=a+c,\quad z=b+c,\]
so $P(A)$ is (up to squares) the product $xyz$.
Suppose for contradiction that $\omega(xyz)=1$, i.e. $x,y,z$ are all powers of a single prime $p$.
Then $p$ divides each of $x,y,z$, hence $x\equiv y\equiv z\equiv 0\pmod p$.
Subtracting gives $b-c=x-y\equiv 0\pmod p$, and similarly $a-c\equiv 0\pmod p$; thus $a\equiv b\equiv c\pmod p$.
Then $x=a+b\equiv 2a\pmod p$, so $2a\equiv 0\pmod p$.
If $p$ is odd, then $a\equiv 0\pmod p$, hence $p\mid a,b,c$.
Dividing $a,b,c$ by $p$ yields a smaller triple with the same property; iterating gives an impossible infinite descent in positive integers.
Therefore $p=2$.
But if $x,y,z$ are powers of 2, then in particular they are all even, so $a,b,c$ must all have the same parity.
If all are odd, then each sum is congruent to $2\pmod 4$, so each of $x,y,z$ has 2-adic valuation exactly 1, forcing $x=y=z=2$, which would imply $a=b=c=1$, contradicting distinctness.
Hence all of $a,b,c$ are even, so dividing by 2 gives a smaller triple with the same property, again yielding infinite descent.
Thus $\omega(xyz)\ge 2$ always.
The example $A=\{1,3,5\}$ has sums $4,6,8$ with prime set $\{2,3\}$, so $\omega(P(A))=2$ and $f(3)=2$. \qed

Corollary 4.3.
$f(4)=2$.

*Proof.* By Lemma 4.1 and Lemma 4.2, $f(4)\ge f(3)=2$.
On the other hand, the explicit set $A=\{1,5,7,11\}$ has all pairwise sums in $\{6,8,12,16,18\}$, hence every pairwise sum has prime factors only from $\{2,3\}$.
Therefore $\omega(P(A))\le 2$, so $f(4)\le 2$. \qed

Lemma 4.4 (A simple upper bound).
For all $n\ge 2$,
\[
f(n) \le \pi(2n),
\]
where $\pi(x)$ is the number of primes $\le x$.

*Proof.* Take $A=\{1,2,\dots,n\}$. Then every factor $(a+b)$ lies in $[3,2n]$, so every prime divisor of $P(A)$ is $\le 2n$.
Hence $\omega(P(A))\le \pi(2n)$ for this particular $A$, and therefore the minimum $f(n)$ is also $\le \pi(2n)$. \qed

FAST REALITY CHECK (computation, heuristic only).
Searching over sets $A\subset\{1,\dots,30\}$ found:
- For $n=2$, minimum $\omega(P(A))$ is 1, achieved by $\{1,2\}$.
- For $n=3$, minimum is 2, achieved by $\{1,2,7\}$.
- For $n=4$, minimum is 2, achieved by $\{1,5,7,11\}$.
- For $n=5$, the minimum within this search window was 3 (example $\{1,2,7,11,25\}$).
This does not determine $f(5)$ globally, but is consistent with growth beyond 2.

5) VERIFICATION

- Lemma 4.1 is a clean divisibility argument: restricting to an $n$-subset only removes factors, so primes cannot be lost when passing from subset to superset.
- Lemma 4.2 carefully rules out the "single prime" case by congruences and infinite descent.
- Corollary 4.3 uses both the monotonicity and an explicit example, so it is rigorous.

6) FINAL

**UNRESOLVED**

(i) Strongest fully proved partial result:
- $f(n)$ is nondecreasing (Lemma 4.1).
- Exact small values: $f(2)=1$, $f(3)=2$ (Lemma 4.2), and $f(4)=2$ (Corollary 4.3).
- A simple universal upper bound $f(n)\le \pi(2n)$ (Lemma 4.4).

(ii) Exact first gap:
- Prove a super-logarithmic lower bound: show that for every $n$-element set $A$, the product of pairwise sums must contain more than $C\log n$ distinct primes for any fixed $C$ as $n\to\infty$.

(iii) Top 3 next moves:
1. Develop a structural theorem for sets $A$ whose pairwise sums have prime factors restricted to a small set $P$; show such $A$ must be small.
2. Use sieve/graph methods: color edges of $K_n$ by a prime divisor of $a_i+a_j$ and bound the number of colors needed.
3. Attempt a "many moduli" argument: for many primes $p$, show there must exist at least one pair sum divisible by $p$ with a new prime, forcing $\omega(P(A))$ large.

(iv) What a minimal counterexample would likely look like:
- A large set $A$ for which all pairwise sums $a+b$ are composed only of primes from a set $P$ with $|P|\asymp \log n$, implying strong congruence alignment of $A$ modulo many primes simultaneously.


