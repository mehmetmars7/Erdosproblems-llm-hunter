% Erdos Problem #432
% URL: https://www.erdosproblems.com/432

Let $A,B\subseteq \mathbb{N}$ be two infinite sets. How dense can $A+B$ be if all elements of $A+B$ are pairwise relatively prime? Asked by Straus, inspired by a problem of Ostmann (see [431] ).

%Erdos problem 432

\medskip
\noindent\textbf{FORMAL RESTATEMENT}

Let $\mathbb N:=\{1,2,3,\dots\}$. Let $A,B\subseteq\mathbb N$ be infinite, and define $S:=A+B=\{a+b:a\in A,\ b\in B\}$. Assume:
\[
\forall s\ne t\in S:\ \gcd(s,t)=1.
\]
Question: how large can $|S\cap[1,x]|$ be as $x\to\infty$ (e.g. what is the maximal possible upper asymptotic density $\overline d(S):=\limsup_{x\to\infty}|S\cap[1,x]|/x$)?

Stress points: (i) $S$ pairwise coprime implies at most one multiple of each prime; (ii) $S$ is a sumset, so $S$ is constrained by additive structure.

\medskip
\noindent\textbf{QUICK LITERATURE/CONTEXT CHECK}

No external results beyond the statement are used here. The problem is motivated by [431].

\medskip
\noindent\textbf{ATTACK PLAN}

Proof track (upper bounds):
1) Use the ``one prime per element'' principle: in a pairwise coprime set, distinct elements have distinct least prime factors, giving $|S\cap[1,x]|\le \pi(x)+1$.
2) Use parity and small primes to constrain $A,B$ (sumset-specific obstruction).

Construction track (lower bounds):
1) Search for finite examples where $A+B$ is large and pairwise coprime; look for patterns to extend to infinite constructions.

I carry out (1) and (2) as rigorous partial bounds and do a small brute force search for (finite) examples.

\medskip
\noindent\textbf{WORK}

\noindent\emph{FAST REALITY CHECK (finite search).}
For each $M\le 10$ I brute-forced all nonempty $A,B\subseteq\{1,\dots,M\}$ with $|A|,|B|\ge 2$ and computed $S=A+B$ (with sums up to $2M$). I then required that all elements of $S$ are pairwise coprime, and maximized $|S|$.
One best example at $M=10$ is
\[A=\{1,3,9\},\quad B=\{4,8,10\},\quad S=\{5,7,9,11,13,17,19\},\]
which indeed is pairwise coprime.
The maxima found were:
\begin{verbatim}
M 6 best|S| 4 |A| 2 |B| 2
M 7 best|S| 5 |A| 2 |B| 3
M 8 best|S| 5 |A| 2 |B| 3
M 9 best|S| 6 |A| 2 |B| 3
M 10 best|S| 7 |A| 3 |B| 3
\end{verbatim}
This suggests nontrivial finite constructions exist, but does not address asymptotic density.

\medskip
\noindent\textbf{Lemma 432.1 (at most one multiple of each prime).}
Let $S\subseteq\mathbb N$ be pairwise coprime: $\gcd(s,t)=1$ for all distinct $s,t\in S$. Then for every prime $p$, there is at most one element of $S$ divisible by $p$.

\noindent\emph{Proof.}
If there were two distinct $s,t\in S$ with $p\mid s$ and $p\mid t$, then $p\mid\gcd(s,t)$, so $\gcd(s,t)\ge p\ge 2$, contradicting pairwise coprimality. \hfill$\Box$

\medskip
\noindent\textbf{Lemma 432.2 (prime-factor injection bound for $A+B$).}
Let $A,B\subseteq\mathbb N$ be infinite and let $S:=A+B$. Assume $S$ is pairwise coprime. Then for every $x\ge 1$,
\[
|S\cap[1,x]|\le \pi(x)+1,
\]
where $\pi(x)$ is the number of primes $\le x$.

\noindent\emph{Proof.}
Let $T:=S\cap[1,x]$. If $1\notin T$, set aside nothing; if $1\in T$, set it aside as a special case because it has no prime divisor. For each $t\in T\setminus\{1\}$, let $p(t)$ be the smallest prime divisor of $t$.

Claim: the map $t\mapsto p(t)$ is injective on $T\setminus\{1\}$. Indeed, if $t_1\ne t_2$ in $T\setminus\{1\}$ had $p(t_1)=p(t_2)=p$, then $p\mid t_1$ and $p\mid t_2$, implying $\gcd(t_1,t_2)\ge p\ge 2$, contradicting that $T\subseteq S$ is pairwise coprime.

Therefore $|T\setminus\{1\}|\le$ (number of primes that occur as $p(t)$). Since $p(t)\le t\le x$, we have $p(t)\le x$, so there are at most $\pi(x)$ possibilities. Hence $|T\setminus\{1\}|\le \pi(x)$, and adding the possible element $1$ gives $|T|\le \pi(x)+1$. \hfill$\Box$

\medskip
\noindent\textbf{Corollary 432.3 (parity constraint for $A,B$).}
Under the hypotheses of Lemma 432.2, $S=A+B$ contains at most one even number (by Lemma 432.1 with $p=2$). In particular, all but at most one element of $S$ is odd. Consequently, $A$ and $B$ cannot both contain infinitely many even integers, and cannot both contain infinitely many odd integers.

\noindent\emph{Proof.}
The first statement is immediate from Lemma 432.1 with $p=2$. If $A$ and $B$ each contained infinitely many even integers, then $A+B$ would contain infinitely many even sums, contradicting ``at most one even element''. The odd case is the same. \hfill$\Box$

\medskip
\noindent\textbf{VERIFICATION}

- Lemma 432.1 is immediate and cannot fail.
- In Lemma 432.2, the only subtlety is handling $1$, which is addressed explicitly.
- Corollary 432.3 only uses that if two infinite sets share a parity infinitely often, then their sumset contains infinitely many even numbers; this is correct.

\medskip
\noindent\textbf{FINAL}

**UNRESOLVED**

(i) Strongest proved partial result: For any infinite $A,B\subseteq\mathbb N$ with $S=A+B$ pairwise coprime, one has the universal counting bound $|S\cap[1,x]|\le \pi(x)+1$ for every $x$ (Lemma 432.2). Also, $S$ contains at most one multiple of any fixed prime (Lemma 432.1), implying strong parity restrictions (Corollary 432.3).

(ii) First gap (crisp): Determine the correct asymptotic order of growth of
\[\sup\{|(A+B)\cap[1,x]|: A,B\subseteq\mathbb N\text{ infinite and }A+B\text{ pairwise coprime}\}\]
as $x\to\infty$ (or even whether one can achieve $|(A+B)\cap[1,x]|\gg \pi(x)$ infinitely often).

(iii) Top 3 next moves:
1. Construction attempt: build explicit infinite $A,B$ such that $A+B$ consists mostly of primes (or prime powers with distinct primes), while controlling cross-sums to avoid shared prime factors.
2. Upper bound: exploit the sumset structure to improve the generic bound $\pi(x)+1$, e.g. show $|(A+B)\cap[1,x]|$ is $o(\pi(x))$ or even $O(x^\theta)$ for some $\theta<1$.
3. Computation: for larger $M$, optimize $|A+B|$ under the pairwise-coprime constraint and inspect structural patterns (e.g. does $A$ tend to be very small or very structured?).

(iv) What a minimal counterexample would likely look like: an extremal pair $(A,B)$ would have nearly all sums odd, and $A+B$ would have to allocate distinct prime factors to distinct sums in a near-injective way, suggesting that most elements of $A+B$ should be prime (or have a unique large prime factor) and that $A,B$ themselves must be very sparse to prevent producing two sums sharing a small prime.


