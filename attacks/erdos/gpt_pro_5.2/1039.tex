
Let $f(z)=\prod_{i=1}^n(z-z_i)\in \mathbb{C}[z]$ with $\lvert z_i\rvert \leq 1$ for all $i$. Let $\rho(f)$ be the radius of the largest disc which is contained in $\{z: \lvert f(z)\rvert< 1\}$. Determine the behaviour of $\rho(f)$. In particular, is it always true that $\rho(f)\gg 1/n$? A problem of Erd\H{o}s, Herzog, and Piranian, who note that $f(z)=z^n-1$ has $\rho(f) \leq \frac{\pi/2}{n}$. Pommerenke \cite{Po61} proved that\[\rho(f) \geq \frac{1}{2en^2}.\]Krishnapur, Lundberg, and Ramachandran \cite{KLR25} proved\[\rho(f) \gg \frac{1}{n\sqrt{\log n}}.\] References [KLR25] M. Krishnapur, E. Lundberg, and K. Ramachandran, On the area of polynomial lemniscates . arXiv:2503.18270 (2025). [Po61] Pommerenke, Ch., On metric properties of complex polynomials . Michigan Math. J. (1961), 97-115.



FORMAL RESTATEMENT
Let f(z)=\prod_{i=1}^n (z-z_i) with |z_i|<=1. Let
  L(f) := { z in C : |f(z)| < 1 }.
Define rho(f) to be the supremum of radii r>0 such that there exists a centre w with the closed disc {z: |z-w|<=r} contained in L(f).
Determine the asymptotic behaviour of rho(f) in terms of n; in particular, is there an absolute c>0 such that rho(f) >= c/n for all such f?

QUICK LITERATURE/CONTEXT CHECK
Only facts explicitly present in the problem statement are treated as established here. In particular, the statement records:
* Example: f(z)=z^n-1 has rho(f) <= (pi/2)/n.
* A lower bound rho(f) >= 1/(2 e n^2) is stated to be known.
* A stronger lower bound rho(f) >> 1/(n*sqrt(log n)) is stated to be known.

ATTACK PLAN
(1) Produce exact rho(f) for very simple families (sanity check).
(2) Prove at least a completely elementary universal lower bound for rho(f) from the factorization (even if weak).
(3) Compare numerically with the model example z^n-1.

WORK
FAST REALITY CHECK (exact for trivial families; numerical for z^n-1).
* For f(z)=z-a (n=1), L(f) is the open disc of radius 1 about a, so rho(f)=1.
* For f(z)=z^n, L(f) is the open unit disc, so rho(f)=1 (Lemma 1039.1).
* A crude numerical search (random centres + ray-bisection over directions) for f(z)=z^n-1 gave the following approximate rho values:
  n=2: rho ~ 0.483
  n=3: rho ~ 0.331
  n=4: rho ~ 0.272
  n=5: rho ~ 0.225
  (These are approximate and only a sanity check; they are consistent with rho being on the order of 1/n.)

Lemma 1039.1 (exact rho for f(z)=z^n).
For f(z)=z^n, one has rho(f)=1.

Proof.
We have |f(z)|<1 iff |z|^n<1 iff |z|<1. Thus L(f) is exactly the open unit disc. The largest disc contained in the open unit disc has radius 1 (take the unit disc itself, or any disc of radius 1-epsilon). Hence rho(f)=1.

Lemma 1039.2 (a completely elementary universal lower bound).
Let f(z)=\prod_{i=1}^n (z-z_i) with |z_i|<=1 and n>=2. Then rho(f) >= 3^{-(n-1)}.

Proof.
Pick any root z_1. Consider the open disc D := {z: |z-z_1| < r} with r=3^{-(n-1)}.
For any z in D, we have |z-z_1|<r. For j!=1, use the triangle inequality:
  |z-z_j| <= |z-z_1| + |z_1-z_j| < r + 2.
Since r<1 for n>=2, we have r+2 < 3, hence |z-z_j| < 3 for all j!=1.
Therefore
  |f(z)| = \prod_{i=1}^n |z-z_i| < r * 3^{n-1} = 1.
So D is contained in L(f), and hence rho(f) >= r = 3^{-(n-1)}.

VERIFICATION
* Lemma 1039.1 is immediate from the definition.
* Lemma 1039.2 uses only triangle inequality and the factorization of f.
* The numerical sanity check was produced by an explicit script; it is not used as proof.

FINAL
UNRESOLVED
(i) Strongest proved partial result here: an explicit universal lower bound rho(f) >= 3^{-(n-1)} (Lemma 1039.2), and exact value rho(z^n)=1 (Lemma 1039.1). Numerical sanity checks for z^n-1 suggest rho(z^n-1) is about 1/n.
(ii) First gap: prove or disprove a polynomial-in-n lower bound of the form rho(f) >= c/n for an absolute constant c>0 (or determine the correct order of the worst case).
(iii) Top 3 next moves:
  1. Improve the elementary bound by exploiting that not all factors |z-z_j| can be simultaneously close to their worst-case upper bound near a root; try to relate \prod_{j!=i} |z_i-z_j| to n and then control f in a neighbourhood.
  2. Search for explicit families with unusually small rho(f) (potential counterexamples to 1/n), e.g. roots distributed near the unit circle in a way that makes all components of L(f) thin.
  3. Rigorous computational exploration for moderate n (say n<=10): optimize over root locations (subject to |z_i|<=1) to minimize rho(f), to guess extremal configurations.
(iv) Minimal counterexample structure (if rho(f) is smaller than c/n): a sequence of polynomials whose sublevel set |f|<1 breaks into many thin components, each with small inradius, perhaps via roots near the unit circle spaced to create narrow corridors.

