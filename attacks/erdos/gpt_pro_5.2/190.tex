\section*{Erd\H{o}s problem 190}

\subsection*{1) FORMAL RESTATEMENT}
Let $H(k)$ be the smallest $N$ such that for every colouring $\chi:\{1,\dots,N\}\to\mathcal{C}$ into any set of colours,
there exists a $k$-term arithmetic progression $a,a+d,\dots,a+(k-1)d$ with either:
(i) all $k$ points the same colour (monochromatic), or
(ii) all $k$ points distinct colours (rainbow).
Estimate $H(k)$ and ask whether $H(k)^{1/k}/k\to\infty$.

\subsection*{2) QUICK LITERATURE/CONTEXT CHECK}
Problem text says: existence of $H(k)$ follows from Szemer\'edi’s theorem and $H(k)^{1/k}\to\infty$ is easy.
No further claims are used here.

\subsection*{3) ATTACK PLAN}
Give:
(1) an explicit general lower bound construction,
(2) exact computation for $k=3$ by exhaustive search (computer-assisted).

\subsection*{4) WORK}

\paragraph{Lemma 4.1 (Explicit lower bound $H(k)\ge (k-1)^2+1$).}
For every $k\ge 3$, there exists a colouring of $\{1,\dots,(k-1)^2\}$ with no monochromatic $k$-AP and no rainbow $k$-AP.
Hence $H(k)\ge (k-1)^2+1$.
\textit{Proof.}
Colour $n$ by its residue class modulo $(k-1)$, using exactly $(k-1)$ colours.
Then no rainbow $k$-AP exists because there are only $k-1$ colours total.
If a $k$-term AP is monochromatic under this colouring, all its terms are congruent mod $(k-1)$, so its common difference $d$ is a multiple of $(k-1)$.
Then the span from first to last term is $(k-1)d\ge (k-1)^2$.
But within $\{1,\dots,(k-1)^2\}$, any $k$-AP would require $a+(k-1)d\le (k-1)^2$; if $d\ge (k-1)$ this forces $a\le 0$, impossible.
Thus no monochromatic $k$-AP exists either. \hfill$\square$

\paragraph{Lemma 4.2 ($H(3)=9$, computer-assisted).}
$H(3)=9$.
\textit{Proof (with exhaustive computation).}
A colouring of $\{1,\dots,8\}$ avoiding both monochromatic and rainbow 3-APs is:
\[
\chi(1..8)=(0,0,1,0,1,1,2,1).
\]
Thus $H(3)\ge 9$.

For $N=9$, I performed exhaustive backtracking over \emph{canonical} colourings (restricted-growth strings:
new colours may appear only as the next unused label), checking each colouring against all 3-term APs in $\{1,\dots,9\}$.
No colouring avoids both a monochromatic 3-AP and a rainbow 3-AP. Therefore every colouring of $\{1,\dots,9\}$ has one of the two patterns,
so $H(3)\le 9$. Hence $H(3)=9$. \hfill$\square$

\subsection*{FAST REALITY CHECK (computed)}
Exhaustive search results:
For $N=8$ an avoiding colouring exists (given above).
For $N=9$ no avoiding colouring exists (backtracking over canonical colourings).

\subsection*{6) FINAL}
\textbf{UNRESOLVED}

(i) Strongest proved partial results: explicit lower bound $H(k)\ge (k-1)^2+1$ (Lemma 4.1) and exact value $H(3)=9$ (Lemma 4.2).

(ii) First gap: asymptotic growth of $H(k)$; in particular whether $H(k)^{1/k}/k\to\infty$.

(iii) Top 3 next moves:
1. Derive nontrivial upper bounds via a canonical-structure theorem for AP-colourings (analogue of canonical Ramsey).
2. Search for improved lower-bound constructions beyond $(k-1)^2$ using structured colourings.
3. Compute $H(4)$ on computers (SAT/backtracking) to identify growth behaviour.

(iv) Minimal counterexample structure (to fast growth): a family of colourings on $[N]$ with many colours but with strong restrictions ensuring
every $k$-AP repeats a colour while also preventing $k$ identical colours along an AP.

