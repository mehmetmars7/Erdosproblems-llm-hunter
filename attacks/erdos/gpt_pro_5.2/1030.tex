
If $R(k,l)$ is the Ramsey number then prove the existence of some $c>0$ such that\[\lim_k \frac{R(k+1,k)}{R(k,k)}> 1+c.\] A problem of Erd\H{o}s and S\'{o}s, who could not even prove whether $R(k+1,k)-R(k,k)>k^c$ for any $c>1$. It is trivial that $R(k+1,k)-R(k,k)\geq k-2$. Burr, Erd\H{o}s, Faudree, and Schelp \cite{BEFS89} proved\[R(k+1,k)-R(k,k)\geq 2k-5.\]See also [544] for a similar question concerning $R(3,k)$, and [1014] for the general off-diagonal case. References [BEFS89] Burr, S. A. and Erd\H{o}s, P. and Faudree, R. J. and Schelp, R. H., On the difference between consecutive {R}amsey numbers . Utilitas Math. (1989), 115--118.

\subsection*{Erd\H{o}s Problem \#1030 --- Solution Attempt}

\textbf{FORMAL RESTATEMENT.}
For integers $k,\ell\ge 2$, let $R(k,\ell)$ be the smallest $n$ such that every red/blue edge-colouring of $K_n$ contains either a red $K_k$ or a blue $K_\ell$.
The problem asks whether there exists a constant $c>0$ such that
\[
\liminf_{k\to\infty}\frac{R(k+1,k)}{R(k,k)} > 1+c.
\]
(Interpreting the problem's ``$\lim_k$'' as at least a liminf, since existence of a strict limit is not known from the statement.)

\textbf{QUICK LITERATURE/CONTEXT CHECK.}
The statement records:
\begin{itemize}
\item the trivial bound $R(k+1,k)-R(k,k)\ge k-2$,
\item the stronger bound $R(k+1,k)-R(k,k)\ge 2k-5$ due to Burr--Erd\H{o}s--Faudree--Schelp \cite{BEFS89}.
\end{itemize}
I do not use any external results beyond what is explicitly in the problem file.

\textbf{ATTACK PLAN.}
\emph{Proof track:} show that $R(k+1,k)$ grows at least multiplicatively faster than $R(k,k)$, e.g. by proving a superlinear difference bound $R(k+1,k)-R(k,k)\ge \Omega(R(k,k))$.

\emph{Disproof track:} attempt to construct colourings showing $R(k+1,k)\le (1+o(1))R(k,k)$, or at least $\liminf$ of the ratio equals $1$.

Here I provide rigorous baseline inequalities and small-$k$ computations.

\textbf{WORK.}

\textbf{Lemma 1 (An elementary additive lower bound).}
For every integer $k\ge 2$,
\[
R(k+1,k) \ge R(k,k) + (k-1).
\]
In particular, $R(k+1,k)-R(k,k)\ge k-1$.

\emph{Proof.}
Let $n:=R(k,k)-1$. By definition of $R(k,k)$, there exists a red/blue colouring of the edges of $K_n$ with no red $K_k$ and no blue $K_k$.
Fix such a colouring on a vertex set $V$ with $|V|=n$.

We build a colouring on $V\cup W$ where $W$ is a set of $k-1$ new vertices.
Colour every edge inside $W$ blue.
Colour every edge between $W$ and $V$ red.
Keep the original colouring on edges inside $V$.
Let $N:=n+(k-1)=R(k,k)+k-2$.

We claim that this colouring of $K_N$ contains neither a red $K_{k+1}$ nor a blue $K_k$.

\emph{No blue $K_k$.}
Any set of $k$ vertices contained entirely in $V$ is not a blue $K_k$ by the defining property of the colouring on $V$.
If a set of $k$ vertices contains at least one vertex from $W$, then it also contains at most $k-1$ vertices from $V$.
But every edge between $W$ and $V$ is red, so such a $k$-set cannot span a blue $K_k$.

\emph{No red $K_{k+1}$.}
Any red clique can contain at most one vertex from $W$, because edges inside $W$ are blue.
So a red $K_{k+1}$ would have to be either entirely inside $V$ (impossible since the colouring on $V$ has no red $K_k$, hence no red $K_{k+1}$),
or consist of one vertex from $W$ plus $k$ vertices from $V$.
In the latter case, the $k$ vertices from $V$ would need to form a red $K_k$ (since all edges from $W$ to $V$ are red), contradicting the property of the colouring on $V$.

Thus the constructed colouring of $K_N$ avoids red $K_{k+1}$ and blue $K_k$, so $R(k+1,k) > N = R(k,k)+k-2$.
Equivalently, $R(k+1,k)\ge R(k,k)+k-1$.
\qed

\medskip
\textbf{Lemma 2 (A simple explicit lower bound for $R(k,k)$).}
For every integer $k\ge 2$,
\[
R(k,k) > (k-1)^2.
\]
Equivalently, $R(k,k)\ge (k-1)^2+1$.

\emph{Proof.}
Let $n:=(k-1)^2$ and partition the $n$ vertices into $k-1$ groups $V_1,\dots,V_{k-1}$ each of size $k-1$.
Colour edges inside each group $V_i$ red, and colour edges between distinct groups blue.

Then there is no red $K_k$ because within any group there are only $k-1$ vertices.
There is no blue $K_k$ because a blue clique can include at most one vertex from each group (edges inside a group are red), so it has size at most $k-1$.
Hence this colouring of $K_n$ avoids both a red and a blue $K_k$, proving $R(k,k)>n$.
\qed

\medskip
\textbf{FAST REALITY CHECK (small cases / computation).}
\begin{itemize}
\item Brute force confirms $R(3,3)=6$.
\item Using a DPLL/SAT search for $(4,3)$, there exists a colouring of $K_8$ with no red $K_4$ and no blue $K_3$, but none exists for $K_9$.
Thus $R(4,3)=9$.
One explicit $K_8$ colouring (red edges listed) avoiding red $K_4$ and blue $K_3$ is:
\[\begin{aligned}
&\{01,02,03,04,05,12,13,16,17,24,26,35,37,45,46,47,57,67\}
\end{aligned}\]
(where $ij$ denotes the edge between vertices $i$ and $j$).
\end{itemize}
For this small case, the ratio is $\frac{R(4,3)}{R(3,3)}=\frac{9}{6}=1.5$.

\textbf{VERIFICATION.}
\begin{itemize}
\item Lemma 1: checked that any red clique can have at most one vertex in $W$ because all edges in $W$ are blue, and that any candidate red $K_{k+1}$ would force a red $K_k$ inside $V$.
\item Lemma 2: checked both obstructions (red and blue) by counting how many vertices can be taken from each part.
\item The small-case $R(4,3)=9$ statement is backed by an exhaustive DPLL/SAT search: it explicitly finds a satisfying assignment for $n=8$ and proves unsatisfiable for $n=9$ under the clauses ``no red $K_4$'' and ``no blue $K_3$''.
\end{itemize}

\textbf{FINAL.} \textbf{UNRESOLVED.}

(i) \emph{Strongest proved partial result.} An explicit additive separation holds: $R(k+1,k)\ge R(k,k)+k-1$ (Lemma 1), and $R(k,k)\ge (k-1)^2+1$ (Lemma 2).
Thus
\[
\frac{R(k+1,k)}{R(k,k)} \ge 1 + \frac{k-1}{R(k,k)} \ge 1 + \frac{k-1}{(k-1)^2+1} = 1 + \Theta\!\left(\frac{1}{k}\right),
\]
which does not give a constant $c>0$.

(ii) \emph{First gap (crisp).} Prove or disprove: there exists $c>0$ and $k_0$ such that for all $k\ge k_0$,
\[R(k+1,k)\ge (1+c)R(k,k).\]

(iii) \emph{Top 3 next moves.}
1. Strengthen Lemma 1 from an additive gap $\Omega(k)$ to a gap $\Omega(R(k,k))$ (even on a subsequence), e.g. by constructing colourings on $(1+o(1))R(k,k)$ vertices avoiding red $K_{k+1}$ and blue $K_k$.
2. Try to relate $R(k+1,k)$ to off-diagonal asymptotics (e.g. $R(k+1,k)\approx R(k,k)$ would suggest very tight off-diagonal control) and test numerically for small $k$ via SAT.
3. Investigate whether known lower-bound constructions for $R(k,k)$ can be modified to avoid red $K_{k+1}$ without significantly shrinking $n$; formally, target a lemma of the form ``from a $(k,k)$-Ramsey colouring on $n$ vertices, build a $(k+1,k)$-Ramsey colouring on $(1+\delta)n$ vertices''.

(iv) \emph{What a minimal counterexample would likely look like.} If the statement is false, one expects an infinite sequence $k\to\infty$ for which there exist colourings on $n\approx (1+o(1))R(k,k)$ vertices that avoid both a red $K_{k+1}$ and a blue $K_k$, forcing $R(k+1,k)/R(k,k)\to 1$ along that sequence.


