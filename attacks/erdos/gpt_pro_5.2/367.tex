\section*{Problem 367: Products of $2$-full parts in short intervals}

\subsection*{FORMAL RESTATEMENT}
For $n\in\mathbb{N}$, define $B_2(n)$ to be the \emph{$2$-full part} of $n$: if $n=\prod_p p^{e_p}$ then
\[
B_2(n) := \prod_{e_p\ge 2} p^{e_p}.
\]
Equivalently, $B_2(n)=n/n'$ where $n'$ is the product of primes dividing $n$ to exact exponent $1$.

The prompt asks, for fixed $k\ge 1$, whether
\[
\prod_{n\le m<n+k} B_2(m) \ll n^{2+o(1)}
\]
and even whether the stronger bound
\[
\prod_{n\le m<n+k} B_2(m) \ll_k n^{2}
\tag{$\dagger$}
\]
holds.

\textbf{Ambiguity / minimal correction.} The ``$n^{2+o(1)}$'' version remains a genuine open question in the prompt. However, the stronger bound $(\dagger)$ is a precise statement. I will give an explicit infinite family disproving $(\dagger)$ for every $k\ge 3$, and I will also verify that $(\dagger)$ is trivially true for $k\le 2$.

\subsection*{QUICK LITERATURE/CONTEXT CHECK}
The Erd\H{o}s-problems discussion thread for \#367 contains an explicit Pell-equation construction (due to W. van Doorn, streamlined/verified by T. Tao) producing infinitely many $n$ such that
\[
\prod_{n\le m<n+3} B_2(m) \gg n^2 \log n,
\]
which immediately rules out $(\dagger)$ for $k\ge 3$.

\subsection*{ATTACK PLAN}
\begin{enumerate}
\item Prove $(\dagger)$ for $k\le 2$ directly from $B_2(m)\le m$ and $\gcd(m,m+1)=1$.
\item For $k\ge 3$, construct infinitely many $n$ for which $B_2(n)=n$ and $B_2(n+1)=n+1$ (via the Pell equation $x^2-8y^2=1$), and in addition force $n+2$ to be divisible by a large power $5^t$.
\item Use $5^t\mid (n+2)$ (with $t\ge 2$) to get $B_2(n+2)\ge 5^t$, and relate $5^t$ to $\log n$ to obtain the lower bound $\gg n^2\log n$.
\end{enumerate}

\subsection*{WORK (complete disproof of $(\dagger)$ for $k\ge 3$)}
\subsubsection*{1. The case $k\le 2$ (trivial upper bound $\ll n^2$)}
Since $B_2(m)\le m$ for all $m$, we have
\[
\prod_{n\le m<n+1} B_2(m) = B_2(n)\le n\le n^2.
\]
For $k=2$, using $\gcd(n,n+1)=1$ is not even necessary:
\[
\prod_{n\le m<n+2} B_2(m) = B_2(n)B_2(n+1)\le n(n+1) \le 2n^2\qquad (n\ge 1).
\]
Thus $(\dagger)$ holds for $k\le 2$.

\subsubsection*{2. Pell solutions and the sequence $n_j$}
Let $\alpha := 3+\sqrt{8}$, so $\alpha^{-1}=3-\sqrt{8}$.
Define integer sequences $(x_j,y_j)$ by
\[
 x_j + y_j\sqrt{8} := \alpha^j \qquad (j\ge 0).
\]
Then $x_j,y_j\in\mathbb{Z}$, $x_j>0$, $y_j\ge 0$, and
\[
 x_j^2 - 8y_j^2 = 1
\]
for all $j$ (taking norms in $\mathbb{Z}[\sqrt{8}]$).
Now set
\[
 n_j := 8y_j^2.
\]
Then
\[
 n_j+1 = 8y_j^2+1 = x_j^2.
\]

\paragraph{Lemma 2.1.} For every $j\ge 0$,
\[
B_2(n_j)=n_j\quad\text{and}\quad B_2(n_j+1)=n_j+1.
\]
\begin{proof}
We have $n_j=8y_j^2=2^3\cdot y_j^2$. In the prime factorization of $y_j^2$, every exponent is even; multiplying by $2^3$ makes the $2$-exponent at least $3$. Hence every prime exponent in $n_j$ is $\ge 2$, so $n_j$ is powerful and $B_2(n_j)=n_j$.
Also $n_j+1=x_j^2$ is a perfect square, hence powerful, so $B_2(n_j+1)=n_j+1$.
\end{proof}

\subsubsection*{3. Forcing a large power of $5$ to divide $n_j+2$}
We will exhibit indices $j=j_t$ such that $5^t\mid (n_{j}+2)$.

First note the explicit computation
\[
\alpha^3 = (3+\sqrt{8})^3 = 99 + 35\sqrt{8} = -1 + 5(20+7\sqrt{8}).
\tag{1}
\]
Similarly $\alpha^{-3}=-1+5(20-7\sqrt{8})$.

\paragraph{Lemma 3.1.} For every $t\ge 1$ there exist integers $a_t,b_t\in\mathbb{Z}$ such that
\[
\alpha^{\pm 3\cdot 5^{t-1}} = -1 + 5^t\bigl(a_t \pm b_t\sqrt{8}\bigr).
\tag{2}
\]
\begin{proof}
We argue by induction on $t$.
For $t=1$, (2) is exactly (1), with $(a_1,b_1)=(20,7)$.
Assume (2) holds for some $t\ge 1$ with the $+$ sign, i.e.
$\alpha^{3\cdot 5^{t-1}}=-1+5^t w$ with $w\in\mathbb{Z}[\sqrt{8}]$.
Then
\[
\alpha^{3\cdot 5^{t}} = \bigl(\alpha^{3\cdot 5^{t-1}}\bigr)^5 = (-1+5^t w)^5.
\]
Expanding by the binomial theorem,
\[
(-1+5^t w)^5 = -1 + 5\cdot 5^t w + \sum_{j=2}^5 \binom{5}{j} (-1)^{5-j} (5^t w)^j.
\]
Every term in the sum over $j\ge 2$ is divisible by $5^{2t}$ in $\mathbb{Z}[\sqrt{8}]$, hence (since $t\ge 1$) divisible by $5^{t+1}$. Therefore
\[
(-1+5^t w)^5 = -1 + 5^{t+1} w'
\]
for some $w'\in\mathbb{Z}[\sqrt{8}]$. Writing $w'=a_{t+1}+b_{t+1}\sqrt{8}$ gives (2) for $t+1$.
The statement for the $-$ sign follows by applying the same argument to $\alpha^{-1}$ (or by conjugation $\sqrt{8}\mapsto -\sqrt{8}$).
\end{proof}

Now define
\[
 j_t := \frac{3\cdot 5^{t-1}-1}{2}\in\mathbb{N}.
\]
(This is an integer because $3\cdot 5^{t-1}$ is odd.) Then $2j_t+1 = 3\cdot 5^{t-1}$.

\paragraph{Lemma 3.2.} For every $t\ge 1$,
\[
5^t \mid (n_{j_t}+2).
\]
\begin{proof}
We first record a closed form for $n_j$.
From $x_j+y_j\sqrt{8}=\alpha^j$ and conjugation,
\[
 x_j = \frac{\alpha^j+\alpha^{-j}}{2},\qquad y_j = \frac{\alpha^j-\alpha^{-j}}{2\sqrt{8}}.
\]
Hence
\[
 n_j = 8y_j^2 = 8\left(\frac{\alpha^j-\alpha^{-j}}{2\sqrt{8}}\right)^2
     = \frac{\alpha^{2j}-2+\alpha^{-2j}}{4}.
\]
Therefore
\[
4(n_j+2)=\alpha^{2j}+6+\alpha^{-2j}.
\tag{3}
\]
Rewrite $\alpha^{2j}=\alpha^{-1}\alpha^{2j+1}$ and $\alpha^{-2j}=\alpha\alpha^{-(2j+1)}$ to get
\[
4(n_j+2)=\alpha^{-1}\alpha^{2j+1}+6+\alpha\alpha^{-(2j+1)}.
\tag{4}
\]
Now take $j=j_t$ so $2j+1=3\cdot 5^{t-1}$. Using Lemma 3.1,
\[
\alpha^{\pm(2j_t+1)}=\alpha^{\pm 3\cdot 5^{t-1}}=-1+5^t(a_t\pm b_t\sqrt{8}).
\]
Substitute into (4) and use $\alpha^{\pm 1}=3\pm\sqrt{8}$:
\begin{align*}
4(n_{j_t}+2)
&=(3-\sqrt{8})\bigl(-1+5^t(a_t+b_t\sqrt{8})\bigr)+6+(3+\sqrt{8})\bigl(-1+5^t(a_t-b_t\sqrt{8})\bigr)\\
&=\underbrace{(3-\sqrt{8})(-1)+(3+\sqrt{8})(-1)+6}_{=0}
 +5^t\Bigl((3-\sqrt{8})(a_t+b_t\sqrt{8})+(3+\sqrt{8})(a_t-b_t\sqrt{8})\Bigr).
\end{align*}
A direct expansion shows the bracket equals $6a_t-16b_t\in\mathbb{Z}$, so
\[
4(n_{j_t}+2)=5^t(6a_t-16b_t),
\]
which implies $5^t\mid 4(n_{j_t}+2)$. Since $\gcd(4,5)=1$, we conclude $5^t\mid (n_{j_t}+2)$.
\end{proof}

\subsubsection*{4. Lower bound for the product and failure of $(\dagger)$ for $k\ge 3$}
Fix $t\ge 2$ and set $n:=n_{j_t}$.
By Lemma 2.1, $B_2(n)=n$ and $B_2(n+1)=n+1$.
By Lemma 3.2, $5^t\mid (n+2)$, hence $v_5(n+2)\ge 2$ and therefore the full power $5^t$ is included in $B_2(n+2)$. In particular,
\[
B_2(n+2)\ge 5^t.
\]
Consequently,
\[
\prod_{n\le m<n+3} B_2(m) = B_2(n)B_2(n+1)B_2(n+2) \ge n(n+1)5^t \ge n^2\,5^t.
\tag{5}
\]

\paragraph{Relating $5^t$ to $\log n$.}
Since $n=n_{j_t}=\frac{\alpha^{2j_t}-2+\alpha^{-2j_t}}{4}<\frac{\alpha^{2j_t}}{2}$ for $j_t\ge 1$, we have
\[
\log n < 2j_t\log\alpha.
\]
Using $2j_t=3\cdot 5^{t-1}-1<3\cdot 5^{t-1}$, we obtain
\[
\log n < 3\cdot 5^{t-1}\log\alpha.
\]
Thus
\[
5^t = 5\cdot 5^{t-1} > \frac{5}{3\log\alpha}\,\log n.
\tag{6}
\]
Combine (5) and (6) to get
\[
\prod_{n\le m<n+3} B_2(m) \ge \frac{5}{3\log\alpha}\, n^2\log n.
\]
This holds for infinitely many $n$ (namely $n=n_{j_t}$, $t\ge 2$), proving
\[
\prod_{n\le m<n+3} B_2(m) \gg n^2\log n\quad\text{infinitely often}.
\]
Finally, for any $k\ge 3$,
\[
\prod_{n\le m<n+k} B_2(m) \ge \prod_{n\le m<n+3} B_2(m),
\]
so the same lower bound holds infinitely often for all $k\ge 3$.

\paragraph{Conclusion (disproof of $(\dagger)$).}
If $(\dagger)$ were true for some $k\ge 3$, we would have $\prod_{n\le m<n+k}B_2(m)\le C_k n^2$ for all large $n$, contradicting that the left-hand side is $\gg n^2\log n$ infinitely often. Hence $(\dagger)$ is false for every $k\ge 3$.

\subsection*{VERIFICATION}
\begin{itemize}
\item For $t=2$, the construction gives $j_2=7$ and a (large) integer $n_{j_2}=8y_7^2$ with $v_5(n_{j_2}+2)=2$, confirming the divisibility phenomenon.
\item The only place where we used $t\ge 2$ is to ensure $5^t$ contributes to $B_2(n+2)$ (since primes with exponent $1$ are removed).
\item All inequalities used to compare $5^t$ and $\log n$ are one-sided and elementary.
\end{itemize}

\subsection*{META-CHECK (common failure modes)}
\begin{itemize}
\item \emph{Confusing ``$2$-full'' vs ``squarefree kernel'':} checked that $B_2$ keeps exactly primes with exponent $\ge 2$.
\item \emph{Forgetting that $t\ge 2$ is needed:} explicitly enforced.
\item \emph{Algebra in $\mathbb{Z}[\sqrt{8}]$:} re-expanded the key bracket to confirm the $\sqrt{8}$ terms cancel.
\item \emph{Off-by-one in $j_t$:} verified $2j_t+1=3\cdot 5^{t-1}$.
\end{itemize}

\subsection*{FINAL}
FINAL: LABEL: \textbf{FULL SOLUTION} \quad SUBLABEL: \textbf{COUNTEREXAMPLE/DISPROOF}.

\subsection*{COMPLETION ESTIMATE}
COMPLETION ESTIMATE: 100\%.

