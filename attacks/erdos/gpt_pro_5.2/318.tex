\section*{Erd\H{o}s Problem \#318}

\subsection*{1) RESTATE}
Let $A\subset\mathbb{N}$ be an infinite arithmetic progression and let $f:A\to\{-1,1\}$ be non-constant.
\begin{enumerate}[label=(\alph*)]
\item Must there exist a finite nonempty set $S\subset A$ such that
\[
\sum_{n\in S}\frac{f(n)}{n}=0\ ?
\]
\item What if $A\subset\mathbb{N}$ is any set of positive (upper) density?
\item What if $A=\{4,9,16,25,\dots\}$ is the set of squares excluding $1$?
\end{enumerate}

\subsection*{2) KNOWN FACTS}
\begin{enumerate}[label=(\alph*)]
\item For $A=\mathbb{N}$ the answer to (1a) is yes (Erd\H{o}s--Straus).
\item For $A$ an arithmetic progression, the answer to (1a) is known to be yes (Sattler).
\item For general positive-density sets, the answer to (1b) is no: there are counterexamples.
\item For squares $\{n^2:n\ge 2\}$, the problem appears open.
\item (Modern tool) Bloom proved that any subset $B\subset\mathbb{N}$ of positive upper density contains a finite $T\subset B$ with $\sum_{t\in T} 1/t=1$.
\end{enumerate}

\subsection*{3) PROOF STRATEGY}
Provide a direct proof of (1a) for arithmetic progressions using Bloom's density theorem on unit fractions.
Then give a parity-based counterexample for (1b).
Finally, record the (apparently) open status for squares.

\subsection*{4) ATTEMPTED PROOF}
\paragraph{Part (1a): arithmetic progressions.}
Let $A=\{a+kd:k\ge 0\}$ with $a,d\in\mathbb{N}$.
Partition $A$ into
\[
P:=\{n\in A:f(n)=+1\},\qquad N:=\{n\in A:f(n)=-1\}.
\]
Since $A$ has positive natural density $1/d$, at least one of $P,N$ has positive upper density in $\mathbb{N}$.

\medskip
\noindent\emph{Lemma 4.1 (multiples of an element stay dense in an arithmetic progression).}
Fix $n_0\in A$. Then
\[
A_{n_0}:=\{m\in A: n_0\mid m\}
\]
is an infinite arithmetic progression, hence has positive density.
\emph{Proof.}
We need solutions to the congruence $a+kd\equiv 0\pmod{n_0}$.
Because $n_0\in A$, we have $n_0\equiv a\pmod d$, so $\gcd(d,n_0)\mid a$ and the congruence has a solution $k_0$.
All solutions are $k\equiv k_0\pmod{n_0/\gcd(d,n_0)}$, hence $A_{n_0}$ is an arithmetic progression with common difference $\operatorname{lcm}(d,n_0)=dn_0/\gcd(d,n_0)$.
\qed

\medskip
\noindent\emph{Lemma 4.2 (Bloom's density theorem, used as a black box).}
If $B\subset\mathbb{N}$ has positive upper density, then there exists a finite $T\subset B$ with $\sum_{t\in T}1/t=1$.
\qed

\medskip
\noindent\emph{Case 1: both $P$ and $N$ have positive upper density.}
By Lemma 4.2 choose finite $S_+\subset P$ and $S_-\subset N$ with
\(\sum_{n\in S_+}1/n=1\) and \(\sum_{n\in S_-}1/n=1\).
Then for $S:=S_+\cup S_-$ we have
\[
\sum_{n\in S}\frac{f(n)}{n}=\sum_{n\in S_+}\frac{1}{n}-\sum_{n\in S_-}\frac{1}{n}=1-1=0.
\]

\medskip
\noindent\emph{Case 2: one sign class has zero upper density.}
Assume without loss of generality that $\overline d(N)=0$ (upper density $0$), so $P$ has the same positive upper density as $A$.
Pick any $n_0\in N$.
By Lemma 4.1, $A_{n_0}$ has positive density.
Since $N\cap A_{n_0}\subset N$ has upper density $0$, the set
\[
M:=P\cap A_{n_0}=A_{n_0}\setminus (N\cap A_{n_0})
\]
has the same positive density as $A_{n_0}$.
Define the scaled set
\[
B:=\left\{\frac{m}{n_0}: m\in M\right\}\subset \mathbb{N}.
\]
Because $M=n_0 B$, $B$ also has positive upper density.
By Lemma 4.2 there exists finite $T\subset B$ with $\sum_{t\in T}1/t=1$.
Let $S_+:=\{n_0 t: t\in T\}\subset M\subset P$.
Then
\[
\sum_{m\in S_+}\frac{1}{m}=\sum_{t\in T}\frac{1}{n_0 t}=\frac{1}{n_0}\sum_{t\in T}\frac{1}{t}=\frac{1}{n_0}.
\]
Finally, set $S:=S_+\cup\{n_0\}$. Since $f(n_0)=-1$ and $f\equiv +1$ on $S_+$,
\[
\sum_{n\in S}\frac{f(n)}{n}=\sum_{m\in S_+}\frac{1}{m}-\frac{1}{n_0}=\frac{1}{n_0}-\frac{1}{n_0}=0.
\]
This completes a proof of (1a) for any arithmetic progression $A$.

\paragraph{Part (1b): positive density sets. Counterexample.}
Let $A:=\{2\}\cup\{\text{all odd positive integers}\}$. Then $A$ has natural density $1/2$.
Define $f(2)=-1$ and $f(n)=+1$ for all odd $n$.
Take any finite nonempty $S\subset A$.
If $2\notin S$, then $\sum_{n\in S} f(n)/n>0$.
If $2\in S$, then
\[
\sum_{n\in S}\frac{f(n)}{n}=\sum_{\substack{n\in S\\ n\ \text{odd}}}\frac{1}{n}-\frac{1}{2}.
\]
But a sum of reciprocals of odd integers is a rational number whose reduced denominator is odd (a common denominator is the product of the odd integers, hence odd; cancelling a gcd cannot introduce a factor $2$).
Therefore it cannot equal $1/2$.
Hence the displayed quantity cannot be $0$.
So (1b) has a negative answer.

\paragraph{Part (1c): squares.}
For $A=\{n^2:n\ge 2\}$, the answer appears to be open.

\subsection*{5) OBSTACLES}
\begin{enumerate}[label=(\alph*)]
\item Part (1a) above relies on Bloom's deep theorem about unit fractions in positive density sets; obtaining an elementary proof specialized to arithmetic progressions would be interesting but is not provided here.
\item For squares, current methods (including density-based ones) do not apply because the squares have zero density and the reciprocal sum $\sum 1/n^2$ converges.
\end{enumerate}

\subsection*{6) FINAL}
\textbf{UNRESOLVED.}
\begin{enumerate}[label=(\roman*)]
\item \textbf{What I tried:} Gave a complete proof of the arithmetic progression case using Bloom's density theorem; gave an explicit counterexample for the positive-density variant.
\item \textbf{Where it fails / stuck:} The squares case remains open; I do not have a proof or counterexample there.
\item \textbf{What might work next:} Look for square-specific Egyptian-fraction identities (reciprocal-square identities) that could simulate the ``density'' argument, or find an obstruction (e.g. via denominator parity/valuation) robust enough to rule out zero sums.
\item \textbf{Confidence:} High confidence in the proof of (1a) conditional on Bloom's theorem, and in the counterexample for (1b). The open status of (1c) is based on current literature reports.
\end{enumerate}

\subsection*{7) WRITE-UP (clean, complete)}
For an arithmetic progression $A$ and a non-constant $f:A\to\{-1,1\}$, let $P$ and $N$ be the $+1$ and $-1$ classes.
If both have positive upper density, Bloom's theorem gives finite subsets of each whose reciprocal sums equal $1$, and their signed union has sum $0$.
If one class (say $N$) has zero upper density, choose $n_0\in N$.
The set of multiples of $n_0$ inside $A$ is itself an arithmetic progression, hence has positive density.
Removing the density-zero subset of those multiples lying in $N$ leaves a positive-density set $M\subset P$ of positive multiples of $n_0$.
Scaling by $n_0$ produces a positive-density set $B\subset\mathbb{N}$, so Bloom's theorem gives a finite $T\subset B$ with $\sum_{t\in T}1/t=1$.
Scaling back gives a finite $S_+\subset P$ with $\sum_{m\in S_+}1/m=1/n_0$, so $S_+\cup\{n_0\}$ yields a signed reciprocal sum $0$.

For general positive-density sets, the claim is false: taking $A=\{2\}\cup\{\text{odds}\}$ and $f(2)=-1$, $f(\text{odd})=+1$ gives a parity obstruction, since a sum of reciprocals of odd integers can never equal $1/2$.

For the set of squares $\{n^2:n\ge 2\}$, the problem appears open.

