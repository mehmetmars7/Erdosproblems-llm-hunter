\section*{Problem 638}

\subsection*{1) FORMAL RESTATEMENT}
Let $\mathcal S$ be a family of \emph{finite} (simple, undirected) graphs.
Assume that for every integer $n\ge 1$ there exists a graph $G_n\in\mathcal S$ such that
\[
  G_n \to (K_3)^{e}_n,
\]
meaning: for every edge-colouring $c:E(G_n)\to\{1,\dots,n\}$ there is a monochromatic triangle.

Question (as written). Must it then be true that for every infinite cardinal $\kappa$ there exists a graph $G$ such that

\begin{enumerate}[label=(\alph*)]
\item every finite subgraph of $G$ belongs to $\mathcal S$, and
\item $G\to (K_3)^{e}_\kappa$ (every $\kappa$-edge-colouring of $G$ contains a monochromatic triangle)?
\end{enumerate}

\subsection*{2) QUICK LITERATURE/CONTEXT CHECK}
The formulation is ambiguous/degenerate unless one assumes additional closure properties of $\mathcal S$ (e.g. being hereditary under taking finite subgraphs). Under the literal reading (no closure assumptions), there are immediate counterexamples (below).  A natural ``minimal correction'' is to add that $\mathcal S$ is closed under taking finite subgraphs (and perhaps has a joint-embedding property so that some infinite $G$ with age contained in $\mathcal S$ even exists); under such strengthened hypotheses the question becomes nontrivial.

\subsection*{3) ATTACK PLAN}
Give a concrete family $\mathcal S$ satisfying the hypothesis (Ramsey-for-$K_3$ graphs exist for every $n$), but such that \emph{no} graph $G$ can have all its finite subgraphs in $\mathcal S$.
This disproves the conclusion already at clause (a), hence answers the literal question in the negative.

\subsection*{4) WORK}
\paragraph{Step 1: Fix Ramsey witnesses.}
For each $n\ge 1$, let $R_3(n)$ denote the (finite) Ramsey number for monochromatic triangles under $n$ edge-colours; i.e.
$R_3(n)$ is the least integer $N$ such that
\[K_N \to (K_3)^e_n.\]
Existence of such $R_3(n)$ follows from the classical finite Ramsey theorem.

\paragraph{Step 2: Define the family $\mathcal S$.}
Set
\[
  \mathcal S := \{\,K_{R_3(n)} : n\in\mathbb N,\ n\ge 1\,\}.
\]
Then for each $n$ we may take $G_n := K_{R_3(n)}\in\mathcal S$, and by definition $G_n\to (K_3)^e_n$.
So the hypothesis of the problem holds.

\paragraph{Step 3: Show the conclusion fails.}
Suppose (for contradiction) that there exists \emph{some} graph $G$ such that every finite subgraph of $G$ belongs to $\mathcal S$.
Pick any vertex $v\in V(G)$. The one-vertex graph $K_1$ is a finite subgraph of $G$ (take vertex set $\{v\}$ and no edges), hence $K_1\in\mathcal S$.
But by construction $\mathcal S$ consists only of complete graphs $K_{R_3(n)}$ with $R_3(n)\ge 3$, so $K_1\notin\mathcal S$.
This contradiction shows that \emph{no} such $G$ exists.

Therefore the asserted implication is false already at the level of (a), so the answer to the question as written is \emph{no}.

\subsection*{5) VERIFICATION / SANITY CHECKS}
\begin{itemize}
\item For each $n$, $G_n:=K_{R_3(n)}\in\mathcal S$ satisfies $G_n\to (K_3)^e_n$ by the definition of $R_3(n)$.
\item Any graph $G$ has a finite subgraph on one vertex (and also on two vertices, etc.). Since $K_1\notin\mathcal S$, no graph can have all finite subgraphs in $\mathcal S$.
\item Hence the conclusion ``for every infinite $\kappa$ there exists $G$ with (a) and (b)'' fails.
\end{itemize}

\subsection*{6) FINAL}
\noindent\textbf{LABEL: FULL SOLUTION}\\
\textbf{SUBLABEL: COUNTEREXAMPLE/DISPROOF}\\
Taking $\mathcal S=\{K_{R_3(n)}:n\ge 1\}$ satisfies the hypothesis, but there is no graph $G$ all of whose finite subgraphs lie in $\mathcal S$ (since $K_1\notin\mathcal S$). Thus the implication is false as stated.

\subsection*{7) COMPLETION ESTIMATE}
\noindent\textbf{COMPLETION: 100\%}

\hrule

