
FORMAL RESTATEMENT

Fix an integer $k\ge 2$.
Let $V$ be the set of all $k$-element subsets of $[2k]=\{1,2,\dots,2k\}$.
Does there exist a coloring $c:V\to\{1,2,\dots,k+1\}$ such that for every $(k+1)$-element subset $A\subset[2k]$, all $k+1$ colors appear among the $k$-subsets of $A$?

Equivalently (as stated in the problem text), let $J(2k,k)$ be the Johnson graph whose vertices are $k$-subsets of $[2k]$ and where two vertices are adjacent if they intersect in exactly $k-1$ elements.
The question is whether $\chi(J(2k,k))=k+1$ for some $k>2$.

QUICK LITERATURE/CONTEXT CHECK

The problem text notes:
- The condition is trivially possible for $k=2$.
- It is equivalent to asking whether $\chi(J(2k,k))=k+1$.
- It reports that a website listing chromatic numbers shows the statement is false for $3\le k\le 8$.
No external results beyond what is proved below are assumed.

ATTACK PLAN

Proof track:
(1) Translate the coloring condition into an extremal partition problem inside the Johnson graph.
(2) Use counting to constrain the structure of a hypothetical $(k+1)$-coloring (Hoffman/ratio bound style, but proved elementarily here).

Disproof track:
(1) Show that $\chi(J(2k,k))>k+1$ for all $k>2$ by deriving a contradiction from the required structure.
(2) At minimum, disprove existence for small $k$ by exact computation.

WORK

Proposition 0 (equivalence of formulations).
A coloring $c$ as in the problem exists if and only if $J(2k,k)$ admits a proper coloring with exactly $k+1$ colors, i.e. $\chi(J(2k,k))\le k+1$.
Since $J(2k,k)$ contains cliques of size $k+1$, this is equivalent to $\chi(J(2k,k))=k+1$.

Proof.
If $A\subset[2k]$ has size $k+1$, then its $k$-subsets are exactly the $k+1$ sets $A\setminus\{a\}$ for $a\in A$.
Any two of these $k$-subsets intersect in exactly $k-1$ elements, hence they form a clique of size $k+1$ in $J(2k,k)$.
If every such clique uses all $k+1$ colors, then in particular adjacent vertices have different colors, so $c$ is a proper coloring.
Conversely, if $c$ is a proper coloring using at most $k+1$ colors, then every clique of size $k+1$ must use $k+1$ distinct colors, hence all colors appear on the $k$-subsets of each $(k+1)$-set.
Finally, because a clique of size $k+1$ exists, $\chi(J(2k,k))\ge k+1$, so $\chi\le k+1$ implies $\chi=k+1$.
$\square$

Lemma 1 (maximum independent set bound).
Let $\mathcal{F}$ be an independent set in $J(2k,k)$ (i.e. a family of $k$-subsets of $[2k]$ such that no two differ by exactly one element).
Then
\[
|\mathcal{F}|\le \frac{1}{k+1}\binom{2k}{k}.
\]
Moreover, equality holds if and only if $\mathcal{F}$ is a Steiner system $S(k-1,k,2k)$, i.e. every $(k-1)$-subset of $[2k]$ is contained in exactly one member of $\mathcal{F}$.

Proof.
A $(k-1)$-subset $S\subset[2k]$ is contained in exactly $k+1$ different $k$-subsets of $[2k]$ (one for each choice of the missing element in $[2k]\setminus S$).
If two distinct members $B_1,B_2\in\mathcal{F}$ both contain the same $(k-1)$-subset $S$, then $|B_1\cap B_2|\ge k-1$.
Because $B_1\ne B_2$ and both have size $k$, this forces $|B_1\cap B_2|=k-1$, so $B_1$ and $B_2$ are adjacent in $J(2k,k)$.
This contradicts that $\mathcal{F}$ is independent.
Therefore, each $(k-1)$-subset $S$ is contained in at most one block of $\mathcal{F}$.

Count incidences between blocks in $\mathcal{F}$ and $(k-1)$-subsets.
Each block $B\in\mathcal{F}$ contains exactly $k$ different $(k-1)$-subsets (obtained by deleting one element of $B$).
The total number of $(k-1)$-subsets of $[2k]$ is $\binom{2k}{k-1}$.
Since each such $(k-1)$-subset is contained in at most one member of $\mathcal{F}$, we have
\[
 k|\mathcal{F}|\le \binom{2k}{k-1}.
\]
Using the identity $\binom{2k}{k-1}=\frac{k}{k+1}\binom{2k}{k}$, we obtain
\[
 |\mathcal{F}|\le \frac{1}{k+1}\binom{2k}{k}.
\]
Equality holds if and only if every $(k-1)$-subset is contained in exactly one member of $\mathcal{F}$, which is precisely the Steiner system condition $S(k-1,k,2k)$.
$\square$

Lemma 2 (structure forced by a $(k+1)$-coloring).
If $J(2k,k)$ has a proper coloring with exactly $k+1$ colors, then the color classes form a partition of $\binom{[2k]}{k}$ into $k+1$ disjoint Steiner systems $S(k-1,k,2k)$.

Proof.
Let $V=\binom{[2k]}{k}$ be the vertex set and suppose $c:V\to\{1,\dots,k+1\}$ is a proper coloring.
For each color $i$, the color class $\mathcal{F}_i=c^{-1}(i)$ is an independent set.
By Lemma~1,
\[
|\mathcal{F}_i|\le \frac{1}{k+1}\binom{2k}{k}.
\]
Summing over $i$ gives
\[
|V|=\binom{2k}{k}=\sum_{i=1}^{k+1} |\mathcal{F}_i|\le (k+1)\cdot \frac{1}{k+1}\binom{2k}{k}=\binom{2k}{k}.
\]
Thus all inequalities must be equalities.
In particular, for every $i$ we have $|\mathcal{F}_i|=\frac{1}{k+1}\binom{2k}{k}$, so each $\mathcal{F}_i$ attains equality in Lemma~1 and is therefore a Steiner system $S(k-1,k,2k)$.
Also the $\mathcal{F}_i$ partition $V$ because every vertex has exactly one color.
$\square$

Fast reality check (computation; exact graph coloring for small $k$).
We computed the chromatic number of $J(2k,k)$ for $k=2,3,4$ by exact backtracking coloring.
The results were:
\[
\chi(J(4,2))=3=k+1,\qquad \chi(J(6,3))=6>4,\qquad \chi(J(8,4))=6>5.
\]
In particular, the desired property fails for $k=3$ and $k=4$.

VERIFICATION

(1) Proposition~0: every edge of $J(2k,k)$ lies in a $(k+1)$-clique coming from a $(k+1)$-subset $A$ (namely $A$ is the union of the two adjacent $k$-subsets). Hence the problem's local condition implies proper coloring.
(2) Lemma~1: the key equivalence “two $k$-sets share a $(k-1)$-subset iff they are adjacent” holds because distinct $k$-sets sharing a $(k-1)$-subset must differ by exactly one element.
(3) Lemma~2: the counting argument forces every color class to be maximum independent, hence Steiner.
(4) The computation provides independent confirmation for small $k$.

FINAL

**UNRESOLVED**

(i) Strongest proved partial result.
We proved a necessary structural condition: if $\chi(J(2k,k))=k+1$ then the $k$-subsets of $[2k]$ must partition into $k+1$ disjoint Steiner systems $S(k-1,k,2k)$ (Lemma~2).
We also computed exactly that $\chi(J(6,3))=6$ and $\chi(J(8,4))=6$, so the property fails for $k=3,4$.

(ii) First gap (crisp statement).
Determine whether there exists any $k>2$ for which such a partition into $k+1$ Steiner systems $S(k-1,k,2k)$ exists (equivalently, $\chi(J(2k,k))=k+1$).

(iii) Top 3 next moves.
1. Nonexistence approach: derive an invariant (parity, modular count, intersection-number constraint) that any large set of $S(k-1,k,2k)$ would have to satisfy, then show it fails for all $k>2$.
2. Existence approach: attempt to explicitly construct such a large set for a candidate $k$ (the problem text mentions $k=6$ as a natural test case) using algebraic/combinatorial designs.
3. Computation: implement SAT/ILP search for $k=5,6$ to test whether a $(k+1)$-coloring exists; if found, extract the corresponding Steiner-system partition.

(iv) Minimal counterexample structure.
If the answer is “yes”, the smallest $k>2$ with $\chi(J(2k,k))=k+1$ would necessarily come with an explicit large set of Steiner systems $S(k-1,k,2k)$; by the computations above, such a minimal $k$ would have to be at least $5$.
If the answer is “no”, then for every $k>2$ one should be able to show that no such large set exists, so a minimal obstruction would likely manifest as a simple integrality or intersection-count contradiction for the Steiner-system partition.

