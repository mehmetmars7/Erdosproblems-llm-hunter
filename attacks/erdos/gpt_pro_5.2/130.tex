% Erdos problem #130
% Attempt for Erdos Problem #130
% Following PROMPT_STRATEGY.MD
% Tools/Constraints:
% - Web browsing available? YES (not used; I restrict to what is stated in 123-137.tex)
% - Computation available (Python)? not needed

1) FORMAL RESTATEMENT

Let $A\subset\mathbb{R}^2$ be an infinite set of points such that
(i) no three points of $A$ are collinear, and
(ii) no four points of $A$ are concyclic.
Define a graph $G(A)$ with vertex set $A$ where two vertices are adjacent iff the Euclidean distance between them is an integer.

Questions:
- How large can the chromatic number $\chi(G(A))$ be?
- How large can the clique number $\omega(G(A))$ be?
- In particular, can $\chi(G(A))$ be infinite?

The text also asks about an infinite complete graph in this metric setting, and notes (by Anning--Erdos 1945) that it is impossible.

2) QUICK LITERATURE/CONTEXT CHECK

(Restricted to statements explicitly present in 123-137.tex.)
- Anning and Erdos (1945) imply the graph cannot contain an infinite complete graph.

3) ATTACK PLAN

- Prove basic combinatorial/geometry constraints implied by "no four cocircular".
- Deduce that an infinite clique cannot occur (using the Anning--Erdos statement from the problem file).
- Give at least one explicit (existence) construction of an infinite $A$ with controlled $\chi$ and $\omega$ to show nontrivial lower bounds.

4) WORK

Lemma 4.1 (At most three neighbors at any fixed integer radius).
Fix $p\in A$ and a positive integer $r$.
Then there are at most 3 points $q\in A$ with $\|p-q\|=r$.

*Proof.* All points $q$ with $\|p-q\|=r$ lie on the circle of radius $r$ centered at $p$.
If there were 4 such points, those 4 points of $A$ would be concyclic, contradicting the hypothesis that no four points of $A$ lie on a circle. \qed

Lemma 4.2 (No infinite clique).
The graph $G(A)$ contains no infinite clique.

*Proof.* An infinite clique in $G(A)$ is an infinite subset $K\subset A$ such that every pair of points in $K$ has integer distance.
The problem text states (Anning--Erdos 1945) that an infinite set of points with all pairwise distances integers is impossible in this setting (in fact such a set must be collinear).
Since $A$ has no three collinear points, it cannot contain such an infinite clique.
Therefore $G(A)$ has no infinite clique. \qed

Proposition 4.3 (Existence of an infinite example with $\chi=3$ and $\omega=3$).
There exists an infinite set $A\subset\mathbb{R}^2$ satisfying the hypotheses (no three collinear, no four concyclic) such that $G(A)$ is a disjoint union of triangles (hence $\omega(G(A))=3$ and $\chi(G(A))=3$).

*Proof (constructive existence by induction).* We build $A$ as a disjoint union of equilateral triangles of side length 1, placed so that:
- within each triangle, all three pairwise distances are 1 (integer), giving a $K_3$ component;
- between distinct triangles, all inter-triangle distances are non-integers (so no extra edges);
- no three points are collinear and no four concyclic.

Inductive step:
Assume we have built a finite set $A_m$ which is a disjoint union of $m$ such triangles and satisfies all geometric conditions and the no-extra-integer-distance condition.
We will place the next triangle far away.

Let $F$ be the union of the following forbidden sets for the location of the next triangle (we choose one vertex first and then the other two relative to it):
- For each pair of points in $A_m$, the line through them (to avoid creating a collinear triple).
- For each triple of points in $A_m$, the unique circle through them (to avoid creating a concyclic quadruple).
- For each point $p\in A_m$ and each integer $r\ge 1$, the circle of radius $r$ centered at $p$ (to avoid creating a new integer distance from any new point to $p$).

For the first two bullets there are finitely many lines/circles; for the third bullet there are countably many circles.
In all cases, each forbidden set is a 1-dimensional subset of the plane.
Hence their union $F$ is a countable union of sets with empty interior, and therefore cannot cover all of $\mathbb{R}^2$.
Choose a point $p_{m+1}\in \mathbb{R}^2\setminus F$.
Now place an equilateral triangle of side 1 with one vertex at $p_{m+1}$ and with the other two vertices obtained by rotating the vector $(1,0)$ by an angle $\theta$ and translating, where $\theta$ is also chosen to avoid the finitely many bad orientations that would place a new vertex on a forbidden line/circle or on a forbidden integer-radius circle around existing points.
Such a choice is possible because only finitely many orientations are excluded at this step.
Let these two new vertices be $q_{m+1},r_{m+1}$.
Set $A_{m+1}:=A_m\cup\{p_{m+1},q_{m+1},r_{m+1}\}$.

By construction:
- the new triangle has integer edges (all 1);
- no new point lies on a forbidden line/circle determined by previous points, so no new collinear triple or concyclic quadruple is formed;
- no new point lies on any integer-radius circle around any old point, so there are no new integer distances between different triangles.

Thus the induction continues, producing an infinite set $A=\bigcup_m A_m$.
The resulting graph $G(A)$ is a disjoint union of triangles, so $\omega(G(A))=3$ and $\chi(G(A))=3$. \qed

5) VERIFICATION

- Lemma 4.1 is immediate from the "no four concyclic" hypothesis.
- Lemma 4.2 uses the Anning--Erdos statement explicitly given in the problem file.
- In Proposition 4.3, at each finite stage we avoid a countable union of circles/lines; such a union cannot cover the plane, so a choice exists. The construction guarantees the graph is exactly a disjoint union of triangles.

6) FINAL

**UNRESOLVED**

(i) Strongest fully proved partial result:
- For any such $A$, each vertex has at most 3 neighbors at any fixed integer distance (Lemma 4.1).
- No infinite clique can occur (Lemma 4.2).
- There exist infinite examples with $\chi=3$ and $\omega=3$ (Proposition 4.3).

(ii) Exact first gap:
- Determine whether $\chi(G(A))$ can be infinite (and more generally, characterize possible growth of $\chi$ and $\omega$ over all such $A$).

(iii) Top 3 next moves:
1. Try to embed known high-chromatic graphs as integer-distance graphs while maintaining the no-three-collinear/no-four-concyclic constraints.
2. Prove upper bounds on $\chi(G(A))$ from geometric constraints (e.g. degree growth restrictions implied by Lemma 4.1 plus incidence bounds).
3. Investigate whether forbidding 4 concyclic points forces bounded chromatic number for integer-distance graphs, perhaps via circle-packing type arguments.

(iv) What a minimal counterexample would likely look like:
- An infinite configuration $A$ producing arbitrarily large finite subgraphs of $G(A)$ with large chromatic number but without large cliques (since infinite cliques are impossible), requiring very controlled integer-distance incidence across many different radii.


