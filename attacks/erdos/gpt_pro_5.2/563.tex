% Erdos problem #563
## 1) FORMAL RESTATEMENT

Let (n\in\mathbb{N}) and (\alpha\in[0,1/2]). Consider the complete simple graph (K_n) on vertex set ([n]={1,2,\dots,n}). A **2‑colouring** is a map
[
\chi:\binom{[n]}{2}\to{\text{red},\text{blue}}.
]
For any (X\subseteq[n]) with (|X|=s), let

* (e_R(X)) be the number of red edges with both endpoints in (X),
* (e_B(X)) be the number of blue edges with both endpoints in (X).

Then
[
e_R(X)+e_B(X)=\binom{s}{2}.
]

Define (F(n,\alpha)) to be the **smallest integer (m)** such that there exists a 2‑colouring (\chi) of (E(K_n)) for which the following holds:

[
\forall X\subseteq[n]\ \text{with}\ |X|\ge m:\quad e_R(X)>\alpha\binom{|X|}{2}\ \text{and}\ e_B(X)>\alpha\binom{|X|}{2}.
\tag{(*)}
]

The claim to prove (as written) is:

> For every (\alpha\in[0,1/2]), there exists a constant (c_\alpha>0) such that as (n\to\infty),
> [
> F(n,\alpha)\sim c_\alpha\log n.
> \tag{Claim}
> ]

### Ambiguity/misstated points (must resolve)

1. **Is (m) allowed to exceed (n)?**
The definition “smallest (m) such that … every (X\subseteq[n]) with (|X|\ge m) …” *allows* (m>n), in which case the condition ((*)) is **vacuous** (there are no such (X)).
This is how the ErdősProblems page states it, without restricting (m\le n). ([Erdős Problems][1])

2. **Endpoint (\alpha=1/2) with strict “more than”:**
For any (s\ge2), it is impossible to have **both** (e_R(X)>\tfrac12\binom{s}{2}) and (e_B(X)>\tfrac12\binom{s}{2}) because their sum is (\binom{s}{2}).
This will force (F(n,1/2)=n+1) (if (m) may exceed (n)), which contradicts (\sim c\log n).

### Minimal corrected statement (standard convention)

The minimal correction consistent with the intended “(\Theta(\log n))” behaviour is to restrict to
[
0\le \alpha < \tfrac12.
\tag{Corrected domain}
]
(Alternatively one could replace “(>)” by “(\ge)” and then treat parity issues, but that is *not* the given statement.)

---

## 2) QUICK LITERATURE/CONTEXT CHECK (web browsing used)

The ErdősProblems statement for #563 matches your “smallest (m)” definition and also states the asymptotic conjecture (F(n,\alpha)\sim c_\alpha\log n) for (0\le\alpha\le1/2). ([Erdős Problems][1])
It also notes that for (\alpha=0) this recovers the classical diagonal Ramsey setting (avoid a monochromatic clique of size (m)). ([Erdős Problems][1])

Separately, a standard survey on Ramsey numbers remarks that “the limit of (R(k,k)^{1/k}), **if it exists** …” indicating that even the existence of a clean exponential growth constant for diagonal Ramsey numbers is not settled in general. ([combinatorics.org][2])
This matters because (\alpha=0) is embedded in the present statement.

---

## 3) ATTACK PLAN

### Disproof / counterexample strategies

1. **Endpoint attack:** test (\alpha=\tfrac12). With strict “(>)”, the requirement demands both colours exceed half the edges inside each large (X), which is arithmetically impossible for any (|X|\ge2). This should force (F(n,1/2)) to be (n+1) (or undefined if one forbids (m>n)), contradicting (\sim c\log n).
2. **Tiny‑(n) brute force sanity check:** compute (F(n,\alpha)) for small (n) and (\alpha) to detect definitional edge cases.

### Proof strategies (for the corrected (\alpha<1/2) version)

1. **Upper bound:** random 2‑colouring + Chernoff/Hoeffding + union bound to get (F(n,\alpha)\le C(\alpha)\log n).
2. **Lower bound:** since ((*)) implies no monochromatic (K_m), derive (F(n,\alpha)\ge F(n,0)) and hence (F(n,\alpha)\ge c\log n) using standard Ramsey bounds.

**I will pursue Disproof Strategy 1** because it yields a complete, gap-free disproof of the statement **as written**.

---

## 4) WORK

### Phase 1 — FAST REALITY CHECK (hand + computation)

**Hand check at (\alpha=1/2):**
Take any (X) with (|X|=s\ge2). Then (e_R(X)+e_B(X)=\binom{s}{2}). If both were (>\tfrac12\binom{s}{2}), their sum would be (>\binom{s}{2}), impossible.

**Brute force (exact) for small (n):**
I exhaustively enumerated all (2^{\binom{n}{2}}) 2‑colourings for (n\le 6) and, for each colouring, tested whether there exists an (m\in{1,\dots,n}) satisfying ((*)).
For (\alpha=0.5), the result was “no such (m\le n)” for every (n=2,3,4,5,6). (This matches the hand argument.)

(For comparison: for (\alpha=0,0.1,0.25), the computed values were (F(3,\alpha)=3), (F(4,\alpha)=3), (F(5,\alpha)=3), (F(6,\alpha)=4), consistent with the classical (R(3,3)=6) phenomenon.)

---

### Core disproof: (\alpha=\tfrac12) breaks the statement

#### Lemma 1 (Half+half is impossible on any nontrivial induced subgraph)

Let (s\ge2). For any 2‑colouring of (E(K_s)), there is **no** way to have simultaneously
[
e_R([s])>\tfrac12\binom{s}{2}
\quad\text{and}\quad
e_B([s])>\tfrac12\binom{s}{2}.
]

**Proof.**
In any 2‑colouring of (K_s),
[
e_R([s]) + e_B([s]) = \binom{s}{2}.
]
If both summands were (>\tfrac12\binom{s}{2}), then adding gives
[
e_R([s]) + e_B([s]) > \binom{s}{2},
]
contradiction. (\square)

#### Lemma 2 (Exact value of (F(n,1/2)) under the literal definition)

Assume (n\ge1) and interpret (F(n,\alpha)) as “the smallest integer (m)” with no restriction (m\le n) (this matches the literal quantifier structure as written on the ErdősProblems page). ([Erdős Problems][1])
Then
[
F(n,1/2)=n+1.
]

**Proof.**

* **Step 1: No (m\le n) can work.**

* If (m=1), then the condition ((*)) must hold for (X) of size (1). But (\binom{1}{2}=0) and (e_R(X)=e_B(X)=0), so the strict inequalities
[
0> \tfrac12\cdot 0
\quad\text{and}\quad
0> \tfrac12\cdot 0
]
both fail.

* If (2\le m\le n), then take any subset (X\subseteq[n]) with (|X|=m). The condition ((*)) requires
[
e_R(X)>\tfrac12\binom{m}{2}
\quad\text{and}\quad
e_B(X)>\tfrac12\binom{m}{2}.
]
This is impossible by Lemma 1 (applied to (s=m)).

Hence no (m\le n) satisfies the defining property.

* **Step 2: (m=n+1) always works (vacuously).**

If (m=n+1), then there is **no** subset (X\subseteq[n]) with (|X|\ge m). Therefore the universal condition ((*)) is vacuously true for **every** 2‑colouring of (K_n). So such an (m) exists.

* **Step 3: Minimality.**
From Step 1, no smaller (m) works; from Step 2, (m=n+1) works. Therefore (F(n,1/2)=n+1). (\square)

#### Consequence: the claimed asymptotic fails

From Lemma 2,
[
\frac{F(n,1/2)}{\log n}=\frac{n+1}{\log n}\longrightarrow \infty
\quad(n\to\infty),
]
so there is **no** finite constant (c_{1/2}) with (F(n,1/2)\sim c_{1/2}\log n).

Thus the statement “for every (0\le\alpha\le1/2)” is **false**.

---

### (Optional but relevant) Minimal corrected statement and what *can* be proved quickly

**Corrected statement:** Restrict to (0\le\alpha<1/2). ([Erdős Problems][1])
Then the ErdősProblems page notes the easy (probabilistic) bounds (F(n,\alpha)\asymp_\alpha \log n). ([Erdős Problems][1])
A standard proof is: random 2‑colouring + Chernoff gives an upper bound (O_\alpha(\log n)); and since the property implies “no monochromatic (K_m)”, one gets a lower bound (\Omega(\log n)) from classical Ramsey upper bounds. (This does **not** resolve the stronger (\sim c_\alpha\log n) limit statement.)

I am not labeling this as a full proof of the corrected “(\sim)” claim, because the endpoint disproof already fully resolves the literal statement, and proving the corrected “(\sim)” would in particular settle (\alpha=0), which connects to deep questions about the exponential growth constant of diagonal Ramsey numbers. ([combinatorics.org][2])

---

## 5) VERIFICATION (adversarial checks)

1. **Quantifier check:**
The disproof uses only the fact that the claim asserts the asymptotic for **every** (\alpha\in[0,1/2]). Exhibiting one (\alpha) for which the asymptotic is false is sufficient.

2. **Edge cases (n=1):**
For (n=1), any (m\le 1) forces checking (X={1}) and fails because strict “(>)” demands (0>0). Thus the smallest (m) is (2=n+1), consistent with Lemma 2.

3. **Potential “escape” via redefining (F) only for (m\le n):**
If one *imposed* (m\le n), then for (\alpha=1/2) there is **no** such (m), so (F(n,1/2)) would be undefined or (+\infty). Either way, the asymptotic (F(n,1/2)\sim c\log n) still fails.

4. **Cross-check with computation:**
Exact brute force for (n\le 6) finds no satisfying (m\le n) for (\alpha=1/2), matching Lemma 1’s impossibility argument.

---

## 6) FINAL

**FULL SOLUTION**
**COUNTEREXAMPLE/DISPROOF**

Take (\alpha=\tfrac12). For any subset (X) with (|X|\ge2), the total number of edges is (\binom{|X|}{2}), so it is impossible for both colours to have **strictly more than half** of these edges (Lemma 1). Therefore no (m\le n) can satisfy the defining property, while (m=n+1) makes the condition vacuous, giving (F(n,1/2)=n+1) (Lemma 2). Hence
[
F(n,1/2)\not\sim c_{1/2}\log n,
]
and the statement “for every (0\le\alpha\le1/2)” is false as written.

[1]: https://www.erdosproblems.com/563 "

Erdős Problem #563

"
[2]: https://www.combinatorics.org/files/Surveys/ds1/ds1v16-2021.pdf "https://www.combinatorics.org/files/Surveys/ds1/ds1v16-2021.pdf"



