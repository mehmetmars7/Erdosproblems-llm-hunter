

A question of Erd\H{o}s and Graham. Suppose $G$ is a graph with $m$ edges and no isolated vertices. Is the Ramsey number $R(G)$ maximised when $G$ is as complete as possible? More precisely, if $m=\tbinom{n}{2}+t$ with $0\leq t<n$, is it true that $R(G)\leq R(H)$, where $H$ is the graph formed by connecting a new vertex to $t$ vertices of $K_n$? 

1) “FORMAL RESTATEMENT”

Let $m\ge 1$ be an integer. Let $\mathcal{G}_m$ be the family of finite simple graphs with exactly $m$ edges and no isolated vertices. For $m=\binom{n}{2}+t$ with $n$ maximal and $0\le t < n$, define $H_m$ to be the graph obtained from $K_n$ by adding one new vertex joined to exactly $t$ vertices of the $K_n$ (note: when $t=0$, this construction produces an isolated vertex; since the quantification restricts to graphs with no isolated vertices, the natural comparison graph is $K_n$ itself, but in the counterexample below $t\neq 0$ so no ambiguity arises).

Conjecture (as stated): For every $G\in\mathcal{G}_m$,
\[
R(G)\le R(H_m),
\]
where $R(\cdot)$ denotes the ordinary (two-colour) Ramsey number of a graph.

I will disprove this conjecture by exhibiting $m=2$ and a graph $G\in\mathcal{G}_2$ with $R(G)>R(H_2)$.

2) “QUICK LITERATURE/CONTEXT CHECK”

The problem statement notes (i) a weaker upper bound question $R(G)\le 2^{O(m^{1/2})}$ is true (Sudakov), and (ii) LouisD observed the conjectured extremiser fails for small $m$ (in particular $2\le m\le 5$ and $7\le m\le 9$). I will give an explicit self-contained counterexample at $m=2$.

3) “ATTACK PLAN”

For $m=2$, compute the proposed extremal graph $H_2$ and its Ramsey number. Then find another $2$-edge graph $G$ with no isolated vertices whose Ramsey number is larger. To make the disproof fully self-contained, prove the exact Ramsey numbers needed.

4) “WORK”

For $m=2$ we have $2=\binom{2}{2}+1$, so $n=2$ and $t=1$. Thus $H_2$ is obtained from $K_2$ by adding one new vertex adjacent to exactly one endpoint of the $K_2$. This is exactly the path $P_3$ on three vertices.

Let $G$ be the disjoint union of two edges, i.e. the matching $2K_2$ on $4$ vertices. Note $G$ has $m=2$ edges and no isolated vertices.

I now compute $R(P_3)$ and $R(2K_2)$.

\textbf{Lemma 1 (Exact value $R(P_3)=3$).}

\emph{Proof.}
A copy of $P_3$ consists of two adjacent edges on three vertices.

- Lower bound: $R(P_3)\ge 3$ because $K_2$ has only one edge and cannot contain $P_3$ in either colour.

- Upper bound: In any red/blue colouring of the edges of $K_3$, there are three edges. By the pigeonhole principle, at least two edges have the same colour. In a triangle, any two distinct edges share a vertex, hence these two same-coloured edges form a monochromatic copy of $P_3$.

Therefore $R(P_3)=3$. $\square$

\textbf{Lemma 2 (Exact value $R(2K_2)=5$).}

\emph{Proof.}
A monochromatic copy of $2K_2$ is a monochromatic matching of size $2$.

\underline{Step 1: $R(2K_2)\ge 5$.}
Consider $K_4$ on vertices $\{0,1,2,3\}$. Colour the three edges among $\{1,2,3\}$ red, and colour the three edges incident to $0$ blue. Then:
- The red edges form a triangle, so any two red edges share a vertex (no red matching of size $2$).
- The blue edges form a star centred at $0$, so any two blue edges share the vertex $0$ (no blue matching of size $2$).
Hence this colouring of $K_4$ contains no monochromatic $2K_2$, and therefore $R(2K_2)\ge 5$.

\underline{Step 2: $R(2K_2)\le 5$.}
Assume for contradiction that there is a red/blue colouring of $K_5$ with no monochromatic matching of size $2$.
Then:
- The set of red edges is \emph{pairwise intersecting} (any two red edges share a vertex), otherwise two disjoint red edges would form a red $2K_2$.
- Similarly, the set of blue edges is pairwise intersecting.

\emph{Claim:} In $K_5$, any pairwise-intersecting set of edges has size at most $4$.

\emph{Proof of claim.}
Take any edge $ab$ in the set. Every other edge in the set must intersect $ab$, so every other edge must be incident to $a$ or to $b$.
If there is an edge $ac$ with $c\notin\{a,b\}$ and an edge $bd$ with $d\notin\{a,b\}$ and $c\neq d$, then $ac$ and $bd$ are disjoint, contradicting pairwise intersection. Therefore all edges in the set must share a common vertex (either $a$ or $b$), i.e. the set is contained in a star, which has exactly $4$ edges in $K_5$.
This proves the claim. $\square$

Applying the claim to both colours, we get
\[
|E_{\text{red}}|\le 4 \quad\text{and}\quad |E_{\text{blue}}|\le 4,
\]
so $|E(K_5)|=|E_{\text{red}}|+|E_{\text{blue}}|\le 8$, contradicting $|E(K_5)|=10$.
Thus every colouring of $K_5$ contains a monochromatic $2K_2$, and hence $R(2K_2)\le 5$.

Combining Step 1 and Step 2 gives $R(2K_2)=5$. $\square$

\textbf{Disproof.}
For $m=2$ the conjectured extremiser is $H_2=P_3$ with $R(H_2)=3$ (Lemma 1). The graph $G=2K_2$ has $R(G)=5$ (Lemma 2). Hence $R(G)>R(H_2)$, contradicting the conjecture.

5) “VERIFICATION”

Fast reality check by exhaustive search:
- Brute force over all $2^{\binom{3}{2}}=8$ colourings of $K_3$ confirms $R(P_3)=3$.
- Brute force over all $2^{\binom{4}{2}}=64$ colourings of $K_4$ finds a colouring avoiding monochromatic $2K_2$ (the star/triangle construction above).
- Brute force over all $2^{\binom{5}{2}}=1024$ colourings of $K_5$ finds none avoiding monochromatic $2K_2$.

6) FINAL

**COUNTEREXAMPLE/DISPROOF**

**Counterexample at $m=2$: $G=2K_2$ beats the proposed extremiser $H_2=P_3$.**

