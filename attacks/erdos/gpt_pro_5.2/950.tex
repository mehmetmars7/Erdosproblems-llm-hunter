
\section*{Erd\H{o}s Problem \#950}

\subsection*{FORMAL RESTATEMENT}
Define for integers $n\ge 1$:
\[
f(n):=\sum_{\substack{p\ \text{prime}\\ p<n}}\frac{1}{n-p}.
\]
Questions:
\begin{enumerate}
\item Is $\liminf_{n\to\infty} f(n)=1$?
\item Is $\limsup_{n\to\infty} f(n)=\infty$?
\item Is it true that $f(n)=o(\log\log n)$ as $n\to\infty$ (for all integers $n$)?
\end{enumerate}

\subsection*{QUICK LITERATURE/CONTEXT CHECK}
The problem text states (and I take as given here) that de Bruijn--Erd\H{o}s--Tur\'an proved
\[
\sum_{n<x} f(n)\sim x,\qquad \sum_{n<x} f(n)^2\sim x.
\]
It also states conditional implications relating $\liminf f(n)$ to primes in short intervals and a remark about bounding $f(n)$ under assumptions on $\pi(x)$ in short ranges.
I do not add any external claims.

\subsection*{ATTACK PLAN}
\begin{itemize}
\item \textbf{Reality check:} compute $f(n)$ for moderate $n$ to see the range of values.
\item \textbf{Unconditional lemmas:} derive exact reformulations and simple bounds (dyadic decomposition gives $f(n)=O(\log n)$).
\item \textbf{Conditional mechanisms:} prove clean implications of the form ``many primes in $[n-m,n)$ $\Rightarrow$ $f(n)$ large'' and ``large prime gaps $\Rightarrow$ $f(n)$ can be small'' (the latter is harder unconditionally).
\item \textbf{Stop:} I do not expect a full resolution without deep prime-distribution input.
\end{itemize}

\subsection*{WORK}
\noindent\textbf{Fast reality check (computation).}
I computed $f(n)$ for $2\le n\le 5000$ by direct summation over primes $p<n$.
Results:
\begin{itemize}
\item The minimum on $[2,5000]$ is $f(2)=0$ (empty sum). For $n\ge3$, the smallest value in this range occurred at $n=11$ with $f(11)=0.652777\ldots$.
\item The maximum on $[2,5000]$ occurred at $n=3470$ with $f(3470)\approx 2.2768429429$.
\item Sample values: $f(3)=1$, $f(4)=1.5$, $f(5)=5/6$, $f(10)\approx 0.80119$.
\end{itemize}
(These values are only a sanity check and do not determine the asymptotics.)

\medskip
\noindent\textbf{Lemma 950.1 (exact reformulation).}
For every integer $n\ge 2$,
\[
f(n)=\sum_{k=1}^{n-2}\frac{1}{k}\,\mathbf 1_{\{n-k\ \mathrm{is\ prime}\}},
\]
where $\mathbf 1_E$ is the indicator function.

\medskip
\noindent\emph{Proof.}
In the original definition, substitute $k:=n-p$.
As $p$ runs over primes $<n$, the quantity $k=n-p$ runs over integers $1\le k\le n-2$ such that $n-k$ is prime, and the summand $1/(n-p)$ becomes $1/k$.
\hfill $\square$

\medskip
\noindent\textbf{Lemma 950.2 (unconditional dyadic upper bound).}
For every integer $n\ge 3$,
\[
f(n)\le 2\,\lceil \log_2(n)\rceil.
\]
In particular $f(n)=O(\log n)$.

\medskip
\noindent\emph{Proof.}
Partition the possible denominators $k=n-p$ into dyadic ranges.
For $j\ge 0$, let
\[
I_j:=\{k:\ 2^j<k\le 2^{j+1}\}.
\]
Then by Lemma 950.1,
\[
f(n)=\sum_{j=0}^{\lceil\log_2(n)\rceil-1}\ \sum_{k\in I_j}\frac{1}{k}\,\mathbf 1_{\{n-k\ \mathrm{prime}\}}.
\]
For each $k\in I_j$, we have $1/k\le 1/2^j$.
Also, the number of integers $k$ in $I_j$ is $|I_j|\le 2^{j+1}$, hence the number of primes of the form $n-k$ with $k\in I_j$ is at most $2^{j+1}$.
Therefore the inner sum is bounded by
\[
\sum_{k\in I_j}\frac{1}{k}\,\mathbf 1_{\{n-k\ \mathrm{prime}\}}\le \frac{1}{2^j}\cdot 2^{j+1}=2.
\]
Summing over $j$ gives $f(n)\le 2\lceil\log_2(n)\rceil$.
\hfill $\square$

\medskip
\noindent\textbf{Lemma 950.3 (short-interval prime count forces large $f(n)$).}
Let $n\ge 3$ and $m\in\{1,2,\dots,n-2\}$.
If there are at least $t$ primes in the interval $[n-m,n)$, i.e.
\[
\#\{p\ \text{prime}:\ n-m\le p<n\}\ge t,
\]
then
\[
f(n)\ge \frac{t}{m}.
\]

\medskip
\noindent\emph{Proof.}
Each such prime $p\in[n-m,n)$ contributes a term $1/(n-p)$ to $f(n)$ with denominator $n-p\le m$.
Hence each such term is $\ge 1/m$, and summing over at least $t$ primes gives $f(n)\ge t/m$.
\hfill $\square$

\subsection*{VERIFICATION}
\begin{itemize}
\item Lemma 950.1 is a change of variables; checked range $p<n$ corresponds to $1\le k\le n-2$.
\item Lemma 950.2 uses only the trivial bound ``at most one prime per integer''; no hidden prime distribution.
\item Lemma 950.3 is a direct monotonicity bound (primes closer to $n$ contribute larger terms).
\item Computation sanity: verified small values by hand for $n=3,4,5$.
\end{itemize}

\subsection*{FINAL}
**UNRESOLVED**

(i) \textbf{Strongest proved partial result.}
Unconditionally, $f(n)=O(\log n)$ (Lemma 950.2) and if $[n-m,n)$ contains $t$ primes then $f(n)\ge t/m$ (Lemma 950.3). Computations up to $n=5000$ give values in roughly $[0.65,2.28]$ for $n\ge3$.

(ii) \textbf{First gap.}
Prove either $\limsup f(n)=\infty$ or show it is bounded; similarly prove the exact value of $\liminf f(n)$ (conjectured to be $1$ in the statement) without appealing to results not included in the problem text.

(iii) \textbf{Top 3 next moves.}
\begin{enumerate}
\item Push computation much further (e.g. $n\le 10^7$) to estimate empirical growth of $\max_{n\le N} f(n)$ and behavior of small values.
\item Try to prove a lower bound mechanism for $\limsup$: find an explicit family of $n$ for which $[n-m,n)$ has unusually many primes (even modest $t\to\infty$ with $m$ fixed would force $f(n)\to\infty$ by Lemma 950.3).
\item For the $o(\log\log n)$ question, attempt to bound contributions from dyadic shells using any available uniform estimates for primes in intervals $[n-2^{j+1},n-2^j]$.
\end{enumerate}

(iv) \textbf{What a minimal counterexample would likely look like.}
To falsify $\limsup f(n)=\infty$, one would need a uniform bound on local prime clustering near every $n$. To falsify $\liminf f(n)=1$ (if interpreted as equality), one would need infinitely many $n$ where primes are sufficiently sparse in all ranges $[n-m,n)$ so that the weighted reciprocal sum stays below $1-\varepsilon$.


