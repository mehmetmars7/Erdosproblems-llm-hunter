% ------------------------------------------------------------
% Erdos problem 193
% ------------------------------------------------------------
\section*{Erd\H{o}s problem 193}

\subsection*{1) FORMAL RESTATEMENT}
Let $S\subset\mathbb{Z}^3$ be finite.
An \emph{$S$-walk} is an infinite sequence $(a_n)_{n\ge 0}\subset\mathbb{Z}^3$ such that $a_{n+1}-a_n\in S$ for all $n$.
Question: Must the set $\{a_n:n\ge 0\}$ contain three distinct collinear points?

\subsection*{2) QUICK LITERATURE/CONTEXT CHECK}
Problem text states: in $\mathbb{Z}^2$ the answer is yes (Gerver--Ramsey [GR79]).
In $\mathbb{Z}^3$ the question is open; the text notes bounds on the maximum number of collinear points can be controlled.

\subsection*{3) ATTACK PLAN}
Provide reductions:
(1) the walk lies in a coset of the lattice generated by $S$,
(2) if $\mathrm{rank}\langle S\rangle_{\mathbb{Z}}\le 2$, the walk is planar (reducing to $\mathbb{Z}^2$ case).
Also provide a computational sanity check for a sample $S$.

\subsection*{4) WORK}

\paragraph{Lemma 4.1 (Lattice-coset containment).}
Let $L=\mathrm{span}_{\mathbb{Z}}(S)\le \mathbb{Z}^3$. Then every $S$-walk $(a_n)$ satisfies
\[
a_n\in a_0+L\quad \text{for all }n\ge 0.
\]
\textit{Proof.}
By induction: $a_0\in a_0+L$. If $a_n\in a_0+L$, then $a_{n+1}=a_n+s$ for some $s\in S\subset L$, hence $a_{n+1}\in a_0+L$.
\hfill$\square$

\paragraph{Lemma 4.2 (Rank reduction).}
If $\mathrm{rank}(L)=1$, then every $S$-walk lies on an affine line in $\mathbb{Z}^3$ (hence trivially contains infinitely many collinear points).
If $\mathrm{rank}(L)=2$, then every $S$-walk lies in an affine plane $a_0+P$ where $P$ is a rank-2 sublattice of $\mathbb{Z}^3$.
\textit{Proof.}
If $\mathrm{rank}(L)=1$ then $L=\mathbb{Z}v$ for some nonzero $v\in\mathbb{Z}^3$, and Lemma 4.1 gives $a_n\in a_0+\mathbb{Z}v$, an affine line.
If $\mathrm{rank}(L)=2$, then $L$ is a rank-2 subgroup of $\mathbb{Z}^3$, hence contained in some 2D real subspace $P_{\mathbb{R}}\subset\mathbb{R}^3$;
then $a_0+L\subset a_0+P_{\mathbb{R}}$, an affine plane. \hfill$\square$

\subsection*{FAST REALITY CHECK (computed)}
For the sample step set $S=\{e_1,e_2,e_3\}$ (positive axis steps), an exhaustive backtrack search from $a_0=(0,0,0)$ found
no $S$-walk longer than 7 steps without creating 3 collinear points; the longest found had 8 visited points and still avoided collinearity.
This is only evidence for this particular $S$ (not a proof for infinite).

\subsection*{6) FINAL}
\textbf{UNRESOLVED}

(i) Strongest proved partial result: every $S$-walk lies in a coset of $\mathrm{span}_{\mathbb{Z}}(S)$ (Lemma 4.1),
so only the rank-3 case is genuinely 3D.

(ii) First gap: handle $\mathrm{rank}(L)=3$ without relying on known $\mathbb{Z}^2$ theorem.

(iii) Top 3 next moves:
1. Prove a 3D analogue of Gerver--Ramsey for special classes of $S$ (e.g. symmetric generating sets).
2. Attempt a density-in-lines argument: show infinitely many points force 3 collinear via pigeonhole on directions.
3. Systematically search for long finite $S$-walks with no 3 collinear to guess counterexample structures.

(iv) Minimal counterexample structure: a rank-3 step set $S$ and an infinite walk whose point set behaves like a discrete space curve
avoiding repeated directions and avoiding long arithmetic structure in any line.

