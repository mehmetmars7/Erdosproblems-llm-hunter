\section*{Erd\H{o}s Problem \#97}

\begin{enumerate}[leftmargin=*,label=\arabic*.]

\item \textbf{Formal Restatement.}

Let $P$ be a convex polygon in $\mathbb{R}^2$ with vertex set $V=\{v_1,\dots,v_n\}$.
Say that a vertex $v\in V$ is \emph{$4$-pinned-repetition-free} if there do \emph{not} exist four \emph{distinct} vertices $x_1,x_2,x_3,x_4\in V\setminus\{v\}$ with
\[
\|v-x_1\|=\|v-x_2\|=\|v-x_3\|=\|v-x_4\|.
\]
Equivalently, in the multiset $\{\|v-x\|:x\in V\setminus\{v\}\}$, every value occurs with multiplicity at most $3$.

\medskip
\noindent\textbf{Question (\#97).} Is it true that every convex polygon $P$ has at least one vertex that is $4$-pinned-repetition-free?

\item \textbf{Quick literature/context check.}

The statement with ``$4$'' replaced by ``$3$'' is false: Danzer produced a convex $9$-gon such that \emph{every} vertex has three other vertices equidistant from it, and Fishburn--Reeds constructed a convex $20$-gon where the witnessing distance can be chosen the \emph{same} for all vertices.\footnote{See the summary and references on the Erd\H{o}s Problems website.\cite{ErdosProblems97,FiRe92}}

Erd\H{o}s discussed this type of question already in 1946 and later writings; in particular, he explicitly expressed hope that a statement of this form might hold for ``$4$'' even though it fails for ``$3$'' (and remarked he could not prove even very weak quantitative variants).\footnote{See, e.g., Erd\H{o}s (1986) and the sources listed on the Erd\H{o}s Problems page.\cite{Er86, ErdosProblems97}}

As of the last edit date shown on the Erd\H{o}s Problems page (27 Oct 2025), the problem is listed as open and falsifiable by a finite counterexample.\cite{ErdosProblems97}

\item \textbf{Attack plan.}

\textbf{Proof track.} Try to show that assuming the negation (every vertex $v$ admits a radius $r(v)$ with at least four vertices on the circle $\partial B(v,r(v))$) forces too many circle--circle intersections. Since two circles with distinct centers meet in at most $2$ points, this becomes an extremal set-system/incidence problem. The convexity constraint might further restrict which incidences are geometrically realizable.

\textbf{Disproof track.} Try to explicitly construct a convex $n$-gon where for each vertex $v$ there is a circle centered at $v$ passing through (at least) four other vertices. One approach is to prescribe an ``incidence design'' $(V,\{S_v\}_{v\in V})$ with $|S_v|\ge 4$ and $|S_u\cap S_v|\le 2$, then attempt to solve the resulting distance-equality constraints in coordinates and finally check convex position.

\textbf{Phase 1 small cases.} Note that for $n\le 4$ the question is trivial (a vertex cannot have four other vertices). For $n=5$ it reduces to asking whether every vertex has all four other vertices at the same distance, which is impossible in $\mathbb{R}^2$ (it would force all pairwise distances equal).

\item \textbf{Work.}

I did not obtain a full proof or a counterexample for general $n$. I did, however, obtain a clean unconditional ``small-$n$'' obstruction that does \emph{not} use convexity.

\begin{theorem}[No $6$-point counterexample, even without convexity]
\label{thm:n6}
There do not exist six points $p_1,\dots,p_6\in\mathbb{R}^2$ such that \emph{for every} $i\in\{1,\dots,6\}$ there is a radius $r_i>0$ for which at least four of the other points lie on the circle $\{x:\|x-p_i\|=r_i\}$.
Equivalently: for every $6$-point set $A\subset\mathbb{R}^2$, there exists $p\in A$ such that no four points of $A\setminus\{p\}$ are equidistant from $p$.
\end{theorem}

\begin{proof}
Suppose for contradiction that $p_1,\dots,p_6$ satisfy the stated property.
For each $i$ choose a circle
\[
C_i := \{x\in\mathbb{R}^2 : \|x-p_i\|=r_i\}
\]
that contains at least four points of $\{p_1,\dots,p_6\}\setminus\{p_i\}$.
Define the index set
\[
S_i := \{j\in\{1,\dots,6\}\setminus\{i\} : p_j\in C_i\}.
\]
Then $|S_i|\ge 4$ for each $i$.

\smallskip
\noindent\emph{Step 1: no $S_i$ can have size $5$.}
If some $|S_i|=5$, then $C_i$ contains all five points $\{p_j:j\ne i\}$. For any $k\ne i$, the circle $C_k$ contains at least $4$ of those same five points, hence $C_i$ and $C_k$ would share at least $4$ common points. But two distinct circles in the plane have at most $2$ intersection points, and $C_i\ne C_k$ because they have different centers $p_i\ne p_k$. Contradiction. Therefore, for each $i$ we must have
\[
|S_i|=4.
\]

\smallskip
\noindent\emph{Step 2: derive a forced ``missing index'' map.}
Since $S_i\subseteq\{1,\dots,6\}\setminus\{i\}$ has size $4$ and $\{1,\dots,6\}\setminus\{i\}$ has size $5$, there is a unique index $f(i)\in\{1,\dots,6\}\setminus\{i\}$ such that
\[
S_i = \{1,\dots,6\}\setminus\{i,f(i)\}.
\]
In words: $C_i$ passes through every point except $p_i$ itself and one ``missing'' point $p_{f(i)}$.

\smallskip
\noindent\emph{Step 3: use a special pair to contradict circle intersection.}
Fix any $i$, and set $j:=f(i)$ (so $j\ne i$).
Then by definition $j\notin S_i$, i.e., $p_j\notin C_i$.
Now consider $S_j$. We know $|S_j|=4$ and $S_j = \{1,\dots,6\}\setminus\{j,f(j)\}$.
Since $f(i)=j$, the set of indices excluded from $S_i\cap S_j$ is
\[
\{i,j,f(i),f(j)\} = \{i,j,j,f(j)\} = \{i,j,f(j)\},
\]
which has size at most $3$. Therefore
\[
|S_i\cap S_j| \;=\; 6-\bigl|\{i,j,f(j)\}\bigr| \;\ge\; 3.
\]
Thus the two circles $C_i$ and $C_j$ share at least three common points $p_k$ with $k\in S_i\cap S_j$. This is impossible because two distinct circles intersect in at most two points.

We have reached a contradiction, so no such six-point configuration exists.
\end{proof}

\noindent\textbf{Consequences.}
In particular, for every convex hexagon (indeed, for every $6$-point set in $\mathbb{R}^2$) there exists a vertex $v$ for which no four other vertices are equidistant from $v$.
So any counterexample to Problem~\#97 must have at least $7$ vertices.

\medskip
\noindent\textbf{Where the approach gets stuck for general $n$.}
For $n\ge 7$, the simple ``missing index'' argument above no longer forces a unique complement representation for each $S_i$, and extremal set systems with $|S_i|=4$ and $|S_i\cap S_j|\le 2$ can exist combinatorially (e.g. via designs). The remaining difficulty is to determine whether such incidence patterns can be realized by circles \emph{centered at the points themselves} \emph{and} with the points in convex position.

\item \textbf{Verification.}

The only geometric input used in Theorem~\ref{thm:n6} is:
(i) distinct circles intersect in at most $2$ points, and
(ii) circles $C_i$ and $C_k$ are distinct because their centers $p_i\ne p_k$.
All other steps are purely combinatorial on the incidence sets $S_i$.
No convexity hypothesis was used, so the conclusion is valid a fortiori for convex polygons with $n\le 6$.

\item \textbf{FINAL: UNRESOLVED.}

\begin{itemize}[leftmargin=*]
\item[(i)] \textbf{Strongest fully proved partial result:}
Theorem~\ref{thm:n6} (no $6$-point configuration can have the ``every point has $4$ equidistant neighbors'' property). In particular, Problem~\#97 holds for all convex polygons with $n\le 6$.
\item[(ii)] \textbf{Most plausible route forward:}
Model the negation by circles $C_v$ centered at each vertex with $|V\cap C_v|\ge 4$, convert to a constrained incidence design $\{S_v\}$ with $|S_u\cap S_v|\le 2$, then incorporate \emph{convex position} constraints (cyclic order/visibility constraints) to rule out realizability for large $n$, or else explicitly solve the distance-equality system for a realizable design.
\item[(iii)] \textbf{First place the proof attempt breaks:}
Beyond $n=6$, the incidence sets $S_v$ need not be complements of size-$2$ subsets, so the forced pair argument $j=f(i)$ that yields $|S_i\cap S_j|\ge 3$ no longer applies.
\item[(iv)] \textbf{What a counterexample would have to look like:}
A convex $n$-gon together with a choice of radii $\{r(v)\}_{v\in V}$ so that for each $v$ at least four vertices lie on the circle centered at $v$ with radius $r(v)$, and such that for distinct $u,v$ the circles share at most two vertices (automatically), yet the system is geometrically realizable in convex position.
\end{itemize}

\item \textbf{Completion estimate.}

\textbf{COMPLETION: 25\%.}

\end{enumerate}

