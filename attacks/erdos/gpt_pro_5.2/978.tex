% Erdos Problem #978

1) FORMAL RESTATEMENT

Let $f\in\mathbb Z[x]$ be irreducible of degree $k>2$.
An integer $N\ne 0$ is called $t$-power-free if there is no prime $p$ with $p^t\mid N$.
The questions are:
(a) for $k>2$ with $k\ne 2^\ell$, does the set $\{n\in\mathbb N: f(n)\text{ is }(k-1)\text{-power-free}\}$ have positive (natural) density?
(b) are there infinitely many $n$ such that $f(n)$ is $(k-2)$-power-free?
In particular (taking $f(n)=n^4+2$ and $k=4$), are there infinitely many $n$ for which $n^4+2$ is squarefree?

Conventions/edge cases:
- We interpret $t$-power-free for negative values via $|f(n)|$.
- The density in (a) is $\lim_{x\to\infty}\frac1x\#\{1\le n\le x: f(n)\text{ is }(k-1)\text{-power-free}\}$ when the limit exists.

2) QUICK LITERATURE/CONTEXT CHECK

The provided statement already records several known results (Erd\H{o}s, Hooley, Heath-Brown, Browning).
A quick web search (2026-01-16) did not surface an explicit resolution of the specific squarefree question for $n^4+2$ beyond what is written in the problem statement, so I treat that question as open here.

3) ATTACK PLAN

Proof track: one expects to use a power-free (squarefree) sieve, reducing the problem to controlling solutions of congruences $f(n)\equiv 0\pmod{p^{k-2}}$ for many primes $p$, and then estimating the remainder term.
Disproof track: look for a local obstruction (a prime $p$ such that $p^{k-2}\mid f(n)$ for all $n$), which would force only finitely many power-free values.

In this write-up I give elementary local computations for the special polynomial $n^4+2$ and a general lemma for the simplest local obstruction $p\mid f(n)$ for all $n$.

4) WORK

FAST REALITY CHECK (for $f(n)=n^4+2$; exact computations)

Count of $1\le n\le N$ for which $n^4+2$ is squarefree:
\[
\begin{array}{c|c|c}
N & \#\{1\le n\le N: n^4+2\text{ squarefree}\} & \text{fraction}\\\hline
10 & 8 & 0.8\\
50 & 38 & 0.76\\
100 & 75 & 0.75\\
500 & 379 & 0.758\\
1000 & 759 & 0.759\\
2000 & 1514 & 0.757\\
5000 & 3787 & 0.7574
\end{array}
\]
Among $1\le n\le 100$, the first few nonsquarefree values occur at
$n=2$ ($n^4+2=18=2\cdot 3^2$), $n=7$ ($2403=3^3\cdot 89$), $n=11$ ($14643=3^2\cdot 1627$).

Lemma 4.1 (a basic local obstruction criterion mod $p$).
Let $p$ be prime and let $\overline f\in\mathbb F_p[x]$ be the reduction of $f$ modulo $p$.
Then the following are equivalent:
(i) $p\mid f(n)$ for every integer $n$.
(ii) $\overline f(a)=0$ for every $a\in\mathbb F_p$.
(iii) $x^p-x$ divides $\overline f(x)$ in $\mathbb F_p[x]$.

Proof.
(i)$\Rightarrow$(ii): reduce the congruence $f(n)\equiv 0\pmod p$ modulo $p$.
(ii)$\Leftrightarrow$(iii): in $\mathbb F_p[x]$ we have the factorization
$x^p-x=\prod_{a\in\mathbb F_p}(x-a)$, so a polynomial vanishes on every element of $\mathbb F_p$ iff it has every $a\in\mathbb F_p$ as a root, equivalently iff it is divisible by $\prod_{a\in\mathbb F_p}(x-a)=x^p-x$.
(ii)$\Rightarrow$(i): if $\overline f$ vanishes on all residues modulo $p$, then $f(n)\equiv 0\pmod p$ for every integer $n$. \qed

Lemma 4.2 (exact $3$-adic divisibility pattern for $n^4+2$ at level $3^2$).
For every integer $n$,
\[
9\mid (n^4+2)\quad\Longleftrightarrow\quad n\equiv 2\text{ or }7\pmod 9.
\]
Moreover, $3\mid(n^4+2)$ if and only if $3\nmid n$.

Proof.
Modulo $3$, one has $n^4\equiv 0$ if $3\mid n$ and $n^4\equiv 1$ if $3\nmid n$.
Thus $n^4+2\equiv 2\pmod 3$ when $3\mid n$ and $n^4+2\equiv 0\pmod 3$ when $3\nmid n$, proving the second claim.
For the $\bmod\ 9$ statement, compute $(r^4+2)\bmod 9$ for residues $r=0,1,\dots,8$:
\[
\begin{array}{c|ccccccccc}
 r & 0&1&2&3&4&5&6&7&8\\\hline
 r^4+2\pmod 9 & 2&3&0&2&6&6&2&0&3
\end{array}
\]
So $9\mid (n^4+2)$ exactly for $n\equiv 2,7\pmod 9$. \qed

Lemma 4.3 (no fixed square divisor for $n^4+2$).
There is no prime $p$ such that $p^2\mid n^4+2$ for all integers $n$.

Proof.
If $p^2\mid n^4+2$ for all $n$, then in particular $p^2\mid 0^4+2=2$, hence $p=2$.
But for $n=1$ we have $1^4+2=3$, which is not divisible by $4$. Contradiction. \qed

5) VERIFICATION

- Lemma 4.1 is purely over $\mathbb F_p$ and uses the standard identity $x^p-x=\prod_{a\in\mathbb F_p}(x-a)$.
- Lemma 4.2 was verified by an explicit residue computation modulo $9$.
- Lemma 4.3 uses only evaluations at $n=0$ and $n=1$, hence is robust.

6) FINAL

**UNRESOLVED**
(i) Strongest proved partial result: for the specific polynomial $n^4+2$, the exact criterion for divisibility by $9$ is $n\equiv 2,7\pmod 9$ (Lemma 4.2), and there is no fixed prime square dividing all values (Lemma 4.3). Computations up to $N=5000$ show $3787$ squarefree values among the first $5000$ inputs (table above).
(ii) First gap (crisp): prove that the set $\{n\ge 1: n^4+2\text{ is squarefree}\}$ is infinite (or give a counterexample by producing a prime square dividing $n^4+2$ for all sufficiently large $n$, which seems incompatible with Lemma 4.3 but could in principle occur intermittently).
(iii) Top 3 next moves:
  1. For each prime $p$, bound the number of residue classes $n\bmod p^2$ with $n^4\equiv -2\pmod{p^2}$ uniformly and sum the resulting local densities to set up a squarefree sieve.
  2. Control the remainder term in the sieve for $n^4+2$ by estimating correlations of the events $p^2\mid(n^4+2)$ for different primes $p$.
  3. Computation: for increasing $N$, factor $n^4+2$ and record which prime squares occur, to guess the main obstructions (e.g. the prominent $3^2$ obstruction coming from Lemma 4.2).
(iv) Minimal counterexample structure: if $n^4+2$ were squarefree only finitely often, then beyond some $N_0$ every value would be divisible by $p(n)^2$ for some prime $p(n)$ depending on $n$; one would expect a structured concentration of solutions to $n^4\equiv -2\pmod{p^2}$ for many primes $p$, contrary to the heuristic that such congruences have $O(1)$ solutions modulo $p^2$.


