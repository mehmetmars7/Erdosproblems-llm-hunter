\section{Erd\H{o}s Problem \#149: Strong chromatic index}

\subsection*{FORMAL RESTATEMENT}
Let $G=(V,E)$ be a finite simple graph with maximum degree $\Delta=\Delta(G)$.
A set of edges $M\subseteq E$ is \emph{strongly independent} if the induced subgraph on the
endpoints of $M$ is a disjoint union of edges (equivalently, $M$ is an \emph{induced matching}).

Let $\chi'_s(G)$ denote the minimum number of strongly independent edge-sets whose union is $E$
(equivalently, the minimum number of colors in a strong edge-coloring). The question is whether
\[
\chi'_s(G)\ \le\ \frac54\,\Delta^2
\]
for all graphs $G$.

\medskip
\noindent\textbf{Ambiguity note.} For odd $\Delta$, the right-hand side is not an integer, and the sharp
Erd\H{o}s--Ne\v{s}et\v{r}il conjecture is usually stated with a slightly smaller (still $\sim \frac54\Delta^2$)
expression. Interpreting the prompt literally, one should read it as
$\chi'_s(G)\le \left\lceil \frac54\Delta^2\right\rceil$.

\subsection*{QUICK LITERATURE/CONTEXT CHECK}
This is the classical Erd\H{o}s--Ne\v{s}et\v{r}il conjecture (1985). It is known that the constant $5/4$
would be best possible via a blow-up construction of $C_5$. The best general asymptotic upper bound currently
recorded in the Erd\H{o}s Problems database is $\chi'_s(G)\le 1.772\,\Delta^2$ for sufficiently large $\Delta$
(Hurley--de Joannis de Verclos--Kang), improving earlier constants $1.835,1.93,1.998$.

It also remains open in full generality whether the \emph{clique number} of $L(G)^2$ is at most $\frac54\Delta^2$,
although it is known under additional forbidden-cycle hypotheses.

\subsection*{DEFINITIONS / SETUP}
\begin{itemize}[leftmargin=2em]
\item \textbf{Strong edge-coloring.} A map $\varphi:E\to [t]$ is a strong edge-coloring if whenever
$e\neq f$ have $\varphi(e)=\varphi(f)$, then $e$ and $f$ are strongly independent (i.e.\ form an induced matching).
\item \textbf{Strong chromatic index.} $\chi'_s(G)$ is the minimum $t$ for which a strong edge-coloring exists.
\item \textbf{Line graph formulation.} Let $L(G)$ be the line graph of $G$ and $L(G)^2$ its square.
Then $\chi'_s(G)=\chi(L(G)^2)$: edges of $G$ are vertices of $L(G)$, and ``strong conflict'' corresponds to
distance at most $2$ in $L(G)$.
\end{itemize}

\subsection*{KNOWN RESULTS I WILL USE}
I only use two elementary facts, each proved below: a greedy upper bound and the blow-up lower bound.

\subsection*{MAIN ATTEMPT}

\subsubsection*{1) Elementary greedy upper bound $\chi'_s(G)\le 2\Delta^2-2\Delta+1$}
\begin{proposition}[Greedy bound]\label{prop:greedy-strong}
For every graph $G$ of maximum degree $\Delta$,
\[
\chi'_s(G)\ \le\ 2\Delta^2-2\Delta+1.
\]
\end{proposition}

\begin{proof}
We color edges one-by-one in an arbitrary order, always choosing the smallest available color.
Fix an edge $e=uv$ being colored. A previously colored edge $f$ is forbidden to share color with $e$ if
$f$ is adjacent to $e$ (shares an endpoint) or if $f$ is adjacent to some edge adjacent to $e$
(i.e.\ in the square of the line graph).

Count the number of edges at strong distance at most $1$ from $e$:
there are at most $(\deg(u)-1)+(\deg(v)-1)\le 2(\Delta-1)$ edges incident to $u$ or $v$ other than $e$.

Now consider edges at strong distance exactly $2$ from $e$. Any such edge must be incident to a neighbor of
$u$ or of $v$ (other than $u,v$ themselves). The vertex $u$ has at most $\Delta-1$ neighbors besides $v$,
and each such neighbor $w$ is incident to at most $\Delta-1$ edges other than $uw$. Thus the number of edges
incident to the neighbors of $u$ (excluding those already counted as incident to $u$) is at most
$(\Delta-1)(\Delta-1)=(\Delta-1)^2$. Similarly for $v$, giving at most $2(\Delta-1)^2$ additional forbidden edges.

Hence, at the time we color $e$, there are at most
\[
2(\Delta-1) + 2(\Delta-1)^2 = 2\Delta^2-2\Delta
\]
previously colored edges whose colors cannot be used. Therefore at most $2\Delta^2-2\Delta$ colors are forbidden,
so with $2\Delta^2-2\Delta+1$ colors available, a greedy choice always succeeds.
\end{proof}

\subsubsection*{2) The blow-up of \texorpdfstring{$C_5$}{C5} yields the lower bound $5\Delta^2/4$}
\begin{proposition}[Tightness example for even $\Delta$]\label{prop:C5blowup}
Let $t\in\mathbb{N}$ and let $G$ be the $t$-blow-up of the cycle $C_5$: replace each vertex of $C_5$
by an independent set $A_0,\dots,A_4$ of size $t$, and replace each edge of $C_5$ by all edges of the
complete bipartite graph between the corresponding parts (indices modulo $5$).
Then $\Delta(G)=2t$ and
\[
\chi'_s(G)=|E(G)| = 5t^2 = \frac54\,\Delta(G)^2.
\]
\end{proposition}

\begin{proof}
Each vertex lies in exactly two bipartite blocks, so $\Delta(G)=2t$.
Also $E(G)$ is the disjoint union of the five complete bipartite graphs between consecutive parts,
so $|E(G)|=5\cdot t^2$.

It remains to show that in $G$ no two distinct edges are strongly independent, i.e.\ every color class in
a strong edge-coloring has size at most $1$. Let $e\neq f$ be edges of $G$.
Project each edge to the corresponding base edge of $C_5$ (i.e.\ remember only which pair of parts it lies between).
In $C_5$, any two distinct edges are at distance at most $2$ in the line graph (indeed $L(C_5)\cong C_5$
and $L(C_5)^2\cong K_5$). Translating back to $G$, this implies that for any two edges $e,f$ there is
either
\begin{itemize}[leftmargin=2em]
\item a shared endpoint (if the corresponding base edges share a vertex), or
\item an adjacency between an endpoint of $e$ and an endpoint of $f$ coming from a base edge connecting the
relevant parts (if the base edges are at line-graph distance $2$).
\end{itemize}
In either case, the induced subgraph on the four endpoints contains an extra edge besides $e$ and $f$,
so $\{e,f\}$ is not a strongly independent set. Hence every strongly independent set has size at most $1$,
and $\chi'_s(G)\ge |E(G)|$. The reverse inequality $\chi'_s(G)\le |E(G)|$ is trivial, so equality holds.
\end{proof}

\subsection*{ADVERSARIAL CHECK}
\begin{itemize}[leftmargin=2em]
\item Proposition~\ref{prop:greedy-strong} only uses an upper bound on the number of conflicting edges;
overcounting is harmless. The conclusion is therefore safe.
\item Proposition~\ref{prop:C5blowup} hinges on the claim that any two distinct edges in the blow-up
are at distance at most $2$ in the line graph. This is true because the base edges of $C_5$ already have this property,
and the blow-up only \emph{adds} adjacencies, never removes them.
\item These arguments do \emph{not} approach the conjectured constant $5/4$ from above; they only show
(1) a coarse universal bound $2\Delta^2+O(\Delta)$ and (2) that $5/4$ would be best possible.
\end{itemize}

\subsection*{FINAL}
\noindent\textbf{UNRESOLVED --- PARTIAL PROGRESS.}

\begin{itemize}[leftmargin=2em]
\item \textbf{Farthest point reached.} I gave complete proofs of (i) the classical greedy bound
$\chi'_s(G)\le 2\Delta^2-2\Delta+1$ and (ii) the $C_5$ blow-up construction showing
$\chi'_s(G)$ can equal $\frac54\Delta^2$ (so the conjectured constant is tight for even $\Delta$).
\item \textbf{Remaining gap.} Proving $\chi'_s(G)\le \frac54\Delta^2$ (or giving a counterexample) for general
graphs remains open; existing best results replace $5/4$ by a larger constant (currently $\approx 1.772$).
\item \textbf{Promising next steps.} Any improvement seems to require either (a) improved control on sparse
neighborhood structure in $L(G)^2$ together with probabilistic coloring methods, or (b) structural decomposition
approaches for extremal configurations resembling the $C_5$ blow-up.
\end{itemize}

\medskip
\noindent\textbf{COMPLETION: 30\%}
