
1) FORMAL RESTATEMENT

A set $A\subset\{1,2,\dots,N\}$ is called **non-dividing** if for every $a\in A$, the integer $a$ does not divide the sum of any nonempty set of distinct elements of $A\setminus\{a\}$.
Let $F(N)$ be the maximum size of a non-dividing subset of $\{1,\dots,N\}$.

The problem asks for estimates of $F(N)$, and in particular asks whether
\[
F(N) > N^{1/2-o(1)}\quad\text{as }N\to\infty.
\]

2) QUICK LITERATURE/CONTEXT CHECK

(Restricted to statements explicitly present in 123-137.tex.)
- Erdos and Straus conjectured $F(N)\ge N^{1/5}$ for large $N$.
- Straus showed $F(N)\ge \exp(c\sqrt{\log N})$.
- Erdos and Rauzy proved $F(N)<3\sqrt{N}+1$.
- The text states: every non-dividing set is non-averaging; the result of Pham and Zakharov implies $F(N)\le N^{1/4+o(1)}$, so the answer to the original $N^{1/2-o(1)}$ question is **no**.

3) ATTACK PLAN

- Prove the elementary implication non-dividing $\Rightarrow$ non-averaging.
- Provide some explicit small-$N$ computations of $F(N)$.
- Use the stated Pham--Zakharov bound (via the implication) to refute the original question.

4) WORK

Lemma 4.1 (Non-dividing implies non-averaging).
If $A$ is non-dividing, then $A$ contains no 3-term arithmetic progression.
Equivalently, there do not exist distinct $x,y,z\in A$ with $y=(x+z)/2$.

*Proof.* Suppose $x,y,z\in A$ are distinct and satisfy $x+z=2y$.
Then $y$ divides the sum $x+z$ (namely $2y$), contradicting the non-dividing property for the element $y\in A$ (take the nonempty subset $\{x,z\}\subset A\setminus\{y\}$). \qed

Lemma 4.2 (A trivial universal lower bound).
For every $N\ge 3$, we have $F(N)\ge 2$.

*Proof.* Take $A=\{N-1,N\}$.
For $a=N$, the only nonempty subset of $A\setminus\{a\}$ sums to $N-1$, not divisible by $N$.
For $a=N-1$, the only nonempty subset sums to $N$, which is not divisible by $N-1$ for $N\ge 3$.
So $A$ is non-dividing and has size 2. \qed

FAST REALITY CHECK (exact computation for $N\le 20$).
By exhaustive search, the exact values of $F(N)$ for $1\le N\le 20$ are:
- $F(N)=1$ for $N=1,2$.
- $F(N)=2$ for $3\le N\le 6$.
- $F(N)=3$ for $7\le N\le 9$.
- $F(N)=4$ for $10\le N\le 20$.
Examples:
- For $N=6$: $\{2,3\}$.
- For $N=9$: $\{4,6,7\}$.
- For $N=20$: $\{5,6,16,17\}$.

Consequence for the "$N^{1/2-o(1)}$" question.
By Lemma 4.1, every non-dividing set is non-averaging.
The problem text states that Pham and Zakharov proved an upper bound for non-averaging sets implying
\[
F(N)\le N^{1/4+o(1)}.
\]
Therefore the proposed lower bound $F(N)>N^{1/2-o(1)}$ is false.

5) VERIFICATION

- Lemma 4.1 is a direct 2-element subset check.
- The brute force values for $N\le 20$ are internally consistent and provide explicit witnesses.
- The final conclusion uses only the implication in Lemma 4.1 plus the upper bound stated in the problem file.

6) FINAL

**UNRESOLVED**

(i) Strongest fully proved partial result:
- Non-dividing sets are non-averaging (Lemma 4.1).
- Consequently (using the stated Pham--Zakharov theorem), $F(N)\le N^{1/4+o(1)}$, so the answer to "$F(N)>N^{1/2-o(1)}$?" is **no**.
- Exact values $F(N)$ for $N\le 20$ were computed.

(ii) Exact first gap:
- Determine the true order of growth of $F(N)$ between the stated lower bounds (e.g. $\exp(c\sqrt{\log N})$) and the stated upper bounds (e.g. $N^{1/4+o(1)}$).

(iii) Top 3 next moves:
1. Improve lower bounds by constructing large non-averaging sets that also satisfy the stronger non-dividing constraint.
2. Improve upper bounds by adapting non-averaging methods to exploit the full divisibility-for-all-subset-sums restriction.
3. Search for structural characterizations: how does forbidding $a\mid\sum S$ for all $S\subset A\setminus\{a\}$ constrain residue distributions modulo $a$?

(iv) What a minimal counterexample would likely look like:
- For any conjectured strong lower bound (say $N^{\theta}$), a counterexample would be a family of sets $A_N$ that are large yet force, for many $a\in A_N$, a subset sum of the others to hit $0\pmod a$; thus understanding subset-sum sets modulo $a$ seems central.


