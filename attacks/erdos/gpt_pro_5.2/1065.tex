% Erdos solutions for problems 1065--1071 (as provided in 1065-1071.tex)
% Generated following PROMPT_STRATEGY.MD.

% Erdos Problem #1065

\subsection*{FORMAL RESTATEMENT}
We ask whether there exist infinitely many primes $p$ of the form
\[
  p = 2^k q + 1 \qquad (k \in \mathbb{N}_0,\ q \text{ prime}).
\]
Equivalently: are there infinitely many primes $p$ such that $p-1$ has the form
\[
  p-1 = 2^k q \quad\text{with $k\ge 0$ and $q$ prime?}
\]
A second question asks the analogous statement with an additional power of $3$:
\[
  p = 2^k 3^\ell q + 1 \qquad (k,\ell \in \mathbb{N}_0,\ q \text{ prime}).
\]

\subsection*{QUICK LITERATURE/CONTEXT CHECK}
The case $k=1$ in the first question is the ``safe prime'' condition $p=2q+1$ with $q$ prime.  The case $q=2$ gives primes of the form $2^m+1$ (Fermat-type).  The problem file itself does not assert any general infinitude results; I therefore treat both questions as open here and only record direct deductions and computations.

\subsection*{ATTACK PLAN}
Reformulate the condition in terms of the prime factorization of $p-1$ (odd part constraints).  Prove exact classifications in degenerate cases ($k=0$, $q=2$).  Then do a ``FAST REALITY CHECK'' by enumerating primes $p\le N$ satisfying each condition for several $N$.

\subsection*{WORK}
\textbf{Lemma 1065.1 (the $k=0$ case is unique).}
If $p$ and $q$ are primes and $p=q+1$, then $(q,p)=(2,3)$.

\emph{Proof.}
If $q=2$ then $p=3$ is prime, giving one solution.
Assume $q>2$. Then $q$ is odd, so $p=q+1$ is an even integer strictly larger than $2$, hence not prime.  Therefore no solutions exist with $q>2$, and the only solution is $(q,p)=(2,3)$. \qed

\textbf{Lemma 1065.2 (the $q=2$ subfamily equals Fermat-type primes).}
Let $p$ be prime. Then $p=2^k q+1$ with $q=2$ and $k\ge 0$ holds if and only if $p$ is of the form $p=2^m+1$ with $m\ge 1$.

\emph{Proof.}
If $q=2$, then $p = 2^k\cdot 2 + 1 = 2^{k+1}+1$, so $p=2^m+1$ with $m=k+1\ge 1$.
Conversely, if $p=2^m+1$ with $m\ge 1$, then writing $m=k+1$ gives $p=2^k\cdot 2+1$, i.e. $q=2$. \qed

\textbf{Lemma 1065.3 (factorization reformulation).}
Let $p$ be an odd prime and write $p-1=2^{v_2(p-1)}\,\mathrm{odd}(p-1)$ where $\mathrm{odd}(p-1)$ is the odd part.
Then $p=2^k q+1$ for some prime $q$ and some $k\ge 0$ if and only if $\mathrm{odd}(p-1)$ is either $1$ or a prime.
Moreover, the representation is unique:
\begin{itemize}
\item if $\mathrm{odd}(p-1)$ is an odd prime, then necessarily $k=v_2(p-1)$ and $q=\mathrm{odd}(p-1)$;
\item if $\mathrm{odd}(p-1)=1$ (so $p-1$ is a power of $2$), then necessarily $q=2$ and $k=v_2(p-1)-1$.
\end{itemize}

\emph{Proof.}
($\Rightarrow$) Suppose $p-1=2^k q$ with $q$ prime.  If $q$ is odd then $\mathrm{odd}(p-1)=q$ is prime.  If $q=2$ then $p-1$ is a pure power of $2$, hence $\mathrm{odd}(p-1)=1$.  In either case $\mathrm{odd}(p-1)\in\{1\}\cup\{\text{primes}\}$.

($\Leftarrow$) If $\mathrm{odd}(p-1)$ is an odd prime $q$, then $p-1=2^{v_2(p-1)}q$ gives $p=2^{v_2(p-1)}q+1$.
If $\mathrm{odd}(p-1)=1$, then $p-1=2^m$ where $m=v_2(p-1)\ge 1$ (since $p$ is odd).  Then
\[
  p-1 = 2^{m-1}\cdot 2,
\]
so $p=2^{m-1}\cdot 2 + 1$ with $q=2$ and $k=m-1$.

For uniqueness: if $\mathrm{odd}(p-1)$ is an odd prime and $p-1=2^{k'}q'$ with $q'$ prime, then $q'$ must be odd (otherwise $q'=2$ forces $\mathrm{odd}(p-1)=1$), so $2^{k'}$ must be the full power of $2$ dividing $p-1$, hence $k'=v_2(p-1)$ and $q'=\mathrm{odd}(p-1)$.
If $\mathrm{odd}(p-1)=1$ and $p-1=2^{k'}q'$ with $q'$ prime, then $q'$ must be $2$ and $k'=v_2(p-1)-1$.
\qed

\textbf{Lemma 1065.4 (analogous reformulation for $2^k3^\ell q$).}
Let $p$ be prime.  Then $p=2^k3^\ell q+1$ for some $k,\ell\ge 0$ and prime $q$ if and only if, after removing all factors of $2$ and $3$ from $p-1$, the remaining factor is either $1$ or a prime.

\emph{Proof.}
Write $p-1 = 2^a3^b r$ where $\gcd(r,6)=1$.  If $r$ is prime, take $(k,\ell,q)=(a,b,r)$.  If $r=1$, then $p-1=2^a3^b$; if $a\ge 1$ take $(k,\ell,q)=(a-1,b,2)$, and if $a=0$ then $b\ge 1$ and take $(k,\ell,q)=(0,b-1,3)$.  Conversely, any representation $p-1=2^k3^\ell q$ has $r=q$ after removing all $2,3$ factors, hence $r$ is $1$ or prime. \qed

\textbf{FAST REALITY CHECK (computation).}
I enumerated primes $p\le N$ and tested the conditions by factoring $p-1$ and checking whether it can be written as $2^k q$ with $q$ prime, and similarly $2^k3^\ell q$ with $q$ prime.  Exact results:
\[
\begin{array}{r|r|r|r|r}
N & \#\{p\le N\} & \#\{p\le N: p=2^k q+1\} & \#\{p\le N: p=2^k3^\ell q+1\} & \text{fractions} \\ \hline
10^3 & 168 & 59 & 117 & 0.351,\ 0.696 \\
10^4 & 1229 & 257 & 580 & 0.209,\ 0.472 \\
10^5 & 9592 & 1470 & 3264 & 0.153,\ 0.340 \\
10^6 & 78498 & 9287 & 20479 & 0.118,\ 0.261
\end{array}
\]
The first few primes $p$ satisfying $p=2^kq+1$ (with their $(k,q)$) are:
\[
3\,(0,2),\ 5\,(1,2),\ 7\,(1,3),\ 11\,(1,5),\ 13\,(2,3),\ 17\,(3,2),\ 23\,(1,11),\ 29\,(2,7),\ 41\,(3,5),\ldots
\]
(Computed by a minimal script using primality testing and dividing $p-1$ by successive powers of $2$ and $3$.)

\subsection*{VERIFICATION}
Lemma~1065.1 is a parity argument and is complete.  Lemma~1065.2 is algebraic and complete.
Lemma~1065.3 is checked by the standard $2$-adic valuation decomposition of $p-1$; it shows that $p=2^kq+1$ with $q$ prime holds exactly when the odd part of $p-1$ is $1$ or prime, and it records the (case-by-case) uniqueness of $(k,q)$.  Lemma~1065.4 follows by extracting maximal powers of $2$ and $3$.
The computational table was obtained by direct enumeration of primes and exact arithmetic on $p-1$.

\subsection*{FINAL}
\textbf{UNRESOLVED.}

(i) \emph{Strongest proved partial result:}
The problem reduces to the arithmetic of $p-1$: for odd primes $p$, the condition $p=2^kq+1$ holds exactly when the odd part of $p-1$ is either $1$ or prime (equivalently, $p-1$ has at most one odd prime factor) (Lemma~1065.3), with complete classification of the degenerate cases $k=0$ and $q=2$ (Lemmas~1065.1--1065.2).  Analogously, $p=2^k3^\ell q+1$ holds exactly when $(p-1)/2^{v_2(p-1)}3^{v_3(p-1)}$ is $1$ or prime (Lemma~1065.4).  Computations up to $10^6$ give the exact counts recorded above.

(ii) \emph{First gap (crisp statement):}
Prove (or disprove) that there are infinitely many primes $p$ for which the odd part of $p-1$ is either $1$ or prime (equivalently, infinitely many primes $p$ with $p-1=2^k q$ for some prime $q$).

(iii) \emph{Top 3 next moves (concrete lemmas/computations):}
\begin{enumerate}
\item Study fixed-$k$ subfamilies: for each fixed $k\ge 1$, attempt to show infinitely many primes of the form $p=2^k q+1$ with $q$ prime (even $k=1$ is already a central special case).
\item For the $2^k3^\ell q+1$ variant, attempt to control primes $p$ for which $(p-1)$ has at most one prime factor outside $\{2,3\}$; a first concrete step is to obtain nontrivial lower bounds on the count of such primes up to $N$.
\item Extend the computations (e.g. to $N\approx 10^8$ if feasible) and record empirical densities as a function of $\log N$, plus the distribution of $(v_2(p-1),v_3(p-1))$ among the solutions.
\end{enumerate}

(iv) \emph{Minimal counterexample structure:}
If only finitely many such primes exist, then there is a bound $P_0$ such that for every prime $p>P_0$, $p-1$ has at least two distinct odd prime factors (for the first question), and for the second question $p-1$ has at least two prime factors outside $\{2,3\}$.


