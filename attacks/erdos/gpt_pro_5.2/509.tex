
\noindent\textbf{FORMAL RESTATEMENT}

Let $f\in\mathbb{C}[z]$ be monic, non-constant, of degree $d\ge 1$. Define the (closed) lemniscate
\[
E(f):=\{z\in\mathbb{C}: |f(z)|\le 1\}.
\]
Interpretation issue: the phrase ``covered by circles'' is ambiguous.

\begin{itemize}
\item \emph{Literal (1-dimensional) interpretation:} circles mean circumferences. Then $E(f)$ typically has nonempty interior (e.g. for $f(z)=z$), so it cannot be covered by countably many circles of finite total radius. Hence that literal reading is impossible.
\item \emph{Standard complex-analysis interpretation (Cartan/Pommerenke context):} ``circles'' means closed discs (each disc determined by its boundary circle). We adopt this corrected interpretation: we ask whether there exist closed discs $\overline{D}(c_j,r_j)$ such that
\[
E(f)\subseteq \bigcup_j \overline{D}(c_j,r_j)
\qquad\text{and}\qquad \sum_j r_j\le 2.
\]
The covering family is understood to be finite or countable so that $\sum_j r_j$ is defined.
\end{itemize}

Edge cases: $d=1$ gives a single closed unit disc, covered with total radius $1$. For $d=2$ the trivial cover by unit discs about the roots already has total radius $2$.

\medskip
\noindent\textbf{QUICK LITERATURE/CONTEXT CHECK}

The problem statement itself records: Cartan proved a universal covering with total radius $\le 2e$; Pommerenke improved the universal constant to $2.59$; and Pommerenke proved the sharp constant $2$ holds when $E(f)$ is connected. (I do not reproduce those proofs here.)

\medskip
\noindent\textbf{ATTACK PLAN}

\begin{itemize}
\item Proof-track idea: relate $E(f)$ to potential theory (logarithmic capacity $=1$ for monic polynomials) and compare capacity to a one-dimensional outer content given by disc coverings; then try to push constants down to $2$.
\item Disproof-track idea: search for high-degree polynomials where $E(f)$ has many separated ``thick'' components forcing large total covering radius. A plausible candidate would be polynomials with many near-multiple roots arranged to make components fat.
\end{itemize}
I did not find either a proof with constant $2$ or a counterexample.

\medskip
\noindent\textbf{WORK}

\noindent\textbf{Fast reality check (small degrees).}

\begin{itemize}
\item If $d=1$, $f(z)=z-a$, then $E(f)=\overline{D}(a,1)$, so one disc of radius $1$ covers it.
\item If $d=2$, writing $f(z)=(z-\alpha)(z-\beta)$ (monic), then
\[
E(f)\subseteq \overline{D}(\alpha,1)\cup \overline{D}(\beta,1)
\]
(because if $|z-\alpha|>1$ and $|z-\beta|>1$ then $|f(z)|>1$). Hence total radius $2$ suffices for $d=2$.
\end{itemize}

\medskip
\noindent\textbf{Lemma 1 (trivial disc cover by unit discs around roots).}
Let $f(z)=\prod_{j=1}^d (z-\alpha_j)$ be monic of degree $d\ge 1$ (roots counted with multiplicity). Then
\[
E(f)\subseteq \bigcup_{j=1}^d \overline{D}(\alpha_j,1).
\]

\textit{Proof.}
Take $z\in\mathbb{C}$ that lies outside every $\overline{D}(\alpha_j,1)$. Then $|z-\alpha_j|>1$ for each $j$, so
\[
|f(z)|=\prod_{j=1}^d |z-\alpha_j| > 1.
\]
Thus $z\notin E(f)$. Contrapositively, every $z\in E(f)$ must belong to at least one $\overline{D}(\alpha_j,1)$.
\hfill$\square$

\medskip
\noindent\textbf{Lemma 2 (each component of $\{|f|<1\}$ contains a zero of $f$).}
Let
\[
\Omega(f):=\{z\in\mathbb{C}: |f(z)|<1\}
\]
(which is open). Then every connected component $U$ of $\Omega(f)$ contains at least one zero of $f$.

\textit{Proof.}
Because $f$ is a non-constant polynomial, $|f(z)|\to\infty$ as $|z|\to\infty$. Hence there exists $R>0$ such that $|f(z)|>1$ whenever $|z|\ge R$, so $\Omega(f)\subset \overline{D}(0,R)$ is bounded. In particular, every connected component $U$ of $\Omega(f)$ is a bounded open set.

Fix such a component $U$ and suppose, for contradiction, that $f$ has no zeros in $U$. Then $f$ is nonzero on $\overline{U}$ as well: if $f(\zeta)=0$ for some $\zeta\in\overline{U}$, then $|f(\zeta)|=0<1$ and continuity would imply $\zeta\in\Omega(f)$, hence $\zeta\in U$, contradicting the assumption.

Define $g(z):=1/f(z)$. Then $g$ is holomorphic on $U$ and continuous on $\overline{U}$. Moreover, for $z\in U$ we have $|f(z)|<1$, so $|g(z)|>1$.

Next, we claim that for every boundary point $\zeta\in\partial U$ we have $|f(\zeta)|=1$, hence $|g(\zeta)|=1$. Indeed, since $\zeta$ is a limit of points $z_j\in U$ with $|f(z_j)|<1$, continuity gives $|f(\zeta)|\le 1$. If $|f(\zeta)|<1$, then by continuity $|f|<1$ in a neighborhood of $\zeta$, implying $\zeta$ is an interior point of $\Omega(f)$, contradicting $\zeta\in\partial U$. Thus $|f(\zeta)|=1$.

Now apply the maximum modulus principle to $g$ on the bounded domain $U$: the maximum of $|g|$ on $\overline{U}$ is attained on $\partial U$. But $|g|=1$ on $\partial U$ and $|g|>1$ on $U$, a contradiction. Therefore $U$ must contain a zero of $f$.
\hfill$\square$

\medskip
\noindent\textbf{VERIFICATION}

\begin{itemize}
\item Lemma 1: checked the contrapositive carefully; it uses only the factorisation of monic polynomials and that if all factors exceed $1$ in modulus, the product exceeds $1$.
\item Lemma 2: uses that $\Omega(f)=\{|f|<1\}$ is bounded for a polynomial $f$, that $|f|=1$ on $\partial U$ for any component $U$, and then applies the maximum modulus principle to $g=1/f$ (which has $|g|>1$ on $U$ but $|g|=1$ on $\partial U$), yielding a contradiction unless $U$ contains a zero.
\item These lemmas do not approach the sharp constant $2$ except in degree $2$; they are consistency checks and structural facts.
\end{itemize}

\medskip
\noindent\textbf{FINAL}

\textbf{UNRESOLVED}

(i) Strongest proved partial result: For degree $d=2$ the desired bound $\sum r_j\le 2$ holds by the trivial cover by unit discs about the two roots (and Lemma 1 gives a degree-$d$ cover with total radius $d$ in general).

(ii) First gap (crisp): Prove or disprove that for every monic polynomial $f$ there is a disc cover of $E(f)$ with total radii $\le 2$.

(iii) Top 3 next moves:
\begin{enumerate}
\item Attempt a potential-theoretic reduction: prove an inequality of the form ``for polynomial lemniscates $E(f)$, the one-dimensional outer content by discs is $\le 2\,\mathrm{cap}(E(f))$'' and then show $\mathrm{cap}(E(f))=1$.
\item Search numerically for extremisers: sample high-degree polynomials, approximate $E(f)$ by level sets, and attempt computational covering optimisations to see whether total radius $>2$ is ever forced.
\item Try a reduction to connected components: show one can cover each component of $\Omega(f)=\{|f|<1\}$ by a disc with radius controlled by something additive over components, and then sum over components.
\end{enumerate}

(iv) Minimal counterexample structure (if false): would require a monic $f$ for which $E(f)$ has many ``thick'' separated components so that any disc cover needs total radius $>2$ despite being able to place many small discs; heuristically this would demand components whose diameters do not shrink fast enough as the number of components grows.


