% Erdos Problem #677

1) FORMAL RESTATEMENT

For integers $n\ge 0$ and $k\ge 1$, define
\[
M(n,k):=\operatorname{lcm}(n+1,n+2,\dots,n+k).
\]
Primary question: Is it true that for all integers $n\ge 0$, $k\ge 1$, and $m\ge n+k$,
\[
 M(n,k)\ne M(m,k)?
\]
That is, do disjoint length-$k$ intervals of consecutive integers always have different least common multiples?

More general question: how many solutions can the equation $M(n,k)=M(m,\ell)$ have under the conditions $m\ge n+k$ and $\ell>1$?

2) QUICK LITERATURE/CONTEXT CHECK

As stated in the problem text:
- The Thue--Siegel theorem implies that for fixed $k$, there are only finitely many solutions with $m\ge n+k$ to $M(n,k)=M(m,k)$.
- Erd\H{o}s knew solutions with different lengths, e.g. $M(4,3)=M(13,2)$ and $M(3,4)=M(19,2)$.

I do not use any other external results here.

3) ATTACK PLAN

- Compute small cases to look for equalities for fixed $k$.
- Prove simple cases ($k=1,2$) completely.
- Derive constraints from prime-adic valuations: $v_p(M(n,k))=\max_{1\le i\le k} v_p(n+i)$.

4) WORK

FAST REALITY CHECK (computation)

I searched for solutions to $M(n,k)=M(m,k)$ with $m\ge n+k$ for $1\le k\le 10$ and $0\le n,m\le 500$. No solutions were found.

As a sanity check for the "mixed length" equation $M(n,k)=M(m,\ell)$, a search with $0\le n,m\le 200$ and $1\le k,\ell\le 6$ found (among others) the known identities
\[
M(4,3)=M(13,2)=210,\qquad M(3,4)=M(19,2)=420,
\]
and also additional ones such as
\[
M(8,4)=M(43,2)=1980,\qquad M(2,5)=M(19,2)=420,\qquad M(1,6)=M(19,2)=420.
\]

Lemma 677.1 (cases $k=1,2$)

For $k=1$ we have $M(n,1)=n+1$, so $M(n,1)=M(m,1)$ implies $n=m$.
For $k=2$ we have $M(n,2)=\operatorname{lcm}(n+1,n+2)=(n+1)(n+2)$, so $M(n,2)=M(m,2)$ implies $n=m$.
In particular, for $k\in\{1,2\}$ there are no solutions with $m\ge n+k$.

Proof.
For $k=1$ this is immediate.
For $k=2$, consecutive integers are coprime, i.e. $\gcd(n+1,n+2)=1$. Hence
\[
M(n,2)=\frac{(n+1)(n+2)}{\gcd(n+1,n+2)}=(n+1)(n+2).
\]
If $(n+1)(n+2)=(m+1)(m+2)$ with $n,m\ge 0$, then expanding gives
\[
 n^2+3n+2=m^2+3m+2\iff (n-m)(n+m+3)=0.
\]
Since $n+m+3>0$, we obtain $n=m$.


Lemma 677.2 (prime-power constraint for primes $p>k$)

Let $k\ge 1$ and let $n\ge 0$. Let $p$ be a prime with $p>k$. Then among the $k$ integers $n+1,\dots,n+k$, at most one is divisible by $p$.
If in fact $p$ divides $n+i$ for some $1\le i\le k$, then
\[
 v_p(M(n,k))=v_p(n+i).
\]
Consequently, if $M(n,k)=M(m,k)$ and $p>k$ divides $M(n,k)$, then there exists some $1\le j\le k$ such that
\[
 v_p(m+j)=v_p(n+i)=v_p(M(n,k)),
\]
so in particular $p^{v_p(n+i)}\mid (m+j)$.

Proof.
If $p>k$, then two distinct numbers in $\{n+1,\dots,n+k\}$ differ by at most $k-1<p$, so they cannot both be multiples of $p$ (two multiples of $p$ differ by at least $p$). Hence at most one is divisible by $p$.

If $p\mid(n+i)$ for some $i$, then $v_p(n+i)\ge 1$ and for every $i'\ne i$ we have $v_p(n+i')=0$. By definition,
\[
 v_p(M(n,k))=\max_{1\le r\le k} v_p(n+r)=v_p(n+i).
\]

Now assume $M(n,k)=M(m,k)$. Then $v_p(M(n,k))=v_p(M(m,k))$. Let $e:=v_p(M(n,k))$. By definition of $M(m,k)$, there exists $j\in\{1,\dots,k\}$ with $v_p(m+j)=e$. This implies $p^e\mid(m+j)$.


5) VERIFICATION

- Lemma 677.1: for $k=2$, checked the gcd claim and the algebra.
- Lemma 677.2: the "at most one multiple" argument uses only $p>k$ and the interval length.

Quantifier check: the main question requires $m\ge n+k$ (disjoint intervals). Lemma 677.1 covers $k\in\{1,2\}$ completely.

6) FINAL

\textbf{UNRESOLVED}

(i) Strongest proved partial result:

For $k=1,2$, there are no solutions with $m\ge n+k$ (Lemma 677.1). For general $k$, primes $p>k$ contribute to $M(n,k)$ through at most one element of the interval, so any equality $M(n,k)=M(m,k)$ forces matching prime-power divisibility conditions across the two intervals (Lemma 677.2). Computations found no same-$k$ equalities for $k\le 10$ and $n,m\le 500$.

(ii) First gap (crisp):

I cannot rule out the existence of a solution $M(n,k)=M(m,k)$ with $k\ge 3$ and $m\ge n+k$.

(iii) Top 3 next moves:

1. Use Lemma 677.2 together with existence of a prime $p$ in $(n, n+k]$ that is sufficiently large (e.g. $>k$) to force strong congruence constraints on $m$; then try to show these constraints are incompatible for multiple such primes.
2. Study the stronger conjecture mentioned in the problem text about equality of prime factor sets of the products $\prod_{i\le k}(n+i)$ and $\prod_{i\le k}(m+i)$.
3. Extend computation to larger ranges and search systematically for same-$k$ equalities, tracking the prime-power structure suggested by Lemma 677.2.

(iv) Minimal counterexample structure:

A minimal counterexample would be a triple $(n,m,k)$ with $k\ge 3$, $m\ge n+k$, such that for every prime $p$ the maximal $p$-adic valuation attained among $n+1,\dots,n+k$ equals the maximal valuation among $m+1,\dots,m+k$. Lemma 677.2 shows that for primes $p>k$, this forces the two intervals to contain matching multiples of $p^e$ at the same exponent $e$, creating a rigid system of congruences that a counterexample would have to satisfy simultaneously.

