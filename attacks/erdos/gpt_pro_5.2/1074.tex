% Erdos Problem #1074

1) FORMAL RESTATEMENT
Define the set S of all integers m >= 1 such that there exists a prime p with two properties:
  (a) p divides (m!+1), and
  (b) p is NOT congruent to 1 modulo m, i.e. (p-1) is not divisible by m.

Define the set P of primes p such that there exists an integer m with the same two properties:
  p divides (m!+1) and (p-1) is not divisible by m.

Questions:
- Does the asymptotic density of S exist?
- Does the asymptotic density of P among the primes exist?

2) QUICK LITERATURE/CONTEXT CHECK
From the problem statement:
- Erdős, Hardy and Subbarao proved that S and P are infinite.
- The first values of S are: 8, 9, 13, 14, 15, 16, 17, ...
- Hardy and Subbarao checked up to m <= 2^10 and found m in S for m >= 8.
No other external results are assumed here.

3) ATTACK PLAN
To study S and P one needs structural constraints on prime divisors of m!+1 (especially congruence information), and at least small exact computations to confirm the initial segment.

4) WORK
Lemma 1074.1 (Any prime divisor of m!+1 exceeds m).
If p is prime and p divides (m!+1), then p > m.
Proof. If p <= m then p divides m!, hence p divides (m!+1 - m!) = 1, impossible. Therefore p>m. QED.

Lemma 1074.2 (Wilson complement identity specialized to a prime divisor of m!+1).
Let p be an odd prime and suppose p divides (m!+1) for some m with 1 <= m <= p-2.
Then (p-1-m)! ≡ (-1)^m (mod p). In particular, if m is odd then p divides ((p-1-m)! + 1).
Proof. Apply Lemma 1072.3 with n=m: for any odd prime p and 0<=m<=p-1,
  m! * (p-1-m)! ≡ (-1)^(m+1) (mod p).
If additionally p | (m!+1), then m! ≡ -1 (mod p). Substituting gives
  (-1) * (p-1-m)! ≡ (-1)^(m+1) (mod p),
hence (p-1-m)! ≡ (-1)^m (mod p). If m is odd then (-1)^m = -1, so p divides ( (p-1-m)! + 1 ). QED.

5) VERIFICATION (small cases)
For m <= 20, m!+1 is small enough to factor exactly, so membership in S can be checked exactly.
The computed S intersect {1,...,20} equals:
  S cap [1,20] = {8, 9, 13, 14, 15, 16, 17, 18, 19, 20}.
Witness primes p (one example each) are:
  m=8  : p=61 divides 8!+1 and (61-1) mod 8 = 4
  m=9  : p=71 divides 9!+1 and (71-1) mod 9 = 7
  m=13 : p=83 divides 13!+1 and (83-1) mod 13 = 4
  m=14 : p=23 divides 14!+1 and (23-1) mod 14 = 8
  m=15 : p=59 divides 15!+1 and (59-1) mod 15 = 13
  m=16 : p=61 divides 16!+1 and (61-1) mod 16 = 12
  m=17 : p=661 divides 17!+1 and (661-1) mod 17 = 14
  m=18 : p=23 divides 18!+1 and (23-1) mod 18 = 4
  m=19 : p=71 divides 19!+1 and (71-1) mod 19 = 13
  m=20 : p=20639383 divides 20!+1 and (p-1) mod 20 != 0.
From these m<=20 computations, the set of Pillai primes P encountered is:
  {23, 29, 59, 61, 67, 71, 83, 137, 139, 269, 479, 661, 537913, 1000357, 1059511, 20639383, 75024347, 3790360487, 117876683047, 1713311273363831}.

6) FINAL
UNRESOLVED
(i) Strongest proved partial result here: any prime divisor p of m!+1 must satisfy p>m (Lemma 1074.1), and such a divisor forces a complementary factorial congruence (Lemma 1074.2). Exact computation confirms the stated initial segment S cap [1,20] = {8,9,13,14,15,16,17,18,19,20}.
(ii) First gap (crisp): Determine whether the limit density of S exists (and similarly for P among primes). Even proving that S has density 1 (or density 0, or does not converge) is not resolved here.
(iii) Top 3 next moves (concrete):
  1. Prove a criterion ensuring m in S that can be checked without fully factoring m!+1, e.g. by guaranteeing a prime divisor p of m!+1 with (p-1) not divisible by m.
  2. Establish upper and lower bounds for |S cap [1,x]| that are within o(x) of each other (or exhibit oscillations).
  3. Compute S for much larger m with rigorous methods (e.g. finding at least one certified prime divisor p of m!+1 and checking (p-1) mod m).
(iv) Minimal counterexample structure: If S failed to have density 1, one would need infinitely many m for which every prime divisor p of m!+1 satisfies (p-1) divisible by m.


