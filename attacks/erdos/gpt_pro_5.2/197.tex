% ------------------------------------------------------------
% Erdos problem 197
% ------------------------------------------------------------
\section*{Erd\H{o}s problem 197}

\subsection*{1) FORMAL RESTATEMENT}
A set $A\subseteq \mathbb{N}$ is said to be \emph{permutable to avoid monotone 3-APs} if there exists a bijection
$\pi:\mathbb{N}\to A$ such that the sequence $(\pi(1),\pi(2),\dots)$ contains no monotone 3-term arithmetic progression
(i.e. no indices $i<j<k$ with $\pi(i),\pi(j),\pi(k)$ in arithmetic progression and monotone in that order).
Question: Can $\mathbb{N}$ be partitioned into two such sets $A,B$?
The problem text notes the analogous question for three sets is open.

\subsection*{2) QUICK LITERATURE/CONTEXT CHECK}
Problem text cites DEGS77 showing: in any permutation of $\mathbb{N}$ there is a monotone 3-AP; hence $\mathbb{N}$ itself is \emph{not}
permutable to avoid. Partition question remains.

\subsection*{3) ATTACK PLAN}
Provide two core structural lemmas:
(1) heredity under subsets,
(2) affine invariance (shift/scale).
Then perform a finite sanity-check: for $N\le 12$, exhaustive search found a partition of $\{1,\dots,N\}$ into two subsets each permutable to avoid (computed).

\subsection*{4) WORK}

\paragraph{Lemma 4.1 (Heredity).}
If a set $A\subseteq \mathbb{N}$ is permutable to avoid monotone 3-APs, then every subset $A'\subseteq A$ is also permutable to avoid monotone 3-APs.
\textit{Proof.}
Let $\pi$ be an avoiding permutation of $A$. Restrict $\pi$ to the indices where $\pi(i)\in A'$. This yields a permutation of $A'$.
Any monotone 3-AP in the restricted sequence would also be a monotone 3-AP in the original sequence, contradiction. \hfill$\square$

\paragraph{Lemma 4.2 (Affine invariance).}
If $A$ is permutable to avoid monotone 3-APs and $u\in\mathbb{Z}$, $v\in\mathbb{N}$, then
\[
u+vA:=\{u+va:\ a\in A\}
\]
is permutable to avoid monotone 3-APs (when intersected with $\mathbb{N}$ as needed).
\textit{Proof.}
Let $\pi$ be an avoiding permutation of $A$. Then $\pi'(i)=u+v\pi(i)$ is a permutation of $u+vA$.
Three values form a 3-term arithmetic progression iff their affine images do, and monotonicity is preserved because $v>0$.
Thus any monotone 3-AP in $\pi'$ would correspond to one in $\pi$, contradiction. \hfill$\square$

\subsection*{FAST REALITY CHECK (computed)}
For finite sets $[N]=\{1,\dots,N\}$, exhaustive backtracking verified:
for each $N\le 12$ there exists a partition $[N]=A\sqcup B$ such that both $A$ and $B$ admit a permutation avoiding monotone 3-APs.
This is only finite evidence (does not imply an infinite partition).

\subsection*{6) FINAL}
\textbf{UNRESOLVED}

(i) Strongest proved partial results: heredity (Lemma 4.1) and affine invariance (Lemma 4.2).
Finite partitions exist for $N\le 12$ by computation.

(ii) First gap: produce an \emph{infinite} partition $\mathbb{N}=A\sqcup B$ with both sets permutable to avoid, or prove impossible.

(iii) Top 3 next moves:
1. Strengthen finite computations: find maximal $N$ where partitions exist and identify extension obstructions.
2. Try to build $A,B$ recursively using affine invariance gadgets (block concatenations).
3. Prove a compactness criterion: if every finite initial segment admits a partition with a coherent extension property, then an infinite partition exists (K\H{o}nig-type).

(iv) Minimal counterexample structure: a proof that \emph{any} partition $A\sqcup B$ forces at least one side to be ``extension-rigid'' so that
every permutation of it eventually contains a monotone 3-AP.

