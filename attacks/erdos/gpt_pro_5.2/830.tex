
FORMAL RESTATEMENT

For $n\in\mathbb{N}$ let $\sigma(n)=\sum_{d\mid n} d$ be the sum-of-divisors function.
A pair $(a,b)\in\mathbb{N}^2$ is called an amicable pair if
\[
\sigma(a)=\sigma(b)=a+b.
\]
Equivalently, writing $s(n)=\sigma(n)-n$ for the sum of proper divisors, amicability is $s(a)=b$ and $s(b)=a$ with $a\ne b$.
Questions:
1. Are there infinitely many amicable pairs?
2. If $A(x)$ counts the number of amicable pairs $1\le a\le b\le x$, is it true that $A(x)>x^{1-o(1)}$ as $x\to\infty$?

QUICK LITERATURE/CONTEXT CHECK

The problem statement records that Erd\H{o}s proved $A(x)=o(x)$ and that Pomerance proved explicit upper bounds of the form
$A(x)\le x\exp(- (\log x)^{1/3})$ and later $A(x)\le x\exp(- (\tfrac12+o(1))(\log x\log\log x)^{1/2})$.
We do not use these results below.

ATTACK PLAN

Proof track:
(1) Derive structural constraints on amicable pairs from the defining equality $\sigma(a)=a+b$.
(2) Try to construct infinite families by parametrizing $s(n)$ (hard; no complete construction here).

Disproof track:
(1) Look for congruence obstructions forcing finiteness (none found here).
(2) Use computation to gauge growth of $A(x)$ at small $x$.

WORK

Lemma 1 (abundant/deficient dichotomy).
Let $(a,b)$ be an amicable pair with $a<b$.
Then $a$ is abundant (i.e. $\sigma(a)>2a$) and $b$ is deficient (i.e. $\sigma(b)<2b$).

Proof.
Because $\sigma(a)=a+b$ and $b>a$, we have
\[
\sigma(a)=a+b> a+a =2a,
\]
so $a$ is abundant.
Similarly, $\sigma(b)=a+b < b+b=2b$ because $a<b$.
Thus $b$ is deficient.
$\square$

Lemma 2 (no primes in amicable pairs).
If $(a,b)$ is an amicable pair, then both $a$ and $b$ are composite integers at least $4$.

Proof.
First, $a\ne 1$ because $\sigma(1)=1$ would force $a+b=1$.
So $a\ge 2$ and similarly $b\ge 2$.

Assume for contradiction that $a$ is prime, $a=p$.
Then $\sigma(p)=1+p$.
The amicable condition $\sigma(a)=a+b$ yields $1+p = p+b$, hence $b=1$.
But then $\sigma(b)=\sigma(1)=1$, while the defining equality requires $\sigma(b)=a+b=p+1>1$, contradiction.
Thus $a$ is not prime. By symmetry, $b$ is not prime.
Therefore both are composite.
Since the smallest composite integer is $4$, this implies $a,b\ge 4$.

$\square$

Fast reality check (computation; exact search).
We computed $A(x)$ for several small values by sieving the sum of proper divisors $s(n)=\sigma(n)-n$ and checking $s(a)=b,\ s(b)=a$.
The number of amicable pairs with $1\le a<b\le x$ found was:
\[
\begin{array}{r|r}
 x & A(x) \\\hline
 10^4 & 5\\
 5\cdot 10^4 & 8\\
 10^5 & 13\\
 2\cdot 10^5 & 20\\
 5\cdot 10^5 & 28\\
 10^6 & 40
\end{array}
\]
The first five pairs (for $x\ge 10^4$) were
\[(220,284),\ (1184,1210),\ (2620,2924),\ (5020,5564),\ (6232,6368),\]
and an odd amicable pair appears by $x=5\cdot 10^4$:
\[(12285,14595).
\]
These data are only sanity checks and do not suggest asymptotics.

VERIFICATION

(1) Lemma~1 is a direct inequality check and depends only on $a<b$.
(2) Lemma~2: the prime case gives an immediate contradiction; symmetry handles both variables.
(3) The computation uses exact divisor-sum sieving, and the listed amicable pairs match standard small examples.

FINAL

**UNRESOLVED**

(i) Strongest proved partial result.
We proved basic structural facts: in any amicable pair with $a<b$, the smaller number is abundant and the larger is deficient (Lemma~1), and neither entry can be prime (Lemma~2).
We also computed $A(x)$ exactly for $x\le 10^6$ (with $A(10^6)=40$).

(ii) First gap (crisp statement).
Prove or disprove: there exist infinitely many amicable pairs $(a,b)$.

(iii) Top 3 next moves.
1. Attempt to build parametric families of amicable pairs by controlling $s(n)$ on structured factorizations (e.g. $n=2^u p^v q^w$).
2. Computation: extend exact sieves to larger $x$ (e.g. $10^8$ or more) and fit empirical growth, searching for patterns that could hint at construction.
3. Prove unconditional lower bounds for $A(x)$ beyond triviality; even showing $A(x)\to\infty$ without explicit pairs would be progress.

(iv) Minimal counterexample structure.
A counterexample to “infinitely many” would mean there exists $X_0$ such that no amicable pair has both entries $>X_0$.
A counterexample to $A(x)>x^{1-o(1)}$ could still allow infinitely many pairs but with very sparse distribution, e.g. $A(x)\le x^{\theta}$ for some fixed $\theta<1$.

