\section*{Problem 774: proportionately dissociated sets vs.\ bounded unions}

\subsection*{1) Formal restatement}
A set \(A\subseteq\mathbb{N}\) is \emph{dissociated} if every relation
\[
\sum_{i=1}^t \varepsilon_i a_i = 0,\qquad a_i\in A\ \text{distinct},\ \varepsilon_i\in\{-1,0,1\},
\]
forces all \(\varepsilon_i=0\).
(For positive integers, this is equivalent to uniqueness of subset sums.)

Say that \(A\) is \emph{proportionately dissociated} if there exists \(\delta>0\) such that every finite \(B\subseteq A\) contains a dissociated \(C\subseteq B\) with \(|C|\ge \delta|B|\).

Assume additionally that \(A\) is a union of finitely many dissociated sets.
Question: must \(A\) in fact be a union of \emph{at most} \(M(\delta)\) dissociated sets for some bound depending only on \(\delta\)?

\subsection*{2) Quick literature/context check (no more than 8 lines)}
\begin{itemize}[leftmargin=*]
\item This question is attributed to Pisier and is discussed by Alon--Erd\H{o}s (it is not settled there).
\item The integer/\(\mathbb{N}\) case is widely recorded as open.
\item The analogous question for \(B_2\) (Sidon) sets is known to have a negative answer (per recent notes on the ErdosProblems site).
\end{itemize}

\subsection*{3) Attack plan}
\begin{itemize}[leftmargin=*]
\item Note the easy direction: if \(A\) is a union of \(m\) dissociated sets then it is proportionately dissociated with \(\delta\ge 1/m\).
\item Examine what the \(\delta\)-property alone yields via greedy extraction (gives \(O_\delta(\log|B|)\) parts for a finite \(B\)).
\item Prove a positive result in a model setting where ``dissociated'' is matroid-independence (e.g.\ \((\mathbb{Z}/2\mathbb{Z})^n\)), to illustrate why the general integer case is harder.
\end{itemize}

\subsection*{4) Work}

\paragraph{Finite union \(\Rightarrow\) proportionality (easy).}
If \(A=D_1\cup\cdots\cup D_m\) with each \(D_i\) dissociated, then for any finite \(B\subseteq A\) one of the intersections \(B\cap D_i\) has size \(\ge |B|/m\), and is dissociated.
So \(A\) is proportionately dissociated with \(\delta\ge 1/m\).

\paragraph{Greedy decomposition gives only logarithmically many dissociated pieces.}
Assume \(A\) is proportionately dissociated with constant \(\delta>0\).
Given a finite \(B\subseteq A\), pick dissociated \(C_1\subseteq B\) with \(|C_1|\ge \delta|B|\), remove it, and iterate.
After \(t\) steps, the remainder has size \(\le (1-\delta)^t|B|\), so \(B\) can be covered by
\[
t=O_\delta(\log|B|)
\]
dissociated sets.
This does \emph{not} give a uniform bound independent of \(|B|\), which is what Problem~774 asks for.

\paragraph{A positive result in a \(2\)-torsion model case.}
Let \(G=(\mathbb{Z}/2\mathbb{Z})^n\).
In \(G\), dissociated sets (in the signed-sum sense) coincide with linearly independent sets over \(\mathbb{F}_2\), because \(-1\equiv 1\pmod 2\) and a nontrivial signed relation is exactly a nontrivial subset-sum relation.

Suppose \(A\subseteq G\) has the property that every finite \(B\subseteq A\) contains an independent subset of size at least \(\delta|B|\).
Then \(\mathrm{rank}(B)\ge \delta|B|\) for all finite \(B\subseteq A\).

For a finite \(B\), the matroid-cover theorem for vector matroids implies that \(B\) can be covered by \(k\) independent sets iff
\(|X|\le k\,\mathrm{rank}(X)\) for all \(X\subseteq B\).
Here \(|X|\le (1/\delta)\mathrm{rank}(X)\) holds for all \(X\subseteq B\), so taking \(k=\lceil 1/\delta\rceil\) gives such a cover.
Hence in this model case the desired conclusion holds with
\[
M(\delta)\le \left\lceil \frac{1}{\delta}\right\rceil.
\]

\paragraph{Why this does not settle the integer case.}
For \(A\subseteq\mathbb{N}\), dissociated sets do not obviously form the independent sets of a matroid (exchange can fail),
so the above matroid-cover argument does not apply directly.
The open question is whether the additional additive structure in \(\mathbb{N}\), together with proportional dissociation and the assumption of a finite cover, forces a uniform bound.

\subsection*{5) Verification}
\begin{itemize}[leftmargin=*]
\item The pigeonhole implication ``union of \(m\) dissociated sets \(\Rightarrow\) \(\delta\ge 1/m\)'' is immediate.
\item The greedy bound \(O_\delta(\log|B|)\) is correct since each extraction removes at least a \(\delta\)-fraction.
\item In \((\mathbb{Z}/2\mathbb{Z})^n\), signed dissociation equals linear independence; the matroid criterion yields the explicit bound \(\lceil 1/\delta\rceil\).
\end{itemize}

\subsection*{6) Final (exactly one label and one sub-label)}
\noindent\textbf{LABEL: UNRESOLVED.}\\
\textbf{SUBLABEL: Gave a model-case positive result (2-torsion groups) and general reductions; the original \(\mathbb{N}\) question remains open.}

\subsection*{7) Completion estimate}
A full resolution would require either constructing a counterexample in \(\mathbb{N}\) for some \(\delta>0\), or proving a structural theorem that forces a uniform cover bound \(M(\delta)\) (likely needing tools beyond greedy decompositions).

%%%%%%%%%%%%%%%%%%%%%%%%%%%%%%%%%%%%%%%%%%%%%%%%%%%%%%%%%%%%%%%%%%%%%%%%%%%%%%%
