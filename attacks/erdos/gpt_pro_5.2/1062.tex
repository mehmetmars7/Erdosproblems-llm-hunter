
\subsection*{FORMAL RESTATEMENT}
For $n\ge 1$, let $f(n)$ be the maximum cardinality of a subset $A\subseteq\{1,2,\dots,n\}$ such that there do not exist three \emph{distinct} elements $a,b,c\in A$ with $a\mid b$ and $a\mid c$.
Equivalently: for each $a\in A$, there is at most one element $b\in A\setminus\{a\}$ with $a\mid b$.
The problem asks for the size and asymptotics of $f(n)$, and in particular about $\lim f(n)/n$.
The statement also records a known estimate (Lebensold): for large $n$,
\[
0.6725\,n\le f(n)\le 0.6736\,n.
\]

\subsection*{QUICK LITERATURE/CONTEXT CHECK}
The numerical bounds $0.6725n\le f(n)\le 0.6736n$ are stated in the problem text as due to Lebensold. I do not use or claim any other external results.

\subsection*{ATTACK PLAN}
Proof-track: (i) derive structural constraints on the divisibility poset induced by $A$; (ii) build explicit large constructions giving lower bounds; (iii) attempt counting/charging arguments for upper bounds.
Disproof-track (for the ``limit irrational'' question): compute $f(n)/n$ for small $n$ to see behavior; but proving irrationality is far beyond small checks.

\subsection*{WORK}
\textbf{FAST REALITY CHECK.}
I computed $f(n)$ exactly for $n\le 30$ by branch-and-bound search over subsets, using the defining constraint.
Exact values:
\begin{verbatim}
 n : f(n)
 1 : 1
 2 : 2
 3 : 2
 4 : 3
 5 : 4
 6 : 4
 7 : 5
 8 : 6
 9 : 6
10 : 7
11 : 8
12 : 8
13 : 9
14 : 10
15 : 10
16 : 11
17 : 12
18 : 12
19 : 13
20 : 14
21 : 14
22 : 15
23 : 16
24 : 16
25 : 17
26 : 18
27 : 19
28 : 20
29 : 21
30 : 21
\end{verbatim}
For instance, $f(30)=21$ (ratio $0.7$), while the simple construction below gives $\lceil 2\cdot 30/3\rceil=20$.

\medskip
\textbf{Lemma 1062.1 (no 3-term chains).}
If $A$ satisfies the condition of the problem, then there do not exist $a<b<c$ in $A$ with $a\mid b$ and $b\mid c$.

\emph{Proof.}
If $a\mid b$ and $b\mid c$, then $a\mid c$. Thus $a$ divides both $b$ and $c$ with $b\ne c$, contradicting the defining property of $A$. \hfill$\square$

\medskip
\textbf{Lemma 1062.2 (elementary $2n/3$ construction).}
For every $n\ge 1$,
\[
f(n)\ge n-\left\lfloor\frac n3\right\rfloor\ \ge\ \left\lceil\frac{2n}{3}\right\rceil.
\]
One explicit choice is
\[
A:=\Big\{\left\lfloor\tfrac n3\right\rfloor+1,\ \left\lfloor\tfrac n3\right\rfloor+2,\ \dots,\ n\Big\}.
\]

\emph{Proof.}
Let $m:=\lfloor n/3\rfloor$ and $A=\{m+1,\dots,n\}$. Then $|A|=n-m$.
Take any $a\in A$. Since $a\ge m+1>n/3$, we have $3a>n$.
Therefore the only possible multiples of $a$ in $\{1,\dots,n\}$ are $a$ itself and possibly $2a$ (since $2a\le n$ may occur, but $3a$ cannot).
In particular, within $A$ the element $a$ can divide at most one other distinct element (namely $2a$ if it lies in $A$).
Hence there do not exist distinct $b,c\in A$ with $a\mid b$ and $a\mid c$. This verifies that $A$ is admissible, and yields $f(n)\ge |A|=n-\lfloor n/3\rfloor\ge \lceil 2n/3\rceil$. \hfill$\square$

\subsection*{VERIFICATION}
Lemma 1062.1 is an immediate logical consequence of the definition.
Lemma 1062.2 checks the only possible multiples of $a>n/3$ inside $\{1,\dots,n\}$ and uses $3a>n$ to rule out two distinct multiples beyond $a$ itself.
The exact small-$n$ values were computed by an explicit search that maintains, for each potential divisor $d$, the count of already-selected multiples of $d$.

\subsection*{FINAL}
\textbf{UNRESOLVED}
\begin{enumerate}
\item[(i)] \textbf{Strongest proved partial result.}
An explicit construction gives $f(n)\ge \lceil 2n/3\rceil$ for all $n$ (Lemma 1062.2). The problem statement records much sharper bounds for large $n$ (Lebensold): $0.6725n\le f(n)\le 0.6736n$. I also computed $f(n)$ exactly for $n\le 30$ (see the table in WORK).
\item[(ii)] \textbf{First gap (crisp).}
Determine the true asymptotic constant $\lim_{n\to\infty} f(n)/n$ (existence, value) and address whether it is irrational.
\item[(iii)] \textbf{Top 3 next moves.}
(1) Computation: extend exact/near-exact optimization for $f(n)$ to much larger $n$ to better estimate $f(n)/n$ and detect patterns.
(2) Upper bounds: develop a charging/double-counting argument that forces $|A|\le (\alpha+o(1))n$ for some explicit $\alpha<0.6736$.
(3) Constructions: search for structured sets improving the $2/3$ construction and explain the near-$0.673$ density seen in known bounds.
\item[(iv)] \textbf{Minimal counterexample structure.}
A counterexample to convergence of $f(n)/n$ would require large oscillations in the optimal density across $n$; such oscillations would likely correlate with arithmetic structure of $n$ (e.g. primorial-like ranges) affecting how many divisibility relations can be packed.
\end{enumerate}


