\section{Round 5 (Problem 30): improving the Singer lower bound via a gap/diameter argument}

\subsection{1) Round-5 Objective}
\textbf{Path (A): proof direction (strengthen explicit \emph{lower} bounds).}
Round~4 produced an explicit infinite family from Singer perfect difference sets:
for every prime power $q$,
\[
h(q^2+q+1)\ge q+1,
\]
implying $\limsup_{N\to\infty}\bigl(h(N)-\sqrt N\bigr)\ge \tfrac12$.

In this round we \emph{strictly strengthen} that conclusion by showing that the same Singer family can be \emph{translated} to fit in a much shorter integer interval.  Concretely, we prove:
\[
h(q^2+1)\ \ge\ q+1\qquad\text{for every prime power }q.
\]
As a consequence,
\[
\limsup_{N\to\infty}\bigl(h(N)-\sqrt N\bigr)\ \ge\ 1,
\]
a strict improvement over Round~4.

The original Erd\H{o}s--Tur\'an question remains open: is it true that for every $\varepsilon>0$,
\[
h(N)=\sqrt N+O_\varepsilon(N^\varepsilon)\, ?
\]

\subsection{2) Round-4 Foundation Used}
We rely on the following vetted Round-4 ingredients.

\begin{itemize}
\item \textbf{(R4 Singer parameter family)} For each prime power $q$ and $v=q^2+q+1$ there exists a perfect difference set
$D\subset \mathbb Z/v\mathbb Z$ of size $|D|=q+1$, hence $h(v)\ge q+1$.
\item \textbf{(R4 modular-to-integer transfer lemma)} If $B\subseteq\{0,1,\dots,v-1\}$ has all ordered differences distinct mod $v$, then $B$ is Sidon in $\mathbb Z$.
\end{itemize}

We will \emph{strengthen} the transfer lemma, and add a new ``large gap'' translation lemma.

\subsection{3) New Insight / Tool (Round 5)}
\textbf{(i) Gap lemma on the cycle.}
Any $k$-subset of a cyclic group $\mathbb Z/v\mathbb Z$ has a translate that lies in an \emph{arc} of length at most $v-\lceil v/k\rceil$.
This is achieved by cutting the circle at a largest gap between consecutive points.

\medskip
\textbf{(ii) Modular Sidon + injectivity $\Rightarrow$ integer Sidon.}
The Round-4 transfer lemma is upgraded: we do \emph{not} need representatives in $\{0,\dots,v-1\}$, only that reduction mod $v$ is injective on the chosen integer representatives (automatic if the diameter is $<v$).

Together, (i)--(ii) show Singer difference sets of size $q+1$ can be realized inside an interval of length $q^2$, yielding a larger additive constant in $h(N)-\sqrt N$.

\subsection{4) Attack Plan (Round 5)}
Round~4 embeds Singer sets in $\{1,\dots,q^2+q+1\}$, giving deviation $\approx \tfrac12$.

To improve this:
\begin{enumerate}
\item Prove the \emph{gap lemma} to find a translate of a size-$(q+1)$ subset of $\mathbb Z/(q^2+q+1)\mathbb Z$ with diameter $\le q^2$.
\item Prove the strengthened \emph{injective modular-to-integer Sidon lemma} (diameter $<v$ suffices).
\item Apply these to Singer perfect difference sets and compare $q+1$ with $\sqrt{q^2+1}$ to obtain $\limsup\ge 1$.
\end{enumerate}

\subsection{5) Work (Round 5)}

\subsubsection{5.1. A translation lemma from a large cyclic gap}
\begin{lemma}[Large gap $\Rightarrow$ small diameter translate]\label{lem:gap_translate}
Let $v\ge 2$ and let $D\subset \mathbb Z/v\mathbb Z$ have size $k\ge 1$.
Then there exists a translate $D+t$ (with $t\in\mathbb Z/v\mathbb Z$) and a choice of integer representatives
\[
B\subset \mathbb Z
\quad\text{with}\quad
B\equiv D+t\pmod v
\]
such that
\[
\mathrm{diam}(B):=\max B-\min B\ \le\ v-\left\lceil\frac{v}{k}\right\rceil.
\]
\end{lemma}

\begin{proof}
Choose standard representatives of $D$ in $\{0,1,\dots,v-1\}$ and list them in increasing order:
\[
0\le d_0<d_1<\cdots<d_{k-1}\le v-1.
\]
Define the cyclic gaps
\[
g_i:=d_{i+1}-d_i\quad(0\le i\le k-2),
\qquad
g_{k-1}:=(d_0+v)-d_{k-1}.
\]
These are positive integers and satisfy $\sum_{i=0}^{k-1} g_i=v$.
Hence $\max_i g_i\ge \left\lceil \frac{v}{k}\right\rceil$.

Let $j$ be an index with $g_j=\max_i g_i$.
Cut the cycle at this largest gap and translate so the element \emph{after} the gap becomes $0$.
More precisely:
\begin{itemize}
\item if $0\le j\le k-2$, translate by $t\equiv -d_{j+1}\pmod v$;
\item if $j=k-1$, translate by $t\equiv -d_0\pmod v$.
\end{itemize}
Under this translation, the elements of $D+t$ lie on the complementary arc of length $v-g_j$.
Choosing integer representatives along that arc produces a set $B\subset \mathbb Z$ whose minimum is $0$ and whose maximum is at most $v-g_j$.
Therefore
\[
\mathrm{diam}(B)\le v-g_j\le v-\left\lceil \frac{v}{k}\right\rceil,
\]
as claimed.
\end{proof}

\subsubsection{5.2. Strengthened modular-to-integer transfer}
\begin{lemma}[Injective reduction + modular Sidon $\Rightarrow$ integer Sidon]\label{lem:injective_transfer}
Let $v\ge 2$ and let $B\subset \mathbb Z$ be finite.
Assume:
\begin{enumerate}
\item the reduction map $\pi:\mathbb Z\to \mathbb Z/v\mathbb Z$ is \emph{injective} on $B$, and
\item $\pi(B)$ is a Sidon set in $\mathbb Z/v\mathbb Z$.
\end{enumerate}
Then $B$ is a Sidon set in $\mathbb Z$.
\end{lemma}

\begin{proof}
Suppose $x+y=x'+y'$ with $x,y,x',y'\in B$.
Reducing modulo $v$ gives $\pi(x)+\pi(y)=\pi(x')+\pi(y')$ in $\mathbb Z/v\mathbb Z$.
Since $\pi(B)$ is Sidon, this implies $\{\pi(x),\pi(y)\}=\{\pi(x'),\pi(y')\}$ as multisets.

Because $\pi$ is injective on $B$, equality of residues forces equality of integers:
either $x=x'$ and $y=y'$, or $x=y'$ and $y=x'$.
Thus $\{x,y\}=\{x',y'\}$ as multisets, i.e.\ $B$ is Sidon in $\mathbb Z$.
\end{proof}

\begin{remark}
A sufficient condition for injectivity of $\pi$ on $B$ is $\mathrm{diam}(B)<v$.
Indeed, if $x\neq y\in B$ then $0<|x-y|\le \mathrm{diam}(B)<v$, so $x\not\equiv y\pmod v$.
\end{remark}

\subsubsection{5.3. Applying Singer + gap lemma: $h(q^2+1)\ge q+1$}
\begin{theorem}[Singer family fits in length $q^2$]\label{thm:singer_q2plus1}
Let $q$ be a prime power. Then there exists an integer Sidon set
\[
S\subseteq \{1,2,\dots,q^2+1\}
\]
with $|S|=q+1$.
Consequently,
\[
h(q^2+1)\ge q+1.
\]
\end{theorem}

\begin{proof}
Let $v:=q^2+q+1$.
By Singer, there exists a perfect difference set $D\subset \mathbb Z/v\mathbb Z$ of size $k:=|D|=q+1$.
In particular, $D$ is Sidon in $\mathbb Z/v\mathbb Z$.

Apply Lemma~\ref{lem:gap_translate} to $D$.
Since
\[
\left\lceil \frac{v}{k}\right\rceil
=\left\lceil \frac{q^2+q+1}{q+1}\right\rceil
=\left\lceil q+\frac{1}{q+1}\right\rceil
=q+1,
\]
we obtain a translate $D+t$ and integer representatives $B\subset\mathbb Z$ with
\[
B\equiv D+t\pmod v,
\qquad
\mathrm{diam}(B)\le v-(q+1)=q^2.
\]
Thus $\mathrm{diam}(B)=q^2<v$, so reduction mod $v$ is injective on $B$.
Moreover $\pi(B)=D+t$ is Sidon in $\mathbb Z/v\mathbb Z$.
Lemma~\ref{lem:injective_transfer} implies that $B$ is Sidon in $\mathbb Z$.

Translate $B$ so that $\min B=1$:
let $S:=B-(\min B)+1$.
Then $\mathrm{diam}(S)=\mathrm{diam}(B)\le q^2$, hence $S\subseteq \{1,\dots,q^2+1\}$, and $|S|=q+1$.
\end{proof}

\subsubsection{5.4. Consequence for the second-order term}
\begin{corollary}[Improved limsup]\label{cor:limsup_ge_1}
One has
\[
\limsup_{N\to\infty}\bigl(h(N)-\sqrt N\bigr)\ \ge\ 1.
\]
\end{corollary}

\begin{proof}
Let $q$ run through prime powers and set $N:=q^2+1$.
By Theorem~\ref{thm:singer_q2plus1}, $h(N)\ge q+1$.

Also,
\[
\sqrt{q^2+1}-q=\frac{1}{q+\sqrt{q^2+1}},
\]
so
\[
q+1-\sqrt{q^2+1}
=1-\frac{1}{q+\sqrt{q^2+1}}
>1-\frac{1}{2q}.
\]
Therefore
\[
h(q^2+1)-\sqrt{q^2+1}\ \ge\ q+1-\sqrt{q^2+1}\ >\ 1-\frac{1}{2q}.
\]
Letting $q\to\infty$ gives the stated $\limsup\ge 1$.
\end{proof}

\subsection{6) Adversarial Verification}
We stress-test the new steps.

\begin{itemize}
\item \textbf{Gap lemma off-by-one.}
The gaps $g_i$ sum to $v$, so $\max g_i\ge \lceil v/k\rceil$ is correct.
Cutting at that gap leaves an arc of length $v-g_j$ containing all points, giving $\mathrm{diam}(B)\le v-g_j$.
\item \textbf{Injectivity.}
We only need injectivity of reduction mod $v$ on $B$, ensured by $\mathrm{diam}(B)<v$.
In the application, $\mathrm{diam}(B)\le q^2$ and $v=q^2+q+1$, so indeed $q^2<v$.
\item \textbf{Wrap-around issues.}
Lemma~\ref{lem:injective_transfer} uses only ``integer equality $\Rightarrow$ modular equality'' and then injectivity to lift back; it never uses modular equality to deduce integer equality without injectivity.
\item \textbf{Consistency with Round 4.}
This improves Round~4 by shrinking the ambient interval from length $q^2+q+1$ to $q^2+1$ while preserving the same cardinality $q+1$.
\end{itemize}

\subsection{7) Final}
\textbf{UNRESOLVED (BUT STRICTLY ADVANCED).}
Round~5 strengthens the explicit lower bounds by proving the new infinite family
\[
h(q^2+1)\ge q+1\qquad(q\ \text{a prime power}),
\]
and therefore
\[
\limsup_{N\to\infty}(h(N)-\sqrt N)\ge 1,
\]
improving the Round~4 bound $\ge \tfrac12$.

The main open question --- whether $h(N)=\sqrt N+O_\varepsilon(N^\varepsilon)$ for every $\varepsilon>0$ --- remains open.

\subsection{8) Completion Estimate}
\textbf{COMPLETION: 75\%}.

\subsection{9) References}
\begin{itemize}
\item T.\ F.\ Bloom (ed.), \emph{Erd\H{o}s Problem \#30}, \texttt{https://www.erdosproblems.com/30} (last edited 19 Oct 2025; accessed 20 Jan 2026).
\item J.\ Singer, \emph{A theorem in finite projective geometry and some applications to number theory}, Trans.\ Amer.\ Math.\ Soc.\ 43(3) (1938), 377--385 (JSTOR: \texttt{1990067}).
\item G.\ Erskine, P.\ Fratri\v{c}, J.\ \v{S}ir\'a\v{n}, \emph{Graphs derived from perfect difference sets}, arXiv:\,1903.02425 (2019) (for a modern review of Singer's construction).
\item D.\ Carter, Z.\ Hunter, K.\ O'Bryant, \emph{On the diameter of finite Sidon sets}, Acta Math.\ Hungar.\ 175 (2025), 108--126; also arXiv:\,2310.20032.
\end{itemize}
