
FORMAL RESTATEMENT

Let $P$ be a set of $n$ points in $\mathbb{R}^2$ such that
(a) no three points of $P$ are collinear, and
(b) no four points of $P$ are concyclic.
For each 3-element subset $\{A,B,C\}\subset P$, let $R(A,B,C)>0$ be the circumradius of the unique circle passing through $A,B,C$.
Let
\[
\mathcal{R}(P)=\{R(A,B,C): \{A,B,C\}\subset P\}
\]
be the set of distinct radii realized by triangles from $P$.
Define
\[
 h(n)=\min_{P\subset\mathbb{R}^2:\ |P|=n,\ P\ \text{satisfies (a),(b)}} |\mathcal{R}(P)|.
\]
The problem asks for estimates/asymptotics of $h(n)$.

QUICK LITERATURE/CONTEXT CHECK

No quantitative bounds are stated in the problem text beyond the definition.
We therefore treat this as a purely geometric extremal question about the number of distinct circumradii determined by $n$ points in general position.

ATTACK PLAN

Proof track:
(1) Prove general lower bounds on $|\mathcal{R}(P)|$ that hold for every admissible $P$.
(2) Seek constructions with few distinct radii to upper bound $h(n)$.

Disproof track:
Not applicable (this is an estimation problem, not a yes/no statement).

WORK

Lemma 1 (pair-based lower bound).
For every admissible point set $P$ of size $n\ge 3$, one has
\[
|\mathcal{R}(P)|\ge \left\lceil\frac{n-2}{2}\right\rceil.
\]
In particular, $h(n)\ge \lceil (n-2)/2\rceil$.

Proof.
Fix two distinct points $A,B\in P$.
For each $C\in P\setminus\{A,B\}$, let $R_C:=R(A,B,C)$ be the circumradius of triangle $ABC$.
We show that any value $r>0$ can occur among the multiset $\{R_C: C\ne A,B\}$ at most twice.

Fix $r>0$.
A circle of radius $r$ passing through $A$ and $B$ must have its center at distance $r$ from both $A$ and $B$.
Thus its center lies in the intersection of the two circles of radius $r$ centered at $A$ and at $B$.
Two circles in the plane intersect in at most two points, hence there are at most two distinct circles of radius $r$ that pass through both $A$ and $B$.
Geometrically, if there are two, their centers lie on opposite sides of the line $AB$.

Now suppose $C_1,C_2\in P\setminus\{A,B\}$ satisfy $R_{C_1}=R_{C_2}=r$.
Then $A,B,C_1$ lie on some circle of radius $r$ through $A,B$, and likewise $A,B,C_2$ lie on some circle of radius $r$ through $A,B$.
If $C_1$ and $C_2$ lie on the same side of line $AB$, then both must lie on the same one of the at most two circles of radius $r$ through $A,B$, so $A,B,C_1,C_2$ are concyclic.
This contradicts the assumption (b) that no four points are concyclic.
Therefore, for a fixed $r$, there is at most one $C$ on each side of $AB$ with $R_C=r$, so at most two points $C$ total.

Since there are $n-2$ points $C$ and each radius value appears at most twice, the number of distinct values among the $R_C$ is at least $\lceil (n-2)/2\rceil$.
Because these radii are a subset of $\mathcal{R}(P)$, we obtain $|\mathcal{R}(P)|\ge \lceil (n-2)/2\rceil$.
$\square$

Lemma 2 (radius-multiplicity bound).
Fix $r>0$.
For any admissible point set $P$ of size $n$, the number of triangles $\{A,B,C\}\subset P$ with circumradius exactly $r$ is at most $\frac{n(n-1)}{3}$.
Consequently $|\mathcal{R}(P)|\ge \lceil (n-2)/2\rceil$.

Proof.
Let $T_r$ be the set of unordered triangles $\{A,B,C\}\subset P$ whose circumradius equals $r$.
Fix an unordered pair of points $\{A,B\}\subset P$.
As in Lemma~1, there are at most two circles of radius $r$ passing through $A$ and $B$.
Each such circle can contain at most one additional point $C\in P\setminus\{A,B\}$, because if it contained two distinct points $C_1,C_2$ then $A,B,C_1,C_2$ would be four concyclic points, forbidden by (b).
Therefore, the pair $\{A,B\}$ is contained in at most $2$ triangles from $T_r$.

Double-count incidences between pairs and triangles:
Each triangle in $T_r$ contains exactly $3$ unordered pairs.
Hence
\[
3|T_r| = \sum_{\{A,B\}\subset P} \#\{\text{triangles in }T_r\text{ containing }\{A,B\}\} \le 2\binom{n}{2}.
\]
Thus $|T_r|\le \frac{2}{3}\binom{n}{2}=\frac{n(n-1)}{3}$.

If $m=|\mathcal{R}(P)|$ is the number of distinct radii, then the total number of triangles satisfies
\[
\binom{n}{3} \le \sum_{\rho\in\mathcal{R}(P)} |T_{\rho}| \le m\cdot \frac{n(n-1)}{3}.
\]
Rearranging gives $m\ge \frac{3\binom{n}{3}}{n(n-1)}=\frac{n-2}{2}$, hence $m\ge \lceil (n-2)/2\rceil$.
$\square$

Fast reality check (computation; exact arithmetic on integer points).
For random integer point sets satisfying (a),(b), we computed the number of distinct circumradii among all $\binom{n}{3}$ triangles using the exact rational formula for $R^2$.
One sample output was:
\[
\begin{array}{r|r|r}
 n & \binom{n}{3} & |\mathcal{R}(P)| \\\hline
 3 & 1 & 1\\
 4 & 4 & 4\\
 5 & 10 & 10\\
 6 & 20 & 20\\
 7 & 35 & 32\\
 8 & 56 & 48
\end{array}
\]
In each case the lower bound $\lceil (n-2)/2\rceil$ was satisfied.
These computations do not estimate $h(n)$ (which is a minimum over configurations) but serve as a sanity check for Lemmas~1--2.

VERIFICATION

(1) Both lemmas rely crucially on the geometric fact: for fixed $A,B$ and fixed radius $r$, there are at most two circles of radius $r$ through $A,B$. This follows because possible centers are intersections of two radius-$r$ circles.
(2) The use of the “no four concyclic” hypothesis is explicit: it ensures a fixed circle contains at most three points, hence at most one triangle for each fixed pair and fixed circle.
(3) Quantifiers: Lemma~1 gives a bound for every admissible $P$, hence it bounds $h(n)$ from below.

FINAL

**UNRESOLVED**

(i) Strongest proved partial result.
We proved the linear lower bound
\[h(n)\ge \left\lceil\frac{n-2}{2}\right\rceil\]
for all $n\ge 3$ (Lemmas~1--2).

(ii) First gap (crisp statement).
Find a matching upper bound up to constants: does there exist a construction of admissible $P$ with $|\mathcal{R}(P)|\le Cn$ for an absolute constant $C$? Equivalently, is $h(n)=\Theta(n)$?

(iii) Top 3 next moves.
1. Construction attempt: explicitly build point sets where many triangles share circumradius values, while maintaining “no four concyclic”; try to achieve $|\mathcal{R}(P)|=O(n)$.
2. Improve the lower bound: prove $h(n)\ge c n$ with $c>1/2$ (or superlinear) using incidence geometry or additive combinatorics on the circumradius formula $R=\frac{abc}{4\Delta}$.
3. Computational search for small $n$: restrict to small integer grids, enumerate admissible point sets, and compute the exact minimum of $|\mathcal{R}(P)|$ for $n\le 8$ to guess the true growth.

(iv) Minimal counterexample structure.
A near-extremal configuration (achieving $|\mathcal{R}(P)|\approx (n-2)/2$) would require that for some fixed pair $A,B$, almost every radius value among triangles $ABC$ is used exactly twice (once on each side of $AB$), and that triangles not involving $A,B$ introduce very few new radii; this suggests a highly constrained, symmetric arrangement.

