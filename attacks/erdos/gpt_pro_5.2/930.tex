% Erdos Problem #930

1) FORMAL RESTATEMENT

Ambiguity check: the statement says ``intervals of consecutive integers''. If $0\in I_i$ for some $i$ then the product is $0$, which is a perfect power ($0=0^t$ for any $t\ge 2$). So the literal statement is false unless one restricts to (strictly) positive integers.

\medskip
\noindent\textbf{Corrected (standard) formulation.}
Fix an integer $r\ge 1$. An \emph{interval of consecutive positive integers} means a set
\[
I=[a,b]\cap\mathbb Z=\{a,a+1,\dots,b\}\subset\mathbb Z_{\ge 1}
\]
with $1\le a\le b$, and its \emph{length} is $|I|=b-a+1$.
Call a positive integer $P$ a \emph{perfect power} if $P=x^t$ for some integers $x\ge 2$ and $t\ge 2$.

The conjecture is:
\[
\forall r\ge 1\ \exists k\ge 1\ \forall\text{ pairwise disjoint intervals }I_1,\dots,I_r\subset\mathbb Z_{\ge1}
\ \Bigl(\min_i |I_i|\ge k\ \Rightarrow\ \prod_{i=1}^r\prod_{m\in I_i} m\ \text{is not a perfect power}\Bigr).
\]

2) QUICK LITERATURE/CONTEXT CHECK

From the provided problem statement: Erd\H{o}s--Selfridge proved the case $r=1$ (the product of consecutive positive integers is never a perfect power). For $r=2$ there are explicit constructions (referenced as [363] in the source) showing that some largeness condition on the intervals in terms of $r$ is necessary. No further results are stated in the source file.

3) ATTACK PLAN

\begin{itemize}
\item \textbf{Disproof track (literal statement):} exploit the $0$ issue to give a counterexample if intervals are allowed to include $0$.
\item \textbf{Proof track (corrected statement):} try to force a prime to appear to exponent $1$ in the total product. A standard sufficient condition is: if the union contains a prime $q$ exceeding half the maximum element, then $q$ appears exactly once.
\item \textbf{Computational track:} for small $r$ (especially $r=2$), brute force small intervals to see what minimum lengths still allow a perfect power.
\end{itemize}

4) WORK

\textbf{Counterexample 930.0 (literal statement if $0$ is allowed).}
Let $r\ge 1$ be arbitrary. Let $k\ge 1$ be any proposed bound. Take
\[
I_1=[0,k-1]\cap\mathbb Z,\qquad I_2=[k,k+(k-1)]\cap\mathbb Z,\quad\dots,\quad I_r=[(r-1)k,rk-1]\cap\mathbb Z.
\]
Then the $I_i$ are pairwise disjoint consecutive-integer intervals, each of length $k$, and $0\in I_1$. Hence
\[
\prod_{i=1}^r\prod_{m\in I_i} m = 0,
\]
which is a perfect power (e.g. $0=0^2$). Therefore \emph{no} $k$ can work in the literal ``integers'' formulation.
\hfill$\square$

\medskip
\textbf{Lemma 930.1 (a large prime forces non-power).}
Let $S\subset\mathbb Z_{\ge 1}$ be a finite nonempty set of positive integers, and let $M:=\max S$. If $S$ contains a prime $q$ with $M/2<q\le M$, then the product $\prod_{m\in S} m$ is not a perfect power.

\emph{Proof.}
Because $q$ is prime and $q\in S$, the exponent of $q$ in the prime factorization of $\prod_{m\in S} m$ is at least $1$.
If $m\in S$ and $q\mid m$, then $m\ge q$. Since $m\le M$ and $2q>M$, the only multiple of $q$ in $\{1,2,\dots,M\}$ is $q$ itself. Hence $q\mid m$ implies $m=q$.
Therefore $q$ divides exactly one element of $S$, namely $q$, and contributes exponent exactly $1$ to the total product.
A perfect power $x^t$ with $t\ge 2$ has every prime exponent divisible by $t$, in particular no prime can appear to exponent $1$. Hence $\prod_{m\in S} m$ is not a perfect power.
\hfill$\square$

\medskip
\textbf{Proposition 930.2 (small $r=2$ constructions exist; $k(2)\ge 4$ is forced).}
There exist disjoint intervals $I_1,I_2$ of consecutive positive integers with $|I_1|\ge 3$ and $|I_2|\ge 3$ such that
\(\prod_{m\in I_1}m\)(\prod_{m\in I_2}m)\) is a perfect power.
In particular, for the corrected statement with $r=2$, no choice $k\le 3$ can work.

\emph{Proof (explicit verification).}
Take
\[
I_1=\{1,2,3,4,5,6\}=[1,6],\qquad I_2=\{8,9,10\}=[8,10].
\]
These are disjoint intervals, with $|I_1|=6\ge 3$ and $|I_2|=3\ge 3$. Their product equals
\[
(1\cdot2\cdot3\cdot4\cdot5\cdot6)\,(8\cdot9\cdot10)=720\cdot720=720^2,
\]
which is a perfect square.
Thus any valid $k(2)$ must satisfy $k(2)\ge 4$.
\hfill$\square$

\medskip
\textbf{FAST REALITY CHECK (local brute force, $r=2$).}
I brute-forced pairs of disjoint intervals contained in $\{1,2,\dots,60\}$ with each interval length $\le 8$ and checked whether the product of the union is a perfect power.
\begin{itemize}
\item For $k=2$ and $k=3$ (minimum interval length), many examples occur; one example for $k=3$ is the square above.
\item For $k=4$ (minimum length $4$) and $k=5$, I found \emph{no} examples in this search window.
\end{itemize}
This is only a sanity check on small ranges and does not prove anything asymptotic.

5) VERIFICATION

\begin{itemize}
\item Counterexample 930.0: all intervals have length $k$ and are disjoint; $0$ is included, so the total product is exactly $0$, which is a perfect power. Checked.
\item Lemma 930.1: the only subtle point is ``$q$ divides no other element of $S$''. This follows from $2q>M$ and $m\le M$ for all $m\in S$.
\item Proposition 930.2: the numerical identity is exact: $1\cdot2\cdot3\cdot4\cdot5\cdot6=720$ and $8\cdot9\cdot10=720$.
\end{itemize}

6) FINAL

\textbf{UNRESOLVED}

(i) \textbf{Strongest proved partial result here.}
For the corrected positive-integer formulation, if the union of the intervals contains a prime $q>\tfrac12\max(\cup I_i)$, then the total product is not a perfect power (Lemma 930.1). Also, for $r=2$ one must have $k(2)\ge 4$ (Proposition 930.2).

(ii) \textbf{First gap (crisp).}
Show that for each fixed $r$ there exists $k=k(r)$ such that every union of $r$ disjoint intervals of length at least $k$ necessarily contains a prime $q$ whose exponent in the total product is not divisible by any $t\ge 2$ (in particular, a prime $q>\tfrac12\max$ would suffice by Lemma 930.1).

(iii) \textbf{Top 3 next moves.}
\begin{enumerate}
\item Try to force a prime $q>\tfrac12\max$ inside (or as a factor of) the topmost interval using quantitative bounds on prime gaps (would require an explicit theorem strong enough to guarantee a prime in short intervals).
\item For $r=2$, do a targeted computation to guess the true smallest $k(2)$ (search larger windows and allow longer intervals), and then look for a structural description of all perfect-power examples.
\item Attempt to adapt the Erd\H{o}s--Selfridge argument for a single interval to unions of intervals by tracking $p$-adic valuations prime-by-prime and showing an unavoidable ``parity defect'' for some prime.
\end{enumerate}

(iv) \textbf{What a minimal counterexample would look like.}
A minimal counterexample (for fixed $r$) would be $r$ disjoint long intervals whose union avoids every prime in $(\tfrac12M,M]$ (where $M$ is the maximum element), and for which every prime exponent in the product is simultaneously divisible by some $t\ge 2$. In particular, every number in the union near $M$ would have to share its prime factors extensively with other numbers in the union so that no prime occurs to exponent $1$.


