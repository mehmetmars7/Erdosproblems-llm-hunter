% Erdos Problem #172
% Attempt for Erdos Problem #172
% Following PROMPT_STRATEGY.MD
% Tools/Constraints:
% - Web browsing available? YES (not used beyond the problem text; no extra literature claims).
% - Computation available (Python/Sage/Mathematica)? YES (not essential here).

PROBLEM

Is it true that in any finite colouring of $\mathbb{N}$ there exist arbitrarily large finite $A$ such that all sums and products of distinct elements in $A$ are the same colour?

OUTPUT

## 1) FORMAL RESTATEMENT

Let $r\ge 1$ and let $c:\mathbb N\to\{1,2,\dots,r\}$ be a coloring.

For a finite set $A\subseteq\mathbb N$, define its **finite sums set**
\[
\mathrm{FS}(A):=\Bigl\{\sum_{x\in S} x: \emptyset\ne S\subseteq A\Bigr\}
\]
and its **finite products set**
\[
\mathrm{FP}(A):=\Bigl\{\prod_{x\in S} x: \emptyset\ne S\subseteq A\Bigr\}.
\]
(“Distinct elements” means we sum/product over subsets, no repetition.)

The question is:

> For every $r$ and every coloring $c$, do there exist finite sets $A$ of arbitrarily large size such that $c$ is constant on $\mathrm{FS}(A)\cup \mathrm{FP}(A)$?

Equivalently: for every $m\in\mathbb N$ does there exist $A\subseteq\mathbb N$ with $|A|=m$ and a color $\gamma$ such that every nonempty subset-sum and subset-product of $A$ has color $\gamma$?

---

## 2) QUICK LITERATURE/CONTEXT CHECK

I will not use or claim any external results beyond what is explicitly written in the problem text.

The problem text reports:

* Hindman proved the analogous statement for an **infinite** $A$ is false with 7 colors.
* Moreira proved that in any finite coloring of $\mathbb N$ there exist $x,y$ with $\{x,x+y,xy\}$ monochromatic.
* Alweiss proved the finite-$A$ statement over $\mathbb Q\setminus\{0\}$.

I do not re-prove these results.

---

## 3) ATTACK PLAN

I do not currently see a purely elementary route to the full statement in $\mathbb N$ because it requires simultaneous additive and multiplicative structure.

Partial work:

1. Record basic necessary/structural observations about what it means for $\mathrm{FS}(A)\cup\mathrm{FP}(A)$ to be monochromatic.
2. Give concrete sanity-check examples for specific colorings (e.g. parity coloring) where the statement is easy.

---

## 4) WORK

### FAST REALITY CHECK / sanity examples

* If $r=1$ (one color), the statement is trivial: any finite $A$ works.
* For the 2-coloring by parity, $c(n)=n\bmod 2$, choosing $A$ to be any finite set of even integers makes all subset sums and products even (hence monochromatic).

These do not address arbitrary colorings but confirm the statement is consistent with some natural colorings.

---

### Lemma 4.1 (a sufficient condition via a monochromatic ideal)

Fix a coloring $c:\mathbb N\to\{1,\dots,r\}$.
If there exist a color $\gamma$ and an integer $m\ge 2$ such that
\[
\forall t\in\mathbb N:\quad c(mt)=\gamma,
\]
then for every $M\in\mathbb N$ there exists a set $A$ with $|A|=M$ such that $c$ is constant equal to $\gamma$ on $\mathrm{FS}(A)\cup\mathrm{FP}(A)$.

**Proof.**
Assume all multiples of $m$ have color $\gamma$.
Let $A\subseteq m\mathbb N$ be any finite set of size $M$ (e.g. $A=\{m,2m,\dots,Mm\}$).

1. Every element of $\mathrm{FS}(A)$ is a sum of multiples of $m$, hence itself a multiple of $m$.
2. Every element of $\mathrm{FP}(A)$ is a product of multiples of $m$, hence itself a multiple of $m$.

Therefore $\mathrm{FS}(A)\cup\mathrm{FP}(A)\subseteq m\mathbb N$, so $c$ takes the constant value $\gamma$ on this union.
$\square$

---

### Lemma 4.2 (parity coloring works with even sets)

Let $c(n)=n\bmod 2$ be the 2-coloring of $\mathbb N$ into evens/odds.
For every $M\in\mathbb N$, the set $A:=\{2,4,6,\dots,2M\}$ satisfies that $\mathrm{FS}(A)\cup\mathrm{FP}(A)$ is monochromatic (all even).

**Proof.**
Every element of $A$ is divisible by $2$.

* Any finite sum of elements of $A$ is divisible by $2$.
* Any finite product of elements of $A$ is divisible by $2$.

Hence every element of $\mathrm{FS}(A)\cup\mathrm{FP}(A)$ is even, so has the same color under $c$.
$\square$

---

## 5) VERIFICATION

* Lemma 4.1 is logically correct but requires a very strong hypothesis (a whole multiplicative ideal being monochromatic), which need not hold in an arbitrary coloring.
* Lemma 4.2 correctly verifies a basic sanity example.
* I did not find (and did not computationally search for) an explicit finite coloring that prevents large $A$; the obstruction seems genuinely global.

---

## 6) FINAL

**UNRESOLVED**

(i) **Strongest fully proved partial result obtained here.**

I proved that the conjecture holds for any coloring in which some full set of multiples $m\mathbb N$ is monochromatic (Lemma 4.1), and verified it for the parity coloring (Lemma 4.2).

(ii) **Exact first gap.**

Remove the very strong hypothesis in Lemma 4.1: prove existence of large $A$ for an *arbitrary* finite coloring of $\mathbb N$.

(iii) **Top 3 next moves (concrete targets).**

1. Try to prove the statement first for $|A|=2$ (i.e. find $a,b$ with $\{a,b,a+b,ab\}$ monochromatic) or $|A|=3$ in any finite coloring.
2. Attempt a reduction to structured colorings using density/regularity or algebraic decomposition (e.g. working inside multiplicative subsemigroups).
3. Explore whether the $\mathbb Q\setminus\{0\}$ result can be transferred to $\mathbb N$ via denominators/clearing, or identify a clear obstruction.

(iv) **What a minimal counterexample would likely look like.**

A minimal counterexample coloring would likely avoid large monochromatic “additively closed” and “multiplicatively closed” patterns simultaneously, perhaps by encoding $p$-adic valuations or logarithmic partitions so that sums and products systematically change color.


