% Erdos Problem #120
% Attempt for Erdos Problem #120
% Following PROMPT_STRATEGY.MD
% Tools/Constraints:
% - Web browsing available? YES (not used; only facts explicitly stated in the problem text)
% - Computation available (Python)? YES (not used)

\section*{Erd\H{o}s Problem \#120}

\subsection*{1) FORMAL RESTATEMENT}
Let $A\subseteq\mathbb R$ be an infinite set.  Does there exist a measurable set $E\subseteq\mathbb R$ of positive Lebesgue measure such that $E$ contains no set of the form
\[
aA+b:=\{aa'+b:\ a'\in A\}
\]
for any real numbers $a\neq0$ and $b$?

Equivalently: must every infinite $A$ admit a positive-measure set $E$ that avoids every \emph{similar copy} of $A$ (scaling and translation)?

\subsection*{2) QUICK LITERATURE/CONTEXT CHECK}
From the problem text:
\begin{itemize}
\item The conjecture is true if $A$ is unbounded or dense in some interval.
\item It suffices to treat $A$ a decreasing sequence tending to $0$.
\item It is false for finite $A$ (Steinhaus).
\item The case $A=\{1,1/2,1/4,\dots\}$ is open.
\end{itemize}
No further sources are used here.

\subsection*{3) ATTACK PLAN}
Prove the easy cases (unbounded $A$; $A$ dense in an interval) in a fully self-contained way, as sanity checks.  The remaining difficult case is when $A$ is countable and accumulates only at $0$.

\subsection*{4) WORK}
\paragraph{Lemma 120.1 (unbounded $A$ case is trivial).}
If $A$ is unbounded, then there exists a positive-measure set $E$ containing no copy $aA+b$.

\emph{Proof.}
Take any bounded measurable set $E$ of positive measure, e.g. $E=[0,1]$.  For any $a\neq0$ and $b$, if $A$ is unbounded then $aA+b$ is also unbounded (since $|aa'+b|\to\infty$ along some sequence $a'\in A$).  Hence $aA+b$ cannot be a subset of the bounded set $E$.  Therefore $E$ avoids every $aA+b$. \qed

\paragraph{Lemma 120.2 (dense-in-interval case via a fat Cantor set).}
Suppose $A$ is dense in some interval $I\subset\mathbb R$ with nonempty interior.  Then there exists a measurable set $E$ of positive measure that contains no copy $aA+b$.

\emph{Proof.}
Fix such an interval $I$.  For any $a\neq0$ and $b$, the set $aA+b$ is dense in the interval $aI+b$ because the map $x\mapsto ax+b$ is a homeomorphism and carries a dense set to a dense set.

Therefore, if a set $E$ contains $aA+b$, then $\overline{E}$ (the closure) contains the interval $aI+b$, hence $\overline{E}$ has nonempty interior.

Choose $E$ to be a \emph{fat Cantor set}: a closed nowhere dense subset of $[0,1]$ with positive Lebesgue measure.  Such a set can be constructed by iteratively removing open middle intervals of total removed length $<1/2$, leaving a closed set of measure $>1/2$ but with empty interior.

Since $E$ is nowhere dense, its closure is itself and has empty interior, so it cannot contain any interval.  In particular, $E$ cannot contain any dense subset of a nontrivial interval.  Therefore $E$ cannot contain any set $aA+b$ (each such set is dense in $aI+b$).
Thus this $E$ has positive measure and avoids all similar copies of $A$. \qed

\subsection*{5) VERIFICATION (FAST REALITY CHECK)}
\begin{itemize}
\item Lemma~120.1: if $A$ unbounded and $a\neq0$, then $aA+b$ unbounded; bounded $E$ excludes it.
\item Lemma~120.2: the key implication is ``$aA+b$ dense in an interval $\Rightarrow$ any containing set has closure with interior''. This is immediate from density.
\item Existence of fat Cantor sets is standard; the described removal process gives a closed set of positive measure with empty interior.
\end{itemize}

\subsection*{6) FINAL}
\textbf{UNRESOLVED.}

(i) \emph{Strongest fully proved partial result obtained here.}
A complete proof of the conjecture in the two easy regimes explicitly mentioned in the problem text: $A$ unbounded (Lemma~120.1) and $A$ dense in an interval (Lemma~120.2).

(ii) \emph{Exact first gap.}
Handle the ``hard'' remaining case: $A$ countable, strictly monotone, and converging to $0$ (e.g. geometric sequences like $\{1,1/2,1/4,\dots\}$).

(iii) \emph{Top 3 next moves.}
\begin{enumerate}
\item For $A$ decreasing to $0$, try to construct $E$ by a scale-by-scale Cantor-type removal that blocks copies at each scale $a$.
\item Identify quantitative density properties of $A$ (e.g. multiplicative structure) that force copies $aA+b$ to intersect most positive-measure sets.
\item Explore Fourier-analytic criteria for a set $E$ to avoid dilates/translates of a given sequence.
\end{enumerate}

(iv) \emph{Minimal counterexample structure.}
A counterexample to the conjecture would be an infinite set $A$ such that \emph{every} positive-measure set $E$ contains some similar copy $aA+b$.  By the reductions in the problem text, a minimal such $A$ would likely be a decreasing sequence tending to $0$ with strong self-similarity (e.g. geometric progressions).


