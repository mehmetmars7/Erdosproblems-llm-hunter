
PROBLEM

Let $N(k,\ell)$ be the minimal $N$ such that for any $f:\{1,\ldots,N\}\to\{-1,1\}$ there must exist a $k$-term arithmetic progression $P$ such that
\[ \left\lvert \sum_{n\in P}f(n)\right\rvert\geq \ell.\]
Find good upper bounds for $N(k,\ell)$. Is it true that for any $c>0$ there exists some $C>1$ such that $N(k,ck)\leq C^k$?

OUTPUT

## 1) FORMAL RESTATEMENT

Fix integers $k\ge 3$ and $\ell\ge 1$.

For $N\ge 1$ and a function $f:[N]:=\{1,2,\dots,N\}\to\{-1,1\}$, and a $k$-term arithmetic progression
\[
P(a,d):=\{a,a+d,\dots,a+(k-1)d\}\subseteq [N],
\]
define its signed sum
\[
S_f(P):=\sum_{n\in P} f(n).
\]

Define $N(k,\ell)$ to be the smallest $N$ such that for **every** $f:[N]\to\{-1,1\}$ there exists a $k$-term AP $P\subseteq [N]$ with
\[
|S_f(P)|\ge \ell.
\]
If no such $N$ exists, set $N(k,\ell)=\infty$ (but for $\ell\le k$ it is finite by van der Waerden-type reasoning).

The question asks for good upper bounds on $N(k,\ell)$, and specifically whether for each $c>0$ there exists $C>1$ with $N(k,ck)\le C^k$.

---

## 2) QUICK LITERATURE/CONTEXT CHECK

I will not use or claim any external results beyond what is explicitly written in the problem text.

The problem text notes:

* When $\ell=k$, $N(k,k)$ equals the van der Waerden number $W(k)$.
* Spencer computed $N(k,1)$ exactly for $k=2^t m$ with $m$ odd.
* Erd\H{o}s proved an exponential lower bound for $N(k,ck)$.

I do not re-prove Spencer’s or Erd\H{o}s’s results.

---

## 3) ATTACK PLAN

1. Prove elementary base-case facts (e.g. $N(k,1)=k$ for odd $k$).
2. Relate $N(k,\ell)$ to van der Waerden numbers to get a trivial upper bound $N(k,\ell)\le W(k)$ for $\ell\le k$.
3. Compute small values of $N(k,\ell)$ for small $(k,\ell)$ by brute force enumeration of all signings.

---

## 4) WORK

### Phase 1: FAST REALITY CHECK (small exact values)

I computed $N(k,\ell)$ for a few small cases by enumerating all $2^N$ functions $f:[N]\to\{-1,1\}$.
The results:

* $N(3,1)=3$.
* $N(3,2)=9$.
* $N(4,1)=13$.
* $N(4,2)=13$ (for even $k$, the sum over a $k$-AP is even, so the conditions $|S|\ge 1$ and $|S|\ge 2$ coincide).

---

### Lemma 4.1 (odd $k$ forces $N(k,1)=k$)

If $k$ is odd, then
\[
N(k,1)=k.
\]

**Proof.**
First, if $N<k$ then there is no $k$-term arithmetic progression in $[N]$, so $N(k,1)\ge k$.

Now take $N=k$. The set $[k]$ itself is a $k$-term arithmetic progression with $a=1$, $d=1$.
For any $f:[k]\to\{-1,1\}$,
\[
S_f([k])=\sum_{n=1}^k f(n).
\]
This is a sum of $k$ odd integers ($\pm 1$), hence is an odd integer. Therefore $S_f([k])\ne 0$, so $|S_f([k])|\ge 1$.
Thus every $f$ has a $k$-term AP with discrepancy at least 1 already in $[k]$, showing $N(k,1)\le k$.
Combining gives equality.
$\square$

---

### Lemma 4.2 (trivial van der Waerden upper bound)

For every $k\ge 3$ and every $1\le \ell\le k$,
\[
N(k,\ell)\le W(k),
\]
where $W(k)$ is the 2-color van der Waerden number.

**Proof.**
Let $N=W(k)$. Given any $f:[N]\to\{-1,1\}$, view it as a 2-coloring of $[N]$ by colors “$+1$” and “$-1$”.
By definition of $W(k)$, there exists a monochromatic $k$-term arithmetic progression $P\subseteq[N]$.
On such a $P$, either $f\equiv +1$ or $f\equiv -1$, so
\[
S_f(P)=\pm k.
\]
Thus $|S_f(P)|=k\ge \ell$. This shows that for $N=W(k)$ every $f$ has a $k$-AP with discrepancy at least $\ell$, i.e. $N(k,\ell)\le W(k)$.
$\square$

---

### Lemma 4.3 (monotonicity)

If $k'\ge k$ and $\ell'\ge \ell$ then
\[
N(k',\ell')\ge N(k,\ell).
\]

**Proof.**
The condition “there exists a $k'$-AP with $|S_f|\ge \ell'$” is stronger than “there exists a $k$-AP with $|S_f|\ge \ell$” because it asks for longer progressions and a larger discrepancy threshold. Therefore achieving the stronger guarantee for all $f$ cannot require a smaller ambient interval.
Formally, if $N$ works for $(k',\ell')$, then in particular it works for $(k,\ell)$ by restricting attention to $k$-APs and using $\ell\le \ell'$. Hence the minimal such $N$ satisfies the stated inequality.
$\square$

---

## 5) VERIFICATION

* Lemma 4.1: the key is that a sum of an odd number of $\pm 1$ values is odd and hence nonzero.
* Lemma 4.2: correct use of the definition of $W(k)$.
* Computed small values are consistent with Lemma 4.1: $N(3,1)=3$. Also $N(3,2)=9$ matches the interpretation that $\ell=2$ for $k=3$ forces a monochromatic 3-AP.

---

## 6) FINAL

**UNRESOLVED**

(i) **Strongest fully proved partial result obtained here.**

Elementary facts:

* For odd $k$, $N(k,1)=k$ (Lemma 4.1).
* For all $\ell\le k$, $N(k,\ell)\le W(k)$ (Lemma 4.2).
* Monotonicity in parameters (Lemma 4.3).

And exact small values: $N(3,1)=3$, $N(3,2)=9$, $N(4,1)=13$, $N(4,2)=13$.

(ii) **Exact first gap.**

Prove nontrivial upper bounds (e.g. $N(k,2)\le C^k$) beyond the van der Waerden bound, which is far too large.

(iii) **Top 3 next moves (concrete targets).**

1. For fixed small $\ell$ (e.g. $\ell=2$), attempt to prove an exponential upper bound via probabilistic/entropy methods on the family of $\pm 1$ colorings.
2. For $\ell=ck$, seek a “density increment” or Fourier-analytic approach: large discrepancy means biased color distribution along an AP.
3. Compute $N(k,2)$ for moderate $k$ by SAT/ILP to guess growth and candidate extremal colorings.

(iv) **What a minimal counterexample would likely look like.**

A counterexample to $N(k,2)\le C^k$ would be a sequence of signings $f_k$ on $[N_k]$ with $N_k$ super-exponential in $k$ such that every $k$-AP has sum in $\{-0,\pm 2\}$ (i.e. extremely balanced). Such a sequence would likely require highly structured, pseudorandom-like sign patterns.


