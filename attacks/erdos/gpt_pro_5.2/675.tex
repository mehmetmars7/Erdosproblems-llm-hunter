% Erdos Problem #675

1) FORMAL RESTATEMENT

A set $A\subset\mathbb N$ (here $\mathbb N=\{1,2,3,\dots\}$) has the \emph{translation property} if for every integer $n\ge 1$ there exists an integer $t_n\ge 1$ such that for all integers $a$ with $1\le a\le n$,
\[
 a\in A\ \Longleftrightarrow\ a+t_n\in A.
\]
Equivalently, the membership pattern of $A$ on $\{1,2,\dots,n\}$ repeats on the translated block $\{1+t_n,2+t_n,\dots,n+t_n\}$.

Questions posed:

(Q1) Does the set $A_{2\square}:=\{m\in\mathbb N: m=x^2+y^2\text{ for some integers }x,y\}$ have the translation property?

(Q2) If the primes are partitioned into $P\sqcup Q$ with each part containing $\gg x/\log x$ primes $\le x$ for all large $x$, does the set
\[
A_P:=\{m\in\mathbb N: \text{every prime divisor of }m\text{ lies in }P\}
\]
have the translation property?

(Q3) Let $A_{\mathrm{sqf}}$ be the set of squarefree integers. Let $t_n^*$ be the minimal positive integer $t$ witnessing the translation property for $n$, i.e.
\[
 t_n^*:=\min\{t\ge 1: \forall a\le n,\ a\in A_{\mathrm{sqf}}\iff a+t\in A_{\mathrm{sqf}}\}.
\]
How fast can $t_n^*$ grow? In particular, is it true that $t_n^*>\exp(n^c)$ for some constant $c>0$?

2) QUICK LITERATURE/CONTEXT CHECK

As stated in the problem text:
- Elementary sieve theory implies $A_{\mathrm{sqf}}$ has the translation property.
- A more general sufficient condition is claimed using Brun's sieve: if $B\subseteq\mathbb N$ is pairwise coprime with $\sum_{b<x}1/b=o(\log\log x)$, then the set of integers not divisible by any $b\in B$ has the translation property.

Note: the problem text contains a notational slip: it says $A=\{n: b\nmid n\ \text{for all }b\in A\}$; the minimal correction consistent with the preceding sentence is
\[
A=\{n: b\nmid n\ \text{for all }b\in B\}.
\]

I do not use any other external results here.

3) ATTACK PLAN

- For (Q3), compute $t_n^*$ for small $n$ to get growth data, and prove structural lemmas showing what congruence conditions automatically preserve (non-)squarefreeness.
- I do not make progress on (Q1) or (Q2) beyond restating them.

4) WORK

FAST REALITY CHECK (squarefree numbers)

I computed $t_n^*$ for $1\le n\le 50$ by brute force search over $t$ (checking the squarefree indicator on $[1,n]$ and $[1+t,n+t]$). The values are piecewise constant with jumps:
\[
\begin{array}{c|l}
 n & t_n^*\\\hline
 1,2 & 1\\
 3,4 & 4\\
 5 & 12\\
 6,7,8 & 28\\
 9,\dots,12 & 36\\
 13,\dots,24 & 180\\
 25,\dots,30 & 900\\
 31,\dots,49 & 2376\\
 50 & 44100
\end{array}
\]
In particular, $t_{50}^*=44100=(2\cdot 3\cdot 5\cdot 7)^2$.

Lemma 675.1 (equivalent "prefix recurrence" formulation)

Let $A\subset\mathbb N$ and define its indicator $\mathbf 1_A(m)\in\{0,1\}$. Then $A$ has the translation property if and only if for every $n\ge 1$ there exists $t_n\ge 1$ such that
\[
(\mathbf 1_A(1),\dots,\mathbf 1_A(n))=(\mathbf 1_A(1+t_n),\dots,\mathbf 1_A(n+t_n)).
\]

Proof.
The definition says that for all $1\le a\le n$ we have $a\in A\iff a+t_n\in A$. This is equivalent to saying $\mathbf 1_A(a)=\mathbf 1_A(a+t_n)$ for all $a\le n$, which is exactly the componentwise equality of the two length-$n$ vectors displayed.


Lemma 675.2 (a sufficient congruence condition that preserves small square divisors)

Fix $n\ge 1$ and let
\[
P_n:=\{\text{primes }p: p^2\le n\}.
\]
Let $t$ be a positive integer satisfying
\[
 t\equiv 0\pmod{p^2}\quad\text{for every }p\in P_n.
\]
Then for every integer $a$ with $1\le a\le n$:

(a) If $a$ is divisible by $p^2$ for some $p\in P_n$, then $a+t$ is divisible by the same $p^2$, hence $a+t$ is not squarefree.

(b) If $a$ is squarefree, then for every $p\in P_n$ we have $p^2\nmid (a+t)$.

Proof.
Fix $p\in P_n$. The congruence condition says $t\equiv 0\pmod{p^2}$, hence
\[
 a+t\equiv a\pmod{p^2}.
\]
Therefore $p^2\mid a$ if and only if $p^2\mid (a+t)$.

(a) If $p^2\mid a$ for some $p\in P_n$, then $p^2\mid(a+t)$, so $a+t$ has a square prime factor and is not squarefree.

(b) If $a$ is squarefree then for every prime $p$ we have $p^2\nmid a$, in particular for every $p\in P_n$. By the equivalence just proved, $p^2\nmid(a+t)$ for all $p\in P_n$.


Remarks on what this does \emph{not} control: even if $a$ is squarefree, the number $a+t$ could still be divisible by $q^2$ for some prime $q$ with $q^2>n$ (e.g. $a=13$ and $t=36$ gives $a+t=49=7^2$). Thus Lemma 675.2 alone cannot produce a full translation for $A_{\mathrm{sqf}}$.

5) VERIFICATION

- Lemma 675.1 is a direct rewriting; quantifiers match.
- Lemma 675.2: the key equivalence $p^2\mid a\iff p^2\mid(a+t)$ uses $t\equiv 0\pmod{p^2}$ and is correct for all integers $a$.
- Computation sanity: the failure example $13+36=49$ is consistent with the jump $t_{13}^*=180$.

6) FINAL

\textbf{UNRESOLVED}

(i) Strongest proved partial result:

I computed the exact minimal shifts $t_n^*$ for squarefree numbers up to $n=50$ (table above). I also proved that choosing $t$ divisible by all prime squares $p^2\le n$ guarantees that the shifted block preserves all square divisibility by such small squares (Lemma 675.2), which is part of what is needed for $A_{\mathrm{sqf}}$.

(ii) First gap (crisp):

For $A_{\mathrm{sqf}}$, after enforcing the small-square congruences of Lemma 675.2, one must still ensure that for every squarefree $a\le n$, the number $a+t$ is not divisible by any prime square $q^2>n$. I do not have a general construction or lower bound mechanism controlling these larger squares.

(iii) Top 3 next moves:

1. Use a sieve/CRT argument to choose $t$ so that simultaneously (a) $t\equiv 0\pmod{p^2}$ for $p^2\le n$ and (b) $a+t$ avoids $0\pmod{q^2}$ for all squarefree $a\le n$ and all primes $q$ with $q^2\le n+t$.
2. Extend the brute-force computation of $t_n^*$ beyond $n=50$ and factor $t_n^*$ to guess a structural formula or at least a growth model.
3. For (Q1) and (Q2), compute $t_n^*$ for these sets for small $n$ to see whether prefix repetition even appears plausible; then attempt to connect any observed constraints to the defining arithmetic structure.

(iv) Minimal counterexample structure:

To show a set $A$ fails the translation property, one would need some $n$ such that the length-$n$ prefix of $\mathbf 1_A$ never reappears at any later position. For $A_{\mathrm{sqf}}$ the computations up to $n=50$ show repeated reoccurrence with moderately sized shifts, so a minimal counterexample (if it existed) would require a first $n$ where the prefix becomes "unique" in the infinite squarefree indicator sequence.


