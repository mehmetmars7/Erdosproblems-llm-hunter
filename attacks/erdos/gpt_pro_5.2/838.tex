% Erdos Problem #838
% URL: https://www.erdosproblems.com/838

Let $f(n)$ be maximal such that any $n$ points in $\mathbb{R}^2$, with no three on a line, determine at least $f(n)$ different convex subsets. Estimate $f(n)$ - in particular, does there exist a constant $c$ such that\[\lim \frac{\log f(n)}{(\log n)^2}=c?\] A question of Erd\H{o}s and Hammer. Erd\H{o}s proved in \cite{Er78c} that there exist constants $c_1,c_2>0$ such that\[n^{c_1\log n}<f(n)< n^{c_2\log n}.\]See also [107] . References [Er78c] Erd\H{o}s, P., Some more problems on elementary geometry . Austral. Math. Soc. Gaz. (1978), 52-54.

\medskip
\noindent\textbf{1) FORMAL RESTATEMENT}

Let $P\subset\mathbb R^2$ be a finite set of $n$ points in \emph{general position} (no three collinear).

\textbf{Definition (convex subset).}
A subset $S\subseteq P$ is called \emph{convex} if every point of $S$ is a vertex of the convex hull $\mathrm{conv}(S)$.
Equivalently: no point of $S$ lies in the convex hull of the other points of $S$.
(For $|S|\le 2$ this holds automatically; for $|S|\ge 3$ it means $S$ is in convex position.)

Let
\[
C(P):=\big|\{S\subseteq P:\ S\text{ is convex and }S\neq\emptyset\}\big|
\]
be the number of nonempty convex subsets of $P$.

Define
\[
f(n):=\min\{C(P): P\subset\mathbb R^2,\ |P|=n,\ P\text{ has no three collinear}\}.
\]
This matches ``the maximal $f(n)$ such that any $n$ points determine at least $f(n)$ convex subsets''.

The question asks for asymptotics of $f(n)$, in particular whether the limit
\[
\lim_{n\to\infty}\frac{\log f(n)}{(\log n)^2}
\]
exists and equals some constant $c$.

\medskip
\noindent\textbf{2) QUICK LITERATURE/CONTEXT CHECK}

I will not use any external results beyond what is stated in the problem text.
The statement records Erd\H{o}s's bounds:
\[
n^{c_1\log n}<f(n)< n^{c_2\log n}
\]
for some constants $c_1,c_2>0$.
These bounds already imply $\log f(n)=\Theta((\log n)^2)$ up to constant factors, but do not settle existence of the limit constant.

\medskip
\noindent\textbf{3) ATTACK PLAN}

\begin{itemize}
\item Prove basic inequalities that relate $f(n)$ to (a) the size of the largest convex subset, and (b) configurations with bounded convex-subset size.
\item Do small-$n$ sanity checks by explicitly counting convex subsets in a few concrete configurations.
\item Identify the first serious gap: producing matching upper/lower constants in $\log f(n)\sim c(\log n)^2$.
\end{itemize}

\medskip
\noindent\textbf{4) WORK}

\textbf{Fast reality check (explicit counts for small $n$ in two configurations).}
For each $n=3,4,\dots,10$, I counted convex subsets in:
(A) $n$ points in convex position (regular $n$-gon), and
(B) a configuration of 3 outer triangle vertices plus $n-3$ interior points.
The exact counts obtained were:
\begin{center}
\begin{tabular}{c|cccccccc}
$n$ & 3&4&5&6&7&8&9&10\\\hline
$C(\text{convex }n\text{-gon})$ &7&15&31&63&127&255&511&1023\\
$C(\text{triangle+interior})$ &7&14&26&46&77&127&228&331
\end{tabular}
\end{center}
Here $C(\text{convex }n\text{-gon})=2^n-1$ because every nonempty subset of vertices of a convex $n$-gon is in convex position.
These computations only illustrate that $C(P)$ can vary substantially with the configuration.

\medskip
\noindent\textbf{Lemma 1 (A single large convex subset yields many convex subsets).}
If $P$ contains a convex subset $S$ of size $m$, then $C(P)\ge 2^m-1$.

\noindent\emph{Proof.}
Assume $S\subseteq P$ is convex with $|S|=m$.
Claim: every nonempty subset $T\subseteq S$ is convex.
Indeed, since $S$ is in convex position, each point of $S$ is a vertex of $\mathrm{conv}(S)$.
For any $T\subseteq S$, the convex hull $\mathrm{conv}(T)$ is a convex polygon (or segment/point) whose vertices are precisely the extreme points of $T$.
But no point of $T$ can lie in the convex hull of the others because that would also place it in $\mathrm{conv}(S\setminus\{\text{that point}\})\subseteq\mathrm{conv}(S)$, contradicting that it was a vertex of $\mathrm{conv}(S)$.
Thus $T$ is convex.
There are $2^m-1$ nonempty subsets $T\subseteq S$, all counted in $C(P)$, so $C(P)\ge 2^m-1$. \hfill$\Box$

\medskip
\noindent\textbf{Lemma 2 (Bounding $C(P)$ when large convex subsets are forbidden).}
If every convex subset of $P$ has size at most $m$, then
\[
C(P)\le \sum_{t=1}^{m} \binom{n}{t}.
\]

\noindent\emph{Proof.}
Every convex subset is, in particular, a subset of $P$.
If no convex subset has size greater than $m$, then the family of convex subsets is contained in the family of all subsets of sizes $1,2,\dots,m$.
There are $\binom{n}{t}$ subsets of size $t$, so summing over $t\le m$ gives the stated bound. \hfill$\Box$

\medskip
\noindent\textbf{Lemma 3 (A very weak unconditional lower bound).}
For every $n\ge 3$, we have $f(n)\ge \binom{n}{3}+\binom{n}{2}+n$.

\noindent\emph{Proof.}
In general position, every 1-point subset and every 2-point subset is convex.
Also, every 3-point subset is a (nondegenerate) triangle and is convex (no three collinear implies each of the three points is a vertex of the triangle).
Thus every $P$ contributes at least $n+\binom{n}{2}+\binom{n}{3}$ convex subsets, so the minimum $f(n)$ is at least this. \hfill$\Box$

\medskip
\noindent\textbf{5) VERIFICATION}

\begin{itemize}
\item Lemma 1 hinges on: ``subset of a convexly independent set is convexly independent.'' This is true because if a point is extreme in the larger set, it is extreme in every subset that contains it.
\item Lemma 2 is purely counting; it does not assume anything about geometry beyond the size cutoff.
\item Lemma 3 uses only the general-position hypothesis to guarantee every triple forms a nondegenerate triangle.
\item The small-$n$ computations were checked by direct convex-hull computation on explicit coordinate sets; degeneracies were avoided by construction.
\end{itemize}

\medskip
\noindent\textbf{6) FINAL}

\textbf{**UNRESOLVED**}

(i) \emph{Strongest proved partial result.} Basic extremal inequalities: existence of a convex $m$-subset forces $\ge 2^m-1$ convex subsets (Lemma 1), while forbidding convex subsets larger than $m$ forces $\le \sum_{t\le m}\binom{n}{t}$ convex subsets (Lemma 2). Also, trivially $f(n)\ge n+\binom{n}{2}+\binom{n}{3}$ (Lemma 3). Small-$n$ computations illustrate variability across configurations.

(ii) \emph{First gap (crisp).} Establish matching asymptotics $\log f(n) = (c+o(1))(\log n)^2$ with an explicit constant $c$, or decide whether the limit $\lim \frac{\log f(n)}{(\log n)^2}$ exists.

(iii) \emph{Top 3 next moves.}
\begin{enumerate}
\item Prove a quantitative lower bound on the \emph{number} of convex $t$-subsets for $t\asymp \log n$ (not just existence), which would directly feed into a lower bound on $f(n)$ of the form $n^{c\log n}$.
\item Construct explicit ``extremal'' point sets with provably few convex subsets, ideally with a sharp estimate $C(P)=\exp((c+o(1))(\log n)^2)$.
\item Use computer search over order types for small $n$ to guess the minimizers of $C(P)$ and detect a pattern that can be generalized.
\end{enumerate}

(iv) \emph{What a minimal counterexample would likely look like.} If the limit constant $c$ does not exist, one would expect two infinite subsequences $n_j$ and $m_j$ for which the minimizing configurations have systematically different ``convex layer'' structure, yielding different accumulation points of $\log f(n)/(\log n)^2$.


