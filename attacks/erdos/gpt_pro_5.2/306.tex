\section*{Problem \#306}

\subsection*{1) FORMAL RESTATEMENT}
Let $a/b\in\mathbb{Q}_{>0}$ with $b$ squarefree. The question asks whether there always exist an integer $k\ge 1$ and integers
\[
  1<n_1<n_2<\cdots<n_k
\]
so that
\[
  \frac{a}{b} = \frac{1}{n_1}+\frac{1}{n_2}+\cdots+\frac{1}{n_k},
\]
and each denominator $n_i$ is the product of two distinct primes (equivalently, $n_i$ is a squarefree semiprime).

\subsection*{2) QUICK LITERATURE/CONTEXT CHECK}
From the ErdosProblems \#306 entry and the Simons Foundation discussion of Butler--Graham (2015):
\begin{itemize}[leftmargin=2em]
\item Erd\H{o}s and Graham conjectured that every integer can be written as a sum of distinct unit fractions whose denominators have exactly two distinct prime factors; this ``two primes'' version remains open.
\item Butler and Graham proved the ``three distinct prime factors'' version for integers (published 2015).
\item An explicit identity (attributed to Graham/Erd\H{o}s) expresses $1$ as a sum of $48$ reciprocals of semiprimes.
\end{itemize}
The general question here asks for \,\emph{every} rational with squarefree denominator, which appears to remain open.

\subsection*{3) ATTACK PLAN}
\begin{enumerate}[leftmargin=2em]
\item \textbf{Necessary conditions.} Identify obvious obstructions (e.g. why $b$ being squarefree is necessary).
\item \textbf{Explicit base identities.} Use known identities (e.g. a semiprime decomposition of $1$) to generate many other rationals via subset sums.
\item \textbf{Search small cases.} For small squarefree $b$ and $a$, search for explicit decompositions with bounded denominators.
\item \textbf{General strategy.} If a systematic method existed, it would likely resemble a ``covering'' argument: build a finite set of allowed denominators whose reciprocal sum is an integer and whose subset sums are rich enough to hit all residue classes needed.
\end{enumerate}

\subsection*{4) WORK}
\paragraph{4.1. Squarefree denominator is necessary.}
If each $n_i$ is a product of two distinct primes, then each $n_i$ is squarefree. Let $L=\operatorname{lcm}(n_1,\dots,n_k)$. Since the lcm of squarefree integers is squarefree, $L$ is squarefree. Writing the sum over the common denominator $L$ gives
\(
\sum_i 1/n_i = m/L
\)
for some integer $m$, hence the reduced denominator divides $L$ and is also squarefree.
Therefore any rational representable as such a sum must have squarefree denominator in lowest terms; this justifies the hypothesis on $b$.

\paragraph{4.2. An explicit 48-term semiprime decomposition of $1$.}
The Simons Foundation article (Dec 10, 2015) records the identity
\[
1=\sum_{j=1}^{48} \frac{1}{d_j}
\]
with the following $48$ denominators, each a product of two distinct primes:
\begin{align*}
&6,21,34,46,58,77,87,115,155,215,287,391,\\
&10,22,35,51,62,82,91,119,187,221,299,689,\\
&14,26,38,55,65,85,93,123,203,247,319,731,\\
&15,33,39,57,69,86,95,133,209,265,323,901.
\end{align*}
(One can verify by direct exact arithmetic that the sum of reciprocals is $1$.)

\paragraph{4.3. Many other rationals via subset sums (explicit examples).}
Because the above list sums to $1$, any subset of the $48$ denominators gives a rational with squarefree denominator.
In particular, the following exact identities hold (all denominators are semiprimes and distinct):
\begin{align*}
\frac{1}{3} &= \frac{1}{6}+\frac{1}{10}+\frac{1}{15},\\
\frac{1}{5} &= \frac{1}{10}+\frac{1}{14}+\frac{1}{35},\\
\frac{1}{2} &= \frac{1}{6}+\frac{1}{10}+\frac{1}{15}+\frac{1}{21}+\frac{1}{26}+\frac{1}{35}+\frac{1}{39}+\frac{1}{65}+\frac{1}{91},\\
\frac{2}{5} &= \frac{1}{6}+\frac{1}{10}+\frac{1}{14}+\frac{1}{38}+\frac{1}{57}+\frac{1}{95}+\frac{1}{133}.
\end{align*}
From these and the $1$-identity one can also get, e.g.,
\(\frac{2}{3}=1-\frac{1}{3}\)
by taking the complement of the denominators $\{6,10,15\}$ inside the 48-term representation of $1$; similarly $\frac{3}{5}=1-\frac{2}{5}$.
These give (long but explicit) semiprime Egyptian decompositions of $2/3$ and $3/5$.

\paragraph{4.4. Computational small-case search.}
A bounded search over semiprime denominators up to $5000$ found short decompositions for several small fractions (including the ones displayed above), suggesting the conjecture is plausible for small squarefree denominators.
However, no general constructive proof for arbitrary squarefree $b$ was obtained.

\subsection*{5) VERIFICATION}
\begin{itemize}[leftmargin=2em]
\item The ``squarefree denominator'' necessity argument is rigorous: the lcm of squarefree integers is squarefree and any reduced denominator divides the lcm.
\item Each displayed denominator in the 48-term list factors as $pq$ with $p\ne q$ primes (e.g. $6=2\cdot 3$, $21=3\cdot 7$, $34=2\cdot 17$, etc.).
\item The identities for $1/3$, $1/5$, $1/2$, $2/5$ can be verified by clearing denominators (or exact arithmetic); for instance
\(1/6+1/10+1/15=(5+3+2)/30=10/30=1/3\).
\end{itemize}

\subsection*{6) FINAL}
\textbf{UNRESOLVED.}
\begin{enumerate}[leftmargin=2em,label=(\roman*)]
\item \textbf{Progress achieved.} I proved that squarefree denominators are necessary, recorded an explicit 48-term decomposition of $1$ into reciprocals of semiprimes, and exhibited explicit decompositions for several other rationals (including $1/2,1/3,1/5,2/5$, and hence also $2/3,3/5$ via complements).
\item \textbf{Where it got stuck.} No method was found to handle an arbitrary target $a/b$ with arbitrary squarefree $b$.
\item \textbf{What next steps would look like.} One would try to develop a general ``covering''/iteration scheme (similar in spirit to Butler--Graham for the three-prime case) that generates, for each squarefree $b$, a finite set of allowed denominators whose reciprocal sums can realize every residue class $a\bmod b$ (after scaling to a common squarefree denominator).
\item \textbf{Falsifiable conjecture.} For every squarefree $b$ and every $1\le a<b$, there exists a semiprime Egyptian fraction expansion of $a/b$ with distinct denominators.
\end{enumerate}

\subsection*{7) COMPLETION ESTIMATE}
To resolve the problem affirmatively, one needs a uniform construction that takes $(a,b)$ with squarefree $b$ and outputs an explicit semiprime-denominator Egyptian expansion. To resolve it negatively, one needs an explicit squarefree $b$ and $a$ for which one can prove that no such expansion exists (e.g. via congruence obstructions), which would require identifying an invariant that all semiprime-denominator sums satisfy but some $a/b$ violates.

