\section*{Problem 750: infinite chromatic number with near-half independent sets in every finite subgraph}

\subsection*{1) FORMAL RESTATEMENT}

Let $f:\mathbb{N}\to\mathbb{N}$ be a function with $f(m)\to\infty$ as $m\to\infty$.
Does there exist a graph $G$ with \emph{infinite} chromatic number $\chi(G)=\infty$ such that
every (induced) subgraph $H$ of $G$ on $m$ vertices satisfies
\[
\alpha(H)\ \ge\ \frac{m}{2} - f(m),
\]
where $\alpha(H)$ denotes the independence number?

(As usual, it suffices to consider induced subgraphs: deleting edges can only increase $\alpha$.)

\subsection*{2) QUICK LITERATURE/CONTEXT CHECK}

Web browsing available? YES.

\begin{itemize}
\item The problem is listed as open on \texttt{erdosproblems.com} (Problem 750).
\item For linear error terms $f(m)=\epsilon m$, this follows from a theorem of Erd\H{o}s, Hajnal and Szemer\'edi (1982) constructing graphs with arbitrarily large chromatic number and whose finite induced subgraphs are ``almost bipartite''.
\end{itemize}

\subsection*{3) ATTACK PLAN}

\begin{itemize}
\item \textbf{Exploit almost-bipartite constructions:} If every $m$-vertex induced subgraph contains a bipartite induced subgraph on $(1-\delta)m$ vertices, then it has an independent set of size at least $(1-\delta)m/2$.
This gives the desired inequality with $f(m)=\delta m/2$.
\item \textbf{Upgrade to infinite chromatic number:} Take a disjoint union of graphs with arbitrarily large finite chromatic number, all satisfying the same local ``almost bipartite'' property.
\item \textbf{Hard part:} To reach arbitrary $f(m)\to\infty$ (e.g.\ $f(m)=\log m$) one would need $\delta=\delta(m)\to 0$, i.e.\ finite subgraphs becoming \emph{asymptotically} bipartite. This is the genuinely open regime.
\end{itemize}

\subsection*{4) WORK}

\paragraph{4.1 A proved special case: $f(m)$ linear.}
The following is a clean implication of the Erd\H{o}s--Hajnal--Szemer\'edi (EHS) theorem.

\medskip
\noindent\textbf{Theorem 4.1 (EHS $\Rightarrow$ Erd\H{o}s conjecture for $f(m)=\epsilon m$).}
Fix $\epsilon>0$.
There exists a graph $G$ with $\chi(G)=\infty$ such that every finite induced subgraph $H$ on $m$ vertices satisfies
\[
\alpha(H)\ \ge\ \Bigl(\frac12-\epsilon\Bigr)m.
\]

\medskip
\noindent\emph{Proof.}
EHS (1982) prove (in particular) that for every $\delta>0$ and every integer $k$ there is a graph $G_{k,\delta}$
with $\chi(G_{k,\delta})>k$ such that every finite vertex set $U$ in $G_{k,\delta}$ contains an induced bipartite subgraph on more than $(1-\delta)|U|$ vertices.
(Equivalently, their function $f^2_{G_{k,\delta}}(n)$ satisfies $f^2_{G_{k,\delta}}(n)>(1-\delta)n$ for all $n$.)

Let $\delta:=2\epsilon$ and for each $k\ge 1$ pick such a graph $G_k:=G_{k,2\epsilon}$.
Now form the disjoint union
\[
G:=\bigsqcup_{k\ge 1} G_k.
\]
Then $\chi(G)=\sup_k \chi(G_k)=\infty$.

Let $H$ be any induced subgraph of $G$ on $m$ vertices. It decomposes as a disjoint union of induced subgraphs $H_k$ of each component $G_k$:
\[
V(H)=\bigsqcup_k V(H_k),\qquad |V(H_k)|=:m_k,\qquad \sum_k m_k=m.
\]
Inside each $H_k$, EHS gives an induced bipartite subgraph on at least $(1-2\epsilon)m_k$ vertices.
Any bipartite graph on $t$ vertices has an independent set of size at least $t/2$ (take the larger side of a bipartition).
Therefore
\[
\alpha(H_k)\ \ge\ \frac{(1-2\epsilon)m_k}{2}=\Bigl(\frac12-\epsilon\Bigr)m_k.
\]
Since components are disjoint, independent sets add:
\[
\alpha(H)\ =\ \sum_k \alpha(H_k)\ \ge\ \Bigl(\frac12-\epsilon\Bigr)\sum_k m_k=\Bigl(\frac12-\epsilon\Bigr)m.
\]
\hfill $\square$

\medskip
Thus the question has a \emph{complete} answer for all functions $f$ with $f(m)\ge \epsilon m$ for some fixed $\epsilon>0$.

\paragraph{4.2 Why the general $f(m)\to\infty$ case is harder.}
If $f(m)=o(m)$, the required lower bound is
\[
\alpha(H)\ \ge\ \frac{m}{2}-o(m),
\]
so every large finite induced subgraph would have to be \emph{asymptotically} bipartite (up to deleting $o(m)$ vertices).
EHS provides only a fixed proportional deletion $\delta m$ (with $\delta$ fixed in advance).
Whether one can make $\delta=\delta(m)\to 0$ while keeping $\chi(G)=\infty$ is exactly the delicate open direction.

\subsection*{5) VERIFICATION}

\begin{itemize}
\item Verified that the disjoint-union argument preserves the $(\frac12-\epsilon)m$ independent-set property because the bound is \emph{linear} in $m$.
\item Verified that the step ``bipartite $\Rightarrow$ independent set $\ge t/2$'' is immediate from the definition of bipartite.
\item The remaining gap is not a proof issue but a genuine strengthening beyond what EHS guarantees.
\end{itemize}

\subsection*{6) FINAL}

\noindent\textbf{UNRESOLVED.}

\begin{itemize}
\item \textbf{What I have:} A complete proof for the Erd\H{o}s conjectured linear regime $f(m)=\epsilon m$ (Theorem 4.1), via EHS (1982).
\item \textbf{Where it breaks:} For sublinear $f(m)$ one would need ``almost bipartite with $o(m)$ deletions'' for every $m$-vertex subgraph, which is not provided by existing arguments in EHS and is the open core.
\item \textbf{Next moves:}
  \begin{enumerate}
  \item Investigate whether EHS-type probabilistic constructions can be iterated with a diagonal/compactness argument to force $\delta(m)\to 0$ on larger and larger scales without collapsing chromatic number.
  \item Search for obstructions: prove lower bounds showing that $\chi(G)=\infty$ forces some finite subgraphs to have $\alpha(H)\le m/2 - c(m)$ with $c(m)\to\infty$ (which would refute the statement for some slowly growing $f$).
  \end{enumerate}
\item \textbf{Minimal lemma that would settle the full problem (one route):}
  For any function $\delta(m)\downarrow 0$, construct a graph $G$ with $\chi(G)=\infty$ such that every $m$-vertex induced subgraph contains an induced bipartite subgraph on at least $(1-\delta(m))m$ vertices.
  Taking $\delta(m)=2f(m)/m$ would then imply $\alpha(H)\ge m/2-f(m)$.
\end{itemize}

\subsection*{7) COMPLETION}

COMPLETION: 60\% (full solution for the linear case; general $f(m)\to\infty$ remains open).