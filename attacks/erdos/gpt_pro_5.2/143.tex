\section*{Erd\H{o}s Problem \#143 (Integer dilation approximation)}

\subsection*{1) FORMAL RESTATEMENT}
Let $\mathcal A\subset (1,\infty)$ be a countably infinite set. Assume the 
\emph{integer-dilation separation condition}
\begin{equation}
\label{eq:143-sep}
\forall x\neq y\in \mathcal A\ \forall k\in\mathbb Z_{\ge 1}:\quad |kx-y|\ge 1.
\end{equation}
Define the truncated harmonic sum
\[
H_{\mathcal A}(X):=\sum_{\substack{x\in\mathcal A\\ x\le X}}\frac1x\qquad (X\ge 1),
\]
and the ``primitive-set analogue'' sum
\[
P_{\mathcal A}:=\sum_{x\in\mathcal A}\frac{1}{x\log x},
\]
where $\log$ is the natural logarithm (so $x>1$ ensures $\log x>0$).

The question asks whether \eqref{eq:143-sep} forces $\mathcal A$ to be sparse, in particular whether it implies either
\begin{align*}
\text{(Q1)}\quad & P_{\mathcal A}<\infty,\\
\text{(Q2)}\quad & H_{\mathcal A}(X)=o(\log X)\quad\text{as }X\to\infty.
\end{align*}

\subsection*{2) QUICK LITERATURE / CONTEXT CHECK}
\begin{itemize}
\item If $\mathcal A\subset\mathbb N$ consists of integers, then \eqref{eq:143-sep} is equivalent to $\mathcal A$ being \emph{primitive} (no element divides another): for integers, $|kx-y|\ge 1$ always holds unless $kx=y$. For primitive integer sets, Erd\H{o}s proved $\sum_{n\in\mathcal A}\frac{1}{n\log n}<\infty$, and later works of Behrend and Erd\H{o}s--S\'ark\H{o}zy--Szemer\'edi gave quantitative bounds for $\sum_{n\le X,\,n\in\mathcal A}\frac1n$.
\item For general real sets $\mathcal A\subset(1,\infty)$, the second implication (Q2) is now known to hold: Koukoulopoulos--Lamzouri--Lichtman (2025) proved that if $\limsup_{X\to\infty}\frac{1}{\log X}\sum_{x\in\mathcal A\cap[1,X]}\frac1x>0$, then for every $\varepsilon>0$ there exist infinitely many distinct $\alpha,\beta\in\mathcal A$ and a positive integer $n$ with $|n\alpha-\beta|<\varepsilon$. The contrapositive yields $H_{\mathcal A}(X)=o(\log X)$ under \eqref{eq:143-sep}.
\item The convergence question (Q1) for general real sets satisfying \eqref{eq:143-sep} appears to remain open.
\end{itemize}

\subsection*{3) ATTACK PLAN}
\textbf{Proof track.}
\begin{enumerate}
\item Extract immediate structural consequences of \eqref{eq:143-sep} (e.g. $1$-separation from the case $k=1$) to obtain baseline bounds.
\item Use the strongest known theorem (Koukoulopoulos--Lamzouri--Lichtman) as a black box to deduce (Q2) rigorously.
\item For (Q1), try to adapt ``primitive set'' arguments: one would like to partition $\mathcal A$ into classes where ``multiplicative structure'' forces a convergent majorant, but in the real setting there is no prime factorization. A plausible route is to convert \eqref{eq:143-sep} into a graph/Diophantine approximation statement and seek a ``density increment'' or ``energy'' inequality strong enough to upgrade (Q2) to (Q1).
\end{enumerate}

\textbf{Disproof track.}
\begin{enumerate}
\item Attempt to construct $\mathcal A$ with relatively large local density (e.g. about one point per interval of length $\log\log x$) while arranging fractional parts so that no integer dilation approximates another element within $1$.
\item Heuristic: to make $\sum \frac{1}{x\log x}$ diverge, one needs $\mathcal A$ to have counting function $A(X)=|\mathcal A\cap[1,X]|$ decaying slowly enough, e.g. $A(X)\asymp X/\log\log X$ would already force divergence of $\int \frac{A'(t)}{t\log t}\,dt$. So a counterexample would need to be much denser than the known examples (like primes) but still satisfy \eqref{eq:143-sep}.
\end{enumerate}

\subsection*{4) WORK}
\paragraph{Phase 1 sanity checks (tiny examples).}
\begin{itemize}
\item \emph{Integer example:} If $\mathcal A$ is the set of prime numbers, then \eqref{eq:143-sep} holds (no prime is a multiple of another prime), and one has $H_{\mathcal A}(X)=\sum_{p\le X}\frac1p\sim\log\log X$ (hence $o(\log X)$) and $\sum_{p}\frac{1}{p\log p}<\infty$.
\item \emph{Trivial upper bound:} Any $\mathcal A$ satisfying \eqref{eq:143-sep} is $1$-separated, hence has at most one element per unit interval.
\end{itemize}

\begin{lemma}[Unit separation]
\label{lem:143-unitsep}
If $\mathcal A$ satisfies \eqref{eq:143-sep}, then for all distinct $x,y\in\mathcal A$ we have $|x-y|\ge 1$.
\end{lemma}
\begin{proof}
Take $k=1$ in \eqref{eq:143-sep}.
\end{proof}

\begin{corollary}[Baseline harmonic bound]
\label{cor:143-baseline}
For all $X\ge 2$,
\[
H_{\mathcal A}(X)\le 1+\sum_{m=2}^{\lfloor X\rfloor}\frac1m\le 1+\log X.
\]
In particular $H_{\mathcal A}(X)=O(\log X)$.
\end{corollary}
\begin{proof}
By Lemma~\ref{lem:143-unitsep}, $\mathcal A$ contains at most one element in each interval $[m,m+1)$ with integer $m\ge 2$. If $x\in\mathcal A\cap[m,m+1)$ then $1/x\le 1/m$. Summing over $m\le X$ gives the first inequality; the second is the standard bound on harmonic numbers.
\end{proof}

\paragraph{Known theorem (used as a black box).}
We record the 2025 theorem in a form tailored to this problem.

\begin{theorem}[Koukoulopoulos--Lamzouri--Lichtman (2025), contrapositive form]
\label{thm:KLL-contrapos}
Let $\mathcal A\subset\mathbb R_{\ge 1}$ be countable and fix $\varepsilon>0$. Assume that there do \emph{not} exist infinitely many pairs $(\alpha,\beta)\in\mathcal A^2$ with $\alpha\ne\beta$ and a positive integer $n$ such that $|n\alpha-\beta|<\varepsilon$. Then
\[
\limsup_{X\to\infty}\frac{1}{\log X}\sum_{\alpha\in\mathcal A\cap[1,X]}\frac{1}{\alpha}=0,
\]
i.e. $H_{\mathcal A}(X)=o(\log X)$.
\end{theorem}

\begin{proof}
This is exactly the contrapositive of the statement proved in the cited work: if the displayed limsup were $>0$, then for every $\varepsilon>0$ there would be infinitely many pairs $(\alpha,\beta)$ with $|n\alpha-\beta|<\varepsilon$ for some $n\in\mathbb Z_{\ge 1}$.
\end{proof}

\begin{corollary}[Resolution of (Q2)]
\label{cor:143-Q2}
If $\mathcal A\subset(1,\infty)$ satisfies \eqref{eq:143-sep}, then
\[
H_{\mathcal A}(X)=o(\log X)\qquad (X\to\infty).
\]
\end{corollary}
\begin{proof}
Under \eqref{eq:143-sep}, for every distinct $\alpha,\beta\in\mathcal A$ and every $n\in\mathbb Z_{\ge 1}$ we have $|n\alpha-\beta|\ge 1$. Hence the hypothesis of Theorem~\ref{thm:KLL-contrapos} holds with (say) $\varepsilon=1$. The conclusion is exactly (Q2).
\end{proof}

\paragraph{Status of (Q1).}
The argument above settles (Q2) but does not appear to imply (Q1). Indeed, there exist abstract sequences $(a_n)$ with $\sum_{a_n\le X} \frac1{a_n}=o(\log X)$ but $\sum_n \frac{1}{a_n\log a_n}=\infty$ (for instance, choosing $a_n$ so that $|\{a_n\le X\}|\asymp X/\log\log X$). The open difficulty is to show that \eqref{eq:143-sep} forbids such ``near $X/\log\log X$'' densities.

\subsection*{5) VERIFICATION}
\begin{itemize}
\item \textbf{Quantifiers:} Corollary~\ref{cor:143-Q2} matches (Q2) exactly: $\lim_{X\to\infty} H_{\mathcal A}(X)/\log X=0$.
\item \textbf{Edge cases:} $x>1$ is used only to ensure $\log x>0$ in (Q1). The (Q2) deduction works already for $\mathcal A\subset[1,\infty)$.
\item \textbf{Self-pairs:} Theorem~\ref{thm:KLL-contrapos} requires $\alpha\ne\beta$, matching \eqref{eq:143-sep}.
\end{itemize}

\subsection*{6) FINAL}
\textbf{UNRESOLVED.} The stronger convergence claim (Q1) remains open (to the best of my current knowledge), but (Q2) is resolved affirmatively.

\begin{enumerate}
\item[(i)] \textbf{Strongest fully proved partial result obtained:}
\begin{itemize}
\item (Q2) holds: $\sum_{x\in\mathcal A,\,x\le X}\frac1x=o(\log X)$ (Corollary~\ref{cor:143-Q2}).
\item Trivial baseline: $H_{\mathcal A}(X)\le 1+\log X$ for $X\ge 2$ (Corollary~\ref{cor:143-baseline}).
\end{itemize}
\item[(ii)] \textbf{Exact first gap:} I do not have a proof upgrading (Q2) to $\sum_{x\in\mathcal A}\frac{1}{x\log x}<\infty$, nor an explicit construction of $\mathcal A$ satisfying \eqref{eq:143-sep} for which that sum diverges.
\item[(iii)] \textbf{Top 3 next moves:}
\begin{enumerate}
\item Try to refine the KLL ``GCD graph'' machinery to obtain a quantitative bound on the counting function $A(X)=|\mathcal A\cap[1,X]|$, strong enough to force $\int \frac{dA(t)}{t\log t}<\infty$.
\item Attempt a transference principle: discretize $\mathcal A$ by integer parts and fractional parts, and show that \eqref{eq:143-sep} enforces an ``approximate primitivity'' property that controls $\sum 1/(x\log x)$.
\item Explore constructive disproof: build $\mathcal A$ by choosing one point in many unit intervals via a Lov\'asz local lemma / entropy-compression argument, arranging to avoid all forbidden neighborhoods $|kx-y|<1$.
\end{enumerate}
\item[(iv)] \textbf{What a minimal counterexample would likely look like:} a set with counting function roughly $A(X)\approx X/(\log\log X)$ (or slightly smaller), spread across unit intervals, with carefully chosen fractional parts so that for each $x\in\mathcal A$ the nearest multiple $kx$ stays at distance $\ge 1$ from every other point. Such a set would make $\sum \frac{1}{x\log x}$ diverge while remaining consistent with $H_{\mathcal A}(X)=o(\log X)$.
\end{enumerate}

\subsection*{7) COMPLETION ESTIMATE}
\textbf{9/10.} The remaining gap (Q1) appears to require genuinely new ideas beyond the current (Q2) technology.


