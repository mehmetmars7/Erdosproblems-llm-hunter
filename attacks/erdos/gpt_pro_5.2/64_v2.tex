\section{Problem 64: The Erd\H{o}s--Gy\'arf\'as conjecture (power-of-two cycle lengths)}

\subsection*{1) ROUND-2 OBJECTIVE}
\textbf{Path (A) --- proof.} The Round-1 output produced (i) an \emph{even} cycle and (ii) a short-cycle bound, but did not force a cycle of length $2^k$. In Round~2 I push the ``structural restriction'' approach: prove the conjecture for additional nontrivial subclasses (notably diameter~$2$ graphs) and derive new necessary conditions for any putative counterexample.

\subsection*{2) ROUND-1 FOUNDATION USED}
I rely on the following Round-1 facts exactly as stated in \texttt{64.tex}:
\begin{itemize}
  \item \textbf{Round-1 Lemma 1.} Every finite graph with minimum degree at least $3$ contains an even cycle.
  \item \textbf{Round-1 Lemma 2.} If $|V(G)|=n$ and $\delta(G)\ge 3$, then $G$ contains a cycle of length at most $2\log_2 n + 3$.
  \item \textbf{Round-1 computation.} The conjecture holds for all graphs on $n\le 7$ vertices.
\end{itemize}
\emph{Note.} The Round-1 remark ``a minimal counterexample must have girth at least $5$'' is too strong: a counterexample may contain triangles (length $3$ is not a power of $2$). The correct immediate necessary condition is only ``no $4$-cycle'' (and no $8$-, $16$-, \dots). I do not use the incorrect girth claim.

\subsection*{3) NEW INSIGHT / TOOL (ROUND-2)}
\begin{enumerate}
  \item A \textbf{new local construction} (Lemma~\ref{lem:diam2-notriangle-edge}) giving an explicit $8$-cycle in any diameter-$2$, $C_4$-free graph that has an edge not contained in any triangle. This is a streamlined ``Case~2'' argument.
  \item A \textbf{published diameter-$2$ verification}: every graph of diameter $2$ and minimum degree at least $3$ has a $4$-cycle or an $8$-cycle (Carr, 2025) \cite{Carr2025}.
  \item \textbf{Sharper computer-aided bounds and $P_k$-free confirmations} (Hegde--Sandeep--Shashank, 2025):
  \begin{itemize}
    \item extensive searches show a counterexample needs at least $17$ vertices, a cubic counterexample at least $30$ vertices, and a bipartite counterexample at least $30$ vertices \cite[pp.~1--2]{HegdeSandeepShashank2025};
    \item the conjecture is proved for all $P_{13}$-free graphs, and a stronger ``$4$- or $8$-cycle'' statement for $P_{12}$-free graphs \cite[Theorems~1--2]{HegdeSandeepShashank2025}.
  \end{itemize}
\end{enumerate}

\subsection*{4) ATTACK PLAN (ROUND-2)}
\textbf{Gap after Round~1.} We know an even cycle exists and some cycle is short, but the set of powers of $2$ is sparse; Round~1 had no mechanism to target the exact lengths $4,8,16,\dots$.

\textbf{Round~2 plan.} Use structural constraints to force small powers of $2$:
\begin{itemize}
  \item Prove the conjecture for the subclass of diameter-$2$ graphs (done via Carr's theorem, plus a simplified subcase proof here).
  \item Deduce that any minimal counterexample must have diameter at least $3$.
  \item Incorporate known computer-aided lower bounds and additional verified classes to shrink the plausible search space of counterexamples.
\end{itemize}

\subsection*{5) WORK (ROUND-2)}

\paragraph{Lemma 3 (new: a diameter-$2$ ``edge not in a triangle'' forces an $8$-cycle).}
\label{lem:diam2-notriangle-edge}
Let $G$ be a finite simple graph with $\operatorname{diam}(G)=2$ and $\delta(G)\ge 3$. Assume $G$ has no $4$-cycle and contains an edge $uv$ such that $u$ and $v$ have no common neighbor (equivalently, the edge $uv$ is not contained in any triangle). Then $G$ contains an $8$-cycle.

\emph{Proof.}
Because $\delta(G)\ge 3$, pick two distinct neighbors $u_1,u_2\in N(u)\setminus\{v\}$ and two distinct neighbors $v_1,v_2\in N(v)\setminus\{u\}$.

\smallskip
\noindent\textbf{Step 1: no cross-adjacencies.}
If some $u_i$ were adjacent to some $v_j$, then
\[
  u\,u_i\,v_j\,v\,u
\]
would be a $4$-cycle, contradicting the assumption. Hence
\begin{equation}
\label{eq:nocross}
  u_i\not\sim v_j\qquad\text{for all }i\in\{1,2\},\ j\in\{1,2\}.
\end{equation}
In particular, $u_1\neq v_1$ and $u_2\neq v_2$.

\smallskip
\noindent\textbf{Step 2: pick two distance-$2$ connectors.}
Since $\operatorname{diam}(G)=2$, we have $\operatorname{dist}(u_1,v_1)\le 2$. By \eqref{eq:nocross} it is not $1$, so it is $2$; choose a vertex $x$ adjacent to both $u_1$ and $v_1$.
Similarly, $\operatorname{dist}(u_2,v_2)=2$; choose a vertex $y$ adjacent to both $u_2$ and $v_2$.

\smallskip
\noindent\textbf{Step 3: distinctness checks (to ensure a simple $8$-cycle).}
We claim the vertices
\[
  u,\ u_1,\ x,\ v_1,\ v,\ v_2,\ y,\ u_2
\]
are pairwise distinct.
\begin{itemize}
  \item $x\neq u$: otherwise $u\sim v_1$, making $v_1$ a common neighbor of $u$ and $v$, contradicting the hypothesis on $uv$.
  \item $x\neq v$: otherwise $v\sim u_1$, again giving a $4$-cycle $u\,u_1\,v\,u$ (since $u\sim v$), or equivalently contradicting \eqref{eq:nocross}.
  \item $x\notin\{u_1,v_1\}$ since $x$ is adjacent to both $u_1$ and $v_1$ while $u_1\not\sim v_1$.
  \item Likewise $y\notin\{u,v,u_2,v_2\}$.
  \item $x\neq y$: if $x=y$, then $u\,u_1\,x\,u_2\,u$ is a $4$-cycle (using $u\sim u_1$, $u\sim u_2$ and $x\sim u_1,u_2$).
\end{itemize}
Also $u_1\neq u_2$ and $v_1\neq v_2$ by construction.

\smallskip
\noindent\textbf{Step 4: the $8$-cycle.}
All edges in the cyclic sequence
\[
  u\! -\! u_1\! -\! x\! -\! v_1\! -\! v\! -\! v_2\! -\! y\! -\! u_2\! -\! u
\]
exist by construction, and by Step~3 the vertices are distinct; hence this is a simple cycle of length $8$.
\qed

\paragraph{Theorem 4 (diameter $2$ case: $4$- or $8$-cycle).}
Avery Carr proved:
\begin{quote}
If $\operatorname{diam}(G)=2$ and $\delta(G)\ge 3$, then $G$ contains a cycle of length $4$ or $8$.
\end{quote}
\cite{Carr2025}. (Carr's proof splits into whether a chosen edge lies in a triangle; Lemma~\ref{lem:diam2-notriangle-edge} gives a short proof of the ``no triangle on the edge'' subcase under the additional assumption ``no $4$-cycle'', while Carr's theorem handles both subcases without assuming $C_4$-freeness.)

\paragraph{Corollary 5 (new restriction on counterexamples).}
Any counterexample to the Erd\H{o}s--Gy\'arf\'as conjecture with $\delta(G)\ge 3$ must satisfy $\operatorname{diam}(G)\ge 3$.

\emph{Proof.} If $\operatorname{diam}(G)=2$, then by Theorem~4 the graph contains a $4$-cycle or an $8$-cycle, i.e., a cycle of length $2^k$ with $k\in\{2,3\}$, contradicting that it is a counterexample. \qed

\paragraph{Known computer-aided lower bounds (beyond Round~1).}
Hegde--Sandeep--Shashank report that extensive searches show:
\begin{itemize}
  \item any counterexample has at least $17$ vertices;
  \item any cubic counterexample has at least $30$ vertices;
  \item any bipartite counterexample has at least $30$ vertices,
\end{itemize}
see \cite[p.~2]{HegdeSandeepShashank2025}.

\paragraph{Known $P_k$-free confirmations (beyond Round~1).}
Hegde--Sandeep--Shashank prove the conjecture for all $P_{13}$-free graphs \cite[Theorem~1]{HegdeSandeepShashank2025}, and in fact show that every $P_{12}$-free graph with $\delta\ge 3$ contains a $4$-cycle or an $8$-cycle \cite[Theorem~2]{HegdeSandeepShashank2025}.

\subsection*{6) ADVERSARIAL VERIFICATION}
\begin{itemize}
  \item \textbf{Lemma~\ref{lem:diam2-notriangle-edge}: quantifiers and edge-cases.}
  The construction requires only $\delta\ge 3$ to choose $u_1,u_2$ and $v_1,v_2$. It also uses \emph{only} the hypotheses ``$\operatorname{diam}=2$'', ``no $4$-cycle'', and ``$uv$ not in any triangle''. The only delicate point is ensuring the cycle is simple. The distinctness checks explicitly rule out identifications such as $x=u$ (which would force $u\sim v_1$ and create a common neighbor of $u$ and $v$) and $x=y$ (which would create a $4$-cycle $u u_1 x u_2 u$). Thus the produced $8$-cycle is genuine.

  \item \textbf{Correction of the Round-1 ``girth'' remark.}
  A counterexample need not be triangle-free; only $4,8,16,\dots$ are forbidden. Hence ``girth at least $5$'' does not follow. None of the Round-1 lemmas depend on that remark, and Round~2 avoids using it.

  \item \textbf{External results usage.}
  Theorem~4 is imported from \cite{Carr2025}; Corollary~5 uses it only via its statement and matches the hypotheses. The numerical lower bounds and $P_k$-free theorems are taken from \cite{HegdeSandeepShashank2025}.
\end{itemize}

\subsection*{7) FINAL}
\textbf{UNRESOLVED (BUT STRICTLY ADVANCED).} The full Erd\H{o}s--Gy\'arf\'as conjecture remains open in general. Round~2 adds:
\begin{itemize}
  \item a clean $8$-cycle construction in the diameter-$2$, $C_4$-free, ``edge not in a triangle'' regime (Lemma~\ref{lem:diam2-notriangle-edge});
  \item a diameter-based exclusion: any counterexample must have diameter at least $3$ (Corollary~5), using Carr's theorem;
  \item stronger verified classes and search bounds from the recent literature (counterexample size $\ge 17$, and $P_{13}$-free verified).
\end{itemize}

\subsection*{8) COMPLETION ESTIMATE (MANDATORY)}
COMPLETION: 45\%.

\subsection*{9) REFERENCES}
\begin{thebibliography}{99}

\bibitem{Erdos1997}
P.~Erd\H{o}s,
\newblock Some old and new problems in various branches of combinatorics,
\newblock \emph{Discrete Mathematics} \textbf{165--166} (1997), 227--231.

\bibitem{LiMo20}
X.~Liu and R.~Montgomery,
\newblock A solution to Erd\H{o}s and Gy\'arf\'as's conjecture for graphs of large average degree,
\newblock \emph{Journal of Combinatorial Theory, Series B} \textbf{142} (2020), 186--208.

\bibitem{HegdeSandeepShashank2025}
A.~S. Hegde, R.~B. Sandeep, and P.~Shashank,
\newblock Erd\H{o}s--Gy\'arf\'as conjecture on graphs without long induced paths,
\newblock arXiv:2410.22842v2 (2025).

\bibitem{Carr2025}
A.~Carr,
\newblock Cycles of length 4 or 8 in graphs with diameter 2 and minimum degree at least 3,
\newblock arXiv:2508.19302 (2025).

\end{thebibliography}
