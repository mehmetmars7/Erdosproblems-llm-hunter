% Erdos Problem #1127

\subsection*{FORMAL RESTATEMENT}
Fix an integer dimension $n\ge 1$ and consider Euclidean space $\mathbb{R}^n$ with its usual metric $d(x,y)=\|x-y\|_2$.
Question: does there exist a partition
\[
\mathbb{R}^n = \bigcup_{j=1}^{\infty} S_j,\qquad S_i\cap S_j=\varnothing\ (i\neq j),
\]
such that for every $j$ the set of pairwise distances inside $S_j$ is \emph{distinct}, i.e.
\[
\forall\{x,y\}\neq\{u,v\}\subset S_j\ (x\neq y,\ u\neq v)\implies \|x-y\|_2\neq \|u-v\|_2?
\]

\subsection*{QUICK LITERATURE/CONTEXT CHECK}
I only record context explicitly stated in the problem file.
The file states:
\begin{itemize}
\item Assuming the continuum hypothesis (CH), this is true for $n=1$ (Erd\H{o}s--Kakutani), for $n=2$ (Davies), and for all $n$ (Kunen).
\item Some dependence on CH is necessary in the sense that if CH is false then (as stated) any finite decomposition of $\mathbb{R}$ has four points determining only four distances.
\end{itemize}

\subsection*{ATTACK PLAN}
\textbf{Proof track ideas.}
\begin{itemize}
\item Under CH, attempt to build each color class as a large $\mathbb{Q}$-linearly independent set (in $n=1$), since that property forces distinct distances (Lemma 1127.1).
\item In higher $n$, search for algebraic independence conditions that force distinct Euclidean distances (hard because distances involve squares and sums).
\end{itemize}
\textbf{Disproof track ideas.}
\begin{itemize}
\item Try to show that in ZFC (without CH) a countable partition must create repeated distances in some cell via a Ramsey-type argument.
\item Construct adversarial colorings and search for forced repeated distances in one color.
\end{itemize}

\subsection*{WORK}
\textbf{Lemma 1127.1 ($\mathbb{Q}$-linear independence implies distinct distances in $\mathbb{R}$).}
Let $S\subset\mathbb{R}$ be linearly independent over $\mathbb{Q}$.
Then for any two unordered pairs of distinct points $\{x,y\}\ne\{u,v\}\subset S$ we have
\[
|x-y|\ne |u-v|.
\]

\emph{Proof.}
Assume for contradiction that $|x-y|=|u-v|$ for two unordered pairs of distinct points from $S$.
Then either
\begin{enumerate}
\item $x-y=u-v$, in which case $x-y-u+v=0$, or
\item $x-y=v-u$, in which case $x-y+u-v=0$.
\end{enumerate}
In both cases we obtain a rational linear relation among the elements of $\{x,y,u,v\}\subset S$ with coefficients in $\{\pm 1\}$, not all zero.
Since $S$ is $\mathbb{Q}$-linearly independent, the only such relation is the trivial one, so all coefficients must be $0$.
That is impossible unless the multiset of variables collapses, forcing $\{x,y\}=\{u,v\}$.
This contradicts $\{x,y\}\ne\{u,v\}$.
Therefore $|x-y|\ne |u-v|$.
\qed

\textbf{Lemma 1127.2 (explicit infinite set with distinct distances).}
The set $S:=\{2^k: k\in\mathbb{N}\}\subset\mathbb{R}$ has all pairwise distances distinct.
Consequently, for every $n\ge 1$ the embedded set $S e_1=\{2^k e_1\}\subset\mathbb{R}^n$ has all pairwise Euclidean distances distinct.

\emph{Proof.}
Consider two pairs with indices $i<j$ and $k<\ell$ and suppose
\[
2^j-2^i = 2^{\ell}-2^k.
\]
Factor both sides:
\[
2^i(2^{j-i}-1)=2^k(2^{\ell-k}-1).
\]
The integers $2^{j-i}-1$ and $2^{\ell-k}-1$ are odd.
Taking the $2$-adic valuation (the exponent of $2$ dividing an integer) of both sides shows $i=k$.
Substituting $i=k$ gives $2^{j-i}-1=2^{\ell-i}-1$, hence $j=\ell$.
Therefore the unordered pairs coincide, so all distances are distinct.
In $\mathbb{R}^n$, points on the $e_1$-axis satisfy $\|2^j e_1-2^i e_1\|_2=|2^j-2^i|$, so the same distinctness holds.
\qed

\textbf{Lemma 1127.3 (no isosceles triangles in one cell).}
If $S\subset\mathbb{R}^n$ has all pairwise distances distinct, then $S$ contains no (nondegenerate) isosceles triangle.

\emph{Proof.}
If $x,y,z\in S$ are distinct and form an isosceles triangle, then two of the three distances among them are equal, e.g. $\|x-y\|_2=\|x-z\|_2$.
But this gives two distinct unordered pairs $\{x,y\}$ and $\{x,z\}$ with the same distance, contradicting the defining property of $S$.
\qed

\textbf{FAST REALITY CHECK (finite sample).}
For $S=\{1,2,4,8,16\}$ the distances are
\[
1,2,3,4,6,7,8,12,14,15
\]
(all distinct). This matches Lemma 1127.2.

\subsection*{VERIFICATION}
\begin{itemize}
\item Lemma 1127.1: the only subtlety is the use of absolute values; both sign cases give a nontrivial rational relation, so independence rules them out.
\item Lemma 1127.2: the $2$-adic valuation argument is rigorous because each side is a product of a power of $2$ and an odd integer, so the exponent of $2$ is unique.
\item Lemma 1127.3 is immediate from the definition of distinct pairwise distances.
\end{itemize}

\subsection*{FINAL}
\textbf{UNRESOLVED}

(i) \textbf{Strongest proved partial result.}
In $\mathbb{R}$, any $\mathbb{Q}$-linearly independent set has distinct pairwise distances (Lemma 1127.1). Independently of set-theoretic issues, explicit infinite examples exist in every $\mathbb{R}^n$ (Lemma 1127.2).

(ii) \textbf{First gap (crisp).}
Decide in ZFC (without additional axioms such as CH) whether there exists a countable partition of $\mathbb{R}^n$ into sets with all pairwise distances distinct.

(iii) \textbf{Top 3 next moves.}
\begin{itemize}
\item Identify a purely geometric/algebraic condition on $S\subset\mathbb{R}^n$ that forces distinct distances and is compatible with building a countable partition.
\item Attempt a ZFC lower-bound obstruction: prove that any countable coloring of $\mathbb{R}^n$ yields repeated distances in some color via a combinatorial/measure/category argument.
\item For $n=1$, analyze whether ``$\mathbb{R}$ is a countable union of $\mathbb{Q}$-linearly independent sets'' can be decided in ZFC.
\end{itemize}

(iv) \textbf{Minimal counterexample structure.}
A negative answer in ZFC would manifest as a proof that in every countable partition of $\mathbb{R}^n$ there exists at least one cell $S_j$ containing two distinct pairs at the same distance; in particular that cell would contain an isosceles triangle (Lemma 1127.3).


