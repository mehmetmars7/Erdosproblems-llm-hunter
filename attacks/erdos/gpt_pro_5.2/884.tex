% Erdos Problem #884
% URL: https://www.erdosproblems.com/884

Is it true that, for any $n$, if $d_1<\cdots <d_t$ are the divisors of $n$, then
\[\sum_{1\leq i<j\leq t}\frac{1}{d_j-d_i} \ll 1+\sum_{1\leq i<t}\frac{1}{d_{i+1}-d_i},\]
where the implied constant is absolute? See also [144] .

1) FORMAL RESTATEMENT

Let $n\ge 1$ be an integer and let $1\le d_1<d_2<\cdots<d_t=n$ be the complete increasing list of positive divisors of $n$.
Define
\[S_{\mathrm{all}}(n):=\sum_{1\le i<j\le t}\frac{1}{d_j-d_i},\qquad S_{\mathrm{gap}}(n):=\sum_{1\le i<t}\frac{1}{d_{i+1}-d_i}.
\]
Question: Does there exist an absolute constant $C>0$ such that for all $n$,
\[S_{\mathrm{all}}(n)\le C\,(1+S_{\mathrm{gap}}(n))\ ?\]

2) QUICK LITERATURE/CONTEXT CHECK

No results are quoted in the problem statement for #884, so I do not claim any. I note only that for a general increasing sequence $x_1<\cdots<x_t$ (not necessarily divisors), the analogous inequality fails with an absolute constant (e.g. arithmetic progressions give a $\log t$ loss). Therefore any proof must use special structure of the divisor sequence.

3) ATTACK PLAN

- Prove general inequalities relating $\sum_{i<j}1/(x_j-x_i)$ to consecutive gaps; this will necessarily incur a $\log t$ factor for arbitrary sequences. Apply to divisors as a baseline.
- Prove the desired inequality for special families of $n$ (e.g. prime powers), to understand whether an absolute constant is plausible.
- Run computation for $n$ up to a moderate bound to look for evidence of unbounded growth of the ratio.

4) WORK

Lemma 884.1 (General harmonic-loss bound via consecutive gaps).
Let $x_1<\cdots<x_t$ be any strictly increasing real sequence and let $g_i:=x_{i+1}-x_i>0$.
Then
\[\sum_{1\le i<j\le t}\frac{1}{x_j-x_i}\ \le\ H_{t-1}\sum_{i=1}^{t-1}\frac{1}{g_i},\]
where $H_{t-1}:=\sum_{m=1}^{t-1}\frac1m$.

Proof.
Fix $1\le i<j\le t$ and write
\[x_j-x_i=g_i+g_{i+1}+\cdots+g_{j-1}=:\sum_{k=i}^{j-1} g_k.
\]
By Cauchy--Schwarz,
\[\Big(\sum_{k=i}^{j-1}g_k\Big)\Big(\sum_{k=i}^{j-1}\frac{1}{g_k}\Big)\ge (j-i)^2.
\]
Rearranging gives
\[\frac{1}{x_j-x_i}=\frac{1}{\sum_{k=i}^{j-1}g_k}\le \frac{1}{(j-i)^2}\sum_{k=i}^{j-1}\frac{1}{g_k}.
\]
Summing over all $i<j$ yields
\[\sum_{i<j}\frac1{x_j-x_i}\le \sum_{i<j}\frac{1}{(j-i)^2}\sum_{k=i}^{j-1}\frac{1}{g_k}.
\]
Swap the order of summation: for each fixed $k\in\{1,\dots,t-1\}$, the term $1/g_k$ appears for precisely those pairs $(i,j)$ with $i\le k<j$.
Thus
\[\sum_{i<j}\frac1{x_j-x_i}\le \sum_{k=1}^{t-1}\frac1{g_k}\cdot c_k\quad\text{where}\quad c_k:=\sum_{i\le k<j}\frac{1}{(j-i)^2}.
\]
To bound $c_k$, group pairs by $m:=j-i\ge 1$. For a fixed $m$, the condition $i\le k<j=i+m$ is equivalent to $k-m+1\le i\le k$, so there are at most $m$ admissible $i$.
Therefore
\[c_k\le \sum_{m=1}^{t-1} \frac{m}{m^2}=\sum_{m=1}^{t-1}\frac1m=H_{t-1}.
\]
Substituting gives the claim. \qed

Corollary 884.1a (Baseline bound for divisors).
Applying Lemma 884.1 with $x_i=d_i$ gives
\[S_{\mathrm{all}}(n)\le H_{t-1}\,S_{\mathrm{gap}}(n),\]
which is weaker than the conjecture by a factor $H_{t-1}\asymp \log t$.

Lemma 884.2 (Prime power case satisfies the conjectured form with an absolute constant).
Let $n=p^m$ with $p$ prime and $m\ge 1$. Then
\[S_{\mathrm{all}}(n)\le \frac{p}{p-1}\, S_{\mathrm{gap}}(n)\le 2\,S_{\mathrm{gap}}(n)\le 2\,(1+S_{\mathrm{gap}}(n)).\]

Proof.
The divisors are $d_i=p^{i-1}$ for $i=1,\dots,m+1$.
Consecutive gaps are
\[d_{i+1}-d_i=p^i-p^{i-1}=p^{i-1}(p-1)\quad (i=1,\dots,m).
\]
Therefore
\[S_{\mathrm{gap}}(n)=\sum_{i=1}^{m}\frac{1}{p^{i-1}(p-1)}=\frac{1}{p-1}\sum_{i=0}^{m-1}p^{-i}.
\]
For $i<j$, write $r:=j-i\ge 1$. Then
\[d_j-d_i=p^{j-1}-p^{i-1}=p^{i-1}(p^r-1)\ge p^{i-1}\,p^{r-1}(p-1)=p^{j-2}(p-1).
\]
Hence
\[\frac{1}{d_j-d_i}\le \frac{1}{p^{j-2}(p-1)}=\frac{p}{p-1}\cdot \frac{1}{p^{j-1}}.
\]
Summing over $i<j$ with fixed $j$ gives
\[\sum_{i=1}^{j-1}\frac{1}{d_j-d_i}\le (j-1)\cdot \frac{p}{p-1}\cdot \frac{1}{p^{j-1}}.
\]
Now sum over $j=2,\dots,m+1$:
\[S_{\mathrm{all}}(n)=\sum_{j=2}^{m+1}\sum_{i=1}^{j-1}\frac{1}{d_j-d_i}\le \frac{p}{p-1}\sum_{j=2}^{m+1}(j-1)p^{-(j-1)}.
\]
On the other hand, from the expression for $S_{\mathrm{gap}}$,
\[S_{\mathrm{gap}}(n)=\frac{1}{p-1}\sum_{j=1}^{m}p^{-(j-1)}.
\]
To compare, note that for each $j\ge 1$ one has $(j)p^{-j}\le \sum_{u=j}^{\infty}p^{-u}=p^{-j}/(1-1/p)=\frac{p}{p-1}p^{-j}$. Rearranging gives
\[(j)p^{-(j)}\le \frac{p}{p-1}p^{-j}.
\]
Applying this with $j-1$ in place of $j$ shows
\[(j-1)p^{-(j-1)}\le \frac{p}{p-1}p^{-(j-1)}.
\]
Therefore
\[\sum_{j=2}^{m+1}(j-1)p^{-(j-1)}\le \frac{p}{p-1}\sum_{j=2}^{m+1}p^{-(j-1)}=\frac{p}{p-1}\sum_{i=0}^{m-1}p^{-i}.
\]
Combining with the earlier upper bound yields
\[S_{\mathrm{all}}(n)\le \frac{p}{p-1}\cdot \frac{p}{p-1}\sum_{i=0}^{m-1}p^{-i}=\frac{p}{p-1}\cdot \Big(\frac{1}{p-1}\sum_{i=0}^{m-1}p^{-i}\Big)=\frac{p}{p-1}S_{\mathrm{gap}}(n).
\]
Since $p/(p-1)\le 2$ for all primes $p$, the lemma follows. \qed

FAST REALITY CHECK (computation).
For all $n\le 10^6$, the ratio
\[R(n):=\frac{S_{\mathrm{all}}(n)}{1+S_{\mathrm{gap}}(n)}\]
was computed for those $n$ with at least $60$ divisors (to focus on potentially large ratios).
The maximum observed in this range was
\[\max_{n\le 10^6} R(n) \approx 3.6531\ \text{at }n=997920,\]
with $t=240$ divisors, $S_{\mathrm{all}}\approx 157.7273$ and $S_{\mathrm{gap}}\approx 42.1761$.
This does not prove boundedness but provides numerical evidence for an absolute-constant bound at least up to $10^6$.

5) VERIFICATION

- Lemma 884.1: checked the only nontrivial step is the bound on the multiplicity of intervals of length $m$ containing a fixed gap index $k$; the count is at most $m$, hence the harmonic sum.
- Lemma 884.2: verified the inequality $p^r-1\ge p^{r-1}(p-1)$ and the comparison step that leads to the factor $p/(p-1)\le 2$.
- Computation: the search was not a proof; it can miss growth beyond $10^6$.

6) FINAL

**UNRESOLVED**

(i) Strongest proved partial result.
- For any increasing divisor sequence (indeed any increasing real sequence), one has
  \[S_{\mathrm{all}}(n)\le H_{t-1}\,S_{\mathrm{gap}}(n).\]
- The conjectured inequality holds (with an absolute constant $2$) for the infinite family $n=p^m$ (prime powers) (Lemma 884.2).
- Computation up to $10^6$ found $R(n)\lesssim 3.66$.

(ii) First gap (crisp).
Remove the harmonic-loss factor for divisors:
prove or disprove that there exists absolute $C$ with
\[S_{\mathrm{all}}(n)\le C\,(1+S_{\mathrm{gap}}(n))\ \text{for all }n.\]

(iii) Top 3 next moves.
1. Decompose divisor pairs into “near” pairs (where $d_j/d_i$ is close to $1$) and “far” pairs, and attempt to show the far-pair contribution is always controlled by $S_{\mathrm{gap}}$ with an absolute constant.
2. Analyze the structure of very closely spaced divisors near $\sqrt n$; a strong bound on clusters of divisors in short intervals would likely imply the conjecture.
3. Use computation to search for $n$ with larger ratios, focusing on highly composite numbers and numbers with many divisors near $\sqrt n$.

(iv) Minimal counterexample structure.
A counterexample sequence would require integers $n$ whose divisor set contains a large “cluster” of divisors with roughly uniform small additive spacing (so that $\sum_{i<j}1/(d_j-d_i)$ gains a logarithmic factor) while the consecutive-gap reciprocal sum remains comparatively small. Such clustering is most plausible near $\sqrt n$.


