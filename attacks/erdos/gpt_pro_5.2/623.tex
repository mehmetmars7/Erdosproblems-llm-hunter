\section*{Problem 623}

\subsection*{FORMAL RESTATEMENT}
Let $X$ be a set with $|X|=\aleph_\omega$. Let $f:[X]^{<\omega}\to X$ be a function defined on the family of all \emph{finite} subsets of $X$ such that
\[
\forall A\in [X]^{<\omega}\qquad f(A)\notin A.
\]
A subset $Y\subseteq X$ is \emph{independent} (for $f$) if
\[
\forall B\in [Y]^{<\omega}\qquad f(B)\notin Y.
\]
The question is: \emph{Must there exist an infinite independent set $Y\subseteq X$?}

\subsection*{QUICK LITERATURE/CONTEXT CHECK}
This is Erd\H{o}s Problem \#623. It is listed as \emph{open} (as of the database access date recorded on the Erd\H{o}s Problems site). The prompt attributes the problem to Erd\H{o}s--Hajnal and notes their result that for $|X|<\aleph_\omega$ the answer is negative, while Erd\H{o}s suggested the $\aleph_\omega$ case may be undecidable.

A nearby (stronger) theme in set theory is the \emph{free-subset property for algebras/structures}, studied e.g. by Koepke: one asks for free subsets of a structure under closure by finitely many (or countably many) finitary operations. Koepke proved that a strong free-subset property at $\omega_\omega$ (the initial ordinal of cardinality $\aleph_\omega$) has high consistency strength (equiconsistent with a measurable cardinal). This suggests that at least some strengthened versions of ``free subset'' phenomena at $\aleph_\omega$ are not settled by elementary ZFC arguments.

\subsection*{ATTACK PLAN}
\begin{enumerate}[label=\arabic*.]
\item Try to relate the one-step ``independence'' demanded here to standard free-subset principles for structures/algebras. If a known free-subset principle implies independence for set mappings, this gives conditional/relative-consistency information.
\item Attempt a direct ZFC construction of an infinite independent set via a recursion exploiting the representation $\aleph_\omega=\bigcup_{n<\omega}\aleph_n$.
\item Failing a full proof, isolate a precise obstruction: where the recursion fails (which finite configurations force the next choice into a small forbidden set), and what additional hypothesis would remove the obstruction.
\end{enumerate}

\subsection*{WORK}
\paragraph{(A) A conditional positive result from a stronger free-subset property.}
Define, for each $m\ge 1$, an $m$-ary operation
\[
 g_m: X^m\to X,\qquad g_m(x_1,\dots,x_m):=f(\{x_1,\dots,x_m\}).
\]
(For repeated coordinates one can fix any convention; it does not affect the argument below.) Consider the structure
\[
\mathcal S := \bigl(X, (g_m)_{m\in\mathbb N_{\ge 1}}\bigr)
\]
in a countable language of function symbols.

Assume a \emph{strong} principle: every such structure on $X$ has a countably infinite subset $Y$ that is \emph{free in the closure sense}, i.e. for every $y\in Y$ we have
\[
 y\notin \langle Y\setminus\{y\}\rangle_{\mathcal S},
\]
where $\langle\cdot\rangle_{\mathcal S}$ denotes the substructure generated by a set under the functions $(g_m)$.

Then $Y$ is independent for $f$ in the sense of the problem. Indeed, if $B\subseteq Y$ is finite and $f(B)\in Y$, put $y:=f(B)$. By hypothesis $f(B)\notin B$, so $B\subseteq Y\setminus\{y\}$. But then applying the operation $g_{|B|}$ to (an ordering of) $B$ produces $y$, so $y\in\langle Y\setminus\{y\}\rangle_{\mathcal S}$, contradicting freeness. Hence $f(B)\notin Y$ for all finite $B\subseteq Y$.

This shows: \emph{any model in which such a strong free-subset property holds at $|X|=\aleph_\omega$ yields a positive answer to Problem 623.}

\paragraph{(B) Why this does not resolve the ZFC question.}
The above implication uses a closure-based freeness principle that is strictly stronger than one-step independence: one-step independence only forbids outputs $f(B)$ landing back in $Y$ when $B\subseteq Y$, whereas closure-based freeness also forbids reaching an element of $Y$ via iterated applications of the $g_m$ using intermediate values outside $Y$. Since the problem only asks for one-step independence, existing high-consistency-strength results about closure-freeness do not automatically settle the original question.

\paragraph{(C) Attempt at a direct construction and the first obstruction.}
A natural attempt is to build an increasing sequence $(y_n)_{n<\omega}$ with $Y=\{y_n:n<\omega\}$ such that
\[
\forall \text{finite }B\subseteq Y\quad f(B)\notin Y.
\]
At stage $n$, having chosen $y_0,\dots,y_{n-1}$, one can certainly ensure
\[
 y_n\notin \{y_0,\dots,y_{n-1}\}\cup\{f(B):B\subseteq\{y_0,\dots,y_{n-1}\}\},
\]
by picking $y_n$ outside a countable (indeed finite) forbidden set.

However, the independence requirement also involves sets $B$ that \emph{contain} the new element $y_n$. For each previously chosen $y_i$, we must avoid future choices $y_j$ hitting values of the form
\[
 f(B\cup\{y_n\})\quad\text{for }B\subseteq\{y_0,\dots,y_{n-1}\},
\]
but these values are not known when choosing $y_n$, and may land anywhere in $X$, potentially forcing conflicts with infinitely many future choices. This is the first genuine gap: controlling all future images involving $y_n$ requires global information about $f$ that a naive stage-by-stage avoidance cannot provide.

\subsection*{VERIFICATION}
The conditional implication in (A) is checked directly: if a closure-free $Y$ existed for the structure built from $f$, then no $f(B)$ with $B\subseteq Y$ could lie in $Y$, because that would generate an element of $Y$ from the others in one step.

The obstruction in (C) is real: the set of constraints involving the new element $y_n$ is not finite at stage $n$ from the standpoint of future steps.

\subsection*{FINAL}
\textbf{UNRESOLVED.}
\begin{itemize}[leftmargin=*]
\item (i) What was proved: A \emph{conditional} positive answer: any sufficiently strong closure-based free-subset principle for countable languages on $|X|=\aleph_\omega$ implies the existence of an infinite independent set for every $f:[X]^{<\omega}\to X$ with $f(A)\notin A$.
\item (ii) Strongest partial result stated here: The above reduction plus the observation that strong free-subset principles at $\omega_\omega$ have high consistency strength in the literature (so the conditional route is plausible in some models).
\item (iii) First gap: A direct ZFC recursion to build $Y$ must control values $f(B\cup\{y_n\})$ that depend on the new point $y_n$ and can later collide with future chosen points; the naive finite-avoidance step does not address these infinitely many future constraints.
\item (iv) Next concrete step: Either (a) find a \emph{ZFC} combinatorial principle at $\aleph_\omega$ (e.g. a suitable reflection/compactness or $\Delta$-system style argument) that lets one thin a large set to an infinite one-step free set for \emph{all} finite arities simultaneously, or (b) establish a relative-consistency separation by forcing a counterexample $f$ with no infinite independent set.
\end{itemize}

\subsection*{COMPLETION ESTIMATE}
COMPLETION: 20\%.

% =====================
% Problem 624
% =====================
