\section{Erd\H{o}s Problem \#51 --- Round 2}

\subsection*{1) ROUND-2 OBJECTIVE}
\textbf{Path (C): obstruction/bounding approach.}
Round~1 identified that large values of $n_a/a$ require $n_a$ to have many small prime divisors (Lemma~1 there), but left open whether this can be forced for the \emph{minimal} preimage.
The most promising next step is therefore to (i) obtain \emph{unconditional} global upper bounds on $n_a/a$ (ruling out any ``fast'' divergence), and (ii) push computation far enough to see whether record values of $n_a/a$ grow at all.

\subsection*{2) ROUND-1 FOUNDATION USED}
We rely on the following Round~1 items (without reproving them):
\begin{itemize}
\item (R1--Lemma~1) $\displaystyle \frac{n}{\varphi(n)}=\prod_{p\mid n}\frac{p}{p-1}$.
\item (R1--Lemma~2) If $a=p-1$ with $p$ prime then $n_a=p$ and $n_a/a\to 1$ along this infinite family.
\item (R1--Lemma~3) For any totient value $a$, $n_a\ge a+1$.
\item (R1--Computation) For $a\le 20000$ the maximum observed ratio was $\approx 2.035$.
\end{itemize}

\subsection*{3) NEW INSIGHT / TOOL (ROUND-2)}
\begin{itemize}
\item \textbf{New structural lemma:} $\varphi(n)\ge \sqrt n$ for all $n\neq 2,6$. This implies \emph{quadratic compression} for inverse totients: if $\varphi(n)=a$ and $a\ge 3$ then $n\le a^2$, hence $n_a\le a^2$.
\item \textbf{New explicit analytic bound:} using Rosser--Schoenfeld's inequality
\[\frac{n}{\varphi(n)}<e^{\gamma}\log\log n+\frac{2.50637}{\log\log n}\qquad(n\ge 3),\]
we obtain an explicit upper bound on $n_a/a$ in terms of $a$ alone.
\item \textbf{Much larger computation:} a totient sieve up to $n\le 10^8$ to locate record values of $n_a/a$.
\end{itemize}

\subsection*{4) ATTACK PLAN (ROUND-2)}
\begin{enumerate}
\item Close a Round~1 gap by proving a nontrivial \emph{unconditional} upper bound for $n_a/a$.
\item Extend the computational search from $n\le 3\cdot 10^5$ (Round~1) to $n\le 10^8$, recording the maximum of $n_a/a$ among all totients realized with minimal preimage $\le 10^8$.
\item Try to identify rigid structural patterns among record-holders as a guide for (future) proofs or counterexamples.
\end{enumerate}

\subsection*{5) WORK (ROUND-2)}

\subsubsection*{5.1. A basic but powerful inequality: $\varphi(n)\ge \sqrt n$ except $n=2,6$}
\begin{lemma}[\,$\varphi(n)\ge \sqrt n$ for $n\neq 2,6$\,]
For every integer $n\ge 1$ with $n\neq 2,6$, we have $\varphi(n)\ge \sqrt n$.
Equivalently, $\varphi(n)^2\ge n$ for all $n\neq 2,6$.
\end{lemma}

\begin{proof}
Define
\[g(n):=\frac{\varphi(n)^2}{n}.
\]
Since $\varphi$ is multiplicative and $n\mapsto 1/n$ is multiplicative, $g$ is multiplicative:
if $(m,n)=1$ then
\[
 g(mn)=\frac{\varphi(mn)^2}{mn}=\frac{\varphi(m)^2}{m}\cdot\frac{\varphi(n)^2}{n}=g(m)g(n).
\]
Hence it suffices to analyze prime powers.
For $p$ prime and $k\ge 1$,
\[
 g(p^k)=\frac{\varphi(p^k)^2}{p^k}=\frac{\bigl(p^{k-1}(p-1)\bigr)^2}{p^k}=p^{k-2}(p-1)^2.
\]
\emph{Odd primes.}
If $p\ge 3$ then for all $k\ge 1$,
\[
 g(p^k)\ge g(p)=\frac{(p-1)^2}{p} > 1,
\]
since $(p-1)^2-p=p^2-3p+1\ge 1$ for $p\ge 3$.
So $g(p^k)>1$ for every odd prime power.

\emph{The prime $2$.}
We have $g(2)=\frac{1}{2}<1$, while for $k\ge 2$,
\[
 g(2^k)=2^{k-2}\ge 1.
\]
Now write $n=2^k m$ with $m$ odd.
If $k\ge 2$ then $g(n)=g(2^k)g(m)\ge 1\cdot 1=1$.
If $k=0$ then $n$ is odd and $g(n)=g(m)\ge 1$ (with equality only at $m=1$).
If $k=1$ then $g(n)=g(2)g(m)=\tfrac12 g(m)$.
This is $<1$ only if $g(m)<2$.
But for odd $m>1$, the smallest possible value of $g(m)$ occurs at $m=3$, giving $g(3)=4/3<2$; for $m\ge 5$, one checks $g(m)\ge g(5)=16/5>2$ because every prime-power factor contributes $g(p^k)>1$ and the least such factor for odd primes is $g(5)=16/5$.
Hence $g(2m)<1$ occurs exactly for $m=1$ (i.e. $n=2$) and $m=3$ (i.e. $n=6$).

Therefore $g(n)\ge 1$ for all $n\neq 2,6$, i.e. $\varphi(n)^2\ge n$ and $\varphi(n)\ge \sqrt n$ for $n\neq 2,6$.
\end{proof}

\begin{corollary}[Quadratic bound for inverse totients]
If $a\ge 3$ is a totient value and $n$ is any solution of $\varphi(n)=a$, then $n\le a^2$.
In particular, $n_a\le a^2$ for all totient values $a\ge 3$.
\end{corollary}

\begin{proof}
Let $\varphi(n)=a$ with $a\ge 3$.
Then $n\neq 2,6$ (since $\varphi(2)=1$ and $\varphi(6)=2$), so by the lemma $n\le \varphi(n)^2=a^2$.
Taking the minimum over solutions gives $n_a\le a^2$.
\end{proof}

\subsubsection*{5.2. An explicit analytic upper bound on $n_a/a$}
We use the following explicit inequality of Rosser--Schoenfeld.

\begin{theorem}[Rosser--Schoenfeld (1962), explicit upper bound]
For every integer $n\ge 3$,
\[
\frac{n}{\varphi(n)}<e^{\gamma}\log\log n+\frac{2.50637}{\log\log n}.
\]
\end{theorem}

\begin{theorem}[Explicit upper bound for $n_a/a$]
Let $a\ge 3$ be a totient value and $n_a:=\min\{n:\varphi(n)=a\}$.
Then
\[
\frac{n_a}{a}<e^{\gamma}\log\log(a^2)+\frac{2.50637}{\log\log(a^2)}
\;=\;
 e^{\gamma}\bigl(\log\log a+\log 2\bigr)+\frac{2.50637}{\log\log a+\log 2}.
\]
In particular, $\frac{n_a}{a}=O(\log\log a)$.
\end{theorem}

\begin{proof}
Since $a\ge 3$, we have $n_a\ge a+1\ge 4$, hence $n_a\ge 3$.
Applying Rosser--Schoenfeld to $n=n_a$ gives
\[
\frac{n_a}{a}=\frac{n_a}{\varphi(n_a)}<e^{\gamma}\log\log n_a+\frac{2.50637}{\log\log n_a}.
\]
By the previous corollary, $n_a\le a^2$.
As $\log\log$ is increasing on $[3,\infty)$, we have $\log\log n_a\le \log\log(a^2)$.
Substituting yields the displayed bound.
The $O(\log\log a)$ conclusion follows immediately.
\end{proof}

\subsubsection*{5.3. Extended computation up to $n\le 10^8$}
\paragraph{Method.}
We computed $\varphi(n)$ for all $1\le n\le 10^8$ by a standard totient sieve.
For each totient value $a\le 10^8$ encountered, we recorded the least $n$ with $\varphi(n)=a$.
Finally we scanned over all such $a$ to find the maximum of $n_a/a$.

\paragraph{Result (record up to $10^8$).}
Among all totients $a$ with minimal preimage $n_a\le 10^8$, the maximum observed ratio was
\[
\max \frac{n_a}{a} \approx 2.041378860109,
\]
attained at
\[
 a=2037248,\qquad n_a=4158795=3\cdot 5\cdot 17\cdot 47\cdot 347.
\]
For this record value,
\[
\frac{n_a}{a}=\frac{4158795}{2037248}=\prod_{p\mid n_a}\frac{p}{p-1}
 =\frac32\cdot\frac54\cdot\frac{17}{16}\cdot\frac{47}{46}\cdot\frac{347}{346}.
\]

\paragraph{Top record-holders (up to $10^8$).}
The next-largest ratios observed are listed below.
Empirically, every entry has the form $n_a=3\cdot 5\cdot 17\cdot 47\cdot q$ with $q$ prime, and hence
$a=\varphi(n_a)=2\cdot 4\cdot 16\cdot 46\cdot (q-1)=5888\,(q-1)$.
\begin{center}
\begin{tabular}{r|r|r|l}
rank & $a$ & $n_a$ & $n_a/a$ \\\hline
1 & 2037248 & 4158795 & 2.0413788601\\
2 & 6983168 & 14226195 & 2.0372121937\\
3 & 8043008 & 16383495 & 2.0369860381\\
4 & 8749568 & 17821695 & 2.0368657058\\
5 & 11929088 & 24293595 & 2.0365006109\\
6 & 12423680 & 25300335 & 2.0364606139\\
7 & 14578688 & 29686845 & 2.0363180144\\
8 & 23587328 & 48023895 & 2.0360040357\\
9 & 25212416 & 51331755 & 2.0359712849\\
10 & 26236928 & 53417145 & 2.0359527228\\
\end{tabular}
\end{center}

\subsection*{6) ADVERSARIAL VERIFICATION}
\begin{itemize}
\item \textbf{Lemma $\varphi(n)\ge \sqrt n$ (except $2,6$).}
We reduced to prime powers via the multiplicativity of $g(n)=\varphi(n)^2/n$.
All prime-power factors satisfy $g(p^k)>1$ except $g(2)=1/2$.
The only way for a product of such factors to fall below $1$ is to include $g(2)$ and multiply it by an odd factor $g(m)<2$; checking odd $m$ shows only $m=1,3$ work, yielding $n=2,6$.
Boundary checks: $\varphi(2)=1<\sqrt2$ and $\varphi(6)=2<\sqrt6$ do fail, as required.
\item \textbf{Rosser--Schoenfeld bound usage.}
We only apply it for $n=n_a\ge 4$, so $\log\log n_a>0$.
The step $\log\log n_a\le \log\log(a^2)$ uses the proven inequality $n_a\le a^2$.
\item \textbf{Computation.}
The record value $n_a=3\cdot 5\cdot 17\cdot 47\cdot 347$ was independently verified by direct factorization and by checking $\varphi(n_a)=a$ via the product formula.
The computation is still finite: it does not exclude larger ratios occurring at $n_a>10^8$.
\end{itemize}

\subsection*{7) FINAL}
\textbf{UNRESOLVED (BUT STRICTLY ADVANCED).}
We proved nontrivial global structure and bounds:
\begin{itemize}
\item $\varphi(n)\ge \sqrt n$ for $n\neq 2,6$, implying $n_a\le a^2$ for every totient $a\ge 3$.
\item Using Rosser--Schoenfeld, we derived an explicit unconditional upper bound
$\displaystyle \frac{n_a}{a}<e^{\gamma}\log\log(a^2)+\frac{2.50637}{\log\log(a^2)}$, hence $n_a/a=O(\log\log a)$.
\item Computation up to $n_a\le 10^8$ finds a \emph{stable} maximum ratio $\approx 2.0413788601$ (no growth beyond the Round~1 range), with record-holders of a very rigid empirical shape.
\end{itemize}
The main Erd\H{o}s question (existence of an infinite set with $n_a/a\to\infty$) remains open.

\subsection*{8) COMPLETION ESTIMATE (MANDATORY)}
\textbf{COMPLETION: 40\%}

\subsection*{9) REFERENCES}
\begin{itemize}
\item J. B. Rosser and L. Schoenfeld, ``Approximate formulas for some functions of prime numbers'', \emph{Illinois J. Math.} \textbf{6} (1962), Theorem~15 (explicit bounds for $n/\varphi(n)$).
\item E. W. Weisstein, ``Totient Function'', MathWorld (states $\varphi(n)\ge \sqrt n$ for $n\neq 2,6$ and cites Kendall--Osborn (1965) and Mitrinovi\'c--S\'andor (1995)).
\end{itemize}
