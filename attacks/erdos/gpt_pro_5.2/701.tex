\section*{Erd\H{o}s Problem \#701}

\subsection*{FORMAL RESTATEMENT}
Let $[n]:=\{1,2,\dots,n\}$.
A family $\mathcal{F}\subseteq 2^{[n]}$ is a \emph{down-set} (hereditary family) if whenever $A\in\mathcal{F}$ and $B\subseteq A$, then $B\in\mathcal{F}$.
A subfamily $\mathcal{F}'\subseteq \mathcal{F}$ is \emph{intersecting} if every member of $\mathcal{F}'$ is nonempty and for all distinct $A,B\in\mathcal{F}'$ we have $A\cap B\ne\varnothing$.

Conjecture (Chv\'{a}tal): For every down-set $\mathcal{F}\subseteq 2^{[n]}$ there exists an element $x\in [n]$ such that for every intersecting subfamily $\mathcal{F}'\subseteq\mathcal{F}$,
\[|\mathcal{F}'|\le |\{A\in\mathcal{F}: x\in A\}|.
\]

\subsection*{QUICK LITERATURE/CONTEXT CHECK}
The problem text states Chv\'{a}tal proved the conjecture under a stronger closure condition, and that special cases have been proved (e.g. covering number $2$). A 2022/2023 paper by Frankl--Kupavskii establishes the covering number $2$ case.
I do not see a claim that the general conjecture is fully resolved as of late 2025.

\subsection*{ATTACK PLAN}
\begin{itemize}
\item \textbf{Proof track:} prove the conjecture in elementary special cases by structural decomposition via maximal elements of the down-set.
\item \textbf{Reality check:} brute-force all down-sets on $[n]$ for small $n$ and verify the conjecture by maximum-clique computation in the intersection graph.
\end{itemize}

\subsection*{WORK}
\textbf{Lemma 701.1 (unique maximal set case).}
Suppose $\mathcal{F}\subseteq 2^{[n]}$ is a down-set with a \emph{unique} maximal element (under inclusion) $M\in\mathcal{F}$. Then the conjecture holds, and in fact for any $x\in M$ and any intersecting $\mathcal{F}'\subseteq\mathcal{F}$ one has
\[|\mathcal{F}'|\le 2^{|M|-1}=|\{A\in\mathcal{F}: x\in A\}|.
\]

\emph{Proof.}
First we show $\mathcal{F}=2^{M}$. Let $A\in\mathcal{F}$. If $A\nsubseteq M$, then $A$ contains some element outside $M$. Any superset $B\supseteq A$ would also contain that element and hence cannot be a subset of $M$. But $M$ is the unique maximal element of $\mathcal{F}$, so $A$ must have a chain of strict supersets in $\mathcal{F}$ ending at $M$, which is impossible if $A\nsubseteq M$. Therefore every $A\in\mathcal{F}$ satisfies $A\subseteq M$. Since $\mathcal{F}$ is a down-set and $M\in\mathcal{F}$, it contains all subsets of $M$, hence $\mathcal{F}=2^{M}$.

Now let $\mathcal{F}'\subseteq 2^{M}$ be intersecting (in particular, all members are nonempty). For each subset $A\subseteq M$ let $\overline{A}:=M\setminus A$.
The map $A\mapsto \overline{A}$ is a fixed-point-free involution on $2^{M}$, pairing each $A$ with its complement.
In an intersecting family, one cannot contain both $A$ and $\overline{A}$, because $A\cap \overline{A}=\varnothing$.
Therefore at most one set from each complementary pair can lie in $\mathcal{F}'$, so
$|\mathcal{F}'|\le 2^{|M|-1}$.
Finally, for any fixed $x\in M$, exactly half of the subsets of $M$ contain $x$, so
$|\{A\subseteq M: x\in A\}|=2^{|M|-1}$.
This is exactly the required bound. \qed

\medskip
\textbf{Lemma 701.2 (pairwise disjoint maximal sets case).}
Let $\mathcal{F}\subseteq 2^{[n]}$ be a down-set, and let $\mathcal{G}$ be the set of maximal elements of $\mathcal{F}$.
Assume that the members of $\mathcal{G}$ are pairwise disjoint.
Then the conjecture holds.

\emph{Proof.}
For each $M\in\mathcal{G}$, let $\mathcal{F}_M:=2^{M}$.
Because $\mathcal{F}$ is a down-set, every set in $\mathcal{F}$ is a subset of some maximal element, so
\[\mathcal{F}=\bigcup_{M\in\mathcal{G}} \mathcal{F}_M.
\]
Moreover, if $M\ne M'$ are distinct maximal sets, then $M\cap M'=\varnothing$ by hypothesis, hence any $A\subseteq M$ and $B\subseteq M'$ are disjoint.
Therefore any intersecting subfamily $\mathcal{F}'\subseteq\mathcal{F}$ can involve sets from at most one block $\mathcal{F}_M$: if it contained both some $A\subseteq M$ and some $B\subseteq M'$ with $M\ne M'$, we would have $A\cap B=\varnothing$, contradicting intersecting.
Thus there is some $M\in\mathcal{G}$ such that $\mathcal{F}'\subseteq 2^{M}$.
Pick any $x\in M$. Then by Lemma 701.1 applied to the down-set $2^{M}$, we have $|\mathcal{F}'|\le 2^{|M|-1}$.
On the other hand, within the whole family $\mathcal{F}$, the star at $x$ consists precisely of the subsets of $M$ containing $x$ (no other maximal block contains $x$), so
$|\{A\in\mathcal{F}: x\in A\}|=2^{|M|-1}$.
Hence $|\mathcal{F}'|\le |\{A\in\mathcal{F}: x\in A\}|$, proving the conjecture in this case. \qed

\medskip
\textbf{FAST REALITY CHECK (computation).}
For $n\le 5$ I enumerated \emph{all} down-sets $\mathcal{F}\subseteq 2^{[n]}$ (equivalently all antichains of maximal elements; there are $3,6,20,168,7581$ of them for $n=1,2,3,4,5$) and for each:
\begin{itemize}
\item computed $\alpha(\mathcal{F})=$ the maximum size of an intersecting subfamily (using a maximum-clique algorithm on the intersection graph of nonempty members), and
\item computed $\max_{x\in[n]} |\{A\in\mathcal{F}: x\in A\}|$.
\end{itemize}
In every case for $n\le 5$ I found $\alpha(\mathcal{F})\le \max_x |\{A\in\mathcal{F}: x\in A\}|$, i.e. the conjecture holds for all down-sets on ground sets of size at most $5$.

\subsection*{VERIFICATION}
\begin{itemize}
\item Definition check: I excluded the empty set from intersecting subfamilies (otherwise the family $\{\varnothing\}$ would create a degenerate obstruction).
\item Lemma 701.1: checked the argument that unique maximal element forces $\mathcal{F}=2^M$.
\item Computation: verified enumeration count for antichains on $[5]$ equals $7581$ and ensured the clique search only used nonempty sets.
\end{itemize}

\subsection*{FINAL}
\textbf{UNRESOLVED.}
\begin{enumerate}
\item[(i)] Strongest proved partial results here:
  \begin{itemize}
  \item Conjecture holds when $\mathcal{F}$ has a unique maximal set (Lemma 701.1).
  \item Conjecture holds when maximal sets are pairwise disjoint (Lemma 701.2).
  \item Exhaustive verification for all down-sets on $[n]$ for $n\le 5$.
  \end{itemize}
\item[(ii)] First gap: extend the argument beyond these decomposable cases, i.e. handle overlapping maximal sets without additional structural assumptions.
\item[(iii)] Top 3 next moves:
  \begin{enumerate}
  \item Try to prove the conjecture for down-sets with exactly two maximal elements with nontrivial overlap by an explicit injection/compression argument.
  \item Develop an induction on $n$ using links/deletions: relate $\mathcal{F}$ to the down-sets $\{A\in\mathcal{F}: x\notin A\}$ and $\{A\setminus\{x\}: x\in A\in\mathcal{F}\}$.
  \item Search for a minimal counterexample by computation for $n=6$ restricted to families with small numbers of maximal elements (since enumerating all down-sets is infeasible at $n=6$).
  \end{enumerate}
\item[(iv)] Minimal counterexample structure: a smallest counterexample would likely be a down-set on small $n$ with several overlapping maximal sets arranged so that \emph{every} star is significantly smaller than a carefully constructed intersecting subfamily; overlaps must be rich enough to allow a large intersecting family but balanced enough to keep all stars small.
\end{enumerate}


