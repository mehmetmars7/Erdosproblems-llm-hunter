
1) \textbf{FORMAL RESTATEMENT}

Let $p_n$ denote the $n$-th prime ($p_1=2,p_2=3,\dots$).
Consider the alternating series
\[
\sum_{n=1}^\infty (-1)^n \frac{n}{p_n}.
\]
Question: Does this series converge?

The problem statement also mentions related conjectures/series involving prime gaps $g_n:=p_{n+1}-p_n$; I treat those as auxiliary.

Stress points:
\begin{itemize}
\item The term $a_n:=n/p_n$ tends to $0$ extremely slowly (heuristically $\sim 1/\log n$), and it is not monotone because small prime gaps can make $a_{n+1}>a_n$.
\item Absolute convergence fails heuristically (since $\sum 1/\log n$ diverges), but proving such comparisons requires nontrivial information about $p_n$.
\end{itemize}

2) \textbf{QUICK LITERATURE/CONTEXT CHECK}

From the problem statement: Tao proved conditional convergence under a strong Hardy--Littlewood prime tuples conjecture. Erd\H{o}s conjectured other alternating series in prime gaps; bounded gaps between primes imply that $\sum (-1)^n 1/(p_{n+1}-p_n)$ does not converge.

I do not use Tao's conditional result; I only prove elementary identities and a simple implication from ``bounded gaps'' to nonconvergence of a gap-reciprocal alternating series, plus numerical exploration of partial sums.

3) \textbf{ATTACK PLAN}

\emph{Proof track:}
\begin{itemize}
\item Try to control partial sums by summation by parts, using information about the variation of $a_n=n/p_n$.
\item Group the series in pairs and relate pair terms to prime gaps $g_n$; attempt to show the paired series is summable.
\end{itemize}

\emph{Disproof track:}
\begin{itemize}
\item Try to show partial sums are unbounded by constructing long runs where $a_n$ stays above a threshold, forcing significant drift. This likely needs deep information on primes.
\item Look for structural reasons $a_n$ oscillates too irregularly for cancellation.
\end{itemize}

4) \textbf{WORK}

\textbf{Lemma 15.1 (exact increment formula for $a_n=n/p_n$ in terms of gaps).}
Let $a_n:=n/p_n$ and $g_n:=p_{n+1}-p_n$.
Then for every $n\ge 1$,
\[
 a_{n+1}-a_n
 = \frac{p_n - n g_n}{p_n(p_n+g_n)}.
\]
In particular, $a_{n+1}>a_n$ if and only if $g_n < p_n/n$.

\textbf{Proof.}
Compute directly:
\[
 a_{n+1}-a_n
 = \frac{n+1}{p_{n+1}}-\frac{n}{p_n}
 = \frac{n+1}{p_n+g_n}-\frac{n}{p_n}
 = \frac{(n+1)p_n - n(p_n+g_n)}{p_n(p_n+g_n)}
 = \frac{p_n - n g_n}{p_n(p_n+g_n)}.
\]
The sign condition follows because the denominator is positive.
\qed

\textbf{Lemma 15.2 (bounded gaps force nonconvergence of the alternating gap-reciprocal series).}
Assume there exists a constant $H\ge 1$ such that $g_n\le H$ for infinitely many $n$.
Then the series
\[
\sum_{n=1}^\infty (-1)^n \frac{1}{g_n}
\]
cannot converge.

\textbf{Proof.}
A necessary condition for convergence of any series $\sum b_n$ is that $b_n\to 0$.
Here $b_n:=(-1)^n\frac{1}{g_n}$ satisfies $|b_n|=1/g_n$.
If $g_n\le H$ for infinitely many $n$, then along that infinite subsequence,
\[
|b_n|=\frac{1}{g_n}\ge \frac{1}{H}.
\]
Therefore $b_n$ does not tend to $0$, so the series cannot converge.
\qed

\textbf{FAST REALITY CHECK / COMPUTATION (partial sums for the main series).}
I computed partial sums
\[
S(N):=\sum_{n=1}^N (-1)^n\frac{n}{p_n}
\]
using exact primes from a sieve.
Results:
\begin{itemize}
\item $S(200{,}000)\approx -0.0157664606$.
\item $S(1{,}000{,}000)\approx -0.0198592163$.
\item Over $1\le N\le 1{,}000{,}000$, the observed minimum was $S(1)=-1/2$ and the observed maximum was $S(2)=1/6$.
\item The last few values at $N=999{,}991$ to $1{,}000{,}000$ were:
\[
\begin{array}{c|c}
N & S(N)\\\hline
999991 & -0.0844341087\\
999992 & -0.0198592244\\
999993 & -0.0844341316\\
999994 & -0.0198592349\\
999995 & -0.0844341711\\
999996 & -0.0198592954\\
999997 & -0.0844342107\\
999998 & -0.0198592559\\
999999 & -0.0844342419\\
1000000 & -0.0198592163
\end{array}
\]
Note the even/odd subsequences differ by the current term magnitude $\approx 0.0646$, which is consistent with the fact that $n/p_n\to 0$ only slowly.
\end{itemize}

5) \textbf{VERIFICATION}

\begin{itemize}
\item Lemma~15.1 is pure algebra and holds exactly.
\item Lemma~15.2 uses only the necessary condition $b_n\to 0$ for convergence.
\item Computation: primes up to $p_{1{,}000{,}000}=15485863$ were generated by a standard sieve; the reported sums are floating-point approximations of rational partial sums.
\end{itemize}

6) \textbf{FINAL}

\textbf{UNRESOLVED}

(i) Strongest proved partial result: Lemma~15.1 gives an exact relationship between the variation of $a_n=n/p_n$ and prime gaps. Lemma~15.2 gives a rigorous implication: any infinite occurrence of bounded gaps forces nonconvergence of the alternating gap-reciprocal series mentioned in the problem statement. Numerically, $S(N)$ for the main series appears bounded and slowly drifting, with $S(10^6)\approx -0.01986$.

(ii) First gap (crisp): Prove convergence or divergence of
\[
\sum_{n=1}^\infty (-1)^n\frac{n}{p_n}
\]
unconditionally.
A concrete analytic obstacle is to control partial sums without assuming a strong prime tuples conjecture.

(iii) Top 3 next moves:
\begin{enumerate}
\item Try to bound the total variation $\sum_n |a_{n+1}-a_n|$ using Lemma~15.1 and information about the distribution of gaps $g_n$; if this sum converged, it would imply convergence of $\sum (-1)^n a_n$ by a standard summation-by-parts argument.
\item Study the paired series $\sum_{m\ge 1}\big( \frac{2m}{p_{2m}}-\frac{2m-1}{p_{2m-1}}\big)$ and attempt to show it is summable; Lemma~15.1 suggests each paired term depends on a prime gap at odd indices.
\item Extend computations (e.g. to $N=10^7$) and track the envelopes of even and odd partial sums to see whether they appear to converge to a common limit; simultaneously compute statistics of when $a_{n+1}>a_n$ to relate to small-gap frequencies.
\end{enumerate}

(iv) Minimal counterexample structure: If the series diverges, a ``minimal'' way this could happen is that the even and odd partial sums drift apart and fail to approach a common limit, despite the term sizes $a_n\to 0$. By Lemma~15.1, increases in $a_n$ are driven by unusually small gaps $g_n<p_n/n$; thus any divergence mechanism would likely correlate with frequent small prime gaps producing systematic bias in the alternating weighting.
