% Erdos Problem #617
% URL: https://www.erdosproblems.com/617

FORMAL RESTATEMENT
Fix an integer $r\ge3$. Consider an $r$-edge-coloring of the complete graph
$K_{r^2+1}$.
The conjecture asks whether there must exist a set $S$ of $r+1$ vertices such
that the induced $K_{r+1}$ on $S$ is missing at least one of the $r$ colors.

Equivalently, there is \emph{no} ``balanced'' $r$-coloring of $K_{r^2+1}$ in which
\emph{every} $K_{r+1}$ uses all $r$ colors.

QUICK LITERATURE/CONTEXT CHECK
I do not use external sources; I only use the statements recorded in the problem
file.
The file states:
\begin{itemize}
\item The statement is false for $r=2$.
\item Erd\H{o}s and Gy\'arf\'as showed it becomes false for infinitely many $r$ if
$K_{r^2+1}$ is replaced by $K_{r^2}$.
\item The conjecture has been proved for $r=3$ and $r=4$.
\end{itemize}

ATTACK PLAN
Translate the ``missing a color'' condition into graph properties of each color
class.
Then:
\begin{itemize}
\item Produce explicit balanced colorings for the variant $K_{r^2}$ (to match the
known failure of that variant).
\item Derive necessary constraints for any putative balanced coloring of
$K_{r^2+1}$.
\end{itemize}
Also check small cases computationally where feasible.

WORK
\textbf{Lemma 1 (balanced $\Leftrightarrow$ small independence number in each color class).}
Let the edges of $K_n$ be colored with colors $\{1,\dots,r\}$.
For a fixed color $c$, let $H_c$ be the spanning subgraph whose edges are exactly
those of color $c$.
Then the coloring is balanced (i.e., every $(r+1)$-set of vertices contains at
least one edge of each color) if and only if
\[
\alpha(H_c)\le r\quad\text{for every }c\in\{1,\dots,r\}.
\]

\emph{Proof.}
Fix a color $c$.
A set $S$ of $r+1$ vertices has \emph{no} edges of color $c$ if and only if $S$ is
an independent set in the graph $H_c$.
Therefore the condition ``every $(r+1)$-set has at least one $c$-colored edge'' is
exactly the condition ``$H_c$ has no independent set of size $r+1$'', i.e.
$\alpha(H_c)\le r$.
Imposing this for every color $c$ is equivalent to balancedness.
\hfill $\square$

\medskip
\textbf{Lemma 2 (explicit balanced $3$-coloring of $K_9$ for the $r^2$ variant).}
There exists a $3$-coloring of the edges of $K_9$ in which \emph{every} $K_4$
contains all $3$ colors.
In particular, the conjectured statement becomes false if one replaces
$K_{r^2+1}$ by $K_{r^2}$, already for $r=3$.

\emph{Proof (explicit construction + exhaustive verification).}
Label the vertices by $\{0,1,2,3,4,5,6,7,8\}$.
Color the $36$ edges of $K_9$ with colors $0,1,2$ as follows.

\underline{Color $0$ (18 edges):}
\[
\{01,04,05,06,12,13,14,23,26,28,35,37,47,48,56,57,68,78\}.
\]
\underline{Color $1$ (9 edges):}
\[
\{02,07,15,18,27,34,36,46,58\}.
\]
\underline{Color $2$ (9 edges):}
\[
\{03,08,16,17,24,25,38,45,67\}.
\]
(Here $ij$ denotes the edge $\{i,j\}$.)

I verified by an exhaustive script that for every $4$-subset $S\subseteq\{0,\dots,8\}$
(there are ${9\choose 4}=126$ of them), the $6$ edges induced by $S$ contain at
least one edge of each of the three colors.
Therefore this is a balanced $3$-coloring of $K_9$.
\hfill $\square$

VERIFICATION
\textbf{FAST REALITY CHECK 1 (the stated $r=2$ failure).}
For $r=2$, balanced means ``no monochromatic triangle''.
Label $K_5$ vertices $0,1,2,3,4$ in cyclic order.
Color the $5$ cycle edges
$01,12,23,34,40$ red and the $5$ diagonals blue.
Every triangle in $K_5$ contains at least one cycle edge and at least one diagonal,
so it uses both colors. Hence the conjecture would indeed be false if one
allowed $r=2$.

\textbf{FAST REALITY CHECK 2 (the $r^2$ variant for $r=3$).}
The coloring in Lemma 2 was checked exhaustively over all $126$ induced $K_4$'s;
no exceptions were found.

FINAL
**UNRESOLVED**
(i) Strongest proved partial results here: the structural equivalence in Lemma 1,
and an explicit balanced coloring for the $r^2$-vertex variant at $r=3$ (Lemma 2),
matching the problem file's statement that the $r^2$ variant fails for infinitely
many $r$.

(ii) First gap (crisp): prove (or disprove) that for every $r\ge3$ there is \emph{no}
balanced $r$-coloring of $K_{r^2+1}$ (i.e., show that some $(r+1)$-set is missing a
color in every $r$-coloring).

(iii) Top 3 next moves:
1. Use Lemma 1 to translate the conjecture into constraints on an edge-partition
of $K_{r^2+1}$ into $r$ graphs $H_c$ each having $\alpha(H_c)\le r$; attempt to show
such a partition is impossible at $n=r^2+1$.
2. Develop a counting argument on $(r+1)$-subsets:
for each color $c$, count how many $(r+1)$-sets are independent in $H_c$ and show
that the sum over colors must be positive.
3. Computational exploration for the next open values of $r$ (beyond those known
in the file), searching either for a balanced coloring (as a disproof) or for a
certificate that none exists.

(iv) Minimal counterexample structure:
A minimal counterexample to the conjecture would be a balanced coloring of
$K_{r^2+1}$.
Equivalently, it would be a partition of the edges into $r$ color-class graphs
$H_1,\dots,H_r$ on $n=r^2+1$ vertices with $\alpha(H_c)\le r$ for every $c$.
Such an object would likely have strong algebraic symmetry (similar in flavor to
finite-geometry constructions) and would in particular give an extremal example
for the independence constraints of Lemma 1.


