
1) “FORMAL RESTATEMENT”

Fix integers $k>r>2$. Let $K_k^r$ denote the complete $r$-uniform hypergraph on $k$ vertices (i.e. every $r$-subset of a fixed $k$-set is an edge).

For each $n\ge k$, let $\mathrm{ex}_r(n,K_k^r)$ be the maximum number of $r$-edges in an $r$-uniform hypergraph on vertex set $[n]:=\{1,\dots,n\}$ that contains no copy of $K_k^r$ as a (not-necessarily-induced) subhypergraph.

The problem asks to determine
\[
\lim_{n\to\infty}\frac{\mathrm{ex}_r(n,K_k^r)}{\binom{n}{r}}\qquad\text{(if the limit exists),}
\]
for each fixed $k>r>2$.

Stress points: existence of the limit is not automatic; even when it exists, identifying the optimal construction is difficult.

2) “QUICK LITERATURE/CONTEXT CHECK”

Only what is explicitly stated in the problem text:

- For $r=2$ (graphs), Tur\'{a}n proved the limit equals $\frac12\left(1-\frac1{k-1}\right)$.

Per integrity rules, I do not use any other external results.

3) “ATTACK PLAN”

Proof track ideas:

- Lower bounds from explicit constructions (balanced $(k-1)$-partite $r$-graphs, or other structured designs).
- Upper bounds by counting missing edges per $k$-set, stability-type arguments, or flag algebra computations (not carried out here).

Disproof/construction track: search small $n$ by brute force to see which constructions appear extremal.

4) “WORK”

\textbf{FAST REALITY CHECK (small exact computations).}

For $r=3$ I computed by brute force the exact values for small $n$:

- For $K_4^3$ (the 3-uniform tetrahedron),
\[
\mathrm{ex}_3(4,K_4^3)=3\ (\text{out of }\binom43=4),\quad
\mathrm{ex}_3(5,K_4^3)=7\ (\text{out of }10),\quad
\mathrm{ex}_3(6,K_4^3)=14\ (\text{out of }20).
\]

- For $K_5^3$,
\[
\mathrm{ex}_3(5,K_5^3)=9\ (\text{out of }10),\quad
\mathrm{ex}_3(6,K_5^3)=18\ (\text{out of }20).
\]

These are sanity checks only; small $n$ behavior can differ from the asymptotic density.

\medskip
\textbf{Proposition 1 (Explicit lower bound via $(k-1)$-partite construction).}
Let $k>r\ge 2$ be fixed. Partition $[n]$ into $k-1$ parts $V_1\cup\cdots\cup V_{k-1}$ (as equal as possible), and let $H$ be the complete $(k-1)$-partite $r$-uniform hypergraph: an $r$-set $e$ is an edge iff it meets each $V_i$ in at most one vertex (equivalently, all $r$ vertices lie in distinct parts).

Then $H$ is $K_k^r$-free. Moreover, writing $|V_i|=n_i$ with $\sum_i n_i=n$,
\[
|E(H)|=\sum_{1\le i_1<\cdots<i_r\le k-1} n_{i_1}n_{i_2}\cdots n_{i_r}.
\]
In particular, for balanced parts $n_i\in\{\lfloor n/(k-1)\rfloor,\lceil n/(k-1)\rceil\}$,
\[
\frac{|E(H)|}{\binom{n}{r}} \to \frac{r!\binom{k-1}{r}}{(k-1)^r}
\qquad (n\to\infty),
\]
so
\[
\liminf_{n\to\infty}\frac{\mathrm{ex}_r(n,K_k^r)}{\binom{n}{r}}\ge \frac{r!\binom{k-1}{r}}{(k-1)^r}.
\]

\emph{Proof.}
First, $H$ is $K_k^r$-free: take any set $S$ of $k$ vertices. Since there are only $k-1$ parts, by the pigeonhole principle two vertices of $S$ lie in the same part, say $V_j$.
Choose any other $r-2$ vertices of $S$ (possible since $r\le k-1$). Then the resulting $r$-set contains two vertices from $V_j$, hence is \emph{not} an edge of $H$. Therefore $S$ does not span all $\binom{k}{r}$ edges, so no copy of $K_k^r$ exists.

Second, to count edges: an edge of $H$ is determined by choosing $r$ distinct parts $V_{i_1},\dots,V_{i_r}$ and then choosing one vertex from each part, giving $n_{i_1}\cdots n_{i_r}$ choices. Summing over $\binom{k-1}{r}$ choices of parts gives the formula.

Finally, for balanced parts $n_i\sim n/(k-1)$, each summand $n_{i_1}\cdots n_{i_r}\sim (n/(k-1))^r$, so
\[
|E(H)|\sim \binom{k-1}{r}\left(\frac{n}{k-1}\right)^r.
\]
Since $\binom{n}{r}\sim \frac{n^r}{r!}$, dividing gives the stated limiting ratio.
\qed

\medskip
\textbf{Proposition 2 (Simple universal upper bound by counting missing edges).}
For all $n\ge k$,
\[
\mathrm{ex}_r(n,K_k^r)\le \left(1-\frac{1}{\binom{k}{r}}\right)\binom{n}{r}.
\]
Equivalently,
\[
\frac{\mathrm{ex}_r(n,K_k^r)}{\binom{n}{r}}\le 1-\frac{1}{\binom{k}{r}}.
\]

\emph{Proof.}
Let $H$ be any $K_k^r$-free $r$-graph on $[n]$ with edge set $E(H)$. Let $M$ be the set of \emph{missing} $r$-edges:
\[
M:=\binom{[n]}{r}\setminus E(H),\qquad |M|=\binom{n}{r}-|E(H)|.
\]
Because $H$ is $K_k^r$-free, every $k$-subset $S\in\binom{[n]}{k}$ misses at least one of its $\binom{k}{r}$ $r$-subsets, i.e. there exists at least one $e\in M$ with $e\subseteq S$.

Count pairs $(S,e)$ where $S\in\binom{[n]}{k}$ and $e\in M$ with $e\subseteq S$.

- Lower bound: each $k$-set $S$ contributes at least one such missing $e$, so the number of pairs is at least $\binom{n}{k}$.

- Upper bound: each missing $r$-set $e\in M$ is contained in exactly $\binom{n-r}{k-r}$ different $k$-sets $S$ (choose the remaining $k-r$ vertices from the $n-r$ outside $e$). Thus the number of pairs is at most
\[
|M|\binom{n-r}{k-r}.
\]

Therefore
\[
\binom{n}{k}\le |M|\binom{n-r}{k-r}.
\]
A direct factorial manipulation gives
\[
\frac{\binom{n}{k}}{\binom{n-r}{k-r}} = \frac{\binom{n}{r}}{\binom{k}{r}}.
\]
(Indeed, both sides equal $\frac{n!}{r!(n-r)!}\cdot\frac{r!(k-r)!}{k!} = \frac{n!(k-r)!}{k!(n-r)!}$.)
So
\[
|M| \ge \frac{\binom{n}{k}}{\binom{n-r}{k-r}} = \frac{\binom{n}{r}}{\binom{k}{r}}.
\]
Hence
\[
|E(H)| = \binom{n}{r}-|M| \le \binom{n}{r}\left(1-\frac{1}{\binom{k}{r}}\right).
\]
Taking the maximum over all such $H$ gives the claim.
\qed

5) “VERIFICATION”

- Proposition 1: The $K_k^r$-freeness argument uses only pigeonhole and the explicit definition of $(k-1)$-partite $r$-edges; checked that it indeed produces a missing $r$-subset inside any $k$-set.
- Proposition 2: Double counting verified; the identity $\binom{n}{k}/\binom{n-r}{k-r}=\binom{n}{r}/\binom{k}{r}$ was checked algebraically.
- Computations: brute force enumerations were run only for small parameters where $2^{\binom{n}{r}}$ is feasible.

6) FINAL

**UNRESOLVED**

(i) Strongest proved partial result here: explicit lower bound
\[
\liminf_{n\to\infty}\frac{\mathrm{ex}_r(n,K_k^r)}{\binom{n}{r}}\ge \frac{r!\binom{k-1}{r}}{(k-1)^r}
\]
from the balanced $(k-1)$-partite construction, and a universal upper bound
\[
\frac{\mathrm{ex}_r(n,K_k^r)}{\binom{n}{r}}\le 1-\frac{1}{\binom{k}{r}}.
\]

(ii) First gap (crisp statement): determine the exact value of
\[
\pi_r(K_k^r):=\lim_{n\to\infty}\frac{\mathrm{ex}_r(n,K_k^r)}{\binom{n}{r}}
\]
(or show the limit exists) for each fixed $k>r>2$.

(iii) Top 3 next moves (concrete targets):

1. Compute $\mathrm{ex}_r(n,K_k^r)$ exactly for the next few $n$ beyond brute force (e.g. via MILP/branch-and-bound) to identify candidate extremal constructions.

2. Prove a stability statement: if an $r$-graph is $K_k^r$-free and has edge density within $o(1)$ of the conjectured optimum, then its vertex set admits an almost $(k-1)$-partite structure.

3. Improve Proposition 2 by weighting $k$-sets and missing edges more carefully (e.g. count \emph{how many} missing edges each $k$-set forces on average) to get an upper bound closer to known constructions.

(iv) Minimal counterexample structure (if the balanced $(k-1)$-partite density is not optimal): one expects a family of $K_k^r$-free $r$-graphs with density strictly larger than $r!\binom{k-1}{r}/(k-1)^r$ whose local structure is not close to a $(k-1)$-partition, likely involving nontrivial designs where many $k$-sets miss different $r$-edges in a highly non-uniform way.


