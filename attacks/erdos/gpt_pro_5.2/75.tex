% Erd\H{o}s Problem #75

\noindent\textbf{FORMAL RESTATEMENT.}
Does there exist a (simple, undirected) graph $G$ with chromatic number $\chi(G)=\aleph_1$ such that the following holds?

For every real $\epsilon>0$ there exists an integer $N_\epsilon$ such that for all integers $n\ge N_\epsilon$ and every subgraph $H$ of $G$ with $\lvert V(H)\rvert=n$, the graph $H$ contains an independent set of size strictly larger than $n^{1-\epsilon}$.

(Recall: an independent set is a vertex set spanning no edges. Also, since removing edges can only \emph{increase} the independence number, it suffices to check the property for induced subgraphs on $n$ vertices.)

\medskip
\noindent\textbf{QUICK LITERATURE/CONTEXT CHECK.}
The problem statement says this is conjectured by Erd\H{o}s--Hajnal--Szemer\'{e}di. It also notes Erd\H{o}s suggested the independent set might even be of linear size $\gg n$. No additional results are used here.

\medskip
\noindent\textbf{ATTACK PLAN.}
\emph{Proof track ideas.}
(1) Try to build an $\aleph_1$-chromatic graph via a set-theoretic construction while enforcing strong local sparsity (few edges on every finite vertex set), since large independent sets force sparsity.
(2) Seek a transfinite inductive construction ensuring that every $n$-vertex induced subgraph has a large independent set, while globally preventing a countable coloring.

\emph{Disproof track ideas.}
(1) Show that the local condition implies $G$ must be countably chromatic (i.e. admits an $\aleph_0$-coloring), contradicting $\chi(G)=\aleph_1$.

I did not reach a complete proof or counterexample. The work below records deterministic implications of the local independent-set condition.

\medskip
\noindent\textbf{WORK.}

\medskip
\noindent\textbf{Lemma 75.1 (independence forces small finite chromatic number).}
Let $H$ be a finite graph with $n:=\lvert V(H)\rvert$ and independence number $\alpha(H)$. Then
\[\chi(H)\ge \frac{n}{\alpha(H)}.\]
In particular, if $\alpha(H)>n^{1-\epsilon}$ then $\chi(H)<n^{\epsilon}$.

\noindent\textbf{Proof.}
A proper coloring of $H$ with $\chi(H)$ colors partitions $V(H)$ into $\chi(H)$ independent color classes. Therefore at least one color class has size at least $n/\chi(H)$. By definition of $\alpha(H)$, we have $\alpha(H)\ge n/\chi(H)$, which rearranges to $\chi(H)\ge n/\alpha(H)$. If $\alpha(H)>n^{1-\epsilon}$ then $\chi(H)\ge n/\alpha(H)< n/n^{1-\epsilon}=n^{\epsilon}$. Since $\chi(H)$ is an integer, this implies $\chi(H)\le \lfloor n^{\epsilon}\rfloor$ and in particular $\chi(H)<n^{\epsilon}$.
\hfill$\square$

\medskip
\noindent\textbf{Lemma 75.2 (Caro--Wei bound and an edge upper bound).}
Let $H$ be a finite graph with $n$ vertices, $m$ edges, and degrees $d(v)$. Then
\[
\alpha(H)\ge \sum_{v\in V(H)}\frac{1}{d(v)+1}\ge \frac{n^2}{2m+n}.
\]
Consequently, if $\alpha(H)>n^{1-\epsilon}$ then necessarily
\[
m<\frac{n^{1+\epsilon}-n}{2}.
\]

\noindent\textbf{Proof.}
\emph{Step 1 (Caro--Wei inequality).}
Choose a uniformly random permutation $\pi$ of the vertex set $V(H)$. Define a random vertex subset $I(\pi)$ by
\[
I(\pi):=\{v\in V(H): \pi(v)<\pi(u)\text{ for every neighbor }u\text{ of }v\}.
\]
If $v$ and $w$ are adjacent, they cannot both be earliest among their own neighborhoods (each would require being earlier than the other), so $I(\pi)$ is an independent set.

For a fixed vertex $v$ with degree $d(v)$, consider the set consisting of $v$ and its $d(v)$ neighbors, which has size $d(v)+1$. In a uniformly random permutation, each of these $d(v)+1$ vertices is equally likely to be the earliest among this set, so
\[
\mathbb P(v\in I(\pi))=\frac{1}{d(v)+1}.
\]
Taking expectations and using linearity of expectation,
\[
\mathbb E\lvert I(\pi)\rvert = \sum_{v\in V(H)} \mathbb P(v\in I(\pi)) = \sum_{v\in V(H)}\frac{1}{d(v)+1}.
\]
Since $I(\pi)$ is always an independent set, $\lvert I(\pi)\rvert\le \alpha(H)$ for every $\pi$, hence $\alpha(H)\ge \mathbb E\lvert I(\pi)\rvert$, proving
\[\alpha(H)\ge \sum_v \frac{1}{d(v)+1}.
\]

\emph{Step 2 (convert to an edge bound).}
The function $x\mapsto 1/(x+1)$ is convex on $[0,\infty)$ (its second derivative is $2/(x+1)^3>0$), so by Jensen's inequality,
\[
\frac{1}{n}\sum_{v}\frac{1}{d(v)+1}\ge \frac{1}{\big(\frac{1}{n}\sum_v d(v)\big)+1}.
\]
But $\sum_v d(v)=2m$, so the right-hand side equals $1/(2m/n+1)=n/(2m+n)$. Multiplying by $n$ gives
\[
\sum_v \frac{1}{d(v)+1}\ge \frac{n^2}{2m+n}.
\]
Combining with Step 1 yields the stated inequalities.

Finally, if $\alpha(H)>n^{1-\epsilon}$, then in particular $\alpha(H)\ge n^{1-\epsilon}+\delta$ for some $\delta>0$, so the inequality $\alpha(H)\ge n^2/(2m+n)$ forces
\[
\frac{n^2}{2m+n}< n^{1-\epsilon} \quad\Longrightarrow\quad 2m+n> n^{1+\epsilon}\quad\Longrightarrow\quad m>\frac{n^{1+\epsilon}-n}{2}.
\]
Taking the contrapositive yields: if $m\ge (n^{1+\epsilon}-n)/2$ then $\alpha(H)\le n^{1-\epsilon}$. Equivalently, if $\alpha(H)>n^{1-\epsilon}$ then $m< (n^{1+\epsilon}-n)/2$, as claimed.
\hfill$\square$

\medskip
\noindent\textbf{FAST REALITY CHECK (small parameter sanity).}
The condition $\alpha(H)>n^{1-\epsilon}$ is asymptotic in $n$. For small $n$, the inequality is easily met for many graphs; for instance, for $n=1,2$ every graph has an independent set of size at least $1>n^{1-\epsilon}$ for any $\epsilon>0$. The nontrivial content is that \emph{every} large finite subgraph must have very large independent sets; Lemma 75.2 shows this forces every such subgraph to have at most about $\frac12 n^{1+\epsilon}$ edges for each fixed $\epsilon>0$.

\medskip
\noindent\textbf{VERIFICATION.}
\begin{itemize}
\item Lemma 75.1: checked that the argument uses only the pigeonhole principle on color classes and does not assume anything about $H$ beyond finiteness.
\item Lemma 75.2: verified that $I(\pi)$ is independent, computed $\mathbb P(v\in I(\pi))$ exactly as $1/(d(v)+1)$, and applied Jensen correctly using $\sum_v d(v)=2m$.
\item Inequality directions: note that large $\alpha(H)$ implies an \emph{upper} bound on $m$ via the contrapositive of $\alpha(H)\ge n^2/(2m+n)$; the final displayed implication is stated in that direction.
\end{itemize}

\medskip
\noindent\textbf{FINAL: \textbf{UNRESOLVED}.}
\begin{itemize}
\item[(i)] \emph{Strongest proved partial result.} The local requirement $\alpha(H)>n^{1-\epsilon}$ forces strong local sparsity: every $n$-vertex subgraph has at most $<\frac12(n^{1+\epsilon}-n)$ edges (Lemma 75.2), and consequently has chromatic number $<n^{\epsilon}$ (Lemma 75.1).
\item[(ii)] \emph{First gap (crisp).} Decide whether there exists an $\aleph_1$-chromatic graph $G$ in which \emph{every} finite induced subgraph on $n$ vertices has at most $O(n^{1+\epsilon})$ edges (for each fixed $\epsilon>0$) while still preventing any countable coloring of $G$.
\item[(iii)] \emph{Top 3 next moves.}
(1) Try to construct $G$ by transfinite recursion on $\omega_1$, at each stage adding a new vertex with carefully controlled edges to earlier vertices so that every finite set stays sparse.
(2) Attempt a disproof via a decomposition theorem: show that graphs in which every finite subgraph has $O(n^{1+\epsilon})$ edges for all $\epsilon>0$ must be countably chromatic.
(3) Explore whether known uncountably chromatic constructions (e.g. comparability graphs from set theory) can be modified to satisfy the edge bound from Lemma 75.2.
\item[(iv)] \emph{Minimal counterexample structure.} A minimal failure of the conjecture would be a graph $G$ with $\chi(G)=\aleph_1$ such that for some fixed $\epsilon_0>0$ there exist arbitrarily large $n$ and $n$-vertex subgraphs $H$ with $\alpha(H)\le n^{1-\epsilon_0}$. By Lemma 75.2 such an $H$ would necessarily have at least $\gtrsim \tfrac12 n^{1+\epsilon_0}$ edges, so any counterexample mechanism must force moderately dense finite subgraphs to appear.
\end{itemize}
