
1) FORMAL RESTATEMENT

Fix an integer $n\ge 2$.

Definition (admissible interval factorisation).
For integers $t\ge 2$ and $m\ge 1$, we say that $n!$ has an admissible $(t,m)$-factorisation if there exist integers
\[
1\le a_1 < a_2 < \cdots < a_t
\]
with
\[
a_t=a_1+m\quad\text{and}\quad n! = \prod_{i=1}^t a_i.
\]
Equivalently, there exist integers $A\ge 1$ and $m\ge 1$ and a subset $S\subseteq\{A,A+1,\dots,A+m\}$ containing both endpoints $A$ and $A+m$ such that
\[
\prod_{s\in S} s = n!.
\]

Definition (the function $f$).
For $n\ge 2$ define
\[
f(n):=\min\{m\ge 1:\; n!\text{ has an admissible }(t,m)\text{-factorisation for some }t\ge 2\}.
\]

The problem asks for the asymptotic behaviour of $f(n)$ as $n\to\infty$, and in particular whether $f(n)=1$ for infinitely many $n$.

For $m\ge 1$ and $N\ge 2$, define
\[
F_m(N):=|\{n\le N:\ f(n)=m\}|.
\]

Edge cases.
For $n=1$, the definition as stated has no solution because $1!=1$ cannot be written as a product of at least two distinct positive integers; throughout I treat $f$ as defined on $n\ge 2$.

2) QUICK LITERATURE/CONTEXT CHECK

I do not introduce any results beyond what is explicitly recorded in the problem statement.

The problem statement records:
- Erd\H{o}s and Graham did not know whether $f(n)=1$ infinitely often.
- For each fixed $m$, Berend--Osgood proved $F_m(N)=o(N)$.
- Bui--Pratt--Zaharescu proved $F_m(N)\ll_m N^{33/34}$.
- Luca proved (conditional on ABC) that $f(n)\to\infty$.

3) ATTACK PLAN

Proof-track ideas (partial results only here):
- Derive unavoidable constraints on any admissible factorisation (e.g. how large $a_t$ must be).
- Give general constructions providing unconditional upper bounds on $f(n)$.
- Reality-check small $n$ by brute force to see what $f(n)$ looks like.

Disproof-track for specific subquestions:
- Try to falsify “$f(n)=1$ infinitely often” by showing $f(n)\neq 1$ for a long initial range (computation only, not a proof).

Chosen path: (i) brute force small $n$, (ii) prove a few unconditional structural lemmas.

4) WORK

PHASE 1 — FAST REALITY CHECK (computed)

I brute-forced $f(n)$ for $2\le n\le 15$ by searching over short intervals $[A,A+m]$ and subsets whose product equals $n!$.
The minimal $m=f(n)$ and one witnessing factor set $\{a_1,\dots,a_t\}$ are:

- $f(2)=1$ via $2!=1\cdot 2$ (interval $[1,2]$).
- $f(3)=1$ via $3!=2\cdot 3$ (interval $[2,3]$).
- $f(4)=2$ via $4!=2\cdot 3\cdot 4$ (interval $[2,4]$).
- $f(5)=2$ via $5!=4\cdot 5\cdot 6$ (interval $[4,6]$).
- $f(6)=2$ via $6!=8\cdot 9\cdot 10$ (interval $[8,10]$).
- $f(7)=2$ via $7!=70\cdot 72$ (interval $[70,72]$).
- $f(8)=4$ via $8!=12\cdot 14\cdot 15\cdot 16$ (interval $[12,16]$).
- $f(9)=6$ via $9!=6\cdot 7\cdot 8\cdot 9\cdot 10\cdot 12$ (interval $[6,12]$).
- $f(10)=7$ via $10!=9\cdot 10\cdot 12\cdot 14\cdot 15\cdot 16$ (interval $[9,16]$).
- $f(11)=6$ via $11!=30\cdot 32\cdot 33\cdot 35\cdot 36$ (interval $[30,36]$).
- $f(12)=9$ via $12!=24\cdot 25\cdot 27\cdot 28\cdot 32\cdot 33$ (interval $[24,33]$).
- $f(13)=9$ via $13!=39\cdot 40\cdot 42\cdot 44\cdot 45\cdot 48$ (interval $[39,48]$).
- $f(14)=9$ via $14!=63\cdot 64\cdot 65\cdot 66\cdot 70\cdot 72$ (interval $[63,72]$).
- $f(15)=12$ via $15!=16\cdot 18\cdot 20\cdot 21\cdot 22\cdot 25\cdot 26\cdot 27\cdot 28$ (interval $[16,28]$).

(These are exact outputs of the script; of course they do not address asymptotics.)

Lemma 1 (trivial universal upper bound).
For every $n\ge 3$ we have $f(n)\le n-2$.

Proof.
For $n\ge 3$,
\[
 n! = 2\cdot 3\cdot 4\cdots n.
\]
Set $t=n-1$ and $a_i=i+1$ for $1\le i\le n-1$. Then $a_1=2$, $a_t=n$, and $a_1<a_2<\cdots<a_t$ are distinct positive integers with
\[
\prod_{i=1}^{t} a_i = 2\cdot 3\cdots n = n!.
\]
The interval length is $m=a_t-a_1=n-2\ge 1$, so this is an admissible factorisation with parameter $m=n-2$. By definition of $f(n)$ as the minimal such $m$, we get $f(n)\le n-2$. \qed

Lemma 2 (a necessary size constraint on the top factor).
Let $n\ge 2$ and suppose $n!=\prod_{i=1}^t a_i$ is an admissible factorisation with $a_1<\cdots<a_t$ and $a_t=a_1+m$. Then $a_t\ge n$.

Proof.
All the integers $a_1,\dots,a_t$ are distinct and satisfy $1\le a_i\le a_t$. Hence the set $\{a_1,\dots,a_t\}$ is a subset of $\{1,2,\dots,a_t\}$.
Therefore
\[
\prod_{i=1}^t a_i \le \prod_{j=1}^{a_t} j = a_t!.
\]
Since $\prod_{i=1}^t a_i = n!$, we have $n!\le a_t!$.
The factorial function is strictly increasing on positive integers, so $n!\le a_t!$ implies $n\le a_t$. \qed

Lemma 3 (characterisation of the case $f(n)=1$).
For $n\ge 2$, $f(n)=1$ if and only if there exists an integer $x\ge 1$ such that
\[
 n! = x(x+1).
\]

Proof.
($\Rightarrow$) If $f(n)=1$, then there is an admissible factorisation with $m=1$. In that factorisation, $a_t=a_1+1$. Since all $a_i$ are distinct and lie in the interval $[a_1,a_1+1]$, the only possibility is $t=2$ and $\{a_1,a_2\}=\{x,x+1\}$ for $x=a_1$. Thus $n!=x(x+1)$.

($\Leftarrow$) Conversely, if $n!=x(x+1)$ for some integer $x\ge 1$, then taking $t=2$, $a_1=x$, $a_2=x+1$ gives an admissible factorisation with $m=1$, hence $f(n)\le 1$. Since $f(n)\ge 1$ by definition, $f(n)=1$. \qed

5) VERIFICATION

- Lemma 1: For $n=3$, the construction is $3!=2\cdot 3$ with $m=1$; for $n=4$, $4!=2\cdot 3\cdot 4$ with $m=2$, consistent.
- Lemma 2 against computed data: in every computed witness above, the largest factor $a_t$ is indeed $\ge n$.
- Lemma 3: the computed values show $f(2)=f(3)=1$ with $2!=1\cdot 2$ and $3!=2\cdot 3$, and no $f(n)=1$ for $4\le n\le 15$.

6) FINAL

**UNRESOLVED**

(i) Strongest fully proved partial result obtained here.
- Unconditional bounds: for all $n\ge 3$, $f(n)\le n-2$ (Lemma 1), and in any admissible factorisation the top factor satisfies $a_t\ge n$ (Lemma 2).
- Exact small values by brute force: $f(n)$ for $2\le n\le 15$ listed in WORK.
- Exact equivalence: $f(n)=1$ iff $n!=x(x+1)$ for some integer $x$ (Lemma 3).

(ii) Exact first gap.
Give any unconditional asymptotic information beyond the trivial upper bound; in particular, prove or disprove that $f(n)=1$ for infinitely many $n$, or (even weaker) prove unconditionally that $f(n)\to\infty$.

(iii) Top 3 next moves (concrete targets).
1. For each fixed $m$, understand the Diophantine constraints on representing $n!$ as a product of divisors in an interval of length $m$; even an explicit, effective upper bound on $F_m(N)$ would be progress.
2. Search systematically (computationally) for additional $n$ with $f(n)=2$ or small $m$, and look for structural patterns in the witnessing intervals.
3. Try to prove an unconditional lower bound of the form $f(n)\ge c\log n$ or $f(n)\ge n^{c}$ for infinitely many $n$ by using prime-factor “spacing” constraints on short-interval divisor products.

(iv) What a minimal counterexample would likely look like.
A counterexample to “$f(n)\to\infty$” would be an infinite sequence $n_j\to\infty$ together with a fixed $m$ such that each $n_j!$ is a product of distinct divisors contained in some interval of length $m$. Such a family would require unexpectedly many divisors of $n!$ to cluster in bounded-length intervals.


