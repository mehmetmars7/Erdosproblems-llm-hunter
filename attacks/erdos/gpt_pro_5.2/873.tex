\section*{Problem \#873 (Erd\H{o}s--Szemer\'edi: small LCM of consecutive terms)}

\subsection*{1. Formal restatement}

Let \(A=\{a_1<a_2<\cdots\}\subseteq\mathbb{N}\) be infinite.
For real \(X\ge1\) and integer \(k\ge1\), define
\[
  F(A,X,k) := \#\Big\{ i\ge1 : \operatorname{lcm}(a_i,a_{i+1},\dots,a_{i+k-1}) < X\Big\}.
\]

\noindent\textbf{Problem.}
Is it true that for every \(\varepsilon>0\) there exists an integer \(k\ge1\) (possibly depending on \(A\) and \(\varepsilon\))
for which
\[
  F(A,X,k) < X^{\varepsilon}
\]
holds for all sufficiently large \(X\)? (The original prompt does not specify whether ``for all \(X\)'' or ``for all large \(X\)'';
most analytic-number-theory inequalities are meant for large \(X\), and this is the interpretation I adopt.)

\subsection*{2. Quick literature/context check (browsing available)}

The Erd\H{o}s Problems website (problem \#873) records an Erd\H{o}s--Szemer\'edi theorem:
for every infinite \(A\),
\[
  F(A,X,3) \ll X^{1/3}\log X,
\]
and that there exist sets \(A\) with
\(F(A,X,3)\gg X^{1/3}\log X\) for infinitely many \(X\).
The website lists the problem status as open.

\subsection*{3. Attack plan}

\begin{enumerate}
\item Record basic monotonicity properties of \(F(A,X,k)\) in \(X\) and \(k\).
\item Use the known \(k=3\) result to deduce that the conjectured statement is true for every \(\varepsilon>1/3\).
\item Identify ``easy'' classes of sequences \(A\) for which the statement is true (very sparse sequences, or sequences with large growth).
\item Explain what additional input seems needed to reach arbitrarily small exponents \(\varepsilon\to0\) for general \(A\).
\end{enumerate}

\subsection*{4. Detailed work}

\subsubsection*{4.1 Elementary monotonicity and trivial bounds}

\begin{lemma}[Monotonicity]
For fixed \(A\) and \(X\), the function \(k\mapsto F(A,X,k)\) is nonincreasing.
For fixed \(A\) and \(k\), the function \(X\mapsto F(A,X,k)\) is nondecreasing.
\end{lemma}

\begin{proof}
If \(k'\ge k\), then
\(\operatorname{lcm}(a_i,\dots,a_{i+k'-1})\ge \operatorname{lcm}(a_i,\dots,a_{i+k-1})\)
for every \(i\), so the condition ``\(<X\)'' becomes harder as \(k\) grows; hence
\(F(A,X,k')\le F(A,X,k)\).
Monotonicity in \(X\) is immediate from the definition.
\end{proof}

\begin{lemma}[Trivial counting bound]
Let \(A(X):=|A\cap[1,X)|\). Then for all \(k\ge1\),
\[
  F(A,X,k)\le \max\{A(X)-k+1,\,0\}\le A(X)\le X.
\]
\end{lemma}

\begin{proof}
If \(\operatorname{lcm}(a_i,\dots,a_{i+k-1})<X\), then in particular \(a_{i+k-1}<X\), so \(i+k-1\le A(X)\),
which implies \(i\le A(X)-k+1\). This gives the first inequality.
The bounds \(A(X)\le X\) and \(\max\{A(X)-k+1,0\}\le A(X)\) are immediate.
\end{proof}

\subsubsection*{4.2 Consequence of the Erd\H{o}s--Szemer\'edi \texorpdfstring{\(k=3\)}{k=3} bound}

Assume the known theorem:
there exists an absolute constant \(C>0\) such that for every infinite \(A\subset\mathbb{N}\),
\begin{equation}
  F(A,X,3) \le C\,X^{1/3}\log X\qquad (X\ge 2).
  \label{eq:ES3}
\end{equation}

Then by monotonicity in \(k\), for every \(k\ge3\),
\(F(A,X,k)\le F(A,X,3)\), hence
\begin{equation}
  F(A,X,k) \ll X^{1/3}\log X\qquad (k\ge3).
  \label{eq:kge3}
\end{equation}

In particular, for any fixed \(\varepsilon>1/3\), taking \(k=3\) yields
\(F(A,X,3)=o(X^{\varepsilon})\) as \(X\to\infty\).
Thus the problem is only genuinely difficult for \(\varepsilon\le 1/3\).

\subsubsection*{4.3 ``Easy'' cases where the conjectured conclusion holds}

\begin{itemize}
\item If \(A(X)\ll X^{\varepsilon}\) for some \(\varepsilon>0\), then taking \(k=1\) already gives
\(F(A,X,1)=A(X)\ll X^{\varepsilon}\).
In particular, for extremely sparse sequences (e.g. exponential growth), the statement is trivial.
\item For the full set \(A=\mathbb{N}\), one expects (and can prove with more work) that
\(F(\mathbb{N},X,k)\) is about \(X^{1/k}\) up to polylog factors, so choosing \(k>1/\varepsilon\) would suffice.
This suggests the conjecture is plausible, but the difficulty is achieving a bound of the form
\(X^{c_k+o(1)}\) uniformly for \emph{all} sets \(A\), with \(c_k\to 0\) as \(k\to\infty\).
\end{itemize}

\subsection*{5. Verification (gap check, edge cases)}

\begin{itemize}
\item The interpretation ``for all sufficiently large \(X\)'' is not explicitly stated in the prompt; if the intent is
\emph{uniform in all \(X\)}, then even the meaning of ``\(<X^{\varepsilon}\)'' at small \(X\) would need conventions.
All standard asymptotic statements are intended for large \(X\).
\item The deduction ``\(k\ge3\Rightarrow F(A,X,k)\le F(A,X,3)\)'' is correct by lcm monotonicity.
\item No attempt was made to reprove \eqref{eq:ES3} here; it is taken as known context.
\end{itemize}

\subsection*{6. Final}

\textbf{UNRESOLVED.}

\begin{enumerate}[label=(\roman*)]
\item \emph{Furthest point reached:}
Explained that the problem is open as stated; proved monotonicity and trivial bounds; deduced that the desired conclusion holds
for every \(\varepsilon>1/3\) using the known Erd\H{o}s--Szemer\'edi estimate for \(k=3\).
\item \emph{Blocking issue:}
The known uniform bound is only at exponent \(1/3\) (up to logs). To settle the question one needs bounds for larger windows \(k\)
that beat \(X^{1/3}\) and in fact tend to \(X^{o(1)}\) as \(k\to\infty\), uniformly in \(A\).
\item \emph{Most plausible next steps:}
Develop a theory controlling how many long blocks can have small lcm, perhaps via:
(i) estimates for the distribution of \(y\)-smooth numbers inside an arbitrary set \(A\),
(ii) structure theory for sets whose consecutive elements have large gcd overlap,
(iii) adapting sum-product or multiplicative energy methods to the lcm constraint.
\item \emph{Small experiments/checks:}
For model sets (\(A=\mathbb{N}\), primes, smooth numbers, products of two primes, etc.) one can numerically estimate
\(F(A,X,k)\) to guess the correct exponent as a function of \(k\), but this does not resolve the general worst-case problem.
\end{enumerate}

% ------------------------------------------------------------
