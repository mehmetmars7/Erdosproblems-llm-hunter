% Erdos Problem #868
% Solutions/notes generated by ChatGPT following PROMPT_STRATEGY.MD

\bigskip
\noindent\textbf{FORMAL RESTATEMENT.}
Let $A\subseteq\mathbb{N}$ (here $\mathbb{N}=\{0,1,2,\dots\}$ or $\{1,2,3,\dots\}$; the distinction does not affect ``all sufficiently large integers'').
Say $A$ is an \emph{asymptotic additive basis of order $2$} if there exists $n_0$ such that every integer $n\ge n_0$ can be written as $n=a+a'$ with $a,a'\in A$.
Let
\[
 r_A(n):=(1_A*1_A)(n)=\bigl|\{(a,a')\in A\times A: a+a'=n\}\bigr|,
\]
(the ordered representation function).
A subset $B\subseteq A$ is a \emph{minimal asymptotic basis of order $2$} if $B$ is an asymptotic basis of order $2$ and for every $b\in B$, the set $(B\setminus\{b\})+(B\setminus\{b\})$ fails to contain infinitely many integers.

Question: If $A$ is an asymptotic basis of order $2$ and $r_A(n)\to\infty$ as $n\to\infty$, must $A$ contain a minimal asymptotic basis of order $2$? What if $r_A(n)\ge \varepsilon\log n$ for all large $n$?

\medskip
\noindent\textbf{QUICK LITERATURE/CONTEXT CHECK.}
The problem statement records:
\begin{itemize}
\item Erd\H{o}s--Nathanson (1979) proved the conclusion under the stronger hypothesis $r_A(n)>(\log(4/3))^{-1}\log n$ for all large $n$.
\item H\"{a}rtter (1956) and Nathanson (1974) constructed additive bases containing no minimal additive bases.
\item Erd\H{o}s--Nathanson (1989) constructed, for any fixed $t$, bases $A$ with $r_A(n)\ge t$ for all large $n$ but containing no minimal asymptotic basis of order $2$.
\end{itemize}
(Per the project integrity constraints, I do not use or assert further literature beyond what is in the problem text.)

\medskip
\noindent\textbf{ATTACK PLAN.}
\begin{itemize}
\item Prove basic implications of $r_A(n)\to\infty$, e.g. robustness under deleting finitely many elements.
\item Try to formalize a Zorn/chain-minimization approach: show every descending chain of asymptotic bases inside $A$ has an intersection that is still an asymptotic basis, which would yield a minimal element.
\item Disproof track: attempt to amplify known constructions with bounded $r_A(n)$ to force $r_A(n)\to\infty$ while still forbidding minimal subbases.
\end{itemize}
I only establish basic robustness lemmas; I do not resolve existence of a minimal subbasis.

\medskip
\noindent\textbf{WORK.}

\medskip
\noindent\textbf{Lemma 868.1 (Finite deletions preserve the basis property under $r_A(n)\to\infty$).}
Assume $A$ is an asymptotic basis of order $2$ and $r_A(n)\to\infty$ as $n\to\infty$.
Then for every finite set $F\subseteq A$, the set $A\setminus F$ is also an asymptotic basis of order $2$.

\smallskip
\noindent\emph{Proof.}
Fix a finite $F\subseteq A$ and put $m:=|F|$.
For each integer $n$, let $R(n)$ be the set of ordered representations $(a,a')\in A\times A$ with $a+a'=n$, so $|R(n)|=r_A(n)$.
Let $R_F(n)\subseteq R(n)$ be those representations that use at least one summand from $F$.
If $f\in F$, then in an ordered representation $(a,a')$ of $n$ the condition $a=f$ forces $a'=n-f$ uniquely, and likewise $a'=f$ forces $a=n-f$ uniquely.
Therefore, for each $n$,
\[
|R_F(n)|\le 2m.
\]
Since $r_A(n)\to\infty$, there exists $n_1$ such that for all $n\ge n_1$,
$r_A(n)>2m$.
For such $n$, we have $|R(n)\setminus R_F(n)|\ge r_A(n)-2m\ge 1$, so there exists an ordered representation $(a,a')$ of $n$ with $a,a'\notin F$.
Thus $n\in (A\setminus F)+(A\setminus F)$ for all $n\ge n_1$, i.e. $A\setminus F$ is an asymptotic basis of order $2$.
\qed

\medskip
\noindent\textbf{Lemma 868.2 (Minimality \texorpdfstring{$\Leftrightarrow$}{<->} infinitely many ``forced'' representations).}
Let $B$ be an asymptotic basis of order $2$.
Then $B$ is minimal (asymptotic) if and only if for every $b\in B$ there are infinitely many integers $n$ such that every representation of $n$ as $n=x+y$ with $x,y\in B$ satisfies $x=b$ or $y=b$.

\smallskip
\noindent\emph{Proof.}
Fix $b\in B$.
If $B$ is minimal, then by definition $B\setminus\{b\}$ fails to represent infinitely many integers $n$.
For each such $n$, every representation of $n$ using $B$ must involve $b$ (otherwise $n$ would still be representable after deleting $b$).
Thus the ``forced representation'' property holds for infinitely many $n$.

Conversely, suppose the forced-representation property holds for $b$ on an infinite set $S\subseteq\mathbb{N}$.
Then for each $n\in S$, there is no representation of $n$ using only $B\setminus\{b\}$, so $n\notin (B\setminus\{b\})+(B\setminus\{b\})$.
Hence deleting $b$ creates infinitely many missing integers, so $B$ is minimal.
\qed

\medskip
\noindent\textbf{FAST REALITY CHECK (sanity examples).}
\begin{itemize}
\item If $A=\mathbb{N}$, then $r_A(n)=n+1\to\infty$, and Lemma 868.1 correctly says deleting finitely many elements preserves being an asymptotic basis.
\item Lemma 868.2 matches the intuition: in a minimal basis, each element must be essential for infinitely many target integers.
\end{itemize}

\medskip
\noindent\textbf{VERIFICATION.}
\begin{itemize}
\item Lemma 868.1: the only nontrivial estimate is $|R_F(n)|\le 2|F|$, justified by the uniqueness of a partner for a fixed summand in an ordered sum representation.
\item Lemma 868.2: both implications are direct unpacking of the definition, with no hidden compactness assumptions.
\end{itemize}

\medskip
\noindent\textbf{FINAL.} \textbf{UNRESOLVED}

(i) \emph{Strongest proved partial result.}
If $A$ is an asymptotic basis of order 2 and $r_A(n)\to\infty$, then $A$ is robust under deleting any finite subset (Lemma 868.1). Minimality can be characterized by having infinitely many integers whose representations are forced to use each given element (Lemma 868.2).

(ii) \emph{First gap (crisp).}
Show that the family of asymptotic bases contained in $A$ has a minimal element under inclusion, e.g. by proving that the intersection of any descending chain of such bases is still an asymptotic basis.
This is exactly where a Zorn-type approach would need a verified chain-intersection property.

(iii) \emph{Top 3 next moves.}
\begin{itemize}
\item Attempt to prove a chain-intersection lemma under quantitative growth (e.g. $r_A(n)\ge \varepsilon\log n$): if $A=A_0\supseteq A_1\supseteq A_2\supseteq\cdots$ is a nested sequence of asymptotic bases with each $A_{m+1}=A_m\setminus\{a_m\}$, show $\bigcap_m A_m$ still represents all large $n$.
\item Try to adapt (or reprove from scratch) the Erd\H{o}s--Nathanson threshold $(\log(4/3))^{-1}\log n$ as an explicit ``sufficient slack'' for iterative deletions.
\item Construct candidate counterexamples by building $A$ in blocks so that $r_A(n)$ grows slowly, but every purported subbasis can be thinned further without losing the basis property.
\end{itemize}

(iv) \emph{Minimal counterexample structure.}
A counterexample would be an asymptotic basis $A$ with $r_A(n)\to\infty$ but such that for every $B\subseteq A$ which is an asymptotic basis, there exists $b\in B$ with $B\setminus\{b\}$ still an asymptotic basis (so no minimal element). Such an $A$ would need representations to be plentiful yet ``spread'' so that no single element is forced infinitely often inside any subbasis.


