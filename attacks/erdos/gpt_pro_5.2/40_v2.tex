\section{Erd\H{o}s Problem \#40 --- Round 2}

\subsection*{1) ROUND-2 OBJECTIVE}
\textbf{Path (B): counterexample / disproof for an infinite family of functions $g$.}

Round~1 isolated that first-moment bounds are far too weak in the regime
$|A\cap[1,N]|\asymp \sqrt N/g(N)$ with $g(N)\to\infty$, and suggested that any negative
answer should come from building a set $A$ with \emph{uniformly bounded} additive
representation counts (a ``$B_2[O(1)]$'' construction) while keeping the required lower
bound on $|A\cap[1,N]|$.

In this round I implement that disproof track for a large explicit class of growth
functions $g$, using a theorem of Erd\H{o}s--R\'enyi on infinite $B_2[g]$ sequences.

\subsection*{2) ROUND-1 FOUNDATION USED}
I use the following vetted Round~1 items.
\begin{itemize}
\item Lemma~40.1: For $m=|A\cap[1,N]|$, some $n\le 2N$ has at least $m^2/(2N-1)$ ordered
representations using summands $\le N$.
\item Lemma~40.2: If $(1_A\!\ast\!1_A)(n)\le R$ for all $n$, then $|A\cap[1,N]|<\sqrt{2RN}$ for
all $N$.
\item Round~1 gap: no construction was given that meets the hypothesis
$|A\cap[1,N]|\gg \sqrt N/g(N)$ (for all large $N$) while keeping $(1_A\!\ast\!1_A)(n)$ bounded.
\end{itemize}

\subsection*{3) NEW INSIGHT / TOOL (ROUND-2)}
The new tool is a classical probabilistic-existence theorem (quoted in a standard survey)
constructing infinite sets with \emph{bounded} two-term representation function and
prescribed polynomial growth.

\medskip
\noindent\textbf{External Theorem (Erd\H{o}s--R\'enyi, 1960; quoted as Theorem~8.2 in
S\'ark\"ozy--S\'os).}
For every $\varepsilon>0$ there exists an integer $\lambda=\lambda(\varepsilon)$ and an
infinite set $A\subset\mathbb N$ such that:
\begin{enumerate}
\item[(i)] $A$ is a $B_2[\lambda]$ set: every $n\in\mathbb N$ has at most $\lambda$ solutions
$a\le a'$ with $a,a'\in A$ to $a+a'=n$.
\item[(ii)] (Polynomial density) There exists $N_0(\varepsilon)$ such that for all $N\ge N_0(\varepsilon)$,
\[|A\cap[1,N]|>N^{\frac12-\varepsilon}.\]
\end{enumerate}

\subsection*{4) ATTACK PLAN (ROUND-2)}
\begin{itemize}
\item \emph{Gap to close from Round~1:} exhibit any divergent $g(N)$ for which the implication in
Problem~40 fails.
\item \emph{Plan:} Fix $\varepsilon>0$ and take $g(N)=N^{\varepsilon}$. Apply the external theorem
above to obtain a set $A$ with $|A\cap[1,N]|\ge N^{1/2-\varepsilon}$ for all large $N$.
Convert the $B_2[\lambda]$ property into a uniform bound on the ordered convolution
$(1_A\ast 1_A)(n)$, showing $\limsup_n (1_A\ast 1_A)(n)<\infty$.
\item \emph{Extension:} the same $A$ also disproves the implication for any $g$ that eventually dominates
$N^{\varepsilon}$ (since the required lower bound $\sqrt N/g(N)$ is then even weaker).
\end{itemize}

\subsection*{5) WORK (ROUND-2)}
\paragraph{5.1. From $B_2[\lambda]$ to bounded ordered convolution.}
For a set $A\subset\mathbb N$, write
\[r_A(n):=(1_A\ast 1_A)(n)=|\{(a,b)\in A\times A: a+b=n\}|\]
for the \emph{ordered} two-sum representation function.
Also write
\[R_A(n):=|\{(a,a')\in A\times A: a\le a',\ a+a'=n\}|\]
for the \emph{nondecreasing} (unordered) representation count.

\begin{lemma}
If $R_A(n)\le \lambda$ for all $n$ (i.e. $A\in B_2[\lambda]$), then
$r_A(n)\le 2\lambda$ for all $n$.
\end{lemma}

\begin{proof}
Fix $n$.
Map each ordered solution $(a,b)\in A\times A$ with $a+b=n$ to the nondecreasing pair
$(\min\{a,b\},\max\{a,b\})$. This map lands in the set counted by $R_A(n)$.
Moreover, every nondecreasing pair $(x,y)$ with $x+y=n$ has at most two preimages:
$(x,y)$ and $(y,x)$, and only one if $x=y$.
Hence $r_A(n)\le 2R_A(n)\le 2\lambda$.
\end{proof}

\paragraph{5.2. A disproof for $g(N)=N^{\varepsilon}$.}
Fix any $\varepsilon>0$ and set
\[g(N):=N^{\varepsilon},\qquad N\in\mathbb N.
\]
Then $g(N)\to\infty$.
Let $A\subset\mathbb N$ be the set produced by the Erd\H{o}s--R\'enyi theorem in
\S3, with the corresponding $\lambda=\lambda(\varepsilon)$.

\begin{proposition}
For $g(N)=N^{\varepsilon}$, the implication in Problem~40 is \emph{false}.
More precisely, there exists an infinite set $A\subset\mathbb N$ and a constant $c>0$ such that
\[|A\cap[1,N]|\ge c\,\frac{\sqrt N}{g(N)}\quad\text{for all sufficiently large $N$},\]
but $\sup_{n\in\mathbb N}(1_A\ast 1_A)(n)<\infty$.
\end{proposition}

\begin{proof}
By property (ii) of the Erd\H{o}s--R\'enyi theorem, for all $N\ge N_0(\varepsilon)$ we have
\[|A\cap[1,N]|>N^{\frac12-\varepsilon}=\frac{\sqrt N}{N^{\varepsilon}}=\frac{\sqrt N}{g(N)}.
\]
Thus the hypothesis of Problem~40 holds with $c=1$ (and $N_0=N_0(\varepsilon)$).

On the other hand, by property (i) the set $A$ satisfies $R_A(n)\le\lambda$ for all $n$.
By Lemma~5.1, this implies
\[(1_A\ast 1_A)(n)=r_A(n)\le 2\lambda\qquad\text{for all }n\in\mathbb N.
\]
Hence $\limsup_{n\to\infty}(1_A\ast 1_A)(n)\le 2\lambda<\infty$, contradicting the
conclusion required in Problem~40.
\end{proof}

\paragraph{5.3. A necessary restriction on $g$.}
The same argument yields an immediate corollary clarifying where a positive answer might
still live.

\begin{corollary}
Fix $\varepsilon>0$ and let $A$ be as above. Then the implication in Problem~40 fails for
\emph{every} function $g$ satisfying $g(N)\ge N^{\varepsilon}$ for all sufficiently large $N$.
Consequently, if the implication in Problem~40 holds for some $g(N)\to\infty$, then necessarily
\[\forall\varepsilon>0:\quad g(N)=o(N^{\varepsilon})\quad (N\to\infty),\]
i.e. $g$ must be subpolynomial ($g(N)=N^{o(1)}$).
\end{corollary}

\begin{proof}
If $g(N)\ge N^{\varepsilon}$ eventually, then for all large $N$,
$\sqrt N/g(N)\le \sqrt N/N^{\varepsilon}=N^{1/2-\varepsilon}$.
But $|A\cap[1,N]|>N^{1/2-\varepsilon}$ for all large $N$ by the theorem, so
$|A\cap[1,N]|\ge \sqrt N/g(N)$ eventually as well.
The boundedness of $(1_A\ast 1_A)(n)$ is unchanged.
The final claim is just the contrapositive.
\end{proof}

\subsection*{6) ADVERSARIAL VERIFICATION}
\begin{itemize}
\item \emph{Ordered vs unordered counts:} The only conversion step is Lemma~5.1. The map
$(a,b)\mapsto(\min\{a,b\},\max\{a,b\})$ has fibers of size $\le 2$, so $r_A(n)\le 2R_A(n)$ is
tight and correct.
\item \emph{Quantifiers in the hypothesis:} The Erd\H{o}s--R\'enyi theorem gives a lower bound
$|A\cap[1,N]|>N^{1/2-\varepsilon}$ for all $N\ge N_0(\varepsilon)$, matching the
``$\forall N\ge N_0$'' hypothesis in Problem~40. Taking $c=1$ is valid.
\item \emph{$g(N)\to\infty$:} For $g(N)=N^{\varepsilon}$ this is immediate.
\item \emph{Does boundedness hold for all $n$ or only large $n$?} The $B_2[\lambda]$ property is
for \emph{all} $n\in\mathbb N$, so $\sup_n (1_A\ast 1_A)(n)\le 2\lambda$ holds globally.
This certainly implies the failure of $\limsup_n(1_A\ast 1_A)(n)=\infty$.
\item \emph{Interaction with Round~1 Lemma~40.2:} Lemma~40.2 gives
$|A\cap[1,N]|\ll \sqrt N$ under bounded representation.
Our constructed set has $|A\cap[1,N]|\asymp N^{1/2-\varepsilon}$, which is consistent.
\end{itemize}

\subsection*{7) FINAL}
\textbf{UNRESOLVED (BUT STRICTLY ADVANCED).}

Round~2 gives an explicit infinite family of divergent functions for which the implication
in Problem~40 fails:
\begin{quote}
for every fixed $\varepsilon>0$, taking $g(N)=N^{\varepsilon}$ (or any $g$ eventually
$\ge N^{\varepsilon}$) does \emph{not} force $\limsup_n (1_A\ast 1_A)(n)=\infty$.
\end{quote}
The remaining open regime is when $g(N)$ is subpolynomial ($g(N)=N^{o(1)}$), e.g.
$g(N)=(\log N)^C$ or $g(N)=\log\log N$.

\subsection*{8) COMPLETION ESTIMATE (MANDATORY)}
COMPLETION: 40\%

\subsection*{9) REFERENCES}
\begin{enumerate}
\item P.~Erd\H{o}s and A.~R\'enyi, \emph{Additive properties of random sequences of positive integers},
Acta Arith.\ \textbf{6} (1960), 83--110.
\item A.~S\'ark\"ozy and V.~T.~S\'os, \emph{On additive representation functions}, in:
\emph{The Mathematics of Paul Erd\H{o}s I} (R.~L. Graham et al., eds.), Springer, 1997.
(See Theorem~8.2 therein for the Erd\H{o}s--R\'enyi construction used above.)
\end{enumerate}


\section{Erd\H{o}s Problem \#40 --- Round 3}

\subsection*{1) ROUND-3 OBJECTIVE}
\textbf{Path (C): fatal obstruction / minimal correction.}

Round~2 produced a rigorous counterexample family showing that the implication in
Problem~\#40 fails whenever the density parameter $g(N)$ is \emph{polynomially large}
(e.g. $g(N)=N^\varepsilon$).  In Round~3, the most promising way to \emph{finish the
classification as far as current progress allows} is therefore not to chase further
counterexamples ad hoc, but to:
\begin{itemize}
\item isolate a clean structural \emph{equivalence} between ``failure of Problem~\#40
for $g$'' and the existence of a sufficiently dense \emph{bounded representation}
($B_2[\lambda]$) set;
\item deduce from Round~2 that \emph{no} $g$ with a positive power growth rate can work;
\item state the resulting \emph{minimal corrected form}: any candidate $g$ must be
\emph{subpolynomial} ($g(N)=N^{o(1)}$).
\end{itemize}
This closes a conceptual gap left after Round~2: it identifies precisely what remains
open after the polynomial counterexamples.

\subsection*{2) ROUND-1/2 FOUNDATION USED}
I rely on the following Round~1/2 items without re-proving them.
\begin{itemize}
\item Round~1 Lemma~40.2: If $(1_A\!\ast\!1_A)(n)\le R$ for all $n$, then
$|A\cap[1,N]|<\sqrt{2RN}$ for all $N$.
\item Round~2 Lemma~5.1: If $A\in B_2[\lambda]$ (unordered representations $\le\lambda$)
then the ordered representation function satisfies $(1_A\ast 1_A)(n)\le 2\lambda$ for all $n$.
\item Round~2 Proposition~5.2 and Corollary~5.3: for every fixed $\varepsilon>0$ the
implication in Problem~\#40 fails for $g(N)=N^{\varepsilon}$ (indeed, for every $g$ with
$g(N)\ge N^{\varepsilon}$ eventually), by an Erd\H{o}s--R\'enyi $B_2[\lambda]$ construction.
\end{itemize}

\subsection*{3) NEW INSIGHT / TOOL (ROUND-3)}
The new ingredient is a \emph{reduction} that turns Problem~\#40 into an existence/nonexistence
statement about dense infinite $B_2[\lambda]$ sets, plus a monotonicity principle.

\begin{definition}[Ordered representation bound]
For $A\subset\mathbb N$ define
\[r_A(n):=(1_A\ast 1_A)(n)=|\{(a,b)\in A\times A: a+b=n\}|\in\mathbb N\cup\{0\}.\]
We say that $A$ has \emph{bounded two-sum representations} if
$\sup_{n\ge 1} r_A(n)<\infty$.
\end{definition}

\begin{lemma}[Reduction to $B_2[\lambda]$ sets]\label{lem:reduction}
Fix $g:\mathbb N\to(0,\infty)$ with $g(N)\to\infty$.
The implication in Problem~\#40 \emph{fails} for this $g$ if and only if there exist
constants $c>0$, $\lambda\in\mathbb N$, and an infinite set $A\subset\mathbb N$ such that
\begin{enumerate}
\item[(i)] (Density) $|A\cap[1,N]|\ge c\,\dfrac{\sqrt N}{g(N)}$ for all sufficiently large $N$;
\item[(ii)] ($B_2[\lambda]$) $A\in B_2[\lambda]$ (equivalently, $r_A(n)\le 2\lambda$ for all $n$).
\end{enumerate}
Equivalently, failure for $g$ is the same as the existence of a bounded-representation set
$A$ with
\[\liminf_{N\to\infty}\ \frac{g(N)\,|A\cap[1,N]|}{\sqrt N}\ >\ 0.\]
\end{lemma}

\begin{proof}
($\Leftarrow$) If such an $A$ exists, then by (ii) and Round~2 Lemma~5.1 we have
$\sup_n r_A(n)\le 2\lambda<\infty$, hence $\limsup_n r_A(n)<\infty$.
Together with (i), this is a direct counterexample to the conclusion of
Problem~\#40.

($\Rightarrow$) If the implication in Problem~\#40 fails for $g$, then there exists an
infinite $A\subset\mathbb N$ and $c>0$ such that the density lower bound in (i) holds,
while $\limsup_n r_A(n)<\infty$.  Set $R:=\sup_n r_A(n)<\infty$.
Define $R_A(n)$ to be the number of \emph{unordered} representations $n=a+a'$ with
$a\le a'$ and $a,a'\in A$.  Then $R_A(n)\le r_A(n)\le R$ for all $n$, so $A\in B_2[R]$.
Taking $\lambda:=R$ gives (ii).  The liminf reformulation is equivalent to (i) by
unpacking the meaning of ``$\gg$''.
\end{proof}

\begin{lemma}[Monotonicity in $g$]\label{lem:mono}
Let $g_1,g_2:\mathbb N\to(0,\infty)$ with $g_i(N)\to\infty$ and assume
$g_2(N)\ge g_1(N)$ for all sufficiently large $N$.
\begin{enumerate}
\item[(a)] If Problem~\#40 fails for $g_1$, then it fails for $g_2$.
\item[(b)] If Problem~\#40 holds for $g_2$, then it holds for $g_1$.
\end{enumerate}
\end{lemma}

\begin{proof}
If $g_2\ge g_1$ eventually then $\sqrt N/g_2(N)\le \sqrt N/g_1(N)$ eventually.
Thus any set $A$ meeting the stronger density requirement for $g_1$ also meets the
weaker density requirement for $g_2$.  Statement (a) follows by transporting a
counterexample; (b) is the contrapositive of (a).
\end{proof}

\subsection*{4) ATTACK PLAN (ROUND-3)}
\begin{itemize}
\item \emph{Gap after Round~2:} we had a family of counterexamples, but no structural
characterisation of what a positive answer would have to exclude.
\item \emph{Round~3 goal:} use Lemma~\ref{lem:reduction} and Lemma~\ref{lem:mono} to:
\begin{enumerate}
\item reformulate Problem~\#40 as a question about \emph{how dense} a bounded-representation
($B_2[\lambda]$) set can be while still satisfying a uniform lower bound at \emph{all}
large $N$;
\item upgrade the Round~2 corollary into a clean ``no polynomial $g$ can work'' statement.
\end{enumerate}
\item \emph{Why this helps:} it isolates the remaining open regime as a single extremal
question about dense infinite $B_2[\lambda]$ sets, clarifying what must be improved to
settle the problem.
\end{itemize}

\subsection*{5) WORK (ROUND-3)}
\paragraph{5.1. A clean necessary condition on $g$.}
Define the (asymptotic) power-growth exponent of $g$ by
\[\alpha(g):=\liminf_{N\to\infty} \frac{\log g(N)}{\log N}\in[0,\infty].\]
Note that $\alpha(g)>0$ is equivalent to: there exists $\varepsilon>0$ such that
$g(N)\ge N^{\varepsilon}$ for all sufficiently large $N$.

\begin{theorem}[Polynomially large $g$ never work]\label{thm:polyfails}
If $g(N)\to\infty$ and $\alpha(g)>0$, then the implication in Problem~\#40 \emph{fails} for $g$.
Equivalently, if Problem~\#40 holds for some $g$, then necessarily
\[\alpha(g)=0\qquad\text{ i.e. }\qquad \forall\varepsilon>0:\ g(N)=o(N^{\varepsilon}).\]
(In other words, $g$ must be subpolynomial: $g(N)=N^{o(1)}$.)
\end{theorem}

\begin{proof}
Assume $\alpha(g)>0$.  Then choose $\varepsilon>0$ and $N_1$ such that
$g(N)\ge N^{\varepsilon}$ for all $N\ge N_1$.
By Round~2 Corollary~5.3, Problem~\#40 fails for every such $g$ (indeed for every $g$ with
$g(N)\ge N^{\varepsilon}$ eventually), because an Erd\H{o}s--R\'enyi $B_2[\lambda]$ set $A$
exists with $|A\cap[1,N]|\ge N^{1/2-\varepsilon}=\sqrt N/N^{\varepsilon}\ge \sqrt N/g(N)$ for all
large $N$ while $\sup_n r_A(n)<\infty$.

The final statement is the contrapositive.
\end{proof}

\paragraph{5.2. Minimal corrected statement and remaining open regime.}
Combining Lemma~\ref{lem:mono}, Lemma~\ref{lem:reduction}, and
Theorem~\ref{thm:polyfails} yields the following sharp ``frontier'' formulation.

\begin{corollary}[Frontier formulation]\label{cor:frontier}
Let $\mathcal G$ denote the class of functions $g(N)\to\infty$ for which Problem~\#40 holds.
Then:
\begin{enumerate}
\item[(i)] (Downward closure) If $g\in\mathcal G$ and $h(N)\le g(N)$ eventually, then $h\in\mathcal G$.
\item[(ii)] (Polynomial exclusion) No $g$ with $\alpha(g)>0$ lies in $\mathcal G$.
\item[(iii)] (Open core) The only possible elements of $\mathcal G$ are subpolynomial
functions $g(N)=N^{o(1)}$ (e.g. $g(N)=(\log N)^C$, $g(N)=\log\log N$, etc.).
Determining whether any such $g$ lies in $\mathcal G$ is equivalent (by
Lemma~\ref{lem:reduction}) to ruling out the existence of bounded-representation
$B_2[\lambda]$ sets with density $\gg \sqrt N/g(N)$ for all large $N$.
\end{enumerate}
\end{corollary}

\subsection*{6) ADVERSARIAL VERIFICATION}
\begin{itemize}
\item \emph{Quantifiers in Lemma~\ref{lem:reduction}:}
Problem~\#40 assumes $|A\cap[1,N]|\gg \sqrt N/g(N)$ for \emph{all sufficiently large $N$}.
This is exactly the statement that the ratio $g(N)|A\cap[1,N]|/\sqrt N$ has positive
\emph{liminf}.  Using limsup or limsup only would be incorrect; the proof uses the
liminf version.

\item \emph{Bounded convolution $\Rightarrow B_2[\lambda]$:}
If $r_A(n)\le R$ for all $n$, then unordered representations satisfy $R_A(n)\le r_A(n)\le R$,
so $A\in B_2[R]$ is valid.

\item \emph{Monotonicity direction:}
The direction in Lemma~\ref{lem:mono}(a) is the one actually used:
weakening the hypothesis (larger $g$) cannot repair a counterexample.  The contrapositive
(b) is consistent.

\item \emph{No hidden dependence on $N$:}
In Theorem~\ref{thm:polyfails} we use a \emph{fixed} $\varepsilon>0$ extracted from
$\alpha(g)>0$; this matches the Round~2 construction which provides a single fixed
$B_2[\lambda(\varepsilon)]$ set.
\end{itemize}

\subsection*{7) FINAL}
\textbf{UNRESOLVED (BUT STRICTLY ADVANCED).}

Round~3 upgrades Round~2 from a family of counterexamples to a structural description:
\begin{quote}
Problem~\#40 is \emph{false} for every $g(N)\to\infty$ with positive power-growth rate
$\alpha(g)>0$ (in particular for every polynomially growing $g$), and the only remaining
possible regime is subpolynomial $g(N)=N^{o(1)}$.
\end{quote}
The remaining core question is equivalent to understanding how dense an infinite
bounded-representation $B_2[\lambda]$ set can be while maintaining a uniform lower bound
at \emph{all} large $N$.

\subsection*{8) COMPLETION ESTIMATE (MANDATORY)}
COMPLETION: 55\%

\subsection*{9) REFERENCES}
\begin{enumerate}
\item P.~Erd\H{o}s and A.~R\'enyi, \emph{Additive properties of random sequences of positive integers},
Acta Arith.\ \textbf{6} (1960), 83--110.
\item A.~S\'ark\"ozy and V.~T.~S\'os, \emph{On additive representation functions}, in:
\emph{The Mathematics of Paul Erd\H{o}s I} (R.~L. Graham et al., eds.), Springer, 1997.
(See Theorem~8.2 for the Erd\H{o}s--R\'enyi construction of dense $B_2[\lambda]$ sequences.)
\item T.~F.~Bloom (ed.), \emph{Erd\H{o}s Problem \#40} (problem statement and status),
Erd\H{o}s Problems website, accessed 2026-01-20.
\end{enumerate}
