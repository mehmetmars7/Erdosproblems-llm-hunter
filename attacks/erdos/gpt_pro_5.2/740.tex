\section*{PROBLEM 3 (\#740): Preserving infinite chromatic number while deleting short odd cycles}

\subsection*{FORMAL RESTATEMENT}
Let $\mathfrak m$ be an infinite cardinal, let $r\in\mathbb{N}$ with $r\ge 1$, and let $G=(V,E)$ be a (simple, undirected) graph with $\chi(G)=\mathfrak m$.

Define the \emph{odd-girth threshold} condition
\[
\text{``$H$ has no odd cycle of length $\le r$''}\quad\Longleftrightarrow\quad
\forall \ell\in\{3,5,7,\dots\}\ (\ell\le r\Rightarrow C_\ell\not\subseteq H).
\]

Question: must there exist a (not necessarily induced) subgraph $H\subseteq G$ such that
\[
\chi(H)=\mathfrak m\qquad\text{and}\qquad H\text{ has no odd cycle of length }\le r\ ?
\]

\subsection*{QUICK LITERATURE/CONTEXT CHECK}
This question is listed as Erd\H{o}s Problem \#740 and marked open in the Erd\H{o}s Problems Project.
The page notes that R\"{o}dl proved the statement for $(\mathfrak m,r)=(\aleph_0,3)$.
For uncountable chromatic number, Komj\'ath and Shelah (1988) discuss the related Erd\H{o}s--Hajnal conjecture that every $\kappa$-chromatic graph contains a $\kappa$-chromatic triangle-free subgraph; they show the answer can be consistently negative for $\kappa=\aleph_1$.

\subsection*{ATTACK PLAN}
\begin{enumerate}[leftmargin=*]
\item \textbf{Proof-track:} use known finitary ``cleaning'' theorems (R\"{o}dl for triangles) plus compactness to deduce the countable case; attempt to iterate for larger odd-girth thresholds.
\item \textbf{Disproof-track:} search for (or cite) constructions of graphs of uncountable chromatic number whose triangle-free (or bounded odd-girth) subgraphs necessarily have smaller chromatic number.
\item \textbf{Best path here:} record the strongest unconditional results (notably the $\aleph_0$, $r=3$ case) and explain why ZFC-resolution for general $(\mathfrak m,r)$ remains out of reach, including known consistency results.
\end{enumerate}

\subsection*{WORK}
\subsubsection*{1. The case $(\mathfrak m,r)=(\aleph_0,3)$ from a finitary input}
We record a standard reduction from a finitary triangle-cleaning theorem to the countably infinite conclusion.

\begin{theorem}[Countable case, assuming finitary R\"{o}dl]
Assume the following finitary statement:
\begin{quote}
For every $k\in\mathbb{N}$ there exists $N(k)\in\mathbb{N}$ such that every finite graph $F$ with $\chi(F)\ge N(k)$ contains a \emph{triangle-free} subgraph $F'\subseteq F$ with $\chi(F')\ge k$.
\end{quote}
Then every graph $G$ with $\chi(G)=\aleph_0$ contains a triangle-free subgraph $H\subseteq G$ with $\chi(H)=\aleph_0$.
\end{theorem}

\begin{proof}
Let $G$ be a graph with $\chi(G)=\aleph_0$.
We build vertex-disjoint finite induced subgraphs $G[S_k]$ with very large chromatic numbers.

\medskip
\noindent\textbf{Step 1: Disjoint high-chromatic finite induced subgraphs exist.}
We claim that for every finite set $S\subseteq V(G)$, the graph $G\setminus S$ still has infinite chromatic number.
Indeed, if $\chi(G\setminus S)\le t$ for some $t\in\mathbb{N}$, then coloring $G\setminus S$ with $t$ colors and giving each vertex of $S$ a fresh new color yields a proper coloring of $G$ with $t+|S|$ colors, contradicting $\chi(G)=\aleph_0$.

Hence, inductively, we may choose finite, pairwise disjoint sets $S_1,S_2,\dots$ such that
\[\chi(G[S_k])\ge N(k)\qquad\text{for each }k\ge 1.\]
(Choose $S_k$ inside $G\setminus (S_1\cup\cdots\cup S_{k-1})$ using de Bruijn--Erd\H{o}s compactness.)

\medskip
\noindent\textbf{Step 2: Apply the finitary input on each piece.}
By the assumed finitary statement, each finite graph $G[S_k]$ has a triangle-free subgraph $H_k\subseteq G[S_k]$ with $\chi(H_k)\ge k$.

\medskip
\noindent\textbf{Step 3: Take the disjoint union as a subgraph of $G$.}
Let $H$ be the subgraph of $G$ with vertex set $\bigcup_{k\ge 1} S_k$ and edge set
\[ E(H):=\bigcup_{k\ge 1} E(H_k).\]
That is, we keep the edges of $H_k$ inside each block and delete \emph{all} edges between distinct blocks.
This is allowed because $H$ is only required to be a (not induced) subgraph.

Then $H$ is triangle-free (a triangle would have to lie within one block, but each $H_k$ is triangle-free), and
\[\chi(H)=\sup_{k\ge 1}\chi(H_k)=\aleph_0\]
because $\chi(H)\ge \chi(H_k)\ge k$ for each $k$.
\end{proof}

\noindent\textbf{Remark.}
The cited Erd\H{o}s Problems Project notes that R\"{o}dl proved the $(\aleph_0,3)$ case (and points to a finitary version).
The argument above explains one way the countable conclusion follows from a finitary triangle-cleaning theorem.

\subsubsection*{2. Uncountable chromatic number: consistency obstacles}
Komj\'ath and Shelah (1988) studied the uncountable analogue of the triangle-cleaning problem and showed (by forcing) that it is consistent with ZFC that there exists a graph $X$ on $\omega_1$ with $\chi(X)=\aleph_1$ such that every triangle-free subgraph of $X$ is only countably chromatic.
In such a model, the answer to the $r=3$ instance of the question is negative for $\mathfrak m=\aleph_1$.
They emphasize that a ZFC counterexample is not known there.

\subsection*{VERIFICATION}
\begin{itemize}[leftmargin=*]
\item In the countable-case theorem, the construction deletes all inter-block edges, ensuring triangle-freeness even if $G$ has many edges between blocks.
\item The key subtlety is obtaining \emph{vertex-disjoint} finite induced subgraphs with large chromatic number; the argument that removing finitely many vertices cannot drop $\chi$ from $\aleph_0$ to finite is correct.
\item The ``uncountable'' paragraph is explicitly stated as a \emph{consistency} statement rather than a ZFC theorem.
\end{itemize}

\subsection*{FINAL}
\textbf{UNRESOLVED (in ZFC) for general $(\mathfrak m,r)$.}

\medskip
\noindent\textbf{(i) Strongest fully proved partial result obtained here.}
Assuming a finitary triangle-cleaning theorem (R\"{o}dl-type), one can rigorously deduce the case $(\mathfrak m,r)=(\aleph_0,3)$: every $\aleph_0$-chromatic graph contains a triangle-free subgraph of chromatic number $\aleph_0$.
Also, there are consistency results (Komj\'ath--Shelah) showing that for $\mathfrak m=\aleph_1$ the $r=3$ statement can fail in some models of ZFC.

\medskip
\noindent\textbf{(ii) First gap.}
A ZFC-resolution (either proof or explicit ZFC counterexample) for uncountable $\mathfrak m$ already when $r=3$ is not supplied by the consistency results; similarly, extending from $r=3$ to larger odd-girth thresholds is open in general.

\medskip
\noindent\textbf{(iii) Top 3 next moves.}
\begin{enumerate}[leftmargin=*]
\item Determine whether the $r=3$ statement for $\mathfrak m=\aleph_1$ is decidable in ZFC (prove it or build an explicit ZFC counterexample).
\item Seek a framework to ``clean'' odd cycles of lengths $\le r$ while preserving large chromatic number, generalizing R\"{o}dl's approach beyond triangles.
\item Analyze whether additional set-theoretic hypotheses (large cardinals, forcing axioms) yield positive or negative answers, clarifying the independence landscape.
\end{enumerate}

\medskip
\noindent\textbf{(iv) What a minimal counterexample would likely look like.}
For $r=3$ and uncountable $\mathfrak m$, a minimal counterexample would be a graph $G$ with $\chi(G)=\mathfrak m$ such that every triangle-free subgraph has chromatic number $<\mathfrak m$.
Known forcing constructions achieve this behavior for $\mathfrak m=\aleph_1$ in some models, suggesting that any ZFC counterexample (if it exists) may require delicate combinatorial set-theory.

\subsection*{COMPLETION ESTIMATE}
COMPLETION: 45\%.


%%%%%%%%%%%%%%%%%%%%%%%%%%%%%%%%%%%%%%%%%%%%%%%%%%%%%%%%%%%%%%%%%%%%%%%%%%%%%%%
\begin{thebibliography}{99}


\bibitem{GyarfasPfree}
M. Bonamy, N. Bousquet, M. Pilipczuk, P. Rz\k{a}\.{z}ewski, S. Thomass\'e, and B. Walczak,
\emph{Degeneracy of $P_t$-free and $C_{\ge t}$-free graphs with no large complete bipartite subgraphs},
\newline arXiv:2012.03686. (Abstract recalls Gy\'{a}rf\'{a}s' 1985 bound for $P_t$-free graphs.)

\bibitem{GyarfasSzTuza}
A. Gy\'{a}rf\'{a}s, E. Szemer\'{e}di, and Z. Tuza,
\emph{Induced subtrees in graphs of large chromatic number},\newline
Discrete Mathematics \textbf{30} (1980), 235--244.

\bibitem{ScottInducedTrees}
A. D. Scott,
\emph{Induced trees in graphs of large chromatic number},\newline
Journal of Graph Theory \textbf{24}(4) (1997), 297--311.\newline
(Topological/subdivision version of Gy\'{a}rf\'{a}s--Sumner.)

\bibitem{KomjathShelah1988}
P. Komj\'ath and S. Shelah,
\emph{For uncountably chromatic graphs},\newline
Journal of Symbolic Logic \textbf{53} (1988), 695--703.\newline
(Consistency results for uncountably chromatic graphs and triangle-free subgraphs; see also the discussion in \S1.)

\end{thebibliography}

