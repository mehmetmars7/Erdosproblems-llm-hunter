
\noindent\textbf{FORMAL RESTATEMENT.}\\
For $n\in\mathbb{N}$ write the prime factorisation of $n(n+1)$ as
\[
 n(n+1) = \prod_i p_i^{k_i}
\]
with distinct primes $p_i$. The question asks whether there exist infinitely many $n$ such that the exponents $k_i$ are \emph{pairwise distinct} (i.e. $k_i\ne k_j$ for $i\ne j$).

\medskip
\noindent\textbf{QUICK LITERATURE/CONTEXT CHECK.}\\
The problem text gives a conditional infinite family: if there are infinitely many primes $p$ for which $8p^2-1$ is prime, then taking $n=8p^2-1$ yields exponent set $\{1,2,3\}$. I will not assume any unproved distributional statements about such primes.

\medskip
\noindent\textbf{ATTACK PLAN.}\\
\emph{Construction track.} Seek explicit parametric families of $n$ forcing a controlled factorisation of $n$ and $n+1$ (e.g. one of them a small-prime power times a prime). Use coprimality of $n$ and $n+1$.

\smallskip
\noindent\emph{Computation track.} Search small $n$ for which the distinct-exponent property holds, to guess patterns/families.

\medskip
\noindent\textbf{WORK.}\\
\emph{Phase 1: fast reality check (computation).}  I checked the condition ``all exponents in $n(n+1)$ are distinct'' by factoring $n$ and $n+1$ separately (they are coprime) and comparing the multiset of exponents.

\begin{itemize}
\item Up to $n\le 5000$, there are $57$ solutions. The first few are
\[
1,3,4,7,8,16,24,27,31,48,63,71,72,107,108,124,127,199,242,243,256,\dots
\]
\item Up to $n\le 10^5$, there are $140$ solutions.
\item Up to $n\le 10^6$, there are $299$ solutions.
\end{itemize}
(These are exact counts from a direct scan using integer factorisation.)

\medskip
\noindent\textbf{Lemma 913.1 (exponents come from $n$ and $n+1$ separately).}  \emph{Let $n\ge 1$. Since $\gcd(n,n+1)=1$, every prime divisor of $n(n+1)$ divides exactly one of $n$ and $n+1$. Hence the multiset of exponents in the prime factorisation of $n(n+1)$ is the disjoint union of the exponent multisets for $n$ and for $n+1$.}

\noindent\textbf{Proof.} The coprimality $\gcd(n,n+1)=1$ implies no prime divides both. If $p^a\parallel n$ (i.e. $p^a\mid n$ but $p^{a+1}\nmid n$), then $p\nmid n+1$ and therefore $p^a\parallel n(n+1)$. Similarly for primes dividing $n+1$. Thus exponents in $n(n+1)$ are exactly those appearing in $n$ together with those appearing in $n+1$, with no overlap of primes. \hfill$\square$

\medskip
\noindent\textbf{Lemma 913.2 (the $8p^2-1$ pattern gives exponents $\{1,2,3\}$).}  \emph{Let $p$ be prime and suppose $q:=8p^2-1$ is also prime. Set $n=q$. Then the exponents in the prime factorisation of $n(n+1)$ are $\{1,2,3\}$ (hence all distinct).}

\noindent\textbf{Proof.} By definition, $n=q$ is prime so $n=q^1$. Also
\[
 n+1 = q+1 = 8p^2 = 2^3\,p^2.
\]
Since $q$ is odd and $p$ is prime, $q,2,p$ are three distinct primes. Therefore
\[
 n(n+1)= q\cdot 2^3\,p^2,
\]
so the exponent multiset is $\{1,3,2\}$, which has three distinct elements. \hfill$\square$

\medskip
\noindent\textbf{VERIFICATION.}\\
\begin{itemize}
\item Lemma~913.1 uses only $\gcd(n,n+1)=1$ and the uniqueness of prime factorisation.
\item Lemma~913.2 verifies line-by-line the factorisation and distinctness of primes.
\item The computation does not prove infinitude; it only confirms that solutions exist and are not extremely rare at small scales.
\end{itemize}

\medskip
\noindent\textbf{FINAL.}\\
\textbf{UNRESOLVED.}

(i) Strongest proved partial result: structural reduction Lemma~913.1, and an explicit conditional infinite family Lemma~913.2 (conditional only on the existence of infinitely many primes $p$ with $8p^2-1$ prime). Computationally, there are $140$ solutions up to $10^5$ and $299$ up to $10^6$.

(ii) First gap (crisp): Prove unconditionally that there are infinitely many $n$ such that the exponents in $n(n+1)$ are all distinct, or else prove that only finitely many exist.

(iii) Top 3 next moves:
\begin{enumerate}
\item Search for other parametric patterns forcing a bounded set of primes with prescribed exponents, e.g. $n$ prime and $n+1=2^a p^b$ with $a,b$ distinct and both $\ne 1$; attempt to make such patterns accessible to sieve methods.
\item Relax primality conditions and try to prove infinitude of solutions with ``almost primes'' (e.g. $n$ having two prime factors with distinct exponents) as a stepping stone; then attempt to upgrade.
\item Extend computation to much larger ranges and classify solutions by the exponent pattern (e.g. $\{1,2,3\}$, $\{1,4\}$, $\{2,3,1\}$, etc.) to identify the dominant families.
\end{enumerate}

(iv) Minimal counterexample structure (if infinitude fails): beyond some $N$, every $n\ge N$ would force at least one repeated exponent among the prime factors of $n$ and $n+1$ (by Lemma~913.1 this repetition could occur within $n$, within $n+1$, or as the same exponent appearing once in $n$ and once in $n+1$).


