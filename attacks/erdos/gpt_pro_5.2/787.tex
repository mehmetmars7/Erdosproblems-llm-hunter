\section*{Problem 787}
\addcontentsline{toc}{section}{Problem 787}

\subsection*{1. Formal restatement}
For a finite set $A\subset\mathbb{R}$, define $\varphi(A)$ to be the largest integer $m$ such that $A$ contains distinct elements $b_1,\dots,b_m$ with
\[
 b_i+b_j\notin A\qquad\text{for all }1\le i<j\le m.
\]
Let
\[
 g(n)\coloneqq \min\{\varphi(A): A\subset\mathbb{R},\ |A|=n\}.
\]
Estimate the growth of $g(n)$.

\subsection*{2. Quick literature/context check}
The statement records the following (we list them only as context).
\begin{itemize}[leftmargin=2em]
\item A greedy algorithm gives $g(n)\gg \log n$.
\item (Choi) $g(n)\ge \log_2 n$.
\item (Ruzsa, unpublished) $g(n)>2\log_3 n-1$.
\item (Sudakov--Szemer\'edi--Vu) $g(n)$ is superlogarithmic.
\item Best known upper bound (Ruzsa): $g(n)<\exp(\sqrt{\log n})$.
\item (Sanders) $g(n)\ge (\log n)^{1+c}$ for some $c>0$.
\item (Beker) the exponent can be taken to be $1+\tfrac{1}{68}$.
\end{itemize}

\subsection*{3. Attack plan}
\begin{enumerate}[leftmargin=2em]
\item Reinterpret the problem as an independent-set problem in a ``sum graph'' on $A$.
\item Prove at least logarithmic lower bounds by bounding degrees in a suitable induced sum graph (e.g., on the positive elements), then apply a standard independent-set lower bound such as Caro--Wei.
\item For upper bounds, search for constructions of $A$ with very large additive energy so that many pairwise sums stay inside $A$, shrinking $\varphi(A)$. (This is the hard direction and uses additive-combinatorial machinery in the best known results.)
\item As a computational reality check for small $n$, brute-force over $A\subset[-M,M]\cap\mathbb{Z}$ and compute $\varphi(A)$ by exhaustive search.
\end{enumerate}

\subsection*{4. Work}
\subsubsection*{4.1. A clean $\boldsymbol{\gg\log n}$ lower bound via Caro--Wei}
The following argument is essentially the one recorded in the Tao--Vu survey on sum-free sets.

\begin{proposition}[Logarithmic lower bound]
\label{prop:g-log}
For all $n\ge 2$,
\[
 g(n)\ge H_{\lfloor n/2\rfloor},
\]
where $H_m=1+\tfrac12+\cdots+\tfrac1m$ is the $m$-th harmonic number. In particular, $g(n)\ge \log(n/2)$ for $n\ge 2$.
\end{proposition}

\begin{proof}
By a Freiman-isomorphism argument one may reduce to the case $A\subset\mathbb{Z}$ (this is standard and we do not reprove it).
Also note that $\varphi(A)=\varphi(-A)$ because $b_i+b_j\in A$ is equivalent to $-(b_i+b_j)\in -A$.
Hence, replacing $A$ by $-A$ if needed, we may assume $A$ contains at least $m\coloneqq\lfloor n/2\rfloor$ positive elements.
List $m$ of them in increasing order:
\[
0<a_1<a_2<\cdots<a_m.
\]

Define a graph $G$ on vertex set $\{a_1,\dots,a_m\}$ by joining $a_i$ and $a_j$ ($i<j$) if $a_i+a_j\in A$.
Any independent set in $G$ is a subset $B\subset A$ of positive elements whose pairwise sums avoid $A$, hence has size at most $\varphi(A)$. Therefore
\[
\varphi(A)\ge \alpha(G),
\]
where $\alpha(G)$ denotes the independence number.

For each fixed $j$, the neighbors of $a_j$ correspond to those $i\ne j$ for which $a_i+a_j\in A$. Because $a_i,a_j>0$ and distinct, each sum $a_i+a_j$ is a positive integer strictly larger than $a_j$, and different $i$ give different sums. Thus these sums must lie among the positive elements of $A$ that are larger than $a_j$, of which there are at most $m-j$. Hence
\[
\deg_G(a_j)\le m-j.
\]

Now apply the Caro--Wei bound for independent sets:
\[
\alpha(G)\ge \sum_{v\in V(G)} \frac{1}{\deg(v)+1}
\ \ge\ 
\sum_{j=1}^m \frac{1}{(m-j)+1}
\ =\ H_m.
\]
Since this holds for every $A$ of size $n$, we get $g(n)\ge H_m$.
\end{proof}

\subsubsection*{4.2. Small computational reality check}
For small $n$ one can brute-force over $A\subset[-M,M]\cap\mathbb{Z}$ and compute $\varphi(A)$ by checking all subsets $B\subset A$. For $M=10$ this search returns the following minima for $n\le 7$:
\[
 g(2)=1,\quad g(3)=1,\quad g(4)=2,\quad g(5)=2,\quad g(6)=2,\quad g(7)=2.
\]
(These match the general lower bound $g(n)\ge H_{\lfloor n/2\rfloor}$, which is $\ge 2$ for $n\ge 4$.)

\subsection*{5. Verification}
\begin{itemize}[leftmargin=2em]
\item Proposition~\ref{prop:g-log} is self-contained and yields the promised $\gg\log n$ growth.
\item The computational check is consistent with the theorem for small $n$ (it is not a proof for general $n$).
\end{itemize}

\subsection*{6. FINAL}
\textbf{UNRESOLVED.}

\medskip
\noindent\textbf{Fail-safe (required):}
\begin{enumerate}[leftmargin=2em,label=(\roman*)]
\item \textbf{Strongest partial results proved here.}
A clean logarithmic lower bound $g(n)\ge H_{\lfloor n/2\rfloor}\sim \log n$ (Proposition~\ref{prop:g-log}).
\item \textbf{First gap that blocks a full solution.}
We do not prove any superlogarithmic lower bound, nor do we reproduce Ruzsa-type constructions giving subpolynomial upper bounds such as $\exp(\sqrt{\log n})$.
\item \textbf{What would close the problem.}
Either (a) a matching upper bound construction showing $g(n)\le (\log n)^{1+o(1)}$ or similar, or (b) a proof that $g(n)$ grows faster than any fixed power of $\log n$ (together with a best-possible rate).
\item \textbf{Why computations didn't settle it.}
Brute force only accesses tiny $n$ and cannot detect the asymptotic regime where superlogarithmic behavior appears.
\end{enumerate}

\subsection*{7. COMPLETION ESTIMATE}
\textbf{30\%.} A baseline $\log n$ lower bound is proved, but the true growth rate remains open.

