\section{Round 2 Objective}

\textbf{Path chosen: (C) Obstruction / correction via rigorous reduction.}

Building strictly on the Round--1 results, the objective of Round--2 is not to restart the investigation, but to sharpen and reduce the conjecture to its strongest known and most standard formulation. The key advance is to show that the normalization used in the problem,
\[
g_n=\frac{p_{n+1}-p_n}{\log n},
\]
has exactly the same set of limit points as the classical normalization
\[
\frac{p_{n+1}-p_n}{\log p_n}
\]
used in the modern literature. This allows all known deep results to be transferred verbatim to the present setting and isolates a single precise analytic obstruction whose resolution would settle the conjecture.


\section{Round--1 Foundation Used}

We rely on the following Round--1 results without reproving them:

\begin{itemize}
\item \textbf{Lemma 1 (Closedness).} The limit--point set $S$ of $(g_n)$ is a closed subset of $[0,\infty]$.
\item \textbf{Lemma 2.} $0\in S \iff \liminf g_n=0$ and $\infty\in S \iff \limsup g_n=\infty$.
\item \textbf{Lemma 3.} If $S$ is dense in $[0,\infty]$, then $S=[0,\infty]$.
\item The definition of $g_n$ and the numerical sanity checks up to $n\le 10^5$.
\end{itemize}


\section{New Insight / Tool Introduced in Round--2}

The new ingredient is a general \emph{normalization invariance lemma}: multiplying a sequence by a factor $r_n\to 1$ does not change its set of limit points (including $\infty$). Applying this to
\[
r_n=\frac{\log p_n}{\log n},
\]
which tends to $1$, shows that the limit--point set defined using $\log n$ coincides exactly with the classical limit--point set defined using $\log p_n$. This reduction allows us to import the strongest known theorems (Merikoski; Banks--Freiberg--Maynard) directly into the present problem.


\section{Attack Plan}

The remaining Round--1 gap is density of $S$. The Round--2 plan is:

\begin{enumerate}
\item Prove that rescaling by $r_n\to 1$ preserves limit points.
\item Prove $\log p_n/\log n\to 1$ using explicit bounds for $p_n$.
\item Deduce that the present set $S$ equals the classical limit--point set $L$.
\item Import the strongest known results about $L$ (large measure, bounded gaps).
\item Isolate the precise analytic obstruction whose improvement would imply $S=[0,\infty]$.
\end{enumerate}


\section{Work}

\subsection{Rescaling Lemma}

Let $(a_n)$ be a sequence in $[0,\infty]$ and let $(r_n)$ be positive with $r_n\to 1$. Define $b_n=a_n r_n$. Then $(a_n)$ and $(b_n)$ have the same set of limit points in $[0,\infty]$.

\paragraph{Proof.}
If $a_{n_k}\to x<\infty$, then $b_{n_k}=a_{n_k}r_{n_k}\to x$. Conversely, if $b_{n_k}\to x<\infty$, then $a_{n_k}=b_{n_k}/r_{n_k}\to x$. For $\infty$, since $r_n\to 1$, eventually $1/2\le r_n\le 2$, so divergence of one sequence forces divergence of the other. \qed


\subsection{Asymptotics of $\log p_n$}

\begin{lemma}
\[
\lim_{n\to\infty}\frac{\log p_n}{\log n}=1.
\]
\end{lemma}

\paragraph{Proof.}
Since $p_n\ge n+1$, we have $\log p_n/\log n\ge \log(n+1)/\log n\to 1$. Explicit bounds of Dusart give
\[
p_n\le n(\log n+\log\log n+O(1)),
\]
hence $\log p_n\le \log n+o(\log n)$, which implies the limsup is at most $1$. \qed


\subsection{Equality of Limit--Point Sets}

Define
\[
g_n=\frac{p_{n+1}-p_n}{\log n},\qquad
h_n=\frac{p_{n+1}-p_n}{\log p_n}.
\]
Let $S$ and $L$ be the respective limit--point sets.

\begin{proposition}
$S=L$.
\end{proposition}

\paragraph{Proof.}
We have $g_n=h_n r_n$ with $r_n=\log p_n/\log n\to 1$. By the rescaling lemma, the two sequences have identical limit--point sets. \qed


\subsection{Imported Consequences}

Identifying $S=L$, we obtain immediately:

\begin{itemize}
\item (Merikoski) For all $T\ge 0$, $\mu(S\cap[0,T])\ge T/3$, and $S$ is syndetic (has bounded gaps).
\item (Banks--Freiberg--Maynard) Quantitative difference--set properties and positive lower density of $S$.
\end{itemize}

\subsection{New Corollary}

\begin{corollary}
For every $x>0$, at least one of $x,2x,3x$ lies in $S$.
\end{corollary}

\paragraph{Proof.}
Merikoski's four--point difference theorem applied to $\{0,x,2x,3x\}$ implies that one of the differences $x,2x,3x$ lies in $L=S$. \qed


\section{Adversarial Verification}

\begin{itemize}
\item Finite truncations do not affect limit points.
\item The rescaling argument handles $\infty$ explicitly.
\item No unproven hypotheses (e.g.\ RH, EH) are used in the reduction.
\item Closedness plus bounded gaps does not imply density, so no logical overreach occurs.
\end{itemize}


\section{Final Status}

\textbf{UNRESOLVED (BUT STRICTLY ADVANCED).}

The conjecture $S=[0,\infty]$ is reduced exactly to the standard formulation for normalized prime gaps. All strongest known partial results now apply directly to the original problem, and a precise analytic obstruction is isolated: improving the key sieve bound in Merikoski's framework would settle the conjecture.


\section{Completion Estimate}

\textbf{COMPLETION: 45\%}


\section{References}

\begin{itemize}
\item P.~Dusart, \emph{Estimates of some functions over primes without R.H.}
\item W.~D.~Banks, T.~Freiberg, J.~Maynard, \emph{On limit points of the sequence of normalized prime gaps}.
\item J.~Merikoski, \emph{Limit points of normalized prime gaps}.
\end{itemize}

\section{ROUND-3 OBJECTIVE}

\textbf{Path chosen: (C) Obstruction / correction (strategy-elimination) + refined necessary form of counterexamples.}

Round--2 reduced the original normalization $\log n$ to the standard $\log p_n$ normalization (showing the limit-point sets coincide) and imported the strongest known measure/syndeticity statements for the classical limit-point set. The Round--1 core gap remains: prove topological density of $S$ (hence $S=[0,\infty]$ by closedness).

In Round--3 we pursue a rigorous \emph{gap-closure obstruction}: we prove that the currently known structural properties of $S$ (closedness, syndeticity, positive-measure lower bounds, and the ``$k$-point difference'' property) \emph{do not logically imply density}. Concretely, we build a closed, syndetic, high-measure set satisfying these same properties while still missing an open interval. This rules out an entire class of argument templates and sharpens the minimal shape any counterexample interval must have.


\section{ROUND-1/2 FOUNDATION USED}

We use (and do not reprove) the following vetted Round--1/2 results:

\begin{itemize}
\item \textbf{(Round--1 Lemma 1)} $S$ is closed in $[0,\infty]$.
\item \textbf{(Round--1 Lemma 3)} If $S$ is dense in $[0,\infty]$, then $S=[0,\infty]$.
\item \textbf{(Round--2 Reduction)} The limit-point set for $\frac{p_{n+1}-p_n}{\log n}$ equals the classical limit-point set for $\frac{p_{n+1}-p_n}{\log p_n}$. Thus we may freely import theorems stated for the latter.
\item \textbf{(Imported in Round--2)} Merikoski: $\mu(S\cap[0,T])\ge T/3$ for all $T\ge 0$ and $S$ is syndetic (bounded additive gaps between limit points).
\item \textbf{(Imported in Round--2)} Banks--Freiberg--Maynard: the $k=9$ ``difference-set'' property (for any $9$ reals, some pairwise difference lies in $S$).
\end{itemize}


\section{NEW INSIGHT / TOOL (ROUND-3)}

\textbf{New tool: a quantitative \emph{relative-gap lemma}} for sets with the $k$-point difference property. It implies any open interval missing from such a set must have bounded multiplicative ratio, and (in the closed case) forces intersection with every interval $[a,(k-1)a]$.

\textbf{New obstruction: a concrete closed, syndetic, high-measure counterexample set} satisfying all currently known structural properties of $S$ but still not dense. This demonstrates that density cannot be deduced from these properties alone and therefore identifies a genuine missing ingredient needed to resolve Erd\H{o}s' conjecture.


\section{ATTACK PLAN (ROUND-3)}

\begin{enumerate}
\item Use the known $k$-point difference property (for $k=4$ and/or $k=9$) to prove a new lemma: if a set $A$ has this property, then it cannot omit an interval $(a,b)$ with $b>(k-1)a$.
\item Combine with closedness to sharpen: a closed $A$ with the property must intersect every $[a,(k-1)a]$.
\item Construct an explicit closed set $A\subset[0,\infty)$ that is syndetic, satisfies $\mu(A\cap[0,T])\ge T/3$ for all $T$, and satisfies the same $k$-point difference property as $S$, yet is not dense.
\item Conclude: any proof of $S=[0,\infty]$ must exploit additional arithmetic structure beyond the currently known abstract properties; and any counterexample interval for $S$ must obey new quantitative restrictions.
\end{enumerate}


\section{WORK (ROUND-3)}

\subsection{Imported theorem: $S$ contains an initial interval $[0,c]$ (ineffective $c>0$)}

In the classical normalization, it is known (via Zhang's bounded gaps) that there exists an ineffective constant $c>0$ such that $[0,c]\subseteq S$. Equivalently, \emph{every} real number in $[0,c]$ is a limit point of normalized prime gaps.

This is strictly stronger than $0\in S$: it gives nonempty interior near $0$.

\medskip

\subsection{Definition: the $k$-point difference property}

Let $A\subseteq[0,\infty)$. For an integer $k\ge 2$, say that $A$ has property $(P_k)$ if for every real numbers
\[
\beta_1\le\beta_2\le\cdots\le\beta_k
\]
we have
\[
A\cap\{\beta_j-\beta_i:1\le i<j\le k\}\neq\emptyset.
\]
By Merikoski's Theorem~1, $S$ has $(P_4)$; by Banks--Freiberg--Maynard, $S$ has $(P_9)$.

\subsection{New Lemma: relative-gap constraint from $(P_k)$}

\begin{lemma}[Relative-gap obstruction]
Assume $A\subseteq[0,\infty)$ has $(P_k)$. Let $a>0$ and $b>(k-1)a$. Then $(a,b)\cap A\neq\emptyset$.
Equivalently: if $(a,b)\cap A=\emptyset$ with $a>0$, then necessarily $b\le (k-1)a$.
\end{lemma}

\paragraph{Proof.}
Assume for contradiction that $(a,b)\cap A=\emptyset$ with $b>(k-1)a$. Choose $\varepsilon>0$ such that
\[
a+\varepsilon<\frac{b}{k-1},
\]
which is possible exactly because $b>(k-1)a$.

Define $k$ points by
\[
\beta_i:=(i-1)(a+\varepsilon),\qquad i=1,2,\dots,k.
\]
Then every difference has the form
\[
\beta_j-\beta_i=(j-i)(a+\varepsilon)\in\{(a+\varepsilon),2(a+\varepsilon),\dots,(k-1)(a+\varepsilon)\}.
\]
Each such number lies in $(a,b)$ since $a+\varepsilon>a$ and $(k-1)(a+\varepsilon)<b$. Hence
\[
\{\beta_j-\beta_i:1\le i<j\le k\}\subset (a,b),
\]
so it is disjoint from $A$, contradicting $(P_k)$. \qed

\subsection{Corollary: closed + $(P_k)$ forces multiplicative syndeticity}

\begin{corollary}
If $A\subseteq[0,\infty)$ is closed and has $(P_k)$, then for every $a>0$,
\[
A\cap[a,(k-1)a]\neq\emptyset.
\]
In particular, since $S$ is closed and has $(P_4)$, we have
\[
S\cap[a,3a]\neq\emptyset\quad\text{for all }a>0,
\]
and any open interval $(a,b)$ disjoint from $S$ must satisfy $b\le 3a$.
\end{corollary}

\paragraph{Proof.}
Fix $a>0$ and suppose $A\cap[a,(k-1)a]=\emptyset$. Since $A$ is closed and $[a,(k-1)a]$ is compact, the distance
\[
d:=\mathrm{dist}\bigl(A,\,[a,(k-1)a]\bigr)
\]
is strictly positive. Put $\delta:=\min(d/2,a/2)$ so that $a-\delta>0$ and
\[
(a-\delta,\,(k-1)a+\delta)\cap A=\emptyset.
\]
But then
\[
(k-1)a+\delta > (k-1)(a-\delta)
\]
(since the difference is $k\delta>0$), hence the interval $(a-\delta,(k-1)a+\delta)$ has right endpoint $>(k-1)$ times its left endpoint. This contradicts the relative-gap lemma. \qed

\subsection{New obstruction example: known structural properties do not imply density}

We now show that the \emph{package} of currently known abstract properties of $S$ is insufficient to deduce density.

\begin{proposition}[Strategy-elimination example]
Let
\[
A:=[0,1]\cup[3,\infty)\subset[0,\infty).
\]
Then:
\begin{enumerate}
\item $A$ is closed.
\item $A$ is syndetic (indeed, every interval of length $2$ intersects $A$).
\item For all $T\ge 0$, $\mu(A\cap[0,T])\ge T/3$.
\item $A$ has $(P_4)$ and $(P_9)$ (in fact it has $(P_k)$ for every $k\ge 4$).
\item $A$ is \emph{not} dense in $[0,\infty)$ (it misses the open interval $(1,3)$).
\end{enumerate}
\end{proposition}

\paragraph{Proof.}
(1) Closedness is immediate.

(2) Let $T\ge 0$. If $T\le 1$, then $[T,T+2]$ intersects $[0,1]\subset A$. If $1<T<3$, then $3\in[T,T+2]\cap A$. If $T\ge 3$, then $[T,T+2]\subset[3,\infty)\subset A$. Hence every interval of length $2$ intersects $A$.

(3) Compute $\mu(A\cap[0,T])$:
\[
\mu(A\cap[0,T])=
\begin{cases}
T, & 0\le T\le 1,\\
1, & 1<T\le 3,\\
1+(T-3)=T-2, & T\ge 3.
\end{cases}
\]
For $0\le T\le 1$, we have $T\ge T/3$. For $1<T\le 3$, we have $1\ge T/3$. For $T\ge 3$, we have $T-2\ge T/3$ since $2T/3\ge 2$.

(4) To prove $(P_4)$, let $\beta_1\le\beta_2\le\beta_3\le\beta_4$. If any consecutive difference satisfies $\beta_{i+1}-\beta_i\le 1$, then that difference lies in $[0,1]\subset A$ and we are done. Otherwise all three consecutive differences exceed $1$, so
\[
\beta_4-\beta_1=(\beta_2-\beta_1)+(\beta_3-\beta_2)+(\beta_4-\beta_3)>3,
\]
hence $\beta_4-\beta_1\in[3,\infty)\subset A$, and we are done. Thus $(P_4)$ holds.
The proof for $(P_9)$ is identical: if no consecutive difference is $\le 1$, then $\beta_9-\beta_1>8$, hence belongs to $[3,\infty)\subset A$.

(5) $A$ is not dense because $(1,3)\cap A=\emptyset$. \qed

\subsection{Round--3 consequence: what a counterexample interval for $S$ must look like}

Assume (for contradiction) that $S\neq[0,\infty]$. By Round--1 Lemma~1, there exists a nonempty open interval $(a,b)\subset[0,\infty)$ with $(a,b)\cap S=\emptyset$.

Combining:
\begin{itemize}
\item the imported fact that $[0,c]\subseteq S$ for some $c>0$ (ineffective), and
\item the corollary from $(P_4)$ (relative-gap obstruction), and
\item syndeticity of $S$ (bounded additive gaps),
\end{itemize}
we obtain the following necessary constraints on any such missing interval $(a,b)$:

\begin{enumerate}
\item $a\ge c$ (since $(0,c)$ is fully contained in $S$).
\item $b\le 3a$ (since $(P_4)$ forbids gaps with multiplicative ratio $>3$).
\item $b-a\le C_S$ for some absolute (ineffective) syndeticity constant $C_S$ (since $S$ intersects every interval of length $C_S$).
\end{enumerate}

Thus any failure of density must come from a \emph{short} additive gap occurring \emph{above} the initial interval $[0,c]$, and with \emph{bounded} multiplicative ratio.

\medskip
\noindent\textbf{Crucial meta-point:} Proposition above shows these constraints and the known abstract properties are \emph{far} from sufficient to force density; any genuine proof of Erd\H{o}s' conjecture must use arithmetic structure beyond closedness + syndeticity + measure bounds + $(P_k)$.


\section{ADVERSARIAL VERIFICATION}

\begin{itemize}
\item \textbf{Quantifier check for $(P_k)$.} The proofs use only that $\beta_1\le\cdots\le\beta_k$; differences are automatically nonnegative, consistent with $A\subset[0,\infty)$.
\item \textbf{Edge case $a\to 0$ in the corollary.} We ensured positivity of the left endpoint by choosing $\delta=\min(d/2,a/2)$ so $a-\delta>0$.
\item \textbf{Tightness of the relative-gap lemma.} The lemma excludes $b>(k-1)a$ but permits $b=(k-1)a$. The example $A=[0,1]\cup[3,\infty)$ has an actual missing interval $(1,3)$ with ratio exactly $3$, confirming the bound is sharp in general.
\item \textbf{Interaction with known results for $S$.} Since the example set $A$ satisfies the same package of known properties as $S$ (closed, syndetic, $\mu(\cdot)\ge T/3$, $(P_4)$ and $(P_9)$), it rigorously demonstrates that these properties alone cannot imply density.
\item \textbf{No circularity.} The example $A$ is purely combinatorial; it does not depend on prime gaps. It is used only to refute a class of deduction strategies, not to claim anything about the actual $S$ beyond logical independence.
\end{itemize}


\section{FINAL}

\textbf{UNRESOLVED (BUT STRICTLY ADVANCED).}

New Round--3 contributions beyond Round--2:
\begin{enumerate}
\item Imported (via the standard literature) the stronger fact that $S$ contains a full initial interval $[0,c]$ for some (ineffective) $c>0.
\item Proved a new relative-gap lemma: if a set has the $k$-point difference property $(P_k)$, it cannot omit an interval $(a,b)$ with $b>(k-1)a$; for closed sets this forces intersection with every $[a,(k-1)a]$.
\item Gave an explicit closed, syndetic, high-measure set $A=[0,1]\cup[3,\infty)$ satisfying $(P_4)$ and $(P_9)$ but not dense, showing that the current package of abstract properties known for $S$ is insufficient to deduce Erd\H{o}s' conjecture.
\item Sharpened the necessary form of any counterexample interval for $S$: it must start above $c$ and satisfy $b\le 3a$ and $b-a\le C_S$.
\end{enumerate}

These results do not resolve density, but they rule out a broad family of ``purely topological/combinatorial'' proof attempts and sharpen the target for any genuine analytic advance.


\section{COMPLETION ESTIMATE}

\textbf{COMPLETION: 55\%}


\section{REFERENCES}

\begin{itemize}
\item W.\,D.\ Banks, T.\ Freiberg, J.\ Maynard, \emph{On limit points of the sequence of normalized prime gaps}.
\item J.\ Merikoski, \emph{Limit points of normalized prime gaps}.
\end{itemize}

\section{ROUND-4 OBJECTIVE}

\textbf{Path chosen: (C) Obstruction / strategy-elimination + sharpness.}

Round--3 isolated a genuine obstruction: from the currently known abstract properties of the
limit-point set $S$ (closedness, syndeticity, positive measure, and the $k$-point difference property
$(P_k)$), one \emph{cannot} deduce density of $S$.
The Round--3 example $A=[0,1]\cup[3,\infty)$ already showed that these properties do not imply
$S=[0,\infty)$.

In Round--4 we strengthen this obstruction in two mathematically substantive ways:

\begin{enumerate}
\item We prove a \emph{sharpness theorem} showing that the Round--3 multiplicative gap constraint
$b\le 3a$ is essentially optimal: for any interval $(a,b)$ with $0<a<b\le 3a$ there exists a closed,
syndetic set satisfying $(P_4)$ and the measure lower bound $\mu(\cdot)\ge T/3$, yet missing $(a,b)$.
\item We build an \emph{infinite family} of such counter-model sets with missing intervals occurring
\emph{arbitrarily far out} while simultaneously containing an \emph{arbitrarily long initial segment}
$[0,L]$ and even having very small syndetic constant. This rules out an entire class of proof
strategies that try to deduce density from the known ``global'' structural properties of $S$.
\end{enumerate}

These results do not resolve Erd\H{o}s' conjecture for the actual set $S$, but they rigorously show
that any successful proof must use additional arithmetic structure beyond $(P_4)$-type
combinatorics, measure bounds, and syndeticity.


\section{ROUND-3 FOUNDATION USED}

We rely on the following Round--3 results (treated as vetted progress):

\begin{itemize}
\item \textbf{(R1 Lemma 1)} The limit-point set $S$ is closed in $[0,\infty]$.
\item \textbf{(R3 Definition)} $(P_k)$: for all $\beta_1\le\cdots\le\beta_k$, the difference set
$\{\beta_j-\beta_i:1\le i<j\le k\}$ meets $S$.
\item \textbf{(R3 Relative-gap lemma)} If a set $A\subset[0,\infty)$ has $(P_k)$ and
$(a,b)\cap A=\emptyset$ with $a>0$, then $b\le (k-1)a$.
In particular, for $(P_4)$ one must have $b\le 3a$.
\item \textbf{(R3 Corollary)} If $A$ is closed and has $(P_k)$, then $A\cap[a,(k-1)a]\neq\emptyset$
for every $a>0$.
\item \textbf{(Imported earlier)} For the actual prime-gap limit-point set $S$, Merikoski proves $(P_4)$,
$\mu(S\cap[0,T])\ge T/3$ for all $T\ge 0$, and $S$ is syndetic; and also cites Pintz that
$[0,c]\subseteq S$ for some ineffective $c>0$.
\end{itemize}


\section{NEW INSIGHT / TOOL (ROUND-4)}

\textbf{New tool: a sharpness construction for $(P_k)$.}

We show that the Round--3 multiplicative gap constraint is best possible in the following strong sense:
whenever $0<a<b\le (k-1)a$, the simple \emph{two-block set}
\[
A_{a,b}:=[0,a]\cup[b,\infty)
\]
is closed, syndetic, satisfies $(P_k)$, and already saturates the general measure lower bound
$\mu(A\cap[0,T])\ge T/(k-1)$ at $T=b$ when $b=(k-1)a$.

In particular, for $k=4$ we obtain a large family of counter-models that satisfy the same abstract
properties currently proved for $S$, while still missing an arbitrary prescribed open interval $(a,b)$
with $b\le 3a$.
This strictly strengthens the Round--3 strategy-elimination example.


\section{ATTACK PLAN (ROUND-4)}

\begin{enumerate}
\item Prove the sharpness construction: $A_{a,b}=[0,a]\cup[b,\infty)$ satisfies $(P_k)$ whenever
$b\le (k-1)a$.
\item Verify that the same $A_{a,b}$ is closed, syndetic, and satisfies the measure bound
$\mu(A_{a,b}\cap[0,T])\ge T/(k-1)$ for all $T\ge 0$ (hence $\ge T/3$ for $k=4$).
\item Specialize to $k=4$ and exhibit an infinite family with:
\begin{itemize}
\item missing intervals located arbitrarily far out (large $a$),
\item arbitrarily long initial segment containment $[0,L]\subset A_{a,b}$,
\item and even very small syndetic constants (e.g.\ $1$).
\end{itemize}
\item Conclude that density of $S$ cannot be deduced from the presently known abstract properties;
a proof of Erd\H{o}s' conjecture must exploit additional arithmetic structure.
\end{enumerate}


\section{WORK (ROUND-4)}

\subsection{Reminder: $(P_k)$}

For $k\ge 2$, a set $A\subseteq[0,\infty)$ has property $(P_k)$ if for every choice of real numbers
$\beta_1\le\cdots\le\beta_k$ we have
\[
A\cap\{\beta_j-\beta_i:1\le i<j\le k\}\neq\emptyset.
\]
(As noted in Round--3, $(P_4)$ implies $(P_k)$ for all $k\ge 4$ by restriction to any four points.)


\subsection{Sharpness construction for $(P_k)$}

\begin{theorem}[Sharpness of the relative-gap constraint]\label{thm:sharpnessPk}
Fix an integer $k\ge 2$ and reals $0<a<b$ with $b\le (k-1)a$.
Define
\[
A_{a,b}:=[0,a]\cup[b,\infty)\subset[0,\infty).
\]
Then:

\begin{enumerate}
\item $A_{a,b}$ is closed and is not dense (it misses the open interval $(a,b)$).
\item $A_{a,b}$ is syndetic (every closed interval of length $b-a$ meets $A_{a,b}$).
\item $A_{a,b}$ satisfies $(P_k)$.
\item For every $T\ge 0$,
\[
\mu(A_{a,b}\cap[0,T])\ \ge\ \frac{T}{k-1}.
\]
In particular, for $k=4$ we have $\mu(A_{a,b}\cap[0,T])\ge T/3$ for all $T\ge 0$.
\end{enumerate}
\end{theorem}

\paragraph{Proof.}
(1) Closedness is clear since $A_{a,b}$ is a union of closed intervals.
Non-density holds since $(a,b)\cap A_{a,b}=\emptyset$.

\smallskip
(2) Let $C:=b-a$. We claim every closed interval $[x,x+C]$ intersects $A_{a,b}$.
If $x\le a$, then $[x,x+C]$ intersects $[0,a]\subset A_{a,b}$.
If $x>a$, then $x+C>x+(b-a)>b$, hence $[x,x+C]$ intersects $[b,\infty)\subset A_{a,b}$.
Thus $A_{a,b}$ is syndetic.

\smallskip
(3) Let $\beta_1\le\cdots\le\beta_k$.
If some consecutive gap $\beta_{i+1}-\beta_i\le a$, then this difference lies in $[0,a]\subset A_{a,b}$.
Otherwise every consecutive gap satisfies $\beta_{i+1}-\beta_i>a$, so
\[
\beta_k-\beta_1=\sum_{i=1}^{k-1}(\beta_{i+1}-\beta_i)>(k-1)a\ge b,
\]
hence $\beta_k-\beta_1\in[b,\infty)\subset A_{a,b}$.
In all cases, some pairwise difference lies in $A_{a,b}$, i.e.\ $(P_k)$ holds.

\smallskip
(4) We compute $\mu(A_{a,b}\cap[0,T])$ explicitly:

\[
\mu(A_{a,b}\cap[0,T])=
\begin{cases}
T, & 0\le T\le a,\\
a, & a<T<b,\\
a+(T-b)=T-(b-a), & T\ge b.
\end{cases}
\]

If $0\le T\le a$, then $\mu=T\ge T/(k-1)$.
If $a<T<b$, then $\mu=a\ge b/(k-1)\ge T/(k-1)$ since $T<b\le (k-1)a$.
If $T\ge b$, then
\[
\mu=T-(b-a)\ \ge\ T-\Bigl(b-\frac{b}{k-1}\Bigr)=\frac{T}{k-1},
\]
because $a\ge b/(k-1)$.
This proves the measure bound for all $T\ge 0$.
\qed

\subsection{Consequences for $k=4$: optimality of the Round--3 constraint}

Specializing Theorem~\ref{thm:sharpnessPk} to $k=4$ yields:

\begin{corollary}[Sharpness at $k=4$]
For every $0<a<b\le 3a$, the set $A_{a,b}=[0,a]\cup[b,\infty)$ is closed, syndetic, satisfies $(P_4)$,
and obeys $\mu(A_{a,b}\cap[0,T])\ge T/3$ for all $T\ge 0$, yet $(a,b)\cap A_{a,b}=\emptyset$.
\end{corollary}

Together with the Round--3 relative-gap lemma (necessity of $b\le 3a$ for any set with $(P_4)$),
this shows that the multiplicative restriction $b\le 3a$ is \emph{optimal}: it cannot be improved
purely from $(P_4)$.

\subsection{A strengthened strategy-elimination family: far-out holes + long initial segment}

Fix any $L>0$ and any $M>L$ with $M\ge \tfrac12$.
Set $a:=M$ and $b:=M+1$ (so $b\le 3a$ holds for $M\ge \tfrac12$) and define
\[
A_M := A_{M,M+1}=[0,M]\cup[M+1,\infty)= [0,\infty)\setminus(M,M+1).
\]
Then:

\begin{itemize}
\item $[0,L]\subseteq A_M$ (indeed, $[0,M]\subseteq A_M$), so $A_M$ contains an arbitrarily long
initial interval.
\item $A_M$ is syndetic with constant $1$ (every closed interval of length $1$ meets $A_M$).
\item By Theorem~\ref{thm:sharpnessPk} (with $k=4$), $A_M$ satisfies $(P_4)$ and
$\mu(A_M\cap[0,T])\ge T/3$ for all $T$.
\item $A_M$ is not dense since it misses the open interval $(M,M+1)$, and this missing interval can be
placed arbitrarily far out by taking $M$ large.
\end{itemize}

\noindent\textbf{Conclusion.}
Even if one strengthens the abstract hypotheses drastically --- closedness, $(P_4)$, the uniform measure
bound $\mu(\cdot)\ge T/3$ \emph{for all $T$}, syndeticity with \emph{tiny} constant, and containment of an
arbitrarily long initial segment --- density still does not follow.
Hence, any proof of Erd\H{o}s' conjecture for the genuine prime-gap set $S$ must use additional
arithmetic structure beyond these global properties.


\section{ADVERSARIAL VERIFICATION}

\begin{itemize}
\item \textbf{Boundary case $b=(k-1)a$.} In the $(P_k)$ proof we use $\beta_k-\beta_1>(k-1)a$,
hence $\beta_k-\beta_1>b$ even when $b=(k-1)a$, so membership in $[b,\infty)$ is valid.
\item \textbf{Syndeticity convention (open vs.\ closed intervals).} The proof uses closed intervals $[x,x+C]$.
If one uses open intervals $(x,x+C)$, then one should take any $C'>b-a$ instead; this does not affect
the strategy-elimination conclusions.
\item \textbf{Measure bound check in the regime $a<T<b$.} The inequality reduces to $a\ge T/(k-1)$ for all
$T<b$, which is guaranteed by $b\le (k-1)a$ and is tight at $T=b$ when $b=(k-1)a$.
\item \textbf{Interaction with Round--3 relative-gap lemma.} Our construction matches the Round--3 necessary
condition $b\le (k-1)a$, showing it is sharp; there is no contradiction.
\item \textbf{Relevance to the actual $S$.} These sets $A_{a,b}$ are \emph{model sets} sharing the same abstract
properties currently proved for $S$. They do not claim anything about primes, but they rigorously show
why these properties alone cannot settle the conjecture.
\end{itemize}


\section{FINAL}

\textbf{UNRESOLVED (BUT STRICTLY ADVANCED).}

Round--4 establishes sharpness and a stronger strategy-elimination principle:

\begin{enumerate}
\item The Round--3 multiplicative restriction on missing intervals for sets with $(P_4)$,
namely $b\le 3a$, is \emph{optimal}: any open interval $(a,b)$ with $b\le 3a$ can be removed from a
closed syndetic set still satisfying $(P_4)$ and $\mu(\cdot)\ge T/3$ for all $T$.
\item There exist counter-model sets with \emph{arbitrarily long} initial intervals and
\emph{arbitrarily far-out} missing intervals, even with syndetic constant $1$ and the full uniform
measure lower bound $T/3$.
\end{enumerate}

Thus any proof of Erd\H{o}s' conjecture for the genuine limit-point set of normalized prime gaps must
use substantially more arithmetic information than the currently known global structural properties.


\section{COMPLETION ESTIMATE}

COMPLETION: 60\%


\section{REFERENCES}

\begin{itemize}
\item W.\,D.\ Banks, T.\ Freiberg, J.\ Maynard, \emph{On limit points of the sequence of normalized prime gaps},
arXiv:1404.5094v2.
\item J.\ Merikoski, \emph{Limit points of normalized prime gaps}, arXiv:1811.03008v3.
\item J.\ Pintz, \emph{On the distribution of gaps between consecutive primes}, arXiv:1407.2213 (for the interval $[0,c]\subseteq L$ result cited by Merikoski).
\end{itemize}
