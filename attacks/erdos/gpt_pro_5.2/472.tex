% Erdos Problem #472
% URL: https://www.erdosproblems.com/472

1) “FORMAL RESTATEMENT”

Let m>=1 and let q_1<...<q_m be primes. We define a (possibly finite) sequence (q_n)_{n>=1} inductively as follows.

For each n>=m, let S_n be the finite set of integers of the form

    q_n + q_i - 1  with  1<=i<=n.

If S_n contains at least one prime, define q_{n+1} to be the smallest prime in S_n. If S_n contains no primes, the construction stops at q_n.

Question: Does there exist some initial finite prime sequence q_1<...<q_m for which this construction never stops (i.e. produces infinitely many primes)?

(Interpretation note: the natural reading is that i ranges over all previously constructed terms 1<=i<=n, not just i<=m.)

2) “QUICK LITERATURE/CONTEXT CHECK”

A quick check of the Erdos Problems page for #472 shows it is listed as OPEN (as of mid-Jan 2026). I do not use any external results beyond the problem statement.

3) “ATTACK PLAN”

Try to prove existence of an infinite run by showing that at each stage n there is always at least one prime among the finitely many candidates q_n+q_i-1. This resembles a finite-shift prime-hitting problem, but the shifts (q_i-1) grow with n.

Try to disprove existence by constructing an unavoidable “dead end”: a prime Q such that Q+q_i-1 is composite for every previous q_i.

In parallel, do computational experiments for small initial sequences to see whether termination happens and what “dead ends” look like.

4) “WORK”

Lemma 472.1 (monotonicity and parity).
Assume the process has defined q_1,...,q_n with n>=m. Then every candidate q_n+q_i-1 is strictly larger than q_n. Hence, if q_{n+1} exists then q_{n+1} > q_n, so the sequence is strictly increasing. Moreover, if all initial primes are odd (in particular if q_1>=3), then every q_n is odd and every step size q_{n+1}-q_n is even.

Proof.
Fix i with 1<=i<=n. Since q_i>=2, we have q_n+q_i-1 >= q_n+1 > q_n, so all candidates exceed q_n, and if a prime candidate exists then the minimum prime candidate also exceeds q_n.
For the parity: if q_i and q_n are odd primes, then q_n+q_i-1 is odd+odd-1 = odd+even = odd. Thus every prime candidate is odd, and by induction every q_{n+1} (when defined) is an odd prime. Then q_{n+1}-q_n is an even difference of odd integers.  QED.

Lemma 472.2 (single-prime start criterion).
If m=1 and q_1=p is prime, then the construction defines q_2 if and only if 2p-1 is prime; in that case q_2=2p-1.

Proof.
When n=1, the only available index is i=1, so S_1 consists of the single integer q_1+q_1-1 = 2p-1. If this integer is prime, then by definition q_2 is the smallest prime in S_1, hence equals 2p-1. If it is not prime, then S_1 contains no primes and the process stops at q_1.  QED.

Lemma 472.3 (gaps come from earlier primes).
Whenever q_{n+1} is defined, there exists some index i<=n such that

    q_{n+1} - q_n + 1 = q_i.

In particular, each “gap plus 1” is itself a prime already present in the sequence.

Proof.
By definition, q_{n+1} has the form q_n + q_i - 1 for some i<=n. Rearranging gives q_{n+1}-q_n+1 = q_i.  QED.

FAST REALITY CHECK (local computation).
I implemented the definition exactly as in the formal restatement (i ranges over 1..n). Here are concrete outputs.

(a) Starting from (3,5), the first 50 terms are:
3, 5, 7, 11, 13, 17, 19, 23, 29, 31, 37, 41, 43, 47, 53, 59, 61, 67, 71, 73, 79, 83, 89, 101, 103, 107, 109, 113, 131, 137, 139, 149, 151, 157, 163, 167, 173, 179, 181, 191, 193, 197, 199, 211, 223, 227, 229, 233, 239, 241.

(b) Brute sampling of initial sequences built from primes < 50:
- Among all single-prime starts (m=1) with p<50, exactly 4 starts survived to length 200 without stopping: (2), (3), (7), (19).
- Among all 2-prime starts (m=2) with both primes <50, 19 starts survived to length 200 without stopping.
(“Survived” means no dead end was encountered within the first 200 generated terms.)

These computations do NOT prove infinitude, but they show the process can persist for hundreds of steps for many starts.

5) “VERIFICATION”

- The monotonicity proof uses only q_i>=2, so it applies even if q_1=2.
- If q_1=2, Lemma 472.1 does not force all q_n odd at n=1, but after the first step the sequence quickly enters odd primes in experiments.
- Lemma 472.2 exactly matches computation: e.g. p=5 gives 2p-1=9 composite, so the process stops immediately at length 1.

6) FINAL

UNRESOLVED

(i) Strongest proved partial result:
The sequence (when defined) is strictly increasing, and for odd starts consists entirely of odd primes with even gaps. Also, for m=1 the process continues iff 2p-1 is prime (Lemma 472.2), and each gap+1 equals a previous prime (Lemma 472.3).

(ii) First gap (crisp statement):
Prove or disprove: there exists an initial finite prime list q_1<...<q_m such that for every n>=m, at least one of the finitely many integers q_n+q_i-1 (1<=i<=n) is prime.

(iii) Top 3 next moves (concrete):
1. Try to construct an explicit “dead end” prime Q for a given start by forcing every Q+q_i-1 to have a small prime divisor via congruence conditions (a covering-system style obstruction).
2. Try to prove a sieve-theoretic lower bound guaranteeing that among the candidate set {q_n+q_i-1} there is a prime for infinitely many n, possibly using that the set of shifts grows with n.
3. Computationally search for starts that maximize survival length and record the factorization patterns of Q+q_i-1 at the first dead end Q (if any), to guess a structural obstruction.

(iv) Minimal counterexample structure (if the answer is “no”):
For every start, there exists a terminal index N and a prime Q:=q_N such that for all i<=N, the integer Q+q_i-1 is composite.


