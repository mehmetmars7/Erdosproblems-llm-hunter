\section*{Problem \#261}
\addcontentsline{toc}{section}{Problem \#261}

\subsection*{1. Formal Restatement}
For $n\in\mathbb{N}$, consider the Diophantine equation
\begin{equation}
\label{eq:261}
\frac{n}{2^n}=\sum_{k=1}^{t}\frac{a_k}{2^{a_k}},
\end{equation}
where $t\ge 2$ and $a_1,\dots,a_t$ are distinct positive integers.

Questions:
\begin{enumerate}[label=(\alph*),leftmargin=2em]
\item Are there infinitely many $n$ for which \eqref{eq:261} is solvable?
\item Is \eqref{eq:261} solvable for \emph{all} $n$?
\item Does there exist a \emph{rational} $x$ for which
\[
 x = \sum_{k=1}^{\infty}\frac{a_k}{2^{a_k}}
\]
has at least $2^{\aleph_0}$ distinct solutions (i.e. uncountably many distinct increasing sequences $(a_k)$)?
\end{enumerate}

\subsection*{2. Quick Literature / Context Check}
Borwein--Loring exhibited an explicit infinite family of $n$ admitting a solution (settling (a) positively), and proposed a greedy-type algorithm and conjectures toward (b). Tengely--Ulas--Zygad\l{}o extended computations and verified solvability for all $n\le 10^4$. The existence of a rational $x$ with continuum-many representations appears open.

\subsection*{3. Attack Plan}
\begin{itemize}[leftmargin=2em]
\item For (a), prove the explicit identity
\[
\frac{n}{2^n}=\sum_{n<k\le n+m}\frac{k}{2^k}\quad\text{for }n=2^{m+1}-m-2,
\]
which yields infinitely many $n$.
\item For (b), one would need a general constructive procedure producing a finite set of indices $\{a_1<\dots<a_t\}$ for each $n$. Greedy strategies (choose the largest $a$ with $a/2^a\le n/2^n$ and iterate) work numerically but lack a proof of termination.
\item For (c), one seeks a rational $x$ with extreme non-uniqueness of representation; known constructions give irrationals with uncountably many representations, and rationals with many (finite) representations, but the ``continuum many'' rational case remains unclear.
\end{itemize}

\subsection*{4. Work}
\subsubsection*{4.1 A complete proof for (a): infinitely many $n$}

\begin{theorem}[Borwein--Loring/Cusick family]
\label{thm:261-infinite}
For every integer $m\ge 1$, let
\[
 n := 2^{m+1}-m-2.
\]
Then
\begin{equation}
\label{eq:261-identity}
\frac{n}{2^n} = \sum_{k=n+1}^{n+m}\frac{k}{2^k},
\end{equation}
which gives a solution to \eqref{eq:261} with $t=m$ and $a_k=n+k$.
In particular, there are infinitely many $n$ for which \eqref{eq:261} is solvable.
\end{theorem}

\begin{proof}
We use the closed form for dyadic ``tails'':
\begin{equation}
\label{eq:tail}
\sum_{k=N+1}^{\infty} \frac{k}{2^k} = \frac{N+2}{2^N},\qquad N\ge 1.
\end{equation}
(One derivation: start from $\sum_{k\ge 1} kx^k = x/(1-x)^2$ and specialize to $x=1/2$, then subtract the first $N$ terms.)

Applying \eqref{eq:tail} twice gives
\[
\sum_{k=n+1}^{n+m}\frac{k}{2^k}
=\sum_{k=n+1}^{\infty}\frac{k}{2^k}-\sum_{k=n+m+1}^{\infty}\frac{k}{2^k}
=\frac{n+2}{2^n}-\frac{n+m+2}{2^{n+m}}.
\]
Therefore \eqref{eq:261-identity} is equivalent to
\[
\frac{n}{2^n}=\frac{n+2}{2^n}-\frac{n+m+2}{2^{n+m}}.
\]
Multiplying by $2^n$ and rearranging yields
\[
\frac{n+m+2}{2^m}=2
\quad\Longleftrightarrow\quad
n+m+2 = 2^{m+1}
\quad\Longleftrightarrow\quad
n=2^{m+1}-m-2,
\]
which is exactly how $n$ was defined.
\end{proof}

\subsubsection*{4.2 Tiny-case computation (evidence toward (b))}
The following simple depth-first search (over finite subsets of indices) can find solutions to \eqref{eq:261} for small $n$:
\begin{verbatim}
Input: n, MaxExponent M
Target := n/2^n
Search over subsets A of {1,2,...,M} with |A|>=2:
    if sum_{a in A} a/2^a == Target then output A
Prune using that the terms a/2^a decrease for a>=2.
\end{verbatim}
A quick run for $1\le n\le 20$ (with $M=400$) finds at least one solution for every such $n$.
For example:
\[
\frac{1}{2}=\frac{3}{2^3}+\frac{6}{2^6}+\frac{8}{2^8},\qquad
\frac{4}{2^4}=\frac{5}{2^5}+\frac{6}{2^6}.
\]
This is consistent with the verified range $n\le 10^4$ in the literature.

\subsubsection*{4.3 Status of (b) and (c)}
No general proof is known (to the best of the sources consulted) that \eqref{eq:261} is solvable for every $n$. Likewise, the existence of a \emph{rational} $x$ with $2^{\aleph_0}$ distinct representations of the form $\sum a_k/2^{a_k}$ appears open.

\subsection*{5. Verification / Sanity Checks}
\begin{itemize}[leftmargin=2em]
\item The tail formula \eqref{eq:tail} can be checked by differentiating the geometric series.
\item In Theorem~\ref{thm:261-infinite}, all $a_k=n+k$ are distinct and $t=m\ge 2$ is achieved for $m\ge 2$; for $m=1$ the identity gives a one-term representation, but $m\ge 2$ already yields infinitely many examples with $t\ge 2$.
\item The displayed small-$n$ examples can be verified by direct arithmetic.
\end{itemize}

\subsection*{6. FINAL}
\textbf{UNRESOLVED}

\subsubsection*{Why stuck / what remains open}
Part (a) is completely solved (infinitely many $n$). The open parts are:
\begin{itemize}[leftmargin=2em]
\item (b) proving solvability for \emph{every} $n$ (or producing a counterexample),
\item (c) finding a \emph{rational} value with continuum-many distinct representations.
\end{itemize}

\subsubsection*{Strongest partial result proved here}
Theorem~\ref{thm:261-infinite} gives an explicit infinite family of $n$ for which \eqref{eq:261} has a solution.

\subsubsection*{Next steps (concrete)}
\begin{enumerate}[leftmargin=2em]
\item Analyze the Borwein--Loring greedy algorithm: prove termination for all $n$, or find a structural obstruction.
\item Use structural constraints from divisibility and size bounds on $a_t$ (as in later work) to reduce (b) to finitely many configurations for each $n$.
\item For (c), attempt to build a rational $x$ admitting infinitely many independent ``local rewrite'' identities to force $2^{\aleph_0}$ representations.
\end{enumerate}

\subsection*{7. Completion Estimate}
\textbf{55\%}. Part (a) is fully proved; parts (b) and (c) remain open.


%%%%%%%%%%%%%%%%%%%%%%%%%%%%%%%%%%%%%%%%%%%%%%%%%%%%%%%%%%%%%%%%%%%%%%%%%%%%%%%
