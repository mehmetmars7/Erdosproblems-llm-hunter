% Erdos Problem #1100

\medskip
\noindent\textbf{FORMAL RESTATEMENT}

\smallskip
\noindent
For an integer $n\ge 1$, let $1=d_1<d_2<\cdots<d_{\tau(n)}=n$ be the increasing list of all positive divisors of $n$.
Define
\[
\tau_\perp(n):=\#\{i\in\{1,\dots,\tau(n)-1\}:\ \gcd(d_i,d_{i+1})=1\}.
\]
Let $\omega(n)$ denote the number of distinct prime divisors of $n$.

\smallskip
\noindent
\textbf{Questions.}
\begin{enumerate}
\item Is it true that $\tau_\perp(n)/\omega(n)\to\infty$ for almost all $n$ (natural density)?
\item Is it true that $\tau_\perp(n)<\exp((\log n)^{o(1)})$ for all $n$?
\item Restrict to squarefree $n$. For $k\ge 1$ define
\[
 g(k):=\max_{\substack{n\ \text{squarefree}\\ \omega(n)=k}} \tau_\perp(n).
\]
Determine the growth rate of $g(k)$.
\end{enumerate}

\medskip
\noindent\textbf{QUICK LITERATURE/CONTEXT CHECK}

\smallskip
\noindent
The problem text attributes the function $\tau_\perp$ to Erd\H{o}s--Hall and records: a trivial bound $\tau_\perp(n)\ge \omega(n)$ (with equality for infinitely many $n$), an extremal lower bound for $\max_{n<x}\tau_\perp(n)$ of the form $>\exp((\log\log x)^{2-\epsilon})$, and exponential bounds on $g(k)$ of the shape $(2^{1/2}+o(1))^k<g(k)<(2-c)^k$. Per the integrity rule for this task, I do not assert any additional literature beyond what is written in the problem file.

\medskip
\noindent\textbf{ATTACK PLAN}

\smallskip
\noindent
\begin{itemize}
\item Prove the ``trivial'' lower bound $\tau_\perp(n)\ge \omega(n)$ carefully from the divisor-order definition.
\item Prove universal upper bounds (e.g. $\tau_\perp(n)\le \tau(n)-1$) and specialize to squarefree $n$ to get $g(k)\le 2^k-1$.
\item Compute $\tau_\perp(n)$ for small $n$ and approximate $g(k)$ for small $k$ by searching over squarefree $n$ supported on small primes.
\end{itemize}

\medskip
\noindent\textbf{WORK}

\smallskip
\noindent\textbf{Lemma 1100.1 ($\tau_\perp(n)\ge \omega(n)$, proved).}
For every integer $n\ge 2$,
\[
\tau_\perp(n)\ge \omega(n).
\]

\smallskip
\noindent\emph{Proof.}
Let $p$ be a prime divisor of $n$. Then $p$ appears in the divisor list $d_1<\cdots<d_{\tau(n)}$ at some position, say $d_{j(p)}=p$.
Because $p$ is prime, every positive divisor of $n$ that is strictly less than $p$ is not divisible by $p$. In particular, the predecessor divisor $d_{j(p)-1}$ (which exists because $p>1=d_1$) satisfies $p\nmid d_{j(p)-1}$. Hence
\[
\gcd(d_{j(p)-1},d_{j(p)})=\gcd(d_{j(p)-1},p)=1.
\]
So the index $i(p):=j(p)-1$ contributes $1$ to $\tau_\perp(n)$.

If $p\ne q$ are distinct prime divisors of $n$, then $p$ and $q$ occur at different positions in the divisor list, hence $i(p)\ne i(q)$. Therefore the set $\{i(p): p\mid n,\ p\ \text{prime}\}$ has size $\omega(n)$ and consists entirely of indices counted by $\tau_\perp(n)$. This proves $\tau_\perp(n)\ge \omega(n)$. \qed

\smallskip
\noindent\textbf{Lemma 1100.2 (universal upper bound and squarefree consequence).}
For every $n\ge 1$,
\[
\tau_\perp(n)\le \tau(n)-1.
\]
In particular, if $n$ is squarefree with $\omega(n)=k$, then $\tau(n)=2^k$ and so
\[
\tau_\perp(n)\le 2^k-1,\qquad\text{hence }\ g(k)\le 2^k-1.
\]

\smallskip
\noindent\emph{Proof.}
There are exactly $\tau(n)-1$ adjacent pairs $(d_i,d_{i+1})$ in the divisor list. Since $\tau_\perp(n)$ counts a subset of these pairs, $\tau_\perp(n)\le \tau(n)-1$.
If $n$ is squarefree with $k$ distinct prime divisors, each divisor corresponds to a choice of subset of the $k$ primes, so $\tau(n)=2^k$. Substituting gives the stated bound. \qed

\smallskip
\noindent\textbf{FAST REALITY CHECK (exact small-$n$ computations).}

\smallskip
\noindent
I computed $\tau_\perp(n)$ exactly for $n\le 2000$ by generating the sorted divisor list and counting coprime adjacencies.
\begin{itemize}
\item The maximum observed ratio $\tau_\perp(n)/\omega(n)$ for $2\le n\le 2000$ was $2.75$, achieved at $n=1260$ with $(\tau_\perp,\omega)=(11,4)$.
\item For primorials $n=2\cdot3\cdot5\cdots p_k$ (product of first $k$ primes), the computed values were:
\[
\begin{array}{r|r|r|r}
 k & n & \omega(n) & \tau_\perp(n) \\\hline
 3&30&3&4\\
 4&210&4&7\\
 5&2310&5&12\\
 6&30030&6&17
\end{array}
\]
so $\tau_\perp(n)/\omega(n)$ increases in this sample.
\end{itemize}

\smallskip
\noindent\textbf{FAST REALITY CHECK (approximate $g(k)$ for small $k$).}

\smallskip
\noindent
Restricting to squarefree $n$ whose $k$ primes are chosen among the $15$ primes $<50$, I exhaustively searched all such $n$ for $1\le k\le 8$ and computed $\tau_\perp(n)$. The maximum values found in this restricted search were:
\[
\begin{array}{r|r|l}
 k & \max \tau_\perp(n) & \text{one maximizing prime set (subset of primes $<50$)} \\\hline
 1&1&(2)\\
 2&2&(2,3)\\
 3&4&(2,3,5)\\
 4&7&(2,3,5,7)\\
 5&13&(2,5,7,11,19)\\
 6&22&(3,5,13,19,29,41)\\
 7&35&(2,3,5,11,23,31,41)\\
 8&58&(5,7,11,13,17,23,41,47)
\end{array}
\]
This is only a sanity check: since the search restricts available primes, these values are lower bounds for the true $g(k)$.

\medskip
\noindent\textbf{VERIFICATION}

\smallskip
\noindent
\begin{itemize}
\item Lemma 1100.1 is checked carefully: it uses the fact that each prime divisor $p$ appears as some divisor $d_j=p$ and that $\gcd(d_{j-1},p)=1$ because $d_{j-1}<p$.
\item Lemma 1100.2 is immediate from the definition of $\tau_\perp$ as a count of adjacent pairs.
\item Computations: for each $n$ I used exact divisor lists (not approximations) and standard gcd computations; for the restricted $g(k)$ search, I enumerated all $k$-subsets of the primes $<50$ and computed the divisor list of their product.
\end{itemize}

\medskip
\noindent\textbf{UNRESOLVED}

\smallskip
\noindent
(i) \emph{Strongest proved partial result here.}
We proved the universal sandwich bounds
\[
\omega(n)\le \tau_\perp(n)\le \tau(n)-1
\]
(Lemmas 1100.1 and 1100.2), and hence for squarefree $n$ with $\omega(n)=k$ we have
\[
 k\le \tau_\perp(n)\le 2^k-1\quad\text{and }\quad g(k)\le 2^k-1.
\]
We also provided exact small-$n$ computations (up to $n\le 2000$) and restricted searches giving lower-bound data for $g(k)$ up to $k=8$.

\smallskip
\noindent
(ii) \emph{First gap (crisp).}
Either prove or disprove:
\begin{enumerate}
\item $\tau_\perp(n)/\omega(n)\to\infty$ for almost all $n$;
\item $\tau_\perp(n)<\exp((\log n)^{o(1)})$ for all $n$;
\item determine the correct exponential growth rate of $g(k)$ for squarefree $n$ with $\omega(n)=k$.
\end{enumerate}

\smallskip
\noindent
(iii) \emph{Top 3 next moves.}
\begin{enumerate}
\item Model the sorted divisor list of a ``typical'' integer $n$ (via random multiplicative structure) and estimate the frequency with which adjacent divisors share a common prime factor, aiming to show $\tau_\perp(n)$ is typically much larger than $\omega(n)$.
\item For $g(k)$, search for explicit prime configurations that force many alternations between coprime consecutive divisors, and prove lower bounds on $\tau_\perp(n)$ in terms of combinatorial properties of subset-product orderings.
\item Extend computation: exact $g(k)$ for slightly larger $k$ (by better search/pruning) and $\tau_\perp(n)$ statistics for much larger $n$ to test whether $\tau_\perp/\omega$ grows slowly or quickly in practice.
\end{enumerate}

\smallskip
\noindent
(iv) \emph{Minimal counterexample structure.}
For the ``almost all $n$'' divergence question, a minimal counterexample would be an infinite-density set of integers $n$ for which $\tau_\perp(n)=O(\omega(n))$, meaning that in the increasing divisor list, almost every adjacent pair shares some prime factor. Such behavior would likely correlate with $n$ having a highly unbalanced prime-power structure (many divisors clustered in chains sharing primes) rather than being squarefree with many moderately sized primes.

