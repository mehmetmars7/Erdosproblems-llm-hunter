\section*{Problem 624}

\subsection*{FORMAL RESTATEMENT}
Let $X$ be a finite set with $|X|=n$. Define $H(n)$ to be the least integer $h$ for which there exists a function
\[
 f:\mathcal P(X)\to X
\]
such that for every subset $Y\subseteq X$ with $|Y|\ge h$ we have
\[
\{f(A):A\subseteq Y\}=X.
\]
Equivalently, for every $Y\subseteq X$ with $|Y|=h$, the restriction $f\!\restriction\mathcal P(Y)$ is surjective onto $X$.

The problem asks to prove that
\[
H(n)-\log_2 n\to\infty\qquad (n\to\infty).
\]

\subsection*{QUICK LITERATURE/CONTEXT CHECK}
This is Erd\H{o}s Problem \#624 and is listed as open. Erd\H{o}s--Hajnal proved
\[
\log_2 n\le H(n) < \log_2 n + (3+o(1))\log_2\log_2 n.
\]
A folklore ``first step'' is to show $H(2^k)\ge k+1$. A short argument attributed to Alon establishes this.
More strongly, the Erd\H{o}s Problems page records (attribution to Alon, personal communication) that for $n=2^k$ and any $f$, there exists $Y$ with $|Y|=k$ such that $\#\{f(A):A\subseteq Y\}<(1-c)2^k$ for some absolute $c>0$ and $k$ large.

\subsection*{ATTACK PLAN}
\begin{enumerate}[label=\arabic*.]
\item Re-derive the basic lower bound $H(n)\ge\log_2 n$ (counting).
\item Present Alon's argument showing $H(2^k)\ge k+1$.
\item Explore whether the collision idea can be iterated (using many $t$-subsets with identical image) to force $H(2^k)\ge k+\omega(1)$, which would imply the desired limit. Identify precisely where the iteration fails.
\end{enumerate}

\subsection*{WORK}
\paragraph{(A) The easy information-theoretic lower bound $H(n)\ge\log_2 n$.}
Fix $h$ and suppose there exists $f$ satisfying the defining property for $H(n)$ with this $h$.
Take any $Y\subseteq X$ with $|Y|=h$. Then $\mathcal P(Y)$ has $2^h$ subsets, so the image
\(\{f(A):A\subseteq Y\}\) has size at most $2^h$. But by requirement it equals $X$, so has size $n$.
Hence $n\le 2^h$, i.e.
\[
 h\ge \log_2 n.
\]
Therefore $H(n)\ge\lceil\log_2 n\rceil$.

\paragraph{(B) Alon's argument for $H(2^k)\ge k+1$.}
Let $n=2^k$ and let $X$ be a set of size $n$. Consider an arbitrary function $f:\mathcal P(X)\to X$.
By the pigeonhole principle, among the ${n\choose 2}$ two-element subsets of $X$, some value $x\in X$ is attained by at least
\[
\frac{{n\choose 2}}{n} = \frac{n-1}{2}
\]
of them. Let $A_1,\dots,A_m$ (with $m\ge\frac{n-1}{2}$) be distinct $2$-subsets with $f(A_i)=x$.

Now take any $k$-element subset $Y\subseteq X$ that contains at least $\lfloor k/2\rfloor$ of the sets $A_i$ (i.e. contains both elements of each of these $A_i$).
For each such $A_i\subseteq Y$, the subset $A_i$ contributes no new image value beyond $x$. Thus the number of distinct images on $\mathcal P(Y)$ is at most
\[
2^k - \lfloor k/2\rfloor + 1,
\]
because compared to the maximal $2^k$ distinct images (one per subset), the $\lfloor k/2\rfloor$ colliding $A_i$'s force at least $\lfloor k/2\rfloor-1$ identifications.
Since $2^k=n$ and $\lfloor k/2\rfloor\ge 1$ for $k\ge 2$, we have
\[
2^k - \lfloor k/2\rfloor + 1 < 2^k = n.
\]
Hence for this $Y$ we get $\#\{f(A):A\subseteq Y\}<n$, so $f$ cannot satisfy the defining property with $h=k$.
Therefore $H(2^k)\ge k+1$.

\paragraph{(C) Why this does not yet imply $H(n)-\log_2 n\to\infty$.}
The above shows only an additive $+1$ gap on the subsequence $n=2^k$.
To prove $H(n)-\log_2 n\to\infty$ one would need to show that for each fixed constant $C$, and all sufficiently large $n$, \emph{no} function $f$ can be surjective on $\mathcal P(Y)$ for every $Y$ of size $\lceil\log_2 n\rceil + C$.
Equivalently, for such $h$, every $f$ must admit some $Y$ of size $h$ for which the range size $\#\{f(A):A\subseteq Y\}$ is $<n$.

A natural strengthening attempt is to produce, for some fixed output value, not merely $\Omega(n)$ colliding $2$-sets, but enough collisions \emph{inside a single $h$-set} to reduce the range below $n$ even when $2^h$ is a constant multiple of $n$. However, when $h=\log_2 n + C$, we have $2^h = 2^C n$, so surjectivity is compatible with an average multiplicity $2^C$ of preimages per element, leaving room for many collisions without forcing a missing value. Forcing a missing value seems to require structure beyond a single pigeonhole collision at fixed set size.

\subsection*{VERIFICATION}
The lower bound (A) is a direct counting argument.
The proof in (B) uses only pigeonhole and the range-size bound $\#\mathrm{im}(f|_{\mathcal P(Y)})\le 2^{|Y|}$ together with a concrete collision count.
The discussion in (C) correctly identifies that, for $h=\log_2 n + C$, collisions alone do not immediately prevent surjectivity because there is slack by the factor $2^C$.

\subsection*{FINAL}
\textbf{UNRESOLVED.}
\begin{itemize}[leftmargin=*]
\item (i) What was proved: The standard lower bound $H(n)\ge\log_2 n$ and (following Alon) the nontrivial step $H(2^k)\ge k+1$.
\item (ii) Strongest partial result stated here: On the Erd\H{o}s Problems page, Alon (personal communication) is reported to have a constant-factor deficit: for $n=2^k$ and any $f$, there exists $|Y|=k$ with $\#\{f(A):A\subseteq Y\}<(1-c)2^k$ for some absolute $c>0$ and large $k$.
\item (iii) First gap: No argument here forces an \emph{unbounded additive} gap $H(n)\ge \log_2 n + \omega(1)$; the collision method only yields an additive constant, and the presence of $2^C$ slack for $h=\log_2 n + C$ blocks a direct extension.
\item (iv) Next concrete step: Develop a mechanism that, for any $f$, produces a set $Y$ of size $\log_2 n + o(\log_2 n)$ for which \emph{many} values of $X$ are simultaneously missing from $\{f(A):A\subseteq Y\}$ (not just one), or else show that the distribution of $f$ on small subsets forces a structured obstruction (e.g. via entropy/compression, VC-dimension, or covering-design lower bounds) that grows with $n$.
\end{itemize}

\subsection*{COMPLETION ESTIMATE}
COMPLETION: 25\%.

% =====================
% Problem 625
% =====================
