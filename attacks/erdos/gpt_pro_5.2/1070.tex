
\subsection*{FORMAL RESTATEMENT}
Given $n$ points in the plane, consider the graph on these $n$ vertices in which two points are adjacent iff their Euclidean distance is exactly $1$.
Let $f(n)$ be the largest integer such that every set of $n$ points contains an independent set of size at least $f(n)$ (equivalently, the minimum possible independence number among all unit-distance graphs on $n$ vertices).
The problem asks for estimates on $f(n)$, and in particular whether $f(n)\ge n/4$.

\subsection*{QUICK LITERATURE/CONTEXT CHECK}
The problem file relates this to the chromatic number $\chi$ of the unit distance graph of $\mathbb{R}^2$ (Hadwiger--Nelson).  It states the general bounds $n/\chi\le f(n)\le 2n/7$, with the upper bound via the Moser spindle.  It also states a lower bound $f(n)\ge n^2/(2m_1+n)$ in terms of the maximal number $m_1$ of unit distances among $n$ points.

\subsection*{ATTACK PLAN}
Prove the general bounds from scratch:
(1) $f(n)\ge n/\chi$ by pigeonholing a coloring of the plane;
(2) $f(n)\ge n^2/(2m+n)$ for any graph with $n$ vertices and $m$ edges, then specialize to $m\le m_1(n)$;
(3) Build an explicit 7-point unit-distance configuration (Moser spindle) with independence number $2$ and replicate it far apart to obtain $f(n)\le 2n/7+O(1)$.
Include a computational ``FAST REALITY CHECK'' verifying the Moser-spindle adjacency and independence number.

\subsection*{WORK}
\textbf{Lemma 1070.1 (chromatic number lower bound).}
Let $\chi$ be the chromatic number of the (infinite) unit distance graph on $\mathbb{R}^2$.
Then for every $n$, $f(n)\ge \lceil n/\chi\rceil$.

\emph{Proof.}
By definition of $\chi$, there exists a coloring of all points of $\mathbb{R}^2$ with $\chi$ colors such that no two points at distance $1$ share a color.
Given any set $P$ of $n$ points, one color class in $P$ has size at least $\lceil n/\chi\rceil$ (pigeonhole principle). Any two points of the same color are not at distance $1$, so that color class is an independent set in the unit-distance graph on $P$.
Thus every $P$ has an independent set of size at least $\lceil n/\chi\rceil$, i.e. $f(n)\ge \lceil n/\chi\rceil$.
\qed

\textbf{Lemma 1070.2 (edge-count lower bound on independence).}
Let $G$ be a finite graph with $n$ vertices and $m$ edges. Then
\[
\alpha(G)\ \ge\ \frac{n^2}{2m+n}.
\]
Consequently, if $m_1(n)$ denotes the maximum possible number of unit-distance edges among $n$ planar points, then
\[
 f(n)\ \ge\ \frac{n^2}{2m_1(n)+n}.
\]

\emph{Proof.}
Let the vertex degrees be $d(v)$, $v\in V(G)$, so that $\sum_{v} d(v)=2m$.
Choose a uniformly random permutation $\pi$ of $V(G)$ and define
\[
I(\pi):=\{v\in V(G): v\text{ appears before all its neighbors in }\pi\}.
\]
Then $I(\pi)$ is an independent set: if $uv$ is an edge, it is impossible that both $u$ and $v$ appear before each other.
Fix a vertex $v$ of degree $d(v)$.  Among the set consisting of $v$ and its $d(v)$ neighbors (a set of size $d(v)+1$), each vertex is equally likely to be the earliest in the random permutation $\pi$, hence
\[
\mathbb{P}(v\in I(\pi))=\frac{1}{d(v)+1}.
\]
By linearity of expectation,
\[
\mathbb{E}|I(\pi)|=\sum_{v\in V(G)} \frac{1}{d(v)+1}.
\]
Therefore there exists at least one permutation $\pi$ with
\[
\alpha(G)\ge |I(\pi)|\ge \sum_{v\in V(G)} \frac{1}{d(v)+1}.
\]
Apply Cauchy--Schwarz to the positive numbers $d(v)+1$:
\[
\sum_{v} \frac{1}{d(v)+1}
\ \ge\ \frac{n^2}{\sum_v (d(v)+1)}
\ =\ \frac{n^2}{2m+n}.
\]
This proves $\alpha(G)\ge n^2/(2m+n)$.
For unit-distance graphs on $n$ planar points, the number of edges is at most $m_1(n)$ by definition, hence $f(n)=\min \alpha\ge n^2/(2m_1(n)+n)$.
\qed

\textbf{Lemma 1070.3 (a 7-vertex unit-distance graph with $\alpha=2$).}
There exists a set of $7$ points in the plane whose unit-distance graph has independence number $2$.
Hence, for every $t\ge 1$,
\[
 f(7t)\le 2t,
\]
and in particular $f(n)\le \frac{2}{7}n + O(1)$.

\emph{Proof (explicit construction and combinatorial bound).}
We give coordinates for a Moser-spindle-type configuration.
Let $\theta>0$ satisfy $2\sqrt{3}\,\sin(\theta/2)=1$.
Define points
\[
A=(0,0),\quad B=(1,0),\quad D=\Bigl(\tfrac12,\tfrac{\sqrt{3}}{2}\Bigr),\quad C=B+D=\Bigl(\tfrac32,\tfrac{\sqrt{3}}{2}\Bigr).
\]
Let $R_\theta$ denote rotation by $\theta$ about the origin and set
\[
E=R_\theta(B),\quad F=R_\theta(D),\quad G=R_\theta(C).
\]
Now check unit distances:
\begin{itemize}
\item In $\{A,B,C,D\}$ we have $|AB|=|AD|=|BD|=|BC|=|CD|=1$ (two equilateral triangles sharing the edge $BD$).
\item In $\{A,E,F,G\}$ we have the analogous five unit distances $|AE|=|AF|=|EF|=|EG|=|FG|=1$ by rotation invariance.
\item Finally, $|CG|=|C-R_\theta(C)|=1$ by the chord-length equation $2|C|\sin(\theta/2)=1$ and $|C|=\sqrt{3}$.
\end{itemize}
Thus the unit-distance graph on the seven vertices $\{A,B,C,D,E,F,G\}$ contains the two triangles $B\!C\!D$ and $E\!F\!G$, and $A$ is adjacent to $B,D,E,F$.
From this structure we show $\alpha\le 2$:
\begin{itemize}
\item If an independent set contains $A$, it cannot contain $B,D,E,F$, and among $\{C,G\}$ it can contain at most one because $C$--$G$ is an edge. So it has size $\le 2$.
\item If an independent set does not contain $A$, then it meets the triangle $\{B,C,D\}$ in at most one vertex and meets the triangle $\{E,F,G\}$ in at most one vertex, so it has size $\le 2$.
\end{itemize}
On the other hand, $\{A,C\}$ is independent (since $A$ is not adjacent to $C$), so $\alpha=2$.

To deduce $f(7t)\le 2t$, place $t$ translated copies of this 7-point configuration so far apart (distance $>2$ between distinct copies) that no unit distances occur between points from different copies.  Any independent set in the union contains at most $2$ vertices from each copy, hence at most $2t$ vertices in total. \qed

\textbf{FAST REALITY CHECK (computation).}
Using the above coordinates and $\theta=2\arcsin(1/(2\sqrt{3}))\approx 0.5856855$ radians, a brute-force distance check (within $10^{-9}$ tolerance) finds exactly $11$ unit-distance pairs, namely
\[
\{AB,AD,AE,AF,BD,BC,CD,CG,EF,EG,FG\},
\]
and exhaustive search over all subsets confirms the maximum independent set size is $2$.

\subsection*{VERIFICATION}
Lemma~1070.1 is a direct pigeonhole argument from the definition of $\chi$.
Lemma~1070.2 is proved via the Caro--Wei expectation bound and Cauchy--Schwarz; all steps are explicit.
Lemma~1070.3 supplies coordinates and checks unit distances by construction; the combinatorial argument for $\alpha=2$ uses only the existence of two triangles and the adjacency of $A$.
The replication argument is verified by choosing inter-copy distances $>2$, ensuring cross-copy distances are $>1$.
The ``FAST REALITY CHECK'' was done by direct computation on the stated coordinates.

\subsection*{FINAL}
\textbf{UNRESOLVED.}

(i) \emph{Strongest proved partial result:}
We have general lower bounds $f(n)\ge \lceil n/\chi\rceil$ (Lemma~1070.1) and $f(n)\ge n^2/(2m_1(n)+n)$ (Lemma~1070.2), and an explicit construction giving $f(n)\le 2n/7+O(1)$ (Lemma~1070.3).  In particular, $\limsup f(n)/n\le 2/7$ via the replicated 7-point configuration.

(ii) \emph{First gap (crisp statement):}
Determine the correct linear growth constant for $f(n)$ (if it exists), and specifically decide whether $f(n)\ge n/4$ holds for all $n$.

(iii) \emph{Top 3 next moves (concrete lemmas/computations):}
\begin{enumerate}
\item Improve upper bounds by finding smaller (or denser) finite unit-distance graphs with smaller independence ratio than $2/7$, and then replicate them.
\item Improve lower bounds by sharpening estimates on $m_1(n)$ (maximum number of unit distances), then plugging into Lemma~1070.2.
\item For moderate $n$ (say $n\le 50$), attempt computational search (with exact algebraic coordinates) for configurations minimizing the independence number, to guess the best constant and candidate extremizers.
\end{enumerate}

(iv) \emph{Minimal counterexample structure:}
A counterexample to $f(n)\ge n/4$ would be a concrete $n$-point set whose unit-distance graph has an independent set of size at most $\lfloor n/4\rfloor$.  Any asymptotic counterexample to a conjectured constant $c$ would require a family of finite unit-distance graphs with independence ratio $<c$.


