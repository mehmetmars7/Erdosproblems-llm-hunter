\section*{Erd\H{o}s problem 187}

\subsection*{1) FORMAL RESTATEMENT}
Let $c:\mathbb{Z}\to\{\text{red},\text{blue}\}$ be a 2-colouring.
For each $d\in\mathbb{N}$, define
\[
L_c(d):=\sup\Bigl\{\ell\ge 2:\exists a\in\mathbb{Z}\ \text{such that } a,a+d,\dots,a+(\ell-1)d\ \text{are all the same colour}\Bigr\}.
\]
We seek the largest function $f:\mathbb{N}\to\mathbb{N}$ such that for every colouring $c$,
\[
L_c(d)\ge f(d)\quad\text{for infinitely many }d.
\]

\subsection*{2) QUICK LITERATURE/CONTEXT CHECK}
Only what appears in the problem text: van der Waerden implies $f(d)\to\infty$ is necessary, Beck gives an upper bound
$f(d)\le (1+o(1))\log_2 d$ via an explicit colouring, and Erd\H{o}s gave a weaker linear-type obstruction using an irrational rotation colouring.

\subsection*{3) ATTACK PLAN}
Give two rigorous components:
(1) a universal trivial lower bound ($f(d)=2$ works),
(2) a fully verified irrational-rotation upper bound of the form $L_c(d)\ll 1/\|d\alpha\|$ and specialize to $\alpha=\sqrt2$ giving $L_c(d)\le 2d+1$.

\subsection*{4) WORK}

\paragraph{Lemma 4.1 (Infinitely many monochromatic pairs).}
For every 2-colouring $c$, the set $\{d\ge 1:\ L_c(d)\ge 2\}$ is infinite. In particular $f(d)=2$ is always admissible.
\textit{Proof.}
At least one colour class is infinite; call it $A\subset\mathbb{Z}$. The difference set $A-A=\{a-a':a,a'\in A\}$ is infinite
(because for any fixed $D$ there are only finitely many pairs with $|a-a'|\le D$, while $A$ has infinitely many pairs).
Every $d\in (A-A)\cap\mathbb{N}$ yields a monochromatic 2-term progression $\{a,a+d\}\subset A$, so $L_c(d)\ge 2$ for infinitely many $d$. \hfill$\square$

\paragraph{Lemma 4.2 (Rotation colouring bound).}
Fix irrational $\alpha$ and colour $n\in\mathbb{Z}$ red iff $\{\alpha n\}\in [0,1/2)$, blue otherwise.
Let $\|x\|:=\min_{m\in\mathbb{Z}}|x-m|$. Then for every $d\ge 1$,
\[
L_c(d)\ \le\ \left\lfloor \frac{1}{2\|\alpha d\|}\right\rfloor+1.
\]
\textit{Proof.}
Let $\delta=\|\alpha d\|\in(0,1/2]$ (irrationality gives $\delta>0$).
Suppose $a,a+d,\dots,a+(\ell-1)d$ are all red. Consider the points
\[
x_t:=\{\alpha(a+td)\}=\{\alpha a+t\alpha d\}\in [0,1/2)\quad (t=0,\dots,\ell-1).
\]
On the circle $\mathbb{R}/\mathbb{Z}$, adding $\alpha d$ each time moves by either $+\delta$ or $-\delta$ (choose direction giving the shorter step).
Thus the set $\{x_t\}$ contains $\ell$ points spaced at least $\delta$ apart along that direction. If all $\ell$ points lie in an interval of length $1/2$,
we must have $(\ell-1)\delta\le 1/2$, hence $\ell\le 1/(2\delta)+1$. The same argument holds for blue.
Therefore $L_c(d)\le \lfloor 1/(2\|\alpha d\|)\rfloor+1$. \hfill$\square$

\paragraph{Lemma 4.3 (Diophantine bound for $\sqrt2$).}
For every integer $q\ge 1$,
\[
\|q\sqrt2\|\ \ge\ \frac{1}{(2\sqrt2+1)q}\ \ge\ \frac{1}{4q}.
\]
\textit{Proof.}
Let $p$ be the nearest integer to $q\sqrt2$, so $\|q\sqrt2\|=|q\sqrt2-p|\le 1/2$.
Then $p/q$ lies within $1/(2q)$ of $\sqrt2$, in particular $p/q\le \sqrt2+1$.
Compute
\[
\left|\sqrt2-\frac{p}{q}\right|=\frac{|2q^2-p^2|}{q^2\left(\sqrt2+\frac{p}{q}\right)}.
\]
Since $2q^2-p^2$ is a nonzero integer (no integer solutions to $p^2=2q^2$ for $q\ge 1$), $|2q^2-p^2|\ge 1$.
Also $\sqrt2+p/q\le \sqrt2+(\sqrt2+1)=2\sqrt2+1$. Hence
\[
\left|\sqrt2-\frac{p}{q}\right|\ge \frac{1}{(2\sqrt2+1)q^2},
\]
and multiplying by $q$ gives $\|q\sqrt2\|\ge 1/((2\sqrt2+1)q)\ge 1/(4q)$. \hfill$\square$

\subsection*{FAST REALITY CHECK (computed)}
For the colouring $c(n)=\mathbf{1}_{\{\sqrt2 n\}\ge 1/2}$, brute checks on $[1,2000]$ gave:
$d=1: L(d)=2$ (bound $3$), $d=2: L(d)=3$ (bound $5$), $d=5: L(d)=8$ (bound $11$), consistent with Lemmas 4.2--4.3.

\subsection*{6) FINAL}
\textbf{UNRESOLVED}

(i) Strongest proved partial result: $f(d)=2$ always works (Lemma 4.1) and there exists an explicit colouring with
$L(d)\le 2d+1$ for all $d$ (Lemmas 4.2--4.3).

(ii) First gap: prove any nontrivial lower bound $f(d)\to\infty$ \emph{quantitatively} (e.g. $f(d)\ge c\log d$ infinitely often) without citing deep theorems.

(iii) Top 3 next moves:
1. Translate the problem into density/recurrence properties of the difference sets of colour classes.
2. Attempt entropy or energy estimates for the distribution of monochromatic runs along residue classes mod $d$.
3. Try to re-derive Beck’s logarithmic upper bound via a self-contained random construction.

(iv) Minimal counterexample structure: a colouring for which $L_c(d)$ stays bounded or grows slower than any target function along all but finitely many $d$; such a colouring
must impose strong pseudorandomness simultaneously across many arithmetic progressions.

