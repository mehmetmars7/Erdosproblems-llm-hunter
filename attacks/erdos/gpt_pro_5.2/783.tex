\section*{Problem \#783 (Coprime set with bounded harmonic sum covering $[1,n]$)}
\addcontentsline{toc}{section}{Problem \#783 (Coprime set with bounded harmonic sum covering $[1,n]$)}

\subsection*{1) FORMAL RESTATEMENT}
\addcontentsline{toc}{subsection}{1) Formal restatement (\#783)}

Fix a constant $C>0$ and let $n$ be large.
We choose a set $A\subseteq\{2,3,\dots,n\}$ subject to
\begin{enumerate}[label=(\roman*),leftmargin=2em]
\item $\gcd(a,b)=1$ for all distinct $a,b\in A$ (pairwise coprime);
\item $\displaystyle \sum_{a\in A} \frac1a \le C$.
\end{enumerate}
Define the ``uncovered'' set
\[
U(A) := \big\{ m\in\{1,2,\dots,n\}: \forall a\in A,\ a\nmid m\big\}.
\]
Equivalently, $U(A)$ are those integers $\le n$ not divisible by any element of $A$.

\medskip
\noindent\textbf{Goal.}
Choose $A$ (depending on $n$ and $C$) to minimize $|U(A)|$.

\medskip
\noindent\textbf{Candidate suggested in the problem.}
Let $q_1>q_2>\cdots$ be the primes in decreasing order starting at the largest prime $\le n$.
Let $k$ be maximal such that $\sum_{i=1}^k 1/q_i \le C$, and take $A=\{q_1,\dots,q_k\}$.
The question asks whether this choice minimizes $|U(A)|$.

\subsection*{2) QUICK LITERATURE/CONTEXT CHECK}
\addcontentsline{toc}{subsection}{2) Quick literature/context check (\#783)}

This problem appears as Erd\H{o}s problem \#783 on \texttt{erdosproblems.com}.  I did not find a definitive published solution.

\subsection*{3) ATTACK PLAN}
\addcontentsline{toc}{subsection}{3) Attack plan (\#783)}

View this as a weighted covering problem.
Each $a\in A$ ``covers'' the multiples $\{a,2a,\dots,\lfloor n/a\rfloor a\}$.  We want to cover as much of $[1,n]$ as possible under a harmonic-cost budget and a coprimality constraint.

\medskip
\noindent\textbf{Two guiding principles.}
\begin{itemize}[leftmargin=2em]
\item \emph{Upper bound by the budget:} ignoring overlaps, $a$ covers about $n/a$ integers and costs $1/a$, so the naive ``efficiency'' is about $n$ covered per unit cost.  Overlaps between different $a$ are the main source of inefficiency.
\item \emph{Avoid overlaps by making all $a$ large:} if $a,b>\sqrt{n}$ and $\gcd(a,b)=1$, then $ab>n$ and hence no integer $\le n$ is divisible by both $a$ and $b$.  Thus for such large $A$ the covered sets are disjoint.
\end{itemize}

\subsection*{4) WORK}
\addcontentsline{toc}{subsection}{4) Work (\#783)}

\subsubsection*{4.1 A universal lower bound}
Let
\[
B(A):=\{m\le n:\exists a\in A\text{ with }a\mid m\}
\]
be the covered set, so $|U(A)| = n-|B(A)|$.
By a union bound,
\[
|B(A)|\le \sum_{a\in A} \left\lfloor\frac{n}{a}\right\rfloor
\le n\sum_{a\in A} \frac1a \le Cn.
\]
Therefore
\begin{equation}
\label{eq:lb}
|U(A)| \ge n - Cn = (1-C)n.
\end{equation}
This bound is nontrivial when $C\le 1$.

\subsubsection*{4.2 A regime where the candidate is (asymptotically) optimal}
A key observation is that when all chosen moduli are $>\sqrt{n}$, there are \emph{no overlaps}.

\begin{lemma}[Disjointness above $\sqrt{n}$]
\label{lem:disjoint}
If $a,b>\sqrt{n}$ and $\gcd(a,b)=1$, then no $m\le n$ is divisible by both $a$ and $b$.
Hence the sets of multiples of elements of $A$ are pairwise disjoint whenever every $a\in A$ satisfies $a>\sqrt{n}$.
\end{lemma}
\begin{proof}
If $a\mid m$ and $b\mid m$ with $\gcd(a,b)=1$, then $ab\mid m$.  But $ab>n$, so $m\ge ab>n$, contradiction.
\end{proof}

Now consider the candidate choice $A$ consisting of all primes $p$ in an interval $[y,n]$, where $y$ is chosen so that $\sum_{p\in[y,n]}1/p\approx C$.

\begin{theorem}[Asymptotic optimality for $C<\log 2$]
\label{thm:C<log2}
Fix $0<C<\log 2$.  For all sufficiently large $n$ there exists a set $A\subseteq\{2,\dots,n\}$ of pairwise coprime integers with $\sum_{a\in A} 1/a\le C$ such that
\[
|U(A)| = (1-C)n + o(n).
\]
In particular, the lower bound \eqref{eq:lb} is asymptotically tight.
Moreover, choosing $A$ to be (essentially) all primes $p\in[n^{e^{-C}},n]$ achieves this.
\end{theorem}

\begin{proof}[Proof sketch]
Let $y:=n^{e^{-C}}$.  Since $C<\log 2$, we have $e^{-C}>1/2$, so $y>\sqrt{n}$.

By Mertens' theorem for primes,
\[
\sum_{p\le x} \frac1p = \log\log x + B_1 + o(1),
\]
so
\[
\sum_{y\le p\le n} \frac1p = \big(\log\log n - \log\log y\big) + o(1)
= \big(\log\log n - \log(e^{-C}\log n)\big) + o(1) = C + o(1).
\]
Thus for $n$ large we can select a subset of primes in $[y,n]$ whose reciprocal sum is $\le C$ but arbitrarily close to $C$.
Call this set $A$.

Because every $p\in A$ satisfies $p\ge y>\sqrt{n}$, Lemma~\ref{lem:disjoint} gives that the sets of multiples are disjoint. Hence
\[
|B(A)| = \sum_{p\in A} \left\lfloor\frac{n}{p}\right\rfloor
= n\sum_{p\in A} \frac1p + O(|A|).
\]
Here $|A|\le \pi(n)=o(n)$, so $O(|A|)=o(n)$.  Since $\sum_{p\in A}1/p = C+o(1)$, we get $|B(A)|=Cn+o(n)$.
Therefore $|U(A)| = n-|B(A)| = (1-C)n+o(n)$, as claimed.
\end{proof}

\paragraph{Interpretation.}
For $C<\log 2$, one can spend the harmonic budget entirely on primes $>\sqrt{n}$, achieving essentially no overlaps and hence the best-possible coverage up to lower-order terms.

\subsubsection*{4.3 What happens for larger $C$ (heuristics)}
When $C\ge \log 2$, the threshold $y=n^{e^{-C}}$ drops below $\sqrt{n}$, so overlaps among multiples of chosen primes become unavoidable.
A natural ``all large primes'' choice $A=\{p: p\ge y\}$ covers precisely the integers $m\le n$ having some prime factor $\ge y$.
Thus the uncovered set becomes the set of $y$--smooth numbers, whose size is traditionally denoted $\psi(n,y)$.
Heuristically (Dickman--de Bruijn), if $y=n^{1/u}$ then
\[
\psi(n,y) \sim n\,\rho(u),
\]
where $\rho$ is the Dickman function.
Here $y=n^{e^{-C}}$ corresponds to $u=e^{C}$, predicting
\[
|U(A)| \approx n\,\rho(e^{C}).
\]
Proving optimality of this choice (or finding a better one) appears difficult.

\subsubsection*{4.4 Small-$n$ exact optimization as a sanity check}
For small $n$ one can brute-force the optimization problem and compare to the ``take the largest primes'' candidate.
For example, for $n=30$ and $C=1.0$:
\begin{itemize}[leftmargin=2em]
\item Candidate (largest primes with reciprocal sum $\le 1$): $A=\{29,23,19,17,13,11,7,5\}$, which leaves $|U(A)|=12$ uncovered.
\item An optimal set (by exact search) achieves $|U(A)|=6$ (one optimal choice is $A=\{2,7,9,17,19,23,25,29\}$ with $\sum 1/a\approx 0.983$).
\end{itemize}
This illustrates that the ``largest primes'' rule is not literally optimal for small $n$, but it does not decide the asymptotic question for $n\to\infty$.

\subsection*{5) VERIFICATION}
\addcontentsline{toc}{subsection}{5) Verification (\#783)}

\begin{itemize}[leftmargin=2em]
\item Inequality \eqref{eq:lb} is immediate and correct (union bound).
\item Lemma~\ref{lem:disjoint} is a direct divisibility argument.
\item Theorem~\ref{thm:C<log2} hinges on two standard facts: Mertens' asymptotic for $\sum_{p\le x}1/p$ and the observation that $C<\log 2$ implies $n^{e^{-C}}>\sqrt{n}$.
\end{itemize}

\subsection*{6) FINAL}
\addcontentsline{toc}{subsection}{6) Final (\#783)}

\paragraph{LABEL: \textbf{UNRESOLVED}.}

\paragraph{What is proved here.}
For $0<C<\log 2$, the minimum possible uncovered set size satisfies
\[
\min_A |U(A)| = (1-C)n + o(n),
\]
and can be achieved by choosing an appropriate set of primes all exceeding $\sqrt{n}$.

\paragraph{What remains open.}
For general fixed $C\ge \log 2$, overlaps become unavoidable and I do not have a proof that the ``largest primes'' choice is optimal.  The natural heuristic connects $|U(A)|$ to counts of smooth numbers, predicting $\approx n\rho(e^{C})$ uncovered for the ``all primes $\ge n^{e^{-C}}$'' choice, but optimality is not established.

\paragraph{Main obstacle.}
One needs a sharp upper bound on the size of the union of multiples under only the harmonic budget and coprimality constraints.  Controlling overlaps in a best-possible way is essentially a sieve-optimization problem.

\paragraph{Next concrete steps.}
\begin{enumerate}[leftmargin=2em]
\item Formulate the optimization as a constrained sieve and attempt to prove that any $A$ with $\sum 1/a\le C$ leaves at least $\psi(n,n^{e^{-C}})$ numbers uncovered (up to lower-order terms).
\item Investigate whether allowing prime powers (one per prime) can ever beat the ``large primes'' strategy for $C\ge \log 2$.
\item Compute the exact optimum for larger $n$ using integer programming with coprimality constraints, to guess the true asymptotic structure.
\end{enumerate}

\paragraph{COMPLETION: 55\%.}

\hrule

\begin{thebibliography}{99}

\bibitem{MaSp24}
S.~Malekshahian and G.~Spiro,
\emph{On a clique-building game of Erd\H{o}s}, arXiv:2410.18304 (2024).

\bibitem{CaPr25}
S.~Cambie and H.~Provoost,
\emph{On edge-colouring-games by Erd\H{o}s, and Bensmail and McInerney}, arXiv:2505.03497v2 (2025).

\bibitem{BEF90}
T.~C.~Brown, P.~Erd\H{o}s, and D.~Z.~Freedman,
\emph{Quasi-progressions and descending waves},
J. Combin. Theory Ser. A \textbf{53} (1990), 81--95.

\bibitem{So07}
J.~Solymosi,
\emph{Note on sumsets of integers},
European J. Combin. \textbf{28} (2007), no.~4, 1067--1071.

\bibitem{CiGr07}
J.~Cilleruelo and A.~Granville,
\emph{Lattice points on circles, squares in arithmetic progressions and sumsets of squares},
in \emph{Additive combinatorics}, CRM Proc. Lecture Notes, vol.~43, Amer. Math. Soc., Providence, RI, 2007, pp.~241--262.

\bibitem{DiEl15}
R.~Dietmann and C.~Elsholtz,
\emph{Hilbert cubes in arithmetic sets},
arXiv:1510.05260 (2015).

\end{thebibliography}

