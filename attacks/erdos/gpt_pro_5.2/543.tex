% Erdos Problem #543

\noindent\textbf{FORMAL RESTATEMENT.}
Let $N\ge 2$.
Define $f(N)$ to be the minimal integer $k$ such that the following statement holds:
for every finite abelian group $G$ with $|G|=N$, if $A\subseteq G$ is chosen uniformly at random among all $k$-element subsets of $G$, then
\[
\Pr\Bigl(\forall g\in G\ \exists S\subseteq A\text{ with } g=\sum_{x\in S} x\Bigr)\ge \frac12.
\]
(Here the empty sum is $0\in G$.)
The question asks whether
\[
 f(N) \le \log_2 N + o(\log\log N)
\]
as $N\to\infty$.

\noindent\textbf{QUICK LITERATURE/CONTEXT CHECK.}
The problem statement reports an upper bound $f(N)\le \log_2 N+O(\log\log N)$ due to Erd\H{o}s--R\'enyi, and also reports Erd\H{o}s's belief that improving $O(\log\log N)$ to $o(\log\log N)$ should be impossible.
Below I give self-contained lower bounds and analyze exactly the case $G\cong (\mathbb Z/2\mathbb Z)^d$.

\noindent\textbf{ATTACK PLAN.}
1) Prove basic counting lower bounds ($2^k\ge N$ is necessary).
2) Study explicit group families where subset-sums have clean algebraic meaning (notably exponent-2 groups).
3) Compute exact small-$N$ probabilities to sanity-check the scale.

\noindent\textbf{WORK.}
\textbf{Fast reality check (exact small-$N$ probabilities).}
For cyclic groups $G=\mathbb Z/N\mathbb Z$ with $N\le 12$, I enumerated all $k$-subsets and computed the exact probability that subset sums cover $G$.
The minimal $k$ with probability $\ge 1/2$ was:
\[
\begin{array}{c|ccccccccccc}
N &2&3&4&5&6&7&8&9&10&11&12\\\hline
\min k &1&3&3&4&4&4&4&5&5&5&5
\end{array}
\]
For elementary $2$-groups $G\cong (\mathbb Z/2\mathbb Z)^d$ (size $N=2^d$), exact enumeration for $d\le 4$ gave:
\[
\begin{array}{c|cccc}
 d &1&2&3&4\\\hline
 N=2^d &2&4&8&16\\\hline
\min k &1&2&3&5
\end{array}
\]

\medskip
\textbf{Lemma 543.1 (counting lower bound).}
For every $N\ge 2$,
\[
 f(N)\ge \lceil\log_2 N\rceil.
\]

\textbf{Proof.}
Fix $G$ with $|G|=N$ and any subset $A\subseteq G$ of size $k$.
The set of all subset sums of $A$ has cardinality at most $2^k$ because there are $2^k$ subsets $S\subseteq A$.
If all elements of $G$ are representable as subset sums, then $N\le 2^k$.
Therefore a necessary condition for the coverage property to hold (for any $G$ and any $A$) is $2^k\ge N$, i.e. $k\ge \lceil\log_2 N\rceil$.
Since $f(N)$ is defined as a minimal such $k$, it must satisfy this bound. \qed

\medskip
\textbf{Proposition 543.2 (exact threshold for $G=(\mathbb Z/2\mathbb Z)^d$ up to $\pm1$).}
Let $G=(\mathbb Z/2\mathbb Z)^d$ with $N=2^d$.
A $k$-subset $A\subseteq G$ has the property that \emph{every} element of $G$ is a subset sum of $A$ if and only if $A$ spans $G$ as an $\mathbb F_2$-vector space.
Moreover:
\begin{itemize}
\item If $k=d$, then
\[
\Pr(A\text{ spans }G)=\prod_{j=0}^{d-1}\frac{2^d-2^j}{2^d-j}.
\]
In particular, for $d\ge 4$ this probability is $<1/2$.
\item If $k=d+1$ and $d\ge 3$, then $\Pr(A\text{ spans }G)>1/2$.
\end{itemize}
Consequently, for $d\ge 4$ one has
\[
 f(2^d)\ge d+1 = \log_2 N +1
\]
(because $f(2^d)$ must work for this specific group).

\textbf{Proof.}
\emph{Step 1: subset sums = linear span.}
In $G=(\mathbb Z/2\mathbb Z)^d$, every element has order $2$, so adding an element twice gives $0$.
Given $A\subseteq G$, any subset sum $\sum_{x\in S} x$ is exactly an $\mathbb F_2$-linear combination of the elements of $A$ with coefficients in $\{0,1\}$.
Conversely, every $\mathbb F_2$-linear combination uses each vector with coefficient $0$ or $1$, hence is a subset sum.
Therefore the set of subset sums is exactly the $\mathbb F_2$-span of $A$.
Thus it equals all of $G$ if and only if $A$ spans.

\emph{Step 2: formula for spanning probability when $k=d$.}
Choose $A$ uniformly among $d$-element subsets.
Expose the elements sequentially: after choosing $j$ linearly independent vectors, their span has size $2^j$.
At the next selection step, there are $2^d-j$ remaining vectors not yet chosen.
Among these, exactly $2^j-j$ lie in the current span but are not already chosen.
Hence the number of remaining vectors outside the span is
\((2^d-j)-(2^j-j)=2^d-2^j\), and the conditional probability that the next chosen vector increases the rank is
\((2^d-2^j)/(2^d-j)\).
Multiplying these conditional probabilities for $j=0,1,\dots,d-1$ yields
\[
\Pr(A\text{ is a basis})=\prod_{j=0}^{d-1}\frac{2^d-2^j}{2^d-j}.
\]
For $d=4$ this equals
\(
\frac{15}{16}\cdot\frac{14}{15}\cdot\frac{12}{14}\cdot\frac{8}{13}=\frac{6}{13}<\frac12.
\)
Now assume $d\ge 5$.
For each $j\in\{0,1,\dots,d-1\}$ we have $2^j>j$ (indeed $2^j\ge j+1$ for all $j\ge 0$), hence
\(2^d-2^j < 2^d-j\) and so every factor \((2^d-2^j)/(2^d-j)\) is $<1$.
Therefore the full product is bounded by the product of its last three factors:
\[
\Pr(A\text{ spans})
\le \prod_{j=d-3}^{d-1}\frac{2^d-2^j}{2^d-j}.
\]
Writing these three factors explicitly,
\[
\prod_{j=d-3}^{d-1}\frac{2^d-2^j}{2^d-j}
=\frac{2^d-2^{d-3}}{2^d-(d-3)}\cdot\frac{2^d-2^{d-2}}{2^d-(d-2)}\cdot\frac{2^d-2^{d-1}}{2^d-(d-1)}.
\]
Factor out $2^d$ from each ratio to get
\[
=\frac{7/8}{1-(d-3)/2^d}\cdot\frac{3/4}{1-(d-2)/2^d}\cdot\frac{1/2}{1-(d-1)/2^d}.
\]
For $d\ge 5$ one has $(d-1)/2^d\le 4/32=1/8$, so each denominator term is at least $1-1/8=7/8$.
Hence
\[
\Pr(A\text{ spans})
\le \frac{21}{64}\cdot\frac{1}{(7/8)^3}=\frac{168}{343}<\frac12.
\]
Combining with the explicit $d=4$ computation, we conclude that for all $d\ge 4$, with $k=d$ the success probability is $<1/2$.

\emph{Step 3: $k=d+1$ succeeds with probability $>1/2$ for $d\ge 3$.}
Failure to span means $A$ is contained in some proper subspace.
Every proper subspace is contained in a codimension-$1$ subspace (a hyperplane).
In $G=(\mathbb Z/2\mathbb Z)^d$, the number of distinct hyperplanes equals the number of nonzero linear functionals, which is $2^d-1$.
Fix a hyperplane $H$; it has size $2^{d-1}$.
The probability that a uniformly random $(d+1)$-subset $A$ lies entirely inside this fixed $H$ is
\(
\binom{2^{d-1}}{d+1}/\binom{2^d}{d+1}.
\)
Using the product formula for binomial ratios,
\[
\frac{\binom{2^{d-1}}{d+1}}{\binom{2^d}{d+1}}=
\prod_{i=0}^{d}\frac{2^{d-1}-i}{2^d-i}.
\]
For each $i\ge 0$,
\(
\frac{2^{d-1}-i}{2^d-i}\le \frac{1}{2}
\)
because $2(2^{d-1}-i)=2^d-2i\le 2^d-i$.
Therefore
\[
\frac{\binom{2^{d-1}}{d+1}}{\binom{2^d}{d+1}}\le 2^{-(d+1)}.
\]
By the union bound over all $2^d-1$ hyperplanes,
\[
\Pr(A\text{ fails to span})
\le (2^d-1)\cdot 2^{-(d+1)}< 2^d\cdot 2^{-(d+1)}=\frac12.
\]
So $\Pr(A\text{ spans})>1/2$ for $d\ge 3$.

\emph{Step 4: implication for $f(2^d)$.}
For $d\ge 4$, we have shown that in this specific group, $k=d$ gives success probability $<1/2$.
Hence any $k$ that works in the definition of $f(2^d)$ must be at least $d+1$.
Thus $f(2^d)\ge d+1=\log_2(2^d)+1$. \qed

\noindent\textbf{VERIFICATION.}
The key coupling in Proposition 543.2 is valid because in exponent-$2$ groups, subset sums coincide with $\mathbb F_2$-linear combinations.
In Step 3, the only estimate used is that a random $(d+1)$-subset lying in a fixed half-size hyperplane has probability at most $2^{-(d+1)}$, proved term-by-term.

\noindent\textbf{FINAL.}\;\textbf{UNRESOLVED.}
\begin{itemize}
\item[(i)] \textbf{Strongest proved partial result.} Always $f(N)\ge \lceil\log_2 N\rceil$ (Lemma 543.1). Moreover, for $N=2^d$ with $d\ge 4$, considering $G\cong(\mathbb Z/2\mathbb Z)^d$ yields the stronger lower bound $f(2^d)\ge d+1=\log_2 N+1$ (Proposition 543.2).
\item[(ii)] \textbf{First gap (crisp statement).} Decide whether there exists a sequence of group orders $N\to\infty$ for which $f(N)\ge \log_2 N + c\log\log N$ (for some absolute $c>0$), which would refute $\log_2 N+o(\log\log N)$; or conversely, prove $f(N)\le \log_2 N+o(\log\log N)$ uniformly over all abelian $|G|=N$.
\item[(iii)] \textbf{Top 3 next moves.}
  \begin{itemize}
  \item Analyze cyclic groups $\mathbb Z/N\mathbb Z$ (or groups with large odd-order components) using Fourier methods for the random subset-sum walk to estimate the expected number of uncovered elements as a function of $k$.
  \item Seek worst-case groups: understand which abelian group structures maximize the probability that a random $k$-subset has many collisions among its subset sums.
  \item Computationally: for moderate $N$ (say $N\le 200$), Monte Carlo estimate the coverage probability in different group types to guess the extremal family.
  \end{itemize}
\item[(iv)] \textbf{Minimal counterexample structure to look for.} To force $f(N)\ge \log_2 N + \omega(1)$, one needs an abelian group in which random subset sums exhibit many collisions so that the multiset of $2^k$ sums fails to cover $G$ even when $2^k$ is moderately larger than $N$; cyclic groups (or groups with large odd exponent) are natural candidates because the subset-sum map is not an $\mathbb F_2$-linear map there.
\end{itemize}


