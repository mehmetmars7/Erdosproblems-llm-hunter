
1) FORMAL RESTATEMENT

Fix an integer $k\ge 2$.

Definition.
For each $n\ge 1$, define $g_k(n)$ by
\[
 g_k(n):=\max\Bigl\{ (a_1+\cdots+a_k)-n:\ a_1,\dots,a_k\in\mathbb N,\ \prod_{i=1}^k a_i!\ \mid\ n!\Bigr\}.
\]
(Here $\mathbb N=\{1,2,3,\dots\}$. Allowing $0$ would not change the maximum since $0!=1=1!$ but $1>0$.)

Questions.
(A) Does $\sum_{n\le x} g_k(n)\sim c_k x\log x$ for some constant $c_k$?
(B) Is there a constant $c_k$ such that for almost all $n<x$, $g_k(n)=c_k\log x+o(\log x)$?

Edge cases.
$g_k(1)$ is well-defined: taking all $a_i=1$ gives $\prod a_i!=1\mid 1!$ and $g_k(1)\ge k-1$.

2) QUICK LITERATURE/CONTEXT CHECK

I only restate what is recorded in the problem statement.

The statement records:
- Erd\H{o}s and Graham noted it is easy to show $g_k(n)\ll_k \log n$ for all $n$, but the best constant is unknown.

3) ATTACK PLAN

Proof-track (partial):
- Prove an explicit $O_k(\log n)$ upper bound (as the statement claims is easy).
- Give trivial lower bounds and compute small values for sanity.

Disproof-track:
- Try to find $n$ for which $g_k(n)$ is unusually large compared to $\log n$ by brute force (small $n$ only).

Chosen path: prove a clean $O_k(\log n)$ bound using the $2$-adic valuation of factorials, and compute $g_k(n)$ for small $n$.

4) WORK

PHASE 1 — FAST REALITY CHECK (computed)

Brute force (over $a_i\in\{1,\dots,n\}$) gives the following values and one witnessing $k$-tuple for $n\le 30$.

For $k=2$:
\[
\begin{array}{c|cccccccccc}
 n&1&2&3&4&5&6&7&8&9&10\\\hline
 g_2(n)&1&1&1&1&1&2&1&1&2&3
\end{array}
\]
and continuing up to $n=30$, the maximum observed is $g_2(30)=2$ with witness $(3,29)$.

For $k=3$ (selected values):
$g_3(10)=5$ with witness $(3,5,7)$, and $g_3(30)=6$ with witness $(6,15,15)$.

Lemma 1 (a universal lower bound).
For all $n\ge 1$ and $k\ge 2$,
\[
 g_k(n)\ge k-1.
\]

Proof.
Take $(a_1,\dots,a_{k-1},a_k)=(1,1,\dots,1,n)$. Then
\[
\prod_{i=1}^k a_i! = (1!)^{k-1}\cdot n! = n!\mid n!.
\]
The corresponding value of $(a_1+\cdots+a_k)-n$ is
\[
((k-1)\cdot 1+n)-n=k-1.
\]
Since $g_k(n)$ is a maximum over all admissible $k$-tuples, $g_k(n)\ge k-1$. \qed

Lemma 2 ($2$-adic valuation identity for factorials).
Let $s_2(m)$ be the sum of the binary digits of $m$.
Then for every integer $m\ge 1$,
\[
 v_2(m!) = m - s_2(m).
\]

Proof.
By Legendre's formula,
\[
 v_2(m!)=\sum_{j\ge 1} \Bigl\lfloor\frac{m}{2^j}\Bigr\rfloor.
\]
Write $m$ in binary as $m=\sum_{t=0}^L \varepsilon_t 2^t$ with $\varepsilon_t\in\{0,1\}$.
For fixed $j\ge 1$,
\[
\Bigl\lfloor\frac{m}{2^j}\Bigr\rfloor = \sum_{t=j}^L \varepsilon_t 2^{t-j}.
\]
Summing over $j\ge 1$ and swapping the order of summation gives
\[
\sum_{j\ge 1}\Bigl\lfloor\frac{m}{2^j}\Bigr\rfloor
=\sum_{t=1}^L \varepsilon_t\sum_{j=1}^{t} 2^{t-j}
=\sum_{t=1}^L \varepsilon_t(2^t-1).
\]
Now $\sum_{t=1}^L \varepsilon_t2^t = m-\varepsilon_0$ and $\sum_{t=1}^L \varepsilon_t = s_2(m)-\varepsilon_0$. Therefore
\[
\sum_{t=1}^L \varepsilon_t(2^t-1)=(m-\varepsilon_0)-(s_2(m)-\varepsilon_0)=m-s_2(m).
\]
This proves $v_2(m!)=m-s_2(m)$. \qed

Proposition 3 (explicit $O_k(\log n)$ upper bound).
For all $n\ge 2$,
\[
 g_k(n)\le k\bigl(\lfloor\log_2 n\rfloor+1\bigr).
\]

Proof.
Let $(a_1,\dots,a_k)$ be any admissible $k$-tuple, so $\prod_{i=1}^k a_i!\mid n!$.
Taking $2$-adic valuations gives
\[
\sum_{i=1}^k v_2(a_i!)\le v_2(n!).
\]
Apply Lemma 2 to each factorial:
\[
\sum_{i=1}^k (a_i-s_2(a_i))\le n-s_2(n).
\]
Rearrange to isolate the sum of the $a_i$:
\[
(a_1+\cdots+a_k)-n \le \sum_{i=1}^k s_2(a_i) - s_2(n)\le \sum_{i=1}^k s_2(a_i).
\]
For each $i$, since $1\le a_i\le n$ (otherwise $a_i!$ would contain a prime factor $>n$ and could not divide $n!$), the binary expansion of $a_i$ has at most $\lfloor\log_2 n\rfloor+1$ digits, hence
\[
 s_2(a_i)\le \lfloor\log_2 n\rfloor+1.
\]
Therefore
\[
(a_1+\cdots+a_k)-n \le k(\lfloor\log_2 n\rfloor+1).
\]
Taking the maximum over all admissible tuples yields the bound on $g_k(n)$. \qed

5) VERIFICATION

- Lemma 2 checked for small $m$ numerically during development; the algebraic proof is self-contained.
- Proposition 3: the only nontrivial ingredient is taking $v_2$ of the divisibility relation; all subsequent steps are explicit inequalities.
- Sanity check: for $k=2$, Proposition 3 gives $g_2(10)\le 2(\lfloor\log_2 10\rfloor+1)=2(3+1)=8$, consistent with the computed value $g_2(10)=3$.

6) FINAL

**UNRESOLVED**

(i) Strongest fully proved partial result obtained here.
- Explicit upper bound $g_k(n)\le k(\lfloor\log_2 n\rfloor+1)$ (Proposition 3).
- Trivial lower bound $g_k(n)\ge k-1$ (Lemma 1).
- Computed exact values for $g_2(n)$ and $g_3(n)$ up to $n=30$ (WORK).

(ii) Exact first gap.
Derive asymptotics for the mean values $\sum_{n\le x} g_k(n)$, or even determine whether $g_k(n)$ typically has size proportional to $\log n$ (and identify the constant).

(iii) Top 3 next moves (concrete targets).
1. Refine the $2$-adic argument by understanding the typical size of $\sum_i s_2(a_i)$ for extremising tuples and how it depends on $n$.
2. Use $p$-adic constraints for multiple primes (not just $p=2$) to improve constants and possibly approach a limiting constant $c_k$.
3. Large-scale computation of $g_k(n)$ for $n$ up to a large bound (say $10^6$ for $k=2$) to estimate $\sum_{n\le x} g_k(n)/(x\log x)$ and test concentration.

(iv) What a minimal counterexample would likely look like.
A counterexample to the conjectured “almost always constant” behaviour for $g_k(n)/\log n$ would likely be a positive-density set of $n$ for which extremising tuples force unusually large binary digit sums (or analogous $p$-adic digit sums), causing $g_k(n)$ to deviate systematically from a single constant multiple of $\log n$.


