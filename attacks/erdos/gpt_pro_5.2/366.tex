
## 1) FORMAL RESTATEMENT

### 1.1. Definitions
For an integer \(k\ge 2\), a positive integer \(n\) is called **\(k\)-full** if
\[
\forall\text{ primes }p:\ p\mid n\ \Rightarrow\ p^k\mid n.
\]
Thus:
* \(2\)-full \(\Leftrightarrow\) powerful/squareful.
* \(3\)-full is sometimes called **cubeful**.

### 1.2. Ambiguity/misstatement in the provided problem text
The problem text asks:

> “Are there any 2-full \(n\) such that \(n+1\) is 3-full?”

but then cites consecutive examples \((8,9)\) with “8 is 3-full and 9 is 2-full” and
\((12167,12168)\) with “\(12167=23^3\) and \(12168=2^3\cdot 3^2\cdot 13^2\)” (again: first is 3-full,
second is 2-full).

So the examples match the **reversed** condition.

I will therefore separate:

* **Literal statement:** find \(n\) such that \(n\) is 2-full and \(n+1\) is 3-full.
* **Corrected statement (minimal change consistent with examples):** find \(n\) such that \(n\) is
  3-full and \(n+1\) is 2-full; and ask whether the only such pairs are \((8,9)\) and
  \((12167,12168)\).

---

## 2) QUICK LITERATURE/CONTEXT CHECK

I only use what is explicitly stated in the problem text:

* The pair \((8,9)\) has 8 = 3-full and 9 = 2-full.
* Another pair is \((12167,12168)\) with \(12167=23^3\) and \(12168=2^3\cdot 3^2\cdot 13^2\).
* It is stated (via OEIS A060355) that there are no other examples for \(n<10^{22}\).

---

## 3) ATTACK PLAN

1. Clarify the intended (corrected) formulation.
2. Prove structural lemmas for 2-full and 3-full integers, especially congruence constraints for
   consecutive pairs.
3. Do a modest computational scan to look for examples in both the literal and corrected directions.

---

## 4) WORK

### Phase 1: FAST REALITY CHECK (computation)

I searched up to \(n\le 2\cdot 10^6\) and found:

* **Literal direction** (2-full \(n\), 3-full \(n+1\)): no examples.
* **Corrected direction** (3-full \(n\), 2-full \(n+1\)): exactly two examples,
  \((n,n+1)=(8,9)\) and \((12167,12168)\).

---

### Lemma 4.1 (Parity constraints for consecutive 2-full/3-full)
Let \(m\ge 2\).

1. If \(n\) is 2-full and even, then \(4\mid n\).
2. If \(n\) is 3-full and even, then \(8\mid n\).

Consequently, in any consecutive pair \((n,n+1)\) with one 2-full and the other 3-full:

* If the 3-full number is even, it is \(0\pmod 8\).
* If the 2-full number is even, it is \(0\pmod 4\).

**Proof.**
1. If \(n\) is 2-full and even, then \(2\mid n\). By definition of 2-full, this implies \(2^2\mid n\),
   i.e. \(4\mid n\).
2. If \(n\) is 3-full and even, then \(2\mid n\). By definition of 3-full, \(2^3\mid n\), i.e. \(8\mid n\).

The consequences are immediate specializations. \(\square\)

---

### Lemma 4.2 (2-full \(\Leftrightarrow a^2 b^3\) with \(b\) squarefree)
A positive integer \(n\) is 2-full if and only if \(n=a^2 b^3\) with \(a,b\in\mathbb N\) and \(b\)
 squarefree.

**Proof.** This is exactly Lemma 4.1 in Problem #364 above, specialized to “2-full = powerful.” \(\square\)

---

### Proposition 4.3 (Verification of the stated examples for the corrected statement)
The pairs \((8,9)\) and \((12167,12168)\) satisfy: first number 3-full, second number 2-full.

**Proof.**

* \(8=2^3\) is 3-full (the only prime divisor is 2, occurring to exponent 3). \(9=3^2\) is 2-full.
* \(12167=23^3\) is 3-full.
  \(12168 = 2^3\cdot 3^2\cdot 13^2\) has prime exponents \(3,2,2\), all \(\ge 2\), hence is 2-full.

\(\square\)

---

## 5) VERIFICATION

* Lemma 4.1 is a direct reading of the definition applied to the prime \(2\).
* Lemma 4.2 was proved fully in Problem #364 (Lemma 4.1 there).
* Proposition 4.3 uses explicit factorizations given in the problem text.
* Computation: the scan is a sanity check only and is not used for any asymptotic or global claim.

---

## 6) FINAL

**UNRESOLVED**

(i) **Strongest fully proved partial result obtained here.**

* Basic necessary congruence constraints for any such consecutive pair (Lemma 4.1).
* Explicit verification that \((8,9)\) and \((12167,12168)\) satisfy the corrected condition
  (Proposition 4.3).
* Computationally, no examples of the literal condition and only those two examples of the corrected
  condition were found for \(n\le 2\cdot 10^6\).

(ii) **Exact first gap.**

For the corrected statement suggested by the examples:

> Prove or disprove that the only solutions to “\(n\) is 3-full and \(n+1\) is 2-full” are
> \((n,n+1)=(8,9)\) and \((12167,12168)\).

(iii) **Top 3 next moves (concrete targets).**

1. Parametrize 2-full numbers as \(a^2 b^3\) (Lemma 4.2) and similarly find a convenient
   parametrization of 3-full numbers to reduce the corrected problem to an explicit Diophantine
   equation.
2. Use gcd constraints: \(\gcd(n,n+1)=1\) forces disjoint prime sets, which is very restrictive when
   both sides have only high prime exponents.
3. Extend computation with a sieve for k-full numbers to search for additional solutions (if any)
   and look for modular patterns that could be proved.

(iv) **What a minimal counterexample would likely look like.**

If the corrected statement has more solutions, the next one would be a much larger pair of
consecutive integers \(n,n+1\) with disjoint prime supports, where \(n\) is cubeful (every prime
exponent \(\ge 3\)) and \(n+1\) is squareful (every prime exponent \(\ge 2\)). By Lemma 4.1, the even
member must be divisible by \(8\) (if it is cubeful) or \(4\) (if it is squareful), so any counterexample
must respect these strong 2-adic constraints.


