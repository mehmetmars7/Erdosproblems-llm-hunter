% Erdos Problem #503

1) FORMAL RESTATEMENT

Fix an integer $d\ge 1$.
Find the maximum possible size of a set $A\subseteq\mathbb R^d$ such that for every triple of \emph{distinct} points $x,y,z\in A$, the triangle with vertices $x,y,z$ is isosceles, i.e.
\[
\forall\ x,y,z\in A\ \text{distinct},\quad \text{at least two of }\lvert x-y\rvert,\ \lvert y-z\rvert,\ \lvert x-z\rvert\ \text{are equal}.
\]
(Equivalently: every 3-point subset determines an isosceles triangle.)

2) QUICK LITERATURE/CONTEXT CHECK

The problem statement records the exact answers $6$ for $d=2$ and $8$ for $d=3$, an upper bound $\lvert A\rvert\le \binom{d+2}{2}$ (Blokhuis), and a lower bound construction of size $\binom{d+1}{2}$ (Alweiss) improved to $\binom{d+1}{2}+1$ (Weisenberg, as reported).

Below I verify (with full proofs) the lower bound constructions.

3) ATTACK PLAN

\emph{Proof track:} give explicit lower bound constructions and verify the isosceles property by an explicit distance calculation.

\emph{Disproof/construction track:} check small dimensions and understand why these constructions fall short at $d=2,3$; try to extend constructions or detect structural obstructions.

4) WORK

Lemma 503.1 (Alweiss construction; only two distances occur).
Let $d\ge 1$ and consider the set
\[
S:=\{e_i+e_j\in\mathbb R^{d+1}: 1\le i<j\le d+1\},
\]
where $e_1,\dots,e_{d+1}$ are the standard basis vectors.
Then for any two distinct points $p,q\in S$, the distance $\lVert p-q\rVert$ is either $\sqrt2$ or $2$.
Consequently, every triangle determined by three distinct points of $S$ is isosceles.

\emph{Proof.}
Take $p=e_i+e_j$ and $q=e_k+e_\ell$ with $\{i,j\}\ne\{k,\ell\}$.
There are two possibilities:

\underline{(a) The pairs share exactly one index.}
For example $k=i$ and $\ell\notin\{i,j\}$.
Then $p-q = (e_i+e_j)-(e_i+e_\ell)=e_j-e_\ell$ has exactly two nonzero coordinates, one $+1$ and one $-1$.
Thus $\lVert p-q\rVert^2=1^2+(-1)^2=2$, so $\lVert p-q\rVert=\sqrt2$.

\underline{(b) The pairs are disjoint.}
Then $p-q=e_i+e_j-e_k-e_\ell$ has four nonzero coordinates, two $+1$ and two $-1$.
Thus $\lVert p-q\rVert^2=1+1+1+1=4$, so $\lVert p-q\rVert=2$.

These are the only possibilities for distinct unordered pairs of indices.
So all pairwise distances in $S$ lie in $\{\sqrt2,2\}$.
Given any three distinct points in $S$, their three mutual distances each lie in a 2-element set, hence by the pigeonhole principle at least two of the three distances are equal. Therefore every such triangle is isosceles.
\qed

Lemma 503.2 (Adding one more point in the same affine subspace).
Let $u:=(2/(d+1),\dots,2/(d+1))\in\mathbb R^{d+1}$ and define $S' := S\cup\{u\}$.
Then every triangle determined by three distinct points of $S'$ is isosceles.
Moreover, $S'$ lies in the affine hyperplane $H:=\{x\in\mathbb R^{d+1}: \sum_{i=1}^{d+1} x_i=2\}$, which has affine dimension $d$, so $S'$ can be viewed isometrically as a subset of $\mathbb R^d$.

\emph{Proof.}
First, note that every point of $S$ has coordinate sum $2$, and $u$ also has coordinate sum $(d+1)\cdot (2/(d+1))=2$, hence $S'\subset H$.
The affine hyperplane $H$ is a translate of the $d$-dimensional linear subspace $\{x: \sum x_i=0\}$, so $H$ has affine dimension $d$ and is isometric to $\mathbb R^d$.

Now fix $p=e_i+e_j\in S$.
Compute $\lVert p-u\rVert^2$:
$p$ has coordinates $1$ in positions $i,j$ and $0$ elsewhere.
Thus $p-u$ has coordinates $1-2/(d+1)$ in positions $i,j$ and $-2/(d+1)$ in the other $d-1$ positions.
Therefore
\[
\lVert p-u\rVert^2
=2\left(1-\frac{2}{d+1}\right)^2+(d-1)\left(\frac{2}{d+1}\right)^2,
\]
which depends only on $d$ and not on $i,j$.
Hence $\lVert p-u\rVert$ is the same constant for all $p\in S$.

Now consider any triangle with vertices among $S'$.
- If all three vertices lie in $S$, it is isosceles by Lemma 503.1.
- If one vertex is $u$ and the other two are $p,q\in S$, then $\lVert u-p\rVert=\lVert u-q\rVert$ by the constant-distance property, so the triangle is isosceles.

Thus every triangle from $S'$ is isosceles.
\qed

FAST REALITY CHECK (dimension $d=1$).
In $\mathbb R^1$, the largest such set has size $3$.
Indeed, if $A\subset\mathbb R$ contains four points $a<b<c<d$, then the triple $(a,b,c)$ being isosceles forces $b-a=c-b$ (the only possible equality among the three distances), and similarly $(b,c,d)$ forces $c-b=d-c$. Thus $a,b,c,d$ form an arithmetic progression with step $t>0$.
But then the triple $(a,c,d)$ has distances $\lvert c-a\rvert=2t$, $\lvert d-c\rvert=t$, and $\lvert d-a\rvert=3t$, which are all distinct, contradicting the required property.
On the other hand, any 3-term arithmetic progression achieves size 3.

5) VERIFICATION

\emph{Lemma 503.1:} the distance computation is explicit and exhaustive for two distinct index-pairs.

\emph{Lemma 503.2:} the added point $u$ was chosen to lie in the same affine hyperplane $\sum x_i=2$ so the configuration remains $d$-dimensional up to isometry; the constant-distance calculation ensures every triangle involving $u$ is isosceles.

6) FINAL
UNRESOLVED

(i) \emph{Strongest proved partial result here:} There exist sets of size $\binom{d+1}{2}$ (Lemma 503.1) and even $\binom{d+1}{2}+1$ (Lemma 503.2) in $\mathbb R^d$ such that every triangle is isosceles.

(ii) \emph{First gap (crisp):} Determine the exact maximum size $M(d)$ of such a set in $\mathbb R^d$ for general $d$ (even asymptotically in $d$), and in particular decide whether $M(d)$ is closer to $\binom{d+1}{2}$, $\binom{d+2}{2}$, or something else.

(iii) \emph{Top 3 next moves:}
1. For small $d\ge 4$, do a computational search (e.g. over rational coordinates up to bounded denominator, or via SAT/ILP on distance patterns) to guess $M(d)$.
2. Try to prove that such sets determine only few distances, or impose strong constraints on Gram matrices, to recover an upper bound of order $d^2$ by an elementary argument.
3. Attempt to classify extremal configurations for $d=2,3$ to see what feature allows $6$ and $8$, then generalize that feature.

(iv) \emph{Minimal counterexample structure (to the lower bound being tight):} an extremal set larger than $\binom{d+1}{2}+1$ would have to introduce additional points while preserving the "two-distance" behavior (or some other mechanism) that forces isosceles triangles; it likely would require a highly symmetric configuration rather than a small perturbation of $S'$.


