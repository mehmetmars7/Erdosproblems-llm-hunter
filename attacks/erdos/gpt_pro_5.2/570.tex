% Erdos Problem #570
% URL: https://www.erdosproblems.com/570

1) FORMAL RESTATEMENT

For $k\ge 3$, the problem asks whether the following inequality holds:

For every graph $H$ with $m=e(H)$ edges and with no isolated vertices,
\[
R(C_k,H) \le 2m+\left\lceil\frac{k-1}{2}\right\rceil.
\]

2) QUICK LITERATURE/CONTEXT CHECK

The problem file notes that this statement was proved for even $k$ by Erd\H{o}s, Faudree, Rousseau, and Schelp [EFRS93], and for $k=3$ by Sidorenko [Si93].

3) ATTACK PLAN

Before attempting a proof, check for small-$m$ obstructions. In particular, test $m=1$ (so $H=K_2$) because $R(C_k,K_2)$ is elementary.

4) WORK

\emph{Fast reality check.}

Take $m=1$, so $H=K_2$. Then $R(C_k,K_2)=k$ (Lemma~570.1 below). For $k\ge 5$ the proposed upper bound evaluates to
\[
2\cdot 1+\left\lceil\frac{k-1}{2}\right\rceil = 2+\left\lceil\frac{k-1}{2}\right\rceil < k,
\]
so the inequality fails.

\medskip

\textbf{Lemma 570.1 (exact value for $H=K_2$).}
For every $k\ge 3$,
\[
R(C_k,K_2)=k.
\]

\textbf{Proof.}
This is exactly Lemma~569.1 specialised to $\ell=k$. (The proof is repeated briefly.)

- In $K_{k-1}$ colour all edges red. Then there is no blue edge and no red $C_k$ (not enough vertices), so $R(C_k,K_2)>k-1$.

- In $K_k$, any colouring either has a blue edge (giving blue $K_2$) or is all red, in which case the red $K_k$ contains a $C_k$.

Hence $R(C_k,K_2)=k$. \qed

\medskip

\textbf{Lemma 570.2 (the proposed inequality fails for all $k\ge 5$).}
For every $k\ge 5$,
\[
R(C_k,K_2)=k > 2+\left\lceil\frac{k-1}{2}\right\rceil.
\]

\textbf{Proof.}
By Lemma~570.1, $R(C_k,K_2)=k$.

For $k\ge 5$ we have $\lceil (k-1)/2\rceil \le k/2$ (since $\lceil x\rceil\le x+1/2$ for half-integers, and here $(k-1)/2$ is either an integer or a half-integer). Thus
\[
2+\left\lceil\frac{k-1}{2}\right\rceil \le 2+\frac{k}{2}.
\]
Now $k>2+k/2$ is equivalent to $k/2>2$, i.e. $k>4$, which holds for all $k\ge 5$. Therefore
\[
2+\left\lceil\frac{k-1}{2}\right\rceil < k = R(C_k,K_2).
\]
So the claimed inequality fails when $H=K_2$ (which has $m=1$ and no isolated vertices).
\qed

\medskip

\textbf{Additional computed checks for $k=5$ and matchings.}
Although the conjectured inequality is false as stated, for $k=5$ and $m\ge 2$ it matches small computed values for matchings:
\[
R(C_5,2K_2)=6 = 2\cdot 2+2,\qquad R(C_5,3K_2)=8=2\cdot 3+2.
\]

5) VERIFICATION

- The counterexample uses $H=K_2$ (one edge, no isolated vertices) and the exact evaluation $R(C_k,K_2)=k$.

- The strict inequality $2+\lceil (k-1)/2\rceil<k$ is checked explicitly in Lemma~570.2.

- The numerical checks $R(C_5,2K_2)=6$ and $R(C_5,3K_2)=8$ were computed locally by exhaustive search/backtracking.

6) FINAL

LABEL: **FULL SOLUTION**

SUBLABEL: **COUNTEREXAMPLE/DISPROOF**

The statement ``for all $k\ge 3$ and all $m$-edge graphs $H$ without isolated vertices, $R(C_k,H)\le 2m+\lceil (k-1)/2\rceil$'' is false.

A concrete counterexample is: take any $k\ge 5$ and take $H=K_2$ (so $m=1$). Then $R(C_k,K_2)=k$ (Lemma~570.1), but the proposed bound gives $2+\lceil (k-1)/2\rceil<k$ (Lemma~570.2), contradicting the inequality.

(If one wants a corrected version consistent with standard conventions, a minimal repair is to add a hypothesis ``$m$ is sufficiently large (depending on $k$)'', or to replace the right-hand side by $\max\{k,\,2m+\lceil (k-1)/2\rceil\}$.
The corrected statement is not proved here.)


