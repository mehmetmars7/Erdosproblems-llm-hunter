
1) FORMAL RESTATEMENT

Let $(n_i)_{i\ge 1}$ be a strictly increasing sequence of integers with $n_i\ge 1$.
For each $i$, fix an integer residue class $a_i\pmod{n_i}$.

Define the set $A\subseteq\mathbb N$ by
\[
A := \{n\in\mathbb N : \forall i\ge 1,\ (n<n_i)\ \lor\ (n\not\equiv a_i \pmod{n_i})\}.
\]
Equivalently, for each fixed $n$, the condition only constrains indices with $n_i\le n$:
\[
 n\in A\ \Longleftrightarrow\ \forall i\text{ with }n_i\le n,\ n\not\equiv a_i\pmod{n_i}.
\]

Define the \emph{logarithmic density} (if it exists) by
\[
\delta_{\log}(A) := \lim_{x\to\infty} \frac{1}{\log x}\sum_{\substack{1\le n\le x\\ n\in A}}\frac{1}{n}.
\]
Question: Must $\delta_{\log}(A)$ exist for every choice of $(n_i,a_i)$?

2) QUICK LITERATURE/CONTEXT CHECK

The provided problem statement only notes that this is a special case of another problem [486]. I do not use or claim any external literature beyond what is stated in the problem file.

3) ATTACK PLAN

Disproof track (ambitious):
- Try to choose the residue classes so that on long intervals $[n_k, n_{k+1})$ the set $A$ alternates between two different periodic sets whose densities differ by a fixed amount, and arrange the interval lengths so that the logarithmic weights do not average out, forcing non-convergence.

Proof track (ambitious):
- Prove that the weighted average defining logarithmic density always converges for this ``growing sieve'' structure, perhaps using an approximate martingale or subadditivity.

What I can deliver: existence of logarithmic density in two nontrivial regimes (finite constraints, and a general ``convergent densities + mild growth'' condition), plus explicit sanity-check examples.

4) WORK

FAST REALITY CHECK (examples + computation)

Example A (nested moduli): take $n_i=2^i$ and $a_i\equiv 0\pmod{2^i}$. For any $n$, if $n$ is odd then it avoids all these congruences; if $n$ is even then it violates the $i=1$ condition ($n\equiv 0\pmod 2$ with $2\le n$). Thus $A$ is exactly the odd integers. The logarithmic density exists and equals $1/2$.

Example B (sparse $A$): take $n_i=2^i$ and $a_i\equiv 2^{i-1}\pmod{2^i}$.
Then for an integer $n\ge 2$, write $n=2^t\cdot u$ with $u$ odd. If $u\ge 3$ then $n\ge 2^{t+1}$ and $n\equiv 2^t\pmod{2^{t+1}}$, so it violates the condition with $i=t+1$. Hence the only $n\ge 2$ in $A$ are powers of $2$. So $A=\{1\}\cup\{2^t:t\ge 0\}$ has logarithmic density $0$.

Numerical check (exact computation of the truncated ratio $\frac{1}{\log X}\sum_{n\le X, n\in A}1/n$):
- For Example A (odd integers), the ratio at $X=10^6$ is about $0.54598$, trending toward $0.5$.
- For Example B (powers of $2$), the ratio at $X=10^6$ is about $0.14476$ and decreasing, consistent with limit $0$.

Lemma 25.1 (finite set of congruence constraints $\Rightarrow$ logarithmic density exists).
Assume only finitely many constraints are present, i.e., there is some $K$ such that we only impose the conditions for $i=1,\dots,K$ (or equivalently, ignore all $i>K$). Then $A$ has a natural density, hence a logarithmic density.

Proof.
Let $L:=\mathrm{lcm}(n_1,\dots,n_K)$. For each $i\le K$, the set of integers congruent to $a_i\pmod{n_i}$ is a union of residue classes modulo $L$. Therefore the set $A$ defined by avoiding these finitely many progressions is also a union of residue classes modulo $L$.

Any union of residue classes modulo $L$ is periodic with period $L$, so its natural density exists and equals $|A\cap\{1,\dots,L\}|/L$.

Whenever the natural density exists, the logarithmic density exists and equals the same value (because partial summation shows that for periodic sets, the weighted sum $\sum_{n\le x, n\in A}1/n$ equals $(\text{density})\log x + O(1)$).

Thus $\delta_{\log}(A)$ exists. $\square$

Lemma 25.2 (log-density exists under convergent stage densities and mild modulus growth).
For each $k\ge 1$, define the periodic set
\[
A^{(k)} := \{n\in\mathbb N: n\not\equiv a_i\pmod{n_i}\text{ for all }1\le i\le k\}.
\]
Let $\delta_k$ be the natural density of $A^{(k)}$ (which exists because $A^{(k)}$ is periodic modulo $\mathrm{lcm}(n_1,\dots,n_k)$).

Assume:

(H1) $\delta_k$ converges to some limit $\delta\in[0,1]$ as $k\to\infty$.

(H2) The modulus growth is such that
\[
\frac{\max_{1\le j\le k} \log\bigl(n_{j+1}/n_j\bigr)}{\log n_k}\ \longrightarrow\ 0\quad (k\to\infty).
\]
(For example, it suffices that the ratios $n_{j+1}/n_j$ are uniformly bounded.)

Then the logarithmic density $\delta_{\log}(A)$ exists and equals $\delta$.

Proof.
Fix $k\ge 1$ and consider integers $n$ with
\[
 n_k \le n < n_{k+1}.
\]
For such $n$, the defining condition for $A$ only involves indices $i$ with $n_i\le n$, which are exactly $i\le k$ by strict increase. Therefore
\[
\mathbf{1}_A(n) = \mathbf{1}_{A^{(k)}}(n)\qquad (n_k\le n < n_{k+1}).\tag{1}
\]

Define the logarithmic counting function
\[
S(x):=\sum_{\substack{1\le n\le x\\ n\in A}}\frac{1}{n}.
\]
Let $K(x)$ be the unique index with $n_{K(x)}\le x < n_{K(x)+1}$.
Then by partitioning into intervals $[n_k, n_{k+1})$ and using (1), we have
\[
S(x) = \sum_{k=1}^{K(x)-1} \sum_{n=n_k}^{n_{k+1}-1} \frac{\mathbf{1}_{A^{(k)}}(n)}{n}
\ +\ \sum_{n=n_{K(x)}}^{\lfloor x\rfloor} \frac{\mathbf{1}_{A^{(K(x))}}(n)}{n}.
\tag{2}
\]

Now we approximate each inner sum by $\delta_k\log(n_{k+1}/n_k)$.
Because $A^{(k)}$ is periodic with some period $L_k$, its indicator has average value $\delta_k$ over each complete period. For $n$ in $[n_k,n_{k+1})$, the weight $1/n$ is slowly varying.
A standard summation-by-blocks argument gives
\[
\sum_{n=n_k}^{n_{k+1}-1} \frac{\mathbf{1}_{A^{(k)}}(n)}{n}
= \delta_k\sum_{n=n_k}^{n_{k+1}-1}\frac{1}{n} + O\left(\frac{L_k}{n_k}\right).
\tag{3}
\]
(Reason: split the interval into consecutive blocks of length $L_k$. On each full block, the average of $\mathbf{1}_{A^{(k)}}$ is exactly $\delta_k$, and $1/n$ varies by at most $O(L_k/n_k^2)$ per term, giving total error $O(L_k/n_k)$ over a block; the leftover partial block contributes at most $O(L_k/n_k)$.)

Also,
\[
\sum_{n=n_k}^{n_{k+1}-1}\frac{1}{n}= \log\frac{n_{k+1}}{n_k} + O\left(\frac{1}{n_k}\right).
\tag{4}
\]
Combining (3) and (4) yields
\[
\sum_{n=n_k}^{n_{k+1}-1} \frac{\mathbf{1}_{A^{(k)}}(n)}{n}
= \delta_k\log\frac{n_{k+1}}{n_k} + O\left(\frac{L_k}{n_k}\right)+O\left(\frac{1}{n_k}\right).
\tag{5}
\]
Since $L_k\le n_1\cdots n_k$ may be enormous, we do not attempt to control the $L_k/n_k$ term uniformly; instead we use a crude bound that is sufficient under (H2):
\[
\left|\sum_{n=n_k}^{n_{k+1}-1} \frac{\mathbf{1}_{A^{(k)}}(n)}{n} - \delta_k\log\frac{n_{k+1}}{n_k}\right|
\le \sum_{n=n_k}^{n_{k+1}-1}\frac{1}{n}
\le \log\frac{n_{k+1}}{n_k} + O\left(\frac{1}{n_k}\right).
\tag{6}
\]
Thus (6) gives a uniform \emph{relative} error control: each block contributes at most its full harmonic weight.

Now divide (2) by $\log x$. The total harmonic weight satisfies
\[
\sum_{k=1}^{K(x)-1}\log\frac{n_{k+1}}{n_k} \le \log x + O(1),
\]
because the telescoping product gives $\prod_{k=1}^{K(x)-1} (n_{k+1}/n_k)= n_{K(x)}/n_1 \le x$.
So the weights
\[
 w_k(x):= \frac{\log(n_{k+1}/n_k)}{\log x}
\]
form an approximate probability distribution (nonnegative and summing to $\le 1+o(1)$).

Using (6), we can write
\[
\frac{S(x)}{\log x} = \sum_{k=1}^{K(x)-1} w_k(x)\,\delta_k\ +\ R(x),
\tag{7}
\]
where the remainder $R(x)$ is bounded in absolute value by
\[
|R(x)|\le \frac{\max_{1\le k\le K(x)} \log(n_{k+1}/n_k) + O(1)}{\log x}.
\tag{8}
\]
The $O(1)$ term accounts for the final partial block and the $O(1)$ discrepancy between $\sum_{n\le x}1/n$ and $\log x$.

By assumption (H2) with $k=K(x)$, the right-hand side of (8) tends to $0$ as $x\to\infty$.

Finally, since $\delta_k\to\delta$ (assumption (H1)) and the weights $w_k(x)$ have maximum weight $\max w_k(x)=\max\log(n_{k+1}/n_k)/\log x\to 0$, the weighted average in (7) converges to $\delta$.

Formally: for any $\varepsilon>0$, choose $K$ such that $|\delta_k-\delta|<\varepsilon$ for all $k\ge K$. Then split the weighted sum in (7) at $K$:
\[
\sum_{k=1}^{K(x)-1} w_k(x)\,\delta_k = \sum_{k< K} w_k(x)\,\delta_k + \sum_{k\ge K} w_k(x)\,\delta_k.
\]
The tail satisfies
\[
\left|\sum_{k\ge K} w_k(x)\,\delta_k - \delta\sum_{k\ge K} w_k(x)\right|\le \varepsilon\sum_{k\ge K} w_k(x)\le \varepsilon(1+o(1)).
\]
The initial finite part has total weight $\sum_{k< K} w_k(x)\le (\sum_{k< K}\log(n_{k+1}/n_k))/\log x = O(1)/\log x\to 0$, so it vanishes. Hence the weighted sum converges to $\delta$.

Combining with $R(x)\to 0$ proves $S(x)/\log x\to\delta$. Therefore $\delta_{\log}(A)$ exists and equals $\delta$. $\square$

5) VERIFICATION

- Lemma 25.1: periodicity modulo the lcm is correct, and the implication ``natural density $\Rightarrow$ logarithmic density'' holds for periodic sets by comparing harmonic sums over residue classes.
- Lemma 25.2: identity (1) is the key observation (only first $k$ constraints matter on $[n_k,n_{k+1})$). The remainder control uses only the size of the harmonic weight of each interval and the hypothesis (H2) to ensure no single interval dominates the logarithmic weighting.
- The proof does not assume coprimality or independence of residue classes; it packages all dependence into $\delta_k$.

6) FINAL

**UNRESOLVED**

(i) Strongest proved partial result:
- Logarithmic density exists for any finite set of congruence constraints (Lemma 25.1).
- More generally, if the stage densities $\delta_k$ converge and no single interval $[n_k,n_{k+1})$ carries a non-negligible proportion of the total logarithmic weight, then $\delta_{\log}(A)$ exists and equals $\lim\delta_k$ (Lemma 25.2).
- Concrete sanity-check examples show both positive and zero logarithmic densities can occur.

(ii) First gap (crisp):
Remove hypotheses (H1)--(H2): decide whether for \emph{arbitrary} increasing moduli $n_i$ and residue classes $a_i\pmod{n_i}$, the limit
\[
\lim_{x\to\infty} \frac{1}{\log x}\sum_{\substack{n\le x\\n\in A}}\frac{1}{n}
\]
must exist.

(iii) Top 3 next moves (concrete):
1. Attempt to construct an explicit oscillating example by forcing $\delta_k$ to alternate between two separated values while arranging $\log(n_{k+1}/n_k)$ weights to not average out.
2. Prove a general convergence theorem for $\delta_k$ itself (does $\delta_k$ always have a limit? can it oscillate?) by analyzing how adding one new congruence class affects the density of the periodic set $A^{(k)}$.
3. Computational exploration: for designed sequences with slowly growing $n_i$ (e.g. $n_i=i$ or primes), numerically estimate the truncated ratio for large $x$ to see whether oscillations are plausible.

(iv) Minimal counterexample structure:
A minimal counterexample to existence of $\delta_{\log}(A)$ would require two subsequences $x_m\to\infty$ and $y_m\to\infty$ along which the ratios $S(x_m)/\log x_m$ and $S(y_m)/\log y_m$ converge to two different limits. Structurally, this would likely come from designing long stretches where $A$ coincides with two different periodic sets of noticeably different densities, with those stretches weighted comparably in logarithmic measure.
