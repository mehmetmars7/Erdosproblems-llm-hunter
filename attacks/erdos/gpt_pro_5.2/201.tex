\section{Erd\H{o}s Problem \#201: Progression-free subsets inside arbitrary $N$-sets}

\subsection*{1) Formal restatement (quantifiers, definitions, edge cases)}
Fix an integer $k\ge 3$.

\paragraph{Definition ($k$-term arithmetic progression).}
A \emph{nontrivial} $k$-term arithmetic progression (AP) in $\mathbb{Z}$ is a sequence
\[
x,\\ x+r,\\ x+2r,\dots,\\ x+(k-1)r
\]
with $r\ne 0$.

\paragraph{Definition ($R_k(N)$).}
Let $R_k(N)$ be the maximum size of a subset $A\subseteq\\{1,2,\dots,N\\}$ containing no nontrivial $k$-term AP.

\paragraph{Definition ($G_k(N)$).}
For a finite set $B\subset\mathbb{Z}$, let $f_k(B)$ be the maximum size of a subset of $B$ containing no nontrivial $k$-term AP.
Define
\[
G_k(N):=\min_{B\subset\mathbb{Z}\\,:\\ \left\lvert B\right\rvert=N} f_k(B).
\]
Thus $G_k(N)$ is the \emph{guaranteed} size of a largest $k$-AP-free subset inside an arbitrary $N$-element set of integers.

\paragraph{Problem statement.}
Determine the size/order of growth of $G_k(N)$, and in particular decide whether
\[
\lim_{N\to\infty}\frac{R_3(N)}{G_3(N)}=1.
\]

\paragraph{Edge cases and invariances.}
Affine maps $x\mapsto ux+v$ with $u\ne 0$ preserve $k$-APs, hence preserve the quantity $f_k(B)$ up to relabeling.

\subsection*{2) Quick literature/context check}
\begin{itemize}[leftmargin=2em]
\item Riddell introduced and studied $G_k$-type quantities \cite{Riddell69}.
\item Trivially $G_k(N)\le R_k(N)$: take $B$ to be an $N$-term arithmetic progression, an affine copy of $[N]$.
\item A key theorem of Koml\'os--Sulyok--Szemer\'edi states that for fixed $k$,
\[
R_k(N)\ll_k G_k(N),
\]
so $G_k(N)$ and $R_k(N)$ are comparable up to a constant factor depending only on $k$ \cite{KSS75}.
\item More recently, Semchankau \cite{Semchankau20} re-derives (and for $k>3$ improves along a dense subsequence) such constant-factor comparisons using ``compression'' (Freiman-homomorphism) ideas; for example for $k=3$ it records the explicit constant $(2^{-15}+o(1))$ from \cite{KSS75}, and for $k>3$ shows $(1/4+o(1))$ along a dense subsequence of $N$.
\end{itemize}

\subsection*{3) Attack plan}
\begin{enumerate}[leftmargin=2em]
\item Establish the trivial inequality $G_k(N)\le R_k(N)$ rigorously.
\item Record the best known (asymptotic) bounds for $R_k(N)$ (Behrend lower bound, Gowers-type upper bounds), and transfer them to $G_k(N)$ using the constant-factor theorem $R_k(N)\ll_k G_k(N)$.
\item Discuss why the sharper asymptotic $R_3(N)/G_3(N)\to 1$ is substantially stronger than constant-factor comparability and seems open.
\item Give tiny-$N$ examples witnessing $G_3(N)<R_3(N)$ (as in the problem statement).
\end{enumerate}

\subsection*{4) Work (rigorous statements we can prove here)}
\paragraph{Lemma 4.1 (trivial inequality).}
For every $k\ge 3$ and $N\ge 1$,
\[
G_k(N)\le R_k(N).
\]
\begin{proof}
Let $B$ be any arithmetic progression of length $N$ in $\mathbb{Z}$, say
\[
B=\\{x,\\ x+r,\dots,\\ x+(N-1)r\\}
\]
with $r\ne 0$.
By affine invariance of $k$-APs, the maximum size of a $k$-AP-free subset of $B$ equals the maximum size of a $k$-AP-free subset of $[N]$, which is $R_k(N)$ by definition.
Therefore $f_k(B)=R_k(N)$ for this particular $B$, and taking a minimum over all $B$ with $\left\lvert B\right\rvert=N$ yields $G_k(N)\le R_k(N)$.
\end{proof}

\paragraph{Known constant-factor comparison.}
Koml\'os--Sulyok--Szemer\'edi proved that for each fixed $k$ there exists a constant $C_k$ such that
\[
R_k(N)\le C_k\\,G_k(N)\qquad\text{for all }N,
\]
equivalently $G_k(N)\ge c_k R_k(N)$ with $c_k=1/C_k>0$ \cite{KSS75}.
We do not reproduce their full proof here; we instead treat it as an external theorem.

\paragraph{Consequences for the order of magnitude of $G_k(N)$.}
Combining Lemma~4.1 and the KSS theorem gives
\[
c_k\\,R_k(N)\\ \le\\ G_k(N)\\ \le\\ R_k(N).
\]
Therefore, \emph{up to multiplicative constants depending only on $k$}, determining $G_k(N)$ is equivalent to determining $R_k(N)$.

In particular, any known upper/lower bounds on $R_k(N)$ transfer to $G_k(N)$:
\begin{itemize}[leftmargin=2em]
\item (Lower bounds.) Behrend-type constructions give, for fixed $k$,
\[
R_k(N)\ge N\exp\big(-C_k\sqrt{\log N}\big)
\]
for some constant $C_k>0$ \cite{Beh46,Semchankau20}. Hence the same form holds for $G_k(N)$ (possibly with a different constant in the exponent due to the factor $c_k$).
\item (Upper bounds.) Quantitative Szemer\'edi/Gowers bounds imply $R_k(N)=o(N)$ and give explicit upper bounds such as
\[
R_k(N)\le \frac{N}{(\log\log N)^{s_k}}
\]
for some $s_k>0$ (for fixed $k$) \cite{Gow01,Semchankau20}. Thus $G_k(N)=o(N)$ as well with comparable explicit bounds.
\end{itemize}

\paragraph{Tiny examples where $G_3(N)<R_3(N)$.}
As recorded in \cite{Semchankau20} (and in the problem statement),
\[
R_3(5)=4\quad\text{(e.g. }\\{1,2,4,5\\}\text{)},\qquad
G_3(5)=3\quad\text{(witness set }\\{1,2,3,4,7\\}\text{)}.
\]
So the interval $[5]$ is \emph{not} the worst case even for very small $N$.

\subsection*{5) Verification / consistency checks}
\begin{itemize}[leftmargin=2em]
\item Lemma~4.1 uses only affine invariance, which is immediate from the definition of AP.
\item The squeeze $c_k R_k(N)\le G_k(N)\le R_k(N)$ is logically correct given the KSS theorem.
\item The transferred bounds are only ``up to constants'' and do not claim sharp exponents.
\end{itemize}

\subsection*{6) Final}
\paragraph{\textbf{UNRESOLVED.}}
\begin{enumerate}[leftmargin=2em]
\item \textbf{Strongest proved partial result included here.}
The trivial inequality $G_k(N)\le R_k(N)$ is proved (Lemma~4.1).
With the external theorem of Koml\'os--Sulyok--Szemer\'edi \cite{KSS75} that $R_k(N)\ll_k G_k(N)$, one obtains
\[
c_k R_k(N)\le G_k(N)\le R_k(N),
\]
so $G_k(N)$ has the same order of magnitude as $R_k(N)$ up to constants depending only on $k$.
\item \textbf{First precise obstacle.}
The question $R_3(N)/G_3(N)\to 1$ asks for \emph{asymptotic equality}, not just a constant-factor comparison.
Known methods (compressions/Freiman-homomorphisms) lose a fixed positive proportion of the set size, and do not currently control this loss down to $o(1)$.
\item \textbf{Most plausible next lemma.}
A plausible intermediate goal is to prove that for every $\varepsilon>0$ and all large $N$,
\[
G_3(N)\ge (1-\varepsilon)R_3(N),
\]
perhaps by showing that every $N$-element set contains a large subset that is Freiman-isomorphic to a reasonably long interval (or to a dense subset of an interval) with only $o(N)$ losses.
Any ``near-lossless'' compression theorem of this type would directly address the ratio limit.
\end{enumerate}

\subsection*{7) Completion estimate}
\[
\textbf{Completion: }45\%.
\]
(We can pin down $G_k$ up to constant factors via known theorems, but the sharper ratio question remains open.)

% ======================================================================

\begin{thebibliography}{99}

\bibitem{Beh46}
F.~A. Behrend,
\emph{On sets of integers which contain no three terms in arithmetic progression},
Proc. Nat. Acad. Sci. U.S.A. \textbf{32} (1946), 331--332.

\bibitem{BloSi20}
T.~F. Bloom and O.~Sisask,
\emph{Breaking the logarithmic barrier in Roth's theorem on arithmetic progressions},
arXiv:2007.03528 (2020).

\bibitem{ESS94}
P.~Erd\H{o}s, A.~S\\'ark\\\"ozy, and T.~S\\'os,
\emph{On sum sets of Sidon sets. I},
J. Number Theory \textbf{47} (1994), 329--347.

\bibitem{ESS95}
P.~Erd\H{o}s, A.~S\\'ark\\\"ozy, and V.~T. S\\'os,
\emph{On sum sets of Sidon sets. II},
Israel J. Math. \textbf{90} (1995), 221--233.

\bibitem{Gow01}
W.~T. Gowers,
\emph{A new proof of Szemer\\'edi's theorem},
Geom. Funct. Anal. \textbf{11} (2001), no.~3, 465--588.

\bibitem{GreenTao08}
B.~Green and T.~Tao,
\emph{The primes contain arbitrarily long arithmetic progressions},
Ann. of Math. \textbf{167} (2008), no.~2, 481--547.

\bibitem{KSS75}
J.~Koml\\'os, M.~Sulyok, and E.~Szemer\\'edi,
\emph{Linear problems in combinatorial number theory},
Acta Math. Acad. Sci. Hungar. \textbf{26} (1975), 113--121.

\bibitem{OBryant04}
K.~O'Bryant,
\emph{A complete annotated bibliography of work related to Sidon sequences},
Electron. J. Combin. \textbf{11} (2004), \#DS11.
(Also available as arXiv:math/0407117.)

\bibitem{PrimePagesAP}
C.~Caldwell and others,
\emph{Primes in arithmetic progression},
The Prime Pages (accessed 2026-01-17).
\url{https://t5k.org/top20/page.php?page=aplist.html}

\bibitem{Riddell69}
J.~Riddell,
\emph{Sets of integers containing no $n$ terms in arithmetic progression},
Nieuw Archief voor Wiskunde (3) \textbf{17} (1969), 204--209.

\bibitem{Semchankau20}
S.~Semchankau,
\emph{Maximal subsets free of arithmetic progressions in arbitrary sets},
arXiv:2010.04490 (2020).

\bibitem{TV06}
T.~Tao and V.~Vu,
\emph{Additive Combinatorics},
Cambridge Studies in Advanced Mathematics, vol.~105, Cambridge Univ. Press, 2006.

\end{thebibliography}
