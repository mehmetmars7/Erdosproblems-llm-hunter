\section*{Problem 788}
\addcontentsline{toc}{section}{Problem 788}

\subsection*{1. Formal restatement}
For each integer $n\ge 2$ define $f(n)$ to be the largest integer $f$ such that:
for every set $B\subset (2n,4n)\cap\mathbb{Z}$, there exists a set $C\subset (n,2n)\cap\mathbb{Z}$ with
\begin{enumerate}[leftmargin=2em]
\item $c_1+c_2\notin B$ for all distinct $c_1,c_2\in C$,
\item and $|B|+|C|\ge f$.
\end{enumerate}
Estimate $f(n)$.

\subsection*{2. Quick literature/context check}
The statement records (as context) the following.
\begin{itemize}[leftmargin=2em]
\item (Adenwalla) $f(n)\gg n^{1/2}$.
\item (Baltz--Schoen--Srivastav) $f(n)\ll (n\log n)^{2/3}$.
\item Conjecture (Choi): $f(n)\le n^{1/2+o(1)}$.
\item Heuristic: random $B$ gives $f(n)=O(n^{2/3+o(1)})$.
\end{itemize}

\subsection*{3. Attack plan}
\begin{enumerate}[leftmargin=2em]
\item Translate the condition on $C$ into an independent-set problem in a graph on the vertex set $(n,2n)\cap\mathbb{Z}$ where edges correspond to forbidden sums lying in $B$.
\item Use degree bounds to produce general lower bounds on the maximum independent set in terms of $|B|$.
\item For upper bounds (not attempted here), one seeks explicit $B$ such that every large $C$ necessarily creates some sum in $B$; constructions often use Sidon sets or probabilistic methods.
\item Compute exact values for small $n$ by brute force over all $B$.
\end{enumerate}

\subsection*{4. Work}
\subsubsection*{4.1. A general lower bound of order $\sqrt n$}
Let $V\coloneqq (n,2n)\cap\mathbb{Z}=\{n+1,\dots,2n-1\}$, so $|V|=n-1$.
Given $B\subset (2n,4n)\cap\mathbb{Z}$, define a graph $G_B$ on vertex set $V$ by joining distinct $x,y\in V$ if $x+y\in B$.
Then a set $C\subset V$ satisfies ``$c_1+c_2\notin B$ for all distinct $c_1,c_2\in C$'' if and only if $C$ is an independent set in $G_B$.
Thus the best possible $C$ has size $\alpha(G_B)$.

\begin{lemma}[Maximum degree bound]
For every $B$, the graph $G_B$ has maximum degree at most $|B|$.
\end{lemma}

\begin{proof}
Fix $x\in V$. Each neighbor $y$ of $x$ must satisfy $x+y\in B$, i.e.
$y=b-x$ for some $b\in B$. For a fixed $b$ there is at most one such $y$.
Hence $\deg(x)\le |B|$.
\end{proof}

\begin{corollary}[Independent set lower bound]
\label{cor:independent-lower}
For every $B$ there exists $C\subset (n,2n)\cap\mathbb{Z}$ with $c_1+c_2\notin B$ for all distinct $c_1,c_2\in C$ and
\[
|C|\ge \frac{n-1}{|B|+1}.
\]
\end{corollary}

\begin{proof}
A greedy algorithm for independent sets gives $\alpha(G)\ge |V|/(\Delta+1)$ for any graph with maximum degree $\Delta$.
Here $|V|=n-1$ and $\Delta\le |B|$.
\end{proof}

Now minimize the guaranteed sum $|B|+|C|$ over $m\coloneqq |B|$.

\begin{proposition}[A clean $\boldsymbol{\sqrt n}$ lower bound for $f(n)$]
For all $n\ge 2$,
\[
 f(n)\ge \left\lceil 2\sqrt{n-1}-1\right\rceil.
\]
\end{proposition}

\begin{proof}
Let $m\coloneqq |B|$. By Corollary~\ref{cor:independent-lower} we can choose $C$ with
$|C|\ge (n-1)/(m+1)$, hence
\[
|B|+|C|\ge m+\frac{n-1}{m+1}.
\]
By AM--GM,
\[
(m+1)+\frac{n-1}{m+1}\ge 2\sqrt{n-1}.
\]
Subtracting $1$ gives $m+\frac{n-1}{m+1}\ge 2\sqrt{n-1}-1$.
Since this holds for every $B$, it holds for the minimum over $B$, i.e.
$f(n)\ge 2\sqrt{n-1}-1$. Finally, $f(n)$ is an integer.
\end{proof}

\subsubsection*{4.2. Exact values for small $n$ (brute force)}
A brute-force search over all subsets $B\subset (2n,4n)\cap\mathbb{Z}$ for $n\le 8$ yields:
\[
\begin{array}{c|ccccccc}
 n&2&3&4&5&6&7&8\\\hline
 f(n)&1&2&3&3&4&4&5
\end{array}
\]
For example, for $n=5$ one extremal choice is $B=\{15\}\subset\{11,\dots,19\}$, which forces any admissible $C\subset\{6,7,8,9\}$ to have size at most $2$, giving $|B|+|C|\le 3$.

\subsection*{5. Verification}
\begin{itemize}[leftmargin=2em]
\item The reduction to an independent-set problem is exact by definition.
\item The degree bound is immediate and makes the $\sqrt n$ lower bound completely elementary.
\item The small-$n$ brute force agrees with the lower bound (often sharply at these sizes), but is not an asymptotic argument.
\end{itemize}

\subsection*{6. FINAL}
\textbf{UNRESOLVED.}

\medskip
\noindent\textbf{Fail-safe (required):}
\begin{enumerate}[leftmargin=2em,label=(\roman*)]
\item \textbf{Strongest partial results proved here.}
A fully elementary bound $f(n)\ge \lceil 2\sqrt{n-1}-1\rceil$.
\item \textbf{First gap that blocks a full solution.}
We do not supply an upper bound of order $n^{1/2+o(1)}$ or $n^{2/3+o(1)}$; in particular we do not reproduce the Baltz--Schoen--Srivastav construction.
\item \textbf{What would close the problem.}
A matching upper bound $f(n)\le n^{1/2+o(1)}$ (or proof of a different asymptotic) would resolve the growth rate; sharpening the exponent between $1/2$ and $2/3$ is the main obstacle.
\item \textbf{Why computations didn't settle it.}
Exact computation is feasible only for tiny $n$ (here $n\le 8$); asymptotics require new ideas.
\end{enumerate}

\subsection*{7. COMPLETION ESTIMATE}
\textbf{40\%.} A clean $\sqrt n$ lower bound is proved and small-$n$ values are computed, but the conjectured asymptotics remain open.

