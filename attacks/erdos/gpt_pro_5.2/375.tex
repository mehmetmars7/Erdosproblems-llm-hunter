% Erdos Problem #375
% Attempt for Erdos Problem #375
% Following PROMPT_STRATEGY.MD
% Tools/Constraints:
% - Web browsing available? YES
% - Computation available (Python/Sage/Mathematica)? YES

## 1) FORMAL RESTATEMENT

Let \(n,k\ge 1\). Assume \(n+1,n+2,\dots,n+k\) are all composite.

**Grimm’s conjecture (as stated):** there exist **distinct primes** \(p_1,\dots,p_k\) such that
\[
 p_i\mid (n+i)\qquad (1\le i\le k).
\]

Equivalently: we can choose, for each of the \(k\) consecutive composite integers, a prime divisor
so that no prime is chosen twice.

---

## 2) QUICK LITERATURE/CONTEXT CHECK

I only record what the provided problem text states:

* The statement is trivial for \(k\le 2\).
* Grimm proved it for \(k\ll \log n/\log\log n\).
* Erd\H{o}s and Selfridge improved to \(k\le (1+o(1))\log n\).
* Ramachandra–Shorey–Tijdeman improved to
  \(k\ll (\log n/\log\log n)^3\).

---

## 3) ATTACK PLAN

**Proof track (partial results).**
1. Prove Grimm’s conjecture for small \(k\) beyond the stated “trivial” range.
2. Reformulate the condition as a matching/assignment problem on prime divisors.

**Disproof track.**
Search computationally for a counterexample in small ranges for modest \(k\).

---

## 4) WORK

### Phase 1: FAST REALITY CHECK (computation)

For each \(3\le k\le 8\), I checked all \(n\le 200000\) with \(n+1,\dots,n+k\) composite and verified
(by backtracking over prime factors) that a choice of distinct primes exists in every case.

Also for \(k=4\) I checked all \(n\le 10^6\) with \(n+1,\dots,n+4\) composite and found no
counterexample.

(These are sanity checks only; no general conclusion from computation alone.)

---

### Lemma 4.1 (Grimm’s conjecture holds for \(k\le 3\))
Assume \(n+1,\dots,n+k\) are composite.

* For \(k=1\) the statement is immediate.
* For \(k=2\) the statement is immediate.
* For \(k=3\) the statement holds for all \(n\).

**Proof.**

\underline{Case \(k=1\).}
Pick any prime divisor \(p_1\mid (n+1)\).

\underline{Case \(k=2\).}
Let \(p_1\) be any prime divisor of \(n+1\) and \(p_2\) any prime divisor of \(n+2\).
Since \(\gcd(n+1,n+2)=1\), no prime divides both, so \(p_1\ne p_2\).

\underline{Case \(k=3\).}
Let \(N_1=n+1\), \(N_2=n+2\), \(N_3=n+3\). All are composite by assumption.
Pick any prime divisor \(p_2\mid N_2\).
Since \(\gcd(N_1,N_2)=\gcd(N_2,N_3)=1\), any primes \(p_1\mid N_1\) and \(p_3\mid N_3\)
will automatically satisfy \(p_1\ne p_2\) and \(p_3\ne p_2\). So the only potential collision is
\(p_1=p_3\).

But \(\gcd(N_1,N_3)=\gcd(n+1,n+3)=\gcd(n+1,2)\in\{1,2\}\).
So \(N_1\) and \(N_3\) share no odd prime factor; the only possible common prime is 2.

If \(n\) is even, then \(N_1,N_3\) are odd, so they share no prime at all; choose any
\(p_1\mid N_1\), \(p_3\mid N_3\), and all three primes are distinct.

If \(n\) is odd, then \(N_1,N_3\) are even, so they both have the prime 2.
If both \(N_1\) and \(N_3\) were powers of 2, then two powers of 2 would differ by 2.
The only powers of 2 differing by 2 are \(2\) and \(4\), but \(2\) is not composite, contradicting
that \(N_1,N_3\) are composite. Hence at least one of \(N_1,N_3\) has an odd prime divisor.

Choose \(p_1\) to be an odd prime divisor of whichever of \(N_1\) or \(N_3\) has one, and choose
\(p_3=2\) for the other even number. Then \(p_1\ne p_3\), and as noted \(p_2\) is distinct from both.

Thus Grimm’s condition holds for \(k=3\). \(\square\)

---

### Lemma 4.2 (Matching formulation)
Fix \(n,k\) with \(n+1,\dots,n+k\) composite.
For each \(i\in\{1,\dots,k\}\), let
\[
P_i := \{\text{primes }p : p\mid (n+i)\}
\]
(the set of prime divisors of \(n+i\)).

Then Grimm’s conjecture for this \((n,k)\) is equivalent to the existence of an injection
\(f:\{1,\dots,k\}\to\mathbb P\) such that \(f(i)\in P_i\) for every \(i\).

**Proof.**
* Given distinct primes \(p_1,\dots,p_k\) with \(p_i\mid(n+i)\), define \(f(i)=p_i\); this is an injection
  with \(f(i)\in P_i\).
* Conversely, given an injection \(f\) with \(f(i)\in P_i\), set \(p_i=f(i)\); then \(p_i\mid(n+i)\)
  and the primes are distinct.

So the conditions are identical. \(\square\)

---

## 5) VERIFICATION

* Lemma 4.1 (\(k=3\)): the only subtle point is excluding “both endpoints are powers of 2” when
  \(n\) is odd; the difference-of-powers check is explicit.
* Lemma 4.2 is a direct rephrasing.
* Computation: I only claim “no counterexample up to the stated bounds”; this does not prove the
  conjecture.

---

## 6) FINAL

**UNRESOLVED**

(i) **Strongest fully proved partial result obtained here.**

* A complete proof that Grimm’s conjecture holds for all \(n\) when \(k\le 3\) (Lemma 4.1).
* A clean matching/injection formulation of the conjecture (Lemma 4.2).
* Computational sanity checks: no counterexample found for \(3\le k\le 8\) up to \(n\le 200000\), and
  for \(k=4\) up to \(n\le 10^6\).

(ii) **Exact first gap.**

Prove or disprove Grimm’s conjecture in full generality:
\[
\forall n,k\ge 1:\ \bigl(\forall i\in\{1,\dots,k\}:\ n+i\text{ composite}\bigr)\Rightarrow
\exists\text{ distinct primes }p_1,\dots,p_k\ \forall i:\ p_i\mid(n+i).
\]

(iii) **Top 3 next moves (concrete targets).**

1. Use Lemma 4.2 to apply matching/SDR techniques: prove that for every subset
   \(I\subseteq\{1,\dots,k\}\), the union \(\bigcup_{i\in I} P_i\) has size at least \(|I|\).
   (This is exactly the Hall-type obstruction one must rule out.)
2. Target the first “new” case \(k=4\) with a full proof (beyond computation), exploiting that
   \(\gcd(n+i,n+j)\) is very small when \(|i-j|\in\{1,2,3\}\).
3. Try to strengthen known partial results by bounding how many of the \(n+i\) can share a given
   prime divisor, and controlling the total number of distinct primes dividing the product
   \(\prod_{i=1}^k(n+i)\).

(iv) **What a minimal counterexample would likely look like.**

A minimal counterexample would be a block of consecutive composites \(n+1,\dots,n+k\) whose prime
factor sets \(P_i\) have a “bottleneck”: there exists a subset \(I\) of indices such that
\(\bigcup_{i\in I} P_i\) contains fewer than \(|I|\) primes, forcing any assignment to reuse a prime.
In such a configuration, many of the \(n+i\) would have to share a very small set of primes (despite
being mostly coprime in adjacent pairs), so any genuine counterexample is expected to have a highly
structured pattern of repeated small prime divisors across the block.
