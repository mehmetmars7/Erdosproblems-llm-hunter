\section*{Erd\H{o}s Problem \#313}

\subsection*{1) RESTATE}
Determine whether there are infinitely many tuples of distinct primes $p_1<\cdots<p_k$ and an integer $m\ge 2$ such that
\[
\frac{1}{p_1}+\cdots+\frac{1}{p_k}=1-\frac{1}{m}.
\]
Equivalently, are there infinitely many integers $m$ for which
\[
\frac{1}{m}+\sum_{p\mid m}\frac{1}{p}=1,
\]
with the sum over prime divisors $p$ of $m$?

\subsection*{2) KNOWN FACTS}
\begin{enumerate}[label=(\alph*)]
\item \textbf{(Denominator rigidity)} If $p_1,\dots,p_k$ are distinct primes then the rational number $\sum_{i=1}^k \frac{1}{p_i}$ is in lowest terms with denominator $p_1\cdots p_k$.

\item \textbf{(For a solution, $m$ is forced)} In any solution of the original equation one must have $m=p_1\cdots p_k$. Consequently there is at most one solution for each $m$.

\item \textbf{(Name of the $m$'s)} The integers $m$ satisfying $\frac{1}{m}+\sum_{p\mid m}\frac{1}{p}=1$ are called \emph{primary pseudoperfect numbers}.

\item \textbf{(Computational status)} As of January 2026, $8$ primary pseudoperfect numbers are known:
\[
2,\ 6,\ 42,\ 1806,\ 47058,\ 2214502422,\ 52495396602,\ 8490421583559688410706771261086.
\]
It is unknown whether there are infinitely many, and it is unknown whether any are odd.

\item \textbf{(A simple ``chain'' lemma)} If $m$ is primary pseudoperfect and $m+1$ is prime, then $m(m+1)$ is also primary pseudoperfect.
\end{enumerate}

\subsection*{3) PROOF STRATEGY}
A natural strategy to prove infinitude is to find a mechanism that generates new solutions from old ones.
One such mechanism is the chain lemma (2e): if $m$ is a solution and $m+1$ is prime, then $m(m+1)$ is a new solution.
Thus, an infinite chain of solutions would follow from proving that the ``Euclid-like'' numbers $m+1$ are prime infinitely often along this recursion. No method is known.

\subsection*{4) ATTEMPTED PROOF}
\paragraph{Step 1: Force $m=p_1\cdots p_k$.}
Let $D:=p_1\cdots p_k$.
Write
\[
\sum_{i=1}^k\frac{1}{p_i}=\frac{A}{D},\qquad A:=\sum_{i=1}^k \frac{D}{p_i}.
\]
Fix $j\in\{1,\dots,k\}$. Modulo $p_j$ we have
\[
A \equiv \frac{D}{p_j}\pmod{p_j},
\]
because every term $D/p_i$ with $i\neq j$ is divisible by $p_j$.
Since $D/p_j=\prod_{i\neq j}p_i$ is not divisible by $p_j$, we get $A\not\equiv 0\pmod{p_j}$.
Therefore $\gcd(A,D)=1$ and $\sum_{i=1}^k\frac{1}{p_i}$ is in lowest terms with denominator exactly $D$.

On the other hand,
\[
1-\frac{1}{m}=\frac{m-1}{m}
\]
is in lowest terms because $\gcd(m-1,m)=1$. Hence its reduced denominator is $m$.
Equality of the two fractions forces $m=D=p_1\cdots p_k$.

\paragraph{Step 2: Reformulate as the primary pseudoperfect equation.}
Substituting $m=p_1\cdots p_k$ gives
\[
\sum_{p\mid m}\frac{1}{p}=1-\frac{1}{m}
\qquad\Longleftrightarrow\qquad
\frac{1}{m}+\sum_{p\mid m}\frac{1}{p}=1.
\]

\paragraph{Step 3: Prove the chain lemma (2e).}
Assume $m$ is primary pseudoperfect, so $\sum_{p\mid m} \frac{1}{p}=1-\frac{1}{m}$.
Let $q:=m+1$ be prime. Then the prime divisors of $mq$ are those of $m$ together with $q$.
Compute
\begin{align*}
\frac{1}{mq}+\sum_{p\mid mq}\frac{1}{p}
&=\frac{1}{mq}+\left(\sum_{p\mid m}\frac{1}{p}\right)+\frac{1}{q}
=\frac{1}{mq}+\left(1-\frac{1}{m}\right)+\frac{1}{q}\\
&=1-\frac{1}{m}+\frac{1}{q}\left(1+\frac{1}{m}\right)
=1-\frac{1}{m}+\frac{m+1}{m(m+1)}
=1.
\end{align*}
So $mq$ is primary pseudoperfect.

\subsection*{5) OBSTACLES}
\begin{enumerate}[label=(\alph*)]
\item The chain lemma reduces infinitude to a primality question for a very sparse sequence of ``Euclid-like'' integers; no known sieve/analytic method applies.
\item Known solutions are extremely sparse and grow rapidly; currently only $8$ are known (as of Jan 2026).
\item Even basic structural questions are open (e.g. existence of any odd primary pseudoperfect number).
\end{enumerate}

\subsection*{6) FINAL}
\textbf{UNRESOLVED.}
\begin{enumerate}[label=(\roman*)]
\item \textbf{What I tried:} Proved the structural reduction $m=p_1\cdots p_k$ and recast the problem as infinitude of primary pseudoperfect numbers; recorded a standard ``chain'' construction $m\mapsto m(m+1)$ when $m+1$ is prime.
\item \textbf{Where it fails / stuck:} Infinitude would follow from producing infinitely many primes of the form $m+1$ along such constructions, or from an entirely different infinite-family construction. No known argument provides this.
\item \textbf{What might work next:} (a) Search for additional recursive constructions beyond the $m+1$ prime lemma; (b) study congruence restrictions/modular constraints from the equation to narrow candidates; (c) use graph-theoretic/structural formulations from the literature on pseudoperfect numbers.
\item \textbf{Confidence:} High confidence in the reductions and lemmas proved here; low confidence about any claim toward infinitude (open problem).
\end{enumerate}

\subsection*{7) WRITE-UP (clean, complete)}
The equation
\(\sum_{i=1}^k \frac{1}{p_i}=1-\frac{1}{m}\)
forces $m=p_1\cdots p_k$ because the left-hand side is a reduced fraction with denominator $p_1\cdots p_k$ while the right-hand side is reduced with denominator $m$.
Thus solutions correspond exactly to \emph{primary pseudoperfect numbers} $m$ satisfying
\(\frac{1}{m}+\sum_{p\mid m} \frac{1}{p}=1\).
Only finitely many such $m$ are currently known (eight as of Jan 2026), and it is unknown whether there are infinitely many.
A simple construction is: if $m$ is primary pseudoperfect and $m+1$ is prime then $m(m+1)$ is primary pseudoperfect.
This yields some ``chains'' of solutions, but proving infinitely many steps in any chain would require infinitely many primes of the form $m+1$ along that chain, which is out of reach.

\hrule

