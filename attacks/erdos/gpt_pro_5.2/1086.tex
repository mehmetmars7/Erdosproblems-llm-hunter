
\subsection*{Erd\H{o}s Problem 1086 (many equal-area triangles)}

\subsubsection*{FORMAL RESTATEMENT}
For a finite set \(P\subset\mathbb{R}^2\) with \(|P|=n\), and a real number \(A>0\), let
\(T_A(P)\) be the set of unordered triples \(\{p,q,r\}\subset P\) with (nonzero) triangle area \(\mathrm{Area}(pqr)=A\).
Define
\[
 g(n):=\max\{\,|T_A(P)|: P\subset\mathbb{R}^2,\ |P|=n,\ A>0\,\}.
\]
Equivalently (by scaling), \(g(n)\) is the maximum possible number of unit-area triangles determined by \(n\) points in the plane.
The problem asks for the asymptotic growth of \(g(n)\).

\subsubsection*{QUICK LITERATURE/CONTEXT CHECK}
The problem statement records that Erd\H{o}s--Purdy proved
\(n^2\log\log n \ll g(n) \ll n^{5/2}\), and that the best known upper bound recorded there is
\(g(n)\ll n^{20/9}\). I do not use further literature.

\subsubsection*{ATTACK PLAN}
\begin{itemize}
\item \textbf{Proof track (partial).} Prove two basic structural lemmas:
(1) for any fixed base segment \(pq\), unit-area third vertices must lie on two parallel lines;
(2) if the point set has bounded collinearity, this yields an \(O(n^2)\) (or \(O(Ln^2)\)) upper bound.
Then give an explicit construction with \(\Omega(n^2)\) unit-area triangles.
\item \textbf{Disproof track.} Not applicable.
\end{itemize}

\subsubsection*{WORK}

\paragraph{Fast reality check.} For the explicit two-line construction below, brute-force counts of unit-area triangles (unordered triples) give:
\(n=6\mapsto 9\), \(n=8\mapsto 20\), \(n=10\mapsto 35\). (Computed exactly.)

\begin{lemma}[Third vertex lies on two parallel lines]
Fix distinct points \(p,q\in\mathbb{R}^2\) and an area \(A>0\). The set of points \(r\in\mathbb{R}^2\) such that \(\mathrm{Area}(pqr)=A\) is the union of two lines parallel to the line \(pq\), each at perpendicular distance \(\frac{2A}{\|p-q\|}\) from \(pq\) (one on each side).
\end{lemma}

\begin{proof}
Let \(\ell\) be the line through \(p\) and \(q\). For any point \(r\),
\(\mathrm{Area}(pqr)=\frac12\|p-q\|\cdot \mathrm{dist}(r,\ell)\), where \(\mathrm{dist}(r,\ell)\) is the perpendicular distance from \(r\) to \(\ell\).
Thus \(\mathrm{Area}(pqr)=A\) iff \(\mathrm{dist}(r,\ell)=\frac{2A}{\|p-q\|}\).
The locus of points at a fixed distance \(h\) from a line is the union of the two lines parallel to it at distances \(h\) on either side. Taking \(h=\frac{2A}{\|p-q\|}\) gives the claim.
\end{proof}

\begin{lemma}[Bound under limited collinearity]
Let \(P\subset\mathbb{R}^2\) be a set of \(n\) points and suppose no line contains more than \(L\) points of \(P\).
Then for any \(A>0\), the number of triangles of area exactly \(A\) determined by \(P\) satisfies
\[
|T_A(P)|\le 2L\binom{n}{2}.
\]
In particular, if \(P\) is in general position (no three collinear, so \(L=2\)), then \(|T_A(P)|\le 4\binom{n}{2}=2n(n-1)\).
\end{lemma}

\begin{proof}
Count triangles by choosing their base \(\{p,q\}\subset P\).
For fixed \(p\ne q\), by the previous lemma, any point \(r\in P\) with \(\mathrm{Area}(pqr)=A\) must lie on one of two lines parallel to \(pq\). Each such line contains at most \(L\) points of \(P\) by hypothesis, so for the base \(\{p,q\}\) there are at most \(2L\) valid choices of \(r\).
Summing over all \(\binom{n}{2}\) choices of base gives \(|T_A(P)|\le 2L\binom{n}{2}\).

For general position, set \(L=2\).
\end{proof}

\begin{lemma}[Two-line construction with \(\Omega(n^2)\) unit-area triangles]
For each integer \(m\ge 3\), let
\(P\) consist of \(m\) points on the line \(y=0\) at \((0,0),(1,0),\dots,(m-1,0)\), and \(m\) points on the line \(y=1\) at \((0,1),(1,1),\dots,(m-1,1)\).
Then \(|P|=n=2m\), and \(P\) determines at least \(2m(m-2)\) triangles of area exactly \(1\).
Consequently, \(g(n)\ge \frac12 n^2 - O(n)\).
\end{lemma}

\begin{proof}
Consider triangles with base on \(y=0\) and third vertex on \(y=1\).
If \(p=(i,0)\) and \(q=(i+2,0)\), then \(\|p-q\|=2\) and the height from any \(r=(j,1)\) to the base line is \(1\), so
\(\mathrm{Area}(pqr)=\frac12\cdot 2\cdot 1=1\), provided \(r\) is not collinear with \(p,q\) (which cannot happen since \(r\) lies on a different horizontal line).
There are \(m-2\) choices of the base pair \((i,0),(i+2,0)\) with \(0\le i\le m-3\), and \(m\) choices of \(r\) on \(y=1\), giving \(m(m-2)\) unit-area triangles.
By symmetry, using bases on \(y=1\) and third vertices on \(y=0\) gives another \(m(m-2)\) unit-area triangles, for a total of at least \(2m(m-2)\).
In terms of \(n=2m\), this is \(\frac12 n^2-2n\).
\end{proof}

\subsubsection*{VERIFICATION}
\begin{itemize}
\item Lemma 1 is exact and uses only the elementary base\(\times\)height formula.
\item Lemma 2 is conditional on the maximum number \(L\) of collinear points; the conclusion specializes cleanly to general position.
\item For the two-line construction, brute-force enumeration confirms counts exceed the proven lower bound:
for \(n=6,8,10\) the exact numbers of unit-area triangles are \(9,20,35\) respectively (vs. the lower bounds \(6,16,30\)).
\end{itemize}

\subsubsection*{FINAL}
\textbf{UNRESOLVED.}
\begin{enumerate}
\item[(i)] \textbf{Strongest proved partial result here:} unconditional construction yields \(g(n)\ge \frac12 n^2-O(n)\); additionally, under the hypothesis ``no line contains more than \(L\) points'' one has \(g(n)\le 2L\binom{n}{2}\) (in particular \(O(n^2)\) in general position).
\item[(ii)] \textbf{First gap:} obtain a nontrivial unconditional upper bound better than \(O(n^3)\) without importing deep incidence machinery beyond what is stated in the problem file.
\item[(iii)] \textbf{Top 3 next moves:}
(1) analyze contributions from ``heavy'' lines (with many points) and show triangles of fixed area force many incidences between points and parallel lines;
(2) attempt a double-counting scheme that reduces the problem to bounding incidences between points and a family of lines determined by pairs;
(3) computationally search for configurations that beat the two-line construction substantially to understand extremizers.
\item[(iv)] \textbf{Minimal counterexample structure (to the heuristic that \(g(n)\) is near \(n^2\) up to polylogs):}
point sets with many points on many parallel lines, arranged so that for many bases \(pq\), both parallel offset lines at the required height each contain \(\Theta(n)\) points.
\end{enumerate}


