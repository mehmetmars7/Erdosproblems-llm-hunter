\providecommand{\qed}{\hfill\textit{(end of proof)}}


Is there a constant $c_t$, where $c_t\to \infty$ as $t\to \infty$, such that if $\mathcal{F}$ is a finite family of finite sets, all of size at least $t$, and for every set $X$ there are $<c_t\lvert X\rvert$ many $A\in \mathcal{F}$ with $A\subseteq X$, then $\mathcal{F}$ has chromatic number $2$ (in other words, has property B)? Erd\H{o}s originally conjectured, in this language, that $c_2=1$, which was proved by Lov\'{a}sz \cite{Lo68}. References [Lo68] Lov\'{a}sz, L., On covering of graphs . Theory of Graphs (Proc. Colloq., Tihany, 1966) (1968), 231-236.

%Erdos problem #1022
\subsection*{Erd\H{o}s Problem \#1022 --- Solution Attempt}

\textbf{FORMAL RESTATEMENT.}
Fix an integer $t\ge 2$. Let $\mathcal{F}$ be a finite family of finite sets (``edges'') and let
\[
V := \bigcup_{A\in\mathcal{F}} A
\]
be the finite ``vertex set''. Assume $|A|\ge t$ for every $A\in\mathcal{F}$. For any subset $X\subseteq V$, define
\[
\mathcal{F}[X] := \{A\in\mathcal{F}: A\subseteq X\},\qquad e(X):=|\mathcal{F}[X]|.
\]
The hypothesis is that for every $X\subseteq V$ we have
\[
 e(X) < c_t\,|X|.
\]
The conclusion ``$\mathcal{F}$ has chromatic number $2$ / has property B'' means: there exists a map
$\chi:V\to\{\text{red},\text{blue}\}$ such that for every $A\in\mathcal{F}$, the restriction $\chi|_A$ is not constant.

The question: do there exist constants $c_t$ (depending only on $t$) with $c_t\to\infty$ as $t\to\infty$ such that the above implication holds for every such finite $\mathcal{F}$?

\textbf{QUICK LITERATURE/CONTEXT CHECK.}
The problem statement itself records that for $t=2$ one can take $c_2=1$ (Lov\'{a}sz \cite{Lo68}).
I do not use any external results beyond what is explicitly in the problem file.

\textbf{ATTACK PLAN.}
\emph{Proof track:}
try to deduce 2-colourability from the local bound $e(X)<c_t|X|$ via an inductive deletion argument
(looking for a low ``degree'' vertex), or via a probabilistic colouring + Local Lemma style condition
(but any such argument must exploit the specific ``subsets $X$'' constraint).

\emph{Disproof track:}
search for explicit non-2-colourable $t$-uniform (or $\ge t$-uniform) hypergraphs where every induced subhypergraph
has at most $O(|X|)$ edges, with the constant of proportionality as large as possible.
Projective-plane type designs are natural candidates.

I follow the disproof/proof tracks only far enough to obtain rigorous partial results and explicit obstructions.

\textbf{WORK.}

\textbf{Lemma 1 (A $c_t=1$ inductive sufficient condition for property B).}
Let $\mathcal{F}$ be a finite family of subsets of a finite set $V$, with $|A|\ge 2$ for all $A\in\mathcal{F}$.
Assume that for every nonempty $X\subseteq V$,
\[
|\{A\in\mathcal{F}: A\subseteq X\}| \le |X|-1.
\]
Then $\mathcal{F}$ has property B.

\emph{Proof.}
We prove by induction on $n:=|V|$.

Base cases: $n\le 1$ are trivial because there are no edges of size $\ge 2$.

Inductive step: assume $n\ge 2$ and the statement holds for all smaller $|V|$.
For each vertex $v\in V$, define its hyperedge-degree
\[
 d(v):=|\{A\in\mathcal{F}: v\in A\}|.
\]
Suppose for contradiction that $d(v)\ge 2$ for every $v\in V$. Then
\[
\sum_{v\in V} d(v) \ge 2|V|=2n.
\]
On the other hand,
\[
\sum_{v\in V} d(v) = \sum_{A\in\mathcal{F}} |A| \ge 2|\mathcal{F}|,
\]
because each edge has size at least $2$. Combining gives $2|\mathcal{F}|\ge 2n$, i.e. $|\mathcal{F}|\ge n$.
But applying the hypothesis to $X=V$ yields $|\mathcal{F}|\le |V|-1=n-1$, a contradiction.
Therefore there exists a vertex $v\in V$ with $d(v)\le 1$.

Let $V':=V\setminus\{v\}$ and let $\mathcal{F}':=\{A\in\mathcal{F}: v\notin A\}$.
For any nonempty $X\subseteq V'$, the sets in $\mathcal{F}'$ contained in $X$ are exactly the sets in $\mathcal{F}$
contained in $X$, hence
\[
|\{A\in\mathcal{F}':A\subseteq X\}| \le |X|-1.
\]
Thus $\mathcal{F}'$ satisfies the same hypothesis on the smaller vertex set $V'$. By the induction hypothesis,
there exists a 2-colouring $\chi':V'\to\{\text{red},\text{blue}\}$ such that no edge of $\mathcal{F}'$ is monochromatic.

We extend $\chi'$ to $V$.
If $d(v)=0$, then $v$ lies in no edge, so assigning either colour to $v$ preserves property B.
If $d(v)=1$, let $A\in\mathcal{F}$ be the unique edge containing $v$. Then all other edges are in $\mathcal{F}'$ and are already non-monochromatic.
If $\chi'$ is not constant on $A\setminus\{v\}$, again either colour on $v$ makes $A$ non-monochromatic.
If instead all vertices of $A\setminus\{v\}$ have the same colour under $\chi'$, assign $v$ the opposite colour.
Then $A$ is not monochromatic.
In all cases we obtain a 2-colouring of $V$ witnessing property B for $\mathcal{F}$.
\qed

\medskip
\textbf{Comment.} Lemma 1 gives a complete proof that the implication in the problem holds with $c_t=1$ for every $t\ge 2$.
However, it does \emph{not} address the requirement $c_t\to\infty$.

\medskip
\textbf{Lemma 2 (A concrete obstruction at $t=3$: the Fano plane forces $c_3\le 1$).}
Let $V=\{1,2,3,4,5,6,7\}$ and let $\mathcal{F}$ be the $7$ triples
\[
\{1,2,3\},\ \{1,4,5\},\ \{1,6,7\},\ \{2,4,6\},\ \{2,5,7\},\ \{3,4,7\},\ \{3,5,6\}.
\]
Then:
\begin{enumerate}
\item[(a)] For every subset $X\subseteq V$, one has $|\{A\in\mathcal{F}:A\subseteq X\}|\le |X|$.
\item[(b)] $\mathcal{F}$ does \emph{not} have property B (i.e. it is not 2-colourable).
\end{enumerate}
Consequently, for $t=3$ the statement of the problem cannot hold for any constant $c_3>1$.

\emph{Proof of (a).}
For $v\in V$, let $d(v)$ be the number of triples in $\mathcal{F}$ containing $v$.
From the explicit list, each vertex appears in exactly $3$ triples (one checks this by inspection of the seven triples).
Fix any $X\subseteq V$ and let $e(X)$ denote the number of triples contained in $X$.
Counting incidences between vertices of $X$ and triples contained in $X$ gives
\[
3e(X) = \sum_{A\in\mathcal{F}[X]} |A| = \sum_{A\in\mathcal{F}[X]} 3.
\]
On the other hand, each $v\in X$ belongs to at most $3$ triples in total, hence also to at most $3$ triples contained in $X$,
so the number of such incidences is at most $\sum_{v\in X} 3 = 3|X|$.
Therefore $3e(X)\le 3|X|$, i.e. $e(X)\le |X|$.

\emph{Proof of (b).}
Assume for contradiction that there is a 2-colouring of $V$ into red and blue such that no triple in $\mathcal{F}$ is monochromatic.
Let $r$ be the number of red vertices (so $0\le r\le 7$).
Under the assumption ``no monochromatic triple'', every triple has type $(2\text{ red},1\text{ blue})$ or $(1\text{ red},2\text{ blue})$.
Let $x$ be the number of triples with exactly $2$ red vertices.

First count red incidences with triples.
Each red vertex lies in exactly $3$ triples, so the total number of (red vertex, triple) incidences equals $3r$.
On the other hand, each of the $x$ triples of type $(2R,1B)$ contributes $2$ red incidences, and each of the $7-x$ remaining triples contributes $1$ red incidence.
Thus
\[
3r = 2x + 1\cdot(7-x) = x+7,
\]
so
\[
x = 3r-7. \tag{1}
\]

Next count red pairs.
There are $\binom{7}{2}=21$ unordered pairs of vertices, and each triple contains $\binom{3}{2}=3$ pairs.
Since $\mathcal{F}$ has $7$ triples, the total number of pair-occurrences across all triples is $7\cdot 3=21$.
A direct check from the explicit list shows that no two distinct triples share two vertices (equivalently: the intersection of any two distinct triples has size at most $1$).
Therefore all $21$ pair-occurrences are distinct, so \emph{every} unordered pair of vertices lies in \emph{exactly one} triple of $\mathcal{F}$.

Consider any red pair $\{u,v\}$. The unique triple containing this pair cannot be monochromatic red, so its third vertex must be blue.
Hence the unique triple containing $\{u,v\}$ is of type $(2R,1B)$.
Conversely, each triple of type $(2R,1B)$ contains exactly one red pair.
Thus the number of red pairs equals $x$.
But the number of red pairs is also $\binom{r}{2}$.
Therefore
\[
\binom{r}{2} = x. \tag{2}
\]
Combining (1) and (2) gives
\[
\frac{r(r-1)}{2} = 3r-7 \quad\Longrightarrow\quad r^2-7r+14=0,
\]
whose discriminant is $49-56=-7<0$, impossible for an integer $r$.
This contradiction shows that no such 2-colouring exists; hence $\mathcal{F}$ is not property B.
\qed

\medskip
\textbf{FAST REALITY CHECK (small cases / computation).}
\begin{itemize}
\item For $t=2$ and $c_2=1$, the condition becomes: for every $X\subseteq V$, the induced subgraph on $X$ has $<|X|$ edges.
I exhaustively checked all labelled graphs on $n\le 6$ vertices; every graph satisfying this condition was bipartite.
Counts: for $n=2,3,4,5,6$ the numbers of graphs satisfying the condition are $2,7,38,291,2932$, and all are bipartite.
\item For the Fano plane family in Lemma 2, I exhaustively checked all $2^7$ vertex 2-colourings and found none avoids a monochromatic triple.
I also checked for every $X\subseteq V$ that $e(X)\le |X|$.
\end{itemize}

\textbf{VERIFICATION.}
\begin{itemize}
\item Lemma 1: the only subtle point is that deleting $v$ and restricting to edges not containing $v$ preserves the hypothesis for all $X\subseteq V'$. This holds because containment $A\subseteq X$ in $V'$ is the same statement in the original hypergraph, and edges containing $v$ cannot be contained in $X\subseteq V'$.
\item Lemma 2(b): the ``pair lies in exactly one triple'' step is justified by (i) total pair-occurrences $=21$ and (ii) no repeated pair-occurrence, i.e. no two triples share two vertices, which can be verified directly from the explicit list (only $\binom{7}{2}=21$ triple-pairs to check).
\item The obstruction conclusion: since $e(V)=7$ and $|V|=7$, for any $c_3>1$ we have $e(V)=7 < c_3\cdot 7$ and similarly $e(X)\le |X|<c_3|X|$ for all nonempty $X$. Thus the hypothesis of the problem holds for that $c_3$, while the conclusion fails.
\end{itemize}

\textbf{FINAL.} \textbf{UNRESOLVED.}

(i) \emph{Strongest proved partial result.} The implication holds with the fixed constant $c_t=1$ for every $t\ge 2$ (Lemma 1). Also, for $t=3$ any choice $c_3>1$ fails (Lemma 2).

(ii) \emph{First gap (crisp).} Prove or disprove: there exists a function $t\mapsto c_t$ with $c_t\to\infty$ such that every finite hypergraph with all edges of size $\ge t$ and satisfying $e(X)<c_t|X|$ for all $X\subseteq V$ is 2-colourable.

(iii) \emph{Top 3 next moves.}
1. Try to construct, for each large $t$, an explicit non-2-colourable $t$-uniform hypergraph with $e(X)\le C|X|$ for all $X$ where $C$ is as large as possible; determine whether $C$ can grow with $t$.
2. Attempt a proof scheme: show the condition $e(X)<c|X|$ forces a vertex of small degree (as in Lemma 1) when $c$ is large relative to $t$, and then extend the deletion argument to $t$-uniform constraints.
3. Try probabilistic colouring + a dependency graph that uses the stronger hypothesis ``for all $X$'' rather than only global $|\mathcal{F}|$; explicitly target a Local Lemma criterion that would allow $c_t$ to grow with $t$.

(iv) \emph{What a minimal counterexample would likely look like.} A non-2-colourable $t$-uniform (or $\ge t$-uniform) hypergraph that is very ``linear'' (few overlaps), with induced subhypergraphs having edge-count comparable to $|X|$ (to satisfy the local sparsity), i.e. a design-like structure where every vertex has bounded degree but the hypergraph still forces a monochromatic edge in every 2-colouring.


