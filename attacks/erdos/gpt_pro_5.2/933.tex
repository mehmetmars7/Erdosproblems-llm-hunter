
1) FORMAL RESTATEMENT

Let $n\ge 1$ be an integer and factor
\[
 n(n+1)=2^k3^l m,
\]
where $k:=v_2(n(n+1))$, $l:=v_3(n(n+1))$, and $m$ is an integer coprime to $6$ (so $(m,6)=1$).
(Here $v_p$ is the usual $p$-adic valuation.)
Let $\log$ denote the natural logarithm.
The question is whether
\[
\limsup_{n\to\infty} \frac{2^k3^l}{n\log n}=\infty.
\]

2) QUICK LITERATURE/CONTEXT CHECK

From the provided problem statement: Mahler proved a more general result implying $2^k3^l<n^{1+o(1)}$. Erd\H{o}s claimed it is easy to see that $2^k3^l>n\log n$ for infinitely many $n$, and Steinerberger noted a simple proof via $n=2^{3^r}$.

3) ATTACK PLAN

\begin{itemize}
\item \textbf{Lower bound constructions:} choose special $n$ so that $v_2(n(n+1))$ and/or $v_3(n(n+1))$ are large and computable via LTE (lifting-the-exponent).
\item \textbf{Try to amplify the ratio:} look for $n$ where one of $n$ and $n+1$ is close to a large power of $2$ and the other is close to a large power of $3$.
\item \textbf{Computation:} scan $n$ up to a moderate bound to see how large the ratio actually gets and what $n$ achieves it.
\end{itemize}

4) WORK

\textbf{Lemma 933.1 (an infinite family with ratio $>1$ via $n=2^{3^r}$).}
For every integer $r\ge 1$, let $n=2^{3^r}$. Then
\[
 v_2(n(n+1))=3^r,\qquad v_3(n(n+1))=r+1,
\]
and hence
\[
\frac{2^k3^l}{n\log n}=\frac{3}{\log 2}>1.
\]
In particular, $2^k3^l>n\log n$ for infinitely many $n$.

\emph{Proof.}
Let $n=2^{3^r}$. Then $n$ is even and $n+1$ is odd, so
\[
v_2(n(n+1))=v_2(n)=3^r.
\]
Next, $2\equiv -1\pmod 3$, so $2^{3^r}\equiv -1\pmod 3$, hence $3\mid (2^{3^r}+1)=n+1$.
We compute the exact 3-adic valuation using the standard LTE identity for odd primes $p$:
if $p\mid (a+b)$ and $p\nmid ab$ and $t$ is odd, then
\[
 v_p(a^t+b^t)=v_p(a+b)+v_p(t).
\]
Apply this with $p=3$, $a=2$, $b=1$, and $t=3^r$ (odd).
Since $3\mid (2+1)$ and $3\nmid 2$, LTE gives
\[
 v_3(2^{3^r}+1)=v_3(2+1)+v_3(3^r)=1+r.
\]
Because $n$ itself is a power of $2$, $v_3(n)=0$, so
\[
 v_3(n(n+1))=v_3(n+1)=r+1.
\]
Therefore $2^k3^l=2^{3^r}3^{r+1}=n\cdot 3^{r+1}$.
Also $\log n = \log(2^{3^r})=3^r\log 2$.
Hence
\[
\frac{2^k3^l}{n\log n}=\frac{n\cdot 3^{r+1}}{n\cdot 3^r\log 2}=\frac{3}{\log 2}.
\]
\hfill$\square$

\medskip
\textbf{Lemma 933.2 (another infinite family via $n=3^{2^r}-1$).}
For every integer $r\ge 1$, let $n=3^{2^r}-1$. Then
\[
 v_3(n(n+1))=2^r,\qquad v_2(n(n+1))=r+2,
\]
and
\[
\lim_{r\to\infty}\frac{2^k3^l}{n\log n}=\frac{4}{\log 3}>1.
\]
In particular, for all sufficiently large $r$ we have $2^k3^l>n\log n$.

\emph{Proof.}
Let $n=3^{2^r}-1$. Then $n+1=3^{2^r}$, so $v_3(n+1)=2^r$ and $v_3(n)=0$ (since $n\equiv -1\pmod 3$). Hence
\[
v_3(n(n+1))=v_3(n+1)=2^r.
\]
Next, $n$ is even and $n+1$ is odd, so $v_2(n(n+1))=v_2(n)=v_2(3^{2^r}-1)$.
We use the $2$-adic LTE formula for odd $a$ and even exponent $t$:
\[
 v_2(a^t-1)=v_2(a-1)+v_2(a+1)+v_2(t)-1.
\]
Apply this with $a=3$ and $t=2^r$.
We have $v_2(3-1)=v_2(2)=1$, $v_2(3+1)=v_2(4)=2$, and $v_2(2^r)=r$.
Therefore
\[
 v_2(3^{2^r}-1)=1+2+r-1=r+2.
\]
So $2^k3^l=2^{r+2}\cdot 3^{2^r}$.
Since $n=3^{2^r}-1\sim 3^{2^r}$ and $\log n\sim 2^r\log 3$ as $r\to\infty$, we obtain
\[
\frac{2^k3^l}{n\log n}
=\frac{2^{r+2}3^{2^r}}{(3^{2^r}-1)\log(3^{2^r}-1)}
\longrightarrow \frac{2^{r+2}}{2^r\log 3}=\frac{4}{\log 3}.
\]
The limit is $>1$, so for all sufficiently large $r$ the ratio exceeds $1$, giving $2^k3^l>n\log n$ infinitely often.
\hfill$\square$

\medskip
\textbf{FAST REALITY CHECK (local computation up to $2{,}000{,}000$).}
I scanned all $2\le n\le 2{,}000{,}000$ and computed
\(R(n):=2^{v_2(n(n+1))}3^{v_3(n(n+1))}/(n\log n)\).
The maximum value observed in this range was
\[
\max_{2\le n\le 2{,}000{,}000} R(n)\approx 4.328085\,=\frac{3}{\log 2},
\]
achieved for example at $n=2$ and $n=8$.
(This is evidence only; it does not bound the true limsup.)

5) VERIFICATION

\begin{itemize}
\item Lemma 933.1: checked the hypotheses of LTE for $v_3(2^{3^r}+1)$: $3\mid 2+1$ and $3\nmid 2\cdot 1$ and exponent $3^r$ is odd.
\item Lemma 933.2: checked the $2$-adic LTE conditions: base $3$ is odd and exponent $2^r$ is even.
\item Computation: the scan used exact $v_2$ via trailing-zero count and exact division for $v_3$.
\end{itemize}

6) FINAL

\textbf{UNRESOLVED}

(i) \textbf{Strongest proved partial result here.}
There are infinitely many $n$ with $2^k3^l>n\log n$ (Lemma 933.1 gives an exact constant ratio $3/\log 2$ along $n=2^{3^r}$; Lemma 933.2 gives another constant $4/\log 3$ along $n=3^{2^r}-1$). Hence
\[
\limsup_{n\to\infty}\frac{2^k3^l}{n\log n}\ \ge\ \frac{3}{\log 2}.
\]

(ii) \textbf{First gap (crisp).}
Either prove that the ratio $\frac{2^k3^l}{n\log n}$ is unbounded (limsup $=\infty$) or produce an explicit upper bound $\frac{2^k3^l}{n\log n}\le C$ for all sufficiently large $n$.

(iii) \textbf{Top 3 next moves.}
\begin{enumerate}
\item Search for $n$ such that one of $n,n+1$ is divisible by an unusually large power of $2$ while the other is divisible by an unusually large power of $3$, in a way that makes $2^k3^l$ close to $n^2$; this reduces to studying near-collisions of the form $2^a u \pm 1=3^b v$ with small $u,v$.
\item Extend the brute-force scan to larger $n$ using log-scale comparisons to see whether any $R(n)$ exceed $3/\log 2$ and to guess growth.
\item Try to bound $v_2(n)$ and $v_3(n\pm 1)$ simultaneously using $p$-adic methods (e.g. lifting-exponent plus constraints from exponential Diophantine equations) to either force boundedness or construct sequences with growing $v_2$ and $v_3$.
\end{enumerate}

(iv) \textbf{What a minimal counterexample would look like.}
If the conjectured limsup $=\infty$ is false, there would exist a constant $C$ and $n_0$ such that for all $n\ge n_0$, the $\{2,3\}$-part of $n(n+1)$ satisfies $2^k3^l\le C\,n\log n$. Any extremal $n$ should then avoid having simultaneously large $2$-adic valuation on one of $n,n+1$ and large $3$-adic valuation on the other.


