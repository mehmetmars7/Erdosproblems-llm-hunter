% Erdos Problem #938

1) FORMAL RESTATEMENT

Let \(A=\{n_1<n_2<\cdots\}\) be the increasing sequence of powerful (squarefull) numbers, i.e. positive integers \(m\) such that for every prime \(p\mid m\) we have \(p^2\mid m\).

Question (main): Are there only finitely many indices \(k\) such that the three consecutive terms \(n_k,n_{k+1},n_{k+2}\) form a 3-term arithmetic progression, i.e.
\[n_{k+1}-n_k = n_{k+2}-n_{k+1}?\]

Related conjecture (stated): There is no triple of powerful numbers of the form \(n,n+1,n+2\).

2) QUICK LITERATURE/CONTEXT CHECK

The problem statement reports only Erd\H{o}s's conjectural remark about \(n,n+1,n+2\). I do not use any external results.

3) ATTACK PLAN

- Use a structural description of powerful numbers to (i) count them and (ii) generate them for computation.
- Compute the initial segment of \(A\) and check for arithmetic-progression triples among consecutive terms.

4) WORK

Fast reality check (explicit computation).
Using the standard parametrisation of powerful numbers and enumerating all powerful numbers \(\le 10^6\):
- There are 2027 powerful numbers \(\le 10^6\).
- There are 3 instances where \(n_k,n_{k+1},n_{k+2}\) are in arithmetic progression:
  \[(1728,1764,1800)\ (\text{common difference }36),\]
  \[(6912,7056,7200)\ (\text{common difference }144),\]
  \[(729000,729316,729632)\ (\text{common difference }316).\]
- There are no triples \(n,n+1,n+2\) of powerful numbers with \(n+2\le 10^6\).

Lemma 938.1 (Powerful numbers are \(a^2 b^3\) with \(b\) squarefree).
A positive integer \(m\) is powerful if and only if it can be written in the form
\[m=a^2 b^3\]
for some integers \(a\ge 1\) and squarefree \(b\ge 1\).

Proof.
(\(\Rightarrow\)) Let \(m=\prod_p p^{e_p}\) with all \(e_p\ge 2\). For each prime \(p\), write \(e_p\) as \(e_p=2u_p+3v_p\) where \(v_p\in\{0,1\}\):
- If \(e_p\) is even, take \(v_p=0\) and \(u_p=e_p/2\).
- If \(e_p\) is odd, then \(e_p\ge 3\); take \(v_p=1\) and \(u_p=(e_p-3)/2\).
Define
\[a:=\prod_p p^{u_p},\qquad b:=\prod_{p:\,v_p=1} p.
\]
Then \(b\) is squarefree by construction (each such prime occurs to exponent 1), and
\[a^2 b^3 = \prod_p p^{2u_p} \prod_{p:\,v_p=1} p^{3} = \prod_p p^{2u_p+3v_p} = \prod_p p^{e_p}=m.
\]

(\(\Leftarrow\)) Conversely, if \(m=a^2 b^3\) with \(b\) squarefree, then for any prime \(p\mid m\), either \(p\mid a\) (so \(p^2\mid a^2\mid m\)) or \(p\mid b\) (so \(p^3\mid b^3\mid m\), hence again \(p^2\mid m\)). Thus every prime divisor of \(m\) divides it to exponent at least 2, i.e. \(m\) is powerful.
QED.

Lemma 938.2 (Crude counting bound).
Let \(N(X)\) be the number of powerful integers \(m\le X\). Then \(N(X)=O(X^{1/2})\).

Proof.
By Lemma 938.1, every powerful \(m\le X\) can be written as \(m=a^2 b^3\) with \(b\) squarefree and \(b^3\le X\), hence \(b\le X^{1/3}\).
For fixed \(b\), the condition \(a^2 b^3\le X\) implies \(a\le \sqrt{X/b^3}=X^{1/2} b^{-3/2}\). Therefore
\[N(X) \le \sum_{\substack{b\le X^{1/3}\\ b\ \mathrm{squarefree}}} \Big\lfloor X^{1/2} b^{-3/2}\Big\rfloor
\le X^{1/2} \sum_{\substack{b\ge 1\\ b\ \mathrm{squarefree}}} b^{-3/2}.
\]
The series \(\sum_{b\ge 1} b^{-3/2}\) converges (p-series with exponent \(3/2>1\)), and the restriction to squarefree \(b\) only decreases the sum. Hence the sum is bounded by a constant independent of \(X\), giving \(N(X)\ll X^{1/2}\). QED.

5) VERIFICATION

- Lemma 938.1: For odd \(e_p\ge2\), we used \(e_p\ge3\) to make \((e_p-3)/2\ge0\), which holds.
- Computation: the three displayed triples were extracted by scanning the sorted list of powerful numbers \(\le 10^6\) and checking consecutive triples. This validates existence of such triples at least up to this bound.

6) FINAL

UNRESOLVED
(i) Strongest proved partial result: Powerful numbers admit the \(a^2b^3\) description (Lemma 938.1) and satisfy the crude count \(N(X)=O(X^{1/2})\) (Lemma 938.2). Computationally, there are at least three arithmetic-progression triples among consecutive powerful numbers up to \(10^6\), including \((1728,1764,1800)\).
(ii) First gap: Prove either that the set \(\{k: n_k,n_{k+1},n_{k+2}\ \text{is an AP}\}\) is finite, or construct an infinite family of such indices while controlling the ``consecutive in the powerful sequence'' condition.
(iii) Top 3 next moves:
  1. Search for parametrised families of AP triples of powerful numbers and test whether the ``no other powerful numbers in between'' property persists.
  2. Develop local density/gap estimates for powerful numbers in intervals \([x,x+\Delta]\) to bound when a given AP triple can be consecutive.
  3. Extend brute-force computation to larger bounds and record the gap patterns around candidate AP triples.
(iv) Minimal counterexample structure: If there are infinitely many such AP triples of consecutive powerful numbers, one expects a structured family \((a_k, a_k+d_k, a_k+2d_k)\) of powerful numbers with unusually large gaps to exclude intermediate powerful values; understanding how powerful numbers cluster in residue classes modulo large squares/cubes may be essential.


