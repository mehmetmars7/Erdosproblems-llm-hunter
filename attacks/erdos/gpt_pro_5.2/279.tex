\section*{Problem 279}

\subsection*{FORMAL RESTATEMENT}

Fix an integer $k\ge 3$. For each prime $p$, choose a residue $a_p\in\mathbb{Z}/p\mathbb{Z}$ and consider the set
\[
S(\mathbf a)\;:=\;\bigcup_{p\ \text{prime}}\ \{a_p+t\,p:\ t\in\mathbb{Z},\ t\ge k\}.
\]
Question: is there a choice of residues $(a_p)_{p\ \mathrm{prime}}$ such that
$\mathbb{Z}\\setminus S(\mathbf a)$ is finite?

The problem also asks about replacing the primes by a more general set $A\subseteq\mathbb{N}$ and
analogous hypotheses on the size of $A$ and on $\sum_{n\in A,\,n\le N}\frac1n$.

\subsection*{QUICK LITERATURE/CONTEXT CHECK}

As stated in the problem text, the case $k=3$ already appears difficult and (to the best of my knowledge)
remains open. The problem suggests it may hold under stronger ``prime-like'' hypotheses on a general set $A$,
and notes that for $k=1$ and $k=2$ the corresponding statement is known for any $A$ with $\sum_{n\in A}1/n=\infty$
(again, this is as stated in the problem source).

\subsection*{ATTACK PLAN}

\begin{itemize}
\item Rephrase the question as an ``asymptotic covering system'' problem:
choose one residue class modulo each prime $p$, but the progression only starts at $a_p+k p$.
\item Observe that for a large integer $m$, only primes $p\le m/k$ can represent $m$.
Thus the available moduli increase with $m$, but slowly.
\item Use heuristics from random residue choices: for large $m$, the chance that $m$ avoids all
chosen classes for primes $p\le m/k$ should be about $\prod_{p\le m/k}(1-1/p)\asymp 1/\log m$
(Mertens). This suggests the uncovered set might be very sparse, but sparseness does not imply finiteness.
\end{itemize}

\subsection*{WORK (PARTIAL RESULTS AND OBSERVATIONS)}

\paragraph{1) Trivialities for small $k$.}
\begin{itemize}
\item If $k=1$ and $A$ is the set of all primes, the choice $a_p\equiv 0$ covers every integer $m>1$:
take $p$ a prime divisor of $m$, so $m=0+(m/p)p$ with $t=m/p\ge 1$.
\item For $k\ge 2$, primes themselves become the obstruction: if $a_p\equiv 0$ for all $p$,
then every composite number is covered with $t\ge 2$, but primes are not (since $p=0+1\cdot p$).
One needs residues that also cover all sufficiently large primes using \emph{smaller} primes.
\end{itemize}

\paragraph{2) A useful reformulation.}
Fix residues $(a_p)$. An integer $m$ is \emph{covered} if there exists a prime $p\le m/k$ with
\[
m\equiv a_p \pmod p.
\]
(Indeed, if $m\equiv a_p\pmod p$ then $m=a_p+t p$ for some $t\ge 0$, and for $m$ large relative to $p$
this will force $t\ge k$; a sufficient condition is $m\ge (k+1)p$ since then $t\ge k$ regardless of the
representative $a_p\in\{0,\dots,p-1\}$.)

So the problem is to choose one residue mod each prime so that every sufficiently large $m$ lies in at least
one chosen class for a prime $p$ that is not too large relative to $m$.

\paragraph{3) Heuristic density.}
If residues were chosen ``randomly and independently'', then for a fixed large $m$ the probability that
$m\not\equiv a_p\pmod p$ is $1-1/p$. If we (very heuristically) pretend these are independent over primes
$p\le m/k$, we get
\[
\mathbb{P}(m\ \text{uncovered})\approx \prod_{p\le m/k}\Big(1-\frac1p\Big)\asymp \frac{c}{\log m}.
\]
This suggests that one might hope to cover all but finitely many integers, but the heuristic is far from
a proof: it only indicates that the uncovered set \emph{might} have density $0$.

\subsection*{VERIFICATION}

For $k=3$, a concrete test would require choosing residues for many primes and checking coverage up to a large range.
Such computations provide evidence but do not resolve the finiteness of the exceptional set; the published problem
explicitly notes that even $k=3$ is difficult.

\subsection*{FINAL}

\textbf{FINAL: UNRESOLVED.} The existence of such a choice of residues for $k\ge 3$ (in particular $k=3$)
is an open problem as posed.

\subsection*{COMPLETION ESTIMATE}

About \textbf{15\%}: restatement + basic observations + heuristic; no proof known for $k\ge 3$.

% ============================================================
