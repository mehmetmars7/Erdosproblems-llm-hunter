% Erdos Problem #1068

\subsection*{FORMAL RESTATEMENT}
Let $G$ be a graph with chromatic number $\chi(G)=\aleph_1$.
Does $G$ necessarily contain a \emph{countable} subgraph $H\subseteq G$ (i.e. $|V(H)|\le\aleph_0$) that is \emph{infinitely connected}:
for every distinct $u,v\in V(H)$ there exist infinitely many pairwise disjoint $u$--$v$ paths in $H$?

\emph{Note on ambiguity.} ``Disjoint paths'' could mean edge-disjoint or internally vertex-disjoint.  In infinite connectivity problems, the natural strengthening is internally vertex-disjoint with shared endpoints allowed; I will use that convention below (it makes the degree consequences sharp).

\subsection*{QUICK LITERATURE/CONTEXT CHECK}
The problem file cites [LS95] and [Sze97].  I will not use any results from these sources (or elsewhere) beyond what is explicitly stated in the problem text.  I only prove general structural lemmas about infinite connectivity and a basic extraction lemma from the hypothesis $\chi(G)=\aleph_1$.

\subsection*{ATTACK PLAN}
First, record elementary consequences of ``infinitely many pairwise internally vertex-disjoint $u$--$v$ paths'' (infinite minimum degree, avoidance of finite vertex sets).  Second, prove a compactness-type lemma: if a graph is not $k$-colorable for finite $k$, then it has a finite subgraph that is not $k$-colorable; this yields a countable subgraph of arbitrarily large finite chromatic number, hence of chromatic number at least $\aleph_0$.  Finally, explain why this still does not force an infinitely connected countable subgraph.

\subsection*{WORK}
\textbf{Lemma 1068.1 (infinite connectivity forces infinite degree).}
Assume $H$ is a graph in which for every pair of distinct vertices $u,v$ there exist infinitely many pairwise internally vertex-disjoint $u$--$v$ paths.
Then every vertex of $H$ has infinite degree.

\emph{Proof.}
Fix a vertex $u\in V(H)$. Choose any $v\ne u$.
Let $P_1,P_2,\ldots$ be infinitely many pairwise internally vertex-disjoint $u$--$v$ paths.
For each $i$, let $w_i$ be the neighbor of $u$ that is the second vertex on $P_i$.
If $w_i=w_j$ for $i\ne j$, then $P_i$ and $P_j$ would share the internal vertex $w_i$ (since $w_i\ne u,v$), contradicting internal vertex-disjointness.
Hence the $w_i$ are all distinct neighbors of $u$, so $u$ has infinitely many neighbors and therefore $\deg_H(u)=\infty$.
\qed

\textbf{Lemma 1068.2 (avoidance of finite vertex sets).}
Let $H$ be infinitely connected in the sense above.  Fix distinct $u,v\in V(H)$ and a finite set $S\subseteq V(H)\setminus\{u,v\}$.  Then there exists a $u$--$v$ path in $H$ whose internal vertices avoid $S$.

\emph{Proof.}
Let $P_1,P_2,\ldots$ be infinitely many pairwise internally vertex-disjoint $u$--$v$ paths.  Each vertex $s\in S$ can lie as an internal vertex on at most one of these paths (otherwise two paths would share $s$ as an internal vertex). Therefore at most $|S|$ of the paths meet $S$ internally.  Since there are infinitely many paths, some $P_i$ avoids $S$ internally.
\qed

\textbf{Lemma 1068.3 (finite obstruction to $k$-colorability for finite $k$).}
Fix a finite $k\ge 1$.  If a graph $G$ is not $k$-colorable, then it has a finite subgraph $F\subseteq G$ that is not $k$-colorable.

\emph{Proof.}
Let $V=V(G)$. Consider the compact product space $X := \{1,2,\dots,k\}^V$ with the product topology (Tychonoff compactness).
For each edge $e=\{x,y\}\in E(G)$, let
\[
X_e := \{c\in X : c(x)\ne c(y)\}.
\]
Then $X_e$ is open-and-closed (it is determined by the coordinates at $x,y$).
A proper $k$-coloring of $G$ is exactly a point in $\bigcap_{e\in E(G)} X_e$.
Assume $G$ is not $k$-colorable. Then
\[
\bigcap_{e\in E(G)} X_e = \emptyset.
\]
Since $X$ is compact and the $X_e$ are closed, the family $\{X_e : e\in E(G)\}$ has the finite intersection property if and only if its total intersection is nonempty.
Because the total intersection is empty, the finite intersection property fails: there exist edges $e_1,\dots,e_m$ such that
\[
X_{e_1}\cap\cdots\cap X_{e_m} = \emptyset.
\]
Let $F$ be the finite subgraph with vertex set consisting of endpoints of $e_1,\dots,e_m$ and edge set $\{e_1,\dots,e_m\}$.  Then a $k$-coloring of $F$ would define a point of $X$ lying in $X_{e_1}\cap\cdots\cap X_{e_m}$, contradicting emptiness. Hence $F$ is not $k$-colorable.
\qed

\textbf{Corollary 1068.4 (countable subgraph of infinite chromatic number).}
If $\chi(G)=\aleph_1$, then $G$ contains a countable subgraph $H_\omega$ with $\chi(H_\omega)\ge \aleph_0$.

\emph{Proof.}
For each $m\in\mathbb{N}$, $G$ is not $m$-colorable, so by Lemma~1068.3 there is a finite subgraph $F_m\subseteq G$ with $\chi(F_m)\ge m+1$.
Let $H_\omega$ be the subgraph induced by $\bigcup_{m\ge 1} V(F_m)$.  This is a countable union of finite sets, hence countable.
If $H_\omega$ were $m$-colorable for some finite $m$, then every subgraph of $H_\omega$ would be $m$-colorable, contradicting that $F_m\subseteq H_\omega$ has chromatic number at least $m+1$.  Therefore $H_\omega$ is not $m$-colorable for any finite $m$, i.e. $\chi(H_\omega)\ge \aleph_0$.
\qed

\textbf{FAST REALITY CHECK (examples).}
\begin{itemize}
\item If $G$ is the complete graph $K_{\aleph_1}$, then $\chi(G)=\aleph_1$ and it contains the countable complete subgraph $K_{\aleph_0}$, which is infinitely connected: between any two vertices $u,v$, the paths $u-w-v$ for distinct intermediate vertices $w$ are pairwise internally disjoint.
\item The corollary shows that uncountable chromatic number forces a countable subgraph of infinite chromatic number, but does not obviously force any connectivity, let alone infinite connectivity.
\end{itemize}

\subsection*{VERIFICATION}
Lemmas~1068.1--1068.2 are direct consequences of the definition of infinitely many internally vertex-disjoint paths.
Lemma~1068.3 is proved using a compactness argument in the explicit finite-$k$ coloring space $\{1,\dots,k\}^V$; no outside graph-theoretic results are invoked.
Corollary~1068.4 is a standard diagonal union of finite obstructions.

\subsection*{FINAL}
\textbf{UNRESOLVED.}

(i) \emph{Strongest proved partial result:}
From $\chi(G)=\aleph_1$ one can extract a countable subgraph of chromatic number at least $\aleph_0$ (Corollary~1068.4).  Independently, infinite connectivity forces strong local structure such as infinite minimum degree and avoidance of finite vertex sets (Lemmas~1068.1--1068.2).

(ii) \emph{First gap (crisp statement):}
Bridge the gap between ``countable subgraph of infinite chromatic number'' and ``countable subgraph that is infinitely connected'': given $\chi(G)=\aleph_1$, prove or refute the existence of a countable $H\subseteq G$ such that every pair of vertices in $H$ has infinitely many internally vertex-disjoint connecting paths.

(iii) \emph{Top 3 next moves (concrete lemmas/computations):}
\begin{enumerate}
\item Try to show that $\chi(G)=\aleph_1$ forces a countable subgraph with large (perhaps infinite) minimum degree, as a stepping stone toward Lemma~1068.1-type necessary conditions.
\item Attempt an ``iterated connector'' construction: build a countable vertex set by repeatedly adding many disjoint paths between previously chosen vertices, and analyze whether uncountable chromatic number guarantees that this process can continue without getting stuck.
\item Search for candidate counterexamples by considering graphs known (or plausibly) to have chromatic number $\aleph_1$ but sparse connectivity, and test whether every countable subgraph necessarily has a finite separator.
\end{enumerate}

(iv) \emph{Minimal counterexample structure:}
A counterexample would be a graph $G$ with $\chi(G)=\aleph_1$ such that every countable subgraph $H\subseteq G$ fails infinite connectivity; by Lemma~1068.2 this means that for each such $H$ there exist $u\ne v$ and a \emph{finite} vertex set $S$ separating $u$ from $v$ (or, equivalently, only finitely many internally disjoint $u$--$v$ paths).


