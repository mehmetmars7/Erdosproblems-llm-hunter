
1) FORMAL RESTATEMENT

Let $p_r$ denote the $r$th prime, and write $g_r:=p_{r+1}-p_r$ for the prime gap.
For a positive integer $n$, let $P^+(n)$ be the largest prime factor of $n$.
The property in the problem is:
\[
\exists\ \text{at least two integers }n\ \text{with }p_r<n<p_{r+1}\ \text{and }P^+(n)<g_r.
\]
The question is whether this happens for infinitely many $r$.

Edge cases: if $g_r=2$ then there are no integers strictly between $p_r$ and $p_{r+1}$; if $g_r=3$ there is exactly one integer in between, so the property cannot hold. Thus any $r$ witnessing the property must have $g_r\ge 4$.

2) QUICK LITERATURE/CONTEXT CHECK

From the provided problem statement: Erd\H{o}s believed the answer is ``yes'' but that such $r$ are sparse, and he could show that the density of $r$ for which at least one such $n$ exists is $0$. The statement was also formalised in Lean in the Google DeepMind Formal Conjectures project (as noted in the source file).

3) ATTACK PLAN

\begin{itemize}
\item \textbf{Try to build explicit $r$:} a sufficient condition is to find a prime gap in which there are two numbers composed only of primes $<g_r$.
\item \textbf{Local sufficient conditions:} if an integer $n$ in the gap is small compared to $g_r$, then having a prime factor $\ge g_r$ is impossible.
\item \textbf{Computation:} scan prime gaps up to some bound and count how many $n$ in each gap satisfy $P^+(n)<g_r$.
\end{itemize}

4) WORK

\textbf{Lemma 932.1 (a size condition that forces $g_r$-smoothness).}
Fix $r$ and set $g=g_r$.
If $n$ is a composite integer satisfying $p_r<n<p_{r+1}$ and $n<2g$, then every prime factor of $n$ is $<g$. (Equivalently, $P^+(n)<g$.)

\emph{Proof.}
Assume $n$ is composite and $n<2g$.
Let $q$ be any prime factor of $n$. If $q\ge g$, then $n=q\cdot m$ for some integer $m\ge 2$ (since $n$ is composite). Hence $n\ge 2q\ge 2g$, contradicting $n<2g$.
Therefore every prime factor $q$ of $n$ satisfies $q<g$, as claimed.
\hfill$\square$

\medskip
\textbf{Proposition 932.2 (explicit small example).}
For $r=4$ we have $(p_r,p_{r+1})=(7,11)$ and $g_4=4$. The integers $8$ and $9$ satisfy $7<8<9<11$ and all their prime factors are $<4$. Hence $r=4$ satisfies the property.

\emph{Proof.}
Compute: $8=2^3$ has prime factor set $\{2\}$, and $9=3^2$ has prime factor set $\{3\}$. Both $2$ and $3$ are strictly less than $g_4=4$.
\hfill$\square$

\medskip
\textbf{FAST REALITY CHECK (local computation up to $200{,}000$).}
I computed all prime gaps with $p_{r+1}\le 200{,}000$ and counted, for each $r$, how many integers $n$ with $p_r<n<p_{r+1}$ satisfy $P^+(n)<g_r$.
In this range I found $20$ values of $r$ for which there are \emph{at least two} such integers.
The first few are:
\begin{center}
\begin{tabular}{r|r|r|r|l}
$r$ & $p_r$ & $p_{r+1}$ & $g_r$ & first two ``good'' $n$\\\hline
4 & 7 & 11 & 4 & 8,9 \\
9 & 23 & 29 & 6 & 24,25 \\
11 & 31 & 37 & 6 & 32,36 \\
15 & 47 & 53 & 6 & 48,50 \\
24 & 89 & 97 & 8 & 90,96 \\
\end{tabular}
\end{center}
(Here ``good'' means all prime factors $<g_r$.)
This is only evidence and does not address infinitude.

5) VERIFICATION

\begin{itemize}
\item Lemma 932.1: the only use of compositeness is to ensure $m\ge 2$ when writing $n=q m$. If $n$ were prime, it could have $P^+(n)=n\ge g$.
\item Proposition 932.2: verified by direct factorization.
\end{itemize}

6) FINAL

\textbf{UNRESOLVED}

(i) \textbf{Strongest proved partial result here.}
A sufficient condition for an integer $n$ in the prime gap to satisfy $P^+(n)<g_r$ is $n<2g_r$ together with compositeness (Lemma 932.1). There are explicit small $r$ with at least two such $n$ (e.g. $r=4$) and computationally many more in small ranges.

(ii) \textbf{First gap (crisp).}
Prove that there are infinitely many $r$ such that at least two integers $n$ with $p_r<n<p_{r+1}$ satisfy $P^+(n)<p_{r+1}-p_r$.

(iii) \textbf{Top 3 next moves.}
\begin{enumerate}
\item Try to explicitly construct prime gaps that contain two numbers whose prime factors are all below the gap size (e.g. by combining known ``large gap'' constructions with smooth numbers).
\item Compute further to much larger bounds to guess the growth rate (frequency) of such $r$, and to look for repeating structural patterns in the good integers.
\item Seek a theoretical sufficient condition on the residue class structure of $p_r$ modulo small primes that guarantees two $g_r$-smooth integers in $(p_r,p_{r+1})$.
\end{enumerate}

(iv) \textbf{What a minimal counterexample would look like.}
A counterexample to the infinitude statement would mean that beyond some $r_0$, every prime gap $(p_r,p_{r+1})$ contains at most one integer whose prime factors are all $<g_r$. In particular, for large $r$ the interval would force most composites to have a prime factor at least the gap size.


