
1) FORMAL RESTATEMENT

Let $\varphi$ be Euler's totient function. Define iterates
\[
\varphi_1(n)=\varphi(n),\qquad \varphi_{k}(n)=\varphi(\varphi_{k-1}(n))\ (k\ge 2).
\]

Define
\[
 f(n):=\min\{k\ge 0:\ \varphi_k(n)=1\},
\]
with the convention $\varphi_0(n)=n$ (so $f(1)=0$).

Questions.
(A) Does $f(n)/\log n$ have a distribution function?
(B) Is $f(n)/\log n$ almost always constant?
(C) For $k=\log\log n$, what can be said about the largest prime factor of $\varphi_k(n)$?

2) QUICK LITERATURE/CONTEXT CHECK

I only restate what is recorded in the problem statement.

The statement records:
- Pillai proved $\log_3 n < f(n) < \log_2 n$ for all large $n$.
- Shapiro proved $f(n)$ is essentially multiplicative.
- Erd\H{o}s--Granville--Pomerance--Spiro proved (conditional on Elliott--Halberstam type input) the answers to (A) and (B) are yes.

3) ATTACK PLAN

Proof-track (partial):
- Prove an elementary upper bound on $f(n)$ using $\varphi(n)\le n/2$ for $n>2$.
- Compute small values for sanity.

Disproof-track:
- Try to find irregular behaviour of $f(n)/\log n$ in small ranges (purely computational evidence).

Chosen path: prove explicit elementary bounds and compute $f(n)$ for $n\le 50$.

4) WORK

PHASE 1 — FAST REALITY CHECK (computed)

Computed values of $f(n)$ for $1\le n\le 50$ (where $f(n)$ counts the number of totient-iterations to reach $1$) start as
\[
( f(1),\dots,f(10))=(0,1,2,2,3,2,3,3,3,3),
\]
and the maximum in $n\le 50$ is $6$ (attained at $n=41,47$ in this range).

Lemma 1 (elementary upper bound $f(n)\le \lceil\log_2 n\rceil$).
For every integer $n\ge 2$,
\[
 f(n)\le \lceil\log_2 n\rceil.
\]

Proof.
We treat $n=2$ separately: $\varphi(2)=1$, so $f(2)=1=\lceil\log_2 2\rceil$.

Now assume $n\ge 3$.
Set $x_0=n$ and $x_j:=\varphi_j(n)$ for $j\ge 1$.
Then $x_1=\varphi(n)\le n-1$.
Also, $x_1$ is even: indeed, for $n>2$ the reduced residue classes modulo $n$ come in pairs $\{a,n-a\}$, so their count $\varphi(n)$ is even.

Claim.
For every even integer $m\ge 2$ one has $\varphi(m)\le m/2$.

Proof of the claim.
If $m$ is even, then any integer coprime to $m$ must be odd. There are exactly $m/2$ odd integers in $\{1,2,\dots,m\}$, so $\varphi(m)\le m/2$. \qed

We now bound the iterates.
Since $x_1$ is even, either $x_1=2$ (in which case $x_2=\varphi(2)=1$) or $x_1\ge 4$.
In all cases, as long as $x_j\ge 2$ we have $x_j$ even and hence by the claim
\[
 x_{j+1}=\varphi(x_j)\le \frac{x_j}{2}.
\]
By induction this yields, for every $k\ge 1$,
\[
 x_k\le \frac{n-1}{2^{k-1}}.
\]
Now take $k:=\lceil\log_2 n\rceil$.
Then $2^{k-1}\ge n/2$, so
\[
 x_k\le \frac{n-1}{2^{k-1}} < \frac{n}{2^{k-1}}\le 2.
\]
Since $x_k$ is a positive integer, this forces $x_k=1$.
Therefore $f(n)\le k=\lceil\log_2 n\rceil$. \qed

Remark.
The above proof is deliberately elementary and yields a correct $O(\log n)$ bound, though it does not reproduce the sharper constant bounds recorded in the statement.

Lemma 2 (exact value on powers of $2$).
For every integer $m\ge 1$,
\[
 f(2^m)=m.
\]

Proof.
For $m\ge 1$, $\varphi(2^m)=2^{m-1}$.
Iterating gives
\[
\varphi_1(2^m)=2^{m-1},\ \varphi_2(2^m)=2^{m-2},\ \dots,\ \varphi_{m-1}(2^m)=2,\ \varphi_m(2^m)=1.
\]
Thus the smallest $k$ with $\varphi_k(2^m)=1$ is $k=m$, i.e. $f(2^m)=m$. \qed

5) VERIFICATION

- The formula $f(2^m)=m$ matches computed values: $f(2)=1$, $f(4)=2$, $f(8)=3$, $f(16)=4$, $f(32)=5$.
- The upper bound from Lemma 1 is consistent with computed values for $n\le 50$.

6) FINAL

**UNRESOLVED**

(i) Strongest fully proved partial result obtained here.
- Exact formula $f(2^m)=m$ (Lemma 2).
- Elementary logarithmic upper bound $f(n)\ll \log n$ (Lemma 1).
- Computed values of $f(n)$ for $n\le 50$.

(ii) Exact first gap.
Establish unconditionally the existence of a limiting distribution for $f(n)/\log n$ or determine whether it is almost surely constant (both of which are only recorded conditionally in the statement).

(iii) Top 3 next moves (concrete targets).
1. Prove sharper deterministic upper and lower bounds for $f(n)$ in terms of the prime factorisation of $n$ (e.g. via multiplicativity of $\varphi$ and typical sizes of $\varphi(n)$).
2. Run large-scale computations of $f(n)$ for $n$ up to, say, $10^7$ to estimate the empirical distribution of $f(n)/\log n$.
3. Investigate the behaviour of the largest prime factor of $\varphi_k(n)$ for slowly growing $k$ via explicit bounds on prime factors of totients.

(iv) What a minimal counterexample would likely look like.
If $f(n)/\log n$ failed to have a distribution function, one would expect a positive-density subset of integers where $f(n)$ is systematically atypical (either unusually large for integers with many small prime factors, or unusually small for integers with special structure), causing persistent oscillations in the empirical distribution.


