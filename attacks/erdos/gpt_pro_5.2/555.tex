
1) FORMAL RESTATEMENT

Fix integers $n\ge 2$ and $k\ge 1$. Let $C_{2n}$ denote the (simple) cycle graph on $2n$ vertices.

For a finite simple graph $G$ and an integer $k\ge 1$, define $R(G;k)$ to be the least integer $m$ such that for every map (``$k$-edge-colouring'')
\[
\chi:E(K_m)\to [k]:=\{1,2,\dots,k\},
\]
there exists a set of vertices $U\subseteq V(K_m)$ and a colour $c\in [k]$ such that the subgraph of $K_m$ induced by $U$ contains a copy of $G$ all of whose edges have colour $c$.
(Equivalently: some colour class contains a subgraph isomorphic to $G$.)

The problem asks to determine $R(C_{2n};k)$ exactly (or at least its correct asymptotic growth), as a function of $(n,k)$.

Edge cases:
- If $k=1$, then $R(C_{2n};1)=2n$ since $K_{2n}$ contains $C_{2n}$.
- If $n=2$, then $C_{2n}=C_4\cong K_{2,2}$.

2) QUICK LITERATURE/CONTEXT CHECK

The problem statement itself records that Erd\H{o}s gave bounds (for fixed $n$ and large $k$)
\[
k^{1+\frac{1}{2n}}\ll R(C_{2n};k)\ll k^{1+\frac{1}{n-1}}.
\]
No further external results are assumed here beyond what is explicitly in the problem file.

3) ATTACK PLAN

Two complementary tracks:

(Proof/upper bounds.)
- Relate $C_{2n}$ to a larger bipartite graph such as $K_{n,n}$ (since $C_{2n}\subseteq K_{n,n}$), then use extremal estimates for $K_{n,n}$-free graphs (K\H{o}v\'ari--S\'os--Tur\'an type) to force a monochromatic $K_{n,n}$, hence a monochromatic $C_{2n}$.
- This yields a polynomial upper bound in $k$ (not sharp, but fully provable from scratch).

(Disproof/lower bounds.)
- Construct explicit $k$-edge-colourings of $K_m$ with no monochromatic $C_{2n}$, e.g. by decomposing $K_m$ into matchings via 1-factorizations so that every colour class is a matching (hence cycle-free).

4) WORK

PHASE 1 — FAST REALITY CHECK

- $k=1$: $R(C_{2n};1)=2n$.
- $n=2$, $k=2$: brute force confirms $R(C_4;2)=6$.
A witness colouring on $K_5$ with no monochromatic $C_4$ is:
red edges $\{01,02,04,12,13\}$ and blue edges $\{03,14,23,24,34\}$ (vertices $\{0,1,2,3,4\}$).
Exhaustive search over all $2^{15}=32768$ 2-colourings of $K_6$ shows every colouring contains a monochromatic $C_4$.
Hence $R(C_4;2)=6$.

PHASE 2 — LANDSCAPE (tools and why)

- 1-factorizations / proper edge-colourings: give explicit cycle-free colour classes.
- Double counting + convexity: standard extremal method for forbidding $K_{s,t}$.
- Extremal numbers $\mathrm{ex}(m,K_{n,n})$: control maximum edges without $K_{n,n}$, hence force $K_{n,n}$ in a dense colour class.
- Pigeonhole: pick densest colour class in a $k$-colouring.

Lemma 555.1 (A linear-in-$k$ lower bound via 1-factorizations).
Let $n\ge 2$ and $k\ge 1$.
Then
\[
R(C_{2n};k)\ge \max\{2n,k+1\},
\]
and if $k$ is odd then
\[
R(C_{2n};k)\ge \max\{2n,k+2\}.
\]

Proof.
The trivial bound $R(C_{2n};k)\ge 2n$ holds because $K_m$ contains no $C_{2n}$ when $m<2n$.

For the $k$-dependence, we give explicit $k$-edge-colourings with no monochromatic cycle (hence no monochromatic $C_{2n}$).

Case 1: $k$ even.
Let $m=k$. Since $m$ is even, the complete graph $K_m$ admits a 1-factorization: its edges can be partitioned into $m-1=k-1$ perfect matchings.
Colour each perfect matching with a distinct colour from $\{1,\dots,k-1\}$ (leave colour $k$ unused).
Each colour class is a matching, hence contains no cycle of any length.
Therefore this is a $k$-colouring of $K_k$ with no monochromatic $C_{2n}$, so $R(C_{2n};k)>k$, i.e. $R(C_{2n};k)\ge k+1$.

Case 2: $k$ odd.
Let $m=k+1$, which is even.
Then $K_{k+1}$ has a 1-factorization into $m-1=k$ perfect matchings.
Colour each matching with a distinct colour from $[k]$.
Again each colour class is a matching, hence acyclic, so there is no monochromatic $C_{2n}$.
Thus $R(C_{2n};k)>k+1$, i.e. $R(C_{2n};k)\ge k+2$.

Combining with the $2n$ bound gives the claim.
\qed

Lemma 555.2 (K\H{o}v\'ari--S\'os--Tur\'an bound for $K_{n,n}$).
Let $n\ge 2$ and let $G$ be a graph on $m$ vertices with no (not-necessarily-induced) subgraph isomorphic to $K_{n,n}$.
Then
\[
e(G)\le \frac12\Big((n-1)^{1/n} m^{2-1/n}+(n-1)m\Big).
\]

Proof.
For an $n$-set $S\subseteq V(G)$, let $c(S)$ denote the number of common neighbours of all vertices in $S$.
If $c(S)\ge n$, then $S$ together with any $n$ of its common neighbours forms a $K_{n,n}$, contradicting the hypothesis.
Therefore $c(S)\le n-1$ for all $S$.

Double count the set of pairs $(S,v)$ where $S\in\binom{V(G)}{n}$ and $v$ is adjacent to every vertex in $S$.
Counting by $S$ gives
\[
\sum_{S\in\binom{V(G)}{n}} c(S)\le (n-1)\binom{m}{n}.
\tag{1}
\]
Counting by $v$ gives $\sum_{v\in V(G)}\binom{d(v)}{n}$, since each vertex $v$ contributes one for each $n$-subset of its neighbourhood.
Hence
\[
\sum_{v\in V(G)}\binom{d(v)}{n}\le (n-1)\binom{m}{n}.
\tag{1'}
\]

We now lower-bound the left-hand side in terms of the average degree.
For an integer $d\ge 0$ we have $\binom{d}{n}=0$ when $d<n$, and for $d\ge n$,
\[
\binom{d}{n}=\frac{d(d-1)\cdots(d-n+1)}{n!}\ge \frac{(d-n+1)^n}{n!}
\]
since each factor $d-j\ge d-n+1$.
Thus for every integer $d\ge 0$,
\[
\binom{d}{n}\ge \frac{(d-n+1)_+^n}{n!},\qquad (x)_+:=\max\{x,0\}.
\]
Apply this with $d=d(v)$ and define
\[
X_v:=(d(v)-n+1)_+\ge 0.
\]
Then
\[
\sum_{v\in V(G)}\binom{d(v)}{n}\ge \frac{1}{n!}\sum_{v\in V(G)} X_v^n.
\]
Because $x\mapsto x^n$ is convex on $[0,\infty)$, Jensen's inequality yields
\[
\frac{1}{m}\sum_v X_v^n\ge \left(\frac{1}{m}\sum_v X_v\right)^n.
\]
Moreover $X_v\ge d(v)-n+1$ for each $v$, hence
\[
\sum_v X_v\ge \sum_v (d(v)-n+1)=2e(G)-m(n-1).
\]
Putting these together gives
\[
\sum_{v\in V(G)}\binom{d(v)}{n}
\ge \frac{m}{n!}\left(\frac{2e(G)}{m}-(n-1)\right)_+^n.
\tag{2}
\]

Insert (2) into (1), and use the bound $\binom{m}{n}\le \frac{m^n}{n!}$ to obtain
\[
\frac{m}{n!}\left(\frac{2e(G)}{m}-(n-1)\right)_+^n\le (n-1)\frac{m^n}{n!}.
\]
Cancel $m/n!$ (for $m\ge 1$) to get
\[
\left(\frac{2e(G)}{m}-(n-1)\right)_+^n\le (n-1)m^{n-1}.
\]
Taking $n$-th roots (both sides are nonnegative once we replace the left by its positive part) yields
\[
\frac{2e(G)}{m}-(n-1)\le (n-1)^{1/n} m^{(n-1)/n}.
\]
Rearranging gives
\[
2e(G)\le (n-1)^{1/n} m^{2-1/n}+(n-1)m,
\]
which is the desired inequality.
\qed

Lemma 555.3 (A coarse polynomial upper bound for $R(C_{2n};k)$).
For fixed $n\ge 2$ and all $k\ge 2$,
\[
R(C_{2n};k)\le \left(4(n-1)^{1/n}k\right)^n+1.
\]

Proof.
Let
\[
C:=4(n-1)^{1/n},\qquad m:=\left\lceil (Ck)^n\right\rceil.
\]
Consider any $k$-edge-colouring of $K_m$.
Among the $k$ colours, pick a colour class with the maximum number of edges; call its graph $G$.
Then
\[
e(G)\ge \frac{1}{k}\binom{m}{2}=\frac{m(m-1)}{2k}.
\tag{3}
\]

We show that $e(G)$ exceeds the maximum possible number of edges in a $K_{n,n}$-free graph on $m$ vertices, hence $G$ contains a $K_{n,n}$, and therefore contains $C_{2n}$ (since $C_{2n}\subseteq K_{n,n}$).

By Lemma 555.2, if $G$ were $K_{n,n}$-free then
\[
e(G)\le \frac12\Big((n-1)^{1/n} m^{2-1/n}+(n-1)m\Big).
\tag{4}
\]
We now show that the right-hand side of (4) is at most $m^2/(4k)$, while (3) gives $e(G)>m^2/(4k)$ for $m\ge 3$.

First term:
since $m^{1/n}\ge Ck=4(n-1)^{1/n}k$, we have
\[
(n-1)^{1/n} m^{2-1/n}=(n-1)^{1/n}\frac{m^2}{m^{1/n}}\le (n-1)^{1/n}\frac{m^2}{4(n-1)^{1/n}k}=\frac{m^2}{4k}.
\]
Second term:
because $m\ge (Ck)^n=4^n(n-1)k^n$, we have
\[
(n-1)m\le \frac{m^2}{4k}
\]
since $(n-1)m\le m\cdot \frac{m}{4k}$ is equivalent to $m\ge 4k(n-1)$, and indeed
\[
m\ge 4^n(n-1)k^n\ge 4k(n-1)\quad\text{for }n\ge 2,k\ge 1.
\]
Therefore the sum in (4) is at most $\frac12(\frac{m^2}{4k}+\frac{m^2}{4k})=\frac{m^2}{4k}$.

On the other hand, for $m\ge 3$ we have $m-1>m/2$, so
\[
\frac{m(m-1)}{2k}>\frac{m^2}{4k}.
\]
Thus (3) contradicts (4), so $G$ is not $K_{n,n}$-free.
Hence $G$ contains a $K_{n,n}$ and therefore a $C_{2n}$ in that colour.

This shows that $K_m\to (C_{2n})_k$, i.e. $R(C_{2n};k)\le m\le (Ck)^n+1$.
\qed

5) VERIFICATION

- Lemma 555.1: each colour class in the 1-factorization colouring is a matching, hence acyclic; thus it forbids $C_{2n}$ for all $n\ge 2$. The parity split (odd/even $k$) matches when $K_m$ is 1-factorizable.

- Lemma 555.2: checked the double count identity $\sum_v \binom{d(v)}{n}=\sum_{S\in\binom{V}{n}} c(S)$ and the key implication ``$c(S)\ge n$ creates $K_{n,n}$''.
The Jensen step is valid because $x\mapsto x^n$ is convex on $[0,\infty)$ and we apply it to $(d(v)-n+1)_+$ implicitly (negative values only weaken the lower bound).

- Lemma 555.3: the sufficient conditions $m^{1/n}\ge 4(n-1)^{1/n}k$ and $m\ge 8k(n-1)$ were chosen to make the algebra one-sided with slack; the final bound is therefore correct though far from sharp.

6) FINAL

**UNRESOLVED**

(i) Strongest proved partial result:
- Explicit constructions give $R(C_{2n};k)\ge \max\{2n,k+1\}$ for all $k$, and $\ge \max\{2n,k+2\}$ for odd $k$ (Lemma 555.1).
- A coarse extremal argument gives $R(C_{2n};k)=O_n(k^n)$ (Lemma 555.3).

(ii) First gap (crisp):
Determine the correct asymptotic exponent of $k$ in $R(C_{2n};k)$ (for fixed $n$), i.e. close the gap between the known bounds
$k^{1+\frac{1}{2n}}\ll R(C_{2n};k)\ll k^{1+\frac{1}{n-1}}$ stated in the problem.

(iii) Top 3 next moves:
1. Prove/derive sharp asymptotics for $\mathrm{ex}(N,C_{2n})$ and insert into the standard ``densest colour class'' argument to improve the upper bound.
2. Build improved lower-bound colourings using $C_{2n}$-free pseudorandom graphs in each colour class (beyond matchings) to push $m$ above linear in $k$.
3. Compute exact values for small $(n,k)$ via SAT/backtracking to guess the true exponent pattern and identify tight constructions.

(iv) Minimal counterexample structure:
A smallest $(n,k)$ for which conjectured asymptotics/exponents fail would correspond to a $k$-colouring of $K_m$ in which every colour class is $C_{2n}$-free yet globally dense; likely each colour class resembles an extremal $C_{2n}$-free graph (bipartite/pseudorandom) with edge density about $m^{1-1/n}$.

