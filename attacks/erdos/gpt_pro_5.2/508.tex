% Erdos Problem #508

1) FORMAL RESTATEMENT

Consider the graph $G$ with vertex set $\mathbb R^2$ and an edge between two points if their Euclidean distance is exactly $1$.
The \emph{chromatic number of the plane} $\chi$ is
\[
\chi := \chi(\mathbb R^2) := \min\big\{k\in\mathbb N:\ \exists\ c:\mathbb R^2\to\{1,\dots,k\}\text{ such that }\lVert x-y\rVert=1\Rightarrow c(x)\ne c(y)\big\}.
\]
The problem asks to determine $\chi$.

2) QUICK LITERATURE/CONTEXT CHECK

The problem statement records that $\chi\ge 3$ (equilateral triangle), that finite unit-distance graphs such as the Moser spindle imply $\chi\ge 4$, and that the best known bounds currently are $5\le \chi\le 7$ (lower bound due to de Grey, as quoted).

Below I prove the classical bounds $4\le \chi\le 7$ in a fully explicit way (using an explicit unit-distance graph and an explicit 7-coloring).

3) ATTACK PLAN

\emph{Lower bound:} exhibit an explicit finite unit-distance graph with chromatic number $4$ (the Moser spindle) and prove it is not 3-colorable.

\emph{Upper bound:} give an explicit 7-coloring of the plane based on a regular hexagon tiling with sufficiently small diameter.

4) WORK

Lemma 508.1 (Explicit Moser spindle; forces 4 colors).
There exist seven points in $\mathbb R^2$ with eleven unit-distance edges forming a graph that is not 3-colorable. Hence $\chi\ge 4$.

\emph{Proof.}
\underline{Step 1: Coordinates.}
Let
\[
O=(0,0),\quad U=(1,0),\quad V=\left(\tfrac12,\tfrac{\sqrt3}{2}\right),\quad C=U+V=\left(\tfrac32,\tfrac{\sqrt3}{2}\right).
\]
Then $\lVert O-U\rVert=\lVert O-V\rVert=\lVert U-V\rVert=\lVert U-C\rVert=\lVert V-C\rVert=1$, so $O,U,V,C$ form a rhombus with angles $60^\circ$ and $120^\circ$, plus the short diagonal $UV$.

Let $\alpha:=\arccos(5/6)$ and let $R_{\alpha}$ denote rotation about the origin by angle $\alpha$.
Define $D:=R_{\alpha}(C)$; then $\lVert D\rVert=\lVert C\rVert=\sqrt3$ and $\lVert C-D\rVert=1$ (because
$\lVert C-D\rVert^2=2\lVert C\rVert^2(1-\cos\alpha)=6(1-5/6)=1$).
Let $\phi:=\arg(D)$ and define
\[
A:=(\cos(\phi+\pi/6),\ \sin(\phi+\pi/6)),\qquad B:=(\cos(\phi-\pi/6),\ \sin(\phi-\pi/6)).
\]
Then $\lVert A\rVert=\lVert B\rVert=1$ and the angle between $A$ and $B$ is $\pi/3$, so $\lVert A-B\rVert=1$.
Moreover,
\[
A+B = 2\cos(\pi/6)\,(\cos\phi,\sin\phi)=\sqrt3\,(\cos\phi,\sin\phi)=D,
\]
so $D=A+B$ and therefore $\lVert A-D\rVert=\lVert B-D\rVert=1$.
Thus $O,A,B,D$ form a second $60/120$ rhombus plus short diagonal $AB$.

Now define a finite graph $H$ on vertex set
\[\{O,U,V,C,A,B,D\}
\]
with edge set consisting of the following eleven pairs:
\[
\{O,U\},\{O,V\},\{U,V\},\{U,C\},\{V,C\},\qquad
\{O,A\},\{O,B\},\{A,B\},\{A,D\},\{B,D\},\qquad
\{C,D\}.
\]
By the distance checks above, every listed edge is a unit-distance pair.
So $H$ is a unit-distance graph in the plane.

\underline{Step 2: $H$ is not 3-colorable.}
Consider the induced subgraph on $\{O,U,V,C\}$.
This is the graph $K_4$ with the edge $OC$ missing (all other pairs among these four vertices are edges).
In any proper 3-coloring of this subgraph, the adjacent vertices $U$ and $V$ must have different colors.
Then $O$ is adjacent to both $U$ and $V$, so $O$ cannot use either of their colors and must use the third color.
Similarly, $C$ is adjacent to both $U$ and $V$, so $C$ also must use the third color.
Hence in any 3-coloring of $\{O,U,V,C\}$ we must have $\mathrm{col}(O)=\mathrm{col}(C)$.

The same argument applied to the $K_4$-minus-edge subgraph on $\{O,A,B,D\}$ (missing edge $OD$) shows that in any 3-coloring we must have $\mathrm{col}(O)=\mathrm{col}(D)$.
Therefore $\mathrm{col}(C)=\mathrm{col}(D)$ in any 3-coloring.
But $C$ and $D$ are adjacent in $H$ (edge $CD$), contradiction.
So $H$ is not 3-colorable, and thus needs at least 4 colors.

Since $H$ is a unit-distance graph in the plane, any proper coloring of $\mathbb R^2$ with no monochromatic unit-distance pair restricts to a proper coloring of $H$.
Therefore $\chi\ge \chi(H)\ge 4$.
\qed

Lemma 508.2 (Explicit 7-coloring by hexagon tiling).
There exists a coloring $c:\mathbb R^2\to\{0,1,\dots,6\}$ such that any two points at distance $1$ have different colors. Hence $\chi\le 7$.

\emph{Proof.}
\underline{Step 1: Hexagon tiling and parameters.}
Fix a regular hexagon tiling of the plane by congruent regular hexagons of side length
\[s:=\frac{2}{5}.
\]
For a regular hexagon of side length $s$, the diameter (maximum distance between two points in the hexagon) is $2s=4/5<1$.
Therefore any two points \emph{within the same hexagon} have distance $<1$, and in particular cannot be at distance exactly $1$.
So it is safe to give each hexagon a single color.

Let $r:=s$ be the circumradius of each hexagon. The distance between centers of two hexagons that share a side equals twice the inradius, i.e.
\[t:=2r\cos(\pi/6)=2s\cdot \frac{\sqrt3}{2}=\sqrt3\,s.
\]
Thus the set of hexagon centers forms a triangular lattice with nearest-neighbor distance $t$.
Choose lattice basis vectors $\mathbf e_1,\mathbf e_2\in\mathbb R^2$ with $\lVert\mathbf e_1\rVert=\lVert\mathbf e_2\rVert=t$ and angle $60^\circ$ between them.
Every hexagon center can be written uniquely as $a\mathbf e_1+b\mathbf e_2$ with integers $a,b$.

\underline{Step 2: Define the 7-coloring on hexagons.}
Color the hexagon centered at $a\mathbf e_1+b\mathbf e_2$ by
\[c(a,b):= (a+3b)\bmod 7\ \in\{0,1,\dots,6\}.
\]
Extend this to a coloring of the whole plane by coloring each point with the color of the hexagon containing it.
(Points on hexagon boundaries may be assigned arbitrarily among the adjacent hexagons; this does not affect correctness because boundaries have no interior and the distance-1 constraint is closed.)

\underline{Step 3: Same-colored hexagon centers are far apart.}
Suppose two hexagon centers have the same color. Then their difference in lattice coordinates is a nonzero integer pair $(\Delta a,\Delta b)\ne (0,0)$ satisfying
\[\Delta a + 3\Delta b \equiv 0\pmod 7.
\]
Write $\Delta a+3\Delta b=7k$ for some integer $k$.
Then
\[
\Delta a^2+\Delta b^2+\Delta a\Delta b
=(7k-3\Delta b)^2+\Delta b^2+(7k-3\Delta b)\Delta b
=49k^2-35k\Delta b+7\Delta b^2
=7(7k^2-5k\Delta b+\Delta b^2).
\]
In particular, $Q:=\Delta a^2+\Delta b^2+\Delta a\Delta b$ is a positive integer multiple of $7$, hence $Q\ge 7$.
The squared Euclidean distance between the two centers is
\[\lVert \Delta a\mathbf e_1+\Delta b\mathbf e_2\rVert^2 = t^2\,Q \ge 7t^2,
\]
so the center-to-center distance is at least $\sqrt7\,t$.

\underline{Step 4: No unit-distance monochromatic pair.}
Take two points $x,y$ of the same color.

- If $x,y$ lie in the same hexagon, then $\lVert x-y\rVert\le 2s=4/5<1$, so they are not at distance $1$.

- If $x,y$ lie in two different hexagons of the same color, let $z_x,z_y$ be the centers of those hexagons.
Then $\lVert z_x-z_y\rVert\ge \sqrt7\,t$ by Step 3.
Also $\lVert x-z_x\rVert\le s$ and $\lVert y-z_y\rVert\le s$ because $s$ is the circumradius.
By the triangle inequality,
\[
\lVert x-y\rVert\ge \lVert z_x-z_y\rVert - \lVert x-z_x\rVert - \lVert y-z_y\rVert \ge \sqrt7\,t -2s.
\]
Using $t=\sqrt3\,s$, this lower bound equals
\[\sqrt7\,\sqrt3\,s-2s = (\sqrt{21}-2)s.
\]
With $s=2/5$, we have $(\sqrt{21}-2)s>1$ (numerically $(\sqrt{21}-2)\cdot 2/5\approx 1.0335$).
Therefore any two same-colored points in different hexagons are at distance strictly greater than $1$.

In all cases, two same-colored points cannot be at distance exactly $1$.
So this defines a proper 7-coloring for the unit-distance graph of the plane, proving $\chi\le 7$.
\qed

FAST REALITY CHECK (computational sanity for Lemma 508.1).
Using the explicit coordinate construction above, I verified numerically (to machine precision) that:
- all 11 listed edges have distance exactly $1$;
- there are no additional unit-distance pairs among the 7 points;
- the resulting graph is not 3-colorable but is 4-colorable.

5) VERIFICATION

\emph{Lemma 508.1:} The forcing argument uses only local constraints in $K_4$ minus one edge, and the contradiction is the presence of edge $CD$.

\emph{Lemma 508.2:} The key quantitative step is the lower bound $\lVert x-y\rVert>1$ for points in distinct same-colored hexagons; the calculation is explicit and uses the exact choice $s=2/5$.

6) FINAL
UNRESOLVED

(i) \emph{Strongest proved partial result here:} A fully explicit proof of the classical bounds $4\le \chi\le 7$ (Lemmas 508.1--508.2). (The problem statement reports the stronger current lower bound $\chi\ge 5$.)

(ii) \emph{First gap (crisp):} Determine whether $\chi$ equals $5$, $6$, or $7$ (the statement in the problem file gives $5\le \chi\le 7$).

(iii) \emph{Top 3 next moves:}
1. Search for smaller/structured finite unit-distance graphs with chromatic number $6$ or $7$, or prove they cannot exist under certain geometric constraints.
2. Improve the upper bound by constructing a periodic coloring with fewer than $7$ colors and rigorously checking the unit-distance constraint.
3. Investigate whether known $5$-chromatic unit-distance graphs can be extended or combined to force $6$ colors, or alternatively whether there is a measurable coloring with $5$ colors.

(iv) \emph{Minimal counterexample structure:} If $\chi\ne 5$, then either (a) there exists a finite unit-distance graph requiring $6$ or $7$ colors, or (b) every unit-distance graph is 5-colorable but no 5-coloring extends to the whole plane; understanding which of these can happen (and under what regularity assumptions on the coloring) seems central.
