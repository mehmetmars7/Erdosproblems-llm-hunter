\section*{Erd\H{o}s Problem \#104}
\addcontentsline{toc}{section}{Erd\H{o}s Problem \#104}

\subsection*{FORMAL RESTATEMENT}
\noindent\textbf{Verbatim problem statement (from file).}
\emph{``Given $n$ points in $\R^2$ the number of distinct unit circles containing at least three points is $o(n^2)$.''}

\medskip
\noindent\textbf{Definitions and conventions.}
Let $P\subset\R^2$ with $|P|=n$.
A \emph{unit circle} means a Euclidean circle of radius $1$.
Let $C(P)$ be the number of distinct unit circles $\Gamma$ such that $|\Gamma\cap P|\ge 3$.
The conjectured asymptotic statement is:
\[\sup_{|P|=n} C(P) = o(n^2)\quad\text{as }n\to\infty.
\]

\subsection*{QUICK LITERATURE/CONTEXT CHECK}
No external browsing was used.
The file states:
\begin{itemize}[leftmargin=2em]
\item Erd\H{o}s proved $C(P)$ can be $\gg n$.
\item A simple pair-counting argument gives $C(P)=O(n^2)$, with a corrected constant upper bound $\frac{n(n-1)}{3}$.
\item A construction (Elekes) gives $\gg n^{3/2}$ such circles.
\end{itemize}
Below I re-prove the $\frac{n(n-1)}{3}$ upper bound and give an explicit elementary construction with $\Omega(n)$ circles.

\subsection*{ATTACK PLAN}
\begin{itemize}[leftmargin=2em]
\item \textbf{Upper bound:} Double count point-pairs vs circles, using that each pair lies on at most two unit circles and each qualifying circle contains at least three pairs.
\item \textbf{Lower bound:} Exhibit an explicit family of point sets with many distinct unit circles, and verify the circle count exactly.
\item \textbf{Hard part (not solved here):} Improve the $O(n^2)$ upper bound to $o(n^2)$.
\end{itemize}

\subsection*{WORK}
\noindent\textbf{FAST REALITY CHECK (tiny $n$).}
\begin{itemize}[leftmargin=2em]
\item $n<3$: $C(P)=0$.
\item $n=3$: $C(P)$ can be $1$ if the three points lie on a unit circle.
\end{itemize}

\begin{lemma}[Pair counting upper bound $\le n(n-1)/3$]
For any set $P$ of $n$ points in $\R^2$,
\[C(P)\le \frac{n(n-1)}{3}.
\]
\end{lemma}

\begin{proof}
Let $\mathcal{G}$ be the set of all unit circles $\Gamma$ with $|\Gamma\cap P|\ge 3$.
So $|\mathcal{G}|=C(P)$.
For each such circle $\Gamma$, the set $\Gamma\cap P$ has size at least $3$, hence contains at least $\binom{3}{2}=3$ unordered pairs of points from $P$.
Thus the total number of incidences between circles in $\mathcal{G}$ and point-pairs in $\binom{P}{2}$ is at least $3C(P)$.

On the other hand, fix an unordered pair $\{p,q\}\subset P$ with $p\ne q$.
How many unit circles can pass through both $p$ and $q$?
If $\|p-q\|>2$ then none.
If $\|p-q\|=2$ then there is exactly one unit circle through $p,q$ (with $pq$ as a diameter).
If $\|p-q\|<2$ then there are exactly two unit circles through $p,q$ (their centers are the two intersection points of circles of radius 1 around $p$ and around $q$).
In all cases, at most two unit circles pass through a fixed pair.
Therefore the total number of (circle, pair) incidences is at most $2\binom{n}{2}=n(n-1)$.

Combine the lower and upper bounds on the same incidence count:
\[3C(P)\le n(n-1)\quad\Rightarrow\quad C(P)\le \frac{n(n-1)}{3}.
\]
\end{proof}

\begin{proposition}[Explicit construction with $\Omega(n)$ circles]
For every integer $m\ge 3$ there exists a set $P\subset\R^2$ of $n=2m-1$ points such that $C(P)\ge m-2=\frac{n-3}{2}$.
In particular, $\sup_{|P|=n}C(P)\gg n$.
\end{proposition}

\begin{proof}
Fix $m\ge 3$ and define
\[P:=\{(i,0): i=0,1,\dots,m\}\ \cup\ \{(i,1): i=1,2,\dots,m-1\}.
\]
Then $|P|=(m+1)+(m-1)=2m$; to match $n=2m-1$ one may remove one endpoint point, but we keep $2m$ for simplicity (this only changes constants).
For each integer $i$ with $0\le i\le m-2$, consider the unit circle $\Gamma_i$ with diameter endpoints $(i,0)$ and $(i+2,0)$.
Its center is at $(i+1,0)$ and radius is $1$.
By Thales' theorem, a point $r$ lies on $\Gamma_i$ iff the angle $\angle (i,0)\,r\,(i+2,0)$ is a right angle.
Directly, check that $(i+1,1)$ lies on $\Gamma_i$ because its distance to the center $(i+1,0)$ is $1$.
Thus $\Gamma_i$ contains the three points
\[(i,0),\ (i+2,0),\ (i+1,1)\in P.
\]
So each $\Gamma_i$ contributes to $C(P)$.

It remains to check that the circles $\Gamma_i$ are \emph{distinct}.
But their centers are $(i+1,0)$, which are distinct for different $i$.
Therefore $\Gamma_i\ne \Gamma_j$ for $i\ne j$.
Hence
\[C(P)\ge |\{\Gamma_i:0\le i\le m-2\}|=m-1.
\]
If we adjust $P$ to have exactly $2m-1$ points by removing one point (say $(m,0)$), then the same argument shows at least $m-2$ circles survive, giving the stated $\Omega(n)$ bound.
\end{proof}

\subsection*{VERIFICATION}
\begin{itemize}[leftmargin=2em]
\item In the upper bound, the only geometric fact needed is: through two points there are at most two unit circles. This was checked by considering intersection of radius-1 circles around each point.
\item In the construction, for each $i$ the circle $\Gamma_i$ is well-defined because the diameter endpoints are distance $2$ apart.
\item Distinctness of circles follows from distinct centers.
\end{itemize}

\subsection*{FINAL}
\textbf{UNRESOLVED.}
\begin{enumerate}[label=(\roman*),leftmargin=2.5em]
\item \textbf{Strongest proved partial result here:}
For any $n$ points, $C(P)\le n(n-1)/3$ (Lemma), and there are explicit configurations with $C(P)\ge c n$ for an absolute $c>0$ (Proposition).
\item \textbf{First gap (crisp):}
Improve the general upper bound from $O(n^2)$ to $o(n^2)$: prove that for every $\varepsilon>0$ and all sufficiently large $n$, any $n$-point set has at most $\varepsilon n^2$ unit circles containing $\ge 3$ points.
\item \textbf{Top 3 next moves (concrete):}
\begin{enumerate}[label=(\alph*),leftmargin=2.5em]
\item Attempt to adapt point-circle incidence bounds (e.g. polynomial partitioning for circles) to bound the number of ``3-rich'' unit circles.
\item Study structured constructions (lattices, Cartesian products) to see whether $n^{3/2}$ is a genuine barrier; implement a computational search for $n\le 50$ to estimate growth of the maximum.
\item Prove intermediate statements: for example, show $C(P)=o(n^2)$ under mild ``general position'' assumptions, then try to remove assumptions.
\end{enumerate}
\item \textbf{What a minimal counterexample would likely look like:}
A family $P_n$ with $|P_n|=n$ and $C(P_n)\ge c n^2$ for some fixed $c>0$, meaning a positive fraction of all point pairs lie on unit circles that are 3-rich; such a family would likely require strong arithmetic structure producing many repeated unit distances.
\end{enumerate}

