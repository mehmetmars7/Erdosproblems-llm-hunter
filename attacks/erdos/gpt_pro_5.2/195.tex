% ------------------------------------------------------------
% Erdos problem 195
% ------------------------------------------------------------
\section*{Erd\H{o}s problem 195}

\subsection*{1) FORMAL RESTATEMENT}
A \emph{permutation of $\mathbb{Z}$} is a bijection $\pi:\mathbb{Z}\to\mathbb{Z}$, viewed as the bi-infinite sequence $(\pi(n))_{n\in\mathbb{Z}}$.
A \emph{monotone $k$-term arithmetic progression in $\pi$} means there exist indices $i_1<i_2<\dots<i_k$ such that the values
$\pi(i_1),\dots,\pi(i_k)$ form a $k$-term arithmetic progression and are monotone (either increasing or decreasing) in that index order.
Let $K^*$ be the largest $k$ such that every permutation of $\mathbb{Z}$ must contain such a monotone $k$-AP.
Determine $K^*$.

\subsection*{2) QUICK LITERATURE/CONTEXT CHECK}
Problem text states: Geneson proved $K^*\le 5$ and Adenwalla improved to $K^*\le 4$. Thus $K^*\in\{3,4\}$ is plausible.

\subsection*{3) ATTACK PLAN}
Provide reductions and invariances:
(1) reduce from permutations of $\mathbb{Z}$ to permutations of $\mathbb{N}$ via subsequences,
(2) show symmetry operations preserve existence/nonexistence of monotone APs.
Then record computational sanity checks in finite truncations.

\subsection*{4) WORK}

\paragraph{Lemma 4.1 (Positive subsequence reduction).}
Let $\pi$ be a permutation of $\mathbb{Z}$ and let $\pi^+$ be the subsequence of $\pi$ consisting of the positive integers (in the order they occur).
Then $\pi^+$ is a permutation of $\mathbb{N}$ (indexed by $\mathbb{N}$). Any monotone $k$-AP appearing in $\pi^+$ also appears in $\pi$.
\textit{Proof.}
Each positive integer appears exactly once in $\pi$, so the subsequence lists each element of $\mathbb{N}$ exactly once, hence is a permutation of $\mathbb{N}$.
If $\pi^+$ contains indices $j_1<\dots<j_k$ giving a monotone $k$-AP in values, these correspond to indices $i_1<\dots<i_k$ in the original sequence $\pi$
with the same values in the same order, hence the same monotone $k$-AP occurs in $\pi$. \hfill$\square$

\paragraph{Lemma 4.2 (Symmetry invariance).}
If $\pi$ contains a monotone $k$-AP, then so do:
(a) the reversed permutation $\pi'(n)=\pi(-n)$, and
(b) the negated permutation $\pi''(n)=-\pi(n)$.
Conversely, if $\pi$ avoids monotone $k$-APs then so do $\pi'$ and $\pi''$.
\textit{Proof.}
Reversal replaces the index order by its reverse but preserves the relative order type (increasing becomes decreasing).
Negation reverses the numerical order of values (increasing AP becomes decreasing AP).
In both cases, the arithmetic-progression property of the values is preserved (negation and index reversal are affine).
Therefore existence/nonexistence of monotone $k$-APs is invariant. \hfill$\square$

\subsection*{FAST REALITY CHECK (finite evidence)}
Finite permutations of $\{1,\dots,n\}$ can avoid monotone 4-APs for moderately large $n$ (heuristic search finds examples for $n=20,25,30$),
so forcing a monotone 4-AP in an \emph{infinite} permutation is not a simple finite obstruction.

\subsection*{6) FINAL}
\textbf{UNRESOLVED}

(i) Strongest proved partial results: reduction from $\mathbb{Z}$ to $\mathbb{N}$ subsequences (Lemma 4.1) and symmetry invariances (Lemma 4.2).

(ii) First gap: decide whether every permutation of $\mathbb{Z}$ must contain a monotone 4-AP (i.e. whether $K^*=4$ or $3$).

(iii) Top 3 next moves:
1. Try to adapt the known unavoidable 3-AP arguments (DEGS77 for $\mathbb{N}$) to 4-APs.
2. Attempt to explicitly construct a permutation of $\mathbb{Z}$ avoiding monotone 4-AP (would show $K^*=3$).
3. Search for extension obstructions: characterize finite avoiders that cannot be extended.

(iv) Minimal counterexample structure (if $K^*=3$): an explicit infinite permutation with the property that for every 4-term AP
$a,a+d,a+2d,a+3d$, the order of their positions is not monotone (neither increasing nor decreasing).

