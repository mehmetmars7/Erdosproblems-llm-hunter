
\noindent\textbf{FORMAL RESTATEMENT.}

For $n\ge 1$, let $m(n)$ be the maximum number of edges in a simple undirected graph $G$ on $n$ vertices with the following forbidden configuration:

There do not exist two cycles $C_1,C_2\subseteq G$ such that
\[
V(C_1)=V(C_2)\quad\text{and}\quad E(C_1)\cap E(C_2)=\varnothing.
\]
(Each $C_i$ is a simple cycle; equality of vertex sets means each is a Hamilton cycle on that vertex set.)

Question: determine/estimate the growth of $m(n)$.

\bigskip
\noindent\textbf{QUICK LITERATURE/CONTEXT CHECK.}

I will not use any literature beyond what is explicitly quoted in the problem statement: a construction with $\gg n\log\log n$ edges and an upper bound $n(\log n)^{O(1)}$ are mentioned there.

\bigskip
\noindent\textbf{ATTACK PLAN.}

\emph{Proof-track ideas.}
(1) Identify small unavoidable subgraphs: show that high density forces two edge-disjoint cycles on some vertex set.
(2) Try Tur\'an-type reductions: show the forbidden configuration implies forbidding some complete graph or other dense pattern, yielding an $O(n^2)$ upper bound.

\emph{Disproof/construction track ideas.}
(1) Explicit sparse constructions with no $4$-regular subgraphs (as referenced in the problem statement) plausibly avoid the forbidden configuration.
(2) For small $n$, brute force compute $m(n)$ exactly to understand the finite extremals.

Here I only prove easy structural upper bounds (quadratic) and compute $m(n)$ for $n\le 7$.

\bigskip
\noindent\textbf{WORK.}

\noindent\textbf{Lemma 585.1 (vacuity for $n\le 4$ and exact small $n$ behavior).}
If $n\le 4$, then no graph on $n$ vertices contains two edge-disjoint cycles with the same vertex set.
Consequently $m(n)=\binom n2$ for $n\le 4$.
If $n=5$, then the forbidden configuration occurs if and only if $G\cong K_5$, hence $m(5)=9$.

\noindent\emph{Proof.}
A cycle uses at least $3$ vertices.
If $n\le 4$, then the only possible vertex sets of cycles have size $3$ or $4$.
On $3$ vertices the complete graph has only $3$ edges, so it cannot contain two edge-disjoint $3$-cycles (which would require $6$ edges).
On $4$ vertices there are at most $6$ edges, while two edge-disjoint $4$-cycles would require $8$ distinct edges; impossible.
Thus the forbidden configuration cannot occur for $n\le 4$, so the complete graph $K_n$ is allowed and $m(n)=\binom n2$.

For $n=5$, any cycle with vertex set of size $5$ is a $5$-cycle using $5$ edges.
Two edge-disjoint $5$-cycles on the same vertex set would require $10$ distinct edges.
But $10$ is the total number of edges on $5$ vertices, so such a pair exists if and only if the induced subgraph on those $5$ vertices is $K_5$, i.e. $G$ contains a $K_5$ subgraph. Since $n=5$, that means $G\cong K_5$.
Therefore, among $5$-vertex graphs, exactly $K_5$ is forbidden, so the maximum allowed edge count is $9$.
\qed

\bigskip
\noindent\textbf{Lemma 585.2 ($K_5$ forces the forbidden configuration; Tur\'an bound for $K_5$-free graphs).}
The complete graph $K_5$ contains two edge-disjoint cycles with the same vertex set.
Therefore, any graph $G$ counted by $m(n)$ is $K_5$-free.
Moreover, any $K_5$-free graph on $n$ vertices has at most
\[
\left\lfloor \frac{3n^2}{8}\right\rfloor
\]
edges.

\noindent\emph{Proof (two parts).}

\emph{Part A: $K_5$ contains two edge-disjoint $5$-cycles on the same vertex set.}
Label the vertices of $K_5$ by $0,1,2,3,4$.
Consider the two cycles
\[
C_1: 0-1-2-3-4-0,\qquad
C_2: 0-2-4-1-3-0.
\]
Their edge sets are
\[
E(C_1)=\{01,12,23,34,40\},\qquad
E(C_2)=\{02,24,41,13,30\}.
\]
These are disjoint sets of edges, and both cycles use all five vertices.
Hence $K_5$ contains the forbidden configuration.
Thus any graph without the forbidden configuration must be $K_5$-free.

\emph{Part B: $K_5$-free graphs have at most $\lfloor 3n^2/8\rfloor$ edges (a self-contained Tur\'an argument for $r=4$).}
Let $G$ be a $K_5$-free graph on $n$ vertices with the maximum possible number of edges.
We first show that $G$ must be complete multipartite, and then that it has at most $4$ parts.

\emph{Step 1 (Zykov symmetrization to get a complete multipartite graph).}
Suppose there exist two nonadjacent vertices $u\ne v$ with different neighborhoods: $N(u)\ne N(v)$.
Construct a new graph $G'$ on the same vertex set by deleting all edges incident to $u$ and then joining $u$ to exactly the vertices in $N(v)$.
Equivalently, $u$ becomes a ``clone'' of $v$ (they have identical neighborhoods).

Claim: $e(G')\ge e(G)$.
Indeed, the new degree of $u$ equals $\deg_G(v)$.
Since $u$ and $v$ were nonadjacent and neighborhoods differ, we have either $\deg_G(v)>\deg_G(u)$ or $\deg_G(v)<\deg_G(u)$.
If $\deg_G(v)\ge \deg_G(u)$ then cloning cannot decrease edges.
If $\deg_G(v)<\deg_G(u)$ we instead clone $v$ to $u$.
In either case, we can perform a cloning operation on some nonadjacent pair without decreasing the edge count.

Claim: $G'$ remains $K_5$-free.
Any $K_5$ in $G'$ that does not include $u$ would already be a $K_5$ in $G$.
If a $K_5$ in $G'$ includes $u$, then the other four vertices must all lie in $N_{G'}(u)=N_G(v)$.
Those four vertices form a $K_4$ in $G'[N_G(v)]$, hence also in $G[N_G(v)]$.
But then in $G$, the vertex $v$ together with that $K_4$ would form a $K_5$, contradiction.
So $G'$ is $K_5$-free.

By repeating this operation, we obtain a $K_5$-free graph with at least as many edges as $G$ in which every pair of nonadjacent vertices have identical neighborhoods.
In such a graph, ``nonadjacent'' is an equivalence relation: if $x$ is nonadjacent to $y$, then $N(x)=N(y)$, and then any $z$ nonadjacent to $y$ has $N(z)=N(y)$, hence $z$ is also nonadjacent to $x$.
Thus the vertex set partitions into independent sets (equivalence classes), and vertices from different classes are adjacent to each other.
Therefore the graph is complete multipartite.

\emph{Step 2 (at most four parts).}
If a complete multipartite graph has at least $5$ parts, then choosing one vertex from each of five parts yields a $K_5$ (since all cross edges are present).
Hence a $K_5$-free complete multipartite graph has at most $4$ parts.
So $G$ is a complete $4$-partite graph with part sizes $n_1,n_2,n_3,n_4$ summing to $n$ (allowing some $n_i=0$).
Its number of edges is
\[
 e(G)=\sum_{1\le i<j\le 4} n_i n_j
 =\frac12\left(\left(\sum_{i=1}^4 n_i\right)^2-\sum_{i=1}^4 n_i^2\right)
 =\frac12\left(n^2-\sum_{i=1}^4 n_i^2\right).
\]
To maximize $e(G)$ for fixed $n$, we must minimize $\sum n_i^2$.
A standard convexity/averaging argument shows that $\sum n_i^2$ is minimized when the $n_i$ are as equal as possible: if $n_i\ge n_j+2$, then replacing $(n_i,n_j)$ by $(n_i-1,n_j+1)$ decreases the sum of squares because
\[
(n_i-1)^2+(n_j+1)^2-(n_i^2+n_j^2)= -2(n_i-n_j)+2<0.
\]
Iterating, we reach a balanced partition with $n_i\in\{\lfloor n/4\rfloor,\lceil n/4\rceil\}$.
In that balanced case one checks directly that
\[
\sum_{i=1}^4 n_i^2\ge 4\left(\frac n4\right)^2=\frac{n^2}{4}
\]
(with equality when $4\mid n$).
Hence
\[
 e(G)=\frac12\left(n^2-\sum n_i^2\right)\le \frac12\left(n^2-\frac{n^2}{4}\right)=\frac{3n^2}{8}.
\]
Since $e(G)$ is an integer, $e(G)\le\lfloor 3n^2/8\rfloor$.
\qed

\bigskip
\noindent\textbf{Fast reality check (exact values for $n\le 7$ by brute force).}

I exhaustively enumerated all graphs on $n\le 7$ labeled vertices and tested whether they contain two edge-disjoint cycles with the same vertex set.
The resulting exact maxima are:
\[
\begin{array}{c|ccccccc}
 n & 1&2&3&4&5&6&7\\\hline
 m(n) & 0&1&3&6&9&12&16
\end{array}
\]
An example achieving $m(6)=12$ is $K_6$ with the triangle on vertices $\{3,4,5\}$ deleted.
An example achieving $m(7)=16$ is $K_7$ with the five edges
$\{25,26,34,36,45\}$ deleted (in the vertex labeling $0,1,2,3,4,5,6$).

\bigskip
\noindent\textbf{VERIFICATION.}

\emph{Lemma 585.2 Part A.} The two $5$-cycles listed are easily checked to be edge-disjoint and spanning.

\emph{Lemma 585.2 Part B.} The symmetrization step preserves $K_5$-freeness because any $K_5$ using the cloned vertex would imply a $K_4$ inside a neighborhood, which would create a $K_5$ with the original vertex.
The balancing step uses an explicit decrease in sum of squares and is therefore rigorous.

\emph{Computation.} The brute force check treated the forbidden configuration as: there exists a vertex subset $S$ such that the induced subgraph on $S$ contains two edge-disjoint Hamilton cycles. It is exact for $n\le 7$.

\bigskip
\noindent\textbf{FINAL.}

**UNRESOLVED**

(i) Strongest proved partial result: any graph avoiding the forbidden configuration is $K_5$-free and hence has at most $\lfloor 3n^2/8\rfloor$ edges (Lemma 585.2). Exact values are computed for $n\le 7$ with $m(7)=16$.

(ii) First gap (crisp): improve the crude quadratic upper bound to match the known near-linear (polylog) upper bounds, i.e. prove $m(n)=o(n^2)$ (or even $m(n)=n(\log n)^{O(1)}$) without importing heavy machinery.

(iii) Top 3 next moves:
1. Find a purely combinatorial mechanism forcing two edge-disjoint Hamilton cycles on some vertex set once $e(G)$ exceeds $Cn\log^C n$.
2. Strengthen the ``$K_5$-free'' observation: identify a much sparser forbidden family implied by the two-cycle condition and prove a sharp extremal bound for that family.
3. Computation: for $n=8,9$ (where full enumeration is too large), use SAT/ILP or backtracking to search for dense examples and guess the extremal structure.

(iv) Minimal counterexample structure (to a conjectured near-linear upper bound): would be a graph with $\omega(n)$ edges but arranged so that every vertex subset supporting a Hamilton cycle supports at most one edge-disjoint Hamilton cycle; heuristically this suggests avoiding any reasonably regular dense subgraph, e.g. graphs engineered to have no $4$-regular subgraphs on any vertex set.


