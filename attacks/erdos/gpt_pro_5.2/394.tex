% Erdos Problem #394
% URL: https://www.erdosproblems.com/394

1) FORMAL RESTATEMENT

Fix integers $k\ge 2$ and $n\ge 1$.

Definition.
Let $t_k(n)$ be the least integer $m\ge 1$ such that
\[
 n\mid \prod_{i=0}^{k-1} (m+i)=m(m+1)\cdots (m+k-1).
\]

The questions in the problem are:

(A) For $k=2$, does there exist a constant $c>0$ such that
\[
\sum_{n\le x} t_2(n) \ll \frac{x^2}{(\log x)^c}\quad (x\to\infty)?
\]

(B) For $k\ge 2$, is it true that
\[
\sum_{n\le x} t_{k+1}(n) = o\!\left(\sum_{n\le x} t_k(n)\right)\quad (x\to\infty)?
\]

Edge cases.
$t_k(1)=1$ for all $k$ because $1$ divides every integer.

2) QUICK LITERATURE/CONTEXT CHECK

I only restate what is recorded in the problem statement.

The statement records:
- Erd\H{o}s and Hall proved $\sum_{n\le x} t_2(n)\ll \frac{\log\log\log x}{\log\log x}x^2$.
- Erd\H{o}s and Hall conjectured $\sum_{n\le x} t_2(n)=o(x^2/(\log x)^c)$ for any $c<\log 2$.
- It is noted that $t_2(p)=p-1$ for prime $p$.

3) ATTACK PLAN

Proof-track (partial):
- Establish basic structural identities for $t_k(n)$.
- Compute small values for sanity.

Disproof-track for the conjectural asymptotics:
- Try to construct families of $n$ for which $t_k(n)$ is forced large, to contradict strong upper bounds.

Chosen path: derive elementary lemmas and do small-$n$ computation; no asymptotic bounds are proved.

4) WORK

PHASE 1 — FAST REALITY CHECK (computed)

Computed values for $t_2(n)$ for $1\le n\le 30$:
\[
( t_2(1),\dots,t_2(30))=(1,1,2,3,4,2,6,7,8,4,10,3,12,6,5,15,16,8,18,4,6,10,22,8,24,12,26,7,28,5).
\]
So
\[
\sum_{n\le 30} t_2(n)=283.
\]
Computed values for $t_3(n)$ for $1\le n\le 20$:
\[
( t_3(1),\dots,t_3(20))=(1,1,1,2,3,1,5,2,7,3,9,2,11,5,3,6,15,7,17,3),
\]
and $\sum_{n\le 20} t_3(n)=104$.

Lemma 1 (monotonicity in $k$).
For every $n\ge 1$ and every $k\ge 2$,
\[
 t_{k+1}(n)\le t_k(n).
\]

Proof.
Let $m=t_k(n)$. Then by definition,
\[
 n\mid m(m+1)\cdots(m+k-1).
\]
Multiplying by the additional factor $(m+k)$ gives
\[
 n\mid m(m+1)\cdots(m+k-1)(m+k)=\prod_{i=0}^{k} (m+i).
\]
Thus the same $m$ is an admissible witness for $t_{k+1}(n)$, so $t_{k+1}(n)\le m=t_k(n)$. \qed

Lemma 2 (binomial-coefficient factorisation).
For all integers $m\ge 1$ and $k\ge 1$,
\[
\prod_{i=0}^{k-1}(m+i)=k!\binom{m+k-1}{k}.
\]

Proof.
Using factorials,
\[
\binom{m+k-1}{k}=\frac{(m+k-1)!}{k!(m-1)!}=\frac{m(m+1)\cdots(m+k-1)}{k!}.
\]
Rearranging gives the claimed identity. \qed

Lemma 3 (existence and a crude universal upper bound).
For all $n\ge 1$ and all $k\ge 1$, $t_k(n)$ is well-defined and satisfies $t_k(n)\le n$.

Proof.
Take $m=n$. Then the product $n(n+1)\cdots(n+k-1)$ is divisible by $n$, hence is divisible by $n$. Thus $m=n$ is a valid witness for $t_k(n)$, proving existence and the bound $t_k(n)\le n$. \qed

Lemma 4 (closed form for primes).
Let $p$ be prime and $k\ge 1$. Then
\[
 t_k(p)=\max\{1,\,p-(k-1)\}.
\]

Proof.
Because $p$ is prime, $p\mid \prod_{i=0}^{k-1}(m+i)$ holds if and only if at least one factor is divisible by $p$, i.e. if and only if the interval $\{m,m+1,\dots,m+k-1\}$ contains a multiple of $p$.

If $p\le k-1$, then the interval starting at $m=1$ already contains $p$, so $t_k(p)=1$.

Assume now $p\ge k$. Among positive integers, the smallest interval of length $k$ that contains $p$ is $\{p-(k-1),\dots,p\}$, i.e. $m=p-(k-1)$. For any $m'<p-(k-1)$, the interval $\{m',\dots,m'+k-1\}$ ends at most at $p-1$, so contains no multiple of $p$ (since the next positive multiple is $p$ itself). Therefore $m=p-(k-1)$ is minimal. \qed

In particular, for $k=2$ this gives $t_2(p)=p-1$.

5) VERIFICATION

- Lemma 4 vs computation: for $p=11$, $t_2(11)=10$ matches; for $k=3$, $t_3(11)=9$ matches the computed list.
- Lemma 1 is consistent with computed values, e.g. $t_3(10)=3\le t_2(10)=4$.
- Bound $t_k(n)\le n$ is consistent with the computed range.

6) FINAL

**UNRESOLVED**

(i) Strongest fully proved partial result obtained here.
- Structural identities: monotonicity $t_{k+1}(n)\le t_k(n)$ (Lemma 1), and $m(m+1)\cdots(m+k-1)=k!\binom{m+k-1}{k}$ (Lemma 2).
- Explicit prime formula $t_k(p)=\max\{1,p-(k-1)\}$ (Lemma 4).
- Small-$n$ computed tables for $t_2(n)$ ($n\le 30$) and $t_3(n)$ ($n\le 20$).

(ii) Exact first gap.
Prove any nontrivial power-saving upper bound on $\sum_{n\le x} t_2(n)$ beyond $o(x^2)$ of the type conjectured, e.g. $\sum_{n\le x} t_2(n)\ll x^2/(\log x)^c$ for some explicit $c>0$.

(iii) Top 3 next moves (concrete targets).
1. Use Lemma 2 to translate divisibility by consecutive products into divisibility properties of binomial coefficients and attempt to average over $m$.
2. Obtain sharp lower bounds on the count of $m\le M$ for which $n\mid m(m+1)$, averaged over $n$; this could feed into bounds for $\sum t_2(n)$.
3. For the comparison question $\sum t_{k+1}=o(\sum t_k)$, try to prove quantitative inequalities between $t_{k+1}(n)$ and $t_k(n)$ on a positive-density set of $n$.

(iv) What a minimal counterexample would likely look like.
A counterexample to a strong upper bound $\sum_{n\le x}t_2(n)\ll x^2/(\log x)^c$ would likely be a dense family of integers $n$ for which the least $m$ with $n\mid m(m+1)$ is forced to be close to $n$ (so that $t_2(n)$ contributes almost linearly), beyond what prime contributions alone account for.


