\section*{Problem 271}

\subsection*{Problem statement (as given)}
For $n$ a positive integer, let $A(n)$ be the Stanley sequence generated by $\{0,n\}$.
Can $a_k$ be explicitly determined?  How fast do they grow?

% Source for statement/status: citeturn1view1turn21view0

\subsection*{1. Formal restatement}

Fix $n\in\mathbb{N}$.  Define the \emph{Stanley sequence} $A(n)=(a_k)_{k\ge 0}$ by:
\[
a_0=0,\qquad a_1=n,
\]
and for $k\ge 1$, $a_{k+1}$ is the smallest integer $>a_k$ such that the set
\[
\{a_0,a_1,\dots,a_k,a_{k+1}\}
\]
contains no nontrivial 3-term arithmetic progression, i.e.\ no distinct $x<y<z$ with $y-x=z-y$.

Questions:
\begin{itemize}[itemsep=2pt]
\item Can we write a closed form for $a_k$ (for general $n$)?
\item What is the asymptotic growth of $a_k$ as $k\to\infty$ (depending on $n$)?
\end{itemize}

\subsection*{2. Quick literature/context check}

The Erd\H{o}s Problems Project lists this as open in general and summarizes known structure for some special $n$ (e.g.\ $n=1$, and more generally $n=3^k$ and $n=2\cdot 3^k$) going back to Odlyzko--Stanley and others. % citeturn1view1

There is an extensive modern literature on Stanley sequences; for example, a 2025 preprint studies ``irregular'' behavior of Stanley sequences beyond the well-structured cases. % citeturn21view0

\subsection*{3. Attack plan}

\begin{enumerate}[label=\textbf{(\alph*)},itemsep=2pt]
\item \textbf{Compute small examples} to see patterns (especially $n=1$ versusS(0,1) and a ``generic'' $n$ like $4$).
\item \textbf{Prove what can be proved cleanly:}
\begin{itemize}[itemsep=2pt]
\item an explicit description and growth rate for $A(1)$;
\item a general, elementary upper bound $a_k \le n + O(k^2)$ valid for all $n$.
\end{itemize}
\item \textbf{Compare} with the conjectural dichotomy ``structured'' vs ``irregular'' Stanley sequences (but not attempt to resolve the full open problem).
\end{enumerate}

\subsection*{4. Work}

\subsubsection*{4.1 Small-case computation}

For $n=1$, the greedy sequence begins
\[
A(1):\ 0,1,3,4,9,10,12,13,27,28,30,31,36,37,39,\dots
\]
For $n=4$, it begins
\[
A(4):\ 0,4,5,6,9,11,13,14,15,19,20,22,23,24,29,\dots
\]
The $n=1$ case is highly structured; $n=4$ already looks less obviously structured.

\subsubsection*{4.2 Explicit determination of $A(1)$ (complete)}

Let
\[
S:=\Bigl\{\sum_{i\ge 0}\varepsilon_i 3^i:\ \varepsilon_i\in\{0,1\}\Bigr\},
\]
i.e.\ the set of nonnegative integers whose base-$3$ expansion uses only digits $0$ and $1$ (no digit $2$).

\begin{lemma}[No carries]\label{lem:nocarry}
If $x=\sum x_i3^i$ and $z=\sum z_i3^i$ with $x_i,z_i\in\{0,1\}$, then the base-$3$ addition $x+z$ has no carries; equivalently, the $i$th ternary digit of $x+z$ is $x_i+z_i\in\{0,1,2\}$.
\end{lemma}

\begin{proof}
Write the usual carry recursion in base $3$:
$c_0=0$ and for each $i\ge 0$,
\[
x_i+z_i+c_i = d_i + 3c_{i+1},\qquad d_i\in\{0,1,2\}.
\]
We show by induction that $c_i=0$ for all $i$.  For $i=0$, $x_0+z_0+c_0\le 2$, so $c_1=0$.
Assume $c_i=0$; then $x_i+z_i+c_i\le 2$ implies $c_{i+1}=0$.  Hence all carries vanish.
\end{proof}

\begin{lemma}[$S$ is 3AP-free]\label{lem:S3free}
There are no distinct $x<y<z$ in $S$ with $x+z=2y$.
\end{lemma}

\begin{proof}
Write $x=\sum x_i3^i$, $y=\sum y_i3^i$, $z=\sum z_i3^i$ with digits in $\{0,1\}$.
By Lemma~\ref{lem:nocarry}, $x+z$ is computed digitwise: the $i$th digit is $x_i+z_i\in\{0,1,2\}$ with no carry.
Also $2y=\sum 2y_i 3^i$ is computed digitwise with digits $2y_i\in\{0,2\}$ and no carries (since $2\cdot 1=2<3$).

From $x+z=2y$, comparing ternary digits gives $x_i+z_i=2y_i$ for each $i$.
But the only way to have $x_i,z_i,y_i\in\{0,1\}$ and $x_i+z_i=2y_i$ is $x_i=z_i=y_i$.
Hence $x=y=z$, contradicting distinctness.
\end{proof}

\begin{lemma}[Every integer has a ``Cantor'' 2-term representation]\label{lem:repH}
For every $t\ge 0$ and every integer $m$ with $0\le m<3^t$, there exist $x,y\in S$ with $0\le x\le y\le m$ such that
\[
m=2y-x,
\]
and moreover if $m\notin S$ then $y<m$.
\end{lemma}

\begin{proof}
We prove this by induction on $t$.

Base $t=0$: the only $m$ is $0$, and $0=2\cdot 0-0$.

Inductive step $t\to t+1$.
Let $0\le m<3^{t+1}$.  Write uniquely
\[
m=a\cdot 3^t + r,\qquad a\in\{0,1,2\},\quad 0\le r<3^t.
\]
Apply the induction hypothesis to $r$ to get $x_0,y_0\in S$ with $r=2y_0-x_0$ and $0\le x_0\le y_0\le r$, and if $r\notin S$ then $y_0<r$.

\emph{Case 1: $a=0$ or $a=1$.}
Set
\[
x:=a\cdot 3^t + x_0,\qquad y:=a\cdot 3^t + y_0.
\]
Then $x,y\in S$ (their ternary digit at position $t$ is $a\in\{0,1\}$ and lower digits come from $x_0,y_0$), and
\[
2y-x = a\cdot 3^t + (2y_0-x_0) = a\cdot 3^t + r = m.
\]
Also $y\le a\cdot 3^t + r = m$, and if $m\notin S$ in this case, then necessarily $r\notin S$, so $y_0<r$ and hence $y<m$.

\emph{Case 2: $a=2$.}
Set
\[
x:=x_0,\qquad y:=3^t + y_0.
\]
Then $x\in S$ and $y\in S$ (digit $t$ equals $1$), and
\[
2y-x = 2(3^t+y_0)-x_0 = 2\cdot 3^t + (2y_0-x_0)=2\cdot 3^t + r = m.
\]
Finally, $y=3^t+y_0\le 3^t+r < 2\cdot 3^t+r=m$, so $y<m$.

This completes the induction.
\end{proof}

\begin{theorem}[Explicit description of $A(1)$]\label{thm:A1}
The Stanley sequence $A(1)$ is exactly the increasing enumeration of $S$, the set of nonnegative integers with ternary digits in $\{0,1\}$.
Equivalently, if $k=\sum_{i\ge 0}\varepsilon_i 2^i$ is the binary expansion of $k$, then
\[
a_k=\sum_{i\ge 0}\varepsilon_i 3^i.
\]
\end{theorem}

\begin{proof}
We prove by induction on $m\ge 0$ that, when the greedy algorithm for $A(1)$ has processed all integers $<m$, the chosen set equals $S\cap[0,m)$.

Induction base $m=0$ is trivial.  Assume it holds for $m$.

\emph{If $m\in S$:} all previously chosen numbers lie in $S$, and $S$ is 3AP-free by Lemma~\ref{lem:S3free}, so adjoining $m$ creates no 3-term arithmetic progression.  Hence the greedy algorithm accepts $m$.

\emph{If $m\notin S$:} choose $t$ with $m<3^t$.  By Lemma~\ref{lem:repH}, there exist $x,y\in S$ with $m=2y-x$ and $y<m$.  In particular $x<y<m$ and $x,y\in S\cap[0,m)$.
By the induction hypothesis, $x$ and $y$ have already been selected before $m$ is considered, so $m$ would complete the 3AP $(x,y,m)$ and is therefore rejected by the greedy rule.

Thus the greedy algorithm accepts exactly the integers in $S$, in increasing order.  The final formula $a_k=\sum\varepsilon_i3^i$ follows because mapping a binary string $(\varepsilon_i)$ to the ternary string $(\varepsilon_i)$ gives an order-preserving bijection $\mathbb{N}\to S$.
\end{proof}

\subsubsection*{4.3 Growth rate of $A(1)$ (complete)}

\begin{corollary}\label{cor:growthA1}
Let $\alpha:=\log_2 3$.  Then $a_k$ for $A(1)$ satisfies
\[
k^{\alpha}\ \le\ a_k\ <\ 3\,k^{\alpha}\qquad(k\ge 1),
\]
so $a_k=\Theta(k^{\log_2 3})$.
\end{corollary}

\begin{proof}
Let $m$ be such that $2^m\le k<2^{m+1}$.  Then in binary, $k$ has at most $m+1$ digits, so by Theorem~\ref{thm:A1}, $a_k$ has ternary digits only $0/1$ and has largest possible value $<\sum_{i=0}^{m}3^i<3^{m+1}$, hence $a_k<3^{m+1}$.
Also $a_{2^m}=3^m$ (binary $10\cdots0$ maps to ternary $10\cdots0$), so $a_k\ge a_{2^m}=3^m$.
Since $3^m=(2^m)^{\log_2 3}\le k^{\log_2 3}$ and $3^{m+1}=3\cdot (2^{m+1})^{\log_2 3}\le 3k^{\log_2 3}$, the stated bounds follow.
\end{proof}

\subsubsection*{4.4 A general quadratic upper bound for all $n$ (complete)}

\begin{theorem}[Elementary bound]\label{thm:quadbound}
For every $n\in\mathbb{N}$ and every $k\ge 1$, the Stanley sequence $A(n)$ satisfies
\[
a_k \le n+\frac{(k-1)(k+2)}{2}.
\]
In particular $a_k = O(k^2)$ uniformly in $n$.
\end{theorem}

\begin{proof}
Fix $k\ge 1$.  Consider the moment \emph{just before} the algorithm chooses $a_k$; at that time the selected set is
\[
\{a_0,a_1,\dots,a_{k-1}\}.
\]
Now look at all integers $m$ with
\[
n\le m < a_k.
\]
Such an $m$ is \emph{not} selected by time $a_k$ is chosen, so when the greedy algorithm inspected $m$ (which happens before choosing $a_k$), it must have rejected $m$ because adding $m$ would create a 3-term AP with two earlier selected elements.

Since $m$ is larger than all elements present at that stage, $m$ can only play the role of the \emph{third} term in a 3AP, so there exist indices $0\le i<j\le k-1$ such that
\[
m = 2a_j - a_i.
\]
For each ordered pair $(i,j)$ with $0\le i<j\le k-1$ there is \emph{at most one} integer $m$ satisfying $m=2a_j-a_i$.  Therefore, the number of rejected integers $m$ in $[n,a_k)$ is at most the number of pairs $(i,j)$ with $0\le i<j\le k-1$, namely $\binom{k}{2}$.

But the interval $[n,a_k)$ contains exactly $a_k-n$ integers, of which exactly $k-1$ are selected (namely $a_1,\dots,a_{k-1}$).  Hence the number of rejected integers in $[n,a_k)$ equals
\[
(a_k-n) - (k-1),
\]
so we have
\[
(a_k-n)-(k-1)\le \binom{k}{2}=\frac{k(k-1)}{2}.
\]
Rearranging gives $a_k \le n + (k-1) + k(k-1)/2 = n + (k-1)(k+2)/2$, as claimed.
\end{proof}

\subsection*{5. Verification}

\begin{itemize}[itemsep=2pt]
\item The proof of Theorem~\ref{thm:A1} hinges on Lemma~\ref{lem:repH}.  I checked the induction carefully: the key is that for $a=2$ we force $y<m$ automatically, which is what makes ``digit 2'' positions genuinely forbidden.
\item Theorem~\ref{thm:quadbound} is a clean counting argument that avoids the earlier ``pairs involving the last element'' overcount: any $m<a_k$ is rejected before $a_k$ is chosen, so the witness pair must come from $\{a_0,\dots,a_{k-1}\}$.
\item These results do not settle the explicit structure of $A(n)$ for general $n$ (the main open part).
\end{itemize}

\subsection*{6. Final}

\textbf{UNRESOLVED.}

\begin{enumerate}[label=\textbf{(\roman*)},itemsep=4pt]
\item \textbf{Farthest-reaching partial results proved here.}
\begin{itemize}[itemsep=2pt]
\item Complete explicit description of $A(1)$ (Theorem~\ref{thm:A1}), and the growth $a_k=\Theta(k^{\log_2 3})$ (Corollary~\ref{cor:growthA1}).
\item A uniform quadratic upper bound $a_k\le n + (k-1)(k+2)/2$ for all $n$ (Theorem~\ref{thm:quadbound}).
\end{itemize}

\item \textbf{Precise obstacle.}
For general $n$, the greedy choices depend on the evolving pattern of forbidden values $2a_j-a_i$.  Unlike $n=1$ (and a few other special $n$), there is no known closed-form description of which candidates survive, and the structure can be ``irregular'' in ways that defeat current techniques.

\item \textbf{Most promising next steps.}
\begin{itemize}[itemsep=2pt]
\item Identify additional classes of $n$ for which $A(n)$ is ``regular'' (self-similar / described by digital constraints), extending the known $n=3^t$ and $n=2\cdot 3^t$ families.
\item Develop invariants that detect and control irregularity (as explored in recent work). % citeturn21view0
\item Improve general bounds (both upper and lower) on $a_k$ and/or the counting function of $A(n)$, possibly with explicit constants.
\end{itemize}

\item \textbf{Belief about truth value.}
It is plausible that most $A(n)$ are ``irregular'' and do \emph{not} admit a simple closed form, but that there are infinitely many structured $n$ with explicit descriptions; current evidence supports a mix rather than a uniform behavior.
\end{enumerate}

\subsection*{7. Completion estimate}
\[
\textbf{55\%}
\]
(I gave complete proofs for $n=1$ and a strong general upper bound, but not an explicit formula or sharp growth for general $n$.)

%%%%%%%%%%%%%%%%%%%%%%%%%%%%%%%%%%%%%%%%%%%%%%%%%%%%%%%%%%%%%%%%%%%%%%%%%%%%%%%
