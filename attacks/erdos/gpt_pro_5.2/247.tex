\section*{Erd\H{o}s Problem \#247}

\subsection*{1) FORMAL RESTATEMENT}
Let $a_1<a_2<\cdots$ be a strictly increasing sequence of positive integers such that
\[
\limsup_{n\to\infty}\frac{a_n}{n}=\infty.
\]
Define the real number
\[
\alpha \coloneqq \sum_{n=1}^{\infty} \frac{1}{2^{a_n}}.
\]
\noindent\textbf{Question.} Must $\alpha$ be \emph{transcendental} (i.e., not algebraic over $\mathbb{Q}$)?

\subsection*{2) QUICK LITERATURE/CONTEXT CHECK}
\begin{itemize}[leftmargin=2em]
\item The Erd\H{o}s Problems page for \#247 states the problem as open and records that Erd\H{o}s proved transcendence under a stronger growth hypothesis (citing \cite{Erdos1975,Erdos1988}) \cite{ErdosProblems247}.
\item The LaTeX view of the page lists references \cite{Erdos1975,Erdos1988} \cite{ErdosProblems247latex}.
\end{itemize}

\subsection*{3) ATTACK PLAN}
Possible routes to a proof (or disproof) include:
\begin{enumerate}[leftmargin=2em]
\item \textbf{Diophantine approximation:} Truncations give dyadic rationals $p_N/2^{a_N}$ approximating $\alpha$ with error $\ll 2^{-a_{N+1}}$. If one could force approximation exponents $>2$ infinitely often from the hypothesis $\limsup a_n/n=\infty$, then Roth's theorem would imply transcendence. However, $\limsup a_n/n=\infty$ does not directly control ratios $a_{n+1}/a_n$.
\item \textbf{Combinatorics/complexity of digit expansions:} $\alpha$ has a binary expansion supported on the set $\{a_n\}$. If one could show that the subword complexity of this expansion is ``too small'' under the hypothesis, then deep theorems on the base-$b$ expansions of algebraic numbers (Adamczewski--Bugeaud type results) might yield transcendence. The hypothesis here is weak enough that the digit pattern can still be very complicated.
\item \textbf{Disproof attempt:} Construct an algebraic number whose binary expansion has 1's at positions $a_n$ with $\limsup a_n/n=\infty$. I do not know any construction of algebraic numbers with such prescribed sparse digit structure.
\end{enumerate}

\subsection*{4) WORK}
\subsubsection*{4.1 The series is always irrational under the stated hypothesis}
Although transcendence is open, the stated growth assumption immediately rules out rationality.

\begin{proposition}[Irrationality under $\limsup a_n/n=\infty$]
If $a_1<a_2<\cdots$ and $\limsup_{n\to\infty} a_n/n=\infty$, then
\[
\alpha=\sum_{n\ge 1} 2^{-a_n}
\]
is irrational.
\end{proposition}

\begin{proof}
Write the (unique) binary expansion of $\alpha$ that does \emph{not} terminate in an infinite tail of 1's. Because the summands are distinct dyadic rationals, we have a well-defined binary digit sequence $(\varepsilon_m)_{m\ge 1}\in\{0,1\}^{\mathbb{N}}$ with
\[
\alpha = \sum_{m=1}^{\infty} \varepsilon_m 2^{-m},
\qquad \varepsilon_m = 1 \iff m\in\{a_n:n\ge 1\}.
\]
(There are no carries because each position $m$ receives at most one contribution $2^{-m}$.)

Assume for contradiction that $\alpha\in\mathbb{Q}$. Then its binary expansion is eventually periodic: there exist integers $M\ge 1$ and $\ell\ge 1$ and bits $b_1,\dots,b_\ell\in\{0,1\}$, not all zero, such that for all $m\ge M$ one has
\[
\varepsilon_m = b_{((m-M)\bmod \ell)+1}.
\]
Let $s\coloneqq b_1+\cdots+b_\ell\ge 1$ be the number of ones in one period. Then among the digits $\varepsilon_M,\dots,\varepsilon_{M+N-1}$, the number of ones is at least $\lfloor N/\ell\rfloor\,s\ge (s/\ell)N-1$.

Equivalently, for all sufficiently large $L$, the number of indices $m\le L$ with $\varepsilon_m=1$ is at least $(s/\ell)(L-M)-O(1)$, i.e. grows \emph{linearly} in $L$ with positive slope.

Now $a_n$ is the position of the $n$th 1 in this binary expansion. Linear growth of the count of ones implies a linear upper bound on $a_n$: there exists a constant $K>0$ such that for all large $n$,
\[
 a_n \le K n.
\]
(For instance, take $K=2\ell/s$ and compare $n$ to the count of ones up to $L=Kn$.)
Thus $\sup_n a_n/n <\infty$, contradicting $\limsup a_n/n=\infty$.

Therefore $\alpha\notin\mathbb{Q}$.
\end{proof}

\subsubsection*{4.2 A sufficient condition for transcendence (stronger than the hypothesis)}
The following criterion shows where ``Liouville-type'' arguments become available.

\begin{proposition}[Unbounded ratio implies Liouville]
If
\[
\limsup_{n\to\infty}\frac{a_{n+1}}{a_n}=\infty,
\]
then $\alpha=\sum_{n\ge 1}2^{-a_n}$ is a Liouville number, hence transcendental.
\end{proposition}

\begin{proof}
Let $p_N/2^{a_N}$ be the $N$-term truncation:
\[
\alpha_N\coloneqq\sum_{n=1}^{N}2^{-a_n}=\frac{p_N}{2^{a_N}}\in\mathbb{Q}.
\]
The tail satisfies
\[
0<\alpha-\alpha_N = \sum_{n>N} 2^{-a_n} \le 2^{-a_{N+1}}\sum_{j\ge 0}2^{-j}=2\cdot 2^{-a_{N+1}}.
\]
Let $q_N\coloneqq 2^{a_N}$. Then
\[
\bigl|\alpha-\tfrac{p_N}{q_N}\bigr| \le 2\,2^{-a_{N+1}} = 2\,q_N^{-a_{N+1}/a_N}.
\]
By hypothesis, for every integer $m\ge 1$ there are infinitely many $N$ with $a_{N+1}/a_N>m+1$. For such $N$ (and $q_N\ge 2$) we have
\[
\bigl|\alpha-\tfrac{p_N}{q_N}\bigr| \le 2\,q_N^{-(m+1)} < q_N^{-m}.
\]
Thus for every $m$ there are infinitely many rational approximations $p_N/q_N$ satisfying $|\alpha-p_N/q_N|<q_N^{-m}$, which is exactly the definition of a Liouville number. Hence $\alpha$ is transcendental.
\end{proof}

\subsection*{5) VERIFICATION}
\begin{itemize}[leftmargin=2em]
\item The irrationality proof only uses the standard fact that rationals have eventually periodic base-$2$ expansions, and that eventual periodicity forces a positive asymptotic density of 1's in the tail if the number is not dyadic-rational.
\item The ``no carries'' claim is justified because the summands $2^{-a_n}$ have disjoint binary positions.
\item The Liouville criterion is a standard truncation argument; the key inequality is the geometric-series bound on the tail.
\end{itemize}

\subsection*{6) FINAL}
\textbf{UNRESOLVED.}
\begin{enumerate}[label=(\roman*),leftmargin=2.5em]
\item \textbf{Strongest fully proved partial result obtained here:} Under the stated hypothesis $\limsup a_n/n=\infty$, the number $\alpha=\sum 2^{-a_n}$ is \emph{irrational} (Proposition 4.1). Also, if $\limsup a_{n+1}/a_n=\infty$ then $\alpha$ is Liouville and hence transcendental (Proposition 4.2).
\item \textbf{First gap preventing a full solution:} The hypothesis $\limsup a_n/n=\infty$ does not force Liouville-type approximations (ratios $a_{n+1}/a_n$ may stay close to $1$) and does not obviously force low digit-complexity. I do not see a mechanism that excludes algebraicity from this weak sparsity condition.
\item \textbf{Most promising next move:} Try to connect the growth condition to quantitative lower bounds on the block complexity (or other automaticity/regularity properties) of the binary expansion of $\alpha$, and then invoke known transcendence criteria for expansions with ``too simple'' structure. One must also handle the possibility that the expansion is very complex on most scales while still having occasional large gaps.
\item \textbf{Smallest plausible counterexample (if false):} An algebraic irrational whose binary expansion contains arbitrarily long stretches with unusually few 1's (so that the 1-positions $a_n$ satisfy $\limsup a_n/n=\infty$) would refute the conjecture. No explicit candidate is known to me.
\end{enumerate}

\subsection*{7) COMPLETION ESTIMATE}
\textbf{Estimated likelihood of completion with additional work:} \emph{Moderate-to-low} (\(\approx 20\%\)). The problem seems to require either a new transcendence criterion tailored to sparse digit sets with weak hypotheses, or new information about possible sparsity patterns in algebraic base-$2$ expansions.


% =========================================================
