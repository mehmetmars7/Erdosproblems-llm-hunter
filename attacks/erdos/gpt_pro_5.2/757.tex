\section*{Problem 757}

\subsection*{1) FORMAL RESTATEMENT}
Let $A\subset\mathbb{R}$ with $|A|=n$. For a finite set $B\subset\mathbb{R}$, write
\[
B-B:=\{x-y: x,y\in B\}.
\]
Assume that $A$ satisfies the \emph{$(4,5)$-difference condition}:
\[
\forall B\subseteq A\ (|B|=4\implies |B-B|\ge 11).
\]
Equivalently, for every $4$-set $B=\{b_1,b_2,b_3,b_4\}$, among the $6$ pairwise distances
$\{|b_i-b_j|:1\le i<j\le 4\}$ at least $5$ are distinct.

A subset $S\subset\mathbb{R}$ is a \emph{Sidon set} (a $B_2$-set) if all sums $s_i+s_j$ with
$s_i\le s_j$ are distinct; equivalently, all nonzero differences $s_i-s_j$ with $s_i>s_j$ are distinct.

Define, for each $n$, the quantity
\[
\sigma(n):=\min\Big\{\max\{|S|: S\subseteq A\ \text{Sidon}\}: A\subset\mathbb{R},\ |A|=n,\ A\ \text{satisfies }|B-B|\ge 11\ \forall B\in\binom{A}{4}\Big\}.
\]
The problem asks for the best constant
\[
 c^*:=\sup\{c>0: \forall n\ \sigma(n)\ge cn\},
\]
(or, equivalently, the asymptotically best guaranteed linear fraction of a Sidon subset in every such $A$).

\subsection*{2) QUICK LITERATURE/CONTEXT CHECK}
The statement as recorded in Erd\H{o}s's problem list (and repeated in later sources) includes:
\begin{itemize}
\item Erd\H{o}s--S\'os: $c^*\ge \tfrac12$.
\item Gy\'arf\'as--Lehel (1995): improved the lower bound and exhibited an upper bound,
\[\tfrac12<c^*<\tfrac35,\]
with the upper bound given by taking $A$ to be the first $n$ Fibonacci numbers.
\end{itemize}
I did not locate a definitive resolution of the exact value of $c^*$.

\subsection*{3) ATTACK PLAN}
A natural first step is to understand what the constraint $|B-B|\ge 11$ forbids.
For a $4$-set, $|B-B|=1+2\cdot\#\{\text{distinct positive distances in }B\}$.
Thus $|B-B|\ge 11$ means \emph{at most one} of the $6$ distances repeats.
The extremal obstruction to being Sidon is an equality of distances
\[
|a-b|=|c-d|\quad\text{with }\{a,b\}\ne\{c,d\}.
\]
Under the $(4,5)$ hypothesis, such equalities are forced to have strong overlap structure.
If one can reduce Sidon-ness in $A$ to forbidding only a small class of local configurations
(e.g. $3$-term arithmetic progressions), then the problem becomes a structured extremal/packing
question (hypergraph independent set problem) on those configurations.

\subsection*{4) WORK}
\paragraph{Lemma 4.1 (no repeated distance on disjoint pairs).}
Assume $A$ satisfies the $(4,5)$-difference condition. Then there do not exist four \emph{distinct}
$a,b,c,d\in A$ such that
\[
|a-b|=|c-d|\ \text{ and }\ \{a,b\}\cap\{c,d\}=\varnothing.
\]
\emph{Proof.}
Assume (for contradiction) that $a,b,c,d$ are distinct and $|a-b|=|c-d|=:r$ with disjoint pairs.
By translating and relabelling if needed, we may assume $a<b$ and $c<d$ so that $b-a=d-c=r$.
Then also
\[
|c-a|=c-a = (c+r)-(a+r)=d-b=|d-b|.
\]
Hence in the $4$-set $\{a,b,c,d\}$ the distance $r$ occurs at least twice (between $a,b$ and $c,d$)
and the distance $|c-a|$ occurs at least twice (between $a,c$ and $b,d$), so there are \emph{at least two}
repeated distances among the $6$ pairwise distances. This contradicts the $(4,5)$-difference condition.
\qed

\paragraph{Corollary 4.2 (Sidon versus 3-term arithmetic progressions inside $A$).}
Let $S\subseteq A$. Then $S$ fails to be Sidon if and only if $S$ contains a (nontrivial)
$3$-term arithmetic progression, i.e. distinct $x,y,z\in S$ with $y-x=z-y$.

\emph{Proof.}
If $S$ contains a $3$-term AP $x<y<z$ with $y-x=z-y$, then the distance $y-x$ is repeated among
pairs $(x,y)$ and $(y,z)$, so $S$ is not Sidon.
Conversely, if $S$ is not Sidon, there exist distinct unordered pairs $\{u,v\}\ne\{u',v'\}$ in $S$
with $|u-v|=|u'-v'|$. By Lemma~4.1 these pairs cannot be disjoint, hence they share a vertex.
Up to relabelling, we have distinct $x,y,z\in S$ with $|x-y|=|y-z|$, which in $\mathbb{R}$ implies
$y$ is the midpoint of $x$ and $z$, i.e. $x,y,z$ form a $3$-term AP.
\qed

\paragraph{Interpretation.}
Under the $(4,5)$-difference condition, the entire Sidon obstruction collapses to the presence of
$3$-term arithmetic progressions. In other words, within such an $A$, finding a large Sidon subset
is equivalent to finding a large subset free of $3$-term APs.

\paragraph{Small-$n$ sanity check (computational).}
A brute-force search over small integer sets (restricted to a bounded range) finds examples of
$(4,5)$-sets and computes the smallest possible maximum Sidon subset size among them.
For instance, within a small search window one can find:
\begin{itemize}
\item $n=4$: example $A=\{0,1,2,5\}$ satisfies the condition; its largest Sidon subset has size $3$.
\item $n=6$: example $A=\{0,1,2,5,8,15\}$ satisfies the condition; its largest Sidon subset has size $4$.
\end{itemize}
These are consistent with a linear guarantee, but of course do not address the optimal constant $c^*$.

\subsection*{5) OBSTRUCTIONS/CAVEATS}
\begin{itemize}
\item Determining $c^*$ requires a sharp extremal analysis of how many $3$-term APs can be forced
by the $(4,5)$ condition and how large an AP-free subset must exist in such structured sets.
\item While Lemma~4.1 and Corollary~4.2 reduce the nature of the obstruction, they do not by
themselves yield the best constant (currently only the interval $\tfrac12<c^*<\tfrac35$ is recorded).
\end{itemize}

\subsection*{6) FINAL}
\textbf{UNRESOLVED.}
\begin{itemize}
\item[(i)] \emph{Where I got stuck:} I did not obtain a new argument pinning down the exact optimal constant $c^*$,
or improving the best known bounds beyond $\tfrac12<c^*<\tfrac35$.
\item[(ii)] \emph{Partial progress:} I proved that the $(4,5)$ condition forbids equal distances on disjoint pairs
(Lemma~4.1) and therefore, within such sets, Sidon subsets are exactly those avoiding $3$-term APs (Cor.~4.2).
\item[(iii)] \emph{What seems next:} Model the family of $3$-term APs in $A$ as a $3$-uniform hypergraph with strong
linearity-type constraints; then apply (or develop) sharp bounds on the independence number under those constraints.
This is plausibly where the improvements from $1/2$ to $>1/2$ arise.
\item[(iv)] \emph{What I believe is true:} The exact value of $c^*$ is likely nontrivial; the Fibonacci example suggests
the extremal configurations are highly structured rather than random.
\end{itemize}

\subsection*{7) COMPLETION ESTIMATE}
$20\%$.

