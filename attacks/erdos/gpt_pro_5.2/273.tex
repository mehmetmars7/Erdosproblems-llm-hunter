\section*{Problem \#273}

\subsection*{FORMAL RESTATEMENT}
A \emph{(distinct/incongruent) covering system} of $\mathbb{Z}$ is a finite collection of congruences
\[
x \equiv a_i \pmod{m_i}\qquad (1\le i\le k),
\]
with integers $m_i\ge 2$ that are \emph{pairwise distinct}, such that every $x\in\mathbb{Z}$ satisfies at least one of the congruences.

The literal wording in the problem statement does not explicitly say ``distinct moduli''.  However, if repetition is allowed then the problem becomes trivial: since $4 = 5-1$, the four congruences $x\equiv 0,1,2,3\pmod 4$ already cover $\mathbb{Z}$ using moduli of the form $p-1$ with $p\ge 5$.  Thus the only nontrivial interpretation is the distinct-modulus version.

\medskip
\noindent\textbf{Corrected statement (standard in covering-systems theory).}
\emph{Does there exist a distinct covering system of $\mathbb{Z}$ in which every modulus is of the form $p-1$ for some prime $p\ge 5$?}

\subsection*{QUICK LITERATURE/CONTEXT CHECK}
This question is attributed to Erd\H{o}s (and is recorded in the Erd\H{o}s problem list as open as of late 2025).  John Selfridge observed an easy covering system whose moduli are all divisors of $360$, and if one allows the prime $p=3$ then $2=p-1$ appears among these moduli, making ``$p-1$''-type moduli easier to arrange.  The restriction $p\ge 5$ forbids modulus $2$, and the existence/nonexistence of such a cover with \emph{distinct} moduli remains open in the public literature.

\subsection*{ATTACK PLAN}
\begin{enumerate}
\item Use the standard density/counting obstruction $\sum_i 1/m_i \ge 1$ for any (not-necessarily-disjoint) cover.
\item Apply that obstruction to the special set of allowed moduli $\{p-1:\ p\text{ prime},\ p\ge 5\}$ to obtain explicit necessary lower bounds on the largest modulus and on the number of congruences.
\item (If pushing further) attempt to combine the special arithmetic structure of $p-1$ with known tools for covering systems (local constraints modulo prime powers; Hough-type probabilistic obstructions; computational SAT/ILP searches on $\mathbb{Z}/M\mathbb{Z}$ for moderate $M$).
\end{enumerate}

\subsection*{WORK}
\paragraph{A basic necessary condition: reciprocal-sum obstruction.}
Let $\{x\equiv a_i\ (\mathrm{mod}\ m_i)\}_{i=1}^k$ be any covering system (moduli need not be distinct for this lemma).
Set $M=\mathrm{lcm}(m_1,\dots,m_k)$ and work in $\mathbb{Z}/M\mathbb{Z}$.
Each congruence $x\equiv a_i\ (\mathrm{mod}\ m_i)$ covers exactly $M/m_i$ residue classes mod $M$.
Since the union of these covered classes is all of $\mathbb{Z}/M\mathbb{Z}$,
\[
M \;\le\; \sum_{i=1}^k \frac{M}{m_i},
\]
hence
\begin{equation}\label{eq:reciprocal_obstruction}
\sum_{i=1}^k \frac{1}{m_i} \;\ge\; 1.
\end{equation}

\paragraph{Consequence for ``$p-1$'' moduli: the largest modulus must be at least $70$.}
Assume we have a distinct covering system whose moduli are all of the form $m_i=p_i-1$ with primes $p_i\ge 5$.
Then~\eqref{eq:reciprocal_obstruction} gives
\[
\sum_i \frac{1}{p_i-1}\ \ge\ 1.
\]
In particular, if all moduli satisfied $m_i\le 66$ (i.e.\ $p_i\le 67$), then even using \emph{all} possible allowed moduli up to $66$ would give a sum strictly less than $1$:
\[
\sum_{\substack{p\text{ prime}\\ 5\le p\le 67}} \frac{1}{p-1}
\;=\;
\frac{160107799}{160240080}
\;\approx\;
0.99917448
\;<\; 1.
\]
Therefore, no such cover can have all moduli $\le 66$.
Equivalently, any distinct covering system with moduli of the form $p-1$ (with $p\ge 5$) must include \emph{at least one modulus $\ge 70$}, i.e.\ must include the modulus $70=71-1$ or some larger allowed modulus.

\paragraph{Consequence: at least $18$ congruences are needed.}
The argument above also implies a lower bound on the number of congruences.
The $17$ smallest allowed moduli are $p-1$ for primes $p=5,7,11,\dots,67$, and their reciprocal sum is the rational number above, $<1$.
Adding the next smallest allowed modulus $70$ pushes the sum above $1$:
\[
\sum_{\substack{p\text{ prime}\\ 5\le p\le 71}} \frac{1}{p-1}
\;=\;
\frac{162396943}{160240080}
\;\approx\;
1.01346020
\;>\; 1.
\]
Hence any such covering system must use \emph{at least $18$} congruences (and, in particular, at least one modulus $\ge 70$).

\subsection*{VERIFICATION}
\begin{itemize}
\item The reciprocal-sum obstruction~\eqref{eq:reciprocal_obstruction} is a one-line check modulo $M=\mathrm{lcm}(m_i)$.
\item The numerical bounds (``$\ge 70$'' and ``$\ge 18$'') come from explicitly summing $\sum_{5\le p\le 67}1/(p-1)$ and $\sum_{5\le p\le 71}1/(p-1)$, which can be verified exactly as rational numbers.
\end{itemize}

\subsection*{FINAL: UNRESOLVED}
No complete proof of existence or nonexistence of a \emph{distinct} covering system with all moduli of the form $p-1$ for primes $p\ge 5$ was obtained here.  The argument above provides only necessary conditions (at least $18$ congruences and at least one modulus $\ge 70$).

\subsection*{COMPLETION ESTIMATE (honest)}
To resolve the problem one would need either:
\begin{itemize}
\item an explicit construction of a distinct covering system using moduli $\{p_i-1\}$ with $p_i\ge 5$, or
\item a new obstruction specific to the arithmetic shape ``prime minus one'' that rules out all such constructions.
\end{itemize}



%%%%%%%%%%%%%%%%%%%%%%%%%%%%%%%%%%%%%%%%%%%%%%%%%%%%%%%%%%%%%%%%%%%%%%%%%%%%%%%
