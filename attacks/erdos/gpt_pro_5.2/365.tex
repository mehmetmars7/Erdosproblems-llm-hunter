\section*{Problem 365: Consecutive powerful numbers}

\subsection*{FORMAL RESTATEMENT}
A positive integer $m$ is \emph{powerful} (a.k.a. \emph{squarefull} or \emph{$2$-full}) if for every prime $p\mid m$ we have $p^2\mid m$.

The prompt asks (paraphrasing) whether every pair of consecutive powerful integers $(n,n+1)$ must arise from a Pell-type construction, and then says ``in other words'' whether one of $n,n+1$ must be a square.

\textbf{Ambiguity / minimal correction.} ``Comes from a Pell equation'' is not a mathematically precise property unless one specifies what counts as ``coming from.'' The only precise embedded claim is:

\medskip
\noindent\textbf{(\*)} \emph{If $n$ and $n+1$ are powerful, then at least one of $n$ or $n+1$ is a perfect square.}

I will disprove (\*) by an explicit counterexample.

\subsection*{QUICK LITERATURE/CONTEXT CHECK}
The prompt itself records that many examples come from Pell equations such as $x^2=8y^2+1$ (giving $8y^2$ powerful and $x^2$ a square), but also that Golomb found a counterexample with $12167,12168$, and that Walker produced infinitely many counterexamples via a Pell-type equation $7^3x^2=3^3y^2+1$. (These items are also summarized on the Erd\H{o}s-problems site and in OEIS A060355.)

\subsection*{ATTACK PLAN}
To disprove (\*), it suffices to find one $n$ such that
(i) $n$ is powerful,
(ii) $n+1$ is powerful,
(iii) neither $n$ nor $n+1$ is a square.

\subsection*{WORK (complete counterexample/disproof)}
Take
\[
 n := 12167.
\]
We claim that $n$ and $n+1$ are both powerful but neither is a square.

\medskip
\noindent\textbf{Step 1: $n$ is powerful.}
We have
\[
12167 = 23^3.
\]
Thus the only prime dividing $12167$ is $23$, and its exponent is $3\ge 2$. Hence $12167$ is powerful.

\medskip
\noindent\textbf{Step 2: $n+1$ is powerful.}
Compute
\[
12168 = 8\cdot 9\cdot 169 = 2^3\cdot 3^2\cdot 13^2.
\]
Every prime dividing $12168$ has exponent at least $2$ (namely $3,2,2$), so $12168$ is powerful.

\medskip
\noindent\textbf{Step 3: neither $n$ nor $n+1$ is a square.}
Since $12167=23^3$ has an odd exponent in its prime factorization, it cannot be a perfect square.
Similarly, $12168$ has $2$-adic valuation $v_2(12168)=3$ (odd), so it also cannot be a square.

\medskip
\noindent\textbf{Conclusion.}
The pair $(12167,12168)$ consists of consecutive powerful numbers, but neither term is a square. This contradicts (\*), hence (\*) is false.

\subsection*{VERIFICATION}
We explicitly exhibited complete factorizations:
\[
12167 = 23^3,\qquad 12168 = 2^3\cdot 3^2\cdot 13^2.
\]
Both are powerful because all prime exponents are $\ge 2$. Neither is a square because at least one exponent is odd.

\subsection*{META-CHECK (common failure modes)}
\begin{itemize}
\item \emph{Definition check:} Powerful means every prime exponent is $\ge 2$; both factorizations satisfy this.
\item \emph{Square check:} A square has all exponents even; both numbers have an odd exponent.
\item \emph{Off-by-one:} Verified that $12168=12167+1$.
\end{itemize}

\subsection*{FINAL}
FINAL: LABEL: \textbf{FULL SOLUTION} \quad SUBLABEL: \textbf{COUNTEREXAMPLE/DISPROOF}.

\subsection*{COMPLETION ESTIMATE}
COMPLETION ESTIMATE: 100\%.

