\section*{Problem 619}

\subsection*{FORMAL RESTATEMENT}

Let $G=(V,E)$ be a finite simple graph. For an integer $r\ge 2$, define $h_r(G)$ as follows (for triangle-free $G$):
\[
 h_r(G)\ :=\ \min\bigl\{|F|:\ F\subseteq \binom{V}{2}\setminus E,\ \text{$G+F$ is triangle-free and has diameter at most $r$}\bigr\}.
\]
(Here $G+F$ denotes the graph on the same vertex set with edge set $E\cup F$.)

The question in the prompt (Erd\H{o}s--Gy\'arf\'as--Ruszink\'o, 1998) asks whether there exists an absolute constant $c>0$ such that
\[
\forall\ \text{connected triangle-free graphs $G$ on $n$ vertices},\qquad h_4(G)\ <\ (1-c)n.
\]

\subsection*{QUICK LITERATURE/CONTEXT CHECK}

Erd\H{o}s--Gy\'arf\'as--Ruszink\'o (1998) prove that for every connected triangle-free $n$-vertex graph,
\[
 h_3(G)\le n-1,
\]
and they also prove that
\[
 h_5(G)\le \frac{n-1}{2}.
\]
They explicitly pose the diameter-$4$ question as an open problem (their Problem~4.3): whether the $n-1$ bound can be improved by a positive fraction.
They remark that without the triangle-free restriction, diameter-$4$ can be achieved by adding at most $n/2$ edges (and this is best possible), referring to work of Chung and Garey.

\subsection*{ATTACK PLAN}

The goal is to force diameter $\le 4$ while avoiding creation of triangles.
A natural sufficient condition is to make the augmented graph have \emph{radius $\le 2$}: if there is a vertex $x$ at distance at most 2 from every vertex, then diameter is at most 4.
However, in a triangle-free graph we may only add an edge $uv$ if $u$ and $v$ have no common neighbor (otherwise we would create a triangle).

Approaches:
\begin{enumerate}[label=(\arabic*),leftmargin=2.5em]
\item Prove the statement for special classes, e.g. bipartite graphs (where any added edge across a bipartition preserves triangle-freeness).
\item Try to extend the matching-based construction of Erd\H{o}s--Gy\'arf\'as--Ruszink\'o that gives diameter $\le 4$ with $n-\nu(G)$ added edges (where $\nu(G)$ is matching number), and find a complementary construction using $O(\nu(G))$ edges that also yields diameter $\le 4$.
\item Explore whether odd cycles force an obstruction (since triangle-free graphs may have large chromatic number). Possibly reduce to controlling a large bipartite ``core'' plus a small exceptional set.
\end{enumerate}

\subsection*{WORK}

\paragraph{A complete positive result for connected bipartite graphs.}
Let $G=(V,E)$ be connected and bipartite with bipartition $V=A\sqcup B$.
Assume without loss that $|A|\le |B|$.
Pick any vertex $a\in A$.
Let
\[
F\ :=\ \{\{a,b\}:\ b\in B\ \text{and}\ \{a,b\}\notin E\}.
\]
That is, we add all missing edges from $a$ to the opposite side $B$.

\begin{claim}
The augmented graph $H:=G+F$ is triangle-free and has diameter at most $4$.
\end{claim}
\begin{proof}
\emph{Triangle-free:}
$H$ is still bipartite with the same bipartition $A\sqcup B$, since all added edges go from $a\in A$ to $B$.
Any bipartite graph is triangle-free.

\emph{Diameter $\le 4$:}
We show that every vertex is within distance $\le 2$ of $a$ in $H$.
If $v\in B$, then $av\in E(H)$ by construction, so $\mathrm{dist}_H(a,v)=1$.
If $v\in A$, then since $G$ is connected, $v$ has a neighbor $b\in B$ in $G$ (hence also in $H$). Then $\mathrm{dist}_H(a,v)\le \mathrm{dist}_H(a,b)+\mathrm{dist}_H(b,v)=1+1=2$.
Thus the eccentricity of $a$ in $H$ is at most $2$, and therefore the diameter of $H$ is at most $4$.
\end{proof}

\begin{claim}
The number of added edges satisfies $|F|\le |B|\le \tfrac{n}{2}$.
\end{claim}
\begin{proof}
We add at most one edge from $a$ to each vertex of $B$, hence $|F|\le |B|$. Since $|A|\le |B|$ and $|A|+|B|=n$, we have $|B|\le n/2$.
\end{proof}

\paragraph{Implication.}
For connected bipartite graphs, the desired inequality holds with $c=1/2$:
\[
\boxed{\text{If $G$ is connected bipartite on $n$ vertices, then } h_4(G)\le \frac{n}{2}.}
\]

\paragraph{Why this does not settle the triangle-free case.}
If $G$ is triangle-free but non-bipartite, adding edges can easily create triangles due to existing odd-cycle structure: an added edge $uv$ is forbidden whenever $u$ and $v$ share a neighbor.
The bipartite ``join one side to a hub'' construction above relies crucially on the existence of a bipartition, which fails in general.

\paragraph{Connection to the matching-based construction in EGR (sketch only).}
Erd\H{o}s--Gy\'arf\'as--Ruszink\'o show (among other things) that if $\nu(G)=t$ is the matching number, then one can build a triangle-free supergraph of diameter at most $4$ by adding at most $n-t$ edges.
This is useful when $t$ is a positive fraction of $n$, but it does not yield a uniform $c>0$ because $t$ can be small (e.g. a star has $t=1$) while already having small diameter.
The missing ingredient is a complementary ``$O(t)$-edge'' diameter-$4$ augmentation to combine with $n-t$.

\subsection*{VERIFICATION}

\begin{itemize}[leftmargin=2.2em]
\item In the bipartite construction, triangle-freeness is immediate because bipartite graphs have no odd cycles, in particular no 3-cycles.
\item The diameter bound is a direct radius-$2$ argument.
\item Edge count is deterministic and worst-case sharp up to constants (balanced complete bipartite graphs already have diameter $2$ so need $0$ edges; paths show that $\Theta(n)$ edges can be needed in some sparse bipartite cases, but the bound $n/2$ is uniform).
\end{itemize}

\subsection*{FINAL}

\textbf{UNRESOLVED}

\begin{enumerate}[label=(\roman*),leftmargin=2.5em]
\item \textbf{Where and why the attempt fails.}
The obstacle is extending the clean bipartite ``make a radius-2 hub'' augmentation to \emph{all} triangle-free graphs. In non-bipartite triangle-free graphs, adding many edges incident to a hub vertex typically creates triangles because endpoints may have common neighbors. I did not find a uniform way to add $<(1-c)n$ edges (with fixed $c>0$) that guarantees diameter $\le 4$ while maintaining triangle-freeness.

\item \textbf{Strongest partial result proved here.}
If $G$ is connected and bipartite on $n$ vertices, then $h_4(G)\le n/2$ by an explicit construction (add all missing edges from a vertex in the smaller side to the opposite side).

\item \textbf{Most plausible next ideas.}
Try to adapt the Erd\H{o}s--Gy\'arf\'as--Ruszink\'o matching-based approach: find an augmentation with $O(\nu(G))$ edges that achieves diameter $\le 4$ (not $\le 5$), and then combine with their $n-\nu(G)$ augmentation to obtain a uniform fraction improvement.
Alternatively, attempt a decomposition of an arbitrary triangle-free graph into a large bipartite subgraph plus a controlled exceptional set, and run the bipartite augmentation on the large part while carefully connecting the exceptional vertices without creating triangles.

\item \textbf{Small cases / computations.}
No exhaustive search was performed. The bipartite lemma can be checked on small examples (paths, even cycles) by hand; it gives the correct qualitative behavior (diameter becomes $\le 4$ after adding $O(n)$ edges).
Determining the true worst-case behavior of $h_4(G)$ over all triangle-free $G$ appears to require ideas beyond this basic augmentation.
\end{enumerate}

\subsection*{COMPLETION ESTIMATE}

COMPLETION ESTIMATE: 30\%.

%%%%%%%%%%%%%%%%%%%%%%%%%%%%%%%%%%%%%%%%%%%%%%%%%%%%%%%%%%%%%%%%%%%%%%%%%%%%%%%
