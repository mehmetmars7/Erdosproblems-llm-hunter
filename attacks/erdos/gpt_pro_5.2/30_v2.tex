\section{Round 2 (Problem 30): Sidon sets in $\{1,\dots,N\}$}

\subsection{1) Round-2 Objective}
\textbf{Path (A): proof / gap-closure.}
Building on Round 1, we close the main gap left by the elementary counting bound by proving the classical Erd\H{o}s--Tur\'an/Lindstr\"om upper bound
\[
h(N)\le \sqrt N+N^{1/4}+1,
\]
thereby improving the Round-1 bound $h(N)\le (\sqrt2+o(1))\sqrt N$. The conjectural refinement
$h(N)=\sqrt N+O_\varepsilon(N^\varepsilon)$ remains open.

\subsection{2) Round-1 Foundation Used}
We use only the following Round-1 result.

\begin{itemize}
\item \textbf{(R1 Lemma 30.1)} If $S=\{a_1<\cdots<a_r\}\subseteq\{1,\dots,N\}$ is Sidon, then all positive differences $a_j-a_i$ $(1\le i<j\le r)$ are distinct.
\end{itemize}

\subsection{3) New Insight / Tool (Round-2)}
\textbf{Order-of-differences summation + telescoping.}
For each order $\nu=j-i$ we sum the order-$\nu$ differences and bound this sum by a telescoping argument.
Summing over $\nu\le m$ gives an upper bound on the sum of many \emph{distinct} positive integers; comparing to the minimal possible sum $1+2+\cdots+D$ yields a quadratic inequality.
Optimizing $m\asymp N^{1/4}$ produces the $N^{1/4}$ error term.

\subsection{4) Attack Plan (Round-2)}
\textbf{Gap after Round 1:} only a counting upper bound $\sim \sqrt{2N}$ was proved.

\textbf{What to prove now:} for every Sidon $S\subseteq\{1,\dots,N\}$,
\[
|S|\le \sqrt N+N^{1/4}+1.
\]

\textbf{Why this closes the gap:} it uses structure (telescoping constraints on low-order differences) rather than mere counting, forcing the correct leading constant $1$.

\subsection{5) Work (Round-2): Lindstr\"om bound}

\begin{theorem}[Lindstr\"om upper bound]\label{thm:lindstrom}
If $S\subseteq\{1,\dots,N\}$ is Sidon and $|S|=r$, then
\[
r\le \sqrt N+N^{1/4}+1.
\]
Hence $h(N)\le \sqrt N+N^{1/4}+1$.
\end{theorem}

\begin{proof}
Write $S=\{a_1<a_2<\cdots<a_r\}$. By (R1 Lemma 30.1), all positive differences $a_j-a_i$ $(i<j)$ are distinct.

\medskip\noindent
\textbf{Step 1: order-$\nu$ sums.}
For $1\le \nu\le r-1$, define
\[
\Sigma_\nu:=\sum_{i=1}^{r-\nu}(a_{i+\nu}-a_i).
\]
\emph{Claim:} $\Sigma_\nu\le \nu N$ for every $\nu$.
Indeed, partition indices $i$ into residue classes mod $\nu$.
For a fixed residue class $k\in\{1,\dots,\nu\}$, the partial sum over $i=k,k+\nu,k+2\nu,\dots$ telescopes to
$a_{\text{last}+\nu}-a_k\le a_r-a_1\le N-1<N$.
Summing over the $\nu$ residue classes gives $\Sigma_\nu<\nu N$, hence $\Sigma_\nu\le \nu N$.

\medskip\noindent
\textbf{Step 2: sum over $\nu\le m$ and compare two bounds.}
Fix an integer $m\ge 1$.
If $m\ge r$, then $r\le m\le \lceil N^{1/4}\rceil\le N^{1/4}+1\le \sqrt N+N^{1/4}+1$ and we are done.
So assume $m\le r-1$.

Let
\[
\Sigma(1,m):=\sum_{\nu=1}^m \Sigma_\nu.
\]
Using $\Sigma_\nu\le \nu N$ gives
\begin{equation}\label{eq:upper}
\Sigma(1,m)\le \left(\sum_{\nu=1}^m \nu\right)N=\frac{m(m+1)}{2}\,N.
\end{equation}

On the other hand, $\Sigma(1,m)$ sums all differences of order at most $m$:
$a_{i+\nu}-a_i$ with $1\le \nu\le m$ and $1\le i\le r-\nu$.
The number of such differences is
\[
D=\sum_{\nu=1}^m (r-\nu)=mr-\frac{m(m+1)}{2}.
\]
These are distinct positive integers (by (R1 Lemma 30.1)), so their sum is at least the minimum possible sum of $D$ distinct positive integers:
\begin{equation}\label{eq:lower}
\Sigma(1,m)\ge 1+2+\cdots+D=\frac{D(D+1)}{2}.
\end{equation}
Combining \eqref{eq:upper} and \eqref{eq:lower} yields
\[
D(D+1)\le m(m+1)N,
\]
hence in particular $D^2\le m(m+1)N$.

\medskip\noindent
\textbf{Step 3: isolate $r$.}
Write
\[
D=mr-\frac{m(m+1)}{2}=m\left(r-\frac{m+1}{2}\right).
\]
Let $s:=r-\frac{m+1}{2}$, so $D=ms$. Then $D^2\le m(m+1)N$ implies
\[
m^2s^2\le m(m+1)N\quad\Rightarrow\quad s^2\le N\left(1+\frac{1}{m}\right),
\]
so
\[
r=s+\frac{m+1}{2}\le \sqrt N\sqrt{1+\frac{1}{m}}+\frac{m+1}{2}.
\]
Using $\sqrt{1+x}\le 1+\frac{x}{2}$ for $x\ge 0$ (with $x=1/m$) gives
\[
r\le \sqrt N+\frac{\sqrt N}{2m}+\frac{m+1}{2}.
\]

\medskip\noindent
\textbf{Step 4: optimize $m$.}
Choose $m=\lceil N^{1/4}\rceil$.
Then $\frac{\sqrt N}{2m}\le \frac{N^{1/4}}{2}$ and $\frac{m+1}{2}\le \frac{N^{1/4}}{2}+1$, hence
\[
r\le \sqrt N+\frac{N^{1/4}}{2}+\left(\frac{N^{1/4}}{2}+1\right)=\sqrt N+N^{1/4}+1.
\]
This proves the theorem.
\end{proof}

\subsection{6) Adversarial Verification}
Potential failure modes are checked as follows.

\begin{itemize}
\item \emph{Quantifiers / edge cases:} if the chosen $m$ exceeds $r-1$ the bound is trivial (handled explicitly).
\item \emph{Distinctness of differences:} used only through (R1 Lemma 30.1).
\item \emph{Telescoping bound:} each residue-class partial sum telescopes to at most $a_r-a_1\le N-1$; using $\nu N$ is safe.
\item \emph{Lower bound on sum:} for $D$ distinct positive integers, the minimal sum is $1+\cdots+D$.
\end{itemize}

\subsection{7) Final}
\textbf{UNRESOLVED (but strictly advanced).}
Round 2 proves the classical bound $h(N)\le \sqrt N+N^{1/4}+1$, a strict improvement over Round 1.
The conjecture $h(N)=\sqrt N+O_\varepsilon(N^\varepsilon)$ remains open.

\subsection{8) Completion Estimate}
\textbf{COMPLETION: 65\%}

\subsection{9) References}
No external references are required for the proof presented here; it is self-contained modulo (R1 Lemma 30.1).
