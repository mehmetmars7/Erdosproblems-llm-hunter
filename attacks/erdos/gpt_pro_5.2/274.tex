\section*{Problem \#274}

\subsection*{FORMAL RESTATEMENT}
Let $G$ be a group and let $a_iH_i$ $(1\le i\le k)$ be left cosets of subgroups $H_i\le G$.
An \emph{exact coset covering} (or \emph{coset partition}) is a finite family $\{a_iH_i\}_{i=1}^k$ such that
\[
G \;=\; \bigsqcup_{i=1}^k a_iH_i
\]
(disjoint union).

Write $d_i=[G:H_i]$ for the (finite) index of $H_i$ in $G$; then $|a_iH_i|=|H_i|=|G|/d_i$ if $G$ is finite.
The question asks whether a nontrivial coset partition can occur with \emph{all indices $d_i$ distinct} (equivalently, all cosets have different sizes in the finite case).

\medskip
\noindent\textbf{Herzog--Sch\"onheim conjecture (general form).}
If $\{a_iH_i\}_{i=1}^k$ is a coset partition of $G$ with all indices $[G:H_i]$ finite, then at least two of the indices are equal.

\subsection*{QUICK LITERATURE/CONTEXT CHECK}
The conjecture was posed by Herzog and Sch\"onheim (1974).  It is proved for finite nilpotent groups (hence for finite abelian groups) by Berger--Felzenbaum--Fraenkel (1986).  Many further partial results are known (e.g.\ for various classes of solvable/subnormal situations).  As of 2025, the general conjecture for arbitrary groups remains open, but there has been recent progress for certain families (e.g.\ symmetric and nonabelian simple groups in a 2025 preprint by Garonzi--Margolis).

\subsection*{ATTACK PLAN}
\begin{enumerate}
\item Prove a clean special case that is fully self-contained (here: $p$-groups) to illustrate the key obstruction.
\item Explain how this feeds into the known nilpotent/abelian results.
\item Summarize the remaining gap to the full conjecture and what kinds of tools seem relevant (group actions, representation theory, subgroup structure, normal series reductions).
\end{enumerate}

\subsection*{WORK}
\paragraph{A fully self-contained special case: finite $p$-groups.}
\begin{theorem}\label{thm:pgroup}
Let $G$ be a finite group of order $p^n$ (a $p$-group).  In any coset partition
\[
G \;=\; \bigsqcup_{i=1}^k a_iH_i
\qquad (k\ge 2),
\]
at least two indices $[G:H_i]$ are equal.  In particular, $G$ cannot be partitioned into cosets of pairwise distinct indices.
\end{theorem}

\begin{proof}
For each $i$, the index $d_i=[G:H_i]$ is a power of $p$, say $d_i=p^{e_i}$ with $0\le e_i\le n$.
Since the cosets are disjoint and all have size $|H_i|=|G|/d_i$, we have
\[
|G|
\;=\;
\sum_{i=1}^k |a_iH_i|
\;=\;
\sum_{i=1}^k |H_i|
\;=\;
\sum_{i=1}^k \frac{|G|}{d_i}.
\]
Dividing by $|G|$ gives an \emph{exact} Egyptian-fraction identity
\[
\sum_{i=1}^k \frac{1}{p^{e_i}} \;=\; 1.
\]
Multiply by $p^{E}$ where $E=\max_i e_i$:
\[
\sum_{i=1}^k p^{E-e_i} \;=\; p^{E}.
\]
If all indices were distinct then all exponents $e_i$ would be distinct, hence all integers $E-e_i$ would be distinct.
Thus the left-hand side would be a sum of \emph{distinct} powers of $p$ (including $p^0=1$ coming from any $i$ with $e_i=E$).
But the base-$p$ expansion of a sum of distinct powers of $p$ has only digits $0$ or $1$ in each position, so it cannot equal $p^{E}$, whose base-$p$ expansion is $1$ followed by $E$ zeros.
This contradiction shows that the indices cannot all be distinct.
\end{proof}

\paragraph{Finite abelian (indeed finite nilpotent) groups.}
A much stronger result is known:
\begin{quote}
Any coset partition of a finite \emph{nilpotent} group into at least two cosets contains two cosets of the same index.
\end{quote}
This was proved by Berger--Felzenbaum--Fraenkel (1986) by translating the problem to a partition of a finite Cartesian product (a ``parallelepiped'') by ``product sets'' whose side lengths are prime powers, and then proving a combinatorial multiplicity theorem for such partitions.

Since every finite abelian group is nilpotent, this yields the statement quoted in the problem: in abelian groups one cannot have an exact coset covering by more than one coset of pairwise different sizes.

\paragraph{What remains open.}
For general (possibly non-nilpotent, possibly infinite) groups the Herzog--Sch\"onheim conjecture is still open in full generality.  There are partial results for various classes of groups and subgroups; recent progress includes verification for broad families such as symmetric groups and nonabelian finite simple groups (as reported in a 2025 preprint by Garonzi--Margolis).

\subsection*{VERIFICATION}
\begin{itemize}
\item The $p$-group proof (Theorem~\ref{thm:pgroup}) is a complete standalone argument using only index counting and uniqueness of base-$p$ representation.
\item The reduction ``finite abelian $\Rightarrow$ finite nilpotent'' is standard: finite abelian groups are nilpotent because they are direct products of their Sylow subgroups.
\item The remaining open status for general groups is consistent with the current research literature (as of late 2025) which contains many partial cases but no general resolution.
\end{itemize}

\subsection*{FINAL: UNRESOLVED}
The general Herzog--Sch\"onheim conjecture for arbitrary groups is not resolved here.
A complete proof is provided for the special case of finite $p$-groups, and the known (published) result for finite nilpotent groups is recorded, which covers all finite abelian groups.

\subsection*{COMPLETION ESTIMATE (honest)}
A full resolution would require either:
\begin{itemize}
\item a general proof for all groups (likely via structural reductions to finite groups and then to composition factors, together with control of coset partitions under extensions/quotients), or
\item an explicit counterexample: a group admitting a finite coset partition with all indices distinct.
\end{itemize}



%%%%%%%%%%%%%%%%%%%%%%%%%%%%%%%%%%%%%%%%%%%%%%%%%%%%%%%%%%%%%%%%%%%%%%%%%%%%%%%
