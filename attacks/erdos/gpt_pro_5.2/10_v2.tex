\section{Round 2 (v2): Infinite family of even integers not $3$-representable}

\subsection{1) ROUND-2 OBJECTIVE}

\noindent\textbf{Path (C): obstruction / correction.}
We prove a structural obstruction strictly beyond Round~1: \emph{there are infinitely many even integers that are not $3$-representable}. Consequently, any universal constant $k$ (if it exists) must satisfy $k\ge 4$.

\subsection{2) ROUND-1 FOUNDATION USED}

We use the following Round~1 results (without re-proving them):

\begin{itemize}
\item \textbf{Lemma 10.1 (binary-weight reformulation).} An integer $n$ is $k$-representable iff there exists $m\ge 0$ with $\mathrm{wt}_2(m)\le k$ such that $n-m$ is prime.
\item \textbf{Lemma 10.2 (parity reduction for even targets).} If $n$ is even and $n=p+S$ with $p$ an odd prime and $\mathrm{wt}_2(S)\le k$, then $S=1+m$ with $m$ even and $\mathrm{wt}_2(m)\le k-1$, so $n-1=p+m$.
\end{itemize}

\subsection{3) NEW INSIGHT / TOOL (ROUND-2)}

\noindent\textbf{New external input: Crocker (1971).}
Crocker constructs infinitely many odd integers $t$ with
\begin{enumerate}
\item $t\equiv -1 \pmod{16}$,
\item $t$ is not representable as $p+2^a$ with $p$ prime and $a\ge 1$,
\item $t$ is not representable as $p+2^a+2^b$ with $p$ prime and $a,b\ge 1$.
\end{enumerate}
(We use this as a cited theorem; see References.)

\medskip
\noindent\textbf{New local lemma (proved here).}
From the congruence $t\equiv -1\pmod{16}$ we deduce a Hamming-weight lower bound
$\mathrm{wt}_2(t-1)\ge 4$ for all $t>15$, which blocks any $k=3$ representation using the even prime $2$.

\subsection{4) ATTACK PLAN (ROUND-2)}

\noindent\textbf{Round~1 gap.} Round~1 verified one explicit even $k=3$ counterexample, but had no infinite mechanism.

\medskip
\noindent\textbf{Plan.} Take $t$ from Crocker's infinite family and set $n=t+1$ (even). If $n$ had a $3$-representation $n=p+m$ with $\mathrm{wt}_2(m)\le 3$, then:
\begin{itemize}
\item if $p$ is odd, Lemma~10.2 forces $t=n-1$ to be representable as $p+m'$ with $m'$ even and $\mathrm{wt}_2(m')\le 2$, i.e. $m'\in\{0,2^a,2^a+2^b\}$ with $a,b\ge 1$; Crocker forbids this;
\item if $p=2$, then $m=n-2=t-1$ and we would need $\mathrm{wt}_2(t-1)\le 3$, contradicted by the new Hamming-weight lemma coming from $t\equiv -1\pmod{16}$.
\end{itemize}
Hence each such $n$ is a $k=3$ counterexample, giving infinitely many.

\subsection{5) WORK (ROUND-2)}

\subsubsection{5.1 Congruence-to-weight lemma}

\textbf{Lemma 5.1.}
Let $r\ge 1$. If $t\equiv -1\pmod{2^r}$ and $t>2^r-1$, then
\[
\mathrm{wt}_2(t-1)\ \ge\ r.
\]
In particular, if $t\equiv -1\pmod{16}$ and $t>15$, then $\mathrm{wt}_2(t-1)\ge 4$.

\textbf{Proof.}
Write $t=2^r q+(2^r-1)$ with an integer $q\ge 1$. Then
\[
 t-1=2^r q+(2^r-2).
\]
The number $2^r-2$ has binary form $(111\cdots 110)_2$ (with exactly $r-1$ ones), and $2^r q$ has its lowest $r$ bits equal to $0$, so there is no carry interaction when adding these two summands. Therefore
\[
\mathrm{wt}_2(t-1)=\mathrm{wt}_2(2^r q)+\mathrm{wt}_2(2^r-2)=\mathrm{wt}_2(q)+(r-1)\ge 1+(r-1)=r.
\]
\qed

\subsubsection{5.2 Infinite family of even integers not $3$-representable}

\textbf{Theorem 5.2 (Crocker, 1971; extracted form).}
There exist infinitely many odd integers $t>15$ such that:
\begin{enumerate}
\item $t\equiv -1\pmod{16}$;
\item $t\neq p+2^a$ for any prime $p$ and any integer $a\ge 1$;
\item $t\neq p+2^a+2^b$ for any prime $p$ and integers $a,b\ge 1$.
\end{enumerate}

\medskip
\textbf{Theorem 5.3 (Round~2 main theorem).}
There exist infinitely many \emph{even} integers $n$ that are \emph{not} $3$-representable.

\textbf{Proof.}
Let $t$ be an odd integer satisfying Theorem~5.2, and set $n:=t+1$ (even).
Assume for contradiction that $n$ is $3$-representable. By Lemma~10.1 there exists $m\ge 0$ with $\mathrm{wt}_2(m)\le 3$ such that $p:=n-m$ is prime, i.e.
\[
 n=p+m,\qquad \mathrm{wt}_2(m)\le 3.
\]
We split into cases.

\smallskip
\noindent\emph{Case 1: $p$ is an odd prime.}
Then $m=n-p$ is odd. Apply Lemma~10.2 with $k=3$ to write $m=1+m'$ where $m'$ is even and $\mathrm{wt}_2(m')\le 2$. Hence
\[
 t=n-1=(p+m)-1=p+m'.
\]
Because $m'$ is even and $\mathrm{wt}_2(m')\le 2$, we have $m'\in\{0,2^a,2^a+2^b\}$ with $a,b\ge 1$ (distinct in the last option). This contradicts Theorem~5.2.

\smallskip
\noindent\emph{Case 2: $p=2$.}
Then $m=n-2=t-1$, so $\mathrm{wt}_2(t-1)\le 3$. But $t\equiv -1\pmod{16}$ and $t>15$, so Lemma~5.1 (with $r=4$) gives $\mathrm{wt}_2(t-1)\ge 4$, contradiction.

\smallskip
Both cases contradict, hence $n$ is not $3$-representable. Since Theorem~5.2 provides infinitely many such $t$, we obtain infinitely many even counterexamples $n=t+1$.
\qed

\subsubsection{5.3 Consequence for the original question}

\textbf{Corollary 5.4.}
If there exists an absolute $k$ such that every integer $n\ge 2$ is $k$-representable, then necessarily $k\ge 4$.

\textbf{Proof.}
Theorem~5.3 gives infinitely many integers that are not $3$-representable, so no universal $k\le 3$ can work.
\qed

\subsection{6) ADVERSARIAL VERIFICATION}

\begin{itemize}
\item \textbf{Repetitions of powers of $2$.} Eliminated by Lemma~10.1: allowing repetition is equivalent to a Hamming-weight bound.
\item \textbf{The $p=2$ edge case.} Handled separately via Lemma~5.1. Note that the exceptional small value $t=15$ would fail Lemma~5.1, but Crocker's family contains infinitely many $t>15$.
\item \textbf{$m=0$ possibility.} For $n=t+1>16$ even, $m=0$ would force $p=n$ an even prime, impossible (only $2$ is even prime).
\item \textbf{Use of Crocker.} We only use the three listed properties (including the imposed congruence $t\equiv -1\!\!\pmod{16}$) and the fact that infinitely many such $t$ exist.
\end{itemize}

\subsection{7) FINAL (EXACTLY ONE)}

\noindent\textbf{UNRESOLVED (BUT STRICTLY ADVANCED).}
We proved (using Round~1 Lemmas~10.1--10.2 and Crocker (1971)) that there are infinitely many even integers not $3$-representable. Hence any universal $k$ must satisfy $k\ge 4$. The existence of a universal $k$ remains open.

\subsection{8) COMPLETION ESTIMATE (MANDATORY)}

\noindent\textbf{COMPLETION: 45\%}

\subsection{9) REFERENCES}

\begin{enumerate}
\item R. Crocker, \emph{On the sum of a prime and of two powers of two}, Pacific J. Math. 36 (1971), 103--107.
\end{enumerate}
