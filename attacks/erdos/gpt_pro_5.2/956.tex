% Erdos problem 956

Erdos Problem 956.

1) FORMAL RESTATEMENT.
For compact sets $C,D\subseteq\mathbb{R}^2$, define the Euclidean distance between them by
\[
\delta(C,D):=\inf\{\|c-d\|: c\in C,\ d\in D\}.
\]
For $n\in\mathbb{N}$, define $h(n)$ as the maximum possible number of pairs $\{x_1,x_2\}$ with $x_1\ne x_2$ in a set $X\subseteq\mathbb{R}^2$ of size $n$ such that there exists a compact convex set $C\subseteq\mathbb{R}^2$ with:

• The translates $\{C+x: x\in X\}$ are pairwise disjoint, and

• For each counted pair $\{x_1,x_2\}$ we have $\delta(C+x_1, C+x_2)=1$.

Question: determine $h(n)$. In particular, is there a constant $c>0$ such that for all sufficiently large $n$,
\[
 h(n) > n^{1+c}?
\]

2) QUICK LITERATURE/CONTEXT CHECK.
The problem file states a known upper bound $h(n)\ll n^{4/3}$ due to Erd\H{o}s and Pach, and asks for superlinear lower bounds.

3) ATTACK PLAN.
(A) Rewrite distances between translates in terms of translation vectors and the Minkowski difference $C-C$.
(B) Compare $h(n)$ to the classical unit-distance extremal function $f(n)$ for point sets.
(C) Do small $n$ sanity checks (what happens for $n=2,3$?).

4) WORK.

FAST REALITY CHECK (tiny $n$).
For $n=2$, choose any compact convex $C$ and two translation vectors so that the two disjoint translates are exactly distance $1$ apart; hence $h(2)=1$.
For $n=3$, taking $C$ to be a disk of radius $r>0$ and placing centers at the vertices of an equilateral triangle of side length $2r+1$ gives three disjoint translates with pairwise distance $1$, so $h(3)=3$.

Lemma 956.1 (Distance between translates via Minkowski difference).
Let $C\subseteq\mathbb{R}^2$ be nonempty and compact. Define the (compact) set
\[
K:=C-C:=\{c_1-c_2: c_1,c_2\in C\}.
\]
Then for all $x,y\in\mathbb{R}^2$,
\[
\delta(C+x, C+y) = \operatorname{dist}(y-x, K),
\]
where $\operatorname{dist}(v,K):=\inf\{\|v-k\|:k\in K\}$.

Proof.
By definition,
\[
\delta(C+x,C+y)=\inf_{c_1,c_2\in C}\| (c_1+x)-(c_2+y)\|
 =\inf_{c_1,c_2\in C}\|(c_1-c_2)-(y-x)\|.
\]
The set of differences $c_1-c_2$ as $c_1,c_2$ range over $C$ is exactly $K$, so the right-hand side equals
$\inf_{k\in K}\|k-(y-x)\|=\operatorname{dist}(y-x,K)$.
\hfill $\square$

Lemma 956.2 (Trivial reduction: $h(n)\ge f(n)$).
Let $f(n)$ denote the maximum number of unit-distance pairs among $n$ points in the plane (the classical unit distance problem).
Then $h(n)\ge f(n)$ for all $n$.

Proof.
Let $P\subset\mathbb{R}^2$ be a set of $n$ points determining $f(n)$ unit-distance pairs.
Let $d_{\min}:=\min\{\|p-q\|:p,q\in P,\ p\ne q\}>0$.
Choose $r>0$ so small that $(1+2r)d_{\min}>2r$.
Let $C$ be the closed disk of radius $r$ centered at the origin.
Define $X:=(1+2r)P:=\{(1+2r)p:p\in P\}$.

Disjointness: for any distinct $p,q\in P$ we have
\[\|(1+2r)p-(1+2r)q\|=(1+2r)\|p-q\|\ge (1+2r)d_{\min}>2r,\]
so the disks $C+(1+2r)p$ and $C+(1+2r)q$ are disjoint.

Unit-distance pairs transfer: if $\|p-q\|=1$, then the distance between the two disks equals
\[\delta\bigl(C+(1+2r)p,\ C+(1+2r)q\bigr)=\|(1+2r)(p-q)\|-2r=(1+2r)\cdot 1-2r=1.\]
Thus every unit-distance pair in $P$ yields a unit-distance pair of disjoint translates.
Therefore $h(n)\ge f(n)$.
\hfill $\square$

5) VERIFICATION.
• Lemma 956.1 is an exact identity obtained by rewriting the infimum.
• Lemma 956.2 depends only on choosing $r$ small enough compared to the minimum separation in $P$.

6) FINAL.
UNRESOLVED.
(i) Strongest proved partial result: $h(n)\ge f(n)$ for all $n$, and for disks $C$ the translate-distance condition reduces to a fixed-distance condition among centers.
(ii) First gap: produce an explicit construction of disjoint translates with more than $n^{1+c}$ pairs at distance $1$ for some fixed $c>0$ (without appealing to external results about $f(n)$ beyond the definition).
(iii) Top 3 next moves:
  (1) Choose special convex bodies $C$ (very elongated, “almost flat”) and try to construct configurations where many translation differences lie on the distance-$1$ offset of $C-C$.
  (2) Attempt to adapt lattice-based constructions: place translation vectors on a grid and choose $C$ so that $\operatorname{dist}(\cdot,C-C)=1$ holds for many displacement vectors.
  (3) Explore whether the disjointness constraint can be relaxed via packing arguments to permit denser graphs than in the classical unit distance problem.
(iv) Minimal counterexample structure: if the conjectured superlinear lower bound fails, then for every convex $C$ and every $X$ of size $n$, the number of differences $y-x$ at distance exactly $1$ from $C-C$ is $O(n^{1+o(1)})$.


