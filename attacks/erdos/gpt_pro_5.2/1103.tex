
\noindent\textbf{FORMAL RESTATEMENT.}
Let $A\subset\mathbb Z$ be an infinite set.
Interpreting growth as in the problem text, we take the minimal correction that $A\subset\mathbb N$ is infinite and write it in increasing order as $A=\{a_1<a_2<\cdots\}$.
Assume that every sum $a_i+a_j$ (allowing $i=j$) is \emph{squarefree}, i.e. for every prime $p$ one has $p^2\nmid (a_i+a_j)$.
Question: how small can $a_j$ be as a function of $j$? Equivalently, how large can $|A\cap[1,X]|$ be as a function of $X$?

\medskip
\noindent\textbf{QUICK LITERATURE/CONTEXT CHECK (from the problem file only).}
The text states that Erd\H{o}s produced an exponentially growing example, expected no polynomial-growth example, and that van Doorn--Tao (2025) prove a lower bound $a_j>0.24j^{4/3}$ and an upper construction $a_j<\exp(5j/\log j)$ for large $j$.
I do not reprove these claims here.

\medskip
\noindent\textbf{ATTACK PLAN.}
\begin{itemize}
\item Extract immediate necessary congruence conditions from the squarefree constraint, especially modulo $p^2$.
\item Use these congruence restrictions to get any unconditional growth lower bound.
\item Reality check: brute force compute the maximum size of a subset of $\{1,\dots,N\}$ with squarefree pairwise sums for small $N$.
\end{itemize}

\medskip
\noindent\textbf{WORK.}

\medskip
\noindent\textbf{Lemma 1 (each element is odd and squarefree).}
If $A+A$ consists entirely of squarefree integers, then every $a\in A$ is odd and squarefree.

\noindent\emph{Proof.}
Take any $a\in A$. Then $2a=a+a\in A+A$, hence $2a$ is squarefree.
If $a$ were even, then $2a$ would be divisible by $4$, contradicting squarefreeness.
So $a$ is odd.
If $p$ is an odd prime with $p^2\mid a$, then $p^2\mid 2a$, again contradicting squarefreeness of $2a$.
Thus no odd prime square divides $a$.
Together with oddness, this means $a$ is squarefree. \qed

\medskip
\noindent\textbf{Lemma 2 (no opposite residues modulo $p^2$).}
Let $p$ be an odd prime and let $S_p\subset\mathbb Z/p^2\mathbb Z$ be the set of residues attained by $A$ modulo $p^2$:
\[S_p:=\{a\bmod p^2: a\in A\}.
\]
Then $S_p\cap(-S_p)=\varnothing$. In particular,
\[|S_p|\le \frac{p^2-1}{2}.
\]

\noindent\emph{Proof.}
If $r\in S_p\cap(-S_p)$, then there exist $a,b\in A$ with
$a\equiv r\pmod{p^2}$ and $b\equiv -r\pmod{p^2}$.
Then $a+b\equiv 0\pmod{p^2}$, so $p^2\mid(a+b)$, contradicting the assumption that $a+b$ is squarefree.
Thus $S_p\cap(-S_p)=\varnothing$.
The map $x\mapsto -x$ is a bijection of $\mathbb Z/p^2\mathbb Z$, so $|S_p|=|-S_p|$.
Because $S_p$ and $-S_p$ are disjoint subsets of a set of size $p^2$, we have $2|S_p|\le p^2$.
Since $p^2$ is odd this implies $|S_p|\le (p^2-1)/2$. \qed

\medskip
\noindent\textbf{Lemma 3 (mod $4$ rigidity).}
All elements of $A$ lie in the same odd residue class modulo $4$; i.e. either $A\subseteq\{n: n\equiv 1\pmod 4\}$ or $A\subseteq\{n: n\equiv 3\pmod 4\}$.

\noindent\emph{Proof.}
By Lemma 1, all $a\in A$ are odd, so $a\equiv 1$ or $3\pmod 4$.
If there exist $a,b\in A$ with $a\equiv 1\pmod 4$ and $b\equiv 3\pmod 4$, then $a+b\equiv 0\pmod 4$.
Thus $4\mid(a+b)$, so $a+b$ is not squarefree, contradicting the hypothesis.
Therefore all elements are congruent to $1\pmod 4$ or all are congruent to $3\pmod 4$. \qed

\medskip
\noindent\textbf{FAST REALITY CHECK (finite search).}
For each $N$ I computed (by backtracking search) the maximum possible size of a subset $A\subseteq\{1,2,\dots,N\}$ such that every $a+b$ with $a,b\in A$ is squarefree.
The exact maxima found were:
\[
\begin{array}{c|c|l}
N & \max|A| & \text{one maximiser }A\\\hline
10 & 2 & \{3,7\}\\
20 & 3 & \{11,15,19\}\\
30 & 4 & \{11,15,19,23\}\\
50 & 6 & \{1,5,29,33,37,41\}\\
60 & 7 & \{11,15,19,23,51,55,59\}\\
100 & 8 & \{23,35,51,59,71,83,87,95\}\\
150 & 12 & \{5,17,29,41,53,65,77,89,105,113,141,149\}\\
200 & 16 & \{7,15,23,51,59,71,87,95,107,115,123,131,151,159,167,195\}
\end{array}
\]
These are small-$N$ sanity checks only; they do not determine asymptotics.

\medskip
\noindent\textbf{VERIFICATION.}
\begin{itemize}
\item Lemma 1: used only that $a+a\in A+A$ and that squarefree means not divisible by $4$ and not divisible by $p^2$ for odd primes.
\item Lemma 2: checked that the conclusion uses $p$ odd so that $p^2$ is odd (no nontrivial element equals its own negative).
\item Lemma 3: verified that mixing residues $1$ and $3$ mod $4$ forces a sum divisible by $4$.
\item Finite search: I verified the reported witness sets by directly checking the squarefree condition for all pairs.
\end{itemize}

\medskip
\noindent\textbf{FINAL: \textbf{UNRESOLVED}.}
(i) \emph{Strongest proved partial result here.} Any such $A$ must consist of odd squarefree integers all in a single residue class mod $4$ (Lemmas 1 and 3), and for each odd prime $p$ the residue set $A\bmod p^2$ avoids opposite residues, hence has size at most $(p^2-1)/2$ (Lemma 2).

(ii) \emph{First gap (crisp).} I do not have an unconditional argument proving any superlinear lower bound $a_j\gg j^{1+\eta}$ for some fixed $\eta>0$ (or ruling out polynomial growth) from only the squarefree-sum condition.

(iii) \emph{Top 3 next moves.}
1. Combine the ``no opposite residues'' condition for many primes $p\le P$ simultaneously via CRT to obtain a strong restriction on the additive energy of $A\cap[1,X]$.
2. Attempt a sieve-type upper bound on $|A\cap[1,X]|$ by counting pairs $(a,b)$ and estimating how often $a+b$ must fall into the set of nonsquarefree integers.
3. Extend the finite search to larger $N$ with heuristic/greedy constructions to guess the correct exponent from data (e.g. log--log slope of $\max|A\cap[1,N]|$).

(iv) \emph{Minimal counterexample structure.} A hypothetically ``slow-growing'' infinite $A$ would need to place its elements into a carefully chosen sparse set of residue classes modulo many squares $p^2$ so that \emph{no} pair of residues ever sums to $0$ mod $p^2$ for any $p$, while still leaving enough freedom to keep adding new elements without violating earlier congruence constraints.


