\section*{Erd\H{o}s Problem \#208}
\addcontentsline{toc}{section}{Erd\H{o}s Problem \#208}

\subsection*{1. FORMAL RESTATEMENT}
Let $1=s_1<s_2<s_3<\cdots$ be the increasing sequence of \emph{squarefree} positive integers.
The question is whether
\[
 s_{n+1}-s_n = o\bigl(s_n^{1/5}\bigr)\qquad (n\to\infty).
\]
Equivalently: for every $\varepsilon>0$ does there exist $N(\varepsilon)$ such that
$s_{n+1}-s_n \le \varepsilon\, s_n^{1/5}$ for all $n\ge N(\varepsilon)$?

\subsection*{2. PHASE 1 --- FAST REALITY CHECK}
\begin{itemize}
\item \textbf{Computed small-scale gap data (squarefree sieve).}
\begin{enumerate}
\item Up to $10^6$, the maximum observed gap between consecutive squarefree numbers is $8$ (e.g. from $217069$ to $217077$).
\item Up to $10^7$, the maximum observed gap is $10$ (e.g. from $8870023$ to $8870033$).
\end{enumerate}
These are far smaller than $x^{1/5}$ at these scales (for $x=10^7$, $x^{1/5}\approx 25.1$).
\end{itemize}

\subsection*{3. KEY DEFINITIONS / LEMMAS}
\begin{enumerate}
\item \textbf{Squarefree indicator.}
Let $\mu$ be the M\"obius function. Then
\[
\mathbf{1}_{\mathrm{sqfree}}(n)=\mu(n)^2.
\]
Also, the identity
\[
\mu(n)^2 = \sum_{d^2\mid n} \mu(d)
\]
holds for all $n\ge 1$.

\item \textbf{Counting squarefree numbers.}
Let
\[
Q(x):=\#\{n\le x: n\text{ squarefree}\}=\sum_{n\le x}\mu(n)^2.
\]
Then
\[
Q(x)=\frac{6}{\pi^2}x + O(\sqrt{x}).
\]

\item \textbf{Immediate (weak) consequence for gaps.}
There exists an absolute constant $C>0$ such that every interval $[x, x+C\sqrt{x}]$ with $x$ sufficiently large contains a squarefree integer.
Equivalently, $s_{n+1}-s_n \ll \sqrt{s_n}$.

\item \textbf{A simple infinite-family lower bound (gaps exist).}
There exist infinitely many gaps of length at least $c\,\dfrac{\log x}{\log\log x}$ between consecutive squarefree numbers, for some absolute $c>0$.
(A concrete proof is given below using CRT with prime squares.)
\end{enumerate}

\subsection*{4. WORK (Main proof or counterexample)}
\paragraph{Lemma 3.2 (squarefree counting with $O(\sqrt{x})$ error).}
\emph{Claim.} $Q(x)=\dfrac{6}{\pi^2}x + O(\sqrt{x})$.

\emph{Proof.}
Using $\mu(n)^2 = \sum_{d^2\mid n}\mu(d)$,
\[
Q(x)=\sum_{n\le x}\mu(n)^2
=\sum_{n\le x}\ \sum_{d^2\mid n}\mu(d)
=\sum_{d\le \sqrt{x}}\mu(d)\,\#\{n\le x: d^2\mid n\}
=\sum_{d\le \sqrt{x}}\mu(d)\,\Bigl\lfloor\frac{x}{d^2}\Bigr\rfloor.
\]
Write $\lfloor x/d^2\rfloor = x/d^2 + O(1)$, giving
\[
Q(x)= x\sum_{d\le \sqrt{x}}\frac{\mu(d)}{d^2} + O(\sqrt{x}).
\]
As $x\to\infty$, the partial sums converge to
\[
\sum_{d=1}^{\infty}\frac{\mu(d)}{d^2} = \frac{1}{\zeta(2)}=\frac{6}{\pi^2},
\]
and truncation at $\sqrt{x}$ introduces an error $\ll \sum_{d>\sqrt{x}}1/d^2\ll 1/\sqrt{x}$, which after multiplying by $x$ is $O(\sqrt{x})$.
Hence $Q(x)=\frac{6}{\pi^2}x+O(\sqrt{x})$. \qed

\paragraph{Corollary 3.3 (gap bound $\ll\sqrt{x}$).}
\emph{Claim.} There is a constant $C>0$ such that for all sufficiently large $x$,
\[
Q(x+C\sqrt{x})-Q(x) \ge 1.
\]
Consequently, $s_{n+1}-s_n \ll \sqrt{s_n}$.

\emph{Proof.}
By Lemma 3.2,
\[
Q(x+C\sqrt{x})-Q(x)
=\frac{6}{\pi^2}C\sqrt{x} + O(\sqrt{x}).
\]
Choosing $C$ larger than the implied constant in the $O(\sqrt{x})$ term makes the right-hand side positive for all large $x$, hence at least $1$ for all large $x$.
Therefore every such interval contains a squarefree integer.
Taking $x=s_n$ gives $s_{n+1}-s_n\le C\sqrt{s_n}$ for all large $n$. \qed

\paragraph{Lemma 3.4 (explicit construction of infinitely many large squarefree gaps).}
\emph{Claim.} There exists $c>0$ such that for infinitely many $X$ there is an interval of length at least
$c\,\dfrac{\log X}{\log\log X}$ containing no squarefree numbers.

\emph{Proof.}
Let $y$ be large and let $\mathcal{P}(y)$ denote the primes $p\le y$.
For each prime $p\le y$ choose a distinct integer $i(p)$ in $\{1,2,\dots,\pi(y)\}$ (for example enumerate the primes in increasing order and set $i(p)$ to be its index).
Impose the congruences
\[
 n\equiv -i(p) \pmod{p^2}\qquad (p\le y).
\]
The moduli $p^2$ are pairwise coprime, so by CRT there exists $n$ satisfying all of them.
Then for each $p\le y$ we have $p^2\mid (n+i(p))$, hence each of the integers
\[
 n+1,\ n+2,\ \dots,\ n+\pi(y)
\]
contains at least one square prime factor and therefore is \emph{not} squarefree.
Thus there is a squarefree gap of length at least $\pi(y)$.
Now set $X:=n$.
The CRT solution satisfies $n\equiv -i(p)\pmod{p^2}$ for all $p\le y$, hence $n$ is bounded by the product
\[
 n \le \prod_{p\le y} p^2 = \exp\Bigl(2\sum_{p\le y}\log p\Bigr).
\]
A standard Chebyshev bound gives $\sum_{p\le y}\log p \asymp y$ and also $\pi(y)\asymp \dfrac{y}{\log y}$.
Therefore $\log X\asymp y$ and
\[
\pi(y) \asymp \frac{y}{\log y} \asymp \frac{\log X}{\log\log X}.
\]
This provides infinitely many $X$ with a squarefree gap of size $\gg \log X/\log\log X$.
Taking $c>0$ to be a suitable absolute constant completes the proof. \qed

\subsection*{5. SANITY CHECK}
\begin{itemize}
\item Lemma 3.2 is consistent with the known density of squarefree numbers: $6/\pi^2\approx 0.6079$.
\item The computed maximal gaps up to $10^7$ (at most $10$) are consistent with the easy upper bound $\ll\sqrt{x}$.
\item The CRT construction in Lemma 3.4 is structurally analogous to the classical constructions of long prime gaps (Erd\H{o}s--Rankin type), but with prime squares.
\end{itemize}

\subsection*{6. FINAL}
\textbf{UNRESOLVED.}

\begin{itemize}
\item[(i)] \textbf{Best partial results proved here.}
We proved:
\begin{enumerate}
\item The classical counting asymptotic $Q(x)=\frac{6}{\pi^2}x+O(\sqrt{x})$.
\item A consequent unconditional gap upper bound $s_{n+1}-s_n\ll\sqrt{s_n}$.
\item A constructive lower bound showing gaps of size $\gg \log x/\log\log x$ occur infinitely often.
\item Small-scale computation: max gap $\le 10$ for squarefrees up to $10^7$.
\end{enumerate}

\item[(ii)] \textbf{First gap/obstruction to a full solution.}
The question asks for a much stronger upper bound: $s_{n+1}-s_n=o(s_n^{1/5})$.
The elementary $O(\sqrt{x})$-type methods above are far from the $x^{1/5}$ scale.
Reaching exponents near $1/5$ requires deep estimates for squarefree numbers in very short intervals (lattice-point counts near curves / exponential sum technology).

\item[(iii)] \textbf{What would be needed next.}
To resolve the problem one would need either:
\begin{enumerate}
\item a proof that every interval $[X,X+X^{1/5}/g(X)]$ (with $g(X)\to\infty$) contains a squarefree number, or
\item equivalently, a global upper bound on the maximal squarefree gap up to $X$ of order $o(X^{1/5})$.
\end{enumerate}
This lies beyond classical inclusion--exclusion and requires the strongest known short-interval methods.
(As noted in the problem statement, much sharper results are known in the literature, including improvements beyond the $1/5$ exponent.)
\end{itemize}

\subsection*{7. OPTIONAL: SHARPENING / DISCUSSION}
\begin{itemize}
\item The lower bound construction in Lemma 3.4 can be optimized (with more sophisticated residue-class choices) to increase the constant in front of $\log x/\log\log x$; the conjectured/known-best constants in the literature are substantially larger.
\item Conditional results (e.g. from the ABC conjecture) can force gaps to be extremely small (polylogarithmic), which would imply $o(x^{1/5})$ immediately.
\end{itemize}

