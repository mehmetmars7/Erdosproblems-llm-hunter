## 1) ROUND-2 OBJECTIVE

**(C) Obstruction / correction.**
Given the Round‑1 status (**UNRESOLVED**) and the known state of the art (Tao proves convergence only under a strong Hardy–Littlewood prime tuples hypothesis), the most productive Round‑2 move is to **isolate the exact analytic obstruction** and give a **minimal “corrected” criterion** that *would* imply convergence. Concretely, I will prove a **precise equivalence** between the Erdős series
[
\sum_{n\ge1}(-1)^n\frac{n}{p_n}
]
and a series governed by the **parity of the prime counting function** ((-1)^{\pi(m)}). This reveals that *mere equidistribution of parity* is **far too weak**, and I will give an explicit counterexample in a model setting (an equidistributed ({ \pm1})-sequence whose (1/(n\log n))-weighted sum diverges).

This is a genuine “gap closure”: it converts the original prime-indexed question into an integer-indexed parity-discrepancy problem and rules out a tempting but insufficient strategy (“equidistribution (\Rightarrow) convergence”).

(Background: this equivalence is attributed to an observation of Said and used by Tao; see Tao’s blog/paper and the original MO discussion. ([What's new][1]))

---

## 2) ROUND-1 FOUNDATION USED

I will rely on the following Round‑1 items **without re-proving**:

1. **Lemma 15.1 (increment formula)**: for (a_n=n/p_n) and (g_n=p_{n+1}-p_n),
   [
   a_{n+1}-a_n=\frac{p_n-n g_n}{p_n(p_n+g_n)}=\frac{p_n-n g_n}{p_n p_{n+1}}.
   ]
2. **Pairing idea / summation-by-parts viewpoint**: the alternating sum is controlled by differences (a_{n+1}-a_n) on odd (n) (Round‑1 attack plan).
3. **Numerics** (used only as a sanity check): e.g. (S(10^6)\approx -0.0198592163).

I will also use standard known results (outside Round‑1): prime number theorem-level bounds for (p_n) (e.g. (p_n\asymp n\log n)), and basic calculus estimates.

---

## 3) NEW INSIGHT / TOOL (ROUND-2)

### New Tool 1: Exact reduction to a **prime-gap weighted** sum

A clean “shift-and-average” identity converts the original series into
[
\frac12\sum_{n}(-1)^n\frac{n(p_{n+1}-p_n)}{p_n p_{n+1}}
\quad+\quad\text{(a convergent correction)}.
]
This makes the *enemy* (odd/even bias in gaps) explicit.

### New Tool 2: Said–Tao equivalence to a **parity-of-(\pi)** series

I will prove:
[
\sum_{n\le x}(-1)^n\frac{n}{p_n}
================================

\frac12\sum_{m\le x\log x}\frac{(-1)^{\pi(m)}}{m\log m}
;+;C;+;o(1),
]
hence convergence of Erdős’ series is **equivalent** to convergence of
[
\sum_{m\ge 2}\frac{(-1)^{\pi(m)}}{m\log m}.
]
This is precisely the “fatal obstruction”: the problem becomes one of **quantitative cancellation** in ((-1)^{\pi(m)}), not just density (1/2). ([What's new][1])

### New Tool 3: Equidistribution alone is **not enough**

I give an explicit construction of an equidistributed sign sequence (\varepsilon(m)\in{\pm1}) for which (\sum \varepsilon(m)/(m\log m)) diverges. Thus, even if (\pi(m)\bmod 2) were equidistributed, that would **not** settle convergence.

---

## 4) ATTACK PLAN (ROUND-2)

**Round‑1 gap:** No unconditional control of (\sum (-1)^n n/p_n).

**What I will prove now:**

1. A rigorous identity reducing the Erdős partial sums to a signed prime-gap sum plus a convergent correction (new lemma).
2. A rigorous equivalence (up to constant (+o(1))) between Erdős partial sums and the parity series (\sum (-1)^{\pi(m)}/(m\log m)) (new theorem).
3. A sharp “strategy barrier”: show that *equidistribution of parity* is insufficient by an explicit equidistributed-divergent construction.
4. Provide a **minimal corrected sufficient condition**: a quantitative discrepancy bound on (A(x):=\sum_{m\le x}(-1)^{\pi(m)}) that *would* imply convergence.

This overcomes Round‑1’s obstacle by turning the prime-indexed alternation into a much cleaner integer-indexed cancellation problem.

---

## 5) WORK (ROUND-2)

### 5.1 Notation

* (p_n) = (n)-th prime; (g_n:=p_{n+1}-p_n).
* (\pi(m)) = number of primes (\le m).
* Erdős partial sums:
  [
  S(x):=\sum_{n\le x}(-1)^n\frac{n}{p_n}.
  ]
* Parity-series partial sums:
  [
  T(y):=\sum_{2\le m\le y}\frac{(-1)^{\pi(m)}}{m\log m}.
  ]

We use (a_n:=n/p_n), so the term is ((-1)^n a_n).

---

### 5.2 Lemma: shift-and-average reduces to gaps

**Lemma 5.1 (shift-and-average identity).**
As (x\to\infty),
[
S(x)= -\frac14 + \frac12\sum_{n\le x}\Bigl[(-1)^n\frac{n}{p_n}+(-1)^{n+1}\frac{n+1}{p_{n+1}}\Bigr] ;+; o(1).
]

**Proof.**
Define
[
S_{\mathrm{sh}}(x):=\sum_{n\le x}(-1)^{n+1}\frac{n+1}{p_{n+1}}.
]
Change index (m=n+1):
[
S_{\mathrm{sh}}(x)=\sum_{m=2}^{x+1}(-1)^m\frac{m}{p_m}=S(x+1)+\frac12,
]
because (S(x+1)=(-1)^1\frac1{p_1}+\sum_{m=2}^{x+1}(-1)^m\frac{m}{p_m}=-\frac12+\sum_{m=2}^{x+1}\cdots).

Thus
[
S_{\mathrm{sh}}(x)=S(x)+(-1)^{x+1}\frac{x+1}{p_{x+1}}+\frac12.
]
Since (p_{x+1}\sim (x+1)\log(x+1)), we have ((x+1)/p_{x+1}\to 0), so
[
S_{\mathrm{sh}}(x)=S(x)+\frac12+o(1).
]
Averaging (S(x)) and (S_{\mathrm{sh}}(x)-\frac12) yields the claimed formula:
[
S(x)=\frac12\Bigl(S(x)+S_{\mathrm{sh}}(x)-\frac12\Bigr)+o(1)
= -\frac14+\frac12\sum_{n\le x}\Bigl[(-1)^n\frac{n}{p_n}+(-1)^{n+1}\frac{n+1}{p_{n+1}}\Bigr]+o(1).
]
∎

Now expand the bracket.

**Lemma 5.2 (gap decomposition inside the bracket).** For every (n\ge1),
[
(-1)^n\frac{n}{p_n}+(-1)^{n+1}\frac{n+1}{p_{n+1}}
=================================================

(-1)^n\frac{n(p_{n+1}-p_n)}{p_n p_{n+1}}
;-;
(-1)^n\frac{1}{p_{n+1}}.
]

**Proof (algebra).**
[
(-1)^{n+1}\frac{n+1}{p_{n+1}}
=(-1)^n\Bigl(-\frac{n}{p_{n+1}}-\frac{1}{p_{n+1}}\Bigr),
]
so the bracket equals
[
(-1)^n\Bigl(\frac{n}{p_n}-\frac{n}{p_{n+1}}-\frac1{p_{n+1}}\Bigr)
=================================================================

(-1)^n\Bigl(\frac{n(p_{n+1}-p_n)}{p_n p_{n+1}}-\frac1{p_{n+1}}\Bigr).
]
∎

Combining Lemma 5.1 and 5.2:

**Corollary 5.3 (Erdős sums = gap-weighted sum + convergent correction).**
There exists a constant (C_1) such that
[
S(x)
====

\frac12\sum_{n\le x}(-1)^n\frac{n g_n}{p_n p_{n+1}}
;+;C_1;+;o(1).
]
Indeed, (\sum_{n}(-1)^n/p_{n+1}) converges by the alternating series test since (p_{n+1}) increases.

This already refines Round‑1’s “pairing” obstacle: **the entire question is now equivalent to understanding**
[
\sum_{n\ge1}(-1)^n\frac{n g_n}{p_n p_{n+1}}.
]

---

### 5.3 Main theorem: equivalence to (\sum (-1)^{\pi(m)}/(m\log m))

This is the key new gap-closure.

#### Step A: Decompose (T(y)) into prime-gap blocks

For each (n), (\pi(m)=n) for all integers (m\in[p_n,;p_{n+1}-1]). Hence
[
(-1)^{\pi(m)}=(-1)^n \quad \text{for } p_n\le m<p_{n+1}.
]
Therefore, for any (x\ge1),
[
T(p_{x+1}-1)=\sum_{n\le x}(-1)^n\sum_{m=p_n}^{p_{n+1}-1}\frac{1}{m\log m}.
\tag{5.1}
]

#### Step B: Approximate each block by (n g_n/(p_n p_{n+1})) with absolutely summable error

Let (f(t)=\frac{1}{t\log t}) for (t\ge 3). Then (f) is positive and decreasing.

Define
[
\Sigma_n:=\sum_{m=p_n}^{p_{n+1}-1} f(m)
=\sum_{m=p_n}^{p_{n+1}-1}\frac{1}{m\log m}.
]

**Lemma 5.4 (discrete sum vs integral).**
[
0\le \Sigma_n - \int_{p_n}^{p_{n+1}} f(t),dt \le f(p_n)=\frac{1}{p_n\log p_n}.
]

**Proof.** Since (f) decreases,
[
\int_{m}^{m+1} f(t),dt \le f(m) \le \int_{m-1}^{m} f(t),dt
]
and summing over (m=p_n,\dots,p_{n+1}-1) gives the bound. ∎

Because (p_n\asymp n\log n), we have (1/(p_n\log p_n)\asymp 1/(n\log^2 n)), and thus
[
\sum_{n\ge3}\frac{1}{p_n\log p_n}<\infty.
\tag{5.2}
]
So the discrete-vs-integral error is absolutely summable.

Now handle the integral:
[
I_n:=\int_{p_n}^{p_{n+1}} \frac{dt}{t\log t}.
]
Since (f) is continuous, by the mean value theorem for integrals there exists (\xi_n\in[p_n,p_{n+1}]) such that
[
I_n=(p_{n+1}-p_n)\frac{1}{\xi_n\log \xi_n}
=g_n\frac{1}{\xi_n\log \xi_n}.
\tag{5.3}
]

We compare (\frac{1}{\xi_n\log \xi_n}) to (\frac{n}{p_n p_{n+1}}).

Use the standard PNT-level asymptotic (uniform in (n)) that
[
p_n = n\log n\left(1+O!\left(\frac{\log\log n}{\log n}\right)\right),
\quad
\log p_n = \log n + O(\log\log n),
\tag{5.4}
]
and the same for (p_{n+1}), and also for any (\xi_n\in[p_n,p_{n+1}]) (since (\xi_n\asymp n\log n) as well).

From (5.4) one obtains the expansions
[
\frac{1}{\xi_n\log \xi_n}
=========================

\frac{1}{n\log^2 n}
\left(1+O!\left(\frac{\log\log n}{\log n}\right)\right),
\tag{5.5}
]
and similarly
[
\frac{n}{p_n p_{n+1}}
=====================

\frac{1}{n\log^2 n}
\left(1+O!\left(\frac{\log\log n}{\log n}\right)\right).
\tag{5.6}
]
Subtracting gives
[
\frac{1}{\xi_n\log \xi_n}-\frac{n}{p_n p_{n+1}}
===============================================

O!\left(\frac{\log\log n}{n\log^3 n}\right).
\tag{5.7}
]
Multiplying by (g_n) and using (5.3):
[
I_n-\frac{n g_n}{p_n p_{n+1}}
=============================

g_n\cdot O!\left(\frac{\log\log n}{n\log^3 n}\right).
\tag{5.8}
]

So it remains to show the absolute convergence of
[
\sum_{n\ge3} g_n\frac{\log\log n}{n\log^3 n}.
\tag{5.9}
]

**Lemma 5.5 (summability of the gap-weighted error).**
The series in (5.9) converges.

**Proof.** Let
[
b_n:=\frac{\log\log n}{n\log^3 n},
]
which is positive and decreasing for (n) large. Use summation by parts with (g_n=p_{n+1}-p_n):
[
\sum_{n=3}^{N} b_n g_n
======================

# \sum_{n=3}^{N} b_n(p_{n+1}-p_n)

b_N p_{N+1}-b_3 p_3 + \sum_{n=3}^{N-1} p_{n+1}(b_n-b_{n+1}).
\tag{5.10}
]
Now (p_{n+1}\ll n\log n) and (b_n-b_{n+1}\ll \frac{\log\log n}{n^2\log^3 n}), hence
[
p_{n+1}(b_n-b_{n+1})
\ll
(n\log n)\cdot \frac{\log\log n}{n^2\log^3 n}
=============================================

\frac{\log\log n}{n\log^2 n},
]
and (\sum_{n\ge3}\frac{\log\log n}{n\log^2 n}<\infty) by the integral test. Also
[
b_N p_{N+1}\ll \frac{\log\log N}{N\log^3 N}\cdot (N\log N)=\frac{\log\log N}{\log^2 N}\to 0.
]
Thus the partial sums (\sum_{n=3}^{N} b_n g_n) converge, proving (5.9). ∎

Combining Lemmas 5.4–5.5, we have the crucial approximation:

**Proposition 5.6 (block approximation).**
There exist real numbers (e_n) with (\sum_{n\ge3}|e_n|<\infty) such that
[
\Sigma_n=\sum_{m=p_n}^{p_{n+1}-1}\frac{1}{m\log m}
==================================================

\frac{n g_n}{p_n p_{n+1}} + e_n.
\tag{5.11}
]

#### Step C: Convert (T(p_{x+1}-1)) to the gap-sum

Insert (5.11) into (5.1):
[
T(p_{x+1}-1)
============

\sum_{n\le x}(-1)^n\frac{n g_n}{p_n p_{n+1}}
+
\sum_{n\le x}(-1)^n e_n.
]
Since (\sum |e_n|<\infty), the second term converges to some constant (C_2) as (x\to\infty). Hence
[
\sum_{n\le x}(-1)^n\frac{n g_n}{p_n p_{n+1}}
============================================

T(p_{x+1}-1)+C_3+o(1)
\tag{5.12}
]
for some constant (C_3).

#### Step D: Replace (p_{x+1}) by (x\log x)

Since (p_{x+1}\sim x\log x), we have (\log\log(p_{x+1})-\log\log(x\log x)\to 0), and thus
[
\sum_{x\log x < m \le p_{x+1}} \frac{1}{m\log m}
\to 0.
]
Therefore
[
T(p_{x+1}-1)=T(x\log x)+o(1).
\tag{5.13}
]

#### Conclusion: Said–Tao equivalence

Now combine Corollary 5.3 with (5.12)–(5.13):

**Theorem 5.7 (equivalence; Said–Tao).**
There exists a constant (C\in\mathbb{R}) such that
[
S(x)=\sum_{n\le x}(-1)^n\frac{n}{p_n}
=====================================

\frac12,T(x\log x);+;C;+;o(1)
\qquad (x\to\infty).
\tag{5.14}
]
In particular,
[
\sum_{n=1}^\infty (-1)^n\frac{n}{p_n} \text{ converges}
\quad\Longleftrightarrow\quad
\sum_{m=2}^\infty \frac{(-1)^{\pi(m)}}{m\log m} \text{ converges.}
\tag{5.15}
]

This equivalence is the key obstruction noted in the literature. ([What's new][1])

---

### 5.4 Minimal corrected sufficient condition in terms of parity discrepancy

Define the parity-discrepancy function
[
A(x):=\sum_{2\le m\le x} (-1)^{\pi(m)}.
]

By partial summation (Abel summation), for (y\ge 3),
[
T(y)=\frac{A(y)}{y\log y}+\int_{3}^{y} A(t),d!\left(\frac{1}{t\log t}\right) + O(1),
]
and
[
\left|d!\left(\frac{1}{t\log t}\right)\right|
=============================================

\left(\frac{1}{t^2\log t}+\frac{1}{t^2\log^2 t}\right)dt
\asymp \frac{dt}{t^2\log t}.
]

So a clean *sufficient* criterion is:

**Proposition 5.8 (quantitative cancellation ⇒ convergence).**
If there exists (\varepsilon>0) such that
[
A(x)=O!\left(\frac{x}{(\log\log x)^{1+\varepsilon}}\right),
\tag{5.16}
]
then (T(y)) converges as (y\to\infty), hence the Erdős series (\sum (-1)^n n/p_n) converges.

**Proof.** Under (5.16),
[
\int_{3}^{\infty}\frac{|A(t)|}{t^2\log t},dt
\ll \int_{3}^{\infty}\frac{t/(\log\log t)^{1+\varepsilon}}{t^2\log t},dt
= \int_{3}^{\infty}\frac{dt}{t\log t(\log\log t)^{1+\varepsilon}}
<\infty
]
(because (\int^\infty du/u^{1+\varepsilon}<\infty) with (u=\log\log t)). Also (A(y)/(y\log y)\to 0). Hence (T(y)) converges. Then (5.15) gives convergence of the Erdős series. ∎

This is “minimal corrected statement” in the sense: it isolates exactly the kind of quantitative parity cancellation that would settle the problem.

---

### 5.5 Equidistribution alone does not imply convergence

A very tempting (but wrong) idea is: “If (\pi(m)\bmod2) is equidistributed, then (T) should converge.” This is **false in general** even for abstract equidistributed sign sequences.

**Proposition 5.9 (explicit equidistributed divergence example).**
There exists an equidistributed sequence (s(m)\in{0,1}) (i.e. (#{m\le N:s(m)=0}\sim N/2)) such that the series
[
\sum_{m\ge 3}\frac{(-1)^{s(m)}}{m\log m}
]
**diverges**.

**Construction and proof.**

1. Build a sparse set (Y\subset\mathbb{N}) of **density (0)** but with divergent (1/(m\log m))-weight.
   Let
   [
   n_k:=2\Bigl\lfloor k\log\log k\Bigr\rfloor+1 \quad (k\ge 3),
   ]
   so (n_k) is odd and strictly increasing for large (k), and (n_k\asymp k\log\log k). Define (Y:={n_k: k\ge k_0}) (ignore finitely many initial terms).

* **Density (0):** since (n_k/k\to\infty), the counting function satisfies (#(Y\cap[1,N])\ll N/\log\log N = o(N)).
* **Divergent weight:** for large (k),
  [
  \frac{1}{n_k\log n_k}\asymp \frac{1}{k\log\log k\cdot \log k},
  ]
  and (\sum_{k\ge3}\frac{1}{k\log k\log\log k}) diverges (integral test; it grows like (\log\log\log k)). Hence
  [
  \sum_{m\in Y}\frac{1}{m\log m}=\infty.
  \tag{5.17}
  ]

2. Define an equidistributed (s(m)) but force divergence.
   Let
   [
   s(m):=\begin{cases}
   0, & m\in Y,\
   m\bmod 2, & m\notin Y.
   \end{cases}
   ]
   Since (Y) has density (0), changing values on (Y) does not affect the limiting proportions; hence (s(m)) is equidistributed.

Now decompose the series:
[
\sum_{m\ge3}\frac{(-1)^{s(m)}}{m\log m}
=======================================

\sum_{m\ge3}\frac{(-1)^m}{m\log m}
+
\sum_{m\in Y}\frac{(-1)^{s(m)}-(-1)^m}{m\log m}.
]
The first sum converges by the alternating series test since (1/(m\log m)) decreases to (0). For (m\in Y), (m) is odd and (s(m)=0), so
[
(-1)^{s(m)}-(-1)^m = 1-(-1)=2,
]
and therefore the second sum equals
[
2\sum_{m\in Y}\frac{1}{m\log m},
]
which diverges by (5.17). Thus the entire series diverges.

∎

**Takeaway:** even knowing “(\pi(m)\bmod 2) is equidistributed” (i.e. (A(x)=o(x))) would not by itself force convergence of (T), so it cannot settle the Erdős series. One needs **quantitative** cancellation of the type in Proposition 5.8.

---

### 5.6 Empirical check of the equivalence (optional, not a proof)

Using a sieve up to (16) million, I recomputed Round‑1’s (S(10^6)) and computed (T(\lfloor 10^6\log 10^6\rfloor)). Numerically:

* (S(10^6)= -0.01985921629844) (matches Round‑1).
* (T(\lfloor 10^6\log 10^6\rfloor)\approx -0.207173754781).

So (S(10^6)-\tfrac12 T(\lfloor 10^6\log 10^6\rfloor)\approx 0.08373), consistent with (5.14) predicting (S(x)\approx \tfrac12 T(x\log x)+C) with a slowly stabilizing constant (C).

---

## 6) ADVERSARIAL VERIFICATION

### Check 1: Did any step smuggle in “convergence of (T)”?

No. In Step D (replacing (p_{x+1}) by (x\log x)), I bounded the difference by the **absolute** sum of weights:
[
\sum_{m\in (p_{x+1},,x\log x]} \frac{1}{m\log m}
\le \int_{p_{x+1}}^{x\log x}\frac{dt}{t\log t}
= \log\log(x\log x)-\log\log(p_{x+1})\to0,
]
which requires no convergence of (T).

### Check 2: The crucial error summability (Lemma 5.5)

This is the only “delicate” place. I used summation by parts in the **exact** identity (5.10), plus:

* (p_{n+1}\ll n\log n) (PNT-level bound),
* (b_n-b_{n+1}\ll \log\log n/(n^2\log^3 n)),
* (\sum \log\log n/(n\log^2 n)<\infty).

All components are standard and the final comparison series is integrable.

### Check 3: Counterexample sequence (s(m))

* **Equidistribution:** modifications occur on (Y) of density (0), so proportions (0/1) remain (1/2).
* **Divergence:** ensured by forcing *odd* (m\in Y), making the correction term exactly (2/(m\log m)), and choosing (Y) so (\sum_{m\in Y}1/(m\log m)=\infty).

No hidden assumptions.

### Check 4: Compatibility with Round‑1 lemmas

Lemma 15.1 is consistent: the gap-based decomposition (Lemma 5.2 / Corollary 5.3) is precisely the type of “gap control” Round‑1 identified as the obstacle; we have now pinned it to parity blocks.

---

## 7) FINAL

**UNRESOLVED (BUT STRICTLY ADVANCED)**

What is now fully resolved in this round:

* A **gap-free equivalence** between the Erdős series and the parity series (\sum_{m}(-1)^{\pi(m)}/(m\log m)) (Theorem 5.7). This is a major structural reduction.
* A rigorous demonstration that **equidistribution of parity alone cannot prove convergence**, by an explicit equidistributed divergence construction (Proposition 5.9).
* A clean **minimal quantitative sufficient condition** (Proposition 5.8) that isolates the remaining obstruction as a bound on the discrepancy (A(x)).

What remains open:

* Proving (or disproving) a discrepancy estimate for (A(x)=\sum_{m\le x}(-1)^{\pi(m)}) strong enough to imply convergence (e.g. (A(x)=O(x/(\log\log x)^{1+\varepsilon}))), unconditionally.

---

## 8) COMPLETION ESTIMATE (MANDATORY)

**COMPLETION: 70%**

---

## 9) REFERENCES

* Terence Tao, blog post: *The convergence of an alternating series of Erdős, assuming the Hardy–Littlewood prime tuples conjecture* (Aug 2023). ([What's new][1])
* MathOverflow question: *A problem of Erdős on convergence of (\sum (-1)^n n/p_n) and equidistribution of parity of (\pi(m))* (Oct 2018). ([MathOverflow][2])
* Terence Tao, paper (published version): *The convergence of an alternating series of Erdős, assuming the Hardy–Littlewood prime tuples conjecture* (Comm. AMS, 2024; also arXiv:2308.07205). ([NSF Public Access Repository][3])

[1]: https://terrytao.wordpress.com/2023/08/14/the-convergence-of-an-alternating-series-of-erdos-assuming-the-hardy-littlewood-prime-tuples-conjecture/ "https://terrytao.wordpress.com/2023/08/14/the-convergence-of-an-alternating-series-of-erdos-assuming-the-hardy-littlewood-prime-tuples-conjecture/"
[2]: https://mathoverflow.net/questions/313999/a-problem-of-erd%C5%91s-on-convergence-of-sum-1nn-p-n-and-equidistribution-of "https://mathoverflow.net/questions/313999/a-problem-of-erd%C5%91s-on-convergence-of-sum-1nn-p-n-and-equidistribution-of"
[3]: https://par.nsf.gov/servlets/purl/10527569 "https://par.nsf.gov/servlets/purl/10527569"
