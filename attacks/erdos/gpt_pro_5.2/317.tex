\section*{Erd\H{o}s Problem \#317}

\subsection*{1) RESTATE}
For each $n\ge 1$, consider sums
\[
S_n(\delta):=\sum_{k=1}^n\frac{\delta_k}{k},\qquad \delta_k\in\{-1,0,1\}.
\]
\begin{enumerate}[label=(\alph*)]
\item \textbf{(Exponential smallness question)} Does there exist an absolute constant $c>0$ such that for every $n\ge 1$ there is a choice of $\delta_1,\dots,\delta_n\in\{-1,0,1\}$ with
\[
0<|S_n(\delta)|<\frac{c}{2^n}\ ?
\]
Equivalently, is
\[
M_n:=\min_{\delta\in\{-1,0,1\}^n\,\colon\,S_n(\delta)\neq 0}|S_n(\delta)|
\]
bounded by $O(2^{-n})$?

\item \textbf{(Strict lcm lower bound question)} Let $L_n:=\operatorname{lcm}(1,2,\dots,n)$. It is always true that either $S_n(\delta)=0$ or $|S_n(\delta)|\ge 1/L_n$.
Is it true that for all sufficiently large $n$,
\[
S_n(\delta)\neq 0\ \Longrightarrow\ |S_n(\delta)|>\frac{1}{L_n}\ ?
\]
\end{enumerate}

\subsection*{2) KNOWN FACTS}
\begin{enumerate}[label=(\alph*)]
\item \textbf{(Obvious non-strict lcm bound)} Multiply by $L_n$:
\[
L_n S_n(\delta)=\sum_{k=1}^n \delta_k\frac{L_n}{k}\in\mathbb{Z}.
\]
Hence $S_n(\delta)=0$ or $|S_n(\delta)|\ge 1/L_n$.

\item \textbf{(Strictness can fail for small $n$)} For $n=4$, $\frac12-\frac13-\frac14=-\frac{1}{12}=-\frac{1}{L_4}$.

\item \textbf{(Reformulation via subset sums)} Let
\[
Q_n:=\left\{\sum_{k\in A}\frac{1}{k}: A\subseteq\{1,\dots,n\}\right\}.
\]
Every $S_n(\delta)$ can be written as $q-q'$ for some $q,q'\in Q_n$ (take $\delta_k=1$ for $k\in A\setminus B$, $\delta_k=-1$ for $k\in B\setminus A$, and $0$ otherwise).
Consequently
\[
M_n=\min_{\substack{q,q'\in Q_n\\ q\neq q'}}|q-q'|.
\]
So the first question asks whether the minimal gap in $Q_n$ is $O(2^{-n})$.

\item \textbf{(Best general upper bounds are much weaker than $2^{-n}$)} Using information on $|Q_n|$ together with a pigeonhole principle argument yields upper bounds of the shape
\[
M_n\le 2^{-\frac{n(\log\log\log n)^{1+o(1)}}{\log n}},
\]
which is far larger than $2^{-n}$.
\end{enumerate}

\subsection*{3) PROOF STRATEGY}
Two complementary routes suggest themselves.
\begin{enumerate}[label=(\alph*)]
\item \textbf{(Constructive)} Explicitly build, for each $n$, a signed harmonic sum with value about $2^{-n}$ or smaller.
\item \textbf{(Entropy/packing)} Prove that the set $Q_n$ is so large and so well-distributed in an interval of length $\asymp \log n$ that two elements must be within $O(2^{-n})$.
\end{enumerate}
Current technology gives only subexponential smallness via (b).

\subsection*{4) ATTEMPTED PROOF}
\paragraph{Step 1: Record the exact arithmetic obstruction.}
Write $S_n(\delta)=A_n/L_n$ with $A_n\in\mathbb{Z}$.
Then
\[
M_n=\frac{m_n}{L_n},\qquad m_n:=\min\{|A|:A\neq 0,\ A\in\{L_n S_n(\delta)\}\}.
\]
The strict inequality in (1b) is equivalent to $m_n\ge 2$.

\paragraph{Step 2: Small-$n$ computation (meet-in-the-middle).}
For $n\le 26$, one can compute $m_n$ exactly by enumerating all $3^{\lfloor n/2\rfloor}$ partial sums on each half and searching for the closest pair.
The results are:
\begin{center}
\begin{tabular}{r|r|r|r}
$n$ & $L_n$ & $m_n$ & $M_n\cdot 2^n$\\\hline
5  & 60             & 2    & 1.067\\
10 & 2520           & 3    & 1.219\\
15 & 360360         & 9    & 0.818\\
20 & 232792560      & 874  & 3.937\\
21 & 232792560      & 874  & 7.874\\
22 & 232792560      & 299  & 5.387\\
23 & 5354228880     & 399  & 0.625\\
24 & 5354228880     & 399  & 1.250\\
25 & 26771144400    & 1995 & 2.500\\
26 & 26771144400    & 1995 & 5.001\\
\end{tabular}
\end{center}
In particular, for all $5\le n\le 26$ we have $m_n\ge 2$, so the strict inequality $|S_n(\delta)|>1/L_n$ holds in this range.
Also, in this range $M_n\cdot 2^n$ is at most $\approx 7.874$.
This is consistent with (but far from proving) the existence of a universal $c$ in (1a).

\subsection*{5) OBSTACLES}
\begin{enumerate}[label=(\alph*)]
\item The pigeonhole argument depends on $|Q_n|$, but $|Q_n|$ is only known to be $\exp\big(n\,f(n)/\log n\big)$ with slowly growing $f(n)$, yielding only subexponential gaps.
\item Constructive control of signed harmonic sums at scale $2^{-n}$ seems to require highly nontrivial cancellations that are not captured by standard Egyptian-fraction identities.
\item The strictness problem (1b) amounts to showing that a certain restricted signed subset-sum of divisors of $L_n$ cannot equal $\pm 1$ for all large $n$; no known invariant rules this out.
\end{enumerate}

\subsection*{6) FINAL}
\textbf{UNRESOLVED.}
\begin{enumerate}[label=(\roman*)]
\item \textbf{What I tried:} Rewrote the questions in terms of the integer numerator $m_n$ after scaling by $L_n$; computed $m_n$ for $n\le 26$; related $M_n$ to minimal gaps in the subset-sum set $Q_n$.
\item \textbf{Where it fails / stuck:} I do not see how to upgrade known size estimates for $Q_n$ (or distribution information) to force $M_n\ll 2^{-n}$, nor how to rule out/ensure the equality case $M_n=1/L_n$ for infinitely many $n$.
\item \textbf{What might work next:} (a) Prove stronger lower bounds on $|Q_n|$ together with equidistribution/concentration estimates; (b) develop a constructive ``binary-expansion''-type algorithm that keeps the partial gap shrinking by a factor $\approx 2$ each step; (c) attack the strictness question by modular obstructions showing $\sum \delta_k(L_n/k)\neq\pm 1$ for all large $n$.
\item \textbf{Confidence:} High confidence in the reformulations and computations shown; low confidence on the global asymptotic behavior (open problem).
\end{enumerate}

\subsection*{7) WRITE-UP (clean, complete)}
Define $S_n(\delta)=\sum_{k\le n}\delta_k/k$ with $\delta_k\in\{-1,0,1\}$.
Then $L_n S_n(\delta)\in\mathbb{Z}$, so any nonzero value has absolute value at least $1/L_n$.
The strict inequality question asks whether the integer $L_n S_n(\delta)$ can equal $\pm 1$ for infinitely many $n$.

For the exponential smallness question, note that $S_n(\delta)$ equals the difference of two subset sums of $\{1,1/2,\dots,1/n\}$, so the minimal nonzero absolute value $M_n$ equals the minimal gap between distinct elements of
\(Q_n=\{\sum_{k\in A}1/k: A\subseteq\{1,\dots,n\}\}\).
Known general bounds on $|Q_n|$ combined with pigeonhole yield only subexponential upper bounds for $M_n$, much weaker than $c/2^n$.
Direct computation up to $n=26$ shows $M_n\cdot 2^n$ stays below $\approx 7.874$ in that range, but this does not settle whether $\sup_n M_n 2^n<\infty$.

\hrule

