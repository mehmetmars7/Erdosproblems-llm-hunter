
1) FORMAL RESTATEMENT

Let $n\ge 2$ and let $\pi(n)$ be the number of primes $\le n$. Fix an integer $k$ with $\pi(n)<k\le n$.
For a subset $A\subseteq\{1,\dots,n\}$ of size $|A|=k$, consider the following property for a given integer $r\ge 1$:
there exist distinct primes $p_1,\dots,p_r$ such that at least $r$ elements $a\in A$ have all their prime divisors contained in $\{p_1,\dots,p_r\}$.
Define $f(k,n)$ to be the smallest integer $r$ such that this property holds for every $A\subseteq\{1,\dots,n\}$ with $|A|=k$.

Ambiguity note (literal vs corrected):
- Literally, the condition ``$a$ is only divisible by primes from $\{p_1,\dots,p_r\}$'' is vacuously true for $a=1$ (since $1$ has no prime divisors). This makes very small-$n$ cases somewhat degenerate.
- A minimal corrected variant sometimes used in number theory is to require the counted elements $a\in A$ to satisfy $a>1$.
In the computations below, I report both conventions for small $n$.

The question asks whether
\[
2\pi(\sqrt n)-f(\pi(n)+1,n)\to\infty\quad\text{as }n\to\infty,
\]
and more generally to estimate $f(k,n)$ when $\pi(n)+1<k=o(n)$.

2) QUICK LITERATURE/CONTEXT CHECK

The provided statement includes asymptotics proved by Erd\H{o}s and Straus for $f(\pi(n)+1,n)$ and for $f(cn,n)$.
A quick web search (2026-01-16) did not surface an explicit resolution of the limit question beyond what is in the problem statement.

3) ATTACK PLAN

Proof track: reinterpret the problem in terms of prime-support sets of elements of $A$ and search for extremal set-system configurations.
Disproof track: attempt to construct sets $A$ of size $\pi(n)+1$ in which prime supports are spread out so that any set of $r$ primes can certify fewer than $r$ elements.

Here I provide two elementary equivalences and exact brute-force values for $f(\pi(n)+1,n)$ for $n\le 24$.

4) WORK

Definition.
For $a\in\{1,\dots,n\}$, let $\operatorname{supp}(a)$ be the set of prime divisors of $a$.

Lemma 4.1 (equivalent ``union of supports'' formulation).
Fix $A\subseteq\{1,\dots,n\}$.
For a given $r\ge 1$, there exist primes $p_1,\dots,p_r$ such that at least $r$ elements of $A$ have all prime divisors in $\{p_1,\dots,p_r\}$
if and only if there exists a subset $B\subseteq A$ with $|B|=r$ such that
\[
\bigl|\,\bigcup_{b\in B}\operatorname{supp}(b)\,\bigr|\le r.
\]

Proof.
($\Rightarrow$) If $p_1,\dots,p_r$ certify at least $r$ elements, choose any $r$ certified elements and call them $B$.
Every prime divisor of every $b\in B$ lies in $\{p_1,\dots,p_r\}$, hence
$\bigcup_{b\in B}\operatorname{supp}(b)\subseteq\{p_1,\dots,p_r\}$ and so its size is $\le r$.

($\Leftarrow$) Conversely, suppose $B\subseteq A$ has $|B|=r$ and the union of supports has size $s\le r$.
Let $Q$ be the set of primes in this union (so $|Q|=s$), and enlarge $Q$ arbitrarily by adding $r-s$ further distinct primes $\le n$ to obtain a set $P$ of exactly $r$ primes.
Then every $b\in B$ has all prime divisors in $P$, so $P$ certifies at least $r$ elements.
\qed

Lemma 4.2 (characterization of the case $r=1$).
For a set $A\subseteq\{1,\dots,n\}$, the property holds with $r=1$ if and only if $A$ contains a prime power (including $1$ under the literal convention).
If one insists on counting only $a>1$, then $r=1$ holds if and only if $A$ contains $p^t$ for some prime $p$ and integer $t\ge 1$.

Proof.
For $r=1$ we choose a single prime $p$ and ask for at least one $a\in A$ whose prime divisors all lie in $\{p\}$.
This means either $a=1$ (empty support) or $a=p^t$ for some $t\ge 1$.
Conversely, any such element is certified by choosing $p$ to be its prime base (or any $p$ if $a=1$). \qed

FAST REALITY CHECK (exact values for $f(\pi(n)+1,n)$ for $n\le 24$)

Let $k=\pi(n)+1$.
I computed $f(k,n)$ by brute force over all subsets $A\subseteq\{1,\dots,n\}$ of size $k$.
I report both the literal convention (allowing $a=1$ to count) and the corrected convention (counting only $a>1$ among the certified elements).

\[
\begin{array}{c|c|c|c|c}
 n & \pi(n) & k=\pi(n)+1 & f(k,n)\ \text{(literal)} & f(k,n)\ \text{(count only }a>1)\\\hline
 2&1&2&1&1\\
 3&2&3&1&1\\
 4&2&3&1&1\\
 5&3&4&1&1\\
 6&3&4&1&1\\
 7&4&5&1&1\\
 8&4&5&1&1\\
 9&4&5&1&1\\
 10&4&5&1&1\\
 11&5&6&1&1\\
 12&5&6&1&1\\
 13&6&7&1&1\\
 14&6&7&1&1\\
 15&6&7&1&1\\
 16&6&7&1&1\\
 17&7&8&1&1\\
 18&7&8&1&1\\
 19&8&9&1&1\\
 20&8&9&1&1\\
 21&8&9&1&2\\
 22&8&9&2&2\\
 23&9&10&1&2\\
 24&9&10&2&2
\end{array}
\]

For example at $n=22$ (so $\pi(n)=8$ and $k=9$), the worst-case set (for the literal convention) is
\[
A=\{6,10,12,14,15,18,20,21,22\},
\]
for which $r=1$ fails (no prime powers in $A$), while $r=2$ works by choosing primes $\{2,3\}$ which certify $\{6,12,18\}$.

5) VERIFICATION

- Lemma 4.1 is a direct translation between ``certifying primes'' and ``union of supports''.
- Lemma 4.2 follows by classifying the integers with prime support contained in a singleton.
- The brute-force table is exact for the stated range $n\le 24$.

6) FINAL

**UNRESOLVED**
(i) Strongest proved partial result: an equivalent formulation of the certification property in terms of subsets with small union of prime supports (Lemma 4.1) and a complete characterization of the case $r=1$ (Lemma 4.2). Exact values of $f(\pi(n)+1,n)$ for $n\le 24$ are given above under both conventions regarding whether $a=1$ counts.
(ii) First gap (crisp): determine the asymptotic behavior of $f(\pi(n)+1,n)$ well enough to decide whether $2\pi(\sqrt n)-f(\pi(n)+1,n)\to\infty$.
(iii) Top 3 next moves:
  1. Using Lemma 4.1, model the multihypergraph of prime supports of elements of $A$ and search for extremal configurations maximizing the minimum $r$ for which an $r$-set of vertices contains $r$ hyperedges.
  2. Construct explicit ``bad'' sets $A$ of size $\pi(n)+1$ that avoid prime powers and control overlaps of prime supports, to obtain lower bounds on $f(\pi(n)+1,n)$.
  3. Computation: for moderate $n$ (beyond $24$), use heuristic/optimization search (rather than exhaustive search) to find candidate worst-case sets $A$ and estimate $f(\pi(n)+1,n)$ numerically.
(iv) Minimal counterexample structure: for the limit question to be true, one would need that for large $n$ the worst-case $A$ forces $f(\pi(n)+1,n)$ to be noticeably smaller than $2\pi(\sqrt n)$; such an $A$ would have to concentrate many elements on a relatively small set of primes while simultaneously preventing the existence of an $r$-subset of $A$ whose prime supports collectively use only $r$ primes.
