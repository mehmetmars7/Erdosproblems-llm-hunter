% Erdos Problem #39

\subsection*{Erd\H{o}s Problem \#39}

\subsection*{FORMAL RESTATEMENT}
A set $A\subseteq\mathbb N$ is a \emph{Sidon set} (a $B_2$ set) if for all $a,b,c,d\in A$,
\[
 a+b=c+d\ \Longrightarrow\ \{a,b\}=\{c,d\}\ \text{(as multisets)}.
\]
Equivalently, all sums $a_i+a_j$ with $i\le j$ are distinct.

The question asks whether there exists an \emph{infinite} Sidon set $A\subseteq\mathbb N$ such that for every $\varepsilon>0$ there exists $c_{\varepsilon}>0$ with
\[
|A\cap[1,N]|\ \ge\ c_{\varepsilon}\,N^{1/2-\varepsilon}\qquad\text{for all sufficiently large }N.
\]

\subsection*{QUICK LITERATURE/CONTEXT CHECK}
The problem statement records several constructions and bounds (greedy $\gg N^{1/3}$; improvements by Ajtai--Koml\'{o}s--Szemer\'{e}di and by Ruzsa; and an Erd\H{o}s liminf statement).  I will not use any results beyond what is explicitly stated; the work below is self-contained basic combinatorial analysis and a small computation.

\subsection*{ATTACK PLAN}
\textbf{Proof track:} Build an explicit infinite Sidon set of near-square-root growth using a structured construction; verify Sidon property and the growth lower bound for all large $N$.

\textbf{Disproof track:} Prove that every infinite Sidon set must have growth $\ll N^{1/2-\delta}$ for some fixed $\delta>0$ along all $N$ (much stronger than the stated liminf$=0$ result), or exhibit an obstruction to maintaining $N^{1/2-\varepsilon}$ for \emph{every} $\varepsilon$.

Here I only obtain standard elementary constraints and re-derive the greedy $N^{1/3}$ lower bound.

\subsection*{WORK}
\paragraph{Fast reality check (local computation: greedy Sidon sequence).}
I generated the first 300 terms of the standard greedy Sidon sequence (start at 1 and repeatedly add the smallest integer preserving the Sidon property).  The first 25 terms are
\[
1,2,4,8,13,21,31,45,66,81,97,123,148,182,204,252,290,361,401,475,565,593,662,775,822.
\]
Counts $|A\cap[1,N]|$ for this greedy set were:
\[
\begin{array}{c|cccccc}
N & 10 & 100 & 10^3 & 10^4 & 10^5 & 5\cdot 10^5\\\hline
|A\cap[1,N]| & 4 & 11 & 27 & 66 & 161 & 296
\end{array}
\]
In particular $|A\cap[1,N]|/N^{1/3}$ is between $\approx 1.86$ and $\approx 3.73$ over these sample points.

\paragraph{Lemma 39.1 (Sidon implies distinct differences for ordered pairs).}
Let $A\subseteq\mathbb Z$ be Sidon.  If $a,b,c,d\in A$ with $a\neq b$ and $c\neq d$ satisfy
\[
 a-b=c-d,
\]
then $a=c$ and $b=d$.

\emph{Proof.}
From $a-b=c-d$ we get $a+d=c+b$.  Since $A$ is Sidon, the equality of sums implies the multisets satisfy $\{a,d\}=\{c,b\}$.  There are two possibilities:
\begin{itemize}
\item $a=c$ and $d=b$, in which case $b=d$ and we are done.
\item $a=b$ and $d=c$, which is impossible because $a\neq b$ by hypothesis.
\end{itemize}
Thus only the first possibility can occur, proving $a=c$ and $b=d$. \qed

\paragraph{Lemma 39.2 (elementary upper bound in an interval).}
If $A\subseteq[1,N]$ is Sidon and $|A|=m$, then
\[
 m(m-1)\le 2(N-1),\qquad\text{hence}\qquad m\le \frac{1+\sqrt{1+8(N-1)}}{2} < \sqrt{2N}+1.
\]

\emph{Proof.}
Consider the set of ordered differences
\[
\Delta:=\{a-b: a,b\in A,\ a\neq b\}.
\]
By Lemma 39.1, all these differences are distinct as $a,b$ range over ordered pairs with $a\neq b$. Therefore
\[
|\Delta|=m(m-1).
\]
But since $A\subseteq[1,N]$, every difference $a-b$ lies in $\{-(N-1),\dots,-1,1,\dots,N-1\}$, which has size $2(N-1)$. Hence $m(m-1)=|\Delta|\le 2(N-1)$.
Solving the quadratic inequality gives the stated bound. \qed

\paragraph{Lemma 39.3 (greedy construction yields $\gg N^{1/3}$).}
Let $(a_n)_{n\ge1}$ be the greedy Sidon sequence defined by $a_1=1$ and for $n\ge1$ letting $a_{n+1}$ be the smallest integer $>a_n$ such that $\{a_1,\dots,a_n,a_{n+1}\}$ is Sidon. Then
\[
 a_{n+1}\le n^3+1\quad\text{for all }n\ge1.
\]
Consequently, for $N\ge 2$,
\[
|\{a_n\le N\}|\ \ge\ \lfloor (N-1)^{1/3}\rfloor.
\]

\emph{Proof.}
Assume $\{a_1,\dots,a_n\}$ is Sidon.  Consider candidates $x>a_n$.  The Sidon condition for the enlarged set fails for $x$ if and only if one of the following occurs:
\begin{itemize}
\item $2x$ equals an existing sum $a_i+a_j$ with $1\le i\le j\le n$;
\item or $x+a_i$ equals an existing sum $a_j+a_k$ with $1\le j\le k\le n$ for some $i$.
\end{itemize}
Let $S_n:=\{a_i+a_j:1\le i\le j\le n\}$ be the set of existing sums.  Since the set is Sidon, all these sums are distinct, so
\[
|S_n|=\binom{n+1}{2}=\frac{n(n+1)}{2}.
\]
Each existing sum $s\in S_n$ forbids at most one candidate $x$ via $2x=s$, namely $x=s/2$ (which is either not an integer or a single integer).  Also, for each pair $(s,a_i)$ with $s\in S_n$ and $a_i\in\{a_1,\dots,a_n\}$, the equation $x+a_i=s$ forbids at most one integer $x=s-a_i$.
Therefore the total number of integers $x$ that are forbidden by either type of collision is at most
\[
|S_n|\ +\ n\,|S_n|\ =\ (n+1)|S_n|\ =\ (n+1)\frac{n(n+1)}{2}=\frac{n(n+1)^2}{2}\ <\ n^3\quad\text{for }n\ge1.
\]
Hence among the integers in the range $\{a_n+1, a_n+2,\dots,a_n+n^3\}$ there must exist at least one admissible candidate. In particular, the next greedy choice satisfies
\[
 a_{n+1}\le a_n+n^3.
\]
Since trivially $a_n\le a_1+\sum_{t=1}^{n-1} t^3 = 1 + (n(n-1)/2)^2 < n^4$ for large $n$, this already implies a polynomial bound; to get the stated form, observe that the argument above does not use $a_n$ at all, only $n$, so we can simply take the admissible candidate among $\{1,2,\dots,n^3+1\}$, giving $a_{n+1}\le n^3+1$.

Finally, if $a_n\le N$ and $a_{n+1}\le n^3+1$, then $n^3+1\le N$ implies $n\le (N-1)^{1/3}$. Therefore $|\{a_k\le N\}|\ge \lfloor (N-1)^{1/3}\rfloor$. \qed

\subsection*{VERIFICATION}
\begin{itemize}
\item Lemma 39.1: the only delicate point is excluding the matching $\{a,d\}=\{b,c\}$; this would force $a=b$, contradicting the hypotheses.
\item Lemma 39.2: checked that ordered differences indeed number $m(m-1)$ and lie in a set of size $2(N-1)$.
\item Lemma 39.3: verified that the forbidden-candidate count is an \emph{upper} bound (some different collisions could forbid the same $x$), so the pigeonhole step is valid.
\item Computation: the script also explicitly rechecked the Sidon property for the first 300 greedy terms.
\end{itemize}

\subsection*{FINAL}
\textbf{UNRESOLVED}

(i) \emph{Strongest proved partial result.}
Every finite Sidon set $A\subseteq[1,N]$ has size $|A|<\sqrt{2N}+1$ (Lemma 39.2), and the greedy infinite Sidon sequence satisfies $|A\cap[1,N]|\ge \lfloor (N-1)^{1/3}\rfloor$ (Lemma 39.3). A local computation for the greedy sequence gives $|A\cap[1,10^5]|=161$.

(ii) \emph{First gap (crisp statement).}
I cannot construct (or rule out) an infinite Sidon set $A$ such that for every $\varepsilon>0$ one has $|A\cap[1,N]|\ge c_{\varepsilon}N^{1/2-\varepsilon}$ for all sufficiently large $N$.

(iii) \emph{Top 3 next moves (concrete).}
\begin{enumerate}
\item Try to build an explicit infinite Sidon set by concatenating finite Sidon sets at increasing scales while preserving the Sidon property across scales (a ``block concatenation'' construction), and track the resulting counting function.
\item Attempt to upgrade the probabilistic ``bounded convolution'' constructions mentioned in the problem text to a genuine Sidon construction by enforcing exact uniqueness rather than $O(1)$ representations.
\item Computation: search (by backtracking/ILP) for large Sidon subsets of $[1,N]$ for moderate $N$ and try to detect a pattern that can be iterated to infinity.
\end{enumerate}

(iv) \emph{What a minimal counterexample would likely look like.}
A ``no'' answer would amount to a universal obstruction forcing a power-saving below $N^{1/2}$ on \emph{all} large $N$ for every infinite Sidon set.  A minimal obstruction would likely be a lemma forcing a relatively large gap in $A$ infinitely often, translating into $|A\cap[1,N]|\le N^{1/2-\delta}$ along a dense set of $N$.


