% Erdos Problem #893

% Erdos Problem #893

\noindent\textbf{1) FORMAL RESTATEMENT}

Let $\tau(n)$ denote the number of positive divisors of the positive integer $n$.
Define
\[
 f(n):=\sum_{1\le k\le n}\tau(2^k-1) \qquad (n\ge 1).
\]
Question: does the ratio $f(2n)/f(n)$ converge as $n\to\infty$?

\emph{Ambiguity.} “Tend to a limit” could mean a finite real limit, or could
allow the extended real value $+\infty$. I will treat both:

\begin{itemize}
\item (Q1) Does $\lim_{n\to\infty} f(2n)/f(n)$ exist as a finite real number?
\item (Q2) Does $\lim_{n\to\infty} f(2n)/f(n)=+\infty$?
\end{itemize}

\medskip
\noindent\textbf{2) QUICK LITERATURE/CONTEXT CHECK}

The problem statement records that Kova\v{c} and Luca (2025) proved
$\limsup_{n\to\infty} f(2n)/f(n)=\infty$, hence (Q1) has a negative answer.
It also records numerical/theoretical evidence suggesting (Q2) may hold.

\medskip
\noindent\textbf{3) ATTACK PLAN}

\begin{itemize}
\item Prove elementary structural lower bounds on $\tau(2^k-1)$ using divisibility
properties of Mersenne numbers.
\item Do a computational sanity check for $n\le 30$ (or similar) to see whether
ratios are stable or exhibit spikes.
\item Identify a crisp next-step lemma that would force $f(2n)/f(n)\to\infty$.
\end{itemize}

\medskip
\noindent\textbf{4) WORK}

\textbf{PHASE 1 — FAST REALITY CHECK (computation).}

I computed $\tau(2^k-1)$ for $1\le k\le 60$, then $f(n)$ and the ratios
$f(2n)/f(n)$ for $1\le n\le 30$.

Selected values of $\tau(2^k-1)$ for $k\le 20$:
\[
\begin{array}{c|cccccccccc}
 k &1&2&3&4&5&6&7&8&9&10\\\hline
 \tau(2^k-1) &1&2&2&4&2&6&2&8&4&8
\end{array}
\qquad
\begin{array}{c|cccccccccc}
 k &11&12&13&14&15&16&17&18&19&20\\\hline
 \tau(2^k-1) &4&24&2&8&8&16&2&32&2&48
\end{array}
\]

Ratios $f(2n)/f(n)$ for $1\le n\le 30$:
\[
\begin{array}{c|cccccccccc}
 n&1&2&3&4&5&6&7&8&9&10\\\hline
 f(2n)/f(n)&3.00&3.00&3.40&3.00&3.545&3.941&4.053&3.741&4.355&4.744
\end{array}
\]
\[
\begin{array}{c|cccccccccc}
 n&11&12&13&14&15&16&17&18&19&20\\\hline
 f(2n)/f(n)&4.953&4.672&4.768&5.208&5.941&5.337&5.466&8.081&8.051&7.086
\end{array}
\]
\[
\begin{array}{c|cccccccccc}
 n&21&22&23&24&25&26&27&28&29&30\\\hline
 f(2n)/f(n)&7.406&7.488&7.719&7.831&8.047&8.337&8.638&8.057&8.095&15.689
\end{array}
\]
The maximum among $n\le 30$ in this computation is $f(60)/f(30)\approx 15.689$.

\medskip
\textbf{Problem-specific lemmas (elementary lower bounds).}

\medskip
\noindent\textbf{Lemma 893.1 (Divisibility along divisors).}
If $d\mid k$ then $2^d-1\mid 2^k-1$.

\emph{Proof.}
Write $k=dr$. Then
\[
2^k-1 = 2^{dr}-1 = (2^d)^r-1 = (2^d-1)\bigl((2^d)^{r-1}+(2^d)^{r-2}+\cdots+1\bigr),
\]
so $2^d-1$ divides $2^k-1$.
\qed

\medskip
\noindent\textbf{Lemma 893.2 (A uniform divisor-count lower bound).}
For every integer $k\ge 1$,
\[ \tau(2^k-1)\ \ge\ \tau(k). \]
Consequently,
\[ f(n)=\sum_{k\le n}\tau(2^k-1)\ \ge\ \sum_{k\le n}\tau(k). \]

\emph{Proof.}
For each divisor $d\mid k$, Lemma~893.1 gives a divisor $2^d-1$ of $2^k-1$.
If $d_1\ne d_2$ then $2^{d_1}-1\ne 2^{d_2}-1$, so this produces at least $\tau(k)$
distinct divisors of $2^k-1$. Therefore $\tau(2^k-1)\ge \tau(k)$.
Summing over $1\le k\le n$ gives the displayed inequality for $f(n)$.
\qed

\medskip
\noindent\textbf{Lemma 893.3 (Crude growth of $f(n)$).}
For $n\ge 2$,
\[ n\log n - n\ \le\ \sum_{k\le n}\tau(k)\ \le\ n(1+\log n). \]
In particular, $f(n)\ge n\log n - n$.

\emph{Proof.}
The identity
\[ \sum_{k\le n}\tau(k) = \sum_{k\le n}\sum_{d\mid k} 1 = \sum_{d\le n}\#\{k\le n: d\mid k\}
= \sum_{d\le n} \Bigl\lfloor\frac{n}{d}\Bigr\rfloor
\]
is standard by swapping the order of summation.
For the upper bound, $\lfloor n/d\rfloor\le n/d$ gives
\[ \sum_{d\le n}\Bigl\lfloor\frac{n}{d}\Bigr\rfloor \le n\sum_{d\le n}\frac{1}{d}
\le n(1+\log n), \]
using $\sum_{d\le n}1/d \le 1+\int_1^n \frac{dt}{t}=1+\log n$.
For the lower bound, $\lfloor n/d\rfloor\ge n/d -1$ gives
\[ \sum_{d\le n}\Bigl\lfloor\frac{n}{d}\Bigr\rfloor \ge n\sum_{d\le n}\frac{1}{d}-n
\ge n\int_1^n \frac{dt}{t} - n = n\log n - n.
\]
Finally, combine with Lemma~893.2 to get $f(n)\ge \sum_{k\le n}\tau(k)$.
\qed

\medskip
\noindent\textbf{5) VERIFICATION}

\begin{itemize}
\item Lemma~893.1 is the standard factorization $x^r-1=(x-1)(x^{r-1}+\cdots+1)$.
\item Lemma~893.2: injectivity of $d\mapsto 2^d-1$ is immediate since $2^d-1$ is
strictly increasing in $d$.
\item Lemma~893.3: the bounds on harmonic sums are correct and fully explicit.
\end{itemize}

\medskip
\noindent\textbf{6) FINAL}

\textbf{UNRESOLVED}

(i) \emph{Strongest proved partial result.}
An elementary lower bound is $f(n)\ge n\log n - n$ (Lemma~893.3).
Computationally, the ratios $f(2n)/f(n)$ for $n\le 30$ range from $3$ up to
approximately $15.689$.

(ii) \emph{First gap (crisp).}
Prove or disprove that $f(2n)/f(n)\to +\infty$ as $n\to\infty$.
Equivalently, show that the contribution of the block $\sum_{n<k\le 2n}\tau(2^k-1)$
dominates $f(n)$.

(iii) \emph{Top 3 next moves.}
\begin{enumerate}
\item Prove a quantitative lower bound of the form
$\sum_{n<k\le 2n}\tau(2^k-1) \ge c(n)\, f(n)$ with $c(n)\to\infty$.
\item Identify a subsequence $n_j$ for which many $k\in(n_j,2n_j]$ have unusually
large $\tau(2^k-1)$ (e.g. by forcing $2^k-1$ to have many small prime factors).
\item Computational: extend the exact factorization/\,divisor-count computation to
larger $k$ (beyond 60) using ECM or specialized Mersenne factorization routines,
and empirically locate the growth of $\max_{n\le X} f(2n)/f(n)$.
\end{enumerate}

(iv) \emph{Minimal counterexample structure (if $f(2n)/f(n)\not\to\infty$).}
A minimal counterexample to divergence would require that for infinitely many
scales $n$, the block $(n,2n]$ contributes only a bounded multiple of the earlier
sum. This would mean that “typical” values of $\tau(2^k-1)$ in $(n,2n]$ are not
significantly larger than those for $k\le n$, and that large spikes (like the one
visible at $k=60$ in the sanity check) occur too sparsely to force divergence.


