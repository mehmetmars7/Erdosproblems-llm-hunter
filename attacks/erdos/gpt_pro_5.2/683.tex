% Erdos Problem #683
% Gaps Between Primes (II)
\item
\textbf{FORMAL RESTATEMENT.}
Let $P(m)$ be the largest prime divisor of $m$ (with $P(1)=1$).
Does there exist an absolute constant $c>0$ such that for all integers $n\ge1$ and $1\le k\le n$,
\[
P\!\binom{n}{k}\ \ge\ \min\bigl(n-k+1,\ k^{1+c}\bigr)?
\]
The problem text notes two related facts (treated as given):
(1) Sylvester--Schur: if $k\le n/2$ then $P\!\binom{n}{k}>k$.
(2) Erd\H{o}s (1935): $P\!\binom{n}{k} \gg k\log k$.

\textbf{QUICK LITERATURE/CONTEXT CHECK.}
I do not import anything beyond the two stated facts above.

\textbf{ATTACK PLAN.}
The inequality has two regimes:
\begin{itemize}
\item If $k$ is close to $n$, then $n-k+1$ is small and one only needs a modest prime divisor.
\item If $k$ is far from $n$ and $k^{1+c}$ is the smaller term, one needs a genuinely \emph{superlinear} prime divisor in $k$.
\end{itemize}
The Sylvester--Schur theorem completely handles the ``$n-k+1$'' regime when $k>n/2$. The difficult part is forcing $P\!\binom{n}{k} \ge k^{1+c}$ for a uniform $c>0$ across the remaining range.

\textbf{WORK.}
\textbf{Lemma 1 (The $k>n/2$ range is automatic from Sylvester--Schur).}
Assume the Sylvester--Schur statement from the problem: for $t\le n/2$, $P\!\binom{n}{t}>t$.
Then for every $n$ and every $k$ with $k>n/2$,
\[
P\!\binom{n}{k}\ \ge\ n-k+1.
\]
Consequently, for $k>n/2$ the desired inequality holds for \emph{every} $c>0$ (since then $\min(n-k+1,k^{1+c})=n-k+1$).

\emph{Proof.}
Let $t:=n-k$ (so $0\le t < n/2$). Then $\binom{n}{k}=\binom{n}{t}$.
By Sylvester--Schur applied to $t\le n/2$, we have $P\!\binom{n}{t}>t=n-k$.
Hence $P\!\binom{n}{k}=P\!\binom{n}{t}\ge (n-k)+1=n-k+1$.
Also $k>n/2$ implies $k^{1+c}>k>n-k+1$, so the minimum is $n-k+1$.
\hfill$\square$

\textbf{Lemma 2 (A small-$n$ obstruction bounds how large $c$ could be).}
If the inequality holds for all $n,k$, then necessarily
\[
c\le \frac{\log 5}{\log 3}-1\approx 0.464974\,.
\]

\emph{Proof.}
Take $(n,k)=(10,3)$. Then $\binom{10}{3}=120$ has $P(120)=5$.
Also $\min(n-k+1,k^{1+c})=\min(8,3^{1+c})$.
For any $c$ such that $3^{1+c}\le 8$, the minimum equals $3^{1+c}$ and the inequality forces
$5\ge 3^{1+c}$, i.e.
\[c\le \frac{\log 5}{\log 3}-1.
\]
This gives an absolute upper bound on any admissible $c$.
\hfill$\square$

\textbf{VERIFICATION (FAST REALITY CHECK).}
I brute-forced the inequality for all pairs $(n,k)$ with $n\le 500$ and numerically maximized the largest $c$ for which it holds simultaneously on this finite range.
The computation found the same obstruction as Lemma~2:
\begin{verbatim}
For N=30,50,100,200,500:
  maximum c for which P(binomial(n,k)) >= min(n-k+1,k^(1+c)) for all n<=N
  is c ~= 0.464974, and the first failing pair when c is increased is (n,k)=(10,3).
\end{verbatim}
This does \emph{not} indicate what happens asymptotically as $n\to\infty$; it only confirms that small cases already cap any possible $c$.

\textbf{FINAL.}
\textbf{UNRESOLVED.}

(i) \textbf{Strongest proved partial result.}
Using the Sylvester--Schur fact stated in the problem, the conjectured inequality holds for all $k>n/2$ (Lemma~1). Also, any feasible constant $c$ must satisfy $c\le \log 5/\log 3 -1$ (Lemma~2).

(ii) \textbf{First gap (crisp).}
One needs an unconditional lower bound of the form
\[P\!\binom{n}{k}\ \ge\ k^{1+c}
\]
uniformly over all $n$ and all $k$ in the regime where $k^{1+c}\le n-k+1$ (in particular, for many $k\le n/2$). The results quoted in the problem ($P>k$ and $P\gg k\log k$) fall short of any $k^{1+c}$.

(iii) \textbf{Top 3 next moves.}
(1) Identify a structural source of large prime divisors of $\binom{n}{k}$ (e.g. via gcd cancellations in the numerator/denominator) that can force a prime of size $k^{1+c}$.
(2) Compute (for large $n$, say $n\le 10^6$) the minimal value of
\[\frac{\log P\!\binom{n}{k}}{\log k}-1
\]
over the subrange where $k^{1+c}\le n-k+1$ to guess the best possible $c$.
(3) Attempt to reduce the problem to a statement about prime factors of products of $k$ consecutive integers and then seek ``large prime factor'' results in that setting.

(iv) \textbf{Minimal counterexample structure.}
A counterexample to the existence of a uniform $c>0$ would be an infinite family of pairs $(n,k)$ with $k$ large and $k^{1+c}\le n-k+1$ such that all prime divisors of $\binom{n}{k}$ are $\le k^{1+c}$ (indeed much smaller), despite $k$ itself being large.


