%Erdos problem 513
\noindent\textbf{FORMAL RESTATEMENT}

Let $f(z)=\sum_{n=0}^{\infty} a_n z^n$ be a transcendental entire function (so infinitely many $a_n\neq 0$ and the radius of convergence is $+\infty$). For $r>0$ define
\[
\mu(r):=\max_{n\ge 0} |a_n| r^n\qquad\text{(maximum term)},
\]
\[
M(r):=\max_{|z|=r} |f(z)|\qquad\text{(maximum modulus)}.
\]
Question: What is the supremum over such $f$ of
\[
\liminf_{r\to\infty} \frac{\mu(r)}{M(r)}\ ?
\]

Edge cases: $f$ polynomial is excluded (``transcendental''). The ratio is always in $(0,1]$ for each fixed $r$ (Lemma 1 gives $\mu(r)\le M(r)$).

\medskip
\noindent\textbf{QUICK LITERATURE/CONTEXT CHECK}

The problem statement notes: the value is trivially in $[1/2,1)$; K\"ov\'ari observed it must be $>1/2$; Clunie--Hayman showed it is $\le 2/\pi-c$ for some absolute $c>0$. I do not reproduce those deeper arguments here.

\medskip
\noindent\textbf{ATTACK PLAN}

\begin{itemize}
\item Proof-track idea (upper bound): exhibit mechanisms forcing $M(r)$ to be noticeably larger than $\mu(r)$ infinitely often (e.g., many comparable large terms aligned in phase).
\item Construction-track idea (lower bound): build $f$ where at most radii, a single term dominates and remaining terms are arranged to cancel so that $M(r)$ stays close to $\mu(r)$.
\end{itemize}
I did not determine the optimal supremum.

\medskip
\noindent\textbf{WORK}

\noindent\textbf{Fast reality check (example $f(z)=e^z$).}
For $f(z)=e^z=\sum_{n\ge 0} z^n/n!$, we have $M(r)=e^r$. Numerically maximising $r^n/n!$ over $n$ gives:

\begin{center}
\begin{tabular}{c|c|c}
$r$ & argmax $n$ of $r^n/n!$ & $\mu(r)/M(r)$\\
\hline
1 & 0 & 0.367879\\
2 & 2 & 0.270671\\
5 & 4 & 0.175467\\
10 & 9 & 0.125110\\
20 & 19 & 0.088835\\
50 & 49 & 0.056325\\
100 & 99 & 0.039861\\
\end{tabular}
\end{center}
so in this example $\mu(r)/M(r)\to 0$.

\medskip
\noindent\textbf{Lemma 1 ($\mu(r)\le M(r)$ for all $r>0$).}
For any entire $f(z)=\sum_{n\ge 0} a_n z^n$ and any $r>0$,
\[
\mu(r)\le M(r).
\]

\textit{Proof.}
Fix $r>0$ and define the $2\pi$-periodic function
\[
F(\theta):=f(re^{i\theta})=\sum_{n\ge 0} a_n r^n e^{in\theta}.
\]
Since the series converges absolutely for each fixed $r$, $F\in L^2([0,2\pi])$ and we may compute
\[
\frac{1}{2\pi}\int_0^{2\pi} |F(\theta)|^2\,d\theta
=\sum_{n\ge 0} |a_n|^2 r^{2n}
\]
by orthogonality of the exponentials $e^{in\theta}$.
Therefore
\[
\max_{\theta} |F(\theta)|^2 \ge \frac{1}{2\pi}\int_0^{2\pi} |F(\theta)|^2\,d\theta
=\sum_{n\ge 0} |a_n|^2 r^{2n} \ge \max_{n\ge 0} |a_n|^2 r^{2n} = \mu(r)^2.
\]
Taking square roots yields $\max_{\theta}|F(\theta)|\ge \mu(r)$, i.e. $M(r)\ge \mu(r)$.
\hfill$\square$

\medskip
\noindent\textbf{Lemma 2 (constant-argument coefficients force $\liminf \mu/M\le 1/2$).}
Assume $f$ is transcendental entire and there exists a fixed $\phi\in\mathbb{R}$ such that $e^{-i\phi} a_n\ge 0$ for all $n$ (i.e., all nonzero coefficients have the same complex argument). Then
\[
\liminf_{r\to\infty} \frac{\mu(r)}{M(r)}\le \frac12.
\]

\textit{Proof.}
Multiply $f$ by the unimodular constant $e^{-i\phi}$; this does not change the ratio $\mu(r)/M(r)$, so we may assume $a_n\ge 0$ for all $n$.
For $r>0$ define the nondecreasing real function
\[
F(r):=f(r)=\sum_{n\ge 0} a_n r^n\in(0,\infty).
\]
For each $r$, $M(r)\ge |f(r)|=F(r)$ since $z=r$ lies on the circle $|z|=r$.

Because $f$ is transcendental, infinitely many $a_n$ are positive. For each $r>0$, let $n(r)$ be the smallest index attaining the maximum term, i.e.
\[
\mu(r)=a_{n(r)} r^{n(r)}.
\]
As $r$ increases, $n(r)$ is nondecreasing and tends to $\infty$. Hence there are infinitely many ``jump radii'' $r_k\to\infty$ at which $n(r)$ changes value.
At such an $r_k$, there exist two distinct indices $m< n$ with
\[
a_m r_k^m = a_n r_k^n = \mu(r_k)
\]
(both attain the maximum). Since all coefficients are nonnegative,
\[
F(r_k)=\sum_{j\ge 0} a_j r_k^j \ge a_m r_k^m + a_n r_k^n = 2\mu(r_k).
\]
Therefore
\[
M(r_k)\ge F(r_k)\ge 2\mu(r_k)
\quad\Longrightarrow\quad
\frac{\mu(r_k)}{M(r_k)}\le \frac12.
\]
Since $r_k\to\infty$, this gives $\liminf_{r\to\infty} \mu(r)/M(r)\le 1/2$.
\hfill$\square$

\medskip
\noindent\textbf{VERIFICATION}

\begin{itemize}
\item Lemma 1: used Parseval/orthogonality; absolute convergence on $|z|=r$ ensures $F\in L^2$ and justifies termwise integration.
\item Lemma 2: the only nontrivial point is existence of infinitely many jump radii. Since $n(r)$ is integer-valued and tends to infinity, it must jump infinitely often; at each jump there is a tie between consecutive maximisers.
\item Lemma 2 does not contradict K\"ov\'ari's observation ($>1/2$ must hold for the supremum), because Lemma 2 only applies to the restricted class of functions with coefficients of constant argument.
\end{itemize}

\medskip
\noindent\textbf{FINAL}

\textbf{UNRESOLVED}

(i) Strongest proved partial result: For any entire $f$, $\mu(r)\le M(r)$ for all $r$ (Lemma 1). For transcendental $f$ whose coefficients have constant argument, $\liminf_{r\to\infty} \mu(r)/M(r)\le 1/2$ (Lemma 2), so any extremiser for the supremum must have genuinely oscillating coefficient arguments.

(ii) First gap (crisp): Determine (or sharply bound) the supremum over transcendental entire $f$ of $\liminf_{r\to\infty} \mu(r)/M(r)$.

(iii) Top 3 next moves:
\begin{enumerate}
\item Try to construct explicit families with controlled cancellation: choose lacunary series with carefully chosen coefficient phases so that at ``tie radii'' the circle maximum modulus is $<2\mu(r)$.
\item Attempt to prove a universal upper bound $<1$ by showing that for any transcendental entire $f$, there are infinitely many $r$ for which at least two large terms can be phase-aligned enough to force $M(r)\ge (1+\delta)\mu(r)$.
\item Numerical exploration: for candidate lacunary phase-tuned series, approximate $\mu(r)$ and $M(r)$ for large $r$ via sampling on circles and identify the behaviour of the liminf.
\end{enumerate}

(iv) Minimal counterexample structure (to any proposed sharp constant): would be an entire function whose large terms are arranged so that for infinitely many radii where multiple terms are of comparable magnitude, their phases cannot be simultaneously aligned on any $|z|=r$ circle, keeping $M(r)$ close to $\mu(r)$.


