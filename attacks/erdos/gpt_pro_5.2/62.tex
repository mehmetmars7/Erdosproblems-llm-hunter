% Erdős Problem #62
% URL: https://www.erdosproblems.com/62

If $G_1,G_2$ are two graphs with chromatic number $\aleph_1$ then must there exist a graph $G$ whose chromatic number is $4$ (or even $\aleph_0$) which is a subgraph of both $G_1$ and $G_2$? Erd\H{o}s also asked \cite{Er87} about finding a common subgraph $H$ (with chromatic number either $4$ or $\aleph_0$) in any finite collection of graphs with chromatic number $\aleph_1$. Every graph with chromatic number $\aleph_1$ contains all sufficiently large odd cycles (which have chromatic number $3$), see [594] . This was proved by Erd\H{o}s, Hajnal, and Shelah \cite{EHS74}. Erd\H{o}s wrote \cite{Er87} that 'probably' every graph with chromatic number $\aleph_1$ contains as subgraphs all graphs with chromatic number $4$ with sufficiently large girth. References [EHS74] Erd\H{o}s, P. and Hajnal, A. and Shelah, S., On some general properties of chromatic numbers . Topics in topology (Proc. Colloq., Keszthely, 1972) (1974), 243-255. [Er87] Erd\H{o}s, P., Some problems on finite and infinite graphs . Logic and combinatorics (Arcata, Calif., 1985) (1987), 223-228.

%Erdos problem 62
\subsection*{FORMAL RESTATEMENT}
A (simple) graph $G$ has chromatic number $\chi(G)$ equal to the least cardinal $\kappa$ such that $V(G)$ can be 
properly colored with $\kappa$ colors.

Given graphs $G_1,G_2$ with $\chi(G_1)=\chi(G_2)=\aleph_1$,
question: must there exist a graph $H$ with $\chi(H)=4$ (or alternatively with $\chi(H)=\aleph_0$) such that $H$ is a 
subgraph of both $G_1$ and $G_2$? Here ``$H$ is a subgraph of $G_i$'' means there exists an injective map 
$f_i:V(H)\to V(G_i)$ such that if $xy\in E(H)$ then $f_i(x)f_i(y)\in E(G_i)$.

\subsection*{QUICK LITERATURE/CONTEXT CHECK}
Only what is explicitly stated in the problem statement is used:
\begin{itemize}
\item Erd\H{o}s asked about common subgraphs of chromatic number $4$ or $\aleph_0$ in two (or finitely many) graphs of 
chromatic number $\aleph_1$.
\item The statement records (as known) that every graph with chromatic number $\aleph_1$ contains all sufficiently large 
odd cycles (chromatic number $3$), attributed to Erd\H{o}s--Hajnal--Shelah.
\item Erd\H{o}s suggested that probably every $\aleph_1$-chromatic graph contains all $4$-chromatic graphs of sufficiently 
large girth.
\end{itemize}
No other external results are assumed.

\subsection*{ATTACK PLAN}
\begin{itemize}
\item Prove basic structural facts that follow formally from $\chi(G)=\aleph_1$ (e.g., existence of finite $k$-chromatic 
subgraphs for every $k$).
\item Try to see whether those facts can force a \emph{common} $4$-chromatic subgraph of two such graphs.
\item Since the main problem is infinite/set-theoretic, computation is not expected to decide it, but we do sanity checks 
on the finite-compactness statements used.
\end{itemize}

\subsection*{WORK}
\textbf{Lemma 1 (Compactness for $k$-colorability).}
Let $G$ be a (possibly infinite) graph and let $k\in\mathbb{N}$.
If every finite induced subgraph of $G$ is $k$-colorable, then $G$ is $k$-colorable.

\emph{Proof.}
Let $V=V(G)$. Consider the set $\Omega:=\{1,2,\dots,k\}^V$ of all functions $f:V\to\{1,\dots,k\}$, equipped with the 
product topology where each factor $\{1,\dots,k\}$ is discrete (and hence compact).
By Tychonoff's theorem, $\Omega$ is compact.

For each edge $uv\in E(G)$, define the closed set
\[
A_{uv}:=\{f\in\Omega: f(u)\ne f(v)\}.
\]
This set is closed because its complement $\{f: f(u)=f(v)\}$ depends only on the coordinates $u,v$ and is clopen in the 
product topology.
A proper $k$-coloring of $G$ is exactly a function $f\in\bigcap_{uv\in E(G)} A_{uv}$.

We show the family $\{A_{uv}:uv\in E(G)\}$ has the finite intersection property.
Take any finite set of edges $F\subseteq E(G)$. Let $W$ be the (finite) set of vertices incident to edges in $F$.
The induced subgraph $G[W]$ is finite, hence $k$-colorable by hypothesis; let $g:W\to\{1,\dots,k\}$ be a proper 
coloring.
Extend $g$ arbitrarily to a function $f:V\to\{1,\dots,k\}$ (e.g., assign color $1$ to vertices in $V\setminus W$).
Then $f\in\bigcap_{uv\in F}A_{uv}$.
So every finite subcollection intersects.

By compactness of $\Omega$, the full intersection $\bigcap_{uv\in E(G)}A_{uv}$ is nonempty.
Any $f$ in this intersection is a proper $k$-coloring of $G$.
\hfill$\square$

\medskip
\textbf{Lemma 2 (Finite $k$-chromatic subgraphs exist at all finite levels).}
If $\chi(G)>k$ for some finite $k$, then $G$ contains a finite induced subgraph $H$ with $\chi(H)=k+1$.
In particular, if $\chi(G)=\aleph_1$, then for every $t\in\mathbb{N}$ there exists a finite induced subgraph of $G$ with 
chromatic number exactly $t$.

\emph{Proof.}
Assume $\chi(G)>k$; equivalently, $G$ is not $k$-colorable.
By Lemma 1 (contrapositive), there must exist a finite induced subgraph $H\subseteq G$ that is not $k$-colorable.
Among all such $H$, choose one with the minimum number of vertices.
Then every proper induced subgraph of $H$ is $k$-colorable (by minimality), but $H$ itself is not.
Therefore $\chi(H)=k+1$.

If $\chi(G)=\aleph_1$, then $\chi(G)>t-1$ for every finite $t$, so applying the above with $k=t-1$ yields a finite induced 
subgraph with chromatic number exactly $t$. \hfill$\square$

\medskip
\textbf{Consequence for the main question.}
Given $G_1,G_2$ with $\chi(G_1)=\chi(G_2)=\aleph_1$, Lemma 2 implies:
\begin{itemize}
\item $G_1$ contains some finite $4$-chromatic subgraph $H_1$,
\item $G_2$ contains some finite $4$-chromatic subgraph $H_2$.
\end{itemize}
However, this does \emph{not} force $H_1\cong H_2$, and the problem asks whether there exists \emph{some} 
$4$-chromatic graph $H$ that appears in both.

\medskip
\textbf{FAST REALITY CHECK.}
This problem is inherently about infinite graphs and uncountable chromatic number.
Computation is not directly relevant to the main existence question.
The only computationally-checkable component here is Lemma 2's finite-minimality argument; it is purely finite and 
self-contained.

\subsection*{VERIFICATION}
\begin{itemize}
\item Lemma 1 uses Tychonoff compactness; the construction of the closed sets $A_{uv}$ and the finite intersection 
property is explicit.
\item Lemma 2's minimization is over finite induced subgraphs and therefore well-founded.
\item Quantifiers: Lemma 2 produces a finite induced subgraph of chromatic number $k+1$ when $\chi(G)>k$; no set-theoretic 
assumptions beyond those used for Lemma 1 are hidden.
\end{itemize}

\subsection*{FINAL}
\textbf{UNRESOLVED}

(i) Strongest proved partial result here: if $\chi(G)=\aleph_1$, then for every finite $t$ the graph $G$ contains a finite 
induced subgraph of chromatic number exactly $t$ (Lemma 2). In particular, each $G_i$ contains some finite 
$4$-chromatic subgraph.

(ii) First gap: given two graphs $G_1,G_2$ with $\chi(G_i)=\aleph_1$, prove (or refute) that the sets
\[
\mathcal{F}_i:=\{\text{isomorphism types of finite $4$-chromatic graphs that occur as subgraphs of }G_i\}
\]
have nonempty intersection.

(iii) Top 3 next moves:
\begin{itemize}
\item Attempt a construction track: try to build two $\aleph_1$-chromatic graphs whose families $\mathcal{F}_1$ and 
$\mathcal{F}_2$ are disjoint by systematically forbidding a partition of the countably many finite $4$-chromatic graphs.
\item Attempt a forcing/unavoidability track: identify a specific finite $4$-chromatic ``template'' that might be 
unavoidable in every $\aleph_1$-chromatic graph (analogous to unavoidable odd cycles in the cited result).
\item Use girth considerations suggested in the statement: investigate whether large-girth $4$-chromatic graphs are 
forced as subgraphs in all $\aleph_1$-chromatic graphs and whether such forcing could be made uniform across two graphs.
\end{itemize}

(iv) Minimal counterexample structure (if the statement were false): two graphs $G_1,G_2$ with chromatic number $\aleph_1$ 
such that every graph that embeds into both has chromatic number at most $3$ (or at most finite), i.e., their common 
subgraph spectrum avoids all $4$-chromatic graphs.


