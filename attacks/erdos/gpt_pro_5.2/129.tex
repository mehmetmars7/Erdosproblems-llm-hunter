% Erdos problem #129
% Solution / disproof for Erdos Problem #129 as written
% Following PROMPT_STRATEGY.MD
% Tools/Constraints:
% - Web browsing available? YES (not used; I restrict to what is stated in 123-137.tex)
% - Computation available (Python)? not needed

1) FORMAL RESTATEMENT

Let $R(n;k,r)$ be the least $N$ such that in every $r$-coloring of the edges of $K_N$ there exists a set of $n$ vertices that does **not** contain a copy of $K_k$ in at least one of the $r$ colors.
(So for $k=3,r=2$, an $n$-set is "bad" if it has no red triangle or no blue triangle.)

**Claim in the problem text.** There exists a constant $C=C(r)>1$ such that
\[
R(n;3,r) < C^{\sqrt{n}}\quad\text{for all }n.
\]

2) QUICK LITERATURE/CONTEXT CHECK

(Restricted to statements explicitly present in 123-137.tex.)
- The text itself states the claim is false for $r=2$ (Erdos--Hajnal--Rado 1965) and sketches the reason: every $n$-set contains $\gg n^2$ edge-disjoint triangles, so a random 2-coloring makes it unlikely for an $n$-set to be triangle-free in a given color.

3) ATTACK PLAN

- Produce a rigorous probabilistic construction for $r=2$ showing $R(n;3,2)\ge c^n$ for some $c>1$.
- This contradicts the stated upper bound $C^{\sqrt{n}}$.

4) WORK

Lemma 4.1 (Many edge-disjoint triangles in $K_n$).
Let $n\ge 3$ and let $m=\lfloor n/3\rfloor$.
Then $K_n$ contains at least $m^2$ edge-disjoint triangles.

*Proof.* Choose disjoint vertex classes $A,B,C$ of size $m$ inside $[n]$.
Consider the complete tripartite subgraph $K_{m,m,m}$ on $A\cup B\cup C$.
Label each class by the group $\mathbb{Z}_m$.
For each pair $(a,b)\in A\times B$, define $c:=a+b\in \mathbb{Z}_m$ and take the triangle $(a,b,c)$.
This produces exactly $m^2$ triangles.
Each edge between $A$ and $B$ is used exactly once by construction.
Given an edge between $B$ and $C$, say $(b,c)$, it appears in exactly one triangle because $a=c-b$ is uniquely determined; similarly for edges between $C$ and $A$.
Hence these $m^2$ triangles are edge-disjoint. \qed

Lemma 4.2 (A fixed $n$-set is very unlikely to be missing a monochromatic triangle).
Let $S$ be a fixed set of $n$ vertices in $K_N$, and color edges red/blue independently with probability $1/2$ each.
Let $t:=\lfloor n/3\rfloor^2$ be the number of edge-disjoint triangles guaranteed by Lemma 4.1 inside $S$.
Then
\[
\mathbb{P}(S\text{ has no red triangle}) \le (7/8)^t,
\qquad
\mathbb{P}(S\text{ has no blue triangle}) \le (7/8)^t,
\]
and hence
\[
\mathbb{P}(S\text{ is bad}) \le 2(7/8)^t.
\]

*Proof.*
For each of the $t$ edge-disjoint triangles $T_1,\dots,T_t$ from Lemma 4.1, the event
$E_i=\{\text{``$T_i$ is entirely red''}\}$ depends only on the colors of the three edges of $T_i$.
Because the triangles are edge-disjoint, these edge-sets are disjoint, and the events $E_i$ are independent.
Also $\mathbb{P}(E_i)=(1/2)^3=1/8$.
Therefore
\[
\mathbb{P}(S\text{ has no red triangle}) \le \mathbb{P}(E_1^c\cap\cdots\cap E_t^c)
= \prod_{i=1}^t (1-1/8) = (7/8)^t.
\]
The blue case is identical. The final inequality is a union bound.
\qed

Lemma 4.3 (Existence of an exponential-size coloring with no bad $n$-set).
There exists an absolute constant $c>1$ such that for all sufficiently large $n$,
\[
R(n;3,2) \ge c^n.
\]

*Proof.* Let $t:=\lfloor n/3\rfloor^2$.
For a given $N$, let $X$ be the random variable counting bad $n$-vertex subsets in the random coloring.
By Lemma 4.2 and linearity of expectation,
\[
\mathbb{E}X \le \binom{N}{n}\cdot 2\left(\frac78\right)^t.
\]
Use the standard bound $\binom{N}{n}\le (eN/n)^n$.
Choose $N:=\exp(\gamma n)$ with a small constant $\gamma>0$ to be fixed.
Then
\[
\log \mathbb{E}X \le n\log\left(\frac{eN}{n}\right) + \log 2 + t\log\left(\frac78\right)
= n\left(\gamma n + O(\log n)\right) + t\log\left(\frac78\right) + O(1).
\]
Since $t\sim n^2/9$ and $\log(7/8)<0$, for any fixed $\gamma$ with
\[
\gamma < -\frac{1}{9}\log\left(\frac78\right)
\]
(and ignoring the lower-order $O(n\log n)$ term for large $n$), the dominant $n^2$ coefficient is negative and thus $\mathbb{E}X<1$ for all sufficiently large $n$.
Hence there exists a coloring with $X=0$, i.e. with no bad $n$-set.
For such an $N$, the definition of $R(n;3,2)$ implies $R(n;3,2)>N$.
Thus $R(n;3,2)\ge \exp(\gamma n)=c^n$ where $c:=e^\gamma>1$.
\qed

Conclusion: disproof of the stated bound.
Lemma 4.3 gives $R(n;3,2)\ge c^n$ for some $c>1$ and large $n$, which contradicts any bound of the form $R(n;3,2) < C^{\sqrt{n}}$.
Therefore the claim "$\exists C(r)$ such that $R(n;3,r) < C(r)^{\sqrt{n}}$" is false as written (already for $r=2$).

5) VERIFICATION

- Lemma 4.1 provides an explicit edge-disjoint packing inside a $K_{m,m,m}$ subgraph.
- Lemma 4.2 correctly uses independence because triangles are edge-disjoint.
- The union bound and expectation argument are standard; the exponent comparison shows one can choose $\gamma>0$.

6) FINAL

LABEL: **FULL SOLUTION**
SUBLABEL: **COUNTEREXAMPLE/DISPROOF**

Explicit disproof: for $r=2$, the probabilistic construction above shows $R(n;3,2)\ge c^n$ for some $c>1$ and all large $n$, contradicting $R(n;3,2)<C^{\sqrt{n}}$.


