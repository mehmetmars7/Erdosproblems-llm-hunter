
\subsection*{Erd\H{o}s Problem 1084 (unit distances under separation)}

\subsubsection*{FORMAL RESTATEMENT}
For an integer \(d\ge 1\) and \(n\in\mathbb{N}\), let \(f_d(n)\) be the minimum integer such that for every set
\(P\subset\mathbb{R}^d\) of size \(|P|=n\) with \(\|x-y\|\ge 1\) for all distinct \(x,y\in P\), the number of unordered pairs \(\{x,y\}\subset P\) with \(\|x-y\|=1\) is at most \(f_d(n)\).
The problem asks for estimates of \(f_d(n)\) (as \(n\to\infty\) with \(d\) fixed).

\subsubsection*{QUICK LITERATURE/CONTEXT CHECK}
The problem statement records: \(f_1(n)=n-1\), \(f_2(n)<3n\) by a degree argument, and a refined planar bound
\(f_2(n)<3n-c n^{1/2}\) due to Erd\H{o}s. It also records a conjectural exact planar value for certain \(n\), and a claimed asymptotic refinement in \(d=3\).
I do not add external results here.

\subsubsection*{ATTACK PLAN}
\begin{itemize}
\item \textbf{Proof track (partial).} Relate \(f_d(n)\) to the kissing number in \(\mathbb{R}^d\) to get a clean linear upper bound. In \(d=2\), exhibit a triangular-lattice construction giving \(3n-O(\sqrt n)\) unit pairs and compute its exact unit-pair count.
\item \textbf{Disproof track.} Not applicable (estimation problem).
\end{itemize}

\subsubsection*{WORK}

\paragraph{Fast reality check.} For \(d=2\), the configuration of one point and a regular hexagon around it gives \(n=7\) and \(12\) unit pairs. A brute-force check on the triangular lattice within radius \(2\) shows this is best possible among 7-point subsets of that lattice.

\begin{definition}[Kissing number constant]
Let \(\tau_d\) be the maximum size of a set \(S\subset \mathbb{S}^{d-1}\) with \(\|u-v\|\ge 1\) for all distinct \(u,v\in S\).
Equivalently, \(\tau_d\) is the maximum number of points on the unit sphere such that mutual distances are at least \(1\).
(Geometrically, \(\tau_d\) equals the maximum number of disjoint radius-\(1/2\) balls that can simultaneously touch a central radius-\(1/2\) ball.)
\end{definition}

\begin{lemma}[Degree bound via kissing number]
Let \(P\subset\mathbb{R}^d\) have all pairwise distances \(\ge 1\). Form the graph \(G\) on vertex set \(P\) with an edge between \(x\ne y\) iff \(\|x-y\|=1\). Then every vertex has degree at most \(\tau_d\), and hence
\[
\#E(G)\le \frac{\tau_d}{2}\,|P|.
\]
In particular, \(f_d(n)\le \frac{\tau_d}{2}n\) for all \(n\).
\end{lemma}

\begin{proof}
Fix \(p\in P\). Let \(N(p)=\{q\in P: \|q-p\|=1\}\) be its unit neighbors.
For any distinct \(q,r\in N(p)\), we have \(\|q-r\|\ge 1\) by the separation assumption.
Map each \(q\in N(p)\) to the unit vector \(u_q=(q-p)/\|q-p\|\in\mathbb{S}^{d-1}\). Then
\(\|u_q-u_r\|=\|q-r\|\) (since \(\|q-p\|=\|r-p\|=1\)), so \(\|u_q-u_r\|\ge 1\).
Thus \(\{u_q: q\in N(p)\}\subset\mathbb{S}^{d-1}\) is a set of size \(|N(p)|\) with pairwise distances at least \(1\), so \(|N(p)|\le \tau_d\) by definition.
The edge bound follows by summing degrees and dividing by \(2\).
\end{proof}

\begin{lemma}[Planar triangular-lattice construction]
Let \(m\ge 0\). Consider the triangular lattice in axial coordinates
\(\Lambda=\{(a,b): a,b\in\mathbb{Z}\}\) with unit-distance edges between points whose differences are
\((\pm1,0),(0,\pm1),(\pm1,\mp1)\).
Let
\[
H_m:=\{(a,b)\in\Lambda: \max(|a|,|b|,|a+b|)\le m\}
\]
(the ``centered hexagon'' of radius \(m\)). Then
\[
|H_m|=3m^2+3m+1,\qquad \#\{\{x,y\}\subset H_m: \|x-y\|=1\}=9m^2+3m.
\]
In particular, this gives a configuration with \(n=3m^2+3m+1\) points and \(3n-(6m+3)=3n-O(\sqrt n)\) unit pairs.
\end{lemma}

\begin{proof}
The count \(|H_m|=3m^2+3m+1\) is standard for a centered hexagon: there is a central point and \(6k\) points on the ``ring'' at lattice distance \(k\) for each \(1\le k\le m\), so
\(|H_m|=1+\sum_{k=1}^m 6k=1+3m(m+1)=3m^2+3m+1\).

Now count unit edges in the induced unit-distance graph on \(H_m\). Every interior vertex has degree \(6\).
The boundary of \(H_m\) is a hexagon with \(6m\) boundary vertices: it has \(6\) corner vertices and \(6(m-1)\) non-corner boundary vertices. A direct neighbor check in axial coordinates shows:
\begin{itemize}
\item each corner boundary vertex has exactly \(3\) neighbors within \(H_m\),
\item each non-corner boundary vertex has exactly \(4\) neighbors within \(H_m\).
\end{itemize}
Thus the total degree deficit from \(6\) is
\[
\mathrm{def}:=\sum_{v\in H_m} (6-\deg(v)) = 6\cdot(6-3) + 6(m-1)\cdot(6-4) =18+12(m-1)=12m+6.
\]
Therefore
\[
2|E|=\sum_{v\in H_m}\deg(v)=6|H_m|-\mathrm{def}.
\]
Plugging \(|H_m|=3m^2+3m+1\) and \(\mathrm{def}=12m+6\) gives
\[
2|E|=6(3m^2+3m+1)-(12m+6)=18m^2+18m+6-12m-6=18m^2+6m,
\]
so \(|E|=9m^2+3m\).

Finally, since \(m\asymp \sqrt n\) when \(n=3m^2+3m+1\), the error term \(6m+3\) is \(O(\sqrt n)\), yielding \(|E|=3n-O(\sqrt n)\).
\end{proof}

\subsubsection*{VERIFICATION}
\begin{itemize}
\item The kissing-number lemma is purely geometric and uses only the separation hypothesis.
\item In the planar construction, I checked the edge counts computationally for \(m=0,1,2,3,4\):
\((|H_m|,|E|)=(1,0),(7,12),(19,42),(37,90),(61,156)\), matching \(3m^2+3m+1\) and \(9m^2+3m\).
\item Note: the problem statement contains a conjectural exact formula \(f_2(3n^2+3n+1)=9n^2+6n\). The natural centered-hexagon triangular-lattice patch above gives \(9n^2+3n\) unit pairs for that number of points; I did not resolve the discrepancy.
\end{itemize}

\subsubsection*{FINAL}
\textbf{UNRESOLVED.}
\begin{enumerate}
\item[(i)] \textbf{Strongest proved partial result here:} for all \(d\), \(f_d(n)\le (\tau_d/2)n\); and in \(d=2\) there exist configurations with \(3n-O(\sqrt n)\) unit pairs.
\item[(ii)] \textbf{First gap:} determine the correct second-order term in \(d=2\) (and in particular the exact value at the ``hexagonal numbers''), and in \(d=3\) prove or disprove the claimed \(6n-\Theta(n^{2/3})\) behaviour.
\item[(iii)] \textbf{Top 3 next moves:}
(1) for \(d=2\), formulate the extremal problem as maximizing edges in a planar unit-distance graph with minimum separation and analyze boundary effects rigorously;
(2) attempt a discrete isoperimetric inequality on the hexagonal/triangular lattice to show optimality of near-hexagonal patches;
(3) computationally optimize for small \(n\) to test the conjectured exact values.
\item[(iv)] \textbf{Minimal counterexample structure (to ``triangular lattice is extremal'' in \(d=2\)):}
a planar packing close to hexagonal but with local defects that increase the unit-edge count without violating separation, likely concentrated near the boundary.
\end{enumerate}


