\subsection*{Erd\H{o}s problem \#335}

\noindent\textbf{1) FORMAL RESTATEMENT.}
Let $A,B\subseteq\mathbb{N}$ have (asymptotic) densities
\[
 d(A)=\lim_{N\to\infty}\frac{|A\cap[1,N]|}{N}=\alpha>0,
 \qquad
 d(B)=\lim_{N\to\infty}\frac{|B\cap[1,N]|}{N}=\beta>0.
\]
Define the sumset $A+B:=\{a+b: a\in A,\ b\in B\}$. The problem asks to characterize all pairs $(A,B)$ such that
\[
 d(A+B)=d(A)+d(B)=\alpha+\beta.
\]

\medskip
\noindent\textbf{2) QUICK LITERATURE/CONTEXT CHECK.}
No external results are assumed beyond the problem statement.

\medskip
\noindent\textbf{3) ATTACK PLAN.}
\begin{itemize}
\item Note basic necessary constraints (e.g. $\alpha+\beta\le 1$).
\item Analyze the periodic case (unions of residue classes), where density reduces to finite group counting.
\item Use periodic examples as sanity checks and as a source of families satisfying the equality.
\end{itemize}

\medskip
\noindent\textbf{4) WORK.}

\medskip
\noindent\textbf{Lemma 335.1 (periodic sets reduce to residue-class counting).}
Fix $m\ge 1$. Let $A,B\subseteq\mathbb{N}$ be unions of residue classes modulo $m$:
\[
 A=\{n\ge 1: n\bmod m\in A_0\},\qquad B=\{n\ge 1: n\bmod m\in B_0\}
\]
for some $A_0,B_0\subseteq\mathbb{Z}/m\mathbb{Z}$. Then the densities exist and
\[
 d(A)=\frac{|A_0|}{m},\qquad d(B)=\frac{|B_0|}{m},\qquad d(A+B)=\frac{|A_0+B_0|}{m},
\]
where $A_0+B_0:=\{a+b\bmod m: a\in A_0,\ b\in B_0\}$.

\noindent\emph{Proof.}
Each residue class modulo $m$ has density $1/m$ in $\mathbb{N}$. Since $A$ is a disjoint union of $|A_0|$ residue classes, $d(A)=|A_0|/m$, and similarly for $B$.

If $n\in A+B$, then $n\equiv a+b\pmod m$ for some residues $a\in A_0$, $b\in B_0$, so $(A+B)\bmod m\subseteq A_0+B_0$. Conversely, if $r\in A_0+B_0$, choose residues $a\in A_0$, $b\in B_0$ with $r\equiv a+b\pmod m$; then every sufficiently large integer $n\equiv r\pmod m$ can be written as $n=(a+tm)+(b+t'm)$ with $t,t'\ge 0$ large enough, hence $n\in A+B$. Thus $A+B$ is also a union of residue classes with residue set $A_0+B_0$, and therefore $d(A+B)=|A_0+B_0|/m$.
\hfill$\square$

\medskip
\noindent\textbf{Lemma 335.2 (an explicit infinite family with $d(A+B)=d(A)+d(B)$).}
Fix any integer $m\ge 4$, and define
\[
A:=\{n\ge 1: n\bmod m\in\{0,1\}\},\qquad
B:=\{n\ge 1: n\bmod m\in\{0,2\}\}.
\]
Then $d(A)=2/m$, $d(B)=2/m$, and $d(A+B)=4/m=d(A)+d(B)$.

\noindent\emph{Proof.}
By Lemma~335.1, $d(A)=|\{0,1\}|/m=2/m$ and $d(B)=2/m$. The residue sumset is
\[
\{0,1\}+\{0,2\}=\{0+0,0+2,1+0,1+2\}=\{0,1,2,3\}\subseteq\mathbb{Z}/m\mathbb{Z}.
\]
For $m\ge 4$ these are four distinct residues, hence $|A_0+B_0|=4$ and Lemma~335.1 gives
$d(A+B)=4/m=d(A)+d(B)$.
\hfill$\square$

\medskip
\noindent\textbf{FAST REALITY CHECK (small periodic example).}
Take $m=5$. Then $d(A)=d(B)=2/5$ and $A+B$ occupies residues $0,1,2,3$ modulo $5$, hence $d(A+B)=4/5=d(A)+d(B)$.

\medskip
\noindent\textbf{5) VERIFICATION.}
Lemma~335.1 is a direct density computation for periodic sets. Lemma~335.2 is an explicit construction checked by residue arithmetic.

\medskip
\noindent\textbf{6) FINAL.}

\noindent\textbf{UNRESOLVED}

\smallskip
\noindent (i) \textbf{Strongest fully proved partial result obtained here.}
We proved a complete reduction for periodic (mod $m$) sets: $d(A+B)=d(A)+d(B)$ is equivalent to $|A_0+B_0|=|A_0|+|B_0|$ in $\mathbb{Z}/m\mathbb{Z}$ (Lemma~335.1), and we exhibited an explicit infinite family of nontrivial examples satisfying the equality (Lemma~335.2).

\smallskip
\noindent (ii) \textbf{Exact first gap.}
For general positive-density sets $A,B$ (not assumed periodic or structured), we do not have a characterization of when $d(A+B)=d(A)+d(B)$ holds.

\smallskip
\noindent (iii) \textbf{Top 3 next moves (concrete targets).}
\begin{enumerate}
\item Prove a structure theorem: show that if $d(A+B)=d(A)+d(B)<1$, then $A$ and $B$ must be ``almost periodic'' (e.g. close in density to unions of residue classes).
\item Analyze the equality case in finite cyclic groups: classify pairs $(A_0,B_0)\subseteq\mathbb{Z}/m\mathbb{Z}$ with $|A_0+B_0|=|A_0|+|B_0|$, then lift to $\mathbb{N}$ via Lemma~335.1.
\item Search computationally over small moduli $m$ for residue-class solutions and look for a pattern that could plausibly extend to a full classification.
\end{enumerate}

\smallskip
\noindent (iv) \textbf{Minimal counterexample structure.}
A counterexample to any proposed ``periodic-only'' characterization would be a pair of sets $A,B$ of positive density with $d(A+B)=d(A)+d(B)$ but whose characteristic functions are not approximable (in density) by periodic sets. Such a pair would have to avoid the usual heuristic that sumsets create substantial overlap and hence increase density beyond the sum.


