\section*{Erd\H{o}s Problem \#146 (Erd\H{o}s--Simonovits conjecture for $r$-degenerate bipartite graphs)}

\subsection*{1) FORMAL RESTATEMENT}
For a fixed graph $H$, let $\mathrm{ex}(n,H)$ denote the Tur\'an (extremal) number: the maximum number of edges in an $n$-vertex simple graph containing no copy of $H$ as a (not necessarily induced) subgraph.

A graph $H$ is \emph{$r$-degenerate} if every (induced) subgraph of $H$ has a vertex of degree at most $r$; equivalently, there is an ordering of $V(H)$ in which each vertex has at most $r$ neighbors later in the order.

\textbf{Conjecture (Erd\H{o}s--Simonovits / Erd\H{o}s).}
If $H$ is bipartite and $r$-degenerate, then
\[
\mathrm{ex}(n,H)=O_H\bigl(n^{2-1/r}\bigr).
\]

\subsection*{2) QUICK LITERATURE / CONTEXT CHECK}
\begin{itemize}
\item A classic special case: if all degrees in one color class of $H$ are at most $r$, then $\mathrm{ex}(n,H)=O(n^{2-1/r})$ (F\"uredi; also Alon--Krivelevich--Sudakov (2003) via dependent random choice).
\item For general $r$-degenerate bipartite $H$, Alon--Krivelevich--Sudakov (2003) proved a weaker universal bound $\mathrm{ex}(n,H)=O\bigl(n^{2-1/(4r)}\bigr)$.
\item The conjectured exponent $2-1/r$ is known for large families of $r$-degenerate bipartite graphs, including all $r$-degenerate blow-ups of trees (Grzesik--Janzer--Nagy 2019/2022).
\item For the $2$-degenerate grid graphs $F_t$ (the $t\times t$ planar grid), Brada\v{c}--Janzer--Sudakov--Tomon proved $\mathrm{ex}(n,F_t)=\Theta(n^{3/2})$, matching the conjectured exponent for $r=2$.
\end{itemize}

\subsection*{3) ATTACK PLAN}
\textbf{Proof track.}
\begin{enumerate}
\item Settle the easiest case $r=1$ completely (forests), to validate the conjecture at the base level.
\item Recall the dependent random choice (DRC) template: show that if $G$ has many edges, then $G$ contains a bipartite pair $(U_1,U_2)$ where every $r$-tuple in $U_i$ has many common neighbors in $U_{3-i}$; then embed $H$ using a degeneracy ordering. This yields the known $2-1/(4r)$ exponent.
\item Identify what extra structure (supersaturation, random walks on auxiliary graphs, tensor power) is needed to upgrade the exponent from $2-1/(4r)$ to $2-1/r$.
\end{enumerate}

\textbf{Disproof track.}
\begin{enumerate}
\item Search for candidate counterexamples among sparse bipartite graphs with bounded degeneracy but unusually large extremal exponent (e.g. graphs with ``high complexity'' in the sense used in the degenerate Tur\'an literature).
\item Any disproof would require an explicit bipartite $r$-degenerate $H$ and a construction of $H$-free graphs with $\gg n^{2-1/r+\delta}$ edges.
\end{enumerate}

\subsection*{4) WORK}
\paragraph{Phase 1 sanity checks (tiny cases).}
\begin{itemize}
\item $r=1$-degenerate graphs are forests. For a fixed tree $T$, it is classical and elementary that $\mathrm{ex}(n,T)=O(n)$.
\item For $r=2$, the conjectured bound is $\mathrm{ex}(n,H)=O(n^{3/2})$ for every $2$-degenerate bipartite $H$. This is consistent with sharp results for $C_4$ and $K_{2,t}$, and with the proven case of grid graphs $F_t$.
\end{itemize}

\paragraph{A complete proof for $r=1$.}

\begin{theorem}[Conjecture holds for $r=1$]
\label{thm:146-r1}
Let $H$ be a fixed forest on $h\ge 2$ vertices. Then
\[
\mathrm{ex}(n,H)\le (h-2)n
\qquad\text{for all }n\ge 1.
\]
In particular, for bipartite $1$-degenerate $H$ one has $\mathrm{ex}(n,H)=O_H(n)=O_H(n^{2-1}).
\end{theorem}
\begin{proof}
Let $G$ be an $n$-vertex graph with $e(G)>(h-2)n$.
We claim $G$ contains $H$.

First, repeatedly delete vertices of degree at most $h-2$ (and all incident edges) until this is no longer possible.
If the process deletes all vertices, then the total number of edges removed is at most $(h-2)n$, contradicting $e(G)>(h-2)n$.
Hence some nonempty subgraph $G'$ remains with minimum degree
\[
\delta(G')\ge h-1.
\]

It suffices to show that $G'$ contains every tree on $h$ vertices (since $H$ is a subgraph of some tree on $h$ vertices, obtained by adding edges between components).
So let $T$ be any tree on $h$ vertices.
We prove by induction on $h$ that every graph with minimum degree at least $h-1$ contains $T$.

Base $h=2$ is trivial.
For $h\ge 3$, let $v$ be a leaf of $T$ and $u$ its neighbor.
Then $T-v$ is a tree on $h-1$ vertices.
Since $\delta(G')\ge h-1\ge (h-1)-1$, by the induction hypothesis $G'$ contains a copy of $T-v$.
Let $u'$ be the vertex in this copy corresponding to $u$.
This copy uses exactly $h-1$ vertices of $G'$.
Because $\deg_{G'}(u')\ge h-1$, the vertex $u'$ has at least one neighbor outside the used $(h-1)$-set.
Embed $v$ to such an unused neighbor of $u'$, extending the embedding to a copy of $T$.

Thus $G'$ contains $T$, hence contains $H$, contradicting $H$-freeness.
Therefore any $H$-free $n$-vertex graph has at most $(h-2)n$ edges.
\end{proof}

\paragraph{Beyond $r=1$.}
For $r\ge 2$, the conjectured exponent $2-1/r$ is open in general. Known methods (e.g. dependent random choice) yield weaker exponents such as $2-1/(4r)$, and additional structure assumptions on $H$ (e.g. being a blow-up of a tree) allow proving the conjectured exponent.

\subsection*{5) VERIFICATION}
\begin{itemize}
\item Theorem~\ref{thm:146-r1} exactly matches the conjectured exponent for $r=1$.
\item The degeneracy notion used (every induced subgraph has a vertex of degree $\le r$) is equivalent to the standard definition of degeneracy.
\item No hidden assumptions: the proof uses only basic graph theory (minimum degree reduction and greedy tree embedding).
\end{itemize}

\subsection*{6) FINAL}
\textbf{UNRESOLVED.} The full conjecture for all $r\ge 2$ remains open, but the base case $r=1$ is proved.

\begin{enumerate}
\item[(i)] \textbf{Strongest fully proved partial result obtained:}
\begin{itemize}
\item Conjecture holds for $r=1$ (forests): $\mathrm{ex}(n,H)\le (|V(H)|-2)n$ (Theorem~\ref{thm:146-r1}).
\item (From the literature) general bound $\mathrm{ex}(n,H)=O(n^{2-1/(4r)})$ for $r$-degenerate bipartite $H$ (Alon--Krivelevich--Sudakov 2003), and sharp $\mathrm{ex}(n,H)=O(n^{2-1/r})$ for large families (e.g. $r$-degenerate blow-ups of trees).
\end{itemize}
\item[(ii)] \textbf{Exact first gap:} I do not have a proof (or counterexample) for the conjectured exponent $2-1/r$ for general bipartite $r$-degenerate graphs when $r\ge 2$.
\item[(iii)] \textbf{Top 3 next moves:}
\begin{enumerate}
\item Strengthen DRC-type lemmas to produce larger structured bipartite subgraphs that support embedding of any $r$-degenerate $H$ with the optimal exponent.
\item Develop supersaturation and counting of ``degenerate'' copies (as in the blow-up-of-trees results) for wider classes of $r$-degenerate graphs.
\item Investigate potential counterexamples via constructions from algebraic graph theory / random algebraic methods, targeting $H$ with high ``complexity'' but still bounded degeneracy.
\end{enumerate}
\item[(iv)] \textbf{What a minimal counterexample would likely look like:} a bipartite $r$-degenerate graph $H$ with bounded average degree but structured so that known embedding/counting arguments fail (e.g. forcing many ``degenerate'' partial embeddings), together with an explicit $H$-free construction achieving $\Omega(n^{2-1/r+\delta})$ edges.
\end{enumerate}

\subsection*{7) COMPLETION ESTIMATE}
\textbf{9/10.} The conjecture is central and has resisted existing general techniques; resolving it likely requires a new conceptual breakthrough or a definitive counterexample.

