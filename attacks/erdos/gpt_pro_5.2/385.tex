\section*{Problem \#385}

\subsection*{FORMAL RESTATEMENT}
For an integer $m\ge 2$, let $p(m)$ denote the \emph{least prime divisor} of $m$.
Define, for integers $n\ge 5$,
\[
F(n)\ :=\ \max_{\substack{m<n\\ m\ \text{composite}}}\ \bigl(m+p(m)\bigr).
\]
The problem asks:

\begin{enumerate}[label=(\alph*)]
\item Is there an $N$ such that for all $n\ge N$ one has $F(n)>n$?
\item Does $F(n)-n\to\infty$ as $n\to\infty$ (i.e.\ for every $M$ there is $N_M$ such that $n\ge N_M\Rightarrow F(n)-n\ge M$)?
\end{enumerate}

\subsection*{QUICK LITERATURE/CONTEXT CHECK}
This is Erd\H{o}s problem \#385 in the Erd\H{o}s Problems database.  It is presented as open there, with the remark that ``plausible conjectures on primes'' suggest $F(n)\le n$ only finitely often and that perhaps $F(n)\ge n+(1-o(1))\sqrt{n}$ might hold; also note the trivial upper bound $F(n)\le n+\sqrt{n}$.  Terence Tao has a detailed blog discussion relating the difficulty of ruling out $F(n)\le n$ to the parity barrier in sieve theory and to controlling gaps between certain almost primes (e.g.\ semiprimes with restricted factor ranges).  The database notes an equivalence to another formulation (problem \#[430] there) and a related question (problem \#[463] there).

\subsection*{ATTACK PLAN}
Write the inequality $m+p(m)>n$ as $p(m) > n-m$.  Thus to prove $F(n)>n$ it suffices to find, for each large $n$, a composite $m$ in some short interval $[n-h,n)$ with least prime factor exceeding $h$ (i.e.\ an $h$-rough composite).
Potential approaches suggested by the literature:
\begin{itemize}
\item Reduce to locating semiprimes (or more generally $y$-rough integers) in intervals of length comparable to $y$.
\item Use sieve-theoretic estimates for rough numbers in short intervals; the main obstruction is that existing methods struggle to guarantee \emph{at least one} survivor uniformly in $n$ at the relevant intermediate scales.
\item Try to build explicit families of ``bad'' $n$ with $F(n)=n$ (disproof of (a)), or conversely show that such families cannot persist beyond some scale (proof of (a)).
\end{itemize}

\subsection*{WORK}
We record several unconditional observations.

\medskip\noindent
\textbf{Lemma 1 (odd $n$ are trivially good).}
If $n\ge 5$ is odd, then $F(n)\ge n+1$ (hence $F(n)>n$).

\begin{proof}
If $n$ is odd and $n\ge 5$, then $m:=n-1$ is an even integer $\ge 4$, hence composite, and $p(m)=2$.  Therefore $m+p(m)= (n-1)+2=n+1$, and since $m<n$ we have $F(n)\ge n+1$.
\end{proof}

\medskip\noindent
\textbf{Lemma 2 (even $n$ satisfy $F(n)\ge n$).}
If $n\ge 6$ is even, then $F(n)\ge n$.

\begin{proof}
Let $m:=n-2$.  Then $m$ is even and $\ge 4$, hence composite, and $p(m)=2$.  Thus $m+p(m)=(n-2)+2=n$, so $F(n)\ge n$.
\end{proof}

\medskip\noindent
\textbf{Lemma 3 (even $n$ with $n-1$ composite are good).}
If $n\ge 6$ is even and $n-1$ is composite, then $F(n)>n$.

\begin{proof}
Take $m:=n-1$.  Then $m<n$ and $m$ is an odd composite, hence $p(m)\ge 3$.  Therefore
\[
m+p(m)\ \ge\ (n-1)+3\ =\ n+2\ >\ n,
\]
so $F(n)\ge m+p(m)>n$.
\end{proof}

\medskip\noindent
\textbf{Corollary 4 (structure of possible counterexamples to (a)).}
If $n\ge 6$ and $F(n)\le n$, then $n$ must be even, $n-1$ must be prime, and necessarily $F(n)=n$.

\begin{proof}
By Lemma 1, no odd $n\ge 5$ can satisfy $F(n)\le n$.  So $n$ is even.  By Lemma 2, $F(n)\ge n$, hence $F(n)\le n$ forces $F(n)=n$.  Finally, if $n-1$ were composite then Lemma 3 would give $F(n)>n$, a contradiction.  Thus $n-1$ is prime.
\end{proof}

\medskip\noindent
\textbf{Lemma 5 (easy limsup lower bound).}
Along the subsequence $n=q^2+1$ with $q$ prime one has $F(n)-n\ge q-1\to\infty$.  In particular,
\[
\limsup_{n\to\infty}\bigl(F(n)-n\bigr)=\infty.
\]

\begin{proof}
Fix a prime $q$ and set $n=q^2+1$.  Take $m:=q^2<n$, which is composite with $p(m)=q$.  Then
\[
F(n)\ \ge\ m+p(m)\ =\ q^2+q\ =\ (q^2+1)+(q-1)\ =\ n+(q-1),
\]
so $F(n)-n\ge q-1\to\infty$ along this subsequence.
\end{proof}

\medskip\noindent
\textbf{Lemma 6 (trivial upper bound).}
For all $n\ge 5$, $F(n)\le n+\sqrt{n}$.

\begin{proof}
If $m<n$ is composite then $p(m)\le \sqrt{m}<\sqrt{n}$.  Hence $m+p(m)<n+\sqrt{n}$, so taking the maximum over such $m$ gives $F(n)\le n+\sqrt{n}$.
\end{proof}

\medskip\noindent
\textbf{Computational sanity check (not a proof).}
A brute-force computation for $n\le 200$ shows that $F(n)=n$ for
\[
n\in\{6,8,12,14,18,20,24,30,32,42,44,48,60,62,72,74,84,90,102,104,108,110,114,132,140,168,182,198,200\},
\]
and $F(n)>n$ for all other $5\le n\le 200$.

\subsection*{VERIFICATION}
The above lemmas are elementary and were checked symbolically:
\begin{itemize}
\item Lemmas 1--3 are direct constructions of a single admissible composite $m<n$.
\item Corollary 4 is a direct logical combination of Lemmas 1--3.
\item Lemma 5 uses the explicit choice $m=q^2$; no hidden assumptions.
\item Lemma 6 uses the classical bound $p(m)\le \sqrt{m}$ for composite $m$.
\end{itemize}
None of these steps address the main difficulty: ruling out infinitely many even $n=p+1$ with $p$ prime and $F(n)=n$.

\subsection*{FINAL}
\textbf{UNRESOLVED.}

Most promising partial results obtained above:
\begin{itemize}
\item Complete reduction: any counterexample to eventual positivity must be even and of the form $n=p+1$ with $p$ prime (Corollary 4).
\item $\limsup (F(n)-n)=\infty$ (Lemma 5), and always $F(n)\le n+\sqrt{n}$ (Lemma 6).
\end{itemize}

Specific barrier:
\begin{itemize}
\item One must control existence of $h$-rough composite numbers in intervals $[n-h,n)$ at intermediate scales (roughly $\log n \ll h \ll \sqrt{n}$), which is tightly connected to the parity problem in sieve theory.
\end{itemize}

Smallest missing step to a full resolution:
\begin{itemize}
\item Prove (or disprove) that for all sufficiently large primes $p$ there exists a composite $m<p+1$ with $m+p(m)>p+1$; equivalently, show that for each large prime $p$ one can find an $h$ and a composite $m\in[p+1-h,p]$ whose least prime factor exceeds $h$.
\end{itemize}

\subsection*{COMPLETION ESTIMATE}
COMPLETION: 35\%

%%%%%%%%%%%%%%%%%%%%%%%%%%%%%%%%%%%%%%%%%%%%%%%%%%%%%%%%%%%%%%%%%%%%%%%%%%%%%%%
