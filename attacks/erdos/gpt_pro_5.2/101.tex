\section*{Erd\H{o}s Problem \#101}
\addcontentsline{toc}{section}{Erd\H{o}s Problem \#101}

\subsection*{FORMAL RESTATEMENT}
\noindent\textbf{Verbatim problem statement (from file).}
\emph{``Given $n$ points in $\R^2$, no five of which are on a line, the number of lines containing four points is $o(n^2)$.''}

\medskip
\noindent\textbf{Definitions and conventions.}
Let $P\subset\R^2$ with $|P|=n$.
Assume:
\[\text{(No-5-collinear)}\qquad \text{every line in }\R^2\text{ contains at most 4 points of }P.
\]
Let $L_4(P)$ be the number of distinct (geometric) lines $\ell$ such that $|\ell\cap P|=4$.
(The ``no five collinear'' hypothesis implies ``containing four points'' is the same as ``containing \emph{at least} four points''.)

\medskip
\noindent\textbf{Claim to prove (asymptotic).}
\[\frac{L_4(P)}{n^2}\to 0\quad\text{as }n\to\infty,\]
uniformly over all such $P$.
Equivalently: for every $\varepsilon>0$ there exists $N(\varepsilon)$ such that if $n\ge N(\varepsilon)$ then $L_4(P)\le \varepsilon n^2$.

\subsection*{QUICK LITERATURE/CONTEXT CHECK}
No external browsing was used.
The file states that there are constructions with $L_4(P)\ge n^{2-O(1/\sqrt{\log n})}$, so any $o(n^2)$ upper bound would have to be very close to sharp.
I do not attempt to reproduce those constructions.

\subsection*{ATTACK PLAN}
\begin{itemize}[leftmargin=2em]
\item \textbf{Upper bound track:} Use double counting (pairs, incidences) and possibly the Szemer\'edi--Trotter theorem to bound the number of 4-rich lines.
\item \textbf{Reduction track:} Relate 4-point lines to counts of collinear triples or to rich-line incidence graphs.
\item \textbf{Disproof track:} Try to construct a configuration with $\Theta(n^2)$ many 4-point lines while keeping no five collinear; this would disprove the conjectured $o(n^2)$ bound.
\end{itemize}

\subsection*{WORK}
\noindent\textbf{FAST REALITY CHECK (very small $n$).}
\begin{itemize}[leftmargin=2em]
\item $n\le 3$: $L_4(P)=0$.
\item $n=4$: if all 4 points are collinear then $L_4(P)=1$; if not, $L_4(P)=0$.
\item $n=5$: one can have $L_4(P)=1$ (four collinear, fifth off the line).
\item $n=7$: one can realize $L_4(P)=2$ by taking two 4-point lines meeting in one point (uses $1+3+3=7$ points).
\end{itemize}

\begin{lemma}[Pair-counting upper bound]
If $P\subset\R^2$ has $n$ points with no five collinear, then
\[L_4(P)\le \frac{\binom{n}{2}}{\binom{4}{2}}=\frac{n(n-1)}{12}.
\]
\end{lemma}

\begin{proof}
Each line $\ell$ with $|\ell\cap P|=4$ contains exactly $\binom{4}{2}=6$ unordered pairs of points from $P$.
Distinct lines contribute disjoint sets of pairs because a pair of distinct points determines a unique line.
Therefore
\[6\,L_4(P)\le \binom{n}{2},
\]
which rearranges to the stated bound.
\end{proof}

\begin{proposition}[4-point lines through a fixed point]
Let $P\subset\R^2$ be a set of $n$ points with no five collinear.
Fix a point $p\in P$ and let $r(p)$ denote the number of lines $\ell$ such that $p\in\ell$ and $|\ell\cap P|=4$.
Then
\[r(p)\le \frac{n-1}{3}.\]
Consequently,
\[L_4(P)=\frac14\sum_{p\in P} r(p)\le \frac14\cdot n\cdot \frac{n-1}{3}=\frac{n(n-1)}{12},\]
recovering the pair-counting bound from a different viewpoint.
\end{proposition}

\begin{proof}
Each such 4-point line through $p$ contains exactly three further points of $P$ besides $p$.
Moreover, two distinct lines through $p$ cannot share any other point of $P$ (if they shared $q\ne p$ then both lines would be the unique line determined by $p$ and $q$).
Hence the $r(p)$ distinct 4-point lines through $p$ collectively contain at least $3r(p)$ distinct points of $P\setminus\{p\}$.
Since $|P\setminus\{p\}|=n-1$, we have $3r(p)\le n-1$, i.e. $r(p)\le (n-1)/3$.

For the second claim, observe that $\sum_{p\in P} r(p)$ counts incidences between points and 4-point lines, and each 4-point line has exactly $4$ incidences.
Thus $\sum_{p\in P} r(p)=4L_4(P)$.
Substituting the bound $r(p)\le (n-1)/3$ gives the displayed inequality for $L_4(P)$.
\end{proof}

\subsection*{VERIFICATION}
\begin{itemize}[leftmargin=2em]
\item The pair-counting lemma is a direct double count of unordered point pairs on 4-point lines; it uses only that distinct point pairs determine unique lines.
\item The proposition bounds the number of 4-point lines through a fixed point by noting that distinct lines through that point use disjoint sets of the remaining points; no external theorems are invoked.
\item Both bounds are $\Theta(n^2)$, so they do not approach the conjectured $o(n^2)$; they are consistent with the near-quadratic constructions mentioned in the file.
\end{itemize}

\subsection*{FINAL}
\textbf{UNRESOLVED.}
\begin{enumerate}[label=(\roman*),leftmargin=2.5em]
\item \textbf{Strongest proved partial result here:}
For any such $P$, $L_4(P)\le n(n-1)/12$ by pair counting.
\item \textbf{First gap (crisp):}
Prove that for every $\varepsilon>0$ and all sufficiently large $n$, every $n$-point set in $\R^2$ with no five collinear has $L_4(P)\le \varepsilon n^2$.
\item \textbf{Top 3 next moves (concrete):}
\begin{enumerate}[label=(\alph*),leftmargin=2.5em]
\item Try to strengthen Szemer\'edi--Trotter-type bounds using the additional constraint ``no five collinear'' to get a saving over the trivial $\Theta(n^2)$.
\item Convert 4-point lines into a 4-uniform hypergraph on $P$ (edges = 4-tuples on a line) and attempt to exploit geometric restrictions to bound the number of edges.
\item Computational experiment: for moderate $n$ (say $n\le 50$), search over known high-incidence constructions (grids, projected grids, algebraic curves) and measure $L_4(P)/n^2$ to see decay rates.
\end{enumerate}
\item \textbf{What a minimal counterexample would likely look like:}
A family $P_n$ with no five collinear and with $L_4(P_n)\ge c n^2$ for some fixed $c>0$ (i.e. a positive density of all $\binom{n}{2}$ pairs lie in 4-point lines). Such a configuration would resemble a near-Steiner system of 4-point lines geometrically realizable in $\R^2$.
\end{enumerate}

