### 1) ROUND-2 OBJECTIVE

**(C) Obstruction/correction path.**
Given the Round‑1 reduction (“divergence ⇒ infinitely often (|A\cap[1,N]|) beats any *summably small* density profile; APs follow if one can beat (r_k(N))”), the most promising strict advance is to **close the gap for the only length currently reachable unconditionally**: (k=3). Concretely, I will:

1. **Strengthen Round‑1 Lemma 2** to a general “summability criterion” (still elementary, dyadic).
2. Combine it with the **Bloom–Sisask bound** (r_3(N)\ll N/(\log N)^{1+c}) to give a **complete, gap‑free proof of the (k=3) case** of the Erdős conjecture (indeed, arbitrarily large 3‑APs).
3. Precisely isolate why the same method does **not** currently extend to (k\ge 4) with the best known (r_k(N)) bounds.

This is a genuine advance beyond Round‑1: it turns the Round‑1 framework into a **full theorem** for (k=3), and it refines the “gap” to an explicit summability condition on (r_k).

---

### 2) ROUND-1 FOUNDATION USED

I will rely on the following Round‑1 results **as black boxes**:

* **Lemma 1 (dyadic decomposition upper bound):**
  With (B_j:=A\cap(2^j,2^{j+1}]),
  [
  \sum_{n\in A\cap[2,\infty)}\frac1n\ \le\ \sum_{j\ge 0}\frac{|B_j|}{2^j}.
  ]
* **Lemma 3 (definition link):** If (|A\cap[1,N]|>r_k(N)) then (A) contains a nontrivial (k)-term AP in ([1,N]).

I will *not* restate their proofs.

---

### 3) NEW INSIGHT / TOOL (ROUND-2)

**New tool:** a **general dyadic “summability criterion”** that strictly strengthens Round‑1 Lemma 2.

Instead of hard-coding ((\log N)(\log\log N)^2), we prove:

* If (|A\cap[1,N]|) is eventually bounded by (N) times any profile whose dyadic reciprocals form a **convergent series**, then (\sum_{n\in A}1/n) converges.

This lets us *match* the known best upper bound on (r_3(N)) (Bloom–Sisask) and deduce the (k=3) Erdős conjecture as a corollary.

---

### 4) ATTACK PLAN (ROUND-2)

**Round‑1 gap:** need an upper bound on (r_k(N)) strong enough that the divergence‑forced density spikes (from Lemma 2) exceed (r_k(N)) infinitely often.

**What we prove now:**

1. A strengthened lemma: if for some (\varepsilon>0),
   [
   |A\cap[1,N]|\ \le\ C\frac{N}{(\log N)^{1+\varepsilon}}
   \quad\text{for all large }N,
   ]
   then (\sum_{n\in A}1/n<\infty).
   (This is a direct dyadic summability argument.)

2. Use Bloom–Sisask:
   [
   r_3(N)\ \ll\ \frac{N}{(\log N)^{1+c}}
   ]
   for some absolute (c>0). ([arXiv][1])
   Hence any 3‑AP‑free (A) satisfies the above cardinality bound, forcing (\sum_{n\in A}1/n<\infty). Contraposition yields the (k=3) Erdős statement.

3. Finally, show **arbitrarily large** 3‑APs exist by applying the same argument **to dyadic blocks**.

**Why this overcomes Round‑1 obstacles:** for (k=3) we now *do* have a decay exponent (>1) in (\log N), which makes the relevant dyadic series summable. For (k\ge 4), current bounds do not yield such summability, and we will make that explicit.

---

### 5) WORK (ROUND-2)

#### 5.1 A strengthened dyadic summability lemma

Let (A\subseteq\mathbb N). Define dyadic blocks (B_j:=A\cap(2^j,2^{j+1}]) as in Round‑1.

**Lemma 4 (general dyadic summability criterion).**
Let ((u_j)*{j\ge 0}) be a sequence of nonnegative reals with (\sum*{j\ge 0}u_j<\infty).
If there exists (J_0) such that for all (j\ge J_0),
[
\frac{|B_j|}{2^j}\ \le\ u_j,
]
then (\sum_{n\in A}\frac1n<\infty).

**Proof.** By Round‑1 Lemma 1,
[
\sum_{n\in A\cap[2,\infty)}\frac1n\ \le\ \sum_{j\ge 0}\frac{|B_j|}{2^j}
= \sum_{j< J_0}\frac{|B_j|}{2^j} + \sum_{j\ge J_0}\frac{|B_j|}{2^j}.
]
The first sum is finite (finitely many terms). The second is (\le \sum_{j\ge J_0}u_j<\infty). Add the possible (n=1) term. ∎

This lemma is strictly more flexible than Round‑1 Lemma 2; Lemma 2 is the special case (u_j\asymp 1/(j(\log j)^2)).

---

#### 5.2 A concrete corollary matching ((\log N)^{1+\varepsilon})

**Corollary 5.**
Fix (\varepsilon>0). Suppose there exist constants (C>0) and (N_0\ge 3) such that for all (N\ge N_0),
[
|A\cap[1,N]|\ \le\ C\frac{N}{(\log N)^{1+\varepsilon}}.
]
Then (\sum_{n\in A}\frac1n<\infty).

Equivalently (contrapositive): if (\sum_{n\in A}1/n=\infty), then **for every** (C>0) and (N_0) there exists some (N\ge N_0) with
[
|A\cap[1,N]|\ >\ C\frac{N}{(\log N)^{1+\varepsilon}}.
]

**Proof.** Choose (J_0) so that (2^{J_0}\ge N_0). For (j\ge J_0),
[
|B_j|\ \le\ |A\cap[1,2^{j+1}]|
\ \le\ C\frac{2^{j+1}}{(\log 2^{j+1})^{1+\varepsilon}}
= C\frac{2^{j+1}}{((j+1)\log 2)^{1+\varepsilon}}.
]
Hence
[
\frac{|B_j|}{2^j}\ \le\ \frac{2C}{(\log 2)^{1+\varepsilon}}\cdot \frac{1}{(j+1)^{1+\varepsilon}}.
]
Let (u_j := \dfrac{2C}{(\log 2)^{1+\varepsilon}}(j+1)^{-(1+\varepsilon)}). Then (\sum_j u_j<\infty) because (1+\varepsilon>1). Apply Lemma 4. ∎

---

#### 5.3 External input: Bloom–Sisask bound on (r_3(N))

**Theorem (Bloom–Sisask).** There exists an absolute constant (c_0>0) such that if (S\subseteq{1,\dots,N}) contains no nontrivial 3‑term arithmetic progression, then
[
|S|\ \ll\ \frac{N}{(\log N)^{1+c_0}}.
]
Equivalently, (r_3(N)\ll N/(\log N)^{1+c_0}). ([arXiv][1])

I will use only the stated bound, not the proof.

---

#### 5.4 Full deduction of the (k=3) Erdős conjecture

**Theorem 6 (Erdős conjecture for (k=3), with arbitrarily large 3‑APs).**
If (A\subseteq\mathbb N) satisfies (\sum_{n\in A}\frac1n=\infty), then (A) contains a nontrivial 3‑term arithmetic progression. In fact, for every (J) there exists a nontrivial 3‑term AP all of whose terms exceed (2^J); hence (A) contains infinitely many 3‑APs with unbounded maximum element.

**Proof (existence of some 3‑AP).**
Assume for contradiction that (A) contains **no** nontrivial 3‑term arithmetic progression. Then for every (N),
[
|A\cap[1,N]|\ \le\ r_3(N).
]
By Bloom–Sisask, there exist constants (C_0>0) and (N_1) such that for all (N\ge N_1),
[
|A\cap[1,N]|\ \le\ r_3(N)\ \le\ C_0\frac{N}{(\log N)^{1+c_0}}.
]
Apply Corollary 5 with (\varepsilon=c_0): this implies (\sum_{n\in A}\frac1n<\infty), contradicting the hypothesis. Therefore (A) must contain a nontrivial 3‑term AP. ∎

**Proof (arbitrarily large 3‑APs).**
Fix (J\ge 0). Suppose for contradiction that (A) contains **no** 3‑term AP with all three terms (>2^J). Then for every (j\ge J), the block
[
B_j = A\cap(2^j,2^{j+1}]
]
is 3‑AP‑free (any 3‑AP inside (B_j) would have all terms (>2^j\ge 2^J)). Translation preserves 3‑APs, so (B_j) is 3‑AP‑free as a subset of an interval of length (2^j); hence
[
|B_j|\ \le\ r_3(2^j).
]
Again by Bloom–Sisask,
[
\frac{|B_j|}{2^j}\ \le\ \frac{r_3(2^j)}{2^j}
\ \ll\ \frac{1}{(\log 2^j)^{1+c_0}}
\asymp \frac{1}{j^{1+c_0}}.
]
Since (\sum_{j\ge J} j^{-(1+c_0)}<\infty), Lemma 4 (with (u_j\asymp j^{-(1+c_0)})) implies the tail (\sum_{n\in A\cap(2^J,\infty)}\frac1n) converges, and adding finitely many initial terms gives (\sum_{n\in A}\frac1n<\infty), contradiction.

Thus, for every (J), there exists some 3‑AP in (A) with all terms (>2^J). In particular, there are infinitely many distinct 3‑APs and their maxima are unbounded. ∎

This completes the (k=3) case in a fully rigorous way.

---

#### 5.5 A clean criterion reducing the (k)-case to a summability condition on (r_k)

This is a structural clarification not present in Round‑1.

**Proposition 7 (dyadic summability criterion via (r_k)).**
Fix (k\ge 3). If
[
\sum_{j=1}^\infty \frac{r_k(2^j)}{2^j}\ <\ \infty,
]
then every (A\subseteq\mathbb N) with (\sum_{n\in A}\frac1n=\infty) must contain a nontrivial (k)-term arithmetic progression (indeed with arbitrarily large terms).

**Proof.** Suppose (A) is (k)-AP‑free. Then each dyadic block (B_j) is (k)-AP‑free, hence (|B_j|\le r_k(2^j)). Round‑1 Lemma 1 gives
[
\sum_{n\in A\cap[2,\infty)}\frac1n\ \le\ \sum_{j\ge 0}\frac{|B_j|}{2^j}
\ \le\ \sum_{j\ge 0}\frac{r_k(2^j)}{2^j}\ <\ \infty,
]
contradicting divergence. The “arbitrarily large” refinement is proved exactly as in Theorem 6 by truncating below (2^J). ∎

**Remark.** A sufficient (not necessary) condition for (\sum r_k(2^j)/2^j<\infty) is an upper bound
[
r_k(N)\ \ll\ \frac{N}{(\log N)^{1+\varepsilon}}
\quad\text{for some }\varepsilon>0,
]
since then (r_k(2^j)/2^j \ll j^{-(1+\varepsilon)}), summable.

For (k=3) this holds by Bloom–Sisask, so Proposition 7 recovers Theorem 6. ([arXiv][1])

---

#### 5.6 Why the Round‑1 strategy still does not reach (k\ge 4)

The Erdős conjecture for all (k) remains listed as open in current problem databases. ([erdosproblems.com][2])
The precise reason, within the dyadic framework, is:

* For (k=4), the best-known bounds are of the form (r_4(N)\ll N/(\log N)^c) for some (small) (c>0). ([erdosproblems.com][2])
  Then (r_4(2^j)/2^j \ll 1/j^c), and (\sum 1/j^c) **diverges** whenever (c\le 1), so Proposition 7 cannot be triggered.

* For general (k\ge 5), the current best bounds (Leng–Sah–Sawhney) are
  [
  r_k(N)\ \ll\ \frac{N}{\exp!\big((\log\log N)^{c_k}\big)}
  \quad (c_k>0).
  ]
  ([erdosproblems.com][2])
  Setting (N=2^j) gives (r_k(2^j)/2^j\ll \exp!\big(-(\log j)^{c_k}\big)), and the series (\sum_j \exp(-(\log j)^{c_k})) still **diverges** (since ((\log j)^{c_k}\ll \log j), so the terms are (\gg 1/j) eventually). Therefore Proposition 7 still cannot be triggered for (k\ge 5) using currently available (r_k) bounds.

So the Round‑1 “compare to (r_k(N))” mechanism is now **fully closed for (k=3)** and **cleanly pinpointed** as failing for (k\ge 4) because known bounds do not make (\sum r_k(2^j)/2^j) summable.

---

### 6) ADVERSARIAL VERIFICATION

I now try to break the new claims.

1. **Did I illegitimately infer divergence of (\sum |B_j|/2^j) from divergence of (\sum_{n\in A}1/n)?**
   Round‑1 Lemma 1 gives
   (\sum_{n\in A\cap[2,\infty)}1/n \le \sum_j |B_j|/2^j).
   If the left is (+\infty), the right cannot be finite. So yes, the right diverges. No issue.

2. **Translation step for dyadic blocks and (r_3(2^j)):**
   I used: if (B_j\subset(2^j,2^{j+1}]) is 3‑AP‑free, then (|B_j|\le r_3(2^j)).
   Justification: translate by (-2^j): (T_j:={n-2^j:n\in B_j}\subset{1,\dots,2^j}). A 3‑AP in (B_j) corresponds bijectively to a 3‑AP in (T_j). Hence (T_j) is 3‑AP‑free, so (|T_j|\le r_3(2^j)), and (|T_j|=|B_j|). Solid.

3. **Quantifier strength for “arbitrarily large 3‑APs”:**
   I proved: for each (J), some block (B_j) with (j\ge J) contains a 3‑AP. That gives a 3‑AP with all terms (>2^j\ge 2^J). This indeed gives unbounded maxima and infinitely many distinct progressions. No hidden reuse of the same small 3‑AP.

4. **Any hidden dependence on monotonicity of logs / constants?**
   Corollary 5 uses (\log 2^{j+1}=(j+1)\log 2), valid for natural logs. The constants ((\log 2)^{-(1+\varepsilon)}) are harmless.

5. **Potential mismatch of constants in Bloom–Sisask (\ll):**
   In Theorem 6, I used only existence of some (C_0,N_1) such that (r_3(N)\le C_0 N/(\log N)^{1+c_0}) for all (N\ge N_1). The arXiv abstract states (|A|\ll N/(\log N)^{1+c}), which exactly provides such constants. ([arXiv][1])
   No need for effective values.

All checks pass.

---

### 7) FINAL (EXACTLY ONE)

**UNRESOLVED (BUT STRICTLY ADVANCED).**

* The full Erdős conjecture (“divergent (\sum 1/n) ⇒ (k)-APs for *every* (k)”) remains open. ([erdosproblems.com][2])
* **Strict advance beyond Round‑1:** I gave a **complete, gap‑free proof of the (k=3) case**, including the stronger conclusion of **arbitrarily large 3‑term APs**, by:

  * strengthening Round‑1 Lemma 2 to Corollary 5 / Lemma 4, and
  * plugging in Bloom–Sisask’s bound on (r_3(N)). ([arXiv][1])
* I also isolated a sharp sufficient condition for general (k): the summability of (\sum r_k(2^j)/2^j) (Proposition 7), and explained why known bounds for (k\ge 4) do not (yet) meet it. ([erdosproblems.com][2])

---

### 8) COMPLETION ESTIMATE (MANDATORY)

**COMPLETION: 45%**

---

### 9) REFERENCES

* T. F. Bloom, O. Sisask, *Breaking the logarithmic barrier in Roth’s theorem on arithmetic progressions*, arXiv:2007.03528 (v2, 2021). ([arXiv][1])
* Erdős Problems (Problem #3), status and summary of best known bounds on (r_k(N)), last edited 28 Dec 2025. ([erdosproblems.com][2])
* J. Leng, A. Sah, M. Sawhney, *Improved Bounds for Szemerédi’s Theorem*, arXiv:2402.17995 (2024). ([arXiv][3])

[1]: https://arxiv.org/abs/2007.03528 "https://arxiv.org/abs/2007.03528"
[2]: https://www.erdosproblems.com/3 "https://www.erdosproblems.com/3"
[3]: https://arxiv.org/pdf/2402.17995 "https://arxiv.org/pdf/2402.17995"

\section{Round 3: Log-weighted resolution and minimal correction}

\subsection{1) ROUND-3 OBJECTIVE}

\textbf{Path (C): obstruction/correction.}
Round~2 established a complete proof of the Erd\H{o}s conjecture for $k=3$ by combining a dyadic summability criterion with the Bloom--Sisask bound on $r_3(N)$. The original statement for general $k$ remains open. In this round I prove a \emph{minimal corrected statement} that is currently unconditional for \emph{all} $k\ge 3$:
\begin{quote}
If $A\subseteq\mathbb{N}$ satisfies $\sum_{n\in A,\ n\ge 3}\frac{1}{n\log n}=\infty$, then $A$ contains arithmetic progressions of every length. Moreover, for every $k\ge 3$ and every $J\ge 0$ there exists a $k$--term progression in $A$ whose terms are all $>2^J$.
\end{quote}

This strengthens the Round~2 $k=3$ theorem in a different direction: it solves a log-weighted variant of the full conjecture using the current best bounds on $r_k(N)$ for all $k$.

\subsection{2) ROUND-1/2 FOUNDATION USED}

I will use the following Round~1/2 results as black boxes.
\begin{itemize}
\item (Round~1 Lemma~1) Dyadic decomposition upper bound: with $B_j:=A\cap(2^j,2^{j+1}]$,
\[\sum_{n\in A\cap[2,\infty)}\frac1n\ \le\ \sum_{j\ge 0}\frac{|B_j|}{2^j}.\]
\item (Round~1 Lemma~3) Definition link: if $|A\cap[1,N]|>r_k(N)$ then $A$ contains a nontrivial $k$--term AP in $[1,N]$.
\item (Round~2 Theorem~6) The unweighted $k=3$ case: if $\sum_{n\in A}1/n=\infty$ then $A$ contains infinitely many $3$--term APs with unbounded maximum element.
\item (Round~2 Proposition~7) Dyadic criterion via $r_k$: if $\sum_{j\ge 1} r_k(2^j)/2^j<\infty$ then any $A$ with $\sum_{n\in A}1/n=\infty$ contains a $k$--term AP (indeed arbitrarily large ones).
\end{itemize}

\subsection{3) NEW INSIGHT / TOOL (ROUND-3)}

\textbf{New tool: weighted dyadic summability.}
Round~2 used dyadic decomposition to bound the harmonic weight $\sum_{n\in A}1/n$ by $\sum_j |B_j|/2^j$. Here I generalize this to weights of the form $1/(n\,\varphi(n))$ for monotone $\varphi$, and specialize to the threshold weight $\varphi(n)=\log n$.

The key point is that, with the \emph{current} best bounds on $r_k(N)$ for $k\ge 4$, the series
\[\sum_{j\ge 2}\frac{r_k(2^j)}{2^j\,j}\]
converges, yielding a full theorem for the log-weighted conjecture.

\subsection{4) ATTACK PLAN (ROUND-3)}

\begin{enumerate}
\item Prove a weighted dyadic inequality: for nondecreasing $\varphi$,
\[\sum_{n\in A\cap[2,\infty)}\frac{1}{n\varphi(n)}\ \le\ \sum_{j\ge 0}\frac{|B_j|}{2^j\,\varphi(2^j)}.\]
\item Combine this with the translation bound $|B_j|\le r_k(2^j)$ for $k$--AP--free sets to obtain
\[\sum_{n\in A\cap[2,\infty)}\frac{1}{n\varphi(n)}\ \le\ \sum_{j\ge 0}\frac{r_k(2^j)}{2^j\,\varphi(2^j)}\qquad (A\text{ $k$--AP--free}).\]
\item Insert the best known upper bounds on $r_k(N)$:
\begin{itemize}
\item $k=3$: $r_3(N)\ll N/(\log N)^{1+c}$ (Bloom--Sisask).
\item $k=4$: $r_4(N)\ll N/(\log N)^{c}$ (Green--Tao).
\item $k\ge 5$: $r_k(N)\ll N\exp(-(\log\log N)^{c_k})$ (Leng--Sah--Sawhney).
\end{itemize}
Taking $\varphi(n)=\log n$ makes the dyadic series converge for each $k$. Contraposition gives the desired corrected statement.
\end{enumerate}

\subsection{5) WORK (ROUND-3)}

\subsubsection{5.1 Weighted dyadic decomposition}

Throughout, $\log$ denotes the natural logarithm. For $A\subseteq\mathbb{N}$ and $j\ge 0$ set $B_j:=A\cap(2^j,2^{j+1}]$.

\begin{lemma}[Weighted dyadic bound]
\label{lem:weighted-dyadic}
Let $\varphi:[2,\infty)\to(0,\infty)$ be nondecreasing. Then
\[
\sum_{n\in A\cap[2,\infty)}\frac{1}{n\varphi(n)}
\ \le\ \sum_{j\ge 0}\frac{|B_j|}{2^j\,\varphi(2^j)}.
\]
\end{lemma}

\begin{proof}
Fix $j\ge 0$ and $n\in(2^j,2^{j+1}]$. Then $n\ge 2^j$ and $\varphi(n)\ge \varphi(2^j)$, so
\[
\frac{1}{n\varphi(n)}\ \le\ \frac{1}{2^j\,\varphi(2^j)}.
\]
Summing over $n\in B_j$ gives
\[
\sum_{n\in B_j}\frac{1}{n\varphi(n)}\ \le\ \frac{|B_j|}{2^j\,\varphi(2^j)}.
\]
Finally sum over $j\ge 0$ and use that the blocks $B_j$ partition $A\cap[2,\infty)$.
\end{proof}

\subsubsection{5.2 From $k$--AP--freeness to dyadic upper bounds}

\begin{lemma}[Translation to $[2^j]$]
\label{lem:translation}
Fix $k\ge 3$. If $A$ contains no nontrivial $k$--term arithmetic progression, then for every $j\ge 0$,
\[
|B_j|\ \le\ r_k(2^j).
\]
\end{lemma}

\begin{proof}
Let $T_j:=\{n-2^j:\ n\in B_j\}$. Then $T_j\subseteq\{1,\dots,2^j\}$ and translation preserves $k$--term arithmetic progressions, so $T_j$ is $k$--AP--free. Hence $|T_j|\le r_k(2^j)$ by definition of $r_k$, and $|T_j|=|B_j|$.
\end{proof}

\begin{proposition}[Weighted $r_k$ criterion]
\label{prop:weighted-rk}
Fix $k\ge 3$ and let $\varphi:[2,\infty)\to(0,\infty)$ be nondecreasing. If $A\subseteq\mathbb{N}$ is $k$--AP--free then
\[
\sum_{n\in A\cap[2,\infty)}\frac{1}{n\varphi(n)}
\ \le\ \sum_{j\ge 0}\frac{r_k(2^j)}{2^j\,\varphi(2^j)}.
\]
Consequently, if the series on the right converges then every set $A$ with
\[\sum_{n\in A,\ n\ge 2}\frac{1}{n\varphi(n)}=\infty\]
must contain a nontrivial $k$--term arithmetic progression.
\end{proposition}

\begin{proof}
Combine Lemma~\ref{lem:weighted-dyadic} with Lemma~\ref{lem:translation}. The final statement is the contrapositive.
\end{proof}

\subsubsection{5.3 Convergence for $\varphi(n)=\log n$ using current $r_k$ bounds}

Set $\varphi(n):=\log n$ for $n\ge 3$ (and extend it arbitrarily to $[2,3]$). Then $\varphi$ is nondecreasing.

\begin{lemma}[A calculus fact]
\label{lem:exp-log-sum}
For every $c>0$ the series
\[
\sum_{j\ge 3}\frac{\exp(- (\log j)^c)}{j}
\]
converges.
\end{lemma}

\begin{proof}
Consider the integral
\[
\int_3^{\infty}\frac{\exp(- (\log x)^c)}{x}\,dx.
\]
With the substitution $t=\log x$ we have $dt=dx/x$, so the integral becomes
\[
\int_{\log 3}^{\infty} \exp(-t^c)\,dt,
\]
which converges for every $c>0$. The integral test gives the claim.
\end{proof}

\begin{theorem}[Log-weighted finiteness for $k$--AP--free sets]
\label{thm:log-weighted-finite}
Fix $k\ge 3$. If $A\subseteq\mathbb{N}$ contains no nontrivial $k$--term arithmetic progression, then
\[
\sum_{n\in A,\ n\ge 3}\frac{1}{n\log n}\;<\;\infty.
\]
\end{theorem}

\begin{proof}
By Proposition~\ref{prop:weighted-rk} with $\varphi(n)=\log n$ it suffices to show that
\[
\sum_{j\ge 2}\frac{r_k(2^j)}{2^j\,j}
\]
converges.

\medskip
\noindent\emph{Case $k=3$.}
Bloom--Sisask proved $r_3(N)\ll N/(\log N)^{1+c_3}$ for some $c_3>0$. Taking $N=2^j$ gives
\[
\frac{r_3(2^j)}{2^j\,j}\ \ll\ \frac{1}{j\,(\log 2^j)^{1+c_3}}\ \asymp\ \frac{1}{j^{2+c_3}},
\]
which is summable.

\medskip
\noindent\emph{Case $k=4$.}
Green--Tao proved $r_4(N)\ll N/(\log N)^{c_4}$ for some $c_4>0$. Again with $N=2^j$,
\[
\frac{r_4(2^j)}{2^j\,j}\ \ll\ \frac{1}{j\,(\log 2^j)^{c_4}}\ \asymp\ \frac{1}{j^{1+c_4}},
\]
which is summable.

\medskip
\noindent\emph{Case $k\ge 5$.}
Leng--Sah--Sawhney proved that for each fixed $k\ge 5$ there exists $c_k>0$ such that
\[
r_k(N)\ll N\exp(- (\log\log N)^{c_k}).
\]
With $N=2^j$ this becomes
\[
\frac{r_k(2^j)}{2^j\,j}
\ \ll\ \frac{\exp(- (\log\log 2^j)^{c_k})}{j}
\ \asymp\ \frac{\exp(- (\log j)^{c_k})}{j},
\]
which is summable by Lemma~\ref{lem:exp-log-sum}.

In all cases the dyadic series converges, proving the claim.
\end{proof}

\subsubsection{5.4 The corrected statement: divergence of $\sum_{n\in A}1/(n\log n)$ forces arbitrarily long progressions}

\begin{theorem}[Log-weighted Erd\H{o}s conjecture (now unconditional)]
\label{thm:log-weighted-erdos}
Let $A\subseteq\mathbb{N}$. If
\[
\sum_{n\in A,\ n\ge 3}\frac{1}{n\log n}=\infty,
\]
then $A$ contains arithmetic progressions of every length. More precisely: for every integer $k\ge 3$ and every $J\ge 0$ there exist integers $a\ge 1$ and $d\ge 1$ such that
\[
\{a,a+d,\dots,a+(k-1)d\}\subseteq A\cap(2^J,\infty).
\]
\end{theorem}

\begin{proof}
Fix $k\ge 3$. Suppose for contradiction that $A$ contains no nontrivial $k$--term arithmetic progression. Then Theorem~\ref{thm:log-weighted-finite} implies
\[
\sum_{n\in A,\ n\ge 3}\frac{1}{n\log n}<\infty,
\]
contradicting the hypothesis. Thus $A$ must contain at least one $k$--term progression.

For the strengthened statement with all terms $>2^J$, fix $J\ge 0$ and assume instead that $A\cap(2^J,\infty)$ is $k$--AP--free. Apply Theorem~\ref{thm:log-weighted-finite} to the set $A\cap(2^J,\infty)$ to deduce
\[
\sum_{n\in A,\ n>2^J}\frac{1}{n\log n}<\infty.
\]
Adding the finitely many terms with $n\le 2^J$ gives $\sum_{n\in A,\ n\ge 3}1/(n\log n)<\infty$, again contradicting the hypothesis. Therefore for each $J$ the set $A\cap(2^J,\infty)$ contains a nontrivial $k$--term progression, as claimed.
\end{proof}

\subsubsection{5.5 A sharp barrier for this method}

Proposition~\ref{prop:weighted-rk} shows that to force $k$--term progressions from divergence of a weight $1/(n\varphi(n))$ it suffices that
\[
\sum_{j\ge 0}\frac{r_k(2^j)}{2^j\varphi(2^j)}<\infty.
\]
For the original (unweighted) conjecture $\varphi\equiv 1$, this becomes the dyadic summability condition from Round~2, namely $\sum_j r_k(2^j)/2^j<\infty$. The current best bounds for $k\ge 4$ do not yield this convergence.

One can also quantify the near-optimality of taking $\varphi(n)=\log n$ for this approach when the $k\ge 5$ bounds have exponent at most $1$.

\begin{lemma}[No improvement to $1/(n(\log n)^\alpha)$ for $\alpha<1$ when $0<c\le 1$]
\label{lem:alpha-barrier}
Fix $0<c\le 1$ and $\alpha<1$. Then
\[
\sum_{j\ge 3}\frac{\exp(- (\log j)^c)}{j^{\alpha}}=\infty.
\]
\end{lemma}

\begin{proof}
Choose $\varepsilon>0$ such that $\alpha+\varepsilon\le 1$. Since $c\le 1$, we have $(\log j)^c/\log j=(\log j)^{c-1}\to 0$. Hence for all sufficiently large $j$,
\[
(\log j)^c\le \varepsilon\log j.
\]
Exponentiating gives $\exp(- (\log j)^c)\ge \exp(-\varepsilon\log j)=j^{-\varepsilon}$ for all sufficiently large $j$, and therefore
\[
\frac{\exp(- (\log j)^c)}{j^{\alpha}}\ge \frac{1}{j^{\alpha+\varepsilon}}.
\]
Since $\alpha+\varepsilon\le 1$, the comparison series $\sum_j 1/j^{\alpha+\varepsilon}$ diverges, so the original series diverges.
\end{proof}

\noindent\textbf{Interpretation.}
In the range $0<c_k\le 1$, Lemma~\ref{lem:alpha-barrier} shows that the current $k\ge 5$ bound $r_k(2^j)/2^j\ll \exp(- (\log j)^{c_k})$ cannot, via Proposition~\ref{prop:weighted-rk}, force finiteness of $\sum_{n\in A}1/(n(\log n)^\alpha)$ for any $\alpha<1$. Thus within this dyadic+$r_k$ strategy, the threshold weight $1/(n\log n)$ is essentially best possible unless one obtains substantially stronger savings in $r_k(N)$ (or proves that the available exponent $c_k$ is $>1$).

\subsection{6) ADVERSARIAL VERIFICATION}

\begin{itemize}
\item \textbf{Monotonicity requirements.} Lemma~\ref{lem:weighted-dyadic} requires $\varphi$ nondecreasing. For $\varphi(n)=\log n$ this holds on $[3,\infty)$, and the finitely many terms $n=2$ (and $n=1$) are irrelevant to divergence/convergence.
\item \textbf{Translation step.} Lemma~\ref{lem:translation} uses the bijection $B_j\leftrightarrow T_j$ by shifting by $2^j$. This preserves $k$--APs because adding a constant to all terms of a progression keeps equal differences.
\item \textbf{Summability in the $k\ge 5$ case.} Lemma~\ref{lem:exp-log-sum} proves convergence of $\sum_j \exp(- (\log j)^c)/j$ for all $c>0$ by a direct integral substitution; no asymptotics are used.
\item \textbf{Quantifiers for arbitrarily large progressions.} In Theorem~\ref{thm:log-weighted-erdos}, the refinement ``all terms $>2^J$'' is verified by applying Theorem~\ref{thm:log-weighted-finite} to the tail set $A\cap(2^J,\infty)$; if that tail were $k$--AP--free then the weighted tail-sum would be finite, contradicting global divergence.
\item \textbf{Compatibility with Round~2.} Theorem~\ref{thm:log-weighted-erdos} does not assume $\sum_{n\in A}1/n=\infty$ and does not imply it; it is a genuinely different (stronger) hypothesis. For $k=3$, Round~2 already proves much more from the weaker harmonic divergence.
\end{itemize}

\subsection{7) FINAL (EXACTLY ONE)}

\textbf{UNRESOLVED (BUT STRICTLY ADVANCED).}
The original Erd\H{o}s conjecture (harmonic divergence $\sum_{n\in A}1/n=\infty$ implies arithmetic progressions of every length) remains open for $k\ge 4$. However, Round~3 gives a complete proof of a minimal corrected statement: divergence of the log-weighted sum $\sum_{n\in A}1/(n\log n)$ forces progressions of every length (indeed with arbitrarily large terms), using current best bounds for $r_k(N)$.

\subsection{8) COMPLETION ESTIMATE (MANDATORY)}

\noindent\textbf{COMPLETION: 60\%}

\subsection{9) REFERENCES}

\begin{thebibliography}{99}
\bibitem{BlSi20} T. F. Bloom and O. Sisask, \emph{Breaking the logarithmic barrier in Roth's theorem on arithmetic progressions}, arXiv:2007.03528.
\bibitem{GrTa17} B. Green and T. Tao, \emph{New bounds for Szemer\'edi's theorem, III: A polylogarithmic bound for $r_4(N)$}, arXiv:1705.01703 (accepted in \emph{Mathematika}, 2017).
\bibitem{LSS24} J. Leng, A. Sah, and M. Sawhney, \emph{Improved bounds for Szemer\'edi's theorem}, arXiv:2402.17995.
\bibitem{ErdosProblems3} T. F. Bloom, \emph{Erd\H{o}s Problem \#3}, https://www.erdosproblems.com/3 (accessed 2026-01-20).
\end{thebibliography}
