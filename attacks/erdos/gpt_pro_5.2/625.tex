\section*{Problem 625}

\subsection*{FORMAL RESTATEMENT}
For a graph $G$, the \emph{cochromatic number} $\zeta(G)$ is the minimum number of colours in a vertex-colouring such that each colour class induces either a clique or an independent set. Let $\chi(G)$ be the usual chromatic number.

Let $G\sim G(n,1/2)$ be the Erd\H{o}s--R\'enyi random graph.
The question is whether
\[
\chi(G)-\zeta(G)\to\infty
\]
\emph{with high probability} (equivalently: for every fixed $M$, $\Pr[\chi(G)-\zeta(G)\ge M]\to 1$).

\subsection*{QUICK LITERATURE/CONTEXT CHECK}
This is Erd\H{o}s Problem \#625 and is listed as open.
It is classical that for $G\sim G(n,1/2)$,
\[
\zeta(G)\le \chi(G)\le (1+o(1))\frac{n}{2\log_2 n},
\]
and also $\zeta(G)\ge \frac{n}{2\log_2 n}$ follows from the fact that $G$ a.a.s. has clique number and independence number $<2\log_2 n$.

Recent progress (2024) by Heckel and independently Steiner shows that $\chi(G)-\zeta(G)$ is \emph{not} bounded with high probability; moreover, if $\chi(G)-\zeta(G)\le f(n)$ with high probability, then $f(n)\ge n^{1/2-o(1)}$ along an infinite sequence of $n$.
Heckel further proved that for any $\varepsilon>0$ one has $\chi(G)-\zeta(G)\ge n^{1-\varepsilon}$ for roughly $95\%$ of all $n$, and conjectured the typical order $\asymp n/(\log n)^3$.

\subsection*{ATTACK PLAN}
\begin{enumerate}[label=\arabic*.]
\item Record the standard asymptotic scale of $\chi(G)$ and the sandwich bounds on $\zeta(G)$.
\item Explain why these bounds alone do not decide whether $\chi(G)-\zeta(G)\to\infty$.
\item Summarize the strongest known partial results and indicate a concrete path that could prove the conjectured divergence (e.g. computing expectations/concentration of an auxiliary parameter that couples to $\zeta$).
\end{enumerate}

\subsection*{WORK}
\paragraph{(A) Baseline asymptotics.}
A standard estimate for the random graph is
\[
\chi(G(n,1/2)) = (1+o(1))\frac{n}{2\log_2 n}.
\]
Also, every cocolouring is a partition of $V(G)$ into parts each of which is either an independent set of $G$ or a clique of $G$.
In $G(n,1/2)$, both the independence number $\alpha(G)$ and clique number $\omega(G)$ are a.a.s. $<2\log_2 n$, hence every colour class in a cocolouring has size $<2\log_2 n$. Therefore any cocolouring uses at least
\[
\frac{n}{2\log_2 n}
\]
colours, giving $\zeta(G)\ge \frac{n}{2\log_2 n}$ a.a.s. Together with $\zeta(G)\le \chi(G)$, we have
\[
\frac{n}{2\log_2 n}\ \le\ \zeta(G)\ \le\ \chi(G)\ \le\ (1+o(1))\frac{n}{2\log_2 n}.
\]
This shows $\chi$ and $\zeta$ are asymptotically of the same order, but leaves open whether their \emph{difference} diverges.

\paragraph{(B) Why the classical bounds do not answer the question.}
The above inequalities allow the difference $\chi(G)-\zeta(G)$ to be as small as $0$ and as large as, say, $o\bigl(\tfrac{n}{\log n}\bigr)$ without contradicting asymptotic order.
To prove divergence one needs a quantitative separation between the optimal pure independent-set partition and the optimal mixed clique/independent partition.

\paragraph{(C) Present strongest known progress and a plausible route.}
The recent results indicate that the difference is not bounded w.h.p. and can be as large as $n^{1/2-o(1)}$ on infinitely many $n$ (and even $n^{1-\varepsilon}$ for most $n$).
These results are consistent with the conjecture that typically
\[
\chi(G)-\zeta(G) \asymp \frac{n}{(\log n)^3},
\]
which would in particular imply $\chi(G)-\zeta(G)\to\infty$.

A concrete strategy discussed informally in the literature is to couple $\zeta(G)$ to an auxiliary random variable defined via two independent random graphs $G_1,G_2\sim G(n,1/2)$: let $X$ be the minimum $k$ such that $[n]$ can be partitioned into $k$ sets, each of which is an independent set in either $G_1$ or $G_2$. One may compare $X$ to $\zeta(G)$ by a coupling argument and then try to compute $\mathbb E[X]$ and prove concentration of $X$ around its mean; combined with known concentration for $\chi(G)$, a separation of expectations of order $\omega(1)$ would yield the desired divergence.

\subsection*{VERIFICATION}
The inequalities in (A) are standard consequences of the typical size of $\alpha(G)$ and $\omega(G)$ and the classical asymptotics for $\chi(G(n,1/2))$.
The remaining steps (B)--(C) are not proofs but identify what additional quantitative input is required.

\subsection*{FINAL}
\textbf{UNRESOLVED.}
\begin{itemize}[leftmargin=*]
\item (i) What was proved: The classical sandwich bounds $\frac{n}{2\log_2 n}\le \zeta(G)\le \chi(G)\le (1+o(1))\frac{n}{2\log_2 n}$ for $G\sim G(n,1/2)$.
\item (ii) Strongest partial result stated here: Recent work (Heckel; Steiner) implies $\chi(G)-\zeta(G)$ is not bounded with high probability, and Heckel gives polynomial lower bounds for most $n$.
\item (iii) First gap: None of the cited results imply that for \emph{every} fixed $M$ we have $\Pr[\chi(G)-\zeta(G)\ge M]\to 1$; they allow the possibility that the difference is small on an infinite subsequence of $n$ with non-negligible probability.
\item (iv) Next concrete step: Prove concentration and asymptotics for a variable tightly coupled to $\zeta(G)$ (e.g. the two-graph partition parameter $X$) to within an additive $o(1)$ or $o(n^{1/2})$ window, and separate its expectation from $\mathbb E[\chi(G)]$ by $\omega(1)$.
\end{itemize}

\subsection*{COMPLETION ESTIMATE}
COMPLETION: 30\%.

