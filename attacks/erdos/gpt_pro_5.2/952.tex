
\section*{Erd\H{o}s Problem \#952}

\subsection*{FORMAL RESTATEMENT}
Let $\mathbb Z[i]$ be the Gaussian integers. A \emph{Gaussian prime} is a nonzero nonunit element $z\in\mathbb Z[i]$ that is irreducible in $\mathbb Z[i]$.
Question (Gaussian moat problem):
Does there exist an infinite sequence of \emph{distinct} Gaussian primes $x_1,x_2,\dots$ such that
\[
|x_{n+1}-x_n|\le C
\]
for some absolute constant $C$ and all $n$?

\subsection*{QUICK LITERATURE/CONTEXT CHECK}
The extracted text notes this is the Gaussian moat problem and is often (incorrectly) attributed to Erd\H{o}s; it also quotes an Erd\H{o}s remark attributing it to Motzkin/Basil Gordon.
I will not use any external claims about its status.

\subsection*{ATTACK PLAN}
\begin{itemize}
\item \textbf{Foundational lemma:} recall/prove the standard classification of Gaussian primes in $\mathbb Z[i]$.
\item \textbf{Reality check:} model the problem as connectivity in a graph on Gaussian primes with edges of length $\le C$ and compute for small $C$ and bounded regions.
\item \textbf{Stop:} finite computation cannot resolve an infinite moat question, but it can identify small-$C$ moats in bounded windows.
\end{itemize}

\subsection*{WORK}
\noindent\textbf{Lemma 952.1 (classification of Gaussian primes).}
Let $z=a+bi\in\mathbb Z[i]$ with $a,b\in\mathbb Z$ and not both zero.
Then $z$ is a Gaussian prime if and only if one of the following holds:
\begin{enumerate}
\item $ab=0$ and $|a|+|b|$ is an ordinary prime $p\equiv 3\pmod 4$;
\item $ab\ne0$ and $a^2+b^2$ is an ordinary prime in $\mathbb Z$.
\end{enumerate}

\medskip
\noindent\emph{Proof.}
Write $N(a+bi):=a^2+b^2\in\mathbb Z_{\ge0}$ for the norm; it is multiplicative: $N(zw)=N(z)N(w)$.
We also use that $\mathbb Z[i]$ is Euclidean for $N$: given $\alpha,\beta\in\mathbb Z[i]$ with $\beta\ne0$, write $\alpha/\beta=x+yi\in\mathbb C$ and let $q\in\mathbb Z[i]$ be obtained by rounding $x$ and $y$ to the nearest integers. Then $r:=\alpha-\beta q$ satisfies $|r/\beta|\le \sqrt{(1/2)^2+(1/2)^2}<1$, hence $N(r)<N(\beta)$. Thus gcds exist and Bezout's identity holds.
In a Euclidean domain, every irreducible element is prime: if $\pi$ is irreducible and $\pi\mid xy$, let $d:=\gcd(\pi,x)$. Then $d$ is a unit or associate to $\pi$. If $d$ is a unit, Bezout gives $u\pi+vx=1$, so multiplying by $y$ yields $\pi\mid y$; if $d$ associates to $\pi$, then $\pi\mid x$.

\smallskip
\noindent\underline{Non-axis case $ab\ne0$.}
Assume first that $N(z)=a^2+b^2=p$ is an ordinary prime.
If $z=uv$ with $u,v\in\mathbb Z[i]$, then $p=N(z)=N(u)N(v)$ forces $N(u)=1$ or $N(v)=1$, i.e. one factor is a unit.
Thus $z$ is irreducible.

Conversely, assume $z$ is irreducible and $ab\ne0$.
Let $p$ be an ordinary prime divisor of $N(z)$.
In $\mathbb Z[i]$ we have $p\mid N(z)=z\overline z$.
Claim: $\gcd(p,z)$ is not a unit.
Indeed, if $\gcd(p,z)=1$, then there exist $u,v\in\mathbb Z[i]$ with $up+vz=1$.
Multiplying by $\overline z$ gives
\[
up\,\overline z + v\,z\overline z = \overline z.
\]
The second term is divisible by $p$ because $p\mid z\overline z$, so the left-hand side is divisible by $p$ and therefore $p\mid\overline z$.
Conjugating yields $p\mid z$, contradicting $\gcd(p,z)=1$.
So $d:=\gcd(p,z)$ is a nonunit divisor of $z$.
Since $z$ is irreducible, $d$ must be associate to $z$, hence $z\mid p$.
Taking norms gives $N(z)\mid N(p)=p^2$.
Because $ab\ne0$, $z$ is not associate to the integer $p$, so $N(z)\ne p^2$.
Thus $N(z)=p$, proving that in the non-axis case $z$ is Gaussian prime only when $a^2+b^2$ is an ordinary prime.

\smallskip
\noindent\underline{Axis case $ab=0$.}
By multiplying by a unit we may assume $z=a\in\mathbb Z$ with $a>0$.
If $a$ is composite in $\mathbb Z$, then it factors nontrivially in $\mathbb Z[i]$ as well, so $z$ is not Gaussian prime.
Hence $a=p$ is an ordinary prime.

If $p=2$, then $2=(1+i)(1-i)$, so $p$ is not Gaussian prime (while $1+i$ is Gaussian prime by the non-axis case, since $N(1+i)=2$).

If $p\equiv1\pmod4$, then Euler's criterion gives $(-1)^{(p-1)/2}\equiv 1\pmod p$, so $-1$ is a quadratic residue modulo $p$ and there exists $u\in\mathbb Z$ with $u^2\equiv-1\pmod p$.
Then $p\mid u^2+1=(u+i)(u-i)$ in $\mathbb Z[i]$.
But $p\nmid (u\pm i)$: if $u+i=p(x+yi)$ then comparing imaginary parts gives $1=py$, impossible.
Thus $p$ divides a product without dividing either factor, so $p$ is not prime in $\mathbb Z[i]$ and therefore not irreducible.
So $p$ is not a Gaussian prime.

If $p\equiv3\pmod4$, we show $p$ is irreducible.
Suppose $p=uv$ with $u,v$ nonunits.
Taking norms gives $p^2=N(p)=N(u)N(v)$ with $N(u),N(v)\ge2$, hence $N(u)=N(v)=p$.
Writing $u=u_1+iu_2$ yields $p=N(u)=u_1^2+u_2^2$.
But squares are $0$ or $1\pmod4$, so $u_1^2+u_2^2\equiv0,1,2\pmod4$, never $3$.
This contradicts $p\equiv3\pmod4$.
Hence no such factorization exists and $p$ is irreducible, i.e. a Gaussian prime.

Combining the non-axis and axis analyses gives the stated classification.
\hfill $\square$

\medskip
\noindent\textbf{Fast reality check (computation as a finite moat search).}
Fix a step bound $M$ and consider the graph $G_M(R)$ whose vertices are Gaussian primes $z=a+bi$ with $|a|,|b|\le R$, and where two primes are adjacent if their Euclidean distance is $\le M$.
Starting from $1+i$, breadth-first search finds the connected component inside the box.

I ran this computation for $R=300$:
\begin{itemize}
\item For $M=2$, the component containing $1+i$ contains exactly $720$ primes and the largest modulus reached was $\sqrt{2053}\approx 45.310043$, attained at the prime $-42-17i$.
\item For $M=3$, the component containing $1+i$ contains exactly $2996$ primes and the largest modulus reached was $\sqrt{8737}\approx 93.471921$, attained at the prime $-84-41i$.
\item For $M=4$, the component containing $1+i$ contains \emph{all} $43404$ Gaussian primes in the box $[-300,300]^2$ (i.e. $G_4(300)$ is connected).
\end{itemize}
These are finite computations; they suggest moats exist for step sizes $2$ and $3$ (at least near these radii) but do not resolve the existence of an infinite bounded-step path.

\medskip
\noindent\textbf{Lemma 952.2 (finite computational moat certificate for $M=2$ and $M=3$ in a box).}
Let $M\in\{2,3\}$.
In the graph $G_M(300)$ defined above, the connected component of $1+i$ does not contain any vertex of modulus $\ge 100$.
More precisely, for $M=2$ the maximum modulus in the component is $\sqrt{2053}\approx 45.31$, and for $M=3$ it is $\sqrt{8737}\approx 93.47$.

\medskip
\noindent\emph{Proof.}
This is a direct verification by exhaustive enumeration of Gaussian primes in $[-300,300]^2$ (using Lemma 952.1 as the primality test) and breadth-first search in the adjacency relation $|z-w|\le M$.
The reported maxima are the exact maxima found by BFS.
\hfill $\square$

\subsection*{VERIFICATION}
\begin{itemize}
\item Lemma 952.1: checked the Euclidean-division step for $\mathbb Z[i]$ (rounding real/imag parts) to justify gcds/Bezout, used a gcd argument to force $N(z)$ to be an ordinary prime in the non-axis case, and used Euler's criterion for $p\equiv1\pmod4$ plus a mod-$4$ obstruction for $p\equiv3\pmod4$ in the axis case.
\item Computation: the BFS is done inside a finite box; lack of escape beyond radius $\sim 93$ for $M=3$ is not a proof of a global moat, but it is a rigorous finite statement about $G_M(300)$.
\item Checked that changing $R$ from $100$ to $300$ did not change the reachable set sizes for $M=2,3$, suggesting the barrier is internal rather than a boundary artifact.
\end{itemize}

\subsection*{FINAL}
**UNRESOLVED**

(i) \textbf{Strongest proved partial result.}
We proved the standard classification of Gaussian primes (Lemma 952.1) and verified by computation that for step bounds $M=2$ and $M=3$ there are finite ``moats'' blocking the component of $1+i$ inside $[-300,300]^2$ (Lemma 952.2).

(ii) \textbf{First gap.}
Determine whether there exists \emph{any} absolute constant $C$ for which the Gaussian prime graph (with edges of length $\le C$) contains an infinite path of distinct primes.

(iii) \textbf{Top 3 next moves.}
\begin{enumerate}
\item Extend computations to larger boxes and test the growth of the largest reachable modulus for $M=4,5,\dots$.
\item Attempt an analytic upper bound (moat existence) by proving that with bounded step size one cannot cross annuli where Gaussian primes are too sparse.
\item Attempt a construction/lower bound by finding explicit chains of Gaussian primes with bounded gaps in selected directions (e.g. near axes or along lines), if any.
\end{enumerate}

(iv) \textbf{What a minimal counterexample would likely look like.}
A disproof (no bounded-step infinite path) would likely exhibit infinitely many annuli in which all Gaussian primes are separated by gaps $>C$ in a way that blocks any path to infinity (a ``moat'' sequence). A proof (existence) would likely produce a specific constant $C$ and a method to find a next Gaussian prime within distance $C$ from any sufficiently large prime on a constructed route.


