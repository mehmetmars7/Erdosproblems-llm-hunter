
Let $h(n)$ count the number of powerful (if $p\mid m$ then $p^2\mid m$) integers in $[n^2,(n+1)^2)$. Estimate $h(n)$. In particular is there some constant $c>0$ such that\[h(n) < (\log n)^{c+o(1)}\]and, for infinitely many $n$,\[h(n) >(\log n)^{c-o(1)}?\] Erd\H{o}s writes it is not hard to prove that $\limsup h(n)=\infty$, and that the density $\delta_l$ of integers for which $h(n)=l$ exists and $\sum \delta_l=1$. A proof that $h(n)$ is unbounded is provided by van Doorn in the comments. De Koninck and Luca \cite{DeLu04} have proved, for infinitely many $n$,\[h(n) \gg \left(\frac{\log n}{\log\log n}\right)^{1/3}.\]They also give the density ($\approx 0.275$) of those $n$ such that $h(n)=1$. References [DeLu04] De Koninck, Jean-Marie and Luca, Florian, Sur la proximit\'{e} des nombres puissants . Acta Arith. (2004), 149--157.

\subsection*{FORMAL RESTATEMENT}
Fix $n\in\mathbb{N}$ with $n\ge 1$. Call an integer $m\ge 1$ \emph{powerful} if for every prime $p$, $p\mid m\Rightarrow p^2\mid m$.
Define
\[
 h(n) := \#\{m\in\mathbb{Z}: n^2\le m < (n+1)^2 \ \text{and $m$ is powerful}\}.
\]
The problem asks for asymptotics/upper-lower bounds for $h(n)$ as $n\to\infty$.
In particular it asks whether there exists a constant $c>0$ such that
\[
 h(n) < (\log n)^{c+o(1)} \quad (n\to\infty)
\]
(i.e. $\log h(n) \le (c+o(1))\log\log n$), and whether for infinitely many $n$ we have
\[
 h(n) > (\log n)^{c-o(1)}.
\]
Edge case conventions: $\log$ is natural logarithm; $o(1)$ means a quantity tending to $0$ as $n\to\infty$.

\subsection*{QUICK LITERATURE/CONTEXT CHECK}
I do not use any external results beyond what is stated in the problem text. The problem text asserts:
(1) $\limsup_{n\to\infty} h(n)=\infty$ (unboundedness),
(2) the densities $\delta_\ell := \lim_{N\to\infty} \#\{1\le n\le N: h(n)=\ell\}/N$ exist and sum to $1$,
(3) for infinitely many $n$, $h(n) \gg (\log n/\log\log n)^{1/3}$.

\subsection*{ATTACK PLAN}
\emph{Proof-track ideas.}
\begin{itemize}
\item Use the structural parametrization of powerful numbers $m=a^2b^3$ with $b$ squarefree to reduce counting to lattice points $(a,b)$ lying in a thin region defined by $n^2\le a^2b^3<(n+1)^2$.
\item Derive global bounds for the counting function $P(x):=\#\{m\le x: m\text{ powerful}\}$ and convert them into average bounds on $h(n)$ via telescoping sums.
\item (Hard) Improve global bounds to \emph{local} bounds in short intervals of length $\asymp n$, aiming for polylogarithmic control.
\end{itemize}
\emph{Disproof/construction-track ideas.}
\begin{itemize}
\item Try to construct $n$ for which many distinct pairs $(a,b)$ yield $a^2b^3$ in $[n^2,(n+1)^2)$, perhaps by forcing $a^2b^3$ to cluster near a square via congruences/CRT.
\item If polylog bounds are false, search computationally for unusually large $h(n)$ and attempt to reverse-engineer structural reasons.
\end{itemize}

\subsection*{WORK}
\paragraph{Fast reality check (explicit computation).}
I computed $h(n)$ by enumerating powerful numbers $m\le (N+1)^2$ using the representation $m=a^2b^3$ with $b$ squarefree, then counting how many fall in each interval $[n^2,(n+1)^2)$. Exact outputs:
\begin{itemize}
\item $h(1),\dots,h(20) = [1,2,1,1,3,1,1,2,1,2,3,1,1,3,2,2,1,2,2,2]$.
\item For $1\le n\le 5000$, $\max h(n)=7$, first attained at $n=3611$.
\item For $1\le n\le 100000$, $\max h(n)=8$, attained at $n\in\{17329,57476,66154,83860,88402\}$.
\item Empirical distribution up to $n\le 100000$ (counts of $h(n)=\ell$ for $\ell\le 8$):
\[
\#\{n\le 100000: h(n)=1\}=28273,\ \#\{h=2\}=40348,\ \#\{h=3\}=22417,\ \#\{h=4\}=7190,\ \#\{h=5\}=1524,\ \#\{h=6\}=219,\ \#\{h=7\}=24,\ \#\{h=8\}=5.
\]
\end{itemize}
These computations are only sanity checks; they are not proofs.

\paragraph{Lemma 942.1 (structure of powerful integers).}
An integer $m\ge 1$ is powerful if and only if there exist integers $a\ge 1$ and squarefree $b\ge 1$ such that
\[
 m=a^2b^3.
\]
Moreover, $b$ is uniquely determined as the product of primes dividing $m$ to an odd exponent.

\emph{Proof.}
Write the prime factorization $m=\prod_p p^{e_p}$ with $e_p\ge 0$ and $e_p=0$ for all but finitely many primes.

($\Rightarrow$) If $m$ is powerful then every $e_p\in\{0\}\cup\{2,3,4,\dots\}$. Define
\[
 b:=\prod_{p: e_p\text{ odd}} p
\]
(the product over primes where $e_p$ is odd). Then $b$ is squarefree by construction.
Now define $a$ by specifying the exponent of each prime $p$ in $a$:
\[
\nu_p(a):=
\begin{cases}
 e_p/2 & \text{if $e_p$ is even},\\
 (e_p-3)/2 & \text{if $e_p$ is odd}.
\end{cases}
\]
This is an integer because if $e_p$ is odd then $e_p\ge 3$ and $e_p-3$ is even.
Then the exponent of $p$ in $a^2b^3$ equals
\[
2\nu_p(a)+3\nu_p(b)=
\begin{cases}
2\cdot(e_p/2)+0=e_p & \text{if $e_p$ is even},\\
2\cdot((e_p-3)/2)+3\cdot 1=e_p & \text{if $e_p$ is odd},
\end{cases}
\]
so $m=a^2b^3$.

($\Leftarrow$) If $m=a^2b^3$ with $b$ squarefree, then in the prime factorization of $m$ every prime exponent is either $2\nu_p(a)$ (if $p\nmid b$) or $2\nu_p(a)+3$ (if $p\mid b$). In either case it is at least $2$ whenever $p\mid m$, so $m$ is powerful.

Uniqueness of $b$: if $m=\prod p^{e_p}$, then $e_p$ is odd if and only if the $p$-exponent in $b^3$ contributes an odd amount (namely $3$), which happens iff $p\mid b$. Thus $b$ is uniquely determined as the product of primes with odd exponent in $m$.
\hfill $\Box$

\paragraph{Lemma 942.2 (global counting bound).}
Let
\[
P(X):=\#\{m\le X: m\text{ powerful}\}.
\]
Then for all $X\ge 1$,
\[
P(X) \le \Big(\sum_{b=1}^{\infty} b^{-3/2}\Big)\, X^{1/2}.
\]
In particular $P(X)=O(X^{1/2})$.

\emph{Proof.}
By Lemma 942.1, each powerful $m\le X$ has a (unique) representation $m=a^2b^3$ with $b$ squarefree. Ignoring the squarefree restriction only increases the count, so
\[
P(X) \le \#\{(a,b)\in\mathbb{N}^2: a^2b^3\le X\}.
\]
Fix $b\ge 1$. The inequality $a^2b^3\le X$ is equivalent to $a\le \sqrt{X}/b^{3/2}$, so the number of admissible $a$ is at most $\sqrt{X}/b^{3/2}$. Therefore
\[
P(X) \le \sum_{b=1}^{\lfloor X^{1/3}\rfloor} \frac{\sqrt{X}}{b^{3/2}} \le \sqrt{X}\sum_{b=1}^{\infty} \frac{1}{b^{3/2}}.
\]
The series $\sum_{b\ge 1} b^{-3/2}$ converges (a $p$-series with $p=3/2>1$), giving the stated bound.
\hfill $\Box$

\paragraph{Proposition 942.3 (bounded average of $h(n)$).}
For every $N\ge 1$,
\[
\frac{1}{N}\sum_{n=1}^N h(n) \le \Big(\sum_{b=1}^{\infty} b^{-3/2}\Big)\,\Big(1+\frac{1}{N}\Big).
\]
In particular, the average value of $h(n)$ over $1\le n\le N$ is $O(1)$.

\emph{Proof.}
The intervals $[n^2,(n+1)^2)$ for $n=1,2,\dots,N$ partition $[1,(N+1)^2)$, hence
\[
\sum_{n=1}^N h(n)=\#\{m\in\mathbb{Z}:1\le m<(N+1)^2\ \text{and $m$ powerful}\}=P((N+1)^2-1)\le P((N+1)^2).
\]
Apply Lemma 942.2 with $X=(N+1)^2$ to get
\[
\sum_{n=1}^N h(n)\le \Big(\sum_{b\ge 1} b^{-3/2}\Big)\,(N+1),
\]
and divide by $N$.
\hfill $\Box$

\paragraph{Lemma 942.4 (trivial pointwise lower bound).}
For all $n\ge 1$, $h(n)\ge 1$.

\emph{Proof.}
The integer $n^2$ lies in $[n^2,(n+1)^2)$ and is a perfect square, hence powerful. So the interval contains at least one powerful integer.
\hfill $\Box$

\subsection*{VERIFICATION}
\begin{itemize}
\item Lemma 942.1: checked parity cases for exponents $e_p=2,3,4,5,6,7$ explicitly; the construction gives $e_p=2u$ if even and $e_p=2u+3$ if odd.
\item Lemma 942.2: the relaxation removing the squarefree restriction only increases the count, so the upper bound is valid.
\item Proposition 942.3: verified that the intervals $[n^2,(n+1)^2)$ are disjoint and cover $[1,(N+1)^2)$.
\item Computation sanity: for $n=1$, interval $[1,4)$ contains powerful $1,4$? Actually $4$ is excluded; so only $1$, giving $h(1)=1$, matching the computed list.
\end{itemize}

\subsection*{FINAL}
**UNRESOLVED**
(i) Strongest proved partial result: the global counting bound $P(X)=O(X^{1/2})$ (Lemma 942.2) implies the bounded-average statement $\frac1N\sum_{n\le N} h(n)=O(1)$ (Proposition 942.3) and the trivial pointwise lower bound $h(n)\ge 1$ (Lemma 942.4).
(ii) First gap: obtain any nontrivial \emph{pointwise} upper bound $h(n)\le (\log n)^{C}$ (or even $h(n)=n^{o(1)}$) valid for all large $n$.
(iii) Top 3 next moves:
\begin{enumerate}
\item Study the Diophantine inequality $n^2\le a^2b^3<(n+1)^2$ with $b$ squarefree via lattice-point counting and try to prove a pointwise bound uniform in $n$.
\item Attempt to control powerful numbers in intervals of length $\asymp \sqrt{x}$ using a square-sieve: if $m$ is powerful then its squarefree kernel $\mathrm{rad}(m)$ satisfies $\mathrm{rad}(m)^2\mid m$; try to bound how often many such kernels can occur in $[n^2,(n+1)^2)$.
\item Extend the computation to much larger $n$ and record the maximizing $n$ and the specific powerful numbers in the maximizing interval; attempt to detect a structural pattern enabling a constructive lower bound.
\end{enumerate}
(iv) Minimal counterexample structure (to a conjectured polylog upper bound): an $n$ for which there exist many pairs $(a,b)$ with $b$ squarefree such that $a^2b^3$ are all squeezed into a length-$2n+1$ window; heuristically this would likely require $n$ to have many small prime factors so that many nearby integers have large square divisors.


