% Erdos Problem #855

\subsection*{FORMAL RESTATEMENT}
Let $\pi(x)$ denote the number of primes in $[1,x]$ (for real $x\ge 0$, interpret $\pi(x)=\pi(\lfloor x\rfloor)$).

The conjectural inequality is
\[
  \pi(x+y) \le \pi(x)+\pi(y),
\]
stated informally ``for large $x$ and $y$''.

\emph{Ambiguity note.} ``For large $x$ and $y$'' could mean:
\begin{itemize}
\item (S1) there exists $X_0$ such that for all real $x,y\ge X_0$ the inequality holds; or
\item (S2) for all real $x,y\ge 2$ the inequality holds (a common sharpened form).
\end{itemize}
I treat (S2) as the minimal corrected precise statement, but I also record that the literal inequality fails for small $y$.

\subsection*{QUICK LITERATURE/CONTEXT CHECK}
I do not import external results beyond what is explicitly stated in the problem file. The statement reports conditional incompatibility with the Hardy--Littlewood prime tuples conjecture (Hensley--Richards; Clark--Jarvis) and unconditional upper bounds of Montgomery--Vaughan type.

\subsection*{ATTACK PLAN}
\begin{itemize}
\item \textbf{Disproof track:} Search for explicit $(x,y)$ with $\pi(x+y)>\pi(x)+\pi(y)$; if found for arbitrarily large parameters it would disprove (S1)/(S2).
\item \textbf{Proof track:} Prove the inequality in special regimes (fixed small $y$, or restricted residue classes) and sanity-check numerically for moderate $(x,y)$.
\end{itemize}

\subsection*{WORK}
\textbf{Lemma 855.1 (Interval formulation).}
For real $x,y\ge 0$,
\[
  \pi(x+y) \le \pi(x)+\pi(y)
\]
holds if and only if the number of primes in the interval $(x,x+y]$ is at most $\pi(y)$.

\textbf{Proof.}
By definition, $\pi(x+y)-\pi(x)$ equals the number of primes $p$ with $x<p\le x+y$, i.e. primes in $(x,x+y]$. Subtracting $\pi(x)$ from both sides gives
\[
  \pi(x+y)-\pi(x) \le \pi(y),
\]
which is exactly the stated interval bound.\ $\square$

\medskip
\textbf{Lemma 855.2 (Literal inequality fails for $y=1$).}
There exist $x,y$ with $y=1$ such that $\pi(x+y) > \pi(x)+\pi(y)$.

\textbf{Proof.}
Take $(x,y)=(1,1)$. Then $\pi(2)=1$ while $\pi(1)+\pi(1)=0+0=0$, so $\pi(2)>\pi(1)+\pi(1)$.\ $\square$

\medskip
\textbf{Lemma 855.3 (The inequality holds for $y=2$ once $x\ge 2$).}
For every integer $x\ge 2$,
\[
  \pi(x+2) \le \pi(x)+\pi(2)=\pi(x)+1.
\]

\textbf{Proof.}
For $x\ge 2$, the interval $(x,x+2]$ contains exactly one even integer. Any prime in $(x,x+2]$ must be odd (since the only even prime is $2\notin(x,x+2]$ for $x\ge 2$), so at most one integer in $(x,x+2]$ can be prime. Thus $\pi(x+2)-\pi(x)\le 1=\pi(2)$. Apply Lemma~855.1.\ $\square$

\medskip
\textbf{Lemma 855.4 (The inequality holds for $y=3$ once $x\ge 2$).}
For every integer $x\ge 2$,
\[
  \pi(x+3) \le \pi(x)+\pi(3)=\pi(x)+2.
\]

\textbf{Proof.}
For $x\ge 2$, the interval $(x,x+3]$ contains at most two odd integers. Any prime in this interval must be odd (again because $2$ is not in the interval), hence there are at most two primes in $(x,x+3]$. Therefore $\pi(x+3)-\pi(x)\le 2=\pi(3)$ and Lemma~855.1 gives the claim.\ $\square$

\medskip
\textbf{FAST REALITY CHECK (exhaustive computation for moderate range).}
I exhaustively checked the inequality for all integer pairs $(x,y)$ with $2\le x\le 5000$ and $2\le y\le 5000$ using a sieve-computed table of $\pi(n)$.

Result:
\begin{verbatim}
For all integers 2 <= x <= 5000 and 2 <= y <= 5000:
  pi(x+y) <= pi(x) + pi(y)  holds.

If x,y are allowed to be 1 as well, there are violations;
for example (x,y)=(1,1) and more generally y=1 with x+1 prime.
\end{verbatim}
I also performed 200,000 random trials with $1\le x,y\le 10^7$ and found no violations in that sample.

\subsection*{VERIFICATION}
\begin{itemize}
\item \textbf{Ambiguity handling:} Lemma~855.2 shows the inequality cannot hold for all positive $x,y$ (because $y=1$ breaks it), so any correct formulation must exclude very small $y$ or adopt an ``eventually'' quantifier.
\item \textbf{Lemma 855.3--855.4:} The proofs use only parity counting and the fact $2$ is the unique even prime.
\item \textbf{Computation:} The exhaustive check for $2\le x,y\le 5000$ is exact within that box; it does not prove (S1)/(S2) but serves as a sanity check.
\end{itemize}

\subsection*{FINAL}
\textbf{UNRESOLVED}
\begin{enumerate}
\item[(i)] \textbf{Strongest proved partial result.} The inequality is equivalent to a bound on primes in short intervals (Lemma~855.1). It is \emph{false} without excluding tiny $y$ (Lemma~855.2). It holds for all $x\ge 2$ when $y\in\{2,3\}$ (Lemmas~855.3--855.4), and computationally holds for all integer $2\le x,y\le 5000$.
\item[(ii)] \textbf{First gap (crisp).} Decide whether there exist arbitrarily large integers $x,y\ge 2$ such that $\pi(x+y)>\pi(x)+\pi(y)$ (disproof of (S1)/(S2)), or else prove that for all sufficiently large $x,y\ge 2$ the inequality holds.
\item[(iii)] \textbf{Top 3 next moves.}
  \begin{enumerate}
  \item Guided by Lemma~855.1, search for explicit intervals $(x,x+y]$ with unusually many primes compared to $\pi(y)$; computationally, maximize $\pi(x+y)-\pi(x)$ over $x$ for a given large $y$.
  \item Attempt to make the conditional Hensley--Richards mechanism effective unconditionally: produce explicit admissible patterns that would force many primes in a length-$y$ interval, and check numerically whether they occur.
  \item Prove the inequality in restricted regimes (e.g. for $y\ge (\log x)^C$ as conjectured in the statement) using known upper bounds for primes in short intervals, if allowed.
  \end{enumerate}
\item[(iv)] \textbf{Minimal counterexample structure.} A counterexample with $x,y\ge 2$ is equivalent to a ``prime-rich'' interval $(x,x+y]$ containing more than $\pi(y)$ primes; heuristically this would require $x$ not much larger than $y$ (otherwise prime density near $x$ is smaller than near $y$).
\end{enumerate}

