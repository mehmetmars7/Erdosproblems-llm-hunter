% Erdos Problem #670

% Problem statement (from 665-670.tex):
% Let $A\subseteq \mathbb{R}^d$ be a set of $n$ points such that all pairwise distances differ by at least $1$. Is the diameter of $A$ at least $(1+o(1))n^2$? The lower bound of $\binom{n}{2}$ for the diameter is trivial. Erd\H{o}s proved the claim when $d=1$.

\noindent\textbf{FORMAL RESTATEMENT.}
Fix $d\ge 1$.
For a finite set $A\subseteq\mathbb{R}^d$ of size $n$, define its diameter
\[\operatorname{diam}(A):=\max\{\|x-y\|: x,y\in A\}.
\]
Assume that for any two distinct unordered pairs $\{x,y\}\ne\{x',y'\}$ with $x\ne y$ and $x'\ne y'$, the corresponding distances satisfy
\[\big|\|x-y\| - \|x'-y'\|\big|\ge 1.
\]
(Equivalently: the multiset of $\binom{n}{2}$ pairwise distances is $1$-separated.)
The conjecture asks whether necessarily
\[
\operatorname{diam}(A) \ge (1+o(1))\, n^2\quad\text{as } n\to\infty.
\]

\medskip
\noindent\emph{Potential ambiguity.}
The phrase ``all pairwise distances differ by at least $1$'' is interpreted above as a separation condition between distances of \emph{distinct pairs}. This is consistent with the statement that the trivial diameter lower bound is $\binom n2$ (see Lemma 670.2).

\bigskip
\noindent\textbf{QUICK LITERATURE/CONTEXT CHECK.}
The prompt states that the lower bound $\binom n2$ is trivial and that Erd\H{o}s proved the $(1+o(1))n^2$ lower bound for $d=1$.
Per project rules, I do not use external sources beyond this.

\bigskip
\noindent\textbf{ATTACK PLAN.}
\begin{itemize}
\item Prove the ``trivial'' lower bound $\operatorname{diam}(A)\ge \binom n2$ carefully, including the necessary preliminary fact that the minimum distance is at least $1$ when $n\ge 3$.
\item Do small-$n$ checks and a computation in $d=1$ (Golomb ruler search) to see the gap between $\binom n2$ and the known/expected $\Theta(n^2)$ behavior.
\item No general improvement beyond $\binom n2$ is proved here for $d\ge 2$; conclude unresolved.
\end{itemize}

\bigskip
\noindent\textbf{WORK.}

\medskip
\noindent\textbf{Fast reality check (tiny $n$).}
For $n=2$, there is only one distance, so the separation condition is vacuous; the diameter can be any positive number.
For $n=3$, one can realize distances $\{1,2,3\}$ by collinear points at $0,1,3$ in $\mathbb{R}$, giving diameter $3=\binom 32$.
For $n=4$, collinear points at $0,1,4,6$ realize all distances $1,2,3,4,5,6$, giving diameter $6=\binom 42$.
So the trivial bound can be tight for small $n$.

\medskip
\noindent\textbf{Lemma 670.1 (Minimum distance is at least $1$ for $n\ge 3$).}
Let $A\subseteq\mathbb{R}^d$ be a set of $n\ge 3$ points satisfying the distance-separation condition.
Then the minimum pairwise distance in $A$ is at least $1$.

\noindent\emph{Proof.}
Let $\delta$ be the minimum distance between distinct points of $A$, and choose distinct points $P,Q\in A$ with $\|P-Q\|=\delta$.
Since $n\ge 3$, choose a third point $R\in A\setminus\{P,Q\}$.
The distances $\|P-R\|$ and $\|Q-R\|$ correspond to distinct unordered pairs from $\{P,Q\}$, so each must differ from $\delta$ by at least $1$.
Also, by minimality of $\delta$, both are at least $\delta$.
Therefore
\[
\|P-R\|\ge \delta+1,\qquad \|Q-R\|\ge \delta+1.
\]
Moreover, since the unordered pairs $\{P,R\}$ and $\{Q,R\}$ are distinct, the separation condition gives
\(|\|P-R\|-\|Q-R\||\ge 1\).
Thus one of these two distances is at least the other plus $1$.
Without loss of generality assume $\|Q-R\|\ge \|P-R\|+1$.
Then combining with $\|P-R\|\ge \delta+1$ yields
\[\|Q-R\|\ge (\delta+1)+1 = \delta+2.
\]
Now apply the triangle inequality in $\mathbb{R}^d$ to the triangle $PQR$:
\[
\|Q-R\| \le \|Q-P\| + \|P-R\| = \delta + \|P-R\|.
\]
Using $\|Q-R\|\ge \|P-R\|+1$ gives
\[
\|P-R\|+1 \le \delta + \|P-R\|,
\]
so $1\le \delta$.
Therefore the minimum distance $\delta\ge 1$.
\hfill$\square$

\medskip
\noindent\textbf{Lemma 670.2 (Trivial diameter lower bound).}
Let $A\subseteq\mathbb{R}^d$ be a set of $n\ge 3$ points satisfying the distance-separation condition.
Then
\[
\operatorname{diam}(A)\ge \binom{n}{2}.
\]

\noindent\emph{Proof.}
There are $N:=\binom{n}{2}$ unordered pairs of distinct points in $A$, hence $N$ pairwise distances.
Let these distances be listed in increasing order:
\[d_1<d_2<\cdots<d_N.
\]
The separation condition says that for any two distinct distances in this list, their difference is at least $1$.
In particular, consecutive distances satisfy $d_{i+1}-d_i\ge 1$ for each $i$.
Summing these inequalities yields
\[
 d_N - d_1 = \sum_{i=1}^{N-1} (d_{i+1}-d_i) \ge N-1.
\]
By Lemma 670.1, $d_1\ge 1$.
Therefore $d_N\ge 1+(N-1)=N=\binom n2$.
Since $d_N$ is one of the pairwise distances, it is at most the diameter, i.e. $d_N\le \operatorname{diam}(A)$.
Hence $\operatorname{diam}(A)\ge \binom n2$.
\hfill$\square$

\medskip
\noindent\textbf{Computation in $d=1$ (optimal integer Golomb rulers for $n\le 7$).}
In one dimension, taking integer points $0=x_1<\dots<x_n=L$ with all pairwise differences distinct automatically satisfies the ``distance differences $\ge 1$'' condition.
I brute-forced the smallest possible $L$ for $n\le 7$ (optimal Golomb rulers), obtaining:
\begin{verbatim}
n=2: L=1,  marks (0,1)
n=3: L=3,  marks (0,1,3)
n=4: L=6,  marks (0,1,4,6)
n=5: L=11, marks (0,1,4,9,11)
n=6: L=17, marks (0,1,4,10,12,17)
n=7: L=25, marks (0,1,4,10,18,23,25)
\end{verbatim}
These illustrate that the diameter can equal $\binom n2$ for $n\le 4$ but must exceed it for $n\ge 5$ in $d=1$.

\bigskip
\noindent\textbf{VERIFICATION.}
\begin{itemize}
\item Lemma 670.1: checked that it uses $n\ge 3$ essentially (to pick a third point $R$). For $n=2$ the claim is false since there is no constraint.
\item Lemma 670.2: verified that the ordering argument uses only the pairwise distance separation and Lemma 670.1.
\item Small-$n$ examples: $\{0,1,3\}$ gives distances $\{1,2,3\}$ and separation $\ge 1$; $\{0,1,4,6\}$ gives distances $\{1,2,3,4,5,6\}$.
\end{itemize}

\bigskip
\noindent\textbf{FINAL.} \textbf{UNRESOLVED}.

\smallskip
\noindent(i) \emph{Strongest proved partial results.}
\begin{itemize}
\item For any $d$ and any $n\ge 3$, the minimum pairwise distance is at least $1$ (Lemma 670.1).
\item Consequently, for any $d$ and any $n\ge 3$, the diameter satisfies $\operatorname{diam}(A)\ge \binom n2$ (Lemma 670.2), matching the ``trivial'' bound stated in the prompt.
\item In $d=1$, small-$n$ computations show the optimal diameter equals $\binom n2$ for $n\le 4$ and exceeds it for $n\ge 5$.
\end{itemize}

\noindent(ii) \emph{First gap (crisp statement).}
For fixed $d\ge 2$, prove or disprove the asymptotic lower bound
\[
\operatorname{diam}(A) \ge (1+o(1))\,n^2
\]
for every $n$-point set $A\subseteq\mathbb{R}^d$ whose pairwise distances are $1$-separated.

\noindent(iii) \emph{Top 3 next moves.}
\begin{enumerate}
\item \textbf{Dimension reduction attempt:} prove that any such $A$ contains a large subset whose points are ``almost collinear'' in a quantitative sense, reducing to the $d=1$ phenomenon.
\item \textbf{Geometric constraints on distance sets:} exploit triangle inequalities across many triples to show that the $\binom n2$ distances cannot be packed into an interval of length $\approx \binom n2$, forcing diameter strictly larger by a linear factor.
\item \textbf{Construction search in $d\ge 2$:} attempt to build explicit families in $\mathbb{R}^2$ or $\mathbb{R}^3$ with diameter close to the trivial lower bound $\binom n2$, which would refute the conjectured $(1+o(1))n^2$ bound.
\end{enumerate}

\noindent(iv) \emph{What a minimal counterexample would likely look like.}
A counterexample (if the conjecture is false) would be a sequence of point sets $A_n\subset\mathbb{R}^d$ with $|A_n|=n$ such that:
\begin{itemize}
\item the $\binom n2$ pairwise distances are all distinct and $1$-separated, but
\item the diameter satisfies $\operatorname{diam}(A_n) \le (c+o(1))n^2$ with some constant $c<1$ (ideally $c\approx 1/2$, matching the trivial lower bound up to lower order).
\end{itemize}
Such a construction would have to realize the distance multiset with unusually ``tight packing'' compatible with triangle inequalities in fixed dimension.
