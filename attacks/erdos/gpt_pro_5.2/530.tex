% Erdos Problem #530
% URL: https://www.erdosproblems.com/530

Let $\ell(N)$ be maximal such that in any finite set $A\subset \mathbb{R}$ of size $N$ there exists a Sidon subset $S$ of size $\ell(N)$ (i.e. the only solutions to $a+b=c+d$ in $S$ are the trivial ones). Determine the order of $\ell(N)$. In particular, is it true that $\ell(N)\sim N^{1/2}$? Originally asked by Riddell \cite{Ri69}. Erd\H{o}s noted the bounds\[N^{1/3} \ll \ell(N) \leq (1+o(1))N^{1/2}\](the upper bound following from the case $A=\{1,\ldots,N\}$). The lower bound was improved to $N^{1/2}\ll \ell(N)$ by Koml\'{o}s, Sulyok, and Szemer\'{e}di \cite{KSS75}. The correct constant is unknown, but it is likely that the upper bound is true, so that $\ell(N)\sim N^{1/2}$. In \cite{AlEr85} Alon and Erd\H{o}s make the stronger conjecture that perhaps $A$ can always be written as the union of at most $(1+o(1))N^{1/2}$ many Sidon sets. (This is easily verified for $A=\{1,\ldots,N\}$ using standard constructions of Sidon sets.) This is discussed in problem C9 of Guy's collection \cite{Gu04}. See also [1088] for a higher-dimensional generalisation. References [AlEr85] Alon, Noga and Erd\H{o}s, P., An application of graph theory to additive number theory . European J. Combin. (1985), 201-203. [Gu04] Guy, Richard K., Unsolved problems in number theory . (2004), xviii+437. [KSS75] Koml\'{o}s, J. and Sulyok, M. and Szemeredi, E., Linear problems in combinatorial number theory . Acta Math. Acad. Sci. Hungar. (1975), 113-121. [Ri69] Riddell, J., On sets of numbers containing no $l$ terms in arithmetic progression . Nieuw Arch. Wisk. (3) (1969), 204-209.

%Erdos problem 530

\subsection*{FORMAL RESTATEMENT}
For an integer $N\ge 1$, define
\[
\ell(N):=\min\Big\{\max\{|S|:S\subseteq A,\ S\ \text{is Sidon}\}: A\subset\mathbb R,\ |A|=N\Big\}.
\]
Here a finite set $S\subset\mathbb R$ is \emph{Sidon} if every solution to $a+b=c+d$ with $a,b,c,d\in S$ is \emph{trivial} in the sense that $\{a,b\}=\{c,d\}$ as multisets (equivalently: the sums $a+b$ with $a\le b$ are all distinct).
The problem asks for the asymptotic order of growth of $\ell(N)$ as $N\to\infty$, and in particular whether $\ell(N)\sim N^{1/2}$.

\subsection*{QUICK LITERATURE/CONTEXT CHECK}
The statement records: Erd\H{o}s' bounds $N^{1/3} \ll \ell(N) \le (1+o(1))N^{1/2}$ and an improvement of the lower bound to $\ell(N)\gg N^{1/2}$ by Koml\'{o}s--Sulyok--Szemer\'{e}di \cite{KSS75}, plus conjectures in \cite{AlEr85}. Per the integrity rule I do not claim or use any result beyond what is explicitly written in the problem text; instead I re-derive the $N^{1/3}$ lower bound and the $\sqrt N$ upper bound for $A=[N]$ from first principles.

\subsection*{ATTACK PLAN}
\emph{Proof-track ideas.}
(i) Upper bound: show for $A=\{1,\dots,N\}$ any Sidon subset has size $O(\sqrt N)$ by counting distinct sums.
(ii) Lower bound: model all nontrivial additive collisions $a+b=c+d$ as edges of a small-uniformity hypergraph and find a large independent set by the probabilistic method; this yields a universal Sidon subset of size $\gg N^{1/3}$.

\emph{Disproof-track ideas.}
To disprove $\ell(N)\sim \sqrt N$ one would need either (a) a family of sets $A$ where every Sidon subset is $o(\sqrt N)$, or (b) a proof that a universal $\gg\sqrt N$ lower bound fails. I do not find an explicit construction here.

\subsection*{WORK}
\begin{lemma}[Upper bound for $A=[N]$]\label{lem:530-upper}
Let $S\subseteq \{1,2,\dots,N\}$ be Sidon and write $m:=|S|$. Then $m\le \sqrt{4N}+1$.
\end{lemma}
\begin{proof}
Because $S$ is Sidon, all sums $a+b$ with $a,b\in S$ and $a\le b$ are distinct. There are exactly $\binom m2+m=\frac{m(m+1)}2$ such unordered pairs. Each sum $a+b$ lies in the interval $[2,2N]$, which contains exactly $(2N-1)$ integers. Therefore
\[
\frac{m(m+1)}2 \le 2N-1.
\]
This implies $m^2 < 4N + m$, hence $m < \sqrt{4N}+1$.
\end{proof}

\begin{lemma}[Universal $\boldsymbol{N^{1/3}}$ lower bound]\label{lem:530-lower}
For every finite set $A\subset\mathbb R$ with $|A|=N$, there exists a Sidon subset $S\subseteq A$ with
\[
|S|\ \ge\ \frac1{8}\,N^{1/3}.
\]
Consequently $\ell(N)\gg N^{1/3}$.
\end{lemma}
\begin{proof}
Call a quadruple $(a,b,c,d)\in A^4$ \emph{bad} if $a+b=c+d$ and $\{a,b\}\ne\{c,d\}$ as multisets. If $S\subseteq A$ contains no bad quadruple, then $S$ is Sidon by definition.

A bad quadruple can involve either $4$ distinct elements or exactly $3$ distinct elements (the $2$-distinct case forces $\{a,b\}=\{c,d\}$). Indeed, if $a+b=c+d$ with $\{a,b\}\ne\{c,d\}$, and some element appears three times then the fourth appearance forces triviality; thus only $3$ or $4$ distinct elements can occur.

\medskip
\noindent\textbf{Step 1: encode bad configurations as hyperedges.}
Construct a $3$-uniform hypergraph $\mathcal H_3$ on vertex set $A$ whose hyperedges are triples $\{x,y,z\}$ of distinct vertices satisfying
\[
x+y=2z,
\]
(which is exactly the form of a bad quadruple with three distinct elements, namely $x+y=z+z$).
Construct a $4$-uniform hypergraph $\mathcal H_4$ on vertex set $A$ whose hyperedges are $4$-sets $\{a,b,c,d\}$ of distinct vertices for which $a+b=c+d$ in some ordering (which is exactly the form of a bad quadruple with four distinct elements).

If $S\subseteq A$ contains no hyperedge of $\mathcal H_3$ and no hyperedge of $\mathcal H_4$, then $S$ contains no bad quadruple, hence $S$ is Sidon.

\medskip
\noindent\textbf{Step 2: bound the number of hyperedges.}
For $\mathcal H_4$: choose an ordered triple $(a,b,c)\in A^3$; then $d$ is forced to be $d=a+b-c$. Hence the number of ordered solutions $(a,b,c,d)\in A^4$ to $a+b=c+d$ is at most $N^3$. Each unordered $4$-set hyperedge of $\mathcal H_4$ corresponds to at least one ordered solution, so
\[
e(\mathcal H_4)\le N^3.
\]

For $\mathcal H_3$: choose an ordered pair $(x,y)\in A^2$ with $x\ne y$; then $z$ is forced to be $z=(x+y)/2$. Hence the number of ordered solutions $(x,y,z)\in A^3$ to $x+y=2z$ with $x\ne y$ is at most $N^2$. Each unordered triple hyperedge corresponds to at least one ordered solution, so
\[
e(\mathcal H_3)\le N^2.
\]

\medskip
\noindent\textbf{Step 3: random sparsification and deletion.}
Choose a random subset $R\subseteq A$ by including each vertex independently with probability $p$. Let $X:=|R|$. Let $Y_3$ be the number of hyperedges of $\mathcal H_3$ fully contained in $R$, and $Y_4$ the number of hyperedges of $\mathcal H_4$ fully contained in $R$. Then
\[
\mathbb E[X]=pN,\qquad \mathbb E[Y_3]=p^3 e(\mathcal H_3)\le p^3 N^2,\qquad \mathbb E[Y_4]=p^4 e(\mathcal H_4)\le p^4 N^3.
\]
Set $p:=\frac12 N^{-2/3}$. Then
\[
\mathbb E[X]=\frac12 N^{1/3},\qquad \mathbb E[Y_3]\le \frac18 N^{1/3},\qquad \mathbb E[Y_4]\le \frac1{16} N^{1/3}.
\]
Hence $\mathbb E[X-Y_3-Y_4]\ge \left(\frac12-\frac18-\frac1{16}\right)N^{1/3}=\frac5{16}N^{1/3}$.

Now, from any specific $R$, delete one arbitrary vertex from each hyperedge contained in $R$ (first for the $3$-edges, then for the $4$-edges, or in any order). This produces a set $S\subseteq R$ containing no $\mathcal H_3$-edge and no $\mathcal H_4$-edge, and satisfying
\[
|S|\ge |R|-Y_3-Y_4 = X-Y_3-Y_4.
\]
Therefore there exists a choice of $R$ for which $|S|\ge \frac5{16}N^{1/3}\ge \frac18 N^{1/3}$. By Step 1, this $S$ is Sidon.
\end{proof}

\paragraph{FAST REALITY CHECK (computation on the example $A=[N]$).}
I computed by exhaustive backtracking the exact maximum size of a Sidon subset of $\{1,\dots,N\}$ for all $N\le 38$ (no timeouts in this range). The results support the $\Theta(\sqrt N)$ behavior in this model case.

\[
\begin{tabular}{cc}
 \hline
$N$ & $\max\{|S|:S\subseteq [N]\ \text{Sidon}\}$\\ \hline
1 & 1\\
2 & 1\\
3 & 2\\
4 & 2\\
5 & 2\\
6 & 3\\
7 & 3\\
8 & 3\\
9 & 3\\
10 & 4\\
11 & 4\\
12 & 4\\
13 & 4\\
14 & 4\\
15 & 5\\
16 & 5\\
17 & 5\\
18 & 5\\
19 & 5\\
20 & 5\\
21 & 6\\
22 & 6\\
23 & 6\\
24 & 6\\
25 & 6\\
26 & 6\\
27 & 6\\
28 & 7\\
29 & 7\\
30 & 7\\
31 & 7\\
32 & 7\\
33 & 7\\
34 & 7\\
35 & 7\\
36 & 8\\
37 & 8\\
38 & 8\\
\hline
\end{tabular}
\]

\subsection*{VERIFICATION}
\begin{itemize}
\item Lemma~\ref{lem:530-upper}: verified that Sidon $\Rightarrow$ all unordered pair sums $a+b$ with $a\le b$ are distinct; counted them as $m(m+1)/2$; range of sums is exactly $\{2,\dots,2N\}$ of size $2N-1$.
\item Lemma~\ref{lem:530-lower}: checked the classification of nontrivial equalities $a+b=c+d$ into either four distinct elements or the three-distinct form $x+y=2z$. Edge-count bounds $e(\mathcal H_3)\le N^2$ and $e(\mathcal H_4)\le N^3$ follow from direct forcing-counts.
\item The deletion argument is deterministic and guarantees that after removing one vertex per surviving hyperedge, no hyperedge remains; thus the output set is Sidon.
\end{itemize}

\subsection*{FINAL}
\textbf{**UNRESOLVED**}

(i) \emph{Strongest proved partial result.} For all $N$, $\ell(N)\gg N^{1/3}$ (Lemma~\ref{lem:530-lower}). Also the example $A=[N]$ forces $\ell(N)=O(\sqrt N)$ via Lemma~\ref{lem:530-upper}.

(ii) \emph{First gap (crisp statement).} Prove (or disprove) that for every $A\subset\mathbb R$ with $|A|=N$ there exists a Sidon subset of size $\ge c\sqrt N$ for some absolute constant $c>0$ (and ideally determine the optimal constant).

(iii) \emph{Top 3 next moves.}
1. Strengthen the $N^{1/3}$ argument by exploiting additive-structure dichotomies: either $A$ has low additive energy (so few collisions) or it has strong structure that can be handled separately.
2. Develop a container-style bound on the family of non-Sidon subsets of $A$ with improved independent set lower bounds.
3. Computationally: search for candidate worst-case sets $A$ (with many additive relations) and numerically estimate the largest Sidon subset size to guess extremal structure.

(iv) \emph{Minimal counterexample structure.} A counterexample to $\ell(N)\asymp \sqrt N$ would be a family of sets $A_N$ of size $N$ with extremely many additive coincidences, arranged so that every subset of size $\gg \sqrt N$ still contains a nontrivial relation $a+b=c+d$ (including the three-distinct $x+y=2z$ type).


