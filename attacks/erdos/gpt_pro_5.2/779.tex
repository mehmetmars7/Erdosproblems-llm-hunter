
Let $n> 1$ and $p_1<\cdots<p_n$ denote the first $n$ primes. Let $P=\prod_{1\leq i\leq n}p_i$. Does there always exist some prime $p$ with $p_n<p<P$ such that $P+p$ is prime? A problem of Deaconescu. Erd\H{o}s expects that the least such prime is much smaller than $P$, and in fact satisfies $p\leq n^{O(1)}$. Deaconescu has verified this conjecture for $n\leq 1000$. With the usual heuristic, we expect that $P+p$ is prime with 'probability' $\approx 1/\log P$, and hence the chance that this fails is $\ll (1-1/\log P)^{P-p_n}\ll \exp(-n^{-cn})$, using $P=n^{(1+o(1))n}$. As Cambie points out, 'the chances of failing are ridiculously small'.


\noindent\textbf{1) FORMAL RESTATEMENT.}

Let $n\ge 2$ and let $p_1<p_2<\cdots<p_n$ be the first $n$ primes. Define
\[
P:=\prod_{i=1}^n p_i.
\]
Question: does there always exist a prime $p$ such that
\[
p_n < p < P\quad\text{and}\quad P+p\text{ is prime}?
\]

\medskip
\noindent\textbf{2) QUICK LITERATURE/CONTEXT CHECK.}

The statement records: Deaconescu verified the conjecture for $n\le 1000$; Erd\H{o}s expects the least such $p$ to be much smaller than $P$, possibly $p\le n^{O(1)}$; and a heuristic probability estimate is given. I do not rely on any external results beyond this.

\medskip
\noindent\textbf{3) ATTACK PLAN.}

\emph{Proof-track ideas.}
\begin{itemize}
\item Interpret the condition $P+p$ prime as requiring the prime $p$ to land in a ``prime translate'' of $P$. Since $P$ is divisible by all small primes $\le p_n$, numbers of the form $P+p$ automatically avoid these small prime divisors (Lemma~779.1), so one only needs to avoid larger factors.
\item Try to use a sieve/covering argument or a probabilistic method to show some prime $p$ in a range makes $P+p$ prime.
\end{itemize}
\emph{Disproof-track ideas.}
\begin{itemize}
\item Try to find $n$ such that $P+p$ is composite for all primes $p$ in $(p_n,P)$; this resembles constructing covering congruences, but the range is enormous.
\end{itemize}
Here: establish basic coprimality lemmas and perform a computational sanity check for modest $n$.

\medskip
\noindent\textbf{4) WORK.}

\noindent\emph{Fast reality check (computation).}
For each $n$ up to $200$, I computed $P$ exactly and searched primes $p>p_n$ until finding the smallest $p$ with $P+p$ prime.
\begin{itemize}
\item Sample outputs:
\begin{itemize}
\item $n=10$: $p_n=29$, smallest $p=61$ (8 prime checks).
\item $n=20$: $p_n=71$, smallest $p=103$ (7 checks).
\item $n=50$: $p_n=229$, smallest $p=293$ (12 checks).
\item $n=100$: $p_n=541$, smallest $p=641$ (16 checks), and $P$ has $220$ decimal digits.
\item $n=200$: $p_n=1223$, smallest $p=1619$ (56 checks), and $P$ has $513$ decimal digits.
\end{itemize}
\item No failures were observed in this range.
\end{itemize}

\medskip
\noindent\textbf{Lemma 779.1 (Automatic avoidance of the first $n$ primes).}
Let $n\ge 2$ and $P=\prod_{i=1}^n p_i$. If $p$ is any prime with $p>p_n$, then for every $i\le n$,
\[
P+p\not\equiv 0\pmod{p_i}.
\]
Equivalently, $\gcd(P,P+p)=1$, so $P+p$ is not divisible by any of the primes $p_1,\dots,p_n$.

\noindent\emph{Proof.}
Fix $i\le n$. Since $p_i\mid P$, we have $P\equiv 0\pmod{p_i}$, so $P+p\equiv p\pmod{p_i}$.
Because $p>p_n\ge p_i$ and $p$ is prime, $p_i$ does not divide $p$, hence $p\not\equiv 0\pmod{p_i}$. Therefore $P+p\not\equiv 0\pmod{p_i}$.

Alternatively, if a prime $q$ divides both $P$ and $P+p$, then $q\mid (P+p)-P=p$, hence $q=p$. But $q\mid P$ implies $q\le p_n$, contradicting $p>p_n$. Thus $\gcd(P,P+p)=1$. \hfill $\square$

\medskip
\noindent\textbf{Lemma 779.2 (Prime factors of $P+p$ are large).}
With $n,P$ as above and $p>p_n$ prime, if $P+p$ is composite then every prime divisor $q$ of $P+p$ satisfies $q>p_n$.

\noindent\emph{Proof.}
If $q\mid (P+p)$ and $q\le p_n$, then $q$ must equal one of $p_1,\dots,p_n$, but Lemma~779.1 shows none of these primes divides $P+p$. Hence $q>p_n$. \hfill $\square$

\medskip
\noindent\textbf{5) VERIFICATION.}

\begin{itemize}
\item Lemma~779.1: checked the modular reduction $P\equiv 0\pmod{p_i}$ and the fact that a smaller prime $p_i$ cannot divide a larger prime $p$.
\item Lemma~779.2: immediate from Lemma~779.1; no hidden cases.
\item Computation used deterministic integer arithmetic for $P$ and primality testing for $P+p$ via exact/strong probable-prime routines; results are certificates for the tested range only.
\end{itemize}

\medskip
\noindent\textbf{6) FINAL.} \textbf{UNRESOLVED}

(i) Strongest proved partial result: for any prime $p>p_n$, the number $P+p$ is automatically coprime to $P$ and hence avoids all primes $\le p_n$ (Lemmas~779.1--779.2). Computation found a valid prime $p$ for every $2\le n\le 200$, with smallest $p$ staying on the order of a few thousand.

(ii) First gap: prove existence of at least one prime $p$ with $p_n<p<P$ such that $P+p$ is prime, for every $n\ge 2$.

(iii) Top 3 next moves:
\begin{enumerate}
\item Try to control the number of composite values among $\{P+p: p\text{ prime},\ p_n<p\le X\}$ for some $X\ll P$ using a sieve upper bound, to deduce at least one prime value.
\item Search for a structural reason the smallest such $p$ might be polynomial in $n$ (as Erd\H{o}s expects), perhaps by showing that in a window $p\le n^C$ the values $P+p$ have few small prime factors (Lemma~779.2 already eliminates the first $n$ primes).
\item Extend computations (e.g., to $n=1000$ as in the statement) while recording the growth of the smallest witness $p$ to guess a sharper conjectural rate.
\end{enumerate}

(iv) Minimal counterexample structure (if false): a smallest $n\ge 2$ such that for every prime $p$ with $p_n<p<P$, the integer $P+p$ is composite. By Lemma~779.2, every such composite $P+p$ would have all prime factors $>p_n$, so a counterexample would amount to a ``covering'' of all these values by large prime divisors.
