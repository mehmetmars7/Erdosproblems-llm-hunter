% Erdős Problem #14
% URL: https://www.erdosproblems.com/14

1) \textbf{FORMAL RESTATEMENT}

Let $A\subseteq\mathbb{N}$. For an integer $n\ge 2$, define the number of \emph{unordered} representations of $n$ as a sum of two elements of $A$ by
\[
 r_A(n):=|\{\{a,b\}: a\in A,\ b\in A,\ a\le b,\ a+b=n\}|.
\]
(So $a=b$ is allowed; it contributes one representation.)
Define
\[
 B:=\{n\in\mathbb{N}: r_A(n)=1\}.
\]
The questions are:
\begin{enumerate}
\item[(Q1)] Is it true that for every $\varepsilon>0$ there exists $N_0(\varepsilon)$ such that for all $N\ge N_0(\varepsilon)$,
\[
|\{1,\dots,N\}\setminus B| \gg_{\varepsilon} N^{1/2-\varepsilon}?
\]
\item[(Q2)] Is it possible that $|\{1,\dots,N\}\setminus B|=o(N^{1/2})$ as $N\to\infty$ for some choice of $A$?
\end{enumerate}

Ambiguity note: some authors count ordered representations $(a,b)$ instead of unordered, or forbid $a=b$. The qualitative questions change only by constant factors in typical regimes, but for small $n$ the distinction matters; I proceed with the (common) convention above.

2) \textbf{QUICK LITERATURE/CONTEXT CHECK}

From the problem statement: Erd\H{o}s claimed a construction with $|\{1,\dots,N\}\setminus B|\ll_\varepsilon N^{1/2+\varepsilon}$ for all large $N$ and also infinitely many $N$ with lower bound $\gg_\varepsilon N^{1/3-\varepsilon}$. Erd\H{o}s--Freud proved a finite analogue giving an upper bound $<2^{3/2}N^{1/2}$ for some $A\subseteq\{1,\dots,N\}$.

I do not use these claims in proofs; I only give elementary constraints and small-$N$ computations.

3) \textbf{ATTACK PLAN}

\emph{Proof track (lower bounds on the complement):}
\begin{itemize}
\item Relate the number of uniquely represented sums to the additive energy of $A\cap[1,N]$; collisions among sums should force many $n$ with $r_A(n)\ne 1$.
\item Use extremal counting: the number of available sums is $N$, but the number of pairs in $A\cap[1,N]$ is quadratic; this tension should force either many missing sums (no representations) or many collisions (multiple representations).
\end{itemize}

\emph{Disproof track (constructing small complement):}
\begin{itemize}
\item Try to build $A$ so that most $n\le N$ have exactly one representation, balancing coverage (most $n$ represented) with Sidon-type uniqueness (few collisions).
\item Computationally search for small $N$ optimal sets and look for a pattern that might extend to an infinite construction.
\end{itemize}

4) \textbf{WORK}

\textbf{Lemma 14.1 (local dependence on initial segment).}
Fix $N\ge 1$. For any $n\le N$, whether $n\in B$ (i.e. $r_A(n)=1$) depends only on the finite set $A\cap\{1,\dots,N\}$.

\textbf{Proof.}
If $n\le N$ and $a,b\in A$ satisfy $a+b=n$, then necessarily $a\le n\le N$ and $b\le n\le N$. Thus every representation of $n$ uses only elements of $A\cap[1,N]$.
Conversely, any representation using elements of $A\cap[1,N]$ is a representation using elements of $A$.
Therefore $r_A(n)=r_{A\cap[1,N]}(n)$ for all $n\le N$.
\qed

\textbf{Lemma 14.2 (small-number obstruction via the minimum of $A$).}
Let $m:=\min A$ (assuming $A\ne\emptyset$). Then for every integer $n$ with $1\le n<2m$, one has $r_A(n)=0$, hence $n\notin B$.
In particular,
\[
|\{1,\dots,N\}\setminus B|\ge \min\{N,2m-1\}.
\]

\textbf{Proof.}
If $a,b\in A$, then $a\ge m$ and $b\ge m$, so $a+b\ge 2m$. Thus no integer $n<2m$ can be represented as $a+b$ with $a,b\in A$, i.e. $r_A(n)=0$.
Counting gives the displayed lower bound for the complement.
\qed

\textbf{Lemma 14.3 (pair-counting upper bound in the collision-free regime).}
Fix $N\ge 1$ and set $A_N:=A\cap\{1,\dots,N\}$ with $|A_N|=t$.
If every sum $a+b$ with $a,b\in A_N$ and $a\le b$ is distinct (i.e. $r_{A_N}(n)\le 1$ for all $n$), then
\[
\binom{t+1}{2} \le 2N-1,
\]
hence
\[
 t \le \frac{\sqrt{16N-7}-1}{2} < 2\sqrt{N}.
\]

\textbf{Proof.}
There are exactly $\binom{t+1}{2}$ unordered pairs $\{a,b\}$ with $a,b\in A_N$ and $a\le b$.
Each such pair produces a sum $s=a+b$ in the range $2\le s\le 2N$.
If all these sums are distinct, then the number of pairs is at most the number of possible sums in $[2,2N]$, which is $2N-1$.
This gives $\binom{t+1}{2}\le 2N-1$. Solving the quadratic inequality $t(t+1)\le 4N-2$ yields the stated bound.
\qed

\textbf{FAST REALITY CHECK / COMPUTATION.}
For each $N\le 20$, I brute-forced over all subsets $A\subseteq\{1,\dots,N\}$ and computed
\[
\mathrm{bad}(A,N):=|\{1,\dots,N\}\setminus B|
\]
with $B$ defined via unordered representations $a\le b$.
The minimum value $\min_A \mathrm{bad}(A,N)$ for $N\le 20$ is:
\[
\begin{array}{c|cccccccccccccccccccc}
N & 1&2&3&4&5&6&7&8&9&10&11&12&13&14&15&16&17&18&19&20\\\hline
\min \mathrm{bad}(A,N) & 1&1&1&1&1&1&1&2&2&2&2&2&2&2&2&2&2&3&3&3
\end{array}
\]
Example optimizers found by brute force include:
\begin{itemize}
\item $N=16$: $A=\{1,2,4,6,7,14\}$ gives complement $\{1,8\}$ (size $2$).
\item $N=20$: $A=\{1,3,4,8,9,11\}$ achieves complement size $3$.
\end{itemize}
These are \emph{not} consistent initial segments of a single infinite $A$; they are separately optimized for each $N$.

5) \textbf{VERIFICATION}

\begin{itemize}
\item Lemma~14.1 uses only positivity of elements of $A$.
\item Lemma~14.2: verified the inequality $a+b\ge 2m$ is sharp when $a=b=m$ (allowed in our convention).
\item Lemma~14.3: checked that all sums lie in $[2,2N]$ and the count of possible sums is $2N-1$.
\item Computation: brute force over $2^N$ subsets is exact for $N\le 20$.
\end{itemize}

6) \textbf{FINAL}

\textbf{UNRESOLVED}

(i) Strongest proved partial result: Lemma~14.2 gives a necessary lower bound on the complement in terms of $\min A$, forcing $1\in A$ if one hopes for $o(N)$ complement. Lemma~14.3 gives a sharp constraint in the extreme ``no collisions'' regime. Computationally, optimizing $A\subseteq[1,N]$ separately for each $N\le 20$ yields very small complements (as low as $2$ or $3$).

(ii) First gap (crisp): Either prove that for every infinite $A\subseteq\mathbb{N}$,
\[
|\{1,\dots,N\}\setminus B| \ge c_\varepsilon N^{1/2-\varepsilon}
\quad\text{for all large }N,
\]
for some $c_\varepsilon>0$, or construct a single infinite $A$ with $|\{1,\dots,N\}\setminus B|=o(N^{1/2})$.

(iii) Top 3 next moves:
\begin{enumerate}
\item Search for a consistent infinite construction: attempt to extend optimized small-$N$ sets by a backtracking algorithm imposing that $A\cap[1,N]$ is the restriction of $A\cap[1,N+1]$.
\item Develop an energy-based inequality: express counts of $r_A(n)=0,1,\ge 2$ in terms of $|A\cap[1,N]|$ and additive energy, then lower bound the number of ``bad'' $n$.
\item Compute exact minima for larger $N$ via ILP/SAT (variables indicating membership of each $a\le N$) to get better evidence about growth of $\min_A \mathrm{bad}(A,N)$.
\end{enumerate}

(iv) Minimal counterexample structure: a counterexample to Q1 (i.e. showing $|[1,N]\setminus B|=o(\sqrt{N})$ is possible) would require an infinite $A$ with $1\in A$ and such that for most $n\le N$ there is exactly one pair $a\le b$ in $A\cap[1,N]$ summing to $n$. This simultaneously demands (a) near-complete coverage of $[1,N]$ by pair-sums and (b) strong Sidon-type control on collisions among sums in the range $\le N$.


