% Erdos Problem #970
% URL: https://www.erdosproblems.com/970

Let $h(k)$ be Jacobsthal's function, defined to as the minimal $m$ such that, if $n$ has at most $k$ prime factors, then in any set of $m$ consecutive integers there exists an integer coprime to $n$. Determine the order of magnitude of $h(k)$. In particular, is it true that\[h(k) \ll k^2?\] That $h(k)\ll k^2$ is a conjecture of Jacobsthal. Iwaniec \cite{Iw78} proved\[h(k) \ll (k\log k)^2.\]The best lower bound known is\[h(k) \gg \frac{(\log k)(\log\log\log k)}{(\log\log k)^2}k,\]due to Ford, Green, Konyagin, Maynard, and Tao \cite{FGKMT18}. This is a more general form of the function considered in [687] . References [FGKMT18] Ford, Kevin and Green, Ben and Konyagin, Sergei and Maynard, James and Tao, Terence, Long gaps between primes . J. Amer. Math. Soc. (2018), 65-105. [Iw78] Iwaniec, Henryk, On the problem of {J}acobsthal . Demonstratio Math. (1978), 225--231.

%Erdos problem 970

\medskip
\noindent\textbf{1) FORMAL RESTATEMENT}\\
Fix $k\in\mathbb{N}$. For an integer $n\ge 1$, let
\[
 j(n) := \min\{m\in\mathbb{N}: \text{every interval of $m$ consecutive integers contains an integer coprime to $n$}\}.
\]
(Here ``interval of $m$ consecutive integers'' means a set of the form $\{t+1,t+2,\dots,t+m\}$ for some $t\in\mathbb{Z}$.  ``Coprime'' means $\gcd(\cdot,\cdot)=1$.)
Define
\[
 h(k) := \max\{ j(n): n\ge 1\text{ and $n$ has at most $k$ prime factors}\}.
\]
\textbf{Ambiguity.} ``At most $k$ prime factors'' can mean either $\Omega(n)\le k$ (counted with multiplicity) or $\omega(n)\le k$ (distinct prime divisors).  In either convention, the value of $j(n)$ depends only on the distinct prime divisors of $n$ (Lemma~\ref{lem:970-rad}), so $h(k)$ is unchanged by replacing $\Omega$ with $\omega$.

The question asks for the order of magnitude of $h(k)$ as $k\to\infty$; in particular whether there exists an absolute constant $C$ such that $h(k)\le C k^2$ for all sufficiently large $k$.

\medskip
\noindent\textbf{2) QUICK LITERATURE/CONTEXT CHECK}\\
The problem statement records: Jacobsthal conjectured $h(k)\ll k^2$; Iwaniec proved $h(k)\ll (k\log k)^2$; and the best known lower bound is linear in $k$ up to slowly varying factors.  I do not use any additional literature facts beyond what is explicitly in the problem statement.

\medskip
\noindent\textbf{3) ATTACK PLAN}\\
\textbf{Proof track:} (a) reduce to squarefree $n$ and interpret $j(n)$ as a longest ``covered'' interval by residue classes $0\pmod p$; (b) try to show extremizers use small primes; (c) seek a universal upper bound in terms of $k$ by bounding the longest run covered by $k$ prime moduli.

\textbf{Disproof track:} try to construct, for some $k$, an integer $n$ with $\le k$ prime factors and a run of length $>c k^2$ in which every integer shares a prime factor with $n$ (a covering-system style construction).

I only reach structural lemmas + small-case computation; the main asymptotic bound remains open here.

\medskip
\noindent\textbf{4) WORK}\\
\textbf{FAST REALITY CHECK (small cases).}  For squarefree $n$ with prime set $P(n)$, $j(n)$ can be computed by scanning one full period mod $n$ (Lemma~\ref{lem:970-period}).  I computed $j(n)$ for $n$ equal to the product of the first $k$ primes $P_k:=\prod_{i=1}^k p_i$ for $1\le k\le 8$:
\[
\begin{array}{c|cccccccc}
 k & 1 & 2 & 3 & 4 & 5 & 6 & 7 & 8\\\hline
 P_k & 2 & 6 & 30 & 210 & 2310 & 30030 & 510510 & 9699690\\
 j(P_k) & 2 & 4 & 6 & 10 & 14 & 22 & 26 & 34
\end{array}
\]
Moreover, in a brute-force search over all squarefree $n$ whose prime divisors are contained in $\{2,3,5,7,11,13,17,19\}$, for each $k\le 6$ the maximum $j(n)$ among those with $\omega(n)=k$ was achieved by $P_k$ and matched the table above.

\medskip
\noindent\textbf{Lemma 970.1 (reduction to squarefree).}\label{lem:970-rad}
For every integer $n\ge 1$, letting $\operatorname{rad}(n)=\prod_{p\mid n}p$ be the squarefree kernel, one has
\[
 j(n)=j(\operatorname{rad}(n)).
\]
In particular, $h(k)$ is unchanged whether ``$n$ has at most $k$ prime factors'' is interpreted via $\Omega(n)$ or $\omega(n)$.

\noindent\textbf{Proof.}
For any integer $x$, $\gcd(x,n)=1$ if and only if no prime divisor of $n$ divides $x$, which is equivalent to $\gcd(x,\operatorname{rad}(n))=1$.
Therefore, for any $m$ and any interval $\{t+1,\dots,t+m\}$, the statement ``there exists $x$ in the interval coprime to $n$'' is equivalent to ``there exists $x$ in the interval coprime to $\operatorname{rad}(n)$''.  Taking the minimal such $m$ gives $j(n)=j(\operatorname{rad}(n))$.
Finally, if $\Omega(n)\le k$ then certainly $\omega(\operatorname{rad}(n))\le k$, and if $\omega(n)\le k$ then also $\Omega(n)\le k$ holds for $n$ squarefree; since maximizing $j(n)$ may be restricted to squarefree $n$, the two conventions agree for $h(k)$. \qed

\medskip
\noindent\textbf{Lemma 970.2 (periodic gap formula).}\label{lem:970-period}
Let $n\ge 2$ be squarefree.  Define the length-$n$ periodic ``bad'' indicator
\[
 b(r)=\begin{cases}1,&\gcd(r,n)>1,\\0,&\gcd(r,n)=1,\end{cases}\qquad (r\in\mathbb{Z}).
\]
Let $L(n)$ be the maximum length of a cyclic run of $1$'s in $\{b(0),b(1),\dots,b(n-1)\}$ (allowing wrap-around from $n-1$ to $0$).  Then
\[
 j(n)=L(n)+1.
\]

\noindent\textbf{Proof.}
Since $b(r)$ depends only on $r\bmod n$, the pattern of integers coprime to $n$ is periodic with period $n$.

If there is a run of $L$ consecutive integers all not coprime to $n$, then any $m\le L$ fails the defining property of $j(n)$, so $j(n)\ge L+1$.
Conversely, if $L(n)$ is the longest cyclic run of residues mod $n$ with $\gcd(\cdot,n)>1$, then any $L(n)+1$ consecutive residues mod $n$ must include at least one residue coprime to $n$.  Lifting back to integers, any interval of $L(n)+1$ consecutive integers contains an integer coprime to $n$, so $j(n)\le L(n)+1$.
Combining the two inequalities yields $j(n)=L(n)+1$. \qed

\medskip
\noindent\textbf{Lemma 970.3 (CRT lower bound $h(k)\ge k+1$).}\label{lem:970-crt-lower}
For every $k\ge 1$, one has $h(k)\ge k+1$.

\noindent\textbf{Proof.}
Fix distinct primes $p_1,\dots,p_k$ and let $n=\prod_{i=1}^k p_i$, which is squarefree with $\omega(n)=k$.  By the Chinese remainder theorem, there exists an integer $t$ such that
\[
 t+i\equiv 0\pmod{p_i}\qquad\text{for each }i=1,2,\dots,k.
\]
Then for each $i$ the integer $t+i$ is divisible by $p_i$, hence $\gcd(t+i,n)>1$.
Therefore the block $\{t+1,\dots,t+k\}$ contains no integer coprime to $n$, showing $j(n)\ge k+1$.
Since $h(k)$ is the maximum of $j(n)$ over all $n$ with $\le k$ prime factors, $h(k)\ge j(n)\ge k+1$. \qed

\medskip
\noindent\textbf{5) VERIFICATION}\\
\textbf{Quantifiers/boundaries.} Lemma~\ref{lem:970-rad} covers $n\ge 1$; for $n=1$ one has $j(1)=1$, which is compatible but irrelevant for maximizing $h(k)$.
Lemma~\ref{lem:970-period} assumes $n$ squarefree; by Lemma~\ref{lem:970-rad} this is without loss for $h(k)$.
Lemma~\ref{lem:970-crt-lower} uses distinct primes; the CRT system is consistent because the moduli are coprime.

\textbf{Computation sanity.} For $k\le 8$ the computed $j(P_k)$ values satisfy $j(P_k)\ge k+1$ as guaranteed by Lemma~\ref{lem:970-crt-lower}.  The computation used the periodic formula of Lemma~\ref{lem:970-period}.

\medskip
\noindent\textbf{6) FINAL}\\
\textbf{UNRESOLVED}

(i) \textbf{Strongest proved partial result.} One may restrict to squarefree $n$ (Lemma~\ref{lem:970-rad}) and compute $j(n)$ as a maximum cyclic bad run modulo $n$ (Lemma~\ref{lem:970-period}).  Universally $h(k)\ge k+1$ by a CRT construction (Lemma~\ref{lem:970-crt-lower}).  Small-case computation gives $j(\prod_{i=1}^k p_i)=(2,4,6,10,14,22,26,34)$ for $1\le k\le 8$.

(ii) \textbf{First gap (crisp).} Prove (or disprove) that there exists an absolute $C$ such that for all sufficiently large $k$, $h(k)\le Ck^2$.

(iii) \textbf{Top 3 next moves.}
\begin{itemize}
\item Prove a monotonicity/exchange lemma: replacing a prime factor of $n$ by a larger prime does not increase $j(n)$; if true, this would reduce $h(k)$ to $j(P_k)$ with $P_k$ the product of the first $k$ primes.
\item Translate the problem to a covering-system formulation: a run of length $L$ corresponds to residue classes $0\pmod p$ covering an interval of length $L$; attempt to bound maximum cover length with $k$ primes.
\item Compute $h(k)$ exactly for moderate $k$ by optimizing over prime sets and locating extremal runs (using Lemma~\ref{lem:970-period}), to guide conjectures about growth.
\end{itemize}

(iv) \textbf{Minimal counterexample structure.} A disproof of $h(k)\ll k^2$ would require some $k$ and a squarefree $n$ with $\omega(n)=k$ for which there is a block of length $\gg k^2$ covered by the union of residue classes $0\pmod p$ for $p\mid n$; heuristically such an $n$ would likely involve many small primes to make the bad set dense and a highly structured alignment of their multiples to produce an unusually long covered interval.


