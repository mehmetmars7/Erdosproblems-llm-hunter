\section*{Problem 373}

\subsection*{1) FORMAL RESTATEMENT}
For integers $n\ge 1$ and $k\ge 2$, consider solutions in integers
\[
 n-1>a_1\ge a_2\ge \cdots \ge a_k\ge 2
\]
of the factorial Diophantine equation
\[
 n! = a_1!\,a_2!\cdots a_k!.
\]
The problem asks to prove that there are only finitely many such solutions $(n; a_1,\dots,a_k)$.

\subsection*{2) KEY DEFINITIONS AND KNOWN FACTS}
\begin{itemize}[leftmargin=2em]
\item Factorial: $n!=1\cdot2\cdots n$.
\item For any prime $p$, if $p\le n$ then $p\mid n!$.
\item If $a<n$ and $p$ is a prime with $a<p\le n$, then $p$ divides $n!$ but does not divide $a!$.
\end{itemize}

\paragraph{Quick literature/context check (browsing available).}
The problem is listed as open as of late 2025. It is conjectured that the only solutions are the ``Hickerson list''
\(9!=2!\,3!\,3!\,7!\), \(10!=6!\,7!\), \(10!=3!\,5!\,7!\), \(16!=14!\,5!\,2!\).
Conditional on the ABC conjecture, Luca proved there are only finitely many solutions. Unconditionally, Luca proved that the set of $n\le x$ admitting a nontrivial solution is very sparse (subexponential in $x$ in an explicit sense). In the case $k=2$ (and in fact for general $k$), there are strong bounds forcing $a_1$ to be extremely close to $n$ (e.g. $a_1\ge n - C\log\log n$ with an explicit constant $C$), and numerical work has ruled out additional $k=2$ solutions up to astronomically large $n$.

\subsection*{3) ATTEMPTED SOLUTION}
\textbf{PHASE 1 (computational reality check).} A brute-force search (by recursive factor selection with pruning) for nontrivial solutions with $n\le 30$ found exactly the classical solutions
\[
\begin{aligned}
9! &= 7!\,3!\,3!\,2!,\\
10! &= 7!\,6!,\\
10! &= 7!\,5!\,3!,\\
16! &= 14!\,5!\,2!,
\end{aligned}
\]
and no others in that range.

\medskip
\textbf{PHASE 2 (proof attempt).} I attempted to prove finiteness by forcing $a_1$ to be extremely close to $n$ via prime divisibility considerations. One can show that the interval $(a_1,n]$ contains no primes, hence $a_1$ must be at least the largest prime $\le n$. This gives a nontrivial constraint (e.g. $a_1>n/2$ by Bertrand), but it does not by itself rule out infinitely many $n$ because large prime gaps exist.

\subsection*{4) DETAILED WORK (with complete reasoning)}
\subsubsection*{4.1 A proved structural constraint: $(a_1,n]$ is prime-free}
\begin{lemma}\label{lem:primefree}
Assume $n! = a_1!\cdots a_k!$ with $n-1>a_1\ge a_2\ge\cdots\ge a_k\ge2$. Then there is no prime $p$ with $a_1<p\le n$.
\end{lemma}

\begin{proof}
Suppose for contradiction that there exists a prime $p$ with $a_1<p\le n$.

Because $p\le n$, we have $p\mid n!$.

On the other hand, for each $i\in\{1,\dots,k\}$ we have $a_i\le a_1<p$, so $p$ does not divide $a_i!$ (since all factors of $a_i!$ are at most $a_i<p$). Hence $p\nmid a_i!$ for all $i$, so $p\nmid a_1!\cdots a_k!$.

This contradicts $n! = a_1!\cdots a_k!$, because the left-hand side is divisible by $p$ while the right-hand side is not.
\end{proof}

\begin{corollary}\label{cor:a1_ge_prime}
Let $p^*(n)$ denote the largest prime $\le n$ (for $n\ge2$). Any solution must satisfy $a_1\ge p^*(n)$.
\end{corollary}

\begin{proof}
If $a_1<p^*(n)$, then the prime $p^*(n)$ lies in $(a_1,n]$, contradicting Lemma~\ref{lem:primefree}.
\end{proof}

\subsubsection*{4.2 A fully proved numerical consequence: $a_1>n/2$ for $n\ge 4$}
\begin{corollary}\label{cor:a1_gt_half}
For any solution with $n\ge 4$, one has $a_1>\frac{n}{2}$.
\end{corollary}

\begin{proof}
By Bertrand's postulate, for every integer $m\ge 2$ there exists a prime $p$ with $m<p<2m$.
Apply this with $m=\lfloor n/2\rfloor\ge 2$ (since $n\ge4$). Then there exists a prime $p$ with
\[
\lfloor n/2\rfloor < p < 2\lfloor n/2\rfloor \le n.
\]
In particular $p\le n$ and $p>n/2$. Hence $p^*(n)\ge p>n/2$.
By Corollary~\ref{cor:a1_ge_prime}, $a_1\ge p^*(n)>n/2$.
\end{proof}

\subsubsection*{4.3 Why this does not yet imply finiteness}
Lemma~\ref{lem:primefree} shows $n-a_1$ is at most the length of the prime gap immediately preceding $n$. Prime gaps are unbounded, so this constraint alone does not rule out infinitely many solutions.

To finish a proof of finiteness one seemingly needs an additional ingredient connecting the factorial-product structure to \emph{large prime factors} of products of short intervals (as suggested by known reductions to bounds on $P(n(n\pm 1))$), or to show that the prime-free interval constraint forces $n-a_1$ into a regime incompatible with balancing the remaining factorial factors.

\subsection*{5) VERIFICATION}
\begin{itemize}[leftmargin=2em]
\item Lemma~\ref{lem:primefree} uses only basic divisibility and is correct: a prime $p$ in $(a_1,n]$ divides $n!$ but cannot divide any $a_i!$ because all $a_i<p$.
\item Corollary~\ref{cor:a1_gt_half} follows from Bertrand's postulate; the inequality chain is checked explicitly.
\item The computational solutions listed in Phase~1 satisfy $n-1>a_1$ and $a_k\ge2$.
\end{itemize}

\subsection*{6) FINAL}
\textbf{UNRESOLVED.}

\smallskip
\noindent\textbf{(i) Strongest fully proved partial result obtained here.}
Lemma~\ref{lem:primefree} and its corollaries: any solution forces the interval $(a_1,n]$ to contain no primes, hence $a_1\ge p^*(n)$ and in particular $a_1>n/2$ for $n\ge4$.

\noindent\textbf{(ii) Weakest specific gap preventing a full proof.}
Even though $(a_1,n]$ must be prime-free, prime gaps can be large; I do not have an argument that combines this with the remaining factorial-product constraints to preclude infinitely many solutions.

\noindent\textbf{(iii) What lemmas would be needed to complete the proof.}
One plausible route would be to prove that for all sufficiently large $n$, the product
\(\prod_{j=a_1+1}^{n} j\)
necessarily has a prime factor exceeding $a_1$ unless $n-a_1$ is extremely small in a way incompatible with the factorial decomposition. Any theorem of the form
\[
\max_{a<j\le n} P(j) > a\quad\text{whenever } n-a \text{ lies in some explicit range}
\]
would begin to convert Lemma~\ref{lem:primefree} into finiteness.

\noindent\textbf{(iv) Minimal explicit computation supporting finiteness.}
The search up to $n\le 30$ finds only the four classical solutions listed in \textbf{PHASE 1}.

\subsection*{7) COMPLETION ESTIMATE (MANDATORY)}
\textbf{COMPLETION: 40\%}.

