\section*{Problem 635. Classify those triangles which can only be cut into a square number of congruent triangles.}

\subsection*{1) [Formal restatement]}
A (Euclidean) triangle $T$ is said to admit an \emph{$N$-triangle congruent dissection} if $T$ can be partitioned into $N$ pairwise congruent triangles (allowing reflections/rotations).  The problem asks for a classification of those triangles $T$ such that
\[
  \{N\ge 1 : T \text{ has an } N\text{-triangle congruent dissection}\}\subseteq \{n^2 : n\in\mathbb{Z}_{\ge 1}\}.
\]
Equivalently: $T$ can be dissected into congruent triangles only in square cardinalities.

\subsection*{2) [Quick literature/context check (if browsing is available)]}
The prompt notes a theorem of Soifer: if the angles and side-lengths of $T$ are both ``integrally independent'' (in the sense used in that literature), then every congruent-triangle dissection of $T$ has a square number of tiles.

A separate but related line of work studies which $N$ occur at all (Problem 634 below) and which angle configurations force strong arithmetic restrictions on $N$.

No complete classification is known to me from the materials indicated in the prompt; what follows are (i) a fully explicit construction showing that \emph{every} triangle always has square dissections, and (ii) a structured reduction of the classification problem to obstructions/invariants commonly used in triangle-tiling theory.

\subsection*{3) [Attack plan]}
\begin{enumerate}
\item Prove explicitly that every triangle can be dissected into $n^2$ congruent triangles for every $n\ge 1$ (this explains why the ``square-only'' condition is about \emph{excluding non-squares}).
\item Identify a robust way to encode a congruent-triangle tiling (edge vectors/angle types/boundary word), and list the standard invariants used to force arithmetic restrictions on $N$ (lattice/edge-count equations, Dehn-like invariants, coloring arguments).
\item Explain why a full classification seems to require controlling when such invariants admit non-square solutions.
\end{enumerate}

\subsection*{4) [Work]}
\paragraph{4.1. A completely explicit $n^2$-tiling for every triangle.}
Let $\triangle ABC$ be any triangle and fix an integer $n\ge 1$.

Let $M$ be the midpoint of segment $BC$. Let $A'$ be the image of $A$ under the half-turn (rotation by $\pi$) about $M$. Then $M$ is the midpoint of both $AA'$ and $BC$, hence the quadrilateral $ABA'C$ is a parallelogram (a quadrilateral whose diagonals bisect each other).

Now subdivide the parallelogram $ABA'C$ into an $n\times n$ grid of congruent parallelograms as follows:
\begin{itemize}
\item Divide side $AB$ into $n$ equal segments and draw through the division points the lines parallel to $AC$.
\item Divide side $AC$ into $n$ equal segments and draw through the division points the lines parallel to $AB$.
\end{itemize}
This produces $n^2$ congruent small parallelograms, each a translate of the fundamental cell with side vectors $\frac1n\overrightarrow{AB}$ and $\frac1n\overrightarrow{AC}$.

In each small parallelogram, draw the diagonal parallel to $BC$ (equivalently: the diagonal with direction $\overrightarrow{AC}-\overrightarrow{AB}$). This splits each small parallelogram into two congruent triangles. Therefore the whole parallelogram $ABA'C$ is partitioned into $2n^2$ congruent triangles.

Finally observe that the diagonal $BC$ of the large parallelogram is itself a union of these small diagonals: in the $n\times n$ grid, $BC$ runs along the ``anti-diagonal'' strip of cells and coincides with the chosen cell-diagonal in each such cell. Hence $BC$ is a union of tile edges, so the half-parallelogram triangle $\triangle ABC$ is a union of whole tiles.

Since $\triangle ABC$ has half the area of the parallelogram and all tiles have equal area, exactly half of the $2n^2$ tiles lie in $\triangle ABC$. Thus $\triangle ABC$ is dissected into precisely $n^2$ congruent triangles.

\paragraph{4.2. What remains for a classification.}
The above shows that every triangle has square congruent-triangle dissections. Therefore, to classify ``square-only'' triangles, one must understand when a triangle admits \emph{some} non-square congruent-triangle dissection.

A congruent-triangle dissection has two intertwined arithmetic features:
\begin{itemize}
\item \emph{Angle arithmetic.} Each angle of $T$ is a sum of tile angles meeting at boundary vertices. This forces the angles of $T$ to lie in the additive semigroup generated by the tile angles. When the angles of $T$ have strong irrational-independence properties, this can prevent nontrivial angle decompositions.
\item \emph{Edge-length arithmetic.} Along each side of $T$, tile edges appear in some order, giving linear relations between side lengths of $T$ and side lengths of the tile with nonnegative integer coefficients. For triangles with strong linear independence among side lengths, such relations can force very rigid tiling patterns.
\end{itemize}

In particular, the Soifer condition cited in the prompt (integral independence of both angles and side lengths) is precisely of the type that blocks non-square tilings via these arithmetic constraints.

\subsection*{5) [Verification]}
\begin{itemize}
\item The $n^2$-construction is checked by: (i) producing $n^2$ congruent parallelogram cells by parallel-line subdivision; (ii) bisecting each cell by a fixed-direction diagonal to get $2n^2$ congruent triangles; (iii) observing the diagonal $BC$ is a union of chosen cell diagonals so the half-parallelogram is tiled; (iv) counting tiles by area to obtain exactly $n^2$.
\item No step uses special properties of $\triangle ABC$ beyond being a nondegenerate triangle.
\end{itemize}

\subsection*{6) [Final]}
\textbf{LABEL: UNRESOLVED}\\
\textbf{SUBLABEL: Partial progress (explicit $n^2$ tiling) + classification remains open.}\\
I proved constructively that every triangle admits dissections into $n^2$ congruent triangles for every $n\ge 1$ (hence ``square-only'' means ``no non-square tilings''). A complete classification of triangles that forbid all non-square congruent-triangle dissections is not obtained here; the problem reduces to understanding when the standard angle/edge arithmetic constraints admit a non-square solution.

\subsection*{7) [Completion estimate]}
About \emph{35\%} complete: the universal $n^2$-construction is fully proved and the main obstruction mechanisms are laid out, but a full necessary-and-sufficient classification is not reached.


