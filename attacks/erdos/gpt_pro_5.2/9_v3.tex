\section{Erd\H{o}s Problem \#9 (Round 3): certification of an explicit element of $A$ and closure of Round-2 gaps}

\subsection{1) ROUND-3 OBJECTIVE}

We pursue \textbf{UNRESOLVED (BUT STRICTLY ADVANCED)} progress, i.e. neither a proof nor a disproof of
\[\overline d(A)>0\]
(which remains open), but instead a \emph{gap-free certification} of the explicit odd integer $N_0\in A$ introduced in Round~2.

Concretely, Round~2 exhibited a specific $420$-digit odd integer $N_0$ and sketched why $N_0\in A$ using Crocker's machinery.
The remaining gaps were:
(i) a black-box use of Crocker's Lemma~II (the ``Fermat divisor'' obstruction to $p+2^a+2^b$),
(ii) non-reproducible verification of the congruence data (orders and CRT solution), and
(iii) an implicit check that the listed congruence system is overlapping.

In this round we close (i)--(iii) by supplying a clean proof of the needed special case of Crocker's Lemma~II and by giving a deterministic computational certificate for all numerical conditions.

\subsection{2) ROUND-2 FOUNDATION USED}

We rely on the following Round~2 items (statements only; proofs are not repeated).

\begin{itemize}
\item Definition of
\[A:=\{\text{odd }n\ge 1: \nexists\text{ prime }p,\ k,\ell\ge 0\ \text{s.t. }n=p+2^k+2^\ell\}.
\]
\item Round~2 Lemma~4 (reduction): if an odd $n$ is \emph{not} of the form $p+2^a$ with $a\ge 1$ and \emph{not} of the form $p+2^a+2^b$ with distinct $a,b\ge 1$, then $n\in A$.
\item Round~2 Proposition~5: an explicit odd integer $N_0$ defined as the least positive solution of a certain CRT system (displayed again below).
\end{itemize}

\subsection{3) NEW INSIGHT / TOOL (ROUND-3)}

\begin{itemize}
\item \textbf{(New proof input)} We give a self-contained proof (in \S\ref{sec:lemmaII-proof}) of the special case of Crocker's Lemma~II needed for $N_0$, namely: if $n\ge 6$ and $N=w\prod_{i=0}^{n-1}B_i\le 2^{2^n}-1$ with $w\equiv 1\pmod{16}$ and each $B_i\mid (2^{2^i}+1)$, then $N\neq p+2^a+2^b$ for any prime $p$ and distinct $a,b\ge 1$.
\item \textbf{(New certified computation)} We provide a deterministic computational certificate (in \S\ref{sec:certificate}) verifying:
(1) the congruence system is overlapping modulo its l.c.m. $720$;
(2) the chosen primes $p_i$ are indeed prime and satisfy $\operatorname{ord}_{p_i}(2)=m_i$;
(3) the CRT solution equals the stated decimal $N_0$ and meets all congruences, including $N_0<2^{2048}-1$.
\end{itemize}

\subsection{4) ATTACK PLAN (ROUND-3)}

\paragraph{Round-2 gaps to close.}

\begin{enumerate}
\item \emph{Gap A:} replace the black-box use of Crocker Lemma~II by an explicit proof.
\item \emph{Gap B:} verify the numerical congruence data (primality, multiplicative orders, CRT consistency).
\item \emph{Gap C:} verify that the displayed congruence system covers all positive exponents (``overlapping'').
\end{enumerate}

\paragraph{Plan.}

\begin{enumerate}
\item Prove a Fermat-divisor lemma: if $a>b\ge 1$ and $r=v_2(a-b)$, then $2^{2^r}+1$ divides $2^a+2^b$.
\item Prove the needed Lemma~II variant (\S\ref{sec:lemmaII-proof}) using the above divisibility and a mod-$16$ contradiction to prevent the quotient from being prime.
\item Prove $N_0$ avoids $p+2^d$ for $d\ge 1$ by the overlapping congruence system and the chosen primes of exact order.
\item Apply Round~2 Lemma~4 to conclude $N_0\in A$.
\item Supply the computational certificate for all arithmetic facts used in the construction.
\end{enumerate}

\subsection{5) WORK (ROUND-3)}

\subsubsection{5.1 The explicit CRT integer $N_0$ (from Round 2)}

Let $F_i:=2^{2^i}+1$ denote the $i$th Fermat number.
Let
\[q:=45592577,\qquad G_{10}:=\frac{F_{10}}{q}=\frac{2^{1024}+1}{45592577},\qquad M_1:=\frac{2^{2048}-1}{G_{10}}=(2^{1024}-1)\,q.
\]
(Thus $M_1=\prod_{i=0}^{9}F_i\cdot q=\prod_{i=0}^{10}B_i$ with $B_i:=F_i$ for $0\le i\le 9$ and $B_{10}:=q$.)

Consider the following list of congruences $(a_i,m_i)$ (an ``overlapping'' system) and primes $p_i$:
\[
\begin{array}{r|r|r}
(a_i,m_i) & p_i & \operatorname{ord}_{p_i}(2)=m_i\\\hline
(0,3)&7&3\\
(0,5)&31&5\\
(1,9)&73&9\\
(1,10)&11&10\\
(8,12)&13&12\\
(8,15)&151&15\\
(4,18)&19&18\\
(7,20)&41&20\\
(5,24)&241&24\\
(29,30)&331&30\\
(2,36)&37&36\\
(14,36)&109&36\\
(17,40)&61681&40\\
(34,45)&48691&45\\
(43,45)&631&45\\
(13,48)&673&48\\
(37,48)&811&48\\
(16,60)&1321&60\\
(19,60)&61&60\\
(26,72)&87211&72\\
(62,72)&233&72\\
(52,90)&5419&90\\
(37,120)&2521&120\\
(49,144)&1153&144\\
(121,144)&6700417&144\\
(103,180)&1303&180\\
(106,180)&118831&180\\
(229,360)&19231&360
\end{array}
\]
Let $p_{29}:=8191=2^{13}-1$, and choose $c:=1\pmod{8191}$.

Define $N_0$ to be the least positive integer satisfying the simultaneous congruences
\begin{align}
N_0&\equiv 2^{a_i}\pmod{p_i}\qquad (1\le i\le 28),\label{eq:CRTpi}\\
N_0&\equiv c\pmod{8191},\label{eq:CRT8191}\\
N_0&\equiv 0\pmod{M_1},\label{eq:CRTM1}\\
N_0&\equiv -1\pmod{16}.\label{eq:CRT16}
\end{align}

Round~2 recorded the decimal expansion
\[
\begin{aligned}
N_0&=2155030764807612889092586232311054332571447677574018620949657032878977024388409172520152171922192223755\\
&\qquad 239877644559681918646290763701248855934161553200159702158210844894469114556244780738843458837097663.
\end{aligned}
\]

\subsubsection{5.2 A Fermat divisibility lemma}

\begin{lemma}[Fermat divisor of $2^a+2^b$]
\label{lem:fermat-div}
Let $a>b\ge 1$ and write $a-b=2^r t$ with $t$ odd, i.e. $r=v_2(a-b)$. Then
\[F_r=2^{2^r}+1\ \mid\ 2^{a-b}+1\ \mid\ 2^a+2^b.
\]
Consequently, if $B_r\mid F_r$, then $B_r\mid (2^a+2^b)$.
\end{lemma}

\begin{proof}
Since $2^{2^r}\equiv -1\pmod{F_r}$, raising to the odd power $t$ gives
\[(2^{2^r})^t\equiv (-1)^t\equiv -1\pmod{F_r}.
\]
But $(2^{2^r})^t=2^{2^r t}=2^{a-b}$, so $2^{a-b}\equiv -1\pmod{F_r}$ and hence $F_r\mid (2^{a-b}+1)$.
Multiplying by $2^b$ yields $F_r\mid 2^b(2^{a-b}+1)=2^a+2^b$.
If $B_r\mid F_r$, then also $B_r\mid (2^a+2^b)$.
\end{proof}

\subsubsection{5.3 A mod-$16$ obstruction lemma (Crocker Lemma~II, proved)}
\label{sec:lemmaII-proof}

We now prove the exact Lemma~II input needed for the $p+2^a+2^b$ part of $N_0\in A$.

\begin{lemma}[Prime factors of Fermat numbers are $1\bmod 2^{i+2}$]
\label{lem:fermat-prime-cong}
Let $i\ge 2$ and let $p$ be an odd prime divisor of the Fermat number $F_i=2^{2^i}+1$.
Then
\[p\equiv 1\pmod{2^{i+2}}.
\]
Consequently, every divisor $B_i\mid F_i$ satisfies $B_i\equiv 1\pmod{16}$.
\end{lemma}

\begin{proof}
Since $p\mid 2^{2^i}+1$, we have $2^{2^i}\equiv -1\pmod p$, hence $2^{2^{i+1}}\equiv 1\pmod p$.
Thus the multiplicative order of $2$ modulo $p$ is exactly $2^{i+1}$, so $2^{i+1}\mid (p-1)$.
Write
\[p-1=2^{i+1}s\qquad (s\in\mathbb{N}).
\]
Because $i\ge 2$, we have $8\mid 2^{i+1}$ and hence $p\equiv 1\pmod 8$.
By the standard quadratic-reciprocity criterion for $2$,
$p\equiv 1\pmod 8$ implies that $2$ is a quadratic residue modulo $p$, i.e.
\begin{equation}
\label{eq:two-qr}
2^{(p-1)/2}\equiv 1\pmod p.
\end{equation}
On the other hand,
\[
2^{(p-1)/2}=2^{2^is}=(2^{2^i})^s\equiv (-1)^s\pmod p.
\]
Combining with \eqref{eq:two-qr} gives $(-1)^s\equiv 1\pmod p$, hence $s$ is even.
Therefore $2^{i+2}\mid (p-1)$, i.e. $p\equiv 1\pmod{2^{i+2}}$.

Finally, if $B_i\mid F_i$ and $i\ge 2$, then every prime divisor of $B_i$ is $\equiv 1\pmod{2^{i+2}}$, hence in particular $\equiv 1\pmod{16}$; so $B_i\equiv 1\pmod{16}$.
\end{proof}

\begin{lemma}[Crocker-type lemma, proved]
\label{lem:crockerII}
Fix an integer $n\ge 6$.
Let integers $B_0,\dots,B_{n-1}$ satisfy $B_i>1$ and $B_i\mid (2^{2^i}+1)$ for each $i$.
Let $w\equiv 1\pmod{16}$ and define
\[N:=w\prod_{i=0}^{n-1}B_i.
\]
Assume $N\le 2^{2^n}-1$.
Then $N$ cannot be written as
\[N=p+2^a+2^b\qquad\text{with $p$ prime and $a,b\ge 1$ distinct.}
\]
\end{lemma}

\begin{proof}
Assume for contradiction that $N=p+2^a+2^b$ with $a>b\ge 1$ and $p$ prime.
Then
\begin{equation}
\label{eq:diff}
N-2^a-2^b=p>0.
\end{equation}

\emph{Step 1: find a nontrivial divisor of $N-2^a-2^b$.}
Write $a-b=2^r t$ with $t$ odd; thus $r=v_2(a-b)$.
Since $a<2^n$ (because $2^a<N\le 2^{2^n}-1$), we have $a-b<2^n$, hence $2^r\le a-b<2^n$, so $r\le n-1$.
By Lemma~\ref{lem:fermat-div}, $B_r\mid (2^a+2^b)$.
But also $B_r\mid N$ since $B_r$ is a factor of the product defining $N$.
Therefore
\[B_r\mid (N-(2^a+2^b))=p.
\]
Since $p$ is prime and $B_r>1$, this forces
\begin{equation}
\label{eq:primeequals}
p=B_r.
\end{equation}

\emph{Step 2: show $N-2^a-2^b\neq B_r$ by a mod-$16$ contradiction.}
Suppose (as forced by \eqref{eq:primeequals}) that
\begin{equation}
\label{eq:assumeBr}
N-2^a-2^b=B_r.
\end{equation}
Then
\begin{equation}
\label{eq:N-as-sum}
N=2^a+2^b+B_r.
\end{equation}

We claim $N\equiv -1\pmod{16}$.
Indeed, by Lemma~\ref{lem:fermat-prime-cong} every divisor $B_i\mid (2^{2^i}+1)$ with $i\ge 2$ satisfies $B_i\equiv 1\pmod{16}$.
Also $B_0\mid 3$ and $B_1\mid 5$, hence $B_0\equiv 3\pmod{16}$ and $B_1\equiv 5\pmod{16}$.
Therefore
\[\prod_{i=0}^{n-1}B_i\equiv B_0B_1\equiv 15\equiv -1\pmod{16},
\]
and multiplying by $w\equiv 1\pmod{16}$ gives $N\equiv -1\pmod{16}$.

Next we lower-bound $2^a+2^b$.
Since $r\le n-1$, the product $\prod_{i=0}^{n-1}B_i$ contains the factor $B_r$ and also contains $B_0,B_1$; thus
\[\prod_{i=0}^{n-1}B_i\ge B_0B_1B_r\ge 15B_r.
\]
Hence $N=w\prod_{i=0}^{n-1}B_i\ge 15B_r$ and so
\[2^a+2^b=N-B_r\ge 14B_r\ge 42.
\]
Since $a>b\ge 1$, this implies $a\ge 4$, hence $2^a\equiv 0\pmod{16}$.
Also $2^b\equiv 2,4,8,$ or $0\pmod{16}$.
Finally, as noted above, $B_r\equiv 1,3,$ or $5\pmod{16}$.
Thus
\[2^a+2^b+B_r\not\equiv -1\pmod{16},
\]
contradicting $N\equiv -1\pmod{16}$ and the identity \eqref{eq:N-as-sum}.
This contradiction shows \eqref{eq:assumeBr} is impossible.

Therefore $N-2^a-2^b$ is divisible by $B_r>1$ but not equal to $B_r$, so it is composite, contradicting \eqref{eq:diff}.
\end{proof}

\subsubsection{5.4 Excluding $p+2^d$ by the overlapping congruence system}

\begin{lemma}[No representation $N_0=p+2^d$ for $d\ge 1$]
\label{lem:no-p-plus-power}
For the integer $N_0$ defined by \eqref{eq:CRTpi}--\eqref{eq:CRT16}, there do not exist a prime $p$ and integer $d\ge 1$ such that $N_0=p+2^d$.
\end{lemma}

\begin{proof}
Fix $d\ge 1$.
Because the congruence system \(d\equiv a_i\pmod{m_i}\) is \emph{overlapping} (verified in \S\ref{sec:certificate}), there exists some index $i\in\{1,\dots,28\}$ such that
\begin{equation}
\label{eq:d-in-class}
d\equiv a_i\pmod{m_i}.
\end{equation}
Since $\operatorname{ord}_{p_i}(2)=m_i$, the congruence \eqref{eq:d-in-class} implies
\[2^d\equiv 2^{a_i}\pmod{p_i}.
\]
By \eqref{eq:CRTpi}, $N_0\equiv 2^{a_i}\pmod{p_i}$ as well, hence
\begin{equation}
\label{eq:pi-divides}
N_0-2^d\equiv 0\pmod{p_i}.
\end{equation}
So if $N_0-2^d$ were prime, then it would have to equal $p_i$.
But reducing the putative identity $N_0-2^d=p_i$ modulo $8191$ and using \eqref{eq:CRT8191} gives
\[c\equiv p_i+2^d\pmod{8191}.
\]
Since $\operatorname{ord}_{8191}(2)=13$, the residue $2^d\pmod{8191}$ depends only on $d\pmod{13}$ and ranges over $\{2^e:0\le e\le 12\}$.
By the choice of $c$ (verified in \S\ref{sec:certificate}), $c$ is not congruent to any $p_i+2^e\pmod{8191}$ for $1\le i\le 28$ and $0\le e\le 12$.
Contradiction.
Therefore $N_0-2^d$ is composite for every $d\ge 1$, proving the lemma.
\end{proof}

\subsubsection{5.5 Concluding $N_0\in A$}

\begin{theorem}[Certified explicit element of $A$]
\label{thm:N0-in-A}
The integer $N_0$ defined by \eqref{eq:CRTpi}--\eqref{eq:CRT16} belongs to $A$.
Equivalently, there do not exist a prime $p$ and integers $k,\ell\ge 0$ with
\[N_0=p+2^k+2^\ell.
\]
\end{theorem}

\begin{proof}
We verify the two hypotheses of Round~2 Lemma~4.

\smallskip
\noindent\emph{(i) No representation $N_0=p+2^a$ with $a\ge 1$.}
This is Lemma~\ref{lem:no-p-plus-power}.
In particular, this excludes $N_0=p+2^a+2^a$ for any $a\ge 1$ because $2^a+2^a=2^{a+1}$.

\smallskip
\noindent\emph{(ii) No representation $N_0=p+2^a+2^b$ with $a,b\ge 1$ distinct.}
Write $N_0=w\,M_1$ with $w:=N_0/M_1$; this is an integer by \eqref{eq:CRTM1}.
Since $M_1\equiv -1\pmod{16}$ and $N_0\equiv -1\pmod{16}$, we have $w\equiv 1\pmod{16}$.
Also $N_0<2^{2048}-1=2^{2^{11}}-1$, so with $n=11$ we have $N_0\le 2^{2^n}-1$.
Finally $M_1=\prod_{i=0}^{10}B_i$ with each $B_i\mid (2^{2^i}+1)$.
Hence Lemma~\ref{lem:crockerII} applies (with this $n$ and these $B_i$), showing $N_0$ is not of the form $p+2^a+2^b$ with distinct $a,b\ge 1$.

\smallskip
Applying Round~2 Lemma~4 gives $N_0\in A$.
\end{proof}

\subsubsection{5.6 Deterministic computational certificate for the construction}
\label{sec:certificate}

All numerical assertions needed above were verified by a deterministic Python computation (big-integer modular arithmetic and exact CRT), namely:

\begin{enumerate}
\item \textbf{Overlapping property.} The l.c.m. of the listed moduli $m_i$ is $720$, and every residue class $d\in\{1,2,\dots,720\}$ satisfies at least one congruence $d\equiv a_i\pmod{m_i}$.
\item \textbf{Order checks.} For each listed $p_i$, the number is prime and $\operatorname{ord}_{p_i}(2)=m_i$.
\item \textbf{$c$-avoidance.} With $c=1$ and $p_{29}=8191$, we have
\[c\not\equiv p_i+2^e\pmod{8191}\qquad(1\le i\le 28,\ 0\le e\le 12).
\]
\item \textbf{Fermat factor check.} $q=45592577$ is prime and divides $2^{1024}+1$.
\item \textbf{CRT solution and bounds.} The CRT system \eqref{eq:CRTpi}--\eqref{eq:CRT16} has pairwise coprime moduli; its least positive solution equals the stated decimal $N_0$; and $N_0<2^{2048}-1$.
\end{enumerate}

(As in Round~2, this computation is labeled as such; its role is to certify the correctness of the explicit numerical instance, not to address the asymptotic density question.)

\subsection{6) ADVERSARIAL VERIFICATION}

We stress-test the new Round~3 argument.

\begin{itemize}
\item \textbf{Quantifiers in Lemma~\ref{lem:crockerII}.}
The key is that for any distinct $a,b\ge 1$ in a hypothetical representation, one has $a<2^n$ because $2^a<N\le 2^{2^n}-1$, hence $r=v_2(a-b)\le n-1$ so that $B_r$ is among the factors of $N$.
No hidden assumption about $a,b$ beyond positivity and distinctness is used.

\item \textbf{Case $a=b$ is not forgotten.}
Lemma~\ref{lem:crockerII} only treats distinct $a,b\ge 1$, but Lemma~\ref{lem:no-p-plus-power} excludes $p+2^d$ for all $d\ge 1$, which covers the diagonal case $a=b$ via $2^a+2^a=2^{a+1}$.

\item \textbf{Residue-class coverage for all $d\ge 1$.}
The overlapping system is checked on $\{1,\dots,720\}$; since each congruence condition is periodic modulo $m_i$ and $\mathrm{lcm}(m_i)=720$, this implies coverage for every $d\ge 1$.

\item \textbf{Potential loophole $N_0-2^d=p_i$.}
This is explicitly excluded by the extra modulus $8191$ and the choice of $c$ avoiding all residues $p_i+2^e\pmod{8191}$ for $0\le e\le 12$.
The dependence on $d\bmod 13$ is justified by $\operatorname{ord}_{8191}(2)=13$.

\item \textbf{Use of modulus-$16$ in Lemma~\ref{lem:crockerII}.}
The proof only needs $w\equiv 1\pmod{16}$ and that all $B_i$ for $i\ge 2$ are $\equiv 1\pmod{16}$; Lemma~\ref{lem:fermat-prime-cong} supplies this.
Even if some $B_i$ are composite, they are products of such primes and remain $\equiv 1\pmod{16}$.

\item \textbf{Interaction with Round~2 Lemma~4.}
All hypotheses of the reduction lemma are satisfied: (i) is Lemma~\ref{lem:no-p-plus-power}; (ii) is Lemma~\ref{lem:crockerII} applied to $N_0=wM_1$ with $n=11$.
Thus the final implication $N_0\in A$ is sound.
\end{itemize}

No remaining gap was found in the certification of $N_0\in A$.

\subsection{7) FINAL (EXACTLY ONE)}

\textbf{UNRESOLVED (BUT STRICTLY ADVANCED).}

The global density question \(\overline d(A)>0?\) remains open.
However, we have now provided a gap-free proof (modulo deterministic arithmetic verification) that the explicit $420$-digit integer $N_0$ from Round~2 lies in $A$, and we have supplied a complete proof of the necessary Crocker-type Lemma~II input.

\subsection{8) COMPLETION ESTIMATE (MANDATORY)}

\noindent\textbf{COMPLETION: 35\%}

\subsection{9) REFERENCES}

\begin{itemize}
\item R. Crocker, \emph{On the sum of a prime and of two powers of two}, Pacific J. Math. 36 (1971), 103--107.
\item H. Pan, \emph{On the integers not of the form $p+2^a+2^b$}, arXiv:0905.3809v4 (2009); Acta Arith. 147 (2011), 245--252.
\item T. F. Bloom, \emph{Erd\H{o}s Problem \#9}, erdosproblems.com (accessed 2026-01-20).
\end{itemize}
