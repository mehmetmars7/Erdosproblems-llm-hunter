
1) “FORMAL RESTATEMENT”

Let p be a prime. Define

    A_p := { k! (mod p) : 1 <= k < p }  ⊆ F_p.

Question: Is it true that |A_p| ~ (1 - 1/e) p as p -> infinity?

2) “QUICK LITERATURE/CONTEXT CHECK”

The problem statement records the conjecture |A_p| ~ (1-1/e)p and some remarks (e.g. A_p/A_p = F_p^* and |A_p|<=p-2). A quick check of the Erdos Problems page for #478 shows it is listed as OPEN (as of mid-Jan 2026). I do not use external results beyond standard modular arithmetic.

3) “ATTACK PLAN”

Prove basic structural facts about A_p (ratio set, unavoidable collisions from Wilson's theorem, etc.). Then do computations for many primes to see whether |A_p|/p stabilizes near 1-1/e.

A proof of the conjecture would require a probabilistic model for when the recurrence k! mod p repeats.

4) “WORK”

Lemma 478.1 (ratio set is everything nonzero).
For any prime p, we have A_p / A_p = F_p^*.

Proof.
Fix any x in F_p^* (so x is represented by an integer 1<=x<=p-1). Then

    x ≡ x! / (x-1)!  (mod p).

Both x! and (x-1)! occur among k! for 1<=k<p (note that for x=1, we can use 1!/1!=1). Hence x belongs to A_p/A_p. Therefore A_p/A_p contains every nonzero residue.  QED.

Lemma 478.2 (trivial lower bound).
|A_p| >= sqrt(p-1).

Proof.
In any finite group, |X/Y| <= |X| |Y|, and in particular |A_p/A_p| <= |A_p|^2. By Lemma 478.1, |A_p/A_p| = |F_p^*| = p-1. Hence p-1 <= |A_p|^2, i.e. |A_p| >= sqrt(p-1).  QED.

Lemma 478.3 (Wilson collision gives |A_p|<=p-2, and p≡3 (mod 4) forces |A_p|<=p-3).

(a) For any prime p, we have |A_p| <= p-2.
(b) If p ≡ 3 (mod 4), then |A_p| <= p-3. In particular, if |A_p|=p-2 then necessarily p ≡ 1 (mod 4).

Proof.
(a) By Wilson's theorem, (p-1)! ≡ -1 (mod p). Then

    (p-2)! ≡ (p-1)! * (p-1)^{-1} ≡ (-1) * (-1) ≡ 1 (mod p),

since (p-1) ≡ -1 and its inverse is also -1 modulo p. But 1! ≡ 1 as well. Thus at least two distinct k (namely k=1 and k=p-2) yield the same residue 1, so among the p-1 factorial values there is at least one repetition, hence |A_p| <= (p-1) - 1 = p-2.

(b) Let h = (p-1)/2. Consider h! modulo p. We compute

    (p-1)! = (1*2*...*h) * ((h+1)*...*(p-1)).

In the second product, substitute k -> p-k to pair terms: (h+1)*...*(p-1) ≡ (-1)^h * (1*2*...*h) (mod p). Therefore

    (p-1)! ≡ (-1)^h * (h!)^2  (mod p).

By Wilson, (p-1)! ≡ -1, so (h!)^2 ≡ (-1)^{h+1} (mod p).
If p ≡ 3 (mod 4), then h=(p-1)/2 is odd, so h+1 is even and (-1)^{h+1}=+1. Hence (h!)^2 ≡ 1 (mod p), so h! ≡ 1 or h! ≡ -1 (mod p).

- If h! ≡ 1, then the residue 1 occurs at k=1, k=h, and k=p-2 (at least three occurrences), so |A_p| <= (p-1) - 2 = p-3.
- If h! ≡ -1, then the residue -1 occurs at k=h and k=p-1, while 1 occurs at k=1 and k=p-2, giving at least two distinct repetitions, again |A_p| <= p-3.

Thus for p≡3 (mod 4) we cannot have |A_p|=p-2. Therefore, if |A_p|=p-2 then p must be ≡1 (mod 4).  QED.

FAST REALITY CHECK (local computation).
I computed |A_p| exactly for all primes p<=10000. Summary:

- Number of odd primes p<=10000 tested: 1228.
- Mean of |A_p|/p over these primes: 0.6319266077934649.
- Median of |A_p|/p over these primes: 0.6323326214052672.
- For comparison, 1 - 1/e = 0.6321205588285577.
- The only prime p<=10000 with |A_p|=p-2 is p=5.

5) “VERIFICATION”

- Lemma 478.1 is verified directly by the identity x = x!/(x-1)! in F_p.
- Lemma 478.3(b) uses only Wilson's theorem and the simple pairing argument in the product (h+1)...(p-1).
- The computation of |A_p| uses the recurrence fac_{k+1} = fac_k*(k+1) (mod p) and counts distinct residues.

6) FINAL

UNRESOLVED

(i) Strongest proved partial result:
We have the universal bounds sqrt(p-1) <= |A_p| <= p-2, and if p≡3 (mod 4) then |A_p| <= p-3. Computationally, |A_p|/p for p<=10000 has mean 0.6319266..., extremely close to 1-1/e.

(ii) First gap (crisp statement):
Prove that |A_p|/p converges and identify the limit; in particular, prove or disprove lim_{p->infty} |A_p|/p = 1 - 1/e.

(iii) Top 3 next moves (concrete):
1. Model the recurrence k! mod p as a random walk on F_p^* with multipliers k, and try to justify a Poisson-collision heuristic leading to 1-1/e.
2. Prove nontrivial upper and lower bounds of the form c p <= |A_p| <= C p with explicit constants c,C close to 1-1/e.
3. Study the distribution of first collision times: analyze the smallest k<l with k!≡l! (mod p), i.e. (k+1)(k+2)...l ≡ 1 (mod p).

(iv) Minimal counterexample structure (if the conjecture is false):
A sequence of primes p_j -> infinity such that |A_{p_j}|/p_j stays bounded away from 1-1/e by some fixed epsilon>0.


