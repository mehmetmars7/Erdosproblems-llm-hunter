% Erdos Problem #1020

\noindent\textbf{FORMAL RESTATEMENT.}
Let $r\ge 3$ and $k\ge 1$ be integers.
For each $n\ge r$, let $f(n;r,k)$ be the maximum number of edges in an $r$-uniform hypergraph $\mathcal{H}$ on an $n$-element vertex set such that $\mathcal{H}$ contains no matching of size $k$ (i.e. no $k$ pairwise vertex-disjoint hyperedges).
The statement in the problem file asserts that for all $r\ge 3$,
\[
(\star)\qquad f(n;r,k)=\max\left(\binom{rk-1}{r},\ \binom{n}{r}-\binom{n-k+1}{r}\right).
\]

\medskip
\noindent\textbf{QUICK LITERATURE/CONTEXT CHECK.}
The problem statement describes $(\star)$ as the Erd\H{o}s matching conjecture (open in general for $r\ge 3$) and lists several ranges where it has been proved.

\medskip
\noindent\textbf{ATTACK PLAN.}
First check whether $(\star)$, as written for all $n$, is even consistent with the trivial bound $f(n;r,k)\le \binom{n}{r}$.
If inconsistent, give an explicit counterexample and then state a minimal corrected version (typically with an assumption like $n\ge rk-1$).

\medskip
\noindent\textbf{WORK.}

\noindent\textbf{Counterexample to the literal statement $(\star)$.}
Take $(r,k,n)=(3,2,3)$.

\textit{Verification of the left-hand side.}
An $r$-uniform hypergraph on $n=3$ vertices with $r=3$ has exactly one possible edge, namely the full vertex set.
In particular, every such hypergraph has at most one edge.
Also, a matching of size $k=2$ would require two disjoint $3$-edges, which is impossible on $3$ vertices.
Therefore the constraint “no matching of size $2$” is vacuous, and hence
\[
f(3;3,2)=\binom{3}{3}=1.
\]

\textit{Verification of the right-hand side.}
Compute
\[
\binom{rk-1}{r}=\binom{5}{3}=10,
\qquad
\binom{n}{r}-\binom{n-k+1}{r}=\binom{3}{3}-\binom{2}{3}=1-0=1.
\]
So the right-hand side of $(\star)$ equals $\max(10,1)=10$.

\textit{Conclusion.}
We have $f(3;3,2)=1\ne 10$, so $(\star)$ is false as a universal identity over all $n$.
\hfill$\square$

\medskip
\noindent\textbf{Minimal corrected statement (consistent with the standard conjecture).}
A natural minimal correction is to restrict to $n\ge rk-1$ (so the first extremal construction “all $r$-sets on $rk-1$ vertices” can actually live inside an $n$-vertex set). One may also explicitly cap by $\binom{n}{r}$:
\[
 f(n;r,k)=\min\Bigl\{\binom{n}{r},\ \max\bigl(\binom{rk-1}{r},\ \binom{n}{r}-\binom{n-k+1}{r}\bigr)\Bigr\}.
\]
The corrected conjecture remains open in general for $r\ge 3$ (as indicated by the problem statement).

\medskip
\noindent\textbf{FAST REALITY CHECK.}
The above counterexample is the smallest possible parameter choice with $r\ge 3$ and $k\ge 2$; it already violates $(\star)$ because $\binom{rk-1}{r}$ can exceed the trivial upper bound $\binom{n}{r}$ when $n<rk-1$.

\medskip
\noindent\textbf{VERIFICATION.}
All computations are exact.
The only subtle point is the definition of “contains no set of $k$ many independent (i.e. pairwise disjoint) edges”: for $n<rk$ this is indeed vacuous, hence $f(n;r,k)=\binom{n}{r}$.
This directly conflicts with $(\star)$ whenever $\binom{rk-1}{r}>\binom{n}{r}$, which occurs for the exhibited parameters.

\medskip
\noindent\textbf{FINAL.}
\textbf{FULL SOLUTION}

\textbf{COUNTEREXAMPLE/DISPROOF.}
The literal identity $(\star)$ is false for $(r,k,n)=(3,2,3)$, since $f(3;3,2)=1$ while the right-hand side equals $10$.


