\section{Round 2: Hypercube isoperimetry and a sharp general lower bound}

\subsection{1) Round-2 Objective}

\textbf{Path (A): proof attempt toward the conjectured bound $N\gg 2^n$.} 
Building on the Round-1 gap---to prove or disprove the existence of an absolute $c>0$ such that $N\ge c\,2^n$ for all $n$---the most concrete promising route is an \emph{anti-concentration} argument for subset sums. In this round we implement a sharp anti-concentration mechanism via the vertex-isoperimetric inequality on the hypercube, obtaining the best-known general lower bound
\[
N\ge \binom{n}{\lfloor n/2\rfloor}.
\]
This still falls short of $N\gg 2^n$ by a factor $\Theta(\sqrt n)$, so the original conjecture remains unresolved.

\subsection{2) Round-1 Foundation Used}

We use the following Round-1 foundations.
\begin{enumerate}
\item The global hypothesis: $A\subseteq\{1,\dots,N\}$, $|A|=n$, and the subset-sum map $S\mapsto \sigma(S)$ is injective on $\mathcal P(A)$.
\item The standard identification of subsets with vertices of the hypercube $\{0,1\}^n$ and the fact that injectivity implies the $2^n$ values are distinct integers.
\end{enumerate}
(We do not use Round-1 Lemmas 2--3 in this round.)

\subsection{3) New Insight / Tool (Round-2)}

\textbf{New tool: hypercube vertex-isoperimetry (Harper) + a Lipschitz property.}

Let $f:\{0,1\}^n\to\mathbb Z$ be the subset-sum function $f(x)=\sum_{i=1}^n x_i a_i$. If $x$ and $y$ differ in exactly one coordinate $i$, then $|f(x)-f(y)|=a_i\le N$, so $f$ is $N$-Lipschitz on the hypercube graph. Harper's theorem lower-bounds the vertex boundary of any family of size $2^{n-1}$ by $\binom{n}{\lfloor n/2\rfloor}$. Combining these forces at least $\binom{n}{\lfloor n/2\rfloor}$ distinct subset sums into an interval of length $N$, yielding $N\ge \binom{n}{\lfloor n/2\rfloor}$.

\subsection{4) Attack Plan (Round-2)}

\textbf{Gap after Round-1:} prove/disprove an absolute-constant bound $N\ge c\,2^n$.

\textbf{Round-2 target claim:} prove the strongest general lower bound accessible from a clean anti-concentration argument:
\[
N\ge \binom{n}{\lfloor n/2\rfloor}.
\]

\textbf{Plan.} Choose a median threshold among the $2^n$ distinct subset sums and take the corresponding ``lower half'' family $\mathcal F$ of vertices. Points in the external vertex boundary $\partial\mathcal F$ must have subset sums within $N$ of the median threshold (Lipschitz). Thus $|\partial\mathcal F|\le N$. Harper gives $|\partial\mathcal F|\ge \binom{n}{\lfloor n/2\rfloor}$, forcing the desired bound.

\subsection{5) Work (Round-2)}

\subsubsection*{5.1 Setup}

Let $A=\{a_1,\dots,a_n\}\subseteq\{1,\dots,N\}$, and assume all subset sums are distinct.
Identify each subset with its indicator vector $x\in\{0,1\}^n$ and define
\[
 f(x):=\sum_{i=1}^n x_i a_i\in\mathbb Z.
\]
Injectivity of subset sums is equivalent to $f$ being injective on $\{0,1\}^n$.

Let $Q_n$ be the hypercube graph on vertex set $\{0,1\}^n$, with $x\sim y$ if they differ in exactly one coordinate.
For $\mathcal F\subseteq\{0,1\}^n$ define the \emph{external vertex boundary}
\[
\partial\mathcal F:=\{y\in\{0,1\}^n\setminus \mathcal F: \exists x\in\mathcal F\text{ with }x\sim y\}.
\]

\subsubsection*{5.2 Lipschitz property}

If $x\sim y$ and they differ in coordinate $i$, then
\[
|f(x)-f(y)|=a_i\le N.
\]
Hence $f$ is $N$-Lipschitz on $Q_n$.

\subsubsection*{5.3 Harper's vertex-isoperimetric inequality (special case)}

We use the following standard statement.

\begin{quote}
\textbf{Harper (special case).} If $\mathcal F\subseteq\{0,1\}^n$ has size $|\mathcal F|=2^{n-1}$, then
\[
|\partial\mathcal F|\ge \binom{n}{\lfloor n/2\rfloor}.
\]
\end{quote}

\subsubsection*{5.4 Median split and localization of the boundary}

Since $f$ is injective, its values on $\{0,1\}^n$ are $2^n$ distinct integers. List them in increasing order:
\[
 s_0<s_1<\cdots<s_{2^n-1}.
\]
Let
\[
 t:=s_{2^{n-1}-1}.
\]
Define the ``lower half'' family
\[
\mathcal F:=\{x\in\{0,1\}^n: f(x)\le t\}.
\]
By definition of $t$ we have $|\mathcal F|=2^{n-1}$.

Now let $y\in\partial\mathcal F$. Then $y\notin\mathcal F$, so $f(y)>t$, and there exists $x\in\mathcal F$ with $x\sim y$. By the Lipschitz property,
\[
 f(y)-f(x)=a_i\le N
\]
for the flipped coordinate $i$. Since $f(x)\le t$ and $f(y)>t$, we obtain
\[
 t< f(y)\le t+N.
\]
As $f(y)$ is an integer, this implies
\[
 f(y)\in \{t+1,t+2,\dots,t+N\},
\]
an interval containing exactly $N$ integers.

Because $f$ is injective, distinct $y\in\partial\mathcal F$ give distinct values $f(y)$. Therefore
\[
|\partial\mathcal F|\le N.
\]

\subsubsection*{5.5 Conclusion: the sharp general lower bound}

By Harper's inequality applied to $\mathcal F$ (which has size $2^{n-1}$),
\[
|\partial\mathcal F|\ge \binom{n}{\lfloor n/2\rfloor}.
\]
Combined with $|\partial\mathcal F|\le N$, we conclude
\[
N\ge \binom{n}{\lfloor n/2\rfloor}.
\]

\subsubsection*{5.6 Asymptotic form (optional corollary)}

By Stirling's approximation,
\[
\binom{n}{\lfloor n/2\rfloor}=\left(\sqrt{\frac{2}{\pi}}+o(1)\right)\frac{2^n}{\sqrt n},
\]
so the bound implies
\[
N\ge \left(\sqrt{\frac{2}{\pi}}+o(1)\right)\frac{2^n}{\sqrt n}.
\]

\subsection{6) Adversarial Verification}

\begin{enumerate}
\item \textbf{Small $n$ checks.} For $n=1$, the bound gives $N\ge 1$; for $n=2$, it gives $N\ge 2$; for $n=3$, it gives $N\ge 3$. All are consistent with known minima and with Round-1 computations.
\item \textbf{Boundary notion.} The argument uses the \emph{external vertex boundary}. The localization step crucially uses $y\notin\mathcal F$ (hence $f(y)>t$) and adjacency to an $x\in\mathcal F$ (hence $f(y)\le f(x)+N\le t+N$). This matches the standard formulation of Harper used here.
\item \textbf{Quantifier/tie issues at the median.} Injectivity of $f$ guarantees strict ordering $s_0<\cdots<s_{2^n-1}$ and ensures $|\mathcal F|=2^{n-1}$ with no ties.
\item \textbf{Off-by-one in interval length.} The set $\{t+1,\dots,t+N\}$ has exactly $N$ integers. Since $f$ is injective, $|\partial\mathcal F|\le N$ is correct.
\end{enumerate}

\subsection{7) Final Status (exactly one)}

\textbf{UNRESOLVED (but strictly advanced).}

We proved the sharp general lower bound
\[
N\ge \binom{n}{\lfloor n/2\rfloor}=\left(\sqrt{\frac{2}{\pi}}+o(1)\right)\frac{2^n}{\sqrt n},
\]
via a median-splitting argument on the hypercube combined with Harper's vertex-isoperimetric inequality and the $N$-Lipschitz property of the subset-sum map.
This strengthens the Round-1 lower bounds $N\ge (2^n-1)/n$ and $N\ge \binom{n}{2}+1$, but still does not prove the conjectured $N\gg 2^n$.

\subsection{8) Completion Estimate}

\noindent\textbf{COMPLETION: 70\%}

\subsection{9) References}

\begin{thebibliography}{9}
\bibitem{BloomErdos1}
T.~F.~Bloom, \emph{Erd\H{o}s Problem \#1: Distinct subset sums} (online problem page).

\bibitem{DFX}
Q.~Dubroff, J.~Fox, and M.~W.~Xu, \emph{A note on the Erd\H{o}s distinct subset sums problem}, arXiv:2006.12988.
\end{thebibliography}
