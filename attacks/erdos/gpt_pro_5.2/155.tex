\section*{Erdos problem \#155}

\subsection*{1) FORMAL RESTATEMENT}
Let $F(N)$ be the maximum size of a Sidon set $A\subseteq\{1,2,\dots,N\}$.
The conjecture is:

\medskip
\noindent For each fixed integer $k\ge 1$ there exists $N_0(k)$ such that for all $N\ge N_0(k)$,
\[
F(N+k)\le F(N)+1.
\]

\subsection*{2) QUICK LITERATURE/CONTEXT CHECK}
The problem file gives no named theorems here beyond the conjecture itself.

\subsection*{3) ATTACK PLAN}
Two easy baselines:
\begin{itemize}
\item Always $F(N+1)\le F(N)+1$ by removing the element $N+1$ if used.
\item Difference-counting gives $F(N)=O(\sqrt N)$.
\end{itemize}
Then do a small-$N$ computation to see when the conjectured stabilization might begin.

\subsection*{4) WORK}
\paragraph{Lemma 155.1 (one-step Lipschitz).}
For all $N\ge 1$,
\[
F(N+1)\le F(N)+1.
\]

\textit{Proof.}
Let $A\subseteq\{1,\dots,N+1\}$ be a Sidon set of maximum size $|A|=F(N+1)$. If $N+1\notin A$ then $A\subseteq\{1,\dots,N\}$ so $|A|\le F(N)$.
If $N+1\in A$, then $A\setminus\{N+1\}$ is Sidon in $\{1,\dots,N\}$ so $|A|-1\le F(N)$, i.e. $|A|\le F(N)+1$.
In both cases $F(N+1)=|A|\le F(N)+1$. \qed

\paragraph{Lemma 155.2 (difference-counting upper bound).}
For all $N\ge 1$,
\[
F(N)\le \frac{1+\sqrt{1+8(N-1)}}{2}.
\]
In particular, $F(N)=O(\sqrt N)$.

\textit{Proof.}
Let $A=\{a_1<\cdots<a_m\}\subseteq\{1,\dots,N\}$ be Sidon. As in Lemma 153.3, the positive differences $a_j-a_i$ for $i<j$ are all distinct and lie in $\{1,\dots,N-1\}$.
Hence $\binom{m}{2}\le N-1$, i.e.
$m(m-1)\le 2(N-1)$. Solving the quadratic inequality for $m$ gives the stated bound. \qed

\paragraph{FAST REALITY CHECK (computed small cases).}
By brute-force backtracking, I computed $F(N)$ exactly for $N\le 40$:
\[
\begin{array}{c|cccccccccc}
N & 1&2&3&4&5&6&7&8&9&10 \\ \hline
F(N) & 1&2&2&3&3&3&4&4&4&4
\end{array}
\]
\[
\begin{array}{c|cccccccccc}
N & 11&12&13&14&15&16&17&18&19&20 \\ \hline
F(N) & 5&5&5&5&5&6&6&6&6&6
\end{array}
\]
\[
\begin{array}{c|cccccccccc}
N & 21&22&23&24&25&26&27&28&29&30 \\ \hline
F(N) & 6&6&7&7&7&7&7&7&7&7
\end{array}
\]
\[
\begin{array}{c|cccccccccc}
N & 31&32&33&34&35&36&37&38&39&40 \\ \hline
F(N) & 8&8&8&8&8&8&8&8&8&8
\end{array}
\]
For this range, the conjectured inequality $F(N+k)\le F(N)+1$ holds for $k=2,3,4,5$ for all $N\ge 4$ (violations only occur for very small $N$).
An example of a maximum Sidon set for $N=40$ is
\[
A=\{1,2,5,12,21,27,35,37\}\qquad(|A|=8).
\]

\subsection*{5) VERIFICATION}
\begin{itemize}
\item Lemma 155.1: removing $N+1$ cannot create a repeated sum, so Sidon-ness is preserved.
\item Lemma 155.2: the ``distinct differences'' argument is identical to Lemma 153.3 and uses only the Sidon property.
\item Computation: verified by exact search (backtracking) over subsets of $\{1,\dots,N\}$.
\end{itemize}

\subsection*{6) FINAL}
\textbf{UNRESOLVED}

\begin{enumerate}
\item[(i)] \textbf{Strongest fully proved partial result obtained here.}
For every $N$ we have the one-step inequality $F(N+1)\le F(N)+1$ (Lemma 155.1) and the general upper bound $F(N)\le (1+\sqrt{1+8(N-1)})/2$ (Lemma 155.2).

\item[(ii)] \textbf{Exact first gap.}
Fix $k\ge 2$. Prove that for all sufficiently large $N$, any Sidon set in $\{1,\dots,N+k\}$ of maximum size contains at most one element from the last $k$ positions $\{N+1,\dots,N+k\}$. Equivalently, show $F(N+k)-F(N)\le 1$ for all large $N$.

\item[(iii)] \textbf{Top 3 next moves (concrete targets).}
\begin{enumerate}
\item Structural lemma: show a maximum Sidon set $A\subseteq[1,N]$ must have a ``stable tail'' (e.g. cannot have two very large elements too close to $N$ without forcing sum collisions).
\item Develop a transfer argument: from a maximum Sidon set in $[1,N+k]$, delete one element among $\{N+1,\dots,N+k\}$ and ``slide'' the remainder into $[1,N]$ without losing size.
\item Extend computations to larger $N$ for fixed small $k$ (e.g. $k=2,3$) to see whether any late violations occur.
\end{enumerate}

\item[(iv)] \textbf{Minimal counterexample structure (if the conjecture is false).}
For some fixed $k$, there would exist infinitely many $N$ and Sidon sets $A\subseteq[1,N+k]$ of size $F(N)+2$ such that $A$ contains at least two elements from $\{N+1,\dots,N+k\}$. Such a family would have to exploit a special arrangement of sums involving the last $k$ integers that does not force sum collisions.
\end{enumerate}


