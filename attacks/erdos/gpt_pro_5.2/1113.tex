
1) FORMAL RESTATEMENT

A positive odd integer \(m\) is a \emph{Sierpi\'nski number} if
\[
\forall k\in\mathbb{Z}_{\ge 0},\quad 2^k m + 1\ \text{is composite}.
\]

A set of primes \(P\) is a \emph{covering set} for \(m\) if
\[
\forall k\in\mathbb{Z}_{\ge 0},\ \exists p\in P\ \text{such that}\ p\mid (2^k m+1).
\]
The question asks:
\[
\text{Does there exist a Sierpi\'nski number }m\text{ for which no \emph{finite} covering set }P\text{ exists?}
\]
Equivalently: does there exist odd \(m\) such that every \(2^k m+1\) is composite but for every finite set of primes \(P\) there is some \(k\) with \(\gcd(2^k m+1,\prod_{p\in P}p)=1\)?

2) QUICK LITERATURE/CONTEXT CHECK

I did not browse the web; I only record what is stated in the problem text:

- Sierpi\'nski (1960) constructed infinitely many Sierpi\'nski numbers using covering systems, giving a positive-density set of such \(m\).
- Erd\H{o}s--Graham asked whether there are Sierpi\'nski numbers not explained by a covering system; the present formulation asks whether there are Sierpi\'nski numbers with no finite covering set.
- The text reports evidence for a concrete candidate: \(m=734110615000775^4\) is proved Sierpi\'nski (Izotov) and is argued (Izotov; Filaseta--Finch--Kozek) to have no finite covering set.
- Filaseta--Finch--Kozek propose a revised conjecture: every Sierpi\'nski number is either a perfect power or has a finite covering set.

3) ATTACK PLAN

Proof direction (show such \(m\) exist):

- Try to adapt Izotov/FFK-style constructions: build \(m\) so that each \(2^k m+1\) has a forced algebraic factor, but in a way that the responsible primes vary with \(k\) and cannot be bounded by a finite set.

Disproof direction (show every Sierpi\'nski \(m\) has a finite covering set):

- Prove a structural theorem: if \(2^k m+1\) is always composite, then the prime factors must fall into finitely many ``periodic'' congruence patterns in \(k\), yielding a finite covering set.

Below I only prove basic lemmas describing how a finite covering set forces a covering system of congruences in the exponent \(k\), and I run a small computation illustrating that ``many initial composites'' does not imply a finite covering set.

4) WORK

\textbf{Lemma 1113.1 (Divisibility by a fixed odd prime is periodic in \(k\)).}
Let \(m\ge 1\) be an integer and let \(p\) be an odd prime with \(p\nmid m\). Let \(t=\operatorname{ord}_p(2)\), the multiplicative order of \(2\) modulo \(p\). Then the set
\[
S_p(m):=\{k\in\mathbb{Z}_{\ge 0}: p\mid (2^k m+1)\}
\]
is either empty or a single residue class modulo \(t\). More precisely: if there exists \(k_0\) with \(p\mid (2^{k_0}m+1)\), then
\[
S_p(m)=\{k\ge 0: k\equiv k_0\pmod t\}.
\]

\emph{Proof.}
The condition \(p\mid (2^k m+1)\) is equivalent to
\(2^k\equiv -m^{-1}\pmod p\), which makes sense since \(p\nmid m\).
If it holds for \(k=k_0\), then for any \(j\ge 0\),
\(2^{k_0+jt}\equiv 2^{k_0}(2^t)^j\equiv 2^{k_0}\cdot 1^j\equiv 2^{k_0}\pmod p\), so \(2^{k_0+jt}m+1\equiv 2^{k_0}m+1\equiv 0\pmod p\). Hence \(k_0+jt\in S_p(m)\) for all \(j\).

Conversely, if \(k\in S_p(m)\), then
\(2^k\equiv -m^{-1}\equiv 2^{k_0}\pmod p\), so \(2^{k-k_0}\equiv 1\pmod p\). By definition of the order \(t\), this implies \(t\mid (k-k_0)\), i.e. \(k\equiv k_0\pmod t\). \qed

\textbf{Lemma 1113.2 (A finite covering set forces a finite covering system in the exponent).}
Let \(m\) be an odd integer. Suppose \(P=\{p_1,\dots,p_s\}\) is a finite covering set of primes for \(m\). Then there exist integers \(t_i\ge 1\) and residues \(a_i\in\{0,1,\dots,t_i-1\}\) (for \(1\le i\le s\)) such that
\[
\forall k\ge 0,\ \exists i\in\{1,\dots,s\}\ \text{with}\ k\equiv a_i\pmod{t_i}.
\]
Moreover one may take \(t_i=\operatorname{ord}_{p_i}(2)\) (and \(a_i\) any solution class for which \(p_i\mid 2^{a_i}m+1\)).

\emph{Proof.}
Fix \(i\). Since \(p_i\) belongs to a covering set, in particular it divides \(2^{k}m+1\) for \emph{some} \(k\), otherwise \(p_i\) would be irrelevant. Because \(m\) is odd, note \(p_i\ne 2\) for \(k\ge 1\) divisibility, and also \(p_i\nmid m\) is necessary: if \(p_i\mid m\) then \(2^k m+1\equiv 1\pmod{p_i}\) for all \(k\), so \(p_i\) could not divide any term.
Thus for each \(i\), Lemma 1113.1 applies and gives either no solutions or a single residue class modulo \(t_i=\operatorname{ord}_{p_i}(2)\). Choose \(a_i\) representing that class.

Now take any \(k\ge 0\). Since \(P\) is a covering set, there exists some \(p_i\in P\) dividing \(2^k m+1\). Then \(k\in S_{p_i}(m)\), hence by Lemma 1113.1 we have \(k\equiv a_i\pmod{t_i}\). This proves the covering of all \(k\) by the finitely many congruence classes. \qed

\textbf{FAST REALITY CHECK (finite computation on ``almost Sierpi\'nski'' behavior).}
I ran a brute-force check over odd \(m\le 5000\), testing compositeness of \(2^k m+1\) for \(k=0,1,\dots,15\). Exactly \(235\) values of \(m\) passed this finite test.
The first few such \(m\) are:
\[
47,\ 103,\ 143,\ 211,\ 217,\ 241,\ 257,\ 259,\ 263,\ 283,\ \dots
\]
For example, for \(m=47\) all \(2^k\cdot 47+1\) are composite for \(0\le k\le 199\), but SymPy's deterministic primality test finds that
\[
47\cdot 2^{583}+1\ \text{is prime}.
\]
So passing many initial exponent checks (and even admitting many small prime divisors early on) does \emph{not} imply \(m\) is Sierpi\'nski, and does not by itself suggest the existence of a finite covering set.

5) VERIFICATION

- Lemma 1113.1: the only requirements are \(p\) prime and \(p\nmid m\) so that \(m^{-1}\bmod p\) exists and the order \(\operatorname{ord}_p(2)\) is defined (since \(p\ne 2\)).
- Lemma 1113.2: we explicitly used that primes in a covering set must be coprime to \(m\); otherwise they never divide \(2^k m+1\).
- Computational check: the statement ``\(47\cdot 2^{583}+1\) is prime'' is exactly the output of SymPy's \texttt{isprime} for that integer; it is a computational fact, not a theoretical proof about general \(m\).

6) FINAL

**UNRESOLVED**

(i) Strongest proved partial result: If \(m\) has a finite covering set \(P\), then the exponents \(k\ge 0\) are covered by finitely many residue classes modulo \(\operatorname{ord}_p(2)\) for \(p\in P\) (Lemmas 1113.1--1113.2). In particular, ``finite covering set'' is equivalent to a strong periodicity/covering-system phenomenon in \(k\).

(ii) First gap (crisp): Exhibit an explicit Sierpi\'nski number \(m\) and prove that \emph{no} finite set of primes can cover all \(2^k m+1\), or else prove that every Sierpi\'nski number admits some finite covering set.

(iii) Top 3 next moves:
1. Take the concrete candidate \(m=734110615000775^4\) mentioned in the problem text and attempt to turn the ``suggests no covering set'' argument into a fully explicit combinatorial obstruction: for every finite \(P\), produce an explicit \(k\) with \(\gcd(2^k m+1,\prod_{p\in P}p)=1\).
2. Computationally search for moderate-size \(m\) for which any attempted finite covering system (residue classes modulo \(\operatorname{ord}_p(2)\) for primes \(p\) up to some bound) fails to cover all \(k\) up to a very large range, while simultaneously checking compositeness of \(2^k m+1\) for those \(k\). (This would try to mimic the ``no small covering set'' obstruction.)
3. Try to prove the FFK-style dichotomy conjectured in the problem text in a special case: e.g. show that if \(m\) is not a perfect power and \(2^k m+1\) is always composite, then there exists a prime \(p\) dividing infinitely many of the terms, and bootstrap to a finite covering set.

(iv) Minimal counterexample structure (if such \(m\) exist): one would expect a Sierpi\'nski \(m\) for which the set of prime divisors of \(2^k m+1\) necessarily grows with \(k\) in a non-periodic way, so that no finite collection of congruence classes (as in Lemma 1113.2) can cover all exponents. The candidate in the text suggests perfect powers may be the natural place to look for such behavior.


