\section{Round 2: extending the structural method from $P_3$ to $P_4$}

\subsection*{1) ROUND-2 OBJECTIVE}
\textbf{Path (A): proof-direction.}  The Round-1 output proved the Erd\H{o}s--Hajnal conclusion for the special case $H=P_3$ by a full structural
characterization (induced-$P_3$-free graphs are disjoint unions of cliques) plus a clique--independent product bound.

In Round 2 I push the same mechanism one notch further: I prove the conjectured \emph{polynomial} conclusion for the next natural induced-path obstruction
\[H=P_4,\]
namely for induced-$P_4$-free graphs (\emph{cographs}).  This is a genuine strengthening beyond Round 1, and it stays aligned with the ``structural decomposition $\Rightarrow$ product bound'' strategy.

\subsection*{2) ROUND-1 FOUNDATION USED}
I rely on the following vetted Round-1 items (no reproving):
\begin{itemize}
\item Round-1 Lemma~1: induced-$P_3$-free graphs are disjoint unions of cliques.
\item Round-1 Lemma~2: for a disjoint union of cliques on $n$ vertices,
\(\alpha(G)\,\omega(G)\ge n\), hence \(\max\{\alpha(G),\omega(G)\}\ge \sqrt n\).
\item Round-1 ``next moves'' suggestion: treat induced-$P_4$-free graphs (cographs) via a structural characterization.
\end{itemize}

\subsection*{3) NEW INSIGHT / TOOL (ROUND-2)}
The new ingredient is the \emph{cograph (induced-$P_4$-free) decomposition}:
\begin{quote}
Every induced-$P_4$-free graph is obtained from $K_1$ by repeatedly taking \emph{disjoint unions} and \emph{joins}.
\end{quote}
Equivalently (Seinsche's theorem), every induced-$P_4$-free graph on at least two vertices is either disconnected, or its complement is disconnected.

The second new ingredient is a strengthening of Round-1 Lemma~2:
\begin{quote}
The product inequality \(\alpha(G)\,\omega(G)\ge |V(G)|\) holds for \emph{all cographs}, not just for disjoint unions of cliques.
\end{quote}
The proof uses that this inequality is preserved by both disjoint union and join.

\subsection*{4) ATTACK PLAN (ROUND-2)}
\textbf{Round-1 gap addressed:} Round 1 solved only $H=P_3$.

\textbf{Claims to prove in Round 2:}
\begin{enumerate}
\item (Structure) If $G$ is induced-$P_4$-free and connected with $|V(G)|\ge 2$, then $\overline G$ is disconnected. Equivalently, every induced-$P_4$-free graph is disconnected or co-disconnected.
\item (Closure + recursion) Induced-$P_4$-free graphs are exactly those built from $K_1$ by disjoint unions and joins.
\item (Quantitative bound) For every induced-$P_4$-free $G$ on $n$ vertices, \(\alpha(G)\omega(G)\ge n\), hence \(\max\{\alpha(G),\omega(G)\}\ge \sqrt n\). In particular, Erd\H{o}s--Hajnal holds for $H=P_4$ with exponent $c(P_4)=\tfrac12$.
\end{enumerate}

\subsection*{5) WORK (ROUND-2)}
\paragraph{Notation.}
$P_k$ denotes the path on $k$ vertices.
A graph is \emph{$P_4$-free} if it has no induced subgraph isomorphic to $P_4$.
Write $G=G_1\uplus G_2$ for a disjoint union (no edges between parts) and $G=G_1*G_2$ for the \emph{join} (add all edges between the parts).

\medskip
\paragraph{Fact 0 (self-complementarity of $P_4$).}
The complement of $P_4$ is isomorphic to $P_4$.
Consequently, $G$ is $P_4$-free iff $\overline G$ is $P_4$-free.

\emph{Justification.}
Label $P_4$ as $1$--$2$--$3$--$4$.  In $\overline{P_4}$ the edges are $13,14,24$, which form the path $3$--$1$--$4$--$2$.
\hfill$\square$

\medskip
\paragraph{Lemma 1 (Seinsche-type decomposition for $P_4$-free graphs).}
Let $G$ be a connected $P_4$-free graph on $n\ge 2$ vertices.
Then there exists a nontrivial partition $V(G)=A\cup B$ with $A,B\neq\emptyset$ such that every vertex of $A$ is adjacent to every vertex of $B$.
In particular, $\overline G$ is disconnected.

\emph{Proof (induction on $n$).}
For $n\le 3$ this is a direct check:
$K_2$, $P_3$, and $K_3$ each admit a nontrivial partition $A,B$ with all edges between $A$ and $B$.

Assume $n\ge 4$ and the statement holds for smaller connected $P_4$-free graphs.
Fix a spanning tree $T$ of $G$, and let $x$ be a leaf of $T$ with neighbor $u$ in $T$.
Then $G-x$ is connected.
Since $G$ is $P_4$-free, so is the induced subgraph $G-x$.
By the induction hypothesis applied to $G-x$, there is a partition
\[V(G)\setminus\{x\}=A\cup B,\qquad A,B\neq\emptyset,\qquad \text{and } A\text{ is complete to }B.\]
Without loss of generality, $u\in A$.

If $x$ is adjacent to every vertex of $B$, then $(A\cup\{x\},B)$ is the desired complete-join partition.
So assume instead that $x$ has a non-neighbor $z\in B$.
Define
\[A_1:=A\cap N(x),\qquad A_0:=A\setminus A_1.\]
Note that $u\in A_1$, hence $A_1\neq\emptyset$.
If $A_0=\emptyset$, then $x$ is adjacent to all of $A$, and $(A,B\cup\{x\})$ is a complete-join partition.
So assume $A_0\neq\emptyset$.

We claim that $A_1$ is complete to $A_0$.
Indeed, take any $y\in A_1$ and any $w\in A_0$.
We have
\begin{itemize}
\item $x$ is adjacent to $y$ (by definition of $A_1$),
\item $y$ is adjacent to $z$ (since $y\in A$ and $z\in B$ and $A$ is complete to $B$),
\item $z$ is adjacent to $w$ (since $w\in A$ and again $A$ is complete to $B$),
\item $x$ is nonadjacent to $z$ (choice of $z\in B\setminus N(x)$),
\item $x$ is nonadjacent to $w$ (since $w\in A_0=A\setminus N(x)$).
\end{itemize}
Thus, if $y$ were nonadjacent to $w$, the induced subgraph on $\{x,y,z,w\}$ would be exactly the path
\[x-y-z-w,\]
contradicting that $G$ is $P_4$-free.
Hence $y$ must be adjacent to $w$ for all such choices, so $A_1$ is complete to $A_0$.

Now observe that every vertex of $A_1$ is adjacent to:
(i) $x$ (by definition of $A_1$),
(ii) $A_0$ (as just proved), and
(iii) $B$ (since $A$ is complete to $B$).
Therefore $A_1$ is complete to $V(G)\setminus A_1$.
Taking the partition $(A_1,\,V(G)\setminus A_1)$ proves the lemma.
\hfill$\square$

\medskip
\paragraph{Corollary 2 (recursive description of $P_4$-free graphs).}
A graph $G$ is $P_4$-free if and only if it can be generated from $K_1$ by repeatedly applying disjoint union $\uplus$ and join $*$.

\emph{Proof.}
The ``if'' direction: $P_4$-freeness is preserved under induced subgraphs, under disjoint union, and under join (a $P_4$ cannot use vertices from both sides of a join because that would create a vertex of degree $3$ in the induced subgraph). Also $K_1$ is $P_4$-free.

For the ``only if'' direction, proceed by induction on $n=|V(G)|$.
If $n=1$, $G=K_1$.
If $n\ge 2$ and $G$ is disconnected, write $G=G_1\uplus G_2$ with both parts nonempty (partition the components into two nonempty groups) and apply induction to each $G_i$.
If $G$ is connected, apply Lemma~1 to obtain a nontrivial partition $V(G)=A\cup B$ such that $A$ is complete to $B$, i.e. $G=G[A]*G[B]$.
Both induced subgraphs $G[A],G[B]$ are $P_4$-free and smaller, so induction applies.
\hfill$\square$

\medskip
\paragraph{Lemma 3 (clique/independence under $\uplus$ and $*$).}
Let $G_1,G_2$ be graphs.
\begin{enumerate}
\item If $G=G_1\uplus G_2$, then
\[\alpha(G)=\alpha(G_1)+\alpha(G_2),\qquad \omega(G)=\max\{\omega(G_1),\omega(G_2)\}.\]
\item If $G=G_1*G_2$, then
\[\alpha(G)=\max\{\alpha(G_1),\alpha(G_2)\},\qquad \omega(G)=\omega(G_1)+\omega(G_2).\]
\end{enumerate}

\emph{Proof.}
(1) In a disjoint union, an independent set may be chosen independently in each component, while a clique must lie entirely within one component.

(2) In a join, every vertex of $G_1$ is adjacent to every vertex of $G_2$, so cliques add across parts; an independent set cannot use vertices from both parts.
\hfill$\square$

\medskip
\paragraph{Lemma 4 (product bound for cographs).}
If $G$ is $P_4$-free on $n$ vertices, then
\[\alpha(G)\,\omega(G)\ge n.\]
Consequently, $\max\{\alpha(G),\omega(G)\}\ge \sqrt n$.

\emph{Proof (induction on $n$ using Corollary~2).}
For $n=1$, $\alpha=\omega=1$.
Assume $n\ge 2$ and the claim holds for smaller $P_4$-free graphs.
By Corollary~2, write $G$ as either a disjoint union $G_1\uplus G_2$ or a join $G_1*G_2$ of two nonempty $P_4$-free graphs on $n_1,n_2$ vertices.
By induction, $\alpha(G_i)\,\omega(G_i)\ge n_i$.

\emph{Disjoint union case.}
Using Lemma~3(1), let $\omega(G)=\max\{\omega_1,\omega_2\}$ and wlog $\omega_1\ge \omega_2$.
Then
\[
\alpha(G)\omega(G)=(\alpha_1+\alpha_2)\omega_1\ge \alpha_1\omega_1+\alpha_2\omega_2\ge n_1+n_2=n.
\]

\emph{Join case.}
Using Lemma~3(2), let $\alpha(G)=\max\{\alpha_1,\alpha_2\}$ and wlog $\alpha_1\ge \alpha_2$.
Then
\[
\alpha(G)\omega(G)=\alpha_1(\omega_1+\omega_2)\ge \alpha_1\omega_1+\alpha_2\omega_2\ge n_1+n_2=n.
\]

Finally, $\max\{\alpha,\omega\}\ge \sqrt{\alpha\omega}\ge \sqrt n$.
\hfill$\square$

\medskip
\paragraph{Proved special case (new in Round 2): $H=P_4$.}
For $H=P_4$, the Erd\H{o}s--Hajnal conclusion holds with exponent $c(P_4)=\tfrac12$:
any induced-$P_4$-free graph $G$ on $n$ vertices has a clique or independent set of size at least $\sqrt n$.

\medskip
\paragraph{Tightness.}
The exponent $\tfrac12$ is best possible (up to rounding) even within cographs:
if $n=m^2$, take $G$ to be the disjoint union of $m$ cliques of size $m$.
Then $G$ is $P_4$-free (already induced-$P_3$-free by Round-1 Lemma~1), and
\[\omega(G)=m,\qquad \alpha(G)=m,\qquad \max\{\alpha(G),\omega(G)\}=\sqrt n.\]

\subsection*{6) ADVERSARIAL VERIFICATION}
\begin{itemize}
\item \textbf{Base cases in Lemma~1.} For $n\le 3$, connected graphs are $K_2,P_3,K_3$, and each has an explicit nontrivial complete-join partition.
\item \textbf{Induction step in Lemma~1.} The spanning-tree leaf $x$ ensures $G-x$ is connected. The dichotomy ``$x$ adjacent to all of $B$'' or not is exhaustive; in the ``not'' case, the construction of $\{x,y,z,w\}$ uses $z\in B\setminus N(x)$ and $w\in A\setminus N(x)$, both ensured by the corresponding branches.
\item \textbf{Inducedness of the $P_4$ in Lemma~1.} The only potentially problematic extra edge in the candidate path $x-y-z-w$ is $yw$; forcing $yw$ to be present is exactly what prevents the induced $P_4$, and yields the final join partition.
\item \textbf{Corollary~2 closure under join.} Any four-vertex induced subgraph meeting both sides of a join has a vertex adjacent to all vertices on the other side, forcing degree $\ge 3$ in the induced subgraph, hence cannot be $P_4$.
\item \textbf{Product inequality algebra.} In Lemma~4, the key inequalities are monotonicity: if $\omega_1\ge\omega_2$ then $\alpha_2\omega_1\ge\alpha_2\omega_2$, and if $\alpha_1\ge\alpha_2$ then $\alpha_1\omega_2\ge\alpha_2\omega_2$.
\item \textbf{Consistency with Round 1.} Disjoint unions of cliques are cographs, and Lemma~4 reduces to Round-1 Lemma~2 in that subfamily; thus Round 2 strictly extends the Round-1 quantitative mechanism.
\end{itemize}

\subsection*{7) FINAL}
\textbf{UNRESOLVED (BUT STRICTLY ADVANCED).}

Beyond the Round-1 special case $H=P_3$, Round 2 proves the Erd\H{o}s--Hajnal \emph{polynomial} conclusion for the larger obstruction $H=P_4$ (cographs), with an explicit sharp exponent $c(P_4)=\tfrac12$.
The full Erd\H{o}s--Hajnal conjecture for general fixed $H$ remains open.

\subsection*{8) COMPLETION ESTIMATE (MANDATORY)}
\textbf{COMPLETION: 35\%}

\subsection*{9) REFERENCES}
Only sources actually used/checked in Round 2:
\begin{itemize}
\item D. Seinsche, \emph{On a property of the class of $n$-colorable graphs}, J. Combin. Theory Ser.\ B \textbf{16} (1974), 191--193. (Classical source for the ``$P_4$-free $\Rightarrow$ disconnected or co-disconnected'' property.)
\item M. G. S. Chung and D. B. West, \emph{Large $P_4$-free graphs with bounded degree}, Journal of Graph Theory \textbf{17} (1993), 109--116. (Contains a short inductive proof of Seinsche's property, used as a consistency check.)
\end{itemize}
