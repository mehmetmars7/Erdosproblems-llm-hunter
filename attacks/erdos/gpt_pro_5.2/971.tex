% Erdos Problem #971
% URL: https://www.erdosproblems.com/971

Let $p(a,d)$ be the least prime congruent to $a\pmod{d}$. Does there exist a constant $c>0$ such that, for all large $d$,\[p(a,d) > (1+c)\phi(d)\log d\]for $\gg \phi(d)$ many values of $a$? Erd\H{o}s \cite{Er49c} could prove this is true for an infinite sequence of $d$. He also proved that, for any $\epsilon>0$,\[p(a,d)< \epsilon \phi(d)\log d\]for $\gg_\epsilon \phi(d)$ many values of $a$. References [Er49c] Erd\H{o}s, P., On some applications of {B}run's method . Acta Univ. Szeged. Sect. Sci. Math. (1949), 57--63.

%Erdos problem 971

\medskip
\noindent\textbf{1) FORMAL RESTATEMENT}\\
For integers $d\ge 1$ and $a\in\mathbb{Z}$, define
\[
 p(a,d):=\min\{p\text{ prime}: p\equiv a\pmod d\}.
\]
\textbf{Ambiguity.} If $\gcd(a,d)>1$, there may be no prime congruent to $a\pmod d$ (e.g. $a\equiv 0\pmod d$ with $d>1$), or there may be a unique such prime dividing $d$.  The natural convention in this question is to restrict to reduced residue classes $a\bmod d$ with $\gcd(a,d)=1$.

Using $\phi(d)$ for Euler's totient and taking $\log$ to be the natural logarithm, the question is:

\textbf{Does there exist a constant $c>0$ such that for all sufficiently large $d$, the set}
\[
 A_d(c):=\{a\in\{1,2,\dots,d\}: \gcd(a,d)=1\text{ and }p(a,d)>(1+c)\,\phi(d)\log d\}
\]
\textbf{has cardinality $|A_d(c)|\ge c'\phi(d)$ for some absolute $c'>0$ independent of $d$?}

\medskip
\noindent\textbf{2) QUICK LITERATURE/CONTEXT CHECK}\\
The statement records two results of Erd\H{o}s: (i) the displayed lower bound holds for an infinite sequence of moduli $d$; (ii) for any fixed $\varepsilon>0$ there are $\gg_\varepsilon \phi(d)$ classes with $p(a,d)<\varepsilon\phi(d)\log d$.  I do not invoke any further external results.

\medskip
\noindent\textbf{3) ATTACK PLAN}\\
\textbf{Proof track:} relate the distribution of $p(a,d)$ to how primes $\le X$ occupy reduced residue classes mod $d$; show that for $X\asymp \phi(d)\log d$ many classes remain empty up to $X$.

\textbf{Disproof track:} attempt to prove that for infinitely many $d$ almost every reduced residue class already contains a prime $\le (1+c)\phi(d)\log d$, contradicting the desired positive proportion of large least primes.

I only obtain unconditional counting lemmas and small numerical checks.

\medskip
\noindent\textbf{4) WORK}\\
\textbf{FAST REALITY CHECK (small moduli).} For $d\le 100$, I computed $p(a,d)$ for all $a$ with $\gcd(a,d)=1$ and checked the proportion of classes satisfying $p(a,d)>(1+c)\phi(d)\log d$.
For $c=0.1$, the average proportion over $2\le d\le 100$ was approximately $0.242$.  Sample values:
\[
\begin{array}{c|c|c|c}
 d & \phi(d) & (1+0.1)\phi(d)\log d & |A_d(0.1)|/\phi(d)\\\hline
 30 & 8 & 29.93 & 1/8\\
 60 & 16 & 72.06 & 1/16\\
 90 & 24 & 118.79 & 3/24\\
 100 & 40 & 202.63 & 5/40
\end{array}
\]
This is only a sanity check at very small $d$.

\medskip
\noindent\textbf{Lemma 971.1 (injective counting bound).}\label{lem:971-injection}
Fix $d\ge 1$ and $X\ge 2$.
Let
\[
B_d(X):=\{a\in\{1,2,\dots,d\}: \gcd(a,d)=1\text{ and } p(a,d)\le X\}.
\]
Then
\[
|B_d(X)|\le \pi(X),
\]
where $\pi(X)$ is the number of primes $\le X$.  Consequently,
\[
|\{a\in\{1,\dots,d\}: \gcd(a,d)=1\text{ and }p(a,d)>X\}|\ge \phi(d)-\pi(X).
\]

\noindent\textbf{Proof.}
For each $a\in B_d(X)$, choose the prime $p(a,d)\le X$.
If $a\ne a'$ are two reduced residue classes mod $d$, then $p(a,d)\equiv a\not\equiv a'\equiv p(a',d)\pmod d$, so $p(a,d)\ne p(a',d)$.
Thus the map $a\mapsto p(a,d)$ from $B_d(X)$ to the set of primes $\le X$ is injective, giving $|B_d(X)|\le \pi(X)$.
The second inequality follows because the reduced residue classes are exactly $\phi(d)$ in number. \qed

\medskip
\noindent\textbf{Lemma 971.2 (distinct least primes across classes).}\label{lem:971-distinct}
For fixed $d\ge 2$, the values $\{p(a,d): 1\le a\le d,\ \gcd(a,d)=1\}$ are $\phi(d)$ distinct primes, none dividing $d$.
In particular,
\[
\max_{\gcd(a,d)=1} p(a,d)\ \ge\ p_{\phi(d)},
\]
where $p_m$ denotes the $m$-th prime.

\noindent\textbf{Proof.}
If $\gcd(a,d)=1$ then any prime $p\equiv a\pmod d$ cannot divide $d$ (otherwise $p\mid d$ would imply $a\equiv 0\pmod p$ and hence $\gcd(a,d)>1$).  Thus $p(a,d)$ never divides $d$.
If $a\ne a'$ are reduced residue classes, then any prime congruent to $a\pmod d$ is not congruent to $a'\pmod d$, so $p(a,d)\ne p(a',d)$.  Therefore these least primes are $\phi(d)$ distinct primes.
Ordering them increasingly gives a list of $\phi(d)$ distinct primes, hence the largest is at least the $\phi(d)$-th prime $p_{\phi(d)}$. \qed

\medskip
\noindent\textbf{5) VERIFICATION}\\
Lemma~\ref{lem:971-injection} only uses injectivity of the map $a\mapsto p(a,d)$ and does not require distribution results for primes.
Lemma~\ref{lem:971-distinct} requires $d\ge 2$; for $d=1$ the definition degenerates.

The computation for $d\le 100$ was verified by checking that every reduced residue class found a least prime within the sieve limit (primes up to $5\times 10^5$ were used).

\medskip
\noindent\textbf{6) FINAL}\\
\textbf{UNRESOLVED}

(i) \textbf{Strongest proved partial result.} For any modulus $d$ and threshold $X$, at most $\pi(X)$ reduced residue classes mod $d$ can have least prime $\le X$ (Lemma~\ref{lem:971-injection}).  Also, the least primes across reduced classes are all distinct (Lemma~\ref{lem:971-distinct}).

(ii) \textbf{First gap (crisp).} Unconditionally show (or refute) that there exist constants $c,c'>0$ such that for all sufficiently large $d$,
\[
|\{a\bmod d: \gcd(a,d)=1,\ p(a,d)>(1+c)\phi(d)\log d\}|\ge c'\phi(d).
\]

(iii) \textbf{Top 3 next moves.}
\begin{itemize}
\item Combine Lemma~\ref{lem:971-injection} with quantitative upper bounds on $\pi(X)$ at $X\asymp \phi(d)\log d$ to force $\phi(d)-\pi(X)\gg \phi(d)$; this would require nontrivial prime-counting input.
\item Try to transfer known ``Maier matrix''/irregularity ideas to least primes in progressions, aiming to show many residue classes avoid primes up to $(1+c)\phi(d)\log d$.
\item Extend computation to larger $d$ (e.g. up to several thousand) and test stability of the proportion for various $c$, to guess plausible constants.
\end{itemize}

(iv) \textbf{Minimal counterexample structure.} A counterexample to the conjectured statement for a given $c$ would be an infinite sequence of moduli $d$ for which almost every reduced residue class contains a prime $\le (1+c)\phi(d)\log d$.  Such a modulus would have to exhibit unusually even early prime coverage across residue classes.


