

\subsection*{FORMAL RESTATEMENT}
Let $\varphi$ denote Euler's totient function. For each integer $n\ge 1$, define
\[
 g(n) := \bigl|\{m\in\mathbb{Z}_{\ge 1} : \varphi(m)=n\}\bigr|.
\]
The question is:

\medskip
\noindent\textbf{Conjecture.} For every $\varepsilon>0$, there exist infinitely many integers $n\ge 1$ such that
\[
 g(n) > n^{1-\varepsilon}.
\]

\medskip
\noindent\textbf{Conventions/edge cases.}
\begin{itemize}
\item $\mathbb{Z}_{\ge 1}=\{1,2,3,\dots\}$.
\item The inequality $g(n) > n^{1-\varepsilon}$ is meaningful for all $n\ge 1$ (for $n=1$, it reads $g(1)>1^{1-\varepsilon}=1$).
\end{itemize}

\subsection*{QUICK LITERATURE/CONTEXT CHECK}
From the problem statement: Pillai proved $\limsup g(n)=\infty$, Erd\H{o}s proved $g(n)>n^c$ for infinitely many $n$ for some $c>0$, and Lichtman obtained infinitely many $n$ with $g(n)>n^{0.71568\cdots}$ by finding many primes $p$ whose $p-1$ has no large prime factor (improving earlier exponents, e.g. Baker--Harman). The statement also notes an implication from a conjecture about shifted primes $p-1$ being $p^{\varepsilon}$-smooth.

In what follows I do \emph{not} use any result beyond what is explicitly written above; instead I prove a few unconditional, elementary lemmas and record computational sanity checks.

\subsection*{ATTACK PLAN}
\begin{itemize}
\item \textbf{Proof track (ambitious):} Try to build many distinct $m$ with the same totient value $n$ by multiplicative constructions. The main obstruction is that keeping $\varphi(m)$ fixed while varying $m$ typically requires deep information about primes $p$ with prescribed smoothness of $p-1$.
\item \textbf{Disproof track:} Attempt to show the conjecture fails by proving a universal upper bound $g(n) \le n^{1-\varepsilon_0}$ for some fixed $\varepsilon_0>0$. This contradicts known results quoted in the problem statement (e.g. $\limsup g(n)=\infty$), so a disproof seems implausible.
\item \textbf{What I will do here:} Provide rigorous small-case computations and two+ problem-specific lemmas that clarify the structure of $g(n)$ (parity/vanishing on odd $n$, and explicit infinite families with $g(n)\ge 2$).
\end{itemize}

\subsection*{WORK}
\paragraph{FAST REALITY CHECK (exact computation for small $n$).}
I computed $g(n)$ exactly for all $1\le n\le 200$ by enumerating $m$ with $\varphi(m)\le 200$ (the maximum $m$ occurring in this range was $840$). The values in this range satisfy:
\begin{itemize}
\item $g(1)=2$ (solutions $m=1,2$).
\item $g(n)=0$ for every odd $n>1$.
\item The largest values up to $200$ are $g(144)=21$ and $g(192)=21$; also $g(72)=17$, $g(96)=17$, $g(120)=17$.
\end{itemize}
These are only sanity checks; they do not address the conjectured exponent $1-\varepsilon$.

\noindent\textbf{Lemma (Totients are even beyond $2$).}
For every integer $m>2$, $\varphi(m)$ is even. Consequently,
\[
 g(n)=0\quad\text{for every odd }n>1,\qquad\text{and}\qquad g(1)=2.
\]

\noindent\textbf{Proof.}
Let $m>2$. Consider the multiplicative group of units $(\mathbb{Z}/m\mathbb{Z})^\times$, which has size $\varphi(m)$. The element $-1\in(\mathbb{Z}/m\mathbb{Z})^\times$ because $\gcd(-1,m)=1$. Also $-1\not\equiv 1\pmod m$ since $m>2$.

Define the map $f:(\mathbb{Z}/m\mathbb{Z})^\times\to(\mathbb{Z}/m\mathbb{Z})^\times$ by $f(x)\equiv -x\pmod m$. This map is an involution with no fixed points: if $f(x)=x$ then $2x\equiv 0\pmod m$. Since $x$ is a unit, this implies $m\mid 2$, hence $m\le 2$, contradicting $m>2$.

Thus $(\mathbb{Z}/m\mathbb{Z})^\times$ decomposes into disjoint 2-cycles under $f$, so its cardinality $\varphi(m)$ is even.

Now if $n$ is odd and $n>1$, there is no $m>2$ with $\varphi(m)=n$, and the only remaining possibilities are $m\in\{1,2\}$ which both satisfy $\varphi(m)=1$. Hence $g(1)=2$ and $g(n)=0$ for odd $n>1$.
\hfill$\square$


\noindent\textbf{Lemma (An infinite family with $g(n)\ge 2$).}
For every odd prime $p$, one has
\[
 g(p-1)\ge 2.
\]
More precisely, $\varphi(p)=\varphi(2p)=p-1$.

\noindent\textbf{Proof.}
Let $p$ be an odd prime. Then $\varphi(p)=p-1$.

Since $\gcd(2,p)=1$ and $\varphi$ is multiplicative on coprime inputs, we have
\[
\varphi(2p)=\varphi(2)\varphi(p)=1\cdot (p-1)=p-1.
\]
The integers $p$ and $2p$ are distinct, so they contribute at least two distinct solutions to $\varphi(m)=p-1$, giving $g(p-1)\ge 2$.
\hfill$\square$


\noindent\textbf{Lemma (One explicit construction of a preimage).}
Let $k\ge 0$ and let $p_1,\dots,p_r$ be distinct odd primes. Define
\[
 m := 2^{k+1}\prod_{i=1}^r p_i,\qquad n := 2^k\prod_{i=1}^r (p_i-1).
\]
Then $\varphi(m)=n$, so $g(n)\ge 1$.

\noindent\textbf{Proof.}
Because the factors $2^{k+1}$ and $\prod p_i$ are coprime, multiplicativity gives
\[
\varphi(m)=\varphi(2^{k+1})\,\varphi\Bigl(\prod_{i=1}^r p_i\Bigr).
\]
We have $\varphi(2^{k+1})=2^k$. Also, since the $p_i$ are distinct primes,
\[
\varphi\Bigl(\prod_{i=1}^r p_i\Bigr)=\prod_{i=1}^r \varphi(p_i)=\prod_{i=1}^r (p_i-1).
\]
Multiplying yields $\varphi(m)=2^k\prod_{i=1}^r (p_i-1)=n$.
\hfill$\square$


\subsection*{VERIFICATION}
\begin{itemize}
\item Lemma (Totients are even beyond $2$): The involution $x\mapsto -x$ on units has no fixed points exactly because $m>2$; for $m=1,2$ the argument fails and indeed $\varphi(1)=\varphi(2)=1$ is odd.
\item Lemma (An infinite family with $g(n)\ge 2$): Requires $p$ odd so that $p$ and $2p$ are distinct and $\varphi(2)=1$ applies; for $p=2$, $p-1=1$ and $g(1)=2$ still, but the displayed equalities change.
\item Lemma (One explicit construction of a preimage): Checked multiplicativity hypotheses: $\gcd(2^{k+1},\prod p_i)=1$ since all $p_i$ are odd.
\item Computation: the brute-force range $m\le 200000$ found all solutions with $\varphi(m)\le 200$ and the maximum $m$ observed was $840$, so the $g(n)$ values for $n\le 200$ are exact.
\end{itemize}

\subsection*{FINAL}
\textbf{UNRESOLVED.}
\begin{enumerate}
\item[(i)] \textbf{Strongest proved partial result.} Elementary structure results: $g(1)=2$ and $g(n)=0$ for odd $n>1$; also $g(p-1)\ge 2$ for every odd prime $p$ (so $g(n)\ge 2$ for infinitely many $n$).
\item[(ii)] \textbf{First gap.} Prove, for arbitrary fixed $\varepsilon>0$, that there exist infinitely many $n$ admitting \emph{at least} $n^{1-\varepsilon}$ distinct solutions to $\varphi(m)=n$.
\item[(iii)] \textbf{Top 3 next moves.}
  \begin{enumerate}
  \item Prove (or computationally test in growing ranges) a quantitative lower bound on the number of primes $p\le x$ for which $p-1$ is $x^{\theta}$-smooth, with $\theta$ as small as possible; then translate such a bound into a lower bound for $g(n)$ via the standard "smooth $p-1$" construction sketched in the problem statement.
  \item Systematically search for explicit families of $n$ with many representations of the form $n=\varphi(m)$ using multiplicativity: enforce many independent "choices" of prime powers in $m$ while keeping $\varphi(m)$ fixed.
  \item Compute $\max_{n\le N} g(n)$ and candidate extremizers for larger $N$ (e.g. $10^6,10^7$) to guess the right structure of large fibers and to test whether $g(n)$ behaves like a large power of $n$ along any explicit family.
  \end{enumerate}
\item[(iv)] \textbf{Minimal counterexample structure (if the conjecture were false).} One would need an $\varepsilon_0>0$ and an $N_0$ such that for all $n\ge N_0$, $g(n)\le n^{1-\varepsilon_0}$. Such a failure would have to persist even on $n$ built from many smooth shifted primes $p-1$; in particular it would likely force a strong scarcity of primes with very smooth $p-1$.
\end{enumerate}


