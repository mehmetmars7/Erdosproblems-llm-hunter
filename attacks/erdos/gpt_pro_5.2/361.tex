\section*{Problem \#361}

\subsection*{1) FORMAL RESTATEMENT}
Fix a real constant $c>0$. For each integer $n\ge 1$ let $m=\lfloor cn\rfloor$ and define
\[
G_c(n):=\max\Bigl\{|A|:\ A\subseteq\{1,2,\dots,m\}\ \text{and}\ n\notin \Sigma(A)\Bigr\},
\]
where $\Sigma(A)=\{\sum_{a\in S} a:\ S\subseteq A\}$ is the set of all subset sums.
The problem asks for the size/asymptotics of $G_c(n)$ as $n\to\infty$ and whether $G_c(n)$ depends irregularly on $n$.

\subsection*{2) CONTEXT AND DEFINITIONS}
Key observation: if $m\ge n$ and all elements are positive, then elements $>n$ cannot participate in a subset sum equal to $n$. Hence for $m\ge n$ the problem splits:
\[
n\notin \Sigma(A)\quad\Longleftrightarrow\quad n\notin \Sigma(A\cap\{1,\dots,n\}).
\]

\subsection*{3) QUICK LITERATURE/CONTEXT CHECK}
The Erd\H{o}s Problems forum thread for \#361 reports computer checks for $c=3/4$ showing irregular dependence on $n$ (e.g.\ for $100\le n\le 104$ the maximum sizes are $34,37,32,38,35$), and also notes the ``trivial'' exact answer when $c\ge 1$.

\subsection*{4) ATTACK PLAN}
I will give a complete proof of the exact optimum in the regime $m\ge n$ (in particular, for every fixed $c>1$ and all sufficiently large $n$):
\[
G_c(n)=\lfloor cn\rfloor-\left\lceil\frac n2\right\rceil.
\]
This resolves the ``$c\ge 1$'' case and shows that in this range the dependence on $n$ is simple (only parity effects via $\lceil n/2\rceil$).
For $c<1$ the general problem appears more delicate and (per the forum computations) can be irregular.

\subsection*{5) DETAILED WORK}
It suffices to prove the following finite statement.

\medskip
\noindent\textbf{Theorem 361.1.}
\emph{Let $n,m\in\mathbb{N}$ with $m\ge n$. Then the maximum size of a set $A\subseteq\{1,\dots,m\}$ such that $n\notin\Sigma(A)$ is}
\[
m-\left\lceil\frac n2\right\rceil.
\]
\medskip

\begin{proof}
Let $A\subseteq\{1,\dots,m\}$ and assume $n\notin\Sigma(A)$.

\smallskip
\noindent\emph{Step 1: reduction to $\{1,\dots,n\}$.}
Write $A=A_{\le n}\cup A_{>n}$ where $A_{\le n}:=A\cap\{1,\dots,n\}$ and $A_{>n}:=A\cap\{n+1,\dots,m\}$.
Any subset sum using an element of $A_{>n}$ is $>n$, hence cannot equal $n$. Therefore $n\notin\Sigma(A)$ iff $n\notin\Sigma(A_{\le n})$.
So
\[
|A|=|A_{>n}|+|A_{\le n}|\le (m-n)+\max\{|B|:\ B\subseteq\{1,\dots,n\},\ n\notin\Sigma(B)\}.
\]
Thus it is enough to compute the maximum size in the case $m=n$.

\smallskip
\noindent\emph{Step 2: the $m=n$ upper bound.}
Let $B\subseteq\{1,\dots,n\}$ with $n\notin\Sigma(B)$.
Then $n\notin B$ (otherwise $\{n\}$ is a subset summing to $n$).
Also, for each $x\in\{1,\dots,n-1\}$ we cannot have both $x$ and $n-x$ in $B$, since then $\{x,n-x\}$ sums to $n$.
The pairs
\[
\{1,n-1\},\ \{2,n-2\},\ \dots,\ \left\{\left\lfloor\frac{n-1}{2}\right\rfloor,\, n-\left\lfloor\frac{n-1}{2}\right\rfloor\right\}
\]
are disjoint. From each such pair, $B$ can contain at most one element.
If $n$ is even, there is one additional unpaired element $n/2$ (since $n-(n/2)=n/2$), which may or may not be in $B$.
Hence
\[
|B|\le
\begin{cases}
\frac{n-1}{2}=\left\lfloor\frac n2\right\rfloor, & n\text{ odd},\\[4pt]
\frac{n-2}{2}+1=\frac n2=\left\lfloor\frac n2\right\rfloor, & n\text{ even},
\end{cases}
\]
so in all cases $|B|\le \lfloor n/2\rfloor$.

\smallskip
\noindent\emph{Step 3: construction (sharpness) for $m=n$.}
Define
\[
B_0 := \{\lceil n/2\rceil,\ \lceil n/2\rceil+1,\ \dots,\ n-1\}\subseteq\{1,\dots,n\}.
\]
Then $|B_0|=n-\lceil n/2\rceil=\lfloor n/2\rfloor$.
We claim $n\notin\Sigma(B_0)$.
Indeed, any one-element subset sum is at most $n-1$.
Any subset of size at least $2$ has sum at least $\lceil n/2\rceil + (\lceil n/2\rceil+1)>n$.
Therefore no subset sum equals $n$.

Thus the maximum over subsets of $\{1,\dots,n\}$ is exactly $\lfloor n/2\rfloor$.

\smallskip
\noindent\emph{Step 4: conclude for general $m\ge n$.}
Taking $A:=B_0\cup\{n+1,\dots,m\}$ gives
\[
|A|=\lfloor n/2\rfloor+(m-n)=m-\lceil n/2\rceil,
\]
and $n\notin\Sigma(A)$ by Step 1. This matches the upper bound, so it is optimal.
\end{proof}

\paragraph{Consequence for fixed $c>1$.}
If $c>1$ is fixed and $m=\lfloor cn\rfloor$, then for all large $n$ one has $m\ge n$, hence
\[
G_c(n)=\lfloor cn\rfloor-\left\lceil\frac n2\right\rceil
=
\left(c-\frac12\right)n + O(1).
\]
So in this range there is no ``irregular'' dependence beyond parity.

\subsection*{6) VERIFICATION AND EDGE CASES}
\begin{itemize}
\item For $c=1$ this gives $G_1(n)=\lfloor n/2\rfloor$, matching small-$n$ checks.
\item The proof uses only the two-element obstruction $x+(n-x)=n$ and the single-element obstruction $\{n\}$; this is enough to cap the size, while the explicit $B_0$ avoids \emph{all} subset sums $n$ because its elements are so large that any two already exceed $n$.
\end{itemize}

\subsection*{7) FINAL}
\textbf{UNRESOLVED (for the full range $c>0$), but with a complete solution for $c\ge 1$.}
\begin{itemize}
\item[(i)] \textbf{Strongest proved partial result:} if $c\ge 1$ and $m=\lfloor cn\rfloor$, then
\[
G_c(n)=m-\left\lceil\frac n2\right\rceil
=
\left(c-\frac12\right)n + O(1),
\]
with the explicit extremal set $A=\{\lceil n/2\rceil,\dots,n-1\}\cup\{n+1,\dots,m\}$.
\item[(ii)] \textbf{First remaining gap:} determine $G_c(n)$ (or even its correct order of magnitude) for fixed $0<c<1$; explain/prove the reported irregular dependence on $n$ in that regime.
\item[(iii)] \textbf{Next moves:} for $c<1$, combine ``pairing'' constraints with modular constructions (e.g.\ taking multiples of the smallest prime not dividing $n$) and optimize; try to prove matching upper bounds.
\item[(iv)] \textbf{Minimal counterexample search:} for fixed $c<1$, compute/estimate $G_c(n)$ across residue classes and prime factorizations of $n$ to identify the correct extremal templates.
\end{itemize}

\noindent\textbf{COMPLETION: 70\%}

