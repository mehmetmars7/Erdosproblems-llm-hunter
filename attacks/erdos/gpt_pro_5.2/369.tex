\section*{Problem 369}

\subsection*{1) FORMAL RESTATEMENT}
Fix real $\varepsilon>0$ and an integer $k\ge 2$. For $n\in\mathbb{N}$, call an integer $m\ge 1$ \emph{$n^{\varepsilon}$-smooth} if every prime divisor $p\mid m$ satisfies $p\le n^{\varepsilon}$ (equivalently $P(m)\le n^{\varepsilon}$, where $P(m)$ denotes the largest prime factor of $m$ and we set $P(1)=1$).

\medskip
\noindent\textbf{Claim (literal statement).} There exists $n_0(\varepsilon,k)$ such that for every $n\ge n_0(\varepsilon,k)$ there is a block of $k$ consecutive integers contained in $\{1,2,\dots,n\}$, all of which are $n^{\varepsilon}$-smooth.

\medskip
\noindent\textbf{Misstatement/ambiguity note.} The literal claim is elementary because the block may be chosen near $1$ and thus does not ``scale with $n$.'' A \emph{minimal nontrivial correction} (consistent with how such questions are usually posed) is for example:

\smallskip
\noindent\textbf{Corrected variant (one natural choice).} Fix $\varepsilon>0$ and $k\ge 2$. Is it true that for all sufficiently large integers $m$ there exist $k$ consecutive integers $m,m+1,\dots,m+k-1$ such that each is \emph{$m^{\varepsilon}$-smooth} (i.e. $P(m+j)\le m^{\varepsilon}$ for $0\le j\le k-1$)?

\subsection*{2) KEY DEFINITIONS AND KNOWN FACTS}
\begin{itemize}[leftmargin=2em]
\item $m$ is $y$-smooth means $P(m)\le y$.
\item If $1\le m\le y$, then $m$ is $y$-smooth because every prime factor of $m$ is $\le m\le y$.
\end{itemize}

\paragraph{Quick literature/context check (browsing available).}
The corrected (nontrivial) variant above is known to hold for \emph{infinitely many} starting points $m$ for every fixed $k$ and every $\varepsilon>0$ (Balog--Wooley). Whether such blocks exist for \emph{all} sufficiently large $m$ (and stronger ``near-$n$'' variants, e.g. constrained to $[n/2,n]$) remain open in general.

\subsection*{3) ATTEMPTED SOLUTION}
\textbf{PHASE 1 (quick tests / sanity).} For any fixed $k$, the consecutive block $1,2,\dots,k$ always lies inside $\{1,\dots,n\}$ once $n\ge k$. If additionally $n^{\varepsilon}\ge k$, then every prime factor of each $m\in\{1,\dots,k\}$ is at most $m\le k\le n^{\varepsilon}$, so the entire block is $n^{\varepsilon}$-smooth.

\medskip
\textbf{PHASE 2 (write a complete proof).} Choose $n_0(\varepsilon,k)$ so that $n_0\ge k$ and $n_0^{\varepsilon}\ge k$; then for all $n\ge n_0$ the block $\{1,\dots,k\}$ works.

\subsection*{4) DETAILED WORK (with complete reasoning)}
Let $\varepsilon>0$ and $k\ge 2$ be fixed.

Define
\[
 n_0(\varepsilon,k) \coloneqq \left\lceil \max\bigl\{k,\; k^{1/\varepsilon}\bigr\} \right\rceil.
\]
Then $n_0(\varepsilon,k)\ge k$ and $n_0(\varepsilon,k)^{\varepsilon}\ge k$.

Now fix any $n\ge n_0(\varepsilon,k)$. Consider the $k$ consecutive integers
\[
 1,2,\dots,k.
\]
Because $n\ge k$, we have $\{1,\dots,k\}\subseteq\{1,\dots,n\}$.

It remains to prove that each $m\in\{1,\dots,k\}$ is $n^{\varepsilon}$-smooth. Let such an $m$ be given, and let $p$ be any prime divisor of $m$. Then $p\le m\le k$. Since $n\ge n_0(\varepsilon,k)\ge k^{1/\varepsilon}$, we have $n^{\varepsilon}\ge k$. Hence
\[
 p \le k \le n^{\varepsilon}.
\]
Because every prime divisor $p$ of $m$ satisfies $p\le n^{\varepsilon}$, the integer $m$ is $n^{\varepsilon}$-smooth.

Therefore $1,2,\dots,k$ is a block of $k$ consecutive integers in $\{1,\dots,n\}$, all $n^{\varepsilon}$-smooth. This holds for every $n\ge n_0(\varepsilon,k)$.

\subsection*{5) VERIFICATION}
We check all requirements for the literal statement.
\begin{itemize}[leftmargin=2em]
\item For $n\ge n_0$, the chosen block $\{1,\dots,k\}$ is contained in $\{1,\dots,n\}$ because $n_0\ge k$.
\item For each $m\in\{1,\dots,k\}$, every prime divisor satisfies $p\le m\le k\le n^{\varepsilon}$ because $n\ge k^{1/\varepsilon}$.
\item Edge cases: if $k=2$ the same proof works; if $\varepsilon$ is small, $k^{1/\varepsilon}$ is large but finite, so $n_0$ exists.
\end{itemize}

\subsection*{6) FINAL}
\textbf{FULL SOLUTION}\hfill\textbf{FULL PROOF}

\subsection*{7) COMPLETION ESTIMATE (MANDATORY)}
\textbf{COMPLETION: 100\%}.


% =============================================================

