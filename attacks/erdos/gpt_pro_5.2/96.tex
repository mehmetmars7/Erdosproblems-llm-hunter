% Erdős Problem #96
% URL: https://www.erdosproblems.com/96

\noindent\textbf{FORMAL RESTATEMENT.}
Let $A\subset\mathbb{R}^2$ be the vertex set of a convex $n$-gon (i.e. $|A|=n$ and the points are in convex position).
Let
\[
U_\mathrm{conv}(A):=|\{\{x,y\}\subset A: |x-y|=1\}|
\]
be the number of unit-distance pairs among the vertices.
Conjecture (Erd\H{o}s--Moser): there exists an absolute constant $C$ such that for all $n$ and all convex $n$-gons,
\[
U_\mathrm{conv}(A)\le Cn.
\]

\medskip
\noindent\textbf{QUICK LITERATURE/CONTEXT CHECK.}
The problem statement reports known bounds: $O(n\log n)$ and more precise variants, and a lower bound construction with $2n-7$ unit distances. I do not re-prove those results here. Instead I record an elementary linear lower bound and an elementary $O(n^{3/2})$ upper bound (which holds without convexity).

\medskip
\noindent\textbf{ATTACK PLAN.}
\begin{itemize}
\item \emph{Lower bound sanity:} show that $\Omega(n)$ unit distances are attainable even with convexity (regular polygon).
\item \emph{Upper bound (weak):} apply the general unit-distance $O(n^{3/2})$ bound (Lemma 90.2) to convex point sets.
\item \emph{(Unresolved) improvement:} exploit convex position to beat the generic $n^{3/2}$ bound, aiming toward $O(n\log n)$ and then $O(n)$.
\end{itemize}

\medskip
\noindent\textbf{WORK.}

\smallskip
\noindent\textbf{Lemma 96.1 (convex linear lower bound).}
For every $n\ge 3$ there exists a convex $n$-gon with at least $n$ unit-distance pairs.
\textit{Proof.}
Take a regular $n$-gon and scale it so that its side length is $1$. Then each of the $n$ sides is a unit distance.
We claim no diagonal has length $1$.
Indeed, in a regular $n$-gon with circumradius $R$, the chord length between vertices at step $k$ is $2R\sin(k\pi/n)$.
After scaling to make the $k=1$ chord equal to $1$, we have $2R\sin(\pi/n)=1$. If $2R\sin(k\pi/n)=1$ also held for some $2\le k\le n-2$, then $\sin(k\pi/n)=\sin(\pi/n)$, which for $0<k\pi/n<\pi$ forces $k=1$ or $k=n-1$.
Thus only adjacent pairs are at distance $1$, giving exactly $n$ unit distances.
\hfill $\square$

\smallskip
\noindent\textbf{Lemma 96.2 (generic $O(n^{3/2})$ upper bound).}
For every convex $n$-gon vertex set $A\subset\mathbb{R}^2$,
\[
U_\mathrm{conv}(A)\le \frac{\sqrt 2}{2}\,n^{3/2}+\frac14 n.
\]
\textit{Proof.}
This is immediate from Lemma 90.2, which bounds the number of unit-distance pairs in \emph{any} $n$-point set in the plane; convexity is an additional constraint and cannot increase the maximum.\hfill $\square$

\smallskip
\noindent\textbf{FAST REALITY CHECK (regular polygons).}
For regular $n$-gons scaled to side length $1$, a direct computation confirms the unit-distance count equals $n$ for $3\le n\le 10$:
\begin{verbatim}
n=3 -> 3 unit pairs
n=4 -> 4 unit pairs
n=5 -> 5 unit pairs
...
n=10 -> 10 unit pairs
\end{verbatim}

\medskip
\noindent\textbf{VERIFICATION.}
\begin{itemize}
\item Lemma 96.1: verified the chord-length formula and the fact $\sin(k\pi/n)=\sin(\pi/n)$ has only the symmetric solutions $k=1,n-1$ in $\{1,\dots,n-1\}$.
\item Lemma 96.2: relies only on Lemma 90.2, whose proof uses the circle-intersection ($\le2$) property and a $K_{2,3}$ extremal bound.
\item Boundary case $n=3$: the regular triangle has 3 unit sides, consistent with Lemma 96.1.
\end{itemize}

\medskip
\noindent\textbf{FINAL.} \textbf{UNRESOLVED}
\begin{enumerate}
\item[(i)] \textbf{Strongest proved partial result.}
There are convex $n$-gons with $\ge n$ unit distances (Lemma 96.1). Conversely, any convex $n$-gon has $O(n^{3/2})$ unit distances by the general bound (Lemma 96.2).
\item[(ii)] \textbf{First gap (crisp).}
Use convexity to improve the $O(n^{3/2})$ bound, ideally proving $U_\mathrm{conv}(A)=O(n)$ for all convex $n$-gons.
\item[(iii)] \textbf{Top 3 next moves.}
\begin{itemize}
\item Prove a per-vertex bound: show that for a convex $n$-gon, each vertex can be at unit distance from at most $O(\log n)$ (or $O(1)$) other vertices, then sum over vertices.
\item Translate unit distances into incidences between vertices and unit circles, and exploit the \emph{cyclic order} on the convex hull to strengthen the $K_{2,3}$-free extremal-graph estimate.
\item Search computationally for convex $n$-gons with many unit distances to guess extremal patterns (e.g. mixtures of arcs/chords), which may suggest the right combinatorial decomposition.
\end{itemize}
\item[(iv)] \textbf{Likely structure of a minimal counterexample.}
A counterexample to linearity would require a convex $n$-gon with superlinear many unit chords. Such a configuration would likely place many vertices on a few arcs of circles of radius 1 around many other vertices, creating a highly constrained ``almost-circular'' structure with many repeated chord lengths.
\end{enumerate}
