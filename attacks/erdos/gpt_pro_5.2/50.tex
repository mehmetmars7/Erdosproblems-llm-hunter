% Erdős Problem #50
% Erdős Problem #50

\subsection*{Erd\H{o}s Problem \#50}
\noindent\emph{Problem statement (from the problem file).}
Schoenberg proved that for every $c\in[0,1]$ the density of $\{n\in\mathbb{N}:\varphi(n)<cn\}$ exists; let this density be $f(c)$. Is it true that there are no $x$ such that $f'(x)$ exists and is positive?

\paragraph{FORMAL RESTATEMENT.}
For $c\in[0,1]$, define
\[
S_c:=\{n\in\mathbb{N}: \varphi(n)<cn\},\qquad f(c):=\lim_{X\to\infty}\frac{1}{X}|S_c\cap [1,X]|,
\]
where existence of the limit is given.
Question: Is it true that for every $x\in(0,1)$, either $f'(x)$ does not exist, or it exists but equals $0$?
Equivalently: is there no $x$ for which the (classical) derivative exists and is strictly positive?

\paragraph{QUICK LITERATURE/CONTEXT CHECK.}
The problem file states the existence of $f(c)$ (Schoenberg) and that Erd\H{o}s proved the distribution function is purely singular.
I do not rely on any further theorems.

\paragraph{ATTACK PLAN.}
\emph{Proof track:} Try to show $f$ is purely singular in a strong sense that forces derivative $0$ a.e. and prevents points of positive derivative.
\emph{Disproof track:} Attempt to locate an interval where $f$ is absolutely continuous, or a point where a local linear increase can be proved.
\emph{Reality check:} Numerically approximate $f(c)$ for moderate cutoffs $X$ to see whether difference quotients stabilize or fluctuate.

\paragraph{WORK.}
\subparagraph{FAST REALITY CHECK (numerics).}
For cutoff $X=200000$, the empirical densities
$f_X(c):=\frac{1}{X}|\{1\le n\le X: \varphi(n)/n<c\}|$
were approximately:
\[
\begin{array}{c|cccccccccc}
 c & 0.2 &0.3&0.4&0.5&0.6&0.7&0.8&0.9\\\hline
 f_X(c) & 0.000125&0.059225&0.240820&0.511165&0.559700&0.678345&0.740715&0.786770
\end{array}
\]
(Values at $c=0.05,0.10$ were $0$ at this cutoff.)
These finite-$X$ approximations show non-smooth behaviour at small scales, but do not resolve differentiability.

\subparagraph{Lemma 1 (prime-support characterization).}
For every $n\ge 1$,
\[
\frac{\varphi(n)}{n}=\prod_{p\mid n}\Big(1-\frac{1}{p}\Big),
\]
and in particular this ratio depends only on the set of prime divisors of $n$ (not on their exponents).

\emph{Proof.}
Write the prime factorization $n=\prod_{i=1}^k p_i^{e_i}$.
Euler's product formula for $\varphi$ gives
$\varphi(n)=\prod_i (p_i^{e_i}-p_i^{e_i-1}) = n\prod_i(1-1/p_i)$.
Dividing by $n$ yields the identity.
Since the right-hand side involves only the primes $p_i$, it is independent of the exponents $e_i$. \qed

\subparagraph{Lemma 2 (a positive-density lower bound for $f(c)$ from finite prime sets).}
Fix a finite set of primes $P$ and let
$\beta(P):=\prod_{p\in P}(1-1/p)$.
Then for every $c\in[0,1]$ with $c>\beta(P)$, one has
\[
 f(c)\ge \frac{1}{\prod_{p\in P}p}.
\]

\emph{Proof.}
Let $T_P:=\{n\in\mathbb{N}: p\mid n\ \forall p\in P\}$ be the set of integers divisible by all primes in $P$.
A standard counting argument gives
\[
\lim_{X\to\infty}\frac{1}{X}|T_P\cap[1,X]|=\frac{1}{\prod_{p\in P}p},
\]
because $T_P$ is exactly the set of multiples of $\prod_{p\in P}p$.
For any $n\in T_P$, Lemma~1 gives
\[
\frac{\varphi(n)}{n}=\prod_{q\mid n}\Big(1-\frac{1}{q}\Big)\le \prod_{p\in P}\Big(1-\frac{1}{p}\Big)=\beta(P),
\]
since adding extra prime factors only decreases the product.
Thus if $c>\beta(P)$ then $T_P\subseteq S_c$.
Taking densities gives $f(c)\ge \mathrm{dens}(T_P)=1/\prod_{p\in P}p$. \qed

\subparagraph{Corollary 3 (for every $c>0$, $f(c)>0$).}
For every $c\in(0,1]$ there exists a finite prime set $P$ with $\beta(P)<c$, hence by Lemma~2, $f(c)>0$.

\emph{Proof.}
It suffices to show there exists a finite set $P$ with $\prod_{p\in P}(1-1/p)$ arbitrarily small.
Using the inequality $\log(1-x)\le -x$ for $x\in(0,1)$, we have
\[
\log\prod_{p\in P}\Big(1-\frac{1}{p}\Big)=\sum_{p\in P}\log\Big(1-\frac{1}{p}\Big)\le -\sum_{p\in P}\frac{1}{p}.
\]
Euler proved $\sum_{p\ \text{prime}} \frac{1}{p}$ diverges (one classical proof uses Euler's product for $\zeta(s)$ as $s\downarrow 1$).
Therefore we can choose $P$ so that $\sum_{p\in P} 1/p > -\log c$, which forces $\beta(P)<c$.
Then Lemma~2 implies $f(c)\ge 1/\prod_{p\in P}p>0$. \qed

\paragraph{VERIFICATION.}
\begin{itemize}
\item Lemma~1 is a direct multiplicative formula check.
\item Lemma~2 uses only the exact density of multiples of a fixed integer and the monotonicity of $\prod_{p\mid n}(1-1/p)$ in the prime-support.
\item Corollary~3 relies on divergence of $\sum 1/p$, which is classical; if one prefers to avoid it, one can replace it by the weaker statement ``there exist $n$ with $\varphi(n)/n$ arbitrarily small'' (still requiring some standard prime-number input).
\end{itemize}

\paragraph{FINAL.} \textbf{UNRESOLVED.}
\begin{enumerate}
\item[(i)] \emph{Strongest proved partial result.} The ratio $\varphi(n)/n$ depends only on the set of prime divisors (Lemma~1). For any finite prime set $P$, the density of $\{n: \varphi(n)/n\le \beta(P)\}$ is at least $1/\prod_{p\in P}p$ (Lemma~2). In particular, $f(c)>0$ for every $c>0$ (Corollary~3).
\item[(ii)] \emph{First gap.} Prove the main claim: show that no point $x\in(0,1)$ can satisfy $f'(x)$ exists and is strictly positive.
\item[(iii)] \emph{Top 3 next moves.}
(1) Derive a tractable expression for $f(c)$ as a measure on prime-supports (squarefree kernels), then analyze its regularity.
(2) Try to show that for any $x$ and any small $\delta$, the increment $f(x+\delta)-f(x)$ cannot be $\gg \delta$ because it is carried by a sparse set of prime-supports.
(3) Compute high-resolution empirical $f_X$ for large $X$ and tiny $\delta$ to look for candidates $x$ where difference quotients appear stable, then attempt to certify or refute them analytically.
\item[(iv)] \emph{Minimal counterexample structure.} A counterexample would be an $x\in(0,1)$ and an $\eta>0$ such that for all sufficiently small $\delta>0$, one has $f(x+\delta)-f(x)\ge \eta\delta$, i.e. $f$ grows at least linearly near $x$.
\end{enumerate}

