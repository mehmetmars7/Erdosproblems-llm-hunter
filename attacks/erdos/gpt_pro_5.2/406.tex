
1) FORMAL RESTATEMENT

For an integer $k\ge 0$, write $2^k$ in base $3$:
\[
2^k=\sum_{j=0}^L d_j 3^j,\quad d_j\in\{0,1,2\},\ d_L\ne 0.
\]

Question A.
Are there only finitely many $k$ such that all ternary digits satisfy $d_j\in\{0,1\}$?
Equivalently, are there only finitely many powers of $2$ that are sums of distinct powers of $3$?

Question B.
Are there only finitely many $k$ such that all ternary digits satisfy $d_j\in\{1,2\}$ (i.e. no digit $0$ appears)?

Edge cases.
$2^0=1$ has ternary representation $1$, which satisfies both digit restrictions.

2) QUICK LITERATURE/CONTEXT CHECK

I only restate what is in the problem statement.

The statement records:
- The only known examples for digits $\{0,1\}$ seem to be $1$, $4$, $256$.
- If only digits $\{1,2\}$ are allowed, $2^{15}$ seems to be the largest such power.
- Saye computed that $2^n$ contains every ternary digit for $16\le n\le 5.9\times 10^{21}$.

3) ATTACK PLAN

Proof-track:
- Translate the digit condition into a Diophantine equation in powers of $2$ and $3$.
- Derive simple congruence restrictions.

Disproof-track:
- Search for additional solutions computationally for moderate $k$.

Chosen path: give exact equivalence lemmas and confirm computationally up to $k\le 60$.

4) WORK

PHASE 1 — FAST REALITY CHECK (computed)

For $0\le k\le 60$:
- The only $k$ for which the ternary digits of $2^k$ are all in $\{0,1\}$ are
\[
 k\in\{0,2,8\},
\]
corresponding to $2^k\in\{1,4,256\}$.
- The $k$ for which the ternary digits of $2^k$ are all in $\{1,2\}$ are
\[
 k\in\{0,1,2,3,4,15\},
\]
so the largest such $k$ in this range is $15$ (value $2^{15}=32768$).

Lemma 1 (base-$3$ digit reformulation).
For $k\ge 0$, $2^k$ has a ternary expansion using only digits $0$ and $1$ if and only if there exist distinct nonnegative integers $e_1<\cdots<e_t$ such that
\[
2^k=\sum_{i=1}^t 3^{e_i}.
\]

Proof.
($\Rightarrow$) If the ternary digits of $2^k$ are in $\{0,1\}$, then
\[
2^k=\sum_{j=0}^L d_j 3^j
\]
with each $d_j\in\{0,1\}$. Let $\{e_1,\dots,e_t\}=\{j:\ d_j=1\}$. Then the sum is exactly $\sum_{i=1}^t 3^{e_i}$.

($\Leftarrow$) Conversely, if $2^k=\sum_{i=1}^t 3^{e_i}$ with distinct exponents, then in base $3$ this is precisely the expansion with digit $1$ in positions $e_i$ and digit $0$ elsewhere. Uniqueness of base-$3$ expansion ensures all digits are $0$ or $1$. \qed

Lemma 2 (a necessary congruence condition for digits $\{0,1\}$).
If $k\ge 0$ and $2^k$ has ternary digits only in $\{0,1\}$, then:
(a) the least significant ternary digit is $1$ (i.e. $2^k\equiv 1\pmod 3$), and
(b) $k$ is even.

Proof.
If all ternary digits are in $\{0,1\}$, then modulo $3$ the number is congruent to its least significant digit $d_0\in\{0,1\}$. Since $2^k$ is not divisible by $3$, we must have $d_0=1$, i.e. $2^k\equiv 1\pmod 3$.

But $2\equiv -1\pmod 3$, so $2^k\equiv (-1)^k\pmod 3$. Therefore $2^k\equiv 1\pmod 3$ forces $(-1)^k\equiv 1\pmod 3$, i.e. $k$ is even. \qed

5) VERIFICATION

- Lemma 1 is a direct unpacking of the definition of base-$3$ representation.
- Lemma 2 matches the computed solutions: $k=0,2,8$ are all even and have $2^k\equiv 1\pmod 3$.

6) FINAL

**UNRESOLVED**

(i) Strongest fully proved partial result obtained here.
- Exact equivalence between “digits $\{0,1\}$ in base $3$” and “sum of distinct powers of $3$” (Lemma 1).
- Necessary condition: any such exponent $k$ must be even (Lemma 2).
- Computation up to $k\le 60$ confirms only $k=0,2,8$ for digits $\{0,1\}$ and confirms $k=15$ as the largest solution in that range for digits $\{1,2\}$.

(ii) Exact first gap.
Prove finiteness (or exhibit infinitely many) exponents $k$ such that $2^k$ is a sum of distinct powers of $3$.

(iii) Top 3 next moves (concrete targets).
1. Strengthen congruence obstructions by working modulo $3^r$ (which reads off the first $r$ ternary digits) and simultaneously modulo $2^s$.
2. Treat $2^k=\sum 3^{e_i}$ as an $S$-unit equation and attempt to apply explicit bounds on solutions in the $(2,3)$-unit group.
3. Push computation further using meet-in-the-middle for the subset-sum in powers of $3$ (for bounded digit length) to search for larger solutions.

(iv) What a minimal counterexample would likely look like.
If infinitely many such $k$ exist, they would likely form a sparse sequence with rapidly growing ternary length and a rigid combinatorial structure of the set of exponents $\{e_i\}$ (since random subset sums of powers of $3$ rarely hit exact powers of $2$).


