%Erdos problem 514
\noindent\textbf{FORMAL RESTATEMENT}

The problem asks:

Given an entire function $f:\mathbb{C}\to\mathbb{C}$, does there exist a path (continuous curve) $L:[0,\infty)\to\mathbb{C}$ with $|L(t)|\to\infty$ as $t\to\infty$ such that for every integer $n\ge 0$,
\[
\left|\frac{f(L(t))}{L(t)^n}\right|\to\infty\quad\text{as }t\to\infty?
\]
Equivalently: along $L$, $|f(z)|$ eventually dominates $|z|^n$ for every fixed $n$.

\medskip
As written, $f$ may be a polynomial. If $f$ is a polynomial of degree $d$, then for $n>d$ we have $|f(z)/z^n|\to 0$ along every path to infinity, so the answer is \emph{no}. The note in the problem statement that Boas proved the first part strongly suggests the intended hypothesis is that $f$ is \emph{transcendental entire}.

I treat:
\begin{itemize}
\item the \emph{literal statement} (all entire functions): false (counterexample below);
\item the \emph{corrected statement} (transcendental entire): not proved here.
\end{itemize}

\medskip
\noindent\textbf{QUICK LITERATURE/CONTEXT CHECK}

The problem statement records that Boas (unpublished) proved the existence of such a path in the transcendental case. The additional quantitative questions (length bounds in terms of $M(r)$ and faster-than-$M(r)^\varepsilon$ growth along some path) remain.

\medskip
\noindent\textbf{ATTACK PLAN}

\begin{itemize}
\item Disproof-track for literal statement: exhibit a polynomial $f$.
\item Proof-track for corrected statement: attempt to build a path through successive ``large value'' regions of $f(z)/z^n$ as $n\to\infty$.
\end{itemize}

\medskip
\noindent\textbf{WORK}

\noindent\textbf{Counterexample to the literal statement.}
Take $f(z)=1$, which is entire. Consider $n=1$. Along any path $L(t)$ with $|L(t)|\to\infty$,
\[
\left|\frac{f(L(t))}{L(t)^1}\right| = \frac{1}{|L(t)|}\to 0
\quad\text{as }t\to\infty.
\]
In particular, it does not tend to $\infty$. Therefore no such path $L$ can exist for $f(z)=1$.

\medskip
\noindent\textbf{Fast reality check.}
The same obstruction holds for any polynomial: if $f$ has degree $d$, then for $n>d$, $|f(z)/z^n|\to 0$ along all paths to infinity.

\medskip
\noindent\textbf{VERIFICATION}

\begin{itemize}
\item The counterexample uses only the definition of the limit: for $f\equiv 1$ and $n=1$, the expression is exactly $1/|z|$.
\item Quantifiers: the claim being refuted is ``for every entire $f$ there exists $L$ such that for every $n$ the limit is $\infty$''. Providing one $f$ with no such $L$ suffices.
\end{itemize}

\medskip
\noindent\textbf{FINAL}

\textbf{FULL SOLUTION}

\textbf{COUNTEREXAMPLE/DISPROOF}

\textbf{Claim (literal reading).} ``For every entire function $f$ there exists a path $L$ to infinity such that for every $n$, $|f(z)/z^n|\to\infty$ along $L$.''

\textbf{Disproof.} Let $f(z)=1$. For $n=1$ and any path to infinity, $|f(z)/z|=1/|z|\to 0$, not $\infty$. Hence the claim is false. \hfill$\square$


