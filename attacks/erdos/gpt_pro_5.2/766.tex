\section*{Problem 766}

\subsection*{1) FORMAL RESTATEMENT}
For a fixed graph $G$, let $\mathrm{ex}(n;G)$ be the Tur\'an extremal number: the maximum number of edges in an
$n$-vertex (simple) graph containing no copy of $G$ as a (not necessarily induced) subgraph.

For integers $k,l$ with $0\le l\le \binom{k}{2}$ define
\[
f(n;k,l):=\min\{\mathrm{ex}(n;G): v(G)=k,\ e(G)=l\}.
\]
The task is to estimate $f(n;k,l)$ for $k<l\le k^2/4$, and to ask whether for fixed $k$ and large $n$ the function
$l\mapsto f(n;k,l)$ is strictly monotone.

\subsection*{2) QUICK LITERATURE/CONTEXT CHECK}
As recorded in the original sources and on later compilations, Dirac and Erd\H{o}s proved independently that for
\[l=\lfloor k^2/4\rfloor+1\quad\text{one has}\quad f(n;k,l)\le \lfloor n^2/4\rfloor+1.
\]
I did not locate a complete resolution for the full range $k<l\le k^2/4$.

\subsection*{3) ATTACK PLAN}
A standard approach is to compare $G$ to complete bipartite graphs and apply the K\H{o}v\'ari--S\'os--Tur\'an (KST)
upper bound. Since $l\le k^2/4$ guarantees the existence of bipartite $k$-vertex graphs with $l$ edges,
for fixed $k$ and large $n$ one expects the minimiser for $f(n;k,l)$ to be bipartite (hence $f(n;k,l)=o(n^2)$).
To go beyond crude bounds one needs to identify, among all $k$-vertex $l$-edge graphs, those with the smallest extremal number.

\subsection*{4) WORK}
\paragraph{A general constructive upper bound via KST.}
Fix $k$ and $l\le k^2/4$. Let $a\in\{1,2,\dots,\lfloor k/2\rfloor\}$ be the smallest integer such that
\[
 l\le a(k-a).
\]
Then there exists a bipartite graph $G$ on $k$ vertices with $l$ edges that is a subgraph of $K_{a,k-a}$.
Since $G\subseteq K_{a,k-a}$, any $K_{a,k-a}$-free graph is also $G$-free, hence
\[
\mathrm{ex}(n;G)\le \mathrm{ex}(n;K_{a,k-a}).
\]
Therefore
\[
 f(n;k,l)\le \mathrm{ex}(n;K_{a,k-a}).
\]
By the K\H{o}v\'ari--S\'os--Tur\'an theorem (for fixed $a$ and $b=k-a$),
\[
\mathrm{ex}(n;K_{a,b})\le (b-1)^{1/a} n^{2-1/a} + \frac{a-1}{2}n.
\]
Consequently, for fixed $k,l$ we obtain the explicit asymptotic upper bound
\[
 f(n;k,l)=O\big(n^{2-1/a}\big),\qquad a=\min\{t: l\le t(k-t)\}.
\]
In particular, if $l$ is close to the maximum $k^2/4$, then $a\approx k/2$ and the exponent is $2-2/k$.

\paragraph{Trivial universal lower bound.}
Since $l>k$ implies every such $G$ contains a cycle, any tree on $n$ vertices is $G$-free; hence
\[
\mathrm{ex}(n;G)\ge n-1\quad\text{for every }G\text{ in the minimisation, so}\quad f(n;k,l)\ge n-1.
\]
This bound is far from sharp in most regimes.

\paragraph{Heuristic for monotonicity.}
Increasing $l$ increases the edge-density of the forbidden graph class, but the minimisation ranges over
all $k$-vertex graphs with exactly $l$ edges; it is not clear that the minimiser changes monotonically with $l$.
Even if $f(n;k,l)$ is nondecreasing in $l$ for large $n$ (a plausible guess), strict monotonicity would require
showing that no two edge-counts share the same asymptotically extremal obstruction, which seems delicate.

\subsection*{5) OBSTRUCTIONS/CAVEATS}
\begin{itemize}
\item Sharp \emph{lower} bounds on $f(n;k,l)$ require showing that \emph{every} $k$-vertex $l$-edge graph has extremal number at least some quantity.
This is hard because the minimiser could be a very special sparse-in-structure bipartite graph.
\item Determining strict monotonicity in $l$ would likely require identifying the extremal graph(s) $G$ achieving the minimum,
for each $l$, at least asymptotically.
\end{itemize}

\subsection*{6) FINAL}
\textbf{UNRESOLVED.}
\begin{itemize}
\item[(i)] \emph{Where I got stuck:} I did not determine the correct asymptotic order of $f(n;k,l)$ for the full range nor settle strict monotonicity.
\item[(ii)] \emph{Partial progress:} I derived an explicit KST-based upper bound $f(n;k,l)=O(n^{2-1/a})$ with $a$ determined by $l\le a(k-a)$, and recorded the trivial universal lower bound $n-1$.
\item[(iii)] \emph{What seems next:} Improve lower bounds by relating any $k$-vertex $l$-edge graph to a forced bipartite substructure (e.g. a fixed $K_{s,t}$, theta graph, or even cycle) whose extremal number is well understood;
then optimise over which such substructures must appear as a function of $k,l$.
\item[(iv)] \emph{Best guess:} For fixed $k$ and large $n$, the minimiser should be bipartite; the correct exponent likely depends on the smallest $s$ for which some $K_{s,t}$-type configuration is unavoidable in all $k$-vertex $l$-edge graphs.
\end{itemize}

\subsection*{7) COMPLETION ESTIMATE}
$10\%$.

