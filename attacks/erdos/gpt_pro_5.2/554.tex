
Let $R(G;k)$ denote the minimal $m$ such that if the edges of $K_m$ are $k$-coloured then there is a monochromatic copy of $G$. Show that 
\[
\lim_{k\to \infty} \frac{R(C_{2n+1};k)}{R(K_3;k)} =0
\]
for any $n\ge 2$.

This is open even for $n=2$ (i.e. for $C_5$).

%Erdos problem 554
1) “FORMAL RESTATEMENT”

For a fixed graph $G$ and integer $k\ge 1$, let $R(G;k)$ be the smallest $m$ such that every $k$-edge-colouring of $K_m$ contains a monochromatic copy of $G$.

Fix $n\ge 2$. The problem asks to prove
\[
\lim_{k\to\infty} \frac{R(C_{2n+1};k)}{R(K_3;k)} = 0.
\]
The problem file states this is open even for $n=2$.

2) “QUICK LITERATURE/CONTEXT CHECK”

The only context given is that the claim is open even for $C_5$. I will therefore only prove elementary bounds and structural lemmas, and then state the precise gap.

3) “ATTACK PLAN”

To prove the limit is $0$, one needs an upper bound on $R(C_{2n+1};k)$ and a lower bound on $R(K_3;k)$ whose ratio tends to infinity.

I can give an explicit construction showing both Ramsey numbers exceed $2^k$ (so this does not separate them), and I can give very coarse upper bounds via clique Ramsey numbers. These are far from sufficient to prove the stated limit.

4) “WORK”

\textbf{Lemma 1 (A $k$-colouring of $K_{2^k}$ with no monochromatic triangle).}
For each $k\ge 1$, there is a $k$-edge-colouring of $K_{2^k}$ with no monochromatic $K_3$. In particular,
\[
R(K_3;k) > 2^k.
\]

\emph{Proof.}
Let the vertex set be $\{0,1\}^k$ (binary strings of length $k$), so there are $2^k$ vertices.
For two distinct vertices $x\neq y$, define the colour of the edge $xy$ to be the least index $i\in\{1,\dots,k\}$ for which $x_i\neq y_i$.

Suppose for contradiction that there is a monochromatic triangle with vertices $a,b,c$ and all three edges coloured $i$. Then for each pair among $\{a,b,c\}$, the first coordinate where the two vertices differ is exactly $i$. In particular, all three vertices agree in coordinates $1,2,\dots,i-1$.
Now look at the $i$th bits $a_i,b_i,c_i\in\{0,1\}$. Among three bits, two are equal; say $a_i=b_i$. Then $a$ and $b$ do \emph{not} differ at coordinate $i$, so their first differing coordinate is strictly greater than $i$, contradicting that the edge $ab$ has colour $i$.
Thus there is no monochromatic triangle.

Therefore this is a $k$-colouring of $K_{2^k}$ without a monochromatic $K_3$, implying $R(K_3;k)>2^k$. $\square$

\textbf{Lemma 2 (The same colouring forbids every monochromatic odd cycle).}
In the colouring from Lemma 1, for each fixed colour $i$, the monochromatic graph of colour $i$ is bipartite, hence contains no odd cycle. In particular, for every $n\ge 2$,
\[
R(C_{2n+1};k) > 2^k.
\]

\emph{Proof.}
Fix a colour $i$. Every edge of colour $i$ joins two vertices that differ at coordinate $i$ (by definition). Therefore all colour-$i$ edges go between the two vertex classes
\[
A:=\{x\in\{0,1\}^k : x_i=0\},\qquad B:=\{x\in\{0,1\}^k : x_i=1\}.
\]
Hence the colour-$i$ graph is bipartite. Bipartite graphs contain no odd cycles, so in particular no $C_{2n+1}$.
Thus the colouring has no monochromatic $C_{2n+1}$, proving $R(C_{2n+1};k)>2^k$. $\square$

\textbf{Lemma 3 (Trivial upper bound via cliques).}
For every $n\ge 2$ and $k\ge 1$,
\[
R(C_{2n+1};k) \le R(K_{2n+1};k).
\]

\emph{Proof.}
A complete graph $K_{2n+1}$ contains $C_{2n+1}$ as a subgraph (take a Hamilton cycle). Therefore any monochromatic $K_{2n+1}$ yields a monochromatic $C_{2n+1}$. Hence the minimal $m$ forcing a monochromatic $C_{2n+1}$ is at most the minimal $m$ forcing a monochromatic $K_{2n+1}$. $\square$

5) “VERIFICATION”

Fast reality check (computer verification of Lemma 1 for small $k$):
- For $k=1$, $K_{2}$ has no triangle, so the construction vacuously works.
- For $k=2,3,4$, a direct programmatic check on the construction on $2^k$ vertices confirms there is no monochromatic triangle.

6) FINAL

**UNRESOLVED**

(i) Strongest proved partial result: Lemmas 1 and 2 give explicit lower bounds $R(K_3;k)>2^k$ and $R(C_{2n+1};k)>2^k$ for every fixed $n\ge 2$. Lemma 3 gives the trivial upper bound $R(C_{2n+1};k)\le R(K_{2n+1};k)$.

(ii) First gap (crisp): Prove a separation between the growth rates of $R(C_{2n+1};k)$ and $R(K_3;k)$ strong enough to imply $R(C_{2n+1};k)=o(R(K_3;k))$ as $k\to\infty$. With only the shared lower bound $>2^k$ and trivial clique upper bounds, the ratio is not controlled.

(iii) Top 3 next moves:
1. Improve lower bounds for $R(K_3;k)$: find constructions or counting arguments showing $R(K_3;k)$ grows super-exponentially in $k$ (beyond $2^k$), in a way that can be compared to known/derived upper bounds for odd cycles.
2. Improve upper bounds for $R(C_{2n+1};k)$: exploit the fixed cycle structure (rather than bounding by cliques) to get $R(C_{2n+1};k)\le \exp(o(k\log k))$ or similarly, and then compare to the best available lower bounds for $R(K_3;k)$.
3. Small-$k$ experiments: compute or bound $R(C_{2n+1};k)$ and $R(K_3;k)$ for small $k$ and $n=2,3$ to look for structural patterns in extremal colourings.

(iv) Minimal counterexample structure: A counterexample to the limit statement would amount to an infinite sequence of integers $k_j\to\infty$ such that $R(C_{2n+1};k_j)$ is bounded below by a positive constant multiple of $R(K_3;k_j)$. Equivalently, one would need $k_j$-colourings of $K_{m}$ with $m$ close to $R(K_3;k_j)$ that still avoid monochromatic $C_{2n+1}$, showing odd-cycle avoidance can persist nearly as long as triangle avoidance.

