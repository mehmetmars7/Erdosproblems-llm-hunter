
\subsection*{Erd\H{o}s Problem 685}

\paragraph{1) FORMAL RESTATEMENT.}
For integers $n\ge 2$ and $1\le k\le n-1$, write
\[
\omega\!\left(\binom{n}{k}\right):=\#\{\,\text{primes }p:\ p\mid \binom{n}{k}\,\}
\]
for the number of \emph{distinct} prime divisors of $\binom{n}{k}$.
The question is whether the following asymptotic statement is true.

\medskip
\noindent\textbf{Conjectured asymptotic.}
Fix $\epsilon>0$. As $n\to\infty$, uniformly for all integers $k$ with
\[
n^\epsilon<k\le n^{1-\epsilon},
\]
do we have
\[
\omega\!\left(\binom{n}{k}\right)
=(1+o(1))\,k\sum_{\substack{\text{prime }p\\ k<p<n}}\frac1p\,?
\]
Here $o(1)$ means a quantity tending to $0$ as $n\to\infty$ (possibly depending on~$\epsilon$).

\paragraph{2) QUICK LITERATURE/CONTEXT CHECK.}
No external literature is used or claimed here beyond what is explicitly written in the problem statement.
The main heuristic is that for primes $p>k$, the event $p\mid \binom{n}{k}$ is equivalent to a simple residue condition
($n\bmod p<k$), which has ``probability'' about $k/p$ if residues behave randomly; summing $k/p$
over primes $p\in(k,n)$ gives the proposed main term.

\paragraph{3) ATTACK PLAN.}
\begin{itemize}
\item \textbf{Proof-track idea:} reduce (most of) $\omega(\binom{n}{k})$ to a sum of indicators
$\sum_{k<p<n}\mathbf 1_{n\bmod p<k}$ (Lemma~685.1 below) and then prove concentration around its mean
$k\sum_{k<p<n}1/p$ via a variance/correlation estimate (e.g.\ large-sieve--type bounds).
\item \textbf{Disproof-track idea:} try to construct $n$ (as a function of $k$) so that $n\bmod p$ lies in $\{0,1,\dots,k-1\}$
for ``too many'' primes $p\in(k,n)$, i.e.\ make $n$ unusually close to a multiple of many primes; then check whether this can beat the
conjectured main term.
\end{itemize}

\paragraph{4) WORK.}
\medskip
\noindent\textbf{Lemma 685.1 (exact divisibility criterion for primes $p>k$).}
Let $n\ge 1$, $1\le k\le n$, and let $p$ be a prime with $p>k$.
Write $n=ap+b$ with integers $a\ge 0$ and $0\le b<p$. Then
\[
p\mid \binom{n}{k}\quad\Longleftrightarrow\quad b<k,
\]
i.e.\ $p\mid \binom{n}{k}$ if and only if $n\bmod p\in\{0,1,\dots,k-1\}$.

\emph{Proof.}
We use Vandermonde's identity:
\[
\binom{ap+b}{k}=\sum_{i=0}^k \binom{ap}{i}\binom{b}{k-i}.
\]
Fix an integer $i$ with $1\le i\le k<p$. We claim $p\mid \binom{ap}{i}$.
Indeed,
\[
\binom{ap}{i}=\frac{ap(ap-1)\cdots (ap-i+1)}{i!}.
\]
Among the $i$ consecutive integers $ap,ap-1,\dots,ap-i+1$ with $i<p$, exactly one is divisible by $p$, namely $ap$.
Hence the numerator has $p$-adic valuation exactly $1$. On the other hand $i!$ is not divisible by $p$ because $i<p$.
Therefore $v_p\!\left(\binom{ap}{i}\right)=1$ and in particular $p\mid \binom{ap}{i}$.

Reducing Vandermonde's sum modulo $p$, all terms with $i\ge 1$ vanish, so
\[
\binom{ap+b}{k}\equiv \binom{b}{k}\pmod p.
\]
If $b<k$ then $\binom{b}{k}=0$, so $p\mid \binom{n}{k}$.
If $b\ge k$, then $\binom{b}{k}$ is a nonzero integer whose numerator and denominator involve only factors $<p$,
so neither is divisible by $p$, hence $p\nmid \binom{b}{k}$ and thus $p\nmid \binom{n}{k}$.
This proves the equivalence. \hfill$\square$

\medskip
\noindent\textbf{Lemma 685.2 (mean value over $n$ for the $p>k$ contribution).}
Fix integers $N>k\ge 1$. For each $n\in\{1,2,\dots,N\}$ define
\[
A(n):=\#\{\,\text{primes }p:\ k<p\le N,\ \ p\mid \binom{n}{k}\,\}.
\]
Then
\[
\frac1N\sum_{n=1}^N A(n)=k\sum_{\substack{\text{prime }p\\ k<p\le N}}\frac1p\;+\;O\!\left(\frac{k\,\pi(N)}{N}\right),
\]
where $\pi(N)$ is the number of primes $\le N$.

\emph{Proof.}
By Lemma~685.1,
\[
A(n)=\sum_{\substack{\text{prime }p\\ k<p\le N}} \mathbf 1_{\,n\bmod p<k}.
\]
Averaging over $n\in\{1,\dots,N\}$ and interchanging sum and average gives
\[
\frac1N\sum_{n=1}^N A(n)=\sum_{\substack{\text{prime }p\\ k<p\le N}}
\left(\frac1N\sum_{n=1}^N \mathbf 1_{\,n\bmod p<k}\right).
\]
Fix such a prime $p$. Partition $\{1,\dots,N\}$ into $\lfloor N/p\rfloor$ full blocks of length $p$
and one final incomplete block of length $<p$.
In each full block, exactly $k$ residues satisfy $n\bmod p<k$, so the count of $n\le N$ with $n\bmod p<k$ is
\[
k\left\lfloor\frac{N}{p}\right\rfloor + r_p,
\qquad\text{with}\qquad 0\le r_p\le k,
\]
where $r_p$ is the contribution of the final incomplete block.
Therefore
\[
\frac1N\sum_{n=1}^N \mathbf 1_{\,n\bmod p<k}
=\frac{k}{p}+O\!\left(\frac{k}{N}\right).
\]
Summing this over primes $k<p\le N$ yields
\[
\frac1N\sum_{n=1}^N A(n)=k\sum_{k<p\le N}\frac1p+O\!\left(\frac{k}{N}\cdot \pi(N)\right),
\]
as claimed. \hfill$\square$

\medskip
\noindent\textbf{FAST REALITY CHECK (local computation).}
I computed $\omega\!\left(\binom{n}{k}\right)$ exactly using Legendre's formula for prime exponents in factorials,
and compared it to the heuristic main term $k\sum_{k<p<n}1/p$ (computed in floating point).
For $n=100000$ (primes up to $n$ are $9592$) the script printed:
\begin{verbatim}
n=100000
k=   10  omega=   16  pred=  15.290817028567766  ratio=1.046379665004641
k=   31  omega=   40  pred=  35.32683562243201   ratio=1.1322836958145384
k=  100  omega=  107  pred=  90.2454977998395    ratio=1.1856547152891908
k=  177  omega=  167  pred= 139.45259620695515   ratio=1.1975395549621968
k=  316  omega=  264  pred= 215.74872169726672   ratio=1.2236457204619655
k= 1000  omega=  636  pred= 507.19205187217455   ratio=1.253962868014912
k= 3162  omega= 1481  pred=1125.679237358206     ratio=1.315650098935534
\end{verbatim}
These are only sanity checks at a single $n$, not evidence of an asymptotic theorem.

\paragraph{5) VERIFICATION.}
\begin{itemize}
\item Lemma~685.1 uses only the hypothesis $p>k$ (so that $p\nmid k!$ and $i!$ for $i\le k$), and otherwise is an exact congruence.
\item Lemma~685.2 is a direct averaging identity and does not assume any distribution of primes beyond their finiteness up to $N$.
\item The conjecture concerns \emph{all} primes dividing $\binom{n}{k}$, including primes $p\le k$; Lemmas~685.1--685.2 address only $p>k$.
A complete proof would have to show that the contribution of $p\le k$ is lower order (or is absorbed into the stated main term).
\end{itemize}

\paragraph{FINAL.}
\noindent\textbf{UNRESOLVED}

\smallskip
\noindent(i) \textbf{Strongest proved partial result.}
For every prime $p>k$, we have an \emph{exact} criterion
\[
p\mid \binom{n}{k}\ \Longleftrightarrow\ n\bmod p<k
\]
(Lemma~685.1), and consequently the mean over $n\le N$ of the count of such primes equals
$k\sum_{k<p\le N}1/p$ up to an explicit $O(k\pi(N)/N)$ error (Lemma~685.2).

\smallskip
\noindent(ii) \textbf{First gap (crisp statement).}
Prove \emph{uniform concentration} for
\[
\sum_{\substack{\text{prime }p\\ k<p<n}} \mathbf 1_{\,n\bmod p<k}
\]
around its mean $k\sum_{k<p<n}1/p$ for all $n^\epsilon<k\le n^{1-\epsilon}$, and also control the contribution of primes $p\le k$
to $\omega\!\left(\binom{n}{k}\right)$.

\smallskip
\noindent(iii) \textbf{Top 3 next moves.}
\begin{enumerate}
\item Compute the second moment over $n\le N$ of $A(n)$ from Lemma~685.2, i.e.\ bound
$\sum_{n\le N}(A(n)-\mathbb E A)^2$ by estimating correlations
$\#\{n\le N:\ n\bmod p<k,\ n\bmod q<k\}$ for distinct primes $p,q$.
\item Attempt a large-sieve--type inequality to show that for most $n$, the residue constraints $n\bmod p<k$ do not hold for too many primes.
\item Isolate primes $p\le k$: express $\omega\!\left(\binom{n}{k}\right)$ as
$\omega_{>k}(\binom{n}{k})+\omega_{\le k}(\binom{n}{k})$ and prove a uniform upper bound on $\omega_{\le k}(\binom{n}{k})$
that is $o\!\left(k\sum_{k<p<n}1/p\right)$ in the regime $n^\epsilon<k\le n^{1-\epsilon}$.
\end{enumerate}

\smallskip
\noindent(iv) \textbf{What a minimal counterexample would likely look like.}
A counterexample would require some $n,k$ with $n^\epsilon<k\le n^{1-\epsilon}$ such that $n$ lies in the ``small residue'' set
$n\bmod p\in\{0,\dots,k-1\}$ for \emph{significantly} more (or fewer) primes $p\in(k,n)$ than the heuristic $k/p$ suggests,
i.e.\ $n$ is unusually close to multiples of many primes in $(k,n)$; additionally, primes $\le k$ would need to contribute non-negligibly.


