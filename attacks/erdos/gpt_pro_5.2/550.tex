% Erdos Problem #550

Let $m_1\leq \ldots \leq m_k$ and $n$ be sufficiently large. If $T$ is a tree on $n$ vertices and $G$ is the complete multipartite graph with vertex class sizes $m_1,\ldots,m_k$ then prove that 
\[
R(T,G) \leq (\chi(G)-1)(R(T,K_{m_1,m_2})-1)+m_1.
\]

Chv\'atal [Ch77] proved that $R(T,K_m)=(m-1)(n-1)+1$.

%Erdos problem 550
1) “FORMAL RESTATEMENT”

Let $m_1\le \cdots \le m_k$ be fixed positive integers, and let $T$ be a tree on $n$ vertices. Let $G$ be the complete $k$-partite graph with part sizes $m_1,\dots,m_k$ (so $\chi(G)=k$).

The task is to prove that for $n$ sufficiently large,
\[
R(T,G)\ \le\ (k-1)\bigl(R(T,K_{m_1,m_2})-1\bigr)+m_1.
\]
Here $R(T,G)$ is the two-colour Ramsey number: the smallest $N$ such that every red/blue colouring of $E(K_N)$ contains either a red copy of $T$ or a blue copy of $G$.

2) “QUICK LITERATURE/CONTEXT CHECK”

The only external fact stated in the problem file is Chv\'atal's theorem: for a tree $T$ on $n$ vertices,
\[
R(T,K_m)=(m-1)(n-1)+1.
\]
I will not assume any other literature.

3) “ATTACK PLAN”

Try to relate the multipartite target $G$ to the bipartite target $K_{m_1,m_2}$ via monotonicity and structural reductions (e.g. by finding $k$ vertex sets with all cross-edges blue). A natural approach is to study the red graph $R$ of a colouring: if $R$ has many connected components, then cross-edges between components are automatically blue, giving multipartite structure; if $R$ has few components, then some component is large and (since it avoids a red $T$) should contain a blue $K_{m_1,m_2}$ by the definition of $R(T,K_{m_1,m_2})$. One would then try to combine this internal bipartite structure with other components to build the full $k$-partite $G$.

I was not able to complete the last step.

4) “WORK”

Set
\[
R_0 := R(T,K_{m_1,m_2}).
\]

\textbf{Lemma 1 (Monotonicity in the blue target).}
If $H$ is a subgraph of $G$ (i.e. $H$ can be obtained from $G$ by deleting vertices and/or edges), then
\[
R(T,H)\le R(T,G).
\]

\emph{Proof.}
Let $N=R(T,G)$. In any red/blue colouring of $K_N$, either there is a red copy of $T$ or a blue copy of $G$. In the latter case, since $H$ is a subgraph of $G$, the same vertex set contains a blue copy of $H$ by restricting to the vertices/edges corresponding to $H$. Hence $N$ also witnesses the Ramsey property for $(T,H)$, so $R(T,H)\le N=R(T,G)$. $\square$

\textbf{Lemma 2 (For $n$ large, the parts $m_i$ are bounded by $R_0-1$).}
For every $n$ and every $m_1,m_2$, we have $R_0\ge n$. In particular, if $n\ge m_k+1$ then $R_0-1\ge m_k\ge m_i$ for all $i\ge 2$.

\emph{Proof.}
By definition, $R_0$ is the smallest $N$ such that every colouring of $K_N$ contains a red copy of $T$ or a blue copy of $K_{m_1,m_2}$. If $N<n$, then $K_N$ does not even have enough vertices to host a copy of $T$, so colouring all edges red produces a colouring with no red $T$ and (since there are no blue edges) no blue $K_{m_1,m_2}$. Hence $N<n$ cannot satisfy the Ramsey property, so $R_0\ge n$.

If $n\ge m_k+1$ then $R_0\ge n\ge m_k+1$ and thus $R_0-1\ge m_k$. $\square$

\textbf{Lemma 3 (Red components yield blue complete multipartite subgraphs).}
Fix a red/blue colouring of $K_N$ and let $R$ be the graph of red edges. Let $C_1,\dots,C_s$ be vertex-disjoint sets such that there are \emph{no} red edges between $C_i$ and $C_j$ for $i\neq j$ (for instance, distinct connected components of $R$).
Then every edge between $C_i$ and $C_j$ is blue, and therefore for any choice of subsets $C_i'\subseteq C_i$ the blue graph contains the complete multipartite graph with parts $C_1',\dots,C_s'$.

\emph{Proof.}
If there is no red edge between $C_i$ and $C_j$, then every edge between them must be blue because every edge of $K_N$ is coloured either red or blue. Thus all cross-edges between the parts are blue, exactly as required for a blue copy of the corresponding complete multipartite graph. $\square$

\textbf{Lemma 4 (Sanity check: the inequality matches Chv\'atal for cliques).}
If $m_1=m_2=\cdots=m_k=1$, then $G=K_k$ and the claimed inequality becomes
\[
R(T,K_k)\le (k-1)(R(T,K_2)-1)+1.
\]
Moreover $R(T,K_2)=n$, so the right-hand side equals $(k-1)(n-1)+1$, which is exactly Chv\'atal's value of $R(T,K_k)$.

\emph{Proof.}
$K_2$ is a single edge. In any colouring of $K_n$, either there is a blue edge or all edges are red; in the latter case the red graph is $K_n$ and contains $T$ (since $|T|=n$). Thus $R(T,K_2)=n$.
Substituting gives $(k-1)(n-1)+1$, matching the stated Chv\'atal theorem. $\square$

\textbf{Where the proof gets stuck.}
Assume $n\ge m_k+1$ so that $m_i\le R_0-1$ for all $i\ge 2$ (Lemma 2). By Lemma 1, it would suffice to prove an inequality for a larger complete multipartite graph whose $(k-1)$ largest parts are all size $R_0-1$, but I do not see how to force such a multipartite structure from the hypothesis “no red $T$” using only the information encoded in $R_0$.

5) “VERIFICATION”

Fast sanity checks:
- For $k=2$ the inequality reads $R(T,K_{m_1,m_2})\le (1)(R(T,K_{m_1,m_2})-1)+m_1$, which is tautologically true.
- For $m_1=\cdots=m_k=1$ (so $G=K_k$), Lemma 4 shows the inequality holds with equality and is consistent with Chv\'atal.

6) FINAL

**UNRESOLVED**

(i) Strongest proved partial result: Lemmas 1–3 give a structural reduction toolkit: monotonicity in the blue target, the bound $R(T,K_{m_1,m_2})\ge n$, and the fact that vertex sets with no red edges between them automatically form a blue complete multipartite graph. Lemma 4 verifies the claimed inequality matches the known exact formula when $G$ is a clique.

(ii) First gap (crisp): Given a red/blue colouring of $K_N$ with $N=(k-1)(R_0-1)+m_1$ and with no red $T$, prove that the blue graph must contain $K_{m_1,\dots,m_k}$. Equivalently, from “no red $T$” plus the definition of $R_0$, extract $k$ disjoint vertex sets $V_1,\dots,V_k$ with $|V_i|=m_i$ such that all cross-edges between $V_i$ and $V_j$ are blue.

(iii) Top 3 next moves:
1. Component-splitting strategy: Formalise an induction on $k$ using the red component decomposition. Show that if the red graph has too few components, then a large component must contain a blue $K_{m_1,m_2}$; then prove a lemma that this internal bipartite structure can be combined with other red components (or further splittings) to create the full $k$-partite $G$.
2. Minimal counterexample analysis: Assume a smallest $N$ and a counterexample colouring on $N-1$ vertices exists. Derive necessary structure (e.g. constraints on the sizes/number of red components and on where blue $K_{m_1,m_2}$ copies can occur) and attempt to contradict the vertex count $N-1$.
3. Special cases: Prove the inequality for $k=3$ first (tripartite $G$), or for restricted multipartite shapes (e.g. $m_2=\cdots=m_k$), then attempt to bootstrap to general $m_3,\dots,m_k$ via Lemma 1.

(iv) Minimal counterexample structure: A counterexample would be a 2-colouring of $K_{N-1}$, where $N=(k-1)(R_0-1)+m_1$, with no red copy of the given $n$-vertex tree $T$ and no blue copy of the complete $k$-partite graph with part sizes $m_1,\dots,m_k$.

