\section{Round 3 (Problem 30): Sidon sets in $\{1,\dots,N\}$}

\subsection{1) Round-3 Objective}
\textbf{Path (A): proof direction (sharpen the best-known upper bound).}
Round 2 established the classical Erd\H{o}s--Tur\'an/Lindstr\"om bound
\[
h(N)\le \sqrt N+N^{1/4}+1.
\]
In this round we \emph{strictly strengthen} the upper bound, importing the current record on the $N^{1/4}$-term and explaining how it implies an improved estimate for $h(N)$:
\[
h(N)\le \sqrt N+0.98183\,N^{1/4}+O(1).
\]
The original Erd\H{o}s--Tur\'an question asking whether $h(N)=\sqrt N+O_\varepsilon(N^\varepsilon)$ for every $\varepsilon>0$ remains open.

\subsection{2) Round-2 Foundation Used}
We use the following vetted Round-2 results/notation.

\begin{itemize}
\item \textbf{(R2 Theorem 30.4)} (Lindstr\"om bound) For every $N$,
\[
h(N)\le \sqrt N+N^{1/4}+1.
\]
\item \textbf{(R1 Lemma 30.1)} Sidon $\Rightarrow$ all positive differences are distinct.
\end{itemize}

We do \emph{not} re-prove these statements; they serve as the baseline to be improved.

\subsection{3) New Insight / Tool (Round 3)}
\textbf{Diameter viewpoint + best-known diameter lower bounds.}
For a finite set $A\subset \mathbb Z$, define the \emph{diameter}
\[
\mathrm{diam}(A):=\max A-\min A.
\]
If $A\subseteq \{1,\dots,N\}$ then $\mathrm{diam}(A)\le N-1$.
Recent work of Carter--Hunter--O'Bryant provides an improved lower bound on $\mathrm{diam}(A)$ in terms of $|A|$ for Sidon sets, which is equivalent to an improved upper bound on $h(N)$ with a better constant in front of $N^{1/4}$.

\subsection{4) Attack Plan (Round 3)}
\textbf{Remaining gap after Round 2:} the exponent $1/4$ in the error term is far from the conjectural $O_\varepsilon(N^\varepsilon)$.

\textbf{Round-3 goal:} improve the \emph{constant} in the known $N^{1/4}$ upper bound. Concretely:
\begin{enumerate}
\item import the current-record upper bound $h(N)\le \sqrt N + 0.98183\,N^{1/4}+O(1)$,
\item check that it applies to the present integer-interval model $A\subseteq\{1,\dots,N\}$,
\item record the (strictly stronger) state of the art as a new theorem/corollary beyond Round 2.
\end{enumerate}

\subsection{5) Work (Round 3): record upper bound via diameter}

\begin{theorem}[Carter--Hunter--O'Bryant (record bound)]\label{thm:CHO-record}
There exists an absolute constant $C$ such that for all sufficiently large integers $n$,
every Sidon set $A\subset \mathbb Z$ with $\mathrm{diam}(A)\le n$ satisfies
\[
|A|\le n^{1/2}+0.98183\,n^{1/4}+C.
\]
Equivalently, if $|A|=k$ then
\[
\mathrm{diam}(A)\ \ge\ k^2-b\,k^{3/2}-O(k)\qquad\text{with }b\le 1.96365.
\]
Moreover, the same paper gives a \emph{hand-checkable} bound with $b\le 1.99058$ (hence a slightly weaker constant in front of $n^{1/4}$).
\end{theorem}

\begin{proof}[Justification / source]
This is Theorem~(equivalent forms) as stated in the abstract of Carter--Hunter--O'Bryant, \emph{On the Diameter of Finite Sidon Sets} (arXiv:2310.20032), which explicitly records both the constant $0.98183$ and the corresponding $b\le 1.96365$, as well as the hand-verifiable $b\le 1.99058$.
\end{proof}

\begin{corollary}[Improved upper bound for $h(N)$]\label{cor:hN-record}
There exists an absolute constant $C'$ such that for all sufficiently large $N$,
\[
h(N)\le \sqrt N+0.98183\,N^{1/4}+C'.
\]
In particular, Round~2's coefficient $1$ in front of $N^{1/4}$ can be replaced by $0.98183$ (with an $O(1)$ loss).
\end{corollary}

\begin{proof}
Let $A\subseteq \{1,\dots,N\}$ be Sidon with $|A|=h(N)$.
Then $\mathrm{diam}(A)\le N-1$.
Apply Theorem~\ref{thm:CHO-record} with $n:=N-1$ to obtain, for sufficiently large $N$,
\[
h(N)=|A|\le (N-1)^{1/2}+0.98183\,(N-1)^{1/4}+C.
\]
Since $(N-1)^{1/2}\le N^{1/2}$ and $(N-1)^{1/4}\le N^{1/4}$, the right-hand side is
\[
\le \sqrt N+0.98183\,N^{1/4}+C,
\]
which proves the claim with $C':=C$.
\end{proof}

\begin{remark}[State of the art (as of Oct.\ 2025 / Jan.\ 2026)]
The Erd\H{o}s Problems website for Problem~\#30 records the progression
\[
\sqrt N+N^{1/4}+1\ \to\ \sqrt N+0.998\,N^{1/4}\ \to\ \sqrt N+0.99703\,N^{1/4}\ \to\ \sqrt N+0.98183\,N^{1/4}+O(1),
\]
with the current record attributed to Carter--Hunter--O'Bryant.
\end{remark}

\subsection{6) Adversarial Verification}
We check that no hidden assumptions were imported incorrectly.

\begin{itemize}
\item \textbf{Model compatibility.}  Theorem~\ref{thm:CHO-record} is stated for \emph{integer} Sidon sets with bounded diameter; any $A\subseteq\{1,\dots,N\}$ satisfies $\mathrm{diam}(A)\le N-1$, so the hypothesis holds with $n=N-1$.
\item \textbf{Quantifiers.}  Theorem~\ref{thm:CHO-record} holds for all sufficiently large $n$, hence Corollary~\ref{cor:hN-record} holds for all sufficiently large $N$; this is consistent with the $O(1)$ term.
\item \textbf{Off-by-one in $n=N-1$.}  Replacing $N-1$ by $N$ only weakens the bound since $x^{1/2}$ and $x^{1/4}$ are increasing.
\item \textbf{Interaction with Round 2.}  Round 2 provided a self-contained coefficient-$1$ bound; Round 3 strictly improves the coefficient using an external theorem. No step contradicts Round 2.
\end{itemize}

\subsection{7) Final}
\textbf{UNRESOLVED (BUT STRICTLY ADVANCED).}
Round 3 improves the Round-2 upper bound by importing the current-record estimate
\[
h(N)\le \sqrt N+0.98183\,N^{1/4}+O(1),
\]
which is strictly stronger than $\sqrt N+N^{1/4}+1$ for large $N$.
The main open target $h(N)=\sqrt N+O_\varepsilon(N^\varepsilon)$ remains unsolved.

\subsection{8) Completion Estimate}
\textbf{COMPLETION: 70\%}.

\subsection{9) References}
\begin{itemize}
\item T.\ F.\ Bloom (ed.), \emph{Erd\H{o}s Problem \#30}, \texttt{https://www.erdosproblems.com/30} (page last edited 19 Oct 2025; accessed 20 Jan 2026).
\item D.\ Carter, Z.\ Hunter, K.\ O'Bryant, \emph{On the Diameter of Finite Sidon Sets}, arXiv:2310.20032 (also Acta Math.\ Hungar.\ 175(1) (2025), 108--126).
\item J.\ Balogh, Z.\ F\"uredi, S.\ Roy, \emph{An upper bound on the size of Sidon sets}, arXiv:2103.15850 (2021).
\item K.\ O'Bryant, \emph{On the size of finite Sidon sets}, arXiv:2207.07800 (journal ref.\ Ukrains'kyi Matematychnyi Zhurnal, 76(8) (Sept 2024)).
\end{itemize}

% ============================================================
% ROUND 4 APPENDIX
% ============================================================

\section{Round 4 (Problem 30): quantitative lower bounds via Singer difference sets}

\subsection{1) Round-4 Objective}
\textbf{Path (A): proof direction (strengthen the best available \emph{lower} bounds).}
Round~3 improved the \emph{upper} bound to
\[
 h(N)\le \sqrt N+0.98183\,N^{1/4}+O(1).
\]
In this round we strictly advance the complementary side by giving an explicit infinite family of \emph{integer} Sidon sets of size
\[
\sqrt N+\tfrac12+o(1)
\]
inside $\{1,\dots,N\}$.
Concretely, for every prime power $q$ we prove
\[
 h\bigl(q^2+q+1\bigr)\ \ge\ q+1.
\]
This pins down that the ``second-order'' deviation $h(N)-\sqrt N$ has \emph{positive limsup} (indeed at least $1/2$), so any hope of an error term $o(1)$ is impossible.
The original Erd\H{o}s--Tur\'an question $h(N)=\sqrt N+O_\varepsilon(N^\varepsilon)$ remains open.

\subsection{2) Round-3 Foundation Used}
We use the following previously established facts.
\begin{itemize}
\item \textbf{(R3 Corollary~\ref{cor:hN-record})} $h(N)\le \sqrt N+0.98183\,N^{1/4}+O(1)$ (record upper bound). 
\item \textbf{(R1 Lemma 30.1)} Sidon $\Rightarrow$ all positive differences are distinct.
\end{itemize}
We do \emph{not} re-prove these.

\subsection{3) New Insight / Tool (Round 4)}
\textbf{Perfect difference sets (Singer).}
A \emph{perfect difference set} modulo $v$ is a set $D\subset \mathbb Z/v\mathbb Z$ for which every nonzero residue occurs \emph{exactly once} as an ordered difference $d-d'$ with $d\neq d'$ in $D$.
Singer (1938) constructed such sets for $v=q^2+q+1$ whenever $q$ is a prime power.
Perfect difference sets are automatically Sidon sets (modulo $v$), and because modular Sidon is stronger than integer Sidon, they yield explicit Sidon sets in initial intervals of length $v$.

\subsection{4) Attack Plan (Round 4)}
\begin{enumerate}
\item Define modular Sidon/perfect difference sets and prove that a perfect difference set modulo $v$ gives an \emph{integer} Sidon set in $\{1,\dots,v\}$.
\item Invoke Singer's theorem for $v=q^2+q+1$ (prime power $q$) to obtain $|D|=q+1$.
\item Convert this to a lower bound on $h(v)$ and then compare $q+1$ to $\sqrt v$ to extract a concrete positive limsup for $h(N)-\sqrt N$.
\end{enumerate}

\subsection{5) Work (Round 4)}

\subsubsection{5.1. Modular-to-integer transfer}
\begin{definition}[Perfect difference set modulo $v$]
Let $v\ge 2$.
A set $D\subset \mathbb Z/v\mathbb Z$ is a \emph{perfect difference set} (mod $v$) if the map
\[
D\times D\setminus\{(d,d):d\in D\}\to \mathbb Z/v\mathbb Z\setminus\{0\},\qquad (d,d')\mapsto d-d'
\]
is a bijection.
\end{definition}

\begin{lemma}[Perfect difference set $\Rightarrow$ modular Sidon]
If $D\subset \mathbb Z/v\mathbb Z$ is a perfect difference set, then all ordered differences $d-d'$ with $d\neq d'$ are distinct modulo $v$. In particular, $D$ is a Sidon set in the additive group $\mathbb Z/v\mathbb Z$.
\end{lemma}

\begin{proof}
The definition says exactly that the ordered differences $d-d'$ ($d\neq d'$) run through each nonzero residue exactly once, hence are distinct. This implies the Sidon property: if $d_1+d_2\equiv d_3+d_4\pmod v$, then rearranging gives $d_1-d_3\equiv d_4-d_2\pmod v$; if the pairs are nontrivial then this would exhibit a repeated nonzero ordered difference.
\end{proof}

\begin{lemma}[Modular Sidon implies integer Sidon for standard representatives]\label{lem:mod_to_int}
Let $v\ge 2$, and let $B\subseteq \{0,1,\dots,v-1\}$.
Assume that the ordered differences $b-b'$ with $b\neq b'$ are all distinct modulo $v$.
Then $B$ is a Sidon set in $\mathbb Z$ (hence also in $\{0,1,\dots,v-1\}$ in the usual integer sense).

Equivalently, if $b_1+b_2=b_3+b_4$ with $b_i\in B$, then $\{b_1,b_2\}=\{b_3,b_4\}$ as multisets.
\end{lemma}

\begin{proof}
Suppose (for contradiction) that $B$ is not Sidon in $\mathbb Z$.
Then there exist $b_1,b_2,b_3,b_4\in B$ with
\[
 b_1+b_2=b_3+b_4
\]
but $(b_1,b_2)$ is not a permutation of $(b_3,b_4)$.
Rearranging gives
\[
 b_1-b_3=b_4-b_2\qquad\text{in }\mathbb Z.
\]
If $b_1=b_3$ then $b_4=b_2$, contradicting nontriviality; thus $b_1\neq b_3$ and similarly $b_4\neq b_2$.
Reducing modulo $v$ yields a repeated nonzero ordered difference in $\mathbb Z/v\mathbb Z$:
\[
 b_1-b_3\equiv b_4-b_2\pmod v,
\]
contradicting the hypothesis.
\end{proof}

\subsubsection{5.2. Singer's construction and an infinite family with $h(N)\ge \sqrt N+\tfrac12-o(1)$}
\begin{theorem}[Singer (existence of perfect difference sets)]\label{thm:singer}
Let $q$ be a prime power and let
\[
 v:=q^2+q+1.
\]
Then there exists a perfect difference set $D\subset \mathbb Z/v\mathbb Z$ of size
\[
|D|=q+1.
\]
\end{theorem}

\begin{proof}[Justification / source]
This is the classical Singer theorem (1938). A modern statement appears, for example, as Observation~6 in the arXiv paper \emph{Forbidden Sidon subsets of perfect difference sets} (Oct 2025), which explicitly states that a perfect difference set modulo $v$ exists for each prime power $q$ with $v=q^2+q+1$ and has size $q+1$.
\end{proof}

\begin{corollary}[Explicit lower bound at Singer parameters]\label{cor:singer_lower}
For every prime power $q$ and $N=q^2+q+1$,
\[
 h(N)\ \ge\ q+1.
\]
In particular,
\[
\limsup_{N\to\infty}\bigl(h(N)-\sqrt N\bigr)\ \ge\ \frac12.
\]
\end{corollary}

\begin{proof}
Let $v=q^2+q+1$ and let $D\subset \mathbb Z/v\mathbb Z$ be a perfect difference set of size $q+1$ from Theorem~\ref{thm:singer}.
Choose the standard representatives $\widetilde D\subseteq \{0,1,\dots,v-1\}$.
By definition, ordered differences in $D$ are all distinct modulo $v$, hence ordered differences in $\widetilde D$ are distinct modulo $v$.
Lemma~\ref{lem:mod_to_int} implies that $\widetilde D$ is a Sidon set in $\mathbb Z$.
Translating by $+1$ gives a Sidon set $S:=\widetilde D+1\subseteq\{1,\dots,v\}$ of size $q+1$.
Therefore $h(v)\ge q+1$.

For the limsup statement, set $N=v=q^2+q+1$.
Write $\alpha:=q+\tfrac12$, so $\alpha^2=q^2+q+\tfrac14$ and hence
\[
N-\alpha^2=\frac34.
\]
Using the identity $\sqrt{\alpha^2+t}=\alpha+\dfrac{t}{\alpha+\sqrt{\alpha^2+t}}$ with $t=3/4$ gives
\[
\sqrt N=\alpha+\frac{\tfrac34}{\alpha+\sqrt N}.
\]
In particular $\sqrt N>\alpha$ and
\[
0<\sqrt N-\alpha=\frac{\tfrac34}{\alpha+\sqrt N}<\frac{\tfrac34}{\alpha}=\frac{3}{4q+2}.
\]
Therefore
\[
q+1-\sqrt N\ =\ \frac12-(\sqrt N-\alpha)\ >\ \frac12-\frac{3}{4q+2}.
\]
Combining with $h(N)\ge q+1$ yields
\[
 h(N)-\sqrt N\ \ge\ \frac12-\frac{3}{4q+2},
\]
and letting $q\to\infty$ along prime powers gives the claimed $\limsup\ge 1/2$.
\end{proof}

\subsection{6) Adversarial Verification}
We stress-test the new argument.
\begin{itemize}
\item \textbf{Modular vs integer Sidon.} Lemma~\ref{lem:mod_to_int} uses only the implication
\[
(b_1+b_2=b_3+b_4\text{ in }\mathbb Z)\ \Rightarrow\ (b_1-b_3\equiv b_4-b_2\pmod v),
\]
so any integer collision would force a modular collision. The hypothesis that ordered differences are distinct modulo $v$ rules this out.
\item \textbf{Wrap-around issue.} There is no wrap-around problem because we never deduce an integer equality from a modular equality; we go in the safe direction: integer equality $\Rightarrow$ modular equality.
\item \textbf{Need for ordered differences.} A perfect difference set gives uniqueness for ordered differences of distinct elements, which is strictly stronger than what is needed; thus the transfer is safe.
\item \textbf{Prime power quantifier.} Theorem~\ref{thm:singer} requires $q$ a prime power; this is explicit in the construction and in the cited sources.
\item \textbf{Asymptotic comparison.} The inequality bounding $\sqrt N-(q+\tfrac12)$ is exact and non-asymptotic, so the $\limsup$ claim is rigorous.
\end{itemize}

\subsection{7) Final}
\textbf{UNRESOLVED (BUT STRICTLY ADVANCED).}
Round~4 contributes a new, explicit, and quantitative lower-bound family:
\[
 h(q^2+q+1)\ge q+1=\sqrt{q^2+q+1}+\frac12-o(1).
\]
Consequently,
\[
\limsup_{N\to\infty}(h(N)-\sqrt N)\ge \frac12.
\]
Together with the Round~3 record upper bound $h(N)\le \sqrt N+0.98183\,N^{1/4}+O(1)$, the open problem remains the dramatic sharpening of the upper error term from $N^{1/4}$ toward $N^{\varepsilon}$.

\subsection{8) Completion Estimate}
\textbf{COMPLETION: 72\%}.

\subsection{9) References}
\begin{itemize}
\item T. F. Bloom (ed.), \emph{Erd\H{o}s Problem \#30}, \texttt{https://www.erdosproblems.com/30} (last edited 19 Oct 2025; accessed 20 Jan 2026).
\item D. Carter, Z. Hunter, K. O'Bryant, \emph{On the Diameter of Finite Sidon Sets}, arXiv:2310.20032 (submitted Oct 2023; journal version Acta Math. Hungar., 2025).
\item B. Alexeev et al., \emph{Forbidden Sidon subsets of perfect difference sets, featuring a human-assisted proof}, arXiv:2510.19804v2 (Oct 2025); see Observation 6 for Singer's existence theorem.
\end{itemize}
