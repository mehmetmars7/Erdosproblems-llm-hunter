% Erdos problem #417

\subsection*{FORMAL RESTATEMENT}
\textbf{Definitions.}
\begin{itemize}
\item For real $x\ge 0$, define
\[
V'(x)=\#\{\varphi(m): m\in\mathbb{N},\ 1\le m\le x\},
\]
(the number of \emph{distinct} totient values attained by inputs $m\le x$), and
\[
V(x)=\#\{\varphi(m): m\in\mathbb{N},\ \varphi(m)\le x\}.
\]
\item Note that $V(x)$ is the same function as in Problem~\#416.
\end{itemize}

\textbf{Questions.}
\begin{itemize}
\item[(Q1)] Does $\displaystyle \lim_{x\to\infty} \frac{V(x)}{V'(x)}$ exist?
\item[(Q2)] If it exists, is it $>1$? (Erd\H{o}s suggested it might be $+\infty$.)
\end{itemize}

\subsection*{QUICK LITERATURE/CONTEXT CHECK}
The problem statement records the trivial inequality $V'(x)\le V(x)$ and that Erd\H{o}s suggested the ratio might tend to infinity. No further results are included in the extracted statement.

\subsection*{ATTACK PLAN}
\textbf{Proof-track ideas (show a limit exists).}
\begin{itemize}
\item Relate $V(x)$ to $V'(x)$ by bounding the smallest $m$ such that $\varphi(m)=n$ in terms of $n$; if most totients $n\le x$ have a representative $m\le x$, then $V(x)/V'(x)$ would stay near $1$.
\item Conversely, if many totients $n\le x$ require minimal preimage $m\gg x$, one might prove $V(x)/V'(x)$ grows.
\end{itemize}
\textbf{Disproof-track ideas (show the limit does not exist or is infinite).}
\begin{itemize}
\item Try to exhibit families of totients $n$ whose minimal preimage grows much faster than $n$; if these occupy a non-negligible fraction of totients up to $x$, the ratio could diverge.
\item Search for oscillation: ranges of $x$ where many totients have small preimages (ratio near $1$) and ranges where few do (ratio large).
\end{itemize}

\subsection*{WORK}
\subsubsection*{Fast reality check (exact computation for small $x$)}
Using the same sieve computation as in Problem~\#416, we obtained exact values of $V(x)$ and $V'(x)$ for $x\le 1000$:
\begin{verbatim}
 x    V(x)   V'(x)   V(x)/V'(x)
 10     6      4       1.5000000000
 20    10      9       1.1111111111
 50    21     18       1.1666666667
 100   38     34       1.1176470588
 200   72     65       1.1076923077
 500  158    141       1.1205673759
 1000 291    264       1.1022727273
\end{verbatim}
These small values suggest $V(x)/V'(x)$ is only slightly above $1$ at these scales, but they do not indicate asymptotic behaviour.

\subsubsection*{Lemma 417.1 (A universal lower bound for $\varphi(n)$)}
\textbf{Lemma.} For every $n\in\mathbb{N}$,
\[
\varphi(n)\ge \sqrt{\frac{n}{2}}.
\]
Equivalently, $n\le 2\varphi(n)^2$.

\textbf{Proof.}
Write the prime factorisation $n=\prod_{i=1}^r p_i^{a_i}$ with distinct primes $p_i$ and integers $a_i\ge 1$.
Then
\[
\varphi(n)=\prod_{i=1}^r p_i^{a_i-1}(p_i-1).
\]
Divide by $\sqrt{n}=\prod_i p_i^{a_i/2}$ to obtain
\[
\frac{\varphi(n)}{\sqrt{n}} = \prod_{i=1}^r (p_i-1)\,p_i^{a_i/2-1}.
\]
We bound each factor.

\emph{Case 1: $p_i=2$.}
Then the factor equals $(2-1)2^{a_i/2-1}=2^{a_i/2-1}$.
If $a_i=1$ this is $2^{-1/2}=1/\sqrt{2}$; if $a_i\ge 2$ this is $\ge 1$.

\emph{Case 2: $p_i\ge 3$.}
First note that $(p_i-1)^2=p_i^2-2p_i+1\ge p_i$ for $p_i\ge 3$ (since $p_i(p_i-3)+1\ge 1$).
Hence $p_i-1\ge \sqrt{p_i}$.
Therefore
\[
(p_i-1)p_i^{a_i/2-1} \ge \sqrt{p_i}\,p_i^{a_i/2-1}=p_i^{(a_i-1)/2}\ge 1,
\]
because $a_i\ge 1$.

Multiplying all factors, every odd prime factor contributes at least $1$, and a factor $2^1$ contributes exactly $1/\sqrt{2}$ while any higher power of $2$ contributes at least $1$.
Thus
\[
\frac{\varphi(n)}{\sqrt{n}} \ge \frac{1}{\sqrt{2}},
\]
which rearranges to $\varphi(n)\ge \sqrt{n/2}$.
\hfill$\square$

\subsubsection*{Lemma 417.2 (Relating $V$ and $V'$)}
\textbf{Lemma.} For every real $x\ge 1$,
\[
V'(x)\le V(x)\le V'(2x^2).
\]

\textbf{Proof.}
\emph{First inequality.}
Every value counted by $V'(x)$ has the form $\varphi(m)$ with $1\le m\le x$.
Since $\varphi(m)\le m$ for all $m$, such a value automatically satisfies $\varphi(m)\le x$, hence it is included among the values counted by $V(x)$. Therefore $V'(x)\le V(x)$.

\emph{Second inequality.}
Let $n$ be any value counted by $V(x)$. Then $n=\varphi(m)\le x$ for some $m\in\mathbb{N}$.
By Lemma~417.1,
\[
\sqrt{\frac{m}{2}}\le \varphi(m)=n\le x,
\]
so $m\le 2x^2$.
Thus every totient value $\le x$ occurs as $\varphi(m)$ for some $m\le 2x^2$.
Therefore the set of values counted by $V(x)$ is a subset of the set of values counted by $V'(2x^2)$, proving $V(x)\le V'(2x^2)$.
\hfill$\square$

\subsection*{VERIFICATION}
\begin{itemize}
\item Lemma~417.1 attains equality at $n=2$ (both sides equal $1$), and holds for $n=1$ since $\varphi(1)=1>1/\sqrt{2}$.
\item In Lemma~417.2, the implication $m\le 2x^2$ from $\varphi(m)\le x$ is checked using Lemma~417.1; there is no circularity.
\item The computational check of $V(x)$ for small $x$ uses exactly the proved bound $m\le 2x^2$.
\end{itemize}

\subsection*{FINAL}
\textbf{UNRESOLVED}
\begin{enumerate}
\item[(i)] \textbf{Strongest proved partial result here.}
A clean comparison
\[
V'(x)\le V(x)\le V'(2x^2)
\]
for all $x\ge 1$ (Lemma~417.2), based on the universal lower bound $\varphi(n)\ge\sqrt{n/2}$ (Lemma~417.1), plus exact computed ratios $V(x)/V'(x)$ for $x\le 1000$.
\item[(ii)] \textbf{First gap (crisp).}
Determine whether the limit
\[
\lim_{x\to\infty}\frac{V(x)}{V'(x)}
\]
exists (finite or infinite). In particular, prove either (a) $\frac{V(x)}{V'(x)}$ is Cauchy as $x\to\infty$, or (b) there exist sequences $x_j,y_j\to\infty$ with $V(x_j)/V'(x_j)$ and $V(y_j)/V'(y_j)$ having different limit points, or diverging.
\item[(iii)] \textbf{Top 3 next moves.}
\begin{itemize}
\item Empirically compute $V(x)$ and $V'(x)$ for much larger $x$ (e.g. $x\le 10^6$) to detect whether $V(x)/V'(x)$ trends upward, stabilises, or oscillates.
\item Study the ``minimal preimage'' function $m_{\min}(n)=\min\{m: \varphi(m)=n\}$ on totients $n$ and quantify how often $m_{\min}(n)\le x$ for $n\le x$.
\item Attempt to strengthen Lemma~417.1 into a tighter deterministic bound on $m$ in terms of $\varphi(m)$ (even a bound of the form $m\le \varphi(m)\,g(\varphi(m))$ with explicit $g$) to improve $V(x)\le V'(\cdot)$ comparisons.
\end{itemize}
\item[(iv)] \textbf{What a minimal counterexample would likely look like.}
If the ratio does not converge (or diverges), a minimal explanation would be an infinite family of totients $n\le x$ whose smallest preimages typically lie far beyond $x$, creating many values counted by $V(x)$ but not by $V'(x)$. Such a family would likely be ``highly composite'' in structure (coming from $m$ with many small prime factors, making $\varphi(m)$ much smaller than $m$) and would need to occur with non-negligible frequency among totients.
\end{enumerate}

