\section*{Problem 376. Infinitely many $n$ with $\binom{2n}{n}$ coprime to $105$?}

\subsection*{FORMAL RESTATEMENT}
Determine whether the set
\[
  \mathcal{S}:=\{n\in\mathbb{N}: \gcd\!\bigl(\binom{2n}{n},105\bigr)=1\}
\]
is infinite.

\subsection*{QUICK LITERATURE / CONTEXT CHECK}
This problem is stated as open in Erd\H{o}s--Graham (1980) and remains open in the Erd\H{o}s problems database.

A key equivalence comes from Kummer's theorem: for a prime $p$, $p\nmid \binom{2n}{n}$ iff there are no carries when adding $n+n$ in base $p$, which is equivalent to every base-$p$ digit of $n$ being $<p/2$.

Erd\H{o}s--Graham--Ruzsa--Straus (1975) prove that for any two odd primes $p,q$ there are infinitely many $n$ with $\binom{2n}{n}$ coprime to $pq$.

For the specific triple $(3,5,7)$, many examples are known computationally (OEIS~A030979 begins
$0,1,10,756,757,3160,3186,3187,3250,\dots$), but infinitude is unknown. Bloom--Croot (2025) give a ``weak'' result for $\ge 3$ primes when the primes are sufficiently large: they produce infinitely many $n$ for which \emph{almost all} base-$p$ digits are $<p/2$, implying $\binom{2n}{n}$ is coprime to the product of those primes up to a factor $\le n^\varepsilon$.

Conditional and quantitative information can be found in work of Han Yu (2022): under Schanuel's conjecture and an additional conjecture on thickness, the set $\mathcal{S}$ is infinite; under Schanuel's conjecture one has a power-saving upper bound $\#(\mathcal{S}\cap[1,N])\ll N^{0.026\dots}$.

\subsection*{ATTACK PLAN}
\begin{enumerate}
\item Use Kummer/Lucas to rewrite the gcd condition as a digit restriction in bases $3,5,7$.
\item Interpret each digit-restricted set as a Cantor-like set; the problem becomes whether three multiplicatively independent Cantor sets have infinite intersection.
\item Current approaches in the literature include: (i) ``carry-poor'' constructions (EGRS for $r=2$; Bloom--Croot for $r\ge 3$ but with exceptions), and (ii) fractal-geometry methods (Newhouse gap lemma, thickness) giving conditional results and upper bounds.
\item A full resolution for $(3,5,7)$ would likely require a new mechanism forcing simultaneous small-digit expansions in three small bases (or proving this cannot happen beyond finitely many $n$).
\end{enumerate}

\subsection*{WORK}
\subsubsection*{1. Digit formulation via Kummer}
Let $p$ be a prime and write the base-$p$ expansion
\[
  n=\sum_{j\ge 0} a_j p^j,\qquad 0\le a_j\le p-1.
\]
Kummer's theorem states that the $p$-adic valuation $\nu_p\!\bigl(\binom{2n}{n}\bigr)$ equals the number of carries when adding $n+n$ in base $p$. In particular,
\[
  p\nmid \binom{2n}{n}
  \iff \text{no carries occur in }n+n\text{ (base }p)
  \iff \forall j,\ 2a_j\le p-1
  \iff \forall j,\ a_j\le \frac{p-1}{2}.
\]
Thus
\[
  \gcd\!\bigl(\binom{2n}{n},105\bigr)=1
  \iff \begin{cases}
  \text{all base-}3\text{ digits of }n\text{ lie in }\{0,1\},\\
  \text{all base-}5\text{ digits of }n\text{ lie in }\{0,1,2\},\\
  \text{all base-}7\text{ digits of }n\text{ lie in }\{0,1,2,3\}.
  \end{cases}
\]
The example $756=(1001000)_3=(11011)_5=(2130)_7$ is one explicit solution.

\subsubsection*{2. What is known unconditionally}
\begin{itemize}
\item (Two primes.) EGRS (1975) prove that for any two odd primes $p,q$ there are infinitely many $n$ such that $\binom{2n}{n}$ is coprime to $pq$. This corresponds to the existence of infinitely many integers whose base-$p$ digits are $<p/2$ and base-$q$ digits are $<q/2$.
\item (Three primes, small.) For $(3,5,7)$ the problem is open. Direct computation gives at least the following 12 solutions up to $20000$:
\[
0,1,10,756,757,3160,3186,3187,3250,7560,7561,7651.
\]
These agree with the initial segment of OEIS~A030979.
\end{itemize}

\subsubsection*{3. Partial progress toward $r\ge 3$}
Bloom--Croot (2025) show that for $r\ge 1$, if $g_1,\dots,g_r$ are distinct coprime integers that are sufficiently large (depending only on $r$), then for every $\varepsilon>0$ there are infinitely many integers $n$ such that, simultaneously for all $i$, all but at most $\varepsilon\log n$ of the base-$g_i$ digits of $n$ are $\le g_i/2$. Their Corollary~1 yields (for $r\ge 3$ and sufficiently large primes $p_i$) a factorisation
\[
  \binom{2n}{n}=n_1 n_2,
  \qquad (n_1,p_1\cdots p_r)=1,\quad n_2\le n^{\varepsilon},
\]
which is a ``weak'' version of the desired coprimality.

Han Yu (2022) gives conditional results in the exact direction of Problem~376: under Schanuel's conjecture and an additional conjecture on thickness of the digit sets, the intersection for bases $(3,5,7)$ is infinite; under Schanuel one also gets a strong power-saving upper bound on the counting function.

\subsection*{VERIFICATION}
The equivalence between $p\nmid\binom{2n}{n}$ and the digit condition is standard via Kummer's carry interpretation; the digit condition for $105$ is just the conjunction for $p\in\{3,5,7\}$. The numerical list up to $20000$ was verified by a brute-force digit check.

\subsection*{FINAL}
\textbf{LABEL: UNRESOLVED.} \\
\textbf{SUB-LABEL: N/A.} \\
The problem for $105=3\cdot 5\cdot 7$ is open. We recorded the exact digit reformulation, the unconditional $r=2$ theorem of Erd\H{o}s--Graham--Ruzsa--Straus, and partial $r\ge 3$ results (Bloom--Croot; Yu), plus computational evidence.

\subsection*{COMPLETION ESTIMATE}
A full solution would require either (i) a new construction producing infinitely many integers with simultaneously small digits in bases $3,5,7$ (no carries in all three bases), or (ii) a proof that the intersection is finite. Existing methods handle $r=2$ strongly, and $r\ge 3$ only in a weakened ``almost all digits'' sense or under strong transcendence/fractal hypotheses.

%%%%%%%%%%%%%%%%%%%%%%%%%%%%%%%%%%%%%%%%%%%%%%%%%%%%%%%%%%%%%%%%%%%%%%%%%%%%%%
