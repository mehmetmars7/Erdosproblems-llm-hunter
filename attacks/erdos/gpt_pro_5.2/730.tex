% Solutions for selected Erd\H{o}s problems.
% Generated 2026-01-16.

% Erdos Problem #730
% URL: https://www.erdosproblems.com/730

Are there infinitely many pairs of integers $n\neq m$ such that $\binom{2n}{n}$ and $\binom{2m}{m}$ have the same set of prime divisors? A problem of Erd\H{o}s, Graham, Ruzsa, and Straus \cite{EGRS75}, who believed there is 'no doubt' that the answer is yes. For example $(87,88)$ and $(607,608)$. Those $n$ such that there exists some suitable $m>n$ are listed as A129515 in the OEIS. A triple of such $n$ for which $\binom{2n}{n}$ all share the same set of prime divisors is $(10003,10004,10005)$. It is not known whether there are such pairs of the shape $(n,n+k)$ for every $k\geq 1$. References [EGRS75] Erd\H{o}s, P. and Graham, R. L. and Ruzsa, I. Z. and Straus, E. G., On the prime factors of $(\sp{2n}\sb{n})$ . Math. Comp. (1975), 83-92.

%Erdos problem 730

\noindent\textbf{1) FORMAL RESTATEMENT.}

For each integer $n\ge 1$, define the \,\emph{prime divisor set}\,
\[
S(n) := \{\,p\text{ prime}: p\mid \binom{2n}{n}\,\}.
\]
Question: are there infinitely many pairs of integers $(n,m)$ with $n\neq m$ such that $S(n)=S(m)$?

Conventions/edge cases:
\begin{itemize}
\item $\binom{2n}{n}=(2n)!/(n!)^2\in\mathbb Z$ for $n\ge 0$; we restrict to $n\ge 1$.
\item ``Same set of prime divisors'' means equality of sets $S(n)=S(m)$ (multiplicities ignored).
\end{itemize}
Stress points: $S(n)$ always contains all primes in $(n,2n]$ (Lemma~730.1), so equality forces strong constraints relating prime intervals.

\medskip
\noindent\textbf{2) QUICK LITERATURE/CONTEXT CHECK.}

The problem statement itself records examples $(87,88)$ and $(607,608)$ and a triple $(10003,10004,10005)$, and notes the related OEIS entry A129515. I do \emph{not} rely on any additional literature claims beyond what is written above.

\medskip
\noindent\textbf{3) ATTACK PLAN.}

\emph{Proof-track ideas.}
\begin{itemize}
\item Use $p$-adic valuations of $\binom{2n}{n}$ to translate membership of $p$ in $S(n)$ into a digit/carry condition (Lemma~730.2 and Corollary~730.3). Try to build infinite families $n\neq m$ with identical carry patterns across all primes $p\le 2\max\{n,m\}$.
\item Focus on the empirically frequent consecutive collisions $S(n)=S(n+1)$ and analyze the ratio
$\binom{2(n+1)}{n+1}/\binom{2n}{n} = \frac{2(2n+1)}{n+1}$ to derive necessary/sufficient conditions for equal prime sets.
\end{itemize}
\emph{Disproof-track ideas.}
\begin{itemize}
\item Try to show $S(n)$ changes infinitely often because of primes in $(n,2n]$; but this does not preclude collisions, and examples suggest collisions exist.
\end{itemize}
Best path here: develop exact criteria for $p\mid\binom{2n}{n}$ and document computational ``reality checks''.

\medskip
\noindent\textbf{4) WORK.}

\noindent\emph{Fast reality check (computation).}
Using a direct Legendre-formula computation of $v_p\bigl(\binom{2n}{n}\bigr)$ for primes $p\le 2n$:
\begin{itemize}
\item Verified the example pair $(87,88)$: both have
\[
S(87)=S(88)=\{2,3,5,7,11,13,19,23,31,47,53,89,97,101,103,107,109,113,127,131,137,139,149,151,157,163,167,173\}.
\]
\item Exhaustive search for $1\le n\le 5000$ found $39$ collisions of prime divisor sets, all of the form $S(n)=S(n+1)$ (consecutive pairs). The first few are
\[(87,88),(199,200),(237,238),(467,468),(607,608),(967,968),\dots\]
and the last few in this range are
\[(4287,4288),(4360,4361),(4479,4480),(4927,4928),(4987,4988).
\]
No set occurred with multiplicity $\ge 3$ for $n\le 5000$.
\end{itemize}

\medskip
\noindent\textbf{Lemma 730.1 (Large primes always divide).}
Let $n\ge 1$ and let $p$ be a prime with $n<p\le 2n$. Then $p\in S(n)$, and moreover
\[v_p\!\left(\binom{2n}{n}\right)=1.\]

\noindent\emph{Proof.}
Write $\binom{2n}{n}=(2n)!/(n!)^2$. Since $p>n$, the factorial $n!$ contains no multiple of $p$, hence $v_p(n!)=0$. Since $p\le 2n<2p$, the factorial $(2n)!$ contains exactly one multiple of $p$ (namely $p$ itself) and no multiple of $p^2$, hence $v_p((2n)!)=1$. Therefore
\[
v_p\!\left(\binom{2n}{n}\right)=v_p((2n)!)-2v_p(n!)=1-0=1,
\]
so in particular $p$ divides $\binom{2n}{n}$ and $p\in S(n)$. \hfill $\square$

\medskip
\noindent\textbf{Lemma 730.2 (Legendre valuation formula for $\binom{2n}{n}$).}
For any integer $n\ge 1$ and any prime $p$,
\[
v_p\!\left(\binom{2n}{n}\right)=\sum_{k\ge 1}\Bigl(\Bigl\lfloor\frac{2n}{p^k}\Bigr\rfloor -2\Bigl\lfloor\frac{n}{p^k}\Bigr\rfloor\Bigr).
\]

\noindent\emph{Proof.}
First, for any integer $N\ge 1$, we prove Legendre's formula
\[
v_p(N!)=\sum_{k\ge 1}\Bigl\lfloor\frac{N}{p^k}\Bigr\rfloor.
\]
Indeed, among the integers $1,2,\dots,N$, exactly $\lfloor N/p\rfloor$ are divisible by $p$, exactly $\lfloor N/p^2\rfloor$ are divisible by $p^2$, etc. Each multiple of $p^k$ contributes at least one factor of $p$ to $N!$, and counting these contributions with multiplicity gives precisely the sum above.

Applying Legendre's formula to $N=2n$ and $N=n$ and using $\binom{2n}{n}=(2n)!/(n!)^2$ yields
\[
v_p\!\left(\binom{2n}{n}\right)=v_p((2n)!)-2v_p(n!)
=\sum_{k\ge 1}\Bigl\lfloor\frac{2n}{p^k}\Bigr\rfloor-2\sum_{k\ge 1}\Bigl\lfloor\frac{n}{p^k}\Bigr\rfloor,
\]
which is the claimed identity. \hfill $\square$

\medskip
\noindent\textbf{Corollary 730.3 (Carry criterion via base-$p$ digit sums).}
Let $s_p(N)$ denote the sum of the base-$p$ digits of a nonnegative integer $N$. Then for $n\ge 1$,
\[
v_p\!\left(\binom{2n}{n}\right)=\frac{2s_p(n)-s_p(2n)}{p-1}.
\]
In particular, $p\in S(n)$ if and only if adding $n+n$ in base $p$ produces at least one carry.

\noindent\emph{Proof.}
Write $n$ in base $p$ as $n=\sum_{j\ge 0} a_j p^j$ with $0\le a_j\le p-1$ and only finitely many nonzero digits. Then
\[
\Bigl\lfloor \frac{n}{p^k}\Bigr\rfloor = \sum_{j\ge k} a_j p^{j-k}.
\]
Summing over $k\ge 1$ gives
\[
\sum_{k\ge 1}\Bigl\lfloor\frac{n}{p^k}\Bigr\rfloor
=\sum_{k\ge 1}\sum_{j\ge k} a_j p^{j-k}
=\sum_{j\ge 1} a_j \sum_{k=1}^{j} p^{j-k}
=\sum_{j\ge 1} a_j\,(p^{j-1}+\cdots+p+1).
\]
The inner geometric sum equals $(p^j-1)/(p-1)$, so
\[
\sum_{k\ge 1}\Bigl\lfloor\frac{n}{p^k}\Bigr\rfloor
=\frac{1}{p-1}\sum_{j\ge 1} a_j(p^j-1)
=\frac{1}{p-1}\Bigl(\sum_{j\ge 0} a_j p^j - \sum_{j\ge 0} a_j\Bigr)
=\frac{n - s_p(n)}{p-1}.
\]
By Lemma~730.2,
\[
v_p\!\left(\binom{2n}{n}\right)=\sum_{k\ge 1}\Bigl(\Bigl\lfloor\frac{2n}{p^k}\Bigr\rfloor -2\Bigl\lfloor\frac{n}{p^k}\Bigr\rfloor\Bigr)
=\frac{2n-s_p(2n)}{p-1}-2\frac{n-s_p(n)}{p-1}
=\frac{2s_p(n)-s_p(2n)}{p-1}.
\]
Now, when adding $n+n$ in base $p$, the digit-sum identity
\[
s_p(2n)=2s_p(n) - (p-1)\cdot(\#\text{ carries})
\]
holds because each carry reduces the digit sum by exactly $p-1$ (a carry replaces a contribution $a+b\ge p$ by $(a+b-p)$ in the current digit and increases the next digit by $1$, net change $-(p-1)$). Therefore $2s_p(n)-s_p(2n)$ is a nonnegative multiple of $p-1$ equal to $(p-1)\cdot(\#\text{ carries})$, proving the stated formula and the carry criterion.
\hfill $\square$

\medskip
\noindent\textbf{5) VERIFICATION.}

\begin{itemize}
\item Lemma~730.1: checked the valuation count carefully: $p\le 2n<2p$ ensures exactly one multiple of $p$ and no multiple of $p^2$ occurs in $(2n)!$.
\item Lemma~730.2: Legendre's formula proof counts multiples of $p^k$ with multiplicity and uses finite support (since $p^k>2n$ gives zero terms).
\item Corollary~730.3: the digit-sum computation is a direct evaluation of the floor-sum; the carry effect on digit sum is verified locally per digit.
\item Computational check: the reported pairs and sets were produced by exact integer floor computations of $v_p$ (no probabilistic primality).
\end{itemize}

\medskip
\noindent\textbf{6) FINAL.} \textbf{UNRESOLVED}

(i) Strongest proved partial result: exact characterizations of $S(n)$ via valuations (Lemma~730.2) and carries (Corollary~730.3), and the unconditional inclusion $(n,2n]\cap\mathbb P\subseteq S(n)$ with $v_p=1$ (Lemma~730.1). Computation found 39 collisions for $n\le 5000$, all consecutive.

(ii) First gap: prove (or disprove) that there are infinitely many distinct $n\neq m$ with $S(n)=S(m)$ (even the special case $S(n)=S(n+1)$ for infinitely many $n$).

(iii) Top 3 next moves:
\begin{enumerate}
\item Analyze the equality condition $S(n)=S(n+1)$ using
$\binom{2(n+1)}{n+1}=\binom{2n}{n}\cdot \frac{2(2n+1)}{n+1}$, aiming for a clean necessary-and-sufficient condition in terms of carries/valuations for primes dividing $2(2n+1)$ and $n+1$.
\item Try to construct parametric families of $n$ where, for every prime $p\le 2n+2$, the carry pattern for $n$ and $n+1$ in base $p$ is identical (e.g., impose congruences on $n$ modulo many prime powers and check stability).
\item Extend the computation to larger ranges with additional statistics (e.g., gap distribution between consecutive collisions, and whether all collisions continue to be consecutive).
\end{enumerate}

(iv) Minimal counterexample structure (if the answer were ``no''): there would exist $N$ such that for all $n\ge N$ and all $m\neq n$, the sets $S(n)$ are pairwise distinct. In particular, beyond $N$ there would be no consecutive collisions, so each step $n\mapsto n+1$ would introduce or delete at least one prime divisor, despite Lemma~730.1 forcing large blocks of primes in $(n,2n]$.


