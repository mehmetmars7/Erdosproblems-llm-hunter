
\paragraph{FORMAL RESTATEMENT.}
Let $C(x)$ denote the number of Carmichael numbers in $[1,x]$.
A composite integer $n$ is a Carmichael number if
\[
a^{n-1}\equiv 1\pmod n\quad\text{for every integer }a\text{ with }\gcd(a,n)=1.
\]
Question: is it true that
\[
C(x)=x^{1-o(1)}\quad\text{as }x\to\infty?
\]
Equivalently, does $\log C(x)/\log x \to 1$?

\paragraph{QUICK LITERATURE/CONTEXT CHECK.}
I did not use external sources. I only use the statements listed in the problem text (Erd\H{o}s upper bound, Pomerance heuristic, AGP lower bound, Harman exponent) as context, without re-proving them.

\paragraph{ATTACK PLAN.}
\emph{Proof-track:} establish Korselt's criterion; compute $C(x)$ for small $x$; attempt simple constructions of many Carmichael numbers.
\emph{Disproof-track:} attempt to show $C(x)$ must be much smaller than $x^{1-o(1)}$ using only elementary constraints.
Best path here: prove Korselt's criterion and basic necessary conditions and do sanity-check computations; the asymptotic remains open.

\paragraph{WORK.}
\textbf{Lemma 1 (Korselt's criterion).}
A positive integer $n$ is a Carmichael number if and only if
\begin{enumerate}
\item $n$ is composite,
\item $n$ is squarefree, and
\item for every prime $p\mid n$, one has $p-1\mid n-1$.
\end{enumerate}

\emph{Proof.}
($\Rightarrow$) Assume $n$ is Carmichael.
Then $n$ is composite by definition.
If $p^2\mid n$ for some prime $p$, take $a=1+p$.
Then $\gcd(a,n)=1$, but by the binomial theorem
\[
(1+p)^{n-1}\equiv 1+(n-1)p \pmod{p^2}.
\]
Since $p\mid n$ implies $n-1\equiv -1\not\equiv 0\pmod p$, the term $(n-1)p$ is nonzero modulo $p^2$, so $(1+p)^{n-1}\not\equiv 1\pmod{p^2}$.
This contradicts the Carmichael condition modulo $n$, hence $n$ must be squarefree.

Now fix a prime $p\mid n$.
Since $n$ is Carmichael, for every integer $a$ coprime to $n$ (hence also coprime to $p$) we have $a^{n-1}\equiv 1\pmod p$.
We claim there exists $g\in(\mathbb{Z}/p\mathbb{Z})^\times$ of order $p-1$.
To see this, note that for any divisor $d$ of $p-1$, the polynomial $X^d-1$ has at most $d$ roots in the field $\mathbb{F}_p$, hence there are at most $d$ elements of $(\mathbb{Z}/p\mathbb{Z})^\times$ whose order divides $d$.
Summing over all proper divisors $d<p-1$ shows that strictly fewer than $p-1$ elements have order dividing a proper divisor, so some element has order exactly $p-1$.
Choose such a $g$ and lift it to an integer $a$ coprime to $n$.
Then $a^{n-1}\equiv 1\pmod p$ implies $g^{n-1}=1$ in the cyclic subgroup generated by $g$ of size $p-1$, hence $p-1\mid n-1$.

($\Leftarrow$) Conversely, assume (1)--(3).
Let $a$ be coprime to $n$.
Because $n$ is squarefree, it is enough to prove $a^{n-1}\equiv 1\pmod p$ for each prime $p\mid n$ and then combine the congruences via the Chinese remainder theorem.
Fix such a $p$. By Fermat's little theorem, $a^{p-1}\equiv 1\pmod p$.
By (3), write $n-1=(p-1)t$.
Then $a^{n-1}=(a^{p-1})^t\equiv 1^t\equiv 1\pmod p$.
Thus $a^{n-1}\equiv 1\pmod p$ for all $p\mid n$, and CRT gives $a^{n-1}\equiv 1\pmod n$.
So $n$ is Carmichael.
\hfill$\square$

\textbf{Lemma 2 (basic consequences).}
Every Carmichael number $n$ is odd and has at least three distinct prime factors.

\emph{Proof.}
By Lemma 1, $n$ is squarefree.
If $2\mid n$ then $n$ has an odd prime divisor $p$ (since $n$ is composite and squarefree).
Condition (3) gives $p-1\mid n-1$.
But $n$ even implies $n-1$ is odd, while $p-1$ is even, so no even number can divide an odd number; contradiction.
Hence $n$ is odd.

If $n=pq$ were the product of two distinct primes, then (3) gives $p-1\mid pq-1$.
Reducing $pq-1$ modulo $p-1$ yields $pq-1\equiv 1\cdot q-1=q-1\pmod{p-1}$, so $p-1\mid q-1$.
Similarly $q-1\mid p-1$, forcing $p-1=q-1$ and hence $p=q$, contradiction.
Therefore $n$ must have at least three prime factors.
\hfill$\square$

\textbf{FAST REALITY CHECK (computation).}
Using Korselt's criterion and a smallest-prime-factor sieve, I computed:
\begin{verbatim}
C(10^3)=1
C(10^4)=7
C(10^5)=16
C(10^6)=43
C(10^7)=105
First 20 Carmichael numbers:
561, 1105, 1729, 2465, 2821, 6601, 8911, 10585, 15841, 29341,
41041, 46657, 52633, 62745, 63973, 75361, 101101, 115921, 126217, 162401
\end{verbatim}

\paragraph{VERIFICATION.}
For $n=561=3\cdot 11\cdot 17$, we have $2\mid 560$, $10\mid 560$, and $16\mid 560$, so Korselt's criterion holds, consistent with it being the first Carmichael number.
Lemma 2 is consistent with the computed list: all listed Carmichael numbers are odd and (by factorization) have at least three prime factors.

\paragraph{FINAL: UNRESOLVED.}
(i) \emph{Strongest proved partial result here.} Korselt's criterion (Lemma 1) gives a complete and efficiently checkable characterization; in particular Carmichael numbers are odd, squarefree, and have at least three prime factors (Lemma 2). Computations show $C(10^7)=105$.
(ii) \emph{First gap.} Prove either a lower bound strong enough to force $\log C(x)/\log x\to 1$, or an upper bound that rules this out, using only the information explicitly allowed in the problem file.
(iii) \emph{Top 3 next moves.} (1) Use Korselt's criterion to design constructive families (e.g. parameterized products of primes) and attempt to count them. (2) Push computation of $C(x)$ to larger $x$ and compare $\log C(x)/\log x$ to $1$ to see empirical trends. (3) Try to translate the known lower-bound constructions (as stated in the problem text) into explicit, checkable parameterizations to understand where the exponent comes from and whether it can approach $1$.
(iv) \emph{What a minimal counterexample would look like.} If the statement $C(x)=x^{1-o(1)}$ is false, one would expect an $\varepsilon>0$ and arbitrarily large $x$ with $C(x)\le x^{1-\varepsilon}$; any approach would likely exploit an obstruction that forces Carmichael numbers to have constrained prime-factor patterns (beyond Korselt's divisibility) that prevent them from being too dense.

