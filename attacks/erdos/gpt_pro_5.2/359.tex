\section*{Problem \#359}

\subsection*{1) FORMAL RESTATEMENT}
Fix an integer $n\ge 1$. Define a strictly increasing integer sequence $(a_i)_{i\ge 1}$ by
$a_1=n$ and for $i\ge 1$ letting $a_{i+1}$ be the \emph{least integer $>a_i$} which is \emph{not} representable as a sum of consecutive earlier terms:
\[
a_{i+1}=\min\Bigl\{m>a_i:\ 
m\neq \sum_{j=u}^v a_j\ \text{for all }1\le u\le v\le i
\Bigr\}.
\]
(For $n=1$, this agrees with the usual ``least positive integer not representable'' definition and produces the classical sequence.)
The question asks what can be said about the density/growth of $\{a_i\}$.
In particular for $n=1$ one asks whether
\[
\frac{a_k}{k}\to\infty
\quad\text{and}\quad
\frac{a_k}{k^{1+c}}\to 0\ \ \text{for every fixed }c>0.
\]

\subsection*{2) CONTEXT AND DEFINITIONS}
Let
\[
S_i := \left\{\sum_{j=u}^v a_j:\ 1\le u\le v\le i\right\}
\]
be the set of all consecutive block sums among the first $i$ terms.
Then $a_{i+1}$ is the least integer $>a_i$ not in $S_i$.

For $n=1$ the resulting sequence is known as MacMahon's ``segmented numbers'' and appears as OEIS A002048:
\[
1,2,4,5,8,10,14,15,\dots
\]

\subsection*{3) QUICK LITERATURE/CONTEXT CHECK}
The Erd\H{o}s Problems page for \#359 notes that this is a problem of MacMahon studied by Andrews, who conjectured
\[
a_k \sim \frac{k\log k}{\log\log k}.
\]
Porubsk\'y proved that for any $\varepsilon>0$ there are infinitely many $k$ with
\[
a_k < (\log k)^\varepsilon\frac{k\log k}{\log\log k},
\]
and also a limsup relation between the counting function $A(x)=|\{i:a_i\le x\}|$ and the prime counting function $\pi(x)$:
\[
\limsup_{x\to\infty}\frac{A(x)}{\pi(x)}\ge \frac1{\log 2}.
\]

\subsection*{4) ATTACK PLAN}
I will prove a basic \emph{unconditional} upper bound on $a_k$ (hence a weak density bound), and provide some computed data illustrating the conjectured growth.
These do not reach the requested limits $a_k/k\to\infty$ and $a_k/k^{1+c}\to 0$ for all $c>0$, which appear to remain open.

\subsection*{5) DETAILED WORK}
\paragraph{A simple counting upper bound for $n=1$.}
Assume $n=1$.
By definition, $a_{k+1}$ is the \emph{least positive integer} not in $S_k$; hence every integer
$1,2,\dots,a_{k+1}-1$ lies in $S_k$. Therefore
\[
a_{k+1}-1 \le |S_k|.
\]
But $S_k$ consists of sums over all consecutive index intervals $[u,v]$ with $1\le u\le v\le k$, and there are exactly $\binom{k+1}{2}=k(k+1)/2$ such intervals. Hence
\[
|S_k|\le \binom{k+1}{2},
\]
and we obtain the unconditional estimate
\[
a_{k+1}\le \binom{k+1}{2}+1 = \frac{k(k+1)}2+1.
\]
In particular,
\[
a_k = O(k^2)\qquad (n=1).
\]
Consequently,
\[
\frac{a_k}{k^{1+c}}\to 0\quad\text{for every fixed }c>1,
\]
but this is far weaker than the asked-for statement (which requires this for \emph{all} $c>0$).

\paragraph{Trivial lower bound.}
Since the sequence is strictly increasing and $a_1=1$, we have $a_k\ge k$ for all $k$.

\paragraph{Numerical evidence (not a proof).}
Computing the first few thousand terms (using a standard efficient generator for A002048), one observes that $a_k/k$ grows slowly and seems consistent with Andrews' conjecture.
For instance:
\[
a_{10}=21,\quad a_{100}=274,\quad a_{1000}=3165,\quad a_{5000}=18524,
\]
so $a_{5000}/5000\approx 3.70$, while $\frac{\log 5000}{\log\log 5000}\approx 3.98$.

\subsection*{6) VERIFICATION AND EDGE CASES}
\begin{itemize}
\item The bound $a_{k+1}\le \binom{k+1}{2}+1$ uses only that $a_{k+1}$ is the least \emph{positive} missing sum (true for $n=1$).
\item For $n>1$, one must be careful: the least positive non-representable integer could be $1<n=a_1$, so the ``least $>a_i$'' convention is needed to keep $(a_i)$ increasing. The same counting argument does not immediately force all integers below $a_{k+1}$ to be representable.
\end{itemize}

\subsection*{7) FINAL}
\textbf{UNRESOLVED.}
\begin{itemize}
\item[(i)] \textbf{Strongest proved partial result here:} for $n=1$ one has the unconditional quadratic upper bound $a_k\le \frac{k(k-1)}2+1$ (equivalently $a_{k+1}\le \binom{k+1}{2}+1$).
\item[(ii)] \textbf{First remaining gap:} any nontrivial lower bound forcing $a_k/k\to\infty$, i.e.\ $a_k\ge k\cdot \omega(1)$.
\item[(iii)] \textbf{Next moves:} understand how many \emph{distinct} consecutive sums exist among $\{a_1,\dots,a_k\}$ and obtain quantitative lower bounds on collisions; relate $A(x)$ to classical additive basis/representation problems.
\item[(iv)] \textbf{Minimal counterexample search:} compute $a_k/k$ for much larger $k$ and look for structural ``phases''; try to prove $a_k \gg k\log k$ on a density-one set of $k$, or identify obstructions.
\end{itemize}

\noindent\textbf{COMPLETION: 25\%}

\hrule

