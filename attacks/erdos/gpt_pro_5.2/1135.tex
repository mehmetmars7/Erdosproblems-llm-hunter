% Erdos Problem #1135

\subsection*{FORMAL RESTATEMENT}
Define $f:\mathbb{N}\to\mathbb{N}$ by
\[
 f(n)=\begin{cases}
 n/2,&\text{if $n$ is even},\\
 (3n+1)/2,&\text{if $n$ is odd}.
 \end{cases}
\]
For $k\ge 1$, let $f^{(k)}$ denote the $k$-fold iterate.
The Collatz conjecture (as stated) asks:
\[
\forall m\in\mathbb{N}_{\ge 1},\ \exists k\ge 1\text{ such that } f^{(k)}(m)=1.
\]

\subsection*{QUICK LITERATURE/CONTEXT CHECK}
I only record context explicitly stated in the problem file.
The file identifies this as the Collatz conjecture, and notes (historically) that Erd\H{o}s referred to it as ``hopeless''.

\subsection*{ATTACK PLAN}
\textbf{Proof track ideas.}
\begin{itemize}
\item Prove closure properties of the set of starting values that reach $1$ (e.g. stability under multiplying by powers of $2$), then attempt to expand this set.
\item Analyze parity sequences and attempt to show average contraction.
\end{itemize}
\textbf{Disproof track ideas.}
\begin{itemize}
\item Search for a nontrivial cycle (other than $1\mapsto 1$) or an orbit that diverges.
\item Use computation to push bounds and look for anomalies.
\end{itemize}

\subsection*{WORK}
\textbf{Lemma 1135.1 (well-definedness and one-step decrease on evens).}
For every $n\in\mathbb{N}$, $f(n)\in\mathbb{N}$. If $n$ is even and $n\ge 2$, then $f(n)=n/2<n$.

\emph{Proof.}
If $n$ is even then $n/2\in\mathbb{N}$.
If $n$ is odd then $3n+1$ is even, so $(3n+1)/2\in\mathbb{N}$.
For even $n\ge 2$, $n/2<n$ is immediate.
\qed

\textbf{Lemma 1135.2 (powers of $2$ reach $1$).}
If $m=2^t$ with $t\ge 0$, then $f^{(t)}(m)=1$.

\emph{Proof.}
For $t=0$ we have $m=1$ and the claim is trivial.
For $t\ge 1$, $m$ is even, so $f(m)=m/2=2^{t-1}$.
Iterating, one obtains $f^{(k)}(2^t)=2^{t-k}$ for $0\le k\le t$.
In particular $f^{(t)}(2^t)=2^0=1$.
\qed

\textbf{Lemma 1135.3 (closure under multiplying by powers of $2$).}
Suppose $m\ge 1$ is such that $f^{(k)}(m)=1$ for some $k\ge 0$.
Then for every $s\ge 0$ we have $f^{(k+s)}(2^s m)=1$.

\emph{Proof.}
Starting from $2^s m$, the first $s$ iterations apply the ``even'' rule repeatedly, yielding
\[
 f^{(s)}(2^s m)=m.
\]
Applying $k$ further iterations gives $f^{(k+s)}(2^s m)=f^{(k)}(m)=1$.
\qed

\textbf{FAST REALITY CHECK (computation up to $10^5$).}
I computed the trajectories for all starting values $m\le 100000$ and verified they reach $1$.
I also recorded (a) the maximum number of steps to reach $1$ among these starts, and (b) the largest value encountered in any such trajectory.
\begin{verbatim}
Checked m<= 1000: all reached 1 under this computation.
  max steps: 113 achieved by m=871
  max value encountered: 125252 first seen starting from m=703
Checked m<= 10000: all reached 1 under this computation.
  max steps: 165 achieved by m=6171
  max value encountered: 13557212 first seen starting from m=9663
Checked m<= 100000: all reached 1 under this computation.
  max steps: 221 achieved by m=77031
  max value encountered: 785412368 first seen starting from m=77671
\end{verbatim}
This is, of course, not a proof for all $m$.

\subsection*{VERIFICATION}
\begin{itemize}
\item Lemmas 1135.1--1135.3 are straightforward and use only integer arithmetic.
\item The computation checks only finitely many starting values.
\item The algorithm used the exact iteration rule given and terminated trajectories when they hit a previously computed value or $1$; the reported maxima were tracked explicitly.
\end{itemize}

\subsection*{FINAL}
\textbf{UNRESOLVED}

(i) \textbf{Strongest proved partial result.}
All powers of $2$ reach $1$ (Lemma 1135.2), and the set of starting values that reach $1$ is closed under multiplication by powers of $2$ (Lemma 1135.3).

(ii) \textbf{First gap (crisp).}
Prove that every $m\ge 1$ reaches $1$ under iteration of $f$, or exhibit an explicit counterexample (a divergent orbit or nontrivial cycle).

(iii) \textbf{Top 3 next moves.}
\begin{itemize}
\item Establish a quantitative contraction-in-expectation estimate over residue classes (e.g. average behavior of $\log f(n)-\log n$) and turn it into a deterministic recurrence.
\item Search for structural invariants preventing cycles and showing eventual descent below the starting value.
\item Extend verified computational bounds with certified methods (interval arithmetic / proof certificates) to cover larger ranges and to study statistics of stopping times.
\end{itemize}

(iv) \textbf{Minimal counterexample structure.}
A minimal counterexample would be either (a) the smallest $m$ whose orbit never hits $1$, which would necessarily avoid all numbers already known to reach $1$, or (b) the smallest nontrivial cycle under $f$.

