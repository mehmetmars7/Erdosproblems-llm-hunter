\section*{Problem 142: asymptotics of $r_k(N)$ (maximum size of a $k$--AP--free set)}

\subsection*{1) FORMAL RESTATEMENT}

Fix an integer $k\ge 3$.
For $N\in\mathbb{N}$ define
\[
 r_k(N) := \max\bigl\{ |A| : A\subseteq [N]=\{1,2,\dots,N\}\ \text{and $A$ contains no $k$--term arithmetic progression}\bigr\}.
\]
Here ``contains a $k$--term arithmetic progression'' means: there exist $a\in\mathbb{Z}$ and $d\in\mathbb{Z}_{>0}$ such that
\[
 a,a+d,\dots,a+(k-1)d \in A.
\]

The problem asks for an \emph{asymptotic formula} for $r_k(N)$ as $N\to\infty$, i.e. an explicit function $f_k(N)$ with
$r_k(N) \sim f_k(N)$.

\subsection*{2) QUICK LITERATURE / CONTEXT CHECK}

\begin{itemize}[leftmargin=2.2em]
\item (Qualitative) Szemer\'edi's theorem implies $r_k(N)=o(N)$ for every fixed $k\ge 3$.

\item (Upper bounds) For $k=3$, Kelley--Meka proved a quasipolynomial density upper bound, and Bloom--Sisask improved it to
\[ r_3(N) \le \exp\bigl(-c(\log N)^{1/9}\bigr)\,N \quad \text{for some }c>0.\]
See \cite{KelleyMeka2023,BloomSisask2023}.
For $k=4$, Green--Tao proved a polylogarithmic bound
\[ r_4(N) \ll N(\log N)^{-c}\]
for some $c>0$ (\cite{GreenTao2017r4}).
For $k\ge 5$, Leng--Sah--Sawhney proved
\[ r_k(N) \ll N\exp\bigl(-(\log\log N)^{c_k}\bigr)\]
for some $c_k>0$ (\cite{LSS2024}).

\item (Lower bounds) Behrend (1946) constructed large $3$--AP--free sets; Rankin generalized to $k$--AP--free sets, and later work refined constants and lower order factors.
A convenient consolidated reference is O'Bryant's survey/paper \cite{OBryant2011}.
More recently, Elsholtz--Hunter--Proske--Sauermann obtained the first quasipolynomial improvement over Behrend beyond lower-order factors (\cite{EHPS2024}).

\item Enormous gaps remain between known upper and lower bounds, even for $k=3$; thus a full asymptotic formula is currently out of reach.
\end{itemize}

\subsection*{3) ATTACK PLAN}

An asymptotic formula would require matching upper and lower bounds up to a $(1+o(1))$ factor.
Even determining the correct \emph{scale} (e.g. $N\exp(-\Theta((\log N)^\alpha))$ with the right exponent $\alpha$) is open for $k=3$.

Concrete partial goals one might try:
\begin{enumerate}[label=(\alph*),leftmargin=2.2em]
\item For fixed $k$, tighten quantitative Szemer\'edi bounds to a form comparable to the best constructions.

\item Improve constructions (lower bounds) to approach the best upper bounds.

\item Identify a candidate ``main term'' (possibly of the form $N\,\exp(-\Theta((\log N)^{\alpha_k}))$) and prove upper and lower bounds of that order.
\end{enumerate}

In the ``work'' below I give a fully proved elementary (but weak) lower bound via the probabilistic method and deletion, and I include small-$N$ sanity checks for $k=3$.

\subsection*{4) WORK}

\subsubsection*{4.1 Elementary probabilistic lower bound (fully proved)}

\begin{proposition}[A simple general lower bound]
\label{prop:poly-lb}
Fix $k\ge 3$.
There is a constant $c_k>0$ such that for all $N\ge 1$,
\[
 r_k(N) \ge c_k\, N^{\frac{k-2}{k-1}}.
\]
One may take for instance $c_k=3/8\,$.
\end{proposition}

\begin{proof}
Let $p := \frac12\,N^{-1/(k-1)}$.
Select a random subset $A\subseteq[N]$ by including each $x\in[N]$ independently with probability $p$.
Then $\mathbb{E}|A| = pN$.

Let $X$ be the number of $k$--term arithmetic progressions contained in $A$.
Let $T$ be the number of $k$--term arithmetic progressions in $[N]$.
Each fixed progression lies in $A$ with probability $p^k$, hence $\mathbb{E}X = T p^k$.

A crude bound $T\le N^2$ holds (indeed $T\le N^2/(2(k-1))$, but we only need $N^2$).
Thus
\[
\mathbb{E}X \le N^2 p^k = N^2\cdot p\cdot p^{k-1}.
\]
By the choice of $p$, we have $p^{k-1} = 2^{-(k-1)} N^{-1}$, hence
\[
\mathbb{E}X \le N^2\cdot p\cdot 2^{-(k-1)}N^{-1} = 2^{-(k-1)}\,pN.
\]
In particular, $\mathbb{E}X \le \tfrac14\,pN$ since $k\ge 3$ implies $2^{-(k-1)}\le 1/4$.

Consider instead the random variable $Y:=|A|-X$. By linearity of expectation,
\[
\mathbb{E}Y = \mathbb{E}|A| - \mathbb{E}X \ge pN - \tfrac14 pN = \tfrac34\,pN.
\]
Hence there exists at least one realization of $A$ for which $|A|-X \ge \tfrac34 pN$.
Starting from such an $A$, repeatedly delete one element from any $k$--term progression currently present.
Each deletion destroys at least one progression, so after at most $X$ deletions we obtain a $k$--AP--free set $B\subseteq A$ with
\[
|B| \ge |A|-X \ge \tfrac34\,pN.
\]

Finally,
\[
\tfrac34 pN = \tfrac34\cdot \tfrac12\,N^{-1/(k-1)}\cdot N = \tfrac{3}{8} N^{(k-2)/(k-1)}.
\]
This proves the claim with $c_k=3/8$.
\end{proof}

\begin{remark}
Proposition~\ref{prop:poly-lb} is far weaker than Behrend/Rankin type lower bounds, but it is completely elementary.
For $k=3$ it gives $r_3(N)\ge c\sqrt{N}$.
\end{remark}

\subsubsection*{4.2 Tiny-case sanity check for $k=3$}

For $k=3$ and small $N$ one can compute $r_3(N)$ exactly by brute force.
For $1\le N\le 20$ the values are:
\[
\begin{array}{c|cccccccccc}
N & 1&2&3&4&5&6&7&8&9&10\\\hline
r_3(N) & 1&2&2&3&4&4&4&4&5&5
\end{array}
\qquad
\begin{array}{c|cccccccccc}
N & 11&12&13&14&15&16&17&18&19&20\\\hline
r_3(N) & 6&6&6&7&8&8&8&8&8&8
\end{array}
\]
(Computed by exhaustive search over all subsets of $[N]$; included only as a reality check.)

\subsection*{5) VERIFICATION / EDGE CASES}

\begin{itemize}[leftmargin=2.2em]
\item The counting bound $T\le N^2$ for the number of $k$--APs in $[N]$ is valid (indeed $T\le \sum_{d\le (N-1)/(k-1)} (N-(k-1)d)\le N\cdot (N-1)/(k-1)\le N^2$).

\item Proposition~\ref{prop:poly-lb} is uniform in $N$ and does not require $N$ large.

\item The deletion step is standard: removing one element from each existing $k$--AP eliminates all progressions.
\end{itemize}

\subsection*{6) FINAL}

\textbf{UNRESOLVED}

\begin{enumerate}[label=(\roman*),leftmargin=2.2em]
\item \textbf{Farthest point reached.}
I proved an elementary lower bound $r_k(N)\ge c_k N^{(k-2)/(k-1)}$ (Proposition~\ref{prop:poly-lb}) and recorded small-$N$ values for $r_3(N)$.
I also summarized state-of-the-art upper bounds (Kelley--Meka; Bloom--Sisask; Green--Tao; Leng--Sah--Sawhney).

\item \textbf{Best partial lemma.}
Proposition~\ref{prop:poly-lb}: a fully self-contained probabilistic construction giving a polynomial-size $k$--AP-free subset.

\item \textbf{Smallest missing step.}
An asymptotic formula would require matching upper and lower bounds. Even for $k=3$ the gap between the best known lower bounds (Behrend-type, now with quasipolynomial improvements) and the best known upper bounds (quasipolynomial decay in density) is enormous, so the ``correct main term'' is unknown.

\item \textbf{Completion estimate.}
Resolving the asymptotics appears to require major new ideas in additive combinatorics/higher-order Fourier analysis beyond current quantitative Szemer\'edi technology.
\end{enumerate}

%%%%%%%%%%%%%%%%%%%%%%%%%%%%%%%%%%%%%%%%%%%%%%%%%%%%%%%%%%%%%%%%%%%%%%%%%%%%%%%
\begin{thebibliography}{99}

\bibitem{Gowers2001}
W.~T. Gowers,
\emph{A new proof of Szemer\'edi's theorem},
Geom. Funct. Anal. \textbf{11} (2001).

\bibitem{GreenTao2008}
B.~Green and T.~Tao,
\emph{The primes contain arbitrarily long arithmetic progressions},
Ann. of Math. (2) \textbf{167} (2008), 481--547.

\bibitem{GreenTao2017r4}
B.~Green and T.~Tao,
\emph{New bounds for Szemer\'edi's theorem, III: A polylogarithmic bound for $r_4(N)$},
Mathematika \textbf{63} (2017), 944--1040.

\bibitem{KelleyMeka2023}
Z.~Kelley and R.~Meka,
\emph{Strong Bounds for 3-Progressions},
arXiv:2302.05537.

\bibitem{BloomSisask2023}
T.~F. Bloom and O.~Sisask,
\emph{An improvement to the Kelley--Meka bounds on three-term arithmetic progressions},
arXiv:2309.02353.

\bibitem{LSS2024}
J.~Leng, A.~Sah, and M.~Sawhney,
\emph{Improved Bounds for Szemer\'edi's Theorem},
arXiv:2402.17995.

\bibitem{EHPS2024}
C.~Elsholtz, Z.~Hunter, L.~Proske, and L.~Sauermann,
\emph{Improving Behrend's construction: Sets without arithmetic progressions in integers and over finite fields},
arXiv:2406.12290.

\bibitem{OBryant2011}
K.~O'Bryant,
\emph{Sets of integers that do not contain long arithmetic progressions},
Electron. J. Combin. \textbf{18} (2011), \#P59.

\bibitem{KourilPaul2008}
M.~Kou\v{r}il and J.~L. Paul,
\emph{The van der Waerden number $W(2,6)$ is 1132},
Exp. Math. \textbf{17} (2008), 53--61.

\bibitem{KS2016}
J.~Kozik and D.~Shabanov,
\emph{Improved algorithms for coloring simple hypergraphs and applications},
arXiv:1405.4295 (version cited in many summaries; result yields improved exponential lower bounds for van der Waerden numbers).

\bibitem{T5KCPAP}
The Prime Pages (Chris Caldwell), ``Consecutive prime arithmetic progressions'' (CPAP-$k$) page.

\bibitem{WikiVDW}
Wikipedia contributors,
\emph{Van der Waerden number}, online article.

\bibitem{WikiCPAP}
Wikipedia contributors,
\emph{Prime arithmetic progression} (section on consecutive prime arithmetic progressions), online article.

\end{thebibliography}


