% Erdős Problem #28
FORMAL RESTATEMENT
Let $A\subseteq\mathbb{N}$ and write
\[
 r_A(n) := (1_A\ast 1_A)(n)=\#\{(a,b)\in A\times A: a+b=n\},
\]
so $r_A(n)$ counts \emph{ordered} representations of $n$ as a sum of two (not necessarily distinct) elements of $A$.
Assume that $A+A$ is cofinite, i.e. there exists $N_0$ such that for all $n\ge N_0$ we have $r_A(n)\ge 1$.
The Erdős--Turán conjecture (order $2$) asserts that
\[
\limsup_{n\to\infty} r_A(n)=\infty.
\]
(Stronger forms, mentioned in the problem text, include $\limsup r_A(n)/\log n>0$.)

QUICK LITERATURE/CONTEXT CHECK
The problem statement records the conjecture of Erdős--Turán and notes stronger conjectures (e.g. a logarithmic lower bound, and a two-set variant with $A+B=\mathbb{N}$ and $a_n/b_n\to 1$).
I will not use any external results beyond what is already stated in the problem file.

ATTACK PLAN
A natural route to proving unboundedness of $r_A$ is to relate large values of $r_A$ to growth of the counting function $A(N):=|A\cap[1,N]|$:
(1) if one can show $A(N)$ is sometimes much larger than $\sqrt{N}$, then averaging forces large $r_A(n)$;
(2) if $A(N)\asymp\sqrt{N}$ (the ``thinnest'' possible scale), one needs a structural argument showing that bounded representation would force $A$ to be ``too Sidon-like'', contradicting the basis property.
A disproof would require constructing an asymptotic basis of order $2$ with uniformly bounded representation function, which would be a very rigid object (see (iv) in FINAL).

WORK
\textit{Fast reality check (finite toy model).}
For each $N\le 18$ I brute-forced all subsets $S\subseteq\{1,\dots,N\}$ with the property that $S+S$ covers the ``top half'' interval $[N+1,2N]$, and minimized
\(\max_{n\in[N+1,2N]} r_S(n)\).
The exact minima I found (with one example $S$ achieving it) are:
\[
\begin{array}{c|c|l}
N & \min_S\max_{n\in[N+1,2N]} r_S(n) & \text{one minimizer }S\\\hline
1&1&\{1\}\\
2&2&\{1,2\}\\
3&2&\{2,3\}\\
4&2&\{1,3,4\}\\
5&2&\{2,4,5\}\\
6&2&\{1,3,5,6\}\\
7&3&\{2,4,6,7\}\\
8&3&\{3,6,7,8\}\\
9&3&\{1,4,7,8,9\}\\
10&3&\{2,5,8,9,10\}\\
11&3&\{3,6,9,10,11\}\\
12&3&\{1,4,7,10,11,12\}\\
13&3&\{2,5,8,11,12,13\}\\
14&4&\{3,5,10,11,13,14\}\\
15&4&\{4,6,11,12,14,15\}\\
16&4&\{5,7,12,13,15,16\}\\
17&4&\{1,6,8,13,14,16,17\}\\
18&4&\{2,7,9,14,15,17,18\}
\end{array}
\]
This toy model is not equivalent to the conjecture, but it is consistent with the heuristic that ``covering a long interval'' forces the representation function to grow (slowly) with scale.

\medskip
Lemma 28.1 (basis property forces $A(N)\gtrsim\sqrt{N}$).
Assume $A+A$ contains every integer $n\ge N_0$.
Then for each $N\ge N_0$ with $S_N:=A\cap[1,N]$ and $m_N:=|S_N|$ one has
\[
\frac{m_N(m_N+1)}{2}\ \ge\ N-N_0+1,
\]
and hence
\[
 m_N\ \ge\ \frac{\sqrt{8(N-N_0+1)+1}-1}{2}.
\]

Proof.
Fix $N\ge N_0$. For every $n\in[N_0,N]$ there exist $a,b\in A$ with $a+b=n$.
Since $n\le N$ and $a,b\in\mathbb{N}$, we necessarily have $1\le a\le n-1\le N-1$ and $1\le b\le N-1$, so in particular $a,b\in S_N$.
Thus the set of sums $S_N+S_N$ contains the whole interval $[N_0,N]$, which has $N-N_0+1$ distinct integers.
On the other hand, the number of \emph{unordered} pairs $(a,b)$ with $a,b\in S_N$ and $a\le b$ is exactly $m_N(m_N+1)/2$, and each such pair produces a single sum $a+b$.
Therefore the number of distinct sums in $S_N+S_N$ is at most $m_N(m_N+1)/2$.
Since $S_N+S_N$ contains $N-N_0+1$ distinct integers, we must have $m_N(m_N+1)/2\ge N-N_0+1$.
Solving this quadratic inequality gives the claimed bound on $m_N$.
\qed

\medskip
Lemma 28.2 (averaging forces one large value of $r_S$).
Let $S\subseteq\{1,\dots,N\}$ be any finite set, and define $r_S(n)=\#\{(a,b)\in S^2: a+b=n\}$.
Then
\[
\max_{2\le n\le 2N} r_S(n)\ \ge\ \frac{|S|^2}{2N-1}.
\]

Proof.
Every ordered pair $(a,b)\in S^2$ contributes to exactly one sum $n=a+b\in\{2,3,\dots,2N\}$.
Hence
\[
\sum_{n=2}^{2N} r_S(n)=|S|^2.
\]
There are $2N-1$ integers in the range $2\le n\le 2N$, so by the pigeonhole principle (or averaging)
\(
\max_{2\le n\le 2N} r_S(n)\ge |S|^2/(2N-1).
\)
\qed

\medskip
Corollary 28.3 (a sufficient growth condition).
If $A\subseteq\mathbb{N}$ satisfies $A+A$ cofinite and there exists a sequence $N_j\to\infty$ with
\(
|A\cap[1,N_j]|\ge \sqrt{N_j}\,\omega_j
\)
where $\omega_j\to\infty$, then $\limsup_{n\to\infty} r_A(n)=\infty$.

Proof.
Apply Lemma 28.2 to $S=A\cap[1,N_j]$.
Then for each $j$ there exists $n_j\in[2,2N_j]$ with
\[
 r_A(n_j)\ge r_S(n_j)\ge \frac{|S|^2}{2N_j-1}\ge \frac{N_j\,\omega_j^2}{2N_j-1}.
\]
As $j\to\infty$, the right-hand side tends to $\infty$ because $\omega_j\to\infty$.
\qed

VERIFICATION
Lemma 28.1 only used the fact that for $n\le N$ any representation $n=a+b$ forces $a,b\le N$, so restricting to $S_N$ is valid.
Lemma 28.2 is a direct double-counting identity.
The brute-force table in the fast reality check used ordered representations (consistent with convolution) and was verified by direct enumeration.

FINAL
**UNRESOLVED**
(i) Strongest proved partial result here: Lemma 28.2 gives
\(\max_{n\le 2N} r_A(n)\ge |A\cap[1,N]|^2/(2N-1)\).
Combined with Lemma 28.1 this forces at least a positive constant lower bound on $\max_{n\le 2N} r_A(n)$ for all large $N$, and Corollary 28.3 proves the conjecture under the additional hypothesis that $|A\cap[1,N]|/\sqrt{N}$ is unbounded.
(ii) First gap: under only the basis hypothesis $A+A$ cofinite, prove that $|A\cap[1,N]|/\sqrt{N}$ must be unbounded (or otherwise force large $r_A(n)$ without strengthening $A(N)$).
(iii) Top 3 next moves:
  (1) Try to show that any asymptotic basis of order $2$ has large additive energy on infinitely many scales, translating (via Lemma 28.2-type arguments) to large $r_A(n)$.
  (2) Explore ``near-minimal'' bases numerically (greedy constructions) and track the growth of $\max_{n\le 2N} r_A(n)$ to guess a lower-order forcing term (e.g. $\gg\log N$).
  (3) Attempt a Fourier-analytic inequality: boundedness of $r_A$ gives $L^\infty$ control of $1_A\ast 1_A$, which might force $1_A$ to have small Fourier mass, conflicting with the covering condition $A+A$ cofinite.
(iv) Minimal counterexample structure: a disproof would require an asymptotic basis $A$ of order $2$ and a constant $K$ such that $r_A(n)\le K$ for all $n$. Such an $A$ would need to be extremely close to a Sidon set (to prevent repeated sums) while simultaneously being dense enough (roughly $\gtrsim\sqrt{N}$ by Lemma 28.1) to cover every large integer.


