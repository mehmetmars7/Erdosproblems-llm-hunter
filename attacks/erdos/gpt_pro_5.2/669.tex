% Erdos Problem #669

% Problem statement (from 665-670.tex):
% Let $F_k(n)$ be minimal such that for any $n$ points in $\mathbb{R}^2$ there exist at most $F_k(n)$ many distinct lines passing through at least $k$ of the points, and $f_k(n)$ similarly but with lines passing through exactly $k$ points. Estimate $f_k(n)$ and $F_k(n)$ - in particular, determine $\lim F_k(n)/n^2$ and $\lim f_k(n)/n^2$.

\noindent\textbf{FORMAL RESTATEMENT.}
Fix an integer $k\ge 2$.
For a finite set $P\subset\mathbb{R}^2$ with $|P|=n$, let
\begin{align*}
L_{\ge k}(P) &:= \{ \ell : \ell \text{ is a line in }\mathbb{R}^2 \text{ and } |\ell\cap P|\ge k\},\\
L_{=k}(P) &:= \{ \ell : \ell \text{ is a line in }\mathbb{R}^2 \text{ and } |\ell\cap P|= k\}.
\end{align*}
Define extremal quantities
\[
F_k(n) := \max_{|P|=n} |L_{\ge k}(P)|,
\qquad
f_k(n) := \max_{|P|=n} |L_{=k}(P)|.
\]
(These are equivalent to the ``minimal such that for any $P$ there exist at most $F_k(n)$ lines'' phrasing.)

The problem asks for asymptotics of $F_k(n)$ and $f_k(n)$ as $n\to\infty$ for fixed $k$, in particular the limits (if they exist)
\[\lim_{n\to\infty} F_k(n)/n^2,\qquad \lim_{n\to\infty} f_k(n)/n^2.\]

\bigskip
\noindent\textbf{QUICK LITERATURE/CONTEXT CHECK.}
The supplied statement records:
\begin{itemize}
\item $f_2(n)=F_2(n)=\binom n2$ (trivial).
\item For $k=3$ (Orchard problem), Burr--Gr\"{u}nbaum--Sloane proved $f_3(n)=\frac{n^2}{6}-O(n)$ and $F_3(n)=\frac{n^2}{6}-O(n)$.
\item A trivial upper bound $F_k(n)\le \binom n2/\binom k2$, hence $\limsup F_k(n)/n^2\le 1/(k(k-1))$.
\end{itemize}
Per project rules, I do not use external sources beyond these statements.

\bigskip
\noindent\textbf{ATTACK PLAN.}
\begin{itemize}
\item Prove the trivial pair-counting upper bound carefully (it is the main unconditional tool available from the prompt).
\item For $k=3$, use the stated asymptotic to deduce the actual limit $\lim F_3(n)/n^2=\lim f_3(n)/n^2=1/6$.
\item Give computational sanity checks on concrete configurations (e.g. integer grids) to understand magnitudes, though these do not claim optimality.
\end{itemize}

\bigskip
\noindent\textbf{WORK.}

\medskip
\noindent\textbf{Fast reality check (tiny $n$).}
For any $n<k$, $F_k(n)=f_k(n)=0$.
For $n=k$, $F_k(k)\ge 1$ (take $k$ collinear points) and clearly $F_k(k)\le 1$, so $F_k(k)=f_k(k)=1$.

\medskip
\noindent\textbf{Lemma 669.1 (Exact-to-at-least relation).}
For every $n$ and $k$,
\[
F_k(n) = \max_{|P|=n} |L_{\ge k}(P)| = \max_{|P|=n} \sum_{j\ge k} |\{\ell: |\ell\cap P|=j\}|.
\]
In particular, for a fixed point set $P$ one has the identity
\[
|L_{\ge k}(P)| = \sum_{j\ge k} |\{\ell: |\ell\cap P|=j\}|.
\]

\noindent\emph{Proof.}
For a fixed $P$, each line $\ell$ with $|\ell\cap P|\ge k$ has a well-defined integer $j=|\ell\cap P|\ge k$.
The families $\{\ell:|\ell\cap P|=j\}$ for different $j$ are disjoint and their union over $j\ge k$ equals $L_{\ge k}(P)$.
Taking cardinalities yields the displayed identity for $P$.
Maximizing over $P$ preserves equality.
\hfill$\square$

\medskip
\noindent\textbf{Lemma 669.2 (Trivial pair-counting upper bound).}
For all integers $n\ge k\ge 2$,
\[
F_k(n) \le \frac{\binom n2}{\binom k2}.
\]
Consequently,
\[
\limsup_{n\to\infty} \frac{F_k(n)}{n^2} \le \frac{1}{k(k-1)}.
\]

\noindent\emph{Proof.}
Fix a set $P$ of $n$ points.
Each line $\ell\in L_{\ge k}(P)$ contains at least $k$ points of $P$, hence contains at least $\binom{k}{2}$ unordered pairs of points from $P$.
On the other hand, any unordered pair of distinct points of $P$ lies on exactly one line (the unique line through them), so the sets of pairs contributed by distinct lines are disjoint.
Therefore the total number of unordered pairs in $P$, namely $\binom n2$, is at least the sum over $\ell\in L_{\ge k}(P)$ of $\binom{k}{2}$.
That is,
\[
\binom n2 \ge |L_{\ge k}(P)|\,\binom k2.
\]
Rearranging gives $|L_{\ge k}(P)|\le \binom n2/\binom k2$.
Maximizing over $P$ yields the same bound for $F_k(n)$.
Dividing by $n^2$ and taking $\limsup$ gives the stated constant upper bound.
\hfill$\square$

\medskip
\noindent\textbf{Lemma 669.3 (Consequences for $k=3$ using the stated asymptotic).}
Assuming the statement in the prompt that
\(F_3(n)=\frac{n^2}{6}-O(n)\)
and
\(f_3(n)=\frac{n^2}{6}-O(n)\),
one has
\[
\lim_{n\to\infty} \frac{F_3(n)}{n^2} = \frac16,
\qquad
\lim_{n\to\infty} \frac{f_3(n)}{n^2} = \frac16.
\]

\noindent\emph{Proof.}
If $F_3(n)=\frac{n^2}{6}-O(n)$ then there is a constant $C>0$ with
\(|F_3(n)-n^2/6|\le Cn\) for all large $n$.
Dividing by $n^2$ gives
\[
\left|\frac{F_3(n)}{n^2}-\frac16\right| \le \frac{C}{n},
\]
which tends to $0$ as $n\to\infty$.
Therefore $F_3(n)/n^2\to 1/6$.
The same argument applies to $f_3(n)$.
\hfill$\square$

\medskip
\noindent\textbf{Computational sanity check (not claiming optimality).}
For an $m\times m$ integer grid $P=\{0,1,\dots,m-1\}^2$ (so $n=m^2$), I computed the number of lines containing at least $k$ grid points for small $m$.
The exact counts found were:
\begin{verbatim}
3x3 grid (n=9):  lines with >=3 points = 8
4x4 grid (n=16): lines with >=3 points = 14, lines with >=4 points = 10
5x5 grid (n=25): lines with >=3 points = 32, lines with >=4 points = 16, lines with >=5 points = 12
6x6 grid (n=36): lines with >=3 points = 58, lines with >=4 points = 22, lines with >=5 points = 18
7x7 grid (n=49): lines with >=3 points = 108, lines with >=4 points = 44, lines with >=5 points = 24
\end{verbatim}
These grow much more slowly than the quadratic upper bound from Lemma 669.2, illustrating that achieving $\Theta(n^2)$ many $k$-rich lines requires much more structured configurations than a square grid.

\bigskip
\noindent\textbf{VERIFICATION.}
\begin{itemize}
\item Lemma 669.2: checked that pair-sets from distinct lines are disjoint because a pair of points determines a unique line.
\item For $k=3$, Lemma 669.2 gives $F_3(n)\le \binom n2/3 = \frac{n^2}{6}-\frac{n}{6}$, matching the stated asymptotic order and consistent with Lemma 669.3.
\item The grid computation was rechecked by recomputing line keys via normalized integer coefficients; it is intended only as a sanity check.
\end{itemize}

\bigskip
\noindent\textbf{FINAL.} \textbf{UNRESOLVED}.

\smallskip
\noindent(i) \emph{Strongest proved partial results.}
\begin{itemize}
\item For all fixed $k\ge 2$, $F_k(n)\le \binom n2/\binom k2$, hence $\limsup F_k(n)/n^2\le 1/(k(k-1))$ (Lemma 669.2).
\item For $k=3$, using the asymptotics stated in the prompt, one gets the exact limits $\lim F_3(n)/n^2=\lim f_3(n)/n^2=1/6$ (Lemma 669.3).
\end{itemize}

\noindent(ii) \emph{First gap (crisp statement).}
For each fixed $k\ge 4$, determine whether the limits
$\lim_{n\to\infty} F_k(n)/n^2$ and $\lim_{n\to\infty} f_k(n)/n^2$ exist, and if so compute them.
In particular, decide whether the trivial upper bound constant $1/(k(k-1))$ is sharp.

\noindent(iii) \emph{Top 3 next moves.}
\begin{enumerate}
\item \textbf{Constructions:} search for algebraic or finite-geometry inspired point sets in $\mathbb{R}^2$ producing $\Theta(n^2)$ many lines with exactly $k$ points for fixed $k\ge 4$.
\item \textbf{Upper bounds beyond pair-counting:} prove stronger restrictions on the number of $k$-rich lines using incidence geometry (e.g. inequalities relating $k$-rich lines to point-line incidences), aiming to show the constant is strictly smaller than $1/(k(k-1))$ when $k\ge 4$.
\item \textbf{Computer search for small $k$:} for $k=4$, do exhaustive/heuristic searches over integer/rational point sets to estimate $F_4(n)$ and guess the correct asymptotic constant.
\end{enumerate}

\noindent(iv) \emph{What a minimal counterexample would likely look like.}
If the upper bound constant $1/(k(k-1))$ is not sharp for some $k\ge 4$, then any near-extremal configuration would have to use many lines each contributing close to the minimum $\binom{k}{2}$ pairs, with few (ideally no) lines containing more than $k$ points, because higher multiplicity lines ``waste'' pairs.
Thus a minimal counterexample to sharpness would likely consist of a configuration where unavoidable geometric constraints force many points to lie on some lines with $>k$ points, depressing the total count of distinct $k$-rich lines below the pair-counting maximum.


