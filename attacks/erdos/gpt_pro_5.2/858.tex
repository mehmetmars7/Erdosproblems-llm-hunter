
\noindent\textbf{FORMAL RESTATEMENT.}

Fix an integer $N\ge 2$. Let $A\subseteq \{1,2,\dots,N\}$. For an integer $t\ge 2$, write $\operatorname{spf}(t)$ for the smallest prime divisor of $t$.

Assume $A$ satisfies:
\begin{quote}
There do not exist $a,b\in A$ and an integer $t\ge 2$ such that $b=a t$ and $\operatorname{spf}(t)>a$.
\end{quote}
Define
\[
F(N):= \max_{A\subseteq [N]\ \text{satisfying the condition}} \frac{1}{\log N}\sum_{n\in A}\frac{1}{n}.
\]
The problem asks to estimate $F(N)$ as $N\to\infty$.

\emph{Ambiguity/misstated-context note.}
The statement in the problem file asserts that Alexander and Erd\H{o}s--S\'ark\"ozy--Szemer\'edi proved that this maximum is $o(1)$. Under the literal finite-$N$ definition above, we give an explicit family with $F(N)\ge \tfrac12+o(1)$, so $F(N)$ cannot be $o(1)$. This suggests the cited $o(1)$ result concerns a different ``maximum'' (e.g. for a fixed infinite set $A$ and $\limsup_{N\to\infty}$), or the condition is different in the original source.

\medskip
\noindent\textbf{QUICK LITERATURE/CONTEXT CHECK.}

The problem file itself cites Alexander and Erd\H{o}s--S\'ark\"ozy--Szemer\'edi as proving an $o(1)$ bound for the stated maximum, and provides an $N$-dependent example set $A$.
We do not assume any additional literature beyond what is in the problem file.

\medskip
\noindent\textbf{ATTACK PLAN.}

First, sanity-check the condition: when $a$ is large (e.g. $a\ge \sqrt N$), the constraint ``$\operatorname{spf}(t)>a$'' becomes impossible for any multiple $b\le N$, making the condition vacuous. This immediately yields a lower bound $F(N)\ge 1/2+o(1)$.

We then brute-force optimise $F(N)$ for small $N$ to see what the true finite-$N$ behaviour looks like.

\medskip
\noindent\textbf{WORK.}

\medskip
\noindent\underline{Lemma 858.1 (The constraint is vacuous for $a\ge \sqrt N$).}

\textbf{Lemma.}
Let $N\ge 2$. Suppose $a,b\in\{1,\dots,N\}$ and $t\ge 2$ satisfy $b=a t$. If $a\ge \sqrt N$, then $\operatorname{spf}(t)\le a$.
In particular, no triple $(a,b,t)$ with $a\ge \sqrt N$ can violate the condition $\operatorname{spf}(t)>a$.

\textbf{Proof.}
From $b=a t\le N$ we get $t\le N/a$. If $a\ge \sqrt N$ then $N/a\le \sqrt N\le a$, hence $t\le a$.
For $t\ge 2$ we have $\operatorname{spf}(t)\le t$ (because the smallest prime factor cannot exceed the number itself). Therefore $\operatorname{spf}(t)\le t\le a$.
\qed

\medskip
\noindent\underline{Lemma 858.2 (A concrete family gives $F(N)\ge 1/2+o(1)$).}

\textbf{Lemma.}
Let $A_N:=\{\lceil \sqrt N\rceil,\lceil \sqrt N\rceil+1,\dots,N\}$. Then $A_N$ satisfies the condition, and
\[
\frac{1}{\log N}\sum_{n\in A_N}\frac{1}{n} = \frac{H_N-H_{\lceil\sqrt N\rceil-1}}{\log N} = \frac12 + o(1)
\quad\text{as }N\to\infty,
\]
where $H_m=\sum_{j=1}^m \frac{1}{j}$.

\textbf{Proof.}
By Lemma 858.1, any $a\in A_N$ satisfies $a\ge \sqrt N$, so no equation $b=a t$ with $b\le N$ can have $\operatorname{spf}(t)>a$. Hence $A_N$ satisfies the defining condition.

For the asymptotic, we use the standard integral bounds for harmonic numbers: for every integer $M\ge 1$,
\[
\int_1^{M+1} \frac{dx}{x} \le H_M \le 1 + \int_1^M \frac{dx}{x},
\]
which imply
\[
\log(M+1) \le H_M \le 1+\log M.
\]
Apply this with $M=N$ and $M=\lceil\sqrt N\rceil-1$. Then
\[
H_N - H_{\lceil\sqrt N\rceil-1}
\ge \log(N+1) - (1+\log(\lceil\sqrt N\rceil-1))
= \tfrac12\log N + O(1),
\]
and similarly
\[
H_N - H_{\lceil\sqrt N\rceil-1}
\le (1+\log N) - \log(\lceil\sqrt N\rceil)
= \tfrac12\log N + O(1).
\]
Dividing by $\log N$ gives the claimed limit $\tfrac12+o(1)$.
\qed

\medskip
\noindent\underline{Lemma 858.3 (The element $1$ is incompatible with any other element).}

\textbf{Lemma.}
If $1\in A$ and $b\in A$ with $b\ge 2$, then $A$ violates the condition.

\textbf{Proof.}
Take $a=1$ and $t=b$. Then $b=a t$ holds. Since $b\ge 2$, $\operatorname{spf}(t)=\operatorname{spf}(b)\ge 2>1=a$.
Thus $(a,b,t)$ is a forbidden solution, contradicting the defining property.
\qed

\medskip
\noindent\underline{Fast reality check (exact optimisation for small $N$).}

Interpreting the problem literally as a finite-$N$ maximisation, we can brute-force compute $F(N)$ for $N\le 20$ (by enumerating all $2^N$ subsets and checking the forbidden condition).
The computed maximisers for $N=20$ give
\[
F(20)\approx 0.6968,
\]
attained for example by
\[
A=\{2,3,4,5,7,8,9,11,12,13,16,17,19\}.
\]
In particular, for these small values the optimum is noticeably larger than the simple lower bound coming from $A_N=[\lceil\sqrt N\rceil,N]$.

\medskip
\noindent\textbf{VERIFICATION.}

\begin{itemize}
\item Lemma 858.1 hinges only on the inequalities $t\le N/a$ and $\operatorname{spf}(t)\le t$; there are no hidden assumptions.
\item Lemma 858.2 checks the forbidden condition by reducing to Lemma 858.1; this is valid because every $a\in A_N$ satisfies $a\ge\sqrt N$.
\item The brute-force calculation for $N\le 20$ is consistent with Lemma 858.3 (all computed maximisers omit $1$).
\end{itemize}

\medskip
\noindent\textbf{FINAL.} \textbf{UNRESOLVED.}

(i) \emph{Strongest proved partial results.}
For the literal finite-$N$ formulation, we proved the unconditional lower bound
\[
F(N)\ge \frac{H_N-H_{\lceil\sqrt N\rceil-1}}{\log N} = \tfrac12+o(1),
\]
by exhibiting an explicit admissible set $A_N=[\lceil\sqrt N\rceil,N]$.
We also proved the structural fact that any forbidden solution must have $a<\sqrt N$.

(ii) \emph{First gap.}
Determine the true asymptotic behaviour of the finite-$N$ maximum $F(N)$: does $F(N)$ converge, and if so to what constant? (The computation up to $N=20$ suggests values $>1/2$.)

(iii) \emph{Top 3 next moves.}
(1) Reformalise the statement in the original Erd\H{o}s source to resolve the apparent conflict between the literal lower bound $\ge 1/2$ and the cited ``$o(1)$'' claim.
(2) For the finite-$N$ version, model the condition as a maximum-weight independent set problem on the graph with vertices $\{1,\dots,N\}$ and forbidden edges $(a,at)$; use branch-and-bound/ILP to compute $F(N)$ for larger $N$ to guess a limit.
(3) Attempt a ``swap''/charging argument comparing inclusion of small $a$ (large weight $1/a$) versus the forced exclusion of many multiples $a p$ with primes $p>a$.

(iv) \emph{What a minimal counterexample would look like.}
If the true statement is that $F(N)=o(1)$, then a counterexample to that would require an admissible family $A_N$ whose harmonic weight is a positive fraction of $\log N$ while still avoiding all forbidden multiplicative relations. The explicit family $[\lceil\sqrt N\rceil,N]$ already achieves fraction $\approx 1/2$ under the literal reading.


