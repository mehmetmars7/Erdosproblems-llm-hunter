
For any graph $H$ is there some $c=c(H)>0$ such that every graph $G$ on $n$ vertices that does not contain $H$ as an induced subgraph contains either a complete graph or independent set on $\geq n^c$ vertices? Conjectured by Erd\H{o}s and Hajnal \cite{ErHa89}, who proved that a complete graph or independent set must exist on\[\geq \exp(c_H\sqrt{\log n})\]many vertices, where $c_H>0$ is some constant. This was improved by Buci\'{c}, Nguyen, Scott, and Seymour \cite{BNSS23} to\[\geq \exp(c_H\sqrt{\log n\log\log n}).\]See also the entry in the graphs problem collection . References [BNSS23] Buci\'C, M. and Nguyen, T. and Scott, A. and Seymour, P., A loglog step towards Erdos-Hajnal . arXiv:2301.10147 (2023). [ErHa89] Erd\H{o}s, P. and Hajnal, A., Ramsey-type theorems . Discrete Appl. Math. (1989), 37-52.

\subsection*{FORMAL RESTATEMENT}
Fix a finite simple graph $H$. A graph $G$ is \emph{induced-$H$-free} if no subset of vertices of $G$ induces a 
subgraph isomorphic to $H$.
Let $\omega(G)$ denote the clique number and $\alpha(G)$ the independence number.

Erd\H{o}s--Hajnal conjecture: For every fixed $H$ there exists a constant $c(H)>0$ such that for all $n\in\mathbb{N}$, 
any induced-$H$-free graph $G$ on $n$ vertices satisfies
\[
\max\{\omega(G),\alpha(G)\}\ge n^{c(H)}.
\]

\subsection*{QUICK LITERATURE/CONTEXT CHECK}
Only what is explicitly stated in the problem statement is treated as context:
Erd\H{o}s and Hajnal proved the weaker bound $\max\{\omega,\alpha\}\ge \exp(c_H\sqrt{\log n})$ for some $c_H>0$, and a 2023 
result of Buci\'{c}--Nguyen--Scott--Seymour improves this to 
$\exp(c_H\sqrt{\log n\log\log n})$.
No other external results are assumed.

\subsection*{ATTACK PLAN}
\begin{itemize}
\item Since the full conjecture is open, aim for fully proved nontrivial special cases (specific $H$) that illustrate the 
mechanism.
\item For a small $H$, try to characterize induced-$H$-free graphs structurally and then bound $\alpha$ or $\omega$.
\item Do tiny $n$ computations for the chosen $H$ as sanity checks.
\end{itemize}
I work out a complete proof for the special case $H=P_3$ (the 3-vertex path).

\subsection*{WORK}
Let $P_3$ be the path on 3 vertices (two edges in a row).

\textbf{Lemma 1 (Induced-$P_3$-free graphs are disjoint unions of cliques).}
If a graph $G$ has no induced subgraph isomorphic to $P_3$, then every connected component of $G$ is a clique.
Equivalently, $G$ is a disjoint union of complete graphs.

\emph{Proof.}
Assume for contradiction that some connected component $C$ of $G$ is not a clique. Then there exist vertices 
$u,w\in V(C)$ that are not adjacent.
Because $C$ is connected, there exists a $u$--$w$ path in $C$.
Choose a shortest such path and write it as $u=v_0,v_1,\dots,v_t=w$.
Since $u$ and $w$ are nonadjacent, $t\ge 2$.
Now consider the three vertices $v_0,v_1,v_2$. By construction, 
$v_0v_1$ and $v_1v_2$ are edges.
Because the path is shortest, $v_0v_2$ cannot be an edge (otherwise $u=v_0,v_2,\dots,w$ would be a shorter $u$--$w$ path).
Thus the induced subgraph on $\{v_0,v_1,v_2\}$ is exactly $P_3$, contradicting that $G$ is induced-$P_3$-free.
Therefore each component is a clique. \hfill$\square$

\medskip
\textbf{Lemma 2 (Clique--independent product bound for disjoint unions of cliques).}
Let $G$ be a disjoint union of cliques with component sizes $s_1,\dots,s_t$ (so $\sum_i s_i=n$).
Then
\[
\alpha(G)=t,\qquad \omega(G)=\max_i s_i,\qquad \text{and hence }\alpha(G)\,\omega(G)\ge n.
\]
In particular, $\max\{\alpha(G),\omega(G)\}\ge \sqrt{n}$.

\emph{Proof.}
Picking one vertex from each component gives an independent set of size $t$, and no larger independent set exists 
(because within a clique no two vertices are independent), so $\alpha(G)=t$.
The largest clique is exactly the largest component, so $\omega(G)=\max_i s_i$.
Then
\[
\alpha(G)\,\omega(G)=t\cdot\max_i s_i\ge \sum_{i=1}^t s_i = n,
\]
since $\max_i s_i\ge s_i$ for each $i$.
Finally, by AM--GM, $\max\{\alpha,\omega\}\ge \sqrt{\alpha\omega}\ge \sqrt{n}$. \hfill$\square$

\medskip
\textbf{Proved special case.}
For $H=P_3$, the Erd\H{o}s--Hajnal conclusion holds with exponent $c(H)=1/2$:
any induced-$P_3$-free graph on $n$ vertices has a clique or independent set of size at least $\sqrt{n}$.

\medskip
\textbf{FAST REALITY CHECK (small $n$).}
For induced-$P_3$-free graphs, Lemma 1 gives a complete characterization, so no brute force is needed.
As a spot-check: for $n=7$, the ``most balanced'' disjoint union of cliques is $K_3\cup K_2\cup K_2$, giving 
$\omega=3$, $\alpha=3$, and $\sqrt{7}\approx 2.65$, consistent with Lemma 2.

\subsection*{VERIFICATION}
\begin{itemize}
\item Lemma 1 hinges on the fact that a shortest path has no chord skipping an intermediate vertex; that implication was 
used only for the first three vertices and is correct.
\item Lemma 2 uses only definitions of $\alpha$ and $\omega$ in a disjoint union of cliques.
\item The proved special case does not extend to general $H$; it is presented explicitly as a special case.
\end{itemize}

\subsection*{FINAL}
\textbf{UNRESOLVED}

(i) Strongest proved partial result here: for $H=P_3$, the conjectured polynomial bound holds with $c(P_3)=1/2$ 
(via Lemmas 1--2).

(ii) First gap: for a general fixed $H$, produce a mechanism (structural decomposition, regularity, or other) that forces 
$\max\{\alpha(G),\omega(G)\}\ge n^{c(H)}$ in every induced-$H$-free graph $G$.

(iii) Top 3 next moves:
\begin{itemize}
\item Prove additional special cases by structural characterization for other small $H$ (e.g., induced-$P_4$-free graphs 
(cographs), induced-$C_5$-free, etc.), aiming for explicit exponents.
\item Attempt to formalize how induced-$H$-free forces large ``homogeneous'' vertex subsets using density increment or 
regularity-type partitions.
\item Computationally search for small $H$ and moderate $n$ to estimate the best possible exponent $c(H)$ and identify 
extremal constructions.
\end{itemize}

(iv) Minimal counterexample structure (if the conjecture were false for some $H$): a family of induced-$H$-free graphs 
$G_n$ with $\omega(G_n),\alpha(G_n)\le n^{o(1)}$, i.e., both clique and independent numbers subpolynomial, while still 
avoiding $H$ as an induced subgraph.


