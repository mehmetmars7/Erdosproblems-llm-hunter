
\noindent\textbf{FORMAL RESTATEMENT.}

Fix an integer $k\ge 2$.
For each $n\ge 1$, define $f_k(n)$ to be the smallest integer $M$ such that:

For every set $P$ of $n$ points in $\mathbb{R}^2$ with the property that no $k+1$ points of $P$ are collinear, the number of distinct lines that contain at least $k$ points of $P$ is at most $M$.

Equivalently,
\[
 f_k(n)=\max\{\#\{\ell: |\ell\cap P|\ge k\}:\ |P|=n,\ P\subset\mathbb{R}^2,\ \text{no }k+1\text{ points collinear}\}.
\]
Under the ``no $k+1$ collinear'' assumption, ``at least $k$'' means \emph{exactly $k$}.

Question: for fixed $k\ge 4$, is $f_k(n)=o(n^2)$ as $n\to\infty$?

\bigskip
\noindent\textbf{QUICK LITERATURE/CONTEXT CHECK.}

I will not use any literature beyond what is explicitly quoted in the problem statement (lower bounds of K\'arteszi, Gr\"unbaum, and Solymosi--Stojakovi\'c; and the remark about $k=3$).

\bigskip
\noindent\textbf{ATTACK PLAN.}

\emph{Proof-track ideas.}
(1) Convert to an incidence counting problem: each $k$-rich line contributes many point pairs; derive universal $O(n^2)$ upper bounds and see whether any improvement to $o(n^2)$ is possible from the ``no $k+1$ collinear'' constraint alone.
(2) Try to leverage planar incidence theorems (Szemer\'edi--Trotter type) to get subquadratic bounds when $k$ is fixed.

\emph{Construction track ideas.}
(1) Build explicit point sets with many $k$-point lines while avoiding $(k+1)$-point lines.
(2) Use probabilistic ``generic position'' arguments to realize desired incidence patterns without accidental extra collinearities.

In this write-up I only prove the trivial $O(n^2)$ upper bound and a very weak linear lower bound construction.

\bigskip
\noindent\textbf{WORK.}

\noindent\textbf{Lemma 588.1 (pair-counting upper bound).}
For all integers $n\ge 1$ and $k\ge 2$,
\[
 f_k(n)\le \frac{\binom{n}{2}}{\binom{k}{2}}.
\]

\noindent\emph{Proof.}
Let $P$ be any set of $n$ points with no $k+1$ collinear, and let $L$ be the set of lines containing at least $k$ points of $P$.
Then each $\ell\in L$ contains exactly $k$ points of $P$, hence contributes exactly $\binom{k}{2}$ unordered pairs of points lying on $\ell$.
Two distinct lines cannot share the same unordered pair of points, because any two points determine a unique line.
Therefore the sets of point-pairs contributed by different $\ell\in L$ are disjoint subsets of the $\binom{n}{2}$ total point-pairs in $P$.
Hence
\[
 |L|\binom{k}{2}\le \binom{n}{2},
\]
so $|L|\le \binom{n}{2}/\binom{k}{2}$.
Taking the maximum over $P$ yields the bound on $f_k(n)$.
\qed

\bigskip
\noindent\textbf{Lemma 588.2 (a simple linear lower bound construction).}
For all integers $n\ge 1$ and $k\ge 2$,
\[
 f_k(n)\ge \left\lfloor\frac nk\right\rfloor.
\]
In fact, there exists a set $P$ of $n$ points in $\mathbb{R}^2$ with no $k+1$ collinear such that at least $\lfloor n/k\rfloor$ distinct lines each contain exactly $k$ points of $P$.

\noindent\emph{Proof.}
Let $t=\lfloor n/k\rfloor$ and let $r=n-tk$.
Choose $t$ distinct lines $\ell_1,\dots,\ell_t$ in the plane.
On each $\ell_i$, choose $k$ points independently from a continuous distribution supported on $\ell_i$ (e.g. choose a random real parameter along the line).
Finally choose the remaining $r$ points independently from a continuous distribution supported on a region of the plane disjoint from all the $\ell_i$.
Let $P$ be the resulting random set of $n$ points.

By construction, each $\ell_i$ contains at least $k$ points of $P$.
We show that with probability $1$:
(a) no line contains $k+1$ points of $P$, and
(b) the only lines containing $k$ points are the designated $\ell_i$.
Either conclusion implies $f_k(n)\ge t$.

Fix any specific $(k+1)$-tuple $Q$ of distinct chosen points.
The event ``$Q$ is collinear'' is an algebraic condition of measure $0$ under a continuous distribution unless $Q$ was forced to lie on one of the $\ell_i$.
But the only forced collinearities are those among the $k$ points on a single $\ell_i$; a $(k+1)$-tuple cannot be fully contained in a single $\ell_i$ because only $k$ points were placed on each.
Thus for each fixed $(k+1)$-tuple $Q$, $\mathbb{P}(Q\text{ collinear})=0$.
There are finitely many $(k+1)$-tuples, so by a union bound over finitely many measure-$0$ events, with probability $1$ no $(k+1)$ points are collinear.
This proves (a).

Similarly, fix any $k$-tuple $R$ of points that is \emph{not} exactly the $k$ points chosen on some single designated line $\ell_i$.
Then $R$ is not forced to be collinear, so again $\mathbb{P}(R\text{ collinear})=0$.
There are finitely many such $k$-tuples, so with probability $1$ none of them is collinear.
Hence the only collinear $k$-tuples are the $k$ points on each $\ell_i$, implying (b).

Therefore there exists at least one deterministic outcome of the random choice with the stated properties, giving $f_k(n)\ge t=\lfloor n/k\rfloor$.
\qed

\bigskip
\noindent\textbf{Fast reality check (numerics for the trivial upper bound).}

For illustration, the pair-counting bound gives
\[f_4(n)\le \binom{n}{2}/6\approx n^2/12,\qquad f_5(n)\le \binom{n}{2}/10\approx n^2/20.\]
For example, for $n=10$ it yields $f_4(10)\le 7$ and $f_5(10)\le 4$.

\bigskip
\noindent\textbf{VERIFICATION.}

\emph{Lemma 588.1.} The only nontrivial step is the uniqueness of the line through a pair of points; this is standard Euclidean geometry.

\emph{Lemma 588.2.} The probabilistic argument uses only that collinearity of a fixed set of randomly chosen points from a continuous distribution has probability $0$ unless forced. The union bound is over finitely many tuples, so there is no measurability subtlety.

\bigskip
\noindent\textbf{FINAL.}

**UNRESOLVED**

(i) Strongest proved partial result: the universal bound $f_k(n)\le \binom{n}{2}/\binom{k}{2}=O(n^2)$ (Lemma 588.1), and a trivial construction giving $f_k(n)\ge\lfloor n/k\rfloor$ (Lemma 588.2).

(ii) First gap (crisp): prove a genuinely subquadratic upper bound $f_k(n)=o(n^2)$ for fixed $k\ge 4$, i.e. improve the pair-counting argument by using additional geometric constraints beyond ``two points determine a line''.

(iii) Top 3 next moves:
1. Prove an incidence bound of the form ``the number of $k$-rich lines in an $n$-point set with no $k+1$ collinear is $O_k(n^{2-\varepsilon_k})$'' using a planar incidence theorem.
2. Analyze whether the known near-quadratic lower bound constructions must still be $o(n^2)$ and identify which structural features prevent reaching $\Theta(n^2)$.
3. Computation: for small $n$ and $k=4$, search (by integer coordinates up to some range) for configurations maximizing 4-point lines to get sharper finite-n intuition.

(iv) Minimal counterexample structure: would be a family of point sets $P_n$ with no $k+1$ collinear but with $\Theta(n^2)$ distinct $k$-point lines. By Lemma 588.1 this would require nearly every point-pair to lie on a $k$-point line, i.e. an incidence structure approaching a block design where each line contains exactly $k$ points and pairs are almost perfectly packed into these lines.


