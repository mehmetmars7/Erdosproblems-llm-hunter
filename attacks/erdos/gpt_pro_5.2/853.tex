% Erdos Problem #853

\subsection*{FORMAL RESTATEMENT}
Let $p_n$ be the $n$th prime and $d_n=p_{n+1}-p_n$ as above. For an integer $x\ge 1$, define
\[
  r(x):= \min\Bigl\{ t\in 2\mathbb{N}:\ \forall n\in\{1,\dots,x\},\ d_n\ne t\Bigr\}.
\]
(So $r(x)$ is the smallest \emph{even} positive integer which does not occur among the first $x$ prime gaps.)

The problem asks whether $r(x)\to\infty$ as $x\to\infty$, and even whether $r(x)/\log x\to\infty$.

\subsection*{QUICK LITERATURE/CONTEXT CHECK}
I do not import external results beyond what is explicitly stated in the problem file.  The evenness restriction is necessary because $d_n$ is even for $n\ge 2$.

\subsection*{ATTACK PLAN}
\begin{itemize}
\item \textbf{Proof-track:} Derive structural constraints on which even gaps can appear among consecutive primes (this connects to conjectures like Polignac's).
\item \textbf{Disproof-track:} Search for an even $t$ that plausibly never appears as a consecutive prime gap (no such $t$ is known to me from the given text), or for computational evidence that $r(x)$ stabilizes.
\item \textbf{Reality-check:} Compute $r(x)$ for moderate $x$ and compare to $\log x$.
\end{itemize}

\subsection*{WORK}
\textbf{Lemma 853.1 (Even gaps past $p_2$).}
For every $n\ge 2$, $d_n$ is even.

\textbf{Proof.}
This is Lemma~852.1.\ $\square$

\medskip
\textbf{Lemma 853.2 (Monotonicity).}
The function $r(x)$ is nondecreasing in $x$.

\textbf{Proof.}
If $x_1\le x_2$, then the set of gaps seen up to index $x_1$, namely $\{d_n:1\le n\le x_1\}$, is a subset of the gaps seen up to $x_2$. Therefore the set of even integers missing up to $x_2$ is a subset of those missing up to $x_1$, so the smallest missing even integer cannot decrease: $r(x_2)\ge r(x_1)$.\ $\square$

\medskip
\textbf{Lemma 853.3 (A pigeonhole upper bound $r(x)\le 2x$).}
For every integer $x\ge 1$, $r(x)\le 2x$.

\textbf{Proof.}
Consider the $x$ even integers $2,4,6,\dots,2x$. If every one of these occurred among $d_1,\dots,d_x$, then in particular at least $x$ of the gaps $d_1,\dots,d_x$ would be even.

But $d_1=p_2-p_1=3-2=1$ is odd, so among $d_1,\dots,d_x$ there are at most $x-1$ even values. Hence it is impossible for all $x$ even integers $2,4,\dots,2x$ to appear among the first $x$ gaps. Therefore at least one even integer $\le 2x$ is missing, and by definition $r(x)$ (the smallest missing even) satisfies $r(x)\le 2x$.\ $\square$

\medskip
\textbf{Lemma 853.4 (Trivial bound via the maximum observed even gap).}
Let $M(x):=\max\{d_n:1\le n\le x\ \text{and}\ d_n\ \text{even}\}$ (for $x\ge 2$). Then $r(x)\le M(x)+2$.

\textbf{Proof.}
Every even integer $t>M(x)$ is absent from $\{d_1,\dots,d_x\}$ by definition of $M(x)$. The smallest even integer strictly larger than $M(x)$ is $M(x)+2$, so the smallest missing even integer is at most $M(x)+2$.\ $\square$

\medskip
\textbf{FAST REALITY CHECK (computed values).}
Using primes up to index $x=10^6$ (so about $p_{10^6}\approx 1.5\times 10^7$), I computed the set of even gaps occurring among $d_1,\dots,d_x$ and extracted $r(x)$.

Results:
\begin{verbatim}
Selected x and r(x):
 r(10)=8
 r(20)=8
 r(50)=16
 r(100)=16
 r(200)=16
 r(500)=30
 r(1,000)=36
 r(2,000)=38
 r(5,000)=46
 r(10,000)=66
 r(20,000)=74
 r(50,000)=92
 r(100,000)=102
 r(200,000)=116
 r(500,000)=140
 r(1,000,000)=156

For x=1,000,000:
 max even gap among d_1..d_x is 154
 distinct even gaps seen: 77 values, min=2, max=154
 all even gaps 2,4,...,154 occur at least once; the first missing even is 156.
\end{verbatim}

\subsection*{VERIFICATION}
\begin{itemize}
\item \textbf{Definition check:} $r(x)$ ignores odd gaps by definition (searches over $t\in 2\mathbb{N}$).
\item \textbf{Lemma 853.3:} The only place the special gap $d_1=1$ is used is to ensure at most $x-1$ even gaps among the first $x$ gaps.
\item \textbf{Computation sanity:} For $x=1$, the gap multiset is $\{1\}$, so the smallest missing even is $2$, hence $r(1)=2$, consistent with monotonicity and the reported larger values.
\item \textbf{Computed claim ``all evens up to 154 appear'':} This was verified by explicitly checking membership of each even $\le 154$ in the computed set of gaps.
\end{itemize}

\subsection*{FINAL}
\textbf{UNRESOLVED}
\begin{enumerate}
\item[(i)] \textbf{Strongest proved partial result.} $r(x)$ is nondecreasing (Lemma~853.2) and satisfies the unconditional bounds $r(x)\le 2x$ (Lemma~853.3) and $r(x)\le M(x)+2$ where $M(x)$ is the maximum even gap seen up to $x$ (Lemma~853.4).
\item[(ii)] \textbf{First gap (crisp).} Prove or disprove: for every fixed even $t$, there exists $n$ with $d_n=t$ (equivalently, $r(x)\to\infty$).
\item[(iii)] \textbf{Top 3 next moves.}
  \begin{enumerate}
  \item Try to show that for each fixed even $t$, the pattern ``two primes at distance $t$ with all in-between composites'' occurs at least once (a very weak form of Polignac-type statements).
  \item Compute $r(x)$ for much larger $x$ and record when each even gap $2m$ first appears; look for systematic obstructions/modular patterns.
  \item Prove nontrivial lower bounds on $r(x)$ (e.g. $r(x)\gg \log x$) by showing that among the first $x$ gaps, not all small even values can be covered.
  \end{enumerate}
\item[(iv)] \textbf{Minimal counterexample structure.} If $r(x)$ fails to go to infinity, then there exists an even $t_0$ such that $d_n\ne t_0$ for all $n$ (i.e. a specific even integer never occurs as a consecutive prime gap).
\end{enumerate}

