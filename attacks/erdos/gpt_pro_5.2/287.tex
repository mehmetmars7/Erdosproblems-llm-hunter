\section*{Erd\H{o}s Problem \#287}

\subsection*{1) FORMAL RESTATEMENT}
For $k\ge 2$, let $2\le n_1<\cdots<n_k$ be distinct integers with
\[1=\frac1{n_1}+\cdots+\frac1{n_k}.
\]
The conjecture asserts that necessarily
\[\max_{1\le i\le k-1} (n_{i+1}-n_i)\ge 3.
\]
Equivalently: there is no Egyptian-fraction representation of $1$ with all consecutive gaps in the denominator set bounded by $2$.

\subsection*{2) QUICK LITERATURE/CONTEXT CHECK}
The bound ``$\ge 3$ would be best possible'' is witnessed by the classical example
\[1=\frac12+\frac13+\frac16,\]
whose denominator gaps are $1$ and $3$. It is known (Erd\H{o}s, 1932) that the weaker bound $\ge 2$ holds, i.e. $1$ is \emph{not} a sum of reciprocals of \emph{consecutive} integers $n,n+1,\dots,n+k-1$ with $n>1$.

A heuristic/conditional route mentioned in the literature: the conjecture would follow (up to finitely many exceptions) from the existence, for all large $N$, of a prime $p\in[N,2N]$ such that $(p+1)/2$ is also prime (a ``Sophie Germain pair'').

\subsection*{3) ATTACK PLAN}
\begin{enumerate}[label=\textbf{P\arabic*.},leftmargin=*]
\item \textbf{Strengthen the Erd\H{o}s (1932) parity method.} The classical ``no consecutive denominators'' result can be proved via $2$-adic valuation of the lcm. Try to adapt this to the ``gaps $\le 2$'' setting by looking at a prime (or prime power) that divides exactly one chosen denominator to an extremal exponent.
\item \textbf{Density/interval viewpoint.} If gaps $\le 2$, the denominators occupy an interval with at most isolated missing points. Translate the equation into a near-harmonic sum and attempt to derive a contradiction from congruences or from Diophantine approximation properties of harmonic numbers.
\item \textbf{Finite counterexample search.} Because the conjecture is falsifiable, perform exhaustive search for counterexamples with $n_k$ bounded.
\end{enumerate}

\subsection*{4) WORK}
\subsubsection*{4.1. Proof of the weaker bound $\max(n_{i+1}-n_i)\ge 2$ (no consecutive denominators)}
\begin{theorem}
There do not exist integers $n\ge 2$ and $k\ge 1$ such that
\[1=\sum_{j=0}^{k-1}\frac1{n+j}.
\]
In particular, any Egyptian-fraction representation of $1$ with distinct denominators must have a gap at least $2$.
\end{theorem}
\begin{proof}
Assume for contradiction that $1=\sum_{j=0}^{k-1} \frac1{n+j}$ with $n\ge 2$. Let
\[L=\mathrm{lcm}(n,n+1,\dots,n+k-1).
\]
Because $n\ge 2$, among $n,n+1,\dots,n+k-1$ there is at least one even number, hence $2\mid L$. Let $t=\max_{0\le j\le k-1} v_2(n+j)$ be the maximal $2$-adic valuation among the denominators; then $v_2(L)=t\ge 1$.

\smallskip
\noindent\emph{Claim:} exactly one of the integers $n,n+1,\dots,n+k-1$ has $2$-adic valuation equal to $t$.
Indeed, if there were two distinct terms with valuation $t$, their difference would be divisible by $2^{t+1}$ (difference of two odd multiples of $2^t$), so the interval length $k-1$ would be at least $2^{t+1}$. But any block of $2^{t+1}$ consecutive integers contains a multiple of $2^{t+1}$, contradicting maximality of $t$. Hence the maximiser is unique.

Now multiply the assumed identity by $L$:
\[L=\sum_{j=0}^{k-1} \frac{L}{n+j}.
\]
Let $d_*$ be the unique denominator with $v_2(d_*)=t$. Then $L/d_*$ is odd (all factors of $2$ cancel), while for every other denominator $d$ we have $v_2(d)\le t-1$ and hence $L/d$ is even. Therefore the right-hand side is (odd) $+$ (sum of evens), hence odd. But the left-hand side $L$ is even. Contradiction.
\end{proof}

\subsubsection*{4.2. Exhaustive search for a counterexample with gaps $\le 2$ and bounded $n_k$}
A brute-force backtracking search was implemented for all strictly increasing sequences $2\le n_1<\cdots<n_k\le N$ with successive gaps in $\{1,2\}$, checking whether $\sum 1/n_i=1$ exactly.

\medskip
\noindent\textbf{Result:} no counterexample exists with $n_k\le 47$. (The search is exhaustive for this range.)

\subsubsection*{4.3. Why this does not yet prove the conjecture}
The problem remains open because the conjecture could still fail only at much larger denominators, and there is currently no general argument ruling out all ``gap $\le 2$'' patterns. The $2$-adic lcm argument in \S4.1 relies heavily on having \emph{all} denominators consecutive; when isolated denominators are removed, the uniqueness-of-maximum-$v_2$ property can fail in a global set.

\subsection*{5) VERIFICATION}
\begin{itemize}
\item The proof in \S4.1 is self-contained and purely elementary.
\item The computation in \S4.2 is exhaustive for $n_k\le 47$ by construction (enumerating all admissible gap-$\le 2$ patterns in $[2,N]$).
\end{itemize}

\subsection*{6) FINAL}
\textbf{UNRESOLVED.}
\begin{enumerate}[label=(\roman*),leftmargin=*]
\item \textbf{Furthest point reached:} gave a complete proof of the known weaker statement ``no consecutive denominators'' (hence gap $\ge 2$), and performed an exhaustive search showing no gap-$\le 2$ counterexample exists with maximum denominator $\le 47$.
\item \textbf{Key gap:} extending the $2$-adic/lcm obstruction to the setting where isolated denominators may be missing from the interval (gaps allowed to be $2$) remains unresolved.
\item \textbf{Most promising next step:} find a prime-power valuation (not necessarily $2$) that is forced to appear in exactly one denominator when gaps are $\le 2$, or relate the conjecture to distribution of Sophie Germain primes (as suggested by the conditional approach).
\item \textbf{Explicit missing lemma that would close it:} a theorem ensuring that in any gap-$\le 2$ denominator set supporting $\sum 1/n_i=1$, there exists a prime $p$ whose exact $p$-adic valuation in the lcm is attained uniquely by one denominator (giving an immediate contradiction mod $p$).
\end{enumerate}

\subsection*{7) COMPLETION ESTIMATE}
\noindent\textbf{COMPLETION: 35\%}

