\section*{Problem \#260}
\addcontentsline{toc}{section}{Problem \#260}

\subsection*{1. Formal Restatement}
Let $a_1<a_2<\cdots$ be a strictly increasing sequence of positive integers such that
\[
\frac{a_n}{n}\to \infty \qquad (n\to\infty).
\]
Define
\[
S\;:=\;\sum_{n=1}^{\infty} \frac{a_n}{2^{a_n}}\,.
\]
Question: Must $S$ be irrational?

\subsection*{2. Quick Literature / Context Check}
This is recorded as an open Erd\H{o}s problem; Erd\H{o}s proved irrationality under stronger growth/separation hypotheses, e.g. $a_{n+1}-a_n\to\infty$ (and also a quantitative lower bound of the form $a_n\gg n\sqrt{\log n\,\log\log n}$). See the problem database summary and references therein.  

\subsection*{3. Attack Plan}
The natural approach (used in the classical proofs under separation conditions) is to analyze the binary expansion.

Write $a_n$ in base $2$:
\[
 a_n = \sum_{j=0}^{L_n} \varepsilon_{n,j}2^j,\qquad \varepsilon_{n,j}\in\{0,1\},\quad L_n:=\lfloor\log_2 a_n\rfloor.
\]
Then
\[
\frac{a_n}{2^{a_n}}=\sum_{j=0}^{L_n}\varepsilon_{n,j}\,2^{-(a_n-j)}.
\]
So each summand contributes a finite ``block'' of binary digits supported on positions $\{a_n-L_n,\dots,a_n\}$.  
If these blocks are eventually disjoint, then there are no carries, and the binary expansion of the tail is a concatenation of blocks separated by runs of zeros. If the zero-run lengths are unbounded, this forces non-eventual periodicity, hence irrationality.

A sufficient condition for eventual block disjointness is that
$a_{n+1}-a_n$ eventually dominates $\log_2 a_{n+1}$ (e.g. if
$(a_{n+1}-a_n)/\log a_{n+1}\to\infty$).  Note that Erd\H{o}s' published
result is stronger: he proved irrationality already under the weaker
hypothesis $a_{n+1}-a_n\to\infty$, which still allows some overlap/carry.

\subsection*{4. Work}
\paragraph{A proved partial result (a simple no-carry criterion).}
We give a self-contained proof of the following standard ``no-carry'' criterion.

\begin{theorem}[Irrationality under eventual block separation]
\label{thm:260-separated}
Let $a_1<a_2<\cdots$ be strictly increasing positive integers and write
$L_n:=\lfloor\log_2 a_n\rfloor$. Suppose that
\begin{equation}
\label{eq:blocksep}
 (a_{n+1}-a_n) - L_{n+1} \to \infty.
\end{equation}
(Equivalently: the ``zero gap'' between the binary digit block contributed by
$a_n/2^{a_n}$ and the next block contributed by $a_{n+1}/2^{a_{n+1}}$ tends to
$\infty$.)
Then $\sum_{n\ge 1} a_n/2^{a_n}$ is irrational.
\end{theorem}

\begin{proof}
For each $n$ write
\[
\frac{a_n}{2^{a_n}}=\sum_{j=0}^{L_n}\varepsilon_{n,j}\,2^{-(a_n-j)},
\qquad L_n=\lfloor\log_2 a_n\rfloor.
\]
Define the ``support interval'' of the $n$th block by
\[
I_n:=\{a_n-L_n,\,a_n-L_n+1,\,\dots,\,a_n\}.
\]
We claim that \eqref{eq:blocksep} implies $I_n\cap I_{n+1}=\emptyset$ for all sufficiently large $n$. Indeed, the smallest exponent in $I_{n+1}$ is $a_{n+1}-L_{n+1}$, and the largest exponent in $I_n$ is $a_n$; thus disjointness follows from
\[
 a_{n+1}-L_{n+1} > a_n
 \quad\Longleftrightarrow\quad (a_{n+1}-a_n)-L_{n+1}>0,
\]
which holds for all large $n$ because the left-hand side tends to $+\infty$ by \eqref{eq:blocksep}.

Fix $N$ large enough that $I_n$ are pairwise disjoint for all $n\ge N$. Consider the tail
\[
T:=\sum_{n=N}^{\infty}\frac{a_n}{2^{a_n}}.
\]
Because the binary supports $I_n$ are disjoint, when adding the dyadic rationals $a_n/2^{a_n}$ there are \emph{no carries at all} in base $2$ for the tail: for each binary digit position $k\in\mathbb{N}$, at most one summand contributes a $1$ to the coefficient of $2^{-k}$. Therefore the binary expansion of $T$ is obtained by placing, for each $n\ge N$, the binary digits of $a_n$ into the positions indexed by $I_n$, with zeros elsewhere.

Next, we show that the binary expansion of $T$ contains arbitrarily long runs of zeros. Indeed, the gap between the end of block $I_n$ and the start of block $I_{n+1}$ has length
\[
\bigl(a_{n+1}-L_{n+1}\bigr) - a_n - 1 
= (a_{n+1}-a_n) - L_{n+1} - 1.
\]
By \eqref{eq:blocksep}, the quantity $(a_{n+1}-a_n)-L_{n+1}$ tends to $+\infty$, so the displayed gap length tends to $+\infty$ as well. Hence there are arbitrarily long strings of consecutive binary digits of $T$ that are zero.

Finally, recall a basic fact: a real number is rational if and only if its base-$2$ expansion is eventually periodic. An eventually periodic binary expansion that is not eventually all zeros has a bounded maximum run length of consecutive zeros (bounded by the period length). Therefore, a binary expansion with arbitrarily long runs of zeros must either terminate (eventually all zeros) or be irrational.

Our tail $T$ cannot terminate because for each $n\ge N$ the binary expansion of $a_n$ has a leading $1$, so each block contributes at least one $1$ digit, and there are infinitely many blocks. Therefore $T$ is irrational.

Since $S = \sum_{n< N} a_n/2^{a_n} + T$ is the sum of a rational number and an irrational number, $S$ is irrational.
\end{proof}

\paragraph{Consequence.}
Condition \eqref{eq:blocksep} certainly holds if, for example,
$(a_{n+1}-a_n)/\log a_{n+1}\to\infty$ (so the gaps dominate the block lengths).
Under this kind of strong separation, the binary-carry issues disappear.
Erd\H{o}s' published theorem is stronger: he proved irrationality already under the
weaker hypothesis $a_{n+1}-a_n\to\infty$.

\subsection*{5. Verification / Sanity Checks}
\begin{itemize}[leftmargin=2em]
\item \emph{No-carry check:} Disjointness of supports $I_n$ ensures no binary digit position receives two $1$'s, hence no carry.
\item \emph{Arbitrarily long zeros $\Rightarrow$ irrational:} An eventually periodic binary expansion has bounded zero-runs unless it terminates.
\item \emph{Tail argument:} Changing finitely many initial terms does not affect rationality.
\end{itemize}

\subsection*{6. FINAL}
\textbf{UNRESOLVED}

\subsubsection*{Why stuck / what remains open}
The actual hypothesis in the problem ($a_n/n\to\infty$) allows the gaps $a_{n+1}-a_n$ to remain bounded (or at least fail to go to infinity), leading to potential complicated carry interactions in base $2$. The above ``block separation'' proof breaks precisely when carries can occur infinitely often.

\subsubsection*{Strongest partial result proved here}
Theorem~\ref{thm:260-separated}: if the gaps between successive binary-digit blocks tend to infinity, i.e.
\[
 (a_{n+1}-a_n)-\lfloor\log_2 a_{n+1}\rfloor\to\infty,
\]
then $\sum a_n/2^{a_n}$ is irrational (by a direct ``no-carry'' and ``arbitrarily long zero run'' argument).  Erd\H{o}s proved irrationality under the weaker assumption $a_{n+1}-a_n\to\infty$.

\subsubsection*{Next steps (concrete)}
\begin{enumerate}[leftmargin=2em]
\item Develop a carry-control invariant: quantify how many carries can propagate when $a_{n+1}-a_n$ is small compared to $\log a_n$, and show this still forces non-eventual periodicity under $a_n/n\to\infty$.
\item Search for a counterexample: attempt to engineer $a_n$ so that the binary expansion becomes eventually periodic via systematic carry propagation.
\item Explore analytic criteria (e.g. Diophantine approximation of dyadic sums) that bypass digit carries, possibly via lower bounds on distance to dyadic rationals.
\end{enumerate}

\subsection*{7. Completion Estimate}
About \textbf{40\%} toward a full resolution: the standard ``gap'' case is settled cleanly, but the core difficulty is the carry-dominated regime allowed by $a_n/n\to\infty$.


%%%%%%%%%%%%%%%%%%%%%%%%%%%%%%%%%%%%%%%%%%%%%%%%%%%%%%%%%%%%%%%%%%%%%%%%%%%%%%%
