\section*{Problem 620}

\subsection*{FORMAL RESTATEMENT}

For a graph $G=(V,E)$ and $W\subseteq V$, let $G[W]$ denote the induced subgraph.
Define
\[
 f(n)\ :=\ \min\Bigl\{\ \max\bigl\{|W|:\ W\subseteq V(G),\ \text{$G[W]$ is triangle-free}\bigr\}:\ |V(G)|=n,\ \text{$G$ is $K_4$-free}\ \Bigr\}.
\]
Equivalently, $f(n)$ is the largest integer $m$ such that every $n$-vertex $K_4$-free graph contains an induced triangle-free subgraph on at least $m$ vertices.
The problem asks to determine the order of growth of $f(n)$.

\subsection*{QUICK LITERATURE/CONTEXT CHECK}

The prompt states that $f(n)=n^{1/2+o(1)}$ is known and records the best bounds (up to polylogarithmic factors):
\[
 \Omega\!\left(\frac{\sqrt{n\log n}}{\log\log n}\right)\ \le\ f(n)\ \le\ O\!\left(\sqrt{n}\,\log n\right).
\]
Earlier work gave weaker polylogarithmic factors in the upper bound (e.g. $\sqrt{n}(\log n)^{120}$), later improved to $O(\sqrt{n}\log n)$.
The ErdosProblems page lists the problem as open in the sense that the precise polylogarithmic factors (and constant-order behavior) are not settled.

\subsection*{ATTACK PLAN}

A robust baseline is to prove the ``$\sqrt n$'' lower bound directly.
The key structural fact in $K_4$-free graphs is that \emph{neighborhoods are triangle-free}: if $v\in V(G)$, then $G[N(v)]$ contains no triangle (else together with $v$ it would form a $K_4$).
Thus either there is a vertex of large degree (yielding a large triangle-free induced neighborhood), or all degrees are small, in which case a greedy bound produces a large independent set (which is triangle-free).

\subsection*{WORK}

\begin{theorem}[Elementary lower bound]
\label{thm:sqrtn}
Every $K_4$-free graph on $n$ vertices contains an induced triangle-free subgraph on at least $\lfloor\sqrt{n}\rfloor$ vertices. Consequently,
\[
 f(n)\ge \lfloor\sqrt{n}\rfloor.
\]
\end{theorem}

\begin{proof}
Let $G$ be a $K_4$-free graph on $n$ vertices and set $s:=\lfloor\sqrt{n}\rfloor$.
Write $\Delta$ for its maximum degree.

\emph{Case 1: $\Delta\ge s$.}
Choose a vertex $v$ with $\deg(v)=\Delta$.
Claim: the induced subgraph $G[N(v)]$ is triangle-free.
Indeed, if $x,y,z\in N(v)$ form a triangle in $G$, then $\{v,x,y,z\}$ spans a $K_4$, contradicting that $G$ is $K_4$-free.
Therefore $G[N(v)]$ is an induced triangle-free subgraph on $|N(v)|=\Delta\ge s$ vertices.

\emph{Case 2: $\Delta\le s-1$.}
A standard greedy bound gives an independent set of size at least $\frac{n}{\Delta+1}$: repeatedly choose a vertex, put it in the independent set, and delete it together with its neighbors; each chosen vertex removes at most $\Delta+1$ vertices.
Thus
\[
\alpha(G)\ \ge\ \frac{n}{\Delta+1}\ \ge\ \frac{n}{s}.
\]
Since $s=\lfloor\sqrt{n}\rfloor$ implies $s^2\le n$, we have $n/s\ge s$, and so $\alpha(G)\ge s$.
An independent set induces an edgeless (hence triangle-free) subgraph, so this yields an induced triangle-free subgraph on at least $s$ vertices.

In both cases, $G$ contains an induced triangle-free subgraph on at least $s=\lfloor\sqrt{n}\rfloor$ vertices.
Since $G$ was arbitrary $K_4$-free, the minimum over such $G$ satisfies $f(n)\ge \lfloor\sqrt{n}\rfloor$.
\end{proof}

\paragraph{Remarks toward the stated best-known bounds.}
The proof above explains the ``$\sqrt n$'' phenomenon: it is essentially forced by the dichotomy ``large degree gives large triangle-free neighborhood'' vs ``bounded degree gives large independent set.''
Improving the lower bound by the factor $\sqrt{\log n}/\log\log n$ and constructing $K_4$-free graphs with no induced triangle-free subgraph larger than $O(\sqrt{n}\log n)$ requires substantially deeper probabilistic and Ramsey-theoretic tools (e.g. the triangle-free process, dependent random choice, and container-type arguments), beyond what is reproduced here.

\subsection*{VERIFICATION}

\begin{itemize}[leftmargin=2.2em]
\item The neighborhood-triangle argument is correct: a triangle inside $N(v)$ would create a $K_4$ with $v$.
\item The greedy independent-set bound $\alpha(G)\ge n/(\Delta+1)$ is standard and holds for all finite graphs.
\item The two cases cover all graphs and produce an induced triangle-free subgraph either as a neighborhood subgraph or as an independent set.
\end{itemize}

\subsection*{FINAL}

\textbf{UNRESOLVED}

\begin{enumerate}[label=(\roman*),leftmargin=2.5em]
\item \textbf{Where and why the attempt fails.}
The problem asks for the true asymptotic order (including the correct polylogarithmic factor) of $f(n)$. The elementary argument here only proves the baseline $f(n)\ge \lfloor\sqrt n\rfloor$.
Closing the gap between the best known lower bound $\Omega(\sqrt{n\log n}/\log\log n)$ and the best known upper bound $O(\sqrt n\log n)$ appears to require sophisticated probabilistic constructions and/or new structural ideas about $K_4$-free graphs and induced triangle-free subgraphs.

\item \textbf{Strongest partial result proved here.}
A complete proof that $f(n)\ge \lfloor\sqrt n\rfloor$ (Theorem~\ref{thm:sqrtn}).

\item \textbf{Most plausible next ideas.}
To improve the lower bound, one needs a stronger dichotomy than ``large degree vs greedy independent set,'' likely using information about triangle structure in $K_4$-free graphs and refined bounds on large triangle-free induced subgraphs (e.g. Shearer-type entropy methods).
To improve the upper bound, one needs an explicit or probabilistic construction of $K_4$-free graphs in which every induced subgraph on $m$ vertices contains a triangle for $m$ as large as possible (currently $m$ about $\sqrt n\log n$).

\item \textbf{Small cases / computations.}
For small $n$, $f(n)$ can in principle be computed by brute force over $K_4$-free graphs, but this grows rapidly.
Theorem~\ref{thm:sqrtn} is consistent with the small-$n$ regime and provides a universally valid baseline.
\end{enumerate}

\subsection*{COMPLETION ESTIMATE}

COMPLETION ESTIMATE: 45\%.

