% Erdos Problem #327
% URL: https://www.erdosproblems.com/327

1) FORMAL RESTATEMENT

Fix an integer $N\ge 1$.
Let $A\subseteq\{1,2,\dots,N\}$.

Condition (C1): for all distinct $a,b\in A$,
\[
(a+b)\nmid ab.
\]

Variant condition (C2): for all distinct $a,b\in A$,
\[
(a+b)\nmid 2ab.
\]

Questions:

(Q1) Under (C1), can $|A|$ be substantially larger than the number of odd integers in $\{1,\dots,N\}$ (which is $\lceil N/2\rceil$)?

(Q2) Under (C2), what is the maximum possible size of $A$?

(Q3) Under either condition, must $|A|=o(N)$ as $N\to\infty$?

Edge cases.
For $N=1$, any $A$ works. For $N\ge 2$, the condition constrains pairs.


2) QUICK LITERATURE/CONTEXT CHECK

I do not use external results beyond the statement.
The statement mentions an elementary argument (van Doorn) that if $|A|\ge (25/28+o(1))N$ then $A$ must contain a forbidden pair for (C1).


3) ATTACK PLAN

Provide:

(1) A structural equivalence between $(a+b)\mid ab$ and ``$\frac1a+\frac1b$ is a unit fraction'', and a useful Diophantine reformulation.

(2) An explicit infinite family of forbidden pairs, showing why certain multiplicative patterns are excluded.

(3) A brute-force maximum-size computation for small $N$ to see what densities are achievable.


4) WORK

PHASE 1: FAST REALITY CHECK (exact maxima for small $N$)

I computed the maximum size of $A\subseteq\{1,\dots,N\}$ satisfying (C1) and (C2) by exhaustive search for $N\le 40$.
The exact maxima found include:

For (C1):
\[
\begin{array}{c|cccc}
N & 20 & 30 & 35 & 40\\\hline
\max |A| & 15 & 23 & 27 & 31
\end{array}
\]
corresponding densities approximately $0.75, 0.7667, 0.7714, 0.775$.

For (C2):
\[
\begin{array}{c|ccc}
N & 20 & 30 & 40\\\hline
\max |A| & 15 & 22 & 31
\end{array}
\]

For (C1) at $N=20$, one maximum example is
\[
A=\{1,2,3,4,5,7,8,9,10,11,13,14,16,17,19\},\quad |A|=15.
\]
So at least for small $N$, $|A|$ can be significantly larger than $\lceil N/2\rceil$.


Lemma 327.1 (unit-fraction equivalence and a Diophantine form).

For integers $a,b\ge 1$, the following are equivalent:

(a) $(a+b)\mid ab$.

(b) There exists an integer $c\ge 1$ such that
\[
\frac1a+\frac1b=\frac1c.
\]

Moreover, if $c=\frac{ab}{a+b}$, then
\[
(a-c)(b-c)=c^2.
\]

Proof.
Compute
\[
\frac1a+\frac1b=\frac{a+b}{ab}.
\]
This is a unit fraction $1/c$ with integer $c$ if and only if $(a+b)/ab=1/c$, i.e.
\[
\frac{ab}{a+b}=c\in\mathbb Z_{\ge 1}.
\]
That is exactly $(a+b)\mid ab$, and in that case $c=ab/(a+b)$.

For the final identity, start from $c=ab/(a+b)$, i.e. $ab=c(a+b)$.
Then
\[
(a-c)(b-c)=ab-c(a+b)+c^2 = c(a+b)-c(a+b)+c^2=c^2.
\]
\qed


Lemma 327.2 (an explicit infinite family of forbidden pairs for (C1)).

Fix integers $m\ge 1$ and $t\ge 1$.
Let
\[
a=(m+1)t,\qquad b=m(m+1)t.
\]
Then $(a+b)\mid ab$.

Proof.
Compute
\[
a+b=(m+1)t+m(m+1)t=(m+1)^2 t.
\]
Also
\[
ab=(m+1)t\cdot m(m+1)t = m(m+1)^2 t^2.
\]
Therefore
\[
\frac{ab}{a+b}=\frac{m(m+1)^2 t^2}{(m+1)^2 t}=mt\in\mathbb Z.
\]
So $(a+b)\mid ab$.
\qed


Lemma 327.3 (odd numbers give a baseline construction for (C1)).

Let $A$ be the set of odd integers in $\{1,\dots,N\}$. Then $A$ satisfies (C1).

Proof.
If $a,b$ are odd and distinct, then $ab$ is odd while $a+b$ is even.
An even number cannot divide an odd number, so $(a+b)\nmid ab$.
\qed


5) VERIFICATION

-- Lemma 327.1: the equivalence is a direct algebraic manipulation; the Diophantine identity follows by expansion.

-- Lemma 327.2: division is exact; quotient $mt$ is an integer.

-- Lemma 327.3: parity argument is correct.

-- Computations: verified by exhaustive subset search for $N\le 40$.


6) FINAL

**UNRESOLVED**

(i) Strongest fully proved partial result obtained here.

* Structural equivalence between forbidden pairs and unit fractions, including the identity $(a-c)(b-c)=c^2$ (Lemma 327.1).
* An explicit infinite family of forbidden pairs (Lemma 327.2).
* Baseline construction of size $\lceil N/2\rceil$ (odd numbers) satisfying (C1) (Lemma 327.3).
* Exact maxima for $N\le 40$ suggest densities around $0.77$ are achievable for (C1) at small $N$.

(ii) Exact first gap.

Determine the true asymptotic maximum density of sets $A\subseteq\{1,\dots,N\}$ avoiding all pairs with $(a+b)\mid ab$ (or $(a+b)\mid 2ab$). In particular, I did not prove $|A|=o(N)$ nor exhibit a positive-density construction for all large $N$ beyond computational evidence.

(iii) Top 3 next moves (concrete targets).

1. Prove a positive-density construction for all large $N$ (matching or improving the $\approx 0.77$ densities seen computationally), with a rigorous verification of (C1).
2. Improve upper bounds on density beyond the stated $(25/28)$ threshold, using the Diophantine parametrization in Lemma 327.1.
3. Analyze the ``forbidden graph'' on $\{1,\dots,N\}$ where edges are forbidden pairs, and study its independence number asymptotically.

(iv) What a minimal counterexample would likely look like.

If $|A|=o(N)$ were true, then any set of positive density would necessarily contain many forbidden pairs.
A minimal counterexample to an $o(N)$ claim would likely be a sequence of sets $A_N$ of size $\delta N$ with fixed $\delta>0$ having a strong multiplicative structure that avoids the parametrized forbidden families of Lemma 327.2.


