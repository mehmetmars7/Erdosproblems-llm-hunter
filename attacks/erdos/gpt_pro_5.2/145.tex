\section*{Erd\H{o}s Problem \#145 (Moments of gaps between squarefree numbers)}

\subsection*{1) FORMAL RESTATEMENT}
Let $1=s_1<s_2<s_3<\cdots$ be the increasing sequence of squarefree positive integers.
For $x\ge 1$, let
\[
N(x):=|\{n:\ s_n\le x\}|.
\]
For $\alpha\ge 0$, define
\[
M_{\alpha}(x):=\frac{1}{x}\sum_{n\le N(x)}(s_{n+1}-s_n)^{\alpha}.
\]
(The sum is finite because $N(x)<\infty$ for finite $x$, and $s_{N(x)+1}$ is defined because the squarefree numbers are infinite.)

The question is: for every fixed $\alpha\ge 0$, does the limit
\[
\lim_{x\to\infty} M_{\alpha}(x)
\]
exist?

\subsection*{2) QUICK LITERATURE / CONTEXT CHECK}
\begin{itemize}
\item Under the $abc$ conjecture, Granville (1998) derived the asymptotic predicted by Erd\H{o}s for all moments of the gaps between squarefree numbers.
\item Unconditionally, Chan (2023) proved an asymptotic $\sum_{s_{k+1}\le x}(s_{k+1}-s_k)^{\gamma}\sim B(\gamma)x$ for $0\le\gamma<3.75$, which in particular implies existence of the limit for $0\le\alpha<3.75$.
\item For the special exponents $\alpha=0$ and $\alpha=1$, the limit exists and can be proved elementarily: $\alpha=0$ is the natural density of squarefree numbers ($6/\pi^2$), and $\alpha=1$ telescopes to $x$ up to a negligible boundary term.
\end{itemize}

\subsection*{3) ATTACK PLAN}
\textbf{Proof track.}
\begin{enumerate}
\item Prove the cases $\alpha=0$ and $\alpha=1$ directly (baseline, fully rigorous).
\item For general $\alpha$, expand moments in terms of local configurations of squarefree/non-squarefree patterns and estimate via inclusion--exclusion. This rapidly leads to difficult correlation sums of $\mu^2(n)$.
\item Use known deep results as black boxes (Chan 2023; conditional Granville 1998) to describe the state of the art.
\end{enumerate}

\textbf{Disproof track.}
\begin{enumerate}
\item A counterexample would require that $M_{\alpha}(x)$ oscillates without settling. This would likely come from persistent, scale-dependent irregularities in the frequency of very large gaps.
\item Heuristically, squarefree numbers behave like a weakly dependent Bernoulli process with density $6/\pi^2$, suggesting $M_{\alpha}(x)$ should converge for all $\alpha$, but converting this to proof is hard.
\end{enumerate}

\subsection*{4) WORK}
\paragraph{Phase 1 sanity check by computation.}
A quick sieve computation up to $10^6$ (not a proof) suggests stabilization of $M_{\alpha}(x)$ for a few sample $\alpha$:
\[
\begin{array}{c|c}
\alpha & M_{\alpha}(10^6)\ \ \hline
0 & 0.607929\\
1 & 1.000004\\
2 & 2.041249\\
3 & 6.624972
\end{array}
\]
These are consistent with the existence of limits (and with $M_0(\infty)=6/\pi^2\approx0.607927$).

\paragraph{Case $\alpha=0$.}
Observe that $(s_{n+1}-s_n)^0=1$ for all $n$, hence
\[
M_0(x)=\frac{1}{x}\sum_{n\le N(x)}1=\frac{N(x)}{x}.
\]
Thus we need $N(x)/x\to 6/\pi^2$.

\begin{lemma}[Counting squarefree numbers]
\label{lem:145-squarefreecount}
Let $Q(x):=|\{n\le x: n\ \text{squarefree}\}|$. Then
\[
Q(x)=\frac{x}{\zeta(2)}+O(\sqrt{x})
\qquad (x\to\infty).
\]
In particular $Q(x)/x\to 1/\zeta(2)=6/\pi^2$.
\end{lemma}
\begin{proof}
Use the identity $\mu^2(n)=\sum_{d^2\mid n}\mu(d)$ (since $\mu^2$ is the indicator of squarefree integers) to write
\[
Q(x)=\sum_{n\le x}\mu^2(n)=\sum_{n\le x}\sum_{d^2\mid n}\mu(d)=\sum_{d\le \sqrt{x}}\mu(d)\Big\lfloor \frac{x}{d^2}\Big\rfloor.
\]
Split
\[
\Big\lfloor \frac{x}{d^2}\Big\rfloor=\frac{x}{d^2}+O(1)
\]
and obtain
\[
Q(x)=x\sum_{d\le\sqrt{x}}\frac{\mu(d)}{d^2}+O\Big(\sum_{d\le\sqrt{x}}1\Big).
\]
The error term is $O(\sqrt{x})$. For the main term, note that $\sum_{d\ge1}\frac{\mu(d)}{d^2}=1/\zeta(2)$ and the tail satisfies $\sum_{d>\sqrt{x}}\frac{|\mu(d)|}{d^2}\le\sum_{d>\sqrt{x}}\frac1{d^2}=O(1/\sqrt{x})$. Therefore
\[
\sum_{d\le\sqrt{x}}\frac{\mu(d)}{d^2}=\frac1{\zeta(2)}+O\Big(\frac1{\sqrt{x}}\Big),
\]
so $Q(x)=\frac{x}{\zeta(2)}+O(\sqrt{x})$.
Finally, $N(x)=Q(x)$ because $N(x)$ counts squarefree integers $\le x$.
\end{proof}

\begin{corollary}[Limit for $\alpha=0$]
\label{cor:145-alpha0}
$\displaystyle \lim_{x\to\infty} M_0(x)=\frac{1}{\zeta(2)}=\frac{6}{\pi^2}$.
\end{corollary}
\begin{proof}
Immediate from Lemma~\ref{lem:145-squarefreecount} and $M_0(x)=N(x)/x$.
\end{proof}

\paragraph{Case $\alpha=1$.}
Let $N=N(x)$. Then the gap sum telescopes:
\[
\sum_{n\le N}(s_{n+1}-s_n)=s_{N+1}-s_1.
\]
Hence
\begin{equation}
\label{eq:145-alpha1-telescope}
M_1(x)=\frac{s_{N(x)+1}-s_1}{x}.
\end{equation}
Since $s_1=1$, it suffices to prove $s_{N(x)+1}/x\to 1$.

\begin{lemma}[Relative next-squarefree gap is $o(x)$]
\label{lem:145-nextgap}
Let $t(x):=s_{N(x)+1}$ be the least squarefree integer strictly larger than $x$. Then
\[
\lim_{x\to\infty}\frac{t(x)}{x}=1.
\]
\end{lemma}
\begin{proof}
We already know from Lemma~\ref{lem:145-squarefreecount} that $N(u)/u\to 1/\zeta(2)$ as $u\to\infty$.
Fix $\varepsilon>0$. Suppose for contradiction that there exist arbitrarily large $x$ with $t(x)>(1+\varepsilon)x$.
For such an $x$, there are no squarefree integers in $(x,(1+\varepsilon)x]$, hence $N((1+\varepsilon)x)=N(x)$.
Then
\[
\frac{N((1+\varepsilon)x)}{(1+\varepsilon)x}=\frac{N(x)}{(1+\varepsilon)x}.
\]
Taking $x\to\infty$ along this subsequence gives
\[
\lim_{x\to\infty}\frac{N((1+\varepsilon)x)}{(1+\varepsilon)x}=\frac{1}{1+\varepsilon}\cdot \lim_{x\to\infty}\frac{N(x)}{x}=\frac{1}{(1+\varepsilon)\zeta(2)}.
\]
But the left-hand limit must equal $1/\zeta(2)$ because $(1+\varepsilon)x\to\infty$ as well. This contradiction shows that for all sufficiently large $x$, $t(x)\le (1+\varepsilon)x$.
Since always $t(x)>x$, we have $1< t(x)/x\le 1+\varepsilon$ for large $x$, hence $\limsup t(x)/x\le 1+\varepsilon$.
As $\varepsilon>0$ is arbitrary, $\limsup t(x)/x\le 1$, and therefore $t(x)/x\to 1$.
\end{proof}

\begin{corollary}[Limit for $\alpha=1$]
\label{cor:145-alpha1}
$\displaystyle \lim_{x\to\infty} M_1(x)=1$.
\end{corollary}
\begin{proof}
Combine \eqref{eq:145-alpha1-telescope} with Lemma~\ref{lem:145-nextgap} and $s_1=1$.
\end{proof}

\paragraph{Beyond $\alpha=1$.}
For $\alpha\ge 2$, proving existence of $\lim_{x\to\infty}M_{\alpha}(x)$ appears to require nontrivial control of higher correlations of $\mu^2(n)$ (equivalently, counts of squarefree patterns), and unconditional results currently only cover a finite range of exponents (e.g. $\alpha<3.75$ by Chan).

\subsection*{5) VERIFICATION}
\begin{itemize}
\item The definitions match the problem statement: we sum over $s_n\le x$ and use the corresponding gap to $s_{n+1}$.
\item The proofs for $\alpha=0$ and $\alpha=1$ are complete and unconditional.
\item Lemma~\ref{lem:145-nextgap} uses only the existence of the density limit for squarefrees (proved in Lemma~\ref{lem:145-squarefreecount}).
\end{itemize}

\subsection*{6) FINAL}
\textbf{UNRESOLVED.} The full statement ``the limit exists for every $\alpha\ge 0$'' remains open unconditionally.

\begin{enumerate}
\item[(i)] \textbf{Strongest fully proved partial result obtained:}
\begin{itemize}
\item The limit exists for $\alpha=0$ and equals $6/\pi^2$ (Corollary~\ref{cor:145-alpha0}).
\item The limit exists for $\alpha=1$ and equals $1$ (Corollary~\ref{cor:145-alpha1}).
\item (From the literature) the limit exists for all $0\le\alpha<3.75$ (Chan 2023) and, assuming $abc$, for all $\alpha\ge 0$ (Granville 1998).
\end{itemize}
\item[(ii)] \textbf{Exact first gap:} I do not have an unconditional argument for existence of $\lim_{x\to\infty}M_{\alpha}(x)$ for all $\alpha\ge 0$ (e.g. for $\alpha\ge 4$), beyond citing the known partial ranges.
\item[(iii)] \textbf{Top 3 next moves:}
\begin{enumerate}
\item Strengthen unconditional estimates for correlation sums of $\mu^2$ in short patterns to push the known range of $\alpha$ upward.
\item Seek a probabilistic/ergodic formulation of squarefree indicator that gives a renewal-type limit theorem for the gap process, from which all moments would follow.
\item Explore whether modern sieve technology (e.g. adaptations of methods used for prime gaps) can control the tail of the gap distribution strongly enough to imply convergence of all moment averages.
\end{enumerate}
\item[(iv)] \textbf{What a minimal counterexample would likely look like:} persistent oscillation in the frequency of long non-squarefree runs at scales comparable to $x$, causing $M_{\alpha}(x)$ to have different subsequential limits. Any such construction seems incompatible with the strong ``random-like'' heuristics for squarefrees, but no impossibility proof is known to me.
\end{enumerate}

\subsection*{7) COMPLETION ESTIMATE}
\textbf{7/10.} There is substantial partial progress (large range of $\alpha$; conditional full result), but the unconditional ``all $\alpha$'' statement still appears difficult.

