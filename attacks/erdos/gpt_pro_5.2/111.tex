% Erdos Problem #111
% Attempt for Erdos Problem #111
% Following PROMPT_STRATEGY.MD
% Tools/Constraints:
% - Web browsing available? YES (not used; only facts explicitly stated in the problem text)
% - Computation available (Python)? YES (not used)

\section*{Erd\H{o}s Problem \#111}

\subsection*{1) FORMAL RESTATEMENT}
Let $G$ be a (possibly infinite) simple undirected graph.  For each $n\in\mathbb N$, define $h_G(n)$ to be the least integer $t$ such that every induced subgraph $H$ of $G$ with $|V(H)|=n$ can be made bipartite by deleting at most $t$ edges.
Equivalently,
\[
 h_G(n):=\max\{\tau(H):\;H\subseteq G,\ |V(H)|=n\},
\]
where $\tau(H)$ is the minimum number of edges that must be deleted from $H$ to make it bipartite.

Question: determine/estimate the growth of $h_G(n)$ as $n\to\infty$.  In particular, is it true that
\[
\frac{h_G(n)}{n}\to\infty\quad\text{for every graph $G$ with }\chi(G)=\aleph_1\ ?
\]

\subsection*{2) QUICK LITERATURE/CONTEXT CHECK}
From the problem text:
\begin{itemize}
\item Any $G$ with $\chi(G)=\aleph_1$ must satisfy $h_G(n)\gg n$ since it contains $\aleph_1$ vertex-disjoint odd cycles of some fixed length.
\item There exists such a $G$ with $h_G(n)\ll n^{3/2}$.
\item Erd\H{o}s conjectured improvement to $\ll n^{1+\varepsilon}$ for all $\varepsilon>0$.
\end{itemize}
No further sources are used here.

\subsection*{3) ATTACK PLAN}
To get quantitative bounds one needs to understand $\tau(H)$ for finite graphs $H$ and then produce (or obstruct) subgraphs $H\subseteq G$ that force large $\tau(H)$.
I provide two problem-specific lemmas: an exact reformulation of $\tau(H)$ in terms of maximum cuts, and a lower bound in terms of edge-disjoint odd cycles.

\subsection*{4) WORK}
\paragraph{Lemma 111.1 (bipartite edge-deletion equals ``edges minus max-cut'').}
For any finite graph $H=(V,E)$,
\[
\tau(H)=|E(H)|-\operatorname{MaxCut}(H),
\]
where $\operatorname{MaxCut}(H)$ is the maximum number of edges crossing a bipartition $V=A\sqcup B$.

\emph{Proof.}
If we partition $V=A\sqcup B$, then the edges crossing the cut form a bipartite subgraph (all edges go between $A$ and $B$).  Keeping only those edges deletes exactly $|E|-|E(A,B)|$ edges and makes the graph bipartite.  Minimizing deletions is therefore equivalent to maximizing $|E(A,B)|$, i.e.
\[
\tau(H)=\min_{A\sqcup B} (|E|-|E(A,B)|)=|E|-\max_{A\sqcup B}|E(A,B)|.
\]
This is precisely $|E(H)|-\operatorname{MaxCut}(H)$. \qed

\paragraph{Lemma 111.2 (edge-disjoint odd cycles force deletions).}
Let $H$ be a finite graph containing $t$ edge-disjoint odd cycles.  Then $\tau(H)\ge t$.

\emph{Proof.}
Every bipartite graph contains no odd cycle.  To make $H$ bipartite, we must destroy each of the $t$ odd cycles by deleting at least one edge from each cycle.  If the cycles are edge-disjoint, the edges needed to break different cycles are distinct, so at least $t$ edges must be deleted in total. Hence $\tau(H)\ge t$. \qed

\paragraph{Corollary 111.3 (linear lower bound from vertex-disjoint odd cycles).}
If $H$ contains $t$ vertex-disjoint odd cycles, then $\tau(H)\ge t$.  In particular, if $H$ is a disjoint union of $t$ odd cycles, then $\tau(H)=t$.

\emph{Proof.}
Vertex-disjoint cycles are edge-disjoint, so Lemma~111.2 applies.  For the ``in particular'': deleting one edge per cycle makes each component a path and hence bipartite, so $\tau(H)\le t$; combined with $\tau(H)\ge t$ gives equality. \qed

\subsection*{5) VERIFICATION (FAST REALITY CHECK)}
\begin{itemize}
\item For $H=C_{2r+1}$ (single odd cycle), Lemma~111.2 gives $\tau(H)\ge1$, and indeed deleting one edge breaks the cycle, so $\tau(H)=1$.
\item Lemma~111.1: check on $H=K_3$. Here $|E|=3$ and $\operatorname{MaxCut}(K_3)=2$, so $\tau=1$, consistent.
\item For disjoint union of $t$ triangles, $|E|=3t$, maxcut is $2t$, so $\tau=t$.
\end{itemize}

\subsection*{6) FINAL}
\textbf{UNRESOLVED.}

(i) \emph{Strongest fully proved partial result obtained here.}
Exact identity $\tau(H)=|E|-\operatorname{MaxCut}(H)$ for finite $H$ (Lemma~111.1), and the lower bound $\tau(H)\ge t$ from $t$ edge-disjoint odd cycles (Lemma~111.2).

(ii) \emph{Exact first gap.}
For graphs $G$ with $\chi(G)=\aleph_1$, determine whether one must have $h_G(n)/n\to\infty$; current lemmas only show $h_G(n)\gg n$ in the presence of linearly many disjoint odd cycles.

(iii) \emph{Top 3 next moves (concrete targets).}
\begin{enumerate}
\item Prove that $\chi(G)=\aleph_1$ forces, for arbitrarily large $n$, an $n$-vertex subgraph $H$ with many edge-disjoint odd cycles (to feed Lemma~111.2).
\item Conversely, attempt to build $\aleph_1$-chromatic graphs where every $n$-vertex subgraph has maxcut very close to $|E|$, i.e. is ``almost bipartite''.
\item Translate the problem to extremal questions about maxcut in $n$-vertex subgraphs of $G$.
\end{enumerate}

(iv) \emph{Minimal counterexample structure.}
A negative answer would be a graph $G$ with $\chi(G)=\aleph_1$ and a constant $C$ such that every $n$-vertex subgraph $H$ satisfies $\tau(H)\le Cn$ for all large $n$.


