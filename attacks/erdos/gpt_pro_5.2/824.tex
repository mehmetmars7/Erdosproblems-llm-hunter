
\subsection*{FORMAL RESTATEMENT}
Let $\sigma(n)=\sum_{d\mid n} d$ be the sum-of-divisors function. For $x>0$, define $h(x)$ to be the number of pairs of integers $(a,b)$ such that
\[
1\le a<b<x,\qquad \gcd(a,b)=1,\qquad \sigma(a)=\sigma(b).
\]
The question is:
\medskip
\noindent\textbf{Conjecture.} Is it true that
\[
 h(x) > x^{2-o(1)}\qquad (x\to\infty)?
\]
Equivalently: is $\log h(x) / \log x \to 2$?

\subsection*{QUICK LITERATURE/CONTEXT CHECK}
From the problem statement: Erd\H{o}s proved $\limsup h(x)/x=\infty$ and claimed a similar proof for the stronger question; Pollack--Pomerance later gave a complete proof that $h(x)/x\to\infty$. The conjecture $h(x)>x^{2-o(1)}$ is substantially stronger.

I do not invoke any external results beyond those explicitly mentioned above.

\subsection*{ATTACK PLAN}
\begin{itemize}
\item \textbf{Proof track (ambitious):} Try to build $\gg x^{2-o(1)}$ coprime pairs with equal $\sigma$ by constructing many collisions of the multiplicative function $\sigma$ among coprime arguments.
\item \textbf{Construction track (doable partially):} Produce explicit infinite families of coprime pairs $(a,b)$ with $\sigma(a)=\sigma(b)$ (even if the count is far from $x^{2-o(1)}$), and compute small $x$ to sanity-check.
\end{itemize}

\subsection*{WORK}
\paragraph{FAST REALITY CHECK (exact computation for small $x$).}
I computed $h(x)$ exactly for $x\in\{20,50,100,200,300,500,1000\}$ by brute force. The results were:
\[
 h(20)=3,\; h(50)=13,\; h(100)=37,\; h(200)=82,\; h(300)=149,\; h(500)=285,\; h(1000)=716.
\]
Sample coprime pairs with equal $\sigma$ under $x=20$ are $(6,11)$ with $\sigma=12$, $(10,17)$ with $\sigma=18$, and $(14,15)$ with $\sigma=24$.

\noindent\textbf{Lemma ($\sigma$ is multiplicative on coprime inputs).}
If $\gcd(m,n)=1$, then $\sigma(mn)=\sigma(m)\sigma(n)$.

\noindent\textbf{Proof.}
Every divisor $d$ of $mn$ can be written uniquely as $d=ab$ with $a\mid m$ and $b\mid n$ because $\gcd(m,n)=1$ (uniqueness follows from unique factorization and disjoint prime supports). Therefore
\[
\sigma(mn)=\sum_{d\mid mn} d = \sum_{a\mid m}\sum_{b\mid n} ab = \Bigl(\sum_{a\mid m} a\Bigr)\Bigl(\sum_{b\mid n} b\Bigr)=\sigma(m)\sigma(n).
\]
\hfill$\square$


\noindent\textbf{Lemma (Squarefree formula).}
If $n$ is squarefree with prime factorization $n=\prod_{i=1}^r p_i$, then
\[
\sigma(n)=\prod_{i=1}^r (p_i+1).
\]

\noindent\textbf{Proof.}
For a prime $p$, the divisors of $p$ are $1,p$, hence $\sigma(p)=1+p=p+1$.
If $n$ is squarefree, the prime factors $p_i$ are distinct, hence pairwise coprime. By repeated application of the multiplicativity of $\sigma$ on coprime inputs,
\[
\sigma\Bigl(\prod_{i=1}^r p_i\Bigr)=\prod_{i=1}^r \sigma(p_i)=\prod_{i=1}^r (p_i+1).
\]
\hfill$\square$


\noindent\textbf{Lemma (A parametric collision producing coprime pairs).}
Let $p,q$ be primes and set $a:=pq$. If $r:=(p+1)(q+1)-1$ is also prime and $r\notin\{p,q\}$, then with $b:=r$ we have
\[
\gcd(a,b)=1\quad\text{and}\quad \sigma(a)=\sigma(b)=(p+1)(q+1).
\]

\noindent\textbf{Proof.}
Since $p,q$ are primes and $a=pq$ is squarefree, the squarefree formula gives
\[
\sigma(a)=(p+1)(q+1).
\]
Since $b=r$ is prime, $\sigma(b)=1+r=(p+1)(q+1)$ by definition of $r$.

For the gcd, because $b$ is prime it suffices to show $b\nmid a$. If $b=p$ then
$p=(p+1)(q+1)-1$, which rearranges to $(p+1)(q+1)=p+1$, forcing $q=0$, impossible. Similarly $b\ne q$.
Thus $b$ is a prime distinct from $p$ and $q$, so $\gcd(a,b)=1$.
\hfill$\square$


\paragraph{Explicit instances of the parametric collision lemma.}
\begin{itemize}
\item $p=2,q=3$: then $r=(3\cdot 4)-1=11$ is prime, and $(a,b)=(6,11)$ gives $\sigma(6)=\sigma(11)=12$.
\item $p=2,q=5$: then $r=(3\cdot 6)-1=17$ is prime, and $(a,b)=(10,17)$ gives $\sigma(10)=\sigma(17)=18$.
\item $p=3,q=5$: then $r=(4\cdot 6)-1=23$ is prime, and $(a,b)=(15,23)$ gives $\sigma(15)=\sigma(23)=24$.
\end{itemize}
These examples also appear among the small-$x$ computation above.

\subsection*{VERIFICATION}
\begin{itemize}
\item Lemma ($\sigma$ is multiplicative on coprime inputs): the bijection between divisors of $mn$ and pairs of divisors of $m$ and $n$ uses $\gcd(m,n)=1$ essentially; without it, multiplicativity fails.
\item Lemma (A parametric collision producing coprime pairs): the condition $r\notin\{p,q\}$ was used to guarantee $\gcd(pq,r)=1$. The proof shows this condition is automatic once $r$ is defined as $(p+1)(q+1)-1$ and $q\ge 2$; I kept it explicit to avoid missing edge cases.
\item Computation: brute force verified the stated $h(x)$ values for the listed $x$.
\end{itemize}

\subsection*{FINAL}
\textbf{UNRESOLVED.}
\begin{enumerate}
\item[(i)] \textbf{Strongest proved partial result.} I exhibited an explicit infinite family of \emph{conditional} collisions (the parametric collision lemma): whenever primes $p,q$ make $r=(p+1)(q+1)-1$ prime, we get a coprime pair $(pq,r)$ with equal $\sigma$. Unconditionally, the small-$x$ computations confirm many such pairs exist at least for modest ranges.
\item[(ii)] \textbf{First gap.} Produce an unconditional lower bound of the form $h(x)\ge x^{1+\delta}$ for some fixed $\delta>0$ using only elementary constructions, let alone the conjectured $x^{2-o(1)}$.
\item[(iii)] \textbf{Top 3 next moves.}
  \begin{enumerate}
  \item Develop systematic collision constructions using squarefree numbers: find many disjoint prime sets $A,B$ with $\prod_{p\in A}(p+1)=\prod_{q\in B}(q+1)$ and control the sizes of $\prod_{p\in A}p$ and $\prod_{q\in B}q$ to be $<x$.
  \item Compute $h(x)$ for larger $x$ (e.g. $10^5$) and classify the dominant collision mechanisms (prime vs semiprime vs higher squarefree) to guess the right asymptotic.
  \item Try to adapt additive-combinatorial ideas to the multiset $\{\log(p+1): p\le x\text{ prime}\}$ to force many multiplicative coincidences among products $\prod(p+1)$.
  \end{enumerate}
\item[(iv)] \textbf{Minimal counterexample structure (if $h(x)\not> x^{2-o(1)}$).} One would need an infinite sequence $x_j\to\infty$ such that for some fixed $\eta>0$, $h(x_j)\le x_j^{2-\eta}$. Such a failure would require that $\sigma$ is "almost injective" on coprime pairs below $x_j$; in particular, it would force strong restrictions on how often products $\prod(p+1)$ can coincide across disjoint prime sets.
\end{enumerate}


