
Let $A\subset \mathbb{R}^2$ be an infinite set for which there exists some $\epsilon>0$ such that in any subset of $A$ of size $n$ there are always at least $\epsilon n$ with no three on a line. Is it true that $A$ is the union of a finite number of sets where no three are on a line? A problem of Erd\H{o}s, Ne\v{s}et\v{r}il, and R\"{o}dl. See also [774] and [847] .

1) “FORMAL RESTATEMENT”

Fix an infinite set $A\subset\mathbb{R}^2$. Assume:

$\exists\,\epsilon>0$ such that $\forall n\in\mathbb{N}_{\ge 1}$ and $\forall S\subset A$ with $|S|=n$, there exists a subset $T\subset S$ with
\[
|T|\ge \epsilon n
\]
and $T$ has \emph{no three collinear points}, i.e. for every affine line $\ell\subset\mathbb{R}^2$ we have $|T\cap \ell|\le 2$.

Question: Does it follow that there exists an integer $m\ge 1$ and a covering
\[
A = A_1\cup\cdots\cup A_m
\]
with each $A_i\subset\mathbb{R}^2$ having no three collinear points (equivalently: a coloring of $A$ with $m$ colors such that no line contains three points of the same color)?

Conventions/edge cases:
- A “subset of size $n$” means an $n$-element subset.
- If $A$ itself has no three collinear points then the answer is “yes” with $m=1$.
- The hypothesis forces $0<\epsilon\le 1$.

2) “QUICK LITERATURE/CONTEXT CHECK”

The problem statement provides no theorems to use directly, only cross-references to other Erd\H{o}s problems. In this solution write-up I do not invoke external literature beyond what is explicitly stated in the problem text.

3) “ATTACK PLAN”

Proof-track ideas:
- View collinear triples in $A$ as edges of a $3$-uniform hypergraph $\mathcal{H}(A)$. The hypothesis says every induced subhypergraph on $n$ vertices has an independent set of size $\ge \epsilon n$. Try to deduce bounded chromatic number of $\mathcal{H}(A)$.
- Use geometry: the hypothesis forces a uniform bound on the number of points of $A$ on any line. Try to leverage bounded “line multiplicity” to obtain a finite coloring.

Disproof/construction ideas:
- Try to build an infinite set $A$ as a union of finite “blocks” $B_j$ that are far apart so that the hereditary independent-set condition holds with a fixed $\epsilon$, while the required number of colors for each $B_j$ tends to infinity.

Best current path in this write-up: extract unavoidable structural consequences (line-multiplicity bounds) and show what they do and do not give (notably an $O(\log n)$ coloring bound for finite subsets, but no uniform bound).

4) “WORK”

FAST REALITY CHECK (small $n$):
- If some 3-point subset of $A$ is collinear, then for $n=3$ the largest subset with no three collinear has size $2$, so the hypothesis forces $3\epsilon\le 2$, i.e. $\epsilon\le 2/3$.

\textbf{Lemma 846.1 (Finite union $\Rightarrow$ linear-sized general position subset).}
If $A=\bigcup_{i=1}^m A_i$ where each $A_i$ has no three collinear points, then the hypothesis holds with $\epsilon=1/m$.

\emph{Proof.}
Let $S\subset A$ be any finite subset of size $n$. Then $S=\bigcup_{i=1}^m (S\cap A_i)$. By the pigeonhole principle there exists $i$ with
\[
|S\cap A_i|\ge n/m.
\]
Let $T:=S\cap A_i$. Then $|T|\ge (1/m)n$ and $T\subset A_i$ has no three collinear points by assumption on $A_i$. \qed

\textbf{Lemma 846.2 (Uniform bound on points of $A$ per line).}
Assume the hypothesis with parameter $\epsilon>0$. Then every affine line $\ell\subset\mathbb{R}^2$ satisfies
\[
|A\cap \ell|\le \left\lfloor \frac{2}{\epsilon}\right\rfloor.
\]
In particular $A$ contains no infinite collinear subset.

\emph{Proof.}
Suppose for contradiction that some line $\ell$ contains $M$ points of $A$ with $M>2/\epsilon$. Let $S\subset A\cap\ell$ be any subset of size $n:=M$ (so $S$ consists of $M$ collinear points). Any subset $T\subset S$ with no three collinear points can contain at most $2$ points of $\ell$, hence must satisfy $|T|\le 2$. But the hypothesis applied to this $S$ demands a $T$ with $|T|\ge \epsilon n = \epsilon M>2$, a contradiction. Therefore $|A\cap\ell|\le 2/\epsilon$, and since the left side is an integer we obtain the stated floor bound. \qed

\textbf{Lemma 846.3 (Logarithmic coloring bound for finite subsets).}
Assume the hypothesis with parameter $\epsilon>0$. Then for every finite $S\subset A$ with $|S|=n\ge 1$, there exists a partition
\[
S = S_1\cup\cdots\cup S_t
\]
with each $S_j$ having no three collinear points and
\[
 t \le 1+\left\lceil \frac{\log n}{\log(1/(1-\epsilon))}\right\rceil.
\]

\emph{Proof.}
Define $S^{(0)}:=S$. If $S^{(j)}$ is nonempty with $|S^{(j)}|=n_j$, apply the hypothesis to $S^{(j)}$ to obtain a subset $S_{j+1}\subset S^{(j)}$ with $|S_{j+1}|\ge \epsilon n_j$ and no three collinear points. Remove it: set $S^{(j+1)}:=S^{(j)}\setminus S_{j+1}$. Then
\[
|S^{(j+1)}| = n_j-|S_{j+1}| \le n_j-\epsilon n_j = (1-\epsilon)n_j.
\]
Iterating gives $|S^{(t)}|\le (1-\epsilon)^t n$. Choose $t$ minimal such that $(1-\epsilon)^t n<1$. Then $S^{(t)}$ must be empty (it has integer size), so $S$ is the disjoint union of $S_1,\dots,S_t$ and each $S_j$ has no three collinear points.

To bound $t$, the inequality $(1-\epsilon)^t n<1$ is equivalent to
\[
t> \frac{\log n}{\log(1/(1-\epsilon))}.
\]
Thus one may take
\[
 t = 1+\left\lceil \frac{\log n}{\log(1/(1-\epsilon))}\right\rceil.
\]
\qed

5) “VERIFICATION”

- Quantifiers: Lemma 846.2 uses the hypothesis for a subset $S$ contained in a single line; this is allowed because $S\subset A$ is arbitrary.
- Boundary cases: if $n=1$ then Lemma 846.3 gives $t\le 1$ and the partition is trivial.
- Lemma 846.3 yields only $t=O_\epsilon(\log n)$, which grows with $n$ and therefore does not directly imply a uniform finite $m$ for all of $A$.

6) FINAL

**UNRESOLVED**
(i) Strongest fully proved partial result: Under the hypothesis, every line meets $A$ in at most $\lfloor 2/\epsilon\rfloor$ points (Lemma 846.2), and every finite $n$-point subset of $A$ can be partitioned into $O_\epsilon(\log n)$ sets with no three collinear points (Lemma 846.3). Also, any finite union of no-three-collinear sets satisfies the hypothesis with $\epsilon=1/m$ (Lemma 846.1).
(ii) First gap (crisp): Prove (or disprove) that the hypothesis forces a partition of the entire infinite $A$ into at most $m=m(\epsilon)$ subsets with no three collinear points (equivalently: bounded chromatic number of the collinearity $3$-uniform hypergraph), i.e. upgrade the $O(\log n)$ bound of Lemma 846.3 to an $O(1)$ bound independent of $n$.
(iii) Top 3 next moves:
  1. Try to construct finite point sets $S\subset\mathbb{R}^2$ for which every subset contains a general-position subset of size at least $\epsilon$ times its size, but the minimum number of colors needed to avoid monochromatic collinear triples tends to infinity; then separate such blocks far apart to build an infinite counterexample.
  2. Attempt a geometric structure theorem: show that bounded line multiplicity together with the hereditary linear-independent-set condition implies bounded chromatic number for the collinearity hypergraph.
  3. Compute/search: for small finite configurations with bounded points per line, brute-force the maximum ratio $\alpha(S)/|S|$ (largest general-position subset) versus the chromatic number (minimum colors to avoid monochromatic collinear triples) to see whether a counterexample pattern emerges.
(iv) Minimal counterexample structure: An infinite $A$ with a uniform bound on $|A\cap\ell|$ for every line, built from finite “blocks” $B_j$ whose internal collinearity constraints force chromatic number $\chi(B_j)\to\infty$, while still ensuring that every subset of $B_j$ has an independent set of size $\ge \epsilon|\cdot|$.


