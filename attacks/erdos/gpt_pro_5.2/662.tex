% Erdos Problem #662
% URL: https://www.erdosproblems.com/662

Consider the triangular lattice with minimal distance between two points $1$. Denote by $f(t)$ the number of distances from any points $\leq t$. For example $f(1)=6$, $f(\sqrt{3})=12$, and $f(3)=18$. Let $x_1,\ldots,x_n\in \mathbb{R}^2$ be such that $d(x_i,x_j)\geq 1$ for all $i\neq j$. Is it true that, provided $n$ is sufficiently large depending on $t$, the number of distances $d(x_i,x_j)\leq t$ is less than or equal to $f(t)$ with equality perhaps only for the triangular lattice? In particular, is it true that the number of distances $\leq \sqrt{3}-\epsilon$ is less than $1$? A problem of Erd\H{o}s, Lov\'{a}sz, and Vesztergombi. This is essentially verbatim the problem description in \cite{Er97e}, but this does not make sense as written; there must be at least one typo. Suggestions about what this problem intends are welcome. Erd\H{o}s also goes on to write 'Perhaps the following stronger conjecture holds: Let $t_1<t_2<\cdots$ be the set of distances occurring in the triangular lattice. $t_1=1$ $t_2=\sqrt{3}$ $t_3=3$ $t_4=5$ etc. Is it true that there is an $\epsilon_n$ so that for every set $y_1,\ldots,$ with $d(y_i,y_j)\geq 1$ the number of distances $d(y_i,y_j)<t_n$ is less than $f(t_n)$?' Again, this is nonsense interpreted literally; I am not sure what Erd\H{o}s intended. References [Er97e] Erd\H{o}s, Paul, Some of my favourite unsolved problems . Math. Japon. (1997), 527-537.

% Solution to Erdos Problem #662

1) FORMAL RESTATEMENT
The problem statement as written is inconsistent/ambiguous, so I separate a literal reading from a minimal corrected reading.

(Literal text.)
Take the triangular lattice scaled so that its minimal interpoint distance is $1$. Define $f(t)$ to be
``the number of distances from any points $\le t$'', with examples $f(1)=6$, $f(\sqrt3)=12$, and $f(3)=18$.
Given $x_1,\ldots,x_n\in\mathbb{R}^2$ with $d(x_i,x_j)\ge 1$ for $i\neq j$, ask whether (for $n$ large depending on $t$) the
``number of distances $d(x_i,x_j)\le t$'' is at most $f(t)$, with equality perhaps only for the triangular lattice.

Ambiguity 1: ``number of distances $d(x_i,x_j)\le t$'' could mean
(a) the number of unordered pairs $\{i,j\}$ with $d(x_i,x_j)\le t$, or
(b) the number of \emph{distinct distance values} $\le t$ attained by the set, or
(c) for a \emph{fixed} $i$, the number of indices $j\neq i$ with $d(x_i,x_j)\le t$ (a local neighbour count).
If interpreted as (a) or (b), the numerical examples for $f(t)$ do not match standard meanings.

Ambiguity 2 / typo: the examples are consistent with $f(t)$ meaning a local neighbour count in the triangular lattice,
but then $f(3)=18$ is incorrect; it should read $f(2)=18$ (see Lemma 1 below).

(Minimal corrected statement.)
Define $f(t)$ to be the number of lattice points of the triangular lattice within Euclidean distance $\le t$ of a fixed lattice point, excluding the point itself.
Given a finite set $\{x_1,\ldots,x_n\}\subset\mathbb{R}^2$ with pairwise distances $\ge 1$, define the local count
$N_t(i):=|\{j\neq i: |x_i-x_j|\le t\}|$.
Conjecture (plausible intent): for each fixed $t$, and for all sufficiently large $n$ (depending on $t$), one has
$\max_i N_t(i)\le f(t)$, with equality only for subsets of the triangular lattice (up to rigid motion).

2) QUICK LITERATURE/CONTEXT CHECK
I do not assume any external circle-packing results beyond what I prove below. The statement itself notes that it does not make sense as written.

3) ATTACK PLAN
First, disprove any literal interpretation that would bound a \emph{global} count of pairs by the constant $f(t)$.
Second, adopt the minimal corrected ``local neighbour count'' formulation and prove whatever rigorous bounds are accessible (e.g. $t=1$ kissing number).

4) WORK

Lemma 1 (what $f(t)$ equals for small $t$ in the triangular lattice, and locating the typo).
Let $\Lambda$ be the triangular lattice generated by $u=(1,0)$ and $v=(\tfrac12,\tfrac{\sqrt3}{2})$, so the minimal nonzero distance in $\Lambda$ is $1$.
For $t\ge 0$, let
$f(t):=|\{p\in \Lambda\setminus\{0\}: |p|\le t\}|$. Then
$f(1)=6$, $f(\sqrt3)=12$, $f(2)=18$, and $f(3)=36$.

Proof.
Every lattice point has the form $p=a u + b v$ with $a,b\in\mathbb{Z}$. A direct computation gives
$|p|^2 = a^2 + ab + b^2$.
Thus $|p|\le t$ is equivalent to $a^2+ab+b^2\le t^2$.
For $t=1$, the only integer solutions with $(a,b)\neq (0,0)$ are the six unit vectors $(\pm1,0),(0,\pm1),(\pm1,\mp1)$, giving $f(1)=6$.
For $t=\sqrt3$, one additionally gets the six solutions with $a^2+ab+b^2=3$, namely permutations/signs of $(1,1)$, totaling $12$.
For $t=2$, one further includes the six solutions with $a^2+ab+b^2=4$ (e.g. $(\pm2,0),(0,\pm2),(\pm2,\mp2)$), giving $18$.
For $t=3$, the additional solutions have $a^2+ab+b^2\in\{7,9\}$, and an explicit enumeration yields $36$ total points within radius $3$.
(I verified these counts by exhaustive enumeration of integer pairs $(a,b)$ in a suitable bounding box.) $\square$

Lemma 2 (kissing number $6$ at radius $t=1$).
Let $S=\{x_0,x_1,\ldots,x_m\}\subset\mathbb{R}^2$ satisfy $|x_i-x_j|\ge 1$ for all $i\neq j$.
If additionally $|x_i-x_0|\le 1$ for all $i=1,\ldots,m$, then $m\le 6$.

Proof.
Fix $i\neq j$ in $\{1,\ldots,m\}$ and let $\theta_{ij}$ be the angle at $x_0$ between the segments $x_0x_i$ and $x_0x_j$.
By the law of cosines,
$|x_i-x_j|^2 = |x_i-x_0|^2 + |x_j-x_0|^2 -2|x_i-x_0||x_j-x_0|\cos\theta_{ij}$.
Using $|x_i-x_j|\ge 1$ and $|x_i-x_0|\le 1$, $|x_j-x_0|\le 1$, we obtain
$1 \le |x_i-x_j|^2 \le 1^2+1^2 - 2\cdot 1\cdot 1\cdot \cos\theta_{ij} = 2-2\cos\theta_{ij}$.
Therefore $\cos\theta_{ij}\le 1/2$ and hence $\theta_{ij}\ge \pi/3=60^\circ$.
Now order the points $x_1,\ldots,x_m$ by their polar angle around $x_0$; consecutive points in this cyclic order have angular separation at least $\pi/3$.
Summing these $m$ separations around the full circle gives $2\pi \ge m(\pi/3)$, so $m\le 6$. $\square$

Counterexample 3 (disproves a literal global-pair-count interpretation).
Interpret ``the number of distances $d(x_i,x_j)\le t$'' as the number of unordered pairs $\{i,j\}$ with $|x_i-x_j|\le t$.
Then for $t=1$ the claimed bound ``$\le f(1)=6$'' is false for arbitrarily large $n$.

Verification.
Take $n$ points to be a large finite subset of the triangular lattice $\Lambda$ (scaled so minimal distance is $1$).
Then every lattice edge of length $1$ contributes a pair at distance $1\le t$, and a finite patch of $\Lambda$ has $\Theta(n)$ such edges.
In particular, for $n\ge 10$ there are certainly more than $6$ unit-distance pairs, contradicting the literal global bound by $f(1)=6$. $\square$

FAST REALITY CHECK (values of $f(t)$).
Using the coordinate formula $|a u+b v|^2=a^2+ab+b^2$ and enumerating integer pairs $(a,b)$, I obtained:
- $f(1)=6$, $f(\sqrt3)=12$, $f(2)=18$, $f(3)=36$.
Thus the example ``$f(3)=18$'' in the problem text is inconsistent with the natural local-neighbour definition and appears to be a typo for ``$f(2)=18$''.

5) VERIFICATION
- Lemma 1: the squared-distance identity $|a u+b v|^2=a^2+ab+b^2$ is a direct computation; the counts for $t\in\{1,\sqrt3,2,3\}$ were verified by exhaustive enumeration.
- Lemma 2: the only inequality step is $|x_i-x_j|^2\le 2-2\cos\theta_{ij}$, which follows by bounding the radii by $1$. This is valid because the radii appear with a negative sign in the cosine term.
- Counterexample 3: any finite patch of the triangular lattice with $n$ vertices has at least $n-1$ edges, and all such edges have length $1$, giving at least $n-1>6$ pairs with distance $\le 1$ when $n\ge 8$.

6) FINAL: **UNRESOLVED**
(i) Strongest proved partial result: under the corrected local-neighbour interpretation, the $t=1$ case holds with the sharp bound $\max_i N_1(i)\le 6$ (Lemma 2). Also, the intended lattice counts satisfy $f(1)=6,f(\sqrt3)=12,f(2)=18,\ldots$ (Lemma 1).
(ii) First gap (crisp): for fixed $t>1$, prove or disprove the corrected inequality $\max_i N_t(i)\le f(t)$ for all finite $1$-separated sets in the plane (perhaps for all $n$, not just large $n$).
(iii) Top 3 next moves (concrete):
  1. For specific radii (e.g. $t=\sqrt3$ or $t=2$), treat the problem as packing circles of radius $1/2$ inside a circle of radius $t+1/2$ and attempt an exact extremal argument.
  2. Try to prove a stability statement: if a point has near-maximal $N_t(i)$, then the neighbours must lie close to the corresponding shells of the triangular lattice.
  3. Computation: for fixed small $t$ (e.g. $t=2$), perform exact search/optimization for maximum $N_t(i)$ using circle packing heuristics to see if values exceed the lattice count.
(iv) Minimal counterexample structure (for the corrected statement): a counterexample would be a 1-separated set with a point $x_0$ having more than $f(t)$ neighbours within distance $t$, i.e. an unusually dense packing of equal circles in a disk. Any such configuration would likely be highly symmetric and live near the boundary of the disk (where area-based bounds are weakest).

