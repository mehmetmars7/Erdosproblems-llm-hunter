\section*{Problem 278}

\subsection*{FORMAL RESTATEMENT}

Fix distinct positive integers $n_1<n_2<\dots<n_r$. For each $i$ choose a residue
$a_i\in \mathbb{Z}/n_i\mathbb{Z}$, and form the union of congruence classes
\[
U(\mathbf a)\;:=\;\bigcup_{i=1}^r \{m\in \mathbb{Z}:\ m\equiv a_i \pmod{n_i}\}.
\]
Because $U(\mathbf a)$ is periodic modulo $N:=\operatorname{lcm}(n_1,\dots,n_r)$, the density
$\operatorname{d}(U(\mathbf a))$ exists and equals $|U(\mathbf a)\cap[0,N-1]|/N$.

\begin{enumerate}
\item (Maximum-density question) Determine, in terms of $n_1,\dots,n_r$, the value of
\[
\max_{\mathbf a}\ \operatorname{d}(U(\mathbf a)).
\]
\item (Minimum-density question) Show that the minimum
\[
\min_{\mathbf a}\ \operatorname{d}(U(\mathbf a))
\]
is attained when all $a_i$ are ``the same'' in the sense that there exists $a\in\mathbb{Z}$
with $a_i\equiv a\pmod{n_i}$ for every $i$ (equivalently, all chosen classes contain a common integer).
\end{enumerate}

\subsection*{QUICK LITERATURE/CONTEXT CHECK}

The second question is known: Simpson proved that among all choices of residues,
the covered density is \emph{minimized} when the classes have a common intersection (e.g.\ all $a_i\equiv 0$),
and in that case inclusion--exclusion gives the explicit value. This is stated as
Lemma~2.3 in Simpson (1986) and is also quoted as Lemma~1 in Sun (1991).
A recent note by Cambie (2025) discusses Problem~278 and explains why a simple closed form for the
\emph{maximum} density should not be expected in general.

\subsection*{ATTACK PLAN}

\begin{itemize}
\item Reduce both questions to a finite combinatorial problem in the cyclic group $\mathbb{Z}/N\mathbb{Z}$.
Each congruence class $a_i \bmod n_i$ lifts to a coset of a subgroup of $\mathbb{Z}/N\mathbb{Z}$.
Thus $\operatorname{d}(U(\mathbf a))$ is exactly the normalized size of a union of cosets in $\mathbb{Z}/N\mathbb{Z}$.
\item For the \textbf{minimum} density, cite Simpson's theorem:
\[
\operatorname{d}\!\left(\bigcup_{i=1}^r a_i(n_i)\right)\ \ge\ \operatorname{d}\!\left(\bigcup_{i=1}^r 0(n_i)\right),
\]
and the right-hand side is computed by inclusion--exclusion because all intersections are nonempty.
\item For the \textbf{maximum} density, record what can be said cleanly:
trivial bounds, easy cases (e.g.\ pairwise coprime moduli), and the finite-group optimization formulation.
No general explicit formula is currently known.
\end{itemize}

\subsection*{WORK}

\paragraph{Step 1: Density as a finite union problem.}
Let $N=\operatorname{lcm}(n_1,\dots,n_r)$. For each $i$,
\[
C_i(\mathbf a)\;:=\;\{x\in\mathbb{Z}/N\mathbb{Z}:\ x\equiv a_i \pmod{n_i}\}
\]
is a coset of the subgroup $H_i:=\{x\in\mathbb{Z}/N\mathbb{Z}:\ x\equiv 0\pmod{n_i}\}$.
Then
\[
\operatorname{d}(U(\mathbf a)) \;=\; \frac{1}{N}\left|\bigcup_{i=1}^r C_i(\mathbf a)\right|.
\]
So both the maximum and minimum density questions are finite optimization problems over residue choices
in the cyclic group $\mathbb{Z}/N\mathbb{Z}$.

\paragraph{Easy cases for the maximum.}
\begin{itemize}
\item If the $n_i$ are pairwise coprime, then by the Chinese Remainder Theorem every intersection
$\bigcap_{j\in J} C_j(\mathbf a)$ is nonempty and has size $N/\operatorname{lcm}(n_j:j\in J)$ \emph{independent of $\mathbf a$}.
Hence $\operatorname{d}(U(\mathbf a))$ is independent of $\mathbf a$, and in particular
\[
\max_{\mathbf a}\operatorname{d}(U(\mathbf a))=\min_{\mathbf a}\operatorname{d}(U(\mathbf a))
=\sum_{i=1}^r \frac1{n_i}-\sum_{i<j}\frac1{\operatorname{lcm}(n_i,n_j)}+\cdots+(-1)^{r+1}\frac1{\operatorname{lcm}(n_1,\dots,n_r)}.
\]
\item Always one has the crude bounds
\[
\max_{\mathbf a}\operatorname{d}(U(\mathbf a))\le \min\!\Big(1,\ \sum_{i=1}^r \frac1{n_i}\Big)
\qquad\text{and}\qquad
\min_{\mathbf a}\operatorname{d}(U(\mathbf a))\ge 0.
\]
\end{itemize}

\paragraph{The minimum density (Simpson).}
Write $a(n):=\{m\in\mathbb{Z}:\ m\equiv a\pmod n\}$. The following is the key lemma.

\begin{lemma}[Simpson, 1986; quoted e.g.\ by Sun, 1991]
For any distinct positive integers $n_1,\dots,n_r$ and any residues $a_i\pmod{n_i}$,
\[
\operatorname{d}\!\left(\bigcup_{i=1}^r a_i(n_i)\right)\ \ge\ \operatorname{d}\!\left(\bigcup_{i=1}^r 0(n_i)\right).
\]
\end{lemma}

Assuming this lemma, the second question is immediate because when all residues are aligned
(e.g.\ $a_i\equiv 0$), every finite intersection is nonempty and has density exactly
$1/\operatorname{lcm}(n_{i_1},\dots,n_{i_s})$. Therefore inclusion--exclusion gives
\begin{align*}
\operatorname{d}\!\left(\bigcup_{i=1}^r 0(n_i)\right)
&=\sum_{i=1}^r\frac1{n_i}-\sum_{1\le i<j\le r}\frac1{\operatorname{lcm}(n_i,n_j)}+\cdots \\
&\hspace{13mm} +(-1)^{r+1}\frac1{\operatorname{lcm}(n_1,\dots,n_r)}.
\end{align*}
Moreover, the lemma shows this is the \emph{minimum} possible covered density over all residue choices.

\subsection*{VERIFICATION}

A quick sanity check with $n_1=2,n_2=4$:
\[
\operatorname{d}(0(2)\cup 0(4))=\operatorname{d}(0(2))=\tfrac12,
\]
and inclusion--exclusion yields $\frac12+\frac14-\frac14=\frac12$.
If instead one takes residues $a_1\equiv 0\pmod2$, $a_2\equiv 1\pmod4$, then
the classes are disjoint mod $4$, giving density $\frac12+\frac14=\frac34>\frac12$,
consistent with ``aligned residues minimize coverage''.

\subsection*{FINAL}

\textbf{FINAL: UNRESOLVED.}
\begin{itemize}
\item The \textbf{maximum-density} question does not currently have a known general closed-form answer
in terms of the $n_i$ alone; it is equivalent to a finite union-of-cosets maximization in $\mathbb{Z}/N\mathbb{Z}$.
\item The \textbf{minimum-density} question is solved: Simpson proved the minimum occurs when
the residue classes share a common integer (e.g.\ all $a_i\equiv 0$), and the minimum value is the
full inclusion--exclusion alternating sum shown above.
\end{itemize}

\subsection*{COMPLETION ESTIMATE}

Roughly \textbf{40\%}: the second part is complete; the first part is open/unsolved in general.

% ============================================================
