% Erdos Problem #169
% Attempt for Erdos Problem #169
% Following PROMPT_STRATEGY.MD
% Tools/Constraints:
% - Web browsing available? YES (not used beyond the problem text; no extra literature claims).
% - Computation available (Python/Sage/Mathematica)? YES (used for small truncations).

PROBLEM

Let $k\geq 3$ and $f(k)$ be the supremum of $\sum_{n\in A}\frac{1}{n}$ as $A$ ranges over all sets of positive integers which do not contain a $k$-term arithmetic progression. Estimate $f(k)$. Is
\[\lim_{k\to \infty}\frac{f(k)}{\log W(k)}=\infty\]
where $W(k)$ is the van der Waerden number?

OUTPUT

## 1) FORMAL RESTATEMENT

Fix an integer $k\ge 3$.

A set $A\subseteq\mathbb N:=\{1,2,3,\dots\}$ is **$k$-AP-free** if it contains no arithmetic progression of length $k$, i.e. there do not exist integers $a\ge 1$ and $d\ge 1$ such that
\[
\{a,a+d,a+2d,\dots,a+(k-1)d\}\subseteq A.
\]

Define
\[
f(k):=\sup\Bigl\{\sum_{n\in A}\frac{1}{n}: A\subseteq\mathbb N\text{ is $k$-AP-free}\Bigr\}\in[0,\infty].
\]

Let $W(k)$ denote the (two-color) van der Waerden number: the smallest $N$ such that every 2-coloring of $\{1,2,\dots,N\}$ contains a monochromatic $k$-term arithmetic progression.

Questions:

1. Give good lower/upper estimates for $f(k)$.
2. Decide whether
\[
\lim_{k\to\infty} \frac{f(k)}{\log W(k)} = \infty.
\]

Edge cases: for each fixed $k$, finite sets are allowed, so $f(k)\ge \sum_{n=1}^{k-1} \frac1n$ always.

---

## 2) QUICK LITERATURE/CONTEXT CHECK

I will not use or claim any external results beyond what is explicitly written in the problem text.

The problem text reports several lower bounds (Berlekamp; Gerver), record values for $f(3),f(4)$, and a statement that improving $f(k)/\log W(k)\ge 1/2$ to any constant $>1/2$ is open.

---

## 3) ATTACK PLAN

Proof-track (partial):

1. Prove unconditional, elementary lower bounds on $f(k)$ (using explicit finite $k$-AP-free sets).
2. Prove the “trivial” inequality $f(k)\ge \tfrac12\log W(k)$ from the definition of $W(k)$.
3. Do a small-$N$ computational reality check for the finite truncation problem
\[
F_k(N):=\max\Bigl\{\sum_{n\in A}\frac1n: A\subseteq\{1,\dots,N\}\text{ is $k$-AP-free}\Bigr\},
\]
noting $F_k(N)\le f(k)$.

Disproof-track for the limit question is out of reach here (it would require upper bounds or constructions beyond elementary arguments).

---

## 4) WORK

### Phase 1: FAST REALITY CHECK (finite truncations)

For small $N$ I computed exact values of $F_k(N)$ by brute force optimization.

*For $k=3$ (no 3-term APs):*
\[
\begin{array}{c|cccccccccc}
N&1&2&3&4&5&6&7&8&9&10\\\hline
F_3(N)&1&1.5&1.5&1.75&1.95&1.95&1.95&1.95&1.986111\dots&2.05
\end{array}
\]
\[
\begin{array}{c|cccccccccc}
N&11&12&13&14&15&16&17&18&19&20\\\hline
F_3(N)&2.140909\dots&2.140909\dots&2.217832\dots&2.289260\dots&2.289260\dots&2.289260\dots&2.289260\dots&2.289260\dots&2.289260\dots&2.289260\dots
\end{array}
\]

*For $k=4$ (no 4-term APs), computed for $1\le N\le 16$:* 
\[
\begin{array}{c|cccccccc}
N&1&2&3&4&5&6&7&8\\\hline
F_4(N)&1&1.5&1.833333\dots&1.833333\dots&2.033333\dots&2.2&2.2&2.325
\end{array}
\]
\[
\begin{array}{c|cccccccc}
N&9&10&11&12&13&14&15&16\\\hline
F_4(N)&2.436111\dots&2.536111\dots&2.536111\dots&2.536111\dots&2.536111\dots&2.556684\dots&2.602777\dots&2.665277\dots
\end{array}
\]

These are **lower bounds** for $f(3),f(4)$ because finite sets are allowed in the supremum definition.

---

### Lemma 4.1 (monotonicity in $k$)

For $k\ge 3$,
\[
f(k)\le f(k+1).
\]

**Proof.**
If a set $A\subseteq\mathbb N$ contains a $(k+1)$-term arithmetic progression, then it certainly contains a $k$-term arithmetic progression (take the first $k$ terms). Contrapositively, if $A$ is $k$-AP-free, then it is also $(k+1)$-AP-free.

Therefore the class of $k$-AP-free sets is contained in the class of $(k+1)$-AP-free sets. Taking suprema of the same functional $\sum_{n\in A}1/n$ over a larger class cannot decrease the value, so $f(k)\le f(k+1)$.
$\square$

---

### Lemma 4.2 (finite-set lower bound)

For every $k\ge 3$,
\[
f(k)\ge \sum_{n=1}^{k-1}\frac{1}{n}.
\]

**Proof.**
The set $A:=\{1,2,\dots,k-1\}$ has size $k-1$, hence it cannot contain any $k$-term arithmetic progression (a $k$-term AP requires $k$ distinct elements). Thus $A$ is $k$-AP-free, and so
\[
f(k)\ge \sum_{n\in A}\frac1n=\sum_{n=1}^{k-1}\frac1n.
\]
$\square$

---

### Lemma 4.3 (the “trivial” $1/2$ bound from $W(k)$)

For every $k\ge 3$,
\[
f(k)\ge \frac12\log W(k),
\]
and hence
\[
\frac{f(k)}{\log W(k)}\ge \frac12.
\]

**Proof.**
By definition of $W(k)$, there exists a 2-coloring of $\{1,2,\dots,W(k)-1\}$ with **no** monochromatic $k$-term arithmetic progression.
Let $R$ and $B$ be the red and blue color classes. Then both $R$ and $B$ are $k$-AP-free sets.

Moreover,
\[
\sum_{n\in R}\frac1n+\sum_{n\in B}\frac1n=\sum_{n=1}^{W(k)-1}\frac1n=:H_{W(k)-1}.
\]
Therefore at least one of the two sums is at least half:
\[
\max\Bigl\{\sum_{n\in R}\frac1n,\ \sum_{n\in B}\frac1n\Bigr\}\ge \frac12 H_{W(k)-1}.
\]
Since $R$ and $B$ are admissible in the definition of $f(k)$, we obtain
\[
f(k)\ge \frac12 H_{W(k)-1}.
\]

It remains to lower bound $H_{W(k)-1}$.
For every integer $m\ge 1$, we have the integral comparison
\[
\frac1i \ge \int_{i}^{i+1}\frac{dx}{x}
\quad\text{for each }i=1,2,\dots,m.
\]
Summing gives
\[
H_m=\sum_{i=1}^m\frac1i \ge \int_{1}^{m+1}\frac{dx}{x}=\log(m+1).
\]
Taking $m=W(k)-1$ yields $H_{W(k)-1}\ge \log W(k)$ and hence
\[
f(k)\ge \frac12\log W(k).
\]
$\square$

---

## 5) VERIFICATION

* Lemma 4.2 is valid because a $k$-term AP requires $k$ distinct terms, so any set of size $<k$ is automatically $k$-AP-free.
* Lemma 4.3 uses only the definition of $W(k)$ and the harmonic-number lower bound $H_m\ge\log(m+1)$, proved by a standard integral estimate.
* The computed truncation values $F_3(N),F_4(N)$ are consistent with Lemma 4.2: e.g. for $k=3$, Lemma 4.2 gives $f(3)\ge 1+1/2=1.5$, and indeed $F_3(2)=1.5$.

---

## 6) FINAL

**UNRESOLVED**

(i) **Strongest fully proved partial result obtained here.**

We proved elementary lower bounds
\[
f(k)\ge \sum_{n=1}^{k-1}\frac1n\qquad\text{and}\qquad f(k)\ge \frac12\log W(k).
\]
We also computed exact finite truncation maxima $F_3(N)$ for $N\le 20$ and $F_4(N)$ for $N\le 16$.

(ii) **Exact first gap.**

Any nontrivial **upper bound** on $f(k)$ (even finiteness of $f(3)$) is beyond what I can prove here, and I cannot improve the bound $f(k)/\log W(k)\ge 1/2$ to any constant $>1/2$.

(iii) **Top 3 next moves (concrete targets).**

1. Prove or disprove finiteness of $f(3)$ by constructing a 3-AP-free set with divergent harmonic series or proving such divergence is impossible.
2. Attempt to build structured $k$-AP-free “digit-restricted” sets (Kempner-type) with provably large harmonic sums and compare to known lower bounds.
3. Seek a mechanism linking large harmonic sum to large van der Waerden numbers to attack whether $f(k)/\log W(k)\to\infty$.

(iv) **What a minimal counterexample would likely look like.**

If $\lim_{k\to\infty} f(k)/\log W(k)=\infty$ is false, then there is a constant $C$ and infinitely many $k$ with $f(k)\le C\log W(k)$. Any such counterexample would require an upper-bounding principle that converts “no $k$-AP” into a strong restriction on harmonic sum, uniformly over all $k$.


