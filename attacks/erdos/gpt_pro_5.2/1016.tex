
\noindent\textbf{FORMAL RESTATEMENT.}
For each integer $n\ge 3$, let $h(n)$ be the minimum integer $h\ge 0$ such that there exists a (simple, undirected) graph $G$ on $n$ vertices with exactly $n+h$ edges and containing a cycle of every length $k$ for $3\le k\le n$. Such a graph is called \emph{pancyclic}. The task is to estimate $h(n)$; in particular, decide whether
\[
h(n)\ge \log_2 n + \log_* n - O(1)
\]
as $n\to\infty$, where $\log_* n$ is the iterated base-2 logarithm.

\medskip
\noindent\textbf{QUICK LITERATURE/CONTEXT CHECK.}
The problem statement reports a claimed lower/upper range of Bondy and later proofs of the displayed bounds, but does not include a full proof in the extracted text.

\medskip
\noindent\textbf{ATTACK PLAN.}
\emph{Proof track:} Convert the edge surplus $h$ into the cycle-space dimension $m-n+1$, which limits the total number of cycles by $2^{m-n+1}$. Since pancyclic graphs require at least $n-2$ distinct cycle lengths, this gives a logarithmic lower bound.
\emph{Construction track:} Build explicit sparse pancyclic graphs starting from a Hamilton cycle and adding chords that force cycles of all lengths.

\medskip
\noindent\textbf{WORK.}

\noindent\textbf{Lemma 1 (cycle-space bound on the number of simple cycles).}
Let $G$ be a connected graph with $n$ vertices and $m$ edges. Let $r:=m-n+1$ be the dimension of the cycle space of $G$ over $\mathbb{F}_2$. Then the number of distinct simple cycles in $G$ is at most $2^r-1$.

\textit{Proof.}
Consider the vector space $\mathcal{C}(G)\subseteq \mathbb{F}_2^{E(G)}$ of all \emph{even subgraphs}, i.e. subsets of edges in which every vertex has even degree (equivalently, the cycle space).
It is standard (and follows, for example, from choosing a spanning tree and the fundamental cycles) that $\dim \mathcal{C}(G)=r$ and hence $|\mathcal{C}(G)|=2^r$.
Each simple cycle $C$ determines a nonzero vector $\mathbf{1}_C\in \mathcal{C}(G)$ (its edge-incidence vector); distinct simple cycles have distinct edge sets, hence distinct vectors.
Therefore the number of simple cycles is at most the number of nonzero vectors in $\mathcal{C}(G)$, namely $2^r-1$.
\hfill$\square$

\medskip
\noindent\textbf{Corollary 1 (Bondy-type logarithmic lower bound).}
For all $n\ge 3$,
\[
h(n)\ge \left\lceil \log_2(n-1)\right\rceil -1.
\]
\textit{Proof.}
If $G$ is pancyclic on $n$ vertices with $m=n+h$ edges, then $G$ is connected and has $r=m-n+1=h+1$.
A pancyclic graph contains at least one simple cycle of each length $3,4,\dots,n$, so it has at least $n-2$ distinct simple cycles.
By Lemma 1, $n-2\le 2^{h+1}-1$, i.e. $2^{h+1}\ge n-1$, which is equivalent to the displayed bound.
\hfill$\square$

\medskip
\noindent\textbf{Lemma 2 (a simple explicit pancyclic construction).}
For every $n\ge 3$ there exists a pancyclic graph on $n$ vertices with $2n-3$ edges. Consequently, $h(n)\le n-3$.

\textit{Proof.}
Label the vertices $v_1,\dots,v_n$.
Start with the Hamilton cycle
\[
C: v_1v_2,\ v_2v_3,\dots,\ v_{n-1}v_n,\ v_nv_1,
\]
which has $n$ edges.
Add the chords $v_1v_i$ for each $i\in\{3,4,\dots,n-1\}$ (these are not edges of $C$), adding $n-3$ edges in total.
Call the resulting graph $G$; it has $n+(n-3)=2n-3$ edges.
For each integer $\ell$ with $3\le \ell\le n-1$, the edges
\[
v_1v_2,\ v_2v_3,\dots,\ v_{\ell-1}v_{\ell},\ v_{\ell}v_1
\]
form a cycle of length $\ell$ (the last edge is the added chord $v_\ell v_1$).
For $\ell=n$ the original Hamilton cycle $C$ gives a cycle of length $n$.
Thus $G$ contains cycles of all lengths $3\le \ell\le n$, i.e. $G$ is pancyclic.
\hfill$\square$

\medskip
\noindent\textbf{FAST REALITY CHECK.}
Brute force over all graphs on $n\le 7$ vertices gives the exact values:
\[
\begin{array}{c|ccccc}
 n & 3 & 4 & 5 & 6 & 7\\\hline
 \min\{e(G):G\text{ pancyclic on }n\text{ vertices}\} & 3 & 5 & 6 & 8 & 9\\
 h(n)=\min e(G)-n & 0 & 1 & 1 & 2 & 2
\end{array}
\]

\medskip
\noindent\textbf{VERIFICATION.}
Lemma 1 injects the set of simple cycles into the nonzero elements of the cycle space; this uses only that every simple cycle is an even subgraph and that distinct cycles have distinct edge sets.
Lemma 2 explicitly exhibits, for each $\ell$, a cycle and checks that its edges exist in $G$.
The brute force check was done by enumerating all labeled graphs for $n\le 7$ and testing existence of cycles of all lengths $3,\dots,n$.

\medskip
\noindent\textbf{FINAL.}
\textbf{UNRESOLVED}

(i) Strongest proved partial result: the clean universal bounds
\[\left\lceil \log_2(n-1)\right\rceil -1\le h(n)\le n-3,\]
plus exact computed values for $n\le 7$.

(ii) First gap (crisp): improve either side to a bound of order $\log n$ (or show $h(n)$ must be super-logarithmic); in particular, bridge the gap between $\Omega(\log n)$ and $O(n)$.

(iii) Top 3 next moves:
1. Strengthen Corollary 1 by proving a substantially smaller upper bound on the number of \emph{distinct cycle lengths} possible in terms of $h$ (not merely the number of cycles).
2. Improve the construction in Lemma 2 by adding $O(\log n)$ chords to a Hamilton cycle and proving it yields cycles of every length (attempting a binary-representation argument).
3. Extend brute-force computation to $n=8,9,10$ using isomorphism pruning (to guess the growth of $h(n)$).

(iv) Minimal counterexample structure: if the conjectured lower bound $h(n)\ge \log_2 n+\log_* n-O(1)$ is false, then there should exist infinitely many $n$ with a pancyclic graph having $n+\log_2 n+o(\log n)$ edges; such graphs would have cycle-space dimension only about $\log n$, so they must realize nearly all $n-2$ cycle lengths using comparatively few independent cycles.


