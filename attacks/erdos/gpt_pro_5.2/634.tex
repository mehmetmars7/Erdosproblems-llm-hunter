\section*{Problem 634. Find all values $n$ such that there exists a triangle which can be cut into $n$ congruent triangles.}

\subsection*{1) [Formal restatement]}
Determine the set
\[
\mathcal{N}:=\{n\in\mathbb{Z}_{\ge 1}: \exists\text{ a triangle }T\text{ that can be partitioned into }n\text{ congruent triangles}\}.
\]
Equivalently, decide for each $n$ whether at least one triangle admits an $n$-triangle congruent dissection.

\subsection*{2) [Quick literature/context check (if browsing is available)]}
From the prompt and widely cited tiling constructions:
\begin{itemize}
\item $n=k^2$ always lies in $\mathcal{N}$ (by the explicit construction proved in Problem 635).
\item Various additional infinite families are known to lie in $\mathcal{N}$ (e.g. $2k^2,3k^2,6k^2$, and $k^2+m^2$ via right-triangle similarity constructions).
\item Beeson (as cited in the prompt) shows that $7,11\notin\mathcal{N}$.
\item Recent work (e.g. Zhang's 2025/2026 arXiv preprint cited on the Erd\H{o}s Problems site) studies tilings where the tile has a $2\pi/3$ angle, giving further structure and conjectures, but not a complete classification of $\mathcal{N}$.
\end{itemize}

\subsection*{3) [Attack plan]}
\begin{enumerate}
\item Separate the problem into (A) ``reptile'' (tile similar to the whole) cases and (B) non-similar congruent tilings.
\item Record unconditional constructions that realize large families of $n$.
\item Record unconditional obstructions (e.g. $n=7,11$) and known necessary conditions from angle/edge arithmetic.
\item Explain why closing the gap to an exact description of $\mathcal{N}$ remains open.
\end{enumerate}

\subsection*{4) [Work]}
\paragraph{4.1. Constructions giving infinite subfamilies of $\mathcal{N}$.}
\begin{itemize}
\item \emph{Squares.} For every $k\ge 1$, there exists a triangle cut into $k^2$ congruent triangles (Problem 635, \S4.1).
\item \emph{Doubles of squares.} Using the same parallelogram grid but taking the entire parallelogram rather than half gives a dissection of a parallelogram into $2k^2$ congruent triangles; choosing a triangular half of a suitable parallelogram and refining yields $2k^2\in\mathcal{N}$.
\item \emph{Sums of two squares (right-triangle similarity).} There are standard ``reptiling'' constructions for right triangles where a larger similar right triangle is formed from a rectangular array of congruent right-triangle tiles, yielding tile counts of the form $k^2+m^2$.
\item \emph{Multiples $3k^2$ and $6k^2$.} Equilateral/60--60--60 and 30--60--90 angle structures allow triangular-lattice style constructions that yield $3k^2$ and $6k^2$ families.
\end{itemize}
(Here the point is not the uniqueness of the construction, but the existence of triangles realizing these $n$.)

\paragraph{4.2. Obstructions.}
The prompt reports that $7,11\notin\mathcal{N}$ (Beeson). Beyond these sporadic nonexistence results, general necessary conditions often come from:
\begin{itemize}
\item \emph{Angle constraints:} each angle of the big triangle is a sum of tile angles at boundary vertices; and the multiset of tile angles around interior vertices must sum to $2\pi$.
\item \emph{Edge constraints:} each side length of the big triangle is a nonnegative integer combination of the tile side lengths, respecting which tile side can lie on which boundary side.
\end{itemize}
These are powerful enough to rule out some small $n$, but (at present) do not yield a complete characterization of $\mathcal{N}$.

\subsection*{5) [Verification]}
\begin{itemize}
\item The existence of $k^2\in\mathcal{N}$ is rigorously verified by the explicit $k^2$-tiling construction in Problem 635.
\item The additional families listed are consistent with standard tiling constructions; the key unresolved step is proving that \emph{no other} $n$ occur beyond the known families and sporadic cases, or else producing new constructions for the remaining $n$.
\end{itemize}

\subsection*{6) [Final]}
\textbf{LABEL: UNRESOLVED}\\
\textbf{SUBLABEL: Partial progress (recorded constructions/obstructions) + full classification open.}\\
I can certify large infinite families inside $\mathcal{N}$ (in particular all squares) and note known sporadic nonexistence ($7,11$), but I do not obtain a complete description of all $n$ for which some triangle admits an $n$-congruent-triangle dissection.

\subsection*{7) [Completion estimate]}
About \emph{25\%} complete: the square-family is proved constructively and the known landscape is organized, but the ``if and only if'' characterization of $\mathcal{N}$ is not reached.


