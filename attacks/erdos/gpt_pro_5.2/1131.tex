% Erdos Problem #1131

\subsection*{FORMAL RESTATEMENT}
Let $n\ge 1$ and let $x_1<\dots<x_n$ be distinct points in $[-1,1]$.
Let $l_k$ be the associated Lagrange fundamental polynomials.
Define
\[
I(x_1,\dots,x_n):=\int_{-1}^1\sum_{k=1}^n |l_k(x)|^2\,dx.
\]
Since all quantities are real on $[-1,1]$, $|l_k(x)|^2=l_k(x)^2$.
Problem: determine
\[
\min_{x_1,\dots,x_n\in[-1,1]\ \text{distinct}} I(x_1,\dots,x_n)
\]
and in particular decide whether
\[
\min I = 2-(1+o(1))\frac{1}{n}\qquad (n\to\infty)
\]
holds.

\subsection*{QUICK LITERATURE/CONTEXT CHECK}
I only record context explicitly stated in the problem file.
The file states bounds
\[
2-O\Bigl(\frac{(\log n)^2}{n}\Bigr)\le \min I\le 2-\frac{2}{2n-1},
\]
with the upper bound witnessed by the roots of the integral of the Legendre polynomial.

\subsection*{ATTACK PLAN}
\textbf{Proof track ideas.}
\begin{itemize}
\item Compute $I$ exactly for small $n$ to guess minimizers; derive Euler--Lagrange-type conditions.
\item Use functional-analytic interpretation: $\sum l_k(x)^2$ is the diagonal of the reproducing kernel for the interpolation projector; attempt trace/minimax bounds.
\end{itemize}
\textbf{Disproof track ideas.}
\begin{itemize}
\item Try to find node sets for which $I$ is strictly smaller than $2-(1+o(1))/n$ or show that the conjectured asymptotic violates known lower bounds.
\end{itemize}

\subsection*{WORK}
\textbf{Lemma 1131.1 (partition of unity).}
For the Lagrange polynomials $l_k$ associated to any distinct nodes,
\[
\sum_{k=1}^n l_k(x)=1\qquad\text{for all }x.
\]

\emph{Proof.}
This is identical to Lemma 1129.1: $p(x)=\sum_k l_k(x)-1$ is a polynomial of degree $\le n-1$ vanishing at $n$ distinct points, hence identically zero. \qed

\textbf{Lemma 1131.2 (exact formula and minimizer for $n=2$).}
Let $n=2$ and let $x_1,x_2\in[-1,1]$ be distinct.
Then
\[
I(x_1,x_2)=\frac{\frac{4}{3}+2(x_1^2+x_2^2)}{(x_1-x_2)^2}
=1+\frac{(x_1+x_2)^2+\frac{4}{3}}{(x_1-x_2)^2}.
\]
In particular,
\[
\min I = \frac{4}{3},
\]
achieved uniquely (up to swapping $x_1,x_2$) by $(x_1,x_2)=(-1,1)$.

\emph{Proof.}
For $n=2$ the Lagrange polynomials are
\[
 l_1(x)=\frac{x-x_2}{x_1-x_2},\qquad l_2(x)=\frac{x-x_1}{x_2-x_1}.
\]
Since $|l_k(x)|^2=l_k(x)^2$, we have
\[
 l_1(x)^2+l_2(x)^2 = \frac{(x-x_2)^2+(x-x_1)^2}{(x_1-x_2)^2}.
\]
Integrating termwise over $[-1,1]$ gives
\[
I(x_1,x_2)=\frac{1}{(x_1-x_2)^2}\left(\int_{-1}^1 (x-x_1)^2\,dx+\int_{-1}^1 (x-x_2)^2\,dx\right).
\]
For any real $a$,
\[
\int_{-1}^1 (x-a)^2\,dx=\int_{-1}^1 (x^2-2ax+a^2)\,dx=\frac{2}{3}+2a^2,
\]
because $\int_{-1}^1 x^2 dx=2/3$, $\int_{-1}^1 x\,dx=0$, and $\int_{-1}^1 1\,dx=2$.
Hence
\[
I(x_1,x_2)=\frac{(\frac{2}{3}+2x_1^2)+(\frac{2}{3}+2x_2^2)}{(x_1-x_2)^2}
=\frac{\frac{4}{3}+2(x_1^2+x_2^2)}{(x_1-x_2)^2}.
\]
Writing $s=x_1+x_2$ and $d=x_1-x_2$, we have $x_1^2+x_2^2=(s^2+d^2)/2$, so
\[
I(x_1,x_2)=\frac{\frac{4}{3}+s^2+d^2}{d^2}=1+\frac{s^2+\frac{4}{3}}{d^2}.
\]
To minimize $I$, we want to minimize $s^2$ and maximize $d^2$ subject to $x_1,x_2\in[-1,1]$.
We always have $s^2\ge 0$, with equality iff $x_2=-x_1$.
Also $|d|=|x_1-x_2|\le 2$, with equality iff $\{x_1,x_2\}=\{-1,1\}$.
Choosing $(x_1,x_2)=(-1,1)$ gives $s=0$ and $d^2=4$, hence
\[
I(-1,1)=1+\frac{0+\frac{4}{3}}{4}=\frac{4}{3}.
\]
For any other choice, either $s^2>0$ or $d^2<4$, making $(s^2+4/3)/d^2> (4/3)/4$ and hence $I>4/3$.
This proves the claimed minimum and uniqueness up to swapping.
\qed

\textbf{Lemma 1131.3 (a crude universal lower bound).}
For any $n\ge 1$ and any distinct nodes $x_1,\dots,x_n\in[-1,1]$,
\[
I(x_1,\dots,x_n)\ge \frac{2}{n}.
\]

\emph{Proof.}
Fix $x\in[-1,1]$.
By Cauchy--Schwarz,
\[
\sum_{k=1}^n l_k(x)^2\ \ge\ \frac{\left(\sum_{k=1}^n l_k(x)\right)^2}{n}.
\]
By Lemma 1131.1, $\sum_k l_k(x)=1$.
Therefore $\sum_k l_k(x)^2\ge 1/n$ for all $x\in[-1,1]$.
Integrating over $[-1,1]$ yields $I\ge \int_{-1}^1 (1/n)\,dx = 2/n$.
\qed

\textbf{FAST REALITY CHECK (small $n$ numerics).}
Using Gauss--Legendre quadrature, I computed $I$ for a few node choices:
\begin{verbatim}
n=1 nodes [0]: I 2.0000000000000004
n=2 nodes [-1,1]: I 1.3333333333333293
n=2 nodes [-0.5,0.5]: I 2.333333333333319
n=3 nodes [-1,0,1]: I 1.59999999999999
n=3 nodes [-0.5,0,0.5]: I 7.599999999999338
\end{verbatim}
A coarse grid search with step $0.1$ over ordered triples $(x_1,x_2,x_3)$ suggested the smallest value occurs at $(-1,0,1)$ with $I\approx 1.6$ on that grid.

\subsection*{VERIFICATION}
\begin{itemize}
\item Lemma 1131.2 is a complete calculation; the optimization step uses only the bounds $s^2\ge 0$ and $|d|\le 2$ in $[-1,1]$.
\item Lemma 1131.3 is correct even though it is far from sharp for large $n$.
\item Numerical values are consistent with the exact $n=2$ minimum $4/3$ and show that bringing nodes away from endpoints can increase $I$ dramatically.
\end{itemize}

\subsection*{FINAL}
\textbf{UNRESOLVED}

(i) \textbf{Strongest proved partial result.}
The exact minimizer for $n=2$ is $\{-1,1\}$ with $\min I=4/3$ (Lemma 1131.2). Universally, $I\ge 2/n$ for all $n$ (Lemma 1131.3).

(ii) \textbf{First gap (crisp).}
Determine the minimising node sets and asymptotics of $\min I(x_1,\dots,x_n)$ as $n\to\infty$, in particular whether $\min I = 2-(1+o(1))/n$.

(iii) \textbf{Top 3 next moves.}
\begin{itemize}
\item For fixed small $n$ (e.g. $n=3,4,5$), perform rigorous (possibly computer-assisted) constrained optimization to identify minimizers and exact minimal values.
\item Develop a variational formula for $I$ in terms of the Gram matrix $G_{ij}=\int_{-1}^1 l_i(x)l_j(x)\,dx$ and attempt to lower bound $\operatorname{tr}(G)$ under interpolation constraints.
\item Try to connect $\sum l_k(x)^2$ to Christoffel functions/orthogonal polynomials to access asymptotics.
\end{itemize}

(iv) \textbf{Minimal counterexample structure.}
A counterexample to the asymptotic $2-(1+o(1))/n$ would be a family of node sets for which $I$ stays bounded away from $2$ by more than a constant multiple of $1/n$, or conversely an explicit construction driving $I$ below $2- c/n$ for some $c>1$.


