
Erdos Problem 960.

1) FORMAL RESTATEMENT.
Fix integers $r,k\ge 2$.
Let $A\subseteq\mathbb{R}^2$ be a set of $n$ points with the property that no line contains $k$ points of $A$.
A line determined by two points of $A$ is called \emph{ordinary} if it contains exactly two points of $A$.
Let $L_\mathrm{ord}(A)$ denote the number of ordinary lines determined by $A$.

Define $f_{r,k}(n)$ to be the smallest integer with the property:

If $|A|=n$, $A$ has no $k$ collinear points, and $L_\mathrm{ord}(A)\ge f_{r,k}(n)$, then there exists a subset $A'\subseteq A$ with $|A'|=r$ such that every line determined by two points of $A'$ is ordinary (in $A$).

Question: estimate $f_{r,k}(n)$. In particular, is $f_{r,k}(n)=o(n^2)$? Even $f_{r,k}(n)=O(n)$?

2) QUICK LITERATURE/CONTEXT CHECK.
The problem file notes the trivial Tur\'an-theorem upper bound
$f_{r,k}(n)\le (1-1/(r-1))n^2/2+1$ and asks for much smaller bounds using geometry and the no-$k$-collinear constraint.

3) ATTACK PLAN.
(A) Reformulate in graph terms: make a graph on $A$ whose edges are ordinary lines.
(B) Apply extremal graph theory (Tur\'an) for a baseline, then look for geometric improvements using the structure of non-ordinary lines.
(C) Check easy parameter regimes: $k=3$ (general position) and small $r$.

4) WORK.

FAST REALITY CHECK.
If $k=3$ (no three points collinear), then every line determined by two points is ordinary, so for every $r\le n$ any $r$-subset $A'\subseteq A$ works. Thus for $k=3$ one can take
$f_{r,3}(n)=0$ (for $n\ge r$).

Lemma 960.1 (Graph reformulation).
Given $A$, define a graph $G(A)$ on vertex set $A$ by joining $x\ne y$ by an edge if and only if the line through $x$ and $y$ is ordinary (contains exactly two points of $A$).
Then there exists $A'\subseteq A$ with $|A'|=r$ such that every line determined by two points of $A'$ is ordinary if and only if $G(A)$ contains a clique $K_r$.

Proof.
By definition, $A'$ has the desired property exactly when every pair of distinct points in $A'$ spans an ordinary line in $A$, i.e. every pair is an edge in $G(A)$. That is precisely the definition of a clique on $r$ vertices.
\hfill $\square$

Lemma 960.2 (Tur\'an upper bound for $f_{r,k}(n)$).
For every $r\ge 2$ and $k\ge 2$,
\[
 f_{r,k}(n)\le \left(1-\frac{1}{r-1}\right)\frac{n^2}{2}+1.
\]

Proof.
Let $A$ be any set of $n$ points and consider the graph $G(A)$ from Lemma 960.1.
If $G(A)$ has more than $\left(1-\frac{1}{r-1}\right)\frac{n^2}{2}$ edges, then by Tur\'an's theorem $G(A)$ contains a $K_r$.
By Lemma 960.1 this yields a subset $A'$ of size $r$ with all pair-lines ordinary.
Since $L_\mathrm{ord}(A)=e(G(A))$, the displayed inequality on $f_{r,k}(n)$ follows.
\hfill $\square$

Lemma 960.3 (Counting ordinary lines via rich lines).
Let $A$ be a set of $n$ points.
For each line $\ell$ determined by $A$, let $t_\ell:=|A\cap \ell|$.
Then
\[
\sum_{\ell} \binom{t_\ell}{2}=\binom{n}{2},
\]
and hence the number of ordinary lines satisfies
\[
L_\mathrm{ord}(A)=\binom{n}{2}-\sum_{\ell:\ t_\ell\ge 3} \binom{t_\ell}{2}.
\]

Proof.
Every unordered pair $\{x,y\}\subseteq A$ determines a unique line $\ell$ and contributes $1$ to $\binom{t_\ell}{2}$ for that line.
Summing $\binom{t_\ell}{2}$ over all lines therefore counts each unordered pair exactly once, giving the first identity.
Now split the sum into the lines with $t_\ell=2$ (ordinary) and those with $t_\ell\ge 3$ (non-ordinary). Each ordinary line contributes exactly $\binom{2}{2}=1$ to the total, so
$\binom{n}{2}=L_\mathrm{ord}(A)+\sum_{t_\ell\ge 3}\binom{t_\ell}{2}$.
Rearranging yields the second identity.
\hfill $\square$

5) VERIFICATION.
• Lemma 960.1 is a direct translation of the definitions.
• Lemma 960.2 uses only Tur\'an's theorem; geometry enters only through the interpretation of edges.
• Lemma 960.3 is a double counting identity; it is consistent with the case $k=3$ where all $t_\ell=2$ and $L_\mathrm{ord}(A)=\binom{n}{2}$.

6) FINAL.
UNRESOLVED.
(i) Strongest proved partial result: The problem is equivalent to forcing a $K_r$ in the ordinary-line graph $G(A)$; hence the trivial bound $f_{r,k}(n)\le (1-1/(r-1))n^2/2+1$ holds. Also $f_{r,3}(n)=0$ for $n\ge r$.
(ii) First gap: improve Tur\'an's purely graph-theoretic threshold using the geometric constraint “no $k$ collinear” (and the structure of missing edges coming from rich lines) to show $f_{r,k}(n)=o(n^2)$ or even $O(n)$.
(iii) Top 3 next moves:
  (1) Use Lemma 960.3 and the no-$k$-collinear condition to bound the number/structure of non-ordinary pairs (non-edges) and deduce that $G(A)$ must contain a large clique at much lower edge density than Tur\'an predicts.
  (2) Study extremal constructions: arrange points in many lines each with $k-1$ points and compute $L_\mathrm{ord}(A)$ and the clique number of $G(A)$.
  (3) For fixed $r,k$, try to prove a Ramsey-type statement for the complement graph built from collinear cliques of size at most $k-1$.
(iv) Minimal counterexample structure: a set $A$ with many ordinary lines (so $G(A)$ is fairly dense) but with no $K_r$ in $G(A)$; equivalently, the ordinary-line graph is $K_r$-free yet has edge count close to the proposed $f_{r,k}(n)$.
