
\subsection*{Erd\H{o}s Problem \#42}

\subsection*{FORMAL RESTATEMENT}
Fix $M\ge1$. For $N\ge1$ let $[1,N]=\{1,\dots,N\}$. A set $S\subseteq[1,N]$ is Sidon if $s_1+s_2=s_3+s_4$ with $s_i\in S$ implies $\{s_1,s_2\}=\{s_3,s_4\}$.

For a set $S\subseteq\mathbb Z$ define the difference set $S-S:=\{s-s': s,s'\in S\}$.

Question: For each fixed $M\ge1$, does there exist $N_0(M)$ such that for all $N\ge N_0(M)$ and every Sidon set $A\subseteq[1,N]$, there exists a Sidon set $B\subseteq[1,N]$ with $|B|=M$ such that
\[
(A-A)\cap(B-B)=\{0\}?
\]

\subsection*{QUICK LITERATURE/CONTEXT CHECK}
No external context is given in the problem statement beyond the definition; I do not use outside results.

\subsection*{ATTACK PLAN}
\textbf{Proof track:} Given $A$, view forbidden differences $D:=(A-A)\setminus\{0\}$ and try to greedily or probabilistically choose $B$ so that (i) $B$ is Sidon and (ii) all nonzero differences of $B$ avoid $D$.

\textbf{Disproof track:} Construct, for some fixed $M$, Sidon sets $A\subseteq[1,N]$ for arbitrarily large $N$ whose difference set $A-A$ is so large (e.g. contains all small differences up to $\approx N/(M-1)$) that any $M$-set $B\subseteq[1,N]$ must create a forbidden difference.

In this write-up I prove two basic lemmas and report exhaustive small-$N$ computations.

\subsection*{WORK}
\paragraph{Fast reality check (exhaustive search for small $N$).}
Using a brute-force search over all Sidon $A\subseteq[1,N]$ for $N\le 15$, I checked the statement for $M=2$ and $M=3$.
\begin{itemize}
\item For $M=2$: the property fails at $N=2,4,7$ with counterexamples $A=(1,2)$, $A=(1,2,4)$, and $A=(1,2,5,7)$ respectively; it holds for all Sidon $A$ for each $N\in\{3,5,6,8,9,10,11,12,13,14,15\}$.
\item For $M=3$: the property fails for every $N$ in $\{4,5,\dots,20\}$ (checked by a targeted search over Sidon sets of size up to 6), with explicit counterexamples for example:
\[
N=20:\quad A=(1,2,4,9,13,19)\ \text{is Sidon and admits no Sidon }B\subseteq[1,20]\text{ of size }3\text{ with }(A-A)\cap(B-B)=\{0\}.
\]
(These computations do not address the asymptotic question ``for all sufficiently large $N$''.)
\end{itemize}

\paragraph{Lemma 42.1 (size of a Sidon difference set).}
Let $A\subseteq[1,N]$ be Sidon with $|A|=m$. Then all ordered differences $a-a'$ with $a\ne a'$ are distinct, and hence
\[
|A-A|=m(m-1)+1.
\]

\emph{Proof.}
By Lemma 39.1 (proved above, and applicable since Sidon is the same $B_2$ condition), if $a-a'=c-c'$ with $a\ne a'$ and $c\ne c'$, then $a=c$ and $a'=c'$. Thus distinct ordered pairs $(a,a')$ with $a\ne a'$ give distinct nonzero differences.
There are exactly $m(m-1)$ ordered pairs with $a\ne a'$, and each contributes one nonzero element of $A-A$.  The remaining differences are those with $a=a'$, which all equal $0$. Therefore $A-A$ consists of $m(m-1)$ distinct nonzero elements and $0$, giving $|A-A|=m(m-1)+1$. \qed

\paragraph{Lemma 42.2 (pigeonhole bound on a small difference inside $B$).}
Let $B\subseteq[1,N]$ have size $M\ge2$ and write its elements in increasing order $b_1<b_2<\dots<b_M$. Then there exists $1\le i\le M-1$ with
\[
 b_{i+1}-b_i\ \le\ \left\lfloor\frac{N-1}{M-1}\right\rfloor.
\]
In particular, $(B-B)$ contains a nonzero difference of absolute value at most $\lfloor (N-1)/(M-1)\rfloor$.

\emph{Proof.}
The successive gaps satisfy
\[
\sum_{i=1}^{M-1}(b_{i+1}-b_i)=b_M-b_1\le N-1.
\]
If every gap were larger than $\lfloor (N-1)/(M-1)\rfloor$, then the sum of the $M-1$ gaps would exceed $N-1$, a contradiction. Thus at least one gap is at most that floor value, and this gap is a nonzero difference present in $B-B$. \qed

\subsection*{VERIFICATION}
\begin{itemize}
\item Lemma 42.1: checked that the reduction to Lemma 39.1 is valid because both use the same Sidon ($B_2$) condition.
\item Lemma 42.2: boundary case $M=2$ gives the trivial bound $b_2-b_1\le N-1$.
\item Computation: the exhaustive search for $N\le15$ is complete for $M=2$ and relies on full enumeration of Sidon $A$; for $M=3$ the search up to $N=20$ was targeted (enumerating Sidon sets $A$ up to size 6) and might miss larger-size counterexamples, but it did produce explicit counterexamples.
\end{itemize}

\subsection*{FINAL}
\textbf{UNRESOLVED}

(i) \emph{Strongest proved partial result.}
For Sidon $A$ of size $m$, the difference set has size $|A-A|=m(m-1)+1$ (Lemma 42.1).  Any $M$-set $B\subseteq[1,N]$ necessarily has some nonzero difference of size at most $\lfloor (N-1)/(M-1)\rfloor$ (Lemma 42.2).  Computations show the statement fails for small $N$ (e.g. $(M,N)=(2,7)$ and $(3,20)$).

(ii) \emph{First gap (crisp statement).}
I cannot determine whether for each fixed $M$ there exists $N_0(M)$ such that the desired $B$ always exists for all $N\ge N_0(M)$, or whether instead there exist arbitrarily large $N$ admitting counterexamples $A$.

(iii) \emph{Top 3 next moves (concrete).}
\begin{enumerate}
\item (Disproof attempt) Construct, for a given $M$, a family of Sidon sets $A\subseteq[1,N]$ whose positive differences contain $\{1,2,\dots,\lfloor (N-1)/(M-1)\rfloor\}$; then Lemma 42.2 would immediately rule out any $B$ of size $M$.
\item (Proof attempt) Treat $D=(A-A)\setminus\{0\}$ as a forbidden-difference set and try a greedy algorithm for choosing $B$; prove that for $N$ large enough (as a function of $M$) the number of forbidden choices at each step is $<N$ uniformly over all Sidon $A$.
\item Computation: extend the search for counterexamples for fixed $M=3,4$ to larger $N$ using ILP/SAT formulations to see whether counterexamples persist and to guess a structural pattern.
\end{enumerate}

(iv) \emph{What a minimal counterexample would likely look like.}
A minimal counterexample for fixed $M$ and large $N$ would be a Sidon set $A$ whose difference set is ``difference-covering'' for small gaps, forcing every $M$-subset $B\subseteq[1,N]$ to contain at least one forbidden difference by Lemma 42.2.  Computational counterexamples for $M=3$ suggest $A$ should be relatively small (size $\approx \sqrt N$) but arranged so that $A-A$ captures many small integers.

