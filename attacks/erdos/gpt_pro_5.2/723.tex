
1) FORMAL RESTATEMENT

Fix an integer $n\ge 1$. Let $P$ be a set of size
\[
|P| = n^2+n+1.
\]
A \emph{finite projective plane of order $n$} (as defined in the problem statement) is a collection $\mathcal L$ of subsets of $P$ (called \emph{lines}) such that:

(a) Every line has size $n+1$: for all $L\in\mathcal L$, $|L|=n+1$.

(b) Every unordered pair of distinct points lies on exactly one line: for all distinct $x,y\in P$, there exists a unique $L\in\mathcal L$ with $\{x,y\}\subseteq L$.

Question: If such $(P,\mathcal L)$ exists, must $n$ be a \emph{prime power} (i.e. $n=p^m$ for some prime $p$ and integer $m\ge 1$)?

Edge cases/conventions: The definition makes sense for $n=1$ and gives a non-degenerate structure. We do not consider $n=0$.

2) QUICK LITERATURE/CONTEXT CHECK

The problem statement itself records:

- Constructions exist whenever $n$ is a prime power.
- The conjecture is verified for $n\le 11$.
- It is open whether a projective plane of order $12$ exists.
- (Bruck--Ryser) If $n\equiv 1\pmod 4$ or $n\equiv 2\pmod 4$, then $n$ must be a sum of two squares (e.g. this rules out $n=6,14$).
- The case $n=10$ was ruled out by computer search.

I do not use any facts beyond what is explicitly stated above.

3) ATTACK PLAN

Proof track ideas (unlikely to close here):
- Extract algebraic structure from incidences (coordinate the plane) and force a finite field of size $n$.
- Use strong arithmetic constraints on incidence matrices (eigenvalues, $p$-ranks) to force prime-power order.

Disproof track ideas:
- Try to construct an explicit incidence structure of order $n$ for a non-prime-power $n$ (smallest candidate is $n=12$).

Given the status, I focus on rigorous internal consequences of the given axioms and small-case verification.

4) WORK

\textbf{FAST REALITY CHECK.}

I verified by explicit construction (and a short script) that projective planes of order $n=1,2,3$ exist:

- $n=1$: $P=\{1,2,3\}$ and $\mathcal L=\{\{1,2\},\{1,3\},\{2,3\}\}$.
- $n=2$: the Fano plane on $P=\{1,\dots,7\}$ with lines
  \[
  \{1,2,3\},\ \{1,4,5\},\ \{1,6,7\},\ \{2,4,6\},\ \{2,5,7\},\ \{3,4,7\},\ \{3,5,6\}.
  \]
- $n=3$: the standard plane $\mathrm{PG}(2,3)$ with $13$ points and $13$ lines of size $4$ (constructed via $1$- and $2$-dimensional subspaces of $\mathbb F_3^3$).

In each case the script checked that every pair of distinct points lies in exactly one line.

\medskip

\textbf{Lemma 1 (design parameters forced by the axioms).}
Assume $(P,\mathcal L)$ satisfies (a)--(b). Then:

(i) Every point lies on exactly $n+1$ lines.

(ii) The number of lines is $|\mathcal L|=n^2+n+1$.

\emph{Proof.}
(i) Fix a point $x\in P$. Consider the set of unordered pairs $\{x,y\}$ with $y\in P\setminus\{x\}$. There are $|P|-1 = n^2+n$ such pairs.

Each line $L\in\mathcal L$ that contains $x$ contributes exactly $|L\setminus\{x\}|=n$ pairs of the form $\{x,y\}$ with $y\in L\setminus\{x\}$.

By axiom (b), every pair $\{x,y\}$ lies on exactly one line, so the pairs $\{x,y\}$ are partitioned by the lines through $x$. Therefore,
\[
(\#\{L\in\mathcal L: x\in L\})\cdot n = n^2+n,
\]
so $\#\{L\in\mathcal L: x\in L\}=n+1$.

(ii) Count incidences $(x,L)$ with $x\in P$, $L\in\mathcal L$, $x\in L$ in two ways.

From (i), each of the $|P|=n^2+n+1$ points lies on $n+1$ lines, giving $|P|(n+1)$ incidences.

On the other hand, each of the $|\mathcal L|$ lines contains $n+1$ points, giving $|\mathcal L|(n+1)$ incidences.

Equating these gives $|\mathcal L|(n+1)=|P|(n+1)$, hence $|\mathcal L|=|P|=n^2+n+1$.
\qed

\medskip

\textbf{Lemma 2 (any two lines meet in exactly one point).}
Assume $(P,\mathcal L)$ satisfies (a)--(b). Then for any two distinct lines $L,M\in\mathcal L$ we have $|L\cap M|=1$.

\emph{Proof.}
First, $|L\cap M|\le 1$: if $L\cap M$ contained two distinct points $x\ne y$, then the pair $\{x,y\}$ would be contained in both $L$ and $M$, contradicting uniqueness in (b).

It remains to rule out $L\cap M=\emptyset$.
Assume for contradiction that $L$ and $M$ are disjoint. Choose a point $p\in L$.

For each $q\in M$, let $N_q\in\mathcal L$ be the unique line containing $\{p,q\}$ (exists and is unique by (b)).
We claim:

(1) For each $q\in M$, $N_q\cap L = \{p\}$.
Indeed, $p\in N_q\cap L$. If there were some $p'\in (N_q\cap L)\setminus\{p\}$, then the pair $\{p,p'\}$ would be contained in both $L$ and $N_q$, forcing $L=N_q$ by uniqueness in (b). But $N_q$ contains $q\in M$ while $L\cap M=\emptyset$, contradiction.

(2) For each $q\in M$, $N_q\cap M = \{q\}$.
Indeed, $q\in N_q\cap M$. If there were $q'\in (N_q\cap M)\setminus\{q\}$, then the pair $\{q,q'\}$ would be contained in both $M$ and $N_q$, contradicting uniqueness in (b).

(3) If $q_1\ne q_2$ in $M$, then $N_{q_1}\cap N_{q_2}=\{p\}$.
We already know $p\in N_{q_1}\cap N_{q_2}$. If there were another point $r\ne p$ in the intersection, then the pair $\{p,r\}$ would be contained in both $N_{q_1}$ and $N_{q_2}$, contradicting uniqueness in (b).

Now count the size of the union
\[
U := M\ \cup\ \bigcup_{q\in M} (N_q\setminus\{p\}).
\]
By (2), for each $q\in M$ the set $N_q\setminus\{p\}$ contains $q$ and $n-1$ other points (since $|N_q|=n+1$).
By (3), the sets $N_q\setminus\{p\}$ for different $q$ are disjoint.
Therefore
\[
|U| = |M| + \sum_{q\in M} |N_q\setminus\{p\}| 
= (n+1) + (n+1)\cdot n 
= n^2+n+1 = |P|.
\]
Hence $U=P$.

But by (1) we have $L\cap (N_q\setminus\{p\})=\emptyset$ for all $q\in M$, and also $L\cap M=\emptyset$ by assumption. Thus $L\cap U = \{p\}$.
So $L$ contains only one point of $U=P$, contradicting $|L|=n+1\ge 2$.

Therefore $L$ and $M$ cannot be disjoint, and since $|L\cap M|\le 1$ we must have $|L\cap M|=1$.
\qed

5) VERIFICATION

- Quantifiers: Lemmas 1--2 assume exactly the axioms given (a)--(b) and the exact ground set size $|P|=n^2+n+1$.
- Boundary case $n=1$: Lemma 1 gives $n+1=2$ lines through each point, and $|\mathcal L|=3$, which matches the explicit example. Lemma 2 holds: any two $2$-subsets of a $3$-set intersect in exactly one point.
- In Lemma 2, the disjointness of the sets $N_q\setminus\{p\}$ is checked via uniqueness of the pair $\{p,r\}$.

6) FINAL

\textbf{UNRESOLVED}

(i) Strongest proved partial result: Any structure satisfying the problem's definition is a symmetric $2$-$(n^2+n+1,\ n+1,\ 1)$ design: every point lies on exactly $n+1$ lines, the number of lines equals the number of points, and any two distinct lines intersect in exactly one point (Lemmas 1--2). Small cases $n=1,2,3$ exist and were computationally verified.

(ii) First gap (crisp): Prove or disprove: if such a $2$-$(n^2+n+1,\ n+1,\ 1)$ symmetric design exists, then $n$ is a prime power.

(iii) Top 3 next moves:
  1. For the smallest open candidate $n=12$, search for additional necessary arithmetic constraints beyond those listed (e.g. incidence-matrix rank obstructions) and check them computationally.
  2. Try to coordinatize the design: show that the algebra of "translations" or automorphisms forces a finite field of size $n$.
  3. Develop a contradiction from prime factorization of $n$ using $p$-adic invariants of the incidence matrix (e.g. constraints on its determinant/eigenvalues).

(iv) Minimal counterexample structure: A counterexample would be a symmetric block design with parameters $v=b=n^2+n+1$ and block size $k=r=n+1$, with $\lambda=1$, where $n$ is not a prime power (the smallest candidate is $n=12$, i.e. $v=157$, $k=13$).


