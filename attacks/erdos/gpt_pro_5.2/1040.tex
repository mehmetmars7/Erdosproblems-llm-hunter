% Erdos Problem #1040
% URL: https://www.erdosproblems.com/1040

Let $F\subseteq \mathbb{C}$ be a closed infinite set, and let $\mu(F)$ be the infimum of\[\lvert \{ z: \lvert f(z)\rvert < 1\}\rvert,\]as $f$ ranges over all polynomials of the shape $\prod (z-z_i)$ with $z_i\in F$. Is $\mu(F)$ determined by the transfinite diameter of $F$? In particular, is $\mu(F)=0$ whenever the transfinite diameter of $F$ is $\geq 1$? A problem of Erd\H{o}s, Herzog, and Piranian \cite{EHP58}, who show that the answer is yes if $F$ is a line segment or disc, and that if the transfinite diameter is $<1$ then $\{ z: \lvert f(z)\rvert < 1\}$ always contains a disc of radius $\gg_F 1$. Erd\H{o}s and Netanyahu \cite{ErNe73} proved that if $F$ is also bounded and connected, with transfinite diameter $0<c<1$, then $\{ z: \lvert f(z)\rvert < 1\}$ always contains a disc of radius $\gg_c 1$. The transfinite diameter of $F$, also known as the logarithmic capacity, is defined by\[\rho(F)=\lim_{n\to \infty}\sup_{z_1,\ldots,z_n\in F}\left(\prod_{i<j}\lvert z_i-z_j\rvert\right)^{1/\binom{n}{2}}.\] References [EHP58] Erd\H{o}s, P. and Herzog, F. and Piranian, G., Metric properties of polynomials . J. Analyse Math. (1958), 125-148. [ErNe73] Erd\H{o}s, P. and Netanyahu, E., A remark on polynomials and the transfinite diameter . Israel J. Math. (1973), 23--25.


% --- SOLUTION TO ERDOS PROBLEM #1040 ---

FORMAL RESTATEMENT
Let F subset C be a closed infinite set. For each n>=1 and each choice z_1,...,z_n in F, form the polynomial f(z)=\prod_{i=1}^n (z-z_i).
Let A(f) denote the planar Lebesgue measure (area) of the sublevel set
  L(f) := { z in C : |f(z)| < 1 }.
Define
  mu(F) := inf_{n>=1, z_1,...,z_n in F} A(f).
The problem asks whether mu(F) is determined by the transfinite diameter (logarithmic capacity) of F, and in particular whether mu(F)=0 whenever the transfinite diameter of F is >=1.

QUICK LITERATURE/CONTEXT CHECK
Only facts explicitly present in the problem statement are treated as established here (e.g. the definition of transfinite diameter/capacity and some affirmative cases).

ATTACK PLAN
(1) Prove basic structural reductions: when F is unbounded, one can separate two zeros far apart and force the lemniscate area to be very small.
(2) For bounded sets, relate mu(F) to how far apart one can choose points and to potential-theoretic quantities (capacity), but do not assume such results.

WORK
FAST REALITY CHECK.
For two points a,b with distance d=10, Lemma 1040.1 below gives area(L((z-a)(z-b))) <= 8*pi/d^2 = 8*pi/100 approx 0.2513.
For d=100, the same bound gives <= 0.002513.
So if F contains points at arbitrarily large distance, mu(F)=0.

Lemma 1040.1 (two-point construction gives an explicit area upper bound).
Let a,b in C with d:=|a-b|>2, and let f(z)=(z-a)(z-b).
Then
  L(f) = { z: |z-a||z-b| < 1 }
has area at most 8*pi/d^2.

Proof.
Let z be any point with |z-a||z-b|<1.
First, we claim that z must lie within distance d/2 of at least one of a or b.
Indeed, if |z-a|>=d/2 and |z-b|>=d/2 simultaneously, then |z-a||z-b| >= d^2/4 > 1 (since d>2), contradicting the defining inequality.
So either |z-a|<d/2 or |z-b|<d/2.

Assume |z-a|<d/2 (the other case is symmetric). Then by the reverse triangle inequality,
  |z-b| >= |a-b| - |z-a| > d - d/2 = d/2.
Thus the condition |z-a||z-b|<1 implies
  |z-a| < 1/|z-b| <= 2/d.
So any such z with |z-a|<d/2 must in fact lie in the smaller disc { |z-a| < 2/d }.
Similarly, any such z with |z-b|<d/2 must lie in { |z-b| < 2/d }.
Therefore
  L(f) subset { |z-a| < 2/d } union { |z-b| < 2/d }.
The right-hand side is a union of two discs of radius 2/d, so its area is at most
  2 * pi * (2/d)^2 = 8*pi/d^2.
This bounds the area of L(f) as claimed.

Corollary 1040.2 (unbounded sets have mu(F)=0).
If F is an unbounded closed infinite subset of C, then mu(F)=0.

Proof.
Since F is unbounded, for every R>0 there exist a,b in F with |a-b|>R.
For such a,b with d=|a-b|>max(R,2), Lemma 1040.1 gives a polynomial f(z)=(z-a)(z-b) with
  A(f) <= 8*pi/d^2 < 8*pi/R^2.
As R->infty, the right-hand side tends to 0, so the infimum mu(F) is 0.

VERIFICATION
Lemma 1040.1 is a direct geometric covering argument, and Corollary 1040.2 is an immediate consequence.

FINAL
UNRESOLVED
(i) Strongest proved partial result here: if F is unbounded then mu(F)=0 (Corollary 1040.2), with an explicit quantitative bound A({| (z-a)(z-b) |<1}) <= 8*pi/|a-b|^2 (Lemma 1040.1).
(ii) First gap: determine mu(F) for bounded closed infinite sets, and decide whether it is determined solely by the transfinite diameter/capacity; in particular, settle whether mu(F)=0 whenever cap(F)>=1 for bounded F.
(iii) Top 3 next moves:
  1. For compact F of capacity 1, attempt to mimic the two-point construction using many points whose mutual distances realize near-maximal Vandermonde products; quantify how this affects the local scales 1/\prod_{j!=i}|z_i-z_j| near each zero.
  2. Try to produce a lower bound on area(L(f)) in terms of capacity for special classes of sets F (e.g. Jordan curves, Cantor-type sets) to see whether capacity alone can control mu(F).
  3. Computational: for simple compact F (e.g. circle, interval, finite unions) numerically minimize area(L(f)) over choices of zeros to guess whether mu(F) can be forced to 0.
(iv) Minimal counterexample structure (if capacity>=1 but mu(F)>0): a compact set F of capacity 1 for which every polynomial with zeros in F has a lemniscate |f|<1 of uniformly positive area; likely requires strong geometric constraints preventing widely separated zeros from amplifying each other.

