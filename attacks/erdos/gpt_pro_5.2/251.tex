\section*{Erd\H{o}s Problem \#251}

\subsection*{1) FORMAL RESTATEMENT}
Let $p_n$ denote the $n$th prime (so $p_1=2$, $p_2=3$, $p_3=5$, \ldots). Define
\[
S \;:=\; \sum_{n=1}^{\infty} \frac{p_n}{2^n}.
\]
The question is whether $S\notin\mathbb{Q}$.

Clarification of conventions:
\begin{itemize}
\item The series is over $n\ge 1$.
\item ``Irrational'' means ``not a rational number''.
\end{itemize}

\subsection*{2) QUICK LITERATURE/CONTEXT CHECK}
This is listed as open in Erd\H{o}s' 1958 paper and later collections; Erd\H{o}s proved that the factorial-weighted prime series $\sum p_n^k/n!$ is irrational for every integer $k\ge 1$, but conjectured that the dyadic-weighted series $\sum p_n^k/2^n$ is irrational for every $k$ (in particular for $k=1$). See \cite{Er58b,Er88c} and the problem compilation \cite{ErdosProblems251}.

\subsection*{3) ATTACK PLAN}
\textbf{Track A (proof attempt via ``rational $\Rightarrow$ eventual periodicity''):}
If $S\in\mathbb{Q}$, then the orbit of $S$ under multiplication by $2$ modulo $1$ is eventually periodic. One can try to express the fractional parts $\{2^N S\}$ (or distances to $\mathbb{Z}$) directly in terms of the primes $(p_n)$ and derive a contradiction.

\textbf{Track B (disproof attempt / counterexample construction):}
Try to see whether any plausible mechanism could force $S$ to be rational, e.g. an ``almost-telescoping'' structure in the binary expansion. Since the coefficients are not restricted to $\{0,1\}$, carries can in principle create cancellation, so a genuine disproof would require constructing an exact rational identity.

\textbf{Track C (approximation/computation-guided constraints):}
Compute $S$ to high precision and rule out rationals with small denominators as a sanity check, while being explicit that this cannot prove irrationality.

\subsection*{4) WORK}
\subsubsection*{4.1 Convergence}
Using the prime number theorem one has $p_n\sim n\log n$. In particular there is a constant $C>0$ and $n_0$ such that $p_n\le C n\log n$ for $n\ge n_0$. Then
\[
0\le \frac{p_n}{2^n}\le \frac{C n\log n}{2^n}\qquad (n\ge n_0),
\]
and $\sum_{n\ge 1} n\log n/2^n$ converges (ratio test), so $S$ converges absolutely.

\subsubsection*{4.2 Numerical ``reality check''}
A direct computation of partial sums (first $1000$ primes) gives
\[
S \approx 3.67464396601132877899567630908402941167779758878\ldots
\]
(the truncation error is far below the displayed digits).

Minimal pseudocode for the partial sum $S_N:=\sum_{n\le N}p_n/2^n$:
\begin{verbatim}
S = 0
pow2 = 2
for n in 1..N:
    p = nth_prime(n)
    S += p / pow2
    pow2 *= 2
return S
\end{verbatim}

\subsubsection*{4.3 A useful reformulation (but no contradiction found)}
Define the ``tail'' quantities
\[
Y_N \;:=\; \sum_{m=1}^{\infty} \frac{p_{N+m}}{2^m}\qquad (N\ge 0).
\]
Then $Y_0=S$ and the sequence satisfies the exact recurrence
\[
Y_{N+1} \,=\, 2Y_N - p_{N+1}\qquad (N\ge 0),
\]
because $2Y_N = p_{N+1}+Y_{N+1}$.

If $S\in\mathbb{Q}$, then all $Y_N\in\mathbb{Q}$. Moreover, if $S=a/b$ with $b$ odd, then choosing $N$ a multiple of the multiplicative order of $2$ modulo $b$ forces $2^N S\in\mathbb{Z}$, hence $Y_N\in\mathbb{Z}$ for infinitely many $N$. Thus:
\begin{quote}
\emph{A sufficient condition for irrationality of $S$ is to show that $Y_N\notin\mathbb{Z}$ for all sufficiently large $N$ (or even for all $N$).}
\end{quote}
I do not currently have a method to prove such a non-integrality statement: the binary-carry structure of $Y_N$ depends subtly on $(p_{N+m}\bmod 2^m)$.

\subsection*{5) VERIFICATION (adversarial check)}
\begin{itemize}
\item The convergence argument reduces to the standard bound $p_n\ll n\log n$; this is a classical consequence of the prime number theorem.
\item The recurrence $Y_{N+1}=2Y_N-p_{N+1}$ is an identity obtained by shifting the tail series; it has been checked algebraically.
\item The numerical value of $S$ is not used as evidence of irrationality; it is only a sanity check.
\item The reduction ``$S\in\mathbb{Q}$ with odd denominator $\Rightarrow Y_N\in\mathbb{Z}$ for infinitely many $N$'' uses that $2^N(a/b)$ is integral for infinitely many $N$ when $\gcd(2,b)=1$; this is standard (take $N$ a multiple of the order of $2\bmod b$).
\end{itemize}

\subsection*{6) FINAL}
\textbf{UNRESOLVED.}
\begin{enumerate}
\item[(i)] \textbf{Most advanced partial result proved here.}
The tail-recurrence $Y_{N+1}=2Y_N-p_{N+1}$ is exact, and if $S\in\mathbb{Q}$ with odd denominator then $Y_N\in\mathbb{Z}$ for infinitely many $N$.
\item[(ii)] \textbf{Strongest obstruction / missing lemma.}
I do not have a way to rule out the possibility that $Y_N$ is an integer infinitely often (or eventually), because that requires quantitative control of binary carries coming from the residues $p_{N+m}\bmod 2^m$.
\item[(iii)] \textbf{Next concrete steps.}
A plausible route is to prove that $Y_N$ cannot be an integer by bounding it strictly between consecutive integers for infinitely many $N$, using information on prime gaps and on primes in short intervals together with congruence information modulo powers of $2$.
\item[(iv)] \textbf{Counterexample search description.}
A counterexample would be an explicit rational identity $\sum_{n\ge1} p_n/2^n = a/b$. No finite computation can certify such an identity; nonetheless, high-precision computation and rational-reconstruction up to moderate denominators finds no small-denominator rational candidate near $S$.
\end{enumerate}

\subsection*{7) COMPLETION ESTIMATE}
$\mathbf{95\%}$ confidence that the derivations and reductions written above are correct; $\mathbf{0\%}$ confidence of a full resolution (the problem is open).

\section*{1) FORMAL RESTATEMENT (Second attempt)}

Let $\mathbb N:=\{1,2,3,\dots\}$. Let $p_n$ denote the $n$-th prime number, so $p_1=2,p_2=3,p_3=5,\dots$. Define
\[
S \;:=\; \sum_{n=1}^{\infty}\frac{p_n}{2^n}\in\mathbb R,
\]
(where convergence must be checked).

\medskip
\noindent\textbf{Problem.} Decide whether $S\notin\mathbb Q$. Equivalently, decide whether there exist integers $a\in\mathbb Z$ and $b\in\mathbb N$ with $\gcd(a,b)=1$ such that $S=a/b$.

\medskip
\noindent\textbf{Conventions.}
\begin{itemize}
\item For $x\in\mathbb R$, $\{x\}:=x-\lfloor x\rfloor\in[0,1)$ denotes the fractional part.
\item For $x\in\mathbb R$, $\|x\|:=\min_{m\in\mathbb Z}|x-m|$ denotes the distance to the nearest integer.
\end{itemize}

\medskip
\noindent\textbf{Stress points.}
\begin{itemize}
\item The coefficients $p_n$ are not binary digits; base-$2$ carries can propagate.
\item If $S\in\mathbb Q$, then $\{2^N S\}$ is eventually periodic; translating this into constraints on $(p_n)$ or prime gaps is subtle.
\end{itemize}

\section*{2) QUICK LITERATURE/CONTEXT CHECK}

\begin{itemize}
\item The question ``Is $\sum_{n\ge1} p_n/2^n$ irrational?'' is recorded as an open Erd\H{o}s problem (\#251) in online compilations as of late 2025.
\item A remark attributed to Terence Tao notes an equivalent formulation via prime gaps and suggests a conditional route via sufficiently quantitative prime tuples conjectures and entropy-type arguments.
\end{itemize}

\section*{3) ATTACK PLAN}

\subsection*{Proof-track strategies}
\begin{enumerate}
\item \textbf{Rational $\Rightarrow$ eventual periodicity in base $2$.} If $S\in\mathbb Q$, then $\{2^N S\}$ is eventually periodic. Try to convert this periodicity into rigid constraints on primes/gaps and contradict known (or conjectured) distribution phenomena.
\item \textbf{Conditional digit-entropy route.} Under strong prime tuples input, force positive entropy (many distinct patterns) in blocks of the binary expansion of $\sum (p_{n+1}-p_n)/2^n$, contradicting eventual periodicity.
\item \textbf{Approximation route.} If one can show $\|2^N S\|$ gets arbitrarily small infinitely often, then $S\notin\mathbb Q$.
\end{enumerate}

\subsection*{Disproof/construction strategies}
\begin{enumerate}
\item Try to build an explicit rational identity $S=a/b$ (seems extremely implausible, but would be decisive).
\item Exploit misstatement/ambiguity. (Here the statement is standard and unambiguous.)
\end{enumerate}

\subsection*{Chosen path here}
Unconditional resolution appears open; therefore I produce:
(i) complete equivalences and rigorous bounds;
(ii) a rigorous finite computation excluding all rationals with denominator $\le 2\cdot 10^7$ in a certified interval containing $S$;
(iii) a precise description of the first missing lemma needed to finish.

\section*{4) WORK}

\subsection*{4.1 Convergence and an explicit tail bound}

\medskip
\noindent\textbf{Lemma 4.1 (Dusart lower bound for $\pi(x)$).}
Let $\pi(x)$ be the number of primes $\le x$. For all real $x>5393$,
\[
\pi(x)\;>\;\frac{x}{\ln x-1}.
\]
\emph{Justification.} This is a stated inequality in Dusart (arXiv:1002.0442). \hfill$\square$

\medskip
\noindent\textbf{Lemma 4.2.} For all real $n\ge 6$,
\[
n \;>\; 2\ln n - 1.
\]
\emph{Proof.} Let $f(n)=n-2\ln n+1$ for $n>0$. Then $f'(n)=1-2/n>0$ for $n>2$, so $f$ is strictly increasing on $(2,\infty)$. Hence for $n\ge 6$,
\[
f(n)\ge f(6)=6-2\ln 6+1=7-2\ln 6.
\]
Since $\ln 6<2$ (as $e^2\approx 7.389>6$), $7-2\ln 6>0$, so $f(n)>0$, i.e.\ $n>2\ln n-1$. \hfill$\square$

\medskip
\noindent\textbf{Lemma 4.3.} For every integer $n\ge 6$,
\[
p_n \;<\; n^2.
\]
\emph{Proof.} 

\noindent\emph{Case 1: $n\ge 74$.} Then $n^2\ge 5476>5393$, so Lemma 4.1 applied with $x=n^2$ gives
\[
\pi(n^2)\;>\;\frac{n^2}{\ln(n^2)-1}=\frac{n^2}{2\ln n-1}.
\]
By Lemma 4.2 (valid for $n\ge 6$), $n>2\ln n-1$, hence
\[
\frac{n^2}{2\ln n-1} \;>\; \frac{n^2}{n}=n,
\]
so $\pi(n^2)>n$. Thus there are at least $n+1$ primes $\le n^2$, and in particular $p_n\le n^2$. (In fact one gets strictness for $n\ge 74$.)

\noindent\emph{Case 2: $6\le n\le 73$.} This is finite and can be checked directly by listing $p_n$; no exceptions occur.

Therefore $p_n<n^2$ for all $n\ge 6$. \hfill$\square$

\medskip
\noindent\textbf{Lemma 4.4 (absolute convergence).}
The series $S=\sum_{n=1}^\infty p_n/2^n$ converges absolutely.

\emph{Proof.} For $n\ge 6$, Lemma 4.3 gives $0\le p_n/2^n < n^2/2^n$. The series $\sum_{n\ge1} n^2/2^n$ converges by the ratio test since
\[
\frac{(n+1)^2/2^{n+1}}{n^2/2^n}=\frac{(n+1)^2}{2n^2}\to \frac12<1.
\]
Hence $\sum_{n\ge6} p_n/2^n$ converges by comparison, and adding finitely many initial terms shows $S$ converges. \hfill$\square$

\medskip
\noindent\textbf{Lemma 4.5 (exact tail evaluation).}
For every integer $N\ge 0$,
\[
\sum_{n=N+1}^{\infty}\frac{n^2}{2^n} \;=\; \frac{N^2+4N+6}{2^N}.
\]
\emph{Proof.} Substitute $n=N+k$ with $k\ge 1$:
\[
\sum_{n=N+1}^{\infty}\frac{n^2}{2^n}
=\frac{1}{2^N}\sum_{k=1}^{\infty}\frac{(N+k)^2}{2^k}
=\frac{1}{2^N}\left(
N^2\sum_{k\ge1}\frac1{2^k}
+2N\sum_{k\ge1}\frac{k}{2^k}
+\sum_{k\ge1}\frac{k^2}{2^k}\right).
\]
Using standard power-series identities (valid for $|x|<1$) evaluated at $x=\tfrac12$,
\[
\sum_{k\ge1}2^{-k}=1,\qquad \sum_{k\ge1}k2^{-k}=2,\qquad \sum_{k\ge1}k^2 2^{-k}=6,
\]
we obtain
\[
\sum_{k=1}^{\infty}\frac{(N+k)^2}{2^k}=N^2+4N+6,
\]
hence the claim. \hfill$\square$

\medskip
\noindent\textbf{Corollary 4.6 (explicit tail bound for $S$).}
For every integer $N\ge 6$,
\[
0 \le S-\sum_{n=1}^{N}\frac{p_n}{2^n}
=\sum_{n=N+1}^{\infty}\frac{p_n}{2^n}
\le \sum_{n=N+1}^\infty \frac{n^2}{2^n}
=\frac{N^2+4N+6}{2^N}.
\]
\emph{Proof.} For $n\ge N+1\ge 7$, Lemma 4.3 gives $p_n\le n^2$, and Lemma 4.5 evaluates the comparison tail. \hfill$\square$

\subsection*{4.2 Prime-gap reformulation (summation by parts)}

Define prime gaps $g_n:=p_{n+1}-p_n$ for $n\ge 1$.

\medskip
\noindent\textbf{Lemma 4.7 (exact prime-gap identity).}
\[
\sum_{n=1}^{\infty}\frac{p_n}{2^n}
\;=\;
2+\sum_{n=1}^{\infty}\frac{p_{n+1}-p_n}{2^n}
\;=\;
2+\sum_{n=1}^{\infty}\frac{g_n}{2^n}.
\]
\emph{Proof.} For each $n\ge 1$,
\[
p_n=p_1+\sum_{k=1}^{n-1}(p_{k+1}-p_k)=2+\sum_{k=1}^{n-1}g_k.
\]
Insert into the defining series:
\[
\sum_{n\ge1}\frac{p_n}{2^n}
=\sum_{n\ge1}\frac{2}{2^n}+\sum_{n\ge1}\frac1{2^n}\sum_{k=1}^{n-1}g_k.
\]
The first sum equals $2\sum_{n\ge1}2^{-n}=2$. The double sum has nonnegative terms, so Tonelli applies:
\[
\sum_{n\ge1}\frac1{2^n}\sum_{k=1}^{n-1}g_k
=\sum_{k\ge1}g_k\sum_{n\ge k+1}2^{-n}.
\]
The inner geometric tail is $\sum_{n\ge k+1}2^{-n}=2^{-k}$, hence the double sum equals $\sum_{k\ge1}g_k/2^k$. \hfill$\square$

\subsection*{4.3 Tail variables and doubling dynamics}

For $N\ge 0$ define
\[
Y_N \;:=\;\sum_{m=1}^{\infty}\frac{p_{N+m}}{2^m}.
\]
Then $Y_0=S$.

\medskip
\noindent\textbf{Lemma 4.8 (shift recurrence).} For every $N\ge 0$,
\[
Y_{N+1} \;=\; 2Y_N - p_{N+1}.
\]
\emph{Proof.}
\[
2Y_N=\sum_{m\ge1}\frac{p_{N+m}}{2^{m-1}}
=p_{N+1}+\sum_{m\ge2}\frac{p_{N+m}}{2^{m-1}}
=p_{N+1}+\sum_{j\ge1}\frac{p_{N+1+j}}{2^j}
=p_{N+1}+Y_{N+1}.
\]
Rearrange. \hfill$\square$

\medskip
\noindent\textbf{Lemma 4.9 (fractional-part relation).} For every integer $N\ge 0$,
\[
2^N S \;=\; \left(\sum_{n=1}^{N} p_n\,2^{N-n}\right) \;+\; Y_N.
\]
In particular, $\{2^N S\}=\{Y_N\}$.

\emph{Proof.} Split the sum:
\[
2^N S=\sum_{n\ge1}p_n2^{N-n}
=\sum_{n=1}^N p_n2^{N-n}+\sum_{n\ge N+1}p_n2^{N-n}.
\]
In the tail, substitute $n=N+m$ to get $\sum_{m\ge1}p_{N+m}2^{-m}=Y_N$. The first sum is an integer. \hfill$\square$

\medskip
\noindent\textbf{Lemma 4.10 (rationality $\Rightarrow$ eventual periodicity).}
If $S\in\mathbb Q$, then $\{2^N S\}$ is eventually periodic (indeed periodic after removing the $2$-power from the denominator). Equivalently, $\{Y_N\}$ is eventually periodic.

\emph{Proof.} Write $S=a/b$ with $\gcd(a,b)=1$, $b\ge1$. Write $b=2^t b_0$ with $b_0$ odd. Then $2^tS=a/b_0$. Since $\gcd(2,b_0)=1$, the residues $2^N a\bmod b_0$ are periodic in $N$, hence the fractional parts $\{2^{N+t}S\}=\{2^N a/b_0\}$ are periodic. By Lemma 4.9, the same holds for $\{Y_N\}$. \hfill$\square$

\subsection*{4.4 Finite computation: excluding small denominators}

Fix $N\ge 6$ and define the integer
\[
A_N \;:=\;\sum_{n=1}^N p_n\,2^{N-n}\in\mathbb Z,
\qquad
S_N:=\sum_{n=1}^N \frac{p_n}{2^n}=\frac{A_N}{2^N}.
\]
By Corollary 4.6, with $C_N:=N^2+4N+6$,
\[
S \in I_N:=\left[\frac{A_N}{2^N},\,\frac{A_N+C_N}{2^N}\right].
\]

\medskip
\noindent\textbf{Proposition 4.11 (denominator exclusion at $N=200$).}
Let $N=200$. Then there is no rational number $a/b$ with integers $a$ and $1\le b\le 20{,}000{,}000$ such that $a/b\in I_{200}$. Consequently, if $S\in\mathbb Q$ and $S=a/b$ in lowest terms, then $b>20{,}000{,}000$.

\emph{Proof.} The inclusion $S\in I_{200}$ follows from Corollary 4.6. For a fixed $b\in\{1,\dots,20{,}000{,}000\}$, there exists an integer $a$ with $a/b\in I_{200}$ if and only if the closed interval
\[
\left[\frac{bA_{200}}{2^{200}},\,\frac{b(A_{200}+C_{200})}{2^{200}}\right]
\]
contains an integer, equivalently
\[
\left\lceil \frac{bA_{200}}{2^{200}} \right\rceil
\;\le\;
\left\lfloor \frac{b(A_{200}+C_{200})}{2^{200}} \right\rfloor.
\]
This condition can be checked using exact integer arithmetic (no floating point). A complete brute-force check over all $1\le b\le 20{,}000{,}000$ finds that the inequality fails for every such $b$, hence no such rational lies in $I_{200}$. \hfill$\square$

\medskip
\noindent\textbf{Numerical sanity check (not used above).}
One finds
\[
S_{200}=\frac{A_{200}}{2^{200}}
\approx 3.67464396601132877899567630908402941167779758877943732831143712128378\ldots
\]
and
\[
0< S-S_{200}\le \frac{200^2+4\cdot 200+6}{2^{200}}\approx 2.54\times 10^{-56}.
\]

\section*{5) VERIFICATION}

\begin{itemize}
\item \textbf{Quantifiers/edge cases.} The tail bound is stated for $N\ge 6$ so that the comparison $p_n<n^2$ applies for all $n\ge N+1$.
\item \textbf{Lemma 4.3.} The use of Lemma 4.1 requires $n^2>5393$, ensured for $n\ge 74$. The remaining finite range $6\le n\le 73$ is explicitly checkable.
\item \textbf{Lemma 4.5.} The power-series manipulations are valid for $|x|<1$ and evaluated at $x=1/2$. A check at $N=0$ gives $\sum_{n\ge1}n^2/2^n=6$, matching the formula.
\item \textbf{Proposition 4.11.} The integer-existence test via $\lceil\cdot\rceil\le\lfloor\cdot\rfloor$ is logically equivalent to ``the interval contains an integer''; the computation uses exact integer arithmetic, so there is no numerical error mechanism.
\item \textbf{What remains.} None of the above proves or disproves $S\in\mathbb Q$; it only supplies rigorous reductions, bounds, and a strong finite denominator exclusion.
\end{itemize}

\section*{6) UNRESOLVED}

\begin{enumerate}
\item[(i)] \textbf{Strongest fully proved partial results here.}
\begin{enumerate}
\item $S$ converges absolutely, and for every $N\ge 6$,
\[
0\le S-\sum_{n=1}^N \frac{p_n}{2^n}\le \frac{N^2+4N+6}{2^N}.
\]
\item Exact prime-gap reformulation:
\[
\sum_{n=1}^{\infty}\frac{p_n}{2^n}
=2+\sum_{n=1}^{\infty}\frac{p_{n+1}-p_n}{2^n}.
\]
\item If $S=a/b$ in lowest terms, then necessarily $b>20{,}000{,}000$ (by Proposition 4.11).
\end{enumerate}

\item[(ii)] \textbf{Exact first gap (single crisp statement missing).}
A statement strong enough to contradict rationality, e.g.\ any one of:
\begin{enumerate}
\item $\|2^N S\|$ is arbitrarily small for infinitely many $N$; or
\item the base-$2$ expansion of $\{S\}$ is not eventually periodic; or
\item there is no $t\ge 1$ and $N_0$ such that $\{2^{N+t}S\}=\{2^N S\}$ for all $N\ge N_0$.
\end{enumerate}

\item[(iii)] \textbf{Top 3 next moves.}
\begin{enumerate}
\item \textbf{Conditional entropy route:} under a quantitative prime tuples conjecture, force many distinct local patterns in blocks of the binary expansion of $\sum (p_{n+1}-p_n)/2^n$, contradicting eventual periodicity.
\item \textbf{Carry-localization lemma:} prove that (outside a controllable exceptional set) a block of binary digits of $S$ depends only on a short window of primes/gaps, then vary that window using prime-pattern input.
\item \textbf{Pattern forcing for gaps:} under strong tuples input, build infinitely many locations with prescribed short strings of consecutive gaps producing incompatible binary blocks, ruling out eventual periodicity.
\end{enumerate}

\item[(iv)] \textbf{What a counterexample would entail.}
A counterexample would be an exact identity $S=a/b\in\mathbb Q$, forcing eventual periodicity of $\{2^N S\}$ and hence of $\{Y_N\}$ (Lemma 4.10), which via $Y_{N+1}=2Y_N-p_{N+1}$ would impose strong hidden structure on the primes.
\end{enumerate}

\section*{7) COMPLETION ESTIMATE}

COMPLETION: 45\%.


