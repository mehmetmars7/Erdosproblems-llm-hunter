\section*{Problem \#875 (infinite admissible sets: growth and gaps)}

\subsection*{1. Formal restatement}

Let \(A=\{a_1<a_2<\cdots\}\subset\mathbb{N}\) be infinite.
For each \(r\ge1\), define
\[
  S_r(A) := \{a_{i_1}+\cdots+a_{i_r} : 1\le i_1<\cdots<i_r\}\subset\mathbb{N}.
\]
Assume \(S_r(A)\cap S_s(A)=\emptyset\) whenever \(r\neq s\); i.e. whenever an integer is representable
as a sum of distinct elements of \(A\), the number of summands is uniquely determined.

\noindent\textbf{Problem.}
How slowly can such a sequence \(A\) grow? Equivalently, how large can the counting function
\(A(X):=|A\cap[1,X]|\) be?
In particular, for which exponents \(c\) is it possible that
\(a_{n+1}-a_n\le n^c\) for all sufficiently large \(n\)?

\subsection*{2. Quick literature/context check (browsing available)}

The Erd\H{o}s Problems website (problem \#875) presents this as an ``infinite version'' of the finite extremal problem \#874.
For the finite problem, one has sharp asymptotics
\(|A\cap[1,N]|\le (2+o(1))\sqrt{N}\) for admissible sets inside \([1,N]\) (Deshouillers; Deshouillers--Freiman).
The problem \#875 is listed as open; in particular, the existence of an admissible infinite sequence with
\(a_{n+1}/a_n\to 1\) is described by Erd\H{o}s as ``not completely trivial''.

\subsection*{3. Attack plan}

\begin{enumerate}
\item Use the finite \([1,N]\) extremal upper bound to obtain universal growth constraints on any infinite admissible \(A\):
\(A(N)=O(\sqrt{N})\) and hence \(a_n\gg n^2\).
\item Translate \(a_n\gg n^2\) into an obstruction to small gaps \(a_{n+1}-a_n\): show \(c<1\) is impossible.
\item Record trivial constructions (e.g. powers of 2) showing existence, but with very fast growth.
\item State the open gap: whether one can achieve near-quadratic growth (and hence \(a_{n+1}/a_n\to1\)).
\end{enumerate}

\subsection*{4. Detailed work}

\subsubsection*{4.1 Universal upper bound on density and quadratic lower bound on growth}

Let \(k(N)\) denote the maximum size of an admissible subset of \([1,N]\cap\mathbb{N}\) (this is problem \#874).
The finite theory gives
\begin{equation}
  k(N) \le (2+o(1))\sqrt{N}\qquad (N\to\infty).
  \label{eq:kN}
\end{equation}
In particular, there is an absolute constant \(C\) such that \(k(N)\le C\sqrt{N}\) for all \(N\ge1\).

Now take an infinite admissible \(A\subset\mathbb{N}\) as in the problem.
Then for every \(N\), the finite set \(A\cap[1,N]\) is admissible in \([1,N]\), hence
\begin{equation}
  A(N):=|A\cap[1,N]| \le k(N) \le C\sqrt{N}.
  \label{eq:countingBound}
\end{equation}
As a consequence,
\begin{equation}
  a_n \ge c\,n^2\qquad\text{for all }n\ge1
  \label{eq:quadratic}
\end{equation}
for some absolute \(c>0\): indeed, \(n=|A\cap[1,a_n]|\le C\sqrt{a_n}\) implies \(a_n\ge (n/C)^2\).
This already shows \(a_{n+1}/a_n\to1\) is at least \emph{compatible} with known lower bounds,
though it is not known whether it can be achieved.

\subsubsection*{4.2 Consequence for gaps: \texorpdfstring{\(c<1\)}{c<1} is impossible}

\begin{proposition}
If \(A\subset\mathbb{N}\) is an infinite admissible set, then it is \emph{impossible} that
\(a_{n+1}-a_n\le n^c\) holds for all sufficiently large \(n\) with any exponent \(c<1\).
Equivalently, any eventual upper bound of the form \(a_{n+1}-a_n\le n^c\) forces \(c\ge1\).
\end{proposition}

\begin{proof}
Assume for contradiction that \(a_{n+1}-a_n\le n^c\) for all \(n\ge n_0\) with some fixed \(c<1\).
Then for \(n\ge n_0\),
\[
  a_n \le a_{n_0} + \sum_{j=n_0}^{n-1} j^c \le a_{n_0} + \int_{0}^{n} x^c\,dx = a_{n_0} + \frac{n^{c+1}}{c+1}.
\]
Thus \(a_n \ll n^{c+1}\). Invert this: for large \(X\), the number of indices with \(a_n\le X\) satisfies
\(A(X)\gg X^{1/(c+1)}\).
Since \(c<1\), we have \(1/(c+1) > 1/2\), hence \(A(X)\gg X^{1/2+\eta}\) for some \(\eta>0\).
This contradicts the universal admissible-set upper bound \eqref{eq:countingBound}, which gives
\(A(X)\ll \sqrt{X}\).
\end{proof}

So the ``gap exponent'' in the question must satisfy \(c\ge1\).
The case \(c=1\) (i.e. gaps \(O(n)\) and hence essentially quadratic growth) is the first plausible threshold.

\subsubsection*{4.3 Existence (very fast growth) and the open quantitative gap}

A trivial existence construction is to take a dissociated set (all subset sums distinct), e.g.
\(A=\{2^0,2^1,2^2,\dots\}\). Then all \(S_r\) are disjoint automatically.
However, this grows exponentially and gives very large gaps.

The open part is whether one can build an infinite admissible \(A\) with growth close to the quadratic lower bound
\eqref{eq:quadratic}, for instance with
\(A(N)\asymp \sqrt{N}\) or with \(a_{n+1}/a_n\to1\), equivalently \(a_n = n^{2+o(1)}\).

\subsection*{5. Verification (gap check, edge cases)}

\begin{itemize}
\item The deduction \eqref{eq:countingBound} from the finite bound \eqref{eq:kN} is immediate: restrictions of an admissible set are admissible.
\item The conversion of \eqref{eq:countingBound} to \eqref{eq:quadratic} uses only monotonicity: \(A(a_n)\ge n\).
\item The obstruction to \(c<1\) depends only on \eqref{eq:countingBound}; any weaker universal bound \(A(N)\ll N^{1/2}\)
would suffice.
\end{itemize}

\subsection*{6. Final}

\textbf{UNRESOLVED.}

\begin{enumerate}[label=(\roman*)]
\item \emph{Furthest point reached:}
Using the finite admissible-set bound in \([1,N]\), proved that any infinite admissible \(A\) satisfies \(A(N)\ll\sqrt{N}\), hence \(a_n\gg n^2\),
and consequently \(a_{n+1}-a_n\le n^c\) is impossible for any \(c<1\).
\item \emph{Blocking issue:}
Whether one can achieve near-quadratic growth (e.g. \(a_n=n^{2+o(1)}\), or even \(a_{n+1}/a_n\to1\), or gaps \(O(n)\)) remains open.
\item \emph{Most plausible next steps:}
Construct an infinite admissible set by an explicit inductive/greedy or probabilistic method that keeps density near the finite extremal
threshold \(\asymp\sqrt{N}\), while controlling cross-length sum collisions across all earlier stages.
One likely needs a ``local-to-global'' mechanism ensuring that admissibility is stable under controlled extensions.
\item \emph{Small experiments/checks:}
A natural computational experiment is to run a greedy algorithm (add the smallest new integer that preserves admissibility)
for the first several thousand terms, measure empirical growth, and look for a plausible asymptotic law.
This does not prove existence of a quadratic-growth sequence but can suggest a conjectured exponent.
\end{enumerate}

% ------------------------------------------------------------
\begin{thebibliography}{99}

\bibitem{Choi1974}
S. L. G. Choi,
\emph{On an extremal problem in number theory},
J. Number Theory \textbf{6} (1974), 105--111.

\bibitem{Deshouillers1995}
J.-M. Deshouillers,
\emph{On an additive problem of Erd\H{o}s and Straus, I},
Israel J. Math. \textbf{92} (1995), 29--43.

\bibitem{DSH}
J. A. Dias da Silva and Y. O. Hamidoune,
\emph{Cyclic spaces for Grassmann derivatives and additive theory},
Bull. London Math. Soc. \textbf{26} (1994), 140--146.

\bibitem{Erdos1962}
P. Erd\H{o}s,
\emph{Some remarks on number theory. III},
Mat. Lapok \textbf{13} (1962), 28--38.

\bibitem{Straus1966}
E. G. Straus,
\emph{On a problem in combinatorial number theory},
J. Math. Sci. \textbf{1} (1966), 77--80.

\end{thebibliography}

