
\noindent\textbf{FORMAL RESTATEMENT.}
Let $N\in\mathbb{N}$, and let $A\subseteq [N]:=\{1,2,\dots,N\}$.  For each integer $n$ define
\[
 r_A(n):=\bigl|\{(a,b)\in A\times A: a\le b,\ a+b=n\}\bigr|.
\]
Assume there exists \emph{at most one} integer $n_0$ such that $r_A(n_0)\ge 2$, and for all $n\ne n_0$ we have $r_A(n)\le 1$.  Let
\[
F(N):=\max\{|A|:A\subseteq [N]\text{ satisfies the above property}\}.
\]
The problem asks for estimates on $F(N)$ and in particular whether
\[
F(N)\le (1+o(1))\frac{2}{\sqrt 3}\,N^{1/2}\qquad (N\to\infty)
\]
holds.

\medskip
\noindent\textbf{QUICK LITERATURE/CONTEXT CHECK.}
The problem statement itself records a lower-bound construction (Erd\H{o}s--Freud) giving
$F(N)\ge (1+o(1))\tfrac{2}{\sqrt 3}N^{1/2}$ via a Sidon set in $[1,N/3]$ and its reflection in $N$.
No upper bound matching this constant is stated in the problem text.
(Per the project integrity constraints, I do not use or assert any additional literature beyond what is stated in the problem file.)

\medskip
\noindent\textbf{ATTACK PLAN.}
\begin{itemize}
\item Proof track: (i) verify the stated Sidon\,+\,reflection construction rigorously; (ii) obtain general upper bounds by counting distinct pair-sums; (iii) try to force ``two-block'' structure when $|A|$ is large.
\item Disproof track: attempt explicit constructions exceeding $\tfrac{2}{\sqrt 3}\sqrt N$ asymptotically, guided by small-$N$ computer search.
\end{itemize}
I succeed only at (i) and at weak upper bounds; no matching $\tfrac{2}{\sqrt 3}$ upper bound is obtained.

\medskip
\noindent\textbf{WORK.}

\medskip
\noindent\textbf{Lemma 864.1 (Sidon reflection gives a single exceptional sum).}
Fix $N$. Let $B\subseteq \{1,2,\dots,\lfloor N/3\rfloor\}$ be a Sidon set in the following (``sum'') sense:
for all $b_1\le b_2$ and $b_3\le b_4$ in $B$, if $b_1+b_2=b_3+b_4$ then $\{b_1,b_2\}=\{b_3,b_4\}$.
Define
\[
A:=B\ \cup\ \{N-b: b\in B\}\ \subseteq\ [N].
\]
Then $A$ satisfies the property of Problem 864 with the unique possible exceptional sum equal to $N$.

\smallskip
\noindent\emph{Proof.}
Write $C:=\{N-b:b\in B\}$, so $A=B\cup C$ and $B\cap C=\varnothing$ because $B\subseteq[1,N/3]$ implies $C\subseteq[2N/3,N-1]$.
Consider two representations
\[
x=a_1+a_2=a_3+a_4\qquad (a_1\le a_2,\ a_3\le a_4,\ a_i\in A).
\]
We show that if $x\ne N$ then the representations coincide.
There are three pair-types:
\begin{itemize}
\item[(1)] $B+B$: both summands in $B$;
\item[(2)] $C+C$: both summands in $C$;
\item[(3)] $B+C$: one summand in $B$, the other in $C$.
\end{itemize}
We first record where these sums can lie.
\begin{itemize}
\item If $x\in B+B$, then $x\le 2\lfloor N/3\rfloor\le 2N/3$.
\item If $x\in C+C$, then $x\ge 2\lceil 2N/3\rceil\ge 4N/3$.
\item If $x\in B+C$, write $x=b+(N-b')=N+(b-b')$ for $b,b'\in B$. Then $|b-b'|\le \lfloor N/3\rfloor-1$, hence
$x\in [N-(\lfloor N/3\rfloor-1),\ N+(\lfloor N/3\rfloor-1)]\subseteq [2N/3+1,\ 4N/3-1]$.
\end{itemize}
Therefore, the ranges of $B+B$, $B+C$ (excluding the endpoint $N$), and $C+C$ are pairwise disjoint.
Consequently, if $x\ne N$ then both representations of $x$ must be of the \emph{same} pair-type.

\emph{Case 1: $x\in B+B$.} Then $a_1,a_2,a_3,a_4\in B$ and the Sidon property of $B$ implies
$\{a_1,a_2\}=\{a_3,a_4\}$.
Since we impose $a_1\le a_2$ and $a_3\le a_4$, the representation is unique.

\emph{Case 2: $x\in C+C$.} Write $a_i=N-b_i$ with $b_i\in B$. Then
\[
(N-b_1)+(N-b_2)=(N-b_3)+(N-b_4)\iff b_1+b_2=b_3+b_4.
\]
Again, the Sidon property of $B$ forces $\{b_1,b_2\}=\{b_3,b_4\}$, hence $\{a_1,a_2\}=\{a_3,a_4\}$.

\emph{Case 3: $x\in B+C$ and $x\ne N$.}
Write $a_1=b_1\in B$, $a_2=N-b_2\in C$ and similarly $a_3=b_3\in B$, $a_4=N-b_4\in C$.
Then
\[
b_1+(N-b_2)=b_3+(N-b_4)\iff b_1-b_2=b_3-b_4.
\]
Rearranging gives $b_1+b_4=b_3+b_2$. By the Sidon property of $B$ applied to the (unordered) sums,
$\{b_1,b_4\}=\{b_3,b_2\}$.
Since $x\ne N$, we have $b_1\ne b_2$ and $b_3\ne b_4$ (otherwise the sum would be $N$).
Thus the only possibility is $b_1=b_3$ and $b_2=b_4$, so the $B+C$ representation is unique.

The only remaining collisions are at $x=N$, where every $b\in B$ contributes the representation $N=b+(N-b)$.
Hence $A$ has at most one sum with multiple representations, namely $N$.
\qed

\medskip
\noindent\textbf{Lemma 864.2 (General upper bound $F(N)\le 2\sqrt N+O(1)$).}
For every $N$ and every admissible $A\subseteq [N]$,
\[
|A|\le \frac{1+\sqrt{16N-15}}{2} \;<\; 2\sqrt N+1.
\]

\smallskip
\noindent\emph{Proof.}
Let $k:=|A|$. Consider the multiset of unordered pairs with repetition
\[
\mathcal{P}:=\{\{a,b\}: a,b\in A,\ a\le b\}.
\]
Then $|\mathcal{P}|=k(k+1)/2$. Each pair $\{a,b\}$ contributes the sum $a+b\in\{2,3,\dots,2N\}$.
Let $n_0$ be the (possible) exceptional sum with $t:=r_A(n_0)\ge 2$ representations; for all other $n$, $r_A(n)\le 1$.
Thus the number of \emph{distinct} sums arising from $\mathcal{P}$ equals $|\mathcal{P}|-(t-1)$.
Since there are only $2N-1$ possible sums in $[2,2N]$,
\[
\frac{k(k+1)}{2}-(t-1)\le 2N-1.
\]
Also $t\le k$ because for fixed $n_0$, each $a\in A$ determines at most one $b=n_0-a\in A$ with $a\le b$.
Therefore
\[
\frac{k(k+1)}{2}-(k-1)\le 2N-1\quad\Longrightarrow\quad \frac{k(k-1)}{2}+1\le 2N-1.
\]
Equivalently $k(k-1)\le 4N-4$, i.e.
$k^2-k-(4N-4)\le 0$, which solves to
$k\le \frac{1+\sqrt{16N-15}}{2}$.
\qed

\medskip
\noindent\textbf{FAST REALITY CHECK (small $N$).}
I ran an exact backtracking search for $F(N)$ for $N\le 40$.
The exact maxima found were
\[
\begin{array}{c|cccc}
N & 10 & 20 & 30 & 40\\\hline
F(N) & 5 & 8 & 9 & 11
\end{array}
\]
One extremal example for each is:
\begin{align*}
N=10:&\ \{1,2,3,7,10\},\\
N=20:&\ \{1,2,4,8,13,17,19,20\},\\
N=30:&\ \{1,2,4,8,13,18,22,24,25\},\\
N=40:&\ \{1,3,4,9,13,20,27,31,36,37,39\}.
\end{align*}
In each case there is exactly one repeated sum.

\medskip
\noindent\textbf{VERIFICATION.}
\begin{itemize}
\item Lemma 864.1: the only point where a collision could arise between different pair-types is overlap of the sum ranges; the interval check shows disjointness except at $N$.
\item Lemma 864.2: the bound uses only (i) counting pairs, (ii) that all sums lie in $[2,2N]$, and (iii) $t\le k$.
\item Computation: verified by exhaustive backtracking (exact) for $N\le 40$.
\end{itemize}

\medskip
\noindent\textbf{FINAL.} \textbf{UNRESOLVED}

(i) \emph{Strongest proved partial result.}
There is an explicit construction (Lemma 864.1) showing $F(N)\ge 2|B|$ when $B$ is a Sidon set in $[1,\lfloor N/3\rfloor]$, and a universal upper bound (Lemma 864.2) giving $F(N)\le 2\sqrt N+1$.

(ii) \emph{First gap (crisp).}
Prove (or disprove) the asymptotic upper bound
$F(N)\le (1+o(1))\tfrac{2}{\sqrt 3}\sqrt N$.
Even a structural theorem forcing an extremal $A$ to be close to a ``two-block reflection'' configuration is missing.

(iii) \emph{Top 3 next moves.}
\begin{itemize}
\item Prove a stability statement: if $|A|\ge (\tfrac{2}{\sqrt 3}-\eta)\sqrt N$ then (after an affine change/translation of the exceptional sum) most of $A$ lies in two intervals of length $\approx N/3$.
\item Strengthen Lemma 864.2 by counting overlaps between $A+A$ coming from different regions of $A$ (e.g. a three-interval decomposition of $[1,N]$) to force $N/3$-type geometry.
\item Extend exact/heuristic computation to larger $N$ (e.g. $N\le 500$) to see whether $F(N)/\sqrt N$ appears to approach $2/\sqrt 3$ or a larger constant.
\end{itemize}

(iv) \emph{Minimal counterexample structure.}
If the conjectured constant $2/\sqrt 3$ is false, one expects a family of sets $A_N$ with $|A_N|\ge (\tfrac{2}{\sqrt 3}+\delta)\sqrt N$ and exactly one heavily-colliding sum; such a counterexample would likely require using more than two ``blocks'' or a non-reflection geometry so that cross-sums avoid colliding despite overlapping sum ranges.


