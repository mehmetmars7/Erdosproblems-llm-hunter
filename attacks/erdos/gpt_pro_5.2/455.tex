% Erdos Problem #455

\textbf{FORMAL RESTATEMENT}

Let $q_1<q_2<\cdots$ be an infinite sequence of primes. Define gaps $g_n:=q_n-q_{n-1}$ for $n\ge 2$.
Assume the gaps are nondecreasing:
\[
\forall n\ge 2:\quad g_{n+1}\ge g_n\quad\text{equivalently}\quad q_{n+1}-q_n\ge q_n-q_{n-1}.
\]
The question is whether necessarily
\[
\lim_{n\to\infty}\frac{q_n}{n^2}=\infty.
\]

\textbf{QUICK LITERATURE/CONTEXT CHECK}

Only what is stated in the problem text is used: Richter (1976) proved
$\liminf_n \frac{q_n}{n^2}>0.352\dots$.
No other external results are imported here.

\textbf{ATTACK PLAN}

\begin{itemize}
\item Prove elementary structural constraints forced by monotone gaps (e.g. gaps must be unbounded; basic growth inequalities).
\item FAST REALITY CHECK: build an explicit example sequence by a greedy algorithm (always pick the smallest next prime respecting the nondecreasing-gap rule) and look at $q_n/n^2$ numerically.
\item Identify the precise missing ingredient: one would need a lower bound on how fast prime gaps can be forced to grow in such a subsequence.
\end{itemize}

\textbf{WORK}

\emph{Fast reality check (one explicit admissible sequence).}
Define greedily: $q_1=2$, $q_2=3$, and for $n\ge 2$ choose $q_{n+1}$ to be the smallest prime $>q_n$ with $q_{n+1}-q_n\ge q_n-q_{n-1}$.
The first terms are
\[
2,3,5,7,11,17,23,29,37,47,59,71,83,97,113,131,149,167,191,223,\dots
\]
and the ratios along this sequence are (exact computation)
\[
\frac{q_{10}}{10^2}=0.47,\quad \frac{q_{20}}{20^2}\approx 0.5575,\quad \frac{q_{50}}{50^2}\approx 1.3964,\quad \frac{q_{100}}{100^2}\approx 2.1313,\quad \frac{q_{500}}{500^2}\approx 3.7713.
\]
This suggests $q_n/n^2$ may grow along at least this example, but this is not a proof and does not address worst-case sequences.

\medskip

\textbf{Lemma 455.1 (gaps must be unbounded).}
In any infinite prime sequence $(q_n)$ with nondecreasing gaps $g_n=q_n-q_{n-1}$, the gaps are unbounded; in fact $g_n\to\infty$.

\emph{Proof.}
Assume for contradiction that $(g_n)$ is bounded above by some integer $G$. Since $(g_n)$ is a nondecreasing sequence of integers, boundedness implies that $g_n$ is eventually constant: there exists $N$ and an integer $d$ such that $g_n=d$ for all $n\ge N$.
Then for all $n\ge N$,
\[
q_{n}=q_N+(n-N)d,
\]
so the tail of the sequence lies in the arithmetic progression $q_N+d\mathbb N$.

Choose a prime $r$ with $r\nmid d$ (there are infinitely many primes, and only finitely many divide $d$).
Because $\gcd(d,r)=1$, the congruence $(q_N+(n-N)d)\equiv 0\pmod r$ has a solution $n\equiv n_0\pmod r$, and hence infinitely many solutions with $n\ge N$.
For such an $n$ we have $r\mid q_n$. Since $q_n$ is prime, this forces $q_n=r$.
But there are infinitely many such $n$, so we would get infinitely many indices with $q_n=r$, contradicting the strict increase $q_n<q_{n+1}$.
Therefore $(g_n)$ is unbounded.

Finally, since $(g_n)$ is nondecreasing and unbounded, it must satisfy $g_n\to\infty$.
\qed

\medskip

\textbf{Lemma 455.2 (growth inequality from monotone gaps).}
For any $m<n$,
\[
q_n\ge q_m+(n-m)\,(q_{m+1}-q_m).
\]
In particular $q_n\ge q_1+(n-1)(q_2-q_1)$ for all $n\ge 1$.

\emph{Proof.}
Write
\[
q_n-q_m=\sum_{j=m+1}^n (q_j-q_{j-1})=\sum_{j=m+1}^n g_j.
\]
Since the gaps are nondecreasing, for every $j\ge m+1$ we have $g_j\ge g_{m+1}=q_{m+1}-q_m$. Thus
\[
q_n-q_m=\sum_{j=m+1}^n g_j\ge \sum_{j=m+1}^n (q_{m+1}-q_m)=(n-m)(q_{m+1}-q_m),
\]
which is the claimed inequality.
\qed

\medskip

\textbf{VERIFICATION}

\begin{itemize}
\item Lemma 455.1 uses only (a) integer-valued monotone bounded $\Rightarrow$ eventually constant, and (b) a modular argument to show an infinite arithmetic progression cannot consist entirely of distinct primes. The modular argument is checked carefully by selecting a prime $r\nmid d$.
\item Lemma 455.2 is a direct telescoping sum + monotonicity lower bound.
\item The greedy sequence sanity check indeed respects $g_{n+1}\ge g_n$ by construction.
\end{itemize}

\textbf{FINAL.} \textbf{UNRESOLVED}

(i) \emph{Strongest proved partial result.} Any such sequence has $g_n\to\infty$ (Lemma 455.1) and satisfies the general growth lower bound $q_n\ge q_m+(n-m)(q_{m+1}-q_m)$ (Lemma 455.2). Computations for one explicit greedy sequence show $q_n/n^2$ increases at least up to $n=500$.

(ii) \emph{First gap (crisp).} Prove that in \emph{every} such prime sequence, the gaps grow at least linearly in $n$ (e.g. $g_n\gg n$), which would imply $q_n\gg n^2$ with a diverging constant, or else explicitly construct an infinite prime sequence with $g_n=o(n)$ (which would refute the claim).

(iii) \emph{Top 3 next moves.}
\begin{enumerate}
\item Try a disproof-by-construction track: can one choose primes near $C n^2$ with increasing gaps $\asymp n$ indefinitely (this is a prime-distribution-in-short-intervals question)?
\item Try a proof track: use modular obstructions stronger than Lemma 455.1 to show that too-slowly-growing gaps force congruence repetitions that create composites.
\item Computation: search algorithmically for ``slow-growth'' admissible sequences (not just greedy) and track the smallest possible $q_n$ for each $n$ under the constraint.
\end{enumerate}

(iv) \emph{Minimal counterexample structure.} A minimal counterexample to $q_n/n^2\to\infty$ would be an infinite sequence of primes with nondecreasing gaps and a constant $C$ such that $q_n\le C n^2$ for infinitely many $n$; such a sequence would force $g_n=O(n)$ along an infinite subsequence while still threading primality constraints.


