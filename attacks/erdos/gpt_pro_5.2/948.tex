
Is there a function $f(n)$ and a $k$ such that in any $k$-colouring of the integers there exists a sequence $a_1<\cdots$ such that $a_n<f(n)$ for infinitely many $n$ and the set\[\left\{ \sum_{i\in S}a_i : \textrm{finite }S\right\}\]does not contain all colours? Erd\H{o}s initially asked whether this is possible with the set being monochromatic, but Galvin showed that this is not always possible, considering the two colouring where, writing $n=2^km$ with $m$ odd, we colour $n$ red if $m\geq F(k)$ and blue if $m<F(k)$ (for some sufficiently quickly growing $F$). This is open even in the case of $\aleph_0$-many colours. This is asking about a variant of Hindman's theorem (see [532] ).

\subsection*{FORMAL RESTATEMENT}
Fix $k\in\mathbb{N}$ and a function $f:\mathbb{N}\to\mathbb{N}$.
A $k$-colouring of the positive integers is a map $c:\mathbb{N}\to\{1,2,\dots,k\}$.
Given a strictly increasing sequence of positive integers $a_1<a_2<\cdots$, define its finite-sums set
\[
\mathrm{FS}(a_i) := \Big\{\sum_{i\in S} a_i : S\subseteq \mathbb{N} \text{ finite, nonempty}\Big\} \subseteq \mathbb{N}.
\]
Question: Do there exist some $k\ge 2$ and some function $f$ such that for every $k$-colouring $c$ there exists an increasing sequence $(a_i)$ with
\begin{enumerate}
\item $a_n < f(n)$ for infinitely many $n$, and
\item $c(\mathrm{FS}(a_i))\ne \{1,2,\dots,k\}$ (i.e. $\mathrm{FS}(a_i)$ omits at least one colour).
\end{enumerate}

\subsection*{QUICK LITERATURE/CONTEXT CHECK}
I do not use external results beyond what is in the problem text.
The text states that the stronger requirement ``$\mathrm{FS}(a_i)$ is monochromatic'' is not always possible under the growth constraint, due to a construction of Galvin using a 2-adic valuation based colouring with a rapidly growing threshold function $F$.

\subsection*{ATTACK PLAN}
\emph{Proof-track ideas.}
\begin{itemize}
\item Try to obtain a positive answer for some $k\ge 3$ by weakening the target from monochromaticity to omitting at least one colour, perhaps allowing a bounded-growth subsequence.
\item Seek finite (compactness-type) reductions: show that if the statement fails, there is a finite colouring on $[1,N]$ obstructing all sequences with many small terms.
\end{itemize}
\emph{Disproof-track ideas.}
\begin{itemize}
\item Generalize Galvin-type colourings to $k\ge 3$ to force any finite-sums set coming from a sequence with infinitely many ``small'' terms to hit all colours.
\end{itemize}

\subsection*{WORK}
\paragraph{Fast sanity checks (toy colourings).}
\begin{itemize}
\item For the 2-colouring by parity, the sequence $a_n=2n$ satisfies $a_n<3n$ for all $n$, and $\mathrm{FS}(a_i)$ consists only of even numbers, hence omits the odd colour. So for some colourings the desired conclusion is easy.
\item However, the question asks for a \emph{uniform} $(k,f)$ that works for \emph{every} $k$-colouring.
\end{itemize}

\paragraph{Lemma 948.1 (for $k=2$, ``omits a colour'' $\Leftrightarrow$ monochromatic).}
Let $k=2$ and let $c:\mathbb{N}\to\{\text{red},\text{blue}\}$ be a 2-colouring.
For any sequence $(a_i)$, the set $\mathrm{FS}(a_i)$ omits a colour if and only if it is monochromatic.

\emph{Proof.}
If $\mathrm{FS}(a_i)$ is monochromatic, it uses only one of the two colours, hence omits the other.
Conversely, if $\mathrm{FS}(a_i)$ omits one of the two colours, then every element of $\mathrm{FS}(a_i)$ has the remaining colour, i.e. it is monochromatic.
\hfill $\Box$

\paragraph{Lemma 948.2 (unique representation for superincreasing sequences).}
Suppose $(a_i)$ satisfies the superincreasing condition
\[
a_{n+1} > \sum_{i=1}^n a_i \quad\text{for all }n\ge 1.
\]
Then every $s\in \mathrm{FS}(a_i)$ has a unique representation $s=\sum_{i\in S} a_i$ with $S$ finite.

\emph{Proof.}
Assume $\sum_{i\in S} a_i = \sum_{i\in T} a_i$ with distinct finite sets $S\ne T$.
Let $j$ be the largest index in the symmetric difference $S\triangle T$.
Without loss of generality, assume $j\in S\setminus T$.
Then
\[
0 = \sum_{i\in S} a_i - \sum_{i\in T} a_i = a_j + \sum_{i\in S\cap[1,j-1]} a_i - \sum_{i\in T\cap[1,j-1]} a_i.
\]
Taking absolute values and using that all terms are positive,
\[
a_j = \Big|\sum_{i\in T\cap[1,j-1]} a_i - \sum_{i\in S\cap[1,j-1]} a_i\Big| \le \sum_{i=1}^{j-1} a_i.
\]
This contradicts the superincreasing hypothesis applied at $n=j-1$, which gives $a_j>\sum_{i=1}^{j-1} a_i$.
Therefore no two distinct finite subsets can yield the same sum.
\hfill $\Box$

\subsection*{VERIFICATION}
\begin{itemize}
\item Lemma 948.1: checked both directions rely only on having exactly two colours.
\item Lemma 948.2: checked the ``largest differing index'' argument forces the contradiction because the remaining sums involve only indices $<j$.
\item Edge conventions: I defined $\mathrm{FS}(a_i)$ using nonempty finite $S$ so that $0$ is excluded (the problem statement does not specify whether $S=\emptyset$ is allowed; including or excluding $0$ does not change the ``omits a colour'' question unless a colouring treats $0$ specially).
\end{itemize}

\subsection*{FINAL}
**UNRESOLVED**
(i) Strongest proved partial result: for $k=2$ the requirement ``$\mathrm{FS}(a_i)$ omits a colour'' is equivalent to ``$\mathrm{FS}(a_i)$ is monochromatic'' (Lemma 948.1). Thus any negative result for the monochromatic version under growth constraints (such as the Galvin construction described in the problem text) automatically blocks $k=2$.
(ii) First gap: produce either (a) some explicit $k\ge 3$ and $f$ and a proof that the statement holds for all $k$-colourings, or (b) for each candidate $(k,f)$ construct a $k$-colouring forcing every sequence with $a_n<f(n)$ infinitely often to have $\mathrm{FS}(a_i)$ hit all $k$ colours.
(iii) Top 3 next moves:
\begin{enumerate}
\item Try to generalize Galvin's 2-adic-threshold colouring to $k\ge 3$ in a way that forces colour-surjectivity on finite-sums sets generated by ``slow-growing'' sequences.
\item Attempt a positive result for $k=3$ by allowing $\mathrm{FS}(a_i)$ to occupy only 2 of the 3 colours: look for a combinatorial selection argument producing a subsequence with controlled growth and restricted colour set.
\item Implement finite searches: for small $k$ and small bounds $f(n)$, brute-force all colourings of $[1,N]$ to see whether every colouring admits a short initial segment whose finite sums miss a colour; look for patterns that could extend to an infinite construction.
\end{enumerate}
(iv) Minimal counterexample structure (to a positive answer): a colouring $c$ for which every sequence $(a_i)$ having infinitely many indices with $a_n<f(n)$ generates a finite-sums set that is colour-surjective; such a colouring would likely encode information at many scales (e.g. via valuations or growth thresholds) so that subset sums inevitably realize all colours.
