\section{Problem 66 --- Round 2 (continuation \& gap-closure mode)}

\subsection*{1) ROUND-2 OBJECTIVE}
\textbf{Path (C): obstruction/correction direction.}  Starting from the Round~1 status \textbf{UNRESOLVED}, I pursue the ``disproof/rigidity'' track by extracting a \emph{strictly stronger necessary consequence} of the existence of a nonzero limit
\[
\frac{(1_A\ast 1_A)(n)}{\log n}\to L\in(0,\infty).
\]
Round~1 only obtained the order of growth $A(x)=\Theta(\sqrt{x\log x})$.  In this round I close that gap: assuming the limit exists, I prove that the counting function in fact has a \emph{full asymptotic with an explicit constant depending on $L$}:
\[
A(x)\sim 2\sqrt{\frac{L}{\pi}}\,\sqrt{x\log x}.
\]
This is a rigid constraint on any potential example and is a substantial strengthening of Round~1.

\subsection*{2) ROUND-1 FOUNDATION USED}
I rely on the following Round~1 items (as black boxes):
\begin{itemize}
\item The formal definitions: $a_n:=1_A(n)$ and $r(n):=(1_A\ast 1_A)(n)=\sum_{k=1}^{n-1}a_ka_{n-k}$ counts \emph{ordered} representations.
\item Corollary~66.1 only as context (I strengthen it).
\end{itemize}
No Round~1 probabilistic computations are used in the new proof.

\subsection*{3) NEW INSIGHT / TOOL (ROUND-2)}
\textbf{New tool: Abelian--Tauberian passage through generating functions/Laplace transforms.}

Let
\[
F(t):=\sum_{n\ge 2} r(n)e^{-tn}=\Big(\sum_{m\ge 1} a_m e^{-tm}\Big)^2\qquad(t>0),
\]
which is the Laplace transform of $r(n)$ and equals the square of the Laplace transform of $a_m$.

From the assumed pointwise asymptotic $r(n)\sim L\log n$ I first derive an \emph{Abelian} asymptotic for $F(t)$ as $t\to 0^+$, then take a square root, and finally apply \emph{Karamata's Tauberian theorem} (Laplace transform form) to recover an explicit asymptotic for the counting function $A(x)=\sum_{m\le x}a_m$.

\subsection*{4) ATTACK PLAN (ROUND-2)}
\textbf{Round~1 gap addressed:} Corollary~66.1 only yields two-sided bounds $c\sqrt{x\log x}\le A(x)\le C\sqrt{x\log x}$.

\textbf{Goal in this round:} upgrade this to an explicit asymptotic with constant.

\textbf{Plan:}
\begin{enumerate}
\item Prove an Abelian lemma: from $r(n)\sim L\log n$, show
\[
F(t)=\sum_{n\ge 2} r(n)e^{-tn}\sim \frac{L}{t}\log\frac1t\qquad(t\to 0^+).
\]
\item Take square roots to obtain the Laplace transform asymptotic for $a_n$:
\[
G(t):=\sum_{m\ge 1} a_m e^{-tm}=\sqrt{F(t)}\sim \sqrt{L}\,t^{-1/2}\,\sqrt{\log\tfrac1t}.
\]
\item Apply Karamata's Tauberian theorem to the nondecreasing function $A(x)$ to conclude
\[
A(x)\sim \frac{\sqrt{L}}{\Gamma(3/2)}\,x^{1/2}\,\sqrt{\log x}=2\sqrt{\frac{L}{\pi}}\,\sqrt{x\log x}.
\]
\end{enumerate}

\subsection*{5) WORK (ROUND-2)}
Throughout, assume
\begin{equation}
\label{eq:limit-assumption}
\frac{r(n)}{\log n}\to L\in(0,\infty),\qquad r(n)=(1_A\ast 1_A)(n).
\end{equation}

\medskip
\noindent\textbf{Lemma 66.3 (model Laplace asymptotic for $\sum \log n\,e^{-tn}$).}
Define
\[
S(t):=\sum_{n\ge 2} (\log n)e^{-tn}\qquad(t>0).
\]
Then as $t\to 0^+$,
\[
S(t)=\frac{1}{t}\log\frac{1}{t}+O\Big(\frac{1}{t}\Big).
\]
In particular $S(t)\sim \frac{1}{t}\log\frac{1}{t}$.

\begin{proof}
Let
\[
I(t):=\int_{1}^{\infty} (\log x)e^{-tx}\,dx.
\]
Substitute $u=tx$ to get
\[
I(t)=\frac{1}{t}\int_{t}^{\infty} \big(\log u-\log t\big)e^{-u}\,du
=\frac{1}{t}\Big(\log\frac1t\int_{t}^{\infty}e^{-u}\,du+\int_{t}^{\infty}(\log u)e^{-u}\,du\Big).
\]
As $t\to 0^+$, $\int_{t}^{\infty}e^{-u}\,du\to 1$ and $\int_{t}^{\infty}(\log u)e^{-u}\,du\to\int_{0}^{\infty}(\log u)e^{-u}\,du$, a finite constant. Hence
\[
I(t)=\frac{1}{t}\log\frac1t+O\Big(\frac{1}{t}\Big).
\]

It remains to compare $S(t)$ and $I(t)$. The function $g_t(x):=(\log x)e^{-tx}$ is smooth, nonnegative, and tends to $0$ as $x\to\infty$. A standard bounded-variation estimate (e.g. summation by parts) gives
\[
\Big|\sum_{n\ge 2} g_t(n)-\int_{1}^{\infty} g_t(x)\,dx\Big|\le g_t(1)+\int_{1}^{\infty}|g_t'(x)|\,dx.
\]
Compute $g_t'(x)=(1/x-t\log x)e^{-tx}$, so
\[
\int_{1}^{\infty}|g_t'(x)|\,dx\le \int_{1}^{\infty}\Big(\frac{1}{x}+t\log x\Big)e^{-tx}\,dx.
\]
After the substitution $u=tx$, both integrals on the right are $O(\log\tfrac1t)$ as $t\to 0^+$, hence the discretization error is $O(\log\tfrac1t)$. Therefore
\[
S(t)=I(t)+O\Big(\log\frac1t\Big)=\frac{1}{t}\log\frac1t+O\Big(\frac{1}{t}\Big),
\]
as claimed.
\end{proof}

\medskip
\noindent\textbf{Lemma 66.4 (Abelian step: Laplace asymptotic for $r(n)$).}
Let
\[
F(t):=\sum_{n\ge 2} r(n)e^{-tn}\qquad(t>0).
\]
Under \eqref{eq:limit-assumption}, as $t\to 0^+$,
\[
F(t)\sim \frac{L}{t}\log\frac1t.
\]

\begin{proof}
Fix $\varepsilon\in(0,1)$. By \eqref{eq:limit-assumption} there exists $N_0(\varepsilon)$ such that for all $n\ge N_0$,
\[
(L-\varepsilon)\log n\le r(n)\le (L+\varepsilon)\log n.
\]
Split
\[
F(t)=\sum_{2\le n< N_0} r(n)e^{-tn}+\sum_{n\ge N_0} r(n)e^{-tn}.
\]
The finite initial sum is $O(1)$ uniformly in $t$. For the tail we get
\[
(L-\varepsilon)\sum_{n\ge N_0}(\log n)e^{-tn}\le \sum_{n\ge N_0} r(n)e^{-tn}\le (L+\varepsilon)\sum_{n\ge N_0}(\log n)e^{-tn}.
\]
Adding back the $O(1)$ initial segment and using Lemma~66.3 (noting that removing finitely many terms does not change the $\sim$-asymptotic) gives
\[
(L-\varepsilon)\Big(\frac{1}{t}\log\frac1t+O\Big(\frac{1}{t}\Big)\Big)+O(1)
\le F(t)\le
(L+\varepsilon)\Big(\frac{1}{t}\log\frac1t+O\Big(\frac{1}{t}\Big)\Big)+O(1).
\]
Divide by $(1/t)\log(1/t)$ and let $t\to 0^+$. The $O(1/t)$ term is smaller than $(1/t)\log(1/t)$ by a factor $1/\log(1/t)\to 0$, and $O(1)$ is negligible. Hence
\[
L-\varepsilon\le \liminf_{t\to 0^+}\frac{F(t)}{(1/t)\log(1/t)}\le \limsup_{t\to 0^+}\frac{F(t)}{(1/t)\log(1/t)}\le L+\varepsilon.
\]
Since $\varepsilon$ is arbitrary, $F(t)/((1/t)\log(1/t))\to L$.
\end{proof}

\medskip
\noindent\textbf{Lemma 66.5 (square-root transfer to the Laplace transform of $A$).}
Define
\[
G(t):=\sum_{m\ge 1} a_m e^{-tm}=\sum_{m\in A} e^{-tm} \qquad(t>0).
\]
Then $F(t)=G(t)^2$ and, under \eqref{eq:limit-assumption},
\[
G(t)\sim \sqrt{L}\,t^{-1/2}\,\sqrt{\log\tfrac1t}\qquad(t\to 0^+).
\]

\begin{proof}
Since $r(n)=\sum_{k=1}^{n-1}a_ka_{n-k}$, we have the identity of absolutely convergent series for $t>0$:
\[
\sum_{n\ge 2} r(n)e^{-tn}=\sum_{n\ge 2}\Big(\sum_{k=1}^{n-1}a_ka_{n-k}\Big)e^{-tn}
=\Big(\sum_{m\ge 1}a_m e^{-tm}\Big)^2,
\]
i.e. $F(t)=G(t)^2$.  For $t>0$, $G(t)>0$ (as soon as $A\neq\emptyset$) and $F(t)>0$, so taking square roots is legitimate. Lemma~66.4 yields
\[
G(t)=\sqrt{F(t)}\sim \sqrt{\frac{L}{t}\log\frac1t}=\sqrt{L}\,t^{-1/2}\,\sqrt{\log\tfrac1t}.
\]
\end{proof}

\medskip
\noindent\textbf{Proposition 66.6 (Tauberian sharpening of Round~1 Corollary 66.1).}
Assume \eqref{eq:limit-assumption}. Then the counting function satisfies the full asymptotic
\[
A(x):=|A\cap[1,x]|\sim 2\sqrt{\frac{L}{\pi}}\,\sqrt{x\log x}\qquad(x\to\infty).
\]
Equivalently,
\[
\lim_{x\to\infty}\frac{A(x)^2}{x\log x}=\frac{4L}{\pi}.
\]

\begin{proof}
We invoke Karamata's Tauberian theorem in Laplace-transform form (for monotone functions):

\smallskip
\emph{Karamata (Laplace form, special case).} Let $A(x)$ be nondecreasing, right-continuous, with $A(x)=0$ for $x<0$, and let
\[
\mathcal L_A(t):=\int_{[0,\infty)} e^{-tx}\,dA(x)\qquad(t>0)
\]
be its Laplace--Stieltjes transform. Suppose that for some $\rho>0$ and slowly varying $\ell$,
\[
\mathcal L_A(t)\sim C\,t^{-\rho}\,\ell(1/t)\qquad(t\to 0^+).
\]
Then
\[
A(x)\sim \frac{C}{\Gamma(\rho+1)}\,x^{\rho}\,\ell(x)\qquad(x\to\infty).
\]
(See, e.g., Bingham--Goldie--Teugels, \emph{Regular Variation}, or Feller, vol.~II, Tauberian theorems chapter.)

\smallskip
In our setting, $A(x)=\sum_{m\le x}a_m$ is nondecreasing and
\[
G(t)=\sum_{m\ge 1}a_m e^{-tm}=\int_{[0,\infty)} e^{-tx}\,dA(x),
\]
so $G(t)$ is exactly the Laplace--Stieltjes transform of $A$. Lemma~66.5 gives
\[
G(t)\sim \sqrt{L}\,t^{-1/2}\,\ell(1/t),\qquad \ell(u):=\sqrt{\log u},
\]
with $\rho=1/2$ and slowly varying $\ell$. Applying Karamata yields
\[
A(x)\sim \frac{\sqrt{L}}{\Gamma(3/2)}\,x^{1/2}\,\sqrt{\log x}.
\]
Since $\Gamma(3/2)=\tfrac{\sqrt\pi}{2}$, this becomes
\[
A(x)\sim 2\sqrt{\frac{L}{\pi}}\,\sqrt{x\log x}.
\]
Squaring and dividing by $x\log x$ gives the equivalent limit $A(x)^2/(x\log x)\to 4L/\pi$.
\end{proof}

\subsection*{6) ADVERSARIAL VERIFICATION}
\begin{itemize}
\item \textbf{Ordered vs unordered representations.}  The identity $F(t)=G(t)^2$ uses ordered convolution. If one used unordered representations, $F(t)$ would instead correspond to $(G(t)^2+G(2t))/2$; constants would change. Round~1 and this round consistently use \emph{ordered} pairs.
\item \textbf{Asymptotic of $\sum (\log n)e^{-tn}$.}  Lemma~66.3 only needs a leading term. The discretization error is at most $O(\log(1/t))$, far smaller than the main term $(1/t)\log(1/t)$, so the $\sim$ claim is stable.
\item \textbf{Taking square roots.}  For $t>0$, $G(t)\ge 0$ and $F(t)=G(t)^2$, so $G(t)=\sqrt{F(t)}$ with no sign ambiguity.
\item \textbf{Karamata hypotheses.}  The function $A(x)$ is nondecreasing and right-continuous; $\ell(u)=\sqrt{\log u}$ is slowly varying. Also $\rho=1/2>0$, so the theorem applies.
\item \textbf{Boundary cases.}  If $L=0$ the same method gives $G(t)=o(t^{-1/2}\sqrt{\log(1/t)})$, hence $A(x)=o(\sqrt{x\log x})$, consistent with Round~1. If $L=\infty$ the method is inapplicable; the statement of the problem asks $L$ finite.
\end{itemize}

\subsection*{7) FINAL (EXACTLY ONE)}
\textbf{UNRESOLVED (BUT STRICTLY ADVANCED).}

Round~2 closes a concrete Round~1 gap: under the existence of a finite nonzero limit $r(n)/\log n\to L$, the counting function must satisfy the \emph{precise} asymptotic
\[
A(x)\sim 2\sqrt{\frac{L}{\pi}}\,\sqrt{x\log x},
\]
which is substantially stronger than $A(x)=\Theta(\sqrt{x\log x})$.

The core existence/nonexistence question remains open: I did not derive a contradiction nor construct such an $A$.

\subsection*{8) COMPLETION ESTIMATE (MANDATORY)}
\noindent\textbf{COMPLETION: 40\%}

\subsection*{9) REFERENCES}
\begin{itemize}
\item N. H. Bingham, C. M. Goldie, J. L. Teugels, \emph{Regular Variation}, Cambridge Univ. Press (for Karamata Tauberian theorem, Laplace-transform form).
\item W. Feller, \emph{An Introduction to Probability Theory and Its Applications}, Vol.~II (Tauberian theorems chapter).
\end{itemize}
