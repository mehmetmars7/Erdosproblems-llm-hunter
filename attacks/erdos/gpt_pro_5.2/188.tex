\section*{Erd\H{o}s problem 188}

\subsection*{1) FORMAL RESTATEMENT}
A 2-colouring of $\mathbb{R}^2$ assigns each point red or blue.
We seek the smallest integer $k$ such that there exists a colouring with:
(A) no two red points at Euclidean distance $1$; and
(B) no $k$-term \emph{unit-step collinear arithmetic progression} of blue points, i.e. no set
\[
\{x, x+v, x+2v,\dots,x+(k-1)v\}\subset \mathbb{R}^2
\]
with $\|v\|=1$ all coloured blue.

\subsection*{2) QUICK LITERATURE/CONTEXT CHECK}
Problem text states: if one drops the unit-step condition in (B), then no such $k$ exists (Alon’s observation).
For the unit-step version, the text records lower bounds $k\ge 6$ (Graham) and $k\ge 9$ (Exoo, unpublished).

\subsection*{3) ATTACK PLAN}
(1) Prove a baseline lower bound $k\ge 3$ (since $k=2$ would forbid blue unit-distance pairs too, impossible).
(2) Give a full proof of Alon’s ``no finite $k$'' for the \emph{unrestricted-step} interpretation.
These are rigorous partial results; the intended unit-step problem remains open.

\subsection*{4) WORK}

\paragraph{Lemma 4.1 ($k\ge 3$).}
There is no 2-colouring of $\mathbb{R}^2$ with \emph{both} colours avoiding unit-distance pairs.
In particular, a colouring satisfying (A) and additionally forbidding blue 2-term unit-step progressions cannot exist; hence $k\ne 2$ and $k\ge 3$.
\textit{Proof.}
Assume for contradiction that no two red points are at distance $1$ and no two blue points are at distance $1$.
Consider an equilateral triangle of side $1$ in $\mathbb{R}^2$ with vertices $p,q,r$.
By pigeonhole, two vertices share a colour; those two are at distance $1$, contradicting the assumption.
Therefore at least one colour must contain a unit-distance pair, so $k=2$ is impossible. \hfill$\square$

\paragraph{Lemma 4.2 (Alon’s observation for unrestricted progressions).}
If condition (B) is interpreted as ``no $k$-term arithmetic progression of blue points on a line'' with \emph{no restriction on step length},
then \emph{no finite $k$ exists}: every colouring with (A) forces arbitrarily long blue arithmetic progressions along some unit-spaced line.
\textit{Proof.}
Fix a line $L$ in $\mathbb{R}^2$ and identify its integer lattice points $\{m\in\mathbb{Z}\}$ spaced by distance $1$.
Restrict the plane-colouring to these points, giving a 2-colouring of $\mathbb{Z}$.
If there are two red points at distance $1$ on $L$, then (A) fails, so assume not.
Thus no two consecutive integers are red, so for every $n\in\mathbb{Z}$, either $n$ is blue or $n-1$ is blue.
Let $B$ be the blue set. Then $B\cup (B+1)=\mathbb{Z}$.

Define a genuine 2-colouring $c$ of $\mathbb{Z}$ by: $c(n)=0$ if $n\in B$, and $c(n)=1$ otherwise (so if $c(n)=1$ then $n-1\in B$).
By van der Waerden’s theorem (for 2 colours), $c$ contains arbitrarily long monochromatic arithmetic progressions.
If such a progression is in colour $0$, it lies in $B$ and is a blue progression.
If it is in colour $1$, then shifting the progression by $-1$ yields a progression in $B$ (since $c(n)=1\Rightarrow n-1\in B$).
Hence $B$ contains arbitrarily long arithmetic progressions (with some step size), so no finite $k$ can forbid them. \hfill$\square$

\subsection*{5) VERIFICATION}
Lemma 4.1 is a one-triangle obstruction; it uses only the existence of an equilateral unit triangle.
Lemma 4.2 uses only the restriction to a unit-spaced line and van der Waerden’s theorem as stated in the problem text.

\subsection*{6) FINAL}
\textbf{UNRESOLVED} (for the intended \emph{unit-step} version)

(i) Strongest proved partial results: $k\ge 3$ (Lemma 4.1). Also, if step-length is not fixed, then no finite $k$ exists (Lemma 4.2).

(ii) First gap: construct \emph{any} colouring satisfying (A) and (B) for some explicit finite $k$, or prove impossibility below the best-known lower bounds.

(iii) Top 3 next moves:
1. Encode unit-distance constraints as a graph/hypergraph colouring problem on discretizations (lattices / tilings).
2. Study whether blue set must contain long unit-step collinear progressions given red is a distance-1 avoiding set.
3. Attempt computational / SAT search on periodic colourings (fundamental domain) to get explicit upper bounds.

(iv) Minimal counterexample structure: a periodic or quasi-periodic colouring with red as a distance-1 avoiding set of positive density and blue
engineered to avoid long collinear unit-step runs in every direction.

