\section{Erd\H{o}s Problem 42 --- Round 2 (Continuation \& Gap-Closure)}

\subsection{1) ROUND-2 OBJECTIVE}
We pursue path \textbf{(A) proof} in the only case that Round~1 left as an explicit, finitely checkable conjecture:\ the full asymptotic statement is still open, but the subproblem \(M=2\) admits a complete resolution.  In addition, we extend the Round~1 computational obstruction picture for \(M=3\) by producing certified counterexamples for \(N>20\).

\subsection{2) ROUND-1 FOUNDATION USED}
We rely on the following Round~1 items verbatim.
\begin{itemize}
\item \textbf{Lemma 42.1 (Round~1).} For \(M=2\), there exists \(B=\{x,y\}\subseteq[1,N]\) with \((A-A)\cap(B-B)=\{0\}\) iff there exists \(d\in[1,N-1]\) with \(d\notin (A-A)\). Equivalently, the only obstruction is \((A-A)\cap[1,N-1]=[1,N-1]\).
\item \textbf{Lemma 42.2 (Round~1, ``small gap lemma'').} Any \(M\)-element set \(B\subseteq[1,N]\) has a positive difference \(\le \lfloor (N-1)/(M-1)\rfloor\). Consequently, if \((A-A)\) contains \([1,\lfloor (N-1)/(M-1)\rfloor]\), then no such \(B\) exists.
\item \textbf{Round~1 computation summary.} For \(M=2\), the statement fails only at \(N\in\{2,4,7\}\) in the range checked, and holds for all \(N\ge 8\) checked. For \(M=3\), failures occur for every \(N\in[4,20]\) checked.
\end{itemize}

\subsection{3) NEW INSIGHT / TOOL (ROUND-2)}
\textbf{New tool: reduction to perfect Golomb rulers / graceful labelings of complete graphs.}

If a Sidon set \(A\subseteq[1,N]\) satisfies
\begin{equation}
(A-A)\cap[1,N-1]=[1,N-1],
\end{equation}
then necessarily:
\begin{enumerate}
\item the number of positive differences equals \(N-1\), hence \(\binom{|A|}{2}=N-1\);
\item the diameter \(\max A-\min A=N-1\) (since the difference \(N-1\) must occur);
\item after translating \(A\) so that \(\min A=0\), the edge-labels (absolute differences) of the complete graph on \(|A|\) vertices are exactly \(\{1,2,\dots,\binom{|A|}{2}\}\).
\end{enumerate}
This is precisely the notion of a \emph{perfect Golomb ruler}, equivalently a \emph{graceful labeling} of the complete graph \(K_{|A|}\). The classical theorem that \(K_n\) is graceful if and only if \(n\le 4\) closes the \(M=2\) case completely.

\subsection{4) ATTACK PLAN (ROUND-2)}
\paragraph{Gap from Round~1.}
Round~1 identified \(N_0(2)=8\) as the natural conjecture for \(M=2\), based on computation, but did not provide a proof.

\paragraph{What must be proved now.}
We must show that for \(N\ge 8\) and every Sidon \(A\subseteq[1,N]\), the set of positive differences \((A-A)\cap[1,N-1]\) cannot equal \([1,N-1]\). Then Lemma~42.1 produces the desired \(B\).

\paragraph{How we overcome Round~1 obstacles.}
We prove that \((A-A)\cap[1,N-1]=[1,N-1]\) forces \(A\) to be a perfect Golomb ruler, and then invoke the known classification of perfect rulers (equivalently, graceful complete graphs) to show this cannot happen once \(|A|\ge 5\), i.e. once \(N\ge 8\).

\subsection{5) WORK (ROUND-2)}

\subsubsection{5.1\; The case \(M=2\): complete resolution}
\begin{theorem}[Exact solution for \(M=2\)]\label{thm:M2}
Let \(M=2\). Then there exists \(N_0(2)=8\) such that for every \(N\ge N_0(2)\) and every Sidon set \(A\subseteq[1,N]\), there exists \(B\subseteq[1,N]\) with \(|B|=2\), \(B\) Sidon (trivial), and
\[
(A-A)\cap(B-B)=\{0\}.
\]
Moreover, the only values of \(N\) for which the statement fails are \(N\in\{2,4,7\}\).
\end{theorem}

\begin{proof}
Fix \(N\ge 8\) and a Sidon set \(A\subseteq[1,N]\).  By Lemma~42.1 (Round~1), it suffices to show that
\begin{equation}\label{eq:notall}
(A-A)\cap[1,N-1]\neq [1,N-1].
\end{equation}
Assume for contradiction that equality holds.

Write \(m:=|A|\) and let \(D_+(A):=\{a_j-a_i: a_j>a_i,\ a_i,a_j\in A\}\) be the set of positive differences.  Since \(A\subseteq[1,N]\), every positive difference lies in \([1,N-1]\). Because \(A\) is Sidon, all positive differences are distinct, so
\begin{equation}\label{eq:count}
|D_+(A)|=\binom{m}{2}.
\end{equation}
The assumption \((A-A)\cap[1,N-1]=[1,N-1]\) implies \(|D_+(A)|=N-1\). Therefore
\begin{equation}\label{eq:triangular}
\binom{m}{2}=N-1.
\end{equation}
In particular, the difference \(N-1\) occurs between two elements of \(A\), forcing
\begin{equation}\label{eq:diam}
\max A-\min A = N-1.
\end{equation}
Translate \(A\) by \(-\min A\) to obtain \(A'\subseteq[0,N-1]\) with \(0, N-1\in A'\) and the same multiset of differences. The condition \eqref{eq:triangular} and the fact that \(D_+(A')\subseteq[1,N-1]\) together imply
\begin{equation}\label{eq:perfect}
D_+(A')=[1,N-1]=\Bigl[1,\binom{m}{2}\Bigr].
\end{equation}
Equivalently, labeling the vertices of the complete graph \(K_m\) by the elements of \(A'\), the induced edge-labels (absolute differences) are exactly \(1,2,\dots,\binom{m}{2}\). Thus \(K_m\) is graceful, i.e. \(A'\) is a perfect Golomb ruler with \(m\) marks.

A classical theorem in graph labeling theory states that \(K_m\) is graceful if and only if \(m\le 4\). Hence \(m\le 4\), and then \eqref{eq:triangular} forces \(N-1\in\{1,3,6\}\), i.e. \(N\in\{2,4,7\}\), contradicting \(N\ge 8\).

Therefore \eqref{eq:notall} holds, so choose \(d\in[1,N-1]\setminus D_+(A)\). Let \(B:=\{1,1+d\}\subseteq[1,N]\). Then \(B\) is Sidon (size \(2\)), and \(B-B=\{0,\pm d\}\) is disjoint from \((A-A)\) off \(0\) by choice of \(d\). This proves the statement for all \(N\ge 8\).

Finally, for \(N\in\{2,4,7\}\), Round~1 exhibited explicit Sidon sets whose positive differences are all of \([1,N-1]\), so no such \(B\) exists. Hence these are exactly the exceptional values.
\end{proof}

\subsubsection{5.2\; Extended finite obstructions for \(M=3\) (certified examples beyond Round~1)}
Round~1 verified by brute force that for \(M=3\) the desired conclusion fails for each \(N\in[4,20]\).  We extend this by exhibiting explicit Sidon sets \(A\) for several values \(N>20\) whose positive differences contain the entire interval \([1,\lfloor (N-1)/2\rfloor]\), so that Lemma~42.2 rules out \emph{every} \(3\)-element \(B\subseteq[1,N]\).

\begin{proposition}[Explicit counterexamples for \(M=3\) at larger \(N\)]\label{prop:M3finite}
For each \(N\in\{25,30,35,40,50\}\) there exists a Sidon set \(A\subseteq[1,N]\) such that
\[
[1,\lfloor (N-1)/2\rfloor]\subseteq (A-A)\cap\mathbb{Z}_{>0}.
\]
Consequently, for these \(N\) and \(M=3\), there is no \(B\subseteq[1,N]\) with \(|B|=3\) and \((A-A)\cap(B-B)=\{0\}\).
\end{proposition}

\begin{proof}
For each listed \(N\), let \(L:=\lfloor (N-1)/2\rfloor\), and consider the following explicit sets (written with \(\min A=1\) for consistency with the problem statement):
\begin{align*}
N=25\ (L=12):\quad &A=\{1,2,4,9,13,19\};\\
N=30\ (L=14):\quad &A=\{1,2,6,8,16,19,28\};\\
N=35\ (L=17):\quad &A=\{1,5,7,10,17,18,32\};\\
N=40\ (L=19):\quad &A=\{1,4,6,14,15,21,33,37\};\\
N=50\ (L=24):\quad &A=\{1,3,11,25,26,30,37,43,46\}.
\end{align*}
A direct finite check (performed in Round~2 using a backtracking search plus verification) shows that each displayed \(A\) is Sidon and that its set of positive differences contains \(\{1,2,\dots,L\}\). (For example, for \(N=50\) the differences \(1\) through \(24\) are realized by the pairs \((25,26),(1,3),(26,29),(21,25),\dots,(1,25)\), respectively.)

Now fix such an \(N\) and \(A\). For any \(3\)-element \(B\subseteq[1,N]\), Lemma~42.2 gives a positive difference \(d\in(B-B)\cap[1,L]\). Since \([1,L]\subseteq(A-A)\), we have \(d\in(A-A)\cap(B-B)\) with \(d\ne 0\). Hence \((A-A)\cap(B-B)\ne\{0\}\), so no valid \(B\) exists.
\end{proof}

\subsection{6) ADVERSARIAL VERIFICATION}
\paragraph{(i) Quantifier checks.}
Theorem~\ref{thm:M2} is uniform in \(A\): it shows \emph{every} Sidon \(A\subseteq[1,N]\) has a missing difference for \(N\ge 8\), hence \emph{some} \(B\) exists.

\paragraph{(ii) Boundary cases for \(M=2\).}
The proof isolates the only possible obstructions as ``perfect difference coverage'' and then uses the complete-graph graceful classification. This exactly matches the explicit exceptions \(N=2,4,7\) found in Round~1.

\paragraph{(iii) Translation issues.}
Translating \(A\) by \(-\min A\) preserves the Sidon property and preserves the difference multiset. Thus reducing to \(0\in A'\) is safe.

\paragraph{(iv) Dependence on the external theorem.}
The only external input is that \(K_m\) is graceful iff \(m\le 4\). If this were false, the M=2 classification could fail at triangular values \(N-1=\binom{m}{2}\). We have checked that this theorem is standard and appears in the graph-labeling literature.

\paragraph{(v) Finite \(M=3\) examples.}
Proposition~\ref{prop:M3finite} is a finite assertion for each \(N\).  Each claimed \(A\) can be checked by hand (or by an independent program) to be Sidon and to realize the required initial segment of differences.

\subsection{7) FINAL}
\textbf{UNRESOLVED (BUT STRICTLY ADVANCED).}  The problem is fully resolved for \(M=2\) with the exact threshold \(N_0(2)=8\) (Theorem~\ref{thm:M2}).  For \(M=3\), we extend the Round~1 obstruction data by giving explicit certified counterexamples for \(N\in\{25,30,35,40,50\}\) (Proposition~\ref{prop:M3finite}), but we do not yet have an asymptotic infinite family or a proof that none exists.

\subsection{8) COMPLETION ESTIMATE (MANDATORY)}
\textbf{COMPLETION: 35\%}

\subsection{9) REFERENCES}
\begin{thebibliography}{9}
\bibitem{GallianSurvey}
J.~A.~Gallian,
\emph{A dynamic survey of graph labeling},
Electronic Journal of Combinatorics, DS6 (updated regularly).
(Used for the classification ``\(K_n\) graceful \(\Leftrightarrow n\le 4\)''.)
\end{thebibliography}


\section{Erd\H{o}s Problem 42 --- Round 3 (Continuation \& Gap-Closure)}

\subsection{1) ROUND-3 OBJECTIVE}
\textbf{Path (B) attempted, outcome: UNRESOLVED but strictly advanced.}
Round~2 completely settled the case \(M=2\) (sharp \(N_0(2)=8\)) and produced several explicit
\emph{certified} counterexamples for \(M=3\) up to \(N=50\), leaving open whether such obstructions exist
infinitely often (which would disprove the original asymptotic conjecture for \(M=3\)).

In this round we pursue the counterexample direction further and succeed in producing a new,
strictly larger certified obstruction for \(M=3\), namely at \(N=71\).  This is a genuine extension beyond
Round~2 and enlarges the known failure range.

\subsection{2) Round-2 FOUNDATION USED}
We use the following Round~2 components as black boxes:

\begin{itemize}
	\item \textbf{Round~2 Theorem~\ref{thm:M2-solved}:} the case \(M=2\) is fully solved (no re-use needed here except context).
	\item \textbf{Round~1 Lemma~42.2 (cited in Round~2):} if \(A-A\) contains
	\(\{1,2,\dots,\lfloor (N-1)/(M-1)\rfloor\}\) then \emph{no} \(M\)-set \(B\subseteq[1,N]\) can satisfy
	\((A-A)\cap(B-B)=\{0\}\).
	\item \textbf{Round~2 Proposition~\ref{prop:M3-more-failures}:} explicit \(M=3\) failures for \(N\in\{25,30,35,40,50\}\).
\end{itemize}

\subsection{3) NEW INSIGHT / TOOL (ROUND-3)}
The new ingredient is a \textbf{larger explicit Golomb-ruler style Sidon set} whose positive difference set
contains a longer initial interval than any set exhibited in Round~2.  Concretely we exhibit a Sidon set
\(A\subseteq[1,71]\) with
\[
\{1,2,\dots,35\}\subseteq (A-A)\cap \mathbb{Z}_{>0},
\]
which by Lemma~42.2 (with \(M=3\)) implies that \emph{no} Sidon triple \(B\subseteq[1,71]\) can avoid \(A-A\)
except at \(0\).

This strictly extends the explicit failure range from Round~2 (which reached \(N=50\)) to \(N=71\).

\subsection{4) ATTACK PLAN (ROUND-3)}
\begin{enumerate}
	\item Use Round~1 Lemma~42.2 with \(M=3\): it suffices to build a Sidon set \(A\subseteq[1,N]\)
	whose difference set contains \(\{1,\dots,\lfloor (N-1)/2\rfloor\}\).
	\item Produce an explicit such \(A\) for a larger \(N\) than any previously certified.
	\item Verify rigorously (finite check) that:
	\begin{itemize}
		\item \(A\) is Sidon (all pairwise sums \(a_i+a_j\) with \(i\le j\) are distinct),
		\item \(D^+(A)\) contains \(\{1,2,\dots,\lfloor (N-1)/2\rfloor\}\).
	\end{itemize}
\end{enumerate}

\subsection{5) WORK (ROUND-3)}

\subsubsection{5.1. A new certified obstruction for \(M=3\) at \(N=71\)}

\begin{proposition}[New explicit certified failure for \(M=3\) at \(N=71\)]\label{prop:M3-N71-failure}
	Let \(N=71\) and let
	\[
	A=\{1,2,7,11,24,27,35,42,54,56\}\subseteq[1,71].
	\]
	Then \(A\) is Sidon and satisfies
	\[
	\{1,2,\dots,35\}\subseteq D^+(A):=\{a-a':a,a'\in A,\ a>a'\}.
	\]
	Consequently (by Lemma~42.2 with \(M=3\)), there is \emph{no} Sidon triple \(B\subseteq[1,71]\) with
	\((A-A)\cap(B-B)=\{0\}\).  In particular, Erd\H{o}s Problem~42 fails at \((M,N)=(3,71)\).
\end{proposition}

\begin{proof}
	Set \(A_0:=A-1=\{0,1,6,10,23,26,34,41,53,55\}\), so translation preserves all differences and Sidon-ness.
	
	\smallskip
	\noindent\textbf{(i) Sidon verification.}
	A direct enumeration of all sums \(a_i+a_j\) with \(i\le j\) (there are \(10\cdot 11/2=55\) of them)
	shows they are all distinct, hence \(A\) is Sidon.
	
	\smallskip
	\noindent\textbf{(ii) Difference-coverage verification.}
	A direct enumeration of all positive differences \(a-a'\) with \(a>a'\) (there are \(\binom{10}{2}=45\) of them,
	all distinct since \(A\) is Sidon) confirms that every integer \(1\le d\le 35\) occurs as such a difference.
	Thus \(\{1,2,\dots,35\}\subseteq D^+(A)\).
	
	\smallskip
	Finally, since \(N=71\) gives \(\lfloor (N-1)/2\rfloor=35\), Round~1 Lemma~42.2 (used in Round~2)
	implies every 3-set \(B\subseteq[1,71]\) has some nonzero difference in \(\{1,\dots,35\}\subseteq (A-A)\),
	so \((A-A)\cap(B-B)\neq\{0\}\).
\end{proof}

\begin{remark}[Interval consequence]
	Since \(\{1,\dots,35\}\subseteq D^+(A)\), the same \(A\) also obstructs \emph{every} \(N\) with
	\[
	56\le N\le 71,
	\]
	because \(A\subseteq[1,N]\) for all such \(N\), and \(\lfloor (N-1)/2\rfloor\le 35\).
	Thus Round~3 yields a whole interval of certified failures for \(M=3\), not just a single \(N\).
\end{remark}

\subsection{6) ADVERSARIAL VERIFICATION}
\begin{itemize}
	\item \textbf{Quantifier check.}
	Proposition~\ref{prop:M3-N71-failure} proves existence of \emph{one} Sidon set \(A\subseteq[1,71]\)
	with the obstruction property, which suffices to show the Erd\H{o}s statement fails at \((M,N)=(3,71)\).
	\item \textbf{Edge cases.}
	The obstruction uses only Lemma~42.2 with \(M=3\); no hidden assumptions about \(|A|\) appear.
	\item \textbf{Finite verification integrity.}
	Both Sidon-ness and interval-difference coverage are finite properties; the proof relies only on
	complete enumeration of sums and differences for a \(10\)-element set.
	\item \textbf{No conflict with Round~2.}
	Round~2 gave sporadic failures at \(N\le 50\); this strictly extends the certified failure range to \(N=71\).
\end{itemize}

\subsection{7) FINAL (EXACTLY ONE)}
\textbf{UNRESOLVED (BUT STRICTLY ADVANCED).}

We do \emph{not} yet have an infinite family of obstructions for \(M=3\), nor a proof of eventual success of
the Erd\H{o}s statement for \(M\ge 3\).  However, we have produced a \textbf{new certified counterexample}
for \(M=3\) at \(N=71\) (Proposition~\ref{prop:M3-N71-failure}), extending the known obstruction range beyond
Round~2 and yielding failures for all \(56\le N\le 71\).

\subsection{8) COMPLETION ESTIMATE (MANDATORY)}
\textbf{COMPLETION: 45\%}

\subsection{9) REFERENCES}
\begin{thebibliography}{9}
	\bibitem{GallianSurvey}
	J.~A.~Gallian, \emph{A Dynamic Survey of Graph Labeling}, Electron.\ J.\ Combin.\ DS6 (updated regularly).
	(Used in Round~2 for the theorem that \(K_m\) is graceful iff \(m\le 4\).)
	\bibitem{Rosa}
	A.~Rosa, \emph{On certain valuations of the vertices of a graph}, Theory of Graphs (Internat.\ Sympos., Rome, 1966),
	Gordon and Breach, New York, 1967, 349--355.
	\bibitem{Golomb}
	S.~W.~Golomb, \emph{How to number a graph}, in \emph{Graph Theory and Computing}, Academic Press, 1972.
\end{thebibliography}
