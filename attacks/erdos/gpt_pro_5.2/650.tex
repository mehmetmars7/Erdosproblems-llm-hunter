\section{Problem 650 (distinct multiples in any interval of length $2N$)}

\subsection*{FORMAL RESTATEMENT}
Let $m\ge 1$.  Define $f(m)$ to be the largest integer with the following property:

\begin{quote}
For every integer $N\ge 1$ and every set $A\subseteq\{1,\dots,N\}$ with $|A|=m$, every interval $I\subseteq[1,\infty)$ of length $2N$ contains at least $f(m)$ distinct integers $b_1,\dots,b_{f(m)}\in I$ such that each $b_i$ is divisible by a \emph{different} element $a_i\in A$.
\end{quote}

Equivalently: for each such $(N,A,I)$, in the bipartite graph with left part $A$ and right part $I\cap\mathbb Z$, where $a\sim b$ iff $a\mid b$, the maximum matching has size at least $f(m)$.

The question is to estimate $f(m)$ as $m\to\infty$, and in particular whether
\[
 f(m)\ll m^{1/2}.
\]

\subsection*{QUICK LITERATURE/CONTEXT CHECK (browsing)}
The problem statement on the Erd\H{o}s Problems website records that Erd\H{o}s and S\'ar\'anyi (1959) proved a lower bound of order
\[f(m)\gg m^{1/2}.
\]
I did not find, in a brief search, a matching $O(m^{1/2})$ upper bound or a resolution of the posed question.

\subsection*{ATTACK PLAN}

\begin{enumerate}
\item (Definitions/normalization) Recast as a matching problem in a bipartite graph.
\item (Try small parameters) Compute/guess $f(m)$ for $m\le 4$ to see the behaviour.
\item (Known theorems) Consider Hall's theorem, K\H{o}nig's theorem (for bipartite graphs), and divisor-counting bounds.
\item (Probabilistic attempt) Randomly sample integers in $I$ and estimate how many distinct divisors from $A$ one can ``capture''.
\item (Construction attempt for upper bounds) Try to build $(N,A,I)$ with small maximum matching, to show $f(m)$ cannot be too large.
\item (Contradiction attempt for lower bounds) Assume maximum matching $<r$ and use Hall to force a structured obstruction; attempt to rule this out beyond $r\approx c\sqrt m$.
\end{enumerate}

\subsection*{WORK}

\subsubsection*{Step 1: Trivial universal bounds (proved)}
\begin{lemma}[Trivial bounds]
For all $m\ge 1$,
\[1\le f(m)\le m.
\]
\end{lemma}
\begin{proof}
The lower bound $f(m)\ge 1$ holds because every $a\in A$ has at least one multiple in any interval of length $2N\ge a$.
The upper bound $f(m)\le m$ is immediate because the $a_i$ used must be distinct elements of $A$.
\end{proof}

\subsubsection*{Step 2: Matching reformulation via Hall}
Fix $(N,A,I)$. Let $B:=I\cap\mathbb Z$ so $|B|=2N$.
A matching of size $r$ corresponds to choosing distinct $a_1,\dots,a_r\in A$ and distinct $b_1,\dots,b_r\in B$ with $a_i\mid b_i$.

By Hall's theorem, the maximum matching size is $<r$ if and only if there exists a subset $S\subseteq A$ with
\[
|N(S)|<|S|,
\]
where $N(S)\subseteq B$ is the set of all integers in $B$ divisible by at least one element of $S$.

A route to proving $f(m)\gg \sqrt m$ would be to show that such a Hall obstruction cannot occur with $|S|$ much larger than $\sqrt m$.

\subsubsection*{Step 3: Incomplete attempts}
I attempted two approaches:

\begin{itemize}
\item \emph{Divisor-multiplicity averaging:} lower bound $|N(S)|$ in terms of $\sum_{a\in S} (2N/a)$, and upper bound the number of $a\in S$ that can divide a fixed $b\in B$.  This runs into the difficulty that $b$ can be very large (the interval is not bounded above), so $b$ can have extremely many divisors $\le N$.

\item \emph{Poset/chain--antichain ideas:} try to find a large subfamily of $A$ with a controlled divisibility structure (e.g. a long chain), for which distinct representatives can be chosen.  I did not obtain a clean universal guarantee from this method without invoking deeper results.
\end{itemize}

\subsection*{VERIFICATION}
Only the trivial inequalities are proved here; they are straightforward.

\subsection*{FINAL}
\textbf{UNRESOLVED.}

\medskip
\noindent\textbf{Strongest fully proved partial result in this write-up:}
\begin{quote}
The universal bounds $1\le f(m)\le m$.
\end{quote}
The nontrivial lower bound $f(m)\gg \sqrt m$ (Erd\H{o}s--S\'ar\'anyi, 1959) and the questioned upper bound $f(m)\ll \sqrt m$ remain outside what I was able to prove from scratch here.

\subsection*{COMPLETION ESTIMATE}
$0.10$.

