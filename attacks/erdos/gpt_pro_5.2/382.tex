\section*{Problem \#382}

\subsection*{1) FORMAL RESTATEMENT}

Let $1\le u\le v$ be integers, and set
\[
N(u,v):=\prod_{m=u}^{v} m,\qquad Q:=P(N(u,v)).
\]
Assume the exponent of $Q$ in $N(u,v)$ is at least $2$, i.e.\ $\nu_Q(N(u,v))\ge 2$ (equivalently, $[u,v]$ is bad in the sense of Problem \#380).

\noindent\textbf{Question 382(a).} Must one have
\[
v-u \;=\; v^{o(1)}\quad\text{as }v\to\infty\text{ along such bad intervals?}
\]

\noindent\textbf{Question 382(b).} Can $v-u$ be arbitrarily large (i.e.\ do bad intervals exist with unbounded length)?

\subsection*{2) QUICK LITERATURE/CONTEXT CHECK}

The Erd\H{o}s Problems forum thread for \#382 records (from Erd\H{o}s--Graham, 1980) that a result of Ramachandra implies $v-u\le v^{1/2+o(1)}$, and that Cram\'er's conjecture on prime gaps would imply $v-u=v^{o(1)}$.\footnote{\url{https://www.erdosproblems.com/forum/thread/382} (accessed 2026-01-17).}
The same thread notes an example with $v-u=13$ (for a very large $u$), though this is presented informally.\footnote{Same source as above; see the comment by Stijn Cambie.}

A general unconditional input one can combine with the ``no primes in bad intervals'' lemma is the existence of primes in short intervals. For example, Baker--Harman--Pintz proved a weak form of the prime number theorem in intervals $[x-x^{0.525},x]$ for large $x$.\footnote{\url{https://arxiv.org/abs/1707.05437} (Alweiss--Luo, 2017, abstract cites Baker--Harman--Pintz).}

\subsection*{3) ATTACK PLAN}

\begin{enumerate}[leftmargin=2.5em]
\item Use the structural lemma from Problem \#380 to obtain unconditional restrictions (prime-free $\Rightarrow$ $v<2u$).
\item Combine prime-free-ness with known bounds for primes in short intervals to get an unconditional upper bound $v-u\ll v^{0.525}$ (still far from $v^{o(1)}$).
\item Search for explicit families/examples of bad intervals to gauge how large $v-u$ can get, in particular by looking near squares $q^2$ of a prime $q$ (since $q^2$ supplies $\nu_q\ge 2$).
\end{enumerate}

\subsection*{4) WORK}

\paragraph{Unconditional prime-free restriction.}
From Problem \#380 we have:

\begin{lemma}
If $[u,v]$ is bad then $[u,v]$ contains no primes, hence $v<2u$.
\end{lemma}

\paragraph{A further unconditional upper bound via primes in short intervals.}
Assume a theorem of the following kind: for all sufficiently large $x$, the interval $[x-x^{\theta},x]$ contains a prime, where $\theta=0.525$ is admissible by Baker--Harman--Pintz (as cited in Alweiss--Luo).\footnote{\url{https://arxiv.org/abs/1707.05437}}
Then any prime-free interval $[u,v]$ with $v$ large must satisfy
\[
v-u < v^{\theta},
\]
because otherwise $[v-v^{\theta},v]\subseteq [u,v]$ would contain a prime.
Consequently, for all sufficiently large bad intervals $[u,v]$,
\[
v-u < v^{0.525}.
\]
This improves the trivial bound $v-u < v$ but is still much weaker than the desired $v^{o(1)}$.

\paragraph{Explicit examples with growing $v-u$.}
One natural way to build bad intervals is to choose a prime $q$ and look for short runs of consecutive $q$-smooth integers (all prime factors $\le q$) that \emph{contain} $q^2$.
If $[u,v]$ is such a run containing $q^2$, then $P(N(u,v))=q$ and $\nu_q(N(u,v))\ge 2$ (because $q^2\mid N(u,v)$), hence $[u,v]$ is bad.

By direct computation one finds, for example:
\[
[1680,1683]\ \text{ is bad with }P(N)=41\text{ and }\nu_{41}(N)=2,
\]
so $v-u=3$.
Further examples with larger length include
\[
[76725,76729]\ \text{ bad with maximal prime }277,\quad (v-u=4),
\]
and
\[
[332925,332930]\ \text{ bad with maximal prime }577,\quad (v-u=5).
\]
These examples show unconditionally that lengths $v-u$ can exceed $3$ and at least reach $5$ for moderately sized $v$.

\subsection*{5) VERIFICATION}

\begin{itemize}[leftmargin=2em]
\item The deduction ``bad $\Rightarrow$ prime-free $\Rightarrow v<2u$'' is fully proved in Problem \#380.
\item The implication ``prime in $[x-x^{0.525},x]$ for all large $x$ $\Rightarrow v-u<v^{0.525}$ for prime-free $[u,v]$'' is a straightforward containment argument, conditional only on the short-interval prime theorem.
\item The explicit examples above were verified by exact integer factorisations:
in each interval, every prime factor of every term is $\le q$ and the term $q^2$ lies inside, forcing $\nu_q\ge 2$.
\end{itemize}

\subsection*{6) FINAL}

\textbf{UNRESOLVED.}

\begin{enumerate}[leftmargin=2.5em]
\item[(i)] \textbf{Strongest partial result obtained.}
Unconditionally, any bad interval contains no primes and thus satisfies $v<2u$. Combining with known results on primes in short intervals (Baker--Harman--Pintz with exponent $0.525$) yields the unconditional bound $v-u < v^{0.525}$ for all sufficiently large bad intervals.
We also exhibited explicit bad intervals with $v-u=5$.
\item[(ii)] \textbf{First hard gap.}
To upgrade $v-u < v^{0.525}$ to $v-u=v^{o(1)}$ would essentially require prime-free intervals to be much shorter than any fixed power of $v$, i.e.\ deep improvements in prime gap technology (or a different argument not mediated by primes).
\item[(iii)] \textbf{Top 3 next moves.}
\begin{enumerate}[leftmargin=2.5em]
\item Try to prove that in a bad interval the maximal prime must arise from a square $q^2$ inside the interval for ``most'' such intervals; then reduce to understanding runs of $q$-smooth integers near $q^2$.
\item Improve the unconditional exponent $0.525$ by importing the best currently known prime-in-short-interval exponents (or conditional exponents under RH/EH/Cram\'er) and translate directly to $v-u$ bounds.
\item For the ``arbitrarily large'' question: search systematically for long $q$-smooth runs containing $q^2$; develop heuristic predictions for their maximal length as $q\to\infty$ and attempt to turn them into rigorous existence via sieve bounds for smooth values.
\end{enumerate}
\item[(iv)] \textbf{Why it might be true.}
If primes appear ``often enough'' in intervals $[x,x+x^{\varepsilon}]$ for every fixed $\varepsilon>0$, then prime-free-ness forces $v-u=x^{o(1)}$. Since bad intervals are prime-free, this would yield the desired $v^{o(1)}$ bound.
\end{enumerate}

\subsection*{7) COMPLETION ESTIMATE}
Estimated probability of a full resolution from these notes alone: \textbf{20\%}.

%%%%%%%%%%%%%%%%%%%%%%%%%%%%%%%%%%%%%%%%%%%%%%%%%%%%%%%%%%%%%%%%%%%%%%%%%%%%%%%
