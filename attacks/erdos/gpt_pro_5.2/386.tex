\section*{Problem \#386}

\subsection*{FORMAL RESTATEMENT}
Let $p_1=2<p_2=3<p_3=5<\cdots$ denote the increasing sequence of primes.
Call an integer $N\ge 2$ a \emph{consecutive-prime product} if there exist indices $1\le i\le j$ such that
\[
N=\prod_{r=i}^j p_r.
\]
Equivalently, $N$ is squarefree and its prime divisors form an interval in the prime sequence.

The problem asks: are there infinitely many pairs of integers $(n,k)$ with $n\ge 4$ and $2\le k\le n-2$ such that the binomial coefficient $\binom{n}{k}$ is a consecutive-prime product?

\subsection*{QUICK LITERATURE/CONTEXT CHECK}
This is Erd\H{o}s problem \#386 in the Erd\H{o}s Problems database, presented as open.  The page gives the example $\binom{21}{2}=2\cdot 3\cdot 5\cdot 7$ and notes several small examples communicated by Weisenberg:
\[
\binom{7}{3}=5\cdot 7,\quad
\binom{10}{4}=2\cdot 3\cdot 5\cdot 7,\quad
\binom{14}{4}=7\cdot 11\cdot 13,\quad
\binom{15}{6}=5\cdot 7\cdot 11\cdot 13.
\]
For $k=2$, the known values of $n$ such that $\binom{n}{2}$ is a product of consecutive primes are $n=4,6,15,21,715$ (OEIS A280992).

\subsection*{ATTACK PLAN}
A consecutive-prime product is extremely rigid: it is squarefree and its prime factors lie in a very short interval on the logarithmic scale (no ``missing primes'').
Potential routes:
\begin{itemize}
\item Use known results on the least and greatest prime factors of $\binom{n}{k}$ to force the set of prime factors to be ``too spread out'' to be consecutive, at least for large $n$ in many $(n,k)$ ranges.
\item Exploit squarefreeness: show that for large $n$ and nontrivial $k$ some small prime must appear with exponent $\ge 2$ in $\binom{n}{k}$, contradicting squarefreeness.
\item Conversely, attempt constructive families (if they exist) by forcing $\binom{n}{k}$ to match a primorial segment (e.g.\ via diophantine constraints), though this seems highly nontrivial.
\end{itemize}

\subsection*{WORK}
We can at least formalise necessary conditions and record computational data.

\medskip\noindent
\textbf{Necessary condition 1 (squarefree).}
If $\binom{n}{k}$ is a consecutive-prime product, then $\binom{n}{k}$ is squarefree.

\begin{proof}
A product $\prod_{r=i}^j p_r$ has each prime appearing with exponent $1$.  Hence it is squarefree.
\end{proof}

\medskip\noindent
\textbf{Computational check for small $n$ (not a proof).}
Searching all $2\le k\le n-2$ for $n\le 200$ yields exactly the following solutions (up to the symmetry $k\leftrightarrow n-k$):
\[
\begin{array}{rcl}
\binom{4}{2}&=&2\cdot 3,\\
\binom{6}{2}&=&3\cdot 5,\\
\binom{7}{3}&=&5\cdot 7,\\
\binom{10}{4}&=&2\cdot 3\cdot 5\cdot 7,\\
\binom{14}{4}&=&7\cdot 11\cdot 13,\\
\binom{15}{2}&=&3\cdot 5\cdot 7,\\
\binom{15}{6}&=&5\cdot 7\cdot 11\cdot 13,\\
\binom{21}{2}&=&2\cdot 3\cdot 5\cdot 7.
\end{array}
\]
No further examples occur with $n\le 200$.  (In particular, for $k=2$ this recovers $n=4,6,15,21$; the next known $k=2$ example is at $n=715$.)

\subsection*{VERIFICATION}
The ``Necessary condition 1'' is immediate.
The listed small solutions were verified by direct factorisation and checking that the prime factors are distinct and consecutive in the prime sequence; symmetry $\binom{n}{k}=\binom{n}{n-k}$ explains paired solutions.

\subsection*{FINAL}
\textbf{UNRESOLVED.}

Most promising partial results obtained above:
\begin{itemize}
\item The property forces $\binom{n}{k}$ to be squarefree, providing a concrete obstruction to exploit.
\item A complete enumeration for $n\le 200$ produces only finitely many sporadic examples (listed), consistent with rarity.
\end{itemize}

Specific barrier:
\begin{itemize}
\item Current factor-distribution theorems for $\binom{n}{k}$ do not seem strong enough to rule out (or construct) infinitely many cases where the entire prime-support is a consecutive block.
\end{itemize}

Smallest missing step to a full resolution:
\begin{itemize}
\item Either (disproof) construct an infinite family $(n_i,k_i)$ with $\binom{n_i}{k_i}$ squarefree and with prime factors forming a consecutive block; or (proof of finiteness) show that for all sufficiently large $(n,k)$ with $2\le k\le n-2$, $\binom{n}{k}$ has either a repeated prime factor or at least two ``separated'' prime factors (missing a prime between them), contradicting the consecutive-prime condition.
\end{itemize}

\subsection*{COMPLETION ESTIMATE}
COMPLETION: 25\%

%%%%%%%%%%%%%%%%%%%%%%%%%%%%%%%%%%%%%%%%%%%%%%%%%%%%%%%%%%%%%%%%%%%%%%%%%%%%%%%
