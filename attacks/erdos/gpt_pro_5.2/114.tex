
\section*{Erd\H{o}s Problem \#114}

\subsection*{1) FORMAL RESTATEMENT}
Let $p(z)\in\mathbb C[z]$ be a monic polynomial of degree $n\ge1$.  Consider the lemniscate
\[
L(p):=\{z\in\mathbb C:\ |p(z)|=1\}.
\]
Assume $L(p)$ is a rectifiable curve (for polynomials it is a finite union of real-analytic arcs except at finitely many singular points).  Let $\ell(p)$ denote its total Euclidean arclength.  Define
\[
 f(n):=\sup\{\ell(p):\ p\text{ monic of degree }n\}.
\]
Question: is $\ell(p)$ maximized by $p(z)=z^n-1$?  (Up to the natural symmetries: rotation and translation as stated in the problem text.)

\subsection*{2) QUICK LITERATURE/CONTEXT CHECK}
The problem text itself records a sequence of bounds and partial results, culminating in the statement that Tao (2025) proved $z^n-1$ is the unique maximizer for all sufficiently large $n$.  I do not re-prove those results here.  I only compute the exact arclength for $p(z)=z^n-1$ and verify the claimed $2n+O(1)$ behavior.

\subsection*{3) ATTACK PLAN}
\begin{itemize}
\item Compute $\ell(z^n-1)$ in a closed form, to at least have the conjectured extremal value explicitly.
\item For small $n$ do a sanity check with explicit parametrizations.
\end{itemize}

\subsection*{4) WORK}
\paragraph{Lemma 114.1 (the case $n=1$ is trivial).}
If $n=1$ and $p(z)=z-a$ is monic linear, then $L(p)$ is the circle $|z-a|=1$ of length $2\pi$.  Hence $f(1)=2\pi$, and every monic linear polynomial is a maximizer.

\emph{Proof.}
$|p(z)|=1$ is exactly $|z-a|=1$.  The Euclidean circumference is $2\pi$. \qed

\paragraph{Lemma 114.2 (exact formula for $\ell(z^n-1)$).}
Let $p_n(z)=z^n-1$.  Then the arclength of $L(p_n)$ is
\[
\ell(p_n)=2^{1/n}\,\mathrm B\Big(\frac{1}{2n},\frac12\Big)
\]
where $\mathrm B$ is the Beta function.  In particular, $\ell(p_n)=2n+O(1)$ as $n\to\infty$.

\emph{Proof.}
On the unit circle, write $z=e^{it}$.  Then
\[
|p_n(e^{it})|=|e^{int}-1|=2\,\big|\sin\frac{nt}{2}\big|.
\]
The lemniscate $|z^n-1|=1$ is the preimage under the map $w=z^n$ of the circle $|w-1|=1$.  A convenient parametrization is obtained by parametrizing that circle:
\[
 w(\theta)=1+e^{i\theta},\qquad 0\le\theta<2\pi.
\]
Then $|w(\theta)-1|=1$, and $|w(\theta)|=|1+e^{i\theta}|=2|\cos(\theta/2)|$.  Choose a branch of the $n$th root and set
\[
 z(\theta)=w(\theta)^{1/n}.
\]
As $\theta$ runs $0\to2\pi$, $w(\theta)$ traverses the circle once and $z(\theta)$ traces all $n$ components of the lemniscate (the $n$ choices of the $n$th root correspond to rotations by $2\pi/n$, contributing equally to length).  Differentiating,
\[
 z'(\theta)=\frac1n w(\theta)^{1/n-1} w'(\theta)=\frac1n w(\theta)^{1/n-1} \cdot i e^{i\theta}.
\]
Hence
\[
|z'(\theta)|=\frac1n |w(\theta)|^{1/n-1}.
\]
Therefore the length contributed by one branch over $\theta\in[0,2\pi]$ is
\[
\int_0^{2\pi} |z'(\theta)|\,d\theta=\frac1n\int_0^{2\pi} |w(\theta)|^{1/n-1}\,d\theta.
\]
There are $n$ congruent branches (rotations), so the total length is
\[
\ell(p_n)=n\cdot \frac1n\int_0^{2\pi} |w(\theta)|^{1/n-1}\,d\theta=\int_0^{2\pi} \big(2|\cos(\theta/2)|\big)^{1/n-1}\,d\theta.
\]
Using symmetry $|\cos(\theta/2)|$ is even about $\theta=\pi$, we get
\[
\ell(p_n)=2\int_0^{\pi} \big(2\cos(\theta/2)\big)^{1/n-1}\,d\theta.
\]
Substitute $u=\theta/2$:
\[
\ell(p_n)=4\int_0^{\pi/2} (2\cos u)^{1/n-1}\,du = 4\,2^{1/n-1}\int_0^{\pi/2} (\cos u)^{1/n-1}\,du.
\]
Now apply the standard Beta integral
\[
\int_0^{\pi/2} (\cos u)^{\alpha-1}(\sin u)^{\beta-1}\,du = \frac12\,\mathrm B\Big(\frac\alpha2,\frac\beta2\Big)
\]
with $\alpha=1/n$ and $\beta=1$.  This yields
\[
\int_0^{\pi/2} (\cos u)^{1/n-1}\,du = \frac12\,\mathrm B\Big(\frac{1}{2n},\frac12\Big).
\]
Plugging in gives
\[
\ell(p_n)=4\,2^{1/n-1}\cdot \frac12\,\mathrm B\Big(\frac{1}{2n},\frac12\Big)=2^{1/n}\,\mathrm B\Big(\frac{1}{2n},\frac12\Big).
\]

For the asymptotic: using $\mathrm B(x,1/2)=\Gamma(x)\Gamma(1/2)/\Gamma(x+1/2)$ and $\Gamma(x)\sim 1/x$ as $x\to0^+$, we have
\[
\mathrm B\Big(\frac{1}{2n},\frac12\Big)\sim \frac{\Gamma(1/2)}{\Gamma(1/2)}\cdot 2n = 2n,
\]
and $2^{1/n}=1+O(1/n)$.  Thus $\ell(p_n)=2n+O(1)$.
\qed

\paragraph{FAST REALITY CHECK (numerical values for $\ell(z^n-1)$).}
Using the closed form in Lemma~114.2, the following values were computed (rounded):
\[
\begin{array}{c|c}
 n & \ell(z^n-1)\\\hline
 1 & 6.283185307\ (\approx 2\pi)\\
 2 & 7.416298709\\
 3 & 9.179724222\\
 4 & 11.07002052\\
 5 & 13.00681138\\
 6 & 14.96573219
\end{array}
\]
These are consistent with $\ell(z^n-1)\approx 2n+\text{constant}$.

\subsection*{5) VERIFICATION}
\begin{itemize}
\item Lemma~114.2 differentiates $z(\theta)=w(\theta)^{1/n}$; branch issues do not affect arclength because we integrate $|z'|$ and then sum over the $n$ rotationally congruent branches.
\item Endpoint singularities occur when $w(\theta)=0$ (i.e. $\theta=\pi$), but the integrand behaves like $|\cos(\theta/2)|^{1/n-1}$, which is integrable since $1/n-1>-1$ for all $n\ge1$.
\end{itemize}

\subsection*{6) FINAL}
\textbf{UNRESOLVED.}

(i) \emph{Strongest fully proved partial result obtained here.}
An exact closed form for the candidate extremal value $\ell(z^n-1)$ and its $2n+O(1)$ asymptotic (Lemma~114.2).

(ii) \emph{Exact first gap.}
A general upper bound $\ell(p)\le \ell(z^n-1)$ for every monic degree-$n$ polynomial $p$.

(iii) \emph{Top 3 next moves.}
\begin{enumerate}
\item Prove a variational inequality showing $z^n-1$ is a global maximizer among monic polynomials (beyond the local/asymptotic results cited in the problem text).
\item Establish sharp universal bounds on $\ell(p)$ in terms of $n$ with the correct leading constant $2$.
\item For small $n$ (e.g. $n=2,3$), attempt classification of maximizers by reducing to a finite-dimensional optimization problem.
\end{enumerate}

(iv) \emph{Minimal counterexample structure.}
A counterexample would be a monic polynomial $p$ of degree $n$ with $\ell(p)>\ell(z^n-1)$.  Such a $p$ would necessarily have a lemniscate with unusually large total curvature/length relative to the $n$-fold covering map $z\mapsto z^n$.


