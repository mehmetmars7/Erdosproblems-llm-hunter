
\subsection*{FORMAL RESTATEMENT}
Let $P\subset\mathbb{R}^2$ be a set of $n$ points with pairwise distances at least $1$:
\[
|x-y|\ge 1\quad\text{for all distinct }x,y\in P.
\]
Define a graph $G(P)$ with vertex set $P$ by joining $x,y\in P$ by an edge iff $|x-y|=1$.
Let
\[
 g(n) := \min_{|P|=n,\ \min_{x\ne y}|x-y|\ge 1}\ \alpha(G(P)),
\]
where $\alpha(G)$ is the independence number.
The problem asks for estimates on $g(n)$ and in particular the behavior of $g(n)/n$ as $n\to\infty$.

\subsection*{QUICK LITERATURE/CONTEXT CHECK}
The problem file states (i) Pollack noted that the four color theorem implies $g(n)\ge n/4$; (ii) constructions give $g(n)\le 6n/19$; (iii) Csizmadia gives $n/4\le g(n)\le (3n+7)/11$.  I will not use any external improvements beyond these statements; instead I prove the basic geometric planarity fact and the $n/4$ consequence.

\subsection*{ATTACK PLAN}
Show $G(P)$ is planar by drawing edges as straight segments and ruling out crossings using the distance-$\ge 1$ constraint.  Then apply 4-colorability to force an independent set of size at least $n/4$.  For an upper bound construction, exhibit configurations whose graphs are disjoint unions of triangles.

\subsection*{WORK}
\textbf{Lemma 1066.1 (crossing unit edges force a short distance).}
Let $A,B,C,D\in\mathbb{R}^2$ and suppose the closed line segments $\overline{AB}$ and $\overline{CD}$ intersect at a point $X$ that is interior to both segments.  If $|AB|=|CD|=1$, then at least one of the four distances $|AC|,|AD|,|BC|,|BD|$ is strictly less than $1$.

\emph{Proof.}
Because $X$ is interior to $\overline{AB}$, we have $|AX|>0$, $|BX|>0$, and $|AX|+|BX|=|AB|=1$. Hence $\min(|AX|,|BX|)\le 1/2$.  Similarly, $\min(|CX|,|DX|)\le 1/2$.
Choose $P\in\{A,B\}$ so that $|PX|\le 1/2$ and choose $Q\in\{C,D\}$ so that $|QX|\le 1/2$.
Then by the triangle inequality,
\[
|PQ| \le |PX|+|XQ| \le \tfrac12+\tfrac12 = 1.
\]
If equality held, then we would have $P,X,Q$ collinear with $X$ between $P$ and $Q$ and also $|PX|=|XQ|=1/2$.  But $P,X$ lies on the line through $A,B$ and $Q,X$ lies on the line through $C,D$.  Since $X$ is an interior intersection point of the two segments, these two lines are not the same line. Hence $P,X,Q$ cannot be collinear, so the inequality is strict and $|PQ|<1$.
Thus one of $|AC|,|AD|,|BC|,|BD|$ is $<1$. \qed

\textbf{Lemma 1066.2 (planarity of $G(P)$).}
For any $P$ as in the problem, the straight-line drawing of $G(P)$ with vertices at the points of $P$ and edges drawn as straight segments has no crossings; in particular $G(P)$ is planar.

\emph{Proof.}
Assume for contradiction that two edges $xy$ and $uv$ of $G(P)$ cross in their interiors when drawn as straight segments. Then $|x-y|=|u-v|=1$ by definition of the edges, and the two segments intersect in their interiors.  By Lemma~1066.1, one of the distances between an endpoint of the first edge and an endpoint of the second edge is strictly less than $1$.  This contradicts the hypothesis that all pairwise distances among points of $P$ are at least $1$.  Therefore no two edges cross, and the drawing is planar. \qed

\textbf{Lemma 1066.3 ($n/4$ independent set lower bound).}
For all $n\ge 1$, $g(n)\ge \lceil n/4\rceil$.

\emph{Proof.}
Fix any admissible $P$ with $|P|=n$. By Lemma~1066.2, $G(P)$ is planar. By the four color theorem, $G(P)$ admits a proper vertex-coloring with at most $4$ colors.  In any $4$-coloring, the largest color class has size at least $\lceil n/4\rceil$ and is an independent set. Hence $\alpha(G(P))\ge \lceil n/4\rceil$.  Taking the minimum over all admissible $P$ gives $g(n)\ge \lceil n/4\rceil$. \qed

\textbf{Lemma 1066.4 (triangle construction gives $\lceil n/3\rceil$ upper bound).}
For each $n\ge 1$, there exists an admissible point set $P$ of size $n$ for which $\alpha(G(P))=\lceil n/3\rceil$. In particular $g(n)\le \lceil n/3\rceil$.

\emph{Proof.}
Write $n=3m+r$ with $r\in\{0,1,2\}$. Place $m$ disjoint equilateral triangles of side length $1$ very far apart from each other (e.g. translate the $i$th triangle by a vector of length $>10$ from the origin and from all previous triangles). This ensures:
\begin{itemize}
\item within each triangle, all three pairwise distances are $1$;
\item between points from different triangles, all distances are $>1$;
\item hence the minimum pairwise distance is $1$.
\end{itemize}
If $r>0$, add $r$ more points far from everything else and also far from each other (distance $>1$), so they are isolated vertices.
In the resulting graph, each triangle forms a $K_3$ component and each extra point is isolated.  Any independent set can contain at most one vertex from each triangle, and it can contain all isolated vertices. Therefore
\[
\alpha(G(P)) = m + r = \lceil n/3\rceil.
\]
This proves existence of a configuration with independence number $\lceil n/3\rceil$, hence $g(n)\le \lceil n/3\rceil$. \qed

\textbf{FAST REALITY CHECK (small $n$).}
From the definitions:
\begin{itemize}
\item $g(1)=1$.
\item $g(2)=1$ since two points at distance $1$ give a single edge.
\item $g(3)=1$ using an equilateral triangle of side $1$.
\item $g(4)=2$: the lower bound gives $g(4)\ge 1$, but $g(4)\ge 2$ because $\alpha(G)=1$ would force $G\cong K_4$, impossible here; and the unit square configuration gives $\alpha=2$ so $g(4)\le 2$.
\item $g(5)=2$: lower bound $\lceil 5/4\rceil=2$ and the construction ``triangle + edge far away'' gives $\alpha=2$.
\item $g(6)=2$: lower bound $\lceil 6/4\rceil=2$ and two disjoint triangles give $\alpha=2$.
\end{itemize}

\subsection*{VERIFICATION}
Lemma~1066.1 was proved purely from triangle inequality and the geometry of an interior crossing.
Lemma~1066.2 uses Lemma~1066.1 and the minimum distance hypothesis.
Lemma~1066.3 uses only the four color theorem (explicitly mentioned in the problem file) and an averaging argument over color classes.
Lemma~1066.4 checks all distance constraints by placing components far apart.
The stated small-$n$ values are justified by explicit configurations and simple impossibility arguments.

\subsection*{FINAL}
\textbf{UNRESOLVED.}

(i) \emph{Strongest proved partial result:}
Every such graph $G(P)$ is planar (Lemma~1066.2), hence $\alpha(G(P))\ge \lceil n/4\rceil$ and therefore $g(n)\ge \lceil n/4\rceil$ (Lemma~1066.3).  A simple explicit construction shows $g(n)\le \lceil n/3\rceil$ (Lemma~1066.4).

(ii) \emph{First gap (crisp statement):}
Determine the true asymptotic constant $\lim_{n\to\infty} g(n)/n$ (if it exists), or improve the current interval between the general lower bound $1/4$ and known constructions.

(iii) \emph{Top 3 next moves (concrete lemmas/computations):}
\begin{enumerate}
\item Attempt to exploit additional geometric structure beyond planarity (e.g. constraints on local configurations in a unit-circle packing/contact graph) to improve the $n/4$ bound.
\item For upper bounds, search for explicit periodic point configurations with minimum distance $\ge 1$ and small independence ratio; a concrete step is to optimize such configurations numerically in bounded windows and then periodize.
\item Compute exact $g(n)$ for small $n$ by enumerating feasible contact graphs of unit disk packings (penny graphs) and taking the minimum independence number among realizable graphs.
\end{enumerate}

(iv) \emph{Minimal counterexample structure:}
A counterexample to any proposed bound $g(n)\ge c n$ would be an explicit admissible configuration $P$ whose unit-distance graph has independence number $<c n$.  Such a configuration must in particular realize a planar graph with unusually small independence ratio and respect the geometric ``no crossings'' constraint of Lemma~1066.2.


