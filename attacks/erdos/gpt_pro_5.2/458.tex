
\noindent\textbf{1) FORMAL RESTATEMENT}

Let $p_k$ denote the $k$th prime (so $p_1=2,p_2=3,\dots$).  For each integer $n\ge 1$ write
\[[1,\dots,n]:=\mathrm{lcm}(1,2,\dots,n).
\]
The problem asks whether for \emph{all} integers $k\ge 1$ one has
\[[1,\dots,p_{k+1}-1]\ <\ p_k\,[1,\dots,p_k].\]

\noindent\textbf{2) QUICK LITERATURE/CONTEXT CHECK}

The problem text itself states that Erd\H{o}s and Graham considered this ``almost certainly'' true but difficult, due to the possibility of many primes $q$ with $p_k<q^2<p_{k+1}$ and due to small-prime effects. I did not find (and do not use) any stated proof/disproof in the extracted file.

\noindent\textbf{3) ATTACK PLAN}

\begin{itemize}
\item Rewrite $[1,\dots,n]$ via prime powers, so that the ratio
$[1,\dots,p_{k+1}-1]/[1,\dots,p_k]$
becomes a product over prime powers that appear between $p_k$ and $p_{k+1}-1$.
Then the desired inequality becomes a bound on that product.
\item Try to bound the ratio by splitting contributions from (a) prime squares $q^2$ in the interval, and (b) higher powers of small primes (e.g. $2^a,3^a,\dots$) that may cross from $\le p_k$ to $\le p_{k+1}-1$.
\item Disproof attempt: search computationally for a $k$ where many such prime powers accumulate and make the ratio $\ge p_k$.
\end{itemize}

\noindent\textbf{4) WORK}

\textbf{Fast reality check (computations).}

I checked the inequality for $1\le k\le 300$ by direct computation of $\mathrm{lcm}(1,\dots,n)$ up to $n=p_{301}-1$. No counterexample was found in this range.

\medskip
\noindent\textbf{Lemma 458.1 (prime-power factorisation of $[1,\dots,n]$).}
For every integer $n\ge 1$,
\[[1,\dots,n] = \prod_{p\le n} p^{\alpha_p(n)},\qquad \alpha_p(n):=\max\{a\ge 0: p^a\le n\}=\left\lfloor \log_p n\right\rfloor.
\]

\textit{Proof.}
Write the prime factorisation of each integer $m\le n$ as $m=\prod_{p} p^{v_p(m)}$.
The least common multiple $[1,\dots,n]$ has, for each prime $p$, exponent equal to the maximum $\max_{1\le m\le n} v_p(m)$. This maximum equals the largest $a$ for which $p^a\le n$, because $p^a$ itself is in $\{1,\dots,n\}$ and has $v_p(p^a)=a$, while any $m\le n$ satisfies $v_p(m)\le a$ whenever $p^{a+1}>n$. Therefore the exponent is $\alpha_p(n)=\lfloor\log_p n\rfloor$, giving the claimed product over primes $p\le n$. \hfill$\square$

\medskip
\noindent\textbf{Lemma 458.2 (ratio as product over prime powers in the gap).}
Let $k\ge 1$. Then
\[\frac{[1,\dots,p_{k+1}-1]}{[1,\dots,p_k]} = \prod_{\substack{p\le p_k\\ \exists a\ge 1\text{ with }p_k < p^a\le p_{k+1}-1}} p.
\]
Equivalently, the ratio is the product (with multiplicity one) of those primes $p\le p_k$ for which the maximal power of $p$ not exceeding $p_{k+1}-1$ is strictly larger than the maximal power of $p$ not exceeding $p_k$.

\textit{Proof.}
By Lemma~458.1,
\[[1,\dots,p_{k+1}-1]=\prod_{p\le p_{k+1}-1} p^{\alpha_p(p_{k+1}-1)},\qquad [1,\dots,p_k]=\prod_{p\le p_k} p^{\alpha_p(p_k)}.
\]
Since $p_{k+1}$ is the next prime after $p_k$, the primes $\le p_{k+1}-1$ are exactly the primes $\le p_k$, so the quotient is
\[\prod_{p\le p_k} p^{\alpha_p(p_{k+1}-1)-\alpha_p(p_k)}.
\]
Now $\alpha_p(m)$ increases by $1$ exactly when $m$ crosses a prime power $p^a$.
Thus $\alpha_p(p_{k+1}-1)-\alpha_p(p_k)$ counts the number of exponents $a$ for which $p_k < p^a\le p_{k+1}-1$.
But for fixed $p$, if there exists at least one such $a$, then multiplying by $p$ once accounts for increasing the exponent by $1$; and since the quotient only needs the total exponent difference, the quotient can be written as the product of $p$ repeated exactly that many times.  In particular, when there is at least one such $a$ the quotient picks up a factor $p$. This yields the displayed product. \hfill$\square$

\medskip
\noindent\textbf{Reformulation of the target inequality.}
Lemma~458.2 shows the desired inequality is equivalent to
\[\prod_{\substack{p\le p_k\\ \exists a\ge 1:~ p_k < p^a\le p_{k+1}-1}} p\ <\ p_k.\]
So any counterexample must have the product of the ``new'' prime bases (coming from prime powers in $(p_k,p_{k+1})$) at least $p_k$.

\medskip
\noindent\textbf{5) VERIFICATION}

\begin{itemize}
\item Lemma~458.1: verified that for each prime $p$ the exponent is $\max_{m\le n} v_p(m)$ and this equals $\lfloor\log_p n\rfloor$.
\item Lemma~458.2: verified the step ``primes up to $p_{k+1}-1$ equal primes up to $p_k$'' uses only the definition of consecutive primes.
\item Computation: I independently checked small $k$ by hand: for $k=1$, $[1]=1<2\cdot [1,2]=2$; for $k=2$, $[1,2]=2<3\cdot [1,3]=6$, etc.
\end{itemize}

\noindent\textbf{6) FINAL}

\textbf{UNRESOLVED}

(i) \emph{Strongest proved partial result:} The inequality is equivalent to a concrete statement about prime powers in the gap $(p_k,p_{k+1})$:
\[[1,\dots,p_{k+1}-1] < p_k[1,\dots,p_k]\iff \prod_{\substack{p\le p_k\\ \exists a:~ p_k<p^a\le p_{k+1}-1}} p < p_k.
\]
Additionally, the inequality holds for all $1\le k\le 300$ by direct computation.

(ii) \emph{First gap (crisp):} Prove that for every $k$ the product of primes $p\le p_k$ whose next prime power crosses from $\le p_k$ to $\le p_{k+1}-1$ is always $<p_k$.

(iii) \emph{Top 3 next moves:}
\begin{enumerate}
\item Isolate the contribution from prime squares: show that there cannot be ``too many'' primes $q$ with $p_k<q^2<p_{k+1}$, or else their product forces a violation.
\item Control small-prime powers (e.g. powers of $2,3,5$) by proving that not too many of them can lie in $(p_k,p_{k+1})$ simultaneously.
\item Extend computation to larger $k$ and inspect near-extremal cases by logging which prime powers $p^a$ fall in the gap.
\end{enumerate}

(iv) \emph{Minimal counterexample structure:} A minimal $k$ violating the inequality would require a prime gap $(p_k,p_{k+1})$ that contains an unusually large collection of prime powers $p^a$ (especially many prime squares $q^2$ just above $p_k$), such that the product of their bases $\ge p_k$.


