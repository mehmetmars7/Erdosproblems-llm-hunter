\section{Erd\H{o}s Problem \#44 --- Round 2}

\subsection*{1) ROUND-2 OBJECTIVE}
\textbf{Path (A): proof.}
Round 1 isolated the key gap: Lemma~1 and Lemma~2 only add $O(\log M)$ new elements, whereas the target size is $\asymp \sqrt{M}$.
In this round I push the \emph{block extension} idea (Round-1 ``next move'' (2)) to obtain an unconditional constant-density extension.
Specifically, I prove a gap-free theorem showing that every Sidon $A\subset [N]$ can be extended to a Sidon subset of some $[M]$ of size at least $(1/\sqrt2-\varepsilon)\sqrt M$.
This does \emph{not} settle the original $(1-\varepsilon)\sqrt M$ target, but it is a strict quantitative advance.

\subsection*{2) ROUND-1 FOUNDATION USED}
I use the Round-1 formal restatement of the problem and the Round-1 identification of the main gap ("need $\Theta(\sqrt M)$ new elements without $M$ exponential").
I do \emph{not} reuse Lemma~1 or Lemma~2 directly; instead I replace the exponential-growth tail with a dense block.

\subsection*{3) NEW INSIGHT / TOOL (ROUND-2)}
New ingredients beyond Round 1:
\begin{itemize}
\item \textbf{Sum-range separation.} Place the new set $B$ inside a top interval of $[M]$ so that
\[A+A \;<\; A+B \;<\; B+B\]
(as sets of integers). This deletes the hardest mixed collisions (e.g. $a+b=b_1+b_2$) by sheer range separation.
\item \textbf{Forbidden-difference pruning lemma.} Starting from any Sidon set $T$, delete at most $|F|$ elements to ensure that the remaining subset has no pairwise differences lying in a prescribed finite forbidden set $F$.
\item \textbf{Dense Sidon blocks from finite geometry.} Use Singer's construction of a perfect difference set (hence Sidon set) of size $q+1$ inside $[q^2+q+1]$ for every prime power $q$.
\end{itemize}

\subsection*{4) ATTACK PLAN (ROUND-2)}
\textbf{Gap from Round 1.} We need a mechanism to add \emph{many} elements in $\{N+1,\dots,M\}$ while preserving Sidon-ness.

\textbf{Plan.} Fix $A\subset [N]$ Sidon and choose a large length $L$ and an interval
\[I:=[N+L+1,\,N+2L]\subset [M],\qquad M:=N+2L.
\]
We will place $B\subset I$ so that:
\begin{enumerate}
\item $B$ itself is Sidon;
\item no nonzero difference of two elements of $B$ equals a difference of two elements of $A$;
\item because $I$ sits far above $[N]$, the sum-ranges $A+A$, $A+B$, $B+B$ are disjoint.
\end{enumerate}
Then $A\cup B$ is Sidon by a direct case analysis.
Choosing $L\approx q^2$ with $q$ large yields $|B|\approx \sqrt L$ and hence $|A\cup B|\approx \sqrt{M/2}=(1/\sqrt2)\sqrt M$.

\subsection*{5) WORK (ROUND-2)}
\subsubsection*{5.1\; A Sidon set has unique differences}
\begin{lemma}[Differences are unique in a Sidon set]
\label{lem:unique-differences}
Let $S\subset \mathbb Z$ be Sidon.
If $x_1-x_2=x_3-x_4\neq 0$ with $x_i\in S$, then $(x_1,x_2)=(x_3,x_4)$.
Equivalently, for each nonzero integer $d$ there is at most one ordered pair $(x,y)\in S^2$ with $x-y=d$.
\end{lemma}
\begin{proof}
From $x_1-x_2=x_3-x_4$ we obtain $x_1+x_4=x_3+x_2$.
By Sidon-ness of $S$, the unordered multisets of summands are equal:
\[\{x_1,x_4\}=\{x_3,x_2\}.
\]
If $x_1=x_2$ then the difference is $0$, excluded, so $x_1\neq x_2$.
Thus the only way the multisets can coincide is $x_1=x_3$ and $x_4=x_2$, which forces $x_2=x_4$.
Hence $(x_1,x_2)=(x_3,x_4)$.
\end{proof}

\subsubsection*{5.2\; Deleting a small set to avoid forbidden differences}
\begin{lemma}[Forbidden-difference pruning]
\label{lem:prune}
Let $S\subset \mathbb Z$ be a finite Sidon set and let $F\subset \mathbb N$ be any finite set.
Then there exists a subset $S'\subseteq S$ such that
\begin{enumerate}
\item for all distinct $u,v\in S'$ we have $|u-v|\notin F$; and
\item $|S'|\ge |S|-|F|$.
\end{enumerate}
\end{lemma}
\begin{proof}
Form a (simple) graph $G$ with vertex set $S$, joining two distinct vertices $u,v$ by an edge if $|u-v|\in F$.
Fix $d\in F$. By Lemma~\ref{lem:unique-differences}, there is at most one unordered pair $\{u,v\}\subset S$ with $|u-v|=d$.
Hence for each $d\in F$ there is at most one edge of $G$ labelled by $d$, and therefore
\[|E(G)|\le |F|.
\]
Now pick a set $C\subseteq S$ by selecting, for each edge of $G$, \emph{one} of its endpoints, and taking the union over edges.
Then $C$ is a vertex cover of $G$ and $|C|\le |E(G)|\le |F|$.
Let $S':=S\setminus C$.
By construction $S'$ spans no edges in $G$, hence has no forbidden differences.
Also $|S'|=|S|-|C|\ge |S|-|F|$.
\end{proof}

\subsubsection*{5.3\; A dense Sidon block (Singer)}
\begin{theorem}[Singer]
\label{thm:singer}
For every prime power $q$ there exists a set $D\subset \mathbb Z/(q^2+q+1)\mathbb Z$ of size $q+1$ such that every nonzero residue class has a unique representation as a difference of two elements of $D$.
In particular, $D$ is a Sidon set in the cyclic group $\mathbb Z/(q^2+q+1)\mathbb Z$.
Choosing representatives in $\{0,1,\dots,q^2+q\}$ gives a Sidon set of integers of size $q+1$ contained in $[0,q^2+q]$.
\end{theorem}
\begin{proof}[Comment]
This is Singer's construction of a perfect difference set; we use only its existence.
The implication ``perfect difference set $\Rightarrow$ Sidon'' follows from rewriting $a+b=c+d$ as $a-c=d-b$ and invoking uniqueness of differences.
If a set is Sidon in $\mathbb Z/m\mathbb Z$, then it is Sidon as a set of integers when represented in $\{0,\dots,m-1\}$, because an integer equality implies the same congruence mod $m$.
\end{proof}

\subsubsection*{5.4\; Main theorem: a universal $1/\sqrt2$-density extension}
\begin{theorem}[Uniform extension with constant $1/\sqrt2$]
\label{thm:main}
Let $N\ge 1$ and let $A\subset [N]$ be Sidon.
Then for every $\varepsilon>0$ there exist $M\ge N$ and $B\subset\{N+1,\dots,M\}$ such that $A\cup B\subset [M]$ is Sidon and
\[
|A\cup B|\ge \Bigl(\frac1{\sqrt2}-\varepsilon\Bigr)\sqrt M.
\]
\end{theorem}
\begin{proof}
We may assume $0<\varepsilon<1/\sqrt2$ (otherwise the lower bound is nonpositive and the claim is trivial).

\medskip
\noindent\textbf{Step 1: choose a dense Sidon block length.}
Let $q$ be a prime power satisfying
\begin{equation}
\label{eq:qchoice}
q\ge \max\Bigl\{\frac N2,\ \frac{N-1}{\sqrt2\,\varepsilon}-1\Bigr\}.
\end{equation}
(Such $q$ exists, e.g. take $q=2^m$ for $m$ large.)
Set
\[L:=q^2+q+1,\qquad M:=N+2L.
\]
By Theorem~\ref{thm:singer} there exists a Sidon set $T\subset [L]$ of size $|T|=q+1$.

\medskip
\noindent\textbf{Step 2: forbid the differences coming from $A$.}
Define the forbidden absolute difference set
\[F_A:=\{ |a-a'| : a,a'\in A,\ a\neq a'\}\subset \{1,2,\dots,N-1\}.
\]
In particular $|F_A|\le N-1$.
Apply Lemma~\ref{lem:prune} to $S=T$ and $F=F_A$ to obtain $T'\subseteq T$ such that:
\begin{enumerate}
\item for all distinct $u,v\in T'$ we have $|u-v|\notin F_A$; and
\item $|T'|\ge |T|-|F_A|\ge (q+1)-(N-1)=q-N+2$.
\end{enumerate}

\medskip
\noindent\textbf{Step 3: place the block in the top half of $[M]$.}
Define
\[B:=\{t+(N+L):\ t\in T'\} \subset [N+L+1,\,N+2L]=[N+L+1,\,M].
\]
Then $B\subset\{N+1,\dots,M\}$, and $B$ is Sidon because it is a translate of the Sidon set $T'$.

\medskip
\noindent\textbf{Step 4: verify $A\cup B$ is Sidon (sum-range separation + cross-sum injectivity).}
Write $S:=A\cup B$.
We compare the ranges of the three kinds of sums:
\begin{align*}
A+A &\subset [2,\,2N],\\
A+B &\subset [\min(A)+\min(B),\ \max(A)+\max(B)]\subset [1+(N+L+1),\ N+(N+2L)]\\
&\subset [N+L+2,\ 2N+2L],\\
B+B &\subset [2\min(B),\ 2\max(B)]\subset [2(N+L+1),\ 2(N+2L)]\\
&= [2N+2L+2,\ 2N+4L].
\end{align*}
Since $L> N$ (indeed $L=q^2+q+1> N$ since $q\ge N/2$), these three intervals are pairwise disjoint and ordered as
\[A+A\;<\;A+B\;<\;B+B.
\]
Therefore, any solution of $x_1+x_2=x_3+x_4$ with $x_i\in S$ must lie entirely within one category.
The $A+A$ and $B+B$ categories are trivial because $A$ and $B$ are Sidon.
It remains to check $A+B$ sums are distinct.

Suppose $a_1+b_1=a_2+b_2$ with $a_i\in A$ and $b_i\in B$.
Rearranging gives $b_2-b_1=a_1-a_2$.
If $a_1\neq a_2$, then $|a_1-a_2|\in F_A$; taking absolute values gives
\[|b_2-b_1|=|a_1-a_2|\in F_A.
\]
But $b_i$ are translates of elements of $T'$, and $T'$ was constructed to avoid all differences in $F_A$.
Hence this is impossible.
Therefore $a_1=a_2$, and then the equality forces $b_1=b_2$.
So all $A+B$ sums are distinct, completing the Sidon verification.

\medskip
\noindent\textbf{Step 5: size lower bound.}
We have $|A\cup B|=|A|+|B|\ge |B|=|T'|\ge q-N+2$.
Also, using $N\le 2q$ from \eqref{eq:qchoice},
\[
M=N+2(q^2+q+1)=2q^2+2q+(N+2)\le 2q^2+4q+2=2(q+1)^2,
\]
so $\sqrt M\le \sqrt2\,(q+1)$.
Thus
\[
\Bigl(\frac1{\sqrt2}-\varepsilon\Bigr)\sqrt M
\le \Bigl(\frac1{\sqrt2}-\varepsilon\Bigr)\sqrt2\,(q+1)=(1-\sqrt2\,\varepsilon)(q+1).
\]
Therefore it suffices to prove $q-N+2\ge (1-\sqrt2\,\varepsilon)(q+1)$.
But this inequality is equivalent to
\[\sqrt2\,\varepsilon\,(q+1)\ge N-1,
\]
which holds by the choice of $q$ in \eqref{eq:qchoice}.
This proves the theorem.
\end{proof}

\subsection*{6) ADVERSARIAL VERIFICATION}
\begin{itemize}
\item \textbf{Repeated summands.} The Sidon definition allows $x_1=x_2$.
In the range separation check, $2b\in B+B$ is still $\ge 2(N+L+1)=2N+2L+2$, strictly larger than any $a+b\le 2N+2L$, so no hidden overlap.
\item \textbf{Cross-sums among themselves.} The only way $a_1+b_1=a_2+b_2$ could fail to force $a_1=a_2$ is if $b_1-b_2$ equals a nonzero difference of $A$.
This is exactly what Lemma~\ref{lem:prune} eliminates.
\item \textbf{Cardinality of $F_A$.} We used only $F_A\subset\{1,\dots,N-1\}$, hence $|F_A|\le N-1$ (no bound on $|A|$ is needed).
\item \textbf{Quantifiers/dependence.} The construction chooses $M$ and $B$ depending on $(N,A,\varepsilon)$, as allowed.
\item \textbf{Extremal small-$N$ cases.} When $N=1$, $F_A=\varnothing$ and the construction works verbatim.
If $A=\varnothing$, then $F_A=\varnothing$ and we simply place a dense Sidon block $B$ in the top interval.
\end{itemize}

\subsection*{7) FINAL}
\textbf{UNRESOLVED (BUT STRICTLY ADVANCED).}
The original problem asks for extension to size $(1-\varepsilon)\sqrt M$.
In Round 2 we proved the following unconditional quantitative extension theorem:
for every Sidon $A\subset [N]$ and every $\varepsilon>0$ there exist $M$ and $B\subset\{N+1,\dots,M\}$ such that $A\cup B$ is Sidon and
\[|A\cup B|\ge \bigl(\tfrac1{\sqrt2}-\varepsilon\bigr)\sqrt M.\]
The structural reason for the $1/\sqrt2$ loss is that we enforced $A+B$ and $B+B$ to lie in disjoint ranges; improving the constant beyond $1/\sqrt2$ requires handling those mixed collisions rather than separating them.

\subsection*{8) COMPLETION ESTIMATE (MANDATORY)}
\textbf{COMPLETION: 55\%}

\subsection*{9) REFERENCES}
\begin{itemize}
\item J. Singer, \emph{A theorem in finite projective geometry and some applications to number theory}, Trans. Amer. Math. Soc. \textbf{43} (1938), 377--385.
\end{itemize}
