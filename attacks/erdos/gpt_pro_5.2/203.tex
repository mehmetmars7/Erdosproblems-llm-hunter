\section*{Erd\H{o}s Problem \#203}
\addcontentsline{toc}{section}{Erd\H{o}s Problem \#203}

\subsection*{1. FORMAL RESTATEMENT}
Does there exist an integer $m$ with $\gcd(m,6)=1$ such that
\[
2^k 3^\ell m + 1\ \text{is \emph{never prime} for any integers }k,\ell\ge 0?
\]
Equivalently: is there $m$ coprime to $6$ such that every number in the set
\[
\{\,2^k 3^\ell m + 1:\ k,\ell\in\mathbb{N}\cup\{0\}\,\}
\]
is composite?

\subsection*{2. PHASE 1 --- FAST REALITY CHECK}
\begin{itemize}
\item \textbf{Immediate simplification.}
If $\gcd(m,6)=1$ then $m$ is odd. For $k=0$ we have $3^\ell m+1$ even.
If additionally $m>1$, then $3^\ell m+1\ge m+1\ge 4$, so it is composite.
Thus the only potentially prime values occur for $k\ge 1$.

\item \textbf{Computation (evidence, not a proof).}
A brute-force search over $m\le 10{,}000$ with $\gcd(m,6)=1$ found that for \emph{every} such $m$ there exists a pair $(k,\ell)$ with
\[
0\le k\le 4,\qquad 0\le \ell\le 10,
\]
for which $2^k3^\ell m+1$ is prime.
In other words: no counterexample exists below $10{,}000$ that avoids primes even in this small exponent window.
\end{itemize}

\subsection*{3. KEY DEFINITIONS / LEMMAS}
\begin{enumerate}
\item \textbf{CRT gadget for forcing compositeness on finitely many pairs.}
Given a finite set $S\subseteq\mathbb{N}_0^2$ of exponent pairs, one can choose congruences on $m$ so that for every $(k,\ell)\in S$ the number $2^k3^\ell m+1$ is divisible by a prescribed prime square $p^2$ (hence composite).

\item \textbf{Finite-box lemma (proved below).}
For any integers $K,L\ge 0$ there exists $m$ with $\gcd(m,6)=1$ such that
$2^k3^\ell m+1$ is composite for all $1\le k\le K$ and $0\le\ell\le L$.
\end{enumerate}

\subsection*{4. WORK (Main proof or counterexample)}
\paragraph{Lemma 3.1 (finite-set forcing via CRT).}
\emph{Claim.} Let $S\subseteq\mathbb{N}_0^2$ be finite. Then there exists an integer $m$ with $\gcd(m,6)=1$ such that for all $(k,\ell)\in S$ the integer $2^k3^\ell m+1$ is composite.

\emph{Proof.}
Choose distinct primes $p_{(k,\ell)}>3$ indexed by $(k,\ell)\in S$.
For each such pair let
\[
M_{(k,\ell)} := p_{(k,\ell)}^2.
\]
Because $p_{(k,\ell)}>3$, we have $\gcd(2^k3^\ell, M_{(k,\ell)})=1$, so the inverse
$(2^k3^\ell)^{-1}\pmod{M_{(k,\ell)}}$ exists.
Impose the congruence
\[
 m \equiv -\bigl(2^k3^\ell\bigr)^{-1} \pmod{M_{(k,\ell)}}.
\]
Then $2^k3^\ell m+1\equiv 0\pmod{p_{(k,\ell)}^2}$, hence $2^k3^\ell m+1$ is divisible by $p_{(k,\ell)}^2$ and therefore composite.
Finally also impose $m\equiv 1\pmod 6$ to guarantee $\gcd(m,6)=1$.
All moduli $6$ and $p_{(k,\ell)}^2$ are pairwise coprime, so by the Chinese Remainder Theorem there exists an $m$ satisfying all congruences simultaneously.
This $m$ has the desired property. \qed

\paragraph{Concrete example (a killed finite box).}
Take the box $\{(k,\ell): 1\le k\le 2,\ 0\le\ell\le 2\}$ and assign primes
\[
(1,0)\mapsto 5,\ (1,1)\mapsto 7,\ (1,2)\mapsto 11,\ (2,0)\mapsto 13,\ (2,1)\mapsto 17,\ (2,2)\mapsto 19.
\]
Solving the CRT system
\[
 m\equiv 1\pmod 6,\qquad m\equiv -\bigl(2^k3^\ell\bigr)^{-1}\pmod{p^2}\ \text{for the above six pairs},
\]
one obtains (one solution)
\[
 m=5904110274637.
\]
Then for each listed $(k,\ell)$, the corresponding value $2^k3^\ell m+1$ is divisible by the specified $p^2$ and hence composite.
(Values with $k=0$ are even, hence also composite for this odd $m>1$.)

\subsection*{5. SANITY CHECK}
\begin{itemize}
\item The CRT construction is robust because forcing divisibility by $p^2$ prevents the ``degenerate'' case where the forced divisor equals the whole number.
\item The concrete $m$ above satisfies $\gcd(m,6)=1$ (since $m\equiv 1\pmod 6$).
\item Direct modular verification for the example:
for each assigned prime $p$ we have $2^k3^\ell m+1\equiv 0\pmod{p^2}$ by construction.
\end{itemize}

\subsection*{6. FINAL}
\textbf{UNRESOLVED.}

\begin{itemize}
\item[(i)] \textbf{Best partial results proved here.}
We proved a strong finite obstruction-free statement: for any finite set of exponent pairs $S$, there exists $m$ (even with $\gcd(m,6)=1$) such that all $2^k3^\ell m+1$ for $(k,\ell)\in S$ are composite.
We also noted the simplification that, for odd $m>1$, all values with $k=0$ are automatically composite.

\item[(ii)] \textbf{First gap/obstruction to a full solution.}
The CRT argument cannot handle \emph{infinitely many} pairs $(k,\ell)$, because it requires imposing infinitely many congruences (or else finding a finite ``covering'' mechanism that repeats periodically in $(k,\ell)$).
Thus the core difficulty is to either:
\begin{enumerate}
\item construct a genuine infinite covering mechanism (analogous to covering congruences for Sierpi\'nski numbers), or
\item prove that no such $m$ can exist (i.e. for every $m$ there are infinitely many prime values among $2^k3^\ell m+1$).
\end{enumerate}

\item[(iii)] \textbf{What would be needed next.}
A successful construction would likely require a 2-dimensional covering system: a finite set of primes $p$ and residue classes of $(k,\ell)$ modulo periods (coming from the multiplicative orders of $2$ and $3$ mod $p$) such that for each class, $2^k3^\ell m\equiv -1\pmod p$ holds.
Conversely, a disproof would likely require deep results about prime values of $2^k3^\ell m+1$ varying over a 2-parameter semigroup.
\end{itemize}

\subsection*{7. OPTIONAL: SHARPENING / DISCUSSION}
\begin{itemize}
\item The finite-set CRT lemma shows that the obstruction is genuinely ``global in $(k,\ell)$'': there is no finite-pattern contradiction.
\item This problem is a two-parameter analogue of the Sierpi\'nski problem (numbers $m$ such that $m2^k+1$ is always composite), but the extra $3^\ell$ parameter makes a direct covering-congruence approach substantially harder.
\end{itemize}

% ----------------------------------------------------------------------
