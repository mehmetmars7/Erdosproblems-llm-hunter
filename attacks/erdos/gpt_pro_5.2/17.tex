% Erdős Problem #17

1) FORMAL RESTATEMENT

Let $p$ be a prime. Let $\mathcal P(p):=\{q\text{ prime}: 2\le q\le p\}$.

Define: $p$ is a \emph{cluster prime} if
\[
\forall n\in\mathbb Z\;\Big( n\ \text{even and }2\le n\le p-3\ \Rightarrow\ \exists q_1,q_2\in\mathcal P(p)\text{ with }n=q_1-q_2\Big).
\]
(For $p\le 3$ the condition is vacuous because there is no even $n$ with $2\le n\le p-3$.)

Question: Are there infinitely many cluster primes?

2) QUICK LITERATURE/CONTEXT CHECK

The provided problem statement already records the main known global information: Blecksmith--Erd\H{o}s--Selfridge (1999) and Elsholtz (2003) proved strong upper bounds on the counting function of cluster primes (in particular, that they have density $0$ in the primes). I do not use (or claim) any further literature beyond what is explicitly stated in the problem file.

3) ATTACK PLAN

Proof track (ambitious):
- Try to show infinitely many $p$ satisfy the required ``prime-pair covering'' condition by a probabilistic/heuristic model for prime gaps, possibly reducing to simultaneously finding at least one prime pair of each even gap $n\le p-3$.

Disproof track (ambitious):
- Try to force failures by exhibiting, for infinitely many primes $p$, an even $n\le p-3$ such that no prime pair $\le p$ has gap $n$. For a fixed $p$, this is equivalent to showing no prime in $[2,p-n]$ forms a prime pair with gap $n$.

What I can actually deliver here is a clean reformulation, a combinatorial necessary condition, and verified small cases.

4) WORK

FAST REALITY CHECK (explicit small cases)

For a fixed prime $p$, define $D(p):=\{q_1-q_2: q_1,q_2\in\mathcal P(p)\}$. Then $p$ is cluster iff every even $n\in[2,p-3]$ lies in $D(p)$.

I ran an exact brute-force check (enumerating primes $\le p$ and checking all even $n\le p-3$) for $p\le 300$.

Result (exact): the cluster primes $\le 300$ are
\[
\begin{aligned}
&2,3,5,7,11,13,17,19,23,29,31,37,41,43,47,53,59,61,67,71,73,79,83,89,\\
&101,103,107,109,113,131,137,139,151,157,163,167,173,179,181,193,197,199,233,239,241,271,277,281,283.
\end{aligned}
\]
The non-cluster primes $\le 300$ are
\[
(97,\,88),\ (127,\,112),\ (149,\,140),\ (191,\,182),\ (211,\,202),\ (223,\,202),\ (227,\,202),\ (229,\,202),\ (251,\,242),\ (257,\,242),\ (263,\,242),\ (269,\,242),\ (293,\,284),
\]
where $(p,\,n_\text{fail})$ records the smallest even $n\le p-3$ that fails.

Example verification of the first failure $p=97$ (no computation needed):
- The claimed failing even number is $n=88\le 97-3$.
- If $q_1-q_2=88$ with primes $q_1,q_2\le 97$, then $q_2=q_1-88\le 97-88=9$.
- The only primes $\le 9$ are $2,3,5,7$, giving $q_1\in\{90,91,93,95\}$, none of which is prime.
So $97$ is not cluster.

Lemma 17.1 (equivalent prime-pair formulation).
For any prime $p$, the following are equivalent:

(a) $p$ is a cluster prime.

(b) For every even integer $n$ with $2\le n\le p-3$ there exists a prime $q\le p-n$ such that $q$ and $q+n$ are both primes (hence both $\le p$).

Proof.
(a)$\Rightarrow$(b): Fix even $n\le p-3$. By (a), there exist primes $q_1,q_2\le p$ with $q_1-q_2=n$. Put $q=q_2$. Then $q\le p-n$ because $q+n=q_1\le p$. Also $q$ and $q+n=q_1$ are prime.

(b)$\Rightarrow$(a): Fix even $n\le p-3$. By (b) there is a prime $q\le p-n$ with $q$ and $q+n$ prime. Set $q_2=q$ and $q_1=q+n$. Then $q_1,q_2\le p$ and $q_1-q_2=n$. This is exactly (a). $\square$

Lemma 17.2 (a combinatorial lower bound on $\pi(p)$ from the cluster property).
Let $p$ be a cluster prime and let $\pi(p)=|\mathcal P(p)|$. Then
\[
\binom{\pi(p)}{2}\ \ge\ \frac{p-3}{2}.
\]
Equivalently, $\pi(p)$ must satisfy
\[
\pi(p)\ge \left\lceil\frac{1+\sqrt{1+4(p-3)}}{2}\right\rceil.
\]

Proof.
Because $p>2$ implies $p$ is odd, every even $n$ with $2\le n\le p-3$ is an even positive integer. Since $p$ is cluster, for each such even $n$ there exist primes $q_1>q_2$ in $\mathcal P(p)$ with $q_1-q_2=n$. In particular, each such $n$ occurs as a \emph{positive} difference between two \emph{distinct} primes $\le p$.

There are exactly $(p-3)/2$ even integers in the interval $[2,p-3]$.

On the other hand, the number of ordered pairs $(q_1,q_2)$ of distinct primes with $q_1>q_2$ is exactly $\binom{\pi(p)}{2}$. Each such ordered pair contributes at most one positive difference $q_1-q_2$, so the set of attainable positive differences has size at most $\binom{\pi(p)}{2}$. Therefore, to cover all $(p-3)/2$ required even differences, we must have $\binom{\pi(p)}{2}\ge (p-3)/2$.

Solving $\pi(p)(\pi(p)-1)/2\ge (p-3)/2$ gives $\pi(p)(\pi(p)-1)\ge p-3$, and the quadratic formula yields the stated lower bound on $\pi(p)$. $\square$

5) VERIFICATION

- Quantifiers: Lemma 17.1 checks the same $n$-range $2\le n\le p-3$ and keeps the constraint $q_1,q_2\le p$; the equivalence is exact.
- Edge cases: For $p\le 3$ the definition is vacuous; Lemma 17.1 still holds because the quantified set of $n$ is empty.
- Lemma 17.2: The only nontrivial step is the counting bound ``distinct positive differences $\le$ number of ordered pairs''; this is correct because the map $(q_1,q_2)\mapsto q_1-q_2$ may collide, so counting pairs is an upper bound on distinct differences.
- Sanity check: for $p=5$, $\pi(5)=3$ and $\binom{3}{2}=3\ge (5-3)/2=1$, consistent.

6) FINAL

**UNRESOLVED**

(i) Strongest proved partial result:
- Exact reformulation (Lemma 17.1).
- A necessary combinatorial constraint $\binom{\pi(p)}{2}\ge (p-3)/2$ for any cluster prime $p$ (Lemma 17.2).
- Verified exact small cases: all cluster primes $\le 300$ listed above, and explicit failure at $p=97$ via $n=88$.

(ii) First gap (crisp):
Prove or disprove
\[
\text{there exist infinitely many primes }p\text{ such that }\forall\text{ even }n\le p-3\ \exists\text{ primes }q,q+n\le p.
\]

(iii) Top 3 next moves (concrete):
1. For each non-cluster prime $p$ observed computationally, the first failing $n$ is close to $p$ (e.g., $97\mapsto 88$). Prove a structural lemma of the form: if $p$ is large and a specific short list of prime gaps up to some bound occurs below $p$, then $p$ is cluster.
2. Attempt to show that if $p$ is cluster then the prime set $\mathcal P(p)$ must contain a prime in each residue class modulo many small moduli (because each even gap imposes a pair constraint); formalize and see whether this is plausible/impossible.
3. Computationally extend the verified range (e.g., to $p\le 10^5$) and tabulate the distribution of the first failing $n$; look for a pattern that could be proved (e.g., failures coming from $n=p-q$ where $q$ is the smallest prime not producing a prime $p-q+q$).

(iv) Minimal counterexample structure (if the answer is ``finitely many''):
A minimal obstruction prime $p$ would come with a smallest even $n\le p-3$ such that the prime-pair system $\{q\le p-n: q,q+n\text{ prime}\}$ is empty; empirically in small data $n$ is large (close to $p$), forcing $q$ to lie among the very small primes. Proving infinitely many obstructions would amount to constructing infinitely many primes $p$ for which some large even $n$ forces $q$ into a finite set that yields no valid $q+n$.


