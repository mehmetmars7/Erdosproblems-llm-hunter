\section*{Problem \#302}

\subsection*{1) FORMAL RESTATEMENT}
Let $N\in\mathbb{N}$ and write $[N]=\{1,2,\dots,N\}$. Define
\[
  f(N) := \max\Big\{ |A| : A\subseteq [N]\ \text{and there do not exist pairwise distinct }a,b,c\in A\ \text{with }\frac1a=\frac1b+\frac1c\Big\}.
\]
The problem asks for asymptotics of $f(N)$ as $N\to\infty$ (``estimate $f(N)$''), and in particular asks whether
\[
  f(N)=(1/2+o(1))N
\]
holds. (As written, this last sentence is a concrete conjectural asymptotic.)

\subsection*{2) QUICK LITERATURE/CONTEXT CHECK}
I checked the ErdosProblems entry for \#302 and its linked comments/notes. The prompt itself already records the main currently-advertised bounds:
\begin{itemize}[leftmargin=2em]
\item Lower bounds: $f(N)\ge (1/2+o(1))N$ from $A=\{\text{odd}\}$ or $A=[\lceil N/2\rceil, N]$; and an improved construction $f(N)\ge (5/8+o(1))N$ attributed to Stijn Cambie.
\item Upper bound: Wouter van Doorn gives $f(N)\le (9/10+o(1))N$.
\end{itemize}
My quick web search did not reveal a newer asymptotic constant determination beyond these notes.

\subsection*{3) ATTACK PLAN}
\begin{enumerate}[leftmargin=2em]
\item \textbf{Structural rewrite.} Rewrite the Diophantine condition in a form that makes combinatorial obstructions transparent.
\item \textbf{Disprove the specific $1/2$-asymptotic.} Exhibit an explicit family of sets $A\subseteq[N]$ with density strictly bigger than $1/2$ and prove they contain no solutions.
\item \textbf{Reality check.} Compute $f(N)$ for small $N$ via exact search (maximum independent set in a 3-uniform hypergraph) to see what densities appear.
\item \textbf{What would remain.} Any true asymptotic will require either a much better construction (raising the lower bound) or a density/supersaturation theorem (lowering the upper bound).
\end{enumerate}

\subsection*{4) WORK}
\paragraph{4.1. Algebraic normal form.}
If $a,b,c\in\mathbb{N}$ satisfy
\(
\frac1a=\frac1b+\frac1c
\)
then
\[
  \frac1a = \frac{b+c}{bc}\quad\Longleftrightarrow\quad bc=a(b+c)\quad\Longleftrightarrow\quad (b-a)(c-a)=a^2.
\]
In particular $bc=a(b+c)$ implies $bc>ab$ and $bc>ac$, so $b>a$ and $c>a$.
Conversely, any factorisation $a^2=de$ with $d,e\in\mathbb{N}$ and $d\ne e$ gives a solution
\[
  b=a+d,\qquad c=a+e
\]
with $b\ne c$ and $\frac1a=\frac1b+\frac1c$.

\paragraph{4.2. A universal inequality: $a<\tfrac{N}{2}$ in any solution inside $[N]$ (with $b\ne c$).}
Assume $1\le a<b<c\le N$ and $\frac1a=\frac1b+\frac1c$. Then
\[
  a=\frac{bc}{b+c} \le \frac{bc}{2\sqrt{bc}} = \frac{\sqrt{bc}}{2}\le \frac{N}{2}
\]
by AM--GM and $\sqrt{bc}\le N$.
Moreover equality $a=N/2$ forces $\sqrt{bc}=N$ and $b+c=2\sqrt{bc}$, hence $b=c=N$, which is excluded by distinctness. Therefore:
\begin{equation}
  \label{eq:a-strict-half}
  \boxed{\text{If }1\le a,b,c\le N\text{ are distinct and }\tfrac1a=\tfrac1b+\tfrac1c,\text{ then }a<\tfrac{N}{2}.}
\end{equation}
This immediately proves that $A=[\lceil N/2\rceil,N]$ is solution-free.

\paragraph{4.3. Parity obstruction.}
If $b$ and $c$ are both odd, then $bc$ is odd and $b+c$ is even, so $(b+c)/bc$ is a rational with even numerator and odd denominator in lowest terms; it cannot equal $1/a$.
Equivalently: there is \emph{no} integer solution to $\frac1a=\frac1b+\frac1c$ with $b,c$ odd.

\paragraph{4.4. A rigorous $(5/8)$-density construction (disproving the $1/2$ asymptotic conjecture).}
Define
\[
  A_N := \{n\in [N] : n\ge \lceil N/2\rceil\}\ \cup\ \{n\in[N] : n\le \lfloor N/4\rfloor\ \text{and $n$ is odd}\}.
\]
Then
\[
  |A_N| = \big(N-\lceil N/2\rceil+1\big) + \#\{\text{odd }n\le \lfloor N/4\rfloor\}
  = \frac{N}{2} + \frac{N}{8} + O(1) = \Big(\frac{5}{8}+o(1)\Big)N.
\]
\begin{theorem}
For every $N$, the set $A_N$ contains no pairwise distinct $a,b,c\in A_N$ with $\frac1a=\frac1b+\frac1c$.
Consequently $f(N)\ge |A_N|=(5/8+o(1))N$, and in particular $f(N)\neq (1/2+o(1))N$.
\end{theorem}
\begin{proof}
Suppose for contradiction that distinct $a,b,c\in A_N$ satisfy $\frac1a=\frac1b+\frac1c$. As noted above, $b,c>a$.
By \eqref{eq:a-strict-half} we have $a< N/2$, so $a$ cannot lie in the upper-half block of $A_N$. Hence $a$ lies in the lower block, so $a$ is \emph{odd} and $a\le N/4$.

We now claim that $b$ and $c$ must both be even. Indeed, if $b,c$ are both odd then there is no solution (paragraph 4.3). If exactly one of $b,c$ is odd, then $b+c$ is odd and $bc$ is even, so from $bc=a(b+c)$ we see that $a$ must be even (odd times $a$ equals an even number), contradicting that $a$ is odd.
Thus $b$ and $c$ are both even.

Since the only elements of $A_N$ below $N/2$ are odd, the even numbers $b,c\in A_N$ must lie in the upper-half block, i.e. $b,c\ge N/2$. Let $b\le c$. Then
\[
  \frac1a = \frac1b+\frac1c < \frac{2}{b}\qquad\Longrightarrow\qquad a > \frac{b}{2} \ge \frac{N}{4}.
\]
This contradicts $a\le N/4$.
Therefore no such triple exists.
\end{proof}

\paragraph{4.5. Small-$N$ computation (exact).}
As a quick reality check, I exhaustively computed $f(N)$ for $N\le 100$ by modeling the valid triples as a 3-uniform hypergraph and running a branch-and-bound maximum independent set search.
The resulting densities are much larger than $5/8$ for these $N$ (e.g. $f(100)=86$), illustrating that the extremal behavior for small $N$ is not yet close to any conjectural asymptotic.
A witness for $N=100$ is $A=[100]\setminus\{6,8,12,14,18,20,22,24,28,30,48,78,84,90\}$.

\subsection*{5) VERIFICATION}
\begin{itemize}[leftmargin=2em]
\item The equivalence $(b-a)(c-a)=a^2$ was checked by direct algebra.
\item Inequality \eqref{eq:a-strict-half} was derived from AM--GM and is strict because $b\ne c$.
\item In the $(5/8)$ construction proof, the key steps are: (i) $a<N/2$ forces $a$ to be odd $\le N/4$; (ii) that forces $b,c$ even; (iii) then $b,c\ge N/2$ forces $a>N/4$; contradiction.
\item Computations (not reproduced here) verified that the displayed witness for $N=100$ contains no forbidden triple.
\end{itemize}

\subsection*{6) FINAL}
\textbf{UNRESOLVED.}
\begin{enumerate}[leftmargin=2em,label=(\roman*)]
\item \textbf{Progress achieved.} I gave a complete proof of the explicit construction $A_N$ of size $(5/8+o(1))N$ with no solutions, which in particular disproves the conjectural asymptotic $f(N)=(1/2+o(1))N$. I also recorded the structural identity $(b-a)(c-a)=a^2$ and the universal constraint $a<N/2$ for any solution in $[N]$.
\item \textbf{Where it got stuck.} Determining the true asymptotic constant (or even improving the $(9/10)$ upper bound or the $(5/8)$ lower bound) appears to require substantially new extremal/combinatorial input; the present work does not bridge that gap.
\item \textbf{What next steps would look like.} (a) Use the factorisation form $(b-a)(c-a)=a^2$ to set up a hypergraph container/supersaturation argument for dense $A$. (b) Search for higher-density constructions guided by exact solutions for moderate $N$ and attempt to prove a general pattern.
\item \textbf{Falsifiable conjecture.} Based on the existing upper bound $f(N)\le (0.9+o(1))N$ and my small-$N$ computations hovering around $\approx 0.86N$, a plausible guess is that $f(N)=(c+o(1))N$ for some constant $c\in[0.85,0.9]$.
\end{enumerate}

\subsection*{7) COMPLETION ESTIMATE}
To ``complete'' the original problem (i.e. pin down the correct asymptotics of $f(N)$), one would need either:
(i) a new construction improving the best-known lower density, or
(ii) a new extremal argument improving the best-known upper density.
Either route would likely require nontrivial new ideas beyond the parity/interval obstructions used above.


% -----------------------------------------------------------------------------
