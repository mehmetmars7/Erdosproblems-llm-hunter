

Let $\alpha$ be the infinite ordinal $\omega^{\omega^2}$. Is it true that in any red/blue colouring of the edges of $K_\alpha$ there is either a red $K_\alpha$ or a blue $K_3$? For comparison, Specker \cite{Sp57} proved this property holds when $\alpha=\omega^2$ and false when $\alpha=\omega^n$ for $3\leq n<\omega$. Chang proved this property holds when $\alpha=\omega^\omega$ (see [590] ). See [592] for the general case. References [Sp57] Specker, Ernst, Teilmengen von Mengen mit Relationen . Comment. Math. Helv. (1957), 302-314.

\bigskip
\noindent\textbf{FORMAL RESTATEMENT}

Fix the ordinal $\alpha := \omega^{\omega^2}$.
Let $[\alpha]^2 := \{\{x,y\}: x,y<\alpha,\ x\neq y\}$.
A \,\emph{red/blue edge-colouring} is a map $c:[\alpha]^2\to\{\text{red},\text{blue}\}$.
Write $K_\alpha$ for the complete graph on vertex set $\alpha$ with edge-set $[\alpha]^2$.

A \,\emph{red $K_\alpha$} means: there exists a subset $X\subseteq \alpha$ such that
(1) $(X,<)$ is order-isomorphic to $(\alpha,<)$ (so $X$ has order-type $\alpha$), and
(2) for all distinct $x,y\in X$, $c(\{x,y\})=\text{red}$.

A \,\emph{blue $K_3$} means: there exist $x<y<z<\alpha$ with
$c(\{x,y\})=c(\{x,z\})=c(\{y,z\})=\text{blue}$.

Question: Is it true that every $c:[\alpha]^2\to\{\text{red},\text{blue}\}$ produces either a red $K_\alpha$ or a blue $K_3$?
(Equivalently: does $\alpha\to(\alpha,3)^2$ hold for $\alpha=\omega^{\omega^2}$?)

\bigskip
\noindent\textbf{QUICK LITERATURE/CONTEXT CHECK}

I will not import any external results beyond what is explicitly stated in the problem text.
The text states: the property holds for $\alpha=\omega^2$ (Specker), fails for $\alpha=\omega^n$ for $3\le n<\omega$ (Specker), and holds for $\alpha=\omega^\omega$ (Chang).
The present case $\alpha=\omega^{\omega^2}$ is presented as open.

\bigskip
\noindent\textbf{ATTACK PLAN}

\emph{Proof-track ideas.}
\begin{itemize}
\item (Inductive block construction) Use the Cantor normal form structure of $\omega^{\omega^2}=\sup_{m<\omega}\omega^{\omega\cdot m}$ and try to build a red copy by iterating a ``build a red $\omega^{\omega\cdot(m+1)}$ from many red $\omega^{\omega\cdot m}$'' step under the hypothesis ``no blue triangle''.
\item (Canonical/closure argument) Attempt to define, for each vertex, a large red-closed set using the triangle-free-blue constraint and then perform transfinite recursion to obtain a red set of the desired order-type.
\end{itemize}

\emph{Disproof-track ideas.}
\begin{itemize}
\item (Specker-style colouring) Try to generalize the known counterexamples for $\omega^n$ ($n\ge 3$) by defining a blue triangle-free graph on $\omega^{\omega^2}$ whose red edges avoid a red copy of $\omega^{\omega^2}$.
\item (Block-diagonal construction) Decompose $\omega^{\omega^2}$ into $\omega$ many intervals and define colors by relative positions so that every large-order-type subset picks up enough blue edges to prevent being all red, yet without creating a blue triangle.
\end{itemize}

I did not succeed in either direction. Below are rigorous partial lemmas and sanity checks.

\bigskip
\noindent\textbf{WORK}

\medskip
\noindent\textbf{FAST REALITY CHECK (finite and simplest infinite cases)}

\begin{itemize}
\item Finite analogue fails quickly: a brute-force check over all $2^6$ red/blue colourings of the edges of $K_4$ finds counterexamples. For instance, colour a single edge blue and all other edges red. Then there is no blue triangle (only one blue edge) and there is no red $K_4$ (since not all edges are red). So ``$4\to(4,3)^2$'' is false.
\item For $\alpha=\omega$ the analogue \emph{does} hold: any red/blue coloring of $K_\omega$ contains either an infinite red clique (hence a red $K_\omega$) or a blue triangle. A full proof is given as Lemma~\ref{lem:omega_case}.
\end{itemize}

\medskip
\noindent\textbf{Lemma 1 (blue neighborhood is a red clique).}\label{lem:blue_nhood_red}
Let $\alpha$ be any ordinal and let $c:[\alpha]^2\to\{\text{red},\text{blue}\}$ be a colouring with no blue triangle.
For any vertex $v<\alpha$, define the blue neighborhood
\[
N_{\text{blue}}(v):=\{u<\alpha: u\neq v\text{ and }c(\{u,v\})=\text{blue}\}.
\]
Then every edge inside $N_{\text{blue}}(v)$ is red; i.e. for all distinct $u,w\in N_{\text{blue}}(v)$ we have $c(\{u,w\})=\text{red}$.

\emph{Proof.}
Fix $v$ and take distinct $u,w\in N_{\text{blue}}(v)$.
By definition $c(\{u,v\})=c(\{w,v\})=\text{blue}$.
If $c(\{u,w\})=\text{blue}$ also held, then $\{u,v,w\}$ would be a blue $K_3$, contradicting the hypothesis.
Therefore $c(\{u,w\})=\text{red}$.
Since $u,w$ were arbitrary, all edges inside $N_{\text{blue}}(v)$ are red.\qed

\medskip
\noindent\textbf{Lemma 2 (the $\omega$ case, constructive).}\label{lem:omega_case}
Let $c:[\omega]^2\to\{\text{red},\text{blue}\}$ be a colouring with no blue triangle.
Then there exists an infinite increasing sequence $(x_n)_{n<\omega}$ of naturals such that every edge $\{x_i,x_j\}$ ($i<j$) is red.
Equivalently, $K_\omega$ contains a red $K_\omega$.

\emph{Proof.}
We build recursively an increasing sequence $x_0<x_1<x_2<\cdots$ together with infinite sets $A_n\subseteq\omega$ satisfying:
\begin{enumerate}
\item $x_n=\min(A_n)$,
\item $A_{n+1}\subseteq A_n\setminus\{x_n\}$,
\item for every $m\le n$ and every $y\in A_{n+1}$, the edge $\{x_m,y\}$ is red.
\end{enumerate}
Start with $A_0:=\omega$ and $x_0:=\min(A_0)=0$.
Suppose $A_n$ and $x_n$ are defined and $A_n$ is infinite.
Partition $A_n\setminus\{x_n\}$ into
\[
B_n:=\{y\in A_n: y\neq x_n,\ c(\{x_n,y\})=\text{blue}\},\qquad
R_n:=\{y\in A_n: y\neq x_n,\ c(\{x_n,y\})=\text{red}\}.
\]
If $B_n$ is infinite, then by Lemma~\ref{lem:blue_nhood_red} the set $B_n$ is a red clique.
Also, by the inductive property (3) with $A_n$ in place of $A_{n+1}$, every earlier $x_m$ ($m<n$) is red-adjacent to every vertex of $A_n$, hence to every vertex of $B_n$.
Therefore $\{x_0,\dots,x_{n-1}\}\cup B_n$ is an infinite red clique.
In particular, we can extract from it an increasing red clique of order type $\omega$, and we are done.

If instead $B_n$ is finite, then $R_n$ is infinite (since $A_n\setminus\{x_n\}$ is infinite).
Define $A_{n+1}:=R_n$ and $x_{n+1}:=\min(A_{n+1})$.
Condition (1) and (2) hold by construction.
Condition (3) holds because for $m<n$, every $x_m$ was red-adjacent to every element of $A_n$ by the inductive hypothesis, and by definition every element of $A_{n+1}=R_n\subseteq A_n$ is red-adjacent to $x_n$.

Thus either the process terminates at some stage (giving an infinite red clique immediately), or it continues for all $n<\omega$.
In the latter case, for any $i<j$, we have $x_j\in A_i$ by repeated application of (2), hence $c(\{x_i,x_j\})=\text{red}$ by (3).
Therefore $(x_n)_{n<\omega}$ is an infinite increasing red clique as required.\qed

\medskip
\noindent\textbf{Lemma 3 (restriction gives a red $\omega^\omega$ inside $\omega^{\omega^2}$ under ``no blue triangle'').}\label{lem:restrict_omegaomega}
Assume the theorem stated in the problem text for $\omega^\omega$:
\emph{every} red/blue colouring of $[\omega^\omega]^2$ contains either a red $K_{\omega^\omega}$ or a blue $K_3$.
Then for $\alpha=\omega^{\omega^2}$, any colouring $c:[\alpha]^2\to\{\text{red},\text{blue}\}$ with no blue triangle contains a red $K_{\omega^\omega}$.

\emph{Proof.}
Since $\omega^\omega<\omega^{\omega^2}$, the initial segment
$S:=\{\xi<\alpha: \xi<\omega^\omega\}$
has order type $\omega^\omega$.
Restrict $c$ to $[S]^2$.
This restricted colouring has no blue triangle (any blue triangle inside $S$ would also be a blue triangle in $\alpha$).
By the assumed theorem for $\omega^\omega$, the restriction must therefore contain a red $K_{\omega^\omega}$.
That is, there exists $T\subseteq S$ of order-type $\omega^\omega$ such that all edges in $[T]^2$ are red.
Since $T\subseteq\alpha$, this is also a red $K_{\omega^\omega}$ in $K_\alpha$.\qed

\bigskip
\noindent\textbf{VERIFICATION}

\begin{itemize}
\item Lemma~\ref{lem:blue_nhood_red}: checked that the only needed hypothesis is ``no blue triangle''; no ordinal-specific assumption is used.
\item Lemma~\ref{lem:omega_case}: the recursion maintains that $A_n$ is infinite unless we already have an infinite red clique from an infinite $B_n$. The conclusion produces an increasing red clique, hence order type $\omega$.
\item Lemma~\ref{lem:restrict_omegaomega}: the only nontrivial input is the $\omega^\omega$ statement, which is explicitly mentioned in the problem text (attributed to Chang). The rest is a restriction argument.
\end{itemize}

\bigskip
\noindent\textbf{FINAL}

\textbf{UNRESOLVED}

(i) \emph{Strongest proved partial result.}
If a colouring $c:[\omega^{\omega^2}]^2\to\{\text{red},\text{blue}\}$ has no blue triangle, then (by restricting to an initial segment of type $\omega^\omega$ and using the $\omega^\omega$ result stated in the problem text) it contains a red $K_{\omega^\omega}$ (Lemma~\ref{lem:restrict_omegaomega}).
In particular, any counterexample to $\omega^{\omega^2}\to(\omega^{\omega^2},3)^2$ must still contain very large red cliques (of type $\omega^\omega$).

(ii) \emph{First gap (crisp).}
Given a blue-triangle-free colouring $c:[\omega^{\omega^2}]^2\to\{\text{red},\text{blue}\}$, prove or refute that there exists $X\subseteq\omega^{\omega^2}$ of order type $\omega^{\omega^2}$ with all edges in $[X]^2$ red.

(iii) \emph{Top 3 next moves.}
\begin{enumerate}
\item Prove an ``upgrade lemma'': assuming no blue triangle, show that from cofinally many red copies of $\omega^{\omega\cdot m}$ one can build a red copy of $\omega^{\omega\cdot(m+1)}$.
\item Develop a canonical form for triangle-free-blue colourings on $\omega^{\omega^2}$ (e.g. by analyzing colors between blocks in the natural decomposition $\omega^{\omega^2}=\sup_{m<\omega}\omega^{\omega\cdot m}$).
\item Search for an explicit Specker-type triangle-free-blue colouring on $\omega^{\omega^2}$ that destroys red copies of order type $\omega^{\omega^2}$; even a concrete candidate on the subordinal $\omega^{\omega\cdot 2}$ would be informative.
\end{enumerate}

(iv) \emph{Minimal counterexample structure.}
A minimal counterexample would be a colouring on $\omega^{\omega^2}$ with no blue triangle and no red $K_{\omega^{\omega^2}}$; by Lemma~\ref{lem:restrict_omegaomega} it would still contain a red $K_{\omega^\omega}$.
Thus any counterexample must allow very large red cliques while preventing their extension to order type $\omega^{\omega^2}$, likely by forcing many blue edges between different ``$\omega^\omega$-blocks'' without ever creating a blue triangle.


