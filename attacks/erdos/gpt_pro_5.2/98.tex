\section*{Erd\H{o}s Problem \#98}

\begin{enumerate}[leftmargin=*,label=\arabic*.]

\item \textbf{Formal Restatement.}

For a finite set $P\subset\mathbb{R}^2$, write
\[
\Delta(P):=\{\|p-q\|:p,q\in P,\ p\ne q\}
\]
for the set of (nonzero) Euclidean distances determined by pairs of points.
Let $\mathcal{G}_n$ be the family of all $n$-point sets $P\subset\mathbb{R}^2$ in \emph{general position} in the sense:
\begin{itemize}[leftmargin=*]
\item no three points of $P$ are collinear;
\item no four points of $P$ are concyclic.
\end{itemize}
Define
\[
h(n):=\min\{\,|\Delta(P)|: P\in\mathcal{G}_n\,\}.
\]

\medskip
\noindent\textbf{Question (\#98).} Does $h(n)/n\to\infty$ as $n\to\infty$?

\item \textbf{Quick literature/context check.}

The Erd\H{o}s Problems page lists this as open (last edited 15 Oct 2025).\cite{ErdosProblems98}
It records that Erd\H{o}s could not even prove $h(n)\ge n$, and that Pach proved an upper bound $h(n)<n^{\log_2 3}$.\cite{ErdosProblems98}
It also records a stronger upper bound by Erd\H{o}s--F\"uredi--Pach--Ruzsa (1993):
\[
 h(n) < n\exp\bigl(c\sqrt{\log n}\bigr)\quad\text{for some constant }c>0.
\]
\cite{ErdosProblems98,EFPR93}

Independently, for \emph{arbitrary} $n$-point sets (with no general-position hypotheses), Guth and Katz proved that the number of distinct distances is at least on the order of $n/\log n$.\cite{GK15}
Since $\mathcal{G}_n$ is a subset of all $n$-point sets, this lower bound applies to $h(n)$ as well.

\item \textbf{Attack plan.}

\textbf{Proof track (toward $h(n)/n\to\infty$).}
Seek a superlinear lower bound on $|\Delta(P)|$ under the combined restrictions ``no three collinear'' and ``no four concyclic.'' Natural approaches:
\begin{itemize}[leftmargin=*]
\item strengthen incidence-based distinct distance arguments by ruling out rich lines and rich circles;
\item ``pinned distance'' methods: show some point determines $\omega(n)$ distinct distances, then $|\Delta(P)|\ge \omega(n)$.
\end{itemize}

\textbf{Disproof track (construct $P\in\mathcal{G}_n$ with $|\Delta(P)|=O(n)$).}
Try to build a highly structured configuration with many repeated distances while avoiding collinear triples and concyclic quadruples. A key obstacle is that small perturbations typically destroy distance multiplicities.

\textbf{Phase 1 small cases.}
Compute/construct $h(n)$ for the smallest $n$ to sanity-check the definition and the general-position constraints.

\item \textbf{Work.}

\subsubsection*{(A) Tiny cases (exact).}

\textbf{Claim 1: $h(3)=1$.}
Take an equilateral triangle; then $|\Delta(P)|=1$, and clearly no three collinear. Hence $h(3)\le 1$. On the other hand, any three non-collinear points determine at least one distance, so $h(3)=1$.

\medskip
\textbf{Claim 2: $h(4)=2$.}
\emph{Upper bound.} Let $A,B,C$ be an equilateral triangle of side length $1$, and let $O$ be its circumcenter (which coincides with centroid). Then
\[
\|A-B\|=\|B-C\|=\|C-A\|=1,\qquad \|O-A\|=\|O-B\|=\|O-C\|=\frac{1}{\sqrt{3}}.
\]
Thus $\Delta(\{A,B,C,O\})=\{1,1/\sqrt{3}\}$ has size $2$. The four points are in general position: no three are collinear, and they are not concyclic because the unique circle through $A,B,C$ is the circumcircle centered at $O$, which does not contain $O$.
So $h(4)\le 2$.

\emph{Lower bound.} A set of four planar points cannot have all six pairwise distances equal (the largest equidistant set in $\mathbb{R}^2$ has size $3$), so $h(4)\ne 1$.
Therefore $h(4)=2$.

\subsubsection*{(B) General bounds from the literature.}

\textbf{Lower bound (all $n$).}
Guth--Katz proved that for any $n$ points in $\mathbb{R}^2$,
\[
|\Delta(P)| \ge c\,\frac{n}{\log n}
\]
for some absolute constant $c>0$.\cite{GK15}
Since $\mathcal{G}_n$ is a subset of all $n$-point sets,
\[
 h(n)=\min_{P\in\mathcal{G}_n}|\Delta(P)| \ge c\,\frac{n}{\log n}.
\]
This is far from the desired superlinear growth, but it is a rigorous baseline.

\medskip
\textbf{Upper bound (general position).}
Erd\H{o}s--F\"uredi--Pach--Ruzsa constructed (for all sufficiently large $n$) $n$-point sets in general position with
\[
|\Delta(P)| < n\exp\bigl(c\sqrt{\log n}\bigr)
\]
for some constant $c>0$.\cite{EFPR93,ErdosProblems98}
Hence
\[
 h(n) \le n\exp\bigl(c\sqrt{\log n}\bigr).
\]

\medskip
\noindent\textbf{Current gap.}
To answer ``$h(n)/n\to\infty$?'' one needs either:
\begin{itemize}[leftmargin=*]
\item a lower bound $h(n)\ge n\cdot\omega(1)$ (even $h(n)\ge n\log\log n$ would suffice), or
\item a construction showing $h(n)=O(n)$.
\end{itemize}
Neither is currently achieved by the bounds above.

\item \textbf{Verification.}

The computations for $h(3)$ and $h(4)$ are explicit and verify the general-position constraints.
For the general bounds:
\begin{itemize}[leftmargin=*]
\item Guth--Katz applies to all point sets, so it applies to $\mathcal{G}_n$ without extra hypotheses.
\item EFPR93 is explicitly a construction in general position as quoted on the Erd\H{o}s Problems page, and thus gives an upper bound on $h(n)$.
\end{itemize}
No claim here asserts the limit $h(n)/n\to\infty$; rather, these bounds delineate what is known.

\item \textbf{FINAL: UNRESOLVED.}

\begin{itemize}[leftmargin=*]
\item[(i)] \textbf{Strongest fully proved partial result:}
For all $n$, $h(n)\ge c\,n/\log n$ (Guth--Katz), and there exist general-position sets with $h(n)\le n\exp(c\sqrt{\log n})$ (EFPR93). Also $h(3)=1$ and $h(4)=2$ exactly.
\item[(ii)] \textbf{Most plausible route forward:}
Either (a) prove a substantially improved lower bound in general position by combining incidence geometry with the prohibition of 4-point cocircularities, or (b) construct a genuinely linear-distance construction $|\Delta(P)|=O(n)$ while certifying no 3 collinear and no 4 concyclic.
\item[(iii)] \textbf{First place the proof attempt breaks:}
Known distinct-distance lower bounds (even with modern polynomial/incidence methods) are essentially linear up to polylog factors; current techniques do not yield superlinear lower bounds from these general-position constraints.
\item[(iv)] \textbf{What a counterexample would have to look like:}
An infinite family $P_n\in\mathcal{G}_n$ with $|\Delta(P_n)|\le Cn$ for a constant $C$ independent of $n$.
\end{itemize}

\item \textbf{Completion estimate.}

\textbf{COMPLETION: 15\%.}

\end{enumerate}

