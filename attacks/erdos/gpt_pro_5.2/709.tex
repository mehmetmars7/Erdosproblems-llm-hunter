% Erdos Problem #709

1) \textbf{FORMAL RESTATEMENT}

For each $n\ge 1$, define $f(n)$ to be the minimal real number with the following property:

For every set $A=\{a_1,\dots,a_n\}\subset\mathbb{N}\cap[2,\infty)$ with $|A|=n$ and $m=\max(A)$, and for every interval $I$ consisting of $\lfloor f(n)m\rfloor$ consecutive integers, there exist \emph{distinct} integers $x_1,\dots,x_n\in I$ such that $a_i\mid x_i$ for all $i$.

(There is a mild ambiguity in the problem statement about whether the interval length is exactly $f(n)m$ or an integer rounding; for the purposes of rigorous inequalities below I will prove statements for integer multiples, which implies corresponding bounds for any reasonable rounding convention.)

2) \textbf{QUICK LITERATURE/CONTEXT CHECK}

Using only what is stated in the problem text: Erd\H{o}s--Sur\'anyi proved
\[(\log n)^c \ll f(n) \ll n^{1/2}\]
for some constant $c>0$.

3) \textbf{ATTACK PLAN}

- Prove easy universal lower/upper bounds by extremal examples and matching arguments.
- For upper bounds, model the problem as a bipartite matching between divisors $a_i$ and integers in the interval.

4) \textbf{WORK}

\textbf{Lemma 709.1 (Trivial lower bound $f(n)\ge 1$).}
For every $n\ge 1$, $f(n)\ge 1$.

\textit{Proof.}
Fix $n\ge 1$. Consider any $A$ with $m=\max(A)$.
If an interval $I$ contains fewer than $m$ consecutive integers, then it is possible to choose $I$ to avoid all multiples of $m$ (e.g. $I=\{1,2,\dots,m-1\}$).
In such an interval there is no integer divisible by $m$, so we cannot assign a distinct $x_i$ to the element of $A$ equal to $m$.
Hence, to guarantee a solution for all intervals, we need interval length at least $m$, i.e. $f(n)\ge 1$.
\hfill$\square$

\textbf{Lemma 709.2 (General upper bound $f(n)\le n$).}
For every $n\ge 1$ and every finite $A\subset\mathbb{N}\cap[2,\infty)$ with $|A|=n$ and $m=\max(A)$, every interval $I$ of $nm$ consecutive integers contains distinct $x_1,\dots,x_n\in I$ with $a_i\mid x_i$ for all $i$.
In particular, $f(n)\le n$.

\textit{Proof.}
Let $A=\{a_1,\dots,a_n\}$ and let $I$ be any interval of $nm$ consecutive integers.
Construct a bipartite graph $H$ with left vertex set $A$ and right vertex set $I$, where we connect $a\in A$ to $x\in I$ iff $a\mid x$.
We seek a matching that covers all left vertices.

By Hall's marriage theorem, it suffices to show that for every subset $S\subseteq A$, its neighborhood
\[
N(S)=\{x\in I:\exists a\in S\text{ with }a\mid x\}
\]
has size at least $|S|$.

Fix a nonempty $S\subseteq A$ and let $d=\min S$.
Every multiple of $d$ that lies in $I$ belongs to $N(S)$ (because it is divisible by $d\in S$), so $|N(S)|$ is at least the number of multiples of $d$ in $I$.
Write $|I|=nm$.
In any block of $nm$ consecutive integers, the number of multiples of $d$ is at least $\lfloor nm/d\rfloor$.
Since $d\le m$, we have $nm/d\ge n$, hence $\lfloor nm/d\rfloor\ge n$.
Therefore $|N(S)|\ge n$.
Finally, because $|S|\le n$, we have $|N(S)|\ge n\ge |S|$, establishing Hall's condition.
Thus $H$ contains a matching of size $n$, i.e. distinct $x_1,\dots,x_n\in I$ with $a_i\mid x_i$.
\hfill$\square$

\textbf{Lemma 709.3 (Exact value for $n=2$: $f(2)=2$ in the integer-multiple model).}
If we restrict attention to intervals of integer length $t\,\max(A)$ with $t\in\mathbb{N}$, then $t=1$ does not suffice for $n=2$, while $t=2$ always suffices. Hence in this integer-multiple model, $f(2)=2$.

\textit{Proof.}
\underline{Lower bound: $t=1$ fails.}
Take $A=\{2,3\}$, so $m=3$.
Consider the interval $I=\{5,6,7\}$ of length $m$.
The only multiple of $3$ in $I$ is $6$, and the only multiple of $2$ in $I$ is also $6$.
So there do not exist distinct $x_1,x_2\in I$ with $2\mid x_1$ and $3\mid x_2$.
Thus $t=1$ fails.

\underline{Upper bound: $t=2$ suffices.}
Lemma 709.2 with $n=2$ shows that any interval of $2m$ consecutive integers admits such a matching.
Therefore $t=2$ suffices for all $A$ with $|A|=2$.
\hfill$\square$

\textbf{FAST REALITY CHECK (limited brute force for $n=2$).}
I brute-forced all $A$ with $|A|=2$, $\max(A)\le 20$, and all interval starts $t\le 500$.
Result: $t=1$ fails (e.g. $A=\{2,3\}$ with interval $\{5,6,7\}$), while $t=2$ worked for all tested instances.

5) \textbf{VERIFICATION}

- Lemma 709.2: The key inequality is that any interval of length $nm$ contains at least $\lfloor nm/d\rfloor\ge n$ multiples of $d\le m$.
This is correct because multiples of $d$ occur every $d$ integers, so among $nm$ consecutive integers there are at least $\lfloor nm/d\rfloor$ multiples.
- Hall is applied correctly: $|N(S)|\ge n\ge |S|$ for all $S$.

6) \textbf{FINAL}

\textbf{UNRESOLVED}

(i) \textbf{Strongest proved partial result.}
I proved the unconditional bounds $1\le f(n)\le n$ (Lemmas 709.1 and 709.2) and, in the integer-multiple model, $f(2)=2$ (Lemma 709.3). I also ran a finite brute-force sanity check for $n=2$ (reported above).

(ii) \textbf{First gap (crisp).}
Close the gap between $f(n)\le n$ and the much smaller upper bound $f(n)\ll n^{1/2}$ stated in the problem text, or determine the true asymptotic behavior of $f(n)$.

(iii) \textbf{Top 3 next moves.}
1. Strengthen the Hall lower bound $|N(S)|\ge n$ by using the \emph{full} divisor structure of $S$ rather than just $\min S$, aiming for shorter intervals.
2. Search for extremal $A$ that make matching difficult (many $a_i$ sharing large common multiples) and try to prove lower bounds from explicit constructions.
3. Implement a computational optimizer for small $n$ to guess the exact $f(n)$ and the structure of extremal sets $A$.

(iv) \textbf{Minimal counterexample structure.}
A minimal obstruction to a candidate bound $f(n)\le c$ would be a set $A$ with $m=\max(A)$ and an interval $I$ of $\approx cm$ consecutive integers such that the divisibility bipartite graph fails Hall's condition for some subset $S\subseteq A$; i.e. the union of multiples in $I$ of elements of $S$ has size $<|S|$.


