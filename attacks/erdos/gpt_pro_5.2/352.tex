## 1) FORMAL RESTATEMENT

Let (\mu) denote Lebesgue measure on (\mathbb R^2). For three points (x_1,x_2,x_3\in\mathbb R^2), define the (unsigned) Euclidean area of the triangle with these vertices by
[
\operatorname{Area}(x_1,x_2,x_3);=;\frac12,\bigl|\det(x_2-x_1,;x_3-x_1)\bigr|,
]
where (\det((a,b),(c,d)):=ad-bc).

The Erdős problem asks:

> Does there exist a constant (c>0) such that for **every** Lebesgue measurable set (A\subseteq\mathbb R^2) with (\mu(A)\ge c), there exist (x_1,x_2,x_3\in A) with (\operatorname{Area}(x_1,x_2,x_3)=1)?

Stress points / edge cases:

* If (\mu(A)=\infty) the statement is known to be true (per Erdős; see context below).
* The nontrivial regime is (0<\mu(A)<\infty), and (as the examples show) (A) must be allowed to have empty interior and be highly disconnected.
* A necessary lower bound for any such (c) is
  [
  c_0=\frac{4\pi}{\sqrt{27}}\approx 2.418,
  ]
  witnessed by a disk of slightly smaller area (details proved in §4).

## 2) QUICK LITERATURE/CONTEXT CHECK  (web browsing: YES)

* The problem is listed as **open** on the Erdős Problems site.
* Mauldin’s 2013 expository paper restates it and records (i) the conjectured best constant (c_0=4\pi/(3\sqrt3)=4\pi/\sqrt{27}), (ii) a reduction to unions of finitely many compact convex sets, and (iii) verification of the conjecture for unions of (n\le 3) compact convex sets (per Freiling–Mauldin).
* Kovač’s 2024 arXiv survey also cites the problem as open in general and mentions Erdős’s easy observation for infinite area / unbounded positive area sets (and notes the right-triangle variant is open).
* There are negative results for variants where “measurable” is relaxed to arbitrary sets with large **outer** Lebesgue measure: Pantsulaia–Rusiashvili claim that no finite constant works for that outer-measure version.
* A MathOverflow thread discusses outer-measure/full-outer-measure triangle-free sets via transfinite induction (again outside the Lebesgue-measurable regime).

So: as of the latest accessible sources here (2024), the measurable-set version remains open.

## 3) ATTACK PLAN

### Proof-track strategies

1. **Local translation/Steinhaus approach:** show that any measurable (A) with (0<\mu(A)<\infty) contains triangles of *all sufficiently small* areas (an interval ((0,\varepsilon))). This can be proved rigorously via (L^1)-continuity of translations and triple intersections.
2. Combine with a **large-area existence** theorem (e.g., Freiling–Mauldin: outer measure (>c_0\Rightarrow) triangle area (>1)) and try to force **exactly** area (1) via an “intermediate value” mechanism. The missing piece is a connectedness/continuity principle for the set of attainable areas in an arbitrary measurable set.
3. More analytic: study the 3-point correlation
   [
   J(u,v)=\mu\bigl(A\cap(A-u)\cap(A-v)\bigr)=\int 1_A(x)1_A(x+u)1_A(x+v),dx,
   ]
   and try to show existence of ((u,v)) with (\det(u,v)=2) and (J(u,v)>0) under (\mu(A)\ge c).

### Disproof/construction strategies

1. Build bounded measurable sets of arbitrarily large area avoiding area (1) triangles by arranging many components (disks/rectangles) with carefully controlled inter-component triangle-area ranges.
2. Use “arithmetic” constructions (lattice-like sets) to force triangle areas into a discrete set missing (1), then thicken while keeping the exclusion.

**Best current path:** I can fully execute strategy (1) on the proof-track: prove that *every* positive finite measure set contains triangles of every sufficiently small area. I can also rigorously establish the lower bound (c\ge c_0). However, the remaining global step “large measure forces area (1)” is exactly what the literature flags as open. I do not see a way to complete a gap-free proof or a genuine measurable counterexample here.

## 4) WORK

### Lemma 0 (Sharp geometric lower bound (c\ge c_0))

Let (B_R\subset\mathbb R^2) be the closed disk of radius (R) centered at the origin. Then the maximum area of any triangle with vertices in (B_R) is
[
\max{\operatorname{Area}(x_1,x_2,x_3): x_i\in B_R}=\frac{3\sqrt3}{4},R^2,
]
achieved by an equilateral triangle on the boundary circle.

**Proof.**
Take any triangle with vertices on (or inside) the circle of radius (R). Enlarge it radially (if necessary) so its vertices lie on the circle; this can only increase its area, so the maximum occurs among inscribed triangles.

For an inscribed triangle with circumradius (R) and angles (A,B,C) ((A+B+C=\pi)), one standard area formula is
[
\operatorname{Area} = 2R^2,\sin A,\sin B,\sin C.
]
(Indeed, side lengths are (a=2R\sin A), (b=2R\sin B), (c=2R\sin C), and (\operatorname{Area}=\frac12 ab\sin C = 2R^2\sin A\sin B\sin C).)

So maximizing area for fixed (R) is equivalent to maximizing (\sin A\sin B\sin C) subject to (A,B,C>0), (A+B+C=\pi). The function (\log\sin\theta) is strictly concave on ((0,\pi)), so by Jensen,
[
\frac{1}{3}\bigl(\log\sin A+\log\sin B+\log\sin C\bigr)\le \log\sin\Bigl(\frac{A+B+C}{3}\Bigr)=\log\sin\Bigl(\frac{\pi}{3}\Bigr),
]
hence (\sin A\sin B\sin C \le \sin^3(\pi/3) = (\sqrt3/2)^3 = 3\sqrt3/8), with equality only when (A=B=C=\pi/3), i.e. the triangle is equilateral.

Therefore the maximum area is
[
2R^2\cdot \frac{3\sqrt3}{8}=\frac{3\sqrt3}{4}R^2,
]
as claimed. ∎

Now choose (R_0) so that this maximal triangle area equals (1):
[
\frac{3\sqrt3}{4}R_0^2=1
\quad\Longrightarrow\quad
R_0^2=\frac{4}{3\sqrt3}=\frac{4}{\sqrt{27}},
\quad
R_0=\frac{2}{3^{3/4}}.
]
Then the disk (B_{R}) with any (R<R_0) contains **no** triangle of area (1), while
[
\mu(B_{R_0})=\pi R_0^2=\pi\cdot\frac{4}{3\sqrt3}=\frac{4\pi}{\sqrt{27}}=:c_0.
]
Hence any universal constant (c) (if it exists) must satisfy (c\ge c_0).

---

### Lemma 1 (Triple-intersection continuity near ((0,0)))

Let (A\subset\mathbb R^2) be Lebesgue measurable with (0<\mu(A)<\infty). Define, for (u,v\in\mathbb R^2),
[
J(u,v):=\mu\bigl(A\cap(A-u)\cap(A-v)\bigr)
=\int_{\mathbb R^2} 1_A(x),1_A(x+u),1_A(x+v),dx.
]
Then (J(u,v)\to \mu(A)) as ((u,v)\to(0,0)). In particular, there exists (\rho>0) such that
[
J(u,v)>0\quad\text{for all }u,v\text{ with }|u|<\rho,\ |v|<\rho.
]

**Proof.**
Let (f:=1_A\in L^1(\mathbb R^2)) (since (\mu(A)<\infty)). For (u\in\mathbb R^2), let (T_u f(x):=f(x+u)).

A standard fact (and easy to prove by approximation of (f) by continuous compactly supported functions in (L^1)) is:

> (|T_u f - f|_{L^1}\to 0) as (u\to 0).

Now compute, using that (f\in{0,1}),
[
\mu(A)-J(u,v)
=\int f(x),dx-\int f(x)f(x+u)f(x+v),dx
=\int f(x)\bigl(1-f(x+u)f(x+v)\bigr),dx.
]
For (a,b\in[0,1]), (1-ab\le (1-a)+(1-b)). Applying this pointwise with (a=f(x+u)), (b=f(x+v)), we get
[
1-f(x+u)f(x+v)\le (1-f(x+u))+(1-f(x+v)).
]
Therefore
[
\mu(A)-J(u,v)
\le \int f(x)\bigl(1-f(x+u)\bigr),dx+\int f(x)\bigl(1-f(x+v)\bigr),dx.
]
But
[
\int f(x)\bigl(1-f(x+u)\bigr),dx
=\mu(A)-\mu\bigl(A\cap(A-u)\bigr).
]
Since (\mu(A)=\mu(A-u)), the symmetric difference identity gives
[
\mu(A\triangle(A-u))=2\bigl(\mu(A)-\mu(A\cap(A-u))\bigr),
]
hence
[
\mu(A)-\mu(A\cap(A-u))=\frac12,\mu(A\triangle(A-u))=\frac12|f-T_{-u}f|*{L^1}.
]
Similarly for (v). Thus
[
0\le \mu(A)-J(u,v)\le \frac12|f-T*{-u}f|*{L^1}+\frac12|f-T*{-v}f|_{L^1}.
]
As (u\to0) and (v\to0), both (L^1)-norms tend to (0). Hence (J(u,v)\to\mu(A)).

Since (\mu(A)>0), choose (\rho>0) so small that whenever (|u|<\rho) and (|v|<\rho),
[
\frac12|f-T_{-u}f|*{L^1}+\frac12|f-T*{-v}f|_{L^1}<\frac{\mu(A)}{2},
]
which forces (J(u,v)>\mu(A)/2>0). ∎

---

### Corollary 2 (Every positive finite-measure set contains triangles of all sufficiently small areas)

Let (A\subset\mathbb R^2) be measurable with (0<\mu(A)<\infty). Then there exists (\varepsilon=\varepsilon(A)>0) such that for every real (a) with (0<a<\varepsilon), there exist (x_1,x_2,x_3\in A) with (\operatorname{Area}(x_1,x_2,x_3)=a).

**Proof.**
Let (\rho>0) be as in Lemma 1: (J(u,v)>0) for all (|u|,|v|<\rho).

Set
[
\varepsilon:=\frac{\rho^2}{8}.
]
Fix any (a\in(0,\varepsilon)). Define
[
u:=\left(\frac{\rho}{2},,0\right),\qquad v:=\left(0,,\frac{4a}{\rho}\right).
]
Then (|u|=\rho/2<\rho). Also, since (a<\rho^2/8),
[
|v|=\frac{4a}{\rho}<\frac{4(\rho^2/8)}{\rho}=\frac{\rho}{2}<\rho.
]
Therefore (J(u,v)>0), so (A\cap(A-u)\cap(A-v)\neq\varnothing). Choose (x) in this intersection. Then
[
x,\quad x+u,\quad x+v\in A.
]
Their triangle area is
[
\operatorname{Area}(x,x+u,x+v)=\frac12|\det(u,v)|.
]
But (\det(u,v)=(\rho/2)\cdot(4a/\rho)-0\cdot 0=2a). Hence the area is (\frac12|2a|=a). ∎

**What this gives:** It completely settles a “small-area” analogue: every measurable set of positive finite area contains triangles of *every sufficiently small* area.

---

### Where this falls short of the Erdős question

Freiling–Mauldin show that if (A) has outer measure (>c_0), then (A) contains a triangle with area (>1).
If (A) is compact convex, that already implies the existence of an area-(1) triangle (shrink a (>1) triangle inside the convex set continuously down to area (1)).

But for general measurable sets, the missing step is precisely:

* from “there exists some triangle in (A) of area (>1)” plus “(A) contains triangles of all sufficiently small areas”
* to “(A) contains a triangle of area exactly (1)”.

That step would follow, for example, if one could prove an **intermediate value property** for triangle areas in arbitrary measurable sets (something like: the set of attainable triangle areas is connected, or contains an interval ((0,M]) up to its supremum). No such general theorem is currently established in the sources above, and I do not see a gap-free way to prove it here.

## 5) VERIFICATION

* **Lemma 0 checks:** The maximization on the circle used only a standard and fully justified formula (\text{Area}=2R^2\sin A\sin B\sin C) and concavity of (\log\sin). No hidden assumptions. The computed (R_0^2=4/(3\sqrt3)=4/\sqrt{27}) gives disk area (c_0=4\pi/\sqrt{27}), consistent with the stated conjectured optimum.
* **Lemma 1 checks:**

  * Requires (f=1_A\in L^1), hence (\mu(A)<\infty). Correct.
  * Uses (L^1)-continuity of translations for (L^1) functions; standard and can be proved by approximation with (C_c) functions.
  * The inequality (1-ab\le (1-a)+(1-b)) for (a,b\in[0,1]) is correct.
  * The relation (\mu(A)-\mu(A\cap(A-u)) = \frac12\mu(A\triangle(A-u))) uses (\mu(A)=\mu(A-u)), true by translation invariance. Correct.
  * Uniformity: once (\rho) is chosen from the translation-continuity estimate, the bound holds simultaneously for all (|u|,|v|<\rho). Correct.
* **Corollary 2 checks:** The chosen (u,v) satisfy (|u|,|v|<\rho) whenever (a<\rho^2/8). Determinant computation gives exactly (2a). Hence area exactly (a). Correct.
* **Quantifiers:** Corollary 2 produces (\varepsilon(A)>0) depending on (A), not a universal (\varepsilon). That dependence is unavoidable in the proof.

## 6) FINAL

**UNRESOLVED**

(i) **Strongest fully proved partial result obtained here:**
For every Lebesgue measurable (A\subset\mathbb R^2) with (0<\mu(A)<\infty), there exists (\varepsilon(A)>0) such that (A) contains triangles of every area (a\in(0,\varepsilon(A))) (Corollary 2). Also, any universal constant (c) (if it exists) must satisfy (c\ge c_0=4\pi/\sqrt{27}) (Lemma 0).

(ii) **Exact first gap blocking a full solution:**
One needs a rigorous principle that forces **exact area (1)** once the set has “enough area.” A clean crystallization is:

> (**Gap statement**) For measurable (A\subset\mathbb R^2), if (\sup{\operatorname{Area}(x_1,x_2,x_3):x_i\in A}>1), must (1) belong to the set of attained triangle areas?

This implication holds for convex (A) (by convex interpolation), but is not proved for arbitrary measurable (A).

(iii) **Top 3 next moves (concrete):**

1. Prove a **connectedness/intermediate value theorem** for the set
   (\mathcal T(A):={\operatorname{Area}(x_1,x_2,x_3):x_i\in A}\subset[0,\infty)), e.g. that ((0,\varepsilon(A))\subset\mathcal T(A)) extends to an interval up to (\sup\mathcal T(A)) under mild hypotheses satisfied when (\mu(A)) is large.
2. Strengthen Freiling–Mauldin’s “(\exists) triangle of area (>1)” into a **robust family** of (>1)-area triangles with vertices at density points, enabling a measurable/topological deformation argument that forces an exact level set at area (1).
3. On the counterexample side, attempt an explicit bounded measurable construction where all triangles have areas in ((0,1-\delta)\cup(1+\delta,\infty)) for some (\delta>0), likely via a highly disconnected union of pieces with carefully controlled two-point and three-point geometry.

(iv) **What a minimal counterexample would likely look like (if the answer is negative):**
A bounded measurable set (A) with arbitrarily large (\mu(A)), extremely disconnected (many separated “islands”), arranged so that:

* within islands, attainable triangle areas stay in ((0,1-\delta)),
* between islands, attainable triangle areas jump over (1) (landing (\ge 1+\delta)),
* and crucially, the “two vertices in one island + third vertex far away” mechanism never allows tuning the area through (1).
  This last requirement seems to force strong directional/thickness constraints across islands, which is a major obstacle to building such examples.

If you want, I can also write up (fully formally) the standard implication “outer measure (>c_0\Rightarrow) triangle area (>1\Rightarrow) area (1)” for **compact convex** (A), but that does not resolve the general measurable case that Erdős asked about.

### Definitions and notation

* (\mu) denotes Lebesgue measure on (\mathbb R^2).
* (\mu^*(E)) denotes **Lebesgue outer measure**:
  [
  \mu^*(E)\ :=\ \inf{\mu(U): U\subseteq\mathbb R^2\ \text{open and }E\subseteq U}.
  ]
* For (x_1,x_2,x_3\in\mathbb R^2), define the (unsigned) area of the triangle with these vertices by
  [
  \operatorname{Area}(x_1,x_2,x_3)\ :=\ \frac12\bigl|\det(x_2-x_1,;x_3-x_1)\bigr|.
  ]
* A set (K\subseteq\mathbb R^2) is **convex** if for all (p,q\in K) and all (t\in[0,1]), ((1-t)p+tq\in K).
* A set is **compact convex** if it is compact and convex.

Finally, define the constant
[
c_0\ :=\ \frac{4\pi}{\sqrt{27}}=\frac{4\pi}{3\sqrt3}.
]

---

## Step 1. The Freiling–Mauldin “(>1)-area triangle” theorem (used as input)

Freiling–Mauldin proved:

> (**FM**) If (A\subseteq\mathbb R^2) satisfies (\mu^*(A) > c_0), then there exist (p,q,r\in A) such that (\operatorname{Area}(p,q,r)>1). ([erdosproblems.com][1])

I will **not** re-prove (FM) here (it is the nontrivial geometric-measure-theoretic content). The rest of the implication—getting *exactly* area (1) from area (>1) under convexity—is elementary but must be written carefully.

---

## Step 2. Convex “area interpolation” lemma (fully proved)

### Lemma (Scaling a triangle inside a convex set)

Let (K\subseteq\mathbb R^2) be convex. Suppose (p,q,r\in K) form a nondegenerate triangle, and set
[
A_0 := \operatorname{Area}(p,q,r) > 0.
]
For each (t\in[0,1]), define
[
q_t := (1-t)p+tq,\qquad r_t := (1-t)p+tr.
]
Then:

1. (p,q_t,r_t\in K) for every (t\in[0,1]);
2. (\operatorname{Area}(p,q_t,r_t) = t^2,A_0) for every (t\in[0,1]).

**Proof.**

1. Because (K) is convex and (p,q\in K), every convex combination ((1-t)p+tq) lies in (K). Hence (q_t\in K). Similarly (r_t\in K). Also (p\in K) by assumption.

2. Compute the area by determinants:
   [
   \operatorname{Area}(p,q_t,r_t)
   =\frac12\bigl|\det(q_t-p,;r_t-p)\bigr|.
   ]
   Now
   [
   q_t-p = ((1-t)p+tq)-p = t(q-p),
   \qquad
   r_t-p = ((1-t)p+tr)-p = t(r-p).
   ]
   Using bilinearity of the determinant in its two vector arguments,
   [
   \det(q_t-p,;r_t-p)=\det(t(q-p),;t(r-p)) = t^2\det(q-p,;r-p).
   ]
   Therefore
   [
   \operatorname{Area}(p,q_t,r_t)
   =\frac12\bigl|t^2\det(q-p,;r-p)\bigr|
   =t^2\cdot \frac12\bigl|\det(q-p,;r-p)\bigr|
   =t^2,\operatorname{Area}(p,q,r)
   =t^2A_0.
   ]
   This proves (2). ∎

### Corollary (Exact area (1) inside a convex set once you have (>1))

Let (K\subseteq\mathbb R^2) be convex. If there exist (p,q,r\in K) with (\operatorname{Area}(p,q,r)>1), then there exist (p',q',r'\in K) with (\operatorname{Area}(p',q',r')=1).

**Proof.**
Let (A_0=\operatorname{Area}(p,q,r)>1). Define (t:=A_0^{-1/2}). Then (t\in(0,1)) and (t^2A_0=1).

Apply the lemma: (p,q_t,r_t\in K) and (\operatorname{Area}(p,q_t,r_t)=t^2A_0=1). Let (p':=p), (q':=q_t), (r':=r_t). ∎

---

## Step 3. Putting it together for compact convex sets

### Proposition (Compact convex case at the (c_0) threshold)

Let (K\subseteq\mathbb R^2) be compact convex. If (\mu^*(K)>c_0), then (K) contains the vertices of a triangle of area exactly (1).

**Proof.**
Assume (\mu^*(K)>c_0). By (FM), there exist (p,q,r\in K) with (\operatorname{Area}(p,q,r)>1). ([erdosproblems.com][1])

Since (K) is convex, the corollary above applies: from one triangle of area (>1) in (K), we can produce (by scaling toward one vertex along line segments that remain in (K)) a triangle with area exactly (1) whose vertices are still in (K).

Thus (K) contains the vertices of a triangle of area (1). ∎

### Remark (Replacing outer measure by measure)

Every compact set (K) is Lebesgue measurable, and for Lebesgue measurable sets one has (\mu(K)=\mu^*(K)) (this is part of the Carathéodory construction of Lebesgue measure as the restriction of outer measure to measurable sets). Therefore, for **compact** (K),
[
\mu(K)>c_0 \quad\Longleftrightarrow\quad \mu^*(K)>c_0,
]
so the proposition can be stated as: if a compact convex (K) has Lebesgue measure (>c_0), then (K) contains an area-(1) triangle.

---

That’s the complete “standard implication” chain:

[
\boxed{\mu^*(K)>c_0 \ \xRightarrow{\ \text{Freiling–Mauldin}\ }\ \exists\triangle\subset K \text{ of area }>1\ \xRightarrow{\ \text{convex scaling}\ }\ \exists\triangle\subset K \text{ of area }1.}
]

(With the only non-elementary input being the Freiling–Mauldin theorem.)

[1]: https://www.erdosproblems.com/352 "
  
    Erdős Problem #352
  
"
