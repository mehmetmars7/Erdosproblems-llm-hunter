
1) FORMAL RESTATEMENT

We treat graphs as simple undirected graphs (no loops, no multiple edges).
For a graph G=(V,E), the chromatic number \chi(G) is the least cardinal \kappa such that V can be written as a disjoint union of \kappa independent sets.

Ambiguity: the problem statement says “subgraph on \aleph_1 vertices”.
For an upper bound condition “\chi(\cdot)\le\aleph_0”, this is equivalent to requiring the induced subgraph on that vertex set to have \chi\le\aleph_0 (since deleting edges cannot increase chromatic number).
Accordingly, in what follows “subgraph on X\subseteq V” means the induced subgraph G[X].

Question (A): Does there exist a graph G with |V|=\aleph_2 and \chi(G)=\aleph_2 such that for every X\subseteq V with |X|=\aleph_1 we have \chi(G[X])\le \aleph_0?

Question (B): Does there exist a graph H with |V(H)|=\aleph_{\omega+1} and \chi(H)=\aleph_1 such that for every Y\subseteq V(H) with |Y|=\aleph_{\omega} we have \chi(H[Y])\le \aleph_0?

Edge cases/conventions: We use \aleph_0=|\mathbb N|. For induced subgraphs of size \aleph_1, “\chi\le \aleph_0” means “countably colorable”.


2) QUICK LITERATURE/CONTEXT CHECK

The problem statement itself records:
- Erd\H{o}s–Hajnal proved: for every finite k there exists a graph with chromatic number \aleph_1 in which every subgraph on fewer than \aleph_k vertices has chromatic number \le \aleph_0.
- A variant in [Er69b] with “=\aleph_0” for every subgraph is (if interpreted as non-induced subgraph) trivially impossible because one can delete all edges; hence the intended formulation is the “\le\aleph_0” version above.

I do not use or assert any additional external results beyond what is stated in the problem file.


3) ATTACK PLAN

Proof-track ideas (existence):
- Try to build G via transfinite recursion/forcing so that global chromatic number is \aleph_2 but any \aleph_1-sized induced subgraph admits a countable coloring (“non-reflection” of chromatic number).
- Try to encode a strong anti-chain/club-guessing structure into edges while controlling all \aleph_1-substructures.

Disproof-track ideas (nonexistence in ZFC):
- Attempt to prove a “reflection” theorem: if \chi(G)=\aleph_2 on \aleph_2 vertices, then some \aleph_1-sized induced subgraph must have uncountable chromatic number.
- Attempt to derive an \aleph_1-sized K_{\aleph_1} (or another uncountably-chromatic configuration) inside any \aleph_2-chromatic graph.

Best path here: I do not know a route to a complete proof or counterexample. I therefore record rigorous structural consequences (lemmas) and isolate the precise gap.


4) WORK

Lemma 918.1 (no uncountable clique).
Assume G is a graph such that for every X\subseteq V(G) with |X|=\aleph_1 we have \chi(G[X])\le \aleph_0.
Then G contains no clique of size \aleph_1.

Proof.
Suppose for contradiction that C\subseteq V(G) is a clique and |C|=\aleph_1.
Then in the induced subgraph G[C], every pair of distinct vertices is adjacent.
In any proper coloring of G[C], distinct vertices must receive distinct colors, hence \chi(G[C])=|C|=\aleph_1.
This contradicts the assumption \chi(G[C])\le \aleph_0. \qed

Lemma 918.2 (every \aleph_1-set contains an \aleph_1 independent set).
Assume G is a graph such that for every X\subseteq V(G) with |X|=\aleph_1 we have \chi(G[X])\le \aleph_0.
Then for every X\subseteq V(G) with |X|=\aleph_1 there exists an independent set I\subseteq X with |I|=\aleph_1.

Proof.
Fix X\subseteq V(G) with |X|=\aleph_1.
By assumption, \chi(G[X])\le \aleph_0, so there exists a coloring c:X\to \mathbb N such that each color class c^{-1}(n) is independent.
If every color class c^{-1}(n) were countable, then X=\bigcup_{n\in\mathbb N} c^{-1}(n) would be a countable union of countable sets and hence countable.
But X has size \aleph_1, contradiction.
Therefore there exists n\in\mathbb N such that c^{-1}(n) is uncountable.
Since any uncountable subset of \omega_1 has cardinality \aleph_1, we conclude |c^{-1}(n)|=\aleph_1.
Let I:=c^{-1}(n). Then I\subseteq X is independent and |I|=\aleph_1. \qed

Lemma 918.3 (singular-cardinal variant: large independent sets of some \aleph_n).
Let \alpha\in(0,1) be irrelevant here; this lemma concerns the second question.
Assume H is a graph such that for every Y\subseteq V(H) with |Y|=\aleph_\omega we have \chi(H[Y])\le \aleph_0.
Then for every such Y and every fixed n\in\mathbb N, there exists an independent set I\subseteq Y with |I|\ge \aleph_n.

Proof.
Fix Y\subseteq V(H) with |Y|=\aleph_\omega and a proper coloring c:Y\to\mathbb N.
Write Y=\bigcup_{m\in\mathbb N} C_m where C_m:=c^{-1}(m) is independent.
Fix n\in\mathbb N.
If every C_m had cardinality <\aleph_n, then since \aleph_n is regular and \mathbb N is countable, we would have
|Y|=\Big|\bigcup_{m\in\mathbb N} C_m\Big|<\aleph_n\cdot\aleph_0=\aleph_n,
contradicting |Y|=\aleph_\omega\ge \aleph_n.
Hence there exists m with |C_m|\ge \aleph_n.
Take I:=C_m. \qed

FAST REALITY CHECK.
There is no finite “toy” analogue that captures the set-theoretic content, but two sanity checks are:
- If one replaces “\aleph_1” by “finite”, de Bruijn–Erd\H{o}s compactness prevents \chi(G) from being larger than the maximum finite chromatic number of its finite subgraphs.
- Here the obstruction is genuinely at the level of uncountable subgraphs, so compactness does not apply.


5) VERIFICATION

- Lemma 918.1: Verified that a clique of size \aleph_1 forces chromatic number \aleph_1 (each vertex must be a distinct color).
- Lemma 918.2: Verified the only set-theoretic input is: a countable union of countable sets is countable. This is standard in ZFC. Also used: any uncountable subset of \omega_1 has size \aleph_1.
- Lemma 918.3: Checked cardinal arithmetic: for regular \aleph_n, a countable union of sets each of size <\aleph_n has size <\aleph_n, so cannot reach \aleph_\omega.
- No hidden use of induced vs non-induced: in both lemmas we use induced subgraphs/colorings of the vertex set.


6) FINAL

**UNRESOLVED**
(i) Strongest proved partial result: Under the “every \aleph_1-induced-subgraph is countably colorable” hypothesis, every \aleph_1-sized vertex set contains an independent subset of size \aleph_1 (Lemma 918.2) and in particular there is no clique of size \aleph_1 (Lemma 918.1). For the \aleph_\omega-sized subgraph hypothesis, one can still force the existence of independent sets of size \aleph_n for every fixed n (Lemma 918.3).
(ii) First gap (crisp): Construct in ZFC (or disprove in ZFC) a graph G with |V|=\aleph_2 and \chi(G)=\aleph_2 such that \chi(G[X])\le\aleph_0 for every X\subseteq V with |X|=\aleph_1.
(iii) Top 3 next moves:
  1. Try to formalize the property of Lemma 918.2 as a “large independent set reflection” and see whether it forces \chi(G)\le\aleph_1 by a transfinite coloring recursion.
  2. Attempt a forcing-style construction: build edges along a \square_{\aleph_1}-sequence or club-guessing sequence while explicitly maintaining countable colorings on all \aleph_1 stages.
  3. Attempt a disproof under additional reflection principles (e.g. show that if certain stationary reflection holds then any \aleph_2-chromatic graph reflects to an \aleph_1-chromatic induced subgraph), and isolate which step uses extra axioms.
(iv) Minimal counterexample structure: A counterexample-to-nonexistence would be a graph on \aleph_2 vertices of chromatic number \aleph_2 that nevertheless satisfies the very strong property “every \aleph_1-subset contains an \aleph_1 independent set”. Any ZFC disproof would have to show these two requirements are incompatible.


