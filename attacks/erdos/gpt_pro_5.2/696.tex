
\section*{Erd\H{o}s Problem \#696}

\subsection*{FORMAL RESTATEMENT}
Let $n\in\mathbb{N}$ with $n\ge 1$.
\begin{itemize}
\item Define $h(n)$ to be the maximum integer $\ell\ge 0$ for which there exist \emph{distinct} primes
  \[p_1<p_2<\cdots<p_\ell\]
  such that $p_i\mid n$ for all $i$ and $p_{i+1}\equiv 1\pmod{p_i}$ for all $1\le i<\ell$.
  (If $n=1$, then $h(1)=0$.)
\item Define $H(n)$ to be the maximum integer $u\ge 1$ for which there exist integers
  \[1\le d_1<d_2<\cdots<d_u\]
  such that $d_i\mid n$ for all $i$ and $d_{i+1}\equiv 1\pmod{d_i}$ for all $1\le i<u$.
  (Note $1\mid n$, so $H(n)\ge 1$ always.)
\end{itemize}
Tasks:
\begin{enumerate}
\item Give meaningful upper/lower bounds (``estimates'') for $h(n)$ and $H(n)$ in terms of $n$ and standard arithmetic functions.
\item Decide whether $H(n)/h(n)\to\infty$ for \emph{almost all} $n$ (natural density $1$).
\end{enumerate}

\subsection*{QUICK LITERATURE/CONTEXT CHECK}
The problem statement in the source file says Erd\H{o}s wrote that $h(n)\to\infty$ for almost all $n$ and believed the normal order is $\log_* n$ (iterated logarithm). The ErdosProblems forum threads summarize this context.\footnote{For example, the ErdosProblems discussion page for \#696 and the related \#695 thread.}
I did not find (in a quick search) any claim that the full ``$H(n)/h(n)\to\infty$ a.a.s.'' statement has been resolved as of late 2025.

\subsection*{ATTACK PLAN}
\begin{itemize}
\item \textbf{Upper bounds:} any chain forces rapid growth of the terms, and the terms must all divide $n$, yielding bounds in terms of $\log n$, $\omega(n)$ (distinct primes), and $\tau(n)$ (divisor count).
\item \textbf{Lower bounds / constructions:} if $\operatorname{lcm}(1,2,\dots,m)\mid n$ then the explicit chain $1,2,\dots,m$ gives $H(n)\ge m$.
\item \textbf{``Almost all'' questions:} relate $H(n)$ to the largest $m$ with $\operatorname{lcm}(1,\dots,m)\mid n$, and relate $h(n)$ to prime factors in ``Pratt-tree''-type congruence chains. I only pursue rigorous elementary bounds here.
\end{itemize}

\subsection*{WORK}
\textbf{Lemma 696.1 (basic monotonicity and trivial bounds).}
For all $n\ge 1$,
\[0\le h(n)\le H(n),\qquad h(n)\le \omega(n),\qquad H(n)\le \tau(n),\]
where $\omega(n)$ is the number of distinct prime factors of $n$ and $\tau(n)$ is the number of positive divisors of $n$.

\emph{Proof.}
A prime chain as in the definition of $h(n)$ is in particular a divisor chain (all primes are divisors), so $h(n)\le H(n)$. In such a chain the primes are distinct and divide $n$, so its length is at most the number of distinct prime divisors: $h(n)\le\omega(n)$. Finally, a divisor chain uses distinct positive divisors of $n$, hence has length at most $\tau(n)$. \qed

\medskip
\textbf{Lemma 696.2 (a uniform upper bound on $h(n)$ from forced growth).}
Let $\ell\ge 2$ and let $p_1<p_2<\cdots<p_\ell$ be primes with $p_{i+1}\equiv 1\pmod{p_i}$ for all $i<\ell$. Then
\begin{enumerate}
\item $p_2\ge 3$ and for every $i\ge 2$ one has $p_{i+1}\ge 2p_i+1>2p_i$.
\item Consequently, for every $i\ge 2$ we have $p_i\ge 3\cdot 2^{i-2}$.
\item Consequently, the product of the primes satisfies
\[\prod_{i=1}^{\ell} p_i\ \ge\ 2\cdot 3^{\ell-1}\cdot 2^{(\ell-1)(\ell-2)/2}.
\]
In particular, if all these primes divide $n$ (hence their product divides $n$) then
\[h(n)\le 3+\sqrt{2\log_2 n}.
\]

\emph{Proof.}
(1) Since $p_1$ is prime, either $p_1=2$ or $p_1\ge 3$. In either case $p_2>p_1$ and $p_2\equiv 1\pmod{p_1}$, so $p_2\ge 3$.
Now fix $i\ge 2$. Then $p_i\ge 3$ is odd. Writing $p_{i+1}=1+k p_i$ for some integer $k\ge 1$, the parity of $p_{i+1}$ is the parity of $1+k$ because $p_i$ is odd. For $p_{i+1}$ to be an odd prime, we must have $k$ even, hence $k\ge 2$. Therefore $p_{i+1}=1+k p_i\ge 1+2p_i$.

(2) From (1), $p_2\ge 3$ and $p_{i+1}>2p_i$ for all $i\ge 2$. By induction, for $i\ge 2$,
$p_i>2^{i-2}p_2\ge 3\cdot 2^{i-2}$.

(3) Multiply the bound in (2) for $i=2,3,\dots,\ell$ and include $p_1\ge 2$:
\[
\prod_{i=1}^{\ell}p_i\ge 2\cdot \prod_{i=2}^{\ell}\bigl(3\cdot 2^{i-2}\bigr)
=2\cdot 3^{\ell-1}\cdot 2^{\sum_{i=2}^{\ell}(i-2)}
=2\cdot 3^{\ell-1}\cdot 2^{(\ell-1)(\ell-2)/2}.
\]
If the chain primes all divide $n$, then the left-hand side divides $n$, so taking $\log_2$ and discarding the nonnegative $(\ell-1)\log_2 3$ term gives
$\tfrac{(\ell-1)(\ell-2)}{2}\le \log_2 n$, hence $\ell\le 3+\sqrt{2\log_2 n}$. \qed

\medskip
\textbf{Lemma 696.3 (explicit lower bound for $H(n)$ via $\operatorname{lcm}(1,\dots,m)$).}
Let $m\ge 1$ and $n\ge 1$ be integers. If $\operatorname{lcm}(1,2,\dots,m)\mid n$, then $H(n)\ge m$.

\emph{Proof.}
If $\operatorname{lcm}(1,\dots,m)\mid n$, then every integer $d\in\{1,2,\dots,m\}$ divides $n$. Consider the divisor chain
\[d_1=1<d_2=2<\cdots<d_m=m.
\]
For each $1\le t<m$ we have $d_{t+1}=t+1\equiv 1\pmod{t}$ because $(t+1)-1=t$ is divisible by $t$. Hence this is a valid chain in the definition of $H(n)$ of length $m$, proving $H(n)\ge m$. \qed

\medskip
\textbf{FAST REALITY CHECK (computation).}
I brute-forced $h(n)$ and $H(n)$ for $2\le n\le 200$ by dynamic programming over prime divisors (for $h$) and over all divisors (for $H$). Results:
\begin{itemize}
\item For every $2\le n\le 200$, one has $H(n)>h(n)$.
\item The maximum values observed on $2\le n\le 200$ were $\max h(n)=3$ (e.g. $n=42$ has chain $2,3,7$) and $\max H(n)=6$ (e.g. $n=60$ has chain $1,2,3,4,5,6$).
\item The largest observed ratio was $H(n)/h(n)=3$ (e.g. $n=180$ has $h(180)=2$ and $H(180)=6$).
\end{itemize}

\subsection*{VERIFICATION}
\begin{itemize}
\item Lemma 696.2: the key parity step uses only that $p_i$ is odd for $i\ge 2$; checked separately that $p_2\ge 3$ always.
\item Lemma 696.3: checked that the congruence $t+1\equiv 1\pmod t$ holds for all $t\ge 1$ and that each $t\mid n$ follows from the $\operatorname{lcm}$ hypothesis.
\item Edge cases: $n=1$ gives $h(1)=0$ and $H(1)=1$ (only divisor is $1$), consistent with definitions.
\end{itemize}

\subsection*{FINAL}
\textbf{UNRESOLVED.}
\begin{enumerate}
\item[(i)] Strongest proved partial results here: 
  \begin{itemize}
  \item Uniform upper bound $h(n)\le 3+\sqrt{2\log_2 n}$ (Lemma 696.2).
  \item Construction: if $\operatorname{lcm}(1,\dots,m)\mid n$ then $H(n)\ge m$ (Lemma 696.3), giving along the subsequence $n=\operatorname{lcm}(1,\dots,m)$ that $H(n)\ge m$ while $h(n)\ll \sqrt{\log n}$.
  \end{itemize}
\item[(ii)] First gap: prove any nontrivial \emph{normal-order} lower bound for $H(n)$ or $h(n)$ beyond these elementary inequalities, strong enough to decide whether $H(n)/h(n)\to\infty$ for almost all $n$.
\item[(iii)] Top 3 next moves:
  \begin{enumerate}
  \item Relate $H(n)$ to $m(n):=\max\{m: \operatorname{lcm}(1,\dots,m)\mid n\}$ and estimate the normal order of $m(n)$.
  \item For $h(n)$, study the directed graph on prime divisors of $n$ with edges $p\to q$ if $q\equiv 1\pmod p$; attempt to show typical $n$ has a long directed path (e.g. of length $\gg \log_* n$).
  \item Computationally sample $h(n),H(n)$ for large random $n$ (say up to $10^{12}$) and track empirical growth rates and correlations with small prime power divisibility.
  \end{enumerate}
\item[(iv)] Minimal counterexample structure (to $H(n)/h(n)\to\infty$ a.a.s.): one would need a positive-density set of $n$ with unusually small divisor-consecutive structure (small $m(n)$), but still with $h(n)$ not too small; i.e. integers missing long runs $1,2,\dots,m\mid n$ while still containing relatively long prime congruence chains.
\end{enumerate}


