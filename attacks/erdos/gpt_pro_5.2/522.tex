% Erdos Problem #522

\noindent\textbf{FORMAL RESTATEMENT.}
Let $\epsilon_0,\dots,\epsilon_n$ be i.i.d. uniform in $\{-1,1\}$ and
\[f_n(z)=\sum_{k=0}^n \epsilon_k z^k.
\]
Let $R_n$ be the number of complex roots of $f_n$ in the closed unit disk $\{z\in\mathbb C:|z|\le 1\}$, counted with multiplicity.
Question: Is it true that, with probability $1$,
\[\frac{R_n}{n/2}\to 1\qquad (n\to\infty)?
\]

\bigskip
\noindent\textbf{QUICK LITERATURE/CONTEXT CHECK.}
The problem statement reports that Yakir proved the weaker claim $R_n/(n/2)\to 1$ in probability, and even that
\[\lim_{n\to\infty}\mathbb P(|R_n-n/2|\ge n^{9/10})=0.
\]

\bigskip
\noindent\textbf{ATTACK PLAN.}
\begin{itemize}
\item \emph{Proof track idea 1:} Upgrade Yakir's convergence-in-probability to almost sure convergence by proving summable tail bounds for deviations $|R_n-n/2|$ on a suitable subsequence and controlling interpolation.
\item \emph{Proof track idea 2:} Use an argument principle representation for $R_n$ in terms of $\arg f_n(e^{i\theta})$ and prove a strong law for this winding number.
\item \emph{Disproof track:} Search for an event (occurring infinitely often) that forces $\Omega(n)$ roots to concentrate inside or outside the unit disk (e.g. polynomials with atypically large $|f_n|$ on $|z|=1$ that bias the zero distribution).
\end{itemize}

\bigskip
\noindent\textbf{WORK.}

\medskip
\noindent\textbf{Fast reality check (small $n$ enumeration; Monte Carlo).}
Using the same computations as in Problem~521, we counted the number of roots in the closed unit disk (numerically, tolerance $|r|\le 1+10^{-8}$).
\begin{verbatim}
Exact enumeration:
 n=4: avg_in=2.375, min_in=1, max_in=4
 n=6: avg_in=3.25,  min_in=1, max_in=6
 n=8: avg_in=4.3984375, min_in=1, max_in=8
 n=10: avg_in=5.09375, min_in=1, max_in=10
Monte Carlo (200 samples):
 n=20: avg_in=10.17
 n=50: avg_in=25.145
 n=100: avg_in=49.77
\end{verbatim}
Ratios $\texttt{avg\_in}/(n/2)$ in these experiments were close to $1$.

\medskip
\noindent\textbf{Lemma 522.1 (reversal symmetry and an expectation identity).}
For any $n\ge 0$, define the reversed polynomial
\[f_n^*(z)=z^n f_n(1/z)=\sum_{k=0}^n \epsilon_k z^{n-k}.
\]
Let $R_n^{<}$ (resp. $R_n^{>}$, $R_n^{=}$) denote the number of roots of $f_n$ with $|z|<1$ (resp. $|z|>1$, $|z|=1$), counted with multiplicity. Then
\[\mathbb E R_n^{<} = \mathbb E R_n^{>} = \frac{n-\mathbb E R_n^{=}}{2}.
\]

\noindent\textbf{Proof.}
First, $f_n^*$ has coefficient vector $(\epsilon_n,\dots,\epsilon_0)$ and thus the same distribution as $f_n$ (i.i.d. coefficients).

Next, $f_n(0)=\epsilon_0\ne 0$, so $0$ is not a root and all roots are nonzero.
If $z$ is a root of $f_n$ of multiplicity $m$, then $1/z$ is a root of $f_n^*$ of multiplicity $m$ (this is immediate from $f_n^*(w)=w^n f_n(1/w)$).
Therefore the multiset of moduli is inverted: $|z|<1$ for $f_n$ corresponds to $|1/z|>1$ for $f_n^*$. Hence, for every realization,
\[R_n^{<}(f_n)=R_n^{>}(f_n^*).
\]
Taking expectations and using $f_n\stackrel{d}{=} f_n^*$ gives
\[\mathbb E R_n^{<} = \mathbb E R_n^{>}.
\]
Finally, since a degree-$n$ polynomial has exactly $n$ roots counting multiplicity, we have the deterministic identity
\[R_n^{<}+R_n^{=}+R_n^{>}=n.
\]
Taking expectations and substituting $\mathbb E R_n^{<}=\mathbb E R_n^{>}$ yields the claimed formula.
\hfill$\square$

\medskip
\noindent\textbf{Lemma 522.2 (deterministic root annulus).}
For every realization of $(\epsilon_k)_{0\le k\le n}$, every root $z$ of $f_n$ satisfies $\tfrac12\le |z|\le 2$.

\noindent\textbf{Proof.}
This is exactly Lemma~521.1 applied to $f_n$.
\hfill$\square$

\bigskip
\noindent\textbf{VERIFICATION.}
\begin{itemize}
\item Lemma~522.1: The key step is that $f_n^*$ has the same distribution as $f_n$; this uses only i.i.d. of the coefficients. Also, $f_n(0)\ne 0$ ensures inversion $z\mapsto 1/z$ is valid for every root.
\item The identity $R_n^{<}+R_n^{=}+R_n^{>}=n$ uses only the fundamental theorem of algebra, counting multiplicity.
\end{itemize}

\bigskip
\noindent\textbf{FINAL.} \textbf{UNRESOLVED}

(i) \emph{Strongest proved partial result:}
An exact expectation identity $\mathbb E R_n^{<}=\mathbb E R_n^{>}=(n-\mathbb E R_n^{=})/2$ (Lemma~522.1), and deterministic root localization $\tfrac12\le |z|\le 2$ (Lemma~522.2).

(ii) \emph{First gap (crisp):}
Control $R_n^{=}$ (roots exactly on the unit circle) and, more importantly, upgrade convergence in probability of $R_n$ to almost sure convergence, i.e. prove that
\[\sum_{n=1}^\infty \mathbb P\big(|R_n-n/2|\ge \eta n\big)<\infty\quad\text{for every fixed }\eta>0
\]
or otherwise obtain an almost sure argument.

(iii) \emph{Top 3 next moves (concrete):}
\begin{enumerate}
\item Prove a quantitative tail bound $\mathbb P(|R_n-n/2|\ge n^{\alpha})\le n^{-2}$ for some $\alpha<1$ (even on a sparse subsequence) to apply Borel--Cantelli.
\item Bound the expected number (or probability) of unit-circle roots $R_n^{=}$ sharply (e.g. show $\mathbb E R_n^{=} = o(n)$), which together with Lemma~522.1 would imply $\mathbb E R_n\sim n/2$.
\item Computation: for moderate $n$ (say $n\le 40$) empirically estimate $\mathbb P(|R_n-n/2|\ge n^{0.9})$ via Monte Carlo to guess the correct deviation scale.
\end{enumerate}

(iv) \emph{Minimal counterexample structure:}
A counterexample to almost sure convergence would likely manifest as infinitely many $n$ for which $|R_n-n/2|\ge c n$ (linear deviations). This would require persistent, recurring structure in the random coefficients that biases the zero set substantially inside/outside the unit circle.


