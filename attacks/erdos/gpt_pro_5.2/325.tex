
1) FORMAL RESTATEMENT

Fix an integer $k\ge 3$.
Define
\[
f_{k,3}(x):=\bigl|\{n\in\mathbb Z_{\ge 0}: n\le x\text{ and }n=a^k+b^k+c^k\text{ for some }a,b,c\in\mathbb Z_{\ge 0}\}\bigr|.
\]

Question: is it true that
\[
f_{k,3}(x)\gg x^{3/k}\quad\text{as }x\to\infty,
\]
or even $f_{k,3}(x)\gg_\epsilon x^{3/k-\epsilon}$?

Edge cases.
$f_{k,3}(x)\ge 1$ for all $x\ge 0$ since $0$ is representable.


2) QUICK LITERATURE/CONTEXT CHECK

I do not use any external results beyond what the statement records:

* Mahler and Erd\H{o}s proved $f_{k,2}(x)\gg x^{2/k}$.
* For $k=3$, Wooley proved $f_{3,3}(x)\gg x^{0.917\dots}$.


3) ATTACK PLAN

Give elementary lower bounds (much weaker than the conjectured $x^{3/k}$), using the fact that a sum of two $k$th powers is also a sum of three with the third equal to $0$.
Also compute small values for sanity.


4) WORK

PHASE 1: FAST REALITY CHECK (small-$x$ computation)

Exact computed values include:

For $k=3$:
\[
\begin{array}{c|ccc}
x & 100 & 500 & 1000\\\hline
f_{3,3}(x) & 29 & 98 & 173
\end{array}
\]

For $k=4$:
\[
\begin{array}{c|ccc}
x & 100 & 500 & 1000\\\hline
f_{4,3}(x) & 15 & 30 & 49
\end{array}
\]


Lemma 325.1 (trivial bound).

For any $k\ge 3$ and $x\ge 0$,
\[
f_{k,3}(x)\ge \lfloor x^{1/k}\rfloor+1.
\]

Proof.
As in Lemma 323.1, each $t^k\le x$ is representable as $t^k+0^k+0^k$. \qed


Lemma 325.2 (inherit a power-law lower bound from two-term sums).

Fix $k\ge 3$. There exists $x_0=x_0(k)$ such that for all $x\ge x_0$,
\[
f_{k,3}(x)\ge \frac{1}{8}\,x^{3/(2k)}.
\]

Proof.
Every integer of the form $a^k+b^k$ is also of the form $a^k+b^k+0^k$.
Thus
\[
f_{k,3}(x)\ge f_{k,2}(x).
\]
Lemma 323.2 (proved above) gives $f_{k,2}(x)\ge \tfrac18 x^{3/(2k)}$ for all $x\ge x_0(k)$.
Therefore the same lower bound holds for $f_{k,3}(x)$. \qed


5) VERIFICATION

-- Lemma 325.2 depends only on the inclusion of representable sets and on Lemma 323.2, which was proved from scratch.

-- Computations: brute force over all triples $(a,b,c)$ with $a^k,b^k,c^k\le x$.


6) FINAL

**UNRESOLVED**

(i) Strongest fully proved partial result obtained here.

* Trivial bound $f_{k,3}(x)\ge \lfloor x^{1/k}\rfloor+1$ (Lemma 325.1).
* For $k\ge 3$ and large $x$, a rigorous power-law bound $f_{k,3}(x)\ge \tfrac18 x^{3/(2k)}$ (Lemma 325.2).

(ii) Exact first gap.

Prove the conjectured exponent $3/k$ (or $3/k-\epsilon$): my argument only gives exponent $3/(2k)$.

(iii) Top 3 next moves (concrete targets).

1. Prove that the set of sums $b^k+c^k$ has size $\gg B^2$ up to $2B^k$ for suitable $B$, which would upgrade the method behind Lemma 323.2 when adding a large $a^k$.
2. Use mean-value estimates for exponential sums (circle method) to convert tuple-counting into distinct-value counting.
3. Computationally search for collisions and density of values $a^k+b^k+c^k$ for small $k$ to guess the true growth exponent.

(iv) What a minimal counterexample would likely look like.

A counterexample to $f_{k,3}(x)\gg x^{3/k}$ would require that the map $(a,b,c)\mapsto a^k+b^k+c^k$ has extremely high collision multiplicity on most of its image, forcing the set of distinct values to be $o(x^{3/k})$ despite having about $\asymp x^{3/k}$ input triples.


