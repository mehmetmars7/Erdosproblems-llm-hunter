
1) FORMAL RESTATEMENT

An integer $N\ge 1$ is called \emph{representable} if there exist a prime $p$, an integer $a\ge 1$, and an integer $b$ with $0\le b<p$ such that
\[
 N=ap^2+b.
\]
Question: Is every sufficiently large integer representable?

Variant questions (from the problem text):
- What if the condition that $p$ is prime is omitted (i.e. $p$ is any integer $\ge 2$)?
- Define $c_N$ as the minimal constant such that $N=ap^2+b$ with $0\le b<c_N p$ for some prime $p\le\sqrt N$. Is it true that $c_N$ is eventually $\le 1$ (the main question), or perhaps $\limsup c_N=\infty$? Is $c_N<N^{o(1)}$?

2) QUICK LITERATURE/CONTEXT CHECK

As stated in the problem text:
- The sieve of Eratosthenes implies almost all integers are of the form $ap^2+b$ with $0\le b<p$.
- The Brun--Selberg sieve implies the number of exceptions in $[1,x]$ is $\ll x/(\log x)^c$ for some $c>0$.
- Erd\H{o}s believed it is unlikely that all sufficiently large integers are representable.
- Selfridge and Wagstaff's preliminary computation suggested infinitely many exceptions even when $p$ is not required to be prime.

I do not use any other external results here.

3) ATTACK PLAN

- Convert the representation condition into an explicit modular condition $N\bmod p^2<p$.
- Run a sieve-style computation to list exceptions for small $N$ and estimate the exception density.
- Prove basic structural lemmas about the sets covered by a fixed prime $p$.

4) WORK

FAST REALITY CHECK (exceptions by computation)

Define $N$ to be representable if there exists a prime $p\le\sqrt N$ with $N\bmod p^2<p$.

- Up to $10{,}000$: there are $1327$ exceptions. The first exceptions are
\[
1,2,3,6,7,14,15,22,23,30,31,34,35,39,42,43,\dots
\]
and the largest exception $\le 10{,}000$ is $9995$.

- Up to $1{,}000{,}000$: there are $83{,}812$ exceptions (about $8.3812\%$ of integers $\le 10^6$).

Lemma 676.1 (equivalent modular condition)

An integer $N\ge 1$ is representable if and only if there exists a prime $p\le\sqrt N$ such that
\[
 N\bmod p^2 \in \{0,1,\dots,p-1\}.
\]

Proof.
($\Rightarrow$) If $N=ap^2+b$ with $a\ge 1$ and $0\le b<p$, then by the division algorithm, $b$ is exactly the remainder of $N$ upon division by $p^2$. Hence $N\bmod p^2=b<p$.
Also $a\ge 1$ implies $N\ge p^2$, hence $p\le\sqrt N$.

($\Leftarrow$) Conversely, suppose there exists a prime $p\le\sqrt N$ such that the remainder $b:=N\bmod p^2$ satisfies $0\le b<p$. Write $N=qp^2+b$ with $q=\lfloor N/p^2\rfloor$. Since $p\le\sqrt N$, we have $q\ge 1$. Setting $a:=q$ gives $N=ap^2+b$ with $a\ge 1$ and $0\le b<p$.


Lemma 676.2 (structure and density for fixed $p$)

Fix a prime $p$. Let
\[
S_p:=\{N\ge 1: \exists a\ge 1,\ 0\le b<p\text{ with }N=ap^2+b\}.
\]
Then $S_p$ consists exactly of those integers whose residue modulo $p^2$ lies in $\{0,1,\dots,p-1\}$. Equivalently, for each integer $a\ge 1$,
\[
 [ap^2,\ ap^2+p-1]\subset S_p,
\]
and $S_p$ contains no other integers in the block $[ap^2,\ (a+1)p^2-1]$.
Moreover, $S_p$ has natural density $1/p$.

Proof.
The residue description is immediate from Lemma 676.1 (with $p$ fixed) and the uniqueness of the remainder modulo $p^2$.
For each $a\ge 1$, the integers in the interval $[ap^2,ap^2+p-1]$ are exactly those congruent to $0,1,\dots,p-1$ modulo $p^2$, so they lie in $S_p$, and no other integer in that length-$p^2$ block has such a residue.

For density: in each full residue class block of size $p^2$, exactly $p$ residues are allowed. Hence among $\{1,2,\dots,Ap^2\}$ there are exactly $A\cdot p$ elements of $S_p$. Dividing by $Ap^2$ gives proportion $p/(p^2)=1/p$. Letting $A\to\infty$ yields natural density $1/p$.


5) VERIFICATION

- Lemma 676.1: checks that $a\ge 1$ forces $p\le\sqrt N$, so the restriction $p\le\sqrt N$ is not extra.
- Lemma 676.2: the density computation uses periodicity modulo $p^2$ and is correct.
- Computation consistency: $N=4$ is representable via $p=2$, $a=1$, $b=0$; $N=6$ is not representable because for $p=2$ we have $6\bmod 4=2\not<2$ and no other prime $p$ has $p^2\le 6$.

6) FINAL

\textbf{UNRESOLVED}

(i) Strongest proved partial result:

Representability is equivalent to the modular condition $N\bmod p^2<p$ for some prime $p\le\sqrt N$ (Lemma 676.1). For each fixed prime $p$, the covered set has density $1/p$ and consists of the first $p$ integers in each block of length $p^2$ (Lemma 676.2). Computations show that exceptions are common for small $N$ (about $8.38\%$ up to $10^6$) but coverage increases with $N$.

(ii) First gap (crisp):

I cannot decide whether the set of exceptions is finite or infinite; in particular I cannot prove that every sufficiently large $N$ has some prime $p\le\sqrt N$ with $N\bmod p^2<p$, nor can I construct an infinite family of exceptions.

(iii) Top 3 next moves:

1. Attempt an explicit infinite construction of exceptions by forcing $N\bmod p^2\in\{p,\dots,p^2-1\}$ simultaneously for many small primes $p$ via CRT.
2. Quantitatively sharpen sieve upper bounds for the exception set using only the modular structure of $S_p$.
3. Extend computation to larger ranges and look for structured families of exceptions (e.g. in certain congruence classes) that might guide a CRT construction.

(iv) Minimal counterexample structure:

An infinite counterexample family would likely be built by choosing $N$ in a residue class modulo $\prod_{p\le y} p^2$ that avoids the first $p$ residues mod $p^2$ for every small prime $p\le y$, and then showing that for such $N$ no larger prime $p>y$ can work either (e.g. because $p^2$ is too large or because the remaining residues are still excluded).


