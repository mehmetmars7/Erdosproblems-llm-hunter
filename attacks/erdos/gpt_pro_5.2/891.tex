
\noindent 1) \textbf{FORMAL RESTATEMENT}

\noindent Let $p_1=2<p_2=3<p_3=5<\cdots$ be the primes, and fix an integer $k\ge 2$. Write
\[
P_k := \prod_{i=1}^k p_i
\]
(the $k$-th primorial). The question asks whether there exists $N(k)$ such that for all integers $n\ge N(k)$ there exists an integer
\[
 m\in [n,n+P_k)
\]
with ``$>k$ many prime factors.''

\noindent \emph{Ambiguity note (distinct vs. counted with multiplicity).}
If ``prime factors'' were counted \emph{with multiplicity} ($\Omega(m)$), then for $n>P_k$ the claim would be immediate: the interval $[n,n+P_k)$ contains a multiple $M$ of $P_k$, and $M\ge 2P_k$, so $\Omega(M)\ge \Omega(P_k)+1 = k+1$. This would make the case $k=2$ with interval length $6$ trivial, contradicting the remark in the problem text that the case $k=2$ is unknown. Therefore the intended meaning must be:

\begin{quote}
$\omega(m)>k$, i.e. $m$ has more than $k$ \emph{distinct} prime divisors.
\end{quote}

\noindent Under this interpretation, the question becomes:

\noindent \textbf{Conjecture 891(k).} For each $k\ge 2$ there exists $N(k)$ such that for all $n\ge N(k)$, the interval $[n,n+P_k)$ contains an integer $m$ with $\omega(m)\ge k+1$.

\noindent The text also notes: (a) Schinzel deduced from a theorem of P\'olya about gaps between $k$-smooth numbers that Conjecture 891(k) holds with $P_k$ replaced by $p_1\cdots p_{k-1}p_{k+1}$; (b) Weisenberg observed that Dickson's conjecture would make Conjecture 891(k) false if $P_k$ is replaced by $P_k-1$.

\noindent 2) \textbf{QUICK LITERATURE/CONTEXT CHECK}

\noindent I use only the information stated in the problem text: Schinzel's implication from P\'olya, and Weisenberg's implication from Dickson's conjecture. No external results are assumed.

\noindent 3) \textbf{ATTACK PLAN}

\noindent \emph{Toward a proof (distinct primes).}
\begin{itemize}
\item For $k=2$, try to prove that six consecutive integers cannot all be of the form $p^a$ or $p^a q^b$ for large $n$, perhaps via congruence restrictions modulo small primorials.
\item For general $k$, try to force one number in $[n,n+P_k)$ to be divisible by many small primes, and then show it must also have at least one additional prime factor.
\end{itemize}
\noindent \emph{Toward a disproof.}
\begin{itemize}
\item The Dickson-based construction in the text gives a blueprint for producing long blocks where every number is ``almost prime'' (bounded $\omega$). Seek unconditional approximations to that blueprint.
\end{itemize}

\noindent 4) \textbf{WORK}

\noindent \textbf{Lemma 891.1 (numbers below $P_k$ cannot have $k$ distinct primes).}
If $1\le m < P_k$, then $\omega(m)\le k-1$.

\noindent \emph{Proof.}
Assume for contradiction that $\omega(m)\ge k$. Then $m$ has $k$ distinct prime divisors $q_1<q_2<\cdots<q_k$. The product of the $k$ smallest primes is $P_k=p_1p_2\cdots p_k$, and since $q_i\ge p_i$ for each $i$ we have
\[
 m \ge q_1q_2\cdots q_k \ge p_1p_2\cdots p_k = P_k,
\]
contradicting $m<P_k$. Hence $\omega(m)\le k-1$.
\hfill$\square$

\medskip
\noindent \textbf{Lemma 891.2 (Weisenberg's Dickson-based counterexample for $P_k-1$).}
Assume Dickson's conjecture in the specific form stated in the problem text: letting
\[
L_k := \operatorname{lcm}(1,2,\dots,P_k),
\]
there exist infinitely many integers $n'$ such that for every integer $m$ with $1\le m<P_k$, the number
\[
q_m := \Big(\frac{L_k}{m}\Big) n' + 1
\]
is prime. Then, for $n:=L_k n' + 1$, every integer in the interval $[n,n+P_k-1)$ has at most $k$ distinct prime divisors.

\noindent \emph{Proof.}
Fix an integer $j$ with $0\le j\le P_k-2$ and put $m:=j+1$. Then $1\le m < P_k$, so $m\mid L_k$ and the integer $q_m=(L_k/m)n'+1$ is prime by assumption. Compute
\[
 n+j = (L_k n' + 1) + j = L_k n' + (j+1) = L_k n' + m = m\Big(\frac{L_k}{m}n' + 1\Big) = m\,q_m.
\]
Thus $n+j$ factors as a product of $m$ and a prime $q_m$. Consequently
\[
 \omega(n+j) \le \omega(m) + 1.
\]
(If $q_m$ happens to divide $m$, then $\omega(n+j)=\omega(m)$; in any case the displayed upper bound holds.)
By Lemma 891.1, $\omega(m)\le k-1$ because $m<P_k$. Therefore $\omega(n+j)\le k$ for every $0\le j\le P_k-2$, which is exactly the claim that all integers in $[n,n+P_k-1)$ have at most $k$ distinct prime divisors.
\hfill$\square$

\medskip
\noindent \textbf{FAST REALITY CHECK (computations).}

\noindent The problem highlights the first open case $k=2$, i.e. $P_2=2\cdot 3=6$, asking whether for all sufficiently large $n$ every interval of $6$ consecutive integers contains some $m$ with $\omega(m)\ge 3$.

\noindent I computed $\omega(m)$ for $m\le 5{,}000{,}010$ using a sieve, and then scanned all interval start points $n\le 5{,}000{,}000$.

\begin{itemize}
\item For length $L=6$ (the open case): there were $69$ ``bad'' starts $n\le 5{,}000{,}000$ for which all $m\in[n,n+6)$ have $\omega(m)\le 2$. The \emph{largest} such start in this range was $n=4372$.

For this last bad interval we have the explicit factorisations
\[
\begin{array}{rcl}
4372&=&2^2\cdot 1093,\\
4373&=&4373\ \text{(prime)},\\
4374&=&2\cdot 3^7,\\
4375&=&5^4\cdot 7,\\
4376&=&2^3\cdot 547,\\
4377&=&3\cdot 1459.
\end{array}
\]
In particular each has at most two distinct prime factors.

\item For Schinzel's ``$p_1p_3$'' variant when $k=2$ (interval length $L=2\cdot 5=10$): in the same scan up to $n\le 5{,}000{,}000$, there were $32$ bad starts with no $\omega\ge 3$ in $[n,n+10)$, and the largest such start was $n=92$.
\end{itemize}

\noindent These computations do \emph{not} prove either statement (they only verify them up to $5\times 10^6$), but they give concrete target counterexample structure.

\noindent 5) \textbf{VERIFICATION}

\noindent Lemma 891.1: checked that it uses only the minimal fact that the product of the $k$ smallest distinct primes is $P_k$, hence any integer with $k$ distinct prime divisors is at least $P_k$.

\noindent Lemma 891.2: verified the key identity $n+j = (j+1)q_{j+1}$ and that $j+1<P_k$ on the whole interval $0\le j\le P_k-2$. The inequality $\omega(n+j)\le \omega(j+1)+1$ is unconditional.

\noindent Computational checks: the sieve-based $\omega$ computation and window scan were rerun with limit $5{,}000{,}010$; the reported counts and last bad starts were obtained directly from that scan.

\noindent 6) \textbf{FINAL}

\noindent \textbf{UNRESOLVED.}

\begin{enumerate}[(i)]
\item \emph{Strongest proved partial result here.} Unconditionally, Lemma 891.1 shows $m<P_k\Rightarrow \omega(m)\le k-1$. Conditionally on the Dickson-type hypothesis stated in the problem text, Lemma 891.2 constructs infinitely many intervals of length $P_k-1$ in which \emph{every} integer has $\omega\le k$.

\item \emph{First gap (crisp).} Prove or disprove Conjecture 891(k): for fixed $k\ge 2$, does every sufficiently long translate of length $P_k$ contain an integer with at least $k+1$ \emph{distinct} prime divisors?

\item \emph{Top 3 next moves.}
  \begin{enumerate}
  \item (Computational) Extend the $k=2$ scan far beyond $5\times 10^6$ and search systematically for any new ``bad'' length-$6$ intervals; if none appear, attempt to classify all bad intervals by congruence restrictions.
  \item (Structural for $k=2$) Prove that if $n,n+1,\dots,n+5$ all satisfy $\omega\le 2$, then (after excluding finitely many cases) their residue classes modulo a small primorial force a contradiction (e.g. show one of them must be divisible by $2\cdot 3\cdot 5$ and another distinct prime).
  \item (General $k$) Try to adapt the Dickson blueprint unconditionally: construct (by sieve/CRT) large $n$ such that each $n+j$ has a controlled factorisation with at most $k$ distinct primes, and decide whether such constructions can be made arbitrarily large without Dickson.
  \end{enumerate}

\item \emph{Minimal counterexample structure.} A counterexample to Conjecture 891(2) would be an arbitrarily large $n$ such that each of $n,n+1,\dots,n+5$ is either a prime power or a product of two prime powers (i.e. has at most two distinct prime divisors). The last bad example below $5\times 10^6$ has the mixed shape ``(even with a large odd prime) / prime / (even with a high power of 3) / (odd with primes 5 and 7) / (even with a large odd prime) / (odd with primes 3 and a large odd prime).''
\end{enumerate}

