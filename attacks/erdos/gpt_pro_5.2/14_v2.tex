\section{Round-2 Objective}
\label{sec:round2-objective}
\textbf{Path pursued: (C) obstruction/correction (necessary structural conditions).}

Round~1 left the main gap: either prove the lower bound in (Q1) for all infinite $A\subseteq\mathbb N$, or build a single infinite $A$ with $|\{1,\dots,N\}\setminus B|=o(N^{1/2})$.

In this round I focus on extracting a \emph{fatal-looking structural obstruction} that any putative counterexample to (Q1)/(Q2) would have to satisfy, and I prove it cleanly. Concretely: small complement forces an ``almost injectivity'' constraint on certain difference patterns among the pairs representing the initial segment. This reduces (Q2) to the existence of an infinite nested family of finite sets with a strong \emph{triangular Sidon/Golomb-ruler} property. This is new, explicit, and would settle the problem if one can rule out (or construct) such families.

\section{Round-1 Foundation Used}
\label{sec:round1-foundation}
I rely on the following Round~1 items (as stated there):
\begin{itemize}
\item Lemma~14.1: for each $N$, membership of $n\le N$ in $B$ depends only on $A\cap[1,N]$.
\item Lemma~14.2: the ``small-number'' obstruction in terms of $\min A$.
\item Lemma~14.3: the pair-counting bound in the fully collision-free regime.
\end{itemize}
No Round~1 proofs are reproved.

\section{New Insight / Tool (Round-2)}
\label{sec:new-insight}
The new tool is a deterministic ``parallelogram'' mechanism converting \emph{repeated differences among small-sum pairs} into \emph{guaranteed multiple representations inside the same initial segment}. This is a local-to-local propagation statement (unlike the global energy heuristics sketched in Round~1).

In particular, when one restricts attention to pairs $(a,b)$ with $a+b\le M$, repeated differences cannot be ``hidden'' by pushing induced collisions above $M$; they necessarily create a collision \emph{within} $[1,M]$.

\section{Attack Plan (Round-2)}
\label{sec:attack-plan}
\textbf{Gap after Round~1.}
The principal gap is the absence of either
\begin{itemize}
\item a general lower bound $|[1,N]\setminus B|\gg_\varepsilon N^{1/2-\varepsilon}$ (Q1), or
\item an explicit infinite construction with $|[1,N]\setminus B|=o(N^{1/2})$ (Q2).
\end{itemize}

\textbf{Round~2 target.}
Assume toward (Q2) that $|[1,N]\setminus B|$ is very small (e.g. $o(N^{1/2})$) along a sequence. Using Lemma~14.1, pass to the finite truncation $S:=A\cap[1,M]$ with $M\asymp N$. Prove that small complement forces strong \emph{difference constraints} on the graph of pairs $(a,b)$ with $a+b\le M$.

\textbf{Why this helps.}
It converts (Q2) into the existence of a nested family $S_M\subset[1,M]$ with:
\begin{enumerate}
\item $|[1,M]\setminus B(S_M)|=o(\sqrt M)$, and
\item an ``almost Golomb ruler'' constraint on differences among the pairs realizing $B(S_M)$.
\end{enumerate}
Any future resolution can now aim directly at ruling out (or constructing) such objects.

\section{Work (Round-2)}
\label{sec:work}
Throughout, for a finite set $S\subseteq\mathbb N$ we write
\[
 r_S(n):=|\{\{a,b\}:a,b\in S,\ a\le b,\ a+b=n\}|,
\qquad B(S):=\{n\in\mathbb N:r_S(n)=1\}.
\]
When additionally $S\subseteq[1,M]$ we abbreviate $\mathrm{Bad}(S;M):=[1,M]\setminus B(S)$.

\subsection{A parallelogram lemma in the ``triangle'' $a+b\le M$}
\label{subsec:parallelogram}
\begin{lemma}[Parallelogram collision forces a bad sum inside the same initial segment]
\label{lem:parallelogram}
Let $M\ge 2$ and let $S\subseteq[1,M]$. Fix an integer $d\ge 1$.
Assume there exist two distinct starting points $a<c$ such that
\[
 a,a+d,c,c+d\in S\quad\text{and}\quad 2a+d\le M,\ \ 2c+d\le M.
\]
Then the integer
\[
 n:=a+c+d
\]
lies in $[1,M]$ and satisfies $r_S(n)\ge 2$, hence $n\in\mathrm{Bad}(S;M)$.
\end{lemma}

\begin{proof}
First, the two displayed inequalities imply $a,c\le (M-d)/2$, hence
\[
 n=a+c+d\le (M-d)/2+(M-d)/2+d=M.
\]
Now note the two (unordered) representations
\[
 n=a+(c+d)=(a+d)+c.
\]
Because $d\ge 1$ and $a<c$, the pairs $\{a,c+d\}$ and $\{a+d,c\}$ are distinct, so $r_S(n)\ge 2$.
\end{proof}

\begin{remark}[Why the ``triangle'' condition matters]
Lemma~\ref{lem:parallelogram} fails without the constraints $2a+d,2c+d\le M$: if two pairs have the same difference but lie high up in $[1,M]$, the induced collision sum $a+c+d$ could exceed $M$ and thus not contribute to $\mathrm{Bad}(S;M)$. The triangle condition prevents this escape.
\end{remark}

\subsection{Many same-difference small-sum pairs force many bad sums}
\label{subsec:samediff}
For $d\ge 1$ define the set of admissible starting points
\[
 P_d(S;M):=\{a\in S: a+d\in S\ \text{and}\ \ 2a+d\le M\}.
\]
Equivalently, $a\in P_d(S;M)$ iff the unordered pair $\{a,a+d\}$ is a representation of some $n\le M$.

\begin{corollary}[Same difference inside $a+b\le M$ yields many distinct bad sums]
\label{cor:distinct-bad-from-one-d}
Let $M\ge 2$, $S\subseteq[1,M]$, and $d\ge 1$. Write $k_d:=|P_d(S;M)|$.
If $k_d\ge 2$, then there exist $k_d-1$ \emph{distinct} integers $n\le M$ with $r_S(n)\ge 2$.
In particular,
\[
 k_d-1\le |\mathrm{Bad}(S;M)|.
\]
\end{corollary}

\begin{proof}
List $P_d(S;M)=\{a_1<\cdots<a_{k_d}\}$. For each $j\in\{2,\dots,k_d\}$ apply Lemma~\ref{lem:parallelogram} to the two pairs $(a_1,a_1+d)$ and $(a_j,a_j+d)$. This shows that
\[
 n_j:=a_1+a_j+d\le M
\]
and $r_S(n_j)\ge 2$. The integers $n_j$ are distinct because $a_j$ are distinct.
Thus $\{n_2,\dots,n_{k_d}\}$ is a set of $k_d-1$ distinct bad sums, giving $k_d-1\le |\mathrm{Bad}(S;M)|$.
\end{proof}

\subsection{A structural necessary condition for (Q2)}
\label{subsec:necessary}
We now translate Corollary~\ref{cor:distinct-bad-from-one-d} into a concrete obstruction for any putative counterexample.

\begin{proposition}[Small complement forces ``triangular'' near-Golomb behavior]
\label{prop:triangular-golomb-obstruction}
Let $A\subseteq\mathbb N$ be infinite, and let $M\ge 2$.
Set $S:=A\cap[1,M]$ and $E_M:=|\mathrm{Bad}(S;M)|=|[1,M]\setminus B|$.
Then:
\begin{enumerate}
\item \textbf{(Size lower bound.)}
The truncation $S$ must satisfy
\[
 \binom{|S|+1}{2}\ \ge\ M-E_M.
\]
Equivalently,
\begin{equation}
\label{eq:size-lb}
 |S|\ \ge\ \frac{\sqrt{1+8(M-E_M)}-1}{2}.
\end{equation}
In particular, if $E_M=o(\sqrt M)$ along a sequence $M\to\infty$, then
\[
 |A\cap[1,M]|\ \ge\ (\sqrt2+o(1))\sqrt M.
\]

\item \textbf{(Per-difference multiplicity bound in the triangle.)}
For every integer $d\ge 1$ one has
\begin{equation}
\label{eq:per-d-bound}
 |P_d(S;M)|\le E_M+1.
\end{equation}
Equivalently: among the pairs $\{a,a+d\}\subseteq S$ that actually contribute to sums $\le M$, the fixed difference $d$ occurs at most $E_M+1$ times.

\item \textbf{(Reduction of (Q2) to an ``almost triangular Golomb ruler'' family.)}
If (Q2) holds, i.e. there exists $A$ with $E_M=o(\sqrt M)$ as $M\to\infty$, then there is an infinite sequence $M_j\to\infty$ and finite sets $S_j\subseteq[1,M_j]$ with $S_j=A\cap[1,M_j]$ such that
\begin{equation}
\label{eq:triangular-golomb-family}
 |S_j|=(\sqrt2+o(1))\sqrt{M_j}
\quad\text{and}\quad
 \sup_{d\ge 1}|P_d(S_j;M_j)|=o(\sqrt{M_j}).
\end{equation}
In particular, all differences inside the ``triangle'' $a+(a+d)\le M_j$ have sub-$\sqrt{M_j}$ multiplicity.
\end{enumerate}
\end{proposition}

\begin{proof}
\emph{(1)} Each $n\in B\cap[1,M]$ has exactly one representation $n=a+b$ with $a\le b$ and $a,b\in S$. Distinct $n$ give distinct unordered pairs $\{a,b\}$. Thus the number of such pairs is exactly $|B\cap[1,M]|=M-E_M$. But $S$ has only $\binom{|S|+1}{2}$ unordered pairs total, hence $\binom{|S|+1}{2}\ge M-E_M$. Solving the quadratic yields \eqref{eq:size-lb}.

\emph{(2)} Fix $d\ge 1$. Let $k_d=|P_d(S;M)|$. If $k_d\le 1$ there is nothing to prove. If $k_d\ge 2$, Corollary~\ref{cor:distinct-bad-from-one-d} produces $k_d-1$ distinct integers in $\mathrm{Bad}(S;M)$, so $k_d-1\le E_M$, i.e. $k_d\le E_M+1$.

\emph{(3)} If $E_M=o(\sqrt M)$, then (1) implies $|S|\ge (\sqrt2+o(1))\sqrt M$. Also, (2) implies $\sup_d |P_d(S;M)|\le E_M+1=o(\sqrt M)$. Taking a subsequence $M_j\to\infty$ gives \eqref{eq:triangular-golomb-family}.
\end{proof}

\begin{remark}[Interpretation]
Equation~\eqref{eq:triangular-golomb-family} is a concrete necessary condition for any counterexample to (Q1)/(Q2): the truncations must have nearly minimal size $\sim \sqrt{2M}$ (forced by mere counting) \emph{and} must avoid large multiplicities of fixed differences among those pairs that actually land inside $[1,M]$.
This is reminiscent of constructing near-extremal Golomb rulers/Sidon-type objects, but in the \emph{triangular} region $a+b\le M$ rather than in the full difference set.
\end{remark}

\section{Adversarial Verification}
\label{sec:adversarial}
I check the new statements against edge cases and quantifiers.
\begin{itemize}
\item \textbf{Ordered/unordered issue.} Everything is stated for unordered representations with $a\le b$ (Round~1 convention). In Lemma~\ref{lem:parallelogram} the two representations are by two \emph{distinct unordered pairs} because $d\ge 1$ and $a<c$; thus $r_S(n)\ge 2$ is valid under the unordered convention.

\item \textbf{The case $d=0$.} The parallelogram mechanism fails for $d=0$ because $a+a=c+c$ type equal-difference does not produce two distinct unordered pairs for the cross-sum. Accordingly Lemma~\ref{lem:parallelogram} and the corollary explicitly require $d\ge 1$.

\item \textbf{Range check.} The triangle hypotheses $2a+d\le M$ and $2c+d\le M$ are precisely what guarantees $n=a+c+d\le M$; without them the derived collision could lie above $M$ and would not contribute to $\mathrm{Bad}(S;M)$.

\item \textbf{Per-$d$ bound in Proposition~\ref{prop:triangular-golomb-obstruction}(2).}
Potential pitfall: bad sums produced for a fixed $d$ could coincide with each other, invalidating $k_d-1\le E_M$.
This is addressed: for fixed $d$ we produced sums $n_j=a_1+a_j+d$ with distinct $a_j$, hence distinct $n_j$.
Overlaps with bad sums coming from other differences do not affect the inequality $k_d-1\le E_M$ because it is a bound for a fixed $d$.

\item \textbf{Use of Round~1 Lemma~14.1.} Proposition~\ref{prop:triangular-golomb-obstruction} is stated directly for $S=A\cap[1,M]$ and $E_M=|[1,M]\setminus B|$; by Lemma~14.1 this $E_M$ indeed depends only on the truncation $S$, so the reduction is well-posed.
\end{itemize}
No hidden assumptions were found beyond $A\subseteq\mathbb N$ and the adopted representation convention.

\section{Final}
\label{sec:final}
\textbf{UNRESOLVED (BUT STRICTLY ADVANCED).}

Round~2 establishes a new, fully rigorous obstruction: if one hopes for $|\{1,\dots,N\}\setminus B|=o(N^{1/2})$, then the truncations $S=A\cap[1,M]$ must simultaneously have size $(\sqrt2+o(1))\sqrt M$ (forced by counting) and must satisfy a strong per-difference multiplicity constraint \eqref{eq:per-d-bound} inside the triangle $a+(a+d)\le M$.
This reduces (Q2) to the existence of an infinite nested family of finite sets with ``almost triangular Golomb ruler'' behavior \eqref{eq:triangular-golomb-family}.

What remains open is whether such families exist; ruling them out (or constructing them) would resolve (Q1)/(Q2).

\section{Completion Estimate}
\label{sec:completion}
\textbf{COMPLETION: 55\%}

\section{References}
\label{sec:references}
No external theorems beyond elementary counting were invoked in Round~2.
