\section*{Problem \#654}

\subsection*{1) FORMAL RESTATEMENT}

Let $X=\{x_1,\dots,x_n\}\subset\mathbb{R}^2$ be a set of $n$ distinct points such that \emph{no four points of $X$ lie on a common circle}.
For each $x\in X$, define
\[
R_X(x)=\left\lvert \{\mathrm{dist}(x,y):y\in X\setminus\{x\}\} \right\rvert.
\]

\medskip
\noindent
\textbf{Question.}
Must there exist a point $x\in X$ with
\[
R_X(x)\ge (1-o(1))n\qquad\text{as }n\to\infty?
\]
Equivalently: does every such $X$ contain a point from which all but $o(n)$ of the $n-1$ distances to other points are distinct?

\subsection*{2) QUICK LITERATURE/CONTEXT CHECK}

\begin{itemize}[leftmargin=2.2em]
\item The problem is attributed to Erd\H{o}s--Pach (1990).
\item A trivial lower bound holds for \emph{every} point: since no circle can contain four points of $X$, for fixed $x\in X$ and fixed radius $r$, the circle of radius $r$ centered at $x$ contains at most $3$ points of $X$.
Hence each distance value from $x$ occurs at most $3$ times, so $R_X(x)\ge (n-1)/3$.
\item No general improvement beyond $(n-1)/3$ is known to me that approaches $(1-o(1))n$.
There is related literature on ``distinct distances under local conditions'' (Fox--Pach--Suk) and on distinct distances in general position, but it does not directly settle this single-point, near-$n$ requirement.
\end{itemize}

\subsection*{3) ATTACK PLAN}

\begin{enumerate}[leftmargin=2.2em]
\item Translate the condition $R_X(x)\le (1-\varepsilon)n$ into a statement about many repeated distances from $x$.
Because multiplicity per distance is at most $3$, having $R_X(x)\le (1-\varepsilon)n$ forces $\Omega(\varepsilon n)$ many radii where the circle centered at $x$ contains 2 or 3 points.
\item Attempt a global contradiction if \emph{every} point has many such repeated radii.
Potential tools: incidence bounds for point--circle systems, and bounds on the number of isosceles triangles (each repeated distance at $x$ yields at least one isosceles triangle with apex $x$).
\item Try to show that forcing too many repeated radii at \emph{all} points necessarily creates a forbidden 4-cocircular configuration.
\end{enumerate}

\subsection*{4) WORK}

\paragraph{The trivial $(n-1)/3$ bound (spelled out).}
Fix $x\in X$.
For each $r>0$, let
\[
M_x(r):=\left\lvert \{y\in X\setminus\{x\}:\mathrm{dist}(x,y)=r\} \right\rvert.
\]
Then $\sum_r M_x(r)=n-1$ and $R_X(x)=\left\lvert \{r:M_x(r)\ge 1\} \right\rvert$.
If $M_x(r)\ge 4$ for some $r$, then the circle of radius $r$ centered at $x$ contains at least four points of $X$ (namely those $y$ with $\mathrm{dist}(x,y)=r$), contradicting the hypothesis.
Therefore $M_x(r)\le 3$ for all $r$.
Hence
\[
 n-1=\sum_r M_x(r)\le 3\,\left\lvert \{r:M_x(r)\ge 1\} \right\rvert=3R_X(x),
\]
so
\begin{equation}\label{eq:trivial654}
R_X(x)\ge \frac{n-1}{3}\qquad\text{for every }x\in X.
\end{equation}

\paragraph{Reformulation via isosceles triangles.}
For fixed $x\in X$, the number of isosceles triangles with apex $x$ equals
\[
T(x):=\sum_r \binom{M_x(r)}{2}.
\]
Because $M_x(r)\le 3$, each term is at most $\binom{3}{2}=3$, hence
\[
T(x)\le 3R_X(x)\le 3(n-1).
\]
Conversely, if $R_X(x)=n-1-s$ (meaning $s$ ``missing'' distinct distances compared to the generic maximum), then the multiset $(M_x(r))_r$ has $\sum_r (M_x(r)-1)=s$.
Under the cap $M_x(r)\le 3$, this forces at most linear many isosceles triangles at $x$:
indeed $\binom{2}{2}=1$ and $\binom{3}{2}=3$, so $T(x)\le 3s$.
Thus the conjectured conclusion $R_X(x)=(1-o(1))n$ is equivalent to the existence of some $x$ with $T(x)=o(n)$.

\medskip
\noindent
\textbf{First gap:}
I do not see a known global upper bound forcing $\min_x T(x)=o(n)$ from the no-4-cocircular condition alone.
The condition allows $T(x)$ to be linear for every $x$ (consistent with \eqref{eq:trivial654}), and showing that some $T(x)$ must actually be $o(n)$ appears to require new structure.

\subsection*{5) VERIFICATION / FAST REALITY CHECK}

\begin{itemize}[leftmargin=2.2em]
\item Generic point sets satisfy the hypothesis (no four cocircular with probability $1$) and typically have $R_X(x)=n-1$ for every $x$, so the desired conclusion holds easily there.
Thus the difficulty is purely worst-case.
\item Collinear $n$-point sets also satisfy the hypothesis (any circle meets a line in at most two points).
In that case, endpoints have $R_X(x)=n-1$, again consistent with the conjectured conclusion.
\item Any potential counterexample must therefore enforce many repeated distances from \emph{every} point while still avoiding any circle containing four of the points.
\end{itemize}

\subsection*{6) FINAL}

\textbf{UNRESOLVED}

\begin{itemize}[leftmargin=2.2em]
\item[(i)] \textbf{Farthest reached:}
Established the trivial universal bound $R_X(x)\ge (n-1)/3$ for every $x$ and reformulated the goal as finding $x$ with only $o(n)$ isosceles triangles as apex.
\item[(ii)] \textbf{Exact first gap:}
No argument is given that improves $(n-1)/3$ to $(1-\varepsilon)n$ for any fixed $\varepsilon>0$, nor a construction showing the opposite.
The missing ingredient is a global constraint that forces at least one point to have very few repeated distances.
\item[(iii)] \textbf{Most promising next moves:}
\begin{enumerate}[leftmargin=2.2em]
\item Develop a quantitative ``circle-richness'' incidence bound adapted to circles \emph{centered at points of $X$} with bounded co-circularity, to show that not all points can have $\Theta(n)$ radii with multiplicity $\ge 2$.
\item Translate repeated distances at many centers into many coincident perpendicular bisectors, then use rich-bisector bounds (of the type used in crossing-number distinct-distance arguments) to force a forbidden cocircularity.
\item Explore whether adding the mild extra assumption ``no three collinear'' (full general position) changes the answer; some known distinct-distance results under local conditions may become more effective in that regime.
\end{enumerate}
\item[(iv)] \textbf{What would finish it:}
\begin{itemize}[leftmargin=2.2em]
\item Either prove that in every $n$-point set with no four cocircular there exists $x$ with $R_X(x)\ge n-n^{\gamma}$ for some $\gamma<1$ (already implies $(1-o(1))n$),
\item or give an explicit construction (for infinitely many $n$) with no four cocircular but with $\max_{x\in X} R_X(x)\le (1-\varepsilon)n$ for some fixed $\varepsilon>0$.
\end{itemize}
\end{itemize}

\subsection*{7) COMPLETION ESTIMATE}

Not complete (open problem as far as I can verify quickly); this writeup isolates equivalent formulations and the current bottleneck.


%%%%%%%%%%%%%%%%%%%%%%%%%%%%%%%%%%%%%%%%%%%%%%%%%%%%%%%%%%%%%%%%%%%%%%%%%%%%%%%
\begin{thebibliography}{99}

\bibitem{Mathialagan2021}
S.~Mathialagan,
\newblock \emph{On bipartite distinct distances in the plane},
\newblock The Electronic Journal of Combinatorics \textbf{28}(4) (2021), \#P4.33.

\bibitem{Elekes1995}
G.~Elekes,
\newblock \emph{Circle grids and bipartite graphs of distances},
\newblock Combinatorica \textbf{15} (1995), 167--174.

\bibitem{CsizmadiaIsmailescu1997}
Gy.~Csizmadia and D.~Ismailescu,
\newblock \emph{Maximum number of different distance counts},
\newblock in: \emph{Intuitive Geometry} (Budapest, 1995),
Bolyai Society Mathematical Studies \textbf{6} (1997), 301--309.

\bibitem{ErdosPach1990}
P.~Erd\H{o}s and J.~Pach,
\newblock \emph{Variations on the theme of repeated distances},
\newblock Combinatorica \textbf{10} (1990), 261--269.

\end{thebibliography}

