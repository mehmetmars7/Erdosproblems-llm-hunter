
Let $\epsilon>0$. Is it true that, for all large $n$, the number of divisors of $n$ in $(n^{1/2},n^{1/2}+n^{1/2-\epsilon})$ is $O_\epsilon(1)$? Erd\H{o}s attributes this conjecture to Ruzsa. Erd\H{o}s and Rosenfeld \cite{ErRo97} proved that there are infinitely many $n$ such that there are four divisors of $n$ in $(n^{1/2},n^{1/2}+n^{1/4})$.

1) FORMAL RESTATEMENT

Fix $\epsilon>0$. For each integer $n\ge 1$, let
\[I_{\epsilon}(n):=\big(\sqrt n,\ \sqrt n+n^{1/2-\epsilon}\big).
\]
Let
\[r_{\epsilon}(n):=\big|\{d\mid n:\ d\in I_{\epsilon}(n)\}\big|.
\]
Question: Is it true that for each $\epsilon>0$ there exist constants $C(\epsilon)$ and $n_0(\epsilon)$ such that
\[r_{\epsilon}(n)\le C(\epsilon)\quad\text{for all integers }n\ge n_0(\epsilon)?
\]

2) QUICK LITERATURE/CONTEXT CHECK

The problem statement notes:
- Erd\H{o}s attributes the conjecture to Ruzsa.
- Erd\H{o}s--Rosenfeld (1997) proved there are infinitely many $n$ with $r_{1/4}(n)\ge 4$ (i.e. four divisors in $(\sqrt n,\sqrt n+n^{1/4})$).
I do not use any additional external results.

3) ATTACK PLAN

- Relate divisors in $(\sqrt n,\sqrt n+H)$ to “near-square” factorizations $n=ab$ with $b-a$ small.
- Prove deterministic inequalities translating divisor-location bounds into bounds on factor differences.
- Compute $r_{\epsilon}(n)$ for $n$ up to a moderate range to sanity-check growth.

4) WORK

Lemma 886.1 (Divisors just above $\sqrt n$ correspond to near-equal factor pairs).
Let $n\ge 1$ and let $H$ satisfy $0<H\le \sqrt n$.
If $d\mid n$ and
\[\sqrt n<d<\sqrt n+H,\]
then with $a:=n/d$ (an integer) one has $a<\sqrt n$ and
\[0<d-a<2H+\frac{H^2}{\sqrt n} \le 3H.
\]
In particular, every such divisor gives a factorization $n=ab$ with $a<b$ and $b-a<3H$.

Proof.
Let $d\mid n$ with $d>\sqrt n$ and set $a=n/d\in\mathbb N$. Then $a<\sqrt n$ since $ad=n<d^2$.
Also
\[d-a = d-\frac{n}{d}=\frac{d^2-n}{d}.
\]
Because $d<\sqrt n+H$,
\[d^2-n < (\sqrt n+H)^2-n = 2H\sqrt n+H^2.
\]
Dividing by $d>\sqrt n$ gives
\[d-a=\frac{d^2-n}{d} < \frac{2H\sqrt n+H^2}{\sqrt n}=2H+\frac{H^2}{\sqrt n}.
\]
If additionally $H\le \sqrt n$, then $H^2/\sqrt n\le H$, hence $d-a<3H$. \qed

Lemma 886.2 (Localization of the complementary divisor).
Under the assumptions of Lemma 886.1 with $H\le \sqrt n$, the complementary divisor $a=n/d$ satisfies
\[\sqrt n-3H<a<\sqrt n.
\]

Proof.
From Lemma 886.1, $d-a<3H$, i.e. $a>d-3H$.
Since $d>\sqrt n$, this gives $a>\sqrt n-3H$. Also $a<\sqrt n$ was proved in Lemma 886.1. \qed

Lemma 886.3 (Connection to the factor-difference set $D(n)$).
Let $n\ge 1$ and $H\le \sqrt n$.
If $d\mid n$ lies in $(\sqrt n,\sqrt n+H)$ and $a=n/d$, then the factor difference $\delta:=d-a$ satisfies
\[\delta\in D(n)\quad\text{and}\quad 0<\delta<3H.
\]
Consequently,
\[r_{\epsilon}(n)\le \big|D(n)\cap\{1,2,\dots,\lfloor 3n^{1/2-\epsilon}\rfloor\}\big|.
\]

Proof.
The factorization $n=ad$ shows $|d-a|\in D(n)$ by definition of $D(n)$ (Problem #885).
Lemma 886.1 gives $0<d-a<3H$. Taking $H=n^{1/2-\epsilon}$ yields the displayed bound on $r_{\epsilon}(n)$. \qed

FAST REALITY CHECK (computation).
For $n\le 10^6$, the maximum observed values of $r_{\epsilon}(n)$ for several $\epsilon$ were:
- $\epsilon=0.25$: max $r_{0.25}(n)=1$ (no instance with $\ge 2$ divisors in $(\sqrt n,\sqrt n+n^{1/4})$ up to $10^6$).
- $\epsilon=0.20$: max $r_{0.20}(n)=4$ at $n=712800$, with divisors in the interval equal to $\{864,880,891,900\}$.
- $\epsilon=0.15$: max $r_{0.15}(n)=5$ at $n=194040$, with divisors $\{441,462,490,495,504\}$.
- $\epsilon=0.10$: max $r_{0.10}(n)=9$ at $n=720720$, with divisors $\{858,880,910,924,936,990,1001,1008,1040\}$.
These computations are only sanity checks and do not address the asymptotic question.

5) VERIFICATION

- Lemma 886.1: verified the inequality chain carefully; the only place a constant appears is when using $H\le\sqrt n$ to deduce $H^2/\sqrt n\le H$.
- Lemma 886.3: checked that the mapping from a divisor $d>\sqrt n$ to the factor difference $d-n/d$ is well-defined and produces an element of $D(n)$.
- Computation: ensured the interval is open; since divisors are integers, boundary inclusion does not affect counts unless $\sqrt n$ is integer or $\sqrt n+H$ integer, but those only change the count by at most 1. The reported divisors are strictly between the endpoints numerically.

6) FINAL

**UNRESOLVED**

(i) Strongest proved partial result.
- Divisors in $(\sqrt n,\sqrt n+H)$ correspond to factor pairs $n=ab$ with $b-a<3H$ (Lemmas 886.1--886.2).
- Such divisors inject into small values of the factor-difference set $D(n)$ (Lemma 886.3), linking this problem to #885.
- Explicit computations up to $10^6$ give small maxima for several $\epsilon$ values.

(ii) First gap (crisp).
For fixed $\epsilon>0$, prove or disprove the existence of an absolute bound $C(\epsilon)$ for
\[r_{\epsilon}(n)=|\{d\mid n:\ \sqrt n<d<\sqrt n+n^{1/2-\epsilon}\}|\]
that holds for all sufficiently large $n$.

(iii) Top 3 next moves.
1. Use Lemma 886.3 to attempt to bound the number of small values in $D(n)$ uniformly in $n$ for a given scaling $n^{1/2-\epsilon}$.
2. Try to construct families of $n$ with many divisors near $\sqrt n$ by forcing many factorizations $n=a(a+\delta)$ with $\delta\ll n^{1/2-\epsilon}$; quantify how many distinct $\delta$ can coexist.
3. Extend computation to search for instances with $r_{1/4}(n)\ge 4$ (known to exist infinitely often per the statement) to guess structural patterns.

(iv) Minimal counterexample structure.
If the conjecture is false for some $\epsilon$, a minimal counterexample would be a sequence $n_m\to\infty$ with $r_{\epsilon}(n_m)\to\infty$. By Lemma 886.1 this would yield infinitely many distinct factorizations $n_m=a(a+\delta)$ with $\delta\ll n_m^{1/2-\epsilon}$, i.e. $n_m$ admitting many “near-square” factorizations with very small factor gaps.
