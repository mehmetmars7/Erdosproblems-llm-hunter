
\noindent\textbf{FORMAL RESTATEMENT.}
A (simple) graph is \emph{triangle-free} if it has no $3$-cycle.
Let $f(n)$ be the maximum possible chromatic number among all triangle-free graphs on $n$ vertices:
\[f(n):=\max\{\chi(G): |V(G)|=n,\ G\ \text{triangle-free}\}.
\]
Similarly, let $g(m)$ be the maximum possible chromatic number among all triangle-free graphs with $m$ edges:
\[g(m):=\max\{\chi(G): |E(G)|=m,\ G\ \text{triangle-free}\}.
\]
Question: estimate $f(n)$ (and $g(m)$).

\medskip
\noindent\textbf{QUICK LITERATURE/CONTEXT CHECK (from the problem file only).}
The text records that bounds for the Ramsey number $R(3,k)$ imply $f(n)\asymp (n/\log n)^{1/2}$ and gives the best known constant-factor bounds
\[(1-o(1))(n/\log n)^{1/2}\le f(n)\le (2+o(1))(n/\log n)^{1/2}.
\]
It also records bounds for $g(m)$: an upper bound $g(m)\le (3^{5/3}+o(1))(m/(\log m)^2)^{1/3}$ and a lower bound $g(m)\gg (m/(\log m)^2)^{1/3}$.
I do not reprove these deep results.

\medskip
\noindent\textbf{ATTACK PLAN.}
\begin{itemize}
\item Provide a fully rigorous (but weaker) construction showing $f(n)\to\infty$ using the Mycielski construction.
\item Prove (elementary) structural lemmas about Mycielski graphs: triangle-freeness, chromatic number increment, and explicit vertex counts under iteration.
\item Reality check: compute small instances of Mycielski graphs and verify triangle-freeness for small parameters.
\end{itemize}

\medskip
\noindent\textbf{WORK.}

\medskip
\noindent\textbf{Definition (Mycielski construction).}
Given a graph $G=(V,E)$ with $V=\{v_1,\dots,v_n\}$, define $\mu(G)$ to be the graph with vertex set
\[V(\mu(G)):=V\ \cup\ \{u_1,\dots,u_n\}\ \cup\ \{w\}
\]
and edges:
\begin{itemize}
\item all original edges $v_iv_j\in E$ remain,
\item for each original edge $v_iv_j\in E$, add edges $u_iv_j$ and $u_jv_i$,
\item connect $w$ to every $u_i$.
\end{itemize}
(No edges are added among the $u_i$.)

\medskip
\noindent\textbf{Lemma 1 (Mycielski preserves triangle-freeness).}
If $G$ is triangle-free, then $\mu(G)$ is triangle-free.

\noindent\emph{Proof.}
Suppose for contradiction that $\mu(G)$ contains a triangle.
If the triangle uses $w$, then it would have the form $w-u_i-x$ with $x$ adjacent to both $w$ and $u_i$.
But $w$ is adjacent only to $u$-vertices, and there are no edges among $u$-vertices, so no triangle can contain $w$.

Thus any triangle lies in $V\cup\{u_1,\dots,u_n\}$.
There are no edges among the $u_i$, so the triangle contains at most one $u$-vertex.

If it contains no $u$-vertex, then it is a triangle entirely in $V$, contradicting that $G$ is triangle-free.
So the triangle contains exactly one $u$-vertex, say $u_i$, and two original vertices $v_j,v_k$.
Edges $u_i v_j$ and $u_i v_k$ exist in $\mu(G)$ only if $v_i v_j\in E(G)$ and $v_i v_k\in E(G)$.
The third edge $v_jv_k$ must also be in $E(G)$ (it is an edge among original vertices).
Therefore $v_i,v_j,v_k$ form a triangle in $G$, again a contradiction.
Hence $\mu(G)$ is triangle-free. \qed

\medskip
\noindent\textbf{Lemma 2 (Mycielski increases chromatic number by $1$).}
For every graph $G$,
\[\chi(\mu(G))=\chi(G)+1.
\]

\noindent\emph{Proof.}
Let $k:=\chi(G)$.

\emph{Upper bound $\chi(\mu(G))\le k+1$.}
Fix a proper $k$-coloring $c:V\to\{1,\dots,k\}$ of $G$.
Color each $v_i$ with $c(v_i)$, color each $u_i$ with $c(v_i)$, and color $w$ with a new color $k+1$.
Edges among the $v_i$ are properly colored by assumption.
There are no $u_i u_j$ edges.
If $u_i v_j$ is an edge in $\mu(G)$, then $v_i v_j$ was an edge of $G$, so $c(v_i)\ne c(v_j)$; hence $u_i$ and $v_j$ get different colors.
Finally, $w$ has color $k+1$ different from every $u_i$.
Thus this is a proper $(k+1)$-coloring of $\mu(G)$, so $\chi(\mu(G))\le k+1$.

\emph{Lower bound $\chi(\mu(G))\ge k+1$.}
Suppose for contradiction that $\mu(G)$ has a proper $k$-coloring $c$ with colors $\{1,\dots,k\}$.
Let $\alpha:=c(w)$ be the color of $w$.
Because $w$ is adjacent to every $u_i$, we have $c(u_i)\ne \alpha$ for all $i$.
Let
\[X:=\{v_i\in V: c(v_i)=\alpha\}
\]
be the set of original vertices that share $w$'s color.
Since $c$ is proper and edges among $V$ in $\mu(G)$ are exactly the edges of $G$, the set $X$ is an independent set in $G$.

Define a new coloring $\tilde c$ of the original vertex set $V$ by
\[\tilde c(v_i):=\begin{cases}
 c(u_i), & v_i\in X,\\
 c(v_i), & v_i\notin X.
\end{cases}
\]
This coloring uses only colors in $\{1,\dots,k\}\setminus\{\alpha\}$, because if $v_i\in X$ then $\tilde c(v_i)=c(u_i)\ne\alpha$ and if $v_i\notin X$ then $c(v_i)\ne\alpha$ by definition.
Hence $\tilde c$ uses at most $k-1$ colors.
We check that $\tilde c$ is proper on every edge $v_iv_j\in E(G)$.

If neither endpoint lies in $X$, then $\tilde c(v_i)=c(v_i)$ and $\tilde c(v_j)=c(v_j)$, and $c(v_i)\ne c(v_j)$ because $c$ is proper.
If exactly one endpoint lies in $X$, say $v_i\in X$ and $v_j\notin X$, then $\tilde c(v_i)=c(u_i)$ and $\tilde c(v_j)=c(v_j)$.
Because $v_iv_j\in E(G)$, the edge $u_i v_j$ is present in $\mu(G)$, so properness of $c$ gives $c(u_i)\ne c(v_j)$.
Therefore $\tilde c(v_i)\ne \tilde c(v_j)$.
If both endpoints lie in $X$, then there is no edge $v_iv_j$ because $X$ is independent.
Thus $\tilde c$ is a proper coloring of $G$ using at most $k-1$ colors, contradicting $\chi(G)=k$.
This contradiction shows that $\chi(\mu(G))\ge k+1$.

Combining the two bounds yields $\chi(\mu(G))=k+1$. \qed

\medskip
\noindent\textbf{Lemma 3 (iterating Mycielski from $C_5$ gives an explicit lower bound for $f(n)$).}
Let $M_3:=C_5$ and define $M_{k+1}:=\mu(M_k)$ for $k\ge 3$.
Then for all $k\ge 3$:
\begin{itemize}
\item $M_k$ is triangle-free,
\item $\chi(M_k)=k$,
\item $|V(M_k)|=3\cdot 2^{k-2}-1$.
\end{itemize}
Consequently, for all $n\ge 5$,
\[f(n)\ge 2+\Bigl\lfloor \log_2\Bigl(\frac{n+1}{3}\Bigr)\Bigr\rfloor.
\]

\noindent\emph{Proof.}
We argue by induction on $k$.
For $k=3$, $M_3=C_5$ is triangle-free and $3$-chromatic, and $|V(M_3)|=5=3\cdot 2^{1}-1$.
Assume the claims for $M_k$.
Then $M_{k+1}=\mu(M_k)$ is triangle-free by Lemma 1 and has chromatic number $\chi(M_k)+1=k+1$ by Lemma 2.
Also, by construction $|V(\mu(G))|=2|V(G)|+1$, so $n_{k+1}:=|V(M_{k+1})|$ satisfies the recurrence $n_{k+1}=2n_k+1$ with $n_3=5$.
Solving gives $n_k=3\cdot 2^{k-2}-1$.

For the bound on $f(n)$: if $n\ge n_k$, we can take $M_k$ and (if needed) add isolated vertices to reach exactly $n$ vertices; adding isolated vertices does not create triangles and does not decrease chromatic number.
Thus $f(n)\ge \chi(M_k)=k$ whenever $n\ge n_k$.
The inequality $n\ge 3\cdot 2^{k-2}-1$ is equivalent to $2^{k-2}\le (n+1)/3$, i.e.
$k\le 2+\log_2((n+1)/3)$.
Taking the largest integer $k$ satisfying this yields the displayed lower bound. \qed

\medskip
\noindent\textbf{FAST REALITY CHECK (small Mycielski graphs).}
I generated the standard Mycielski graphs $M_k$ for $k=3,4,5,6,7$ (starting from $C_5$) and counted triangles.
The results were:
\[
\begin{array}{c|c|c}
 k & |V(M_k)| & \#\text{triangles in }M_k\\\hline
 3 & 5 & 0\\
 4 & 11 & 0\\
 5 & 23 & 0\\
 6 & 47 & 0\\
 7 & 95 & 0
\end{array}
\]
This matches Lemma 1 and the vertex-count formula in Lemma 3.

\medskip
\noindent\textbf{VERIFICATION.}
\begin{itemize}
\item Lemma 1: enumerated all possible triangle placements and reduced to a triangle in $G$.
\item Lemma 2: the recoloring argument relies crucially on the fact that $X=\{v_i:c(v_i)=c(w)\}$ is independent in $G$ (because $c$ is a proper coloring on $V$).
\item Lemma 3: checked the recurrence $|V(\mu(G))|=2|V(G)|+1$ and solved it explicitly.
\end{itemize}

\medskip
\noindent\textbf{FINAL: \textbf{UNRESOLVED}.}
(i) \emph{Strongest proved partial result here.} I proved from scratch that the Mycielski construction preserves triangle-freeness (Lemma 1) and increases chromatic number by $1$ (Lemma 2). Iterating from $C_5$ gives explicit triangle-free graphs with chromatic number $k$ on $3\cdot 2^{k-2}-1$ vertices, hence $f(n)\ge 2+\lfloor\log_2((n+1)/3)\rfloor$ (Lemma 3).

(ii) \emph{First gap (crisp).} I did not provide a self-contained derivation of the correct order of growth $f(n)\asymp \sqrt{n/\log n}$ (or any nontrivial upper bound of that form); the missing inputs are the deep Ramsey/independence-number estimates cited in the problem text.

(iii) \emph{Top 3 next moves.}
1. Work out in detail the implication ``$R(3,k)\asymp k^2/\log k$ $\Rightarrow$ $f(n)\asymp \sqrt{n/\log n}$'' in both directions, clearly isolating which Ramsey bounds are used where.
2. For $g(m)$, combine explicit constructions (e.g. sparse high-chromatic triangle-free graphs) with edge bounds to obtain elementary lower bounds, then compare to the $(m/(\log m)^2)^{1/3}$ scale cited.
3. Perform small-$n$ computations for triangle-free graphs (e.g. via known catalogues) to sanity-check $f(n)$ for $n\le 20$ and see the onset of growth.

(iv) \emph{Minimal counterexample structure.} A counterexample to the quoted asymptotic scale would require a triangle-free graph on $n$ vertices whose chromatic number deviates substantially from the Ramsey-predicted regime, i.e. either an unexpectedly large chromatic number without creating large independent sets, or an unexpectedly small chromatic number despite the absence of large independent sets; both scenarios are governed by extremal behavior of triangle-free graphs and Ramsey theory.
