
Let $1\leq a_1<a_2<\cdots$ be a sequence of integers such that no $a_i$ is the sum of consecutive $a_j$ for $j<i$. Is it true that\[\limsup \frac{a_n}{n}=\infty?\]Or even\[\lim \frac{1}{\log x}\sum_{a_n<x}\frac{1}{a_n}=0?\] Erd\H{o}s writes that it is easy to see that $\liminf a_n/n<\infty$ is possible, and that one can have\[\sum_{a_n< x}\frac{1}{a_n}\gg \log\log x.\]The upper density of such a sequence can be $1/2$, but Erd\H{o}s thought it probably could not be $>1/2$. In fact this is false - Freud \cite{Fr93} constructed a sequence with upper density $19/36$. See also [359] and [867] . References [Fr93] R. Freud, Adding numbers - on a problem of P. Erd\H{o}s . James Cook Mathematical Notes (1993), 6199-6202.

\medskip
\noindent\textbf{1) FORMAL RESTATEMENT}

Let $(a_n)_{n\ge 1}$ be a strictly increasing sequence of positive integers.
Define prefix sums $s_0:=0$ and $s_n:=\sum_{i=1}^n a_i$.
For indices $1\le j\le k$, define the consecutive-block sum
\[
S(j,k):=a_j+a_{j+1}+\cdots+a_k = s_k-s_{j-1}.
\]

The hypothesis is:
\[
\forall i\ge 1,\ \forall 1\le j\le k < i,\quad a_i \neq S(j,k).
\]
In words: no term equals the sum of a consecutive block of earlier terms.

The questions are:

(Q1) Must $\limsup_{n\to\infty} a_n/n = \infty$?

(Q2) Must $\lim_{x\to\infty} \frac{1}{\log x}\sum_{a_n<x}\frac{1}{a_n}=0$?

\medskip
\noindent\textbf{2) QUICK LITERATURE/CONTEXT CHECK}

I will not invoke any external results beyond what is stated in the problem text.
The statement records (without proof here):
\begin{itemize}
\item It is possible that $\liminf a_n/n<\infty$.
\item It is possible to have $\sum_{a_n<x} 1/a_n \gg \log\log x$.
\item Upper density can be $1/2$ (and in fact $19/36$ by Freud).
\end{itemize}
The main questions (Q1) and the stronger average-reciprocal condition (Q2) are posed.

\medskip
\noindent\textbf{3) ATTACK PLAN}

\begin{itemize}
\item Re-express the constraint in a way amenable to counting/greedy arguments.
\item Use brute-force search for small $n$ to see how small $a_n/n$ can be forced for finite prefixes.
\item Prove at least two rigorous lemmas: (a) an equivalent prefix-sum formulation, and (b) a general extension lemma giving an explicit (though weak) upper bound on how large the next admissible term must be.
\end{itemize}

\medskip
\noindent\textbf{4) WORK}

\textbf{Fast reality check (small finite searches).}
Two exact computations were run:
\begin{itemize}
\item Greedy construction (always choose the smallest admissible next integer) produces the first 10 terms
\[
1,2,4,5,8,10,14,15,16,21
\]
and the first 50 terms end at $a_{50}=132$ (so $a_{50}/50=2.64$).
\item Exhaustive backtracking for the finite-prefix feasibility problem
\[a_i\le 2i\ \text{for all }1\le i\le L\]
shows: such a prefix exists for $L\le 24$ but does \emph{not} exist for $L\ge 25$.
In particular, no length-25 sequence can satisfy $a_i\le 2i$ for all $i\le 25$.
\end{itemize}
These are sanity checks only; they do not resolve the asymptotic questions.

\medskip
\noindent\textbf{Lemma 1 (Prefix-sum/difference reformulation).}
The defining constraint is equivalent to:
\[
\forall i\ge 1,\quad a_i \notin \{s_k-s_j:\ 0\le j<k\le i-1\}.
\]

\noindent\emph{Proof.}
For fixed $i$, a consecutive sum of earlier terms has the form
\[
S(j,k)=a_j+\cdots+a_k = s_k-s_{j-1}
\]
with $1\le j\le k\le i-1$, equivalently $0\le j-1<k\le i-1$.
Thus the set of all consecutive sums among indices $<i$ is exactly $\{s_k-s_j:0\le j<k\le i-1\}$.
Therefore ``$a_i$ is not a sum of consecutive earlier terms'' is exactly the displayed exclusion. \hfill$\Box$

\medskip
\noindent\textbf{Lemma 2 (A general extension bound; existence of an admissible next term).}
Let $a_1<\dots<a_{m}$ be any finite initial segment satisfying the constraint up to index $m$.
Then there exists an integer $a_{m+1}>a_m$ such that the extended sequence still satisfies the constraint up to index $m+1$, and moreover one can choose
\[
a_{m+1}\le a_m + \binom{m}{2}+1.
\]

\noindent\emph{Proof.}
Consider the set $F_m$ of all sums of consecutive blocks drawn from $a_1,\dots,a_m$:
\[
F_m:=\{S(j,k): 1\le j\le k\le m\}.
\]
There are exactly $\binom{m+1}{2}$ choices of a pair $(j,k)$ with $1\le j\le k\le m$, so $|F_m|\le \binom{m+1}{2}$ (some different blocks could in principle yield the same sum).
In particular, the number of \emph{distinct} forbidden values is at most $\binom{m+1}{2}$.

Now look at the interval of integers
\[
I:=\{a_m+1, a_m+2,\dots, a_m+\binom{m}{2}+1\}.
\]
This interval has length $\binom{m}{2}+1$.
Even in the worst case, at most $|F_m|\le \binom{m+1}{2}$ integers are forbidden in total, and in particular at most $\binom{m}{2}$ integers can be forbidden \emph{within} $I$ while still leaving at least one integer unaccounted for (because $I$ has $\binom{m}{2}+1$ elements).
Therefore there exists some $x\in I$ with $x\notin F_m$.
Set $a_{m+1}:=x$.
Then $a_{m+1}>a_m$ and $a_{m+1}$ is not a sum of consecutive terms among $a_1,\dots,a_m$, so the constraint holds at index $m+1$.
The upper bound $a_{m+1}\le a_m+\binom{m}{2}+1$ is immediate from the definition of $I$. \hfill$\Box$

\medskip
\noindent\textbf{Remark (what Lemma 2 does and does not say).}
Lemma 2 guarantees that \emph{some} infinite sequence exists (iterate the extension), and it yields a very weak global upper bound on the growth of the \emph{greedy-minimal} choice.
It does \emph{not} imply any lower bound on $a_n$ and therefore does not resolve (Q1).

\medskip
\noindent\textbf{5) VERIFICATION}

\begin{itemize}
\item Lemma 1 is a direct rewriting using prefix sums.
\item In Lemma 2, the only delicate point is counting forbidden values: there are $\binom{m+1}{2}$ consecutive blocks, hence at most that many distinct sums.
Choosing an interval of length $\binom{m}{2}+1$ ensures at least one admissible integer exists.
\item The backtracking feasibility results for the constraint $a_i\le 2i$ up to length $L$ were exhaustive (i.e., the search explored all increasing sequences with those bounds).
\end{itemize}

\medskip
\noindent\textbf{6) FINAL}

\textbf{**UNRESOLVED**}

(i) \emph{Strongest proved partial result.} The constraint admits an equivalent prefix-sum exclusion (Lemma 1) and always allows extension with an explicit (weak) bound on the next admissible term (Lemma 2). Computationally: a prefix with $a_i\le 2i$ exists up to length 24 but not to length 25.

(ii) \emph{First gap (crisp).} Prove or disprove: for every infinite sequence satisfying the constraint, $\limsup a_n/n=\infty$.

(iii) \emph{Top 3 next moves.}
\begin{enumerate}
\item Try to prove an upper bound on the \emph{upper density} of such sequences by showing that dense intervals force some term to be representable as a consecutive sum of earlier terms.
\item Attempt to show that bounded ratio $a_n\le Cn$ implies the consecutive-sum set $F_m$ becomes dense enough to hit a future term, yielding a contradiction.
\item Extend the finite backtracking computations to larger $L$ with a relaxed bound $a_i\le C i$ (vary $C$), to estimate the smallest feasible $C$ for long prefixes and guess whether $C$ must grow.
\end{enumerate}

(iv) \emph{What a minimal counterexample would likely look like.} A counterexample to (Q1) would be an infinite sequence with bounded $a_n/n$ (equivalently, positive lower asymptotic density) while simultaneously avoiding all consecutive-sum representations. Such a sequence would have to be ``additively pseudorandom'' enough that the set of consecutive sums misses the set itself indefinitely.


