\section*{Erd\H{o}s Problem \#102}
\addcontentsline{toc}{section}{Erd\H{o}s Problem \#102}

\subsection*{FORMAL RESTATEMENT}
\noindent\textbf{Verbatim problem statement (from file).}
\emph{``Let $c>0$ and $h_c(n)$ be such that for any $n$ points in $\R^2$ such that there are $\ge cn^2$ lines each containing more than three points, there must be some line containing $h_c(n)$ many points. Estimate $h_c(n)$. Is it true that, for fixed $c>0$, we have $h_c(n)\to\infty$?''}

\medskip
\noindent\textbf{Definitions and conventions.}
Fix $c>0$.
For each $n\ge 1$, define $h_c(n)$ to be the smallest integer $m$ with the property:
\begin{quote}
For every point set $P\subset\R^2$ with $|P|=n$, if the number of distinct lines $\ell$ with $|\ell\cap P|\ge 4$ is at least $c n^2$, then there exists a line $\ell$ with $|\ell\cap P|\ge m$.
\end{quote}
The question is to estimate the growth of $h_c(n)$, and in particular whether for each fixed $c>0$ one necessarily has $h_c(n)\to\infty$ as $n\to\infty$.

\subsection*{QUICK LITERATURE/CONTEXT CHECK}
No external browsing was used.
The file notes:
\begin{itemize}[leftmargin=2em]
\item ``It is not even known if $h_c(n)\ge 5$''.
\item A proposed lower bound $\gg_c n^{1/2}$ was suggested at some point but is false.
\item A (higher-dimensional) grid construction, after random projection into $\R^2$, shows that $h_c(n)\ll n^{1/\log(1/c)}$.
\end{itemize}
Below I prove only elementary constraints that follow directly from double counting in $\R^2$.

\subsection*{ATTACK PLAN}
\begin{itemize}[leftmargin=2em]
\item \textbf{Upper-bound (construction) track:} Try to explicitly build planar point sets with $\ge c n^2$ rich lines but with small maximum collinearity, giving an upper bound on $h_c(n)$.
\item \textbf{Lower-bound (forcing) track:} From $\ge c n^2$ rich lines, use incidence counting to force a line with many points.
\item \textbf{Bridge track:} Relate this problem to Problem \#101 about 4-point lines with no 5 collinear: if Problem \#101 is true, then for any fixed $c>0$ and large $n$, the $cn^2$ rich lines cannot all be 4-point lines, so some line must contain at least 5 points.
\end{itemize}

\subsection*{WORK}
\noindent\textbf{FAST REALITY CHECK.}
The definition implies trivially $4\le h_c(n)\le n$ whenever the hypothesis ``$\ge cn^2$ rich lines'' can hold.
Also, since the total number of distinct lines determined by $n$ points is at most $\binom{n}{2}$, the condition forces $c\le 1/2$.
A sharper constraint appears in Lemma~\ref{lem:c-upper}.

\begin{lemma}[A necessary constraint on $c$]\label{lem:c-upper}
Let $P\subset\R^2$ be a set of $n$ points, and let $L_{\ge 4}(P)$ be the number of distinct lines $\ell$ with $|\ell\cap P|\ge 4$.
Then
\[L_{\ge 4}(P)\le \frac{\binom{n}{2}}{\binom{4}{2}}=\frac{n(n-1)}{12}.
\]
In particular, if $L_{\ge 4}(P)\ge c n^2$ for some fixed $c>0$ and infinitely many $n$, then necessarily $c\le 1/12$.
\end{lemma}

\begin{proof}
Each line $\ell$ with $|\ell\cap P|\ge 4$ contains at least $\binom{4}{2}=6$ distinct unordered pairs of points from $P$.
Different lines contribute disjoint sets of pairs (a pair of points determines a unique line).
Therefore
\[6\,L_{\ge 4}(P)\le \binom{n}{2},
\]
which implies the first inequality.
If $L_{\ge 4}(P)\ge c n^2$ then $c n^2\le n(n-1)/12$, hence $c\le (n-1)/(12n)\to 1/12$ as $n\to\infty$.
So for infinitely many $n$ we must have $c\le 1/12$.
\end{proof}

\begin{proposition}[Many rich lines force a point on linearly many rich lines]
Let $P\subset\R^2$ have $n$ points and let $L_{\ge 4}(P)$ be as above.
Let $I$ be the number of incidences between points of $P$ and rich lines (lines meeting $P$ in at least 4 points):
\[I=\sum_{\ell:\ |\ell\cap P|\ge 4} |\ell\cap P|.
\]
Then $I\ge 4\,L_{\ge 4}(P)$.
In particular, if $L_{\ge 4}(P)\ge c n^2$ then there exists a point $p\in P$ that lies on at least $4c n$ rich lines.
\end{proposition}

\begin{proof}
Every rich line contains at least 4 points of $P$, so summing $|\ell\cap P|$ over rich lines gives $I\ge 4 L_{\ge 4}(P)$.
Now write $I$ also as a sum over points:
\[I=\sum_{p\in P} r(p),\]
where $r(p)$ is the number of rich lines containing $p$.
Thus the average value of $r(p)$ over $p\in P$ is $I/n$.
If $L_{\ge 4}(P)\ge c n^2$, then $I\ge 4c n^2$, so average $r(p)\ge 4c n$.
Hence there exists $p$ with $r(p)\ge 4c n$.
\end{proof}

\begin{corollary}[A second derivation of $c\le 1/12$]
If $L_{\ge 4}(P)\ge c n^2$ then necessarily $c\le (n-1)/(12n)$.
\end{corollary}

\begin{proof}
From the proposition, there is a point $p\in P$ on $r(p)\ge 4c n$ rich lines.
Distinct lines through $p$ intersect only at $p$.
Since each such line contains at least 3 additional points besides $p$, the union of these $r(p)$ lines contains at least $1+3r(p)$ distinct points.
Therefore $1+3r(p)\le n$, so $r(p)\le (n-1)/3$.
Combine with $r(p)\ge 4c n$ to get $4c n\le (n-1)/3$, i.e. $c\le (n-1)/(12n)$.
\end{proof}

\subsection*{VERIFICATION}
\begin{itemize}[leftmargin=2em]
\item All arguments are purely combinatorial and use only basic geometry facts: (i) a pair of points determines a unique line, (ii) two distinct lines through the same point intersect only at that point.
\item The proposition does not by itself force a line with many points; it forces a \emph{point} on many rich lines, which is much weaker than the desired conclusion $h_c(n)\to\infty$.
\end{itemize}

\subsection*{FINAL}
\textbf{UNRESOLVED.}
\begin{enumerate}[label=(\roman*),leftmargin=2.5em]
\item \textbf{Strongest proved partial result here:}
If a set of $n$ points determines at least $c n^2$ lines with $\ge 4$ points, then necessarily $c\le 1/12$ (Lemma~\ref{lem:c-upper}), and some point lies on at least $4c n$ such rich lines (Proposition).
\item \textbf{First gap (crisp):}
Prove (or disprove) that for each fixed $c\in(0,1/12]$, one must have $h_c(n)\to\infty$ as $n\to\infty$.
\item \textbf{Top 3 next moves (concrete):}
\begin{enumerate}[label=(\alph*),leftmargin=2.5em]
\item Attempt to show a strengthening of the ``many rich lines through one point'' conclusion: e.g. from $r(p)\gg n$ rich lines through $p$, force a line with $\gg \log n$ points by repeated intersection/energy arguments.
\item Build explicit planar constructions with $L_{\ge 4}(P)\ge c n^2$ for some absolute $c>0$ and with bounded collinearity (to disprove $h_c(n)\to\infty$).
\item Explore whether polynomial partitioning or incidence geometry can give a dichotomy: either many 4-point lines (relating to Problem \#101) or existence of a 5-point (or larger) line.
\end{enumerate}
\item \textbf{What a minimal counterexample would likely look like:}
Fix $c>0$ and construct arbitrarily large $n$ point sets $P_n$ with $L_{\ge 4}(P_n)\ge c n^2$ but with a uniform bound $\max_{\ell}|\ell\cap P_n|\le M$ for some constant $M$ (or slowly growing $M$). Such a counterexample would behave like a geometric realization of a dense block design with small block size.
\end{enumerate}

