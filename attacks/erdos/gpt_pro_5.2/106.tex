% Erdős Problem #106
% URL: https://www.erdosproblems.com/106

Draw $n$ squares inside the unit square with no common interior point. Let $f(n)$ be the maximum possible sum of the side-lengths of the squares. Is $f(k^2+1)=k$? In \cite{Er94b} Erd\H{o}s dates this conjecture to 'more than 60 years ago'. Erd\H{o}s proved that $f(2)=1$ in an early mathematical paper for high school students in Hungary. Newman proved (in personal communication to Erd\H{o}s) that $f(5)=2$. It is trivial from the Cauchy-Schwarz inequality that $f(k^2)=k$. Erd\H{o}s also asks for which $n$ is it true that $f(n+1)=f(n)$. It is easy to see that $f(k^2+1)\geq k$, by first dividing the unit square into $k^2$ smaller squares of side-length $1/k$, and then replacing one square by two smaller squares of side-length $1/2k$. Hal\'{a}sz \cite{Ha84} gives a construction that shows $f(k^2+2)\geq k+\frac{1}{k+1}$, and in general, for any $c\geq 1$,\[f(k^2+2c+1)\geq k+\frac{c}{k}\]and\[f(k^2+2c)\geq k+\frac{c}{k+1}.\]Hal\'{a}sz also considers the variants where we replace a square by a parallelogram or triangle. Erd\H{o}s and Soifer \cite{ErSo95} and Campbell and Staton \cite{CaSt05} have conjectured that, in general, for any integer $-k<c<k$, $f(k^2+2c+1)=k+\frac{c}{k}$, and proved the corresponding lower bound. Praton \cite{Pr08} has proved that this general conjecture is equivalent to $f(k^2+1)=k$. Baek, Koizumi, and Ueoro \cite{BKU24} have proved $g(k^2+1)=k$, where $g(\cdot)$ is defined identically to $f(\cdot)$ with the additional assumption that all squares have sides parallel to the sides of the unit square. More generally, they prove that $g(k^2+2c+1)=k+c/k$ for any $-k<c<k$, which determines all values of $g(\cdot)$. References [BKU24] Baek, J. and Koizumi, J. and Ueoro, T., A note on the Erd\H{o}s conjecture about square packing . arXiv:2411.07274 (2024). [CaSt05] Campbell, Connie and Staton, William, A square-packing problem of Erd\H{o}s . Amer. Math. Monthly (2005), 165--167. [Er94b] Erd\H{o}s, Paul, Some problems in number theory, combinatorics and combinatorial geometry . Math. Pannon. (1994), 261-269. [ErSo95] Erd\H{o}s, Paul and Soifer, Alexander, Squares in a square . Geombinatorics (1995), 110--114. [Ha84] Hal\'{a}sz, Sylvia, Packing a convex domain with similar convex domains . J. Combin. Theory Ser. A (1984), 85--90. [Pr08] Praton, I., Packing squares in a square . Math. Mag. (2008), 358-361.

% Attempt for Erdős Problem #106
% Following PROMPT_STRATEGY.MD
% Tools/Constraints:
% - Web browsing available? YES
% - Computation available (Python/Sage/Mathematica)? YES

OUTPUT
1) FORMAL RESTATEMENT
Let $U=[0,1]^2$ be the unit square.

A (closed) square $S\subseteq U$ has a side-length $s(S)>0$ (no restriction on
orientation). A family of squares $S_1,\dots,S_n\subseteq U$ is *admissible* if
their interiors are pairwise disjoint:
\[\operatorname{int}(S_i)\cap\operatorname{int}(S_j)=\emptyset\quad\text{for all }i\neq j.\]

Define
\[f(n):=\sup\Bigl\{\sum_{i=1}^n s(S_i):\; S_1,\dots,S_n\subseteq U\text{ admissible}\Bigr\}.
\]
(The problem statement phrases this as a maximum; using $\sup$ avoids an
attainment issue.)

Main conjecture asked in the problem block:
\[\forall k\in\mathbb N:\quad f(k^2+1)=k\,?\]

Secondary question asked in the problem block:
Determine for which $n$ one has $f(n+1)=f(n)$.

2) QUICK LITERATURE/CONTEXT CHECK
Web browsing available: YES, but in this write-up I do **not** use external
sources. I only rely on statements already present in the PROBLEM block.

3) ATTACK PLAN
The trivial global bound is the *area/Cauchy--Schwarz* inequality:
\[\sum s(S_i)^2\le 1\implies \sum s(S_i)\le \sqrt n.\]
This immediately yields $f(k^2)=k$ (sharp via the $k\times k$ grid construction).

To prove $f(k^2+1)=k$ one needs a *strict* improvement over $\sqrt{k^2+1}$, i.e.
show that the geometric constraints beyond area force $\sum s(S_i)\le k$.
I do not complete such an argument here.

4) WORK
FAST REALITY CHECK (small cases / sanity checks)
- $n=1$: One square of side $1$ fills $U$, so $f(1)=1$.
- $n=4$: Partition $U$ into a $2\times 2$ grid of side $1/2$ squares. Sum of
  side-lengths $=4\cdot(1/2)=2$. The general upper bound (Lemma 4.2 below) gives
  $f(4)\le\sqrt 4=2$, hence $f(4)=2$.
- $n=5$: Taking $k=2$ in Lemma 4.5 gives an explicit packing with total
  side-length $2$, so $f(5)\ge 2$, consistent with the problem text claim
  $f(5)=2$.

Lemma 4.1 (Area bound).
For any admissible squares $S_1,\dots,S_n\subseteq U$,
\[\sum_{i=1}^n s(S_i)^2\le 1.\]

Proof.
Each square $S_i$ has area $\operatorname{area}(S_i)=s(S_i)^2$.
Because the interiors are pairwise disjoint, the squares overlap (if at all)
only on boundaries, which have area $0$. Therefore the area of the union is the
sum of the areas:
\[\operatorname{area}\Bigl(\bigcup_{i=1}^n S_i\Bigr)=\sum_{i=1}^n \operatorname{area}(S_i)=\sum_{i=1}^n s(S_i)^2.\]
Since $\bigcup_i S_i\subseteq U$ and $\operatorname{area}(U)=1$, we get
$\sum_i s(S_i)^2\le 1$. $\square$

Lemma 4.2 (Cauchy--Schwarz upper bound).
For any admissible squares $S_1,\dots,S_n\subseteq U$,
\[\sum_{i=1}^n s(S_i)\le \sqrt n.\]
Consequently $f(n)\le \sqrt n$ for all $n$.

Proof.
Apply Cauchy--Schwarz to the vectors $(s(S_1),\dots,s(S_n))$ and $(1,\dots,1)$:
\[(\sum_{i=1}^n s(S_i))^2\le (\sum_{i=1}^n 1^2)(\sum_{i=1}^n s(S_i)^2)=n\sum_{i=1}^n s(S_i)^2.\]
By Lemma 4.1, $\sum_i s(S_i)^2\le 1$, hence
\[(\sum_i s(S_i))^2\le n\quad\Rightarrow\quad \sum_i s(S_i)\le\sqrt n.\]
Taking the supremum over admissible families gives $f(n)\le\sqrt n$. $\square$

Corollary 4.3 (Exact value at perfect squares).
For every integer $k\ge 1$,
\[f(k^2)=k.\]

Proof.
Upper bound: Lemma 4.2 gives $f(k^2)\le\sqrt{k^2}=k$.
Lower bound: Lemma 4.4 constructs an admissible family with sum $k$, so
$f(k^2)\ge k$. Hence equality. $\square$

Lemma 4.4 ($k\times k$ grid construction).
For every integer $k\ge 1$ there is an admissible family of $k^2$ squares in $U$
with total side-length exactly $k$.

Proof.
Partition $U$ into a $k\times k$ grid of congruent axis-parallel squares, each
of side $1/k$. These $k^2$ squares have disjoint interiors and cover $U$.
Their total side-length is
\[k^2\cdot\frac{1}{k}=k.\]
$\square$

Lemma 4.5 (Elementary lower bound at $k^2+1$).
For every integer $k\ge 1$,
\[f(k^2+1)\ge k.\]

Proof.
Start from the $k\times k$ grid in Lemma 4.4 (total side-length $k$).
Pick one of the $k^2$ grid squares of side $1/k$. Replace it by two smaller
axis-parallel squares of side $1/(2k)$ placed side-by-side inside that grid
cell. This keeps interiors disjoint and contained in $U$.

The new family has $(k^2-1)+2=k^2+1$ squares and total side-length
\[(k^2-1)\cdot\frac{1}{k} + 2\cdot\frac{1}{2k} = \frac{k^2-1}{k}+\frac{1}{k}=k.\]
So $f(k^2+1)\ge k$. $\square$

5) VERIFICATION
- Lemma 4.1 uses only additivity of area up to measure-zero boundaries.
- Lemma 4.2 uses Cauchy--Schwarz and Lemma 4.1.
- Lemma 4.5: the replacement increases the count by $+1$ while preserving total
  side-length exactly; the two $1/(2k)$ squares fit inside the removed
  $1/k$-square because $2\cdot(1/(2k))=1/k$.

6) FINAL
**UNRESOLVED**

(i) Strongest fully proved partial result obtained here.
- For all $n$, $f(n)\le \sqrt n$ (Lemma 4.2).
- For all $k\ge 1$, $f(k^2)=k$ (Corollary 4.3).
- For all $k\ge 1$, $f(k^2+1)\ge k$ (Lemma 4.5).

(ii) Exact first gap.
Show the matching upper bound $f(k^2+1)\le k$ for all integers $k\ge 1$.
Equivalently: rule out any admissible packing of $k^2+1$ squares in $U$ with
\[\sum_{i=1}^{k^2+1} s(S_i) > k.\]

(iii) Top 3 next moves (concrete targets).
1. Prove a *stability* refinement of Lemma 4.2 at $n=k^2+1$: if
   $\sum s(S_i)$ is close to $k$, then many squares must have side close to
   $1/k$. Try to make this quantitative.
2. Use that $k^2+1$ squares force at least two squares to fall in the same cell
   of a $k\times k$ coarse partition; attempt to convert this to a strict upper
   bound on total side-length via a local packing inequality inside one cell.
3. Attempt a linear-programming / interval-projection bound: choose a direction
   $\theta$ and bound total projected lengths of the squares along $\theta$ by
   the projection length of $U$; optimize over $\theta$ and use the fact that a
   square's projection length depends on its orientation.

(iv) Minimal counterexample structure (if the conjecture is false).
A smallest $k\ge 1$ for which there exists an admissible family of $k^2+1$
(non-degenerate) squares inside $U$ whose total side-length exceeds $k$.
Such a counterexample would necessarily beat the trivial area bound by a
$\Theta(1/k)$ margin, i.e. have $\sum s(S_i)\in (k,\sqrt{k^2+1}]$.