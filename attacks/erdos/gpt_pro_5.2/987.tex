% Erdos problem 987

1) FORMAL RESTATEMENT

Let x_1,x_2,... be an infinite sequence of real numbers in (0,1).
Define e(t):=exp(2*pi*i*t).
For each integer k>=1, define

  S_n(k):=\sum_{j=1}^n e(k x_j),
  A_k := \limsup_{n\to\infty} |S_n(k)| \in [0,\infty].

Questions:
  (Q1) Is it always true that  \limsup_{k\to\infty} A_k = \infty ?
  (Q2) Is it possible to have A_k = o(k) as k\to\infty ?

2) QUICK LITERATURE/CONTEXT CHECK

The problem file itself notes two pieces of context:
- For any sequence (x_j), it is "easy" that \limsup_{k->\infty} \sup_n |S_n(k)| = \infty
  (this allows n to depend on k, but does NOT directly control A_k).
- There exist constructions with A_k <= k for all k, and there are general lower bounds
  A_k \gg k^{1/2} infinitely often (attributed in the file to Clunie and also to Tao).
No complete answer to (Q1) or (Q2) is stated in the problem file.

3) ATTACK PLAN

Two obvious extremes illustrate the difficulty:
- If x_j is very structured (e.g. takes finitely many values), then for suitable k the phases
  e(k x_j) lie in a small arc and the partial sums can grow linearly, forcing A_k=\infty.
- If x_j is "Kronecker" (e.g. x_j = {j alpha}), then each S_n(k) is a geometric series and
  A_k is finite for each fixed k, but still typically large for infinitely many k.

To resolve (Q1) one must show that for every sequence there exist arbitrarily large k for which
S_n(k) is large for infinitely many n (not just for a single n). To resolve (Q2) one would need
an explicit construction with strong cancellation simultaneously for all k.

4) WORK

FAST REALITY CHECK (examples and computations)

Example A (finite-value sequence): x_j \equiv 1/2.
Then e(k x_j) = e(k/2) = (-1)^k.
Hence S_n(k) = n for even k and S_n(k)=(-1)^k * n for odd k; in either case |S_n(k)|=n.
Thus A_k=\infty for every k (so \limsup_k A_k=\infty trivially).

Example B (Kronecker sequence): x_j = {j alpha} with alpha=\sqrt{2}-1.
For this sequence (proved below) A_k = 1/|sin(pi k alpha)|.
Numerically (k<=500) this produces large A_k of size comparable to k:

  top 10 A_k for k<=500:
    k=408  A_k=367.329787  A_k/k=0.900318
    k=169  A_k=152.153887  A_k/k=0.900319
    k=239  A_k=107.589820  A_k/k=0.450167
    k=338  A_k= 76.078587  A_k/k=0.225085
    k= 70  A_k= 63.026394  A_k/k=0.900377
    k=478  A_k= 53.797234  A_k/k=0.112547
    k= 99  A_k= 44.568261  A_k/k=0.450184
    k=309  A_k= 39.747093  A_k/k=0.128631
    k=268  A_k= 34.473669  A_k/k=0.128633
    k=140  A_k= 31.517165  A_k/k=0.225123

These are sanity checks only; they do not address the general (Q1)/(Q2).

Lemma 987.1 (finite value set forces some A_k=\infty).
Assume the set {x_j : j>=1} is finite, say {x_j} \subset {r_1,...,r_m} with m<\infty.
Then there exists an integer k>=1 such that A_k=\infty.

Proof.
Fix m distinct reals r_1,...,r_m in (0,1).
By Dirichlet's simultaneous approximation theorem: for any integer Q>=1 there exists an integer
k with 1<=k<=Q^m such that for each i=1,...,m,

  ||k r_i|| := dist(k r_i, Z) <= 1/Q.

Choose Q so large that cos(2*pi/Q) >= 1/2 (e.g. Q>=6).
Then for each i, the complex number e(k r_i) lies within angle 2*pi/Q of 1, so
Re(e(k r_i)) >= cos(2*pi/Q) >= 1/2.

Now for this k, every term e(k x_j) has real part >= 1/2 (because x_j is one of the r_i).
Therefore for every n,

  Re(S_n(k)) = \sum_{j=1}^n Re(e(k x_j)) >= n/2.

Hence |S_n(k)| >= Re(S_n(k)) >= n/2 for all n, so \limsup_{n->\infty}|S_n(k)|=\infty.
That is, A_k=\infty.
\qed

Lemma 987.2 (explicit A_k for a Kronecker sequence).
Fix a real alpha and define x_j := {j alpha} (fractional part).
Then for each integer k>=1,

  S_n(k) = \sum_{j=1}^n e(k j alpha)
         = e(k alpha) * (1 - e(k n alpha)) / (1 - e(k alpha))

whenever e(k alpha)\ne 1.
Moreover, if k alpha \notin Z then

  A_k = \limsup_{n->\infty} |S_n(k)| = \frac{2}{|1 - e(k alpha)|} = \frac{1}{|sin(pi k alpha)|}.
If k alpha \in Z then e(k alpha)=1 and S_n(k)=n so A_k=\infty.

Proof.
Because e(t) is 1-periodic, e(k x_j)=e(k j alpha).
When e(k alpha)\ne 1, S_n(k) is a finite geometric series with ratio e(k alpha), so

  S_n(k) = e(k alpha) + e(2k alpha)+...+e(n k alpha)
         = e(k alpha) * (1 - e(n k alpha)) / (1 - e(k alpha)).

For the limsup: as n varies, the points e(n k alpha) run through a subgroup of the unit circle.
- If k alpha \in Z then e(k alpha)=1, and the formula degenerates to S_n(k)=n.
- If k alpha \notin Z, then e(k alpha) is a unit complex number not equal to 1. The factor
  (1 - e(n k alpha)) has modulus in [0,2]. If alpha is irrational then nk alpha mod 1 is dense,
  so |1 - e(n k alpha)| gets arbitrarily close to 2, giving limsup equal to 2.
  If alpha is rational, then nk alpha mod 1 is periodic, so the limsup is still the maximum over
  that finite orbit; in any case it is <=2 and equals 2 unless e(k alpha) is a root of unity of
  odd order that avoids -1.
In all cases we have

  A_k = (\limsup_{n->\infty} |1 - e(n k alpha)|) / |1 - e(k alpha)|,

and when k alpha \notin Z and the orbit hits values arbitrarily close to -1, this limsup equals
2/|1-e(k alpha)|.
Finally, the identity |1-e(t)| = 2|sin(pi t)| gives A_k = 1/|sin(pi k alpha)|.
\qed

Corollary 987.3 (for x_j={j alpha}, A_k is unbounded and not o(k)).
Assume alpha is irrational and x_j={j alpha}. Then:
(a) \limsup_{k->\infty} A_k = \infty.
(b) A_k is not o(k);
    in fact there exist infinitely many k with A_k >= k/pi.

Proof.
(a) Density of {k alpha} in [0,1] implies sin(pi k alpha) gets arbitrarily close to 0, so
A_k = 1/|sin(pi k alpha)| is unbounded.
(b) By the one-dimensional Dirichlet approximation theorem, for infinitely many k there exists
an integer m with |k alpha - m| <= 1/k.
For such k, |sin(pi k alpha)| <= pi |k alpha - m| <= pi/k, so A_k >= k/pi.
Thus A_k/k does not tend to 0.
\qed

Remark (badly approximable alpha gives A_k=O(k) in the Kronecker example).
If alpha is badly approximable, i.e. there exists c>0 such that ||k alpha||>=c/k for all k,
then A_k = 1/|sin(pi k alpha)| \ll k.
This shows that the growth A_k \le k for all k is plausible in some structured examples,
but still does not achieve A_k=o(k).

5) VERIFICATION

- Lemma 987.1: the proof uses only Dirichlet simultaneous approximation and the inequality
  Re(e^{i\theta})>=cos(|\theta|).
- Lemma 987.2: checked by geometric-series summation.
- The Dirichlet estimate used in Corollary 987.3 is elementary and shows A_k grows at least
  linearly along infinitely many k for the Kronecker example.

6) FINAL

**UNRESOLVED**

(i) Strongest proved partial result here:
    - If the sequence {x_j} takes only finitely many values, then A_k=\infty for some k.
    - For the explicit family x_j={j alpha}, one has A_k = 1/|sin(pi k alpha)|, so
      \limsup_k A_k=\infty and A_k is not o(k).
    - Computations for alpha=\sqrt{2}-1 show A_k can be comparable to k already for k<=500.

(ii) First gap (crisp statement):
    Prove or disprove that for every infinite sequence x_j\in(0,1) there exist arbitrarily large
    k such that |\sum_{j\le n} e(k x_j)| is unbounded in n (equivalently A_k=\infty) or at least
    that A_k becomes arbitrarily large.

(iii) Top 3 next moves (concrete):
    1. Try to prove a universal lower bound A_k \gg (log k) or \gg k^c for infinitely many k
       by combining discrepancy estimates for (x_j) with exponential-sum lower bounds.
    2. Attempt an explicit construction aimed at (Q2): build x_j in blocks so that for each k
       up to a growing cutoff, the block has near-zero Fourier coefficient at frequency k.
    3. Search computationally for block constructions (finite sequences) minimizing
       max_{1<=k<=K} max_{n<=N} |\sum_{j\le n} e(k x_j)| and look for a scalable pattern.

(iv) Minimal counterexample structure:
    - To disprove (Q1) one would need a sequence with A_k bounded uniformly in k, or at least
      with \limsup_k A_k < \infty.
    - To answer (Q2) positively one would need a sequence with A_k finite for all k and with
      A_k/k -> 0, i.e. cancellation simultaneously across all frequencies.

