% Erdos Problem #1092
% URL: https://www.erdosproblems.com/1092

\textbf{FORMAL RESTATEMENT}

Fix an integer $r\ge 2$.
For a graph $G$ and integer $m\ge 1$, consider the property:

(P$_m$): Every (not necessarily induced) subgraph $H\subseteq G$ on $m$ vertices can be written as a union $H=H_0\cup H_1$ where $\chi(H_0)\le r$ and $|E(H_1)|\le t$.

Define $f_r(m)$ to be the largest integer $t$ (as a function of $m$) such that the following implication holds for all graphs $G$:

If $G$ satisfies (P$_m$) with this $t=f_r(m)$ for every $m$ (i.e. for each $m$-vertex subgraph $H$ we can decompose with at most $f_r(m)$ edges in $H_1$), then $\chi(G)\le r+1$.

The question asks whether $f_2(m)\gg m$ (linear growth), and more generally whether $f_r(m)\gg_r m$.

\textbf{QUICK LITERATURE/CONTEXT CHECK}

I will not use external results beyond what is explicitly stated in the problem text.
The text states: Tang notes that a construction of R\"{o}dl disproves the first question, so that $f_2(n)\not\gg n$.

\textbf{ATTACK PLAN}

\begin{itemize}
\item First, rewrite the local condition in an equivalent ``edge-deletion to become $r$-colorable'' form.
\item Then try to understand what kinds of global obstructions to $(r+1)$-colorability can coexist with the requirement that every small subgraph is close to $r$-colorable.
\item A full disproof would require an explicit family of graphs with large chromatic number where every $m$-vertex subgraph can be made $r$-colorable by deleting only $o(m)$ (or at least $O(m)$) edges; I do not currently have such an explicit construction.
\end{itemize}

\textbf{WORK}

\textbf{Lemma 1092.1 (Union form $\Leftrightarrow$ edge-deletion form).}
Let $H$ be a graph and let $t\ge 0$.
Then the following are equivalent:
\begin{enumerate}
\item $H$ can be written as $H=H_0\cup H_1$ where $\chi(H_0)\le r$ and $|E(H_1)|\le t$.
\item There exists a set of edges $F\subseteq E(H)$ with $|F|\le t$ such that $\chi(H-F)\le r$.
\end{enumerate}

\emph{Proof.}
(1)$\Rightarrow$(2): take $F=E(H_1)$. Then $H-F$ is a subgraph of $H_0$, hence $\chi(H-F)\le \chi(H_0)\le r$.

(2)$\Rightarrow$(1): take $H_0=H-F$ and $H_1$ to be the graph on the same vertex set with edge set $F$. Then $H=H_0\cup H_1$ and $|E(H_1)|=|F|\le t$.
\hfill$\square$

\textbf{Lemma 1092.2 (A trivial upper bound and monotonicity).}
For every $m\ge 1$ and $r\ge 2$,
\[0\le f_r(m)\le \binom{m}{2},
\]
and $f_r(m)$ is nondecreasing in $m$.

\emph{Proof.}
The upper bound $f_r(m)\le \binom{m}{2}$ is trivial because any $m$-vertex subgraph $H$ has at most $\binom{m}{2}$ edges, so deleting all edges makes it $r$-colorable.
Also $f_r(m)\ge 0$ by definition.

For monotonicity: suppose $m_1\le m_2$.
If a graph $G$ satisfies the hypothesis of the defining implication with the bounds $f_r(\cdot)$, then in particular every $m_1$-vertex subgraph satisfies it.
Thus any admissible choice of $f_r(m_2)$ cannot be smaller than an admissible choice of $f_r(m_1)$ when seeking the pointwise maximum function.
Formally, if $t$ works for size $m_1$ but not for size $m_2$, then the maximal allowable value at $m_2$ cannot be less than that at $m_1$.
\hfill$\square$

\textbf{FAST REALITY CHECK}

For $r=2$, Lemma 1092.1 shows the hypothesis can be phrased as:
``every $m$-vertex subgraph can be made bipartite by deleting at most $f_2(m)$ edges.''
In particular, if $f_2(m)=0$ for all $m$, then the hypothesis says every subgraph is bipartite, hence $G$ itself is bipartite and $\chi(G)\le 2\le 3$.
This sanity-checks the direction of the implication.

\textbf{VERIFICATION}

\begin{itemize}
\item Lemma 1092.1 is a direct translation between ``add at most $t$ edges'' and ``delete at most $t$ edges''.
\item Lemma 1092.2's monotonicity argument depends on interpreting $f_r$ as the pointwise maximal function for which the global implication holds; this matches the phrasing ``$f_r(n)$ be maximal such that ...''.
\end{itemize}

\textbf{FINAL}

\textbf{UNRESOLVED}

(i) \emph{Strongest proved partial result.} The local hypothesis is equivalently: every $m$-vertex subgraph $H$ has an edge set $F$ of size $\le f_r(m)$ whose removal makes $H$ $r$-colorable (Lemma 1092.1). Trivial bounds give $0\le f_r(m)\le \binom{m}{2}$ and monotonicity in $m$ (Lemma 1092.2).

(ii) \emph{First gap (crisp).} Produce (or verify in full detail) an explicit family of graphs $G$ with $\chi(G)\ge 4$ such that for some constant $c>0$ and for every subgraph $H$ on $m$ vertices, one can delete at most $c m$ edges from $H$ to make it bipartite. Such a family would explicitly disprove $f_2(m)\gg m$.

(iii) \emph{Top 3 next moves.}
\begin{itemize}
\item Try to construct dense graphs that are ``locally close to bipartite'' via blow-ups of sparse high-chromatic graphs, controlling how many edges must be deleted in every induced subgraph.
\item Attempt a probabilistic construction: choose a random graph with a planted bipartite structure plus a small random perturbation, then estimate (uniformly over all vertex subsets) the minimum edge-deletion number to bipartite.
\item For small $m$, brute-force search for 4-chromatic graphs maximizing the property ``every $m$-vertex subgraph becomes bipartite after deleting $\le t$ edges'' to guess the right order of $f_2(m)$.
\end{itemize}

(iv) \emph{Minimal counterexample structure.} A minimal counterexample to the conjectured linear lower bound for $f_2$ would be a 4-chromatic graph $G$ in which every induced subgraph $H$ has a small edge set hitting all odd cycles (an odd-cycle edge transversal) of size $O(|V(H)|)$.
Such a graph must contain odd cycles globally but in every local view they must be ``almost removable'' by deleting few edges.


