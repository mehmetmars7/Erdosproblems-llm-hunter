\section*{Problem 741 (Burr--Erd\H{o}s): splitting a set whose sumset has positive density}

\subsection*{1) FORMAL RESTATEMENT}

Let $\mathbb{N}:=\{1,2,3,\dots\}$ and for $A\subseteq \mathbb{N}$ write
\[
A+A := \{a+a': a,a'\in A\}.
\]
For a set $S\subseteq \mathbb{N}$ define its \emph{upper} and \emph{lower} (asymptotic) densities by
\[
\overline d(S):=\limsup_{N\to\infty}\frac{|S\cap [1,N]|}{N},
\qquad
\underline d(S):=\liminf_{N\to\infty}\frac{|S\cap [1,N]|}{N}.
\]

\medskip
\noindent\textbf{(Q1)} Assume $\overline d(A+A)>0$. Must there exist a partition $A=A_1\sqcup A_2$
such that
\[
\overline d(A_1+A_1)>0
\quad\text{and}\quad
\overline d(A_2+A_2)>0?
\]

\medskip
\noindent\textbf{(Q2)} Call $A\subseteq \mathbb{N}$ an \emph{asymptotic basis of order $2$} if $A+A$ contains all sufficiently large integers.
Is there such a basis $A$ with the Ramsey-type property that for \emph{every} partition
$A=A_1\sqcup A_2$, the sets $A_1+A_1$ and $A_2+A_2$ cannot \emph{both} have bounded gaps?
(Here ``bounded gaps'' means $S\subseteq \mathbb{N}$ is \emph{syndetic}: $\exists L$ such that every interval of length $L$
contains an element of $S$.)

\subsection*{2) QUICK LITERATURE/CONTEXT CHECK}

Web browsing available? YES.

\begin{itemize}
\item The problem is recorded as open on \texttt{erdosproblems.com} (Problem 741), and is attributed to Burr and Erd\H{o}s.
\item A related (stronger) direction in additive number theory: if $A$ is an asymptotic basis of order $2$ with sufficiently large representation function $r_{2,A}(n)$, then $A$ can be partitioned into two disjoint asymptotic bases (Erd\H{o}s--Nathanson; see also Nathanson's survey on extremal problems for sumsets).
\end{itemize}

\subsection*{3) ATTACK PLAN}

\begin{itemize}
\item \textbf{Positive direction:} Try to 2-color $A$ so that many sums $n$ have a monochromatic representation in each color.
A natural attempt is a random 2-coloring plus a second-moment/LLL argument on the events $n\notin A_i+A_i$.
This is more plausible when many $n$ have many representations $n=a+a'$ with $a,a'\in A$.
\item \textbf{Negative direction:} Try to construct $A$ with $\overline d(A+A)>0$ but whose sum representations are so ``fragile'' that any partition destroys density of one monochromatic sumset.
Such an $A$ would likely have very small representation counts $r_{A}(n)$ on a positive-density subset of $A+A$.
\item \textbf{Intermediate reductions:} Identify extra hypotheses (e.g.\ $d(A)>0$, or large $r_A(n)$ on a dense set) under which (Q1) is provable; and identify structural obstructions for (Q2).
\end{itemize}

\subsection*{4) WORK}

\paragraph{4.1 A warm-up case where (Q1) is easy.}
If $A$ itself has positive lower density, then one can force a partition with both monochromatic sumsets having positive density.

\medskip
\noindent\textbf{Lemma 4.1.}
If $\underline d(A)=\alpha>0$, then there exists a partition $A=A_1\sqcup A_2$ with
$\underline d(A_1)\ge \alpha/3$ and $\underline d(A_2)\ge \alpha/3$.
Consequently,
\[
\underline d(A_1+A_1)>0
\quad\text{and}\quad
\underline d(A_2+A_2)>0.
\]

\medskip
\noindent\emph{Sketch of proof.}
(1) Build $A_1$ by a block construction: choose disjoint intervals $I_k=[N_k+1,N_k+M_k]$ with $M_k\to\infty$
and on each block select about one third of the points of $A\cap I_k$ into $A_1$, putting the rest in $A_2$.
By arranging the blocks so that $|A\cap I_k|$ is close to $\alpha M_k$ (possible along a subsequence witnessing the liminf),
one can ensure $\underline d(A_i)\ge \alpha/3$.

(2) For any finite $B\subseteq [1,N]$, one has the elementary inequality in the integers:
\[
|B+B|\ge 2|B|-1,
\]
because the sums $b_{\min}+B$ and $b_{\max}+B$ already cover an interval of length $|B|$ and these translate-intervals overlap in at most one point.
Applying this with $B=A_i\cap[1,N]$ gives
\[
|(A_i+A_i)\cap [2,2N]|\ge 2|A_i\cap[1,N]|-1.
\]
Divide by $2N$ and take $\liminf_{N\to\infty}$ to deduce $\underline d(A_i+A_i)\ge \underline d(A_i)>0$.
\hfill $\square$

\medskip
This does \emph{not} resolve (Q1), because $\overline d(A+A)>0$ does not force $\underline d(A)>0$ (e.g.\ one can have $|A\cap[1,N]|\asymp \sqrt N$).

\paragraph{4.2 Representation-count heuristics for the hard case.}
Write the (ordered) representation function
\[
r_A(n):=(\mathbf 1_A * \mathbf 1_A)(n)=|\{(a,a')\in A^2:\ a+a'=n\}|.
\]
If a positive-density set of $n$ have \emph{many} representations, then a random 2-coloring of $A$ should, heuristically,
leave many $n$ with at least one red-red representation and many $n$ with at least one blue-blue representation.
However, without quantitative lower bounds on $r_A(n)$ on a positive-density set, a clean averaging argument seems out of reach.

\paragraph{4.3 Link to partitioning bases (motivation for Q2).}
If $A$ is an asymptotic basis of order $2$ and $r_A(n)$ grows at least like a constant multiple of $\log n$
for all sufficiently large $n$, then one can partition $A$ into two disjoint asymptotic bases (Erd\H{o}s--Nathanson-type results).
In that regime, for that particular partition both $A_1+A_1$ and $A_2+A_2$ are cofinite, hence have bounded gaps.
Therefore any example for (Q2) must come from a basis with \emph{small} (or highly irregular) representation function.

\paragraph{4.4 What a counterexample to (Q1) would have to look like.}
A counterexample would be a set $A$ with $\overline d(A+A)>0$ but for every $2$-coloring of $A$
(at the level of a partition) at least one monochromatic sumset has density $0$.
Such a phenomenon would suggest that the sumset $A+A$ is ``mostly'' generated by \emph{cross-color} pairs,
so one would need $A$ where many sums are forced to use pairs from two different substructures.
No explicit construction of this type is currently known to me.

\subsection*{5) VERIFICATION}

\begin{itemize}
\item Checked that Lemma 4.1 uses only the elementary finite-set inequality $|B+B|\ge 2|B|-1$.
\item The block-selection argument is standard for producing subsets of prescribed lower density along a liminf subsequence; it does not require any additivity structure.
\item No step above claims to solve the Burr--Erd\H{o}s question; it only isolates an easy case and clarifies why the sparse case is hard.
\end{itemize}

\subsection*{6) FINAL}

\noindent\textbf{UNRESOLVED.}

\begin{itemize}
\item \textbf{What I have:} (Q1) holds if $A$ itself has positive lower density (Lemma 4.1). For asymptotic bases with sufficiently large representation function, known results allow partitions into two bases, so (Q2) fails in that regime.
\item \textbf{Where it breaks:} When $A$ has density $0$ but $A+A$ has positive density, current arguments do not control how monochromatic representations distribute under a partition.
\item \textbf{Next moves:}
  \begin{enumerate}
  \item Prove a robust ``many representations on a dense set'' lemma from $\overline d(A+A)>0$ (or show this is false).
  \item Try a probabilistic coloring plus second-moment/LLL on a \emph{thinned} set of sums where $r_A(n)$ is large.
  \item For (Q2), attempt to build an asymptotic basis with representation function concentrated on cross-block pairs, forcing long gaps in at least one monochromatic sumset under any 2-coloring.
  \end{enumerate}
\item \textbf{Minimal lemma that would settle (Q1) (one possible route):}
  Find a constant $\eta>0$ and a subset $S\subseteq A+A$ with $\overline d(S)>0$ such that every $n\in S$ has at least $R$ \emph{pairwise disjoint} representations $n=a_i+a_i'$ with $(a_i,a_i')$ using distinct elements of $A$.
  Then a random 2-coloring would make $n\in A_1+A_1$ with probability $\ge 1-(3/4)^R$ (similarly for $A_2$), and taking $R$ large would force positive density in both colors.
\end{itemize}

\subsection*{7) COMPLETION}

COMPLETION: 25\% (easy dense case + structural discussion; main problem remains open).

\hrule

