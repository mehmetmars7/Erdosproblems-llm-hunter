%Erdos problem 888

\noindent 1) \textbf{FORMAL RESTATEMENT}

\noindent For each $n\ge 1$, let $f(n)$ be the maximum size of a set $A\subseteq\{1,2,\dots,n\}$ with the property:

\begin{quote}
For all $a,b,c,d\in A$ with $a\le b\le c\le d$, if $abcd$ is a perfect square, then $ad=bc$.
\end{quote}

\noindent \emph{Ambiguity note.} Because $A$ is a set, elements of $A$ are distinct, but the statement does not explicitly say whether the quadruple $(a,b,c,d)$ is required to consist of four \emph{distinct} elements of $A$ or whether repetitions are allowed (e.g. $(a,b,b,c)$). The condition as written quantifies over $a,b,c,d\in A$ and allows equality in $a\le b\le c\le d$, so the literal reading allows repetitions. In what follows, I adopt this literal (stronger) reading; any set satisfying it also satisfies the ``distinct quadruples only'' variant.

\noindent The question asks for the asymptotic order of $f(n)$ (e.g. is $f(n)=o(n)$?).

\noindent 2) \textbf{QUICK LITERATURE/CONTEXT CHECK}

\noindent The problem text attributes the question to Erd\H{o}s--S\'ark\"ozy--S\'os and reports (without proof here) that Erd\H{o}s claimed S\'ark\"ozy proved $f(n)=o(n)$. The problem text also notes a lower bound $f(n)\gg n/\log n$ coming from taking $A$ to be the primes.

\noindent 3) \textbf{ATTACK PLAN}

\noindent \emph{Proof-direction (upper bounds).}
\begin{itemize}
\item Convert the condition ``$abcd$ square'' into parity constraints on prime exponents (squarefree kernels), turning the problem into forbidding certain 4-term relations; try to show this forces sparseness.
\item Search for a combinatorial reformulation (hypergraph of forbidden 4-tuples) and apply container-style counting or energy bounds.
\end{itemize}

\noindent \emph{Disproof-direction (large constructions).}
\begin{itemize}
\item Look for multiplicative constructions beyond primes (e.g. numbers with restricted squarefree part) that still avoid nontrivial square quadruples.
\item Use computation for moderate $n$ to guess the true growth of $f(n)$ under the intended interpretation.
\end{itemize}

\noindent 4) \textbf{WORK}

\noindent \textbf{Lemma 888.1 (Primes give a valid set).}
Let $A$ be the set of primes $\le n$. Then $A$ satisfies the stated property (both in the ``repetitions allowed'' and ``distinct only'' interpretations). Consequently $f(n)\ge \pi(n)$.

\noindent \emph{Proof.}
Take $a\le b\le c\le d$ in $A$ (so each is prime) and assume $abcd$ is a perfect square.
Write the prime factorization of $abcd$: since $a,b,c,d$ are primes, each of these primes occurs with exponent equal to its multiplicity among $\{a,b,c,d\}$. For $abcd$ to be a square, every prime exponent must be even.
Hence every prime appearing among $a,b,c,d$ must appear an even number of times.
Because there are four primes total, the multiset $\{a,b,c,d\}$ must be either:
\begin{itemize}
\item $\{p,p,p,p\}$ for some prime $p$, or
\item $\{p,p,q,q\}$ for two distinct primes $p<q$.
\end{itemize}
Under the imposed ordering $a\le b\le c\le d$, the second case forces $a=b=p$ and $c=d=q$.
In either case we have $ad=bc$ (both sides equal $p^2$ in the first case, and both equal $pq$ in the second).
Thus the property holds for all such quadruples, and $|A|=\pi(n)$ gives $f(n)\ge \pi(n)$.
\hfill$\square$

\medskip
\noindent \textbf{Lemma 888.2 (Square-product pairs force a geometric mean).}
Assume $A\subseteq\{1,\dots,n\}$ satisfies the stated property (with repetitions allowed).
Let $a,c\in A$ with $a<c$ and suppose $ac$ is a perfect square.
Then there is \emph{at most one} element $b\in A$ with $a<b<c$.
Moreover, if such a $b$ exists, then $b^2=ac$ (hence $b=\sqrt{ac}\in\mathbb{N}$).

\noindent \emph{Proof.}
Assume $a<c$ are in $A$ and $ac$ is a square.
Let $b\in A$ satisfy $a<b<c$.
Consider the nondecreasing quadruple $(a,b,b,c)$; it is allowed because repetitions are permitted and $a\le b\le b\le c$.
Its product is
\[
 a\cdot b\cdot b\cdot c = b^2\,(ac).
\]
Because $b^2$ is a perfect square and $ac$ is assumed to be a perfect square, their product $b^2(ac)$ is a perfect square.
Therefore the defining property of $A$ applies and yields
\[
 a\,c = b\,b = b^2.
\]
So any such $b$ must equal $\sqrt{ac}$.
In particular, there is at most one element of $A$ strictly between $a$ and $c$.
\hfill$\square$

\medskip
\noindent \textbf{Fast reality check (computation, literal ``repetitions allowed'' version).}
I brute-forced the maximum size $f(n)$ for $n\le 15$ by enumerating all subsets $A\subseteq\{1,\dots,n\}$ and checking the condition for all nondecreasing quadruples $a\le b\le c\le d$ with entries in $A$ (allowing repetitions).
The exact maxima found were:
\[
\begin{array}{c|ccccccccccccccc}
 n & 1&2&3&4&5&6&7&8&9&10&11&12&13&14&15\\\hline
 f(n) & 1&2&3&3&4&5&6&6&6&7&8&8&9&10&10
\end{array}
\]
One maximizing example for $n=15$ is
\[
A=\{1,2,3,5,6,7,10,11,13,14\},\qquad |A|=10,
\]
and it was verified directly (by exhaustive quadruple checking) that this $A$ has no violating quadruple.

\noindent 5) \textbf{VERIFICATION}

\noindent Lemma 888.1: the proof uses only unique factorization and the characterization of square integers by even prime exponents.

\noindent Lemma 888.2: the key step is that $b^2(ac)$ is a square if and only if $ac$ is a square, because $b^2$ is always a square.
Then the defining implication gives $ac=b^2$.

\noindent Computation: the brute check tested exactly the implication ``if $abcd$ is a square then $ad=bc$'' over all nondecreasing quadruples with repetition; this matches the literal interpretation.

\noindent 6) \textbf{FINAL}

\noindent \textbf{UNRESOLVED}
\begin{enumerate}
\item[(i)] \textbf{Strongest proved partial result here.} The set of primes gives a valid construction of size $\pi(n)$ (Lemma 888.1), so $f(n)\ge \pi(n)\gg n/\log n$. Also, if $a,c\in A$ have $ac$ square, then any $b\in A$ with $a<b<c$ is forced to equal $\sqrt{ac}$ (Lemma 888.2).
\item[(ii)] \textbf{First gap (crisp).} Prove a nontrivial upper bound on $f(n)$ (ideally $f(n)=o(n)$, or any explicit $f(n)\le n/(\log n)^{\delta}$) from the square-quadruple implication.
\item[(iii)] \textbf{Top 3 next moves.}
  \begin{itemize}
  \item Reformulate the constraint in terms of squarefree kernels: write each $x\in A$ as $x=s(x)\,t(x)^2$ with $s(x)$ squarefree, and analyze which 4-tuples of $s(x)$ can yield a square product without forcing $ad=bc$.
  \item Push the computation to larger $n$ under both interpretations (repetitions allowed vs. distinct quadruples only) to guess the true growth rate and to detect structure in extremal sets.
  \item Try to prove that many elements sharing the same squarefree kernel create many square quadruples; quantify this to obtain an upper bound by double counting.
  \end{itemize}
\item[(iv)] \textbf{Minimal counterexample structure.} To refute $f(n)=o(n)$ one would need a sequence $n_t\to\infty$ and sets $A_t\subseteq\{1,\dots,n_t\}$ with $|A_t|\ge \delta n_t$ for some fixed $\delta>0$ such that every nondecreasing quadruple from $A_t$ with square product satisfies $ad=bc$. Such a construction would have to prevent the ``parity cancellations'' of prime exponents that normally generate many square products among dense sets.
\end{enumerate}

