% Erdos Problem #866
% Solutions/notes generated by ChatGPT following PROMPT_STRATEGY.MD

\bigskip
\noindent\textbf{FORMAL RESTATEMENT.}
Fix an integer $k\ge 3$. For $N\in\mathbb{N}$ define $g_k(N)$ to be the least integer $g\ge 0$ such that:
for every set $A\subseteq [2N]:=\{1,2,\dots,2N\}$ with
\[
|A|\ge N+g,
\]
there exist \emph{distinct positive integers} $b_1,\dots,b_k$ such that for every $1\le i<j\le k$ the pairwise sum $b_i+b_j$ lies in $A$.
(\emph{Ambiguity note.} The problem text says ``integers $b_i$'' without stating distinctness/positivity. If the $b_i$ could repeat, the problem becomes degenerate (one even sum in $A$ would allow taking all $b_i$ equal), contradicting the nontrivial values reported in the statement. Thus I adopt ``distinct positive'' as the minimal correction consistent with the rest of the text.)

The problem asks to estimate $g_k(N)$ as a function of $N$.

\medskip
\noindent\textbf{QUICK LITERATURE/CONTEXT CHECK.}
The problem statement itself records several results of Choi--Erd\H{o}s--Szemer\'edi and later work, including:
$g_3(N)=2$, $g_4(N)\ll 1$ (and an explicit bound $\le 2032$),
$g_5(N)\asymp\log N$, $g_6(N)\asymp N^{1/2}$,
and general upper/lower bounds in $k$.
(Per the project integrity constraints, I do not use or assert further literature beyond what is in the problem text.)

\medskip
\noindent\textbf{ATTACK PLAN.}
\begin{itemize}
\item For $k=3$, derive a concrete characterization in terms of triples in $A$ and check small $N$ computationally.
\item For general $k$, I do not attempt to reproduce the deep bounds stated in the problem file; instead I record basic necessary conditions and sanity checks.
\end{itemize}

\medskip
\noindent\textbf{WORK.}

\medskip
\noindent\textbf{Lemma 866.1 ($k=3$ reformulation via triangle inequalities).}
Let $A\subseteq[2N]$. There exist distinct positive integers $b_1,b_2,b_3$ with
\[
 b_1+b_2\in A,\quad b_1+b_3\in A,\quad b_2+b_3\in A
\]
if and only if there exist distinct $x,y,z\in A$ such that
\[
 x+y>z,\quad x+z>y,\quad y+z>x,\quad\text{and}\quad x+y+z\equiv 0\pmod 2.
\]

\smallskip
\noindent\emph{Proof.}
($\Rightarrow$) Let $x=b_1+b_2$, $y=b_1+b_3$, $z=b_2+b_3$.
Then $x,y,z\in A$ and are distinct because if $x=y$ then $b_2=b_3$, etc.
Also $x+y+z=2(b_1+b_2+b_3)$ is even.
Finally, triangle inequalities hold because, e.g.
$x+y=(b_1+b_2)+(b_1+b_3)=2b_1+b_2+b_3> b_2+b_3=z$ since $b_1>0$, and similarly for the others.

($\Leftarrow$) Given distinct $x,y,z\in A$ satisfying the parity and strict triangle inequalities, define
\[
 b_1:=\frac{x+y-z}{2},\qquad b_2:=\frac{x+z-y}{2},\qquad b_3:=\frac{y+z-x}{2}.
\]
Parity ensures each $b_i\in\mathbb{Z}$. The strict triangle inequalities ensure $b_i>0$.
A direct calculation shows $b_1+b_2=x$, $b_1+b_3=y$, $b_2+b_3=z$, so all required pairwise sums lie in $A$.
Distinctness of $x,y,z$ implies $b_1,b_2,b_3$ are distinct (e.g. $b_1=b_2\iff y=z$).
\qed

\medskip
\noindent\textbf{Lemma 866.2 (Lower bound $g_3(N)\ge 2$ for $N\ge 2$).}
For $N\ge 2$, let
\[
A:=\{2\}\ \cup\ \{1,3,5,\dots,2N-1\}\ \subseteq [2N].
\]
Then $|A|=N+1$ and $A$ contains no triple of distinct positive integers $b_1,b_2,b_3$ with all pairwise sums in $A$.
In particular $g_3(N)\ge 2$.

\smallskip
\noindent\emph{Proof.}
Every sum of two distinct positive integers is at least $1+2=3$.
In $A$, the only even element is $2$. By Lemma 866.1, any $(b_1,b_2,b_3)$ configuration would produce three distinct sums $x,y,z\in A$ with even total $x+y+z$.
But among three distinct elements of $A$, at most one can be even, so $x+y+z$ would be even only if there are exactly two odd sums and one even sum.
The only even sum available is $2$.
However, $2$ cannot be expressed as a sum of two distinct positive integers.
Therefore no such configuration exists.
\qed

\medskip
\noindent\textbf{FAST REALITY CHECK (small $N$, $k=3$).}
Using the characterization in Lemma 866.1, I exhaustively computed $g_3(N)$ for $2\le N\le 10$.
The exact values found were:
\[
\begin{array}{c|ccccccccc}
N & 2&3&4&5&6&7&8&9&10\\\hline
g_3(N) & 3&3&2&2&2&2&2&2&2
\end{array}
\]
So for $N\ge 4$ the computation is consistent with the statement $g_3(N)=2$.

\medskip
\noindent\textbf{VERIFICATION.}
\begin{itemize}
\item Lemma 866.1: checked both directions algebraically; the triangle inequalities are exactly positivity of the reconstructed $b_i$.
\item Lemma 866.2: the obstruction is that the only even sum present is $2$, which cannot be a sum of two distinct positive integers.
\item Computation: brute force over all $A\subseteq[2N]$ for $N\le 10$.
\end{itemize}

\medskip
\noindent\textbf{FINAL.} \textbf{UNRESOLVED}

(i) \emph{Strongest proved partial result.}
For $k=3$, Lemma 866.1 gives an exact reformulation in terms of finding an even-perimeter triangle $(x,y,z)$ inside $A$; Lemma 866.2 gives a sharp lower bound $g_3(N)\ge 2$ for $N\ge 2$. Computation shows $g_3(N)=2$ for $4\le N\le 10$.

(ii) \emph{First gap (crisp).}
For general $k\ge 4$, I do not prove any nontrivial asymptotic upper bound on $g_k(N)$ beyond the trivial $g_k(N)\ge 0$.
Even for $k=3$, I do not provide a complete gap-free proof that every $A\subseteq[2N]$ with $|A|\ge N+2$ contains an even-perimeter triangle triple for all large $N$.

(iii) \emph{Top 3 next moves.}
\begin{itemize}
\item Prove the $k=3$ upper bound $g_3(N)\le 2$ for all $N\ge 4$ by a direct combinatorial argument on dense subsets of $[2N]$ guaranteeing an even-perimeter triangle.
\item Generalize Lemma 866.1 to $k=4$ by characterizing 6 pairwise sums as an ``additive $K_4$'' and derive finite forbidden-configuration conditions.
\item Use computer search for small $N$ to guess the true constant value of $g_4(N)$ and identify extremal constructions.
\end{itemize}

(iv) \emph{Minimal counterexample structure.}
A minimal counterexample for fixed $k$ would be a set $A\subseteq[2N]$ with $|A|=N+g_k(N)-1$ that avoids all $k$-cliques of pairwise sums; for $k=3$ this means avoiding all triples $x<y<z$ with $x+y>z$ and even total sum.
Such an extremal $A$ would likely concentrate on one parity class together with a carefully chosen sparse subset of the other parity (as in Lemma 866.2).


