% Erdos Problem #575

1) FORMAL RESTATEMENT

Let $\mathcal{F}$ be a finite family of finite simple graphs.
Define $\mathrm{ex}(n;\mathcal{F})$ to be the maximum number of edges in an $n$-vertex graph containing no subgraph isomorphic to any $F\in\mathcal{F}$.

Assume $\mathcal{F}$ contains at least one bipartite graph.
The question is whether there always exists a \emph{bipartite} graph $G\in\mathcal{F}$ and a constant $C_{\mathcal{F}}>0$ such that for all $n$,
\[
\mathrm{ex}(n;G) \le C_{\mathcal{F}}\, \mathrm{ex}(n;\mathcal{F}).
\]
(Using Vinogradov notation: $\mathrm{ex}(n;G)\ll_{\mathcal{F}} \mathrm{ex}(n;\mathcal{F})$.)

2) QUICK LITERATURE/CONTEXT CHECK

The source file lists this as a problem of Erd\H{o}s and Simonovits and points to related problem [180] and the graphs collection.
The Erd\H{o}s Problems website currently lists #575 as open (as of the access date shown there). 

3) ATTACK PLAN

Proof-track ideas:
- Identify natural sufficient conditions on $\mathcal{F}$ under which $\mathrm{ex}(n;\mathcal{F})$ is within a constant factor of $\min\{\mathrm{ex}(n;H):H\in\mathcal{F}\text{ bipartite}\}$.
- Use subgraph containment relations inside $\mathcal{F}$: if one bipartite member is contained in all others, the problem becomes trivial.

Disproof-track ideas:
- Search for a family $\mathcal{F}$ with multiple bipartite members such that avoiding all of them simultaneously forces a much smaller edge count than avoiding any one of them.
- Attempt small $n$ brute force over families of small graphs to see whether a counterexample pattern emerges.

I provide two unconditional ``sanity'' lemmas (covering some degenerate and some structured cases) but do not resolve the general question.

4) WORK

\textbf{FAST REALITY CHECK (sanity on degenerate cases).}
- If $K_2\in\mathcal{F}$ then $\mathrm{ex}(n;\mathcal{F})=0$ for all $n$, and choosing $G=K_2$ makes the desired inequality trivial.
- If $\mathcal{F}$ contains a bipartite graph $G$ that is a subgraph of every $F\in\mathcal{F}$, then forbidding $\mathcal{F}$ is the same as forbidding $G$.

These are formalized below.

\medskip
\textbf{Lemma 575.1 (trivial case when $K_2\in\mathcal{F}$).}
If $K_2\in\mathcal{F}$ then $\mathrm{ex}(n;\mathcal{F})=0$ and taking $G=K_2$ satisfies
$\mathrm{ex}(n;G)\ll_{\mathcal{F}}\mathrm{ex}(n;\mathcal{F})$.
\emph{Proof.}
A graph contains no $K_2$ subgraph if and only if it has no edges.
So forbidding $K_2$ forces $e(G)=0$, hence $\mathrm{ex}(n;\mathcal{F})=0$.
Also $\mathrm{ex}(n;K_2)=0$.
Thus $\mathrm{ex}(n;K_2)=0\le C\cdot 0$ for any constant $C$.
\qed

\medskip
\textbf{Lemma 575.2 (subgraph-dominating bipartite member).}
Suppose there exists a bipartite graph $G\in\mathcal{F}$ such that $G$ is a subgraph of every $F\in\mathcal{F}$.
Then for every $n$,
\[
\mathrm{ex}(n;\mathcal{F})=\mathrm{ex}(n;G).
\]
In particular, the conjectured inequality holds with this $G$ and constant $C_{\mathcal{F}}=1$.
\emph{Proof.}
Let $H$ be any $n$-vertex graph.
If $H$ contains some $F\in\mathcal{F}$, then (since $G\subseteq F$) the same copy of $F$ contains a copy of $G$; hence $H$ contains $G$.
Therefore, if $H$ is $G$-free then $H$ is $F$-free for every $F\in\mathcal{F}$, i.e. $H$ is $\mathcal{F}$-free.
This shows that the class of $G$-free graphs is a subset of the class of $\mathcal{F}$-free graphs, so
$\mathrm{ex}(n;G)\le \mathrm{ex}(n;\mathcal{F})$.

Conversely, because $G\in\mathcal{F}$, any $\mathcal{F}$-free graph is in particular $G$-free.
So the class of $\mathcal{F}$-free graphs is a subset of the class of $G$-free graphs, giving
$\mathrm{ex}(n;\mathcal{F})\le \mathrm{ex}(n;G)$.

Combining the two inequalities yields equality.
\qed

\medskip
\textbf{Why the general case is hard.}
Outside Lemma~575.2, forbidding multiple graphs can interact nontrivially: graphs that are extremal for one forbidden subgraph can be structurally very different from extremal graphs for another.
The question asks whether this interaction can ever reduce $\mathrm{ex}(n;\mathcal{F})$ by more than a constant factor compared to every bipartite member of $\mathcal{F}$.

5) VERIFICATION

- Lemma~575.1 and Lemma~575.2 follow directly from the definition of subgraph containment and the monotonicity $\mathrm{ex}(n;\mathcal{F})\le \mathrm{ex}(n;F)$ for each $F\in\mathcal{F}$.
- In Lemma~575.2, both inclusions of forbidden classes were checked carefully: (i) $G$-free implies $\mathcal{F}$-free by $G\subseteq F$ for all $F$, and (ii) $\mathcal{F}$-free implies $G$-free because $G\in\mathcal{F}$.

6) FINAL

\textbf{UNRESOLVED}

(i) Strongest fully proved partial result:  
The conjecture holds in the degenerate case $K_2\in\mathcal{F}$ (Lemma~575.1) and in the structured case where some bipartite $G\in\mathcal{F}$ is a subgraph of every member of $\mathcal{F}$ (Lemma~575.2).

(ii) First gap (crisp):  
Either prove that for every finite family $\mathcal{F}$ containing a bipartite graph there exists a bipartite $G\in\mathcal{F}$ with $\mathrm{ex}(n;G)=O_{\mathcal{F}}(\mathrm{ex}(n;\mathcal{F}))$, or exhibit a concrete family $\mathcal{F}$ for which
\[\frac{\mathrm{ex}(n;G)}{\mathrm{ex}(n;\mathcal{F})}\to\infty\quad\text{for every bipartite }G\in\mathcal{F}.
\]

(iii) Top 3 next moves (concrete):
1. Systematically search for a counterexample among small families $\mathcal{F}$ built from standard bipartite obstructions (even cycles, complete bipartite graphs, trees) by comparing growth exponents.
2. Prove a ``winner-takes-all'' principle for families whose bipartite members have distinct known extremal exponents: show $\mathrm{ex}(n;\mathcal{F})$ is within a constant of the smallest exponent member.
3. Try to construct $\mathcal{F}=\{G_1,G_2\}$ with incomparable bipartite graphs where known extremal constructions for $G_1$ necessarily contain $G_2$ and vice versa, potentially forcing a new (smaller) extremal order.

(iv) Minimal counterexample structure:  
A counterexample would require a family $\mathcal{F}$ with at least two bipartite members such that avoiding all of them forces a substantially sparser graph than avoiding any one of them (ratio diverging with $n$). This would likely involve two bipartite graphs with extremal constructions of very different ``geometry'' so that the intersection of the two avoidance properties severely restricts edge density.


