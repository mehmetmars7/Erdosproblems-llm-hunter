% Erdos Problem #501

1) FORMAL RESTATEMENT

Let $m^*$ denote Lebesgue \emph{outer} measure on $\mathbb R$.
For each $x\in\mathbb R$, suppose we are given a set $A_x\subset\mathbb R$ such that:
- $A_x$ is bounded, and
- $m^*(A_x)<1$.

Call a set $X\subseteq\mathbb R$ \emph{independent} (for the map $x\mapsto A_x$) if for all distinct $x,y\in X$ we have $x\notin A_y$.

Question 1: Must there exist an \emph{infinite} independent set $X\subseteq\mathbb R$?

Question 2: If additionally each $A_x$ is \emph{closed} and has (Lebesgue) measure $m(A_x)<1$, must there exist an independent set of size $3$?

2) QUICK LITERATURE/CONTEXT CHECK

The problem statement reports: (a) Erd\H{o}s--Hajnal proved existence of arbitrarily large finite independent sets under the assumptions of Question 1; (b) Gladysz proved existence of an independent set of size $2$ under the assumptions of Question 2; (c) Hechler showed Question 1 can fail under CH; (d) Newelski--Pawlikowski--Seredy\'nski proved Question 1 has a positive answer if all $A_x$ are closed.

In what follows I do \emph{not} use these results as black boxes. Instead I prove a conditional version of the finite independent-set statement under an additional measurability hypothesis.

3) ATTACK PLAN

\emph{Proof track:} try to build an independent set by a measure/selection argument. The key difficulty is that the set
\[E:=\{(u,v)\in\mathbb R^2:\ u\in A_v\}
\]
need not be measurable in $\mathbb R^2$ without extra hypotheses on the assignment $v\mapsto A_v$.

\emph{Construction/disproof track:} understand how non-measurability of $E$ could obstruct Fubini-type estimates; identify where additional definability assumptions (Borel/analytic) restore them.

4) WORK

The following lemmas are \emph{conditional} on a measurability assumption for the relation $E$.
This is not part of the problem statement, so these are partial results only.

Assumption (measurability of the membership relation).
Assume that the set $E=\{(u,v): u\in A_v\}\subset\mathbb R^2$ is Lebesgue measurable.
(For example, this would hold if $(u,v)\mapsto \mathbf 1_{u\in A_v}$ were Borel measurable; the problem statement does not assume this.)

Lemma 501.1 (Independent pair under measurability).
Under the measurability assumption on $E$, there exist distinct $x,y\in\mathbb R$ such that $x\notin A_y$ and $y\notin A_x$.

\emph{Proof.}
Fix an interval $I\subset\mathbb R$ of length $L>2$.
Consider the measurable sets
\[E_I:=E\cap (I\times I),\qquad E_I^T:=\{(u,v)\in I\times I : (v,u)\in E\}.
\]
By Fubini/Tonelli (applied to the indicator of $E_I$),
\[
 m_2(E_I)=\int_{v\in I} m_1(\{u\in I: (u,v)\in E\})\,dv
=\int_{v\in I} m_1(A_v\cap I)\,dv.
\]
For each $v$, $m_1(A_v\cap I)\le m^*(A_v)<1$, hence $m_2(E_I)<\int_I 1\,dv = L$.
The same argument gives $m_2(E_I^T)<L$.
Therefore
\[
 m_2(E_I\cup E_I^T)\le m_2(E_I)+m_2(E_I^T) < 2L.
\]
But $m_2(I\times I)=L^2$, and since $L>2$ we have $L^2>2L$.
Hence $(I\times I)\setminus (E_I\cup E_I^T)$ has positive measure and is nonempty.
Pick $(x,y)$ in this complement with $x\ne y$ (the diagonal has measure $0$).
Then $(x,y)\notin E$ implies $x\notin A_y$, and $(x,y)\notin E^T$ implies $y\notin A_x$.
Thus $\{x,y\}$ is an independent set. \qed

Lemma 501.2 (Arbitrarily large finite independent sets under measurability).
Under the measurability assumption on $E$, for every integer $m\ge 1$ there exists an independent set $X\subset\mathbb R$ with $\lvert X\rvert\ge m$.

\emph{Proof.}
Fix $m\ge 1$. Choose an interval $I$ of length $L:=8m$.
Let $N:=4m$ and sample $N$ i.i.d. points $X_1,\dots,X_N$ uniformly from $I$.
Define an undirected graph $G$ on vertex set $\{1,\dots,N\}$ by putting an edge $\{i,j\}$ (with $i\ne j$) if either $X_i\in A_{X_j}$ or $X_j\in A_{X_i}$.
If $S\subset\{1,\dots,N\}$ is an independent set in this graph, then by definition for all distinct $i,j\in S$ we have $X_i\notin A_{X_j}$ and $X_j\notin A_{X_i}$; hence $\{X_i:i\in S\}$ is an independent set in the sense of the problem.

We bound the expected number of edges in $G$.
For $i\ne j$,
\[
\mathbb P[X_i\in A_{X_j}]
=\frac{1}{L^2}\,m_2(E\cap (I\times I))
=\frac{1}{L^2}\int_{v\in I} m_1(A_v\cap I)\,dv
\le \frac{1}{L^2}\int_I 1\,dv
=\frac{1}{L}.
\]
Similarly $\mathbb P[X_j\in A_{X_i}]\le 1/L$, so by the union bound
\[\mathbb P[\{i,j\}\text{ is an edge of }G]\le \frac{2}{L}.
\]
Let $M$ be the number of edges of $G$. By linearity of expectation,
\[
\mathbb E[M]
\le \binom{N}{2}\cdot \frac{2}{L}
=\frac{N(N-1)}{L}
<\frac{N^2}{L}.
\]
With $N=4m$ and $L=8m$, this gives $\mathbb E[M]<2m$.
By Markov's inequality, $\mathbb P[M\le 4m]>0$.
Fix an outcome with $M\le 4m$.

For any finite graph with $N$ vertices and $M$ edges, the average degree is $\bar d=2M/N$, and there exists an independent set of size at least $N/(\bar d+1)=N^2/(2M+N)$ (this follows from the Caro--Wei bound or a simple averaging argument).
Applying this with $M\le 4m$ and $N=4m$ gives
\[
\alpha(G)\ge \frac{N^2}{2M+N}\ge \frac{(4m)^2}{2\cdot (4m) + 4m}=\frac{16m^2}{12m}=\frac{4m}{3}>m.
\]
Choosing $m$ vertices in such an independent set yields an independent set of size $m$ in $\mathbb R$.
\qed

FAST REALITY CHECK

These lemmas are not about small-$n$ computation but about measure-theoretic structure. A quick sanity example where an infinite independent set exists: if $A_x=[x,x+1/2]$ for all $x$, then any strictly decreasing infinite sequence $x_1>x_2>\cdots$ is independent, since $x_i\notin [x_j,x_j+1/2]$ whenever $i<j$.

5) VERIFICATION

\emph{Where measurability is used:} Lemmas 501.1--501.2 use Fubini/Tonelli on $\mathbf 1_{E\cap (I\times I)}$. Without measurability of $E$, Fubini-type identities can fail, so these proofs do not resolve the original question.

\emph{Quantifiers:} The constructed independent sets are finite (size $m$ arbitrary) and depend on the measurability assumption; they do not imply existence of an infinite independent set.

6) FINAL
UNRESOLVED

(i) \emph{Strongest proved partial result here:} If the membership relation $E=\{(u,v):u\in A_v\}$ is Lebesgue measurable, then there exist independent sets of arbitrary finite size (Lemma 501.2), and in particular an independent pair (Lemma 501.1).

(ii) \emph{First gap (crisp):} Remove the extra hypothesis that $E$ is measurable: prove or disprove in ZFC that for every assignment $x\mapsto A_x$ with each $A_x$ bounded and $m^*(A_x)<1$ there exists an infinite independent set.

(iii) \emph{Top 3 next moves:}
1. Try to show $E$ is measurable (or sufficiently definable) under natural additional hypotheses (e.g. all $A_x$ Borel/closed) and see whether the argument can be iterated transfinitely to produce an infinite independent set.
2. Attempt a direct transfinite recursion construction of an infinite independent set that avoids needing Fubini on $E$.
3. Search for explicit (ZFC) pathological set-mappings $x\mapsto A_x$ of small outer measure that might block infinite free sets; pin down whether any such construction must use additional set-theory (e.g. CH).

(iv) \emph{Minimal counterexample structure:} a counterexample in ZFC to Question 1 would likely define a highly non-measurable relation $E\subset\mathbb R^2$ with all vertical sections $E_v=A_v$ of outer measure $<1$, but such that $E\cup E^T$ covers $X\times X$ for every infinite $X\subset\mathbb R$ (forcing an edge in every infinite subset).


