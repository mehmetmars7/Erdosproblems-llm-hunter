
1) FORMAL RESTATEMENT

Let \(\mathfrak{m}\) be an infinite cardinal with
\(\aleph_0<\mathfrak{m}<\mathfrak{c}=2^{\aleph_0}\).
Let \(\mathcal{F}=\{f_\alpha\}_{\alpha\in I}\) be a family of entire functions \(f_\alpha:\mathbb{C}\to\mathbb{C}\) such that
\[
\forall z_0\in\mathbb{C},\quad \bigl|\{f_\alpha(z_0):\alpha\in I\}\bigr|\le \mathfrak{m}.
\]
Question: must \(|I|\le \mathfrak{m}\)?

2) QUICK LITERATURE/CONTEXT CHECK

I did not browse the web; I only record what is stated in the problem text:

- Erd\H{o}s (1964) answered Wetzel's question: if each point has only countably many values (\(\mathfrak{m}=\aleph_0\)), then if \(\mathfrak{c}>\aleph_1\) the family is countable, while if \(\mathfrak{c}=\aleph_1\) this can fail.
- Hayman notes it is ``easy to see'' the answer is yes if \(\mathfrak{m}^+<\mathfrak{c}\).
- The case \(\mathfrak{m}^+=\mathfrak{c}\) can be undecidable: there are models with \(\mathfrak{c}=\aleph_2\) where the answer is yes for \(\mathfrak{m}=\aleph_1\) (Kumar--Shelah) and models where the answer is no (Schilhan--Weinert).

3) ATTACK PLAN

- Prove the ``easy'' case \(\mathfrak{m}^+<\mathfrak{c}\) using only the identity theorem and a counting/diagonal argument.
- Understand the remaining case \(\mathfrak{m}^+=\mathfrak{c}\) as set-theoretically sensitive; without reproducing forcing arguments, we can at least isolate why the naive counting proof breaks exactly at this boundary.

4) WORK

\textbf{Lemma 1119.1 (Coincidence sets of distinct entire functions are countable).}
If \(f,g\) are distinct entire functions, then
\(Z(f-g)=\{z\in\mathbb{C}: f(z)=g(z)\}\)
is at most countable.

\emph{Proof.}
Let \(h=f-g\), a nonzero entire function. Zeros of a nonzero holomorphic function are isolated. Hence for each \(n\in\mathbb{N}\), the intersection \(Z(h)\cap \overline{D(0,n)}\) is finite because \(\overline{D(0,n)}\) is compact and cannot contain infinitely many isolated points without an accumulation point.
Therefore
\(Z(h)=\bigcup_{n\ge 1} (Z(h)\cap \overline{D(0,n)})\)
is a countable union of finite sets, hence countable. \qed

\textbf{Lemma 1119.2 (Affirmative answer when \(\mathfrak{m}^+<\mathfrak{c}\)).}
Assume \(\mathfrak{m}^+<\mathfrak{c}\). Let \(\mathcal{F}=\{f_\alpha\}_{\alpha\in I}\) be a family of entire functions such that for every \(z_0\in\mathbb{C}\) there are at most \(\mathfrak{m}\) distinct values among \(\{f_\alpha(z_0):\alpha\in I\}\). Then \(|I|\le \mathfrak{m}\).

\emph{Proof.}
Suppose for contradiction that \(|I|>\mathfrak{m}\). Then there exists a subfamily \(\{f_\alpha\}_{\alpha\in J}\) with \(|J|=\mathfrak{m}^+\).
For distinct \(\alpha,\beta\in J\), define
\(E_{\alpha,\beta}:=\{z\in\mathbb{C}: f_\alpha(z)=f_\beta(z)\}\).
By Lemma 1119.1 each \(E_{\alpha,\beta}\) is countable.
Let
\[
U:=\bigcup_{\substack{\alpha,\beta\in J\\ \alpha\ne\beta}} E_{\alpha,\beta}.
\]
The index set of pairs \((\alpha,\beta)\) has cardinality \(|J|^2=(\mathfrak{m}^+)^2=\mathfrak{m}^+\) (for infinite cardinals, \(\kappa^2=\kappa\)). Thus \(U\) is a union of \(\mathfrak{m}^+\) countable sets, so
\(|U|\le \mathfrak{m}^+\cdot \aleph_0=\mathfrak{m}^+\).
Since \(\mathfrak{m}^+<\mathfrak{c}=|\mathbb{C}|\), we can choose \(z_0\in\mathbb{C}\setminus U\).
For this \(z_0\), no two distinct functions in the subfamily agree at \(z_0\); equivalently the values \(\{f_\alpha(z_0):\alpha\in J\}\) are all distinct. Hence
\(|\{f_\alpha(z_0):\alpha\in J\}|=|J|=\mathfrak{m}^+>\mathfrak{m}\), contradicting the hypothesis that at each point there are at most \(\mathfrak{m}\) values.
Therefore \(|I|\le \mathfrak{m}\). \qed

\textbf{FAST REALITY CHECK.}
- If \(\mathcal{F}\) is the family of constant entire functions with constant values in some set \(S\subset\mathbb{C}\) of size \(\mathfrak{m}\), then at each \(z_0\) there are exactly \(\mathfrak{m}\) values and \(|\mathcal{F}|=\mathfrak{m}\), so the bound \(|\mathcal{F}|\le\mathfrak{m}\) is sharp.
- Lemma 1119.2 specializes to Wetzel's countable-valued case when \(\mathfrak{m}=\aleph_0\) and \(\aleph_1<\mathfrak{c}\), matching the qualitative statement in the problem text.

5) VERIFICATION

- Lemma 1119.1 uses only the isolation of zeros and compactness of closed disks.
- Lemma 1119.2: the key counting step is that a union of \(\mathfrak{m}^+\) countable sets has size at most \(\mathfrak{m}^+\). This uses standard cardinal arithmetic \(\mathfrak{m}^+\cdot\aleph_0=\mathfrak{m}^+\) for infinite \(\mathfrak{m}^+\).
- The proof explicitly uses \(\mathfrak{m}^+<\mathfrak{c}\) to ensure \(\mathbb{C}\setminus U\ne\emptyset\); if \(\mathfrak{m}^+=\mathfrak{c}\), the argument fails exactly at this step.

6) FINAL

**UNRESOLVED**

(i) Strongest proved partial result: If \(\mathfrak{m}^+<\mathfrak{c}\), then any family of entire functions with pointwise value-set size \(\le\mathfrak{m}\) has cardinality \(\le\mathfrak{m}\) (Lemma 1119.2).

(ii) First gap (crisp): Settle the case \(\mathfrak{m}^+=\mathfrak{c}\) in ZFC: either prove \(|\mathcal{F}|\le\mathfrak{m}\) must hold in every model, or construct (in ZFC) a counterexample family of size \(>\mathfrak{m}\).

(iii) Top 3 next moves:
1. Try to strengthen Lemma 1119.2 by replacing the use of \(\mathfrak{m}^+<\mathfrak{c}\) with a different counting invariant (e.g. using a carefully chosen set of points of size \(\mathfrak{m}^+\) with special topological properties) that might still leave some point outside the coincidence union even when \(\mathfrak{m}^+=\mathfrak{c}\).
2. Attempt to construct explicit large Wetzel-type families under additional axioms (as in the problem text) and isolate which combinatorial principles are necessary; this can help pinpoint the exact independence mechanism.
3. Explore intermediate hypotheses between CH and \(\mathfrak{m}^+<\mathfrak{c}\) (e.g. Martin's axiom fragments) to see if they force either direction.

(iv) Minimal counterexample structure (if the answer can be ``no''): a counterexample must be a family \(\mathcal{F}\) of size \(>\mathfrak{m}\) in which many pairs of functions agree on many points (so that the coincidence union can cover all of \(\mathbb{C}\)), yet at each individual point only \(\le\mathfrak{m}\) distinct values occur. In particular, the naive ``pick a point where all values are distinct'' argument must fail because the union of coincidence sets is large enough to cover \(\mathbb{C}\), which is only possible when the index set of pairs has size at least \(\mathfrak{c}\), i.e. at the boundary \(\mathfrak{m}^+=\mathfrak{c}\).


