\section{ROUND-2 OBJECTIVE}

\textbf{Path (A): proof-oriented progress.}
Round~1 ended \textbf{UNRESOLVED} with two structural lemmas (evenness; powers of two) and exact values of \(h(n!)\) for \(n\le 9\).
In this round I focus on two gaps that are \emph{logically prior} to any positive answer to (Q1)--(Q2):

\begin{itemize}
\item develop a \emph{structural calculus} for \(h(\cdot)\) under natural operations on practical numbers (in particular, multiplication);
\item obtain \emph{rigorous lower bounds} for \(h(n!)\) (and more generally \(h(m)\)) showing what rates are even possible.
\end{itemize}

This direction is the most promising given Round~1 because it yields unconditional constraints and reduction principles that any eventual upper bound (e.g.\ \(h(n!)<(\log n)^{O(1)}\)) must respect.


\section{ROUND-1 FOUNDATION USED}

I use the following Round~1 results as black boxes:

\begin{itemize}
\item Lemma~18.1: if \(m>1\) is practical then \(2\mid m\).
\item Lemma~18.2: for \(k\ge 1\), \(2^k\) is practical and \(h(2^k)=k\).
\item Exact computations: for \(1\le n\le 9\),
\[
h(1!)=0,\ h(2!)=1,\ h(3!)=2,\ h(4!)=3,\ h(5!)=4,\ h(6!)=5,\ h(7!)=5,\ h(8!)=6,\ h(9!)=7.
\]
\end{itemize}


\section{NEW INSIGHT / TOOL (ROUND-2)}

\begin{enumerate}
\item \textbf{Multiplicative closure and subadditivity of \(h\).}
If \(a,b\) are practical, then \(ab\) is practical and
\[
h(ab)\le h(a)+h(b).
\]
This gives a nontrivial ``calculus'' for constructing new practical numbers with controlled \(h\).

\item \textbf{A counting lower bound for \(h(m)\) in terms of \(\tau(m)\).}
If \(m\) is practical with \(\tau(m)\) positive divisors, then necessarily
\[
\sum_{t=0}^{h(m)} \binom{\tau(m)-1}{t}\ \ge\ m.
\]
This converts lower bounds for \(h(m)\) into divisor-count estimates.

\item \textbf{An elementary estimate \(\log \tau(n!) = O(n)\).}
Using only the trivial bound ``\#primes in an interval \(\le\) \#integers in that interval'' plus the observation that \(v_p(n!)=\lfloor n/p\rfloor\) for \(p>\sqrt n\), one gets
\[
\log \tau(n!) \le S\,n + O(\sqrt n\log n)
\quad\text{with } S:=\sum_{k\ge 1}\frac{\log(k+1)}{k(k+1)}<\frac43.
\]
Combined with the counting lemma, this yields the new unconditional lower bound
\[
h(n!) \gg \log n.
\]

\item \textbf{Extended exact computations.}
I extend the exact computation of \(h(n!)\) from \(n\le 9\) to \(n\le 11\) (details in \S\ref{subsec:comp}).
\end{enumerate}


\section{ATTACK PLAN (ROUND-2)}

\textbf{Round~1 gap(s).} The main open gap is an \emph{upper bound} for \(h(n!)\) (and for infinitely many practical \(m\)) that is polylogarithmic in \(\log n\) (respectively polylogarithmic in \(\log\log m\)).

\textbf{Round~2 goals.} I address complementary necessities:

\begin{itemize}
\item Prove a structural lemma that allows one to \emph{build} practical numbers from smaller ones while controlling \(h\) (Lemma~18.3).
\item Prove a general \emph{lower bound mechanism} (Lemma~18.4) and apply it to factorials by bounding \(\tau(n!)\) (Lemma~18.6), obtaining \(h(n!)\ge c\log n\) (Corollary~18.7).
\item Extend the computation of \(h(n!)\) to \(n=10,11\) to constrain plausible growth rates and test compatibility with \(\Theta(\log n)\).
\end{itemize}

These steps overcome a Round~1 obstacle: without an a priori lower bound, it is unclear whether a target like \(h(n!)<(\log n)^{O(1)}\) is even close to the truth. The new lower bound shows \(h(n!)\) cannot be \(o(\log n)\), so any ``best possible'' bound must be at least order \(\log n\).


\section{WORK (ROUND-2)}

\subsection{Lemma 18.3: multiplicative closure and subadditivity of \(h\)}

\begin{lemma}[Lemma~18.3]\label{lem:subadd}
Let \(a,b\) be practical integers. Then \(ab\) is practical and
\[
h(ab)\le h(a)+h(b).
\]
\end{lemma}

\begin{proof}
Fix \(x\) with \(1\le x<ab\). Write \(x\) uniquely as
\[
x = a q + r,\qquad 0\le q<b,\quad 0\le r<a.
\]
Because \(b\) is practical and \(q<b\), there exist \emph{distinct} divisors \(d_1,\dots,d_s\mid b\) with
\[
q=\sum_{i=1}^s d_i,\qquad s\le h(b).
\]
Multiplying by \(a\) gives
\[
a q = \sum_{i=1}^s (a d_i),
\]
and each \(a d_i\) is a divisor of \(ab\), with all \(a d_i\) distinct.

Similarly, because \(a\) is practical and \(r<a\), there exist \emph{distinct} divisors \(e_1,\dots,e_t\mid a\) with
\[
r=\sum_{j=1}^t e_j,\qquad t\le h(a).
\]
Since \(r<a\), this representation cannot use the divisor \(a\); hence \(e_j<a\) for all \(j\).

Now note the two divisor sets are disjoint:
\[
\{a d_i\}_{i=1}^s \subseteq \{a,a\cdot 2,\dots\}\subseteq [a,ab],
\qquad
\{e_j\}_{j=1}^t \subseteq [1,a-1].
\]
Therefore
\[
x = a q + r = \sum_{i=1}^s (a d_i) + \sum_{j=1}^t e_j
\]
is a sum of \emph{distinct} divisors of \(ab\), using at most \(s+t\le h(b)+h(a)\) terms.

Since this holds for every \(1\le x<ab\), the number \(ab\) is practical and \(h(ab)\le h(a)+h(b)\).
\end{proof}

\begin{remark}
Lemma~\ref{lem:subadd} implies that practical numbers form a multiplicative semigroup. This closure property was not used in Round~1 and will be useful for constructing large practical numbers from smaller components.
\end{remark}


\subsection{Lemma 18.4: a counting lower bound for \(h(m)\)}

Write \(\tau(m)\) for the number of positive divisors of \(m\), and let \(D(m)\) be the set of \emph{proper} divisors of \(m\) (i.e.\ all divisors \(<m\)), so \(|D(m)|=\tau(m)-1\).

\begin{lemma}[Lemma~18.4]\label{lem:counting}
Let \(m\) be practical. Then necessarily
\[
\sum_{t=0}^{h(m)} \binom{\tau(m)-1}{t} \ \ge\ m.
\]
\end{lemma}

\begin{proof}
For each \(t\ge 0\), the number of ways to choose \(t\) distinct proper divisors is \(\binom{\tau(m)-1}{t}\).
Hence the total number of subsets of \(D(m)\) of size at most \(h(m)\) is
\[
N_h := \sum_{t=0}^{h(m)} \binom{\tau(m)-1}{t}.
\]
Each such subset determines \emph{one} integer sum (possibly \(>m\), and different subsets may yield the same sum). Therefore the number of \emph{distinct} integers representable as a sum of at most \(h(m)\) distinct proper divisors is \(\le N_h\).

By definition of \(h(m)\), every integer \(x\in\{1,2,\dots,m-1\}\) is representable in this way, and \(x=0\) is represented by the empty sum. Thus at least \(m\) distinct integers (\(0,1,\dots,m-1\)) are representable, so \(N_h\ge m\).
\end{proof}

\begin{corollary}[A convenient explicit form]\label{cor:counting}
If \(m\) is practical and \(\tau=\tau(m)\), then
\[
m \le (h(m)+1)(\tau-1)^{h(m)} \le \exp\!\big(h(m)\big(1+\log(\tau-1)\big)\big),
\]
and hence
\[
h(m)\ \ge\ \frac{\log m}{1+\log(\tau(m)-1)}.
\]
\end{corollary}

\begin{proof}
Using \(\binom{\tau-1}{t}\le (\tau-1)^t\) for each \(t\le h(m)\),
\[
\sum_{t=0}^{h(m)} \binom{\tau-1}{t}\ \le\ \sum_{t=0}^{h(m)} (\tau-1)^t\ \le\ (h(m)+1)(\tau-1)^{h(m)}.
\]
Combine this with Lemma~\ref{lem:counting}. For the exponential bound use \(\log(h(m)+1)\le h(m)\) and exponentiate.
\end{proof}

\begin{remark}
Lemma~\ref{lem:counting} is \emph{only} a necessary condition (the subset-sum map is not injective). Nevertheless, it gives robust lower bounds once \(\tau(m)\) is controlled.
\end{remark}


\subsection{Lemma 18.6: an elementary bound for \(\tau(n!)\)}

\begin{lemma}[Lemma~18.6]\label{lem:tau-factorial}
Let \(\tau(n!)\) be the divisor-counting function. Define the convergent constant
\[
S \ :=\ \sum_{k=1}^{\infty}\frac{\log(k+1)}{k(k+1)}.
\]
Then
\[
\log \tau(n!) \ \le\ S\,n + O(\sqrt n\log n)\qquad (n\to\infty),
\]
and in particular there exists an absolute \(n_0\) such that for all \(n\ge n_0\),
\[
\log \tau(n!) \le 2n.
\]
Moreover \(S<\log 2 + \frac{\pi^2}{6}-1<\frac43\).
\end{lemma}

\begin{proof}
Write \(v_p(n!)\) for the exponent of a prime \(p\) in \(n!\). Then
\[
\tau(n!)=\prod_{p\le n}\big(v_p(n!)+1\big),
\qquad
\log\tau(n!)=\sum_{p\le n}\log\big(v_p(n!)+1\big).
\]

\medskip
\noindent\textbf{Step 1: primes \(p\le \sqrt n\).}
For all such primes, \(v_p(n!)\le n\), hence \(\log(v_p(n!)+1)\le \log(n+1)\).
The number of primes \(\le \sqrt n\) is at most \(\lfloor\sqrt n\rfloor\), so
\[
\sum_{p\le \sqrt n}\log\big(v_p(n!)+1\big)\ \le\ \sqrt n\,\log(n+1).
\]

\medskip
\noindent\textbf{Step 2: primes \(p>\sqrt n\).}
If \(p>\sqrt n\), then \(p^2>n\), so \(v_p(n!)=\lfloor n/p\rfloor\).
For each integer \(k\in\{1,2,\dots,\lfloor\sqrt n\rfloor\}\), the primes with \(\lfloor n/p\rfloor=k\) lie in the interval
\[
\frac{n}{k+1} < p \le \frac{n}{k}.
\]
Hence the number of such primes is at most the number of integers in that interval:
\[
\#\Big\{p:\ \frac{n}{k+1} < p \le \frac{n}{k}\Big\}
\ \le\ \Big\lfloor\frac{n}{k}\Big\rfloor-\Big\lfloor\frac{n}{k+1}\Big\rfloor
\ \le\ \frac{n}{k(k+1)}+1.
\]
Therefore
\begin{align*}
\sum_{p>\sqrt n}\log\big(v_p(n!)+1\big)
&=\sum_{p>\sqrt n}\log\Big(\Big\lfloor\frac{n}{p}\Big\rfloor+1\Big)\\
&\le \sum_{k=1}^{\lfloor\sqrt n\rfloor}\Big(\frac{n}{k(k+1)}+1\Big)\log(k+1)\\
&= n\sum_{k=1}^{\lfloor\sqrt n\rfloor}\frac{\log(k+1)}{k(k+1)}\ +\ \sum_{k=1}^{\lfloor\sqrt n\rfloor}\log(k+1).
\end{align*}
The final sum is \(O(\sqrt n\log n)\), and the partial sums of the series defining \(S\) increase to \(S\). Thus
\[
\sum_{p>\sqrt n}\log\big(v_p(n!)+1\big)\ \le\ S\,n + O(\sqrt n\log n).
\]

\medskip
Combining Steps~1--2 yields the main estimate.

\medskip
\noindent\textbf{Step 3: bounding \(S\).}
Rewrite
\[
\frac{\log(k+1)}{k(k+1)}=\log(k+1)\Big(\frac1k-\frac1{k+1}\Big).
\]
For partial sums \(S_N:=\sum_{k=1}^N \frac{\log(k+1)}{k(k+1)}\), a telescoping rearrangement gives
\[
S_N = \log 2 + \sum_{k=2}^N \frac{\log(k+1)-\log k}{k} - \frac{\log(N+1)}{N+1}
\ \le\ \log 2 + \sum_{k=2}^N \frac{\log(1+1/k)}{k}.
\]
Using \(\log(1+u)\le u\) for \(u\ge 0\), we obtain
\[
S_N \le \log 2 + \sum_{k=2}^N \frac{1}{k^2}\ \le\ \log 2 + \sum_{k=2}^\infty \frac{1}{k^2}
= \log 2 + \frac{\pi^2}{6}-1 < \frac43.
\]
Letting \(N\to\infty\) yields \(S<\log 2 + \frac{\pi^2}{6}-1<\frac43\).

\medskip
Finally, since \(O(\sqrt n\log n)=o(n)\), there exists \(n_0\) such that \(S\,n + O(\sqrt n\log n)\le 2n\) for all \(n\ge n_0\).
\end{proof}


\subsection{Corollary 18.7: a rigorous lower bound \(h(n!)\gg \log n\)}

\begin{corollary}[Corollary~18.7]\label{cor:hn-lower}
There exists an absolute constant \(c>0\) such that for all sufficiently large \(n\),
\[
h(n!) \ \ge\ c\,\log n.
\]
More quantitatively, for all \(n\ge n_0\) as in Lemma~\ref{lem:tau-factorial},
\[
h(n!)\ \ge\ \frac{\log(n!)}{1+\log(\tau(n!)-1)}\ \ge\ \frac{\log n -1}{3}.
\]
\end{corollary}

\begin{proof}
Apply Corollary~\ref{cor:counting} with \(m=n!\). By Lemma~\ref{lem:tau-factorial}, for \(n\ge n_0\) we have \(\log(\tau(n!)-1)\le \log\tau(n!)\le 2n\), so
\[
h(n!)\ \ge\ \frac{\log(n!)}{1+2n}\ \ge\ \frac{\log(n!)}{3n}.
\]
Using the standard integral bound
\[
\log(n!)=\sum_{k=1}^n \log k \ \ge\ \int_1^n \log x\,dx\ =\ n\log n-n+1,
\]
we get \(h(n!)\ge (\log n-1)/3\) for all \(n\ge n_0\), proving the claim with \(c=1/4\), say.
\end{proof}

\begin{remark}
This shows \(h(n!)\) cannot be \(o(\log n)\). Thus any positive answer to the ``more strongly'' part of (Q2) must have exponent \(C\ge 1\) in a bound \(h(n!)\le (\log n)^C\).
\end{remark}


\subsection{Extended exact computations for \(h(n!)\) up to \(n=11\)}\label{subsec:comp}

\paragraph{Method (exact).}
For \(m=n!\) I compute the smallest \(h\) such that every \(x\in\{0,1,\dots,m-1\}\) is a sum of \(\le h\) \emph{distinct} proper divisors of \(m\).
Let \(D\) be the list of all proper divisors of \(m\). For a fixed \(H\) I maintain bitsets \(B_0,\dots,B_H\) where
\(B_t[s]=1\) iff \(s\) is representable as a sum of exactly \(t\) distinct divisors from the processed portion of \(D\).
Updating by each divisor \(d\in D\) uses the standard bounded-cardinality subset-sum recurrence
\[
B_t \leftarrow B_t \ \lor\ (B_{t-1}\ll d)\qquad (t=H,H-1,\dots,1),
\]
with truncation to the range \(0\le s\le m-1\).
Then \(h(m)\) is the smallest \(t\) for which \(B_0\lor\cdots\lor B_t\) has all bits \(0,1,\dots,m-1\) equal to \(1\).
This is exact (not heuristic) and also certifies practicality.

\paragraph{Results (exact).}
Combining the Round~1 table (\(n\le 9\)) with new exact computations for \(n=10,11\) yields:
\[
\begin{array}{c|c|c}
 n & n! & h(n!)\\\hline
 1 & 1 & 0\\
 2 & 2 & 1\\
 3 & 6 & 2\\
 4 & 24 & 3\\
 5 & 120 & 4\\
 6 & 720 & 5\\
 7 & 5040 & 5\\
 8 & 40320 & 6\\
 9 & 362880 & 7\\
 10 & 3628800 & 7\\
 11 & 39916800 & 7
\end{array}
\]
In particular, \(h(n!)\) stays at \(7\) for \(9\le n\le 11\), suggesting (non-rigorously) very slow growth.


\section{ADVERSARIAL VERIFICATION}

\begin{itemize}
\item \textbf{Lemma~18.3 (subadditivity):} the only delicate point is distinctness across the ``\(aq\)'' and ``\(r\)'' parts. This is secured because \(r<a\) implies no representation of \(r\) can use the divisor \(a\), so all divisors used for \(r\) are \(<a\), while all divisors used for \(aq\) are multiples of \(a\) and hence \(\ge a\). Edge cases:
  \begin{itemize}
  \item If \(a=1\), then \(x=1\cdot q + 0\) and the argument reduces to practicality of \(b\); also \(h(1)=0\).
  \item If \(b=1\), symmetric.
  \end{itemize}

\item \textbf{Lemma~18.4 (counting):} the argument only uses that each subset of \(\le h\) proper divisors produces at most one sum; collisions can only \emph{decrease} the number of representable integers, so the inequality is a necessary condition. Using proper divisors (excluding \(m\)) is essential because \(x<m\) cannot use the divisor \(m\).

\item \textbf{Lemma~18.6 (bounding \(\tau(n!)\)):}
  \begin{itemize}
  \item The identity \(v_p(n!)=\lfloor n/p\rfloor\) for \(p>\sqrt n\) is correct since \(p^2>n\Rightarrow \lfloor n/p^2\rfloor=0\).
  \item The bound on primes in \((n/(k+1),n/k]\) uses only that primes are integers; replacing primes by integers cannot undercount.
  \item The constant bound \(S<\log 2+\pi^2/6-1\) is rigorous and does not rely on numerics.
  \end{itemize}

\item \textbf{Corollary~18.7:} the only nontrivial input beyond Lemma~18.6 is the integral lower bound \(\log(n!)\ge n\log n-n+1\), which is standard and checked directly.

\item \textbf{Computations:} the bitset DP is an exact implementation of the bounded subset-sum recurrence for distinct divisors, and the search over \(t\) is monotone, so the minimal \(t\) found is the exact \(h(n!)\). As a sanity check, for \(n=10,11\) the DP fails for \(t\le 6\) and succeeds for \(t=7\), confirming minimality.
\end{itemize}


\section{FINAL}

\textbf{UNRESOLVED (BUT STRICTLY ADVANCED).}

New rigorous progress beyond Round~1:

\begin{itemize}
\item (Structure) Practical numbers are closed under multiplication and \(h\) is subadditive: \(h(ab)\le h(a)+h(b)\) (Lemma~18.3).
\item (Lower bounds) A general necessary counting condition for \(h(m)\) in terms of \(\tau(m)\) (Lemma~18.4), yielding the explicit inequality
\(\displaystyle h(m)\ge \frac{\log m}{1+\log(\tau(m)-1)}\).
\item (Factorials) An elementary estimate \(\log\tau(n!)=O(n)\) (Lemma~18.6), implying the unconditional lower bound \(h(n!)\gg \log n\) (Corollary~18.7). In particular \(h(n!)\) cannot be \(o(\log n)\).
\item (Computation) Exact values extended to \(n\le 11\): \(h(10!)=h(11!)=7\).
\end{itemize}

What remains open is an upper bound of the form \(h(n!)\le (\log n)^{O(1)}\) (or even \(h(n!)\le n^{o(1)}\)), and likewise the existence of infinitely many practical \(m\) with \(h(m)\le (\log\log m)^{O(1)}\).
The new lower bound shows that, for factorials, the \emph{best conceivable} polylogarithmic upper bound must have exponent at least \(1\).


\section{COMPLETION ESTIMATE}

\textbf{COMPLETION: 45\%}


\section{REFERENCES}

\begin{itemize}
\item (Round~1 output) Lemmas~18.1--18.2 and the table of exact values for \(h(n!)\) for \(n\le 9\).
\item Standard calculus fact: \(\displaystyle \int_1^n \log x\,dx = n\log n-n+1\).
\item Basel sum: \(\displaystyle \sum_{k=1}^\infty \frac1{k^2}=\frac{\pi^2}{6}\).
\end{itemize}
