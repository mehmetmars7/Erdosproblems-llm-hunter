
1) \textbf{FORMAL RESTATEMENT}

A positive integer $s$ is \emph{squarefree} if no prime square divides it (equivalently: for every prime $p$, $p^2\nmid s$).

Question: For every odd integer $n\ge 3$, does there exist an exponent $e\in\mathbb{Z}_{\ge 0}$ and a squarefree integer $s\ge 1$ such that
\[
 n = s + 2^e?
\]
(If one literally includes $n=1$, then the only possibility would be $1=s+2^e$ with $s\ge 1$, impossible; so the natural quantifier range is $n\ge 3$.)

Stress points:
\begin{itemize}
\item $e=0$ forces $s=n-1$ even; $e\ge 1$ forces $s$ odd.
\item A disproof would amount to a single odd $n$ such that every $n-2^e$ is divisible by some square.
\end{itemize}

2) \textbf{QUICK LITERATURE/CONTEXT CHECK}

From the problem statement: Odlyzko checked up to $10^7$; Hercher verified up to $2^{50}\approx 1.12\times 10^{15}$. Granville--Soundararajan relate this to Wieferich primes. Erd\H{o}s could prove the statement (with one power of $2$) for ``almost all'' $n$.

I do not invoke any of these external results in proofs; I only use elementary counting of squarefree numbers and a direct computation for small ranges.

3) \textbf{ATTACK PLAN}

\emph{Proof track:}
\begin{itemize}
\item Show that for each fixed $e$, $n-2^e$ is squarefree for a positive proportion of odd $n$ (easy via density of squarefree numbers).
\item Try to upgrade from ``positive proportion'' to ``all'' using many different exponents $e$ (e.g. show the union over $e$ covers all odds), which likely needs strong uniformity in squarefree distribution.
\end{itemize}

\emph{Disproof track:}
\begin{itemize}
\item Attempt to construct an odd $n$ such that for every $e$ the number $n-2^e$ is divisible by some square, via a covering system of congruences forcing $n\equiv 2^e \pmod{q_e^2}$.
\end{itemize}

4) \textbf{WORK}

\textbf{Lemma 11.1 (squarefree counting with explicit main term).}
Let
\[Q(N):=|\{1\le m\le N: \text{$m$ is squarefree}\}|.
\]
Then
\[
Q(N)=\frac{6}{\pi^2}N + O(\sqrt{N}).
\]

\textbf{Proof.}
Let $\mu$ be the Möbius function.
For each $m\ge 1$, the indicator of squarefreeness satisfies
\[
\mathbf{1}_{\mathrm{sqfree}}(m)=\sum_{d^2\mid m} \mu(d).
\]
(Reason: if $m$ is divisible by $p^2$ for some prime $p$, then the sum over $d^2\mid m$ includes both $d$ and $pd$ with opposite Möbius values and cancels; if $m$ is squarefree, then the only squares dividing $m$ are $1$, so the sum equals $\mu(1)=1$.)

Summing over $1\le m\le N$ and changing order,
\[
Q(N)=\sum_{m\le N}\sum_{d^2\mid m}\mu(d)
=\sum_{d\le \sqrt{N}} \mu(d)\sum_{m\le N,\ d^2\mid m}1
=\sum_{d\le \sqrt{N}} \mu(d)\Big\lfloor \frac{N}{d^2}\Big\rfloor.
\]
Write $\lfloor N/d^2\rfloor = N/d^2 + O(1)$. Then
\[
Q(N)=N\sum_{d\le\sqrt{N}}\frac{\mu(d)}{d^2} + O\Big(\sum_{d\le\sqrt{N}}1\Big)
= N\sum_{d\le\sqrt{N}}\frac{\mu(d)}{d^2} + O(\sqrt{N}).
\]
The series $\sum_{d=1}^\infty \mu(d)/d^2$ converges absolutely and equals $1/\zeta(2)=6/\pi^2$.
Moreover, the tail satisfies
\[
\sum_{d>\sqrt{N}}\Big|\frac{\mu(d)}{d^2}\Big| \le \sum_{d>\sqrt{N}}\frac{1}{d^2}=O(N^{-1/2}).
\]
Therefore
\[
\sum_{d\le\sqrt{N}}\frac{\mu(d)}{d^2}=\frac{6}{\pi^2}+O(N^{-1/2}),
\]
and plugging into the previous expression gives
\[
Q(N)=\frac{6}{\pi^2}N+O(\sqrt{N}).
\]
\qed

\textbf{Lemma 11.2 (positive proportion of odd integers are representable with a fixed power of 2).}
Fix an exponent $e\ge 0$.
Let
\[
S_e(N):=|\{1\le n\le N: n\text{ odd and } n-2^e\ge 1 \text{ and } (n-2^e)\text{ squarefree}\}|.
\]
Then $S_e(N)=c_e\,\frac{N}{2}+O(\sqrt{N})$ where
\[
 c_e = \begin{cases}
  \frac{4}{\pi^2} & e=0,\\
  \frac{8}{\pi^2} & e\ge 1.
 \end{cases}
\]
In particular, for each fixed $e$ a positive proportion of odd integers are representable as ``squarefree + $2^e$''.

\textbf{Proof.}
If $e\ge 1$ then $2^e$ is even, so $n$ is odd if and only if $m:=n-2^e$ is odd. The map $m\mapsto n=m+2^e$ is a bijection between odd squarefree $m\le N-2^e$ and odd $n\le N$ with $n-2^e$ squarefree. Hence
\[
S_e(N)=|\{1\le m\le N-2^e: m\text{ odd and squarefree}\}|.
\]
Similarly, if $e=0$ then $n$ odd corresponds to $m=n-1$ even, so
\[
S_0(N)=|\{1\le m\le N-1: m\text{ even and squarefree}\}|.
\]
Thus it suffices to count odd/even squarefree numbers.

Let $Q(N)$ be as in Lemma~11.1.
The count of odd squarefree numbers up to $N$ equals
\[
Q_{\mathrm{odd}}(N):=|\{1\le m\le N: m\text{ squarefree and odd}\}| = Q(N)-Q_{\mathrm{even}}(N).
\]
An even integer $m$ is squarefree iff $m=2u$ where $u$ is odd and squarefree (because squarefree forbids $4\mid m$ and forbids odd prime squares).
So $Q_{\mathrm{even}}(N)=Q_{\mathrm{odd}}(\lfloor N/2\rfloor)$.

Now apply Lemma~11.1 twice:
\[
Q(N)=\frac{6}{\pi^2}N+O(\sqrt{N}),
\qquad
Q\Big(\Big\lfloor\frac{N}{2}\Big\rfloor\Big)=\frac{6}{\pi^2}\frac{N}{2}+O(\sqrt{N}).
\]
Writing $Q_{\mathrm{odd}}(N)=Q(N)-Q_{\mathrm{even}}(N)=Q(N)-Q_{\mathrm{odd}}(\lfloor N/2\rfloor)$ and solving this recursion at the level of main terms gives densities:
\[
Q_{\mathrm{odd}}(N)=\frac{4}{\pi^2}N+O(\sqrt{N}),
\qquad
Q_{\mathrm{even}}(N)=\frac{2}{\pi^2}N+O(\sqrt{N}).
\]
(Indeed $\frac{4}{\pi^2}+\frac{2}{\pi^2}=\frac{6}{\pi^2}$ and $Q_{\mathrm{even}}(N)=Q_{\mathrm{odd}}(N/2)$ is consistent with the constants.)

Therefore:
\begin{itemize}
\item If $e\ge 1$, then $S_e(N)=Q_{\mathrm{odd}}(N-2^e)=\frac{4}{\pi^2}(N-2^e)+O(\sqrt{N})=\frac{8}{\pi^2}\cdot\frac{N}{2}+O(\sqrt{N})$.
\item If $e=0$, then $S_0(N)=Q_{\mathrm{even}}(N-1)=\frac{2}{\pi^2}N+O(\sqrt{N})=\frac{4}{\pi^2}\cdot\frac{N}{2}+O(\sqrt{N})$.
\end{itemize}
This gives the claimed constants $c_e$.
\qed

\textbf{FAST REALITY CHECK / COMPUTATION.}
I computed all odd $n\le 10^6$ and checked whether there exists $e$ with $2^e\le n$ such that $n-2^e$ is squarefree (using a sieve that marks multiples of squares).
Result: among odd $n\le 10^6$, the only failure is $n=1$; every odd $3\le n\le 10^6$ has at least one representation.

5) \textbf{VERIFICATION}

\begin{itemize}
\item Lemma~11.1: the Möbius inversion identity $\mathbf{1}_{\mathrm{sqfree}}(m)=\sum_{d^2\mid m}\mu(d)$ is checked by considering whether $m$ has a squared prime factor. The error term is controlled by the trivial bound $\sum_{d\le\sqrt{N}}1=O(\sqrt{N})$.
\item Lemma~11.2: the only nontrivial step is the characterization of even squarefree numbers as $2\times$ odd squarefree numbers; this follows from the definition of squarefree (no prime square divides $m$).
\item Computation: squarefreeness was determined exactly by precomputing a boolean table where any multiple of $p^2$ is marked non-squarefree.
\end{itemize}

6) \textbf{FINAL}

\textbf{UNRESOLVED}

(i) Strongest proved partial result: Lemma~11.2 shows that for every fixed $e$ a positive proportion of odd integers are representable as ``squarefree + $2^e$'', with explicit constants (derived from Lemma~11.1). Computationally, every odd $3\le n\le 10^6$ has such a representation.

(ii) First gap (crisp): Show that \emph{every} odd $n\ge 3$ has at least one exponent $e$ with $n-2^e$ squarefree, i.e.
\begin{quote}
$\forall$ odd $n\ge 3\ \exists e\ge 0: \text{$n-2^e\ge 1$ and } \forall p\ (p^2\nmid (n-2^e)).$
\end{quote}

(iii) Top 3 next moves:
\begin{enumerate}
\item Attempt a covering-congruence disproof search: force $n-2^e$ to be divisible by a square for each $e$ by imposing congruences modulo $q_e^2$ (for many $e$) and combine using CRT.
\item Prove uniform distribution of squarefree numbers in the specific set of shifts $n-2^e$ (for $e$ up to $\log_2 n$), strong enough to ensure the union over $e$ covers every $n$.
\item Computationally, perform a directed search for counterexamples by keeping track of the set of exponents $e$ for which $n-2^e$ is forced non-squarefree by small square divisors, and try to build $n$ satisfying all such constraints.
\end{enumerate}

(iv) Minimal counterexample structure: a minimal odd counterexample $n$ would satisfy that for every $e$ with $2^e\le n-1$, the number $n-2^e$ is divisible by some square $q^2>1$. Equivalently, the set $\{n-2^e: 0\le e\le \lfloor\log_2 n\rfloor\}$ would avoid the squarefree numbers entirely, which would require $n$ to satisfy many simultaneous congruence conditions modulo various squares.


