
1) FORMAL RESTATEMENT

A set $A\subseteq\mathbb N$ is an \emph{additive basis of order $2$} (an \emph{asymptotic basis of order $2$}) if there exists an integer $N_0$ such that for every integer $n\ge N_0$ there exist $a,a'\in A$ with
\[
n=a+a'.
\]

Given such an $A$, the problem asks whether there must exist a subset
\[
B=\{b_1<b_2<\cdots\}\subseteq A
\]
which is also an additive basis of order $2$ and for which the limit
\[
\lim_{k\to\infty}\frac{b_k}{k^2}
\]
\emph{does not exist}.

Edge cases.
Any basis $A$ is infinite. The ratio $b_k/k^2$ is always nonnegative.


2) QUICK LITERATURE/CONTEXT CHECK

I use only what is in the statement: Erd\H{o}s asked the stronger question with $B=A$, and Cassels produced a counterexample (a basis $A$ for which $a_k/k^2$ does converge).
I do not use any other external facts.


3) ATTACK PLAN

I do not see a complete proof or disproof.
I will provide basic necessary growth constraints for any order-2 basis, since the question is about the scale $k^2$.

Specifically:

(1) Show that any basis $A$ must satisfy a lower bound on its counting function $A(x)=|A\cap\{1,\dots,x\}|$ of order $\gg\sqrt{x}$ for large $x$.

(2) Convert that into an upper bound $a_k\ll k^2$ for the $k$th element $a_k$ of $A$.

These are problem-specific lemmas with complete proofs.


4) WORK

PHASE 1: FAST REALITY CHECK

There is no meaningful finite brute-force test for an \emph{infinite} asymptotic basis. As a sanity check, note:

* $A=2\mathbb N\cup\{1\}$ is an order-2 basis (all large even numbers are even+even; all large odd numbers are $1+$even).
* Cassels' example (mentioned in the statement) shows that the limit $a_k/k^2$ can exist for some bases.


Lemma 326.1 (counting function lower bound for any order-2 basis).

Let $A\subseteq\mathbb N$ be an additive basis of order $2$.
Then there exists $X_0$ such that for all $x\ge X_0$,
\[
|A\cap\{1,2,\dots,x\}|\ \ge\ \sqrt{\frac{x}{2}}.
\]

Proof.
Because $A$ is a basis of order $2$, there exists $N_0$ such that every $n\ge N_0$ can be written as $n=a+a'$ with $a,a'\in A$.
Fix $T\ge N_0$.
Then every integer $n\in[T,2T]$ can be written as $n=a+a'$ with $a,a'\in A$.
In such a representation, necessarily $a\le n\le 2T$ and $a'\le n\le 2T$.
So $a,a'\in A\cap\{1,\dots,2T\}$.
Hence
\[
[T,2T]\cap\mathbb Z \subseteq (A\cap\{1,\dots,2T\}) + (A\cap\{1,\dots,2T\}).
\]
The set on the right is the set of all sums of two elements from $A\cap\{1,\dots,2T\}$.
The number of \emph{distinct} sums is at most the number of ordered pairs, i.e.
\[
\bigl| (A\cap\{1,\dots,2T\}) + (A\cap\{1,\dots,2T\}) \bigr|\ \le\ |A\cap\{1,\dots,2T\}|^2.
\]
But the left side contains all integers in $[T,2T]$, which are $T+1$ distinct integers.
Therefore
\[
|A\cap\{1,\dots,2T\}|^2 \ge T+1.
\]
Taking square roots gives
\[
|A\cap\{1,\dots,2T\}|\ge \sqrt{T+1} \ge \sqrt{T}.
\]
Now set $x=2T$. For $x\ge 2N_0$ we have $T=x/2\ge N_0$, and thus
\[
|A\cap\{1,\dots,x\}|\ge \sqrt{x/2}.
\]
So the claimed inequality holds for all $x\ge X_0:=2N_0$. \qed


Lemma 326.2 (quadratic upper bound on the $k$th element of a basis).

Let $A\subseteq\mathbb N$ be an additive basis of order $2$, and enumerate it as $a_1<a_2<\cdots$.
Then there exists $k_0$ such that for all $k\ge k_0$,
\[
a_k\le 2k^2.
\]

Proof.
Let $X_0$ be as in Lemma 326.1.
Take any $k$ such that $a_k\ge X_0$.
Then
\[
|A\cap\{1,\dots,a_k\}|\ge \sqrt{a_k/2}.
\]
But $|A\cap\{1,\dots,a_k\}|=k$ by definition of $a_k$.
So $k\ge \sqrt{a_k/2}$, i.e. $a_k\le 2k^2$.
Taking $k_0$ such that $a_{k_0}\ge X_0$ completes the proof. \qed


5) VERIFICATION

-- Lemma 326.1: the crucial observation is that to represent $n\le 2T$ as a sum of two naturals, both summands are automatically $\le 2T$; hence restricting to $A\cap[1,2T]$ is valid.

-- Lemma 326.2: correctly inverts the counting bound by evaluating it at $x=a_k$.


6) FINAL

**UNRESOLVED**

(i) Strongest fully proved partial result obtained here.

* Any order-2 basis must have counting function $A(x)\gg\sqrt{x}$ (Lemma 326.1).
* Consequently, its $k$th element satisfies $a_k\ll k^2$ (Lemma 326.2).

(ii) Exact first gap.

Given an arbitrary order-2 basis $A$, construct a subbasis $B\subseteq A$ whose growth oscillates enough that $b_k/k^2$ has no limit. I do not have a general method to thin an arbitrary basis while controlling representations so finely.

(iii) Top 3 next moves (concrete targets).

1. Develop a ``thinning with stability'' lemma: if $A$ is a basis, remove elements in blocks to force oscillation while preserving the basis property.
2. Investigate whether there exist minimal order-2 bases with a regular asymptotic $a_k\sim c k^2$; if such exist, they would obstruct this problem.
3. For model examples (e.g. $A=2\mathbb N\cup\{1\}$), explicitly construct subbases with nonconvergent $b_k/k^2$ to understand what ingredients are needed.

(iv) What a minimal counterexample would likely look like.

A counterexample would be a basis $A$ such that every subbasis $B\subseteq A$ has $b_k/k^2$ convergent. This would likely require $A$ to be ``essentially minimal'' (so there is very little freedom to delete elements) together with a strong rigidity forcing quadratic growth with a limiting constant.


