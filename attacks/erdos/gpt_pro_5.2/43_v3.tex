\section{Erd\H{o}s Problem \#43 (Round 3)}

\subsection{1) ROUND-3 OBJECTIVE}

\textbf{Path chosen: (A) --- pursue a full proof of Question~(1), while closing concrete gaps and repairing any incorrect intermediate claims encountered in Round~2 material.}

Round~2 left Question~(1) unresolved, but (i) produced strong constructions (showing Question~(2) is false), and (ii) supplied exact small-$N$ data suggesting the excess
\[
E(N):=g(N)-\binom{f(N)}{2}
\]
may be bounded by a small constant.
In this round I focus on two gap-closures that are prerequisites for any proof strategy based on the Round~2 toolkit:
\begin{enumerate}
\item identify and rigorously repair an error in the ``Tao-type'' convolution argument recorded in \texttt{43.tex};
\item correct and extend the exact computation of $g(N)$ beyond the Round~2 range.
\end{enumerate}

\subsection{2) Round-2 FOUNDATION USED}

I rely on the following Round~2 facts (treated as vetted unless explicitly corrected below):
\begin{itemize}
\item (Uniqueness of differences) If $S\subset\mathbb Z$ is Sidon, then for every $d\ne 0$ there is at most one ordered pair $(s,s')\in S^2$ with $s-s'=d$.
\item (Cross-sum injectivity) If $A,B\subset[N]$ are Sidon and $(A-A)\cap(B-B)=\{0\}$, then the map $(a,b)\mapsto a+b$ on $A\times B$ is injective, hence $|A||B|\le 2N-1$.
\item (Barreto/Bose--Chowla construction) For infinitely many $N$ there exist $A,B\subset[N]$ with $(A-A)\cap(B-B)=\{0\}$ and $|A|=|B|=(1/\sqrt2+o(1))\sqrt N$, yielding $\binom{|A|}{2}+\binom{|B|}{2}=\tfrac12 N+o(N)$.
\item (Round~2 computations) Exact values were computed up to $N=30$ with observed $\max_{N\le 30}E(N)=6$.
\end{itemize}

\subsection{3) NEW INSIGHT / TOOL (ROUND-3)}

\begin{itemize}
\item \textbf{Gap repair:} the inequality used in \texttt{43.tex} (attributed to Tao) is shown to be \emph{tautological as written} and therefore cannot yield the claimed bound $|A|^2+|B|^2\le N+O(\sqrt N)$. I give the corrected pointwise bound and the resulting (still too weak) inequality, isolating exactly what extra ingredient would be needed.
\item \textbf{New structural lemma:} cross-\emph{difference} injectivity $(a,b)\mapsto a-b$ for $A\times B$ (a companion to cross-sum injectivity).
\item \textbf{Corrected/extended computation:} exact $f(N),g(N),E(N)$ are recomputed and extended to $N\le 33$, correcting an internal inconsistency in the Round~2 table.
\end{itemize}

\subsection{4) ATTACK PLAN (ROUND-3)}

\textbf{Gaps remaining after Round~2.}
\begin{enumerate}
\item The Round~2 write-up records a ``Tao lemma'' that purportedly implies $|A|^2+|B|^2\le N+O(\sqrt N)$; this would be a major step toward understanding the extremal regime. However, the argument (as currently written in \texttt{43.tex}) does not actually produce such a bound.
\item The computational evidence stopped at $N=30$ and contains at least one numerical inconsistency.
\end{enumerate}

\textbf{Plan to advance.}
\begin{enumerate}
\item Prove that the displayed inequality in \texttt{43.tex} is tautological and cannot upper-bound $|A|^2+|B|^2$; then state the corrected diagonal term and derive the true inequality obtained from the same method.
\item Provide corrected exact values of $g(N)$ and $E(N)$ up to $N=33$.
\end{enumerate}

\subsection{5) WORK (ROUND-3)}

\subsubsection{5.1\; Repairing the ``Tao-type'' convolution argument}

Let $A,B\subset[N]$ be Sidon and satisfy $(A-A)\cap(B-B)=\{0\}$.
Write
\[
\Delta_A(n):=(\mathbf 1_A*\mathbf 1_{-A})(n)=\#\{(a,a')\in A^2: a-a'=n\},
\]
and similarly $\Delta_B$.
By Sidonicity, $\Delta_A(n)\le 1$ for $n\ne 0$, and $\Delta_A(0)=|A|$.
Because nonzero differences are disjoint between $A$ and $B$,
\begin{equation}\label{eq:pointwise-correct}
\Delta_A(n)+\Delta_B(n)\le (|A|+|B|)\mathbf 1_{\{0\}}(n)+1\qquad\text{for all }n\in\mathbb Z.
\end{equation}

\medskip
\noindent\textbf{What is written in \texttt{43.tex}.}
The file \texttt{43.tex} instead records the weaker pointwise bound
\begin{equation}\label{eq:pointwise-too-weak}
\Delta_A(n)+\Delta_B(n)\le (|A|^2+|B|^2)\mathbf 1_{\{0\}}(n)+1,
\end{equation}
which is true but grossly nonsharp at $n=0$ (since $\Delta_A(0)+\Delta_B(0)=|A|+|B|$).
Using \eqref{eq:pointwise-too-weak}, the argument then chooses the weight
\(
 w:=\mathbf 1_{[1,H]}*\mathbf 1_{-[1,H]}
\)
and obtains (as in \texttt{43.tex})
\begin{equation}\label{eq:tao-inequality-as-written}
\frac{(|A|^2+|B|^2)H^2}{N+H}\le (|A|^2+|B|^2)H + H^2.
\end{equation}
\textbf{Claim:} \eqref{eq:tao-inequality-as-written} is \emph{tautological} and implies no bound on $|A|^2+|B|^2$.
Indeed, since $H^2/(N+H)\le H$ for $N,H\ge 1$, we have
\[
\frac{(|A|^2+|B|^2)H^2}{N+H}\le (|A|^2+|B|^2)H\le (|A|^2+|B|^2)H+H^2,
\]
so \eqref{eq:tao-inequality-as-written} holds for \emph{every} $A,B$ and gives no information.
In particular, the step ``choose $H\asymp\sqrt N$ to deduce $|A|^2+|B|^2\le N+O(\sqrt N)$'' is invalid.

\medskip
\noindent\textbf{What the same method yields with the correct diagonal term.}
If one replaces \eqref{eq:pointwise-too-weak} by the sharp bound \eqref{eq:pointwise-correct} and repeats the identical convolution/\(\ell^2\) steps (Cauchy--Schwarz for the lower bound on $\|\mathbf 1_A*\mathbf 1_{[1,H]}\|_2^2$ and similarly for $B$), one instead obtains
\begin{equation}\label{eq:tao-inequality-corrected}
\frac{(|A|^2+|B|^2)H^2}{N+H}\le (|A|+|B|)H + H^2.
\end{equation}
This is now genuinely nontrivial, but it still falls short of bounding $|A|^2+|B|^2$ by $N+O(\sqrt N)$ because the right-hand side depends on $|A|+|B|$.
Any attempt to extract a sharp ``$1/\sqrt 2$'' regime from \eqref{eq:tao-inequality-corrected} would require an additional independent inequality relating $|A|+|B|$ to $|A|^2+|B|^2$ \emph{beyond} the trivial Cauchy--Schwarz bound $|A|+|B|\le \sqrt{2(|A|^2+|B|^2)}$.
(Equivalently, one needs some cancellation mechanism or a mean-zero weight in place of the nonnegative $w$ above.)

\subsubsection{5.2\; New lemma: cross-difference injectivity}

\begin{lemma}[Cross-difference injectivity]\label{lem:cross-diff}
Let $A,B\subset[N]$ be Sidon and satisfy $(A-A)\cap(B-B)=\{0\}$. Then the map
\[
\varphi: A\times B\to\mathbb Z,\qquad \varphi(a,b)=a-b
\]
is injective. In particular, $|A||B|\le 2N-1$ (since $a-b\in [-(N-1),N-1]$).
\end{lemma}

\begin{proof}
Suppose $a_1-b_1=a_2-b_2$ for $(a_i,b_i)\in A\times B$. Rearranging gives
\(
 a_1-a_2=b_1-b_2.
\)
If $a_1\ne a_2$, then $a_1-a_2\ne 0$ is a nonzero element of $(A-A)\cap(B-B)$, contradiction.
Thus $a_1=a_2$, and then $b_1=b_2$.
\end{proof}

\subsubsection{5.3\; Corrected and extended exact computations (up to $N=33$)}

\paragraph{Method.}
In $\mathbb Z$, a Sidon set is equivalent to a Golomb ruler: all positive differences $d=b-a$ $(b>a)$ are distinct.
For each $N\le 33$ I enumerated all Golomb-ruler subsets $S\subset[N]$ with $\min S=1$ by backtracking, maintaining a bitmask of used positive differences in $\{1,\dots,N-1\}$.
Each such $S$ yields its positive-difference mask $D(S)\subseteq\{1,\dots,N-1\}$ with weight $|D(S)|=\binom{|S|}{2}$.
Then
\[
 g(N)=\max\{\;|D(S_1)|+|D(S_2)|\;:\; D(S_1)\cap D(S_2)=\emptyset\;\}
\]
was computed by scanning masks in descending weight and taking the best disjoint pair.

\paragraph{Consistency check against Round~2.}
Round~2 reported $E(26)=5$ with $f(26)=7$, which would force
$g(26)=\binom72+5=21+5=26$.
But always $g(N)\le N-1$ (there are only $N-1$ possible positive differences), so $g(26)\le 25$.
This contradiction proves the Round~2 table contained an error (at least at $N=26$).

\paragraph{Results.}
The corrected exact values for $24\le N\le 33$ are:
\[
\begin{array}{c|cccccccccc}
N & 24&25&26&27&28&29&30&31&32&33\\\hline
f(N) & 6&6&7&7&7&7&7&7&7&7\\
 g(N)&21&21&22&24&24&25&27&27&27&27\\
E(N)=g(N)-\binom{f(N)}{2}
&6&6&1&3&3&4&6&6&6&6
\end{array}
\]
In particular, for all $N\le 33$,
\begin{equation}\label{eq:Emax33}
E(N)\le 6.
\end{equation}

\paragraph{Additional explicit lower bounds from MathOverflow.}
The MathOverflow discussion of Problem~43 gives explicit pairs $(A,B)$ with
\(\binom{|A|}{2}+\binom{|B|}{2}>\binom{f(N)}{2}\)
for $N=14,23,35$.
In particular, for $N=35$ it exhibits $\binom{|A|}{2}+\binom{|B|}{2}=30$ while $f(35)=8$ gives $\binom{f(35)}{2}=28$, hence $E(35)\ge 2$.
These examples remain consistent with the empirical upper bound \eqref{eq:Emax33}.

\subsection{6) ADVERSARIAL VERIFICATION}

\paragraph{(i) Lemma~5.1 diagnosis.}
The tautology check for \eqref{eq:tao-inequality-as-written} is airtight: it reduces to $H^2/(N+H)\le H$.
Thus no rearrangement can bound $|A|^2+|B|^2$.
Any future ``convolution with a nonnegative kernel'' approach must therefore (a) retain the sharp diagonal term $|A|+|B|$ rather than $|A|^2+|B|^2$, and/or (b) use a sign-changing (mean-zero) weight to exploit the negativity of $\Delta_A(n)+\Delta_B(n)-1$ for most $n$.

\paragraph{(ii) Cross-difference lemma edge cases.}
If $a_1=a_2$ then the proof forces $b_1=b_2$; if $b_1=b_2$ similarly.
The only substantive case is $a_1\ne a_2$, where disjointness of nonzero differences is exactly the needed hypothesis.
No torsion issues arise because we work in $\mathbb Z$.

\paragraph{(iii) Computational checks.}
Each reported $g(N)$ satisfies $g(N)\le N-1$.
Moreover, the large values $g(31)=g(32)=g(33)=27$ are consistent with the simple pigeonhole restriction: if $|A|=7$ then $|D(A)|=21$ leaving at most $10$ differences for $B$, forcing $|B|\le 5$.

\subsection{7) FINAL}

\textbf{UNRESOLVED (BUT STRICTLY ADVANCED).}

Round~3 progress beyond Round~2:
\begin{itemize}
\item The step in \texttt{43.tex} asserting that a Tao-style inequality implies $|A|^2+|B|^2\le N+O(\sqrt N)$ is conclusively identified as incorrect: the displayed inequality is tautological.
\item The sharp pointwise bound \eqref{eq:pointwise-correct} is recorded and the correctly derived inequality \eqref{eq:tao-inequality-corrected} is given.
\item Lemma~\ref{lem:cross-diff} (cross-difference injectivity) is added.
\item Exact computations are corrected and extended to $N\le 33$, with $\max_{N\le 33}E(N)=6$.
\end{itemize}

\subsection{8) COMPLETION ESTIMATE (MANDATORY)}

\textbf{COMPLETION: 55\%}

\subsection{9) REFERENCES}

\begin{itemize}
\item \texttt{43.tex} (problem statement and Round~1/2 write-up supplied in this task).
\item MathOverflow thread ``Erd\H{o}s problem 43'' (for explicit example pairs at $N=14,23,35$).
\item Standard equivalence in $\mathbb Z$ between Sidon sets and Golomb rulers (uniqueness of positive differences).
\end{itemize}
