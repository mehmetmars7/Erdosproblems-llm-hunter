\section*{Erd\H{o}s Problem \#295}

\subsection*{1. FORMAL RESTATEMENT}

For an integer $N\ge1$, define $k(N)$ to be the least integer $k$ such that there exist integers
\[
N\le n_1<n_2<\cdots<n_k
\]
with
\[
\sum_{i=1}^k \frac1{n_i}=1.
\]
The problem asks whether
\[
\lim_{N\to\infty}\bigl(k(N)-(e-1)N\bigr)=+\infty.
\]

\subsection*{2. QUICK LITERATURE/CONTEXT CHECK (only if browsing is available)}

The Erd\H{o}s Problems site attributes to Erd\H{o}s--Straus (1971) the bounds
\[
-c < k(N)-(e-1)N \ll \frac{N}{\log N}
\]
for some constant $c>0$ (so $k(N)=(e-1)N+O(N/\log N)$ and the deviation is bounded below by a negative constant). The question is whether the deviation necessarily grows without bound.

\subsection*{3. ATTACK PLAN}

\begin{enumerate}
\item Prove easy, self-contained necessary conditions on $k(N)$ using harmonic sums (these immediately yield $k(N)\ge (e-1)N-O(1)$).
\item Check small cases by hand (find exact $k(N)$ for small $N$) to see what the deviation looks like.
\item Outline a route one might try for showing unbounded deviation: understand the ``granularity'' of achievable sums of distinct reciprocals near $1$ when restricted to denominators $\ge N$, and show a growing ``integrality gap'' between $1$ and the closest such sums.
\end{enumerate}

\subsection*{4. WORK}

\subsubsection*{4.1 A universal harmonic-series lower bound}

\begin{proposition}[Consecutive-denominator domination]
\label{prop:harmonicdom}
If $N\le n_1<\cdots<n_k$ and $\sum_{i=1}^k 1/n_i = 1$, then
\[
1\le \sum_{i=0}^{k-1}\frac1{N+i}.
\]
\end{proposition}
\begin{proof}
Since $n_i\ge N+i-1$ for each $i$, we have $1/n_i\le 1/(N+i-1)$. Summing gives
\[
1=\sum_{i=1}^k \frac1{n_i}\le \sum_{i=1}^k \frac1{N+i-1}=\sum_{i=0}^{k-1}\frac1{N+i}.
\]
\end{proof}

Thus, to even have a chance of writing $1$ as a sum of $k$ distinct reciprocals with all denominators $\ge N$, it is necessary that the first $k$ reciprocals starting at $N$ already sum to at least $1$.

\subsubsection*{4.2 Converting to an explicit bound using an integral}

Using $\frac1{N+i}\le \int_{N+i-1}^{N+i}\frac{dx}{x}$ (because $1/x$ decreases), we have
\[
\sum_{i=0}^{k-1}\frac1{N+i}
\le \int_{N-1}^{N+k-1}\frac{dx}{x}
=\log\Bigl(\frac{N+k-1}{N-1}\Bigr).
\]
Combining with Proposition~\ref{prop:harmonicdom} gives:

\begin{corollary}[A clean asymptotic lower bound]
For all $N\ge2$,
\[
 k(N)\ge (e-1)(N-1)+1.
\]
In particular $k(N)\ge (e-1)N - (e-2)$.
\end{corollary}
\begin{proof}
From Proposition~\ref{prop:harmonicdom}, $1\le\sum_{i=0}^{k-1}1/(N+i)\le \log((N+k-1)/(N-1))$. Hence $\log((N+k-1)/(N-1))\ge 1$, so $(N+k-1)/(N-1)\ge e$, i.e.
$k\ge e(N-1)-(N-1)+1$.
\end{proof}

This recovers (in a very elementary way) the right main term $(e-1)N$ up to an additive constant.

\subsubsection*{4.3 Small cases}

\begin{itemize}
\item $N=1$: $k(1)=1$ since $1=1/1$.
\item $N=2$: $k(2)=3$ via $1=\frac12+\frac13+\frac16$.
\item $N=3$: one has $k(3)=5$, because there is a $5$--term representation
\[
1=\frac13+\frac14+\frac15+\frac16+\frac1{20},
\]
and (by the explicit enumeration of all $4$--term decompositions in the solution to Problem~\#293 above) every $4$--term decomposition uses the denominator $2$, hence none can have all denominators $\ge 3$.
\end{itemize}

\subsection*{5. VERIFICATION}

\begin{itemize}
\item Proposition~\ref{prop:harmonicdom} is a direct termwise comparison and is airtight.
\item The integral comparison $\sum_{i=0}^{k-1}1/(N+i)\le\int_{N-1}^{N+k-1}dx/x$ follows from monotonicity of $1/x$ on each unit interval.
\item The small-case identities were checked by exact arithmetic.
\end{itemize}

\subsection*{6. FINAL (exactly ONE label and ONE sub-label)}

\noindent\textbf{LABEL: UNRESOLVED}\\
\textbf{SUBLABEL: (PARTIAL RESULTS)}

\begin{itemize}
\item \textbf{What I proved here:} an elementary, self-contained lower bound
\(k(N)\ge (e-1)(N-1)+1\), and exact values for $k(1),k(2),k(3)$.
\item \textbf{What remains:} determine whether $k(N)-(e-1)N\to\infty$.
\item \textbf{A concrete possible direction:} quantify how closely one can approximate $1$ from below by sums of distinct reciprocals in $[N,\infty)$ using only $\approx (e-1)N+O(1)$ terms, and show that exact equality forces a growing ``extra budget'' in the number of terms.
\end{itemize}

\subsection*{7. COMPLETION ESTIMATE}

I would rate this as \textbf{25\%} complete: the main-term lower bound is easy and done, but the central question about the divergence of the deviation has not been resolved.

%%%%%%%%%%%%%%%%%%%%%%%%%%%%%%%%%%%%%%%%%%%%%%%%%%%%%%%%%%%%%%%%%%%%%%%%%%%%%%%
