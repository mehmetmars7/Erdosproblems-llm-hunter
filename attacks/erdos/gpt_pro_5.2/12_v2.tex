\section{Erd\H{o}s Problem \#12 --- Round 2 (Continuation \& Gap-Closure Mode)}

\subsection*{1) ROUND-2 OBJECTIVE}

\textbf{Path (C): obstruction/correction (strategy-level), plus a provable strengthened construction.}
Round~1 verified the classical ``prime-square'' example
\[
A_1=\{p^2:\ p\equiv 3\!\!\pmod 4\}
\]
satisfies~(\*), but it only achieves size $\asymp \sqrt{N}/\log N$.
The most promising next step is to \emph{amplify the construction} while also
\emph{isolating why the most naive amplification strategies cannot reach}
\[
\liminf_{N\to\infty}\frac{|A\cap[1,N]|}{\sqrt N}>0
\qquad\text{(Q1).}
\]
In this round I do two things:
\begin{itemize}
\item I prove a Gaussian-$p$-adic valuation lemma (new tool) and use it to build
a broad family of infinite sets of \emph{squares} satisfying~(\*), strictly extending
Round~1's construction.
\item I prove an explicit \emph{barrier} for a whole class of ``private prime'' (distinguishing-prime) constructions:
this strategy can never reach $\gg\sqrt N$ elements up to~$N$.
\end{itemize}
Questions (Q1)--(Q3) remain open in full generality, but the results below
strictly advance the construction side and rule out a natural approach to~(Q1).

\subsection*{2) ROUND-1 FOUNDATION USED}

I rely on the following Round~1 results, without re-proving them:
\begin{itemize}
\item \textbf{Lemma 12.1 (Round~1).} For fixed $a\in A$, among $A_{>a}$ there are no
distinct $b,c$ with $b\equiv -c\pmod a$ (hence at most one class from each pair $\{r,-r\}$).
\item \textbf{Corollary 12.1a (Round~1).} If $A$ is infinite and satisfies~(\*), then $1\notin A$.
\item \textbf{Lemma 12.2 (Round~1).} $A_1=\{p^2:\ p\equiv 3\pmod 4\}$ satisfies~(\*).
\end{itemize}

\subsection*{3) NEW INSIGHT / TOOL (ROUND-2)}

The new tool is an \textbf{exact $p$-adic valuation formula} for sums of two squares at primes
$p\equiv 3\pmod 4$:
\[
v_p(x^2+y^2)=2\min\bigl(v_p(x),v_p(y)\bigr)\qquad (p\equiv 3\!\!\!\pmod 4).
\]
This upgrades the ``$-1$ is a nonresidue'' argument used in Round~1's Lemma~12.2 from
a mod-$p$ contradiction to a \emph{prime-power} statement, and it lets us
build large families $A=\{n^2:\ n\in S\}$ satisfying~(\*) provided $S$ is chosen to avoid
a simple divisibility pattern (an antichain/primitive condition).

\subsection*{4) ATTACK PLAN (ROUND-2)}

\textbf{Round~1 gap (constructive).} The only fully verified infinite example in Round~1 was $A_1=\{p^2:\ p\equiv 3\pmod 4\}$.
The gap is to \emph{systematically enlarge} such constructions and to understand what prevents
a straightforward enlargement from achieving $\gg \sqrt N$ elements.

\textbf{Claims to prove in Round~2.}
\begin{enumerate}
\item Prove the $p$-adic valuation lemma for primes $p\equiv 3\pmod 4$.
\item Use it to prove a general construction theorem:
if $S$ is a suitable divisibility-antichain of integers supported on primes $\equiv 3\pmod 4$,
then $A=\{n^2:\ n\in S\}$ satisfies~(\*).
\item Exhibit explicit infinite choices of $S$ (hence $A$) that strictly generalize $A_1$.
\item Prove an obstruction: any ``private prime'' strategy (each element has a new $p\equiv 3\pmod 4$
that never appears later) cannot yield $\gg \sqrt N$ elements up to~$N$.
\end{enumerate}

\subsection*{5) WORK (ROUND-2)}

\paragraph{\textbf{Corollary 12.1b (new, immediate from Round~1 Lemma~12.1).}}
If $A$ is infinite and satisfies~(\*), then $2\notin A$.
More generally, for any $a\in A$, there is \emph{at most one} $b\in A$ with $b>a$ and $a\mid b$.

\emph{Justification.}
In Lemma~12.1 take the residue class $0\bmod a$, which is self-opposite; then among $A_{>a}$ there
cannot be two distinct multiples of~$a$.
For $a=2$, both residue classes $0,1$ are self-opposite mod~$2$, so among $A_{>2}$ there can be at most
one even and at most one odd element; hence $A$ cannot be infinite.

\bigskip

\paragraph{\textbf{Lemma 12.3 (new: $p$-adic valuation of a sum of two squares).}}
Let $p$ be an odd prime with $p\equiv 3\pmod 4$.
For any integers $x,y$, not both $0$, one has
\[
v_p(x^2+y^2)=2\min\bigl(v_p(x),v_p(y)\bigr).
\]

\emph{Proof.}
Write $x=p^\alpha u$ and $y=p^\beta v$ with $\alpha,\beta\ge 0$ and $p\nmid uv$.
Without loss of generality assume $\alpha\le \beta$.
Then
\[
x^2+y^2=p^{2\alpha}\Bigl(u^2+p^{2(\beta-\alpha)}v^2\Bigr).
\]
If $\beta>\alpha$, then the bracket is congruent to $u^2\not\equiv 0\pmod p$, so $v_p(x^2+y^2)=2\alpha$.

If $\beta=\alpha$, then we must show $u^2+v^2\not\equiv 0\pmod p$.
Indeed, if $u^2\equiv -v^2\pmod p$ and $p\nmid v$, then $(uv^{-1})^2\equiv -1\pmod p$,
so $-1$ would be a quadratic residue modulo $p$, contradicting $p\equiv 3\pmod 4$.
Thus again the bracket is nonzero mod~$p$ and $v_p(x^2+y^2)=2\alpha=2\min(\alpha,\beta)$.
\qed

\paragraph{\textbf{Corollary 12.3a.}}
If $p\equiv 3\pmod 4$ is prime and $p^{2e}\mid (x^2+y^2)$, then $p^e\mid x$ and $p^e\mid y$.

\emph{Proof.}
Lemma~12.3 gives $2\min(v_p(x),v_p(y))=v_p(x^2+y^2)\ge 2e$.
Hence $\min(v_p(x),v_p(y))\ge e$.

\bigskip

\paragraph{\textbf{Theorem 12.4 (new: Gaussian-antichain square construction).}}
Let $S\subset\mathbb N_{\ge 2}$ be an infinite set such that:
\begin{enumerate}
\item[(i)] every prime divisor of every $s\in S$ satisfies $p\equiv 3\pmod 4$;
\item[(ii)] $S$ is \emph{primitive}: for distinct $s,t\in S$ one does not have $s\mid t$.
\end{enumerate}
Define
\[
A(S):=\{s^2:\ s\in S\}.
\]
Then $A(S)$ satisfies property~(\*): there do not exist distinct $a,b,c\in A(S)$ with $b,c>a$ and $a\mid (b+c)$.

\emph{Proof.}
Assume for contradiction that $a,b,c\in A(S)$ are distinct with $b,c>a$ and $a\mid (b+c)$.
Write $a=s^2$, $b=t^2$, $c=u^2$ with $s,t,u\in S$, and note $t,u>s$ since $t^2,u^2>s^2$.
The condition $a\mid(b+c)$ becomes
\[
s^2\mid (t^2+u^2).
\]
Let $p$ be any prime divisor of $s$; by (i) we have $p\equiv 3\pmod 4$.
If $v_p(s)=e\ge 1$, then $p^{2e}\mid s^2\mid (t^2+u^2)$, so by Corollary~12.3a we get $p^e\mid t$ and $p^e\mid u$.
Since this holds for every prime $p\mid s$, we conclude $s\mid t$ and $s\mid u$.

But $s\mid t$ with $s\neq t$ contradicts primitivity~(ii) (and likewise for $u$).
Hence no such triple exists.
\qed

\paragraph{\textbf{Corollary 12.4b (explicit infinite families extending Round~1).}}
Fix an integer $k\ge 1$, and let
\[
S_k:=\{p_1p_2\cdots p_k:\ p_i\ \text{distinct primes with }p_i\equiv 3\pmod 4\}.
\]
Then $S_k$ is primitive (indeed an antichain under divisibility), so
\[
A_k:=A(S_k)=\{(p_1\cdots p_k)^2:\ p_i\equiv 3\pmod 4\ \text{distinct}\}
\]
satisfies~(\*).
For $k=1$ this recovers the Round~1 example $A_1=\{p^2:\ p\equiv 3\pmod 4\}$.

\emph{Proof.}
If $s,t\in S_k$ and $s\mid t$, then the (squarefree) prime support of $s$ is a subset of that of $t$.
Since both have exactly $k$ distinct primes, this forces $s=t$.
Thus $S_k$ is primitive, so Theorem~12.4 applies.

\bigskip

\paragraph{\textbf{Proposition 12.5 (size of $A_k$; analytic input).}}
Let $k\ge 1$ be fixed. Then, as $N\to\infty$,
\[
|A_k\cap[1,N]|\;=\;|S_k\cap[1,\sqrt N]|
\;\asymp_k\;
\frac{\sqrt N}{\log N}\,(\log\log N)^{k-1}.
\]
In particular, for every fixed $k$,
\[
\liminf_{N\to\infty}
\frac{|A_k\cap[1,N]|}{\sqrt N}\,\frac{\log N}{(\log\log N)^{k-1}}
\;>\;0.
\]

\emph{Justification.}
This is a standard Landau--Sathe--Selberg type asymptotic for squarefree $k$-almost primes,
together with the fact that primes $\equiv 3\pmod 4$ have relative density $1/2$ among the primes.
(Only the order of magnitude $\asymp_k$ is used here.)

\bigskip

\paragraph{\textbf{Proposition 12.6 (new: ``private prime'' obstruction for square-based constructions).}}
Consider square-based constructions $A=\{s^2:\ s\in S\}$ where each $s\in S$ has a \emph{private} prime
$p_s\equiv 3\pmod 4$ dividing $s$ such that $p_s\nmid t$ for all $t\in S$ with $t>s$.
Then for every $N$,
\[
|A\cap[1,N]|=|S\cap[1,\sqrt N]|\ \le\ \pi_{3\bmod 4}(\sqrt N),
\]
where $\pi_{3\bmod 4}(X)$ counts primes $p\le X$ with $p\equiv 3\pmod 4$.
In particular, this strategy can never achieve $|A\cap[1,N]|\gg \sqrt N$.

\emph{Proof.}
Define a map $\varphi:S\cap[1,\sqrt N]\to\{p\le \sqrt N:\ p\equiv 3\pmod 4\}$
by $\varphi(s)=p_s$.
If $\varphi(s)=\varphi(t)=p$ with $s<t$, then $p=p_s$ divides $t$, contradicting that $p_s$ is private to~$s$.
Hence $\varphi$ is injective, so $|S\cap[1,\sqrt N]|\le \pi_{3\bmod 4}(\sqrt N)$.
\qed

\subsection*{6) ADVERSARIAL VERIFICATION}

\textbf{(i) Prime-power subtlety in Lemma 12.3.}
The key step is excluding $u^2+v^2\equiv 0\pmod p$ when $p\equiv 3\pmod 4$ and $p\nmid uv$;
this is exactly the ``$-1$ nonresidue'' property used in Round~1, now localized to the unit part.
No step uses $p=2$, and indeed $p=2$ is excluded.

\textbf{(ii) Dependence on the ``square'' structure.}
Theorem~12.4 crucially uses that $b+c=t^2+u^2$ is a sum of \emph{two squares}.
Without $A$ being a set of squares (or more generally Gaussian norms), Corollary~12.3a is unavailable.
So the theorem is not claiming a general resolution of (Q1)--(Q3), only a principled amplification
of the prime-square example.

\textbf{(iii) Quantifiers in (\*).}
In Theorem~12.4, if $a=s^2\in A(S)$ and $b=t^2,c=u^2\in A(S)$ satisfy $b,c>a$, then indeed $t,u>s$.
Thus $s\mid t$ contradicts primitivity, so the forbidden configuration cannot arise.
Distinctness of $a,b,c$ is used only at the final step (to rule out the trivial equality case).

\textbf{(iv) Obstruction proposition sanity.}
The injection argument in Proposition~12.6 is robust: it uses only the definition of ``private prime'',
not property~(\*). Hence it truly rules out a wide class of potential constructions for (Q1).

\subsection*{7) FINAL}

\textbf{UNRESOLVED (BUT STRICTLY ADVANCED).}

\begin{itemize}
\item The original questions (Q1)--(Q3) remain open in full generality.
\item Round~2 provides a new structural tool (Lemma~12.3) and a broad construction principle (Theorem~12.4)
that strictly extends Round~1's verified example.
\item Explicit families $A_k$ are produced for every $k\ge 1$ (Corollary~12.4b), giving
\[
|A_k\cap[1,N]|\asymp_k \frac{\sqrt N}{\log N}(\log\log N)^{k-1},
\]
which is asymptotically larger than the $k=1$ (prime-square) construction for every fixed $k\ge 2$.
\item A key obstruction is isolated: any square-based strategy that assigns each element a new
``private'' $p\equiv 3\pmod 4$ cannot reach $\gg\sqrt N$ (Proposition~12.6).
Thus a positive answer to (Q1), if true, must involve substantial prime re-use without a private-prime certificate.
\end{itemize}

\subsection*{8) COMPLETION ESTIMATE (MANDATORY)}

\noindent\textbf{COMPLETION: 35\%}

\subsection*{9) REFERENCES}

\begin{itemize}
\item P.~Erd\H{o}s and A.~S\'ark\"ozy, original formulation and density-zero result (see bibliographic pointers collected in the Erd\H{o}s Problems database entry for Problem~\#12).
\item E.~M.~Wright, \emph{A Simple Proof of a Theorem of Landau}, Proc. Camb. Phil. Soc. \textbf{50} (1954), 304--309. (For the classical asymptotic of $k$-almost primes used in Proposition~12.5.)
\item Standard algebraic number theory fact: for $p\equiv 3\pmod 4$, $p$ is inert in $\mathbb Z[i]$, giving Lemma~12.3 (proved directly above).
\end{itemize}
