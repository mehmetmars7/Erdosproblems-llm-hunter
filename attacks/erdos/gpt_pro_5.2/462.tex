% Erdos Problem #462
% Solution

\noindent\textbf{1) FORMAL RESTATEMENT}

Let $p(n)$ denote the least prime factor of $n$ (so if $n$ is prime then $p(n)=n$).
The extracted problem states that there exists a constant $c>0$ such that
\[S(x):=\sum_{\substack{n<x\\ n\ \textrm{not prime}}}\frac{p(n)}{n}\ \sim\ c\,\frac{x^{1/2}}{(\log x)^2}.
\]
It then asks whether there exists a constant $C>0$ such that
\[T(x):=\sum_{x\le n\le x+C x^{1/2}(\log x)^2}\frac{p(n)}{n}\ \gg\ 1\]
for all sufficiently large $x$.

\textbf{Ambiguity.} In the second sum, primes $n$ are \emph{not} explicitly excluded. If primes are included, then each prime contributes $p(n)/n=1$ and the question becomes closely related to guaranteeing primes in intervals of length $\asymp x^{1/2}(\log x)^2$ (a Legendre-type prime gap question).  The reference is ambiguous about whether primes should be excluded.  In what follows I focus on the \emph{corrected} version consistent with the first sum:
\[T_{\mathrm{comp}}(x):=\sum_{\substack{x\le n\le x+C x^{1/2}(\log x)^2\\ n\ \textrm{not prime}}}\frac{p(n)}{n},\]
and ask whether $T_{\mathrm{comp}}(x)\gg 1$ for all large $x$.

\noindent\textbf{2) QUICK LITERATURE/CONTEXT CHECK}

The Erd\H{o}s problems page for \#462 explicitly notes this ambiguity and remarks that if primes are included the question is essentially a weaker form of Legendre's conjecture, while if primes are excluded it is related to the distribution of semiprimes $pq$ with $p,q\approx x^{1/2}\log^{O(1)}x$ in short intervals. I do not use any external results beyond the asymptotic stated in the problem.

\noindent\textbf{3) ATTACK PLAN}

\begin{itemize}
\item Use the decomposition by least prime factor: write each composite $n$ uniquely as $n=p\,m$ with $p=p(n)$ and deduce $p(n)/n=1/m$.
\item Derive consequences of the global asymptotic for averages of the short-interval sums.
\item Compute the corrected short-interval sum numerically for moderately large $x$ to sanity-check.
\end{itemize}

\noindent\textbf{4) WORK}

\textbf{Fast reality check (computations, corrected version excluding primes).}

For several values of $x\le 1.5\times 10^6$ I computed
\[\sum_{\substack{x\le n\le x+ C\sqrt{x}(\log x)^2\\ n\ \mathrm{composite}}}\frac{p(n)}{n}
\]
with $C=1$.  The observed values were about $3.73$--$4.27$ for the sample points $x\in\{10^3,5\cdot 10^3,10^4,5\cdot 10^4,10^5,5\cdot 10^5,10^6,1.5\cdot 10^6\}$.

\medskip
\noindent\textbf{Lemma 462.1 (trivial pointwise bound).}
If $n$ is composite then $p(n)\le \sqrt{n}$, and hence
\[0<\frac{p(n)}{n}\le \frac1{\sqrt{n}}.
\]

\textit{Proof.}
If $n$ is composite, write $n=ab$ with $1<a\le b<n$. Then $a\le\sqrt{n}$.
The least prime factor $p(n)$ is at most $a$, so $p(n)\le\sqrt{n}$. Dividing by $n$ gives $p(n)/n\le 1/\sqrt{n}$. \hfill$\square$

\medskip
\noindent\textbf{Lemma 462.2 (decomposition by least prime factor).}
Define $a_n:=\frac{p(n)}{n}$ if $n$ is composite and $a_n:=0$ if $n$ is prime or $n=1$.
Then for every $x\ge 2$,
\[\sum_{n<x} a_n = \sum_{\text{primes }p}\ \sum_{\substack{m<x/p\\ m\ge 2\\ p(m)\ge p}} \frac1m.
\]
Here $p(m)\ge p$ means that $m$ has no prime factor $<p$.

\textit{Proof.}
Every composite integer $n\ge 4$ has a unique least prime factor $p=p(n)$, and can be written uniquely as $n=p m$ with $m\ge 2$.
Moreover, $p(n)=p$ if and only if (i) $p$ is prime, (ii) $n=pm$, and (iii) $m$ has no prime factor smaller than $p$, i.e. $p(m)\ge p$.
For such $n=pm$, one has
\[a_n = \frac{p(n)}{n} = \frac{p}{pm} = \frac1m.
\]
Summing $a_n$ over all composite $n<x$ and grouping terms by $p=p(n)$ yields exactly the double sum on the right-hand side. \hfill$\square$

\medskip
\noindent\textbf{Lemma 462.3 (average short-interval mass from the global asymptotic).}
Assume the asymptotic stated in the problem:
\[S(x):=\sum_{n<x} a_n\sim c\frac{x^{1/2}}{(\log x)^2}.\]
Fix $C>0$ and define $L(x):=\lfloor C x^{1/2}(\log x)^2\rfloor$.  Consider the corrected short-interval sums
\[T_{\mathrm{comp}}(x):=\sum_{x\le n\le x+L(x)} a_n\quad (a_n\ge 0).\]
Then as $X\to\infty$,
\[\frac1{X}\sum_{x=X}^{2X} T_{\mathrm{comp}}(x)\ \ge\ cC(\sqrt{2}-1)+o(1).\]
In particular, there exist arbitrarily large $x$ for which $T_{\mathrm{comp}}(x)\gg_C 1$.

\textit{Proof.}
Let
\[L_{\min}:=\lfloor C X^{1/2}(\log X)^2\rfloor.
\]
For all integers $x\in[X,2X]$, monotonicity of the function $x\mapsto x^{1/2}(\log x)^2$ implies $L(x)\ge L_{\min}$ for all sufficiently large $X$. Since $a_n\ge 0$, we have
\[T_{\mathrm{comp}}(x)=\sum_{x\le n\le x+L(x)} a_n\ \ge\ \sum_{x\le n\le x+L_{\min}} a_n.\]
Summing this inequality over $x=X,X+1,\dots,2X$ and swapping the order of summation gives
\[\sum_{x=X}^{2X} T_{\mathrm{comp}}(x)\ \ge\ \sum_{x=X}^{2X}\ \sum_{x\le n\le x+L_{\min}} a_n
=\sum_{n} a_n\,\#\{x\in[X,2X]: x\le n\le x+L_{\min}\}.
\]
Now fix any integer $n$ with $X+L_{\min}\le n\le 2X$. For each integer $j\in\{0,1,\dots,L_{\min}-1\}$ let $x=n-j$. Then $x\in[X,2X]$ and $x+L_{\min}=n-j+L_{\min}\ge n$, so $x\le n\le x+L_{\min}$. Hence
\[\#\{x\in[X,2X]: x\le n\le x+L_{\min}\}\ \ge\ L_{\min}.
\]
Therefore
\[\sum_{x=X}^{2X} T_{\mathrm{comp}}(x)\ \ge\ L_{\min}\sum_{n=X+L_{\min}}^{2X} a_n
=L_{\min}\bigl(S(2X+1)-S(X+L_{\min})\bigr).
\]
Because $L_{\min}=o(X)$, the asymptotic for $S$ implies
\[S(2X+1)-S(X+L_{\min})\ =\ c\frac{(2X)^{1/2}}{(\log X)^2}-c\frac{X^{1/2}}{(\log X)^2}+o\!\left(\frac{X^{1/2}}{(\log X)^2}\right)
= c(\sqrt{2}-1)\frac{X^{1/2}}{(\log X)^2}+o\!\left(\frac{X^{1/2}}{(\log X)^2}\right).
\]
Multiplying by $L_{\min}=C X^{1/2}(\log X)^2+o\bigl(X^{1/2}(\log X)^2\bigr)$ yields
\[\sum_{x=X}^{2X} T_{\mathrm{comp}}(x)\ \ge\ cC(\sqrt{2}-1)X+o(X).
\]
Dividing by $X$ proves the claimed average lower bound. Since an average over $x\in[X,2X]$ is bounded below by a positive constant for large $X$, at least one $x\in[X,2X]$ must satisfy $T_{\mathrm{comp}}(x)\gg_C 1$; letting $X\to\infty$ gives arbitrarily large such $x$. \hfill$\square$

\medskip
\noindent\textbf{5) VERIFICATION}

\begin{itemize}
\item Checked that the ambiguity about primes is real (it is also noted in the discussion thread).
\item Lemma~462.2: verified uniqueness of decomposition $n=p\,m$ with $p$ least prime factor.
\item Lemma~462.3: verified the lower bound on the counting multiplicity $\#\{x: x\le n\le x+L(x)\}$ uses only monotonicity of $L(x)$ on $[X,2X]$ for large $X$ and that $L(x)\gg \sqrt{X}(\log X)^2$.
\item Computations: used a least-prime-factor sieve up to $2\cdot 10^6$ and prefix sums of $a_n$ (excluding primes).
\end{itemize}

\noindent\textbf{6) FINAL}

\textbf{UNRESOLVED}

(i) \emph{Strongest proved partial result:} After excluding primes (the corrected reading), we have the exact decomposition
\[\sum_{n<x,\,n\ \mathrm{composite}}\frac{p(n)}{n}=\sum_{p\ \mathrm{prime}}\ \sum_{\substack{m<x/p\\ m\ge 2,\ p(m)\ge p}} \frac1m\]
(Lemma~462.2), and assuming the stated global asymptotic one can prove that the short-interval sum has positive \emph{average} size $\asymp cC$ over $x\in[X,2X]$ (Lemma~462.3), hence is $\gg 1$ for infinitely many $x$. Computations for $x\le 1.5\cdot 10^6$ and $C=1$ show values around $4$.

(ii) \emph{First gap (crisp):} Upgrade the average lower bound to a pointwise one: prove there exists $C>0$ and $c_0>0$ such that for \emph{every} sufficiently large $x$,
\[\sum_{\substack{x\le n\le x+C\sqrt{x}(\log x)^2\\ n\ \mathrm{composite}}}\frac{p(n)}{n}\ge c_0.
\]

(iii) \emph{Top 3 next moves:}
\begin{enumerate}
\item Show that intervals of length $\asymp \sqrt{x}(\log x)^2$ always contain enough semiprimes $pq$ with $p,q\approx \sqrt{x}\,\log^{O(1)}x$, since such terms contribute $\approx 1/q$.
\item Prove a lower bound for the count of integers $n$ in $[x,x+\sqrt{x}(\log x)^2]$ whose least prime factor lies in a controlled range (e.g. $p(n)\approx \sqrt{x}\,\log^A x$).
\item Numerically search for $x$ where the corrected interval sum is unusually small (relative to its average) to guess the obstruction structure.
\end{enumerate}

(iv) \emph{Minimal counterexample structure:} A minimal counterexample (under the corrected reading) would be a large $x$ for which the interval $[x,x+C\sqrt{x}(\log x)^2]$ contains very few composites with least prime factor $\gg \sqrt{x}/\log^{O(1)}x$; heuristically this would mean the interval avoids most semiprimes $pq$ with $p,q\approx \sqrt{x}\,\log^{O(1)}x$ and is dominated by numbers with very small least prime factor (whose contributions $p(n)/n$ are tiny).
