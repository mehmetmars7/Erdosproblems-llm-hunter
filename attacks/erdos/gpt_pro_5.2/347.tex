%Erdos problem #347
\subsection*{Erd\H{o}s problem \#347}

\noindent\textbf{1) FORMAL RESTATEMENT.}
Let $A=\{a_1\le a_2\le\cdots\}$ be a nondecreasing sequence of positive integers.
For any subsequence $A'$ (obtained by deleting terms), define
\[
P(A'):=\left\{\sum_{n\in B} n:\ B\subseteq A'\ \text{finite}\right\},
\]
the set of subset sums of \emph{distinct indices} of $A'$.
The (natural) density of a set $S\subseteq\mathbb{N}$ is
\[
 d(S)=\lim_{N\to\infty}\frac{|S\cap[1,N]|}{N}
\]
when it exists.

The question asks whether there exists a sequence $A$ such that
\begin{enumerate}
\item $\displaystyle \lim_{n\to\infty}\frac{a_{n+1}}{a_n}=2$, and
\item for every cofinite subsequence $A'$ (i.e. deleting finitely many terms), one has $d(P(A'))=1$.
\end{enumerate}

\medskip
\noindent\textbf{2) QUICK LITERATURE/CONTEXT CHECK.}
No external results are assumed.

\medskip
\noindent\textbf{3) ATTACK PLAN.}
\begin{itemize}
\item Prove necessary conditions for $d(P(A'))=1$ to hold for all cofinite subsequences (in particular, avoid modular obstructions in every tail).
\item Do small computational experiments with natural ``ratio $\to 2$'' candidates to see how dense their subset sums are, and how sensitive density is to deleting one term.
\end{itemize}

\medskip
\noindent\textbf{4) WORK.}

\medskip
\noindent\textbf{Lemma 347.1 (modular obstruction for density $1$ in all tails).}
Suppose there exists an integer $m\ge 2$ such that all but finitely many terms of $A$ are divisible by $m$.
Then there exists a cofinite subsequence $A'$ with $d(P(A'))\le 1/m<1$.
In particular, a necessary condition for the problem's requirement is:
\begin{quote}
For every $m\ge 2$, infinitely many terms of $A$ are not divisible by $m$.
\end{quote}

\noindent\emph{Proof.}
Delete the finitely many terms not divisible by $m$ to obtain a cofinite subsequence $A'$ all of whose terms are multiples of $m$.
Then every subset sum of $A'$ is a multiple of $m$, so $P(A')\subseteq m\mathbb{N}$.
Therefore $|P(A')\cap[1,N]|\le \lfloor N/m\rfloor$ for all $N$, hence $d(P(A'))\le 1/m$.
\hfill$\square$

\medskip
\noindent\textbf{Lemma 347.2 (a counting necessary condition at finite stage).}
Let $A_N:=\{a_1,\dots,a_N\}$ and let $S_N:=\sum_{i=1}^N a_i$.
Then $P(A_N)\subseteq[0,S_N]$ and $|P(A_N)|\le 2^N$.
Consequently, the density of $P(A_N)$ inside $[0,S_N]$ is at most
\[
\frac{|P(A_N)|}{S_N+1}\le \frac{2^N}{S_N+1}.
\]
In particular, if for some cofinite tail one expects $d(P(A'))=1$, then for large $N$ in that tail one must have $S_N\lesssim 2^N$ at least along a subsequence.

\noindent\emph{Proof.}
Every subset sum of $A_N$ is between $0$ and $S_N$.
There are $2^N$ choices of subsets of $\{1,\dots,N\}$, so $|P(A_N)|\le 2^N$. Dividing by $S_N+1$ gives the stated bound.
\hfill$\square$

\medskip
\noindent\textbf{FAST REALITY CHECK (computational experiments with ratio $\to 2$ candidates).}
I tested the candidate sequence $a_n=2^n+1$ (which satisfies $a_{n+1}/a_n\to 2$).
For $N=20$, the set of subset sums of the first $20$ terms has density about $0.458$ inside its natural range $[0,S_N]$.
If one deletes a single term (e.g. the $10$th), the density drops to about $0.229$ at $N=20$.
Thus this natural ``$2^n$ with perturbation'' candidate is far from satisfying $d(P(A'))=1$ robustly under deletions.

I also tested the mixed sequence $1,2,3,6,12,24,\dots$ (ratio tends to $2$ after the initial segment): its full subset sums fill every integer up to $S_N$ for finite $N$, but deleting $1$ already drops the finite-stage density to $2/3$, illustrating how sensitive density can be to removing a single small element.

\medskip
\noindent\textbf{5) VERIFICATION.}
Lemma~347.1 is a direct modular density bound.
Lemma~347.2 is a direct counting argument.
The computations are exact for the tested finite prefixes.

\medskip
\noindent\textbf{6) FINAL.}

\noindent\textbf{UNRESOLVED}

\smallskip
\noindent (i) \textbf{Strongest fully proved partial result obtained here.}
We proved necessary conditions for the requested robustness: in particular, every modulus $m\ge 2$ must divide only finitely many terms (Lemma~347.1), and finite-stage density forces $\sum_{i\le N}a_i$ not to be too large compared to $2^N$ (Lemma~347.2). Small experiments suggest natural $a_n\approx 2^n$ candidates do not have subset-sum density near $1$, and deletion of a single term can cut density substantially.

\smallskip
\noindent (ii) \textbf{Exact first gap.}
We do not have either (a) an explicit construction of such a sequence $A$, or (b) a proof that no such sequence can exist. The first missing theoretical step is understanding how $a_{n+1}/a_n\to 2$ constrains the asymptotic coverage (density) of subset sums in every tail.

\smallskip
\noindent (iii) \textbf{Top 3 next moves (concrete targets).}
\begin{enumerate}
\item Attempt an impossibility proof by showing that for sequences with ratio $\to 2$, deleting one suitably chosen term forces a positive-density hole (e.g. via residue-class or carry constraints).
\item Attempt a constructive approach by building redundancy across scales (so that no single term is ``essential''), while keeping consecutive ratios tending to $2$.
\item Strengthen Lemma~347.2 into an ``if and only if'' criterion relating tail completeness/density to inequalities of the form $a_{n+1}\le 1+\sum_{i\le n} a_i$ along the tail.
\end{enumerate}

\smallskip
\noindent (iv) \textbf{Minimal counterexample structure.}
If no such sequence exists, a minimal obstruction would likely arise from a cofinite subsequence $A'$ for which $P(A')$ omits an entire residue class modulo some $m$ (giving density $\le 1-1/m$), or from a persistent gap phenomenon where some interval lengths grow proportionally to $N$.
Conversely, any successful construction must defeat all modular obstructions simultaneously (Lemma~347.1) and ensure that tail subset sums cover all but $o(N)$ integers up to $N$.

