% Erdos Problem #796

\subsection*{FORMAL RESTATEMENT}
Fix $k\ge 2$. Let $g_k(n)$ be the largest possible size of a set $A\subset\{1,2,\dots,n\}$ such that every integer $m$ has fewer than $k$ representations
\[
 m=a_1a_2
\]
with $a_1<a_2$ and $a_1,a_2\in A$.
The problem asks for the second order term of $g_k(n)$, in particular for $k=3$.

\subsection*{QUICK LITERATURE/CONTEXT CHECK}
From the problem statement: Erd\H{o}s determined the main term: if $r$ is defined by $2^{r-1}<k\le 2^r$, then
\[
 g_k(n) \sim \frac{(\log\log n)^{r-1}}{(r-1)!\,\log n}\,n.
\]
It is asked whether
\[g_k(n)=\frac{(\log\log n)^{r-1}}{(r-1)!\,\log n}\,n + O\Bigl(\frac{n}{\log n}\Bigr)
\]
and, specifically for $k=3$ (so $r=2$), whether
$g_3(n)=\frac{\log\log n}{\log n}n + C\frac{n}{\log n}+o\bigl(\frac{n}{\log n}\bigr)$
for a constant $C$.

Below I give an explicit $k=3$ construction with a complete verification of the ``$<3$ representations'' condition, and I compute exact $g_3(n)$ for $n\le 20$.

\subsection*{ATTACK PLAN}
1) Provide a concrete family of sets $A$ where each element has a distinguished large prime factor $>\sqrt n$; show this forces at most two product representations.

2) Express the size of this construction in terms of prime-counting functions (no asymptotics claimed beyond what is in the statement).

3) Reality check: brute-force $g_3(n)$ for small $n$.

\subsection*{WORK}
\textbf{Lemma 796.1 (A $k=3$-valid construction using a unique large prime factor).}
Fix $n\ge 2$. Let
\[
Q:=\{q\le n: q\text{ prime and }q>\sqrt n\}.
\]
Define
\[
A:=\{pq: q\in Q,\ p\text{ prime},\ p<q,\ pq\le n\} \subset\{1,\dots,n\}.
\]
Then every integer $m$ has at most $2$ representations $m=a_1a_2$ with $a_1<a_2$ and $a_1,a_2\in A$. In particular $A$ is feasible for $g_3(n)$.

\emph{Proof.}
Each $a\in A$ has a prime factor $q>\sqrt n$ and, because $a\le n$, this factor is \emph{unique}: if $q'\ne q$ were another prime $>\sqrt n$ dividing $a$, then $qq'>n$ would divide $a$, impossible.
So every $a\in A$ can be written uniquely as $a=p(a)\,q(a)$ where $q(a)\in Q$ is its unique prime factor $>\sqrt n$ and $p(a)<\sqrt n$ is the remaining prime factor.

Now fix an integer $m$ and suppose we have a representation
\[
 m=a_1a_2
\]
with distinct $a_1,a_2\in A$.
Write $a_i=p_i q_i$ with $q_i>\sqrt n$ prime and $p_i$ prime $<\sqrt n$.
Then
\[
 m = (p_1q_1)(p_2q_2) = (p_1p_2)(q_1q_2).
\]
In the prime factorization of $m$, the primes $>\sqrt n$ appearing are exactly $q_1$ and $q_2$ (counted with multiplicity; if $q_1=q_2$ then it appears twice).
Therefore, in any other representation $m=b_1b_2$ with $b_j\in A$, the multiset $\{q(b_1),q(b_2)\}$ must equal $\{q_1,q_2\}$.
There are only two assignments of these large primes to the two factors: either
\[(q(b_1),q(b_2))=(q_1,q_2)\quad\text{or}\quad(q(b_1),q(b_2))=(q_2,q_1).
\]
Once this assignment is fixed, the remaining prime factors of $m$ below $\sqrt n$ must match as well. Since $p_1,p_2$ are primes, the only ways to split $p_1p_2$ into a product of two primes are $(p_1,p_2)$ or $(p_2,p_1)$.
Thus, up to swapping the two factors, there is at most one representation when $q_1=q_2$ or $p_1=p_2$, and at most two representations when $p_1\ne p_2$ and $q_1\ne q_2$ (corresponding exactly to swapping the large primes).
Counting only representations with $a_1<a_2$ removes the double-counting from swapping the two factors.
Hence the number of such representations is at most $2$ for every $m$.
\hfill$\square$

\medskip
\textbf{Lemma 796.2 (Size formula for the construction).}
With $A$ as in Lemma 796.1,
\[
|A| = \sum_{\substack{q\in Q}} \pi\!\left(\left\lfloor \frac{n}{q}\right\rfloor\right),
\]
where $\pi(x)$ is the number of primes $\le x$.

\emph{Proof.}
Fix $q\in Q$. The elements of $A$ with distinguished large prime factor $q$ are exactly $pq$ with $p$ prime and $p\le n/q$. (The condition $p<q$ is automatic because $q>\sqrt n$ implies $n/q<\sqrt n<q$.)
There are exactly $\pi(\lfloor n/q\rfloor)$ such primes $p$.
Summing over all $q\in Q$ counts each element of $A$ exactly once by uniqueness of the factor $q$.
\hfill$\square$

\medskip
\textbf{Fast reality check (exact computation of $g_3(n)$ for $n\le 20$).}
I brute-forced $g_3(n)$ exactly by checking all subsets $A\subset\{1,\dots,n\}$ for $n\le 20$.
The exact values found are:
\begin{center}
\begin{tabular}{c|cccccccccccccccccccc}
$n$&1&2&3&4&5&6&7&8&9&10&11&12&13&14&15&16&17&18&19&20\\\hline
$g_3(n)$&1&2&3&4&5&6&7&8&9&10&11&11&12&13&14&15&16&16&17&17
\end{tabular}
\end{center}
(For example, for $n=12$ one optimal set has size $11$; including all of $\{1,\dots,12\}$ fails because $12=1\cdot 12=2\cdot 6=3\cdot 4$ gives three representations.)

\subsection*{VERIFICATION}
-- Lemma 796.1 hinges on the uniqueness of a prime factor $>\sqrt n$ for elements $\le n$.
This uniqueness makes the ``large prime'' factors in any product representation identifiable from $m$ itself.

-- Lemma 796.2 is an exact counting identity (no asymptotics used).

-- The brute-force values for $g_3(n)$ confirm the basic obstruction already at $n=12$.

\subsection*{FINAL}
UNRESOLVED

(i) \textbf{Strongest proved partial result here:} The explicit set $A$ from Lemma 796.1 is feasible for $g_3(n)$ and has size
\[|A|=\sum_{\sqrt n<q\le n\atop q\ \mathrm{prime}} \pi(\lfloor n/q\rfloor).
\]
Moreover, exact values of $g_3(n)$ for $n\le 20$ are listed above.

(ii) \textbf{First gap (crisp):} Prove (or disprove) that for $k=3$ there exists a constant $C$ such that
\[
 g_3(n)=\frac{\log\log n}{\log n}n + C\frac{n}{\log n}+o\Bigl(\frac{n}{\log n}\Bigr).
\]
Even establishing an $O(n/\log n)$ error term without identifying $C$ would be a major step.

(iii) \textbf{Top 3 next moves:}
\begin{enumerate}
\item Analyze the construction in Lemma 796.1 asymptotically by estimating the sum $\sum_{q>\sqrt n} \pi(n/q)$ more precisely (and compare it to $\frac{\log\log n}{\log n}n$) to see whether it captures the conjectured second-order constant.
\item For computation: use integer programming / backtracking to compute $g_3(n)$ for larger $n$ (say up to $n=60$ or $100$) and fit the data to $\frac{\log\log n}{\log n}n + C\frac{n}{\log n}$ to guess $C$.
\item Structural: attempt a stability theorem: show that any near-extremal $A$ must consist mainly of integers with a distinguished ``large prime'' factor (or, more generally, a product of primes from disjoint blocks), reflecting the $2^r$ threshold in the main-term theorem.
\end{enumerate}

(iv) \textbf{Minimal counterexample structure to watch for:} If the $O(n/\log n)$ conjecture is false for $k=3$, a minimal obstruction would be a family of sets $A$ of near-maximal size whose elements have many small prime factors arranged so that products admit unusually many near-collisions without producing three exact coincidences. Such a configuration would have to avoid the ``distinguished large prime'' structure that makes representation counts rigid.

