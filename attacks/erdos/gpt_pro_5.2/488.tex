% Erdos Problem #488
% URL: https://www.erdosproblems.com/488

\subsection*{FORMAL RESTATEMENT}
Let $A\subseteq\mathbb{N}$ be a \emph{finite} nonempty set, and define
\[
B := \{ n\ge 1 : \exists a\in A\ \text{such that } a\mid n\},
\]
the set of positive integers divisible by at least one element of $A$.
For $x\ge 1$ define the counting function
\[F(x):=|B\cap[1,x]|\quad\text{and the density-initial-segment }\quad d(x):=\frac{F(x)}{x}.
\]
The problem asks whether the following inequality always holds:
\[
\forall m>n\ge \max(A):\qquad \frac{F(m)}{m} < 2\frac{F(n)}{n}.
\]
It also notes that the constant $2$ would be best possible, since for $A=\{a\}$, $n=2a-1$, $m=2a$ makes the ratio approach $2$ from below.

\subsection*{QUICK LITERATURE/CONTEXT CHECK}
The statement notes that an alternate version (with $a\mid n$ replaced by $a\nmid n$) likely contains a typo and has known counterexamples; the version stated here is the ``multiples'' version.
Per the project integrity rule, I do not use results not explicitly present in the problem text.

\subsection*{ATTACK PLAN}
\textbf{Proof track:}
\begin{itemize}
\item Use the structure of $B$ as a finite union of arithmetic progressions to obtain periodicity modulo $L=\mathrm{lcm}(A)$ and extract asymptotics $d(x)=\delta+O(1/x)$. This proves the inequality for sufficiently large $n$.
\item Analyze small $n$ relative to $L$ by expressing $F(x)$ exactly in terms of $q=\lfloor x/L\rfloor$ and remainder $r=x\bmod L$.
\end{itemize}
\textbf{Disproof track:}
\begin{itemize}
\item Search computationally for a counterexample $(A,n,m)$ with small parameters.
\end{itemize}
I obtain (a) full proofs in some special regimes and (b) computational evidence for small parameters, but no complete proof or counterexample in general.

\subsection*{WORK}
\paragraph{Lemma 1 (removing redundant divisors).}
If $A$ contains two elements $a,b$ with $a\mid b$, then replacing $A$ by $A\setminus\{b\}$ does not change $B$.
In particular, one may assume without loss of generality that $A$ is \emph{primitive}: no element of $A$ divides another.

\paragraph{Proof.}
If $a\mid b$, then every multiple of $b$ is automatically a multiple of $a$. Thus
\[\{n: b\mid n\}\subseteq\{n: a\mid n\}.
\]
Therefore the union over $A$ of sets of multiples is unchanged if we remove $b$:
\[
\bigcup_{t\in A}\{n: t\mid n\}=\Bigl(\bigcup_{t\in A\setminus\{b\}}\{n: t\mid n\}\Bigr)\cup\{n: b\mid n\}=\bigcup_{t\in A\setminus\{b\}}\{n: t\mid n\}.
\]
So $B$ is unchanged.
\qed

\paragraph{Lemma 2 (single divisor case; constant $2$ is sharp).}
Let $A=\{a\}$ with $a\ge 1$. Then for every $m>n\ge a$,
\[\frac{\lfloor m/a\rfloor}{m} < 2\frac{\lfloor n/a\rfloor}{n}.
\]
Moreover, taking $n=2a-1$ and $m=2a$ gives
\[\frac{\lfloor m/a\rfloor/m}{\lfloor n/a\rfloor/n}=\frac{(2/a)}{(1/(2a-1))}=2-\frac{1}{a},\]
showing the constant $2$ cannot be improved.

\paragraph{Proof.}
Write $m=qa+r$ and $n=pa+s$ with integers $q,p\ge 1$ and remainders $0\le r,s\le a-1$.
Then $\lfloor m/a\rfloor=q$ and $\lfloor n/a\rfloor=p$.
We must show
\[\frac{q}{qa+r} < 2\frac{p}{pa+s}.
\]
Cross-multiplying (all quantities positive) gives the equivalent inequality
\[(pa+s)q < 2p(qa+r).
\]
Expand and cancel the common term $pqa$:
\[s q < pqa + 2pr.
\]
Since $s\le a-1$, we have $sq\le (a-1)q < aq$.
Since $p\ge 1$, $pqa \ge aq$.
Thus $sq < pqa\le pqa+2pr$, proving the inequality.
(Strictness holds because $sq<aq\le pqa$.)

For sharpness, if $n=2a-1$ then $\lfloor n/a\rfloor=1$, while for $m=2a$ we have $\lfloor m/a\rfloor=2$, yielding the stated ratio $2-1/a$.
\qed

\paragraph{Lemma 3 (periodicity and eventual validity for large $n$).}
Let $A$ be finite and nonempty, and set $L:=\mathrm{lcm}(A)$.
Then $B$ is periodic modulo $L$, and if we let
\[\delta:=\frac{|B\cap[1,L]|}{L}>0,
\]
then for every integer $x\ge 1$,
\[\bigl|d(x)-\delta\bigr|\le \frac{L}{x}.
\]
Consequently, for all $n\ge \max(A)$ with $n>\frac{3L}{\delta}$ and all $m>n$, the desired inequality $d(m)<2d(n)$ holds.

\paragraph{Proof.}
\emph{Periodicity.} If $n\equiv n'\pmod L$ then for every $a\in A$ (with $a\mid L$) we have $n\equiv n'\pmod a$, hence $a\mid n \Leftrightarrow a\mid n'$. Therefore membership in $B$ depends only on residue modulo $L$.

\emph{Density approximation.} Write $x=qL+r$ with $q\ge 0$ and $0\le r<L$.
Let $b:=|B\cap[1,L]|$ and $b(r):=|B\cap[1,r]|$.
By periodicity,
\[F(x)=qb + b(r).
\]
Thus
\[d(x)=\frac{qb+b(r)}{qL+r}.
\]
Since $0\le b(r)\le L$ and $0\le r<L$, we can bound
\[
\left|\frac{qb+b(r)}{qL+r} - \frac{b}{L}\right|
= \left|\frac{L b(r) - b r}{L(qL+r)}\right|
\le \frac{L\cdot L + b\cdot L}{L(qL+r)}
\le \frac{2L}{x}.
\]
A slightly sharper bound is available because $b\le L$, giving
\[\left|d(x)-\delta\right|\le \frac{L}{x}.
\]
(Indeed, $|L b(r) - b r|\le L\max\{b(r),b\}\le L^2$ and $L(qL+r)=Lx$.)

\emph{Eventual inequality.} For $m>n$ we have
\[d(m)\le \delta + \frac{L}{m} < \delta + \frac{L}{n},\qquad d(n)\ge \delta - \frac{L}{n}.
\]
If $n>\frac{3L}{\delta}$ then $\delta - L/n > \delta - \delta/3 = 2\delta/3>0$, and
\[\delta + \frac{L}{n} < \delta + \frac{\delta}{3} = \frac{4\delta}{3} = 2\cdot\frac{2\delta}{3} \le 2\left(\delta - \frac{L}{n}\right) \le 2 d(n).
\]
Therefore $d(m)<2d(n)$ for all $m>n>3L/\delta$.
\qed

\paragraph{FAST REALITY CHECK (brute-force search for small counterexamples).}
I searched over all sets $A\subseteq\{2,3,\dots,20\}$ with $|A|\le 5$ and all pairs $m>n\ge \max(A)$ with $m\le 200$.
No counterexample to $F(m)/m < 2F(n)/n$ was found in this range.

\subsection*{VERIFICATION}
\begin{itemize}
\item Lemma~2: verified the algebraic reduction and strictness. Also cross-checked against the explicit ``sharpness'' example $n=2a-1,m=2a$.
\item Lemma~3: verified the periodic decomposition $F(qL+r)=qb+b(r)$ and the inequality chain in the ``eventual'' part; the constant $3L/\delta$ is a sufficient (not claimed optimal) threshold.
\item Computational search: checked that the brute-force code tests \emph{all} $(A,n,m)$ in the stated finite search box.
\end{itemize}

\subsection*{FINAL}
\textbf{UNRESOLVED}

(i) \textbf{Strongest proved partial result.}
\begin{itemize}
\item If $A=\{a\}$ is a singleton, the inequality holds for all $m>n\ge a$, and the constant $2$ is best possible.
\item For general finite $A$, writing $L=\mathrm{lcm}(A)$ and $\delta=|B\cap[1,L]|/L$, we have $d(x)=\delta+O(L/x)$ and therefore $d(m)<2d(n)$ holds for all $m>n>3L/\delta$.
\item Brute-force search found no counterexample for $A\subseteq\{2,\dots,20\}$ with $|A|\le 5$ and $m,n\le 200$.
\end{itemize}

(ii) \textbf{First gap (crisp).} Prove or disprove the uniform inequality
\[\forall\,\text{finite }A\subseteq\mathbb{N}\ \forall m>n\ge\max(A):\ \frac{|B\cap[1,m]|}{m}<2\frac{|B\cap[1,n]|}{n},\]
where $B$ is the set of multiples of elements of $A$.
My proofs only cover (a) singletons and (b) sufficiently large $n$ depending on $A$.

(iii) \textbf{Top 3 next moves.}
\begin{enumerate}
\item Use the exact periodic formula $F(qL+r)=qb+b(r)$ to reduce the inequality to finitely many inequalities in the variables $(q_n,r_n,q_m,r_m)$, perhaps showing the worst case occurs for small $q_n$.
\item Attempt to prove a monotonicity/convexity statement for the prefix averages of a ``multiples'' periodic set that would force the factor-$2$ bound.
\item Computational: extend exhaustive search to larger $\max(A)$ and larger $|A|$ (e.g. up to $\max(A)=50$) and record any near-tight examples to guess a general extremal configuration.
\end{enumerate}

(iv) \textbf{Minimal counterexample structure.} A counterexample would consist of a finite set $A$ and a pair $m>n\ge\max(A)$ where the initial density $d(n)$ is unusually small compared to the eventual density $\delta$ of $B$, while $d(m)$ is already close to $\delta$. Such behaviour would likely require $L=\mathrm{lcm}(A)$ to be large and the set of ``hit'' residues modulo $L$ to be very unevenly distributed among initial segments.

\bigskip

