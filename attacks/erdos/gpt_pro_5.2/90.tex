
\noindent\textbf{FORMAL RESTATEMENT.}
For $n\ge 2$, let
\[
U(n):=\max\bigl\{\,|\{\{x,y\}: x,y\in A,\ x\ne y,\ |x-y|=1\}| : A\subset\mathbb{R}^2,\ |A|=n\,\bigr\}
\]
be the maximum possible number of unit-distance pairs among $n$ points in the Euclidean plane.
Question (Erd\H{o}s unit distance conjecture, asymptotic form): is it true that
\[
U(n)\le n^{1+O(1/\log\log n)}\qquad (n\to\infty)?
\]
Equivalently, does $\log U(n)/\log n\to 1$ as $n\to\infty$ up to a $O(1/\log\log n)$ error?

\medskip
\noindent\textbf{QUICK LITERATURE/CONTEXT CHECK.}
The problem statement notes: (a) lattice-point constructions show that an upper bound of the form $n^{1+o(1)}$ would be best possible; (b) an elementary argument gives $U(n)=O(n^{3/2})$; (c) the best known upper bound stated there is $O(n^{4/3})$.
I prove the elementary $O(n^{3/2})$ bound and a linear lower bound, without using external incidence theorems.

\medskip
\noindent\textbf{ATTACK PLAN.}
\begin{itemize}
\item \emph{Lower bound:} exhibit explicit $n$-point sets with many unit distances (a square grid).
\item \emph{Upper bound:} model unit distances as incidences between points and unit circles, use the fact two equal circles meet in at most two points, and apply a $K_{2,3}$-free extremal graph bound.
\end{itemize}

\medskip
\noindent\textbf{WORK.}

\smallskip
\noindent\textbf{Lemma 90.1 (grid lower bound: $\Omega(n)$).}
Let $k\ge 2$ and let $A$ be the $k\times k$ integer grid
\[
A:=\{(i,j): 0\le i,j\le k-1\},\qquad n=|A|=k^2.
\]
Then the number of unit-distance pairs in $A$ is exactly
\[
U_A:=2k(k-1)=2n-2\sqrt n.
\]
\textit{Proof.}
A unit distance in this grid occurs exactly between horizontally adjacent points or vertically adjacent points.
There are $k$ rows, each containing $k-1$ horizontal unit segments, giving $k(k-1)$ horizontal unit pairs.
Similarly there are $k$ columns, each containing $k-1$ vertical unit segments, giving another $k(k-1)$.
No other pair has Euclidean distance 1 because any other difference vector has squared length at least $2$.
Thus $U_A=2k(k-1)=2k^2-2k=2n-2\sqrt n$.\hfill $\square$

\smallskip
\noindent\textbf{Lemma 90.2 (elementary upper bound: $O(n^{3/2})$).}
For every $A\subset\mathbb{R}^2$ with $|A|=n$,
\[
|\{\{x,y\}\subset A: |x-y|=1\}|\le \frac{\sqrt 2}{2}\,n^{3/2}+\frac14 n.
\]
\textit{Proof.}
This is the special case $r=1$ of Lemma 89.1. For completeness: build the bipartite incidence graph between $n$ unit circles centered at points of $A$ and the $n$ points of $A$. Two unit circles intersect in at most two points, so the bipartite graph is $K_{2,3}$-free. The extremal counting argument (Claim 2 in Lemma 89.1) gives at most $\sqrt2\,n^{3/2}+\tfrac12 n$ directed incidences, i.e. at most half as many undirected unit-distance pairs.\hfill $\square$

\smallskip
\noindent\textbf{FAST REALITY CHECK (small grids).}
For the $k\times k$ grid, Lemma 90.1 gives unit-distance pairs $2k(k-1)$. Numerically:
\begin{verbatim}
k=2  n=4    unit pairs=4
k=3  n=9    unit pairs=12
k=4  n=16   unit pairs=24
k=5  n=25   unit pairs=40
k=10 n=100  unit pairs=180
\end{verbatim}
So the construction is linear in $n$.

\medskip
\noindent\textbf{VERIFICATION.}
\begin{itemize}
\item Lemma 90.1: checked that the only integer vectors of squared length 1 are $(\pm1,0),(0,\pm1)$.
\item Lemma 90.2: the $K_{2,3}$-free condition hinges only on ``two circles meet in at most two points'', which is valid for Euclidean circles.
\item Scaling: the bound is for distance $1$ but is invariant under rigid motions; scaling would change the distance but not the combinatorics.
\end{itemize}

\medskip
\noindent\textbf{FINAL.} \textbf{UNRESOLVED}
\begin{enumerate}
\item[(i)] \textbf{Strongest proved partial result.}
An elementary incidence/graph argument gives $U(n)=O(n^{3/2})$ with an explicit constant (Lemma 90.2). A $k\times k$ integer grid gives $U(n)\ge 2n-2\sqrt n$ for $n=k^2$ (Lemma 90.1).
\item[(ii)] \textbf{First gap (crisp).}
Improve the upper bound exponent below $3/2$ by an argument that uses a special feature of the Euclidean metric, aiming toward $n^{1+o(1)}$.
\item[(iii)] \textbf{Top 3 next moves.}
\begin{itemize}
\item Strengthen the extremal-graph step (which only used $K_{2,3}$-freeness) by incorporating geometric restrictions on how many unit circles can pass through many points in \emph{structured} configurations.
\item Prove sharper bounds for the incidence graph using planar geometry (e.g., crossing number for drawings of the unit-distance graph, or polynomial partitioning).
\item Computational experimentation: for moderate $n$, search for point sets with unusually large unit-distance graphs to guess extremal structure.
\end{itemize}
\item[(iv)] \textbf{Likely structure of a minimal counterexample.}
A set attaining very large $U(n)$ must have many points lying on many unit circles in a highly correlated way (since random sets have nearly all distances distinct). Any counterexample to $n^{1+o(1)}$ would likely involve arithmetic structure (lattice-like sets) arranged to create many repeated unit distances.
\end{enumerate}


