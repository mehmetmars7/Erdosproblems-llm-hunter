
1) FORMAL RESTATEMENT

Fix integers $n\ge 1$ and $k\ge 2$.

A \emph{$k$-sunflower} (also called a $\Delta$-system) is a collection of $k$ sets $S_1,\dots,S_k$ such that there exists a set $C$ (the \emph{core}) with
\[
S_i\cap S_j = C\quad\text{for all }i\ne j.
\]
Equivalently, the sets $P_i:=S_i\setminus C$ are pairwise disjoint.

Let $f(n,k)$ be the smallest integer with the following property:

Every family $\mathcal F$ of sets, each of size exactly $n$ (i.e., $n$-uniform), with
\[
|\mathcal F|\ge f(n,k)
\]
contains a $k$-sunflower.

Question: For each fixed $k$, does there exist a constant $c_k>0$ such that for all $n$,
\[
f(n,k) < c_k^n\ ?
\]

2) QUICK LITERATURE/CONTEXT CHECK

The provided problem statement already lists the classical bound $f(n,k)\le (k-1)^n n!$ (Erd\H{o}s--Rado) and the current-best upper bounds of the form $(Ck\log n)^n$ (Alweiss--Lovett--Wu--Zhang and subsequent refinements). I do not use or claim any additional literature beyond what is explicitly stated in the problem file.

3) ATTACK PLAN

Proof track (ambitious):
- Remove the $\log n$ factor from the best known bounds by strengthening the combinatorial/probabilistic arguments in the modern sunflower proofs.

Disproof track (ambitious):
- Construct, for fixed $k$ and arbitrarily large $n$, $n$-uniform families of size $\exp(\Omega(n\log n))$ without $k$-sunflowers, which would refute the existence of $c_k^n$ upper bounds.

What I can deliver: a complete proof of the classical Erd\H{o}s--Rado sunflower bound, plus exact small-case values for $(n,k)=(2,3)$.

4) WORK

FAST REALITY CHECK (tiny parameters)

- For $k=2$, any two sets form a $2$-sunflower (with core their intersection), so $f(n,2)=2$.
- For $n=1$, a $k$-sunflower is just $k$ singletons (all pairwise intersections are empty), so $f(1,k)=k$.
- For $(n,k)=(2,3)$, I give an exact value $f(2,3)=7$ below, and I verified by brute force over all graphs on up to $7$ vertices that $6$ is the maximum size of a $3$-sunflower-free family of $2$-sets.

Lemma 20.1 (the $k=2$ case).
For all $n\ge 1$,
\[
f(n,2)=2.
\]

Proof.
A $2$-sunflower is a pair of sets $S_1,S_2$; taking the core $C=S_1\cap S_2$ gives $S_1\cap S_2=C$ vacuously. Thus every family $\mathcal F$ with $|\mathcal F|\ge 2$ contains a $2$-sunflower, so $f(n,2)\le 2$.

On the other hand, a family of size $1$ contains no pair of distinct sets, hence no $2$-sunflower of size $2$. Therefore $f(n,2)\ge 2$.

Combining gives $f(n,2)=2$. $\square$

Lemma 20.2 (Erd\H{o}s--Rado sunflower lemma, classical bound; proved here).
For all integers $n\ge 1$ and $k\ge 2$, any $n$-uniform family $\mathcal F$ with
\[
|\mathcal F| > (k-1)^n\,n!
\]
contains a $k$-sunflower. Consequently,
\[
f(n,k)\le (k-1)^n\,n!+1.
\]
(If one uses the alternative convention that $f(n,k)$ is the minimal threshold for the property "$|\mathcal F|>f(n,k)$ implies a sunflower", then the same proof gives $f(n,k)\le (k-1)^n n!$.)

Proof.
We prove the contrapositive by induction on $n$:

\emph{Claim.} If an $n$-uniform family $\mathcal F$ contains no $k$-sunflower, then
\[
|\mathcal F|\le (k-1)^n\,n!.
\]

Base case $n=1$.
Then $\mathcal F$ is a family of singletons. Any $k$ distinct singletons form a $k$-sunflower with core $\varnothing$, so sunflower-freeness forces $|\mathcal F|\le k-1=(k-1)^1\cdot 1!$.

Induction step.
Assume the claim holds for $(n-1)$-uniform families, where $n\ge 2$, and let $\mathcal F$ be an $n$-uniform family with no $k$-sunflower.

Let $t$ be the maximum size of a subfamily of pairwise disjoint sets in $\mathcal F$. If $t\ge k$, then any $k$ disjoint sets form a $k$-sunflower with core $\varnothing$, contradicting sunflower-freeness. Hence
\[
t\le k-1.
\]
Fix a maximal disjoint subfamily $S_1,\dots,S_t\in\mathcal F$ and let
\[
U:=S_1\cup\cdots\cup S_t.
\]
Then $|U|=tn\le (k-1)n$.

Maximality implies that every $S\in\mathcal F$ intersects $U$: if some $S\in\mathcal F$ were disjoint from $U$, then $S$ would be disjoint from each $S_i$, contradicting maximality of $\{S_1,\dots,S_t\}$.

For each element $x\in U$, define the subfamily
\[
\mathcal F_x:=\{S\in\mathcal F: x\in S\}.
\]
Count the pairs $(S,x)$ with $S\in\mathcal F$ and $x\in S\cap U$:
\[
\sum_{x\in U} |\mathcal F_x|\;=\;\sum_{S\in\mathcal F} |S\cap U|\;\ge\;\sum_{S\in\mathcal F} 1\;=\;|\mathcal F|,
\]
where we used $|S\cap U|\ge 1$ for every $S\in\mathcal F$.
Therefore, there exists some $x\in U$ such that
\[
|\mathcal F_x|\ge \frac{|\mathcal F|}{|U|}\ge \frac{|\mathcal F|}{tn}\ge \frac{|\mathcal F|}{(k-1)n}.
\tag{1}
\]

Now form an $(n-1)$-uniform family by deleting $x$ from every set in $\mathcal F_x$:
\[
\mathcal G:=\{S\setminus\{x\}: S\in\mathcal F_x\}.
\]
Then $|\mathcal G|=|\mathcal F_x|$.
If $\mathcal G$ contained a $k$-sunflower $T_1,\dots,T_k$ with core $C$, then the sets
\[
T_1\cup\{x\},\dots,T_k\cup\{x\}
\]
would lie in $\mathcal F$ and would form a $k$-sunflower with core $C\cup\{x\}$, contradicting the assumption that $\mathcal F$ has no $k$-sunflower.
Hence $\mathcal G$ is also $k$-sunflower-free.

By the induction hypothesis applied to $\mathcal G$, we have
\[
|\mathcal G|\le (k-1)^{n-1}(n-1)!.
\tag{2}
\]
Combining (1), (2), and $|\mathcal G|=|\mathcal F_x|$ yields
\[
\frac{|\mathcal F|}{(k-1)n}\le |\mathcal F_x|=|\mathcal G|\le (k-1)^{n-1}(n-1)!.
\]
Multiplying by $(k-1)n$ gives
\[
|\mathcal F|\le (k-1)^n\,n!.
\]
This completes the induction and proves the claim. Taking the contrapositive gives the first statement, and the inequality for $f(n,k)$ follows from the definition. $\square$
Lemma 20.3 (exact value for $f(2,3)$).
We have
\[
f(2,3)=7.
\]

Proof.
An $n=2$ uniform family is a set of 2-element subsets of some ground set; equivalently, it is the edge set of a simple graph $G$.

A $3$-sunflower of 2-sets has either:
- core $\varnothing$: three pairwise-disjoint 2-sets, i.e., a matching of size $3$ in $G$;
- core of size $1$: three 2-sets all sharing a single vertex and otherwise disjoint, i.e., a star $K_{1,3}$ in $G$.

Thus $G$ is $3$-sunflower-free iff it has no matching of size $3$ and maximum degree at most $2$.

(Construction with 6 edges.) Take $G$ to be the disjoint union of two triangles. Then $|E(G)|=6$, the maximum degree is $2$, and the maximum matching size is $2$ (one edge from each triangle). Hence $G$ is $3$-sunflower-free. Therefore $f(2,3)\ge 7$.

(Upper bound: any 7 edges force a sunflower.) Let $G$ be a graph with $|E(G)|\ge 7$.

- If $\Delta(G)\ge 3$, then $G$ contains a star $K_{1,3}$, giving a 3-sunflower with core of size $1$.

- Otherwise $\Delta(G)\le 2$, so every connected component of $G$ is a path or a cycle.
  In a path on $t$ vertices there are $t-1$ edges and a matching of size $\lfloor t/2\rfloor$.
  In a cycle on $t$ vertices there are $t$ edges and a matching of size $\lfloor t/2\rfloor$.

  If $G$ has $7$ edges and $\Delta(G)\le 2$, then $G$ must have at least $7$ vertices (because each component satisfies $|E|\le |V|$). Write $v=|V(G)|\ge 7$.

  For graphs with $\Delta\le 2$, the matching number is at least $\lfloor v/2\rfloor$ because each component (path or cycle) has a matching covering at least half its vertices, and matchings from different components are disjoint. Therefore
  \[
  \nu(G)\ge \left\lfloor\frac{v}{2}\right\rfloor \ge 3.
  \]
  Thus $G$ contains a matching of size $3$, giving a 3-sunflower with empty core.

In all cases, $|E(G)|\ge 7$ implies the existence of a $3$-sunflower. Hence $f(2,3)\le 7$.

Together with $f(2,3)\ge 7$ from the construction, we conclude $f(2,3)=7$. $\square$

5) VERIFICATION

- Lemma 20.1: checks the definition of a $2$-sunflower and shows both $f(n,2)\le 2$ and $f(n,2)\ge 2$.
- Lemma 20.2: the only potentially delicate points are (a) that a maximal pairwise-disjoint subfamily has size $\le k-1$ (else $k$ disjoint sets form a sunflower with empty core), (b) that maximality forces every set of $\mathcal F$ to intersect the union $U$, and (c) that deleting a fixed element $x$ from all sets containing it preserves sunflower-freeness (a sunflower in the deleted family would lift by adding $x$ back). Each of these is explicitly justified in the proof.
- Lemma 20.3: the translation between $3$-sunflowers of $2$-sets and (stars $K_{1,3}$) / (matchings of size $3$) is checked by analyzing the possible core sizes. The upper bound uses only the structure of graphs with maximum degree $\le 2$.

6) FINAL

**UNRESOLVED**

(i) Strongest proved partial result:
- Classical general upper bound proved from scratch: any $n$-uniform family with $|\mathcal F|>(k-1)^n n!$ contains a $k$-sunflower (Lemma 20.2). Consequently $f(n,k)\le (k-1)^n n!+1$.
- Exact small cases: $f(n,2)=2$ for all $n$ (Lemma 20.1) and $f(2,3)=7$ (Lemma 20.3).

(ii) First gap (crisp):
Prove or disprove: for each fixed $k$ there exists a constant $c_k>0$ such that for all $n$,
\[
f(n,k)\le c_k^n.
\]

(iii) Top 3 next moves (concrete):
1. Try to strengthen Lemma 20.2's recursion by using a more efficient way to find a high-frequency element (or a structured subset of elements) without paying the full $n!$ factor; any recurrence that replaces the factor $n$ by $O(1)$ would move toward $c_k^n$.
2. Extend exact computations of $f(n,3)$ for $n=3,4,5$ by brute force / ILP on small universes to search for extremal sunflower-free families and identify candidate constructions.
3. Systematically test structured families (e.g. families defined by restrictions on intersection sizes, or coming from vector spaces / error-correcting codes) for sunflower-freeness and measure their size as a function of $n$.

(iv) Minimal counterexample structure:
If the exponential bound $c_k^n$ is false for some fixed $k$, a minimal counterexample would be a sequence of sunflower-free $n$-uniform families $\mathcal F_n$ whose sizes beat every $c^n$ (so $\log|\mathcal F_n|/n\to\infty$), while avoiding $k$ sets with a common core and disjoint petals.

