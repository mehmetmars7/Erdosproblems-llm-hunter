% Erdos Problem #600
% URL: https://www.erdosproblems.com/600

Let $e(n,r)$ be minimal such that every graph on $n$ vertices with at least $e(n,r)$ edges, each edge contained in at least one triangle, must have an edge contained in at least $r$ triangles. Let $r\geq 2$. Is it true that\[e(n,r+1)-e(n,r)\to \infty\]
as $n\to \infty$? Is it true that\[\frac{e(n,r+1)}{e(n,r)}\to 1\]
as $n\to \infty$? Ruzsa and Szemer\'{e}di \cite{RuSz78} proved that $e(n,r)=o(n^2)$ for any fixed $r$.

\noindent\textbf{1) FORMAL RESTATEMENT}\par
Fix integers $n\ge 1$ and $r\ge 2$.
For a simple undirected graph $H$ on vertex set $[n]:=\{1,2,\dots,n\}$, write $t_H(e)$ for the number of triangles of $H$ containing the edge $e$.

Define $e(n,r)$ as the minimal integer $E$ such that every graph $H$ on $n$ vertices satisfying
\begin{enumerate}
\item $|E(H)|\ge E$, and
\item $t_H(e)\ge 1$ for every edge $e\in E(H)$,
\end{enumerate}
must contain an edge $e$ with $t_H(e)\ge r$.

Edge case: if $r>n-2$, then no edge in any $n$-vertex graph can lie in $r$ triangles (since $t_H(e)\le n-2$). In this case the defining implication can only hold vacuously, and the natural minimal choice is $e(n,r)={n\choose 2}+1$.

\noindent\textbf{2) QUICK LITERATURE/CONTEXT CHECK}\par
The problem statement records: for fixed $r$, $e(n,r)=o(n^2)$ (Ruzsa--Szemer\'{e}di). I do not use additional external results.

\noindent\textbf{3) ATTACK PLAN}\par
\begin{itemize}
\item Proof-track idea: understand extremal graphs with bounded triangle-multiplicity per edge ($t_H(e)\le r-1$) but with every edge in a triangle ($t_H(e)\ge 1$). Compare the maximum edge counts for successive $r$.
\item Disproof-track idea: attempt to build (for fixed $r$) very dense graphs where every edge lies in many triangles but still keep the maximum per edge below $r$, and see whether successive maxima differ by $O(1)$.
\item Computation track: brute force for small $n$ to see how $e(n,r)$ behaves.
\end{itemize}

\noindent\textbf{4) WORK}\par
\textbf{Fast reality check (exact computation for $n\le 7$).}
I exhaustively searched all graphs on $n\le 7$ vertices.
For each $(n,r)$ I computed
\[
M(n,r):=\max\{|E(H)|: H \text{ on } n \text{ vertices, } 1\le t_H(e)\le r-1 \text{ for all edges }e\},
\]
and then $e(n,r)=M(n,r)+1$.
The exact values found are:
\[
\begin{array}{c|cccc}
 n & e(n,2) & e(n,3) & e(n,4) & e(n,5) \\\hline
 3 & 4 & 4 & 4 & 4 \\
 4 & 4 & 7 & 7 & 7 \\
 5 & 7 & 9 & 11 & 11 \\
 6 & 7 & 13 & 13 & 16 \\
 7 & 10 & 16 & 17 & 19 
\end{array}
\]
(For example, $e(7,4)=17$ means there exists a 7-vertex graph with 16 edges, every edge in at least one triangle, and no edge in 4 triangles, but any such graph with 17 edges forces an edge in at least 4 triangles.)

\medskip
\noindent\textbf{Lemma 600.1 (reformulation as a maximum).}\\
Let
\[
M(n,r):=\max\{|E(H)|: H \text{ on } n \text{ vertices, } 1\le t_H(e)\le r-1 \text{ for all edges }e\}.
\]
Then $e(n,r)=M(n,r)+1$.

\noindent\emph{Proof.}
By definition, $e(n,r)$ is the smallest $E$ such that every graph $H$ with $|E(H)|\ge E$ and $t_H(e)\ge 1$ for all edges contains some edge with $t_H(e)\ge r$.
Equivalently, $E-1$ is the largest number of edges for which there exists a counterexample graph $H$ with
\begin{enumerate}
\item $|E(H)|=E-1$,
\item $t_H(e)\ge 1$ for all edges, and
\item $t_H(e)\le r-1$ for all edges.
\end{enumerate}
That largest value is exactly $M(n,r)$. Hence $e(n,r)=M(n,r)+1$. \qed

\medskip
\noindent\textbf{Lemma 600.2 (monotonicity).}\\
For all $n\ge 1$ and $r\ge 2$:
\begin{enumerate}
\item $e(n,r+1)\ge e(n,r)$ (nondecreasing in $r$);
\item $e(n+1,r)\ge e(n,r)$ (nondecreasing in $n$).
\end{enumerate}

\noindent\emph{Proof.}
(1) If a graph has an edge in at least $r+1$ triangles, then it has an edge in at least $r$ triangles. Thus any edge threshold that forces an edge in $r+1$ triangles also forces an edge in $r$ triangles, so the minimal threshold for $r+1$ cannot be smaller: $e(n,r+1)\ge e(n,r)$.

(2) Suppose we had $e(n+1,r)<e(n,r)$. Take an $n$-vertex graph $H$ witnessing that $e(n,r)$ is minimal, i.e., with $|E(H)|=e(n,r)-1$, every edge in at least one triangle, and no edge in $r$ triangles. Add one isolated vertex to form a graph $H'$ on $n+1$ vertices. Then $H'$ still has $|E(H')|=e(n,r)-1$ edges, every edge still lies in a triangle (unchanged), and still no edge lies in $r$ triangles. This contradicts the definition of $e(n+1,r)$ if $e(n+1,r)\le e(n,r)-1$. Hence $e(n+1,r)\ge e(n,r)$. \qed

\medskip
\noindent\textbf{Lemma 600.3 (vacuous range $r>n-2$).}\\
If $r>n-2$ then $e(n,r)={n\choose 2}+1$.

\noindent\emph{Proof.}
In any graph on $n$ vertices, every edge $e$ can be extended to a triangle by choosing a third vertex, giving at most $n-2$ possibilities. Hence $t_H(e)\le n-2$ always. If $r>n-2$ then the conclusion ``there exists an edge contained in at least $r$ triangles'' is impossible for every graph $H$.
Therefore the defining implication can only hold when the antecedent is false, i.e., when $|E(H)|\ge e(n,r)$ is impossible.
The smallest integer larger than the maximum possible number of edges ${n\choose 2}$ is ${n\choose 2}+1$, and for this choice the antecedent ``$|E(H)|\ge e(n,r)$'' is false for all graphs, so the implication holds vacuously. Minimality follows. \qed

\noindent\textbf{5) VERIFICATION}\par
\begin{itemize}
\item Lemma 600.1 is just unpacking of ``minimal such that ...'' and is logically correct.
\item Lemma 600.2(2) uses adding an isolated vertex; this preserves ``each edge lies in at least one triangle'' and preserves per-edge triangle counts.
\item Lemma 600.3 matches the computed small cases: e.g., for $n=7$ and $r=6>5=n-2$ the exhaustive search gave $e(7,6)=22={7\choose 2}+1$.
\item Computation check: for $n\le 7$ the brute force enumerated all $2^{\binom{n}{2}}$ graphs; the reported $e(n,r)$ values are therefore exact for those $n$.
\end{itemize}

\noindent\textbf{6) FINAL}\par
\textbf{UNRESOLVED}

(i) \emph{Strongest proved partial result.} Exact values of $e(n,r)$ for $n\le 7$ and $r\le 5$ are computed above. In general, $e(n,r)=M(n,r)+1$ (Lemma 600.1), $e$ is nondecreasing in both $n$ and $r$ (Lemma 600.2), and $e(n,r)={n\choose 2}+1$ for $r>n-2$ (Lemma 600.3).

(ii) \emph{First gap (crisp).} Prove or disprove that for every fixed $r\ge 2$,
\[
 e(n,r+1)-e(n,r)\to\infty\quad\text{as } n\to\infty,
\]
or, failing that, produce an explicit infinite sequence of $n$ for which $e(n,r+1)-e(n,r)$ stays bounded.

(iii) \emph{Top 3 next moves.}
\begin{enumerate}
\item Computational: push exact/upper-bound searches to larger $n$ for small $r$ (e.g., $r=2,3,4$) using SAT/ILP rather than brute force, to see whether successive maxima differ by $O(1)$ or grow.
\item Structural: characterize extremal graphs for $M(n,r)$ (all edges in at least one triangle, triangle-multiplicity per edge bounded by $r-1$). Even a coarse structural lemma that such graphs have $o(n^2)$ edges (as stated in the problem) with an explicit quantitative bound could inform the difference/ratio questions.
\item Inequality approach: find two-sided bounds for $M(n,r)$ as a function of $r$ (for fixed $r$), and then compare $M(n,r+1)-M(n,r)$.
\end{enumerate}

(iv) \emph{What a minimal counterexample would look like.} To refute $e(n,r+1)-e(n,r)\to\infty$ for a fixed $r$, one would need an infinite sequence $n_j\to\infty$ and graphs $H_j$ on $n_j$ vertices with
\begin{enumerate}
\item $|E(H_j)|=M(n_j,r+1)$,
\item every edge in at least one triangle,
\item every edge in at most $r$ triangles,
\end{enumerate}
but with $M(n_j,r+1)-M(n_j,r)$ bounded independently of $j$.


