\section{Round 3: finite-range CRT obstructions and a lower bound on the least ``good'' exponent}

\subsection{1) ROUND-3 OBJECTIVE}
\textbf{Path pursued: (C) obstruction / correction of proof-strategy.}

Round~2 reduced any genuine counterexample to a covering problem on the exponent interval $E_L=\{0,1,\dots,L\}$ by residue classes modulo orders $\operatorname{ord}_{p^2}(2)$ (Corollary~11.3.1), and isolated a Wieferich-type obstruction to reusing non-Wieferich primes (Corollary~11.4.2).  However, the global quantifier gap
\[
\forall\ \text{odd }n\ge 3\ \exists e\ge 0:\ n-2^e\ \text{squarefree}
\]
remains open.

In this round I prove a \emph{sharp obstruction to several natural proof strategies}: for any fixed finite set of exponents $E$, one can force \emph{all} shifts $n-2^e$ ($e\in E$) to be divisible by prime squares via CRT.  In particular, no approach that searches only a bounded set of exponents (independent of $n$) can ever prove the conjecture.

Moreover, using a mild prime-existence input (Bertrand's postulate), I quantify this obstruction and show that along an explicit infinite sequence of $n$, the least exponent $e$ for which $n-2^e$ could possibly be squarefree must satisfy
\[
 e \ \gtrsim\ \sqrt{\log_2 n}.
\]
This is strictly stronger than any result in Rounds~1--2, and it rules out an entire class of ``small-$e$'' strategies.

\subsection{2) Round-2 FOUNDATION USED}
I use the following Round~2 results as black boxes:
\begin{enumerate}
\item \textbf{Lemma~11.3:} For fixed odd prime $p$, the set $\mathcal E_p(n)=\{e\ge 0: p^2\mid(n-2^e)\}$ is either empty or a single residue class modulo $\operatorname{ord}_{p^2}(2)$.
\item \textbf{Corollary~11.3.1:} A counterexample $n$ forces a covering of $E_L$ by such residue classes.
\item \textbf{Lemma~11.5:} Any counterexample can be reduced modulo the product of its certifying prime squares.
\end{enumerate}
(While the new constructions below do not logically require Lemma~11.3, they are best interpreted as producing explicit exponent-coverings of the type isolated in Corollary~11.3.1.)

\subsection{3) NEW INSIGHT / TOOL (ROUND-3)}
New content introduced here:
\begin{itemize}
\item \textbf{Theorem~11.6 (finite-range CRT killing):} for any finite set of exponents $E$, there are infinitely many odd $n$ for which every $n-2^e$ ($e\in E$) is divisible by an odd prime square.
\item \textbf{Theorem~11.7 (quantitative obstruction):} choosing $E=\{0,1,\dots,K\}$ and selecting primes using Bertrand's postulate yields an explicit upper bound on the constructed $n$ in terms of $K$, implying the lower bound $e_{\min}(n)\gtrsim\sqrt{\log n}$ along an infinite sequence.
\end{itemize}

\subsection{4) ATTACK PLAN (ROUND-3)}
\textbf{Gap remaining after Round~2.}
We still lack a method to guarantee, for each odd $n$, \emph{some} exponent $e\le \lfloor\log_2(n-1)\rfloor$ such that $n-2^e$ is squarefree.

\textbf{Round~3 goal.}
Instead of attempting to close the universal quantifier directly, we:
\begin{enumerate}
\item prove that CRT can force \emph{arbitrarily long finite} blocks of exponents to be ``bad'' (all shifts divisible by squares),
\item thereby rule out strategies that only test finitely many exponents independent of $n$,
\item quantify the size of such CRT constructions to obtain a lower bound on how large the smallest ``good'' exponent must sometimes be.
\end{enumerate}
This overcomes a key ambiguity from Round~2: the exponent-covering condition is \,easy\, to satisfy on a \emph{fixed} finite set of exponents, so any genuine resolution must exploit the \emph{self-referential} dependence $L\approx\log_2 n$.

\subsection{5) WORK (ROUND-3)}

\subsubsection{5.1. A CRT construction killing any prescribed finite set of exponents}

\begin{theorem}[Finite-range CRT killing]\label{thm:finiteCRT}
Let $E\subset\mathbb Z_{\ge 0}$ be finite.  Then there exist infinitely many odd integers $n$ such that for every $e\in E$ the integer $n-2^e$ is divisible by the square of an odd prime (hence is not squarefree).

In particular, for every $K\ge 0$ there exist infinitely many odd $n$ such that
\[
\forall e\in\{0,1,\dots,K\}:\quad n-2^e\ \text{is not squarefree}.
\]
\end{theorem}

\begin{proof}
Choose pairwise distinct odd primes $\{p_e: e\in E\}$.
Consider the system of congruences
\begin{equation}\label{eq:CRTsystem}
 n\equiv 1\pmod 2,\qquad
 n\equiv 2^e\pmod{p_e^2}\ \ (e\in E).
\end{equation}
The moduli $2$ and $\{p_e^2: e\in E\}$ are pairwise coprime, so by the Chinese remainder theorem there exists an integer solution $n_0$ to \eqref{eq:CRTsystem}.  Any such $n_0$ is odd (by $n\equiv 1\pmod2$) and satisfies
\[
 p_e^2\mid (n_0-2^e)\qquad(e\in E),
\]
so every $n_0-2^e$ is divisible by an odd prime square and therefore is not squarefree.

All solutions are congruent modulo
\[
M:=2\prod_{e\in E} p_e^2.
\]
Hence $n=n_0+tM$ is also a solution for every integer $t\ge 0$, producing infinitely many odd integers $n$ with the stated property.
\end{proof}

\begin{corollary}[Obstruction to finite-$e$ proof strategies]\label{cor:finiteObstruction}
Fix $K\ge 0$.  There exist infinitely many odd integers $n$ such that \emph{no} representation
\[
 n=s+2^e\qquad(s\ \text{squarefree})
\]
is possible with $e\le K$.  Equivalently, any proof of the conjecture must necessarily allow the witnessing exponent $e$ to grow (in general) with $n$.
\end{corollary}

\begin{proof}
Apply Theorem~\ref{thm:finiteCRT} with $E=\{0,1,\dots,K\}$.
\end{proof}

\subsubsection{5.2. Quantifying the obstruction: a lower bound on the least possible good exponent}

To turn Theorem~\ref{thm:finiteCRT} into a statement relating $K$ and the size of the constructed $n$, we need a mild way to choose the primes $p_e$ with explicit size control.

\begin{lemma}[Bertrand's postulate (external input)]\label{lem:bertrand}
For every integer $m>1$ there exists a prime $p$ with
\[
 m<p<2m.
\]
\end{lemma}

\begin{theorem}[Quantitative finite-range killing]\label{thm:quantitative}
For every integer $K\ge 0$ there exists an odd integer $n$ such that:
\begin{enumerate}
\item $n-2^e$ is not squarefree for every $e\in\{0,1,\dots,K\}$;
\item $n<2^{K^2+5K+6}$.
\end{enumerate}
Consequently, along an infinite sequence of odd integers $n$ one has the lower bound
\[
 e_{\min}(n)\ \ge\ \sqrt{\log_2 n}\ -\ 2,
\]
where $e_{\min}(n)$ denotes the least exponent $e\ge 0$ for which $n-2^e$ could be squarefree (i.e. the least $e$ such that $n-2^e$ is squarefree, if such an $e$ exists; otherwise interpret $e_{\min}(n)=+\infty$).
\end{theorem}

\begin{proof}
\textbf{Step 1: choose controlled primes.}
For each $e\in\{0,1,\dots,K\}$, apply Lemma~\ref{lem:bertrand} with $m=2^{e+1}$ to choose a prime $p_e$ satisfying
\begin{equation}\label{eq:peInterval}
 2^{e+1}<p_e<2^{e+2}.
\end{equation}
These primes are automatically odd and are pairwise distinct because the intervals $(2^{e+1},2^{e+2})$ are disjoint as $e$ varies.

\textbf{Step 2: CRT system.}
Apply Theorem~\ref{thm:finiteCRT} with $E=\{0,1,\dots,K\}$ and the primes $p_e$ chosen above.
Let $M=2\prod_{e=0}^K p_e^2$ and let $n_0$ be the least positive solution to the congruences
\[
 n\equiv 1\pmod 2,\qquad n\equiv 2^e\pmod{p_e^2}\ (0\le e\le K).
\]
Then $n_0<M$ and $n_0-2^e$ is divisible by $p_e^2$ for all $0\le e\le K$.

To ensure $n$ is large enough to make the subsequent inequality meaningful (and still preserve the congruences), set
\[
 n:=n_0+M.
\]
Then $M\le n<2M$ and still $p_e^2\mid(n-2^e)$ for all $0\le e\le K$.
This proves (1).

\textbf{Step 3: bound the size of $n$.}
From \eqref{eq:peInterval} we have $p_e<2^{e+2}$ and hence $\log_2 p_e<e+2$.  Therefore
\[
\log_2 M
= 1+2\sum_{e=0}^K \log_2 p_e
< 1+2\sum_{e=0}^K (e+2)
=1+2\Big(\frac{K(K+1)}{2}+2(K+1)\Big)
=K^2+5K+5.
\]
Since $n<2M$, we obtain
\[
\log_2 n<\log_2(2M)=1+\log_2 M< K^2+5K+6,
\]
which is equivalent to $n<2^{K^2+5K+6}$ and proves (2).

\textbf{Step 4: deduce the lower bound on $e_{\min}(n)$.}
By construction, all exponents $e\le K$ are ``bad'' (the shifts are not squarefree). Hence $e_{\min}(n)\ge K+1$ (or $e_{\min}(n)=+\infty$).
From $\log_2 n< K^2+5K+6=(K+\tfrac52)^2-\tfrac14$, we have $K+\tfrac52>\sqrt{\log_2 n+\tfrac14}$ and thus
\[
K+1>\sqrt{\log_2 n+\tfrac14}-\tfrac32\ge \sqrt{\log_2 n}-2.
\]
Therefore $e_{\min}(n)\ge \sqrt{\log_2 n}-2$ along the constructed sequence.
\end{proof}

\begin{remark}[Interpretation in the language of Round~2 coverings]
Theorem~\ref{thm:finiteCRT} produces, for any finite exponent set $E$, a ``certificate'' set of primes $\{p_e: e\in E\}$ such that $E$ is covered by the residue classes corresponding to the congruences $n\equiv 2^e\pmod{p_e^2}$.  This is the simplest possible exponent covering: each prime square is used to kill exactly one specified exponent.

The genuine conjecture only fails if one can kill the \emph{moving} interval $E_L$ where $L=\lfloor\log_2(n-1)\rfloor$.  Theorem~\ref{thm:quantitative} shows that killing an initial interval $\{0,\dots,K\}$ is feasible even when $K$ grows like $\sqrt{\log n}$, hence any proof must exploit structure beyond such ``one-prime-per-exponent'' CRT constructions.
\end{remark}

\subsection{6) ADVERSARIAL VERIFICATION}
\textbf{CRT coprimality.}  In \eqref{eq:CRTsystem}, the moduli are $2$ and $p_e^2$ for distinct odd primes $p_e$, hence pairwise coprime. CRT applies without further hypotheses.

\textbf{Parity.}  Because the nontrivial moduli are odd, the congruences $n\equiv 2^e\pmod{p_e^2}$ do not determine parity. Adding $n\equiv 1\pmod 2$ forces $n$ odd and remains compatible.

\textbf{Squarefree definition.}  If an odd prime square $p_e^2$ divides $n-2^e$, then $n-2^e$ is not squarefree by definition.  No other hidden assumption is used.

\textbf{Edge cases ($K=0,1$).}  For $K=0$, one can take $p_0=3$ and obtain solutions $n\equiv 1\pmod{18}$, so $n-1$ is divisible by $9$ (hence not squarefree); this matches the theorem.  For $K=1$, choosing $p_0=3,p_1=5$ yields an explicit CRT system with solutions, forcing both $n-1$ and $n-2$ to be divisible by odd prime squares.

\textbf{Use of Bertrand.}  Lemma~\ref{lem:bertrand} is the only external input.  The disjoint-interval choice ensures distinct primes and yields the claimed size bound.  If Bertrand were replaced by a weaker prime-existence bound, the qualitative Theorem~\ref{thm:finiteCRT} would remain valid (it needs only infinitely many distinct odd primes), but the quantitative inequality in Theorem~\ref{thm:quantitative} would weaken.

\textbf{Interaction with Round~2 Wieferich obstruction.}  The present constructions do not reuse primes across multiple exponents, so Corollary~11.4.2 is not invoked.  This is consistent: Theorem~\ref{thm:finiteCRT} deliberately chooses the least efficient covering (one prime per exponent).  Any attempt to make a true counterexample must cover $E_L$ with far fewer primes, hence must reuse primes, where the Wieferich obstruction becomes relevant.

\subsection{7) FINAL}
\textbf{UNRESOLVED (BUT STRICTLY ADVANCED).}

We still do not have a proof (or disproof) of the global statement for all odd $n\ge 3$.  However, Round~3 proves two new theorems that substantially sharpen the obstruction landscape:
\begin{itemize}
\item For any fixed finite exponent set $E$, CRT produces infinitely many odd $n$ for which all $n-2^e$ ($e\in E$) are divisible by odd prime squares (Theorem~\ref{thm:finiteCRT}).  Hence any proof strategy confined to finitely many exponents cannot succeed.
\item Quantitatively, there exist infinitely many odd $n$ for which every $e\le \sqrt{\log_2 n}-2$ is ``bad'' (Theorem~\ref{thm:quantitative}).  Thus the least possible exponent witnessing the conjectured representation must sometimes be at least on the order of $\sqrt{\log n}$.
\end{itemize}

\subsection{8) COMPLETION ESTIMATE (MANDATORY)}
\textbf{COMPLETION: 62\%}

\subsection{9) REFERENCES}
\begin{itemize}
\item Bertrand's postulate (Chebyshev): for every $m>1$ there is a prime $p$ with $m<p<2m$.  (Used only for Lemma~\ref{lem:bertrand} and Theorem~\ref{thm:quantitative}.)
\end{itemize}
