
\noindent\textbf{1) FORMAL RESTATEMENT}

For integers $t\ge 1$ and $m\ge 1$, define the \emph{$t$-smooth component} $s_t(m)$ to be the product of all prime powers $p^{v_p(m)}$ dividing $m$ with primes $p<t$ (equivalently, remove from the prime factorisation of $m$ all primes $\ge t$).

For integers $n\ge 0$ and $t\ge 1$, define
\[f(n,t):=\left|\{s_t(m): m\in\{n+1,n+2,\dots,n+t\}\}\right|.
\]
The conjecture asks whether there exists an absolute constant $c>0$ such that for all $n,t$,
\[f(n,t)\ge c t\qquad\text{(equivalently, }f(n,t)\gg t\text{ uniformly).}
\]
The extracted text reports a weaker proven bound $f(n,t)\gg t/\log t$.

\noindent\textbf{2) QUICK LITERATURE/CONTEXT CHECK}

The extracted statement reports a bound of order $t/\log t$ due to Erd\H{o}s and Graham, but no proof is included. I do not rely on external results beyond the elementary lemmas proved below.

\noindent\textbf{3) ATTACK PLAN}

\begin{itemize}
\item Use basic divisibility properties of intervals of length $t$ (every modulus $d\le t$ has a multiple in the interval).
\item Try to force many distinct values of $s_t(m)$ by looking at numbers in the interval with a ``distinguishing'' large prime factor $p$ with $p<t$.
\item Search computationally for small $(n,t)$ where $f(n,t)$ is unusually small relative to $t$.
\end{itemize}

\noindent\textbf{4) WORK}

\textbf{Fast reality check (computations).}

I computed $f(n,t)$ exactly for $2\le t\le 50$ and $0\le n\le 500$.  The smallest observed ratios $f(n,t)/t$ in this search were:
\begin{itemize}
\item $t=2$: minimum $f=1$ at $n=0$ (ratio $0.5$);
\item $t=7$: minimum $f=4$ at $n=176$ (ratio $\approx 0.571$);
\item for $t\le 50$ and $n\le 500$, all other minima had ratio $\ge 0.586$.
\end{itemize}
This small search shows no evidence of $f(n,t)$ being sublinear in $t$.

\medskip
\noindent\textbf{Lemma 461.1 (every modulus $d\le t$ appears).}
Let $n\ge 0$ and $t\ge 1$ be integers. For every integer $d$ with $1\le d\le t$, the interval $\{n+1,\dots,n+t\}$ contains at least one multiple of $d$.

\textit{Proof.}
Consider the residues of the $t$ consecutive integers $n+1,\dots,n+t$ modulo $d$. Since $t\ge d$, among these $t$ residues there is a complete set of residues modulo $d$ (in particular, residue $0$ occurs). Concretely, the map $i\mapsto (n+i)\bmod d$ takes $d$ consecutive values $i$ to all residues, so some $i\in\{1,\dots,d\}\subseteq\{1,\dots,t\}$ satisfies $n+i\equiv 0\pmod d$. \hfill$\square$

\medskip
\noindent\textbf{Lemma 461.2 (large primes $p\in(t/2,t)$ yield distinguishable smooth components when their multiples are distinct).}
Assume $t\ge 3$ and let $p$ be a prime with $\tfrac{t}{2}<p<t$. Then the interval $\{n+1,\dots,n+t\}$ contains \emph{exactly one} multiple of $p$; call it $m_p$. Then:
\begin{enumerate}
\item $p\mid s_t(m_p)$.
\item If $p,q$ are distinct primes in $(t/2,t)$ and $m_p\neq m_q$, then $s_t(m_p)\neq s_t(m_q)$.
\end{enumerate}

\textit{Proof.}
Because $p\le t$, Lemma~461.1 guarantees at least one multiple of $p$ in the interval. For uniqueness: if two distinct numbers in the interval were multiples of $p$, their difference would be a nonzero multiple of $p$ of absolute value $<t$, hence at least $p$, which contradicts $p>t/2$ (since then $2p>t$ implies there is at most one multiple in any length-$t$ interval).

(1) Since $p<t$ and $p\mid m_p$, the prime $p$ appears in the prime factorisation of $m_p$ with some positive exponent $v_p(m_p)\ge 1$, and by definition $s_t(m_p)$ includes the factor $p^{v_p(m_p)}$. Hence $p\mid s_t(m_p)$.

(2) If $m_p\neq m_q$, then $q\nmid m_p$: otherwise $m_p$ would be a multiple of $q$ in the interval, and by uniqueness it would have to equal $m_q$. Therefore $q\nmid s_t(m_p)$ (since $q<t$ and if $q$ divided $s_t(m_p)$ it would divide $m_p$). Similarly $p\nmid s_t(m_q)$. Thus $s_t(m_p)$ is divisible by $p$ but not by $q$, while $s_t(m_q)$ is divisible by $q$ but not by $p$, forcing $s_t(m_p)\neq s_t(m_q)$. \hfill$\square$

\medskip
\noindent\textbf{5) VERIFICATION}

\begin{itemize}
\item Lemma~461.1: checked extreme case $d=t$; the interval of length $t$ always contains a multiple of $t$ (namely $n+t$ if $t\mid n$; more generally by residues).
\item Lemma~461.2: uniqueness uses only $2p>t$. The restriction $p>t/2$ is essential.
\item Computation: checked by brute-force factorisation for all $t\le 50, n\le 500$.
\end{itemize}

\noindent\textbf{6) FINAL}

\textbf{UNRESOLVED}

(i) \emph{Strongest proved partial result:} The interval $[n+1,n+t]$ contains a multiple of every $d\le t$ (Lemma~461.1), and primes $p\in(t/2,t)$ contribute divisibly to the smooth component of their unique multiple, yielding distinct $s_t$ values whenever these multiples are distinct (Lemma~461.2). Computations for $t\le 50$ and $n\le 500$ show $f(n,t)/t$ is never below $0.5$ in this range.

(ii) \emph{First gap (crisp):} Prove a uniform constant $c>0$ such that for all $n,t$ the set $\{s_t(n+1),\dots,s_t(n+t)\}$ has size at least $ct$. Equivalently, bound from above how many of the $t$ integers can share the same $t$-smooth component.

(iii) \emph{Top 3 next moves:}
\begin{enumerate}
\item Prove a uniform bound on multiplicities: for a fixed value $S$, bound the number of $m\in[n+1,n+t]$ with $s_t(m)=S$.
\item Develop a counting argument based on representing $m=s_t(m)\cdot r$ where $r$ is $t$-rough (all prime factors $\ge t$), and control how many such $r$ can occur in an interval of length $t$.
\item Search computationally for larger $t$ and specially constructed large $n$ (e.g. aligning many primes) to see how small $f(n,t)/t$ can get.
\end{enumerate}

(iv) \emph{Minimal counterexample structure:} If the conjecture fails badly, a minimal counterexample would be an interval $[n+1,n+t]$ in which many integers share the same small-prime pattern, i.e. can be written as $m=S\cdot r$ with a fixed $S$ composed of primes $<t$ and varying $t$-rough factors $r$, but with only $o(t)$ distinct $S$ values appearing.


