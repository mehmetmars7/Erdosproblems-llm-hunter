% Erdos Problem #181
% Attempt for Erdos Problem #181
% Following PROMPT_STRATEGY.MD
% Tools/Constraints:
% - Web browsing available? YES (not used beyond the problem text; no extra literature claims).
% - Computation available (Python/Sage/Mathematica)? YES (used for small-n Ramsey computation for $Q_2=C_4$).

PROBLEM

Let $Q_n$ be the $n$-dimensional hypercube graph (so that $Q_n$ has $2^n$ vertices and $n2^{n-1}$ edges). Prove that
\[R(Q_n) \ll 2^n.\]

OUTPUT

## 1) FORMAL RESTATEMENT

Let $Q_n$ denote the $n$-dimensional hypercube graph with vertex set $\{0,1\}^n$ and edges between vertices at Hamming distance $1$.
So $|V(Q_n)|=2^n$.

For a graph $H$, let $R(H)$ be its (diagonal) Ramsey number: the smallest integer $N$ such that every 2-coloring of the edges of $K_N$ contains a monochromatic copy of $H$.

The conjecture/problem asks to prove the existence of an absolute constant $C>0$ such that for all $n$,
\[
R(Q_n)\le C\,2^n.
\]

---

## 2) QUICK LITERATURE/CONTEXT CHECK

I will not use or claim any external results beyond what is explicitly written in the problem text.

The problem text reports:

* Trivial bound: $R(Q_n)\le R(K_{2^n})\le C^{2^n}$.
* Current best known bound: $R(Q_n)\ll 2^{(2-c)n}$ for some small $c>0$.

I do not re-prove those improvements.

---

## 3) ATTACK PLAN

I cannot prove the desired linear-in-$2^n$ bound. I will instead:

1. Record trivial lower and upper bounds with full proofs.
2. Compute $R(Q_n)$ for the first nontrivial cases $n=1,2$ as a sanity check.

---

## 4) WORK

### Phase 1: FAST REALITY CHECK (tiny $n$)

* $Q_1$ is a single edge, so $R(Q_1)=2$.
* $Q_2$ is a 4-cycle $C_4$. By brute force enumeration of all 2-edge-colorings of $K_m$ for $m\le 6$, I found
\[
R(Q_2)=R(C_4)=6.
\]

---

### Lemma 4.1 (trivial lower bound)

For every $n\ge 1$,
\[
R(Q_n)\ge 2^n.
\]

**Proof.**
If $N<2^n$, then the complete graph $K_N$ has fewer than $2^n$ vertices and therefore cannot contain $Q_n$ as a (labeled) subgraph at all, regardless of edge-coloring.
Hence $N$ does not satisfy the Ramsey property for $Q_n$, so by definition $R(Q_n)>N$.
Taking $N=2^n-1$ gives $R(Q_n)\ge 2^n$.
$\square$

---

### Lemma 4.2 (trivial tower-type upper bound via clique Ramsey)

For every $n\ge 1$,
\[
R(Q_n)\le R\bigl(K_{2^n}\bigr)\le \binom{2^{n+1}-2}{2^n-1}\le 4^{2^n}=2^{2^{n+1}}.
\]

**Proof.**
Since $Q_n$ has $2^n$ vertices, it is a (spanning) subgraph of the complete graph $K_{2^n}$ on the same number of vertices.
Therefore any monochromatic $K_{2^n}$ contains a monochromatic copy of $Q_n$, and hence
\[
R(Q_n)\le R(K_{2^n}).
\]

It remains to upper bound the diagonal clique Ramsey numbers.
For integers $s,t\ge 1$, let $R(s,t)$ denote the smallest integer $N$ such that every red/blue coloring of the edges of $K_N$ contains either a red $K_s$ or a blue $K_t$.
We first prove the standard recursion
\[
R(s,t)\le R(s-1,t)+R(s,t-1). \tag{$\*$}
\]

Fix $s,t\ge 2$ and set $N:=R(s-1,t)+R(s,t-1)$.
Consider an arbitrary red/blue coloring of the edges of $K_N$.
Pick a vertex $v$.
Let $X$ be the set of vertices joined to $v$ by a red edge, and $Y$ the set of vertices joined to $v$ by a blue edge.
Then $X\cup Y$ is the $N-1$ vertices other than $v$, so $|X|+|Y|=N-1$.

* If $|X|\ge R(s-1,t)$, look at the induced coloring on the complete graph $K_X$.
  By definition of $R(s-1,t)$, $K_X$ contains either a red $K_{s-1}$ or a blue $K_t$.
  In the first case, adjoining $v$ gives a red $K_s$ (because $v$ is red-adjacent to all vertices of $X$).
  In the second case, we already have a blue $K_t$.

* If $|Y|\ge R(s,t-1)$, look at the induced coloring on $K_Y$.
  By definition of $R(s,t-1)$, $K_Y$ contains either a red $K_s$ or a blue $K_{t-1}$.
  In the first case we are done.
  In the second case, adjoining $v$ gives a blue $K_t$ (because $v$ is blue-adjacent to all vertices of $Y$).

Since $|X|+|Y|=R(s-1,t)+R(s,t-1)-1$, at least one of the inequalities $|X|\ge R(s-1,t)$ or $|Y|\ge R(s,t-1)$ must hold.
Thus every coloring of $K_N$ yields a red $K_s$ or a blue $K_t$, proving $(\*)$.

Next we prove by induction on $s+t$ that
\[
R(s,t)\le \binom{s+t-2}{s-1}. \tag{$\dagger$}
\]

**Base cases.** If $s=1$ or $t=1$, then $R(1,t)=R(s,1)=1$ because a single vertex is a monochromatic $K_1$.
Also $\binom{s+t-2}{s-1}=\binom{t-1}{0}=1$ (or symmetrically), so $(\dagger)$ holds.

**Inductive step.** Assume $s,t\ge 2$ and that $(\dagger)$ holds for all pairs with smaller sum.
By $(\*)$ and the induction hypothesis,
\[
R(s,t)\le R(s-1,t)+R(s,t-1)
\le \binom{s+t-3}{s-2}+\binom{s+t-3}{s-1}.
\]
By Pascal's identity, the right-hand side equals $\binom{s+t-2}{s-1}$, proving $(\dagger)$.

Now set $s=t:=T$.
Then the diagonal clique Ramsey number satisfies
\[
R(K_T)=R(T,T)\le \binom{2T-2}{T-1}.
\]
Finally, since the maximum binomial coefficient is at most the sum of all binomial coefficients,
\[
\binom{2T}{T}\le \sum_{j=0}^{2T}\binom{2T}{j}=(1+1)^{2T}=4^T.
\]
Also $\binom{2T-2}{T-1}\le \binom{2T}{T}$ (each factor in the product formula increases), hence
\[
R(K_T)\le \binom{2T-2}{T-1}\le 4^T.
\]

Taking $T=2^n$ gives
\[
R(Q_n)\le R(K_{2^n})\le \binom{2^{n+1}-2}{2^n-1}\le 4^{2^n}=2^{2^{n+1}}.
\]
$\square$

## 5) VERIFICATION

* Lemma 4.1: correct because an $N$-vertex graph cannot contain a graph with more than $N$ vertices.
* Lemma 4.2: verified: (a) $Q_n\subseteq K_{2^n}$, so any monochromatic $K_{2^n}$ contains a monochromatic copy of $Q_n$; (b) the recursion $R(s,t)\le R(s-1,t)+R(s,t-1)$ and the binomial upper bound $R(s,t)\le \binom{s+t-2}{s-1}$ were proved inside Lemma 4.2; (c) $\binom{2t-2}{t-1}\le 4^t$ follows since the largest binomial coefficient is at most the sum $\sum_{j=0}^{2t}\binom{2t}{j}=4^t$.
* Computation sanity check: $R(C_4)=6$ is consistent with known small Ramsey behavior and was verified by exhaustive search for $m\le 6$.

---

## 6) FINAL

**UNRESOLVED**

(i) **Strongest fully proved partial result obtained here.**

Elementary bounds with full proofs:
\[
2^n\le R(Q_n)\le 2^{2^{n+1}}.
\]
Also, exact tiny values: $R(Q_1)=2$ and $R(Q_2)=6$.

(ii) **Exact first gap.**

Prove a *linear* upper bound $R(Q_n)=O(2^n)$ (or even a substantially improved exponential bound $2^{(1+o(1))n}$).

(iii) **Top 3 next moves (concrete targets).**

1. Develop an embedding lemma that finds a monochromatic $Q_n$ inside any 2-colored $K_{C2^n}$ by iterative neighborhood intersection, exploiting the layered structure of $Q_n$.
2. Prove a density-type statement: in any red/blue coloring of $K_{N}$, one color contains a large subgraph with pseudo-random/expanding properties sufficient to embed $Q_n$.
3. Compute/estimate $R(Q_3)$ experimentally for small $N$ using SAT/heuristics to get intuition for the true growth.

(iv) **What a minimal counterexample would likely look like.**

A counterexample to $R(Q_n)=O(2^n)$ would be a sequence of 2-colorings of $K_{c2^n}$ (for arbitrarily large $n$ and any fixed $c$) that avoid monochromatic $Q_n$ in both colors. Such colorings would likely need to be highly structured (e.g., based on affine geometries or product constructions) to block embeddings at all scales.
