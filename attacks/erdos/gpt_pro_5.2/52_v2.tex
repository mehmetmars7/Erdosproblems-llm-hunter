\section{Erd\H{o}s Problem \#52 --- Round 2 (gap-closure via incidence geometry)}

\subsection{1) ROUND-2 OBJECTIVE}
\textbf{Path pursued: (A) proof-track (partial, but strictly stronger than Round 1).}

The original target statement
\[
\max\{|A+A|,|AA|\} \gg_{\varepsilon} |A|^{2-\varepsilon}\qquad(\forall\varepsilon>0)
\]
remains open. In Round~2 we close the \,\emph{first explicit gap left by Round~1}\, by proving an unconditional \emph{superlinear} lower bound of the form
\[
\max\{|A+A|,|AA|\} \gg |A|^{5/4}
\]
for all finite $A\subset\mathbb Z$. This is a substantial qualitative upgrade over the linear bound proved in Round~1.

\subsection{2) ROUND-1 FOUNDATION USED}
We treat the Round~1 work as correct and use it as follows.
\begin{enumerate}
\item \textbf{Round~1 Lemma~1:} $|A+A|\ge 2|A|-1$ for finite $A\subset\mathbb Z$.
\item \textbf{Round~1 Lemma~2:} if $A$ contains a nonzero element then $|AA|\ge |A|$.
\item \textbf{Round~1 Corollary~3:} $\max(|A+A|,|AA|)\ge 2|A|-1$ for all finite nonempty $A\subset\mathbb Z$.
\item \textbf{Monotonicity under passing to subsets:} if $B\subseteq A$ then $|B+B|\le |A+A|$ and $|BB|\le |AA|$.
\end{enumerate}
No Round~1 proofs are repeated.

\subsection{3) NEW INSIGHT / TOOL (ROUND-2)}
The new tool is the \textbf{Szemer\'edi--Trotter incidence theorem} for points and lines in $\mathbb R^2$, combined with an \textbf{Elekes-type line--point construction} encoding the additive and multiplicative structure of $A$.

This yields a quantitative bound on the product $|A+A|\,|AA|$, and hence on $\max(|A+A|,|AA|)$.

\subsection{4) ATTACK PLAN (ROUND-2)}
\textbf{Gap after Round~1.} Round~1 explicitly identified as the first major gap:
\begin{quote}
prove any superlinear bound $\max(|A+A|,|AA|)\ge |A|^{1+c}$ in $\mathbb Z$ by self-contained arguments.
\end{quote}

\textbf{Claims to prove now.}
\begin{enumerate}
\item[(i)] (Incidence lower bound) Construct a set of lines $\mathcal L$ and points $\mathcal P$ so that the number of incidences satisfies $I(\mathcal P,\mathcal L)\gtrsim |A|^3$.
\item[(ii)] (Incidence upper bound) Apply Szemer\'edi--Trotter to bound $I(\mathcal P,\mathcal L)$ in terms of $|\mathcal P|=|A+A|\,|AA|$ and $|\mathcal L|\asymp |A|^2$.
\item[(iii)] Rearrange to obtain $|A+A|\,|AA|\gtrsim |A|^{5/2}$, hence $\max(|A+A|,|AA|)\gtrsim |A|^{5/4}$.
\end{enumerate}

\textbf{Why this overcomes Round~1 obstacles.} Round~1 stayed within elementary ordering/injection arguments, which cap out at linear growth. The incidence method imports a deep but standard geometric inequality (Szemer\'edi--Trotter) whose strength is precisely of the right scale to force \emph{superlinear} expansion.

\subsection{5) WORK (ROUND-2)}
\subsubsection{5.1 Reduction to a ``well-behaved'' subset}
Let $A\subset\mathbb Z$ be finite nonempty and put $n:=|A|$.
If $n=1$ the claims are trivial, so assume $n\ge 2$.

Remove $0$ if present: let $A_0:=A\setminus\{0\}$.
If $A_0=\varnothing$ then $A=\{0\}$ and Round~1 already gives $\max(|A+A|,|AA|)=1$.
Otherwise $|A_0|\ge n-1$ and, by monotonicity,
\[
|A+A|\ge |A_0+A_0|,\qquad |AA|\ge |A_0A_0|.
\]
Thus it suffices to prove the desired bound for sets not containing $0$, at the cost of changing the implicit constant.

Next, reduce to a set with at least half its elements of the same sign.
Let $A^+:=A_0\cap \mathbb Z_{>0}$ and $A^-:=(-A_0)\cap \mathbb Z_{>0}$.
Then $|A^+|+|A^-|=|A_0|\ge n-1$, so one of them has size at least $(n-1)/2$.
Choose $B$ to be that larger one; then $B\subset \mathbb Z_{>0}$ and
\[
|B|\ge \frac{n-1}{2}.
\]
By monotonicity, $\max(|A+A|,|AA|)\ge \max(|B+B|,|BB|)$, so it suffices to prove a superlinear bound for finite $B\subset\mathbb Z_{>0}$.

Henceforth, \emph{assume} $A\subset\mathbb Z_{>0}$ and $0\notin A$, and write $n:=|A|$.

\subsubsection{5.2 Szemer\'edi--Trotter theorem (external input)}
We use the following standard form.

\paragraph{Theorem (Szemer\'edi--Trotter).}
There exists an absolute constant $C_{\mathrm{ST}}>0$ such that for any finite set $\mathcal P$ of $m$ points and any finite set $\mathcal L$ of $\ell$ distinct lines in $\mathbb R^2$, the number of incidences
\[
I(\mathcal P,\mathcal L):=|\{(p,\ell)\in \mathcal P\times \mathcal L:\; p\in \ell\}| 
\]
satisfies
\[
I(\mathcal P,\mathcal L)\le C_{\mathrm{ST}}\big(m^{2/3}\ell^{2/3}+m+\ell\big).
\]

\subsubsection{5.3 Elekes line--point construction}
Define the sumset and product set sizes
\[
S:=|A+A|,\qquad P:=|AA|.
\]
Consider the point set
\[
\mathcal P := (A+A)\times (AA)\subset \mathbb R^2,
\]
so $|\mathcal P|=SP$.

For each ordered pair $(a,b)\in A\times A$, define the line
\[
\ell_{a,b}:\quad y=a(x-b).
\]
Let
\[
\mathcal L :=\{\ell_{a,b}: (a,b)\in A\times A\}.
\]
Because $A\subset\mathbb Z_{>0}$, all slopes $a$ are nonzero. Moreover, if $\ell_{a,b}=\ell_{a',b'}$ as sets then comparing slopes gives $a=a'$ and then comparing intercepts gives $-ab=-ab'$, hence $b=b'$. Therefore all $n^2$ lines are distinct and
\[
|\mathcal L|=n^2.
\]

\subsubsection{5.4 Incidence lower bound $I(\mathcal P,\mathcal L)\ge n^3$}
Fix $a,b\in A$. For each $c\in A$, define the point
\[
p_{a,b,c}:=(b+c,ac)\in \mathbb R^2.
\]
Then $b+c\in A+A$ and $ac\in AA$, so $p_{a,b,c}\in \mathcal P$. Also
\[
ac = a\big((b+c)-b\big),
\]
so $p_{a,b,c}\in \ell_{a,b}$.

As $c$ varies in $A$, the $x$--coordinates $b+c$ are distinct (because $c\mapsto b+c$ is injective), hence the points $p_{a,b,c}$ are distinct for fixed $(a,b)$. Therefore each line $\ell_{a,b}$ is incident to at least $n$ distinct points of $\mathcal P$.

Summing over the $n^2$ distinct lines, we obtain
\[
I(\mathcal P,\mathcal L) \ge n^2\cdot n = n^3.
\]

\subsubsection{5.5 Incidence upper bound and rearrangement}
Apply Szemer\'edi--Trotter to $(\mathcal P,\mathcal L)$ with $m=|\mathcal P|=SP$ and $\ell=|\mathcal L|=n^2$:
\[
I(\mathcal P,\mathcal L)
\le C_{\mathrm{ST}}\big( (SP)^{2/3}(n^2)^{2/3} + SP + n^2\big)
= C_{\mathrm{ST}}\big( n^{4/3}(SP)^{2/3} + SP + n^2\big).
\]
Combined with $I(\mathcal P,\mathcal L)\ge n^3$, this yields
\begin{equation}
\label{eq:ST-main}
 n^3 \le C_{\mathrm{ST}}\big( n^{4/3}(SP)^{2/3} + SP + n^2\big).
\end{equation}

\paragraph{Claim.} There exists an absolute constant $c_0>0$ such that for all $n\ge 2$,
\begin{equation}
\label{eq:SP-lb}
SP \ge c_0\, n^{5/2}.
\end{equation}

\paragraph{Proof of the claim.}
Let $X:=SP$.
If $X\ge n^{5/2}$ then \eqref{eq:SP-lb} holds with any $c_0\le 1$.
Assume now that $X<n^{5/2}$.
Then
\[
\frac{n^{4/3}X^{2/3}}{X} = \frac{n^{4/3}}{X^{1/3}} > \frac{n^{4/3}}{(n^{5/2})^{1/3}} = n^{1/2}\ge 1,
\]
so $n^{4/3}X^{2/3} \ge X$. Hence
\eqref{eq:ST-main} implies
\[
 n^3 \le C_{\mathrm{ST}}\big( 2n^{4/3}X^{2/3} + n^2\big).
\]
For $n\ge 2C_{\mathrm{ST}}$ we have $n^3-C_{\mathrm{ST}}n^2 = n^2(n-C_{\mathrm{ST}})\ge \tfrac12 n^3$, so
\[
\tfrac12 n^3 \le 2C_{\mathrm{ST}} n^{4/3}X^{2/3}.
\]
Rearranging gives
\[
X^{2/3} \ge \frac{1}{4C_{\mathrm{ST}}} n^{5/3}
\quad\Rightarrow\quad
X \ge \Big(\frac{1}{4C_{\mathrm{ST}}}\Big)^{3/2} n^{5/2}.
\]
Thus \eqref{eq:SP-lb} holds for all $n\ge 2C_{\mathrm{ST}}$ with $c_0:=(4C_{\mathrm{ST}})^{-3/2}$.
For the finitely many $2\le n<2C_{\mathrm{ST}}$, decrease $c_0$ if needed so that \eqref{eq:SP-lb} holds uniformly.
\hfill $\square$

\subsubsection{5.6 From $SP$ to $\max(S,P)$}
From \eqref{eq:SP-lb} and the AM--GM inequality,
\[
\max(S,P) \ge \sqrt{SP} \ge \sqrt{c_0}\, n^{5/4}.
\]
This proves:

\paragraph{Theorem (Elekes--type sum-product bound over $\mathbb Z$).}
There exists an absolute constant $c>0$ such that for every finite nonempty set $A\subset\mathbb Z$,
\[
\max\{|A+A|,|AA|\}\ge c\,|A|^{5/4}.
\]

\paragraph{Proof (reduction back to general $A\subset\mathbb Z$).}
We proved the theorem for $A\subset\mathbb Z_{>0}$, $0\notin A$, with some absolute $c$.
For general $A\subset\mathbb Z$, choose $B\subset\mathbb Z_{>0}$ with $|B|\ge (|A|-1)/2$ as in \S5.1.
Then by monotonicity,
\[
\max(|A+A|,|AA|)\ge \max(|B+B|,|BB|) \ge c\,|B|^{5/4}\ge c\,2^{-5/4}(|A|-1)^{5/4}.
\]
Absorb the factor $2^{-5/4}$ and the $(|A|-1)^{5/4}\asymp |A|^{5/4}$ loss into the implicit constant to get the stated bound for all $A$.
\hfill $\square$

\subsection{6) ADVERSARIAL VERIFICATION}
We stress-test the new argument against hidden degeneracies.
\begin{enumerate}
\item \textbf{Duplicate lines:} If $0\in A$, then the family $y=a(x-b)$ includes the horizontal line $y=0$ repeated for every $b$ when $a=0$. We avoided this by reducing to $A\subset\mathbb Z_{>0}$ (so $a\neq 0$) before defining $\mathcal L$. In the reduced setting, $(a,b)\mapsto \ell_{a,b}$ is injective.

\item \textbf{Overcounting incidences:} The lower bound uses that each line $\ell_{a,b}$ is incident to \emph{at least} $n$ \emph{distinct} points of $\mathcal P$. Distinctness holds because $c\mapsto b+c$ is injective, so the $x$--coordinates along the line differ. Even if different triples $(a,b,c)$ yield the same point for different $(a,b)$, incidences are counted per \emph{(point,line)} pair, so this does not reduce the incidence count.

\item \textbf{Sign issues:} The construction itself works over all reals (negative slopes allowed). We reduced to positives only to avoid $a=0$ and keep the ``all lines distinct'' verification simple. The monotonicity argument shows this reduction cannot make the conclusion false.

\item \textbf{Small $n$:} The rearrangement step used $n\ge 2C_{\mathrm{ST}}$ to subtract the $+n^2$ term. For smaller $n$, the theorem is salvaged by decreasing the absolute constant $c$, and Round~1 already provides a uniform linear lower bound which covers all finite exceptional cases.

\item \textbf{Quantifiers:} All constants ($C_{\mathrm{ST}},c_0,c$) are absolute; no dependence on $A$ occurs. The reduction from $A$ to $B$ loses at most a fixed factor $2^{5/4}$.
\end{enumerate}
No hidden assumptions remain.

\subsection{7) FINAL (EXACTLY ONE)}
\textbf{UNRESOLVED (BUT STRICTLY ADVANCED).}

We have rigorously proved the superlinear bound
\[
\max(|A+A|,|AA|)\gg |A|^{5/4}
\]
for all finite $A\subset\mathbb Z$, closing the primary Round~1 gap (``prove any superlinear exponent''). The original conjecture $2-\varepsilon$ remains open.

\subsection{8) COMPLETION ESTIMATE (MANDATORY)}
\textbf{COMPLETION: 55\%}.

\subsection{9) REFERENCES}
Only references actually used in this round are listed.
\begin{enumerate}
\item E. Szemer\'edi and W. T. Trotter, \emph{Extremal problems in discrete geometry}, Combinatorica \textbf{3} (1983), 381--392. (Szemer\'edi--Trotter incidence theorem.)
\item G. Elekes, \emph{On the number of sums and products}, Acta Arith. \textbf{81} (1997), 365--367. (Incidence-based sum--product bound.)
\item T. F. Bloom, \emph{On the sum-product problem for integers with few prime factors}, arXiv:2512.04931 (2025). (For context on current record exponents; not used in proofs.)
\item ErdosProblems.com, \emph{Erd\H{o}s Problem \#52} (accessed 2026). (For context on current record exponents; not used in proofs.)
\end{enumerate}
