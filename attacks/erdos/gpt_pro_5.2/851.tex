% Erdos Problem #851
% URL: https://www.erdosproblems.com/851

Let $\epsilon>0$. Is there some $r\ll_\epsilon 1$ such that the density of integers of the form $2^k+n$, where $k\geq 0$ and $n$ has at most $r$ prime divisors, is at least $1-\epsilon$? Romanoff \cite{Ro34} proved that the set of integers of the form $2^k+p$ (where $p$ is prime) has positive lower density. See also [205] . References [Ro34] Romanoff, N. P., \"{U}ber einige S\"Atze der additiven Zahlentheorie . Math. Ann. (1934), 668-678.

1) “FORMAL RESTATEMENT”

Ambiguity note: “$n$ has at most $r$ prime divisors” can mean either
- $\omega(n)\le r$ (at most $r$ distinct prime divisors), or
- $\Omega(n)\le r$ (at most $r$ prime factors counted with multiplicity).
The phrase “prime divisors” most naturally means $\omega(n)$. I record statements for $\Omega$ when they remain valid.

For $r\in\mathbb{N}$ define the set
\[
S_r := \{m\in\mathbb{N}: \exists k\in\mathbb{N}_{\ge 0},\ \exists n\in\mathbb{N}_{\ge 1}\text{ with } m=2^k+n \text{ and } \omega(n)\le r\}.
\]
(One may analogously define $S_r^{(\Omega)}$ using $\Omega(n)\le r$.)

Let $d(S_r)$ denote the (natural) density if the limit exists:
\[
d(S_r)=\lim_{N\to\infty}\frac{|S_r\cap\{1,2,\dots,N\}|}{N},
\]
and otherwise interpret “density at least $1-\epsilon$” as lower density $\underline d(S_r)=\liminf_{N\to\infty} |S_r\cap[1,N]|/N$.

Question: For every $\epsilon>0$, does there exist an integer $r=r(\epsilon)$ (written $r\ll_\epsilon 1$) such that $\underline d(S_r)\ge 1-\epsilon$?

2) “QUICK LITERATURE/CONTEXT CHECK”

The problem statement notes Romanoff's theorem: the set $\{2^k+p\}$ with $p$ prime has positive lower density. No other results are stated in the problem text. In this write-up I do not invoke external literature beyond what is explicitly stated.

3) “ATTACK PLAN”

Proof-track ideas:
- Try to show that for most integers $m$, among the $\lfloor\log_2 m\rfloor+1$ differences $m-2^k$ (with $2^k<m$), at least one has $\omega(m-2^k)\le r$. This resembles a “many chances” sieve/large deviations problem.

Disproof-track ideas:
- Try to construct a positive-density set of integers $m$ such that all $m-2^k$ have many prime factors, for any fixed $r$.

Best current path in this write-up: prove monotonicity/covering lemmas that are unconditional and do computational sanity checks for moderate $N$.

4) “WORK”

FAST REALITY CHECK (computation):
For $N=200000$ and $N=2000000$ I checked representability of all $2\le m\le N$ as $m=2^k+n$ with $n\ge 1$ and either $\omega(n)\le r$ or $\Omega(n)\le r$.
- For $N=200000$, the observed fractions for $\omega(n)\le r$ were:
  $r=1$: $0.5558$; $r=2$: $0.9884$; $r=3$: $1.0000$ (every $m\le 200000$ is representable).
  For $\Omega(n)\le r$: $r=1$: $0.5544$; $r=2$: $0.9015$; $r=3$: $0.9932$ (1353 exceptions up to $200000$); $r=4$: $0.9999$ (15 exceptions up to $200000$).
- For $N=2000000$, the observed fractions were:
  for $\omega(n)\le r$: $r=1$: $0.534549$; $r=2$: $0.9827465$; $r=3$: $0.9999995$ (the only exception is $m=1$).
  for $\Omega(n)\le r$: $r=3$: $0.988699$; $r=4$: $0.9997815$ (437 exceptions up to $2\cdot 10^6$).
These computations are evidence only; they do not establish any limiting density.

\textbf{Lemma 851.1 (Small numbers have few prime factors).}
Let $n\in\mathbb{N}_{\ge 1}$.
- If $\Omega(n)$ denotes the number of prime factors counted with multiplicity, then
\[
2^{\Omega(n)}\le n.
\]
In particular, if $n\le 2^r$ then $\Omega(n)\le r$.
- Consequently also $\omega(n)\le \Omega(n)\le r$ for $n\le 2^r$.

\emph{Proof.}
Write the prime factorization $n=\prod_{i=1}^t p_i$ where primes are listed with multiplicity, so $t=\Omega(n)$ and each $p_i\ge 2$. Then
\[
n=\prod_{i=1}^t p_i \ge \prod_{i=1}^t 2 = 2^t = 2^{\Omega(n)}.
\]
If $n\le 2^r$, this inequality implies $2^{\Omega(n)}\le 2^r$ and hence $\Omega(n)\le r$. The inequality $\omega(n)\le \Omega(n)$ is immediate because counting distinct prime divisors is at most counting with multiplicity. \qed

\textbf{Lemma 851.2 (Monotonicity in $r$).}
For integers $r\le r'$, we have $S_r\subseteq S_{r'}$. Consequently the lower density $\underline d(S_r)$ is nondecreasing in $r$.

\emph{Proof.}
If $m\in S_r$, then $m=2^k+n$ with $\omega(n)\le r$. Since $r\le r'$, we also have $\omega(n)\le r'$, so the same representation shows $m\in S_{r'}$. The density monotonicity follows from set inclusion. \qed

\textbf{Lemma 851.3 (A guaranteed covered neighborhood of each power of two).}
Fix $r\ge 1$. For every integer $k\ge 0$ and every integer $n$ with $1\le n\le 2^r$, the integer $m=2^k+n$ lies in $S_r$ (and also lies in the $\Omega$-version $S_r^{(\Omega)}$).

\emph{Proof.}
We have $m=2^k+n$ with the displayed $n$. By Lemma 851.1, $n\le 2^r$ implies $\Omega(n)\le r$, hence also $\omega(n)\le r$. Therefore $m\in S_r$ by definition. \qed

5) “VERIFICATION”

- Lemma 851.1 is a direct consequence of the prime factorization and the bound $p_i\ge 2$.
- Lemma 851.2 is purely set-theoretic.
- Lemma 851.3 correctly uses Lemma 851.1 to certify $\omega(n)\le r$ for $n\le 2^r$.
- The computational checks depend on the interpretation of “prime divisors” ($\omega$ vs $\Omega$); both were tested.

6) FINAL

**UNRESOLVED**
(i) Strongest fully proved partial result: The representable sets $S_r$ are nested in $r$ (Lemma 851.2). Every integer within distance $2^r$ above a power of two is representable with an $n$ having at most $r$ prime factors (Lemma 851.3). Computations for $N\le 2\cdot 10^6$ suggest very high coverage already for small $r$ (especially for the $\omega$ interpretation), but this is only numerical evidence.
(ii) First gap (crisp): Prove (or disprove) that for every $\epsilon>0$ there exists a fixed $r=r(\epsilon)$ such that the lower density of $S_r$ is at least $1-\epsilon$.
(iii) Top 3 next moves:
  1. Develop a sieve estimate for the probability that $m-2^k$ has $\omega\le r$ for at least one $k\le \log_2 m$, with uniform control in $m$; attempt a second-moment or union-bound-with-correlation-control argument.
  2. Try to prove that for most $m$, the multiset $\{m-2^k: 2^k<m\}$ contains a “typical” integer (in terms of prime-factor counts), and quantify how $\omega$ behaves under this family of shifts.
  3. On the disproof side, search for structured obstructions: sets of $m$ for which all $m-2^k$ are forced to have many prime factors (e.g. by congruence constraints mod many small primes), and test whether such obstructions can have positive density.
(iv) Minimal counterexample structure (if false): For some fixed $r$, a positive-density set of integers $m$ such that for every $k$ with $2^k<m$, the difference $m-2^k$ has more than $r$ prime divisors (in the chosen sense). Such $m$ would have to avoid being within $2^r$ of any power of two (by Lemma 851.3) and would need to enforce “many-prime-factor” behavior simultaneously across about $\log_2 m$ different shifts.
