% Erdos Problem #122
% Attempt for Erdos Problem #122
% Following PROMPT_STRATEGY.MD
% Tools/Constraints:
% - Web browsing available? YES (not used; only facts explicitly stated in the problem text)
% - Computation available (Python)? YES (not used)

\section*{Erd\H{o}s Problem \#122}

\subsection*{1) FORMAL RESTATEMENT}
Let $f:\mathbb N\to\mathbb N$ be an arithmetic function.  Consider any function $F:\mathbb N\to(0,\infty)$ such that
\[
\frac{f(n)}{F(n)}\to 0\quad\text{for almost all }n
\]
(i.e. for all $\varepsilon>0$ the set $\{n: f(n)/F(n)>\varepsilon\}$ has asymptotic density $0$).

Question: for which functions $f$ is it true that there exist infinitely many $x\in\mathbb N$ such that
\[
\frac{\#\{n\in\mathbb N:\ n+f(n)\in (x,x+F(x))\}}{F(x)}\to\infty\ ?
\]
(Here $(x,x+F(x))$ denotes the open interval in $\mathbb R$; equivalently we count those $n$ with $x<n+f(n)<x+F(x)$.)

\subsection*{2) QUICK LITERATURE/CONTEXT CHECK}
From the problem text:
\begin{itemize}
\item Erd\H{o}s--Pomerance--S\'ark\"ozy proved the property holds for $f(n)=d(n)$ (divisor function) and for $f(n)=\omega(n)$ (number of distinct prime factors).
\item Erd\H{o}s believed it is false for $f(n)=\varphi(n)$ or $f(n)=\sigma(n)$.
\end{itemize}
No further sources are used here.

\subsection*{3) ATTACK PLAN}
Try to identify simple necessary obstructions: classes of functions $f$ for which the conclusion cannot hold for \emph{some} admissible choice of $F$.  I give two such obstructions (bounded $f$, and eventually nondecreasing $f$), which already show the property is highly nontrivial and requires strong oscillation.

\subsection*{4) WORK}
\paragraph{Lemma 122.1 (bounded $f$ cannot satisfy the conclusion).}
Suppose $f$ is bounded: $f(n)\le B$ for all $n$.  Then $f$ \emph{does not} have the stated property.

\emph{Proof.}
Choose any function $F(n)\to\infty$ (e.g. $F(n)=n$).  Then $f(n)/F(n)\to0$ for all $n$ and hence ``for almost all $n$''.

For any $x$, if $n+f(n)\in (x,x+F(x))$, then necessarily $n\in (x-B,\,x+F(x))$.  Therefore
\[
\#\{n:\ n+f(n)\in(x,x+F(x))\}\le \#\big(\mathbb N\cap(x-B,\,x+F(x))\big)\le F(x)+B+1.
\]
Dividing by $F(x)$ yields
\[
\frac{\#\{n:\ n+f(n)\in(x,x+F(x))\}}{F(x)}\le 1+\frac{B+1}{F(x)}\to 1.
\]
In particular, this ratio cannot tend to $\infty$ along any sequence of $x$.  Thus bounded $f$ fails the required property. \qed

\paragraph{Lemma 122.2 (eventually nondecreasing $f$ cannot satisfy the conclusion).}
Suppose $f$ is eventually nondecreasing: there exists $N_0$ such that $f(n+1)\ge f(n)$ for all $n\ge N_0$.  Then $f$ does \emph{not} have the stated property.

\emph{Proof.}
Define $g(n):=n+f(n)$.  For $n\ge N_0$, we have
\[
 g(n+1)-g(n)=(n+1+f(n+1))-(n+f(n))=1+(f(n+1)-f(n))\ge 1.
\]
Hence $g$ is strictly increasing on $\{N_0,N_0+1,\dots\}$.

Fix any admissible $F$ with $f(n)/F(n)\to0$ for almost all $n$.  For any $x$, the set
\(
\{n\ge N_0:\ g(n)\in(x,x+F(x))\}
\)
consists of integers $n$ whose strictly increasing images lie in an interval of length $F(x)$.  Since $g$ increases by at least $1$ each step, there can be at most $\lceil F(x)\rceil$ such $n$ (more formally: the values $g(n)$ are distinct integers separated by $\ge1$, so an open interval of length $F(x)$ contains at most $\lceil F(x)\rceil$ of them).
Thus for all sufficiently large $x$,
\[
\#\{n:\ g(n)\in(x,x+F(x))\}\le F(x)+1.
\]
Dividing by $F(x)$ gives a uniform bound
\[
\frac{\#\{n:\ n+f(n)\in(x,x+F(x))\}}{F(x)}\le 1+\frac{1}{F(x)}\le 2
\]
for all large $x$.  Hence the ratio cannot tend to $\infty$.
Therefore any eventually nondecreasing $f$ fails the property. \qed

\subsection*{5) VERIFICATION (FAST REALITY CHECK)}
\begin{itemize}
\item Lemma~122.1 applies to $f(n)\equiv 1$ (bounded).  Indeed $n+1$ hits any interval of length $F(x)$ in about $F(x)$ ways, ratio $\asymp 1$.
\item Lemma~122.2 applies to $f(n)=n$ or $f(n)=n^2$ (eventually increasing).  The map $n\mapsto n+f(n)$ is strictly increasing, so the count in any length-$F(x)$ interval is $O(F(x))$ and the normalized ratio is bounded.
\end{itemize}

\subsection*{6) FINAL}
\textbf{UNRESOLVED.}

(i) \emph{Strongest fully proved partial result obtained here.}
Two rigorous obstruction lemmas: bounded $f$ cannot satisfy the property (Lemma~122.1), and eventually nondecreasing $f$ cannot satisfy it (Lemma~122.2).

(ii) \emph{Exact first gap.}
Beyond these obstructions, characterize which highly oscillatory arithmetic functions (such as $d(n)$ or $\omega(n)$) do satisfy the ``local repetition'' phenomenon for every admissible $F$.

(iii) \emph{Top 3 next moves.}
\begin{enumerate}
\item Try to prove a sufficient condition in terms of distribution of values of $f(n)$ in short intervals (e.g. many repeats) that forces the normalized count to become large.
\item Test the conjectured negative cases $\varphi(n)$ and $\sigma(n)$ by attempting to construct an explicit admissible $F$ that keeps the normalized ratio bounded.
\item Explore probabilistic models: treat $f(n)$ as a random-like multiplicative function and estimate clustering of $n+f(n)$.
\end{enumerate}

(iv) \emph{Minimal counterexample structure.}
If Erd\H{o}s' belief is correct for $\varphi$ or $\sigma$, a minimal counterexample would be a specific admissible $F$ such that, along all sequences of $x\to\infty$, the interval $(x,x+F(x))$ contains only $O(F(x))$ values of $n+f(n)$ with a uniformly bounded proportionality constant.
