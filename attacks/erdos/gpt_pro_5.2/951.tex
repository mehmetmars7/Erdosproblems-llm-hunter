
\section*{Erd\H{o}s Problem \#951}

\subsection*{FORMAL RESTATEMENT}
Let $(a_i)_{i\ge1}$ be a strictly increasing sequence of real numbers with $a_1>1$.
Assume that for every pair of distinct finitely supported tuples $k=(k_i)_{i\ge1}$ and $\ell=(\ell_i)_{i\ge1}$ of nonnegative integers,
\[
\left|\prod_i a_i^{k_i} - \prod_i a_i^{\ell_i}\right|\ge 1.
\]
Let $A(x):=\#\{i: a_i\le x\}$.
Question: is it true that $A(x)\le \pi(x)$ for all $x\ge1$?

\subsection*{QUICK LITERATURE/CONTEXT CHECK}
The problem text calls such a sequence a set of Beurling prime numbers and mentions (without proof here) a conjecture of Beurling about the counting of multiplicative semigroup elements.
I will not use any external results.

\subsection*{ATTACK PLAN}
\begin{itemize}
\item \textbf{Reality check:} understand immediate consequences of the spacing condition, in particular constraints on the $a_i$ themselves.
\item \textbf{Prove unconditional lemmas:} (a) injectivity of the exponent map; (b) $a_i\ge 2$ and $|a_i-a_j|\ge1$; (c) trivial bound $A(x)\le x-1$.
\item \textbf{Special-case proof attempt:} show the conjectured bound $A(x)\le \pi(x)$ holds if the $a_i$ are integers (or rationals), using prime-factor exponent vectors and linear independence.
\item \textbf{Stop:} the full real-valued problem seems to require new ideas.
\end{itemize}

\subsection*{WORK}
\noindent\textbf{Fast reality check (tiny examples).}
\begin{itemize}
\item The usual primes $a_i=p_i$ satisfy the condition because distinct prime-power products are distinct integers, hence differ by at least $1$.
\item The sequence $a_1=2$, $a_2=4$ fails: $4=2^2$ gives difference $0$ between tuples $(0,1,0,\dots)$ and $(2,0,0,\dots)$.
\end{itemize}

\medskip
\noindent\textbf{Lemma 951.1 (basic consequences of the spacing hypothesis).}
Under the hypothesis:
\begin{enumerate}
\item For each $i$, $a_i\ge 2$.
\item For $i\ne j$, $|a_i-a_j|\ge 1$.
\item The map $\Phi$ from finitely supported $k\in\mathbb Z_{\ge0}^{(\mathbb N)}$ to $\prod_i a_i^{k_i}$ is injective.
\end{enumerate}

\medskip
\noindent\emph{Proof.}
(1) Compare the tuple $k$ with $k_i=1$ and $k_{m\ne i}=0$ to the zero tuple $\ell\equiv0$.
Then the inequality gives $|a_i-1|\ge1$. Since $a_i>1$, it follows $a_i\ge2$.

(2) Compare the tuple with one $1$ at position $i$ to the tuple with one $1$ at position $j$.
Then $|a_i-a_j|\ge1$.

(3) If $\Phi(k)=\Phi(\ell)$ for distinct $k\ne\ell$, then the left-hand side of the hypothesis is $0$, contradicting $\ge1$.
\hfill $\square$

\medskip
\noindent\textbf{Lemma 951.2 (trivial growth bound).}
For every $x\ge2$, one has $A(x)\le \lfloor x\rfloor-1$.

\medskip
\noindent\emph{Proof.}
By Lemma 951.1(1), every $a_i\ge2$.
By Lemma 951.1(2), the set $\{a_i\}$ is $1$-separated.
Thus in the interval $[2,x]$ one can fit at most $\lfloor x\rfloor-1$ many points with pairwise distance at least $1$.
Formally, order the $a_i\le x$ as $b_1<\cdots<b_m$. Then $b_m-b_1\ge (m-1)\cdot 1$; since $b_1\ge2$ and $b_m\le x$, we have $x-2\ge m-1$, so $m\le x-1$ and hence $m\le\lfloor x\rfloor-1$.
\hfill $\square$

\medskip
\noindent\textbf{Lemma 951.3 (the conjectured bound holds for integer sequences).}
Assume in addition that each $a_i$ is an integer $\ge 2$.
Then for every real $x\ge2$,
\[
A(x)=\#\{i: a_i\le x\}\le \pi(x).
\]

\medskip
\noindent\emph{Proof.}
Fix $x\ge2$ and let $I:=\{i: a_i\le x\}$, so $|I|=A(x)$.
For each $i\in I$, write the prime factorization
\[
a_i=\prod_{p\le x} p^{v_p(a_i)},
\]
where $v_p(a_i)\in\mathbb Z_{\ge0}$ is the usual $p$-adic valuation.
Define the exponent vector
\[
v(i):=(v_p(a_i))_{p\le x}\in \mathbb Z_{\ge0}^{\pi(x)}.
\]

\emph{Claim:} The vectors $\{v(i): i\in I\}$ are linearly independent over $\mathbb Z$.
Indeed, suppose there is an integer relation
\[
\sum_{i\in I} c_i\, v(i)=0\quad\text{in }\mathbb Z^{\pi(x)},
\]
with not all $c_i$ zero.
Write $c_i=c_i^+-c_i^-$ with $c_i^+,c_i^-\in\mathbb Z_{\ge0}$ and $c_i^+c_i^-=0$.
Then the vector equation implies (componentwise for each prime $p\le x$)
\[
\sum_i c_i^+ v_p(a_i)=\sum_i c_i^- v_p(a_i),
\]
which in turn implies equality of integers
\[
\prod_{i\in I} a_i^{c_i^+}=\prod_{i\in I} a_i^{c_i^-}.
\]
The exponent tuples $(c_i^+)_{i\in I}$ and $(c_i^-)_{i\in I}$ are distinct because some $c_i\ne0$.
This contradicts Lemma 951.1(3), which says distinct exponent tuples give distinct products.
So no nontrivial integer relation exists, proving the claim.

Now the vectors $v(i)$ lie in the free abelian group $\mathbb Z^{\pi(x)}$ of rank $\pi(x)$.
A set of $\mathbb Z$-linearly independent vectors in $\mathbb Z^{\pi(x)}$ has size at most the rank, i.e. at most $\pi(x)$.
Therefore $A(x)=|I|\le \pi(x)$.
\hfill $\square$

\subsection*{VERIFICATION}
\begin{itemize}
\item Lemma 951.1: checked that comparing exponent tuples with single $1$ vs all $0$ forces $a_i\ge2$; comparing different singletons forces $|a_i-a_j|\ge1$.
\item Lemma 951.2: verified inequality $b_m-b_1\ge m-1$ for ordered $1$-separated points.
\item Lemma 951.3: the key step is that any nontrivial integer dependence among prime-exponent vectors yields an equality between two distinct monomials in the $a_i$, forbidden by the spacing hypothesis.
\end{itemize}

\subsection*{FINAL}
**UNRESOLVED**

(i) \textbf{Strongest proved partial result.}
We proved $a_i\ge2$ and the $a_i$ are pairwise $1$-separated, giving the trivial bound $A(x)\le x-1$ (Lemmas 951.1--951.2). If the $a_i$ are integers, then the desired inequality $A(x)\le\pi(x)$ holds (Lemma 951.3).

(ii) \textbf{First gap.}
Extend Lemma 951.3 from integer sequences to arbitrary real sequences satisfying the spacing condition, i.e. find an invariant replacing prime-factor exponent vectors that forces $A(x)\le\pi(x)$.

(iii) \textbf{Top 3 next moves.}
\begin{enumerate}
\item Translate the condition into a packing problem in $\mathbb R$ for the multiplicative semigroup $\{\prod a_i^{k_i}\}$: it is $1$-separated in the Euclidean metric. Seek entropy/geometry-of-numbers bounds relating the number of generators $\le x$ to how many semigroup elements fit below $x$.
\item Examine whether one can force many semigroup elements $\le x$ from having too many generators $\le x$, contradicting the $1$-separation.
\item Search for explicit constructions with $A(x) > \pi(x)$ (to disprove) by choosing reals with carefully spaced logarithms and checking the $1$-separation numerically for initial segments.
\end{enumerate}

(iv) \textbf{What a minimal counterexample would likely look like.}
A counterexample would require a real sequence $(a_i)$ with many terms $\le x$ (significantly more than $x/\log x$) while keeping \emph{all} distinct products $\prod a_i^{k_i}$ at distance $\ge1$ from each other. This would likely involve very sparse multiplicative interactions (strong multiplicative independence) and rapidly growing gaps to prevent near-collisions of large products.


