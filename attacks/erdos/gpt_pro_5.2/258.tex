\section*{Erd\H{o}s Problem \#258}

\subsection*{1) FORMAL RESTATEMENT}
Let $(a_n)_{n\ge1}$ be a sequence of integers with $a_n\to +\infty$ as $n\to\infty$.
(Interpreting the statement literally requires $a_n\neq0$ for all $n$; in practice one assumes $a_n\in\mathbb{N}$ and hence $a_n\ge 1$ with $a_n\ge2$ eventually.)
Define
\[
T(a_1,a_2,\dots):=\sum_{n=1}^{\infty}\frac{\tau(n)}{a_1a_2\cdots a_n},
\]
where $\tau(n)$ is the divisor-counting function.
The problem asks whether
\[
(\ddagger)\qquad \forall (a_n)\subset\mathbb{Z}\text{ with }a_n\to\infty:\ T(a_1,a_2,\dots)\notin\mathbb{Q}.
\]

\subsection*{2) QUICK LITERATURE/CONTEXT CHECK}
\begin{itemize}[leftmargin=2.2em]
\item Erd\H{o}s--Straus (1971) proved $(\ddagger)$ holds if $(a_n)$ is monotone nondecreasing (i.e. $a_{n-1}\le a_n$ for all $n$) \cite{ErSt71}.
\item Erd\H{o}s (1948) proved that for every integer $t\ge2$, the Lambert series $\sum_{n\ge1} d(n)/t^n$ is irrational \cite{Er48}; this is related but does not directly cover $a_n\to\infty$.
\item As of late 2025/early 2026 the unrestricted case (no monotonicity) remains listed as open on the Erd\H{o}s Problems website \cite{ErP258}.
\end{itemize}

\subsection*{3) ATTACK PLAN}
\begin{enumerate}[leftmargin=2.2em]
\item Recognise $T(a_1,a_2,\dots)$ as a Cantor-series type expansion with ``bases'' $a_n$ and ``digits'' $\tau(n)$.
\item Try two directions:
  \begin{itemize}
  \item Extend the Erd\H{o}s--Straus proof from monotone $a_n$ to general $a_n\to\infty$.
  \item Attempt to construct a counterexample sequence $(a_n)$ forcing $T$ to be rational (e.g. via telescoping or mixed-radix periodicity).
  \end{itemize}
\end{enumerate}
I did not succeed in either direction.

\subsection*{4) WORK}
\paragraph{Convergence sanity check.}
Since $a_n\to\infty$, there exists $N$ such that $a_n\ge2$ for all $n\ge N$. Then for $n\ge N$,
\[
\left\lvert a_1\cdots a_n\right\rvert\ge \left\lvert a_1\cdots a_N\right\rvert\,2^{n-N}.
\]
Also $\tau(n)\le n$ for all $n$. Hence the tail is bounded by a convergent series:
\[
\sum_{n\ge N}\frac{\tau(n)}{\left\lvert a_1\cdots a_n\right\rvert}
\le \frac{1}{\left\lvert a_1\cdots a_N\right\rvert}\sum_{n\ge N}\frac{n}{2^{n-N}}<\infty.
\]
So $T(a_1,a_2,\dots)$ is well-defined.

\paragraph{Tiny numerical experiment (illustrative only).}
For the monotone sequence $a_n=n+1$ (which satisfies $a_n\to\infty$), the first $50$ terms give
\[
\sum_{n=1}^{50}\frac{\tau(n)}{2\cdot 3\cdots (n+1)}\approx 0.94529965343349825190\dots
\]
This is consistent with convergence and with irrationality, but of course gives no proof.

\paragraph{Known conditional resolution under monotonicity.}
Erd\H{o}s--Straus \cite{ErSt71} prove that if $a_1\le a_2\le\cdots$ and $a_n\to\infty$, then $T(a_1,a_2,\dots)$ is irrational.
The present problem is to remove the monotonicity hypothesis.

\paragraph{A tempting (but invalid here) telescoping construction.}
If one could take $a_n=\tau(n)+1$, then
\[
\frac{\tau(n)}{\prod_{j\le n}(\tau(j)+1)}
=\frac{1}{\prod_{j\le n-1}(\tau(j)+1)}-\frac{1}{\prod_{j\le n}(\tau(j)+1)},
\]
so the series would telescope to $1$.
However $\tau(n)$ does \emph{not} tend to $\infty$ (it is merely unbounded), so $a_n=\tau(n)+1$ fails the hypothesis $a_n\to\infty$.
This illustrates why some growth/regularity condition on $a_n$ may be necessary to rule out rationality.

\subsection*{5) VERIFICATION}
\begin{itemize}[leftmargin=2.2em]
\item The convergence argument uses only $a_n\ge2$ eventually and the trivial bound $\tau(n)\le n$.
\item The telescoping observation is algebraically correct, and it explains the remark (seen in Erd\H{o}s--Straus) that for other arithmetic functions (e.g. $\phi$ or $\sigma$) one \emph{can} force a rational sum by choosing $a_n=\phi(n)+1$ or $\sigma(n)+1$; for $\tau(n)$ this exact choice does not satisfy $a_n\to\infty$.
\end{itemize}

\subsection*{6) FINAL}
\noindent \textbf{UNRESOLVED}

\medskip
\noindent (i) \textbf{Strongest fully proved partial result obtained here.}
The series $T(a_1,a_2,\dots)$ converges for every integer sequence with $a_n\to\infty$ (proved above). The conjectured irrationality is known to hold when $(a_n)$ is monotone nondecreasing by Erd\H{o}s--Straus \cite{ErSt71}.

\smallskip
\noindent (ii) \textbf{First gap.}
Removing monotonicity breaks the existing proof strategy: one needs a way to control carries / mixed-radix expansions or to adapt divisor-function congruence arguments without assuming $a_{n-1}\le a_n$.

\smallskip
\noindent (iii) \textbf{Top 3 next moves.}
\begin{enumerate}[leftmargin=2.2em]
\item Re-express $T$ as a Cantor series and apply (or sharpen) known rationality criteria for Cantor-series expansions with varying bases $a_n$, attempting to exploit that $\tau(n)$ is ``not too large'' compared to $a_n\to\infty$.
\item Search for a counterexample sequence $(a_n)$ by imposing congruences on $a_n$ (and hence on partial products) that might force eventual periodicity of the expansion.
\item Attempt to weaken monotonicity to a growth condition (e.g. $a_{n+1}\ge a_n^{1+\delta}$ often enough) and see whether Erd\H{o}s--Straus methods extend; this might provide intermediate theorems.
\end{enumerate}

\smallskip
\noindent (iv) \textbf{Minimal counterexample structure.}
A counterexample would be a sequence $a_n\to\infty$ with $\sum_n \tau(n)/(a_1\cdots a_n)\in\mathbb{Q}$. Because the terms are positive, such a sequence would have to enforce exact mixed-radix ``eventual periodicity'' or telescoping/cancellation at the level of partial products, without violating $a_n\to\infty$. Any such construction would likely require very non-monotone behaviour (since monotone sequences are known to give irrationality).

\subsection*{7) COMPLETION ESTIMATE (MANDATORY)}
\noindent COMPLETION: 30\%.

%%%%%%%%%%%%%%%%%%%%%%%%%%%%%%%%%%%%%%%%%%%%%%%%%%%%%%%%%%%%%%%%%%%%%%%%%%%%%%%
\begin{thebibliography}{99}

\bibitem{ErSz59}
P.~Erd\H{o}s and G.~Szekeres,
\emph{On the product $\prod_{k=1}^n(1-z^{a_k})$},
Publ. Math. Inst. Hung. Acad. Sci. \textbf{4} (1959), 29--34.

\bibitem{At61}
F.~V. Atkinson,
\emph{On a problem of Erd\H{o}s and Szekeres},
Canad. Math. Bull. \textbf{4} (1961), 7--12.

\bibitem{Od82}
A.~Odlyzko,
\emph{Minima of cosine sums and maxima of polynomials on the unit circle},
J. London Math. Soc. \textbf{26} (1982), 412--420.

\bibitem{BoCh18}
J.~Bourgain and M.-C. Chang,
\emph{On a paper of Erd\H{o}s and Szekeres},
J. Anal. Math. \textbf{136} (2018), 253--271.

\bibitem{BeKo96}
A.~S. Belov and S.~V. Konyagin,
\emph{An estimate of the free term of a non-negative trigonometric polynomial with integer coefficients},
Izvestiya Math. \textbf{60} (1996), 1123--1182.

\bibitem{Tang25}
Q.~Tang,
\emph{An improved lower bound for Erd\H{o}s--Szekeres products},
arXiv:2509.14182 (2025).

\bibitem{Er48}
P.~Erd\H{o}s,
\emph{On arithmetical properties of Lambert series},
J. Indian Math. Soc. (N.S.) (1948), 63--66.

\bibitem{Er68d}
P.~Erd\H{o}s,
\emph{Some problems and results on irrationality},
Math. Student \textbf{36} (1968), 222--226.

\bibitem{Er88c}
P.~Erd\H{o}s,
\emph{On the irrationality of certain series: problems and results},
in \emph{New Advances in Transcendence Theory} (Cambridge University Press, 1988).

\bibitem{KoTa24}
V.~Kova\v{c} and T.~Tao,
\emph{On several irrationality problems for Ahmes series},
arXiv:2406.17593 (2024; version cited on arXiv as of 2025).

\bibitem{TaTe25}
T.~Tao and J.~Ter\"{a}v\"{a}inen,
\emph{Quantitative correlations and some problems on prime factors of consecutive integers},
arXiv:2512.01739 (2025).

\bibitem{ErSt71}
P.~Erd\H{o}s and E.~G. Straus,
\emph{Some number theoretic results},
Pacific J. Math. \textbf{36} (1971), 635--646.

\bibitem{ErP258}
T.~F. Bloom (ed.),
\emph{Erd\H{o}s Problem \#258 (website summary, accessed 2026-01-17)},
\url{https://www.erdosproblems.com/258}.

\end{thebibliography}

