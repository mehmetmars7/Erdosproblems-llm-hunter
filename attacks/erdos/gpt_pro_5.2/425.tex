% Erdos Problem #425
% URL: https://www.erdosproblems.com/425

FORMAL RESTATEMENT
(Notation correction.) The statement defines $F(n)$ using $\{1,\dots,N\}$ but then uses $n$ elsewhere; we interpret this as $A\subseteq\{1,\dots,n\}$.

For $n\in\mathbb N$, let $F(n)$ be the maximum possible size of a set $A\subseteq\{1,\dots,n\}$ with the property:
\[
\text{for all distinct }a,b,c,d\in A,\ \{a,b\}\ne\{c,d\}\implies ab\ne cd.
\]
Equivalently, the map $\{a,b\}\mapsto ab$ from unordered pairs of distinct elements of $A$ is injective.

Question 1: Does there exist a constant $c$ such that
\[
F(n)=\pi(n)+\bigl(c+o(1)\bigr)\,n^{3/4}(\log n)^{-3/2}\quad (n\to\infty)?
\]
The problem statement records Erd\H{o}s’s bounds: there exist constants $0<c_1\le c_2$ with
\[
\pi(n)+c_1 n^{3/4}(\log n)^{-3/2}\le F(n)\le \pi(n)+c_2 n^{3/4}(\log n)^{-3/2}.
\]

Question 2 (higher products): Fix an integer $r\ge 2$. If $A\subseteq\{1,\dots,n\}$ is such that for every two distinct increasing $r$-tuples $(a_1<\cdots<a_r)$ and $(b_1<\cdots<b_r)$ from $A$ we have
\[
 a_1\cdots a_r\ne b_1\cdots b_r,
\]
is it true that
\[
|A|\le \pi(n)+O\bigl(n^{(r+1)/(2r)}\bigr)?
\]

QUICK LITERATURE/CONTEXT CHECK
The bounds with $n^{3/4}(\log n)^{-3/2}$ and the existence of some constants $c_1,c_2$ are explicitly stated in the problem text. No other external results are used here.

ATTACK PLAN
Proof track ideas.
1) Reprove (or partially reprove) upper bounds via counting products and factoring by small/large primes (graph method).
2) Build explicit constructions beyond primes (lower bounds) by adding carefully chosen composites while maintaining product-uniqueness.

Disproof track ideas.
1) Search for oscillations in $F(n)-\pi(n)$ that would preclude a single constant $c$ in the main term.
2) For the $r$-fold version, attempt to construct sets violating the claimed exponent.

WORK
Fast reality check (exact values for small $n$ by brute force).
A complete backtracking search for $n\le 20$ gives the exact values:
\[
\begin{array}{c|cccccccccccccccccccc}
 n&1&2&3&4&5&6&7&8&9&10&11&12&13&14&15&16&17&18&19&20\\\hline
 F(n)&1&2&3&4&5&5&6&6&7&7&8&9&10&10&10&11&12&12&13&13
\end{array}
\]
For example, for $n=20$ one maximal set is
\[
A=\{1,2,4,6,7,9,11,13,15,16,17,19,20\}
\]
of size $13$.

Lemma 425.1 (prime-based lower bound).
For every $n\ge 2$,
\[
F(n)\ge \pi(n)+1.
\]

Proof.
Let $A:=\{1\}\cup\{p\le n: p\text{ is prime}\}$. Then $|A|=\pi(n)+1$.
We show that all products $ab$ for distinct $a,b\in A$ are distinct.
Consider two unordered pairs $\{a,b\}$ and $\{c,d\}$ of distinct elements of $A$ with $ab=cd$.
We must show $\{a,b\}=\{c,d\}$.

If one of $a,b$ equals $1$, then $ab$ is the other element, which is prime. Thus $ab$ is prime, so $cd$ is prime as well. A product of two distinct elements of $A$ is prime only if one factor is $1$ and the other is that prime. Therefore both pairs must be of the form $\{1,p\}$ with the same prime $p$, hence equal.

Otherwise, $a,b,c,d$ are primes. Then $ab$ and $cd$ are products of two primes. By unique prime factorisation in $\mathbb N$, the multiset of prime factors of $ab$ is uniquely determined, so the multiset $\{a,b\}$ equals the multiset $\{c,d\}$, i.e. $\{a,b\}=\{c,d\}$.
Thus the pair-product map is injective on $A$, so $A$ is admissible and $F(n)\ge |A|=\pi(n)+1$.
\hfill $\square$

Lemma 425.2 (trivial counting upper bound).
For any admissible $A\subseteq\{1,\dots,n\}$,
\[
|A|\le \frac{1+\sqrt{1+8n^2}}{2}.
\]

Proof.
Let $m:=|A|$. The number of unordered pairs of distinct elements of $A$ is $\binom{m}{2}=m(m-1)/2$.
Each product of a pair is an integer between $1\cdot 2=2$ and $n\cdot (n-1)<n^2$, hence at most $n^2$.
Since the products are all distinct by assumption, we must have
\[
\binom{m}{2}\le n^2.
\]
That is $m(m-1)\le 2n^2$, equivalently $m^2-m-2n^2\le 0$.
Solving this quadratic inequality for $m\ge 0$ yields
\[
 m\le \frac{1+\sqrt{1+8n^2}}{2}.
\]
\hfill $\square$

VERIFICATION
- Lemma 425.1: the only nontrivial input is unique prime factorisation in $\mathbb N$ and the observation that a prime cannot factor as a product of two integers $>1$.
- Lemma 425.2: uses only that pair products lie in $\{1,2,\dots,n^2\}$ and are distinct.
- Brute force check: the backtracking search explicitly checked the injectivity of the pair-product map for all candidate sets for $n\le 20$.

FINAL
**UNRESOLVED**
(i) Strongest proved partial result: elementary bounds $\pi(n)+1\le F(n)\le (1+\sqrt{1+8n^2})/2$ (Lemmas 425.1–425.2). The problem statement also records much sharper bounds of order $\pi(n)+\Theta\bigl(n^{3/4}(\log n)^{-3/2}\bigr)$, but no new proof is given here.
(ii) First gap (crisp): determine whether $F(n)-\pi(n)$ has an asymptotic main term $c\,n^{3/4}(\log n)^{-3/2}$ with a single constant $c$.
(iii) Top 3 next moves:
  1. Re-derive the $n^{3/4}(\log n)^{-3/2}$ upper bound from scratch with explicit constants, then see whether the argument can be sharpened to a limit constant.
  2. Constructive side: build explicit admissible sets of size $\pi(n)+\text{(large)}$ by adding composites with controlled prime factor patterns, and track the exact number of added elements.
  3. Computation: compute or tightly bound $F(n)$ for substantially larger $n$ (heuristics/ILP/branch-and-bound) to see whether $(F(n)-\pi(n))\,(\log n)^{3/2}/n^{3/4}$ stabilises.
(iv) Minimal counterexample structure: if a constant $c$ does not exist, then along two subsequences $n_j$ and $m_j$ the normalised excess
\[
\frac{(F(n)-\pi(n))(\log n)^{3/2}}{n^{3/4}}
\]
would have two different limit points; a minimal obstruction would therefore be a structural phenomenon (e.g. dependence on the distribution of primes in certain size ranges) that forces persistent fluctuations larger than $o(1)$ after normalisation.


