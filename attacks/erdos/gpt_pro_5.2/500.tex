
1) FORMAL RESTATEMENT

For an integer $n\ge 1$, let $H=(V,E)$ be a \emph{$3$-uniform hypergraph} on $\lvert V\rvert=n$, meaning $E\subseteq \binom{V}{3}$. Let $K_4^3$ denote the $3$-uniform complete hypergraph on $4$ vertices, i.e. a $4$-vertex set $S\subseteq V$ for which \emph{all} $\binom{4}{3}=4$ triples in $\binom{S}{3}$ belong to $E$. Define
\[
\mathrm{ex}_3(n,K_4^3)\;:=\;\max\{\lvert E\rvert:\; H\text{ is a $3$-uniform hypergraph on $n$ vertices containing no }K_4^3\}.
\]
The problem asks for $\mathrm{ex}_3(n,K_4^3)$ (exactly, or at least asymptotically as $n\to\infty$).

Edge cases: for $n\le 3$ there is no $4$-set, hence every $3$-graph is $K_4^3$-free and $\mathrm{ex}_3(n,K_4^3)=\binom{n}{3}$.

2) QUICK LITERATURE/CONTEXT CHECK

The problem statement records Tur\'an's construction giving
\[\mathrm{ex}_3(n,K_4^3)\ge \left(\frac{5}{9}+o(1)\right)\binom{n}{3},\]
and an upper bound $\mathrm{ex}_3(n,K_4^3)\le 0.5611666\binom{n}{3}$ (Razborov, as quoted in the problem file). I do not use any external results beyond what is explicitly stated there.

3) ATTACK PLAN

\emph{Proof track:} verify Tur\'an's construction is $K_4^3$-free and count its edges (gives a rigorous lower bound). Then sanity-check small $n$ by exhaustive computation.

\emph{Disproof track:} try to beat density $5/9$ by small-$n$ search and/or by perturbing the 3-part construction; look for a $K_4^3$-free configuration with higher density at small $n$.

4) WORK

Definition (Tur\'an 3-part construction).
Fix a partition $V=X_1\sqcup X_2\sqcup X_3$ (indices modulo $3$, so $X_4=X_1$). Define a $3$-graph $T(V)$ with edge set $E(T)$ consisting of all triples $\{a,b,c\}\subseteq V$ of either type:

(1) \emph{(1,1,1)-type:} one vertex in each $X_1,X_2,X_3$; or

(2) \emph{cyclic (2,1)-type:} two vertices in $X_i$ and one vertex in $X_{i+1}$ for some $i\in\{1,2,3\}$.

Lemma 500.1 (The Tur\'an construction is $K_4^3$-free).
For every partition $V=X_1\sqcup X_2\sqcup X_3$, the $3$-graph $T(V)$ contains no copy of $K_4^3$.

\emph{Proof.}
Let $S\subseteq V$ be any $4$-element set. We show that at least one of the four triples in $\binom{S}{3}$ is \emph{not} an edge of $T(V)$.

Let $(a_1,a_2,a_3)$ be the occupancy numbers where $a_i:=\lvert S\cap X_i\rvert$ so $a_1+a_2+a_3=4$.

\underline{Case 1: $(4,0,0)$ up to permutation.}
Then some part, say $X_1$, contains all four vertices. Every triple in $\binom{S}{3}$ lies entirely in $X_1$, hence is of type $(3,0,0)$ and is \emph{not} allowed by the definition of $E(T)$ (which allows only types $(1,1,1)$ and $(2,1,0)$ with a cyclic orientation). Therefore $S$ does not span $K_4^3$.

\underline{Case 2: $(3,1,0)$ up to permutation.}
Say $S$ has three vertices in $X_1$ and one in $X_2$. Then the triple consisting of the three vertices in $X_1$ is again of type $(3,0,0)$ and is not an edge. Hence $S$ does not span $K_4^3$.

\underline{Case 3: $(2,2,0)$ up to permutation.}
Say $S$ has two vertices in $X_1$ and two in $X_2$. The four triples in $\binom{S}{3}$ consist of:
- two triples of type $(2\text{ in }X_1,\;1\text{ in }X_2)$, and
- two triples of type $(1\text{ in }X_1,\;2\text{ in }X_2)$.
By definition, $E(T)$ includes triples with two vertices in $X_i$ and one in $X_{i+1}$, but \emph{excludes} triples with two vertices in $X_{i+1}$ and one in $X_i$.
Thus, among these four triples, exactly one of the two types can be present (depending on the cyclic direction), and the other type is absent. Hence $S$ does not span $K_4^3$.

\underline{Case 4: $(2,1,1)$ up to permutation.}
Say $S$ has two vertices in $X_1$, one vertex in $X_2$, and one vertex in $X_3$.
Label the two vertices in $X_1$ by $u,v$, the vertex in $X_2$ by $w$, and the vertex in $X_3$ by $z$.
The four triples in $\binom{S}{3}$ are
\[\{u,v,w\},\ \{u,v,z\},\ \{u,w,z\},\ \{v,w,z\}.
\]
The last two triples $\{u,w,z\}$ and $\{v,w,z\}$ have one vertex in each part, hence are of type $(1,1,1)$ and therefore belong to $E(T)$.
Among the first two triples, $\{u,v,w\}$ has type $(2\text{ in }X_1,\;1\text{ in }X_2)$, while $\{u,v,z\}$ has type $(2\text{ in }X_1,\;1\text{ in }X_3)$.
Exactly one of $X_2$ or $X_3$ equals $X_{1+1}$ in the cyclic order, so by definition of $E(T)$ exactly one of these two triples is present and the other is absent. Therefore at least one triple among the four is missing, so $S$ does not span $K_4^3$.

These cases exhaust all possibilities for $(a_1,a_2,a_3)$. Hence no $4$-set $S$ spans all four triples, i.e. $T(V)$ is $K_4^3$-free. \qed

Lemma 500.2 (Edge count; asymptotic density $5/9$ for equal parts).
Assume $n=3m$ and $\lvert X_1\rvert=\lvert X_2\rvert=\lvert X_3\rvert=m$. Then
\[
\lvert E(T)\rvert 
= m^3 + 3\binom{m}{2}m
= \frac{m^2}{2}(5m-3)
= \left(\frac{5}{9}+O\left(\frac{1}{n}\right)\right)\binom{n}{3}.
\]

\emph{Proof.}
Count edges by type.

(1) Edges with one vertex in each part: choose one from each part in $m^3$ ways.

(2) Cyclic (2,1)-edges: for each $i\in\{1,2,3\}$, choose $2$ vertices from $X_i$ and $1$ vertex from $X_{i+1}$. This gives $\binom{m}{2}\cdot m$ edges for each $i$, hence $3\binom{m}{2}m$ in total.

Summing gives
\[
\lvert E(T)\rvert = m^3 + 3\binom{m}{2}m = m^3 + 3\cdot \frac{m(m-1)}{2}\cdot m = \frac{m^2}{2}(5m-3).
\]
Also $\binom{n}{3}=\binom{3m}{3} = \frac{(3m)(3m-1)(3m-2)}{6} = \frac{9}{2}m^3 + O(m^2)$.
Therefore
\[
\frac{\lvert E(T)\rvert}{\binom{n}{3}}
= \frac{\frac{5}{2}m^3 + O(m^2)}{\frac{9}{2}m^3 + O(m^2)}
= \frac{5}{9} + O\left(\frac{1}{m}\right)
= \frac{5}{9} + O\left(\frac{1}{n}\right).
\]
\qed

FAST REALITY CHECK (exact small $n$).
I exhaustively searched all $3$-graphs on $n\le 6$ vertices and found:
\[
\mathrm{ex}_3(1,K_4^3)=0,\ \mathrm{ex}_3(2,K_4^3)=0,\ \mathrm{ex}_3(3,K_4^3)=1,\ \mathrm{ex}_3(4,K_4^3)=3,\ \mathrm{ex}_3(5,K_4^3)=7,\ \mathrm{ex}_3(6,K_4^3)=14.
\]
Moreover, the Tur\'an construction achieves $7$ edges at $n=5$ (parts $2,2,1$) and $14$ edges at $n=6$ (parts $2,2,2$), so it is optimal for these $n$.

5) VERIFICATION

\emph{Quantifiers/edge cases:} For $n\le 3$, the definition of $K_4^3$ makes the forbidden configuration impossible, so $\mathrm{ex}_3(n,K_4^3)=\binom{n}{3}$; this matches the computed values $0,0,1$.

\emph{Lemma checks:} Lemma 500.1 enumerates all possible $4$-vertex distributions among three parts; in each case, an explicit missing triple is identified.

\emph{Computation check:} Exhaustive enumeration for $n=6$ checks $2^{\binom{6}{3}}=2^{20}=1{,}048{,}576$ hypergraphs, feasible and exact.

6) FINAL
UNRESOLVED

(i) \emph{Strongest proved partial result here:} Tur\'an's 3-part construction is $K_4^3$-free and has edge density $\frac{5}{9}+O(1/n)$ (Lemmas 500.1--500.2). Exact values for $n\le 6$ are computed and match this construction.

(ii) \emph{First gap (crisp):} Prove that every $K_4^3$-free $3$-graph on $n$ vertices satisfies $\lvert E\rvert \le \left(\frac{5}{9}+o(1)\right)\binom{n}{3}$, i.e. the Tur\'an construction is asymptotically optimal.

(iii) \emph{Top 3 next moves:}
1. Compute $\mathrm{ex}_3(n,K_4^3)$ for $7\le n\le 10$ via ILP/branch-and-bound to test stability and see whether extremals resemble the 3-part construction.
2. Prove a \emph{stability} statement: if a $K_4^3$-free $3$-graph has density $\ge 5/9-\varepsilon$, then its vertex set admits a near-balanced 3-partition with most edges of the Tur\'an types.
3. Derive a clean analytic inequality bounding edge density in terms of the densities of the three parts (a first step toward a purely combinatorial upper bound).

(iv) \emph{Minimal counterexample structure (if $5/9$ is not optimal):} a $K_4^3$-free $3$-graph with density $>5/9$ would likely require a structured partition into three (or more) parts with a modified allowed-type rule, while still avoiding the $(2,1,1)$ obstruction that forced a missing triple in Lemma 500.1.


