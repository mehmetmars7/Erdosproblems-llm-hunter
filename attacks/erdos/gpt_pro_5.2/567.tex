## 1) FORMAL RESTATEMENT

### Conventions

* All graphs are **finite, simple, undirected**.
* For graphs (A,B), the (two–color) **Ramsey number** (R(A,B)) is the least integer (N) such that for every red/blue coloring of the edges of the complete graph (K_N), either the red graph contains a (not-necessarily-induced) copy of (A) or the blue graph contains a copy of (B). Equivalently: for every graph (X) on (N) vertices, either (X) contains (A) or (\overline X) contains (B).
* “(\ll m)” means “(\le C,m)” for some constant (C) depending only on (G), not on (H) or (m).

### The fixed graphs (G)

Let (G) be one of:

* (Q_3): the 3-dimensional cube graph (8 vertices, 12 edges).
* (K_{3,3}): complete bipartite graph with parts of size 3 (6 vertices, 9 edges).
* (H_5): the graph obtained from (C_5) by adding two vertex-disjoint chords; equivalently (K_4^*), the graph obtained from (K_4) by subdividing one edge once.

### The variable graph (H)

Let (H) be any graph with

* (e(H)=m) edges,
* **no isolated vertices** (i.e., (\delta(H)\ge 1)).

Note: the “no isolated vertices” condition implies (v(H)\le 2m).

### Question interpreted as a statement

For each (G\in{Q_3,K_{3,3},H_5}), is it true that

[
\exists C_G>0\ \forall H\ (e(H)=m,\ \delta(H)\ge 1)\ \Longrightarrow\ R(G,H)\le C_G,m\ ?
]

If true, we say (G) is **Ramsey size-linear**.

---

## 2) QUICK LITERATURE/CONTEXT CHECK  (web-browsing available)

* The notion “Ramsey size-linear” and the question whether certain specific graphs (notably (K_4^*), i.e. your (H_5)) are size-linear are explicitly discussed by Bradač–Gishboliner–Sudakov (SIAM JDM 2024). They state the general “(m_2(H)\le 2) implies size-linear?” question is still out of reach, and specifically note they **cannot** answer the (K_4^*) case, but they prove a strong partial result for bipartite (F): (R(K_4^*,F)=O(e(F))).
* They also prove that every subdivision of (K_4) on at least 6 vertices is Ramsey size-linear.
* The Erdős Problems website entry for #567 indicates the question is still posed as open (and records the partial results above).
* For Turán numbers of cubes: a 2024 arXiv paper by Janzer & Sudakov recalls the long-standing bound of Erdős–Simonovits ( \mathrm{ex}(n,Q_3)=O(n^{8/5})) and says it remains the best known upper bound for (Q_3).

So: as of the latest sources I checked, the full linear-in-(m) statement for (G\in{Q_3,K_{3,3},H_5}) is **not recorded as solved**, though there are substantial partial results (notably for (H_5) against bipartite (H)).

---

## 3) ATTACK PLAN

### Proof-track ideas

1. **Induction / decomposition on (H) by removing vertices of small degree**, trying to show that if no red (G) exists then some vertex has a huge blue neighborhood in which one can inductively embed (H-v). This typically needs a robust “either embed in blue or force (G) in red” dichotomy that is hard for non-clique (G).
2. **Universality/packing approach**: if red is (G)-free, red has structural sparsity; attempt to show the blue complement is universal for all graphs with (m) edges. This resembles graph packing/universality with constraints driven by (G)-free extremal structure.
3. **Dependent random choice**: extract a large set in blue with large common neighborhoods to embed (H). This tends to work when (H) has bounded degeneracy/max degree, but here (H) is arbitrary (only (e(H)=m)).

### Disproof-track ideas

1. Try to force (R(G,K_n)) to be superquadratic in (n) (then taking (H=K_n) gives (e(H)\sim n^2) and violates linearity). For these graphs, the usual “(m_2(G)>2)” obstruction is not available, making this hard.
2. Construct explicit colorings with no red (G) and no blue (H) for (N\gg m), by choosing red as an extremal (G)-free graph and choosing (H) so that its complement avoids embedding. This amounts to finding large (G)-free graphs whose complements avoid a carefully chosen (m)-edge (H).

**Best path I can complete rigorously:** I cannot currently close either the full proof or an explicit superlinear counterexample family. What I *can* do gap-free is (i) derive rigorous **polynomial** upper bounds (R(G,H)=O(m^c)) from extremal-number bounds and (ii) record (with citation) the best-known partial linear result for (H_5) vs bipartite (H).

So I proceed with both tracks until the first genuine gap, then give a fully proved partial result package.

---

## 4) WORK

### PHASE 0: Stress points

* Worst-case (H) ranges from a matching ((v(H)=2m), (\Delta=1)) to a clique on (\Theta(\sqrt m)) vertices.
* The claim must hold **uniformly for all (H)**; thus any embedding argument must handle both very sparse and very dense graphs (H).
* For bipartite fixed (G) (like (K_{3,3},Q_3)), red (G)-free graphs are sparse (Turán exponents (<2)), hence blue is dense. But density alone does not imply universality for arbitrary (H).

---

### PHASE 1: Tiny cases (by hand)

* If (m=1), (H=K_2), then (R(G,K_2)=v(G)) for any (G) (since avoiding a blue edge forces all edges red, hence red contains (G) iff (N\ge v(G))). So the bound (R(G,H)\le C m) holds for (m=1) with (C\ge v(G)).
* If (H) is a matching of (m) edges, then (R(G,H)\ge 2m) is forced by needing (v(H)=2m) vertices in the blue copy. So any linear constant (C_G) must satisfy (C_G\ge 2). This is consistent.

No small-case contradiction appears.

---

### PHASE 2: Tool landscape (why relevant)

* **Extremal numbers** (\mathrm{ex}(N,G)): relate “red is (G)-free” to red edge sparsity.
* **Independence number lower bounds** (greedy/Caro–Wei): link red sparsity to existence of large blue cliques, hence bounds on (R(G,K_n)).
* **Kővári–Sós–Turán type counting**: gives (\mathrm{ex}(n,K_{s,t})=O(n^{2-1/s})), in particular (K_{3,3}) has exponent (5/3).
* **Monotonicity**: (H\subseteq K_{v(H)}) implies (R(G,H)\le R(G,K_{v(H)})).
* **Graph packing / universality**: would be the natural route to the linear bound, but requires a strong packing statement beyond what I can prove here.

---

### PROOF TRACK (attempt; stops at first gap)

#### Attempted induction on (H) by removing a vertex

Let (v\in V(H)) with neighbors (N_H(v)). In a coloring on (N) vertices with no red (G), if some vertex (x) has a blue neighborhood of size at least (R(G,H-v)), then within that blue neighborhood we get a blue copy of (H-v), and since (x) is blue-adjacent to all vertices of that copy, we can map (v\mapsto x) and obtain a blue copy of (H).

So, **if no blue (H)** exists, then every vertex has blue degree at most (R(G,H-v)-1), so every vertex has red degree at least
[
N-1-(R(G,H-v)-1)=N-R(G,H-v).
]

At this point, to force a red copy of (G) from large minimum red degree one would need a statement of the form:

> (**Desired lemma**) For (G\in{Q_3,K_{3,3},H_5}) there is a function (f_G(t)) such that any graph with minimum degree (\ge f_G(t)) necessarily contains (G), and moreover (f_G(t)) can be made (\le c,\deg_H(v)) so that summing over the induction yields (O(m)).

But for these (G), minimum-degree conditions of the form (\delta\ge c) (constant) are not sufficient, and even (\delta\ge \alpha N) does force (G) (by Turán), but the induction does not provide such a strong (\alpha) uniformly when (\deg_H(v)) can be small (e.g. matchings).

**First gap (proof track):** I cannot replace the “large minimum red degree” conclusion by an unconditional guarantee of a red copy of (G) that is strong enough to close an (O(m)) induction uniformly over all degree patterns in (H).

---

### DISPROOF TRACK (attempt; no counterexample found)

The most direct counterexample strategy would be:

* Take (H=K_n) so (m\sim n^2).
* Show (R(G,K_n)) grows faster than (c n^2), e.g. (R(G,K_n)\ge c n^2\log n) or (n^{2+\varepsilon}).

However:

* The standard Spencer-type general lower bound gives (R(G,K_n)\gtrsim \tilde\Omega(n^{m_2(G)})). For (K_{3,3}) and (H_5), (m_2=2), which is compatible with (\Theta(n^2)) up to polylog factors.
* I do not have a method to prove a lower bound exceeding (c n^2) for these (G), nor an explicit family (H) with (e(H)=m) for which I can rigorously build a coloring on (N\gg m) avoiding both red (G) and blue (H).

**First gap (disproof track):** I cannot exhibit an explicit infinite family (H_m) and explicit 2-colorings on (N\ge c(m),m) with (c(m)\to\infty) that avoid both red (G) and blue (H_m).

---

### Fully proved partial results

Even though I cannot resolve linearity, I can give rigorous **polynomial** upper bounds for (R(G,H)) for these (G), and a rigorous recorded partial linear bound for (H_5) vs bipartite graphs.

#### Lemma 1 (clique reduction)

For any graphs (G,H),
[
R(G,H)\le R(G,K_{v(H)}).
]

**Proof.** Since (H) is a subgraph of the complete graph (K_{v(H)}), any blue copy of (K_{v(H)}) contains a blue copy of (H). Thus forcing a blue (K_{v(H)}) is stronger than forcing a blue (H), hence (R(G,H)\le R(G,K_{v(H)})). ∎

Also, if (H) has (m) edges and no isolated vertices, then (v(H)\le 2m).

#### Lemma 2 (extremal-number ⇒ off-diagonal Ramsey upper bound)

Let (G) be fixed. Suppose there exist constants (C>0) and (\varepsilon\in(0,1]) such that for all (N),
[
\mathrm{ex}(N,G)\le C,N^{2-\varepsilon}.
]
Then there is a constant (C') (depending only on (C,\varepsilon)) such that for all (n),
[
R(G,K_n)\le C',n^{1/\varepsilon}.
]

**Proof.** Let (N) be any integer, and consider a 2-coloring of (K_N) with no red copy of (G). Then the red graph (R) has at most (e(R)\le \mathrm{ex}(N,G)\le C N^{2-\varepsilon}) edges, so its average degree is
[
\bar d(R)=\frac{2e(R)}{N}\le 2C,N^{1-\varepsilon}.
]

A standard greedy algorithm gives an independent set in any graph of size at least (N/(\bar d+1)): iteratively pick a vertex of degree at most (\bar d), put it in the independent set, and delete it and its neighbors; each step deletes at most (\bar d+1) vertices, so at least (N/(\bar d+1)) steps occur. Hence
[
\alpha(R)\ge \frac{N}{\bar d(R)+1}\ge \frac{N}{2C N^{1-\varepsilon}+1}.
]

If (\alpha(R)\ge n), then the complement (blue graph) contains a (K_n), and we are done. Therefore, to avoid a blue (K_n), we must have
[
\frac{N}{2C N^{1-\varepsilon}+1}< n.
]

For (N\ge (2C)^{1/(1-\varepsilon)}), we have (2C N^{1-\varepsilon}\ge 1), so
[
\alpha(R)\ge \frac{N}{2C N^{1-\varepsilon}+1}\ge \frac{N}{4C N^{1-\varepsilon}}=\frac{N^{\varepsilon}}{4C}.
]
Thus if (N\ge (2C)^{1/(1-\varepsilon)}) and still (\alpha(R)<n), then (N^{\varepsilon}/(4C)<n), i.e.
[
N< (4C,n)^{1/\varepsilon}.
]

Hence choosing
[
C' := \max\Bigl{(2C)^{1/(1-\varepsilon)} ,, (4C)^{1/\varepsilon}\Bigr}
]
ensures that for all (N\ge C' n^{1/\varepsilon}), any red (G)-free coloring forces a blue (K_n). Therefore (R(G,K_n)\le C' n^{1/\varepsilon}+1), proving the claim. ∎

---

#### Lemma 3 (Kővári–Sós–Turán bound for (K_{3,3}), self-contained)

There is an absolute constant (C_0) such that every (N)-vertex graph with no (K_{3,3}) has at most (C_0 N^{5/3}+4N) edges. In particular, (\mathrm{ex}(N,K_{3,3})=O(N^{5/3})).

**Proof.**

**Step 1: reduce to bipartite graphs up to factor 2.**
Let (X) be any graph on (N) vertices with (e(X)=e) and no (K_{3,3}). Choose a random bipartition (V(X)=A\cup B). Each edge crosses with probability (1/2), so the expected number (e(A,B)) of crossing edges is (e/2). Hence there exists a bipartition with
[
e(A,B)\ge e/2.
]
Let (Y) be the bipartite subgraph consisting only of the crossing edges. Then (Y\subseteq X), hence (Y) is also (K_{3,3})-free.

So it suffices to bound the number of edges in a (K_{3,3})-free bipartite graph, then multiply by 2.

**Step 2: bipartite bound.**
Let (Y) be a bipartite graph with parts (A,B), (|A|=a), (|B|=b), and (e(Y)=E), with no copy of (K_{3,3}).

For each (u\in B), let (d(u)) be its degree into (A). Count the number of pairs ((S,u)) where (u\in B) and (S\subseteq N_Y(u)) is a 3-element subset. This count is
[
\sum_{u\in B}\binom{d(u)}{3}.
]
On the other hand, for any 3-element subset (S\subseteq A), the number of common neighbors of (S) in (B) is at most (2): if (S) had 3 distinct common neighbors in (B), then those 3 vertices of (B) together with (S) would span a (K_{3,3}). Therefore each 3-set (S\subseteq A) contributes to at most 2 terms in the above sum, so
[
\sum_{u\in B}\binom{d(u)}{3}\ \le\ 2\binom{a}{3}\ \le\ \frac{a^3}{3}.
\tag{1}
]

Now use the inequality, valid for every integer (d\ge 0),
[
\binom{d}{3}=\frac{d(d-1)(d-2)}{6}\ \ge\ \frac{\bigl(\max{d-2,0}\bigr)^3}{6}.
\tag{2}
]
(Indeed, for (d\le 2) both sides are (0) or RHS is (\le 0); for (d\ge 2), note (d(d-1)\ge (d-2)^2), hence (d(d-1)(d-2)\ge (d-2)^3).)

Let (x_u := \max{d(u)-2,0}\ge 0). Then by (2),
[
\sum_{u\in B}\binom{d(u)}{3}\ \ge\ \frac{1}{6}\sum_{u\in B} x_u^3.
]
By Hölder’s inequality (or power-mean inequality),
[
\Bigl(\sum_{u\in B} x_u\Bigr)^3\ \le\ b^2 \sum_{u\in B} x_u^3
\quad\Longrightarrow\quad
\sum_{u\in B} x_u^3\ \ge\ \frac{\bigl(\sum_{u\in B} x_u\bigr)^3}{b^2}.
]
Also (x_u\ge d(u)-2) for each (u), hence
[
\sum_{u\in B} x_u\ \ge\ \sum_{u\in B}(d(u)-2)\ =\ E-2b.
]
Combining gives
[
\sum_{u\in B}\binom{d(u)}{3}
\ \ge\ \frac{1}{6}\cdot \frac{(E-2b)^3}{b^2}.
\tag{3}
]

Now (1) and (3) imply
[
\frac{1}{6}\cdot \frac{(E-2b)^3}{b^2}\ \le\ \frac{a^3}{3}
\quad\Longrightarrow\quad
(E-2b)^3 \le 2 a^3 b^2.
]
Taking cube roots,
[
E \le 2b + 2^{1/3} a, b^{2/3}.
\tag{4}
]

**Step 3: return to general graphs.**
We had (e/2\le E), so from (4),
[
e \le 2E \le 4b + 2^{4/3} a b^{2/3} \le 4N + 2^{4/3} N^{5/3}.
]
So we can take (C_0=2^{4/3}). ∎

---

#### Corollary 4 (fully proved polynomial bound for (R(K_{3,3},H)))

There is an absolute constant (C) such that for every graph (H) with (m) edges and no isolated vertices,
[
R(K_{3,3},H) \le C, m^3.
]

**Proof.**
From Lemma 3, (\mathrm{ex}(N,K_{3,3})\le C_0 N^{5/3}+4N \le C_1 N^{5/3}) for all sufficiently large (N), so Lemma 2 applies with (\varepsilon=1/3) and gives
[
R(K_{3,3},K_n)\le C_2 n^3.
]
Now apply Lemma 1 and (v(H)\le 2m):
[
R(K_{3,3},H)\le R(K_{3,3},K_{v(H)})\le R(K_{3,3},K_{2m})\le C_2(2m)^3.
]
Absorb constants into (C). ∎

---

#### Corollary 5 (polynomial bound for (R(Q_3,H)); two versions)

**Version A (self-contained, weaker):**
Since (Q_3\subseteq K_{4,4}), we have (\mathrm{ex}(N,Q_3)\le \mathrm{ex}(N,K_{4,4})). A Kővári–Sós–Turán-type argument (same style as Lemma 3 with (s=t=4)) gives (\mathrm{ex}(N,K_{4,4})=O(N^{7/4})), hence by Lemma 2 with (\varepsilon=1/4),
[
R(Q_3,H)\le O(m^4).
]

**Version B (stronger, using a known Turán bound):**
Erdős–Simonovits proved (\mathrm{ex}(N,Q_3)=O(N^{8/5})), and this remains the best known upper bound for (Q_3).
Applying Lemma 2 with (\varepsilon=2/5) yields (R(Q_3,K_n)=O(n^{5/2})), and then Lemma 1 with (v(H)\le 2m) gives
[
R(Q_3,H) \le R(Q_3,K_{2m}) = O(m^{5/2}).
]

Either way, we get a fully rigorous polynomial upper bound; Version B is sharper but imports the known (\mathrm{ex}(N,Q_3)) bound.

---

#### Corollary 6 (partial linearity for (H_5) against bipartite (H), from the literature)

Let (H_5=K_4^*). For every bipartite graph (F) with no isolated vertices,
[
R(H_5,F)=O(e(F)).
]
This is Theorem 3 of Bradač–Gishboliner–Sudakov.

This matches the partial result you quoted.

---

## 5) VERIFICATION

### Quantifier/boundary checks

* The reductions (v(H)\le 2m) and (R(G,H)\le R(G,K_{v(H)})) are unconditional and correct for all graphs (H) with no isolated vertices.
* Lemma 3 explicitly proves a correct KST-type bound for (K_{3,3})-free graphs:

  * The random bipartition step is valid (linearity of expectation).
  * The double counting of (\sum_{u\in B} \binom{d(u)}{3}) and the “common neighbors (\le 2)” implication are exact.
  * Inequality (\binom{d}{3}\ge \frac{(\max{d-2,0})^3}{6}) is checked casewise and uses only algebra.
  * Hölder inequality usage is standard: ((\sum x_i)^3\le b^2\sum x_i^3).
* Lemma 2 relies only on the greedy independent-set bound and algebraic rearrangement; no hidden assumptions.
* Corollary 4 follows cleanly: ex-bound → Ramsey vs clique → general (H) by clique reduction.

### Attempted refutation of Lemma 3

Could there be a mistake where a triple in (A) having 3 common neighbors in (B) does not imply a (K_{3,3})? It does: “three vertices in (A) adjacent to three vertices in (B)” exactly forms (K_{3,3}) (all cross-edges exist by the definition of “common neighbor”), so the implication is correct.

So the partial results proved above are internally consistent.

---

## 6) FINAL

**UNRESOLVED**

(i) **Strongest fully proved partial result obtained here (with proofs above):**

* (R(K_{3,3},H)=O(m^3)) for every (m)-edge graph (H) with no isolated vertices (Corollary 4).
* (R(Q_3,H)=O(m^4)) unconditionally by reduction to (K_{4,4}); and (R(Q_3,H)=O(m^{5/2})) assuming the known bound (\mathrm{ex}(n,Q_3)=O(n^{8/5})).
* For (H_5=K_4^*), the linear bound (R(H_5,F)=O(e(F))) holds for every bipartite (F) (literature Theorem 3).

(ii) **Exact first gap (single crisp statement I could not prove):**

> For (G\in{Q_3,K_{3,3},H_5}), show that every red (G)-free graph on (N=Cm) vertices has complement containing **every** (m)-edge graph (H) with no isolated vertices (i.e., prove a universality/packing statement strong enough to imply (R(G,H)\le Cm) uniformly in (H)).

(iii) **Top 3 next moves (concrete lemmas/approaches to target):**

1. **Packing lemma tuned to “few forbidden copies” rather than max degree:** prove that if (R) is (K_{3,3})-free (or (Q_3)-free) then (\overline R) is universal for all graphs with (m) edges on (\le 2m) vertices once (N\ge Cm). This likely needs a refined container/entropy argument rather than degree-based packing.
2. **Hybrid embedding scheme:** split (H) into a dense core (C) of size (O(\sqrt m)) and a sparse remainder (S), embed (C) using a “blue pseudorandomness” lemma forced by red (G)-freeness, then extend greedily to (S). The missing piece is a robust structural lemma about the blue graph derived from forbidding red (G).
3. **Lower-bound construction search:** attempt explicit colorings using near-extremal (G)-free red graphs with controlled complement structure (e.g., polarity graphs, norm graphs, generalized quadrangles) and choose (H) as an “anti-universal” graph for those complements.

(iv) **What a minimal counterexample would likely look like (if the statement is false):**

* A family (H_m) with (e(H_m)=m) where (H_m) is neither too sparse (easy to embed) nor too dense (could sit inside large cliques), but “structured” to exploit a large (G)-free red graph whose complement avoids (H_m).
* Likely (H_m) would be **bipartite-ish but not too** (for (H_5)), or would have a degree distribution and expansion properties that fail in complements of known extremal (K_{3,3})-free (or cube-free) constructions.


## 1) FORMAL RESTATEMENT

### Conventions

* All graphs are finite, simple, undirected.
* For graphs (A,B), the (two–colour) Ramsey number (R(A,B)) is the least integer (N) such that every red/blue colouring of the edges of (K_N) contains either a red copy of (A) or a blue copy of (B). Equivalently: every (N)-vertex graph (X) contains (A) or (\overline X) contains (B).
* “(X\ll Y)” means “(X \le C,Y) for some absolute constant (C)” (here (C) may depend on the fixed graph (G) but **not** on (H) or (m)).

### The literal question and the minimal corrected statement

The phrase “Let (G) be either (Q_3) or (K_{3,3}) or (H_5)” is ambiguous between:

* (i) “for **each** (G) in this set, does the property hold?”, vs.
* (ii) “for **some** (G) in this set, does the property hold?”

**Minimal corrected statement consistent with standard usage** in “is (G) Ramsey size linear?” is:

> For each fixed (G\in{Q_3, K_{3,3}, H_5}), does there exist a constant (C_G) such that for every graph (H) with (m=e(H)) edges and **no isolated vertices**, we have
> [
> R(G,H)\le C_G, m , ?
> ]

(Here “no isolated vertices” means every vertex of (H) has degree at least (1), hence (|V(H)|\le 2m).)

### Definitions of the three candidate graphs

* (K_{3,3}): complete bipartite graph with parts of size 3.
* (Q_3): the 3D cube graph (8 vertices, 12 edges).
* (H_5): (C_5) with two vertex-disjoint chords (equivalently (K_4) with one edge subdivided once, often denoted (K_4^*)).

**What I will do below:** I focus on (G=K_{3,3}), and prove a *linear* bound (R(K_{3,3},H)=O(m)) for a substantial subclass of graphs (H) (those with (\Delta(H)\le m^{1/3})). I do **not** reach a full proof (or disproof) for all (H), so the overall problem remains unresolved in this response.

---

## 2) QUICK LITERATURE/CONTEXT CHECK

* The “Ramsey size-linear” notion and the specific open question for (G=K_{3,3}) trace back to Erdős–Faudree–Rousseau–Schelp (early 1990s) and Erdős’s 1995 problems list.
* Bradač–Gishboliner–Sudakov prove that every subdivision of (K_4) on at least 6 vertices is Ramsey size-linear, and they also prove (R(H_5,H)\ll m) for bipartite (H) (among other results).
* Wigderson (2025) constructs infinitely many *minimally non–Ramsey size-linear* graphs (showing there are infinitely many obstructions), but this does not settle the specific status of (K_{3,3}), (Q_3), or (H_5).
* I did not find (in the limited web checks) any definitive 2025–2026 resolution stating that (K_{3,3}) is or is not Ramsey size-linear.

---

## 3) ATTACK PLAN

### Proof-track ideas

1. **Degree-trimming + greedy embedding (works when (\Delta(H)) is small enough).**
   If red is (K_{3,3})-free on (N) vertices, it has at most (O(N^{5/3})) edges (Kővári–Sós–Turán), hence average red degree (O(N^{2/3})). Throw away high red-degree vertices; on the remaining large set (U), every vertex has at most (O(N^{2/3})) red neighbours, i.e. (O(N^{2/3})) forbidden choices.
   Then greedily embed (H) into blue provided each new vertex only needs to satisfy at most (O(N^{1/3})) adjacency constraints—i.e. roughly (\Delta(H)\lesssim m^{1/3}) when (N\sim m).

2. **Hybrid core–remainder scheme (not completed here).**
   Embed the “dense core” of (H) (where degrees are large) by some structural/pseudorandomness lemma forced by red (K_{3,3})-freeness; then embed the low-degree remainder greedily.

3. **Packing lemma beyond max degree (not completed here).**
   Adapt Sauer–Spencer/Bollobás–Eldridge style “minimal conflict mapping” but replace max-degree hypotheses with a “few forbidden configurations” hypothesis (here: no (K_{3,3})).

### Disproof-track ideas

1. Find (K_{3,3})-free graphs on (N) vertices with extremely small independence number (\alpha) (much smaller than (N^{1/2})), giving large lower bounds for (R(K_{3,3},K_n)) and potentially breaking linearity by taking (H=K_n) (where (m\sim n^2)).
2. Use explicit extremal constructions (norm graphs, generalized polygons, etc.) and choose (H_m) “anti-universal” for their complements.

### Path chosen now

I will **fully execute** proof-track strategy (1) and obtain a **complete, gap-free** linear bound for the subclass (\Delta(H)\le m^{1/3}). Then I will identify the precise point where extending to arbitrary (H) fails.

---

## 4) WORK

### Lemma 4.1 (Kővári–Sós–Turán bound for (K_{3,3}), stated)

There exists an absolute constant (c_{\mathrm{KST}}) such that every (K_{3,3})-free graph (R) on (N) vertices satisfies
[
e(R)\le c_{\mathrm{KST}}, N^{5/3}+N.
]
In particular the average degree of (R) is at most
[
\overline d(R);=;\frac{2e(R)}{N};\le;2c_{\mathrm{KST}},N^{2/3}+2.
]

*Justification:* This is the classical Kővári–Sós–Turán theorem for (K_{s,t}) with (s=t=3), which gives an upper bound of order (N^{2-1/3}=N^{5/3}).

For concreteness in what follows, fix a numerical constant
[
c_{\mathrm{KST}}:=2,
]
so we may use
[
e(R)\le 2N^{5/3}+N,\qquad \overline d(R)\le 4N^{2/3}+2.
]
(This dominates the true asymptotic constant and is valid for all (N\ge 1).)

---

### Lemma 4.2 (Large low-red-degree subset)

Let (R) be any graph on (N) vertices with average degree (\overline d(R)\le D_{\mathrm{avg}}). Then there exists a subset (U\subseteq V(R)) with
[
|U|\ge \frac{N}{2}
\qquad\text{and}\qquad
\deg_R(u)\le 2D_{\mathrm{avg}}\ \text{ for all }u\in U.
]

**Proof.**
Let
[
B:={v\in V(R): \deg_R(v)>2D_{\mathrm{avg}}}.
]
Then
[
\sum_{v\in V(R)}\deg_R(v) ;>; \sum_{v\in B}\deg_R(v);>; |B|\cdot 2D_{\mathrm{avg}}.
]
But (\sum_v\deg_R(v)=N\overline d(R)\le ND_{\mathrm{avg}}). Hence
[
ND_{\mathrm{avg}} > |B|\cdot 2D_{\mathrm{avg}}\quad\Rightarrow\quad |B|<\frac{N}{2}.
]
Let (U:=V(R)\setminus B). Then (|U|\ge N/2) and every (u\in U) has (\deg_R(u)\le 2D_{\mathrm{avg}}), as required. ∎

---

### Lemma 4.3 (Greedy embedding criterion into the complement)

Let (B) be a graph on vertex set (U) (think: blue graph) and let (F:=\overline B[U]) be its complement on (U) (think: red edges within (U)). Suppose:

1. (|U|=M).
2. Every vertex has at most (t) non-neighbours in (B) (equivalently, (\Delta(F)\le t)).
3. (H) is a graph on (n) vertices with maximum degree (\Delta(H)\le \Delta).
4. The numerical inequality holds:
   [
   M - n - \Delta t \ \ge\ 1.
   ]

Then (B) contains a (not necessarily induced) copy of (H).

**Proof.**
Fix an arbitrary ordering (v_1,\dots,v_n) of the vertices of (H). We construct an injective map (\varphi:V(H)\to U) step-by-step so that every edge of (H) maps to an edge of (B).

Inductive hypothesis at step (i): (\varphi) is already defined on ({v_1,\dots,v_{i-1}}), is injective there, and maps every edge among these vertices to an edge of (B).

At step (i), let
[
S_i:={v_j: 1\le j<i,\ v_jv_i\in E(H)}
]
be the earlier neighbours of (v_i). Then (|S_i|\le \deg_H(v_i)\le \Delta).

For each already embedded neighbour (v_j\in S_i), we require (\varphi(v_i)) to be adjacent in (B) to (\varphi(v_j)). The set of vertices in (U) *not* adjacent to (\varphi(v_j)) in (B) is exactly the set of non-neighbours of (\varphi(v_j)) in (B), and by assumption this has size at most (t).

Therefore, the set of vertices in (U) that fail at least one adjacency requirement to (S_i) has size at most (|S_i|t\le \Delta t) (union bound).

Also, we must avoid the ((i-1)) already used images (\varphi(v_1),\dots,\varphi(v_{i-1})).

Hence the set of valid choices for (\varphi(v_i)) has size at least
[
M - (i-1) - \Delta t \ \ge\ M - (n-1) - \Delta t \ \ge\ 1
]
by assumption (4). So we can choose (\varphi(v_i)) satisfying all constraints. This preserves injectivity and makes all edges from (v_i) to earlier vertices blue, completing the inductive step.

After (n) steps, (\varphi) embeds (H) as a subgraph of (B). ∎

---

### Theorem 4.4 (A proved partial Ramsey size-linear bound for (K_{3,3}))

There is an explicit absolute constant (C=8000) such that the following holds.

For every integer (m\ge 1) and every graph (H) with

* (e(H)=m),
* no isolated vertices,
* maximum degree (\Delta(H)\le m^{1/3}),

we have
[
R(K_{3,3},H)\le 8000,m.
]

**Proof.**
Let (N:=8000m). Consider any red/blue colouring of the edges of (K_N) with **no red copy of (K_{3,3})**. Let (R) be the red graph and (B) the blue graph on the same (N) vertices.

By Lemma 4.1 (with (c_{\mathrm{KST}}=2)), since (R) is (K_{3,3})-free,
[
e(R)\le 2N^{5/3}+N.
]
Hence the average red degree satisfies
[
\overline d(R)=\frac{2e(R)}{N}\le \frac{2(2N^{5/3}+N)}{N}=4N^{2/3}+2 \le 4N^{2/3}+4.
]
Apply Lemma 4.2 with (D_{\mathrm{avg}}:=4N^{2/3}+4). We get a vertex subset (U\subseteq V(K_N)) of size
[
|U|\ge \frac N2 = 4000m,
]
such that every (u\in U) has red degree
[
\deg_R(u)\le 2D_{\mathrm{avg}} = 8N^{2/3}+8.
]
Now consider the blue graph induced on (U), namely (B[U]). For any (u\in U), the number of **blue non-neighbours inside (U)** is at most the number of **red neighbours inside (U)**, which is at most (\deg_R(u)). Therefore every vertex in (B[U]) has at most
[
t:=8N^{2/3}+8
]
non-neighbours in (B[U]).

Let (n:=|V(H)|). Since (H) has no isolated vertices and (e(H)=m),
[
\sum_{v\in V(H)}\deg_H(v)=2m \quad\Rightarrow\quad n\le 2m.
]
We are assuming (\Delta(H)\le m^{1/3}); set (\Delta:=m^{1/3}).

We now verify the greedy-embedding inequality of Lemma 4.3 for the host (B[U]):

* (M:=|U|\ge 4000m).
* (n\le 2m).
* (t=8N^{2/3}+8 = 8(8000m)^{2/3}+8).

Compute ((8000)^{1/3}=20), hence ((8000)^{2/3}=400). Therefore
[
t = 8\cdot 400\cdot m^{2/3} +8 = 3200,m^{2/3}+8.
]
So
[
\Delta t \le m^{1/3}(3200,m^{2/3}+8)=3200m+8m^{1/3}.
]
Hence
[
M - n - \Delta t
\ \ge\ 4000m - 2m - (3200m+8m^{1/3})
\ =\ 798m - 8m^{1/3}.
]
For every integer (m\ge 1),
[
798m - 8m^{1/3} \ge 798 - 8 = 790 \ge 1.
]
Thus the condition (M-n-\Delta t\ge 1) holds, and Lemma 4.3 guarantees that (B[U]), hence (B), contains a copy of (H).

Therefore every red/blue colouring of (K_N) contains either a red (K_{3,3}) or a blue (H). By definition of Ramsey numbers, (R(K_{3,3},H)\le N=8000m). ∎

---

### What this theorem *does* and *does not* accomplish

* It **does** give a genuine **linear** bound (R(K_{3,3},H)=O(m)) for a broad family of (H): those with (\Delta(H)\le m^{1/3}).
* It **does not** resolve the original statement for general (H), since (H) may have (\Delta(H)) as large as (\Theta(m)) (e.g. stars) or (\Theta(\sqrt m)) (e.g. cliques on (\Theta(\sqrt m)) vertices).

(Stars are not actually a problem for this method because they have large max degree but very small order; cliques are a serious obstruction for this particular greedy scheme.)

---

## 5) VERIFICATION

### Quantifier check

* Theorem 4.4 asserts: (\exists C) (here (8000)) such that (\forall m\ge 1), (\forall H) with (e(H)=m), no isolated vertices, and (\Delta(H)\le m^{1/3}), we have (R(K_{3,3},H)\le Cm).
  All quantifiers are explicit and constants are independent of (m) and (H).

### Boundary cases

* (m=1): Then (H) is (K_2), (\Delta(H)=1\le 1^{1/3}). The bound gives (R(K_{3,3},K_2)\le 8000), which is true (indeed (R(K_{3,3},K_2)=2)).
* Small (m) generally: the numerical inequality in the proof was checked for all (m\ge 1) via the explicit (790) margin.

### Internal consistency / no hidden assumptions

* The only nontrivial external input is Lemma 4.1 (Kővári–Sós–Turán). I used it only for (K_{3,3})-free red graphs, which is exactly the hypothesis in the Ramsey setting.
* The greedy embedding (Lemma 4.3) is fully proved with an explicit union bound and a verified positivity condition.

### Attempt to break the greedy embedding argument

The critical step is bounding the forbidden set for embedding a vertex (v_i) by (|S_i|t). This is valid because:

* each already-embedded neighbour excludes at most (t) vertices (its blue non-neighbours),
* and the union bound gives a valid overestimate regardless of overlaps.
  Thus the argument is safe (it may be inefficient, but correctness is not in doubt).

---

## 6) FINAL

**UNRESOLVED**

(i) **Strongest fully proved partial result obtained here.**
For (G=K_{3,3}), I proved the explicit linear bound
[
R(K_{3,3},H)\le 8000,m
]
for every (m)-edge graph (H) with no isolated vertices and (\Delta(H)\le m^{1/3}) (Theorem 4.4, with a complete proof).

(ii) **Exact first gap.**
To prove full Ramsey size-linearity for (K_{3,3}), one must handle graphs (H) with vertices that may have (>!m^{1/3}) neighbours **among already embedded vertices**, i.e. embeddings that require finding a common blue neighbour of sets of size (\gg m^{1/3}).
My current method only guarantees nonempty common blue neighbourhood for sets of size (\le m^{1/3}) via the crude union bound (|\bigcup N_R|\le \sum \deg_R), and it gives no control beyond that. This is the first point where the proof cannot be extended without a substantially sharper structural lemma about (K_{3,3})-free red graphs.

(iii) **Top 3 next moves (concrete lemmas/approaches).**

1. **Refined “union of neighbourhoods” lemma using (K_{3,3})-freeness.** Prove that in any (K_{3,3})-free graph (R) on (N) vertices, for “typical” (or appropriately chosen) sets (S) of size up to (\Theta(\sqrt m)), the union (\bigcup_{v\in S} N_R(v)) is significantly smaller than the union-bound (\sum_{v\in S}\deg_R(v)). Even a bound of the form (|\bigcup N_R(v)|\le O(|S|^{1-\eta}N)) for some (\eta>0) in a suitable regime could push the greedy embedding threshold past (m^{1/3}).
2. **Hybrid embedding scheme (dense core + sparse remainder).** Embed a core (C\subseteq V(H)) containing all vertices of degree (>!m^{1/3}) (size (\le 2m^{2/3})) using a structural property of complements of (K_{3,3})-free graphs; then embed the remainder by the proven greedy method. The missing lemma is a robust way to embed an arbitrary graph on (O(m^{2/3})) vertices into the complement of a (K_{3,3})-free graph on (O(m)) vertices.
3. **Lower-bound construction search.** Try to build explicit 2-colourings from near-extremal (K_{3,3})-free red graphs with specially structured complements (polarity/norm graphs/generalized polygons), and pick (H_m) to be “anti-universal” for those complements; this could yield a superlinear lower bound if such complements systematically avoid some (m)-edge family.

(iv) **What a minimal counterexample would likely look like (if the statement is false).**

* A family ((H_m)) with (e(H_m)=m), no isolated vertices, where (H_m) has a **medium-size dense core** (to defeat my (\Delta\le m^{1/3}) method) but is not so dense that it trivially embeds in “almost complete” complements.
* The obstruction would likely exploit a red (K_{3,3})-free graph (R) whose complement (\overline R) is dense but lacks some specific “universality” feature needed to embed (H_m) (e.g., fails to contain a certain structured core, or has a partition/geometry blocking (H_m)).

