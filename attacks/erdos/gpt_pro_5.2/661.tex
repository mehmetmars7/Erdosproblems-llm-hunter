% Erdos Problem #661
% URL: https://www.erdosproblems.com/661

Are there, for all large $n$, some points $x_1,\ldots,x_n,y_1,\ldots,y_n\in \mathbb{R}^2$ such that the number of distinct distances $d(x_i,y_j)$ is\[o\left(\frac{n}{\sqrt{\log n}}\right)?\] One can also ask this for points in $\mathbb{R}^3$. In $\mathbb{R}^4$ Lenz observed that there are $x_1,\ldots,x_n,y_1,\ldots,y_n\in \mathbb{R}^4$ such that $d(x_i,y_j)=1$ for all $i,j$, taking the points on two orthogonal circles. More generally, if $F(2n)$ is the minimal number of such distances, and $f(2n)$ is minimal number of distinct distances between any $2n$ points in $\mathbb{R}^2$, then is $F =o(f)$? See also [89] .

% Solution to Erdos Problem #661

1) FORMAL RESTATEMENT
For each $n\ge 1$, consider two $n$-point sets $X=\{x_1,\ldots,x_n\}\subset\mathbb{R}^2$ and $Y=\{y_1,\ldots,y_n\}\subset\mathbb{R}^2$.
Let
\[
D(X,Y):=\bigl|\{\,|x_i-y_j|:1\le i,j\le n\,\}\bigr|
\]
be the number of distinct bipartite distances.
Question: do there exist, for all sufficiently large $n$, choices of $X,Y$ with
\[
D(X,Y)=o\!\left(\frac{n}{\sqrt{\log n}}\right)\ ?
\]
(One may also ask the same question in $\mathbb{R}^3$.)

2) QUICK LITERATURE/CONTEXT CHECK
I did not find (and do not assume) a standard theorem resolving this; the problem appears open on erdosproblems.com as of early 2026.

3) ATTACK PLAN
Proof-track ideas (to show existence of very small $D(X,Y)$):
- Use special configurations where $|x_i-y_j|$ depends on a low-complexity parameter, e.g. $x_i$ and $y_j$ constrained to curves/lines so distances become a 1D additive structure.
- Try to imitate the $\mathbb{R}^4$ Lenz construction by seeking near-constant distances via orthogonality-like constraints (circles/spheres), but in $\mathbb{R}^2$ this is much harder.

Disproof-track ideas (to show lower bounds):
- Incidence-geometry: interpret distance equations $|x-y|=r$ as incidences between points and circles, then try to force many radii.
- Reduce to sumset/difference-set bounds in special cases (e.g. points on parallel lines).

I can give a clean $O(n)$ construction and show that in the ``two parallel lines'' model one cannot beat $\Omega(n)$.

4) WORK

Lemma 1 (a simple $O(n)$ construction in $\mathbb{R}^2$).
For each $n$, let
$X:=\{(i,0): i=0,1,\ldots,n-1\}$ and $Y:=\{(j,1): j=0,1,\ldots,n-1\}$.
Then $D(X,Y)\le n$.

Proof.
For $x=(i,0)\in X$ and $y=(j,1)\in Y$ we have $|x-y|^2=(i-j)^2+1$.
Thus distinct distances are in bijection with distinct values of $(i-j)^2$ as $i,j$ range over $\{0,\ldots,n-1\}$.
The integer difference $i-j$ ranges over $\{-(n-1),\ldots,n-1\}$, so $|i-j|$ ranges over $\{0,1,\ldots,n-1\}$ and $(i-j)^2$ takes at most $n$ values.
Therefore $|x-y|$ takes at most $n$ values and $D(X,Y)\le n$. $\square$

Lemma 2 (parallel-line lower bound).
Suppose $X\subset L_0$ and $Y\subset L_1$ where $L_0,L_1$ are two parallel lines at fixed distance $h>0$, and write the coordinates so that
$L_0=\{(t,0):t\in\mathbb{R}\}$ and $L_1=\{(t,h):t\in\mathbb{R}\}$.
Let $A:=\{a: (a,0)\in X\}$ and $B:=\{b: (b,h)\in Y\}\subset\mathbb{R}$. Then
$D(X,Y)=|\{|a-b|:a\in A,b\in B\}|$. In particular, if $|A|=|B|=n$ then $D(X,Y)\ge n$.

Proof.
The squared distance between $(a,0)$ and $(b,h)$ is $(a-b)^2+h^2$, so distinct distances are determined exactly by distinct values of $|a-b|$.
To lower bound these, let $m:=\min(A\cup B)$.
If $m\in A$, then $m\le b$ for all $b\in B$, hence $\{|m-b|:b\in B\}=\{b-m:b\in B\}$ has size $|B|=n$ because $b\mapsto b-m$ is injective on $B$.
Thus $|\{|a-b|:a\in A,b\in B\}|\ge n$.
If instead $m\in B$, the same argument with roles reversed gives at least $|A|=n$ distinct absolute differences. $\square$

FAST REALITY CHECK (brute force on the $3\times 3$ grid $\{0,1,2\}^2$).
I minimized $D(X,Y)$ over all choices of $X,Y\subset\{0,1,2\}^2$ with $|X|=|Y|=n$. Results (squared-distance distinct counts, then witnesses):
- $n=1$: min found $D=1$, e.g. $X=\{(0,0)\}$, $Y=\{(0,0)\}$.
- $n=2$: min found $D=1$, e.g. $X=\{(0,0), (1,1)\}$, $Y=\{(0,1), (1,0)\}$.
- $n=3$: min found $D=2$, e.g. $X=\{(0,0), (0,2), (1,1)\}$, $Y=\{(0,1), (1,0), (1,2)\}$.
- $n=4$: min found $D=2$, e.g. $X=\{(0,0), (0,2), (1,1), (2,0)\}$, $Y=\{(0,1), (1,0), (1,2), (2,1)\}$.
- $n=5$: min found $D=4$, e.g. $X=\{(0,0), (0,2), (1,1), (2,0), (2,2)\}$, $Y=\{(0,0), (0,2), (1,1), (2,0), (2,2)\}$.
These checks show that extremely small $D(X,Y)$ can occur for small $n$ (e.g. $D=1$ for $n=2$), but do not address asymptotics.

5) VERIFICATION
- Lemma 1: I checked that $(i-j)^2$ indeed takes exactly the $n$ values $0^2,1^2,\ldots,(n-1)^2$ by taking $j=0$ and $i=0,\ldots,n-1$. So $D(X,Y)=n$ for that construction.
- Lemma 2: the key injectivity step uses that the minimum element $m$ of $A\cup B$ lies in exactly one of the sets or both; in all cases, distances from $m$ to the opposite set produce $n$ distinct values.

6) FINAL: **UNRESOLVED**
(i) Strongest proved partial result: there is a simple construction with $D(X,Y)=n$ (Lemma 1), and in the special model where $X$ and $Y$ lie on two parallel lines one always has $D(X,Y)\ge n$ (Lemma 2), so parallel-line constructions cannot achieve $o(n)$.
(ii) First gap (crisp): either construct $X,Y\subset\mathbb{R}^2$ with $D(X,Y)=o(n/\sqrt{\log n})$ or prove a lower bound $D(X,Y)\gg n/\sqrt{\log n}$ for all $X,Y$.
(iii) Top 3 next moves (concrete):
  1. Try 2D analogues of Lenz-type orthogonality: points on circles/ellipses with constrained chord lengths, and count the resulting distance spectrum.
  2. Incidence approach: for each radius $r$, circles of radius $r$ centered at points of $X$ contain at most $n$ points of $Y$; bound total incidences to force many radii.
  3. Computation: search over structured families (grids, cyclic polygons, lattice slices) for unusually small $D(X,Y)$ to identify candidate extremizers.
(iv) Minimal counterexample structure (if the answer is ``yes''): one would need configurations where the $n^2$ distances $|x_i-y_j|$ collapse into $o(n/\sqrt{\log n})$ radii, meaning that for most $x\in X$ the set of distances to $Y$ is itself very small and highly overlapping across different $x$, suggesting strong algebraic or incidence structure (many points of $Y$ lying on many common circles with centers in $X$).

