% Attempt for Erdos Problem #87
% Following PROMPT_STRATEGY.MD
% Tools/Constraints:
% - Web browsing available? NO (restricted)
% - Computation available (Python/Sage/Mathematica)? YES, not used

OUTPUT
1) FORMAL RESTATEMENT
For a graph $H$, let $R(H)$ denote the diagonal Ramsey number $r(H,H)$. For $k\ge 2$, let
$R(k)=R(K_k)$. The questions are:
(Q1) For every $\epsilon>0$, does there exist $k_0$ such that for all $k\ge k_0$ and all graphs $G$
with $\chi(G)=k$, we have $R(G)>(1-\epsilon)^k R(k)$?
(Q2) Does there exist $c>0$ and $k_0$ such that for all $k\ge k_0$ and all $k$-chromatic $G$,
$R(G)>cR(k)$?

2) QUICK LITERATURE/CONTEXT CHECK
Browsing is not available. I only record what the problem statement itself claims, without proof.

3) ATTACK PLAN
Proof strategies:
1. Prove a universal lower bound on $R(G)$ in terms of $\chi(G)$ via random colouring.
2. Compare this bound to the best available lower bounds on $R(k)$ (not proved here).

Disproof strategies:
1. Construct $k$-chromatic graphs with Ramsey numbers far smaller than $R(k)$.

Chosen path: provide a complete proof of the Wigderson-type bound $R(G)\ge 2^{(k-3)/2}$ and
isolate the gap to the conjectured ratio $R(G)/R(k)$.

4) WORK
Lemma 1 (Subgraph monotonicity).
If $H$ is a subgraph of $G$, then $R(H)\le R(G)$.

Proof.
Every two-colouring of $K_{R(G)}$ contains a monochromatic copy of $G$, hence also of $H$. ∎

Lemma 2 (Existence of a $k$-critical subgraph).
If $\chi(G)=k$, then $G$ contains an induced subgraph $H$ with $\chi(H)=k$ and
$\chi(H-v)\le k-1$ for every vertex $v$.

Proof.
Iteratively delete vertices that do not reduce chromatic number. The process terminates with a
minimal induced subgraph of chromatic number $k$. ∎

Lemma 3 (Minimum degree in a $k$-critical graph).
If $H$ is $k$-vertex-critical, then $\delta(H)\ge k-1$.

Proof.
If a vertex $v$ has degree at most $k-2$, then a $(k-1)$-colouring of $H-v$ can be extended to $v$,
contradicting $\chi(H)=k$. ∎

Theorem 4 (Wigderson-type lower bound).
If $\chi(G)=k\ge 2$, then
\[R(G)\ge 2^{(k-3)/2}.\]

Proof.
Let $H$ be a $k$-critical induced subgraph of $G$ (Lemma 2). Let $v=|V(H)|$ and $e=|E(H)|$.
By Lemma 3, $2e\ge v(k-1)$, so $e\ge v(k-1)/2$.
Let $N=\lfloor 2^{(k-1)/2-1}\rfloor$. Consider a random red/blue colouring of $K_N$.
For a fixed injective map $\varphi:V(H)\hookrightarrow [N]$, the probability that all $e$ edges of
$\varphi(H)$ are monochromatic is $2\cdot 2^{-e}=2^{1-e}$. The number of injective maps is at most
$N^v$. Hence
\[\mathbb{E}[X]\le N^v 2^{1-e}\le 2\left(\frac{N}{2^{(k-1)/2}}\right)^v\le 2\left(\frac12\right)^v<1.
\]
Therefore there exists a colouring with no monochromatic copy of $H$, so $R(H)>N$ and hence
$R(G)\ge R(H)>N\ge 2^{(k-3)/2}$. ∎

5) VERIFICATION
Edge cases:
- $k=2$: $R(G)\ge 2$ is trivial.
- $k=3$: the bound gives $R(G)\ge 1$, still consistent.

Quantifier checks:
The theorem is unconditional and applies to every $k$-chromatic graph. It does not imply the
ratio bounds in (Q1)--(Q2) without additional information about $R(k)$.

6) FINAL
**UNRESOLVED**

(i) Strongest fully proved partial result:
For any $k$-chromatic graph $G$, $R(G)\ge 2^{(k-3)/2}$ (Theorem 4).

(ii) First gap:
Relating $R(G)$ to $R(k)$ multiplicatively requires sharper lower bounds on $R(k)$ or sharper
upper bounds on $R(G)$ for special $G$; neither is achieved here.

(iii) Top 3 next moves:
1. Improve the universal lower bound on $R(G)$ beyond $2^{k/2}$.
2. Establish better explicit lower bounds for $R(k)$ to compare with Theorem 4.
3. Identify families of $k$-chromatic graphs with particularly small Ramsey numbers.

(iv) Minimal counterexample structure (if (Q1) is false):
An infinite sequence of $k$-chromatic graphs $G_k$ with $R(G_k)$ asymptotically much smaller than
$R(k)$, showing the ratio $R(G_k)/R(k)$ tends to $0$.
