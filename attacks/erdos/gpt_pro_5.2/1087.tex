% Erdos Problem #1087

\subsection*{Erd\H{o}s Problem 1087 (``degenerate'' 4-sets with a repeated distance)}

\subsubsection*{FORMAL RESTATEMENT}
Let \(P\subset\mathbb{R}^2\) be a set of \(n\) points.
Call a 4-element subset \(Q\subset P\) \emph{degenerate} if among the \(\binom{4}{2}=6\) pairwise distances \(\{\|x-y\|: x,y\in Q, x\ne y\}\) there exist two distinct unordered pairs \(\{x,y\}\ne\{u,v\}\) with \(\|x-y\|=\|u-v\|\).
Let
\[
 f(n):=\max\{\,\#\{Q\subset P: |Q|=4,\ Q\text{ degenerate}\}: P\subset\mathbb{R}^2,\ |P|=n\,\}.
\]
Equivalently, \(f(n)\) is the smallest number such that every \(n\)-point set has at most \(f(n)\) degenerate 4-subsets.
The problem asks for estimates of \(f(n)\), in particular whether \(f(n)\le n^{3+o(1)}\).

\subsubsection*{QUICK LITERATURE/CONTEXT CHECK}
The problem statement records bounds of Erd\H{o}s--Purdy:
\(n^3\log n \ll f(n) \ll n^{7/2}\).
I do not use further external results.

\subsubsection*{ATTACK PLAN}
\begin{itemize}
\item \textbf{Proof track (partial).} Provide a clean combinatorial upper bound on the number of degenerate 4-sets in terms of distance multiplicities (counts of equal-length pairs). Provide an explicit construction (regular \(n\)-gon) yielding \(\Omega(n^3)\) degenerate 4-sets.
\item \textbf{Disproof track.} Not applicable.
\end{itemize}

\subsubsection*{WORK}

\paragraph{Fast reality check.} For \(P\) the vertices of a regular \(n\)-gon, brute-force computation shows:
for \(n=6,7,8,9,10,11\), \emph{every} 4-subset is degenerate (so the count equals \(\binom{n}{4}\)); for \(n=12\), there are \(447\) degenerate 4-subsets out of \(\binom{12}{4}=495\).

\begin{lemma}[Degenerate 4-sets controlled by distance multiplicities]
For a finite set \(P\subset\mathbb{R}^2\), and each distance value \(r>0\), let
\(a_r\) be the number of unordered pairs \(\{x,y\}\subset P\) with \(\|x-y\|=r\).
Then the number \(D(P)\) of degenerate 4-subsets of \(P\) satisfies
\[
 D(P)\le \sum_{r} \binom{a_r}{2}.
\]
\end{lemma}

\begin{proof}
For each distance value \(r\), form the graph \(G_r\) on vertex set \(P\) whose edges are exactly the pairs at distance \(r\). Then \(|E(G_r)|=a_r\).
If \(Q\subset P\) is a degenerate 4-set, by definition there exist two distinct unordered pairs
\(e_1,e_2\subset Q\) with \(|e_1|=|e_2|=2\) and \(\mathrm{dist}(e_1)=\mathrm{dist}(e_2)=r\) for some \(r\).
In particular, \(e_1\) and \(e_2\) are two distinct edges of \(G_r\).
Thus each degenerate 4-set contributes at least one unordered pair of edges from some \(G_r\).
For each fixed \(r\), the total number of unordered edge-pairs in \(G_r\) is \(\binom{a_r}{2}\), and the number of degenerate 4-sets that can be ``certified'' by distance \(r\) is at most that.
Summing over \(r\) gives
\(D(P)\le \sum_r \binom{a_r}{2}\).
\end{proof}

\begin{lemma}[Regular \(n\)-gon gives \(\Omega(n^3)\) degenerate 4-sets]
Let \(P\) be the vertex set of a regular \(n\)-gon in \(\mathbb{R}^2\), with \(n\ge 6\). Then
\[
 f(n)\ge D(P)\ge c\,n^3
\]
for an absolute constant \(c>0\) (e.g. \(c=1/48\) works for all sufficiently large \(n\)).
\end{lemma}

\begin{proof}
In a regular \(n\)-gon, for each step size \(k\in\{1,2,\dots,\lfloor n/2\rfloor\}\), all chords connecting vertices with cyclic separation \(k\) have the same length; denote this length by \(\ell_k\).
For each fixed \(k\) (with \(1\le k< n/2\)), there are exactly \(n\) edges of length \(\ell_k\): namely \(\{i,i+k\}\) for \(i\in\mathbb{Z}/n\mathbb{Z}\).
Among these \(n\) edges, at most \(2n\) unordered pairs of edges share a vertex (each edge shares a vertex with at most \(2\) other edges of the same step size at each endpoint), so the number of unordered \emph{disjoint} edge pairs of length \(\ell_k\) is at least
\[
\binom{n}{2}-2n = \frac{n(n-1)}{2}-2n \ge \frac{n^2}{3}
\]
for all \(n\ge 6\).
Each disjoint pair of edges \(\{a,b\}\) and \(\{c,d\}\) of the same length determines a 4-set \(\{a,b,c,d\}\) which is degenerate (it contains two equal distances).

Let \(T\) be the number of pairs \((k,\{e_1,e_2\})\) where \(1\le k\le \lfloor n/3\rfloor\) and \(e_1,e_2\) are disjoint edges of length \(\ell_k\).
By the bound above, for each such \(k\) there are at least \(n^2/3\) choices of \(\{e_1,e_2\}\), so
\[
T \ge \left\lfloor\frac{n}{3}\right\rfloor\cdot \frac{n^2}{3} \ge \frac{n^3}{12}-\frac{n^2}{9}.
\]
On the other hand, a fixed 4-set \(Q\) can contribute to at most \(3\) such pairs \((k,\{e_1,e_2\})\), because a 4-set has exactly \(3\) ways to partition its vertices into two disjoint unordered pairs.
Therefore \(T\le 3D(P)\), hence
\[
D(P)\ge \frac{T}{3} \ge \frac{n^3}{36}-\frac{n^2}{27}.
\]
Taking \(c=1/48\) works for all sufficiently large \(n\), and in any case this proves \(D(P)=\Omega(n^3)\).
Since \(f(n)\) is the maximum possible number of degenerate 4-sets over all \(n\)-point sets, \(f(n)\ge D(P)\).
\end{proof}

\subsubsection*{VERIFICATION}
\begin{itemize}
\item In Lemma 1, the injection ``degenerate 4-set \(\to\) some distance class edge-pair'' may have multiplicities (a 4-set can be certified by multiple distances), but that only strengthens the inequality in the correct direction.
\item In Lemma 2, the disjoint-edge-pair count is a lower bound; even if the constant \(2n\) is loose, it suffices to obtain \(\Omega(n^3)\).
\item Computation for regular \(n\)-gons: for \(n\le 11\) all \(\binom{n}{4}\) quadruples are degenerate; for \(n=12\) the exact number of degenerate quadruples is \(447\) (out of \(495\)).
\end{itemize}

\subsubsection*{FINAL}
\textbf{UNRESOLVED.}
\begin{enumerate}
\item[(i)] \textbf{Strongest proved partial result here:} explicit construction shows \(f(n)\ge c n^3\). Also, for any point set \(P\), the number of degenerate 4-sets is at most \(\sum_r \binom{a_r}{2}\) where \(a_r\) are multiplicities of each realized distance.
\item[(ii)] \textbf{First gap:} prove a universal upper bound \(f(n)\le n^{3+o(1)}\) (or construct configurations with \(\gg n^{3+\varepsilon}\) degenerate 4-sets).
\item[(iii)] \textbf{Top 3 next moves:}
(1) bound \(\sum_r \binom{a_r}{2}\) sharply using incidence geometry to control distance multiplicities \(a_r\);
(2) look for configurations (e.g. integer lattices) where many distances repeat and estimate the resulting degenerate-4 count computationally;
(3) attempt to count non-degenerate 4-sets (all 6 distances distinct) from below to deduce upper bounds on \(f(n)\).
\item[(iv)] \textbf{Minimal counterexample structure (to \(f(n)\le n^{3+o(1)}\)):}
sets with many distance classes each containing \(\Theta(n)\) pairs, and with many of those pairs disjoint so that \(\sum_r \binom{a_r}{2}\) is \(\gg n^{3+\varepsilon}\).
\end{enumerate}

