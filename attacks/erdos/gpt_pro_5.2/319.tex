\section*{Problem \#319}

\subsection*{1) FORMAL RESTATEMENT}
For each integer $N\ge 1$, let $F(N)$ be the maximum cardinality of a set $A\subseteq \{1,2,\dots,N\}$ for which there exists a sign function
\[
\delta:A\to\{-1,+1\},\qquad n\mapsto \delta_n,
\]
such that
\[
\sum_{n\in A}\frac{\delta_n}{n}=0,
\]
and moreover
\[
\sum_{n\in A'}\frac{\delta_n}{n}\neq 0
\quad\text{for every nonempty proper subset } \varnothing\neq A'\subsetneq A.
\]
The problem asks for good bounds/asymptotics for $F(N)$ as $N\to\infty$.

\subsection*{2) CONTEXT AND DEFINITIONS}
Writing $A=P\sqcup Q$ where $P=\{n\in A:\delta_n=+1\}$ and $Q=\{n\in A:\delta_n=-1\}$, the condition is equivalent to
\[
\sum_{p\in P}\frac1p=\sum_{q\in Q}\frac1q,
\]
together with a \emph{minimality} condition: there do not exist nonempty proper subcollections $P'\subseteq P$, $Q'\subseteq Q$ with
$\sum_{p\in P'}\frac1p=\sum_{q\in Q'}\frac1q$.

A particularly simple way to guarantee minimality is to take $P=\{1\}$ and $Q=B$ for some $B\subseteq\{2,\dots,N\}$ with $\sum_{b\in B}\frac1b=1$.
Then any proper sub-sum over $B$ is $<1$.

\subsection*{3) QUICK LITERATURE/CONTEXT CHECK}
The Erd\H{o}s Problems page for \#319 records an observation of Adenwalla that a linear-size lower bound follows from a theorem of Croot on Egyptian fractions in short intervals: there exists $B\subset [(\frac1e-o(1))N,N]$ with $\sum_{b\in B}1/b=1$, which yields $|A|\ge (1-\frac1e+o(1))N$ by taking $A=B\cup\{1\}$ with signs $+1$ on $1$ and $-1$ on $B$.

\subsection*{4) ATTACK PLAN}
I will (i) state the needed consequence of Croot's theorem precisely; (ii) give a complete proof that it implies
\[
F(N)\ge \left(1-\frac1e+o(1)\right)N;
\]
and (iii) check the ``no proper subset sums to $0$'' condition carefully.

This does \emph{not} resolve the full asymptotic for $F(N)$, but it does give a fully rigorous linear lower bound of the claimed constant.

\subsection*{5) DETAILED WORK}
\paragraph{Step 1: A usable consequence of Croot's theorem.}
Croot proved (among other things) that for any fixed rational $r>0$ and sufficiently large $X$ there exist integers
\[
X < n_1 < n_2 < \cdots < n_k < (e^r+o_r(1))X
\]
such that $\sum_{i=1}^k \frac1{n_i}=r$.
Specializing to $r=1$ and choosing $X=X(N)$ with $(e+o(1))X\le N$ (e.g.\ $X=N/(e+\varepsilon_N)$ for any $\varepsilon_N\to 0^+$) gives:

\medskip
\noindent\textbf{Lemma 319.1 (existence of an interval representation of $1$).}
\emph{There exists a function $\alpha_N\to 1/e$ and, for all sufficiently large $N$, a finite set of integers}
\[
B\subseteq \{\lceil \alpha_N N\rceil,\lceil \alpha_N N\rceil+1,\dots,N\}
\]
\emph{such that}
\[
\sum_{b\in B}\frac1b = 1.
\]
\medskip

\paragraph{Step 2: This forces $B$ to have (almost) full size in the interval.}
Let $I_N=\{\lceil \alpha_N N\rceil,\dots,N\}$. Using the integral test (or the asymptotic $H_n=\log n+\gamma+o(1)$),
\[
\sum_{m\in I_N}\frac1m
=
\log\!\Big(\frac{N}{\alpha_N N}\Big) + o(1)
=
\log\!\Big(\frac1{\alpha_N}\Big)+o(1)
=
1+o(1)
\quad (N\to\infty),
\]
since $\alpha_N\to 1/e$.

Because $B\subseteq I_N$ and $\sum_{b\in B}\frac1b=1$, we have
\[
\sum_{m\in I_N\setminus B}\frac1m
=
\sum_{m\in I_N}\frac1m - 1
=
o(1).
\]
But every $m\in I_N\setminus B$ satisfies $m\le N$, hence $1/m\ge 1/N$. Therefore
\[
\frac{|I_N\setminus B|}{N}
\le
\sum_{m\in I_N\setminus B}\frac1m
=
o(1),
\]
so $|I_N\setminus B|=o(N)$, i.e.
\[
|B|=|I_N|-o(N)=\bigl(N-\lceil \alpha_N N\rceil +1\bigr)-o(N)
=
\left(1-\frac1e+o(1)\right)N.
\]

\paragraph{Step 3: Build $A$ and verify minimality.}
Let
\[
A := B\cup\{1\}\subseteq\{1,\dots,N\},
\qquad
\delta_1:=+1,\ \delta_b:=-1\ (b\in B).
\]
Then
\[
\sum_{n\in A}\frac{\delta_n}{n}
=
1-\sum_{b\in B}\frac1b
=
0.
\]
Now let $\varnothing\neq A'\subsetneq A$.

\begin{itemize}
\item If $1\notin A'$, then $A'\subseteq B$ and $\sum_{n\in A'}\frac{\delta_n}{n}=-\sum_{b\in A'}\frac1b<0$, so it is nonzero.
\item If $1\in A'$, write $A'=\{1\}\cup B'$ with $B'\subsetneq B$. Then
\[
\sum_{n\in A'}\frac{\delta_n}{n}
=
1-\sum_{b\in B'}\frac1b.
\]
Since $B'$ is a \emph{proper} subset of $B$ and all terms are positive,
$\sum_{b\in B'}\frac1b < \sum_{b\in B}\frac1b = 1$, hence the above sum is $>0$, in particular nonzero.
\end{itemize}

Thus $A$ satisfies all requirements and
\[
|A|=|B|+1 \ge \left(1-\frac1e+o(1)\right)N.
\]

\subsection*{6) VERIFICATION AND EDGE CASES}
\begin{itemize}
\item The construction uses only positivity of the reciprocals to ensure strict inequality for proper subsets.
\item For small $N$ this asymptotic statement is not informative; e.g.\ for $N=6$ one can take $A=\{1,2,3,6\}$ with signs $(+,-,-,-)$ since $1=\frac12+\frac13+\frac16$.
\end{itemize}

\subsection*{7) FINAL}
\textbf{UNRESOLVED.}
\begin{itemize}
\item[(i)] \textbf{Strongest proved partial result:} $F(N)\ge (1-\frac1e+o(1))N$ via Croot's theorem and the explicit $A=B\cup\{1\}$ construction above.
\item[(ii)] \textbf{First remaining gap:} obtain any nontrivial \emph{upper bound} $F(N)\le (1-\varepsilon)N$ for some fixed $\varepsilon>0$, or determine the true asymptotic constant for $F(N)/N$.
\item[(iii)] \textbf{Next moves:} study structure of minimal solutions of $\sum_{p\in P}1/p=\sum_{q\in Q}1/q$ with $P,Q\subseteq[1,N]$; try to rule out ``too many'' denominators by showing unavoidable sub-equalities among large sets of unit fractions.
\item[(iv)] \textbf{Minimal counterexample search:} compute $F(N)$ for moderate $N$ by integer/rational exact search to guess the constant (if it exists), and to see whether extremals always have a ``single positive term'' form.
\end{itemize}

\noindent\textbf{COMPLETION: 45\%}

\hrule

