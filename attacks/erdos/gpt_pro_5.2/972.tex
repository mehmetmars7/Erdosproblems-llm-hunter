
Let $\alpha>1$ be irrational. Are there infinitely many primes $p$ such that $\lfloor p\alpha\rfloor$ is also prime? Vinogradov \cite{Vi48} proved that the sequence $\{p\alpha\}$ is uniformly distributed for every irrational $\alpha$, and hence there are infinitely many primes $p$ of the shape $p=\lfloor n\alpha\rfloor$ for every irrational $\alpha>1$. Indeed, this occurs if and only if\[\frac{p}{\alpha}\leq n<\frac{p+1}{\alpha},\]which is true if and only if $\{p\alpha^{-1}\}>1-\alpha^{-1}$, which happens infinitely often by the uniform distribution of $\{p\alpha^{-1}\}$. References [Vi48] Vinogradov, I. M., On an estimate of trigonometric sums with prime numbers . Izv. Akad. Nauk SSSR Ser. Mat. (1948), 225--248.


\medskip
\noindent\textbf{1) FORMAL RESTATEMENT}\\
Fix an irrational real number $\alpha>1$.
The question is whether
\[
\#\{p\text{ prime}: \lfloor \alpha p\rfloor\text{ is prime}\}=\infty.
\]
Equivalently, do there exist infinitely many prime pairs $(p,q)$ such that $q$ is prime and
\[
 q=\lfloor \alpha p\rfloor \quad\Longleftrightarrow\quad q\le \alpha p<q+1.
\]

\medskip
\noindent\textbf{2) QUICK LITERATURE/CONTEXT CHECK}\\
The problem statement recalls Vinogradov's uniform distribution result for $\{p\alpha\}$ (and $\{p\alpha^{-1}\}$) and explains how it implies infinitely many primes of the form $\lfloor n\alpha\rfloor$ for irrational $\alpha>1$.  The present problem asks for simultaneous primality of $p$ and $\lfloor \alpha p\rfloor$, which is not resolved by that uniform distribution statement alone.

\medskip
\noindent\textbf{3) ATTACK PLAN}\\
\textbf{Proof track:} reinterpret the condition as a thin set of prime pairs with a linear constraint and attempt a sieve for the pair $(p,\lfloor\alpha p\rfloor)$.

\textbf{Disproof track:} attempt to find an $\alpha$ for which $\lfloor\alpha p\rfloor$ is forced to have a small prime factor for all large $p$ (e.g. near an integer); however $\alpha$ is irrational so only approximate congruence obstructions are possible.

I reach only equivalences and computational evidence.

\medskip
\noindent\textbf{4) WORK}\\
\textbf{FAST REALITY CHECK (sample $\alpha$).} For two irrationals, I searched primes $p\le 10^6$ and counted those for which $\lfloor \alpha p\rfloor$ is also prime.
\begin{itemize}
\item $\alpha=\sqrt2$: there are $6047$ such primes $p\le 10^6$; the first few pairs $(p,\lfloor\sqrt2 p\rfloor)$ are $(2,2),(5,7),(29,41),(31,43),(59,83)$.
\item $\alpha=\varphi=(1+\sqrt5)/2$: there are $6036$ such primes $p\le 10^6$; the first few pairs $(p,\lfloor\varphi p\rfloor)$ are $(2,3),(7,11),(11,17),(23,37),(37,59)$.
\end{itemize}
This does not prove infinitude, but it checks that there is no immediate obstruction for these examples.

\medskip
\noindent\textbf{Lemma 972.1 (ratio window).}\label{lem:972-ratio}
Let $\alpha>1$ be real and let $p$ be a positive integer.  Put $q=\lfloor \alpha p\rfloor$.
Then
\[
 q\text{ is prime}\ \Longleftrightarrow\ \text{there is a prime }q\text{ with }\alpha-\frac1p<\frac{q}{p}\le \alpha.
\]

\noindent\textbf{Proof.}
By definition, $q=\lfloor\alpha p\rfloor$ is the unique integer satisfying $q\le \alpha p < q+1$.
Dividing by $p>0$ gives
\[
\frac{q}{p}\le \alpha < \frac{q+1}{p} = \frac{q}{p}+\frac1p,
\]
which rearranges to $\alpha-\frac1p<\frac{q}{p}\le\alpha$.
The equivalence with primality is immediate since $q$ is exactly $\lfloor\alpha p\rfloor$. \qed

\medskip
\noindent\textbf{Lemma 972.2 (inversion via fractional parts).}\label{lem:972-inversion}
Let $\alpha>1$ be irrational.
For an integer $q\ge 1$, define $p:=\lfloor q/\alpha\rfloor+1$.
Then
\[
\lfloor \alpha p\rfloor = q
\quad\Longleftrightarrow\quad
\Big\{\frac{q}{\alpha}\Big\} > 1-\frac{1}{\alpha},
\]
where $\{x\}=x-\lfloor x\rfloor$ is the fractional part.
Consequently, the original question is equivalent to asking whether there are infinitely many primes $q$ such that $p=\lfloor q/\alpha\rfloor+1$ is prime and $\{q/\alpha\}>1-1/\alpha$.

\noindent\textbf{Proof.}
Because $\alpha$ is irrational, $q/\alpha\notin\mathbb{Z}$ for every integer $q$, so $p=\lfloor q/\alpha\rfloor+1$ is the unique integer satisfying
\[
\frac{q}{\alpha}<p\le \frac{q}{\alpha}+1.
\]
Multiplying by $\alpha$ gives
\[
 q< \alpha p\le q+\alpha.
\]
Thus $\lfloor\alpha p\rfloor=q$ holds exactly when $\alpha p<q+1$ (since $\alpha p>q$ already), i.e.
\[
\alpha\Big(\lfloor q/\alpha\rfloor+1\Big) < q+1.
\]
Write $q/\alpha=\lfloor q/\alpha\rfloor+\{q/\alpha\}$; then the left-hand side is
\[
\alpha\lfloor q/\alpha\rfloor+\alpha = q-\alpha\{q/\alpha\}+\alpha.
\]
So $\alpha p<q+1$ is equivalent to
\[
q-\alpha\{q/\alpha\}+\alpha < q+1
\quad\Longleftrightarrow\quad
\{q/\alpha\} > 1-\frac1\alpha.
\]
This proves the equivalence.  The final sentence just substitutes ``$q$ prime and $p$ prime'' into the equivalence. \qed

\medskip
\noindent\textbf{5) VERIFICATION}\\
Lemma~\ref{lem:972-inversion} uses irrationality to guarantee $q/\alpha\notin\mathbb{Z}$, ensuring $p=\lfloor q/\alpha\rfloor+1$ is the unique candidate integer mapped to $q$ by $p\mapsto\lfloor\alpha p\rfloor$.  If $\alpha$ were rational, this step could fail when $q/\alpha\in\mathbb{Z}$.

The computation used sieving up to $\lfloor \alpha\cdot 10^6\rfloor$ to test primality of $\lfloor\alpha p\rfloor$.

\medskip
\noindent\textbf{6) FINAL}\\
\textbf{UNRESOLVED}

(i) \textbf{Strongest proved partial result.} The condition is equivalent to finding infinitely many prime pairs $(p,q)$ with $q/p$ lying in the shrinking interval $(\alpha-1/p,\alpha]$ (Lemma~\ref{lem:972-ratio}), or equivalently prime $q$ with $p=\lfloor q/\alpha\rfloor+1$ prime and $\{q/\alpha\}>1-1/\alpha$ (Lemma~\ref{lem:972-inversion}).  Computations for $\alpha=\sqrt2$ and $\alpha=\varphi$ show thousands of solutions up to $p\le 10^6$.

(ii) \textbf{First gap (crisp).} Prove (or disprove) that for every irrational $\alpha>1$, there are infinitely many primes $p$ such that $\lfloor\alpha p\rfloor$ is prime.

(iii) \textbf{Top 3 next moves.}
\begin{itemize}
\item Treat $(p,\lfloor\alpha p\rfloor)$ as a two-dimensional prime pattern and attempt a sieve/prime-tuple heuristic in the thin region $q\approx \alpha p$ with error $<1$.
\item Use Lemma~\ref{lem:972-inversion} to recast the problem as primes $q$ in a Beatty-type sequence with prime-index constraint, and test whether known distribution results for primes in Beatty sequences can be combined with a second sieve.
\item Extend computations to larger $p$ (e.g. $10^7$ or $10^8$) for several $\alpha$ (including badly approximable and very well approximable irrationals) to see if the count matches a stable heuristic.
\end{itemize}

(iv) \textbf{Minimal counterexample structure.} A counterexample would be an irrational $\alpha>1$ for which all but finitely many primes $p$ yield a composite $\lfloor\alpha p\rfloor$.  Any such obstruction would likely have to come from subtle correlations between $p$ and the fractional parts $\{\alpha p\}$ that force $\lfloor\alpha p\rfloor$ into residue classes with small prime factors.


