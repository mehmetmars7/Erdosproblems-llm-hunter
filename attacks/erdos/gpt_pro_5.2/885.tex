% Erdos Problem #885
% URL: https://www.erdosproblems.com/885

For integer $n\geq 1$ we define the factor difference set of $n$ by
\[D(n) = \{\lvert a-b\rvert : n=ab\}.\]
Is it true that, for every $k\geq 1$, there exist integers $N_1<\cdots<N_k$ such that
\[\lvert \cap_i D(N_i)\rvert \geq k?\]
A question of Erd\H{o}s and Rosenfeld \cite{ErRo97}, who proved this is true for $k=2$. Jim\'{e}nez-Urroz \cite{Ji99} proved this for $k=3$ and Bremner \cite{Br19} proved this for $k=4$.

1) FORMAL RESTATEMENT

For $n\in\mathbb N$, define
\[D(n):=\{ |a-b| : a,b\in\mathbb N,\ ab=n\}.
\]
(So $0\in D(n)$ iff $n$ is a perfect square.)

Question: For every integer $k\ge 1$, do there exist integers $N_1<\cdots<N_k$ such that
\[\big|D(N_1)\cap D(N_2)\cap\cdots\cap D(N_k)\big|\ge k\ ?\]

2) QUICK LITERATURE/CONTEXT CHECK

The problem statement itself reports proofs for $k=2,3,4$ (with references). Per the integrity rule, I do not add further literature. I focus on structural lemmas and explicit small-$k$ examples verified directly.

3) ATTACK PLAN

- Rewrite the condition “$d\in D(n)$” in several equivalent arithmetic forms, to expose Diophantine structure.
- Provide explicit verified examples for $k=2$ and $k=3$ as a sanity check.
- Attempt a small computational search for $k\ge 4$ in a bounded range (not a proof); record the outcome.

4) WORK

Lemma 885.1 (Divisor-pair description).
For every $n\ge 1$,
\[D(n)=\{\,\tfrac{n}{d}-d\ :\ d\mid n,\ d\le \sqrt n\,\}.
\]

Proof.
If $ab=n$ with $a\le b$, then $a\mid n$ and $b=n/a$, so $|a-b|=b-a=n/a-a$.
Conversely, for any divisor $d\mid n$ with $d\le \sqrt n$, let $b=n/d\ge d$; then $db=n$ and $b-d\in D(n)$.
This gives the stated equality of sets. \qed

Lemma 885.2 (Quadratic form characterization).
Let $n\ge 1$ and $d\ge 0$ be integers. Then
\[d\in D(n)\ \Longleftrightarrow\ \exists a\in\mathbb N\text{ such that }n=a(a+d).
\]

Proof.
($\Rightarrow$) If $d\in D(n)$, there exist $a,b\in\mathbb N$ with $ab=n$ and $|a-b|=d$. Without loss of generality $b\ge a$, so $b=a+d$ and $n=ab=a(a+d)$.
($\Leftarrow$) If $n=a(a+d)$ for some $a\in\mathbb N$, then $n$ has the factor pair $(a,a+d)$ and the difference of the factors is $d$, hence $d\in D(n)$.
\qed

Lemma 885.3 (Square-completion test).
Let $n\ge 1$ and $d\ge 0$. Then $d\in D(n)$ if and only if there exists an integer $m$ such that
\[m^2=4n+d^2\quad\text{and}\quad m\equiv d\pmod 2.
\]
In that case one can take $a=(m-d)/2$ in Lemma 885.2.

Proof.
By Lemma 885.2, $d\in D(n)$ iff $n=a(a+d)$ for some $a\in\mathbb N$.
Compute
\[4n+d^2=4a(a+d)+d^2=(2a+d)^2.
\]
So if $d\in D(n)$, then $m:=2a+d$ satisfies $m^2=4n+d^2$ and clearly $m\equiv d\pmod 2$.

Conversely, if $m^2=4n+d^2$ and $m\equiv d\pmod2$, then $a:=(m-d)/2$ is an integer and
\[a(a+d)=\frac{m-d}{2}\cdot\frac{m+d}{2}=\frac{m^2-d^2}{4}=n.
\]
Thus $d\in D(n)$. \qed

FAST REALITY CHECK (explicit examples for $k=2$ and $k=3$).
The following examples were found and verified by direct factor pairs.

Example for $k=2$:
\[N_1=171360,\quad N_2=196560.
\]
Their common factor differences include $606$ and $1321$:
- $196560=234\cdot 840$ and $171360=210\cdot 816$ give difference $606$.
- $196560=135\cdot 1456$ and $171360=119\cdot 1440$ give difference $1321$.
Hence $|D(N_1)\cap D(N_2)|\ge 2$.

Example for $k=3$:
\[N_1=28560,\quad N_2=139440,\quad N_3=185640.
\]
Their common factor differences include $454,674,1663$:
- Difference $454$:
  $185640=260\cdot 714$, $139440=210\cdot 664$, $28560=56\cdot 510$.
- Difference $674$:
  $185640=210\cdot 884$, $139440=166\cdot 840$, $28560=40\cdot 714$.
- Difference $1663$:
  $185640=105\cdot 1768$, $139440=80\cdot 1743$, $28560=17\cdot 1680$.
Thus $|D(N_1)\cap D(N_2)\cap D(N_3)|\ge 3$.

Additional bounded search (not exhaustive).
A backtracking search among “highly-divisible” numbers $\le 10^6$ did not find a $k=4$ instance within the tested candidate pool (this is not a contradiction to the stated $k=4$ theorem; the first known examples may occur above this range or outside the heuristic pool).

5) VERIFICATION

- Lemma 885.1: verified bijection between factor pairs $(a,b)$ with $a\le b$ and divisors $a\le \sqrt n$.
- Lemma 885.3: checked parity condition is necessary so that $a=(m-d)/2$ is integral.
- The explicit examples were verified by exhibiting factor pairs for each claimed difference, so membership in each $D(N_i)$ is line-by-line checked.

6) FINAL

**UNRESOLVED**

(i) Strongest proved partial result.
- Exact equivalences for membership $d\in D(n)$ were proved (Lemmas 885.1--885.3).
- Explicit verified instances exist for $k=2$ and $k=3$ (given above), matching the statements in the problem text.

(ii) First gap (crisp).
Give a construction (or disproof) that for every $k$ produces distinct integers $N_1<\cdots<N_k$ and $k$ distinct integers $d_1,\dots,d_k$ such that
\[\forall i\in\{1,\dots,k\}\ \forall j\in\{1,\dots,k\}\ \exists a_{i,j}\in\mathbb N\text{ with }N_i=a_{i,j}(a_{i,j}+d_j).\]

(iii) Top 3 next moves.
1. Use Lemma 885.3 to rephrase the problem as finding many $N_i$ for which $4N_i+d^2$ is a square for many common $d$; attempt to build parametric families.
2. Search computationally for a $k=5$ example to guess a pattern of differences and factor pairs, then try to prove the pattern.
3. Explore whether combining solutions of the two-difference equation $a(a+d)=b(b+e)$ (a quadratic Diophantine equation) can be iterated to force many common differences.

(iv) Minimal counterexample structure.
If the conjecture fails, let $k_0$ be the smallest $k$ for which no such $N_1<\cdots<N_k$ exist. A minimal counterexample would manifest as: for every choice of $k_0$ integers, the intersection of their $D(\cdot)$ sets has size at most $k_0-1$. Such a failure would likely already appear for $k_0=5$, so computational searches for $k=5$ are structurally informative.


