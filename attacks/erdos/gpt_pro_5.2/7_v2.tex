### 1) ROUND-2 OBJECTIVE

**(C) Obstruction / correction.**
Round‑1’s “formal restatement” did **not** explicitly require the moduli to be *distinct*, while the subsequent investigation (and the classic Erdős–Selfridge “odd covering problem”) implicitly does. Without distinctness, the posed question has an immediate trivial “yes.” The **minimal corrected statement** consistent with Round‑1’s intent and the literature is:

> **Odd covering problem (corrected):** Does there exist a **covering system with distinct moduli**, all **odd integers > 1**?

This is the standard “odd covering system” question and is still listed as open in current references. ([Erdős Problems][1])

I will (i) close the definitional gap and (ii) strictly advance the distinct‑moduli investigation by proving a **new general obstruction lemma** (union bound + CRT) that **replaces** Round‑1’s brute‑force Proposition 2 and **extends** it to a larger finite range.

---

### 2) ROUND-1 FOUNDATION USED

I rely on the following Round‑1 results/ideas (without re‑proving them):

1. **Lemma 1 (Round‑1):** If ({x\equiv a_i\pmod{m_i}}) covers (\mathbb Z), then (\sum_i 1/m_i \ge 1).
2. **Reduction mod (L=\mathrm{lcm}(m_i)):** coverage of (\mathbb Z) ⇔ coverage of all residues mod (L).
3. **Proposition 2 (Round‑1):** computationally, no cover using exactly one class for each modulus in ({3,5,7,9,11,13,15}).

I will **strengthen** Proposition 2 by giving a short *non‑computational* proof, and then extend the method.

---

### 3) NEW INSIGHT / TOOL (ROUND-2)

**New Lemma (Coprime‑block obstruction via union bound + CRT).**
If the moduli split into blocks whose lcms are pairwise coprime, and in each block the reciprocal sum is (<1), then **no choice of residues can cover**: there is always an integer simultaneously avoiding every congruence.

This is strictly new relative to Round‑1: it converts certain “candidate lists” into **deterministic impossibility** results without enumeration, and it scales to bigger modulus lists.

---

### 4) ATTACK PLAN (ROUND-2)

**Gaps after Round‑1:**

* (Gap 0) The problem statement ambiguity: distinct moduli were not enforced, yet Round‑1 proceeded as if they were.
* (Gap 1) Proposition 2 was proved only by exhaustive computation; no structure was extracted.
* (Gap 2) No extension beyond the ( \le 15) modulus universe.

**Plan to overcome:**

1. **Close Gap 0:** Note the trivial non‑distinct odd covering; then state the minimal corrected “distinct odd moduli” version (the actual open problem). ([Erdős Problems][1])
2. **Close Gap 1:** Prove the coprime‑block lemma and use it to re‑prove Proposition 2 *uniformly for all residues*.
3. **Close Gap 2:** Apply the lemma to all odd moduli (\le 31) to get a **new unconditional lower bound**: any distinct odd covering system must use some modulus (\ge 33).

---

### 5) WORK (ROUND-2)

#### 5.1 The definitional correction

* If moduli need **not** be distinct, the original question “all moduli odd?” is trivially **YES**:
  [
  {x\equiv 0!!\pmod 3,;;x\equiv 1!!\pmod 3,;;x\equiv 2!!\pmod 3}
  ]
  covers (\mathbb Z) because every integer has a residue (0,1,) or (2) modulo (3).

* The **nontrivial** problem (and the one Round‑1 was actually addressing) is the **distinct‑moduli** odd covering problem, as standardly formulated. ([Erdős Problems][1])

From here on, “odd covering system” means: **distinct** odd moduli (>1). ([Erdős Problems][1])

---

#### 5.2 New Lemma: Coprime‑block obstruction

**Lemma 2 (Coprime‑block obstruction).**
Let (M) be a finite set of moduli (m\ge 2), and for each (m\in M) fix a residue (a_m\pmod m).
Suppose (M) can be partitioned into blocks
[
M = M_1 \sqcup \cdots \sqcup M_r
]
such that the integers
[
L_j := \mathrm{lcm}(M_j)
]
are **pairwise coprime**, and for every (j),
[
\sum_{m\in M_j} \frac{1}{m} < 1.
]
Then there exists an integer (x) such that
[
x\not\equiv a_m \pmod m \quad \text{for every } m\in M.
]
In particular, no selection of one congruence class for each (m\in M) can cover (\mathbb Z).

**Proof.**
Fix (j). Work modulo (L_j). For each (m\in M_j), the congruence (x\equiv a_m\pmod m) corresponds to a set of residues modulo (L_j) of size exactly (L_j/m) (because (m\mid L_j)). Let that residue-set be (R_{m}\subseteq \mathbb Z/L_j\mathbb Z).

By the union bound,
[
\left|\bigcup_{m\in M_j} R_m\right|
\le \sum_{m\in M_j} |R_m|
= \sum_{m\in M_j} \frac{L_j}{m}
= L_j\sum_{m\in M_j}\frac{1}{m}
< L_j.
]
Hence the union is a **proper** subset of (\mathbb Z/L_j\mathbb Z), so we can choose a residue class (b_j \pmod{L_j}) outside it. That means:
[
b_j \not\equiv a_m \pmod m \quad\text{for all } m\in M_j.
]

Now the (L_j) are pairwise coprime, so by the Chinese Remainder Theorem there exists an integer (x) satisfying
[
x \equiv b_j \pmod{L_j}\quad (j=1,\dots,r).
]
Fix any (m\in M_j). Since (m\mid L_j), the congruence (x\equiv b_j\pmod{L_j}) implies (x\equiv b_j\pmod m), and therefore (x\not\equiv a_m\pmod m). This holds for every (m\in M). ∎

---

#### 5.3 Strengthened replacement for Round‑1 Proposition 2 (no computation needed)

**Proposition 2′ (uniform nonexistence for ({3,5,7,9,11,13,15})).**
For **every** choice of residues ((a_3,a_5,a_7,a_9,a_{11},a_{13},a_{15})), there exists an integer (x) avoiding all congruences
[
x\not\equiv a_m \pmod m \quad (m\in{3,5,7,9,11,13,15}).
]
Hence no covering system can consist of exactly one class for each modulus in ({3,5,7,9,11,13,15}).

**Proof.**
Partition the moduli into two blocks:
[
M_1={3,5,9,15},\qquad M_2={7,11,13}.
]
Their lcms are (L_1=\mathrm{lcm}(3,5,9,15)=45) and (L_2=\mathrm{lcm}(7,11,13)=1001), and (\gcd(45,1001)=1).

Compute the reciprocal sums (equivalently, the union‑bound counts modulo the block lcms):

* Modulo (45): sizes are (45/3=15,;45/5=9,;45/9=5,;45/15=3), so
  [
  15+9+5+3=32<45;;\Longrightarrow;;\sum_{m\in M_1}\frac1m=\frac{32}{45}<1.
  ]
* Modulo (1001): sizes are (1001/7=143,;1001/11=91,;1001/13=77), so
  [
  143+91+77=311<1001;;\Longrightarrow;;\sum_{m\in M_2}\frac1m=\frac{311}{1001}<1.
  ]

Thus Lemma 2 applies, producing an (x) avoiding all seven congruences. ∎

**Strict strengthening over Round‑1:** this removes the exhaustive search entirely and proves a stronger, *residue‑uniform* impossibility statement.

---

#### 5.4 New extension: no distinct odd covering system with all moduli (\le 31)

Let
[
M_{\le 31}={m\in\mathbb Z_{\ge2}: m\ \text{odd},\  m\le 31}
={3,5,7,9,11,13,15,17,19,21,23,25,27,29,31}.
]

**Theorem 3 (new finite-range obstruction).**
For every choice of residues ({a_m\pmod m}*{m\in M*{\le 31}}), there exists an integer (x) such that
[
x\not\equiv a_m \pmod m\quad\text{for all }m\in M_{\le 31}.
]
In particular, **no** covering system can have all distinct odd moduli (\le 31). Consequently, any distinct odd covering system (if it exists) must include some modulus (\ge 33).

**Proof.**

**Step 1: Partition into coprime blocks.**
Consider the block of all ({3,5,7})-smooth odd moduli up to 31:
[
A:={3,5,7,9,15,21,25,27}.
]
Let (L_A=\mathrm{lcm}(A)=3^3\cdot 5^2\cdot 7=27\cdot 25\cdot 7=4725).

The remaining moduli in (M_{\le 31}) are the primes
[
P:={11,13,17,19,23,29,31}.
]
Each (p\in P) is coprime to (4725), and the primes are mutually coprime. So we have a partition into blocks
[
M_{\le 31}=A ;\sqcup; {11};\sqcup;{13};\sqcup;\cdots;\sqcup;{31},
]
whose lcms are (4725,11,13,17,19,23,29,31), pairwise coprime.

**Step 2: Check the “sum < 1” condition on each block.**

* For each singleton prime block ({p}), clearly (\sum_{m\in{p}}1/m = 1/p<1).

* For the block (A), compute sizes modulo (L_A=4725):
  [
  \frac{4725}{3}=1575,;
  \frac{4725}{5}=945,;
  \frac{4725}{7}=675,;
  \frac{4725}{9}=525,;
  \frac{4725}{15}=315,;
  \frac{4725}{21}=225,;
  \frac{4725}{25}=189,;
  \frac{4725}{27}=175.
  ]
  Summing,
  [
  1575+945+675+525+315+225+189+175 = 4624 < 4725.
  ]
  Therefore
  [
  \sum_{m\in A}\frac{1}{m}=\frac{4624}{4725}<1.
  ]

**Step 3: Apply Lemma 2 and conclude.**
Lemma 2 applies to this coprime block decomposition, so for any choice of residues ({a_m}) there is an integer (x) avoiding every congruence (x\equiv a_m\pmod m) for all (m\in M_{\le 31}). Hence no selection of residue classes on those moduli can cover (\mathbb Z).

Finally, if a distinct odd covering system had all moduli (\le 31), its modulus set would be a subset of (M_{\le 31}); but any cover for a subset would extend to a cover for the full set (M_{\le 31}) by adding arbitrary extra congruences (union only grows), contradicting the impossibility just proved. Therefore no distinct odd covering system can have maximum modulus (\le 31), so any such system must contain some modulus (\ge 33). ∎

---

### 6) ADVERSARIAL VERIFICATION

* **Quantifiers / “for all residues” check:**
  Lemma 2 and its applications do not depend on any specific residue choices; the union bound uses only the *sizes* (L_j/m), which are residue‑independent. Thus Proposition 2′ and Theorem 3 truly hold uniformly over all residue assignments.

* **CRT hypotheses:**
  In Proposition 2′, (\gcd(45,1001)=1) (prime factors ({3,5}) vs ({7,11,13})).
  In Theorem 3, (4725=3^3\cdot 5^2\cdot 7) is coprime to each of (11,13,17,19,23,29,31), and those primes are mutually coprime; CRT applies.

* **Potential loophole: “but a covering might omit some small moduli”:**
  Addressed: if a subset of moduli admits a covering, then adding more congruences on additional moduli cannot destroy coverage (union is monotone). So disproving coverage for the **full** set (M_{\le 31}) automatically disproves coverage for every subset.

* **Boundary case where (\sum 1/m =1):**
  Not used; every block we applied Lemma 2 to satisfies a **strict** inequality, so the union bound yields a strict deficit in residue count, guaranteeing a complement residue exists.

All checks pass; no hidden assumption beyond distinctness of moduli in the “odd covering problem” interpretation.

---

### 7) FINAL

**UNRESOLVED (BUT STRICTLY ADVANCED)**

* The original question **without** distinctness is trivial “YES” (e.g. the three classes mod (3)).
* Under the **minimal corrected** and standard interpretation (distinct odd moduli (>1)), the existence problem remains open. ([Erdős Problems][1])
* **Strict advances beyond Round‑1:**

  1. A **new general obstruction lemma** (Lemma 2) using union bound + CRT.
  2. A **noncomputational, residue‑uniform** proof strengthening Round‑1 Proposition 2.
  3. A **new finite-range obstruction:** no distinct odd covering system can have all moduli (\le 31) (hence any candidate must include some modulus (\ge 33)).

---

### 8) COMPLETION ESTIMATE (MANDATORY)

**COMPLETION: 35%**

---

### 9) REFERENCES

* Erdős Problems page for Problem #7 (odd covering system) — definition/context and current open status. ([Erdős Problems][1])

[1]: https://www.erdosproblems.com/prizes/25/open "https://www.erdosproblems.com/prizes/25/open"
