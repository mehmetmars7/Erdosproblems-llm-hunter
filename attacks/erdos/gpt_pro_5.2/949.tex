
\section*{Erd\H{o}s Problem \#949}

\subsection*{FORMAL RESTATEMENT}
Let $\mathfrak c:=|\mathbb R|$.

\medskip
\noindent\textbf{Literal statement (as written).}
Given a set $S\subset\mathbb R$ such that there do not exist $a,b,c\in S$ with $a+b=c$, must there exist a set $A\subseteq\mathbb R\setminus S$ with $|A|=\mathfrak c$ such that
\[
A+A:=\{x+y:x,y\in A\}\subseteq \mathbb R\setminus S?
\]
(Edge cases: the condition rules out $0\in S$ because $0+0=0$ would be a solution.)

\medskip
\noindent\textbf{Variant mentioned in the problem text.}
Assume instead that $S$ is \emph{Sidon} in $(\mathbb R,+)$, meaning:
for any $s_1,s_2,s_3,s_4\in S$, if $s_1+s_2=s_3+s_4$ then $\{s_1,s_2\}=\{s_3,s_4\}$.
Must there exist $A\subseteq\mathbb R\setminus S$ with $|A|=\mathfrak c$ and $A+A\subseteq\mathbb R\setminus S$?

\subsection*{QUICK LITERATURE/CONTEXT CHECK}
The extracted problem text records:
(i) the general sum-free version above (no $a+b=c$ in $S$) is open on the ErdosProblems site;
(ii) Erd\H{o}s suggested strengthening to Sidon sets;
(iii) a positive proof for the Sidon variant was posted in the comments (attributed there to ``AlphaProof'').
I will not use any external results beyond what is stated in the problem text; I reproduce full proofs below.

\subsection*{ATTACK PLAN}
\begin{itemize}
\item \textbf{Proof track (partial):} Prove the statement when $|S|<\mathfrak c$ by taking a maximal $A$ (Zorn) with $A\cap S=\emptyset$ and $(A+A)\cap S=\emptyset$, and use maximality plus a cardinality bound to force $|A|=\mathfrak c$.
\item \textbf{Proof track (Sidon variant):} If $S$ is Sidon and $|S|=\mathfrak c$, give an explicit construction of $A$ (a translate of $S\setminus\{a\}$) and verify $A\cap S=\emptyset$ and $(A+A)\cap S=\emptyset$ using the Sidon property.
\item \textbf{Disproof track (general case):} Try to build a sum-free $S$ of size $\mathfrak c$ so ``additively thick'' that for every $A\subseteq\mathbb R\setminus S$ of size $\mathfrak c$, one has $(A+A)\cap S\neq\emptyset$. No such construction is known to me from the given text, so I will end \textbf{UNRESOLVED} for the literal statement unless I find an explicit counterexample.
\end{itemize}

\subsection*{WORK}
\noindent\textbf{Fast reality check (toy examples).}
\begin{itemize}
\item $S=(1,2)$ is sum-free since $a,b>1\Rightarrow a+b>2$, hence $a+b\notin(1,2)$. Taking $A=[0,0.4)$ gives $A\cap S=\emptyset$ and $A+A=[0,0.8)\cap S=\emptyset$.
\item If $S$ is countable, then one expects plenty of room to choose $A$ of size $\mathfrak c$ avoiding $S$ and $S$-hits in $A+A$.
\end{itemize}

\medskip
\noindent\textbf{Lemma 949.1 (small $S$ case; no Sidon or sum-free hypothesis needed).}
Let $S\subset\mathbb R$ satisfy $|S|<\mathfrak c$. Then there exists $A\subseteq\mathbb R\setminus S$ with $|A|=\mathfrak c$ such that $(A+A)\cap S=\emptyset$.

\medskip
\noindent\emph{Proof.}
Consider the collection
\[
\mathcal F:=\{A\subseteq\mathbb R:\ A\cap S=\emptyset\ \text{and}\ (A+A)\cap S=\emptyset\},
\]
ordered by inclusion.
If $\{A_i\}_{i\in I}$ is a chain in $\mathcal F$, then $A:=\bigcup_{i\in I}A_i$ satisfies $A\cap S=\emptyset$ and
\[
(A+A)\cap S\subseteq \bigcup_{i\in I}(A_i+A_i)\cap S=\emptyset
\]
because for any $x,y\in A$ there exists $j\in I$ with $x,y\in A_j$ (chain property), hence $x+y\in A_j+A_j$.
Thus every chain has an upper bound, so by Zorn's lemma there exists a maximal $A\in\mathcal F$.

We claim $|A|=\mathfrak c$.
By maximality, for every $x\in\mathbb R\setminus A$ at least one of the following holds:
\begin{enumerate}
\item $x\in S$ (otherwise $A\cup\{x\}$ would still avoid $S$),
\item $2x\in S$ (otherwise $2x\notin S$ so the new sum $x+x$ would not hit $S$),
\item there exists $a\in A$ with $a+x\in S$ (otherwise all new sums $a+x$ would avoid $S$).
\end{enumerate}
Equivalently,
\begin{equation}\label{eq:maximal-cover}
\mathbb R\subseteq A\ \cup\ \bigcup_{a\in A}(S-a)\ \cup\ S\ \cup\ (S/2),
\end{equation}
where $S-a:=\{s-a:s\in S\}$ and $S/2:=\{s/2:s\in S\}$.

Since $|S|<\mathfrak c$, also $|S\cup (S/2)|<\mathfrak c$.
Hence $|(\mathbb R\setminus S)\cap (\mathbb R\setminus S/2)|=\mathfrak c$.
From \eqref{eq:maximal-cover},
\[
(\mathbb R\setminus S)\cap (\mathbb R\setminus S/2)\subseteq A\cup \bigcup_{a\in A}(S-a).
\]
Now each translate $S-a$ has cardinality $|S|$, and there are $|A|$ many translates. Therefore
\[
\left|A\cup \bigcup_{a\in A}(S-a)\right|\le |A|+|A|\,|S|.
\]
If $|A|<\mathfrak c$, then (since $|S|<\mathfrak c$ and infinite-cardinal arithmetic gives $|A|+|A||S|=\max(|A|,|S|)<\mathfrak c$) the right-hand side would be $<\mathfrak c$, contradicting that the left-hand side has size $\ge \mathfrak c$.
So $|A|=\mathfrak c$.
\hfill $\square$

\medskip
\noindent\textbf{Lemma 949.2 (Sidon sets of size continuum; explicit construction).}
Assume $S\subset\mathbb R$ is Sidon and $|S|=\mathfrak c$. If $S$ contains some $a\ne 0$, define
\[
A:=\bigl((S\setminus\{a\})-a/2\bigr)\setminus S.
\]
Then $|A|=\mathfrak c$, $A\cap S=\emptyset$, and $(A+A)\cap S=\emptyset$.

\medskip
\noindent\emph{Proof.}
By definition $A\cap S=\emptyset$.
We first show $|A|=\mathfrak c$.
It suffices to show that
\begin{equation}\label{eq:intersection-at-most-one}
\bigl((S\setminus\{a\})-a/2\bigr)\cap S
\ \text{has at most one element}.
\end{equation}
Indeed, the translate $(S\setminus\{a\})-a/2$ has cardinality $|S\setminus\{a\}|=\mathfrak c$, and removing a set of size at most one does not change cardinality $\mathfrak c$.

To prove \eqref{eq:intersection-at-most-one}, suppose $x$ lies in the intersection.
Then $x\in S$ and $x=s-a/2$ for some $s\in S\setminus\{a\}$, i.e. $s=x+a/2\in S$.
Thus $x,\ x+a/2\in S$.
If there were two distinct such $x_1\ne x_2$ in the intersection, we would have
\[
(x_1+a/2)+x_2 = (x_2+a/2)+x_1,
\]
with all four terms in $S$.
By the Sidon property, the unordered pairs must coincide:
\[
\{x_1+a/2,\ x_2\}=\{x_2+a/2,\ x_1\}.
\]
Since $a\ne 0$, we cannot have $x_1+a/2=x_1$ or $x_2+a/2=x_2$.
The only way the sets coincide is then $x_1+a/2=x_2$ and $x_2+a/2=x_1$, which implies $a=0$, contradiction.
Hence the intersection has at most one element, proving \eqref{eq:intersection-at-most-one} and therefore $|A|=\mathfrak c$.

Next we prove $(A+A)\cap S=\emptyset$.
Take any $x,y\in A$.
Then there exist $s,t\in S\setminus\{a\}$ with
\[
x=s-a/2,\qquad y=t-a/2.
\]
So
\[
x+y = s+t-a.
\]
Suppose for contradiction that $x+y\in S$.
Then we have an additive relation in $S$:
\[
(s+t) = (x+y)+a,
\]
where $s,t\in S\setminus\{a\}$ and $(x+y),a\in S$.
By the Sidon property applied to $s+t=(x+y)+a$, we must have
\[
\{s,t\}=\{x+y,a\}.
\]
But $a\notin\{s,t\}$ by construction ($s,t\in S\setminus\{a\}$), contradiction.
Therefore $x+y\notin S$ for all $x,y\in A$, i.e. $(A+A)\cap S=\emptyset$.
\hfill $\square$

\medskip
\noindent\textbf{Corollary 949.3 (Sidon variant is true).}
If $S\subset\mathbb R$ is Sidon, then there exists $A\subseteq\mathbb R\setminus S$ with $|A|=\mathfrak c$ such that $A+A\subseteq\mathbb R\setminus S$.

\medskip
\noindent\emph{Proof.}
If $|S|<\mathfrak c$, apply Lemma~949.1.
If $|S|=\mathfrak c$ and $S$ contains some $a\ne0$, apply Lemma~949.2.
(If $S=\{0\}$, take any $A$ of size $\mathfrak c$ avoiding $0$ and with $0\notin A+A$, e.g. $A\subset (0,\infty)$.)
\hfill $\square$

\subsection*{VERIFICATION}
\begin{itemize}
\item Lemma 949.1: Checked that unions of chains preserve $(A+A)\cap S=\emptyset$ using the chain property (both summands land in some $A_j$). The maximality covering \eqref{eq:maximal-cover} was derived by contrapositive: if $x$ avoids $S$, $S/2$, and all $S-a$, then adding $x$ keeps sums out of $S$.
\item Cardinal arithmetic in Lemma 949.1 uses only that if $\kappa,\mu<\mathfrak c$ are infinite cardinals, then $\kappa+\kappa\mu=\max(\kappa,\mu)<\mathfrak c$; this is valid in ZFC.
\item Lemma 949.2: Verified the ``intersection at most one'' claim uses Sidon and $a\ne0$; if $a=0$ then construction degenerates. The $A+A$ disjointness uses that $a\notin S\setminus\{a\}$.
\item Corollary 949.3: handled the degenerate case $S=\{0\}$ separately.
\end{itemize}

\subsection*{FINAL}
**UNRESOLVED**

(i) \textbf{Strongest proved partial result.}
If $|S|<\mathfrak c$ (no further hypotheses), there exists $A\subseteq\mathbb R\setminus S$ with $|A|=\mathfrak c$ and $A+A\subseteq\mathbb R\setminus S$ (Lemma 949.1). Moreover, if $S$ is Sidon (any cardinality), the answer is \emph{yes} (Corollary 949.3).

(ii) \textbf{First gap.}
For a general sum-free set $S\subset\mathbb R$ of cardinality $\mathfrak c$ (no Sidon assumption), show either:
\begin{quote}
there exists $A\subseteq\mathbb R\setminus S$ with $|A|=\mathfrak c$ and $(A+A)\cap S=\emptyset$,
\end{quote}
or construct an explicit counterexample $S$.

(iii) \textbf{Top 3 next moves.}
\begin{enumerate}
\item Try to adapt Lemma 949.2 without Sidon: find a structured large subset $T\subseteq \mathbb R\setminus S$ with $T+T$ avoiding $S$ by choosing a translate/dilate $T=U-t$ for some large $U\subseteq S$.
\item Attempt a counterexample: build sum-free $S$ of size $\mathfrak c$ such that for every continuum-sized $A\subseteq \mathbb R\setminus S$ one has $(A+A)\cap S\ne\emptyset$ (``$S$ hits every large sumset'').
\item Study maximal (under inclusion) sum-free sets $S$ in $(\mathbb R,+)$ and analyze the structure of $\mathbb R\setminus S$; determine whether maximality forces existence of the required $A$.
\end{enumerate}

(iv) \textbf{What a minimal counterexample would likely look like.}
A counterexample (if it exists) must have $|S|=\mathfrak c$ and must accumulate at $0$ (otherwise a sufficiently small neighborhood of $0$ would give an $A$ with $A+A$ still inside that neighborhood and disjoint from $S$). It must also be additively ``thick'' in the sense that $S$ intersects $A+A$ for every continuum-sized $A\subseteq\mathbb R\setminus S$.


