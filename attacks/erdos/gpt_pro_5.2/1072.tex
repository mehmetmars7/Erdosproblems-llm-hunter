% Erdos Problem #1072

1) FORMAL RESTATEMENT
For a prime p, define
  f(p) := the least integer n >= 1 such that p divides (n! + 1).
Wilson's theorem implies (p-1)! ≡ -1 (mod p), hence p divides ((p-1)!+1), so f(p) is always defined and f(p) <= p-1.

Question A: Are there infinitely many primes p with f(p) = p-1?
Question B: Does f(p)/p -> 0 for almost all primes p?
(Here "almost all primes" means: #{p <= x : the property fails} = o(pi(x)) as x -> infinity.)

2) QUICK LITERATURE/CONTEXT CHECK
The problem statement itself notes:
- f(p) <= p-1 by Wilson.
- It is conjectured there are infinitely many primes with f(p)=p-1, but also a heuristic that the number of such primes <= x is o(x/log x).
No other external results are assumed here.

3) ATTACK PLAN
Two directions are natural:
(A) Structural: understand solutions of n! ≡ -1 (mod p) for n < p, and derive constraints on where they can occur.
(B) Data: compute f(p) for small primes to sanity-check the range of behaviors.
I provide two problem-specific lemmas that give (i) the universal Wilson upper bound and (ii) a sharp characterization of the case when the witnessing factorial is smaller than p.

4) WORK
Lemma 1072.1 (Wilson bound).
For every prime p, f(p) <= p-1.
Proof. Wilson's theorem gives (p-1)! ≡ -1 (mod p). Therefore p divides ((p-1)!+1), so the minimum n with p | (n!+1) satisfies n <= p-1. QED.

Lemma 1072.2 (When n! < p, the divisibility forces p = n!+1).
Let p be prime and suppose p | (n!+1) for some integer n >= 1.
If n! < p, then p = n!+1 and moreover f(p) = n.
Conversely, if n!+1 is itself prime p, then f(p)=n.
Proof. If p | (n!+1) then p <= n!+1. If also n! < p then n!+1 <= p, so n!+1 = p.
Now assume p = n!+1. For any m < n we have m!+1 < n!+1 = p, hence p cannot divide m!+1. Therefore f(p)=n.
The converse implication is exactly the same argument read in reverse. QED.

Lemma 1072.3 (Complementary factorial identity and a consequence).
Let p be an odd prime and let n be an integer with 0 <= n <= p-1. Then
  n! * (p-1-n)! ≡ (-1)^(n+1) (mod p).
In particular, if 1 <= n <= p-2 and p | (n!+1), then (p-1-n)! ≡ (-1)^n (mod p), and also the complementary product (n+1)(n+2)...(p-1) ≡ 1 (mod p).
Proof. Start from Wilson: (p-1)! ≡ -1 (mod p). Also
  (p-1)! = n! * product_{k=n+1}^{p-1} k.
Rewrite the complementary product by changing variables: for j=1,...,p-1-n, we have p-j ≡ -j (mod p), hence
  product_{k=n+1}^{p-1} k = product_{j=1}^{p-1-n} (p-j) ≡ (-1)^(p-1-n) * (p-1-n)! (mod p).
Therefore
  -1 ≡ (p-1)! ≡ n! * (-1)^(p-1-n) * (p-1-n)! (mod p).
Multiply both sides by (-1)^(p-1-n). Since p is odd, (-1)^(p-n) = (-1)^(n+1). This gives
  n! * (p-1-n)! ≡ (-1)^(n+1) (mod p).
Now assume additionally that p | (n!+1), so n! ≡ -1 (mod p). Substituting into the identity yields
  (-1) * (p-1-n)! ≡ (-1)^(n+1) (mod p),
so (p-1-n)! ≡ (-1)^n (mod p).
Also, from (p-1)! ≡ n! * product_{k=n+1}^{p-1} k and (p-1)! ≡ -1 and n! ≡ -1, we get
  (-1) ≡ (-1) * product_{k=n+1}^{p-1} k (mod p),
hence product_{k=n+1}^{p-1} k ≡ 1 (mod p). QED.

5) VERIFICATION (small cases)
I computed f(p) for all primes p < 500 by direct modular multiplication.
- Primes p < 500 with f(p) = p-1 are:
  2, 3, 5, 13, 17, 31, 37, 41, 53, 73, 89, 97, 101, 107, 113, 151, 157, 167, 173, 181, 197, 211, 223, 229, 241, 281, 283, 313, 331, 337, 349, 353, 373, 409, 421, 433, 439, 457, 487.
  Count: 39 out of 95 primes < 500.
- Examples where f(p) is much smaller than p: f(103)=6, f(269)=9, f(479)=15.
These computations are only sanity checks; they do not settle Questions A or B.

6) FINAL
UNRESOLVED
(i) Strongest proved partial result here: f(p) is always defined and f(p) <= p-1 (Lemma 1072.1). If p divides (n!+1) with n! < p then necessarily p = n!+1 and f(p)=n (Lemma 1072.2). Also, any solution n < p-1 forces explicit complementary congruences (Lemma 1072.3).
(ii) First gap (crisp): Prove or disprove that there are infinitely many primes p for which no n <= p-2 satisfies n! ≡ -1 (mod p) (equivalently f(p)=p-1). Separately, prove or disprove that for almost all primes p there exists some n = o(p) with n! ≡ -1 (mod p).
(iii) Top 3 next moves (concrete):
  1. Try to show that for most primes p there exists n <= p^alpha (for some fixed alpha<1) with n! ≡ -1 (mod p), by studying distribution of factorial residues modulo p.
  2. Search for a mechanism forcing f(p)=p-1 along an infinite family (e.g. show that n!+1 avoids 0 mod p for all n<=p-2 in that family).
  3. Extend computations to much larger primes to estimate the frequency of f(p)=p-1 and the empirical distribution of f(p)/p.
(iv) Minimal counterexample structure: To refute Question B (f(p)/p -> 0 for almost all primes), one would need a fixed c>0 and a positive-density set of primes p for which n! ≠ -1 (mod p) for every n <= c p.


