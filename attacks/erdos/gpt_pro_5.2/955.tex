
Erdos Problem 955.

1) FORMAL RESTATEMENT.
For $n\in\mathbb{N}$ define
\[
 s(n):=\sigma(n)-n=\sum_{\substack{d\mid n\\ d<n}} d,
\]
the sum of the proper divisors of $n$.
For a set $A\subseteq\mathbb{N}$, write $A$ has (natural) density $0$ if
\[
\lim_{x\to\infty} \frac{|A\cap [1,x]|}{x}=0.
\]
The conjecture asks:

If $A\subseteq\mathbb{N}$ has density $0$, must the preimage
$s^{-1}(A):=\{n\in\mathbb{N}: s(n)\in A\}$ also have density $0$?

2) QUICK LITERATURE/CONTEXT CHECK.
The problem file attributes this conjecture to Erd\H{o}s, Granville, Pomerance and Spiro.
It records several partial results (e.g. for very sparse $A$, and results depending on bounds for $s(n)$) and notes that there exist sets $A$ of positive density with $s^{-1}(A)=\varnothing$.

3) ATTACK PLAN.
(A) Prove robust inequalities relating $s(n)$ to elementary parameters of $n$ (size, smallest prime factor, divisor count).
(B) Use these inequalities to control $\{n\le x: s(n)\in A\}$ for special classes of sparse sets $A$.
(C) Run a sieve for small $x$ to see how often $s(n)$ hits various zero-density targets (squares, primes, etc.).

4) WORK.

FAST REALITY CHECK (computations for small targets).
Using a divisor-sum sieve up to $200000$, I computed the proportions
\[\#\{n\le X: s(n)\text{ is a square}\}/X\quad\text{and}\quad \#\{n\le X: s(n)\text{ is prime}\}/X.\]
The results were:
\[
\begin{array}{c|c|c||c|c}
X & \#\{s(n)\text{ square}\} & \text{ratio} & \#\{s(n)\text{ prime}\} & \text{ratio}\\\hline
10^2 & 36 & 0.36 & 15 & 0.15\\
10^3 & 216 & 0.216 & 125 & 0.125\\
10^4 & 1422 & 0.1422 & 1045 & 0.1045\\
5\cdot 10^4 & 5593 & 0.11186 & 4655 & 0.0931\\
10^5 & 10312 & 0.10312 & 8801 & 0.08801\\
2\cdot 10^5 & 19070 & 0.09535 & 16734 & 0.08367
\end{array}
\]
These ratios decrease over this range, consistent with (but not proving) density $0$ for these particular choices of $A$.

Lemma 955.1 (A universal lower bound for composite $n$).
If $n$ is composite, then
\[
 s(n)\ge 1+2\sqrt{n}.
\]

Proof.
Since $n$ is composite, choose a divisor $d$ with $1<d\le \sqrt{n}$.
Then $d$ and $n/d$ are both proper divisors of $n$ (because $d>1$ and $n/d<n$).
Therefore
\[
 s(n)\ge 1 + d + \frac{n}{d}.
\]
By AM--GM, $d + n/d \ge 2\sqrt{n}$, so $s(n)\ge 1+2\sqrt{n}$.
\hfill $\square$

Lemma 955.2 (Lower bound using the smallest prime factor).
Let $n\ge 2$ and let $p$ be the smallest prime divisor of $n$.
Then
\[
 s(n)\ge \frac{n}{p}+1.
\]

Proof.
The integer $n/p$ is a proper divisor of $n$ because $p\ge 2$.
Also $1$ is a proper divisor of $n$.
Since $s(n)$ is the sum of all proper divisors, it is at least the sum of these two:
$s(n)\ge 1+n/p$.
\hfill $\square$

Lemma 955.3 (Crude universal upper bound in terms of the divisor function).
Let $\tau(n):=\#\{d:d\mid n\}$ be the number of positive divisors of $n$.
Then
\[
 s(n)\le (\tau(n)-1)\,n.
\]

Proof.
There are exactly $\tau(n)-1$ proper divisors of $n$.
Each proper divisor $d$ satisfies $d\le n$.
Summing the $\tau(n)-1$ terms gives $s(n)\le (\tau(n)-1)n$.
\hfill $\square$

5) VERIFICATION.
• Lemma 955.1: the choice of $d\le \sqrt{n}$ is standard and always possible for composite $n$.
• Lemma 955.2: uses only two explicit proper divisors.
• Lemma 955.3: very weak but unconditional.
• Computations: the sieve method computed $s(n)$ exactly by adding each divisor $d$ to multiples $2d,3d,\dots$.

6) FINAL.
UNRESOLVED.
(i) Strongest proved partial result: For composite $n$, $s(n)\ge 1+2\sqrt{n}$ and $s(n)\ge 1+n/p$ where $p$ is the least prime factor of $n$; also $s(n)\le (\tau(n)-1)n$.
(ii) First gap: turn these pointwise inequalities into a density statement for *arbitrary* zero-density sets $A$, i.e. show $\#\{n\le x: s(n)\in A\}=o(x)$ from the hypothesis $|A\cap[1,y]|=o(y)$.
(iii) Top 3 next moves:
  (1) Prove that for most $n\le x$, the value $s(n)$ lies in an interval $[c_1 n, c_2 n\log\log n]$ with a distribution that is sufficiently “mixing” to prevent concentration on a zero-density set.
  (2) For a given sparse $A$, try to bound solutions of $s(n)=a$ with $a\in A$ by structural restrictions on $n$ (e.g. on its prime factorization), then sum over $a\le y$.
  (3) Run targeted computations for adversarial-looking zero-density sets (e.g. squares, powers, very thin unions of intervals) to guess what the hardest $A$ might look like.
(iv) Minimal counterexample structure: a zero-density $A$ and a positive-density set $B$ such that $s(B)\subseteq A$; equivalently, $s(n)$ would have to fall into a very sparse set for a positive proportion of integers $n$.


