

\paragraph{FORMAL RESTATEMENT.}
Let $\sigma(n)=\sum_{d\mid n} d$ be the sum-of-divisors function for $n\in\mathbb{N}$.
Call $n$ \emph{$k$-perfect} if $\sigma(n)=k n$ for some integer $k\ge 1$.
Question: along any sequence of $k$-perfect numbers $n\to\infty$, must the corresponding multipliers satisfy
\[
k = o(\log\log n),
\]
meaning: for every $\varepsilon>0$ there exists $N(\varepsilon)$ such that if $n\ge N(\varepsilon)$ is $k$-perfect then $k\le \varepsilon\log\log n$.
Edge cases: $n=1$ is $1$-perfect; $k=2$ are the usual perfect numbers.

\paragraph{QUICK LITERATURE/CONTEXT CHECK.}
I did not use external sources. I only use the definitions and context stated in the problem text (e.g. that examples exist up to $k=11$).

\paragraph{ATTACK PLAN.}
\emph{Proof-track:} relate $k=\sigma(n)/n$ to the Euler product over prime powers in $n$; compare to $n/\varphi(n)$ and attempt to control $n/\varphi(n)$ in terms of $\log\log n$.
\emph{Disproof-track:} search for (computational) $k$-perfect numbers with unusually large $k$; attempt constructions where $n$ has many small prime factors.
Best path here: derive rigorous elementary bounds on $k$ (too weak to settle $o(\log\log n)$), and record sanity-check computations.

\paragraph{WORK.}
\textbf{Lemma 1 (Euler product and comparison with $n/\varphi(n)$).}
If $n=\prod_{p^a\parallel n} p^a$, then
\[
\frac{\sigma(n)}{n}=\prod_{p^a\parallel n}\frac{p^{a+1}-1}{p^a(p-1)}
=\prod_{p^a\parallel n}\frac{1-p^{-(a+1)}}{1-p^{-1}}.
\]
In particular,
\[
\frac{6}{\pi^2}\,\frac{n}{\varphi(n)}\le \frac{\sigma(n)}{n}\le \frac{n}{\varphi(n)}.
\]

\emph{Proof.}
The factorization of $\sigma$ over prime powers gives $\sigma(p^a)=(p^{a+1}-1)/(p-1)$, hence $\sigma(n)=\prod\sigma(p^a)$ and the displayed Euler product for $\sigma(n)/n$.
For the upper bound, each factor has numerator $\le 1$, so
\[
\frac{\sigma(n)}{n}\le \prod_{p\mid n}\frac{1}{1-p^{-1}}=\frac{n}{\varphi(n)}.
\]
For the lower bound, since $a+1\ge 2$ we have $p^{-(a+1)}\le p^{-2}$, so $1-p^{-(a+1)}\ge 1-p^{-2}$, giving
\[
\frac{\sigma(n)}{n}\ge \prod_{p\mid n}\frac{1-p^{-2}}{1-p^{-1}}
=\Bigl(\prod_{p\mid n}(1-p^{-2})\Bigr)\frac{n}{\varphi(n)}.
\]
Because each $(1-p^{-2})\in(0,1)$, the product over the subset $\{p\mid n\}$ is at least the product over all primes:
$\prod_{p\mid n}(1-p^{-2})\ge \prod_{p}(1-p^{-2})=1/\zeta(2)=6/\pi^2$.
\hfill$\square$

\textbf{Lemma 2 (elementary bound $n/\varphi(n)\le e\,\omega(n)$).}
Let $\omega(n)$ be the number of distinct prime divisors of $n$. Then
\[
\frac{n}{\varphi(n)}=\prod_{p\mid n}\frac{p}{p-1}\le e\,\omega(n)\le e\,\log_2 n.
\]
Consequently, for any $k$-perfect $n$ one has $k\le e\log_2 n$.

\emph{Proof.}
For each prime $p\mid n$, $\frac{p}{p-1}=1+\frac{1}{p-1}$.
Using $\log(1+x)\le x$ for $x>-1$,
\[
\log\Bigl(\frac{n}{\varphi(n)}\Bigr)=\sum_{p\mid n}\log\Bigl(1+\frac{1}{p-1}\Bigr)
\le \sum_{p\mid n}\frac{1}{p-1}.
\]
Order the distinct prime divisors as $p_1<p_2<\cdots<p_{\omega(n)}$.
The $j$th smallest prime satisfies $p_j\ge j+1$ (there are only $j$ integers in $\{2,3,\dots,j+1\}$), hence $p_j-1\ge j$, so
\[
\sum_{p\mid n}\frac{1}{p-1}\le \sum_{j=1}^{\omega(n)}\frac{1}{p_j-1}\le \sum_{j=1}^{\omega(n)}\frac{1}{j}
=H_{\omega(n)}\le 1+\log\omega(n).
\]
Exponentiating yields $n/\varphi(n)\le e\,\omega(n)$.
Finally $n\ge 2^{\omega(n)}$ gives $\omega(n)\le \log_2 n$.
For $k$-perfect $n$, Lemma 1 gives $k=\sigma(n)/n\le n/\varphi(n)$.
\hfill$\square$

\textbf{FAST REALITY CHECK (computation).}
I searched all $n\le 10^6$ for which $\sigma(n)/n$ is an integer. The results were:
\begin{verbatim}
(1,1), (6,2), (28,2), (120,3), (496,2), (672,3), (8128,2),
(30240,4), (32760,4), (523776,3).
\end{verbatim}
Thus among $n\le 10^6$ the largest multiplier found was $k=4$ (attained at $n=30240$ and $n=32760$).

\paragraph{VERIFICATION.}
Lemma 1: For $n=12$ one has $\sigma(12)/12 = 28/12 = 7/3$ and $n/\varphi(n)=12/4=3$, so indeed $7/3\le 3$.
Also $(6/\pi^2)(n/\varphi(n)) \approx 0.6079\cdot 3\approx 1.8237 \le 7/3$.
Lemma 2: For $n=30$, $n/\varphi(n)=(2/1)(3/2)(5/4)=15/4=3.75$ while $e\omega(n)=3e\approx 8.15$.

\paragraph{FINAL: UNRESOLVED.}
(i) \emph{Strongest proved partial result here.} For every $n$,
\[
\frac{6}{\pi^2}\frac{n}{\varphi(n)}\le \frac{\sigma(n)}{n}\le \frac{n}{\varphi(n)}\le e\,\omega(n)\le e\log_2 n,
\]
so in particular every $k$-perfect $n$ satisfies $k\le e\log_2 n$.
(ii) \emph{First gap.} Prove (or disprove) that for $k$-perfect $n$ one necessarily has $n/\varphi(n)=o(\log\log n)$ (or even $O(\log\log n)$), which via Lemma 1 would force $k=o(\log\log n)$.
(iii) \emph{Top 3 next moves.} (1) Establish an explicit upper bound for $n/\varphi(n)$ in terms of $\log\log n$ using only results allowed in the problem file. (2) Investigate whether $k$-perfect numbers necessarily have bounded $\omega(n)$ or other restrictions tightening Lemma 2. (3) Extend computations of known $k$-perfect numbers to record $k$ versus $\log\log n$.
(iv) \emph{What a minimal counterexample would look like.} A sequence of $k$-perfect numbers $n$ with $k$ comparable to $c\log\log n$ (or larger) would force $n/\varphi(n)$ large, hence $n$ with very many distinct small prime divisors and $\sigma(n)/n$ close to $n/\varphi(n)$ (i.e. large exponents on those primes so that the factors $1-p^{-(a+1)}$ are near $1$).


