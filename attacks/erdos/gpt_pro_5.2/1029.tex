
If $R(k)$ is the Ramsey number for $K_k$, the minimal $n$ such that every $2$-colouring of the edges of $K_n$ contains a monochromatic copy of $K_k$, then\[\frac{R(k)}{k2^{k/2}}\to \infty.\] In \cite{Er93} Erd\H{o}s offers \$100 for a proof of this and \$1000 for a disproof, but says 'this last offer is to some extent phoney: I am sure that [this] is true (but I have been wrong before).' Erd\H{o}s and Szekeres \cite{ErSz35} proved\[k2^{k/2} \ll R(k) \leq \binom{2k-1}{k-1}.\]One of the first applications of the probabilistic method pioneered by Erd\H{o}s gives\[R(k) \geq (1+o(1))\frac{1}{\sqrt{2}e}k2^{k/2},\]which Spencer \cite{Sp75} improved by a factor of $2$ to\[R(k) \geq (1+o(1))\frac{\sqrt{2}}{e}k2^{k/2}.\]See also [77] for a more general problem concerning $\lim R(k)^{1/k}$, and discussion of upper bounds for $R(k)$. References [Er93] Erd\H{o}s, Paul, Some of my favorite solved and unsolved problems in graph theory . Quaestiones Math. (1993), 333-350. [ErSz35] Erd\H{o}s, P. and Szekeres, G., A combinatorial problem in geometry . Compos. Math. (1935), 463-470. [Sp75] Spencer, Joel, Ramsey's theorem---a new lower bound . J. Combinatorial Theory Ser. A (1975), 108--115.

\subsection*{Erd\H{o}s Problem \#1029 --- Solution Attempt}

\textbf{FORMAL RESTATEMENT.}
For each integer $k\ge 2$, let $R(k)$ denote the diagonal Ramsey number $R(k,k)$: the least integer $n$ such that every 2-colouring of the edges of the complete graph $K_n$ contains a monochromatic copy of $K_k$.
The conjecture is
\[
\lim_{k\to\infty} \frac{R(k)}{k2^{k/2}} = +\infty.
\]
Equivalently: for every constant $C>0$ there exists $k_0$ such that for all $k\ge k_0$ one has $R(k)\ge C\,k2^{k/2}$.

\textbf{QUICK LITERATURE/CONTEXT CHECK.}
The problem statement itself records:
\begin{itemize}
\item the Erd\H{o}s--Szekeres upper bound $R(k)\le \binom{2k-1}{k-1}$,
\item a probabilistic-method lower bound with constant $\frac{1}{\sqrt{2}e}$ up to $(1+o(1))$,
\item Spencer's improvement by a factor $2$.
\end{itemize}
I do not use any external results beyond what is explicitly in the problem file.

\textbf{ATTACK PLAN.}
\emph{Proof track:} try to amplify the known probabilistic lower bounds by a factor that tends to infinity (e.g. by iterated random constructions or by showing ``random colouring works'' at larger $n$ than the first-moment threshold).

\emph{Disproof track:} try to show $R(k)\le Ck2^{k/2}$ for some constant $C$ (or even along an infinite subsequence), which would require a new type of upper bound.

Given current tools available in this session, I focus on producing rigorous baseline bounds and sanity checks, and then state the precise gap.

\textbf{WORK.}

\textbf{Lemma 1 (Erd\H{o}s--Szekeres recursion bound).}
For integers $s,t\ge 2$,
\[
R(s,t) \le \binom{s+t-2}{s-1}.
\]
In particular,
\[
R(k)=R(k,k) \le \binom{2k-2}{k-1} < \binom{2k-1}{k-1}.
\]

\emph{Proof.}
We first prove the standard recursion
\[
R(s,t) \le R(s-1,t) + R(s,t-1). \tag{R}
\]
Consider any 2-colouring (red/blue) of the edges of $K_N$ with
$N := R(s-1,t)+R(s,t-1)$.
Fix a vertex $v$. Let $R_v$ be the set of red neighbours of $v$ and $B_v$ the set of blue neighbours; then
$|R_v|+|B_v|=N-1$.
If $|R_v|\ge R(s-1,t)$, then by definition of $R(s-1,t)$ the induced complete graph on $R_v$ contains either
(i) a red $K_{s-1}$ or (ii) a blue $K_t$.
In case (i), adding the vertex $v$ gives a red $K_s$ (since $v$ is joined by red edges to all vertices of that $K_{s-1}$).
In case (ii) we already have a blue $K_t$.
Similarly, if $|B_v|\ge R(s,t-1)$, then the induced graph on $B_v$ contains either a red $K_s$ or a blue $K_{t-1}$; in the latter case adding $v$ yields a blue $K_t$.
Therefore every colouring of $K_N$ contains a red $K_s$ or a blue $K_t$, proving (R).

Now define $f(s,t):=\binom{s+t-2}{s-1}$.
It satisfies the same recursion
\[
 f(s,t)=f(s-1,t)+f(s,t-1),
\]
with boundary values $f(1,t)=f(s,1)=1$.
Also $R(1,t)=R(s,1)=1$ under the convention that $K_1$ is monochromatic.
By induction on $s+t$ using (R) and the recursion for $f$, we obtain $R(s,t)\le f(s,t)$.
Specializing to $s=t=k$ yields $R(k)\le \binom{2k-2}{k-1}$.
\qed

\medskip
\textbf{Lemma 2 (A fully explicit probabilistic lower bound at the $k2^{k/2}$ scale).}
For every integer $k\ge 3$,
\[
R(k) > \left\lfloor \frac{k}{e\sqrt{2}}\,2^{k/2}\right\rfloor.
\]
Equivalently, there exists a 2-colouring of the edges of $K_n$ with
$n=\left\lfloor \frac{k}{e\sqrt{2}}\,2^{k/2}\right\rfloor$
containing no monochromatic copy of $K_k$.

\emph{Proof.}
Fix $n$ and colour each edge of $K_n$ independently red/blue with probability $1/2$ each.
For each $k$-subset $S\subseteq [n]$ (vertex set), let $E_S$ be the event that $S$ spans a monochromatic $K_k$.
There are exactly two monochromatic colour choices, so
\[
\mathbb{P}(E_S)=2\cdot 2^{-\binom{k}{2}} = 2^{1-\binom{k}{2}}.
\]
Let $X$ be the number of monochromatic $K_k$ subgraphs. By linearity of expectation,
\[
\mathbb{E}[X]=\binom{n}{k} 2^{1-\binom{k}{2}}.
\]
We will show that for $n=\left\lfloor \frac{k}{e\sqrt{2}}\,2^{k/2}\right\rfloor$ and $k\ge 3$ we have $\mathbb{E}[X]<1$.
Then some colouring has $X=0$, implying $R(k)>n$.

To bound $\binom{n}{k}$, use
\[
\binom{n}{k} \le \frac{n^k}{k!}. \tag{3}
\]
We also need a concrete lower bound on $k!$.
\emph{Claim.} For every integer $k\ge 1$,
\[
 k! \ge \sqrt{2k}\,\left(\frac{k}{e}\right)^k. \tag{4}
\]
\emph{Proof of the claim.}
The function $\ln x$ is concave on $(0,\infty)$.
For each integer $1\le i\le k$, concavity implies that the average value of $\ln x$ on $[i-\tfrac12,i+\tfrac12]$ is at most $\ln i$:
\[
\int_{i-1/2}^{i+1/2} \ln x\,dx \le 1\cdot \ln i.
\]
Summing over $i=1,2,\dots,k$ gives
\[
\int_{1/2}^{k+1/2} \ln x\,dx \le \sum_{i=1}^k \ln i = \ln(k!).
\]
Evaluating the integral,
\[
\int_{1/2}^{k+1/2} \ln x\,dx = \bigl[x\ln x - x\bigr]_{1/2}^{k+1/2}
= (k+\tfrac12)\ln(k+\tfrac12) - (k+\tfrac12) + \tfrac12\ln 2 + \tfrac12.
\]
Since $\ln(k+\tfrac12)\ge \ln k$ for $k\ge 1$, this is at least
\[
(k+\tfrac12)\ln k - (k+\tfrac12) + \tfrac12\ln 2 + \tfrac12
= k\ln k - k + \tfrac12\ln(2k).
\]
Therefore
\[
\ln(k!) \ge k\ln k - k + \tfrac12\ln(2k).
\]
Exponentiating yields (4).
\hfill $\triangle$

Using (3) and (4),
\[
\mathbb{E}[X]\le 2\cdot \frac{n^k}{k!}\cdot 2^{-\binom{k}{2}}
\le 2\cdot \frac{n^k}{\sqrt{2k}(k/e)^k}\cdot 2^{-\binom{k}{2}}
= \sqrt{\frac{2}{k}}\,\left(\frac{en}{k}\right)^k 2^{-\binom{k}{2}}.
\]
Now choose $n\le \frac{k}{e\sqrt{2}}\,2^{k/2}$.
Then
\[
\left(\frac{en}{k}\right)^k \le \left(\frac{1}{\sqrt{2}}\,2^{k/2}\right)^k = 2^{-k/2} 2^{k^2/2} = 2^{\binom{k}{2}}.
\]
Substituting gives
\[
\mathbb{E}[X] \le \sqrt{\frac{2}{k}}.
\]
For every $k\ge 3$, $\sqrt{2/k}<1$, hence $\mathbb{E}[X]<1$.
Therefore there exists a colouring with $X=0$, i.e. no monochromatic $K_k$.
This implies $R(k)>n$.
\qed

\medskip
\textbf{FAST REALITY CHECK (small cases / computation).}
By brute force over all $2^{\binom{n}{2}}$ edge-colourings for $n=5,6$:
\begin{itemize}
\item There exists a 2-colouring of $K_5$ with no monochromatic triangle, but none for $K_6$.
Hence $R(3)=6$.
\end{itemize}
(Explicit example for $K_5$: colour the edges of a 5-cycle red and all remaining edges blue.)

\textbf{VERIFICATION.}
\begin{itemize}
\item Lemma 1: the recursion (R) is checked vertex-by-vertex and uses only the defining property of $R(s,t)$.
\item Lemma 2: the only analytic input is the elementary factorial bound (4), proved from concavity of $\ln$; all inequalities are explicit and checked for $k\ge 3$.
\end{itemize}

\textbf{FINAL.} \textbf{UNRESOLVED.}

(i) \emph{Strongest proved partial result.} A completely explicit constant-factor lower bound holds:
$R(k) > \left\lfloor \frac{k}{e\sqrt{2}}2^{k/2}\right\rfloor$ for all $k\ge 3$ (Lemma 2). Also $R(k)\le \binom{2k-2}{k-1}$ (Lemma 1).
Thus $\frac{R(k)}{k2^{k/2}}$ is bounded below by $\frac{1}{e\sqrt{2}}+o(1)$, but this does not diverge.

(ii) \emph{First gap (crisp).} Prove (or disprove) that $R(k)\ge f(k)\,k2^{k/2}$ for some function $f(k)\to\infty$.
Equivalently: show that for every fixed $C$ and all large $k$, there exists a 2-colouring of $K_{Ck2^{k/2}}$ with no monochromatic $K_k$.

(iii) \emph{Top 3 next moves.}
1. Try to push the probabilistic method beyond the first-moment threshold, e.g. prove that with positive probability a random colouring has at most $o(1)$ monochromatic $K_k$ and can be ``repaired'' to eliminate them while preserving size.
2. Attempt to build explicit constructions (iterative products/stepping-up style) yielding an extra factor in $k$ (or more) over the baseline $k2^{k/2}$.
3. Computationally: implement a SAT/DPLL search for small $k$ to see whether $R(k)/(k2^{k/2})$ is increasing and to look for structural patterns in near-extremal colourings.

(iv) \emph{What a minimal counterexample would likely look like.} A disproof would require a uniform constant $C$ and an infinite sequence $k\to\infty$ with $R(k)\le Ck2^{k/2}$, i.e. 2-colourings of $K_{\approx Ck2^{k/2}}$ that necessarily force monochromatic $K_k$ at this scale.
A proof would require exhibiting colourings on significantly larger $n$ (e.g. $n\ge k^{1+\varepsilon}2^{k/2}$ infinitely often, or stronger).


