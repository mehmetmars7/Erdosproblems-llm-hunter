% Erdos Problem #456

\textbf{FORMAL RESTATEMENT}

For each integer $n\ge 1$ let
\begin{itemize}
\item $p_n$ be the smallest prime congruent to $1\pmod n$;
\item $m_n$ be the smallest positive integer $m$ such that $n\mid \phi(m)$, where $\phi$ is Euler's totient.
\end{itemize}
Questions:
\begin{enumerate}
\item Is $m_n<p_n$ for almost all $n$?
\item Does $p_n/m_n\to\infty$ for almost all $n$?
\item Are there infinitely many primes $p$ such that $p-1$ is the only $n$ for which $m_n=p$?
\end{enumerate}

\textbf{QUICK LITERATURE/CONTEXT CHECK}

Only what is stated in the problem text is used: Linnik's theorem implies $p_n\le n^{O(1)}$; trivially $m_n\le p_n$; if $n=q-1$ for a prime $q$ then $m_n=p_n$; Erd\H{o}s claims it is easy to show infinitely many $n$ with $m_n<p_n$, and that $m_n/n\to\infty$ for almost all $n$; van Doorn notes that for $n=2^{2k+1}$ one has $m_n\le 2n$ and $p_n\ge 2n+1$.
No additional external results are imported here.

\textbf{ATTACK PLAN}

\begin{itemize}
\item Prove the basic inequality $m_n\le p_n$ and characterize the equality case $n=q-1$.
\item Prove van Doorn's explicit infinite family $n=2^{2k+1}$ giving strict inequality $m_n<p_n$.
\item FAST REALITY CHECK: compute $(p_n,m_n)$ for small $n$.
\end{itemize}

\textbf{WORK}

\emph{Fast reality check (exact computation for small $n$).}
By brute force for $2\le n\le 100$ (exact primality and totient), there are $14$ values with $m_n<p_n$; the first few are:
\[
(n,p_n,m_n,p_n/m_n)=(8,17,15,1.133\dots),\ (20,41,25,1.64),\ (24,73,35,2.085\dots),\ (32,97,51,1.901\dots),\ (55,331,121,2.735\dots).
\]
Many $n$ (including all $n=q-1$ with prime $q\le 101$) have $m_n=p_n$ in this range.

\medskip

\textbf{Lemma 456.1 (trivial upper bound $m_n\le p_n$).}
For every $n\ge 1$, $m_n\le p_n$.

\emph{Proof.}
By definition, $p_n$ is a prime with $p_n\equiv 1\pmod n$, so $n\mid (p_n-1)$. Since $\phi(p_n)=p_n-1$, we have $n\mid \phi(p_n)$, so $p_n$ is an admissible candidate in the defining minimum for $m_n$. Therefore $m_n\le p_n$.
\qed

\medskip

\textbf{Lemma 456.2 (the equality case $n=q-1$).}
If $n=q-1$ for some prime $q$, then $m_n=p_n=q$.

\emph{Proof.}
If $n=q-1$, then $q\equiv 1\pmod n$, so $p_n\le q$. On the other hand, any prime $p\equiv 1\pmod n$ has the form $p=1+tn$ with $t\ge 1$, hence $p\ge 1+n=q$. Thus $p_n=q$.

For $m_n$, note that $m=q$ satisfies $n\mid \phi(q)=q-1=n$, so $m_n\le q$.
Conversely, if $m<q$ then $\phi(m)\le m-1<q-1=n$, so $\phi(m)$ cannot be divisible by $n$. Thus no $m<q$ works, and $m_n\ge q$. Therefore $m_n=q=p_n$.
\qed

\medskip

\textbf{Lemma 456.3 (van Doorn family $n=2^{2k+1}$ gives strict inequality).}
Let $k\ge 1$ and $n:=2^{2k+1}$. Then
\[
m_n\le 2n\qquad\text{and}\qquad p_n\ge 2n+1.
\]
In particular $m_n<p_n$ for all such $n$, so there are infinitely many $n$ with $m_n<p_n$.

\emph{Proof.}
First, take $m:=2n=2^{2k+2}$. Then
\[
\phi(m)=\phi(2^{2k+2})=2^{2k+1}=n,
\]
so $n\mid\phi(m)$. Hence $m$ is admissible and $m_n\le m=2n$.

Second, any prime $p\equiv 1\pmod n$ has the form $p=1+tn$ for some integer $t\ge 1$. If $t=1$ then $p=n+1=2^{2k+1}+1$. Since $2\equiv -1\pmod 3$, we have $2^{2k+1}\equiv (-1)^{2k+1}\equiv -1\pmod 3$, hence $2^{2k+1}+1\equiv 0\pmod 3$. For $k\ge 1$, $n+1>3$, so $n+1$ is composite and cannot equal $p_n$.
Therefore $t\ge 2$, giving $p_n\ge 1+2n=2n+1$.
Combining with $m_n\le 2n$ yields $m_n<p_n$.
\qed

\medskip

\textbf{VERIFICATION}

\begin{itemize}
\item Lemma 456.1 uses only $\phi(p)=p-1$ for prime $p$.
\item Lemma 456.2's lower bound $m_n\ge q$ uses $\phi(m)\le m-1$ for $m\ge 2$; this is elementary and correct.
\item Lemma 456.3 checks the congruence modulo $3$ carefully and requires $k\ge 1$ so that $n+1>3$.
\item The computed data agrees with Lemma 456.3 for $n=8,32,128$ (in the computed range $n=8,32$): $m_8=15\le 16<p_8=17$ and $m_{32}=51\le 64<p_{32}=97$.
\end{itemize}

\textbf{FINAL.} \textbf{UNRESOLVED}

(i) \emph{Strongest proved partial result.} Always $m_n\le p_n$ (Lemma 456.1). Equality holds for $n=q-1$ with prime $q$ (Lemma 456.2). There are infinitely many $n$ with strict inequality $m_n<p_n$, e.g. $n=2^{2k+1}$ with $k\ge 1$ (Lemma 456.3).

(ii) \emph{First gap (crisp).} Prove (or disprove) that $m_n<p_n$ for \emph{almost all} $n$, i.e. that the equality family $n=q-1$ (and any other possible equality families) has density $0$ among the integers.

(iii) \emph{Top 3 next moves.}
\begin{enumerate}
\item Classify all $n$ for which $m_n=p_n$ can occur: determine necessary/sufficient structural conditions on $n$ that force the minimal $m$ with $n\mid\phi(m)$ to be prime.
\item For typical $n$, attempt to upper bound $m_n$ by constructing small $m$ with $\phi(m)$ highly divisible (e.g. $m$ with many small prime factors), and compare with available upper bounds on $p_n$.
\item Computation: tabulate $p_n/m_n$ for $n$ up to, say, $10^5$ (with optimized totient/divisibility search) to test whether $p_n/m_n$ appears to grow on a density-$1$ set.
\end{enumerate}

(iv) \emph{Minimal counterexample structure.} A minimal counterexample to ``$m_n<p_n$ for almost all $n$'' would require a positive-density set of $n$ such that the smallest $m$ with $n\mid\phi(m)$ is itself prime and is also the smallest prime $\equiv 1\pmod n$; such $n$ would likely force unusually large constraints on $\phi(m)$ for all $m<p_n$.

