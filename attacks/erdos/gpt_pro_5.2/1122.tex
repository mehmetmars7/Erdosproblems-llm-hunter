

\subsection*{FORMAL RESTATEMENT}
Let $f:\mathbb{N}\to\mathbb{R}$ be an \emph{additive function} in the sense that
\[
\forall a,b\in\mathbb{N},\ (a,b)=1 \implies f(ab)=f(a)+f(b).
\]
Define
\[
A:=\{n\in\mathbb{N}_{\ge 1}:\ f(n+1)<f(n)\}.
\]
Assume that
\[
\lvert A\cap[1,X]\rvert=o(X)\qquad (X\to\infty),
\]
i.e. $\lvert A\cap[1,X]\rvert/X\to 0$ as $X\to\infty$.
Question: must there exist a constant $c\in\mathbb{R}$ such that
\[
\forall n\ge 1,\quad f(n)=c\log n?
\]
(Here $\log$ is the natural logarithm; $\mathbb{N}=\{1,2,3,\dots\}$.)

\subsection*{QUICK LITERATURE/CONTEXT CHECK}
I only record context explicitly stated in the problem file.
The file states:
(i) Erd\H{o}s proved $f(n)=c\log n$ if $A$ is empty, and also if $f(n+1)-f(n)=o(1)$.
(ii) Mangerel proved the conclusion under a stronger sparsity hypothesis $\lvert A\cap[1,X]\rvert\ll X/(\log X)^{2+c}$ (some $c>0$) plus a technical ``no very large values'' condition on primes.

\subsection*{ATTACK PLAN}
\textbf{Proof track ideas.}
\begin{itemize}
\item Try to show that ``few descents'' forces $f$ to be close to a monotone function, then use additivity to identify the monotone additive functions as $c\log n$.
\item Study $g(n):=f(n)-c\log n$ (still additive) and attempt to show $g$ must be asymptotically constant, then use additivity to conclude $g\equiv 0$.
\end{itemize}
\textbf{Disproof track ideas.}
\begin{itemize}
\item Attempt to define $f$ on prime powers so that $f$ deviates from $c\log n$, but correlates with the arithmetic of $n$ vs $n+1$ to avoid descents on most $n$.
\item Attempt a ``sparse perturbation'' of $c\log n$ supported on a very sparse set of primes/prime powers, then test whether the descent set can be made density $0$.
\end{itemize}

\subsection*{WORK}
\textbf{Lemma 1122.1 (normalization at $1$).}
For any additive $f:\mathbb{N}\to\mathbb{R}$ (in the coprime sense), one has $f(1)=0$.

\emph{Proof.}
Since $(1,1)=1$, additivity gives $f(1\cdot 1)=f(1)+f(1)$, i.e. $f(1)=2f(1)$. Subtracting $f(1)$ from both sides yields $f(1)=0$. \qed

\textbf{Lemma 1122.2 (prime-power decomposition).}
Let $f$ be additive. If $n=\prod_{i=1}^r p_i^{a_i}$ is the prime factorization of $n$ (distinct primes $p_i$ and exponents $a_i\ge 1$), then
\[
 f(n)=\sum_{i=1}^r f\bigl(p_i^{a_i}\bigr).
\]

\emph{Proof.}
Write $n=u\,v$ where $u=p_1^{a_1}$ and $v=\prod_{i=2}^r p_i^{a_i}$. Then $(u,v)=1$, so $f(n)=f(u)+f(v)$. If $r\ge 2$, apply the same argument to $f(v)$, splitting off $p_2^{a_2}$ from the remaining coprime factor, and iterate. After $r-1$ iterations one obtains the displayed sum. \qed

\textbf{Lemma 1122.3 (subtracting $c\log$ preserves additivity).}
Fix $c\in\mathbb{R}$ and define $g(n):=f(n)-c\log n$. Then $g$ is additive (in the same coprime sense).

\emph{Proof.}
Let $(a,b)=1$. Using additivity of $f$ and the identity $\log(ab)=\log a+\log b$, we compute
\[
 g(ab)=f(ab)-c\log(ab)=(f(a)+f(b)) - c(\log a+\log b)=g(a)+g(b).
\]
Thus $g$ is additive. \qed

\textbf{FAST REALITY CHECK (small $X$ computations).}
The condition $\lvert A\cap[1,X]\rvert=o(X)$ is very strong. As a sanity check, I computed $\lvert A\cap[1,X]\rvert$ for several standard additive functions that are \emph{not} of the form $c\log n$:
\begin{itemize}
\item $f(n)=\omega(n)$ (number of distinct prime factors),
\item $f(n)=\Omega(n)$ (number of prime factors counted with multiplicity),
\item $f(n)=\log\operatorname{rad}(n)=\sum_{p\mid n}\log p$.
\end{itemize}
For each, I computed $A=\{n\le X: f(n+1)<f(n)\}$.
\begin{verbatim}
X=1000
  log: |A∩[1,1000]|=0  ratio=0.000000
  omega: |A∩[1,1000]|=346  ratio=0.346000
  Omega: |A∩[1,1000]|=424  ratio=0.424000
  lograd: |A∩[1,1000]|=345  ratio=0.345000

X=5000
  log: |A∩[1,5000]|=0  ratio=0.000000
  omega: |A∩[1,5000]|=1809  ratio=0.361800
  Omega: |A∩[1,5000]|=2137  ratio=0.427400
  lograd: |A∩[1,5000]|=1701  ratio=0.340200

X=20000
  log: |A∩[1,20000]|=0  ratio=0.000000
  omega: |A∩[1,20000]|=7354  ratio=0.367700
  Omega: |A∩[1,20000]|=8597  ratio=0.429850
  lograd: |A∩[1,20000]|=6787  ratio=0.339350

X=100000
  log: |A∩[1,100000]|=0  ratio=0.000000
  omega: |A∩[1,100000]|=37494  ratio=0.374940
  Omega: |A∩[1,100000]|=43077  ratio=0.430770
  lograd: |A∩[1,100000]|=33910  ratio=0.339100
\end{verbatim}
This does not prove anything about the conjecture, but it indicates that ``generic'' additive functions have a descent set $A$ of positive density (at least for these examples).

\subsection*{VERIFICATION}
\begin{itemize}
\item Lemma 1122.1 uses only the defining property at $(1,1)=1$ and is correct.
\item Lemma 1122.2 uses only repeated splitting into coprime factors; it does not assume complete additivity.
\item Lemma 1122.3 uses that $\log$ is exactly additive for products.
\item The computations above were performed by sieve-based evaluation of $\omega,\Omega,\log\operatorname{rad}$ and direct counting of descents.
\end{itemize}

\subsection*{FINAL}
\textbf{UNRESOLVED}

(i) \textbf{Strongest proved partial result.}
For any additive $f$, $f(1)=0$ and $f$ decomposes over prime powers as $f(n)=\sum_{p^a\parallel n} f(p^a)$ (Lemmas 1122.1--1122.2). Also, for any $c\in\mathbb{R}$, $g(n):=f(n)-c\log n$ remains additive (Lemma 1122.3).

(ii) \textbf{First gap (crisp).}
Show that the density-zero descent condition
\[
\lvert\{n\le X: f(n+1)<f(n)\}\rvert=o(X)
\]
forces an additive function $f$ to satisfy $f(n)=c\log n$ for all $n$.

(iii) \textbf{Top 3 next moves.}
\begin{itemize}
\item Prove a rigidity lemma: if $\lvert A\cap[1,X]\rvert=o(X)$ then there exists $c$ such that $f(p)=c\log p+o(\log p)$ for primes $p\to\infty$.
\item Prove a prime-power control lemma: under the same hypothesis, for each fixed prime $p$ one has $f(p^k)=k f(p)+o(1)$ as $k\to\infty$ (or a suitable uniform variant).
\item Attempt a ``consecutive integers'' transfer: prove that if $f$ is close to $c\log$ on a set of integers of density $1$, then additivity forces $f\equiv c\log$.
\end{itemize}

(iv) \textbf{Minimal counterexample structure.}
A counterexample would be an additive $f$ with $f\not\equiv c\log$ for every $c$, but such that $f(n+1)\ge f(n)$ for all but $o(X)$ values $n\le X$. Such an $f$ would have to exploit special correlations between the prime-power decompositions of consecutive integers.


