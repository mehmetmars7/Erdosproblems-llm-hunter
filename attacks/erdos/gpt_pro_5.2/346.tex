\subsection*{Erd\H{o}s problem \#346}

\noindent\textbf{1) FORMAL RESTATEMENT.}
We consider an increasing sequence $A=\{a_1<a_2<\cdots\}\subseteq\mathbb{N}$ and its subset-sum set
\[
P(A)=\left\{\sum_{n\in B} n:\ B\subseteq A\ \text{finite}\right\}.
\]
A sequence $A$ is \emph{complete} if every sufficiently large integer lies in $P(A)$.
The problem asks whether there exists a sequence $A$ such that:
\begin{enumerate}
\item[(1)] For every finite subset $B\subseteq A$, the set $A\setminus B$ is complete.
\item[(2)] For every infinite subset $B\subseteq A$, the set $A\setminus B$ is \emph{not} complete.
\end{enumerate}
The statement reports that Graham exhibited an example with these properties.
It further asks: if additionally $a_{n+1}/a_n\ge 1+\varepsilon$ for all $n$, must one have
\[
\lim_{n\to\infty}\frac{a_{n+1}}{a_n}=\frac{1+\sqrt5}{2}=:\varphi\ ?
\]
It also notes that if $a_{n+1}/a_n>\varphi$ for all $n$, then property (2) is ``easy''.

\medskip
\noindent\textbf{2) QUICK LITERATURE/CONTEXT CHECK.}
We do not use Graham's result beyond acknowledging it is asserted in the statement. We focus on proving the ``easy'' implication about ratios.

\medskip
\noindent\textbf{3) ATTACK PLAN.}
\begin{itemize}
\item Prove a standard gap lemma: if a term exceeds $1+$ the sum of all previous terms, completeness fails.
\item Use the ratio hypothesis $a_{n+1}\ge c a_n$ with $c>\varphi$ to force such a gap in any subsequence obtained by deleting infinitely many terms.
\item Do a small numerical sanity check illustrating the gap phenomenon.
\end{itemize}

\medskip
\noindent\textbf{4) WORK.}

\medskip
\noindent\textbf{Lemma 346.1 (gap obstruction to completeness).}
Let $B=\{b_1<b_2<\cdots\}\subseteq\mathbb{N}$ and let $S_m:=\sum_{i=1}^m b_i$.
If for some $m$ one has
\[
 b_{m+1} > S_m+1,
\]
then $B$ is not complete. Indeed, every integer in the interval $(S_m,\,b_{m+1})$ fails to lie in $P(B)$.

\noindent\emph{Proof.}
Any subset sum of elements among $\{b_1,\dots,b_m\}$ is at most $S_m$.
Any subset sum that uses $b_{m+1}$ is at least $b_{m+1}$.
Therefore no subset sum can lie strictly between $S_m$ and $b_{m+1}$.
In particular, the integers $S_m+1,\dots,b_{m+1}-1$ are missing from $P(B)$, so $B$ cannot be complete.
\hfill$\square$

\medskip
\noindent\textbf{Proposition 346.2 (uniform ratio $>\varphi$ forces property (2)).}
Assume $A=\{a_1<a_2<\cdots\}$ satisfies a uniform ratio bound
\[
\frac{a_{n+1}}{a_n} \ge c\quad\text{for all }n,
\]
for some constant $c>\varphi$.
Then for every infinite subset $B\subseteq A$, the set $A\setminus B$ is not complete.
Equivalently: deleting infinitely many terms from $A$ always destroys completeness.

\noindent\emph{Proof.}
Let $C:=A\setminus B=\{b_1<b_2<\cdots\}$ be the remaining infinite subsequence.
Since $C$ is a subsequence of $A$, it also satisfies $b_{i+1}/b_i\ge c$ for all $i$.
Hence for each $m$,
\[
\sum_{i=1}^m b_i \le b_m\sum_{j=0}^{m-1} c^{-j} \le b_m\cdot \frac{c}{c-1}.
\]
Now, because $B$ is infinite, there are infinitely many indices $m$ such that at least one term of $A$ lies strictly between $b_m$ and $b_{m+1}$; in particular, for infinitely many $m$ we have $b_{m+1}\ge c^2 b_m$ (skipping at least one ratio-$\ge c$ step multiplies by at least $c^2$).
Fix such an $m$. Then
\[
 b_{m+1} \ge c^2 b_m.
\]
Since $c>\varphi$ is equivalent to $c^2>\frac{c}{c-1}$ (because $c^2>\frac{c}{c-1}\iff c^2-c-1>0$), we have
\[
 b_{m+1} - \sum_{i=1}^m b_i \ge c^2 b_m - \frac{c}{c-1}b_m 
 = b_m\Bigl(c^2-\frac{c}{c-1}\Bigr) >0.
\]
For all sufficiently large such $m$, the right-hand side exceeds $1$, hence $b_{m+1} > \sum_{i=1}^m b_i + 1$.
Applying Lemma~346.1 shows that $C$ is not complete.
\hfill$\square$

\medskip
\noindent\textbf{FAST REALITY CHECK (illustration with a toy ratio).}
Take a toy geometric sequence $a_n=\lceil c^n\rceil$ with $c=1.7>\varphi$.
If we delete every other term, the remaining subsequence has ratio about $c^2\approx 2.89$, and Lemma~346.1 predicts large gaps in subset sums. A small computation for the first few terms indeed shows missing intervals once a big ratio jump occurs (not included here as a theorem).

\medskip
\noindent\textbf{5) VERIFICATION.}
Lemma~346.1 is a standard and fully explicit gap argument.
Proposition~346.2 is a direct inequality manipulation that uses the equivalence $c>\varphi\iff c^2>c/(c-1)$.

\medskip
\noindent\textbf{6) FINAL.}

\noindent\textbf{UNRESOLVED}

\smallskip
\noindent (i) \textbf{Strongest fully proved partial result obtained here.}
We proved the ``easy'' direction stated in the problem: if $a_{n+1}/a_n\ge c>\varphi$ uniformly, then deleting infinitely many terms necessarily destroys completeness (Proposition~346.2).

\smallskip
\noindent (ii) \textbf{Exact first gap.}
The main open component here is the converse-type question: under the weaker hypothesis $a_{n+1}/a_n\ge 1+\varepsilon$, does the existence of a sequence satisfying (1)--(2) force $a_{n+1}/a_n\to\varphi$?
We have no argument establishing such a limiting ratio.

\smallskip
\noindent (iii) \textbf{Top 3 next moves (concrete targets).}
\begin{enumerate}
\item Prove a stability version of Proposition~346.2: if deletions always destroy completeness, show that ratios cannot stay bounded away from $\varphi$ on either side.
\item Analyze Graham's claimed example (as given in the statement) and compute its ratio behavior to see whether it converges to $\varphi$; then attempt to extract a general mechanism.
\item Establish necessary conditions for property (1) (robust completeness under finite deletions), perhaps in terms of lower bounds on local additive coverage or on $\sum 1/a_n$.
\end{enumerate}

\smallskip
\noindent (iv) \textbf{Minimal counterexample structure.}
A counterexample to ``ratio must tend to $\varphi$'' would be a sequence satisfying (1)--(2) and $a_{n+1}/a_n\ge 1+\varepsilon$ but with $\limsup a_{n+1}/a_n>\varphi+\delta$ or $\liminf a_{n+1}/a_n<\varphi-\delta$ for some $\delta>0$. Such a sequence would have to balance being complete after any finite deletion with being fragile under any infinite deletion, yet without the Fibonacci-like ratio.


