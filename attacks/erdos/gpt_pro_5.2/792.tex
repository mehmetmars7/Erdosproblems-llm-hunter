% Erdos Problem #792

\subsection*{FORMAL RESTATEMENT}
For $n\ge 1$, let $f(n)$ be the largest integer with the property:
for every set $A\subset\mathbb Z$ with $|A|=n$, there exists a subset $B\subset A$ with $|B|\ge f(n)$ such that $B$ is \emph{sum-free} in the 3-term sense:
there are no distinct $a,b,c\in B$ with $a+b=c$.
The task is to estimate $f(n)$.

\subsection*{QUICK LITERATURE/CONTEXT CHECK}
From the problem statement: Erd\H{o}s proved $f(n)>n/3$ and conjectured the true order is $n/2$. A random-set argument gives an upper bound $f(n)\le n/2+o(n)$, and there are quoted refinements
$f(n)\ge n/3 + \tfrac13\log n$ (Alon--Kleitman) and $f(n)\le n/2-\log n$ (Eberhard--Green--Manners).

Below I reprove the classical $n/3$ lower bound by an elementary ``random dilation on the circle'' argument.

\subsection*{ATTACK PLAN}
1) Show that selecting elements whose fractional parts lie in the middle third of $[0,1)$ produces a sum-free set.

2) Average over the dilation parameter to guarantee a choice giving $\ge n/3$ elements.

3) Reality check: brute force worst cases for $n\le 8$ inside $[-8,8]$.

\subsection*{WORK}
\textbf{Lemma 792.1 (Middle-third selection is sum-free).}
Fix a real number $\theta$. For any finite $A\subset\mathbb Z$, define
\[
B_\theta := \{a\in A: \{\theta a\}\in (1/3,2/3)\},
\]
where $\{x\}$ denotes the fractional part of $x$ in $[0,1)$.
Then $B_\theta$ contains no distinct $a,b,c$ with $a+b=c$.

\emph{Proof.}
Suppose $a,b,c\in B_\theta$ are distinct. Then
$\{\theta a\},\{\theta b\},\{\theta c\}\in (1/3,2/3)$.
If $a+b=c$ held, then in $\mathbb R/\mathbb Z$ we would have
\[
\{\theta c\} = \{\theta(a+b)\} = \{\theta a + \theta b\}.
\]
But $\theta a$ and $\theta b$ have fractional parts in $(1/3,2/3)$, so their sum has fractional part in
$(2/3,4/3)$, which modulo $1$ lies in $(0,1/3)\cup(2/3,1)$.
This set is disjoint from $(1/3,2/3)$, so $\{\theta a+\theta b\}\notin (1/3,2/3)$, contradiction.
Therefore $a+b\ne c$ for all distinct $a,b,c\in B_\theta$.
\hfill$\square$

\medskip
\textbf{Lemma 792.2 (Averaging gives $|B_\theta|\ge n/3$ for some $\theta$).}
For every finite $A\subset\mathbb Z$ with $|A|=n$, there exists $\theta\in[0,1]$ such that
\[|B_\theta|\ge \lceil n/3\rceil.
\]
In particular, $f(n)\ge \lceil n/3\rceil$.

\emph{Proof.}
Choose $\theta$ uniformly at random from $[0,1]$.
Fix $a\in A$ with $a\ne 0$. Then $\{\theta a\}$ is uniform on $[0,1)$, so
\[\mathbb P\bigl(\{\theta a\}\in (1/3,2/3)\bigr)=1/3.
\]
If $a=0$, then $\{\theta a\}=0\notin(1/3,2/3)$ for all $\theta$, so its contribution is $0$.
Therefore
\[
\mathbb E|B_\theta| = \sum_{a\in A} \mathbb P(a\in B_\theta) \ge (n-1)\cdot\frac13.
\]
In particular, $\mathbb E|B_\theta|\ge n/3 - 1/3$.
Since $|B_\theta|$ is an integer-valued random variable, there must exist some $\theta$ with
$|B_\theta|\ge \lceil n/3\rceil$.
By Lemma 792.1 such a $B_\theta$ is sum-free, proving the claim.
\hfill$\square$

\medskip
\textbf{Fast reality check (small brute force in a bounded window).}
For each $n\le 8$, I brute-forced over all $A\subset[-8,8]$ with $|A|=n$ and computed the maximum size of a 3-term sum-free subset $B\subset A$, then minimized over those $A$.
The restricted-window minima found were:
\begin{center}
\begin{tabular}{c|cccccccc}
$n$ & 1 & 2 & 3 & 4 & 5 & 6 & 7 & 8\\\hline
min max-size in $[-8,8]$ & 1 & 2 & 2 & 2 & 3 & 3 & 4 & 4
\end{tabular}
\end{center}
Again, this is only a sanity check (the true $f(n)$ is a minimum over \emph{all} integer sets $A$).

\subsection*{VERIFICATION}
-- Lemma 792.1 uses only modular arithmetic on $\mathbb R/\mathbb Z$ and the fact that the middle third is disjoint from the sumset of two copies of itself.

-- In Lemma 792.2 the boundary points $1/3,2/3$ have Lebesgue measure $0$, so open vs closed intervals do not affect the expectation.

\subsection*{FINAL}
UNRESOLVED

(i) \textbf{Strongest proved partial result here:} $f(n)\ge \lceil n/3\rceil$ for all $n$, by Lemmas 792.1--792.2.

(ii) \textbf{First gap (crisp):} Determine whether $f(n)/n\to 1/2$ or even prove any constant strictly bigger than $1/3$ in the limit:
\[\liminf_{n\to\infty} \frac{f(n)}{n} > \frac13.
\]

(iii) \textbf{Top 3 next moves:}
\begin{enumerate}
\item Try to refine the interval method by using unions of intervals in $\mathbb R/\mathbb Z$ whose sumset avoids itself, aiming for density $>1/3$ and quantifying the improvement as a function of $n$.
\item Compute (or bound) exact $f(n)$ for small $n$ by searching worst-case $A$ in increasing windows, to guess extremal configurations and whether the constant drifts toward $1/2$.
\item Develop an upper-bound construction (a family of sets $A$) that forces the largest sum-free subset to have density at most some $c<1/2$, to test the conjectured $1/2$ limit.
\end{enumerate}

(iv) \textbf{Minimal counterexample structure to watch for:} Worst-case sets $A$ that minimize the largest sum-free subset typically have many additive relations $a+b=c$ (high additive energy), e.g. long arithmetic progressions or dense chunks of generalized progressions. Any improvement over $1/3$ must exploit a structural dichotomy: either $A$ is structured (then extract a large sum-free subset by a tailored construction), or $A$ is pseudorandom (then a probabilistic selection gives near-$1/2$ density).

\bigskip

