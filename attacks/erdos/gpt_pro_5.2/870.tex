% Erdos Problem #870

\noindent\textbf{FORMAL RESTATEMENT.}
Fix an integer $k\ge 3$.

\begin{itemize}
\item An \emph{additive basis of order $k$} is a set $A\subseteq\mathbb N$ (here $\mathbb N=\{1,2,3,\dots\}$) such that there exists $N_0$ with the property that every integer $n\ge N_0$ can be written as a sum of at most $k$ elements of $A$.
\item A \emph{minimal} additive basis of order $k$ is a basis $B$ of order $k$ such that for every $b\in B$ the set $B\setminus\{b\}$ is \emph{not} a basis of order $k$.
\item For $t\in\{0,1,\dots,k\}$ and $n\in\mathbb Z$, let $r_{A,\le t}(n)$ be the number of representations of $n$ as a sum of at most $t$ elements of $A$, where:
  \begin{itemize}
  \item repetition of summands is allowed;
  \item order of summands is ignored (so we count nondecreasing $t'$-tuples for some $0\le t'\le t$);
  \item we include the empty sum as the unique representation of $0$ (so $r_{A,\le t}(0)\ge 1$).
  \end{itemize}
  Define $r_A(n):=r_{A,\le k}(n)$.
\end{itemize}

(Counting ordered representations instead changes $r_A(n)$ by at most a factor $k!$ for fixed $k$, so the qualitative question ``$\ge c\log n$'' is insensitive to this convention.)

\medskip
\noindent\textbf{Problem.}
Does there exist a constant $c=c(k)>0$ such that for every additive basis $A$ of order $k$, if
\[
 r_A(n)\ge c\log n\qquad\text{for all sufficiently large }n,
\]
then $A$ contains a minimal additive basis of order $k$?

\medskip
\noindent\textbf{QUICK LITERATURE/CONTEXT CHECK.}
The problem text itself states:
\begin{itemize}
\item Erd\H{o}s--Nathanson proved a $k=2$ positive result under the hypothesis $1_A\ast 1_A(n)>(\log(4/3))^{-1}\log n$ for all large $n$.
\item H\"artter and Nathanson proved there exist additive bases containing no minimal additive basis.
\end{itemize}
I do not use any additional literature beyond what is explicitly stated above.

\medskip
\noindent\textbf{ATTACK PLAN.}
Two natural tracks.
\begin{enumerate}
\item \emph{Proof track:} try to delete elements of $A$ while preserving the basis property, and prove that the representation-count lower bound prevents an infinite deletion process without reaching a minimal basis.
\item \emph{Disproof track:} try to modify known ``no minimal subbasis'' constructions by adding redundancy so that $r_A(n)\gg\log n$ still holds.
\end{enumerate}
I can formalize how representation counts change under deletion (Lemma~1) and how minimality is witnessed (Lemma~2), but I do not close the global deletion argument.

\medskip
\noindent\textbf{WORK.}

\noindent\textbf{Lemma 1 (exact change in $r_{A,\le k}$ when deleting one element).}
Let $A\subset\mathbb N$, fix $k\ge 1$, and fix $a\in A$. For every integer $n$,
\[
 r_{A,\le k}(n)
 = r_{A\setminus\{a\},\le k}(n)
 + \sum_{j=1}^{k} r_{A\setminus\{a\},\le k-j}(n-ja).
\]
(We use the convention that $r_{B,\le t}(0)$ counts the empty sum as one representation for every $t\ge 0$ and $r_{B,\le t}(m)=0$ for $m<0$.)

\noindent\emph{Proof.}
Count representations of $n$ by the multiplicity $j\in\{0,1,\dots,k\}$ of the summand $a$.

If $j=0$, then the representation uses only elements of $A\setminus\{a\}$ and contributes $r_{A\setminus\{a\},\le k}(n)$.

If $j\ge 1$, remove the $j$ copies of $a$ from the multiset. The remaining multiset is a representation of $n-ja$ using at most $k-j$ elements from $A\setminus\{a\}$, hence is counted by $r_{A\setminus\{a\},\le k-j}(n-ja)$. Conversely, from any representation of $n-ja$ using at most $k-j$ elements of $A\setminus\{a\}$, adjoining $j$ copies of $a$ yields a unique representation of $n$ using at most $k$ elements of $A$ in which $a$ appears exactly $j$ times.

This gives a bijection for each fixed $j$ and summing over $j=0,1,\dots,k$ yields the identity.
\hfill$\square$

\medskip
\noindent\textbf{Lemma 2 (minimality $\Leftrightarrow$ infinitely many witnesses for each element).}
Let $B\subset\mathbb N$ be an asymptotic basis of order $k$. Then $B$ is minimal of order $k$ if and only if for every $b\in B$ and every integer $N$ there exists $n\ge N$ such that $r_{B\setminus\{b\}}(n)=0$.

\noindent\emph{Proof.}
($\Rightarrow$) If $B$ is minimal then for each $b\in B$, the set $B\setminus\{b\}$ is not a basis of order $k$. Therefore, for every threshold $N$ there exists $n\ge N$ not representable as a sum of at most $k$ elements of $B\setminus\{b\}$, i.e. $r_{B\setminus\{b\}}(n)=0$.

($\Leftarrow$) If for each $b\in B$ and each $N$ there is an $n\ge N$ with $r_{B\setminus\{b\}}(n)=0$, then $B\setminus\{b\}$ fails to represent arbitrarily large integers, hence is not a basis. Thus removing any $b$ destroys the basis property, so $B$ is minimal.
\hfill$\square$

\medskip
\noindent\textbf{Where the argument stalls.}
Lemma~1 makes the effect of deleting one element completely explicit, but converting the hypothesis
$r_A(n)\ge c\log n$ into a deletion scheme that guarantees $r(n)\ge 1$ for all large $n$ \emph{after infinitely many deletions} is the main missing step.
The natural obstruction is that even if each deletion decreases $r(n)$ only by a bounded amount depending on $n$, an infinite sequence of deletions can drive $r(n)$ to $0$ on an infinite subsequence of $n$.

\medskip
\noindent\textbf{VERIFICATION (FAST REALITY CHECK).}
\begin{itemize}
\item Numerical check of the constant appearing in the $k=2$ statement:
\[
(\log(4/3))^{-1}\approx 3.47605949678.
\]
\item Sanity check of Lemma~1 by brute force on small sets.
I tested $200$ random instances with $k=3$, random $A\subset\{1,\dots,15\}$ of size $6$, random $a\in A$, and checked the identity for all $n\le 60$ (with the convention that $0$ has one empty representation). All $200/200$ tests passed.
\end{itemize}

\medskip
\noindent\textbf{FINAL.} \textbf{UNRESOLVED}.

\noindent(i) \emph{Strongest proved partial result here.}
Lemma~1 (exact multiplicity decomposition under deletion) and Lemma~2 (minimality equivalent to infinitely many witness integers for each element).

\noindent(ii) \emph{First gap (crisp).}
Given a basis $A$ with $r_A(n)\ge c\log n$ for all large $n$, prove existence of a subset $B\subseteq A$ that is still a basis of order $k$ and satisfies the witness property of Lemma~2 for every $b\in B$.

\noindent(iii) \emph{Top 3 next moves (concrete).}
\begin{enumerate}
\item Prove a quantitative ``many deletions force failure'' bound: if $D\subset A$ is deleted, lower bound $r_{A\setminus D}(n)$ in terms of $r_A(n)$ and an explicit statistic of $D$ (derived from Lemma~1) that can be controlled when $r_A(n)\gg\log n$.
\item Try a probabilistic thinning of $A$ to a random $B$ with $r_B(n)\ge 1$ for all $n$ in long ranges, then diagonalize; afterward attempt to enforce minimality by a controlled deletion using Lemma~2.
\item Attempt a counterexample construction by augmenting a known ``no minimal subbasis'' basis (as mentioned in the problem text) with structured redundancy that boosts $r_A(n)$ to $\gg\log n$.
\end{enumerate}

\noindent(iv) \emph{Minimal counterexample structure.}
A counterexample would be a basis $A$ of order $k\ge 3$ with $r_A(n)\ge c\log n$ for all large $n$ but such that every subbasis $B\subseteq A$ contains an element $b\in B$ with $B\setminus\{b\}$ still a basis (so no minimal basis exists). Equivalently, there is an infinite strictly descending chain of subbases with no minimal element under inclusion.


