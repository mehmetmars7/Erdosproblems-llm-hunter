
\subsection*{Erd\H{o}s Problem 687}

\paragraph{1) FORMAL RESTATEMENT.}
Let $x\ge 2$ be real, and let $\mathcal P(x):=\{p:\ p\text{ prime},\ p\le x\}$.
For each $p\in\mathcal P(x)$ choose a residue class $a_p\pmod p$.
Say that an integer $n$ is \emph{covered} if $n\equiv a_p\pmod p$ for at least one prime $p\le x$.
Define $Y(x)$ to be the maximal integer $y\ge 0$ for which there exist choices $\{a_p\}_{p\le x}$ such that every integer in $[1,y]$
is covered.

The problem asks for good estimates for $Y(x)$ as $x\to\infty$, in particular whether one can prove $Y(x)=o(x^2)$
or even $Y(x)\ll x^{1+o(1)}$.

\paragraph{2) QUICK LITERATURE/CONTEXT CHECK.}
The problem statement already records:
\begin{itemize}
\item an upper bound due to Iwaniec: $Y(x)\ll x^2$,
\item a lower bound due to Ford--Green--Konyagin--Maynard--Tao: $Y(x)\gg x\frac{\log x\log\log\log x}{\log\log x}$,
\item a conjecture of Maier--Pomerance: $Y(x)\ll x(\log x)^{2+o(1)}$.
\end{itemize}
No additional literature is used or claimed here.

\paragraph{3) ATTACK PLAN.}
\begin{itemize}
\item \textbf{Main reformulation:} encode the residue choices by a single shift $t$ modulo the primorial $P(x):=\prod_{p\le x}p$ (Lemma~687.1),
so that the problem becomes: maximize the length of an interval of consecutive integers containing no number coprime to $P(x)$.
\item \textbf{Proof-track:} try to upper-bound the longest run of consecutive integers all having a prime factor $\le x$ (i.e.\ all non-coprime to $P(x)$),
using sieve bounds on gaps between $x$-rough numbers.
\item \textbf{Disproof-track (for very strong conjectures):} attempt to build explicit residue choices/shift producing intervals of length close to $x^2$
or larger, guided by constructions for large prime gaps.
\end{itemize}

\paragraph{4) WORK.}
\medskip
\noindent\textbf{Lemma 687.1 (CRT shift reformulation).}
Let $P(x):=\prod_{p\le x}p$.
Given a choice of residue classes $\{a_p\pmod p\}_{p\le x}$, there exists a unique residue class $t\pmod{P(x)}$ such that
\[
t\equiv -a_p \pmod p\quad\text{for every prime }p\le x.
\]
For this $t$, an integer $n$ satisfies $n\equiv a_p\pmod p$ for some $p\le x$ if and only if $n+t$ is divisible by some prime $p\le x$, i.e.
\[
\exists\,p\le x:\ n\equiv a_p\pmod p\quad\Longleftrightarrow\quad \gcd(n+t,P(x))>1.
\]
Consequently, $Y(x)$ equals the maximum length of a run of consecutive integers all \emph{not} coprime to $P(x)$:
\[
Y(x)=\max_{t\in\mathbb Z}\ \max\{\,y\ge 0:\ \gcd(t+1,P(x))>1,\dots,\gcd(t+y,P(x))>1\,\}.
\]

\emph{Proof.}
For distinct primes $p\le x$, the moduli are coprime, so by the Chinese remainder theorem there exists a unique $t\pmod{P(x)}$
solving the simultaneous congruences $t\equiv -a_p\pmod p$.

For a fixed prime $p\le x$ and integer $n$, we have
\[
n\equiv a_p\pmod p\quad\Longleftrightarrow\quad n-a_p\equiv 0\pmod p
\quad\Longleftrightarrow\quad n+t\equiv 0\pmod p
\]
because $t\equiv -a_p\pmod p$.
Thus $n$ is covered by at least one congruence class $a_p\pmod p$ if and only if $n+t$ is divisible by at least one prime $p\le x$.
Since $P(x)$ is the product of all primes $\le x$, this is equivalent to $\gcd(n+t,P(x))>1$.

The final displayed formula for $Y(x)$ is exactly the definition of $Y(x)$ rewritten in terms of this equivalence (covering $[1,y]$ corresponds
to choosing $t$ such that $t+1,\dots,t+y$ are all divisible by some prime $\le x$). \hfill$\square$

\medskip
\noindent\textbf{Lemma 687.2 (elementary lower bound $Y(x)\ge x-1$).}
For every real $x\ge 2$, one has $Y(x)\ge \lfloor x\rfloor-1$.

\emph{Proof.}
Let $X:=\lfloor x\rfloor$ and set $t=1$ in the reformulation of Lemma~687.1.
We claim that every integer $n\in[1,X-1]$ is covered by the residue choices $a_p\equiv -1\equiv p-1\pmod p$ for all primes $p\le x$.
Indeed, for such $n$ we have $n+1\le X\le x$, hence $n+1$ has some prime divisor $p\le n+1\le x$.
Then $p\mid (n+1)$ implies $n\equiv -1\equiv p-1\pmod p$, so $n$ lies in the chosen class for that prime $p$.
Thus $[1,X-1]$ is fully covered and $Y(x)\ge X-1$. \hfill$\square$

\medskip
\noindent\textbf{Lemma 687.3 (trivial upper bound from counting totatives).}
Let $P(x)=\prod_{p\le x}p$ and $\varphi$ be Euler's totient function. Then
\[
Y(x)\le P(x)-\varphi(P(x)).
\]

\emph{Proof.}
By Lemma~687.1, for any shift $t$ the pattern of coprimality of $t+1,t+2,\dots$ with $P(x)$ is periodic modulo $P(x)$.
In a complete residue system modulo $P(x)$, there are exactly $\varphi(P(x))$ residues coprime to $P(x)$ and hence
$P(x)-\varphi(P(x))$ residues that are not coprime.

If an interval of consecutive integers contains no number coprime to $P(x)$, then within one period of length $P(x)$ it can cover at most
all the non-coprime residues, i.e.\ it cannot have length exceeding $P(x)-\varphi(P(x))$; otherwise, among the first $P(x)-\varphi(P(x))+1$
distinct residues in $[1,P(x)]$ one would be coprime (pigeonhole, since only $P(x)-\varphi(P(x))$ residues are non-coprime).
Thus the maximal such run length is at most $P(x)-\varphi(P(x))$, which equals the stated bound on $Y(x)$. \hfill$\square$

\medskip
\noindent\textbf{FAST REALITY CHECK (local computation).}
Using the Lemma~687.1 reformulation, I computed $Y(x)$ for small $x$ by scanning residues modulo $P(x)$ and finding the maximal gap between
consecutive totatives of $P(x)$. The exact values for integer $x\le 19$ are:
\begin{verbatim}
x= 2  Y(x)= 1
x= 3  Y(x)= 3
x= 4  Y(x)= 3
x= 5  Y(x)= 5
x= 6  Y(x)= 5
x= 7  Y(x)= 9
x= 8  Y(x)= 9
x= 9  Y(x)= 9
x=10  Y(x)= 9
x=11  Y(x)=13
x=12  Y(x)=13
x=13  Y(x)=21
x=14  Y(x)=21
x=15  Y(x)=21
x=16  Y(x)=21
x=17  Y(x)=25
x=18  Y(x)=25
x=19  Y(x)=33
\end{verbatim}
These are exact for the computed $x$ and are consistent with the easy bound $Y(x)\ge x-1$.

\paragraph{5) VERIFICATION.}
\begin{itemize}
\item Lemma~687.1 is a direct CRT equivalence; the key point is that choosing one residue class mod each prime is equivalent to choosing
a single shift $t$ modulo their product.
\item Lemma~687.2 is checked: for each $n\le x-1$, the integer $n+1$ has some prime factor $\le x$, guaranteeing coverage by $a_p\equiv -1$.
\item Lemma~687.3 is very weak asymptotically because $P(x)$ grows super-exponentially in $x$, but it is logically correct.
\end{itemize}

\paragraph{FINAL.}
\noindent\textbf{UNRESOLVED}

\smallskip
\noindent(i) \textbf{Strongest proved partial result.}
An exact reformulation: $Y(x)$ is the maximum length of a run of consecutive integers all not coprime to the primorial
$P(x)=\prod_{p\le x}p$ (Lemma~687.1). In particular, $Y(x)\ge \lfloor x\rfloor-1$ (Lemma~687.2) and
$Y(x)\le P(x)-\varphi(P(x))$ (Lemma~687.3).

\smallskip
\noindent(ii) \textbf{First gap (crisp statement).}
Prove a nontrivial upper bound improving $Y(x)\ll x^2$ (e.g.\ $Y(x)=o(x^2)$) from only the covering/CRT formulation, without assuming
any results not already in the problem statement.

\smallskip
\noindent(iii) \textbf{Top 3 next moves.}
\begin{enumerate}
\item Translate $Y(x)$ bounds into bounds on maximal gaps between integers coprime to $P(x)$, and attempt to apply/derive sieve bounds on gaps between
$x$-rough numbers (integers with no prime factor $\le x$).
\item Perform more computations for larger $x$ (as far as feasible) to see whether $Y(x)$ appears closer to $x(\log x)^2$ or to $x^2$.
\item Try to prove that any run of length $L$ with all numbers divisible by primes $\le x$ forces the existence of a prime gap of comparable length;
then import (only) those prime-gap bounds explicitly allowed by the problem statement.
\end{enumerate}

\smallskip
\noindent(iv) \textbf{What a minimal counterexample would likely look like.}
A counterexample to $Y(x)=o(x^2)$ would require a sequence $x_j\to\infty$ and shifts $t_j$ such that the interval
$[t_j+1,t_j+Y(x_j)]$ contains no integer coprime to $P(x_j)$, with $Y(x_j)$ of order $c\,x_j^2$ or larger; equivalently, extremely long
gaps between integers coprime to the primorials.


