## 1) FORMAL RESTATEMENT

### Literal statement (as written)

Let (A={a_1<a_2<\cdots}) be an infinite strictly increasing sequence of **integers**.
For each integer (n), let (f(n)) be the number of pairs ((u,v)) of indices with (1\le u\le v) such that
[
n=\sum_{u\le i\le v} a_i.
]
Question: does there exist such an (A) for which (f(n)\to\infty) as (n\to\infty)?
Or even (f(n)\ge 2) for all sufficiently large (n)?

### Ambiguities / stress points

1. **Sign of the (a_i).** The sequence is said to be “integers,” not “positive integers.” If negative terms are allowed, then for large positive (n) only finitely many negative terms can be used (since the sequence is strictly increasing, it contains only finitely many negative integers), so the asymptotic question is essentially about the positive tail anyway. But many standard reformulations (convex sets, convolutions on (\mathbb Z_{\ge0})) implicitly assume positivity.
2. **Domain of (n).** The question “as (n\to\infty)” indicates (n\in\mathbb N) (positive integers). I will take (n\ge 1).
3. **Does (u=v) count?** Yes: the definition allows (u=v). This matches the example (a_n=n), where (f(n)) equals the number of odd divisors of (n), which includes the divisor (1) (corresponding to the length-(1) representation).

### Minimal corrected statement (standard conventions)

I will work with:

> **Corrected problem.**
> Let ((a_i)*{i\ge1}) be a strictly increasing sequence of **positive** integers. For each (n\in\mathbb N), define
> [
> f(n):=#{(u,v)\in\mathbb N^2: 1\le u\le v,\ \sum*{i=u}^v a_i = n}.
> ]
> Does there exist such ((a_i)) with (f(n)\to\infty) as (n\to\infty)?
> Or even with (f(n)\ge 2) for all sufficiently large (n)?

This is the interpretation consistent with the “convex set” reformulation and with all cited examples.

### Equivalent convex/difference formulation (made explicit)

Define prefix sums
[
s_0:=0,\qquad s_m:=\sum_{i=1}^m a_i\quad(m\ge1),
]
and let (S:={s_m:m\ge0}\subset\mathbb Z).

Then for (1\le u\le v),
[
\sum_{i=u}^v a_i = s_v-s_{u-1}.
]
Hence
[
f(n)=#{(i,j)\in\mathbb Z_{\ge0}^2: i<j,\ s_j-s_i=n}.
]
In additive combinatorics notation, if (1_S) is the indicator of (S), then
[
(1_S\circ 1_S)(n):=\sum_x 1_S(x),1_S(x+n) = #{(x,y)\in S^2: y-x=n},
]
so (f(n)=(1_S\circ 1_S)(n)) (up to the harmless convention of ordered vs. (i<j)).

Moreover, because (a_{m+1}>a_m), the consecutive gaps (s_{m+1}-s_m=a_{m+1}) are strictly increasing, so (S) is a **convex** set in the standard discrete sense (“successive differences strictly increasing”).

So the corrected problem is equivalent to:

> Is there an infinite convex set (S\subset\mathbb Z) such that ((1_S\circ 1_S)(n)\to\infty) as (n\to\infty)?
> Or even ((1_S\circ 1_S)(n)\ge2) for all large (n)?

---

## 2) QUICK LITERATURE/CONTEXT CHECK (web available)

The Erdős Problems site lists Problem #358 as **OPEN** and repeats the same reformulation about convex sets and the convolution (1_A\circ 1_A).

The site also notes Erdős–Moser considered the primes case and conjectured an infinite (\limsup) of the number of representations, and even positivity of the upper density of representable integers was not proved there.

I did not find (via web search) any subsequent paper claiming a resolution (positive or negative) of the full “(f(n)\to\infty)” or “eventually (\ge2)” questions in the generality stated. The “OPEN” flag on the Erdős Problems site as of Oct 2025 is consistent with that.

---

## 3) ATTACK PLAN

### Proof-track ideas (existence)

1. **Probabilistic construction:** define an increasing integer sequence with random perturbations (e.g., (a_i=Bi+X_i) with (X_i) i.i.d. bounded), and try to show by Borel–Cantelli that for each fixed (K), only finitely many (n) have (f(n)<K). Key obstacle: strong “for all large (n)” requirement and dependence between different interval-sum events.
2. **Block/concatenation construction:** build ((a_i)) in stages so that on stage (k), every (n) in some large interval gains (\ge k) new representations, and stages cover all sufficiently large (n). Key obstacle: designing a finite increasing block whose consecutive sums hit *every* integer in a target interval with high multiplicity.

### Disproof-track ideas (nonexistence)

1. **Show infinitely many (n) with (f(n)\le1)** for every increasing sequence, perhaps by extracting “uniquely representable” differences in the convex prefix-sum set.
2. **Energy/structure bounds on convex sets**: use known constraints on additive energy of convex sets to force many differences to have small multiplicity. Key obstacle: energy bounds control averages/second moments, but the question is about pointwise lower bounds (“all large (n)”).

**Best path I can partially execute here:** formal equivalences + rigorous universal bounds + sanity computations + identify the exact bottleneck for both tracks.

---

## 4) WORK (rigorous partial results)

### Lemma 1 (equivalence to differences of a convex set)

Let ((a_i)*{i\ge1}) be strictly increasing positive integers, define (s_0=0), (s_m=\sum*{i=1}^m a_i), and (S={s_m:m\ge0}).

Then for every (n\in\mathbb N),
[
f(n)=#{(i,j)\in\mathbb Z_{\ge0}^2:\ i<j,\ s_j-s_i=n}.
]

**Proof.** For any (1\le u\le v),
[
\sum_{i=u}^v a_i = (a_1+\cdots+a_v)-(a_1+\cdots+a_{u-1})=s_v-s_{u-1}.
]
Thus solutions ((u,v)) correspond bijectively to pairs ((i,j)=(u-1,v)) with (0\le i<j) and (s_j-s_i=n). ∎

Moreover, (S) is convex in the discrete sense because consecutive differences satisfy (s_{m+1}-s_m=a_{m+1}) and (a_{m+1}>a_m). ∎

---

### Lemma 2 (for fixed length, at most one start)

Fix (L\ge1). Define
[
S_L(u):=\sum_{i=u}^{u+L-1} a_i\qquad (u\ge1).
]
Then (S_L(u)) is strictly increasing in (u). Consequently, for each (n), there is **at most one** (u) such that (S_L(u)=n).

**Proof.**
[
S_L(u+1)-S_L(u)=\sum_{i=u+1}^{u+L} a_i - \sum_{i=u}^{u+L-1} a_i = a_{u+L}-a_u.
]
Since ((a_i)) is strictly increasing, (a_{u+L}>a_u), hence (S_L(u+1)>S_L(u)). ∎

So:
[
f(n)=#{L\ge1:\ \exists u\text{ with }S_L(u)=n}.
]

---

### Lemma 3 (universal pointwise upper bound (f(n)\ll \sqrt n))

Assume (a_1\ge1) (true for positive integers). If (n=\sum_{i=u}^{u+L-1} a_i), then
[
n\ \ge\ 1+2+\cdots+L=\frac{L(L+1)}2.
]
Hence (L\le \frac{-1+\sqrt{1+8n}}2 < \sqrt{2n}), and therefore
[
f(n)\ \le\ \left\lfloor \frac{-1+\sqrt{1+8n}}2\right\rfloor\ <\ \sqrt{2n}.
]

**Proof.** Since ((a_i)) is strictly increasing in positive integers, (a_i\ge i) for all (i) (because (a_1\ge1) and (a_{i}\ge a_1+(i-1)\ge i)). Therefore
[
\sum_{i=u}^{u+L-1} a_i \ \ge \ \sum_{i=u}^{u+L-1} i \ \ge\ \sum_{i=1}^L i=\frac{L(L+1)}2.
]
Combine with Lemma 2 (at most one representation per length) to obtain (f(n)\le#{L:\frac{L(L+1)}2\le n}), giving the stated bound. ∎

This shows any positive answer can only grow at most on the order of (\sqrt n).

---

### Lemma 4 (counting bound on total representations up to (N))

For any strictly increasing positive integer sequence ((a_i)),
[
\sum_{n=1}^N f(n)\ \le\ N\cdot H_N \ \le\ N(1+\log N),
]
where (H_N=\sum_{L=1}^N \frac1L) is the (N)-th harmonic number.

**Proof.** Each representation of some (n\le N) corresponds to an interval ([u,u+L-1]) whose sum is (\le N). For fixed (L), define (S_L(u)) as above. If (S_L(u)\le N), then using (a_i\ge i),
[
S_L(u)\ge \sum_{i=u}^{u+L-1} i = \frac{L(2u+L-1)}2 \ge Lu.
]
So (Lu\le N), i.e. (u\le N/L). Thus the number of possible starts (u) for length (L) intervals with sum (\le N) is at most (\lfloor N/L\rfloor). Summing over lengths:
[
\sum_{n=1}^N f(n)
\le \sum_{L=1}^N \left\lfloor \frac{N}{L}\right\rfloor
\le \sum_{L=1}^N \frac{N}{L}
= N H_N
\le N(1+\log N).
]
∎

So even in the “best possible” universe, the average of (f(n)) over (n\le N) cannot exceed (\asymp \log N). This still leaves room for (f(n)\to\infty) (e.g. as slowly as (\log\log n)), but it constrains what a positive solution could look like.

---

### Lemma 5 (the example (a_n=n): (f(n)) = number of odd divisors of (n))

Let (a_i=i). Then (f(n)) equals the number of odd positive divisors of (n).

**Proof.** A representation
[
n = u+(u+1)+\cdots+v
]
has length (L=v-u+1\ge1). The sum is
[
n = \frac{L(u+v)}{2}=\frac{L(2u+L-1)}2.
]
Equivalently,
[
2n = L(2u+L-1).
]

**(A) From a representation to an odd divisor.**

* If (L) is odd, then (2u+L-1) is even, so (L\mid n). Thus (L) is an **odd divisor** of (n).
* If (L) is even, write (L=2k). Then
  [
  n = k(2u+2k-1),
  ]
  and the factor (2u+2k-1) is odd, hence it is an **odd divisor** of (n).

So each representation yields an odd divisor of (n).

**(B) From an odd divisor to a representation.**
Let (d) be an odd divisor of (n), and write (n=dm).

* If (d<2m+1) (equivalently (m-\frac{d-1}{2}\ge1)), define
  [
  L=d,\qquad u=m-\frac{d-1}{2}.
  ]
  Then (L) is odd, (u\in\mathbb N), and
  [
  u+(u+1)+\cdots+(u+L-1)=L\left(u+\frac{L-1}{2}\right)=d\left(m-\frac{d-1}{2}+\frac{d-1}{2}\right)=dm=n.
  ]
* If (d\ge 2m+1), define
  [
  L=2m,\qquad u=\frac{d-2m+1}{2}.
  ]
  Since (d) is odd, (u\in\mathbb Z), and (d\ge2m+1) gives (u\ge1). Then
  [
  u+(u+1)+\cdots+(u+L-1)=L\left(u+\frac{L-1}{2}\right)
  =2m\left(\frac{d-2m+1}{2}+\frac{2m-1}{2}\right)=2m\cdot\frac d2=dm=n.
  ]

Exactly one of the inequalities (d<2m+1) or (d\ge 2m+1) holds, so each odd divisor (d) produces exactly one valid representation. This gives a bijection between representations and odd divisors. ∎

In particular, (f(2^k)=1) for all (k\ge0), so (f(n)\not\to\infty) for this (A).

---

### PHASE 1 tiny computational falsification attempts (finite checks only)

Because the problem is about *eventual* behavior, finite computation cannot resolve it, but it can quickly falsify naive constructions. For example, if one tries the “almost all integers but delete powers of 2” subsequence (a common first idea), then computing (f(n)) for (n\le 200) already shows many (n) with (f(n)<2) (so this attempt cannot possibly prove (f(n)\ge2) eventually without major further modification).
(Verified by a direct brute-force prefix-sum computation in Python over the initial segment up to 200.)

This is only a sanity check: it does **not** rule out existence of some other, more delicate construction.

---

## 5) VERIFICATION (adversarial self-check)

* **Quantifiers:**

  * Lemma 3 uses the assumption (a_1\ge1) (positive integers). If (a_1) were negative, the bound (a_i\ge i) fails; that is exactly why I separated “literal” vs “corrected.”
* **Lemma 2 monotonicity:**

  * Relies only on strict increase (a_{u+L}>a_u), valid for any strictly increasing integer sequence (no positivity needed).
* **Lemma 4:**

  * Uses (a_i\ge i), hence positivity again.
* **Lemma 5 bijection:**

  * Checked against (n=9): odd divisors (1,3,9) correspond to lengths (1,3,2) respectively (via the two-case construction). So the “large odd divisor gives even length” subtlety is handled.

No lemma here claims to resolve the existence/nonexistence; they are all unconditional bounds or exact computations.

---

## 6) FINAL

**UNRESOLVED**

(i) **Strongest fully proved partial results obtained here**

* Exact reformulation: (f(n)) is the difference-representation function of the convex prefix-sum set (S) (Lemma 1).
* Universal pointwise bound (f(n) < \sqrt{2n}) for any strictly increasing positive integer sequence (Lemma 3).
* Universal average bound (\sum_{n\le N} f(n)\le N(1+\log N)) (Lemma 4), implying any “(f(n)\to\infty)” would have to grow at most logarithmically on average.
* For (a_i=i), (f(n)) equals the number of odd divisors of (n) (Lemma 5), hence (\limsup f(n)=\infty) but (f(n)\not\to\infty).

(ii) **First crisp gap (what I could not prove)**

* I cannot prove either:

  * **Existence:** construct ((a_i)) with (f(n)\to\infty) (or even (f(n)\ge2) eventually), because I cannot produce a block/recursive design or a probabilistic model with a *uniform-in-(n)* lower tail bound strong enough for a Borel–Cantelli argument; **nor**
  * **Nonexistence:** show that every increasing positive integer sequence has infinitely many (n) with (f(n)\le 1) (or with (f(n)) bounded), because known convex-set energy controls do not readily give a pointwise lower bound obstruction.

(iii) **Top 3 next moves**

1. **Probabilistic method with disjoint-interval independence:** choose a concrete random increasing model (e.g., (a_i=Bi+X_i) with bounded i.i.d. (X_i)) and, for each (n), select a large family of lengths whose “candidate intervals” are disjoint, to get independent Bernoulli events and tail bounds for (f(n)). The key missing lemma is a *number-theoretic* statement ensuring that for each (n) there are many such lengths for which the required residue lies in the bulk of the distribution.
2. **Block design for covering intervals:** explicitly build blocks (B_k) of increasing integers so that every integer in a large target interval has (\ge k) representations by consecutive sums **within** the block, then concatenate blocks. The missing piece is a constructive gadget: a finite strictly increasing block with high-multiplicity “consecutive-sum covering.”
3. **Attempt a disproof via “unique differences” in convex sets:** prove that in any infinite convex set (S), there are infinitely many differences (n) with ((1_S\circ1_S)(n)=1). This likely needs a new structural lemma stronger than current additive-energy bounds.

(iv) **What a minimal counterexample/solution would likely look like**

* Any **positive** solution must keep ((a_i)) **near-linear** in growth (fast-growing sequences make interval sums too sparse), and must “break” the classical power-of-two obstruction seen for (a_i=i) in a way that works **uniformly for all large (n)**.
* Any **negative** theorem would need to show an unavoidable infinite set of (n) with small representation count forced by convexity/monotonicity—something not captured by average-energy estimates alone.

(As of the Erdős Problems site status, the full problem remains listed OPEN.)

Here’s a much more “worked” expansion of the three next moves + the “minimal shape” heuristics, with a few **fully proved** structural lemmas that make the heuristics precise.

---

## A. Structural necessities behind “near‑linear” (really: subquadratic)

Let (A={a_1<a_2<\cdots}\subset\mathbb N) and
[
f(n)=#{(u,v):1\le u\le v,\ \sum_{i=u}^v a_i=n}.
]
Write
[
m(X):=\max{j:\ a_j\le X}\quad(\text{so }a_{m(X)}\le X<a_{m(X)+1}).
]

### Lemma A1 (interval-count upper bound up to (X))

For every (X\in\mathbb N),
[
\sum_{n\le X} f(n)\ \le\ \frac{m(X)(m(X)+1)}2.
]

**Proof.** If (\sum_{i=u}^v a_i\le X), then in particular (a_v\le X), hence (v\le m(X)). Thus every representation of an integer (\le X) is an interval ([u,v]) with (1\le u\le v\le m(X)). There are exactly (m(X)(m(X)+1)/2) such index-pairs ((u,v)), so the total number of representations of all integers (\le X) is at most that. ∎

This is extremely basic but it immediately yields hard necessary growth constraints.

### Lemma A2 (subquadratic growth is **necessary** for (f(n)\to\infty))

If (f(n)\to\infty) as (n\to\infty), then
[
\frac{m(X)^2}{X}\to\infty\quad\text{as }X\to\infty,
]
equivalently
[
a_n=o(n^2)\quad\text{as }n\to\infty.
]

**Proof.** Fix (K\in\mathbb N). Since (f(n)\to\infty), there exists (N_K) such that (f(n)\ge K) for all (n\ge N_K).

Let (X\ge 2N_K). Then for all integers (n\in [X/2,X]) we have (n\ge N_K), hence (f(n)\ge K). Therefore
[
\sum_{n\le X} f(n)\ \ge\ \sum_{X/2\le n\le X} f(n)\ \ge\ \left\lfloor\frac X2\right\rfloor \cdot K\ \ge\ \frac X2\cdot K - K.
]
Combine with Lemma A1:
[
\frac{m(X)(m(X)+1)}2\ \ge\ \frac X2\cdot K - K.
]
For large (X) (depending on (K)) this implies (m(X)^2\ge c_K X) with (c_K\asymp K). Since (K) is arbitrary, this forces (m(X)^2/X\to\infty).

To convert this to a statement about (a_n): take (X=a_n). Then (m(a_n)\ge n), so
[
\frac{n^2}{a_n}\ \le\ \frac{m(a_n)^2}{a_n}\ \to\infty,
]
i.e. (a_n=o(n^2)). ∎

So the slogan “near-linear” can be made rigorous as:

> **Any positive solution to (f(n)\to\infty) must have (a_n) strictly subquadratic** (indeed (a_n=o(n^2))).

That’s a real constraint: it rules out (for example) (a_n\sim c n^2), (a_n\sim n^2\log n), anything (\gg n^2).

### Lemma A3 (exponential/lacunary growth kills representability)

If there exists (\lambda>1) such that (a_{n+1}\ge \lambda a_n) for all sufficiently large (n), then the set ({n:\ f(n)\ge 1}) has asymptotic density (0) (in particular (f(n)\ge1) eventually is impossible).

**Proof.** Fix (\lambda>1) and assume (a_{n+1}\ge \lambda a_n) for all (n\ge n_0). Then (a_n\ge a_{n_0}\lambda^{n-n_0}) for (n\ge n_0), hence (m(X)=O(\log X)).

But by Lemma A1,
[
#{n\le X:\ f(n)\ge 1}\ \le\ \sum_{n\le X} f(n)\ \le\ \frac{m(X)(m(X)+1)}2\ =\ O((\log X)^2).
]
Thus
[
\frac{#{n\le X:\ f(n)\ge 1}}{X}\ \to\ 0.
]
∎

So anything “too fast” (in the lacunary/exponential sense) is dead on arrival. This is the other rigorous underpinning of the heuristic “don’t grow too fast.”

---

## B. Next move 1 — Probabilistic method with disjoint-interval independence (made concrete)

The cleanest probabilistic framework is:

* Choose a deterministic “shape” (b_i) (increasing, typically (b_i\approx i) or (i^{1+\varepsilon})).
* Perturb by independent “noise” (X_i) to randomize sums but keep strict increase.

A common trick to guarantee strict increase **without conditioning** is
[
a_i := M i + X_i,
]
where (M\ge 2) is fixed and (X_i\in{0,1,\dots,M-1}) are i.i.d. uniform. Then
[
a_{i+1}-a_i = M + (X_{i+1}-X_i)\ge 1,
]
so ((a_i)) is strictly increasing almost surely.

Now fix (n). For each length (L), define
[
S_L(u):=\sum_{i=u}^{u+L-1} a_i = M\sum_{i=u}^{u+L-1} i + \sum_{i=u}^{u+L-1} X_i.
]
Let (Y_{u,L}:=\sum_{i=u}^{u+L-1} X_i). Then (Y_{u,L}) is a sum of (L) i.i.d. uniform digits.

### Where “disjoint-interval independence” comes from

If you can pick many candidate intervals ([u_k,u_k+L_k-1]) that are **pairwise disjoint**, then the corresponding (Y_{u_k,L_k}) are independent, hence the events
[
E_k := {S_{L_k}(u_k)=n}
]
are independent (because each depends only on the (X_i) inside its interval).

So you want:

1. many disjoint candidate intervals ([u_k,u_k+L_k-1]),
2. each with (\mathbb P(E_k)) not too small,
3. and you want this **uniformly in (n)** (or at least with failure probabilities summable in (n)).

### The first “hard lemma” you need (the one I called number-theoretic)

For each (L), the deterministic part
[
D_L(u):=M\sum_{i=u}^{u+L-1} i = M\cdot \frac{L(2u+L-1)}2
]
moves by (\Delta=D_L(u+1)-D_L(u)=ML).
The random part (Y_{u,L}) has support ([0,(M-1)L]).
So for each length (L), there is **at most one** start (u) for which (n-D_L(u)) even lands in ([0,(M-1)L]). Call it (u_L(n)) if it exists.

Then
[
\mathbb P\big(\exists u:\ S_L(u)=n\big)=
\begin{cases}
\mathbb P\big(Y_{u_L(n),L}=y_L(n)\big) & \text{if a candidate exists}\
0 & \text{otherwise}
\end{cases}
]
with (y_L(n):=n-D_L(u_L(n))).

The **bulk-vs-tail** issue is: for a sum of i.i.d. digits, the point probability (\mathbb P(Y=y)) is of order (\sim \frac{1}{M\sqrt L}) only when (y) sits near the center of the distribution; near the edges it can be astronomically smaller.

So the missing lemma really is:

> **Bulk lemma (informal).** For each large (n), there are many lengths (L\le c\sqrt n) such that the required residue (y_L(n)) lies in a “bulk window” where (\mathbb P(Y=y_L(n))\gtrsim 1/(M\sqrt L)).

If one could prove such a lemma *with enough lengths whose candidate intervals are also disjoint*, then one gets a Chernoff-style tail bound and can aim for a Borel–Cantelli argument to force (f(n)\to\infty) for a single realization.

### A clean *disjointness* lemma you can actually prove

Even before the bulk lemma, you want a deterministic disjointness mechanism: lengths that force their candidate starts far apart.

Heuristic approximation from (D_L(u)\approx MLu): to hit (n), you need (u\approx n/(ML)). So larger lengths live at smaller indices, giving a natural ordering that can produce disjointness.

A typical deterministic claim (you can prove with careful inequalities once you fix how you choose (u_L(n))) is of the form:

> If (L_k=k^2) and (k\le c, n^{1/5}), then the “candidate intervals” ([u_{L_k}(n),u_{L_k}(n)+L_k-1]) are disjoint.

This kind of lemma is purely algebraic (no probability), and it’s exactly what lets you reduce (f(n)) to a sum of independent indicators over (k).

**But:** without the bulk lemma, disjointness alone gives you only a lower bound on independence structure, not a usable probability lower bound.

So the probabilistic track is very sharply bottlenecked at: **prove bulk for each (n)** (or make the construction blockwise so you only need bulk for finitely many (n) at a time).

---

## C. Next move 2 — Block design for covering intervals (why it’s hard, and what a “gadget” must do)

The “block” idea is: build (A) in stages so that on stage (t) you guarantee a lower bound (f(n)\ge t) for all (n) in some interval ([N_t,N_{t+1})), and you make (N_t\to\infty). Because later terms are huge, they don’t create new representations for earlier (n), so the property is stable.

The fundamental obstacle is Lemma 2 from earlier: **for a fixed length (L), each (n) has at most one representation.** Hence (f(n)\ge t) means (n) must be representable in at least (t) *distinct lengths*.

So a block “gadget” must force, for every (n) in the target interval, solutions for many distinct lengths (L)—but the set of lengths is allowed to depend on (n).

### The right mental model: multiple covering by sparse monotone sequences

For each length (L), the values (S_L(u)) (as (u) varies) form a strictly increasing sequence with gaps (\ge L). So “length (L)” provides a sparse set of integers (density (\lesssim 1/L)).

To get (f(n)\ge t) uniformly, you need the union over many lengths to provide a **(t)-fold cover** of the integers in your target interval.

This is strikingly reminiscent of **covering systems** (Erdős’ world): families of congruence classes ({n\equiv r_L\pmod L}) whose union covers all integers (or covers each integer (t) times). In the special case (a_i=i), each length (L) indeed hits exactly one residue class mod (L) (up to parity issues), hence (f(n)) becomes a divisor-type counting function. To make (f(n)) uniform, you’d like to “program” the residue classes for each (L) in a way that creates a (t)-cover.

So the natural gadget target becomes:

> **Gadget goal (informal).** Construct (a_i) so that for many lengths (L), the set ({S_L(u)}) behaves like a translate of an arithmetic progression of step (L), but with a controllable translate that you can choose length-by-length to implement a multiple covering system.

This is exactly why the block design *feels* plausible but is technically brutal: you must control simultaneously a huge collection of overlaps between different lengths.

What’s “minimal” for a gadget?

* It must be long enough to give each length (L) enough starts (u) (at least (1) start per (n) you want to cover).
* It must keep (a_i) increasing (and compatible with previous blocks).
* It must avoid global constraints that force arithmetic-structure like the (a_i=i) case (where you’re stuck with divisor obstructions).

At present, I do not have a worked finite gadget that provably yields even (t=2) on a growing interval; that’s exactly the missing ingredient.

---

## D. Next move 3 — Disproof via “unique differences” in convex sets (what would need to happen)

Reformulate via prefix sums (s_0=0<s_1<s_2<\cdots) with strictly increasing gaps:
[
f(n)=#{(i,j): i<j,\ s_j-s_i=n}.
]
So (f(n)) is the multiplicity function of differences in the convex set (S={s_i}).

A disproof of “eventually (\ge2)” would follow from proving:

> (\forall) infinite convex (S), (\exists) infinitely many (n) with (|S\cap(S+n)|\le 1).

### Why known “energy” bounds don’t immediately give this

There are strong results that convex sets have relatively small additive energy (few additive quadruples), which imply that *many* differences have small multiplicity **on average** inside finite truncations (S\cap[0,X]).

But the fatal snag is stability:

* A difference that is unique among the first (m) elements need not stay unique once you add more elements—later pairs can create the same difference.
* So to get infinitely many (n) with global multiplicity (1), you need “rigid” uniqueness: differences that cannot be replicated later because of monotonicity/convexity.

A plausible route is to study the “new differences” created by adding a new largest point (s_m):
[
{s_m-s_i:\ 0\le i<m}.
]
These are all distinct (trivially) for fixed (m). The dream would be to show that among these, many are **forced** to be globally unique forever. But I don’t currently see a monotonicity principle strong enough to prevent later replication (s_{m'}-s_{i'}=s_m-s_i).

So the disproof track likely requires a genuinely new “no-repeat” mechanism for convex sets, beyond what energy alone controls.

---

## E. “Minimal counterexample/solution shape” revisited (now with precise constraints)

The two proved constraints above are the cleanest formal versions of the heuristics:

1. **If (f(n)\to\infty), then (a_n=o(n^2)).**
   (Lemma A2.)
   So any candidate construction must keep (A) well below quadratic growth.

2. **If (A) grows even mildly exponentially, representability becomes extremely sparse.**
   (Lemma A3.)
   So any solution must be at most subexponential; in practice one expects “polynomial-ish,” plausibly close to linear, but *the proven barrier is quadratic*.

3. The “powers of two obstruction” for (a_n=n) is not a universal obstruction; it’s an obstruction to that particular rigid residue-class structure.
   Any successful construction must break that rigidity, i.e. avoid the situation where each length (L) essentially forces a single congruence class (divisor-function behavior).

---

## What I would do next if I were forced to push one track hard

If you want the probabilistic track to become an actual proof, I’d recommend focusing on a **blockwise random construction** (so you only need uniform control over (n) in a finite interval at each stage), and aim to prove the following finite statement:

> For each (t), there exists a finite increasing block (B_t) (chosen randomly from a structured distribution) such that with probability (1-\varepsilon_t) (where (\sum \varepsilon_t<\infty)), every (n\in[N_t,N_{t+1})) has at least (t) interval-sum representations using intervals wholly inside (B_t).

Then by fixing one realization where all but finitely many blocks succeed, you’d get (f(n)\to\infty).

The technical heart would still be: producing many **independent** candidate intervals for each (n) with **not-too-small** point probabilities—i.e. a usable version of the “bulk lemma.”

---

If you want, I can pick **one** of these tracks and write out the exact “next lemma target” in a fully formal way (statement with constants/ranges), and then try to prove it or show precisely why it fails (with a counterexample), but the three proven lemmas above are the main rigorous content that justifies the heuristics you quoted back.
