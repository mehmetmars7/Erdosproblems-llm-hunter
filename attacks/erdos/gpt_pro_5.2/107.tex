% Erdos Problem #107
% Attempt for Erdos Problem #107
% Following PROMPT_STRATEGY.MD
% Tools/Constraints:
% - Web browsing available? YES (not used; only facts explicitly stated in the problem text)
% - Computation available (Python)? YES (not used)

\section*{Erd\H{o}s Problem \#107}

\subsection*{1) FORMAL RESTATEMENT}
For an integer $n\ge3$, let $f(n)$ be the least integer $N$ such that every set $P\subset\mathbb R^2$ of $N$ points with \emph{no three collinear} contains $n$ points in \emph{convex position}, i.e. $n$ points that are the vertices of a convex $n$-gon (equivalently, no one of the $n$ points lies in the convex hull of the other $n-1$).

The conjectured exact value is
\[
 f(n)=2^{n-2}+1\quad(n\ge3).
\]

\subsection*{2) QUICK LITERATURE/CONTEXT CHECK}
I only record what the problem text states:
\begin{itemize}
\item Erd\H{o}s--Szekeres proved the bounds $2^{n-2}+1\le f(n)\le \binom{2n-4}{n-2}+1$.
\item Several improvements exist; the best stated bound is $f(n)\le 2^{n+O(\sqrt{n\log n})}$.
\end{itemize}
No other results are assumed/proved here.

\subsection*{3) ATTACK PLAN}
\begin{itemize}
\item \textbf{Proof track (full conjecture):} show every set of $2^{n-2}+1$ points contains a convex $n$-gon, likely via a sharp recursion or a refined partial-order argument.
\item \textbf{Disproof track:} exhibit, for some $n$, a configuration of $2^{n-2}+1$ points with no convex $n$-gon.
\end{itemize}
In this writeup I give a complete proof of the base case $f(4)=5$ and basic structural lemmas, but I do not resolve the general conjecture.

\subsection*{4) WORK}
\paragraph{Lemma 107.1 (the case $n=4$: $f(4)=5$).}
One has $f(4)=5$.

\emph{Proof.}
\emph{Lower bound $f(4)\ge5$.}  There exists a $4$-point set with no convex quadrilateral: take three vertices of a triangle and a fourth point strictly inside that triangle.  Any four-point subset is the whole set, and one point lies in the convex hull of the other three, so there is no convex $4$-gon.  Thus $f(4)\ge5$.

\emph{Upper bound $f(4)\le5$.}  Let $P$ be any set of $5$ points with no three collinear.  Consider all triangles formed by triples of points in $P$, and choose a triangle $\triangle ABC$ of \emph{minimum (Euclidean) area}.

\emph{Claim:} no other point of $P$ lies strictly inside $\triangle ABC$.
Indeed, if some point $D\in P$ lies inside $\triangle ABC$, then the three triangles $\triangle ABD$, $\triangle BCD$, $\triangle CAD$ form a partition of $\triangle ABC$ into three nondegenerate triangles.  In particular, each of these triangles has area strictly smaller than $\triangle ABC$, contradicting the minimality of $\triangle ABC$.

Therefore, the remaining two points of $P\setminus\{A,B,C\}$ lie \emph{outside} the triangle $\triangle ABC$.  Hence the convex hull $\operatorname{conv}(P)$ cannot be a triangle (a triangular convex hull would contain all points in or on that triangle, with at least two points inside, contradicting the claim).  Thus $\operatorname{conv}(P)$ has at least $4$ vertices, and those $4$ vertices form a convex quadrilateral.

So every $5$-point set contains a convex $4$-gon, i.e. $f(4)\le5$.  Together with the lower bound, $f(4)=5$. \qed

\paragraph{Lemma 107.2 (monotonicity and trivial bounds).}
For $n\ge3$:
\begin{enumerate}
\item $f(n+1)\ge f(n)$.
\item $f(n)\ge n$.
\end{enumerate}

\emph{Proof.}
(2) is immediate since fewer than $n$ points cannot contain $n$ points in convex position.

For (1): by definition, any set of $f(n+1)$ points in general position contains $n+1$ points in convex position.  Removing any one of these $n+1$ vertices leaves $n$ points still in convex position.  Hence any set of $f(n+1)$ points contains a convex $n$-gon, so $f(n)\le f(n+1)$, i.e. $f(n+1)\ge f(n)$. \qed

\subsection*{5) VERIFICATION (FAST REALITY CHECK)}
\begin{itemize}
\item $f(3)=3$: any $3$ noncollinear points form a triangle.
\item Lemma~107.1 gives $f(4)=5$.
\item The conjecture predicts $f(5)=2^{3}+1=9$, consistent with the statement in the problem text that Tur\'an and Makai proved $f(5)=9$.
\end{itemize}

\subsection*{6) FINAL}
\textbf{UNRESOLVED.}

(i) \emph{Strongest fully proved partial result obtained here.}
A complete proof that $f(4)=5$ (Lemma~107.1), plus basic monotonicity/trivial bounds (Lemma~107.2).

(ii) \emph{Exact first gap.}
Proving (or disproving) the general identity $f(n)=2^{n-2}+1$ for all $n\ge3$.

(iii) \emph{Top 3 next moves (concrete targets).}
\begin{enumerate}
\item Re-prove the general Erd\H{o}s--Szekeres upper bound $f(n)\le \binom{2n-4}{n-2}+1$ in a fully self-contained way as a stepping stone.
\item Attempt to sharpen the induction/recursion to match the lower bound $2^{n-2}+1$.
\item Computationally search for extremal configurations for $n=6,7$ to test structure and potential stability.
\end{enumerate}

(iv) \emph{Minimal counterexample structure.}
A disproof would require some $n$ and an explicit set of $2^{n-2}+1$ points in general position containing no convex $n$-gon; it would have to beat the classical Erd\H{o}s--Szekeres construction of size $2^{n-2}$ without creating a convex $n$-tuple.


