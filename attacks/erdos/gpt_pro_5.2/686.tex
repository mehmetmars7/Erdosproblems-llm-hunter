% Erdos Problem #686
% Solution attempt generated from 685-689.tex

\subsection*{Erd\H{o}s Problem 686}

\paragraph{1) FORMAL RESTATEMENT.}
We interpret the problem in the standard number-theoretic way:
\emph{$k,m,n$ are nonnegative integers}, with $k\ge 2$ and $m\ge n+k$.
For such $(k,m,n)$ define
\[
R(k,m,n):=\frac{\prod_{i=1}^k (m+i)}{\prod_{i=1}^k (n+i)}.
\]
Question: Is it true that for every integer $N\ge 2$ there exist integers $k\ge 2$ and $m\ge n+k\ge 0$ such that
$R(k,m,n)=N$?

\emph{Ambiguity note.} If one allows negative $n$ or $m$, then denominators can vanish or signs can appear; the problem statement
suggests the intended domain is nonnegative integers with $m\ge n+k$.

\paragraph{2) QUICK LITERATURE/CONTEXT CHECK.}
No external results are used or claimed here beyond what is in the problem statement.
Algebraically, $R(k,m,n)$ is a ratio of rising factorials and can be rewritten as a ratio of binomial coefficients (Lemma~686.1).

\paragraph{3) ATTACK PLAN.}
\begin{itemize}
\item \textbf{Proof-track idea:} rewrite $R(k,m,n)$ as a ratio of binomial coefficients; attempt to realize arbitrary $N$ by forcing
prime-adic valuations via choices of $(k,m,n)$.
\item \textbf{Disproof-track idea:} search for a small $N$ not representable; then try to prove an invariant that forbids it (e.g.\ constraints on
prime valuations forced by disjointness of the two length-$k$ intervals).
\end{itemize}

\paragraph{4) WORK.}
\medskip
\noindent\textbf{Lemma 686.1 (binomial coefficient reformulation).}
For integers $k\ge 1$ and $m,n\ge 0$,
\[
R(k,m,n)=\frac{\binom{m+k}{k}}{\binom{n+k}{k}}.
\]

\emph{Proof.}
We have
\[
\prod_{i=1}^k (m+i)=\frac{(m+k)!}{m!},\qquad \prod_{i=1}^k (n+i)=\frac{(n+k)!}{n!}.
\]
Therefore
\[
R(k,m,n)=\frac{(m+k)!}{m!}\cdot \frac{n!}{(n+k)!}
=\frac{(m+k)!}{k!\,m!}\bigg/\frac{(n+k)!}{k!\,n!}
=\frac{\binom{m+k}{k}}{\binom{n+k}{k}}.
\]
\hfill$\square$

\medskip
\noindent\textbf{Lemma 686.2 (the case $k=2$ reduces to a Pell-type square condition).}
Fix an integer $N\ge 2$. There exist integers $m\ge n+2\ge 0$ such that
\[
R(2,m,n)=N
\]
if and only if there exists an integer $n\ge 0$ for which
\[
D(n):=4N(n+1)(n+2)+1
\]
is a perfect square; in that case $m$ is uniquely determined by
\[
m=\frac{-3+\sqrt{D(n)}}{2}.
\]

\emph{Proof.}
The equality $R(2,m,n)=N$ is exactly
\[
\frac{(m+1)(m+2)}{(n+1)(n+2)}=N,
\]
i.e.
\[
(m+1)(m+2)=N(n+1)(n+2).
\]
Expanding gives
\[
m^2+3m+2 = N(n^2+3n+2).
\]
View this as a quadratic equation in $m$:
\[
m^2+3m+(2-N(n^2+3n+2))=0.
\]
Its discriminant is
\[
\Delta = 3^2-4\cdot 1\cdot (2-N(n^2+3n+2))
=1+4N(n^2+3n+2)=4N(n+1)(n+2)+1.
\]
Thus the quadratic has an integer solution $m$ if and only if $\Delta$ is a perfect square, say $\Delta=s^2$.
In that case, the quadratic formula gives
\[
m=\frac{-3\pm s}{2}.
\]
Since $m\ge 0$ and $s\ge 1$, only the $(-3+s)/2$ branch is possible. This yields the stated formula.
Finally, the constraint $m\ge n+2$ is an additional inequality one must check once $m$ is computed; it is automatic in some cases and not in others.
\hfill$\square$

\medskip
\noindent\textbf{FAST REALITY CHECK (local computation).}
Brute search (bounded parameters) found many representations but also some persistent ``missing'' integers in the searched ranges.

\begin{itemize}
\item For $N=2,3,5,\dots,30$, searching over $2\le k\le 6$, $0\le n\le 30$, $n+k\le m\le 60$ found representations for all
$N\in\{2,3,5,6,7,8,9,10,\dots,30\}$ except $N=4$ and $N=25$ in that box. Example outputs:
\begin{verbatim}
2 -> (k,m,n)=(2,19,13)  since (20*21)/(14*15)=2
3 -> (k,m,n)=(2, 8, 4)  since ( 9*10)/( 5* 6)=3
9 -> (k,m,n)=(3,25,11)  since (26*27*28)/(12*13*14)=9
16-> (k,m,n)=(3,13, 4)  since (14*15*16)/( 5* 6* 7)=16
\end{verbatim}
\item Expanding the search to $2\le k\le 8$, $0\le n\le 200$, $n+k\le m\le 2000$ still found \emph{no} representation for $N=4,25,49$.
\item In the special case $k=2$, Lemma~686.2 allows a fast scan of $n$ for the square condition $4N(n+1)(n+2)+1=s^2$.
For $N=4,25,49$ no solutions were found for $0\le n\le 200000$ (this is not a proof of nonexistence).
\end{itemize}

\paragraph{5) VERIFICATION.}
\begin{itemize}
\item Lemma~686.1 is a direct factorial manipulation and is exact.
\item Lemma~686.2 is an exact algebraic equivalence for $k=2$ (discriminant criterion).
\item The computational searches are bounded; they only show absence of solutions within the tested ranges and cannot certify impossibility.
\end{itemize}

\paragraph{FINAL.}
\noindent\textbf{UNRESOLVED}

\smallskip
\noindent(i) \textbf{Strongest proved partial result.}
The representation problem is equivalent to asking whether every $N\ge 2$ is a ratio of binomial coefficients
\[
N=\binom{m+k}{k}\big/\binom{n+k}{k}
\]
with $k\ge 2$ and $m\ge n+k\ge 0$ (Lemma~686.1). For $k=2$, existence is equivalent to the explicit square/discriminant condition
of Lemma~686.2.

\smallskip
\noindent(ii) \textbf{First gap (crisp statement).}
Either (a) prove that for every $N\ge 2$ there exists some $k\ge 2$ and $m\ge n+k$ with $R(k,m,n)=N$, or
(b) exhibit a specific $N$ (e.g.\ $N=4$ or $25$ suggested by bounded search) and prove rigorously that no such $(k,m,n)$ exist.

\smallskip
\noindent(iii) \textbf{Top 3 next moves.}
\begin{enumerate}
\item Analyze the diophantine geometry for fixed small $k$ (starting with $k=2,3$): characterize all integer solutions to
$\binom{m+k}{k}=N\binom{n+k}{k}$ and decide for which $N$ solutions exist.
\item Search for an invariant from prime-adic valuations: for each odd prime $q$, compare $v_q((m+1)_k)$ and $v_q((n+1)_k)$ and see
whether $v_q(R(k,m,n))$ must take some restricted set of values.
\item If a counterexample is suspected (e.g.\ $N=4$), attempt to prove that $R(k,m,n)$ cannot equal that $N$ by forcing an odd prime divisor
in the ratio (e.g.\ show existence of a prime dividing the numerator block with exponent exceeding that in the denominator block).
\end{enumerate}

\smallskip
\noindent(iv) \textbf{What a minimal counterexample would likely look like.}
A minimal counterexample would be a small integer $N$ whose prime factorization is ``too simple'' to be realized as the quotient of two
length-$k$ consecutive products from disjoint intervals. Bounded search suggests $N=4$ and $25$ as candidates, but no proof is known here.


