% Erdos problem 986

1) FORMAL RESTATEMENT

Let R(k,n) denote the (off-diagonal) graph Ramsey number: the least integer N such that
every simple graph G on N vertices contains either
- a clique K_k, or
- an independent set of size n.

Conjecture:
For each fixed integer k>=3 there exist constants c=c(k)>0 and C=C(k)>0 such that for all
sufficiently large integers n,

  R(k,n) >= C * n^{k-1} / (log n)^c.

2) QUICK LITERATURE/CONTEXT CHECK

The problem file states that this conjecture is known for k=3 (Spencer) and for k=4
(Mattheus--Verstraete). It also quotes general upper/lower bounds (not needed for the
arguments below).

3) ATTACK PLAN

- Prove easy structural facts: recursion and the standard polynomial upper bound.
- Prove a baseline probabilistic lower bound (weaker than the conjecture but completely
  self-contained).
- Do a small-case computation to sanity-check the definition (R(3,3)).

4) WORK

FAST REALITY CHECK (exact small computation): R(3,3)=6.

I verified this by brute force over all graphs on N vertices for N<=6.
A graph "witnesses R(3,3)>N" if it has no triangle and no independent set of size 3.
The computation found witnesses for N=3,4,5 and none for N=6.
One witness on 5 vertices is the 5-cycle.

Exact output from a minimal script:
\begin{verbatim}
n 3 exists True
n 4 exists True
n 5 exists True
n 6 exists False
example mask for n=5 220
edges [(0, 3), (0, 4), (1, 2), (1, 4), (2, 3)]
\end{verbatim}

(Here "exists False" at N=6 means there is no graph on 6 vertices avoiding both a triangle
and an independent triple.)

---

Lemma 986.1 (Ramsey recursion).
For integers k,n>=2,

  R(k,n) <= R(k-1,n) + R(k,n-1).

Proof.
Let N = R(k-1,n) + R(k,n-1). Consider any graph G on N vertices and fix a vertex v.
Partition the remaining N-1 vertices into
  A = neighbors of v,
  B = non-neighbors of v.
Then |A|+|B|=N-1.

If |A| >= R(k-1,n), then the induced subgraph G[A] contains either
- a K_{k-1}, in which case adding v gives a K_k in G, or
- an independent set of size n, which is also independent in G.

If |A| < R(k-1,n), then |B| >= N-1-(R(k-1,n)-1) = R(k,n-1). In that case G[B] contains
either
- a K_k (done), or
- an independent set of size n-1; together with v (which is non-adjacent to all of B)
  this forms an independent set of size n in G.

Thus every graph on N vertices contains K_k or an independent n-set, so R(k,n) <= N.
\qed

Corollary 986.2 (Erdos--Szekeres binomial upper bound).
For k,n>=1,

  R(k,n) <= \binom{k+n-2}{k-1}.

Proof.
Define U(k,n)=\binom{k+n-2}{k-1}. One checks the same recursion
U(k,n)=U(k-1,n)+U(k,n-1), with base cases U(1,n)=U(k,1)=1.
Using Lemma 986.1 and induction on k+n gives R(k,n) <= U(k,n).
\qed

---

Lemma 986.3 (A self-contained probabilistic lower bound).
Fix an integer k>=3. For all sufficiently large n, there exists a graph on at least

  (1/(2(10k)^{k/2})) * (n/log n)^{k/2}

vertices with no K_k and with independence number < n. Consequently,

  R(k,n) >= c_k * (n/log n)^{k/2}

for a positive constant c_k depending only on k.

Proof.
Let n be large (so that log n>0 and the estimates below hold). Define

  N := \left\lfloor (n/(10k\,\log n))^{k/2} \right\rfloor,
  p := (5k\,\log n)/n.

Consider the random graph G ~ G(N,p).

Step 1: bound the expected number X of independent sets of size n.
For a fixed n-subset S of vertices, the probability it is independent is (1-p)^{\binom{n}{2}}.
Hence

  E[X] = \binom{N}{n} (1-p)^{\binom{n}{2}}.

Use the crude bounds \binom{N}{n} <= (eN/n)^n and (1-p)^{\binom{n}{2}} <=
exp(-p\,n(n-1)/2) <= exp(-p\,n^2/3) for n>=2. Thus

  E[X] <= exp( n\log(eN/n) - (p n^2)/3 ).

Now compute a simple upper bound on \log(eN/n): since N <= (n/(10k\log n))^{k/2},

  eN/n <= e * n^{k/2-1} / (10k\log n)^{k/2}.

For fixed k and large n this is <= n^{k/2} (say), hence \log(eN/n) <= (k/2)\log n.
Therefore

  n\log(eN/n) <= (k/2) n\log n,
  (p n^2)/3 = (5k\log n)/n * n^2/3 = (5k/3) n\log n.

So the exponent is <= ((k/2) - (5k/3)) n\log n = -(7k/6) n\log n,
which is < -10 for large n. In particular E[X] < 1/4.
Thus P(X>=1) <= E[X] < 1/4 (Markov), so with probability > 3/4 the random graph has
no independent set of size n.

Step 2: bound the expected number Y of K_k subgraphs.
For each k-subset T, the probability it forms a clique is p^{\binom{k}{2}}. Hence

  E[Y] = \binom{N}{k} p^{\binom{k}{2}} <= N^k p^{\binom{k}{2}}.

Let t=\binom{k}{2}=k(k-1)/2. Using N <= (n/(10k\log n))^{k/2} and p=(5k\log n)/n,

  N^k p^t
  <= (n/(10k\log n))^{k^2/2} * (5k\log n/n)^{k(k-1)/2}
  = ( (5k)/(10k) )^{t} * (n^{k/2}/(\log n)^{k/2}) * (1/(10k))^{k/2}
  = (1/2)^{t} * (n/(10k\log n))^{k/2}.

Since N is the floor of (n/(10k\log n))^{k/2}, for large n we have
E[Y] <= (1/2)^{t} (N+1) <= (1/2)^{t}*2N <= N/4,
because t>=3 for k>=3.
Then P(Y > N/2) <= E[Y]/(N/2) <= (N/4)/(N/2)=1/2 by Markov; in fact the above gives <=1/2.
So with probability at least 1/2, both events occur:
- X=0 (no independent n-set), and
- Y <= N/2 (at most N/2 copies of K_k).

Fix a realization of G with these two properties.
Delete one vertex from each K_k present in G (choose arbitrarily one vertex per clique).
Since there are at most N/2 cliques, we delete at most N/2 vertices, leaving an induced
subgraph G' on at least N/2 vertices.

- G' has no K_k by construction.
- G' has no independent set of size n: any independent n-set in G' would also be an
  independent n-set in the original G because deleting vertices does not create new edges.

Thus R(k,n) > |V(G')| >= N/2.
Substituting the definition of N yields the claimed lower bound with
c_k = 1/(2(10k)^{k/2}).
\qed

5) VERIFICATION

- The brute-force computation checks the definition: it finds a 5-vertex witness and
  verifies none exists on 6 vertices, consistent with the classical statement R(3,3)=6.
- Lemma 986.1 recursion proof checked by neighborhood partition.
- Lemma 986.3: all probabilistic inequalities are explicit (Markov + union bound).
  Deleting vertices cannot create a new independent set of a fixed size.

6) FINAL

**UNRESOLVED**

(i) Strongest proved partial result here:
    - An elementary upper bound: R(k,n) <= binom{k+n-2}{k-1}.
    - A self-contained probabilistic lower bound: R(k,n) >= c_k (n/log n)^{k/2}.
    - Exact small-case computation: R(3,3)=6.

(ii) First gap (crisp statement):
    Improve the exponent in the lower bound from k/2 to k-1 (up to polylog factors),
    i.e. prove that for fixed k>=5 the largest K_k-free graphs with independence number < n
    can have N on the order n^{k-1}/(log n)^{O(1)}.

(iii) Top 3 next moves (concrete):
    1. Try to adapt sparse random / semi-random constructions to push the exponent from k/2
       to (k+1)/2 (already nontrivial) without importing heavy external theorems.
    2. For a fixed k (say k=5), attempt a dedicated construction and compute numerically the
       best N achievable by heuristic search to guess the right polylog.
    3. Re-express the conjecture as a statement about maximal size of K_k-free graphs with
       bounded independence number and look for a container-style counting argument.

(iv) Minimal counterexample structure:
    A fixed k>=3 for which there exists an infinite sequence n_i->infty with
      R(k,n_i) << n_i^{k-1}/(log n_i)^{c}
    for every c>0; equivalently, for infinitely many n there is a K_k-free graph on about
    n^{k-1}/(log n)^{c} vertices with independence number < n failing for all c.

