%Erdos problem #165
\section*{Erdos problem \#165}

\subsection*{1) FORMAL RESTATEMENT}
For each integer $N\ge 1$, let $f(N)$ be the smallest size of a set $A\subseteq [N]$ such that the difference set
\[
A-A:=\{a-a':a,a'\in A\}
\]
contains every integer $1,2,\dots,N$.
The question asks whether $f(N)=N^{1-o(1)}$.

\paragraph{Ambiguity/typo check.}
If $[N]=\{1,2,\dots,N\}$, then $A-A\subseteq\{-(N-1),\dots,N-1\}$, so it cannot contain $N$.
Thus the intended $[N]$ must include both endpoints $0$ and $N$ (e.g. $[N]=\{0,1,\dots,N\}$), or the target differences should be $\{1,\dots,N-1\}$.
In what follows I adopt the natural interpretation $[N]=\{0,1,\dots,N\}$ so that difference $N$ is attainable.

\subsection*{2) QUICK LITERATURE/CONTEXT CHECK}
The problem file states:
\begin{itemize}
\item $f(N)\ge c N^{2/3}$ (Erd\H{o}s--Olson).
\item $f(N)\le N/2+2$ (Replanki--Svr\v{s}ek--Szab\'{o}--Wagner).
\item $f(N)\le N/2+1$ for large $N$ (Hefty--Horn--King--Pfender).
\end{itemize}
I do not use these as black boxes; I re-prove a simple $\le N/2+2$ construction and a crude $\ge \sqrt N$ counting lower bound.

\subsection*{3) ATTACK PLAN}
Give explicit construction for an $O(N)$ upper bound and basic counting for a $\Omega(\sqrt N)$ lower bound; compute exact $f(N)$ for small $N$.

\subsection*{4) WORK}
\paragraph{Lemma 165.1 (simple explicit upper bound $f(N)\le \lfloor N/2\rfloor+2$).}
Let $m:=\lfloor N/2\rfloor$. Then the set
\[
A:=\{0,1,2,\dots,m\}\ \cup\ \{N\}
\]
satisfies $\{1,2,\dots,N\}\subseteq A-A$. Hence
\[
f(N)\le (m+1)+1=\lfloor N/2\rfloor+2.
\]

\textit{Proof.}
First, for $1\le d\le m$, we have $d=d-0\in A-A$ since both $d$ and $0$ lie in $A$.
Second, for $m< d\le N$, write $d=N-(N-d)$. Because $d>m$ we have $N-d < N-m\le m$ (since $m=\lfloor N/2\rfloor$), so $N-d\in\{0,1,\dots,m\}\subseteq A$.
Also $N\in A$. Thus $d=N-(N-d)\in A-A$.
Therefore every $d\in\{1,\dots,N\}$ lies in $A-A$. \qed

\paragraph{Lemma 165.2 (counting lower bound $f(N)\ge (1+\sqrt{1+4N})/2$).}
If $A\subseteq\{0,1,\dots,N\}$ satisfies $\{1,\dots,N\}\subseteq A-A$ and $|A|=m$, then
\[
m(m-1)\ge N\quad\text{and hence}\quad m\ge \frac{1+\sqrt{1+4N}}{2}.
\]
In particular, $f(N)\ge \sqrt N$ up to constants.

\textit{Proof.}
For each $d\in\{1,\dots,N\}$ choose one ordered pair $(a_d,b_d)\in A^2$ with $a_d-b_d=d$.
These pairs are all distinct because their differences are distinct.
There are $N$ such ordered pairs with $a_d\ne b_d$, and the number of ordered pairs with distinct elements is at most $m(m-1)$.
Thus $m(m-1)\ge N$. Solving gives the claimed bound. \qed

\paragraph{FAST REALITY CHECK (computed exact values for $N\le 12$).}
By brute force over subsets of $\{0,1,\dots,N\}$, the exact values of $f(N)$ for $1\le N\le 12$ are:
\[
\begin{array}{c|cccccccccccc}
N&1&2&3&4&5&6&7&8&9&10&11&12\\\hline
f(N)&2&3&3&4&4&4&5&5&5&6&6&6
\end{array}
\]
One achieving set for each $N$ (not unique):
\[
\begin{array}{c|l}
N & A \\\hline
1&(0,1)\\
2&(0,1,2)\\
3&(0,1,3)\\
4&(0,1,3,4)\\
5&(0,1,3,5)\\
6&(0,2,5,6)\\
7&(0,1,3,6,7)\\
8&(0,1,3,6,8)\\
9&(0,1,3,7,9)\\
10&(0,1,4,8,9,10)\\
11&(0,1,4,8,9,11)\\
12&(0,2,5,10,11,12)
\end{array}
\]

\subsection*{5) VERIFICATION}
\begin{itemize}
\item Lemma 165.1: the two ranges $d\le \lfloor N/2\rfloor$ and $d>\lfloor N/2\rfloor$ are covered by differences inside the initial segment and differences from $N$ to the initial segment.
\item Lemma 165.2: injection from required differences to ordered pairs with distinct entries.
\item Computation: exhaustive for $N\le 12$.
\end{itemize}

\subsection*{6) FINAL}
\textbf{UNRESOLVED}

\begin{enumerate}
\item[(i)] \textbf{Strongest fully proved partial result obtained here.}
A simple explicit construction gives $f(N)\le \lfloor N/2\rfloor+2$ (Lemma 165.1), while counting gives $f(N)\ge (1+\sqrt{1+4N})/2$ (Lemma 165.2).

\item[(ii)] \textbf{Exact first gap.}
Bridge the gap between $\Omega(\sqrt N)$ and $O(N)$, in particular decide whether $f(N)$ must be $N^{1-o(1)}$ (almost linear) or can be substantially smaller.

\item[(iii)] \textbf{Top 3 next moves (concrete targets).}
\begin{enumerate}
\item Improve the lower bound by using additive-combinatorial structure of difference covers: show $A-A$ cannot contain a full interval unless $|A|$ is large.
\item Search for better explicit constructions beating $N/2+2$ substantially (if possible), perhaps via multi-scale sets.
\item Compute $f(N)$ for $N\le 50$ by ILP/SAT to guess the growth and test conjectures.
\end{enumerate}

\item[(iv)] \textbf{Minimal counterexample structure (if $f(N)=N^{1-o(1)}$ is false).}
One would need an explicit family $A_N\subseteq\{0,1,\dots,N\}$ with $|A_N|\le N^{1-\varepsilon}$ for some fixed $\varepsilon>0$ and with $\{1,\dots,N\}\subseteq A_N-A_N$. Such sets would provide ``difference bases'' with sublinear size.
\end{enumerate}


