
Let $F(x)$ be the maximal $k$ such that there exist $n+1,\ldots,n+k\leq x$ with $\tau(n+1),\ldots,\tau(n+k)$ all distinct (where $\tau(m)$ counts the divisors of $m$). Estimate $F(x)$. In particular, is it true that\[F(x) \leq (\log x)^{O(1)}?\]In other words, is there a constant $C>0$ such that, for all large $x$, every interval $[x,x+(\log x)^C]$ contains two integers with the same number of divisors? A problem of Erd\H{o}s and Mirsky \cite{ErMi52}, who proved that\[\frac{(\log x)^{1/2}}{\log\log x}\ll F(x) \ll \exp\left(O\left(\frac{(\log x)^{1/2}}{\log\log x}\right)\right).\]Erd\H{o}s \cite{Er85e} claimed that the lower bound could be improved via their method 'with some more work' to $(\log x)^{1-o(1)}$. Beker has improved the upper bound to\[F(x) \ll \exp\left(O\left((\log x)^{1/3+o(1)}\right)\right).\]Cambie has observed that Cram\'{e r's conjecture} implies that $F(x) \ll (\log x)^2$, and furthermore if every interval in $[x,2x]$ of length $\gg \log x$ contains a squarefree number (see [208] ) then every interval of length $\gg (\log x)^2$ contains two numbers with the same number of divisors, whence\[F(x) \ll (\log x)^2.\]See [1004] for the analogous problem with the Euler totient function. This problem is discussed in problem B18 of Guy's collection \cite{Gu04}. References [Er85e] Erd\H{o}s, P., Some problems and results in number theory . Number theory and combinatorics. Japan 1984 (Tokyo, Okayama and Kyoto, 1984) (1985), 65-87. [ErMi52] Erd\H{o}s, P. and Mirsky, L., The distribution of values of the divisor function {$d(n)$} . Proc. London Math. Soc. (3) (1952), 257--271. [Gu04] Guy, Richard K., Unsolved problems in number theory . (2004), xviii+437.

\subsection*{FORMAL RESTATEMENT}
Let $\tau(m)$ denote the number of positive divisors of $m\in\mathbb{N}$.
For $x\ge 1$, define $F(x)$ to be the maximum integer $k\ge 1$ for which there exists an integer $n$ such that
\[
 n+1, n+2,\dots,n+k \le x
\]
and the values $\tau(n+1),\dots,\tau(n+k)$ are pairwise distinct.
The problem asks to estimate $F(x)$ as $x\to\infty$, and in particular asks whether $F(x)\le (\log x)^{O(1)}$.

\subsection*{QUICK LITERATURE/CONTEXT CHECK}
I do not use external results beyond what is in the problem text.
The text gives known bounds of Erd\H{o}s--Mirsky and subsequent improvements (all treated here as context, not re-proved).

\subsection*{ATTACK PLAN}
\emph{Proof-track ideas.}
\begin{itemize}
\item Use upper bounds on the range of $\tau$ on $[1,x]$ to obtain (weak) upper bounds on $F(x)$.
\item Try to control collisions $\tau(n+i)=\tau(n+j)$ within short intervals via multiplicative structure.
\item For lower bounds, attempt CRT-based constructions to prescribe prime-power divisibility patterns of consecutive integers so as to force distinct divisor counts.
\end{itemize}
\emph{Disproof-track ideas.}
\begin{itemize}
\item Search computationally for unusually long runs of distinct $\tau$ values to see if growth could exceed any polylog.
\end{itemize}

\subsection*{WORK}
\paragraph{Fast reality check (explicit computation).}
I computed $F(x)$ exactly for several $x$ by computing $\tau(n)$ for all $n\le x$ and using a sliding-window algorithm to find the longest block of pairwise distinct values.
Exact outputs:
\begin{itemize}
\item $F(10)=3$, attained on $[4,6]$ (i.e. $\tau(4)=3,\tau(5)=2,\tau(6)=4$).
\item $F(100)=6$, attained on $[76,81]$.
\item $F(1000)=7$, first attained on $[270,276]$.
\item $F(10000)=9$, first attained on $[3718,3726]$.
\item $F(200000)=10$, first attained on $[95499,95508]$.
\item $F(10^6)=11$, attained on $[590890,590900]$.
\end{itemize}
These are sanity checks; they do not indicate the true asymptotic.

\paragraph{Lemma 945.1 (divisor function formula).}
If $n=\prod_{i=1}^t p_i^{e_i}$ is the prime factorization of $n\ge 1$ (distinct primes $p_i$ and integers $e_i\ge 1$), then
\[
\tau(n)=\prod_{i=1}^t (e_i+1).
\]

\emph{Proof.}
A positive divisor $d$ of $n$ has the form $d=\prod_{i=1}^t p_i^{f_i}$ where each exponent $f_i$ satisfies $0\le f_i\le e_i$; conversely every such choice gives a divisor.
For each $i$ there are exactly $e_i+1$ choices of $f_i$, and choices across different primes are independent. Thus the total number of divisors is the product $\prod_{i=1}^t (e_i+1)$.
\hfill $\Box$

\paragraph{Lemma 945.2 (trivial upper bound $\tau(n)\le 2\sqrt{n}$).}
For every integer $n\ge 1$,
\[
\tau(n)\le 2\lfloor \sqrt{n}\rfloor \le 2\sqrt{n}.
\]

\emph{Proof.}
Every divisor $d\le \sqrt{n}$ pairs with a distinct divisor $n/d\ge \sqrt{n}$. There are at most $\lfloor\sqrt{n}\rfloor$ divisors $\le \sqrt{n}$, so counting their pairs gives at most $2\lfloor\sqrt{n}\rfloor$ total divisors. (If $n$ is a perfect square, the divisor $\sqrt{n}$ pairs with itself, so the inequality remains valid.)
\hfill $\Box$

\paragraph{Lemma 945.3 (odd values of $\tau$).}
For $n\ge 1$, $\tau(n)$ is odd if and only if $n$ is a perfect square.

\emph{Proof.}
Write $n=\prod p_i^{e_i}$. By Lemma 945.1, $\tau(n)=\prod (e_i+1)$ is odd if and only if every factor $(e_i+1)$ is odd, i.e. every $e_i$ is even. This is equivalent to $n$ being a perfect square.
\hfill $\Box$

\paragraph{Corollary 945.4 (very weak global bound on $F(x)$).}
For every $x\ge 1$,
\[
F(x) \le 2\sqrt{x}.
\]

\emph{Proof.}
If $n+1,\dots,n+k\le x$ and $\tau(n+1),\dots,\tau(n+k)$ are pairwise distinct, then these are $k$ distinct positive integers each bounded by $\max_{m\le x}\tau(m)$. By Lemma 945.2, $\max_{m\le x}\tau(m)\le 2\sqrt{x}$. Hence $k\le 2\sqrt{x}$.
\hfill $\Box$

\subsection*{VERIFICATION}
\begin{itemize}
\item Lemma 945.2: checked the square case explicitly: if $n=t^2$, then the pairing counts $t$ only once, so $\tau(n)\le 2t-1\le 2t$.
\item Corollary 945.4: verified the logic only uses that $\tau(m)$ are positive integers in $[1,2\sqrt{x}]$.
\item Computation sanity: for $x=10$, the longest block found is $4..6$ with divisor counts $3,2,4$ distinct; any block of length $4$ inside $\{1,\dots,10\}$ fails (checked by the program).
\end{itemize}

\subsection*{FINAL}
**UNRESOLVED**
(i) Strongest proved partial result: a very weak universal upper bound $F(x)\le 2\sqrt{x}$ (Corollary 945.4), coming from $\tau(n)\le 2\sqrt{n}$.
(ii) First gap: prove any polylogarithmic upper bound $F(x)\le (\log x)^C$ (or produce evidence/counterexample showing $F(x)$ can be much larger).
(iii) Top 3 next moves:
\begin{enumerate}
\item Try to implement (from scratch) a CRT-based lower-bound construction that forces $\tau(n+i)$ to take prescribed, pairwise distinct values while keeping $n+k\le x$.
\item Attempt to prove an upper bound by showing that in any interval of length $L$ there must exist $i\ne j$ with $\tau(n+i)=\tau(n+j)$; reduce to counting solutions of multiplicative equations and apply divisor/sieve bounds.
\item Extend exact computations to larger $x$ (e.g. $10^8$ or higher) and record the structure of the maximizing intervals to see whether they resemble known constructions (e.g. many values of $\tau$ close to each other or highly composite-like behavior).
\end{enumerate}
(iv) Minimal counterexample structure (to a conjectured polylog upper bound): an exceptionally long interval $[n+1,n+k]$ in which each integer has a \emph{distinct} divisor count, which likely forces unusually rigid and varied prime-exponent patterns across consecutive integers.


