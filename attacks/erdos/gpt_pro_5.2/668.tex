
\noindent\textbf{FORMAL RESTATEMENT.}
For $n\ge 1$, let
\[
u(n) := \max\{\#\{\{i,j\}: 1\le i<j\le n,\ \|x_i-x_j\|=1\}\ :\ x_1,\dots,x_n\in\mathbb{R}^2\ \text{distinct}\}
\]
be the maximum possible number of unit distances among $n$ points in the Euclidean plane.
Let $\mathcal{M}(n)$ denote the set of congruence classes (under Euclidean isometries of $\mathbb{R}^2$) of $n$-point configurations achieving $\nu(n)$.
The problem asks:
\begin{enumerate}
\item Does $|\mathcal{M}(n)|\to\infty$ as $n\to\infty$?
\item Is it true that $|\mathcal{M}(n)|>1$ for all sufficiently large $n$?
\end{enumerate}

\medskip
\noindent\emph{Ambiguity.}
The text asks ``always $>1$ for $n>3$'' but immediately notes $|\mathcal{M}(4)|=1$.
So the literal statement ``$|\mathcal{M}(n)|>1$ for all $n>3$'' is already false (with counterexample $n=4$).
A minimal corrected version would ask ``for all $n>4$'' or ``for all sufficiently large $n$''.

\bigskip
\noindent\textbf{QUICK LITERATURE/CONTEXT CHECK.}
The supplied statement reports computational evidence (up to $n\le 21$) suggesting $|\mathcal{M}(n)|=1$ for many small $n$, though only checked up to graph isomorphism rather than congruence.
Per project rules, I do not use external sources beyond what is written in the prompt.

\bigskip
\noindent\textbf{ATTACK PLAN.}
\begin{itemize}
\item \textbf{Reality check for very small $n$:} determine $\nu(n)$ and $|\mathcal{M}(n)|$ for $n=2,3,4$ by direct geometric reasoning.
\item \textbf{Main questions:} no general method is available here; I will provide only rigorous small-$n$ structure and then state the main problem as unresolved with concrete next steps.
\end{itemize}

\bigskip
\noindent\textbf{WORK.}

\medskip
\noindent\textbf{Fast reality check (tiny $n$).}
For $n=2$, $\nu(2)=1$ and $|\mathcal{M}(2)|=1$ (two points at distance $1$).
For $n=3$, $\nu(3)=3$ and $|\mathcal{M}(3)|=1$ (equilateral triangle of side $1$).
For $n=4$, we prove below that $\nu(4)=5$ and $|\mathcal{M}(4)|=1$.

\medskip
\noindent\textbf{Lemma 668.1 (Upper bound $\nu(4)\le 5$).}
For any $4$ distinct points in $\mathbb{R}^2$, the number of unit distances among them is at most $5$.

\noindent\emph{Proof.}
There are $\binom{4}{2}=6$ unordered pairs of points.
Suppose for contradiction that all $6$ pairs are at distance $1$.
Pick any three of the points, call them $A,B,C$.
Then $|AB|=|BC|=|CA|=1$, so $\triangle ABC$ is an equilateral triangle of side length $1$.
Now consider the fourth point $D$.
The conditions $|DA|=|DB|=1$ mean that $D$ lies in the intersection of the circles of radius $1$ centered at $A$ and $B$.
In the plane, two circles of equal radius with centers at distance $1$ intersect in exactly two points: these are precisely the two vertices of equilateral triangles on base $AB$.
One of these intersection points is $C$; let the other be $C'$ (the reflection of $C$ across the line $AB$).
Thus $D\in\{C,C'\}$.
Since the four points are distinct, $D\ne C$, so $D=C'$.
But then the distance between $C$ and $C'$ is $|CC'|=\sqrt{3}$ (this is a standard computation from equilateral triangles, and can be verified by coordinates, e.g. $A=(0,0)$, $B=(1,0)$, $C=(1/2,\sqrt3/2)$, $C'=(1/2,-\sqrt3/2)$).
In particular $|CC'|\ne 1$, contradicting the assumption that all pairs are at distance $1$.
Therefore at least one pair is not a unit distance, so there are at most $5$ unit distances.
\hfill$\square$

\medskip
\noindent\textbf{Lemma 668.2 (Characterization and uniqueness at $n=4$).}
One has $\nu(4)=5$.
Moreover, any configuration of $4$ points in $\mathbb{R}^2$ with exactly $5$ unit distances is congruent to two unit equilateral triangles joined along a common edge.
In particular $|\mathcal{M}(4)|=1$.

\noindent\emph{Proof.}
\emph{Existence:} Let $A=(0,0)$ and $B=(1,0)$.
Let $C=(1/2,\sqrt3/2)$ and $D=(1/2,-\sqrt3/2)$.
Then $\triangle ABC$ and $\triangle ABD$ are equilateral with side length $1$.
Therefore the five distances $|AB|,|AC|,|BC|,|AD|,|BD|$ equal $1$.
The remaining distance is $|CD|=\sqrt3\ne 1$.
So this configuration has exactly $5$ unit distances.
By Lemma 668.1, this is maximal, hence $\nu(4)=5$.

\emph{Uniqueness:} Let $\{P_1,P_2,P_3,P_4\}$ be any configuration with $5$ unit distances.
By Lemma 668.1, not all $6$ pairwise distances can be $1$, so there is exactly one unordered pair (say $\{P_3,P_4\}$) with $|P_3P_4|\ne 1$, and every other pair has distance $1$.
In the unit-distance graph (vertices $P_i$, edges between unit-distance pairs), this means the graph is $K_4$ minus a single missing edge.
The two vertices incident to the missing edge have degree $2$, and the other two have degree $3$.
Renaming if necessary, we may assume $P_1$ and $P_2$ are the degree-$3$ vertices.
Then
\[
|P_1P_2|=|P_1P_3|=|P_2P_3|=1
\quad\text{and}\quad
|P_1P_2|=|P_1P_4|=|P_2P_4|=1.
\]
In particular, both triples $(P_1,P_2,P_3)$ and $(P_1,P_2,P_4)$ form equilateral triangles of side $1$ with the same base $P_1P_2$.
As in Lemma 668.1, for fixed $P_1,P_2$ with $|P_1P_2|=1$, there are exactly two points in the plane at distance $1$ from both $P_1$ and $P_2$.
Therefore $\{P_3,P_4\}$ must equal this pair of equilateral-triangle vertices on the two sides of the line $P_1P_2$.
Consequently the configuration is congruent to the explicit model above (two equilateral triangles glued on a common edge), and there is only one congruence class.
\hfill$\square$

\bigskip
\noindent\textbf{VERIFICATION.}
\begin{itemize}
\item The proof of Lemma 668.1 uses only basic circle intersection facts and the explicit computation $|CC'|=\sqrt3$.
\item Lemma 668.2 checks both existence and uniqueness, and the uniqueness argument uses only that the unit-distance graph must be $K_4$ minus one edge.
\item Boundary cases $n=2,3$ are consistent with standard geometry and were sanity-checked.
\end{itemize}

\bigskip
\noindent\textbf{FINAL.} \textbf{UNRESOLVED}.

\smallskip
\noindent(i) \emph{Strongest proved partial results.}
\begin{itemize}
\item For $n=3$, $\nu(3)=3$ and the maximizer is unique up to congruence (equilateral triangle).
\item For $n=4$, $\nu(4)=5$ and the maximizer is unique up to congruence (two equilateral triangles joined by an edge), hence $|\mathcal{M}(4)|=1$ (Lemmas 668.1--668.2).
\item Therefore the literal subquestion ``$|\mathcal{M}(n)|>1$ for all $n>3$'' is false (counterexample $n=4$).
\end{itemize}

\noindent(ii) \emph{First gap (crisp statement).}
Determine whether $|\mathcal{M}(n)|\to\infty$ as $n\to\infty$.
Even the weaker question ``does there exist some $n_0$ such that $|\mathcal{M}(n)|>1$ for all $n\ge n_0$?'' is open here.

\noindent(iii) \emph{Top 3 next moves.}
\begin{enumerate}
\item \textbf{Rigorous small-$n$ certification:} for $5\le n\le 10$, attempt a full congruence-level classification of maximizers by combining exact distance-geometry constraints with computer-assisted solving (not just graph isomorphism).
\item \textbf{Stability analysis:} prove that near-extremal unit distance graphs must be close (in a precise sense) to a specific geometric template; then classify all exact extremizers as perturbations of that template.
\item \textbf{Search for multiplicity mechanisms:} identify a construction that, for infinitely many $n$, admits a continuous or discrete family of noncongruent realizations while keeping the unit-distance count maximal.
\end{enumerate}

\noindent(iv) \emph{What a minimal counterexample would likely look like.}
\begin{itemize}
\item If $|\mathcal{M}(n)|\to\infty$ is false, there should exist an infinite sequence $n_t\to\infty$ such that the extremal unit-distance graph(s) have essentially unique geometric realization up to congruence.
\item If instead the ``eventually $>1$'' claim is false, one would need infinitely many $n$ with a unique maximizer (as already happens at $n=3,4$).
\end{itemize}


