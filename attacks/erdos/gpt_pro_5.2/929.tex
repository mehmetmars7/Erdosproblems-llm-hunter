% Erdos Problem #929

1) FORMAL RESTATEMENT

Let k\ge 2 be an integer.
For x\ge 2, define the property (\*)_{k,x}(n):
\[(\*)_{k,x}(n)\quad\text{means: for each } i\in\{1,2,\dots,k\},\; n+i \text{ is divisible by at least one prime } p\le x.\]
Equivalently, each of the k consecutive integers n+1,\dots,n+k has a prime factor \le x.

Let A_{k,x}:=\{n\in\mathbb N: (\*)_{k,x}(n) \text{ holds}\}.
Define S(k) to be the minimal real x such that A_{k,x} has positive (natural) density, i.e.
\[ \limsup_{N\to\infty} \frac{|A_{k,x}\cap[1,N]|}{N} > 0. \]
(For fixed k, the intended meaning of “positive density set of n” in the problem statement is: the asymptotic density exists and is >0, or at least that the upper density is >0; for our purposes we will show that once A_{k,x} is nonempty it already contains an arithmetic progression and hence has positive density.)

Question: Estimate S(k) as k\to\infty. In particular, is S(k)\ge k^{1-o(1)}?


2) QUICK LITERATURE/CONTEXT CHECK

The problem statement itself records:
- A lower bound from Rosser’s sieve: S(k) > k^{1/2-o(1)}.
- A trivial upper bound: S(k)\le k+1 (e.g. take n\equiv 1 \pmod{(k+1)!}).
- An improved upper bound from large prime gaps (Ford–Green–Konyagin–Maynard–Tao):
  \(S(k) \ll k \frac{\log\log\log k}{\log\log k\,\log\log\log\log k}.\)

I do not use or assert any additional external results beyond what is stated in the problem file.


3) ATTACK PLAN

Proof-track ideas (lower bounds on S(k)):
- Translate the condition into a covering problem modulo the primorial \(P(x)=\prod_{p\le x} p\) and apply sieve lower bounds to show one cannot cover an interval of length k using only primes \le x when x is too small.

Construction-track ideas (upper bounds on S(k)):
- Use Chinese remainder / covering congruences to force each of n+1,\dots,n+k to be divisible by a chosen small prime, yielding explicit arithmetic progressions.

Best path here: I can prove exact equivalences and the trivial upper bound rigorously, and I can compute small values of S(k) exactly for small k by exhaustive search modulo primorials. The asymptotic lower bound S(k)\ge k^{1-o(1)} remains open.


4) WORK

Lemma 929.1 (nonempty \Rightarrow\ positive density).
Fix k\ge 2 and x\ge 2.
Let M(x):=\prod_{p\le x} p (the product of all primes \le x).
If there exists n_0 such that (\*)_{k,x}(n_0) holds, then A_{k,x} contains the entire arithmetic progression \{n_0 + t M(x): t\in\mathbb Z_{\ge 0}\}.
In particular, A_{k,x} has natural density at least 1/M(x), hence positive.

Proof.
Assume (\*)_{k,x}(n_0) holds.
Fix i\in\{1,\dots,k\}.
By definition, there exists a prime p_i\le x such that p_i divides n_0+i.
Since p_i\le x, the prime p_i is one of the factors of M(x), so p_i divides M(x).
Therefore for any t\ge 0,
\[ n_0 + tM(x) + i \equiv n_0 + i \pmod{p_i}, \]
so p_i divides n_0+tM(x)+i.
Thus for each i, the integer n_0+tM(x)+i is divisible by a prime \le x.
Hence (\*)_{k,x}(n_0+tM(x)) holds for every t\ge 0.
So the full arithmetic progression is contained in A_{k,x}.
The progression has density exactly 1/M(x), so A_{k,x} has density at least 1/M(x)>0. \qed

Corollary 929.1a.
For fixed k, S(k) equals the minimal x such that there exists at least one integer n with (\*)_{k,x}(n).

Proof.
If A_{k,x} is nonempty then by Lemma 929.1 it has positive density, so such x is admissible for S(k).
Conversely, if A_{k,x} has positive density then it is certainly nonempty.
Thus “positive density” and “nonempty” are equivalent for A_{k,x}, and taking minima over x gives the corollary. \qed

Lemma 929.2 (trivial upper bound S(k)\le k+1).
For every integer k\ge 2, we have S(k)\le k+1.

Proof.
Let x:=k+1 and take n satisfying
\[ n \equiv 1 \pmod{(k+1)!}. \]
Then for each i\in\{1,\dots,k\}, we have
\[ n+i = 1 + i + t (k+1)! = (i+1) + t (k+1)! \]
for some integer t, hence (i+1) divides n+i because (i+1) divides (k+1)!.
Therefore n+i is divisible by a prime factor of (i+1), and every prime factor of (i+1) is \le i+1\le k+1=x.
So (\*)_{k,x}(n) holds.
By Corollary 929.1a, this implies S(k)\le x=k+1. \qed

FAST REALITY CHECK (exact small-k computation).
Using Corollary 929.1a, I computed S(k) for k\le 30 by brute force search modulo primorials.
I also computed, for each x, the maximum length L(x) of a run of consecutive integers all having a prime factor \le x.
Exact values found:
- L(2)=1,
  L(3)=3,
  L(5)=5,
  L(7)=9,
  L(11)=13,
  L(13)=21,
  L(17)=25,
  L(19)=33.
From these, the exact S(k) for k\le 30 is:
  S(1)=2,
  S(2)=3,
  S(3)=3,
  S(4)=5,
  S(5)=5,
  S(6)=7,
  S(7)=7,
  S(8)=7,
  S(9)=7,
  S(10)=11,
  S(11)=11,
  S(12)=11,
  S(13)=11,
  S(14)=13,
  S(15)=13,
  S(16)=13,
  S(17)=13,
  S(18)=13,
  S(19)=13,
  S(20)=13,
  S(21)=13,
  S(22)=17,
  S(23)=17,
  S(24)=17,
  S(25)=17,
  S(26)=19,
  S(27)=19,
  S(28)=19,
  S(29)=19,
  S(30)=19.

An explicit witness for k=33 and x=19 is n=60043, since each of 60044,60045,\dots,60076 has a prime factor \le 19; by Lemma 929.1 this yields a positive-density arithmetic progression of solutions.


5) VERIFICATION

- Lemma 929.1: Checked that using M(x)=\prod_{p\le x}p ensures every prime p\le x divides M(x), so divisibility persists under translation by M(x).
- Lemma 929.2: Checked that for each i, i+1 divides (k+1)!, hence divides n+i.
- Computational check: Verified the run length L(19)=33 by direct gcd computation against the primorial 2\cdot3\cdot5\cdot7\cdot11\cdot13\cdot17\cdot19=9699690.


6) FINAL

**UNRESOLVED**
(i) Strongest proved partial result: For each fixed (k,x), existence of a single n with the property implies a positive-density infinite arithmetic progression of such n (Lemma 929.1), and in particular S(k) can be characterized as the minimal x for which at least one such n exists (Corollary 929.1a). Also S(k)\le k+1 holds by an explicit congruence construction (Lemma 929.2). Exact values of S(k) were computed for k\le 30.
(ii) First gap (crisp): Prove a near-linear lower bound S(k)\ge k^{1-o(1)} (or exhibit a family of coverings showing S(k)\le k^{1-\delta} for some fixed \delta>0 infinitely often).
(iii) Top 3 next moves:
  1. Reformulate the problem as a covering congruence problem: choose for each prime p\le x one residue class a_p (mod p) so that {1,\dots,k} is covered by \bigcup_{p\le x}\{i: i\equiv a_p\ (\mathrm{mod}\ p)\}. Use this to search algorithmically for better constructions and to attempt lower bounds.
  2. Attempt to prove a sieve lower bound on the Jacobsthal-type function for primorials: show the maximum run of integers not coprime to \prod_{p\le x} p is \le x^{1+o(1)} (or similar), then translate to S(k) lower bounds.
  3. Extend brute force computations of S(k) to larger k (within feasible primorial sizes) to guess the true order of growth and test the plausibility of S(k)\ge k^{1-o(1)}.
(iv) Minimal counterexample structure: A counterexample to S(k)\ge k^{1-o(1)} would provide a sequence k_j\to\infty and primes bounded by x_j\le k_j^{1-\delta} together with residue classes mod those primes that cover all indices 1..k_j; equivalently, runs of length k_j in the sequence of integers all having a prime factor \le x_j with x_j substantially smaller than k_j.
