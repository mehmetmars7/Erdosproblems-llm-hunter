% Erdos Problem #596
% URL: https://www.erdosproblems.com/596

For which graphs $G_1,G_2$ is it true that {UL} {LI} for every $n\geq 1$ there is a graph $H$ without a $G_1$ but if the edges of $H$ are $n$-coloured then there is a monochromatic copy of $G_2$, and yet{/LI} {LI} for every graph $H$ without a $G_1$ there is an $\aleph_0$-colouring of the edges of $H$ without a monochromatic $G_2$. {/UL} Erd\H{o}s and Hajnal originally conjectured that there are no such $G_1,G_2$, but in fact $G_1=C_4$ and $G_2=C_6$ is an example. Indeed, for this pair Ne\v{s}et\v{r}il and R\"{o}dl established the first property and Erd\H{o}s and Hajnal the second (in fact every $C_4$-free graph is a countable union of trees). Whether this is true for $G_1=K_4$ and $G_2=K_3$ is the content of [595] .

\bigskip
\noindent\textbf{FORMAL RESTATEMENT}

Fix finite graphs $G_1,G_2$.
Consider the two properties:
\begin{enumerate}
\item[(P1)] For every integer $n\ge 1$, there exists a graph $H$ which is $G_1$-free (contains no copy of $G_1$) such that for every edge-colouring $\kappa:E(H)\to[n]$ there exists a monochromatic copy of $G_2$ in $H$.
\item[(P2)] For every $G_1$-free graph $H$, there exists an edge-colouring $\kappa:E(H)\to\omega$ such that $H$ has no monochromatic copy of $G_2$.
\end{enumerate}

Problem: classify the pairs $(G_1,G_2)$ for which both (P1) and (P2) hold.
The statement notes that $(G_1,G_2)=(C_4,C_6)$ is such a pair.

\bigskip
\noindent\textbf{QUICK LITERATURE/CONTEXT CHECK}

I will not import external results beyond what is explicitly stated in the problem text.
The text asserts that $(C_4,C_6)$ satisfies (P1) (due to Ne\v{s}et\v{r}il--R\"{o}dl) and (P2) (due to Erd\H{o}s--Hajnal), and even notes a strengthening: every $C_4$-free graph is a countable union of trees.
The general classification problem is open-ended.

\bigskip
\noindent\textbf{ATTACK PLAN}

\emph{Understanding (P2).}
Rewrite (P2) as a countable decomposition problem: a countable edge-colouring avoiding monochromatic $G_2$ is the same as writing $E(H)$ as a countable union of $G_2$-free subgraphs.
Seek structural reasons why $G_1$-freeness might force such a decomposition (as for $C_4$-free graphs being a countable union of trees).

\emph{Understanding (P1).}
(P1) is a ``Folkman-type'' property: for each finite $n$, some $G_1$-free graph is edge-Ramsey for $G_2$ under $n$ colours.
Try to understand which $(G_1,G_2)$ admit such finite-colour Ramsey obstructions.

I did not obtain a classification. I provide a few rigorous equivalences and structural lemmas relevant to the stated example $(C_4,C_6)$.

\bigskip
\noindent\textbf{WORK}

\medskip
\noindent\textbf{FAST REALITY CHECK (the example $(C_4,C_6)$ for $n=1$)}

For $n=1$, property (P1) for $(C_4,C_6)$ asks for a $C_4$-free graph $H$ such that any 1-colouring (i.e. no colouring choice) contains a monochromatic $C_6$, i.e. simply $H$ contains a $C_6$.
Taking $H=C_6$ works and is $C_4$-free.
For $n\ge 2$ the existence of such $H$ is nontrivial and is attributed in the problem text to Ne\v{s}et\v{r}il--R\"{o}dl.

\medskip
\noindent\textbf{Lemma 1 (reformulating (P2) as a decomposition).}\label{lem:P2_decomp}
Fix graphs $G_1,G_2$.
For a $G_1$-free graph $H=(V,E)$, the following are equivalent:
\begin{enumerate}
\item There exists an edge-colouring $\kappa:E\to\omega$ with no monochromatic copy of $G_2$.
\item The edge-set $E$ can be written as a countable union $E=\bigcup_{n<\omega} E_n$ such that each subgraph $(V,E_n)$ is $G_2$-free.
\end{enumerate}

\emph{Proof.}
(1)$\Rightarrow$(2): Let $E_n:=\kappa^{-1}(\{n\})$. Then $E=\bigcup_n E_n$. If some $(V,E_n)$ contained a copy of $G_2$, that copy would be monochromatic of colour $n$, contradicting (1).

(2)$\Rightarrow$(1): Given $E=\bigcup_n E_n$ with each $(V,E_n)$ $G_2$-free, define a colouring $\kappa$ by $\kappa(e)=\min\{n:e\in E_n\}$.
Then for each $n$, the colour class $\kappa^{-1}(\{n\})$ is a subgraph of $(V,E_n)$ and hence is $G_2$-free.
Therefore there is no monochromatic copy of $G_2$.\qed

\medskip
\noindent\textbf{Lemma 2 (countable union of trees suffices to avoid monochromatic cycles).}\label{lem:trees_avoid_cycles}
Let $H=(V,E)$ be a graph.
If $E$ is a countable union of edge-sets each of which forms a forest (in particular a tree), then $H$ has an $\aleph_0$-edge-colouring with no monochromatic cycle.
In particular, it has an $\aleph_0$-edge-colouring with no monochromatic $C_6$.

\emph{Proof.}
Assume $E=\bigcup_{n<\omega}E_n$ where each $(V,E_n)$ is a forest.
Colour every edge $e\in E_n$ with colour $n$ using the ``least index'' rule to make it a genuine colouring (as in Lemma~\ref{lem:P2_decomp}).
For each $n$, the colour class is a subgraph of a forest, hence is a forest, hence contains no cycles of any length.
Therefore no cycle (and in particular no $6$-cycle) can be monochromatic.\qed

\medskip
\noindent\textbf{Lemma 3 (a basic structural property of $C_4$-free graphs).}\label{lem:C4_codegree}
If a graph $H$ is $C_4$-free, then any two distinct vertices have at most one common neighbour.

\emph{Proof.}
Let $u\neq v$ be vertices.
Suppose, for contradiction, that $u$ and $v$ have two distinct common neighbours $x\neq y$.
Then the four distinct vertices $u,x,v,y$ form a 4-cycle
\[
 u-x-v-y-u
\]
because the edges $ux,xv,vy,yu$ all exist by ``common neighbour''.
This is a copy of $C_4$ in $H$, contradicting that $H$ is $C_4$-free.
Therefore $u$ and $v$ have at most one common neighbour.\qed

\bigskip
\noindent\textbf{VERIFICATION}

\begin{itemize}
\item Lemma~\ref{lem:P2_decomp} is purely about re-encoding a cover as a colouring (and conversely). It matches the formulation of (P2).
\item Lemma~\ref{lem:trees_avoid_cycles} correctly implies the $C_4,C_6$ example's (P2) provided we accept the problem-text claim ``every $C_4$-free graph is a countable union of trees''.
\item Lemma~\ref{lem:C4_codegree} checks the standard implication from two common neighbours to a (not-necessarily-induced) 4-cycle.
\end{itemize}

\bigskip
\noindent\textbf{FINAL}

\textbf{UNRESOLVED}

(i) \emph{Strongest proved partial result.}
Property (P2) for a pair $(G_1,G_2)$ is equivalent to: every $G_1$-free graph is a countable union of $G_2$-free subgraphs (Lemma~\ref{lem:P2_decomp}).
In the example $(C_4,C_6)$, the further sufficient condition ``countable union of trees'' indeed implies (P2) (Lemma~\ref{lem:trees_avoid_cycles}).

(ii) \emph{First gap (crisp).}
Give a structural characterization of pairs $(G_1,G_2)$ for which both (P1) and (P2) hold.
Even for specific natural pairs beyond $(C_4,C_6)$ (e.g. $(K_4,K_3)$, which is Problem~595), no classification is established here.

(iii) \emph{Top 3 next moves.}
\begin{enumerate}
\item Systematically analyze (P2) as an ``arboricity-type'' condition: for which $G_1$ does $G_1$-freeness force countable decomposition into $G_2$-free graphs? Try first when $G_2$ is a cycle.
\item For (P1), attempt to identify which $G_2$ admit $n$-colour edge-Ramsey obstructions inside $G_1$-free graphs (a generalization of Folkman-type existence).
\item Explore whether (P2) can ever hold when $G_2$ is a clique (e.g. $K_3$) and $G_1$ is also a clique (e.g. $K_4$), by seeking either a decomposition theorem or a transfinite counterexample.
\end{enumerate}

(iv) \emph{Minimal counterexample structure.}
To refute a proposed classification, one would need either:
(a) a pair $(G_1,G_2)$ where for every $n$ there exists a $G_1$-free graph $H$ that is edge-Ramsey for $G_2$ under $n$ colours, \emph{but} some $G_1$-free graph admits no countable decomposition into $G_2$-free subgraphs; or
(b) a pair where (P2) always holds but (P1) fails for some $n$.
For clique pairs like $(K_4,K_3)$, any counterexample must be uncountable and have triangles spread so that every countable edge-colouring traps a monochromatic triangle.

