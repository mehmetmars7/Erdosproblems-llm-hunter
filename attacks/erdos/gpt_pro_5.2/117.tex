% Erdos Problem #117
% Attempt for Erdos Problem #117
% Following PROMPT_STRATEGY.MD
% Tools/Constraints:
% - Web browsing available? YES (not used; only facts explicitly stated in the problem text)
% - Computation available (Python)? YES (not used)

\section*{Erd\H{o}s Problem \#117}

\subsection*{1) FORMAL RESTATEMENT}
Fix $n\in\mathbb N$.  Consider (finite) groups $G$ with the property:

\begin{quote}
Every subset $S\subseteq G$ with $|S|>n$ contains two distinct commuting elements (i.e. $\exists x\neq y\in S$ with $xy=yx$).
\end{quote}

Equivalently, the maximum size of a pairwise \emph{noncommuting} set in $G$ is at most $n$.

Define $h(n)$ to be the least integer such that every such group $G$ can be covered by at most $h(n)$ abelian subgroups.
The task is to estimate $h(n)$ as sharply as possible.

\subsection*{2) QUICK LITERATURE/CONTEXT CHECK}
From the problem text:
\begin{itemize}
\item Pyber proved there exist constants $c_2>c_1>1$ with $c_1^n<h(n)<c_2^n$.
\item Erd\H{o}s notes the lower bound was already known to Isaacs.
\end{itemize}
No additional results are assumed here.

\subsection*{3) ATTACK PLAN}
Translate the covering problem into graph theory via the noncommuting graph, then relate coverings by abelian subgroups to colorings/centralizers.

\subsection*{4) WORK}
\paragraph{Lemma 117.1 (reformulation via the noncommuting graph).}
Let $G$ be a finite group and define its \emph{noncommuting graph} $\Gamma(G)$ to be the simple graph with vertex set $V=G$ and an edge $\{x,y\}$ iff $xy\neq yx$.
Then:
\begin{enumerate}
\item A subset $S\subseteq G$ is pairwise commuting iff it is an independent set in $\Gamma(G)$.
\item The hypothesis ``every subset of size $>n$ contains a commuting pair'' is equivalent to $\omega(\Gamma(G))\le n$, where $\omega$ is the clique number.
\item If $G$ is covered by $t$ abelian subgroups, then $\chi(\Gamma(G))\le t$.
\end{enumerate}

\emph{Proof.}
(1) By definition of edges, $S$ has no edges among its vertices iff every pair commutes.

(2) A clique in $\Gamma(G)$ is a set of pairwise noncommuting elements.  The stated property says no set of size $n+1$ is pairwise noncommuting, i.e. the maximum clique size is at most $n$.

(3) If $G=\bigcup_{i=1}^t A_i$ with each $A_i$ abelian, then each $A_i$ is pairwise commuting and hence an independent set in $\Gamma(G)$ by (1).  Coloring each vertex by an index $i$ with $x\in A_i$ gives a proper $t$-coloring.  Thus $\chi(\Gamma(G))\le t$. \qed

\paragraph{Lemma 117.2 (cover by centralizers from a maximal noncommuting set).}
Let $G$ be a finite group and let $S\subseteq G$ be a maximal (by inclusion) pairwise noncommuting set.  Then
\[
G=\bigcup_{s\in S} C_G(s),
\]
where $C_G(s)=\{g\in G: gs=sg\}$ is the centralizer of $s$.

\emph{Proof.}
Take any $g\in G$.  If $g$ failed to commute with every $s\in S$, then $S\cup\{g\}$ would still be pairwise noncommuting (since $g$ is noncommuting with each element of $S$), contradicting maximality.  Hence there exists $s\in S$ such that $gs=sg$, meaning $g\in C_G(s)$.  Therefore $g$ lies in the union of the centralizers. \qed

\paragraph{Remark 117.3 (why this is nontrivial).}
Each $C_G(s)$ need not be abelian.  So Lemma~117.2 does not directly yield a cover by abelian subgroups, but it shows that small maximal noncommuting sets force $G$ to be a union of relatively structured subgroups.

\subsection*{5) VERIFICATION (FAST REALITY CHECK)}
\begin{itemize}
\item For $n=1$, the hypothesis says every 2-element subset contains a commuting pair, hence every pair of elements commutes and $G$ is abelian.  Then $h(1)=1$.
\item In Lemma~117.2, maximality is essential: if $S$ is not maximal, the union of centralizers may miss elements that commute with none of $S$.
\end{itemize}

\subsection*{6) FINAL}
\textbf{UNRESOLVED.}

(i) \emph{Strongest fully proved partial result obtained here.}
Graph-theoretic translation of the hypothesis (Lemma~117.1) and the centralizer covering lemma (Lemma~117.2).

(ii) \emph{Exact first gap.}
Turn the structural information ``$\omega(\Gamma(G))\le n$'' into an explicit upper bound on the number of abelian subgroups needed to cover $G$ (equivalently, control $\chi(\Gamma(G))$ by $\omega(\Gamma(G))$ within the class of noncommuting graphs).

(iii) \emph{Top 3 next moves.}
\begin{enumerate}
\item Study extremal examples: groups with large noncommuting graphs but bounded clique number (e.g. families of almost-simple groups) to estimate growth rates.
\item Try to prove that each centralizer $C_G(s)$ (or a bounded number of its abelian subgroups) already covers a positive fraction of $G$, giving exponential-type bounds.
\item Translate to bounds on the index of the center or on conjugacy class sizes, since commuting relations are tightly linked to centralizers.
\end{enumerate}

(iv) \emph{Minimal counterexample structure.}
A counterexample to a proposed bound $h(n)\le C^n$ with small $C$ would be a sequence of finite groups $G_n$ with maximal noncommuting set size $\le n$ but requiring more than $C^n$ abelian subgroups to cover them.


