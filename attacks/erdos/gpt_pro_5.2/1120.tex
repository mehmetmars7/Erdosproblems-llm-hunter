
1) FORMAL RESTATEMENT

Fix an integer \(n\ge 1\). Let \(f\in\mathbb{C}[z]\) be a monic polynomial of degree \(n\) whose roots \(\alpha_1,\dots,\alpha_n\) satisfy \(|\alpha_j|\le 1\).
Define the lemniscate sublevel set
\[
E(f):=\{z\in\mathbb{C}: |f(z)|\le 1\}.
\]
For such \(f\), define
\[
\ell(f):=\inf\{\operatorname{length}(\gamma): \gamma\text{ is a rectifiable path in }E(f),\ \gamma(0)=0,\ |\gamma(1)|=1\}.
\]
(We interpret ``a path joining \(0\) to \(|z|=1\)'' as a continuous curve with one endpoint at \(0\) and the other on the unit circle; rectifiability is imposed to make ``length'' well-defined.)

The natural extremal quantity is
\[
L_n:=\sup\{\ell(f): f\text{ monic degree }n\text{ with all roots }|\alpha_j|\le 1\}.
\]
The question asks for the behavior of \(L_n\), in particular how large \(\ell(f)\) can be as a function of \(n\).

2) QUICK LITERATURE/CONTEXT CHECK

I did not browse the web; I only record what is stated in the problem text:

- Clunie and Netanyahu (via personal communication reported in Hayman) showed that such a path always exists.
- The trivial lower bound is \(\ell(f)\ge 1\), achieved by \(f(z)=z^n\).
- Erd\H{o}s suggested \(L_n\to\infty\) with \(n\) but not too fast.

3) ATTACK PLAN

- Prove basic necessary properties of \(E(f)\): e.g. show \(E(f)\) always meets the unit circle \(|z|=1\), and record easy lower bounds on \(\ell(f)\).
- Try to get any nontrivial upper bound on \(\ell(f)\) (or \(L_n\)) in terms of \(n\). (I do not succeed beyond trivialities.)

4) WORK

\textbf{Lemma 1120.1 (Existence of a point on the unit circle with \(|f|\le 1\)).}
Let \(f\) be monic of degree \(n\) with all roots \(|\alpha_j|\le 1\). Then there exists \(\zeta\) with \(|\zeta|=1\) such that \(|f(\zeta)|\le 1\). Equivalently, \(E(f)\cap\{z:|z|=1\}\ne\emptyset\).

\emph{Proof.}
Write \(f(z)=\prod_{j=1}^n (z-\alpha_j)\) with \(|\alpha_j|\le 1\).
A standard Jensen formula for polynomials states that
\[
\frac{1}{2\pi}\int_0^{2\pi} \log|f(e^{i\theta})|\,d\theta
= \log|a_n| + \sum_{|\alpha_j|>1} \log|\alpha_j|,
\]
where \(a_n\) is the leading coefficient. Here \(a_n=1\) and there are no roots with \(|\alpha_j|>1\), so the right-hand side equals \(0\). Hence
\[
\frac{1}{2\pi}\int_0^{2\pi} \log|f(e^{i\theta})|\,d\theta = 0.
\]
If \(|f(e^{i\theta})|>1\) for all \(\theta\), then \(\log|f(e^{i\theta})|>0\) for all \(\theta\), forcing the integral to be positive, contradiction. Therefore there exists some \(\theta\) with \(\log|f(e^{i\theta})|\le 0\), i.e. \(|f(e^{i\theta})|\le 1\). Taking \(\zeta=e^{i\theta}\) completes the proof. \qed

\textbf{Lemma 1120.2 (Universal lower bound \(\ell(f)\ge 1\)).}
For any path \(\gamma\) with \(\gamma(0)=0\) and \(|\gamma(1)|=1\), its Euclidean length satisfies \(\operatorname{length}(\gamma)\ge 1\). Consequently \(\ell(f)\ge 1\) for every admissible \(f\), and thus \(L_n\ge 1\) for all \(n\).

\emph{Proof.}
For any rectifiable curve \(\gamma\), the length dominates the straight-line distance between endpoints:
\(\operatorname{length}(\gamma)\ge |\gamma(1)-\gamma(0)|\) (triangle inequality applied to polygonal approximations, then take infimum).
Here \(\gamma(0)=0\) and \(|\gamma(1)|=1\), so \(|\gamma(1)-\gamma(0)|=1\). \qed

\textbf{Lemma 1120.3 (The case \(n=1\): \(L_1=1\)).}
If \(n=1\), every admissible polynomial has the form \(f(z)=z-\alpha\) with \(|\alpha|\le 1\), and for every such \(f\) we have \(\ell(f)=1\). Hence \(L_1=1\).

\emph{Proof.}
For \(f(z)=z-\alpha\),
\(E(f)=\{z:|z-\alpha|\le 1\}\) is the closed disk of radius \(1\) centered at \(\alpha\). Since \(|\alpha|\le 1\), this disk contains \(0\).
By Lemma 1120.1 there exists \(\zeta\) with \(|\zeta|=1\) and \(|\zeta-\alpha|\le 1\), i.e. \(\zeta\in E(f)\cap\{|z|=1\}\).
Because \(E(f)\) is convex, the straight segment from \(0\) to \(\zeta\) lies in \(E(f)\). Its length is \(|\zeta-0|=1\), so \(\ell(f)\le 1\). Lemma 1120.2 gives \(\ell(f)\ge 1\), so \(\ell(f)=1\) for all such \(f\). \qed

\textbf{FAST REALITY CHECK (non-rigorous numerical exploration for small \(n\)).}
To sanity-check that paths exist and to get a feel for the geometry, I discretized the square \([-1.2,1.2]^2\) with mesh size \(h=0.02\), marked grid points satisfying \(|f(z)|\le 1\), and ran a shortest-path search on the resulting grid graph from the grid point nearest \(0\) to grid points lying within distance \(\approx h/2\) of the unit circle. For random degree-4 polynomials built from 4 random roots in the unit disk, the algorithm typically found paths of length very close to \(1\) (often within a few percent), consistent with the trivial lower bound. This computation is only a coarse heuristic because:
(a) the endpoint is only approximately on \(|z|=1\),
(b) the grid introduces discretization artifacts, and
(c) the path is constrained to grid edges.

5) VERIFICATION

- Lemma 1120.1: the only nontrivial input is Jensen's formula for polynomials; the specialization to monic polynomials with all roots \(|\alpha_j|\le 1\) gives an average log-modulus of \(0\), forcing some point with \(|f|\le 1\) on \(|z|=1\).
- Lemma 1120.2 is purely metric and applies to any rectifiable path.
- Lemma 1120.3 checks the base case \(n=1\) completely.

6) FINAL

**UNRESOLVED**

(i) Strongest proved partial result: \(E(f)\) always meets the unit circle (Lemma 1120.1), and every admissible \(f\) satisfies the universal lower bound \(\ell(f)\ge 1\) (Lemma 1120.2). In the base case \(n=1\), the extremal value is exactly \(L_1=1\) (Lemma 1120.3).

(ii) First gap (crisp): Prove any nontrivial upper bound on \(L_n\) in terms of \(n\) (or prove \(L_n\to\infty\)), i.e. control how long the shortest path inside \(E(f)\) might have to be in the worst case.

(iii) Top 3 next moves:
1. Use the branched-covering viewpoint: \(E(f)=f^{-1}(\overline{\mathbb{D}})\). Try to relate existence/length of a path from \(0\) to \(|z|=1\) inside \(E(f)\) to the location of critical values of \(f\) relative to the unit disk, potentially via a ``connected component'' criterion.
2. Search for explicit candidate extremizers for \(L_n\) by arranging roots near \(|z|=1\) so that the lemniscate \(|f|=1\) develops narrow ``necks'' and forces long detours; then try to bound neck widths using product structure \(\prod|z-\alpha_j|\).
3. Attempt to bound \(\ell(f)\) in terms of geometric quantities of \(E(f)\) (e.g. diameter, conformal modulus of an annulus separating \(0\) from \(|z|=1\) inside \(E(f)\)), and then bound those quantities using potential theory and the root constraint \(|\alpha_j|\le 1\).

(iv) Minimal counterexample structure (if \(L_n\) grows quickly): one would expect a polynomial whose sublevel set \(E(f)\) contains \(0\) but connects to the unit circle only through a long, thin corridor (high conformal modulus), with the corridor formed by the interaction of many root factors \(|z-\alpha_j|\) being slightly larger than 1 except along a winding route.
