
1) FORMAL RESTATEMENT

For an integer $n\ge 1$, a divisor $d\mid n$ is called a unitary divisor (notation $d\parallel n$)
if $\gcd(d,n/d)=1$.
Define the sum of unitary divisors
\[
  \sigma^*(n):=\sum_{d\parallel n} d.
\]
A number $n\ge 1$ is called unitary perfect if
\[
  \sigma^*(n)=2n.
\]
Question: Are there only finitely many unitary perfect numbers?

2) QUICK LITERATURE/CONTEXT CHECK

The provided text states:
- There are no odd unitary perfect numbers.
- Five unitary perfect numbers are known: $6,60,90,87360,146361946186458562560000$.
I will not assume any deeper classification, but I will verify computationally the small ones and prove
some standard structural lemmas.

3) ATTACK PLAN

Proof track:
- Derive a multiplicative formula for $\sigma^*(n)$ (Lemma 4.1).
- Translate the unitary perfect condition into a product equation $\prod (1+1/p^k)=2$.
- Prove simple structural consequences (evenness, constraints on number of prime factors).

Construction/disproof track:
- Perform a finite computational search up to a moderately large bound to confirm known solutions
  and look for new ones.

4) WORK

FAST REALITY CHECK (computation)

I searched all $2\le n\le 10^7$ by factoring each $n$ via a smallest-prime-factor sieve and evaluating
$\sigma^*(n)$ using Lemma 4.1.
Result: the only unitary perfect numbers in this range are
\[
  6,\ 60,\ 90,\ 87360.
\]
(No new examples up to $10^7$.)

Lemma 4.1 (Formula for $\sigma^*(n)$)

Let $n=\prod_{i=1}^r p_i^{\alpha_i}$ be the prime factorization of $n$.
Then the unitary divisors of $n$ are exactly the numbers
\[
  d=\prod_{i=1}^r p_i^{\varepsilon_i\alpha_i}\quad\text{with }\varepsilon_i\in\{0,1\},
\]
and therefore
\[
  \sigma^*(n)=\prod_{i=1}^r (1+p_i^{\alpha_i}).
\]
In particular, $\sigma^*$ is multiplicative.

Proof.
A divisor $d\mid n$ can be written uniquely as $d=\prod p_i^{\beta_i}$ with $0\le \beta_i\le \alpha_i$.
Then $n/d=\prod p_i^{\alpha_i-\beta_i}$.
The condition $\gcd(d,n/d)=1$ holds iff for every $i$ we do not have $p_i$ dividing both $d$ and $n/d$,
which is equivalent to: for each $i$, either $\beta_i=0$ or $\beta_i=\alpha_i$.
Thus unitary divisors correspond to choosing, independently for each prime, whether to include the full
prime power $p_i^{\alpha_i}$ or not.
Summing over all $2^r$ choices factors:
\[
  \sigma^*(n)=\prod_{i=1}^r (1+p_i^{\alpha_i}).
\]
Multiplicativity follows from the product form.
$\square$

Corollary 4.2 (Unitary perfect condition as a product equation)

If $n=\prod p_i^{\alpha_i}$ is unitary perfect, then
\[
  \prod_{i=1}^r \left(1+\frac{1}{p_i^{\alpha_i}}\right)=2.
\]

Proof.
Divide the identity $\sigma^*(n)=2n$ by $n=\prod p_i^{\alpha_i}$ and use Lemma 4.1.
$\square$

Lemma 4.3 (No odd unitary perfect numbers)

There is no odd integer $n$ with $\sigma^*(n)=2n$.

Proof.
Assume $n$ is odd and write $n=\prod_{i=1}^r p_i^{\alpha_i}$ with all $p_i$ odd.
Then each factor $1+p_i^{\alpha_i}$ is even.
Hence $\sigma^*(n)=\prod (1+p_i^{\alpha_i})$ is divisible by $2^r$.
But $2n$ has exact $2$-adic valuation $v_2(2n)=1$.
Therefore we must have $r=1$.
So $n=p^\alpha$ for some odd prime $p$.
Then $\sigma^*(n)=1+p^\alpha$, and $\sigma^*(n)=2n$ becomes
$1+p^\alpha = 2p^\alpha$, i.e. $p^\alpha=1$, impossible.
Thus no odd unitary perfect number exists.
$\square$

Lemma 4.4 (Constraint relating $v_2(n)$ to the number of odd prime factors)

Let $n=2^a m$ with $m$ odd, and write $m=\prod_{i=1}^r p_i^{\alpha_i}$ with distinct odd primes $p_i$.
If $n$ is unitary perfect then necessarily
\[
  r\le a+1.
\]

Proof.
Using Lemma 4.1,
\[
  \sigma^*(n)=(1+2^a)\prod_{i=1}^r (1+p_i^{\alpha_i}).
\]
The factor $(1+2^a)$ is odd.
Each $(1+p_i^{\alpha_i})$ is even, hence contributes at least one factor of $2$.
Therefore
\[
  v_2(\sigma^*(n))\ge r.
\]
But $\sigma^*(n)=2n$ implies $v_2(\sigma^*(n))=v_2(2n)=a+1$.
Hence $r\le a+1$.
$\square$

5) VERIFICATION

- Lemma 4.1: checked by direct prime-exponent analysis; no missing cases.
- Lemma 4.3: the key step is comparing 2-adic valuations, which is exact and elementary.
- Computation: sieve-based factorization up to $10^7$ found exactly the four known small examples.
  (This is a sanity check, not a proof of finiteness.)

6) FINAL

**UNRESOLVED**

(i) Strongest proved partial result.
- Structural formula $\sigma^*(n)=\prod (1+p^\alpha)$ and the equivalent product equation
  $\prod (1+1/p^\alpha)=2$.
- No odd unitary perfect numbers (Lemma 4.3).
- Exhaustive search up to $10^7$ finds only $6,60,90,87360$.

(ii) First gap (crisp statement).
Prove that the Diophantine equation
\[
  \prod_{i=1}^r \left(1+\frac{1}{p_i^{\alpha_i}}\right)=2
\]
has only finitely many solutions in distinct primes $p_i$ and exponents $\alpha_i\ge 1$.

(iii) Top 3 next moves.
1. Use the inequality $\log(1+x)\le x$ to show
   $\sum 1/p_i^{\alpha_i} \ge \log 2$, then combine with lower bounds on $p_i^{\alpha_i}$
   to constrain the set of possible prime powers.
2. Improve Lemma 4.4 by controlling higher 2-adic valuations of $1+p^\alpha$ for odd primes,
   potentially forcing stronger restrictions on the exponents and the number of odd primes.
3. Implement a branch-and-bound search using the product equation (Corollary 4.2) to enumerate all
   possible prime-power combinations below a given size bound, potentially extending verified search
   far beyond $10^7$.

(iv) Minimal counterexample structure.
If infinitely many unitary perfect numbers exist, there must be an infinite family with growing sets
of prime powers $p_i^{\alpha_i}$ such that the product $\prod(1+1/p_i^{\alpha_i})$ stays exactly $2$.
This would require a delicate balance between introducing new small prime-power reciprocals and
adjusting exponents so as not to overshoot; one would expect many large prime powers with very small
reciprocals, and tight 2-adic constraints from Lemma 4.4.

