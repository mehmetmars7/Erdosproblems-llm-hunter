% Erdős Problem #91
% URL: https://www.erdosproblems.com/91

\noindent\textbf{FORMAL RESTATEMENT.}
For $n\ge 2$ define
\[
g(n):=\min\{D(A): A\subset\mathbb{R}^2,\ |A|=n\},
\]
where $D(A)$ is the number of distinct Euclidean distances determined by unordered pairs from $A$.
Call two finite point sets $A,B\subset\mathbb{R}^2$ \emph{similar} if $B$ is obtained from $A$ by a similarity transformation (composition of a rigid motion and a uniform scaling; optionally allowing reflection---either choice does not affect the arguments below).
Let $\mathcal{M}_n$ be the set of similarity classes of minimisers:
\[
\mathcal{M}_n:=\{[A]_{\sim}: |A|=n,\ D(A)=g(n)\}.
\]
Question: prove that for all sufficiently large $n$ there are at least two non-similar minimisers, i.e.
\[
|\mathcal{M}_n|\ge 2\qquad\text{for all large }n.
\]

\medskip
\noindent\textbf{QUICK LITERATURE/CONTEXT CHECK.}
The problem statement gives small-$n$ facts: for $n=3$ the equilateral triangle is the unique minimiser; for $n=4$ there are at least two non-similar minimisers; for $n=5$ the regular pentagon is claimed to be the unique minimiser (with a cited proof). I do not use results beyond these statements; instead I verify the $n=4$ and the value $g(5)=2$ directly.

\medskip
\noindent\textbf{ATTACK PLAN.}
\begin{itemize}
\item \emph{Reality check:} verify the minimisation claims for $n=4,5$ directly by elementary arguments.
\item \emph{General strategy (unresolved):} attempt to build, for large $n$, two different-looking configurations $A,B$ with $D(A)=D(B)=g(n)$, e.g. by producing two distinct constructions achieving the same (conjecturally optimal) order of magnitude.
\end{itemize}
The general strategy step remains open; we record small-$n$ verified structure.

\medskip
\noindent\textbf{WORK.}

\smallskip
\noindent\textbf{Lemma 91.1 ($n=4$: two non-similar minimisers).}
We have $g(4)=2$, and there are at least two non-similar point sets $A\subset\mathbb{R}^2$ with $|A|=4$ and $D(A)=2$.
\textit{Proof.}
\emph{Step 1: $g(4)\ge 2$.} If $D(A)=1$, then all $\binom{4}{2}=6$ pairwise distances in $A$ are equal to some $r>0$. But in the plane, there is no set of 4 points that are pairwise equidistant: among any three of the points we get an equilateral triangle of side $r$; a fourth point at distance $r$ from all three would have to lie in the intersection of three circles of radius $r$ around the triangle's vertices. The first two circles intersect in at most two points, and neither of those points is at distance $r$ from the third vertex (the third circle), so the triple intersection is empty. Thus $D(A)\ne 1$ and $g(4)\ge 2$.

\emph{Step 2: two explicit examples with $D=2$.}
\begin{itemize}
\item \emph{Square.} Take the unit square with vertices $(0,0),(1,0),(1,1),(0,1)$. The pairwise distances are $1$ (edges) and $\sqrt2$ (diagonals), so $D=2$.
\item \emph{Two equilateral triangles sharing an edge.} Take points
$B=(0,0)$, $C=(1,0)$, $A=(\tfrac12,\tfrac{\sqrt3}{2})$, $D=(\tfrac12,-\tfrac{\sqrt3}{2})$.
Then $AB=AC=BC=1$ and $BD=CD=BC=1$, while $AD=\sqrt3$. Thus the distinct distances are $\{1,\sqrt3\}$, so again $D=2$.
\end{itemize}
\emph{Step 3: non-similarity.}
In the square example, the ratio of the larger distance to the smaller is $\sqrt2$, while in the two-triangles example the ratio is $\sqrt3$. Similarity transformations preserve all distance ratios, hence these two configurations are not similar.

Combining Steps 1--3 shows $g(4)=2$ and $|\mathcal{M}_4|\ge 2$.\hfill $\square$

\smallskip
\noindent\textbf{Lemma 91.2 ($n=5$: the minimum value is 2).}
We have $g(5)=2$.
\textit{Proof.}
As in Lemma 91.1, $g(5)\ge 2$ because 5 points cannot be pairwise equidistant in the plane.
It remains to exhibit 5 points with exactly 2 distinct distances.
Take a regular pentagon with side length $1$. In a regular pentagon, the pairwise distances are either the side length or the diagonal length, and there are no other distances among vertex pairs.
More concretely, on a circle of radius $R=\bigl(2\sin(\pi/5)\bigr)^{-1}$, let $p_k=(R\cos(2\pi k/5),R\sin(2\pi k/5))$ for $k=0,1,2,3,4$. Then consecutive vertices satisfy $|p_{k+1}-p_k|=1$ by the choice of $R$, while nonconsecutive pairs are all diagonals and have the same length (by rotational symmetry). Therefore $D(\{p_0,\dots,p_4\})=2$, so $g(5)\le 2$.
Thus $g(5)=2$.\hfill $\square$

\smallskip
\noindent\textbf{FAST REALITY CHECK (explicit distance computations).}
A direct computation of squared distances for the configurations above gives:
\begin{verbatim}
Square: squared distances = {1, 2}
Two equilateral triangles sharing an edge: squared distances = {1, 3}
Regular pentagon (side=1): distances = {1, 1.6180339...}
\end{verbatim}
so indeed each has exactly two distinct distances.

\medskip
\noindent\textbf{VERIFICATION.}
\begin{itemize}
\item The nonexistence of 4 (or 5) pairwise equidistant points in $\mathbb{R}^2$ follows from the fact that two circles intersect in at most two points.
\item The regular pentagon argument uses only rotational symmetry: all sides equal; all diagonals equal; no other pairs.
\item Non-similarity check is by the invariant distance ratio $\max\{|x-y|\}/\min\{|x-y|\}$.
\end{itemize}

\medskip
\noindent\textbf{FINAL.} \textbf{UNRESOLVED}
\begin{enumerate}
\item[(i)] \textbf{Strongest proved partial result.}
We verified $g(4)=2$ and exhibited two non-similar minimisers for $n=4$ (Lemma 91.1). We also proved $g(5)=2$ by exhibiting a regular pentagon with two distances (Lemma 91.2).
\item[(ii)] \textbf{First gap (crisp).}
For large $n$, prove existence of two non-similar minimisers: construct $A,B$ with $|A|=|B|=n$, $D(A)=D(B)=g(n)$, and $A\not\sim B$.
\item[(iii)] \textbf{Top 3 next moves.}
\begin{itemize}
\item For moderately small $n$ (say $6\le n\le 12$), run constrained numeric optimisation / discrete searches (e.g. restricting to small integer grids) to guess candidate minimisers and test for non-uniqueness.
\item Prove stability/rigidity results: if $A$ is a minimiser with near-grid structure, show that small deformations strictly increase $D(A)$, then look for a \emph{different} structured family with the same $D$.
\item Attempt a ``two constructions'' approach: produce two explicit infinite families (e.g. square grids vs. certain circular-arc constructions) with the same asymptotic $D$ and then prove that this asymptotic is optimal for $g(n)$.
\end{itemize}
\item[(iv)] \textbf{Likely structure of a minimal counterexample.}
A minimal counterexample would be an $n_0$ such that every minimiser for $g(n_0)$ is unique up to similarity, while for all smaller $n$ there are at least two classes. Such a unique minimiser would likely be extremely symmetric/rigid (analogous to the equilateral triangle for $n=3$), suggesting a highly constrained distance multiset.
\end{enumerate}


