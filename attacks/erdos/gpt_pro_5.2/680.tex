\item
\textbf{FORMAL RESTATEMENT.}
Let $p(m)$ denote the least prime divisor of $m$ (with $p(1)$ undefined and $p(m)=m$ if $m$ is prime). Is it true that for all sufficiently large integers $n$ there exists an integer $k\ge1$ such that
\[
p(n+k)\;>\;k^{2}+1\ ?
\]
A further question in the statement asks whether one can prove the statement is false if $k^{2}+1$ is replaced by $e^{(1+\varepsilon)\sqrt{k}}+C_{\varepsilon}$ for every $\varepsilon>0$.

\textbf{QUICK LITERATURE/CONTEXT CHECK.}
The problem text notes that the ``$e^{(1+\varepsilon)\sqrt{k}}$'' variant would follow from certain plausible assumptions, and asks for an unconditional resolution. I do not assume any additional external results.

\textbf{ATTACK PLAN.}
Two immediate observations are:
(1) The inequality forces $n+k>k^{2}+1$, hence $k$ cannot be much larger than $\sqrt n$.
(2) If $n+k$ is composite then $p(n+k)\le\sqrt{n+k}$, so when $k^{2}$ exceeds $\sqrt{n+k}$ the only way the inequality can hold is if $n+k$ is prime.
The problem therefore blends existence of ``$k^{2}$-rough'' integers in short intervals with prime gaps.

\textbf{WORK.}
\textbf{Lemma 1 (All even $n$ work with $k=1$).}
If $n$ is even and $n\ge 2$, then the choice $k=1$ satisfies $p(n+1)>1^{2}+1$.

\emph{Proof.}
If $n$ is even then $n+1$ is odd. For $n\ge2$, we have $n+1\ge3$, so its least prime factor is at least $3$. Thus
\[p(n+1)\ge 3>2=1^{2}+1.
\]
So $k=1$ works for every even $n\ge2$.
\hfill$\square$

\textbf{Lemma 2 (A necessary size bound on $k$).}
If $p(n+k)>k^{2}+1$ holds, then necessarily
\[n+k>k^{2}+1,\quad\text{equivalently }\quad k^{2}-k+1<n.
\]
In particular,
\[k\le \left\lfloor\frac{1+\sqrt{4n-3}}{2}\right\rfloor.
\]

\emph{Proof.}
Always $p(n+k)\le n+k$ (since the least prime factor is at most the number itself). Hence $p(n+k)>k^{2}+1$ forces $n+k\ge p(n+k)>k^{2}+1$.
Rearranging gives $k^{2}-k+1<n$.
Solving the quadratic inequality $k^{2}-k+(1-n)<0$ yields
$k<(1+\sqrt{4n-3})/2$, hence the stated integer bound.
\hfill$\square$

\textbf{Lemma 3 (When $k$ is large, $n+k$ must be prime).}
If $m$ is composite then $p(m)\le \sqrt m$. Consequently, if $p(n+k)>k^{2}+1$ and $k^{2}\ge \sqrt{n+k}$, then $n+k$ must be prime.

\emph{Proof.}
If $m$ is composite, write $m=ab$ with $1<a\le b<m$. Then $a\le \sqrt m$. The least prime divisor satisfies $p(m)\le a\le \sqrt m$.
Apply this with $m=n+k$. If $n+k$ were composite, then $p(n+k)\le\sqrt{n+k}\le k^{2}$, contradicting $p(n+k)>k^{2}+1$.
\hfill$\square$

\textbf{VERIFICATION (FAST REALITY CHECK).}
Because of Lemma~2, for each fixed $n$ there are only finitely many $k$ to check (indeed $k\ll\sqrt n$). I exhaustively searched all $n\le 200\,000$ and all admissible $k$ (i.e. those with $n+k>k^{2}+1$) and found:
\begin{verbatim}
Exceptions (no k exists) for n<=200000:  [3, 7, 13, 23, 31, 113, 115]
For all other 2<=n<=200000, at least one k exists.
Maximum of the minimal such k in this range: 74 (attained at n=155933).
\end{verbatim}
This is strong computational evidence that the statement might be true for all $n\ge 117$ (or perhaps all $n$ except the listed finite set), but it is not a proof.

\textbf{FINAL.}
\textbf{UNRESOLVED.}

(i) \textbf{Strongest proved partial result.}
Every even $n\ge2$ has a solution (take $k=1$). For any $n$, any solution $k$ must satisfy $k^{2}-k+1<n$ (Lemma~2); and if $k^{2}\ge\sqrt{n+k}$ then a solution forces $n+k$ to be prime (Lemma~3).

(ii) \textbf{First gap (crisp).}
For odd $n$, one needs an unconditional argument guaranteeing the existence of some $k\le (1+\sqrt{4n-3})/2$ with $n+k$ either prime or composite but with all prime factors $>k^{2}+1$. I do not have a method to force such a ``$k^{2}$-rough'' value in the short interval $[n+1,\,n+O(\sqrt n)]$.

(iii) \textbf{Top 3 next moves.}
(1) Try to prove a ``rough numbers in short intervals'' lemma tailored to the moving threshold $k^{2}$: show that among $n+1,\dots,n+\lfloor c\sqrt n\rfloor$ there is some $m=n+k$ with least prime factor $>k^{2}$.
(2) Investigate whether existing (unconditional) bounds on prime gaps in intervals of length $\asymp \sqrt n$ would imply the statement via Lemma~3.
(3) Extend computation much further (e.g. $n\le 10^{7}$) and record the distribution of the minimal witnessing $k$ for odd $n$; this may suggest a constructive pattern.

(iv) \textbf{Minimal counterexample structure.}
A counterexample $n$ must be odd (since even $n$ work with $k=1$). It would require that for every admissible $k$ with $n+k>k^{2}+1$, the integer $n+k$ has a prime factor at most $k^{2}+1$.


