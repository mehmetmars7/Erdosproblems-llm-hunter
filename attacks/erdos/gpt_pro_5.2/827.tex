% Erdos Problem #827

\subsection*{FORMAL RESTATEMENT}
\textbf{Ambiguity note.} The phrase ``points in general position'' is not defined in the excerpt. In planar combinatorial geometry it often means ``no three collinear''; sometimes, in circle-related problems, it also includes ``no four concyclic''. I treat both:
\begin{itemize}
\item \textbf{Literal (minimal) convention:} general position means no three points are collinear.
\item \textbf{Stronger convention:} no three collinear and no four points lie on a common circle.
\end{itemize}

Given $k\ge 3$, define $n_k$ to be the smallest integer $N$ (if it exists) such that every set of $N$ points in $\mathbb{R}^2$ in general position contains a subset $S$ of $k$ points with the property:
\begin{quote}
For all distinct triples $T\subset S$ with $|T|=3$, the circumradius of the circle through $T$ is different for different $T$.
\end{quote}
(Each triple is non-collinear under both conventions, so it determines a unique circumcircle and circumradius.)

The problem asks to determine $n_k$.

\subsection*{QUICK LITERATURE/CONTEXT CHECK}
The excerpt states: Erd\H{o}s asked whether $n_k$ exists, claimed an argument giving $n_k\le k+2\binom{k-1}{2}\binom{k-1}{3}$, but this was incorrect; Martinez--Rold\'an-Pensado gave a corrected argument proving $n_k\ll k^9$.

Below I prove exact $n_3$ and give explicit lower-bound constructions for $n_4$ by counterexamples, using exact computations of circumradii.

\subsection*{ATTACK PLAN}
\begin{itemize}
\item \textbf{Small $k$ exact values:} Determine $n_3$ directly. For $k=4$, try to decide $n_4$ by explicit counterexamples for $N=4,5$ and then look for a proof for $N=6$ (not achieved here).
\item \textbf{General $k$:} Without reproducing the Martinez--Rold\'an-Pensado argument, I will not claim a full existence proof; I only record the bound stated in the excerpt as context.
\end{itemize}

\subsection*{WORK}
\paragraph{FAST REALITY CHECK (circumradius computation formulas).}
For a non-collinear triangle with side lengths $a,b,c$ and area $\Delta$, its circumradius is
\[
R = \frac{abc}{4\Delta}.
\]
For explicit integer-coordinate examples, I compute $R^2$ exactly via
\[
R^2 = \frac{a^2 b^2 c^2}{16\Delta^2} = \frac{a^2 b^2 c^2}{4\,\operatorname{Area2}^2},
\]
where $\operatorname{Area2}=2\Delta$ is the doubled signed area.

\noindent\textbf{Lemma (Exact value of $n_3$).}
Under either convention of general position, $n_3=3$.

\noindent\textbf{Proof.}
We must find the minimal $N$ such that any $N$ points in general position contain $3$ points whose $\binom{3}{3}=1$ triple has "all circumradii distinct".

If $N=3$, any 3 points in general position are non-collinear, hence determine a unique circle with some radius $R$. Since there is only one triple, the set of circumradii has size 1 and is vacuously pairwise distinct. Thus every 3-point set in general position contains a good 3-subset (itself).

If $N<3$, no $k=3$ subset exists at all. Therefore the minimal such $N$ is $3$.
\hfill$\square$


\noindent\textbf{Lemma (A 4-point counterexample: $n_4>4$).}
Under the minimal convention (no three collinear), $n_4>4$.

\noindent\textbf{Proof.}
Take four points on a common circle, e.g.
\[
A=(1,0),\;B=(0,1),\;C=(-1,0),\;D=(0,-1).
\]
No three of these are collinear, so this set is in general position under the minimal convention.

Any triple among $\{A,B,C,D\}$ lies on the same circle $x^2+y^2=1$, so every triangle has circumradius $1$. Therefore, for the only 4-point subset (the full set), the four triples determine circles of the same radius, hence not all radii are distinct. Thus $N=4$ does not guarantee a good 4-subset, so $n_4>4$.
\hfill$\square$


\noindent\textbf{Lemma (A stronger 5-point counterexample: $n_4\ge 6$ even with no four concyclic).}
Consider the five points
\[
P_1=(-2,-2),\;P_2=(-2,-1),\;P_3=(-1,1),\;P_4=(-1,2),\;P_5=(0,0).
\]
Then:
\begin{enumerate}
\item No three of these points are collinear.
\item No four of these points are concyclic (equivalently: for each 4-subset, not all four triangle circumradii coincide).
\item Every 4-subset of $\{P_1,\dots,P_5\}$ contains two distinct triangles with the same circumradius.
\end{enumerate}
Consequently, $n_4\ge 6$ under either convention of general position.

\noindent\textbf{Proof.}
\textbf{Step 1: no three collinear.} The set has two points with $x=-2$ ($P_1,P_2$), two points with $x=-1$ ($P_3,P_4$), and one point with $x=0$ ($P_5$), so no vertical line contains three points. A direct check of the remaining potential collinearities can be done by verifying that for each triple $(P_i,P_j,P_k)$ the doubled area
\[
\operatorname{Area2}(P_i,P_j,P_k)=(x_j-x_i)(y_k-y_i)-(y_j-y_i)(x_k-x_i)
\]
is nonzero. In fact, a direct computation gives the 10 doubled areas:
\[
\operatorname{Area2}(P_1,P_2,P_3)=-1,\;\operatorname{Area2}(P_1,P_2,P_4)=-1,\;\operatorname{Area2}(P_1,P_2,P_5)=-2,\;\operatorname{Area2}(P_1,P_3,P_4)=1,\;\operatorname{Area2}(P_1,P_3,P_5)=-4,\;\operatorname{Area2}(P_1,P_4,P_5)=-6,\;\operatorname{Area2}(P_2,P_3,P_4)=1,\;\operatorname{Area2}(P_2,P_3,P_5)=-3,\;\operatorname{Area2}(P_2,P_4,P_5)=-5,\;\operatorname{Area2}(P_3,P_4,P_5)=-1.
\]
None of these is $0$, so no three points are collinear.

\textbf{Step 2: compute circumradii squared for each 4-subset.} For each 4-subset, there are 4 triangles. I list the four values of $R^2$ (as exact rationals) computed via $R^2=\frac{a^2b^2c^2}{4\,\operatorname{Area2}^2}$.

\smallskip
\noindent(a) Quadruple $\{P_1,P_2,P_3,P_4\}$:
\[
R^2\in\Bigl\{\frac{25}{2},\frac{85}{2},\frac{85}{2},\frac{25}{2}\Bigr\}.
\]
Thus at least two circumradii coincide (in fact two pairs coincide), and not all four coincide.

\smallskip
\noindent(b) Quadruple $\{P_1,P_2,P_3,P_5\}$:
\[
R^2\in\Bigl\{\frac{25}{2},\frac{5}{2},\frac{5}{2},\frac{25}{18}\Bigr\}.
\]
Thus there is a repeated value $5/2$, and not all four coincide.

\smallskip
\noindent(c) Quadruple $\{P_1,P_2,P_4,P_5\}$:
\[
R^2\in\Bigl\{\frac{85}{2},\frac{5}{2},\frac{85}{18},\frac{5}{2}\Bigr\}.
\]
Thus there is a repeated value $5/2$, and not all four coincide.

\smallskip
\noindent(d) Quadruple $\{P_1,P_3,P_4,P_5\}$:
\[
R^2\in\Bigl\{\frac{85}{2},\frac{5}{2},\frac{85}{18},\frac{5}{2}\Bigr\}.
\]
Again there is a repeated value $5/2$, and not all four coincide.

\smallskip
\noindent(e) Quadruple $\{P_2,P_3,P_4,P_5\}$:
\[
R^2\in\Bigl\{\frac{25}{2},\frac{25}{18},\frac{5}{2},\frac{5}{2}\Bigr\}.
\]
Again there is a repeated value $5/2$, and not all four coincide.

In each case, the list of four radii contains a repeat, so no 4-subset has all four circumradii distinct.

\textbf{Step 3: conclude.} Since this 5-point configuration is in general position under either convention (Steps 1 and the "not all equal" part of Step 2), but contains no good 4-subset, it follows that $n_4\ge 6$.
\hfill$\square$


\subsection*{VERIFICATION}
\begin{itemize}
\item Lemma (Exact value of $n_3$): The vacuity for $k=3$ is correct because there is exactly one triple.
\item Lemma (A 4-point counterexample: $n_4>4$): The 4 points are concyclic and non-collinear in triples, so it is a valid counterexample under the minimal convention. Under the stronger convention (excluding 4 concyclic) this example is not allowed, which is why the 5-point counterexample lemma was given.
\item Lemma (A stronger 5-point counterexample: $n_4\ge 6$ even with no four concyclic): The radii-squared lists were computed exactly (rational arithmetic) for each 4-subset; each list contains a duplicate, so each 4-subset fails the distinctness requirement. Also no list is constant, so no 4-subset is concyclic.
\end{itemize}

\subsection*{FINAL}
\textbf{UNRESOLVED.}
\begin{enumerate}
\item[(i)] \textbf{Strongest proved partial result.} Exact value $n_3=3$; and an explicit 5-point configuration in strong general position with no good 4-subset, implying $n_4\ge 6$.
\item[(ii)] \textbf{First gap.} Decide whether $n_4=6$ (i.e. prove or disprove that every 6-point set in general position contains a 4-point subset with all four triangle circumradii distinct), and more generally determine $n_k$ for $k\ge 4$.
\item[(iii)] \textbf{Top 3 next moves.}
  \begin{enumerate}
  \item Computational search over integer-coordinate point sets with $N=6$ to try to find a counterexample to $n_4\le 6$ (i.e. a 6-point set with no good quadruple), which would show $n_4\ge 7$.
  \item Prove a structural lemma: if in a 6-point set every quadruple has two equal circumradii, then some four points must be concyclic (contradicting strong general position). This would establish $n_4=6$ under the stronger convention.
  \item For general $k$, model the problem as an anti-Ramsey problem in a 3-uniform hypergraph where triples are colored by circumradius; seek a general anti-Ramsey bound tailored to the geometric constraints of circumradii.
  \end{enumerate}
\item[(iv)] \textbf{Minimal counterexample structure (if small $n_4$ conjectures fail).} A minimal counterexample to $n_4=6$ would be 6 points in (strong) general position such that \emph{every} 4-subset has a repeated circumradius among its 4 triangles. The 5-point example above suggests a counterexample may feature near-parallel pairs and repeated local distance patterns causing two triangles to share the same $R^2$ value (e.g. repeated rational values like $5/2$).
\end{enumerate}

