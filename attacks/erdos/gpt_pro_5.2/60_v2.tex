\section{Erd\H{o}s Problem \#60 (Round 2): supersaturation just above $\mathrm{ex}(n,C_4)$}

\subsection*{1) ROUND-2 OBJECTIVE}
\textbf{Path (A): attempt at a full proof, by closing the quantitative gap left in Round 1.}

Round 1 produced a robust global lower bound on $\#C_4(G)$ from degree/codegree convexity, but it only forces many $4$-cycles when $e(G)$ exceeds $\tfrac12 n^{3/2}$ by a \emph{constant proportion}. The conjecture asks for the far more delicate regime
\[
 e(G)=\mathrm{ex}(n,C_4)+1 \qquad\Rightarrow\qquad \#C_4(G)=\Omega(\sqrt n).
\]
In this round I introduce a sharper (``linearized'') inequality relating $\#C_4(G)$ to the surplus of $2$-paths, and I push the unconditional supersaturation range substantially closer to the extremal threshold (though still not down to the $+1$ regime).

\subsection*{2) ROUND-1 FOUNDATION USED}
I rely on the following Round-1 results exactly as stated there.
\begin{itemize}
\item \textbf{(R1--Lemma 1)} Codegree formula:
\[\#C_4(G)=\frac12\sum_{\{u,v\}} \binom{c(u,v)}{2},\]
where $c(u,v)=|N(u)\cap N(v)|$ and the sum is over unordered vertex-pairs.
\item \textbf{(R1--Lemma 2)} Cauchy--Schwarz lower bound in terms of
\[S:=\sum_{v}\binom{d(v)}{2}=\sum_{\{u,v\}}c(u,v)\qquad\text{and}\qquad N:=\binom{n}{2}:\]
\[\#C_4(G)\ge \frac14\Big(\frac{S^2}{N}-S\Big).\]
\item \textbf{(R1--Lemma 3)} Jensen/convexity lower bound on $S$ from $m:=e(G)$:
\[S\ge \frac{2m^2}{n}-m.\]
\item \textbf{(R1--Corollary 4)} Consequence: if $m\ge (\tfrac12+\varepsilon)n^{3/2}$ then $\#C_4(G)=\Omega_\varepsilon(n^2)$.
\item \textbf{(R1--Computation)} Exact small-$n$ values: $\mathrm{ex}(7,C_4)=9$ and every $7$-vertex graph with $10$ edges has at least one $C_4$; similarly $\mathrm{ex}(6,C_4)=6$ and there is a $7$-edge example with exactly one $C_4$.
\end{itemize}

\subsection*{3) NEW INSIGHT / TOOL (ROUND-2)}
\begin{enumerate}
\item A new lower bound that is \emph{linear} (rather than quadratic) in the ``$2$-path surplus'':
\[\#C_4(G)\ \ge\ \frac12\Big(M-\binom{n}{2}\Big),\]
where $M:=\sum_v\binom{d(v)}{2}$ is the number of (unordered) length-$2$ paths.
This bound appears (implicitly) in recent work and is the key to improved supersaturation when the edge surplus $t$ is subpolynomially small compared to $n^{3/2}$.
\item New input from the literature beyond Round 1:
\begin{itemize}
\item Qiao proved that for each $6\le n\le 11$ there exists a graph with $\mathrm{ex}(n,C_4)+1$ edges and \emph{exactly one} $C_4$, while for $n\in\{12,13\}$ the minimum is $2$ copies \cite[Theorem~5]{Qiao20}.
\item He--Ma--Yang proved the Erd\H{o}s--Simonovits conjecture for the infinite family $n=q^2+q+1$ with even $q$, and also proved strong supersaturation for larger edge surpluses $t$ (in particular, $\#C_4(G)\ge \tfrac12 t\sqrt n$ for $t\ge 3n^{1.2625}$) \cite[Proposition~1.10]{HeMaYa19}.
\end{itemize}
\end{enumerate}

\subsection*{4) ATTACK PLAN (ROUND-2)}
\textbf{Round-1 gap.} The Round-1 inequality $\#C_4\ge \tfrac14(S^2/N-S)$ is only useful when $S$ is at least a constant factor larger than $N$, which does not occur near the extremal threshold $m\approx \tfrac12 n^{3/2}$.

\textbf{New plan.}
\begin{itemize}
\item Prove a sharper inequality $\#C_4\ge \tfrac12(M-\binom{n}{2})$ in terms of the number $M$ of $2$-paths.
\item Lower bound $M$ from $m$ by Jensen (Round-1 Lemma~3) and relate the relevant ``surplus parameter'' to $t=m-\mathrm{ex}(n,C_4)$ using known lower bounds on $\mathrm{ex}(n,C_4)$.
\item This yields an unconditional bound $\#C_4(G)\ge \Omega(t\sqrt n)$ for $t$ as small as $n^{1.2625}$ (a major strengthening over Round 1), while isolating the remaining obstruction for $t=1$.
\end{itemize}

\subsection*{5) WORK (ROUND-2)}

\subsubsection*{5.1 A linearized codegree lower bound}
\paragraph{Lemma 5.1 (linear $2$-path surplus bound).}\label{lem:linear_surplus}
Let $G$ be a simple graph on $n$ vertices. Let
\[M:=\sum_{v\in V(G)}\binom{d(v)}{2}=\sum_{\{u,v\}} c(u,v)
\qquad\text{and}\qquad N:=\binom{n}{2}.
\]
Then
\begin{equation}
\label{eq:linear_surplus}
\#C_4(G)\ \ge\ \frac12\,\max\Big\{0,\,M-N\Big\}.
\end{equation}

\paragraph{Proof.}
By (R1--Lemma~1),
\[\#C_4(G)=\frac12\sum_{\{u,v\}} \binom{c(u,v)}{2}.\]
For any integer $k\ge 0$ we have
\[\binom{k}{2}=\frac{k(k-1)}{2}\ \ge\ \max\{0,k-1\}.\]
Therefore
\[
\sum_{\{u,v\}}\binom{c(u,v)}{2}\ \ge\ \sum_{\{u,v\}}\max\{0,c(u,v)-1\}
=\sum_{\{u,v\}:\,c(u,v)\ge 1}(c(u,v)-1).
\]
Let $N_1:=|\{\{u,v\}:c(u,v)\ge 1\}|$. Then
\[
\sum_{\{u,v\}:\,c(u,v)\ge 1}(c(u,v)-1)=\sum_{\{u,v\}}c(u,v)-N_1=M-N_1\ge M-N,
\]
because $N_1\le N$. Combining this with (R1--Lemma~1) yields
\[\#C_4(G)\ge \frac12(M-N).
\]
If $M<N$ then the right-hand side is negative, so we may replace it by $\frac12\max\{0,M-N\}$ without weakening the inequality. \hfill$\square$

\subsubsection*{5.2 Reiman-type bound recovered (sanity check)}
For a $C_4$-free graph, every pair of vertices has codegree $\le 1$ (else two common neighbors would form a $4$-cycle), hence $M=\sum_{\{u,v\}}c(u,v)\le N$. This is consistent with Lemma~\ref{lem:linear_surplus}.

Combining $M\le N$ with Jensen (R1--Lemma~3), i.e.
\[M\ge \frac{2m^2}{n}-m,\]
one gets the classical quadratic upper bound
\begin{equation}
\label{eq:reiman}
\frac{2m^2}{n}-m\le \binom{n}{2}
\qquad\Longrightarrow\qquad
m\le \frac{n}{4}\bigl(1+\sqrt{4n-3}\bigr),
\end{equation}
which is a standard general upper bound on $\mathrm{ex}(n,C_4)$ (often attributed to Reiman).

\subsubsection*{5.3 Improved supersaturation for moderately large surplus $t$}
Round 1 only forced many $C_4$ when $t=\Omega(n^{3/2})$. Lemma~\ref{lem:linear_surplus} enables a much smaller $t$ once we know a sufficiently good general lower bound on $\mathrm{ex}(n,C_4)$.

\paragraph{Proposition 5.2 (supersaturation for $t\ge 3n^{1.2625}$).}
There exists $n_0$ such that for all $n\ge n_0$ and all real $t\ge 3n^{1.2625}$,
\[
 e(G)\ge \mathrm{ex}(n,C_4)+t\qquad\Longrightarrow\qquad \#C_4(G)\ge \frac{t\sqrt n}{2}.
\]

\paragraph{Proof (following \cite[Proposition~1.10]{HeMaYa19}).}
Let $m:=e(G)=\mathrm{ex}(n,C_4)+t$.
Write
\begin{equation}
\label{eq:def_s}
 m = \frac12\bigl(n^{3/2}+n\bigr)+s
\qquad\text{for some real }s.
\end{equation}
He--Ma--Yang show that for all sufficiently large $n$ one has
\begin{equation}
\label{eq:ex_lower}
\mathrm{ex}(n,C_4)\ \ge\ \frac12\bigl(n^{3/2}-3n^{1.2625}+n\bigr)
\qquad\text{(see \cite[Proposition~7.3]{HeMaYa19}).}
\end{equation}
Subtracting \eqref{eq:ex_lower} from \eqref{eq:def_s} yields
\[s\ge t-\frac32 n^{1.2625} \ge \frac{t}{2}\qquad\text{since }t\ge 3n^{1.2625}.
\]
Next, Jensen gives a lower bound on the number $M$ of $2$-paths:
\[
M=\sum_{v}\binom{d(v)}{2}\ \ge\ n\binom{\frac{2m}{n}}{2}.
\]
Using \eqref{eq:def_s}, the average degree is
\[\frac{2m}{n}=\frac{n^{3/2}+n+2s}{n}=\sqrt n+1+\frac{2s}{n}.
\]
A direct expansion gives
\[
 n\binom{\sqrt n+1+\frac{2s}{n}}{2}
=\binom{n}{2}+2s\sqrt n+\frac{2s^2}{n}\; +\; \frac12\bigl(n^{3/2}+n\bigr)+s
\ \ge\ \binom{n}{2}+2s\sqrt n+\frac{2s^2}{n},
\]
so
\begin{equation}
\label{eq:M_lower}
M\ \ge\ \binom{n}{2}+2s\sqrt n+\frac{2s^2}{n}.
\end{equation}
Applying Lemma~\ref{lem:linear_surplus} gives
\[
\#C_4(G)\ \ge\ \frac12\Big(M-\binom{n}{2}\Big)\ \ge\ s\sqrt n+\frac{s^2}{n}\ \ge\ \frac{t\sqrt n}{2},
\]
using $s\ge t/2$ and $s^2/n\ge 0$. \hfill$\square$

\paragraph{Remark 5.3.}
Proposition~5.2 strictly strengthens Round 1 by lowering the needed edge surplus from $\Theta(n^{3/2})$ to $\Theta(n^{1.2625})$.
However, it is still far from the conjectured $t=1$ case.

\subsubsection*{5.4 Updated small-$n$ information for $t=1$}
Define
\[h(n,1):=\min\{\#C_4(G): |V(G)|=n,\ e(G)=\mathrm{ex}(n,C_4)+1\}.\]
Round 1 verified $h(6,1)=h(7,1)=1$ and $h(5,1)=2$.
Qiao proved the following extension.

\paragraph{Theorem 5.4 (Qiao \cite[Theorem~5]{Qiao20}).}
For each $6\le n\le 11$ there exists an $n$-vertex graph with $\mathrm{ex}(n,C_4)+1$ edges and \emph{exactly one} copy of $C_4$; for $n\in\{12,13\}$ every graph with $\mathrm{ex}(n,C_4)+1$ edges has at least two $C_4$'s.

Thus the exact values of $h(n,1)$ currently known for $5\le n\le 13$ are:
\[
\begin{array}{c|ccccccccc}
 n & 5&6&7&8&9&10&11&12&13\\\hline
 h(n,1)&2&1&1&1&1&1&1&2&2
\end{array}
\]

\subsubsection*{5.5 What still blocks the conjecture $h(n,1)=\Omega(\sqrt n)$}
Lemma~\ref{lem:linear_surplus} shows that to prove $h(n,1)\ge c\sqrt n$ it would suffice to prove a lower bound of the form
\[
M\ge \binom{n}{2}+\Omega(\sqrt n)
\qquad\text{whenever }e(G)\ge \mathrm{ex}(n,C_4)+1.
\]
Equivalently, one needs to force an \emph{excess of $2$-paths over unordered pairs} of order $\sqrt n$ at the $+1$ edge level.

Current unconditional approaches (including Round-1 convexity and the prime-gap based estimate \eqref{eq:ex_lower}) only force such an excess once the edge surplus $t$ dominates the uncertainty in the best known general lower bounds on $\mathrm{ex}(n,C_4)$, which is currently on the order $n^{1.2625}$ in \cite{HeMaYa19}.

A conceptually different input seems necessary for $t=1$:
\begin{itemize}
\item either a global structural/stability theorem describing extremal and near-extremal $C_4$-free graphs for \emph{all} $n$ (beyond the special family $n=q^2+q+1$),
\item or a new local counting principle showing that any one-edge extension of an extremal $C_4$-free configuration necessarily creates $\Omega(\sqrt n)$ many $C_4$'s.
\end{itemize}

\subsection*{6) ADVERSARIAL VERIFICATION}
\begin{itemize}
\item \textbf{Lemma~\ref{lem:linear_surplus} edge cases.} If $M\le N$, the bound gives $\#C_4\ge 0$, which is tautologically true. If $M>N$, the proof reduces to the inequality $\binom{k}{2}\ge k-1$ for $k\ge 1$ and the trivial bound $N_1\le N$; there are no hidden regularity assumptions.
\item \textbf{Consistency with $C_4$-free graphs.} If $G$ is $C_4$-free then all codegrees are $\le 1$, so $\sum \binom{c(u,v)}{2}=0$ and also $M\le N$. Hence Lemma~\ref{lem:linear_surplus} does not contradict extremality.
\item \textbf{Proposition~5.2 dependence.} The only nontrivial external input is the lower bound \eqref{eq:ex_lower} on $\mathrm{ex}(n,C_4)$ for general $n$, sourced to \cite{HeMaYa19}. The remaining steps are algebra plus Jensen plus Lemma~\ref{lem:linear_surplus}.
\item \textbf{Quantifier check.} The conjecture concerns (eventually all) $n$ and $t=1$; Proposition~5.2 only applies for $t\ge 3n^{1.2625}$, so it does not settle the target statement.
\end{itemize}

\subsection*{7) FINAL}
\textbf{UNRESOLVED (BUT STRICTLY ADVANCED).}

Round 2 adds:
(i) the sharper lower bound \eqref{eq:linear_surplus};
(ii) a substantially improved unconditional supersaturation theorem for edge surplus $t\ge 3n^{1.2625}$ (yielding $\#C_4\ge \tfrac12 t\sqrt n$);
and (iii) updated exact small-$n$ values of $h(n,1)$ through $n=13$ from \cite{Qiao20}.
The main conjecture $h(n,1)=\Omega(\sqrt n)$ remains open in general.

\subsection*{8) COMPLETION ESTIMATE (MANDATORY)}
COMPLETION: 55\%

\subsection*{9) REFERENCES}
\begin{thebibliography}{99}
\bibitem{BloomErdos60}
T.~F.~Bloom,
\emph{Erd\H{o}s Problem \#60}, Erd\H{o}s problems website (problem page last edited 18 Nov 2025).

\bibitem{HeMaYa19}
Z.~He, J.~Ma, and T.~Yang,
\emph{Stability and supersaturation for the 4-cycle}, arXiv:1912.00986.

\bibitem{Qiao20}
P.~Qiao,
\emph{On a problem of Erd\H{o}s about graphs whose size is the Tur\'an number plus one},
arXiv:2001.11723.
\end{thebibliography}
