
\noindent\textbf{FORMAL RESTATEMENT.}
For a finite set $A\subseteq\mathbb N$, define
\[
S(A):=\Bigl\{\frac{a}{\gcd(a,b)}: a,b\in A\Bigr\}.
\]
Define $h(n)$ to be the largest function such that for every $A\subseteq\mathbb N$ with $|A|=n$ we have $|S(A)|\ge h(n)$.
Equivalently,
\[
 h(n)=\min_{A\subseteq\mathbb N,\ |A|=n} |S(A)|.
\]
Estimate the growth of $h(n)$.

\noindent\textbf{QUICK LITERATURE/CONTEXT CHECK.}
The problem statement reports that Erd\H{o}s--Szemer\'edi proved
$n^{1/2}\ll h(n)\ll n^{1-c}$ for some $c>0$.
Below I give a self-contained proof of the elementary lower bound $h(n)\ge \lceil\sqrt n\rceil$ and the trivial upper bound $h(n)\le n$.

\noindent\textbf{ATTACK PLAN.}
1) Find an intrinsic reformulation of $S(A)$ in terms of reduced fractions.
2) Prove universal lower bounds by injecting a large set (e.g. the ratio set) into $S(A)\times S(A)$.
3) Give explicit constructions for upper bounds and compute small $n$ by brute force.

\noindent\textbf{WORK.}
\textbf{Fast reality check (small $n$ search).}
I searched all $n$-element subsets $A\subseteq\{1,2,\dots,20\}$ for $n\le 8$ and computed the minimum size of $S(A)$ among those.
The best values found (as upper bounds on $h(n)$) were:
\[
\begin{array}{c|cccccccc}
 n&1&2&3&4&5&6&7&8\\\hline
 \min_{A\subseteq[20],|A|=n} |S(A)| & 1&2&3&4&5&6&6&7
\end{array}
\]
One achieving set for $n=7$ is $A=\{2,3,4,6,9,12,18\}$, where $|S(A)|=6$.

\medskip
\textbf{Lemma 539.1 (ratio-set injection gives $|S(A)|\ge \sqrt{|A|}$).}
Let $A\subseteq\mathbb N$ be finite with $|A|=n$.
Let
\[
R(A):=\Bigl\{\frac{a}{b}: a,b\in A\Bigr\}\subseteq\mathbb Q.
\]
Then $|R(A)|\le |S(A)|^2$.
In particular $|S(A)|\ge \lceil\sqrt n\rceil$, hence $h(n)\ge \lceil\sqrt n\rceil$.

\textbf{Proof.}
Fix $a,b\in A$ and let $g=\gcd(a,b)$. Then
\[
\frac{a}{b}=\frac{a/g}{b/g}.
\]
The numerator $a/g$ belongs to $S(A)$ by definition.
The denominator $b/g$ also belongs to $S(A)$ because
\(
 b/g = b/\gcd(b,a)
\)
and $a\in A$.
Moreover, the fraction $(a/g)/(b/g)$ is in lowest terms because $\gcd(a/g,b/g)=1$.
Thus every ratio $a/b\in R(A)$ has a unique reduced representation $x/y$ with $x,y\in S(A)$ and $\gcd(x,y)=1$.
Therefore the map $R(A)\to S(A)\times S(A)$ sending $a/b$ to its reduced pair $(x,y)$ is injective, and
$|R(A)|\le |S(A)|^2$.

Finally, fix any $b_0\in A$. The map $a\mapsto a/b_0$ is injective as a map $A\to\mathbb Q$, so $|R(A)|\ge n$.
Combining, $n\le |R(A)|\le |S(A)|^2$, hence $|S(A)|\ge \sqrt n$ and the stated ceiling bound follows. \qed

\medskip
\textbf{Lemma 539.2 (trivial upper bound $h(n)\le n$).}
For every $n\ge 1$, $h(n)\le n$.

\textbf{Proof.}
Take $A=\{1,2,\dots,n\}$. For any $a,b\in A$, $\gcd(a,b)\ge 1$, so
\(
 a/\gcd(a,b)\le a\le n.
\)
Thus $S(A)\subseteq\{1,2,\dots,n\}$.
On the other hand, for each $a\in A$ we have $a/\gcd(a,1)=a$, so $\{1,2,\dots,n\}\subseteq S(A)$.
Therefore $|S(A)|=n$, showing $h(n)\le n$. \qed

\noindent\textbf{VERIFICATION.}
Lemma 539.1 hinges on uniqueness of reduced fraction representation: if $a/b=x/y$ in lowest terms then $(x,y)$ is unique. Since both numerator and denominator come from the same construction $t/\gcd(t,\cdot)$, both lie in $S(A)$.

\noindent\textbf{FINAL.}\;\textbf{UNRESOLVED.}
\begin{itemize}
\item[(i)] \textbf{Strongest proved partial result.} For all $n$,
\[\lceil\sqrt n\rceil\le h(n)\le n.\]
Also, brute force in $[1,20]$ shows $h(7)\le 6$ and $h(8)\le 7$ (upper bounds).
\item[(ii)] \textbf{First gap (crisp statement).} Improve either side toward the reported nontrivial upper bound $h(n)\ll n^{1-c}$ or sharpen the lower bound beyond $\sqrt n$, without invoking results outside the provided problem statement.
\item[(iii)] \textbf{Top 3 next moves.}
  \begin{itemize}
  \item Seek explicit structured families $A$ (e.g. multiplicative grids, divisor sets with constraints) for which $|S(A)|$ provably grows like $n^\theta$ for some $\theta<1$.
  \item Strengthen Lemma 539.1 by lower bounding $|R(A)|$ by more than $n$ in terms of $n$ for every $A$ (a sum-product style statement), which would improve the $\sqrt n$ bound.
  \item Computationally: perform heuristic search for $n=9,10,\dots$ with larger ambient range to guess the true minimal $|S(A)|$ and infer candidate constructions.
  \end{itemize}
\item[(iv)] \textbf{Minimal counterexample structure to look for.} A set $A$ with very small $|S(A)|$ must have a very small ratio set $R(A)$ as well (since $|R(A)|\le |S(A)|^2$). Thus any extremal construction should make many ratios $a/b$ coincide after reduction, meaning heavy multiplicative structure and repeated gcd patterns across many pairs.
\end{itemize}


