% Erdos Problem #1013

1) FORMAL RESTATEMENT

For integer k>=1, define h_3(k) to be the smallest integer n such that there exists a triangle-free graph G on n vertices with chromatic number chi(G)=k.
(If one instead defines chi(G)>=k, the minimum n is the same, because any graph with chi>=k contains a k-critical induced subgraph with chi exactly k, and deleting vertices preserves triangle-freeness.)

Tasks:
(A) Find an asymptotic for h_3(k) as k->infinity.
(B) Prove that
  lim_{k->infinity} h_3(k+1) / h_3(k) = 1.

2) QUICK LITERATURE/CONTEXT CHECK

Only what is explicitly stated in the provided problem file:
- Known bounds:
    (log k / log log k) * k^2 << h_3(k) << (log k) * k^2.
- The lower bound is attributed in the problem text to Graver--Yackel.
- The upper bound is said to follow from Shearer's upper bound for the Ramsey number R(3,k).

3) ATTACK PLAN

Proof-track ideas:
- Use standard tools relating chromatic number to independence number: chi >= n/alpha.
  For triangle-free graphs, alpha is constrained by Ramsey theory and by degree-based bounds.
- Use explicit constructions to give upper bounds on h_3(k), e.g. Mycielski's construction (triangle-free, raises chromatic number by 1).
- For the ratio limit, look for near-multiplicativity/subadditivity properties of h_3(k).

Disproof-track ideas:
- Not applicable (this is an asymptotic/existence determination problem).

I provide fully proved basic lower and upper recurrences (very weak vs the quoted bounds), and verify small cases by explicit construction.

4) WORK

FAST REALITY CHECK (small k, explicit graphs)

Exact values / verified constructions:
- h_3(1)=1: the 1-vertex empty graph is triangle-free and 1-colorable.
- h_3(2)=2: a single edge K2 is triangle-free and has chi=2.
- h_3(3)=5: C5 (cycle on 5 vertices) is triangle-free and has chi=3; no triangle-free graph on <=4 vertices has an odd cycle, hence all are bipartite, so chi<=2.

Mycielski constructions (verified by brute-force exact coloring for these sizes):
- Starting from C5 (n=5, chi=3), one Mycielski step produces a triangle-free graph on 11 vertices with chi=4.
- A second step produces a triangle-free graph on 23 vertices with chi=5.
So h_3(4) <= 11 and h_3(5) <= 23 (upper bounds, not claimed optimal).

Lemma 1013.1 (k-critical subgraph and minimum degree).
Let G be a graph with chi(G)=k>=2.
Then G contains a subgraph H with chi(H)=k such that every proper subgraph of H has chromatic number <= k-1 (a k-critical subgraph).
Moreover, any k-critical graph H satisfies minimum degree delta(H) >= k-1.

Proof.
Existence of a k-critical subgraph: starting from G, iteratively delete vertices v such that chi(G-v)=k (i.e. deleting v does not drop the chromatic number). This process must terminate because the number of vertices decreases at each step. Let H be the resulting induced subgraph. By construction chi(H)=k, and for every vertex v in H, chi(H-v) <= k-1, so H is k-critical.

Minimum degree: let H be k-critical. Suppose for contradiction that some vertex v has degree at most k-2.
Since H-v is (k-1)-colorable (by criticality), fix a proper coloring of H-v with colors {1,...,k-1}.
Vertex v has at most k-2 neighbors, so among the k-1 colors there is at least one color not used on the neighbors of v. Assign that color to v; this extends the (k-1)-coloring to all of H, contradicting chi(H)=k.
Therefore every vertex has degree at least k-1, i.e. delta(H) >= k-1.  QED.

Lemma 1013.2 (a self-contained Mantel bound, used to lower bound h_3(k)).
If G is triangle-free on n vertices with e edges, then e <= n^2/4.

Proof.
Let d(v) be the degree of v. Then sum_v d(v) = 2e.
Because G is triangle-free, for every edge uv, the neighborhoods N(u) and N(v) are disjoint (otherwise a common neighbor would create a triangle u-v-w-u). Therefore d(u)+d(v) <= n for every edge uv.
Sum this inequality over all edges uv in E(G):
  sum_{uv in E} (d(u)+d(v)) <= e * n.
But the left-hand side equals sum_{v} d(v)^2, because each vertex v appears in exactly d(v) edge-terms, contributing d(v) each time.
Hence
  sum_v d(v)^2 <= e n.
By Cauchy-Schwarz,
  sum_v d(v)^2 >= (sum_v d(v))^2 / n = (2e)^2 / n = 4e^2 / n.
Combine:
  4e^2 / n <= e n.
If e>0, divide both sides by e to get 4e/n <= n, i.e. e <= n^2/4. If e=0 the inequality is trivial.
QED.

Lemma 1013.3 (a very weak lower bound: h_3(k) >= 2(k-1)).
For k>=2, any triangle-free graph with chromatic number k has at least 2(k-1) vertices. Consequently
  h_3(k) >= 2(k-1).

Proof.
Let G be triangle-free with chi(G)=k>=2 and let H be a k-critical subgraph from Lemma 1013.1.
Let n=|V(H)| and e=e(H).
By Lemma 1013.1, delta(H) >= k-1, so e >= n(k-1)/2 by the handshake lemma.
By Mantel's theorem (Lemma 1013.2), e <= n^2/4 because H is triangle-free.
Thus
  n(k-1)/2 <= n^2/4.
If n>0, divide by n/4 to obtain 2(k-1) <= n. Hence n>=2(k-1), and since h_3(k) is the minimal possible n, h_3(k) >= 2(k-1).
QED.

Lemma 1013.4 (Mycielski construction: h_3(k+1) <= 2 h_3(k) + 1).
Let G be a triangle-free graph on n vertices with chi(G)=k.
Then there exists a triangle-free graph M(G) on 2n+1 vertices with chi(M(G))=k+1.
Consequently h_3(k+1) <= 2 h_3(k) + 1.

Proof.
Construction of M(G):
- Start with a copy of G on vertex set {v_1,...,v_n}.
- Add new vertices {u_1,...,u_n}.
- Add one more vertex w.
- For each edge v_i v_j in G, add edges u_i v_j and u_j v_i.
- Add edges w u_i for all i.
No edges are added among the u_i themselves.

Triangle-freeness: any triangle would have to use at most one vertex from the set {u_i} because that set is independent. Also w is adjacent only to u_i, so any triangle containing w would need two u-vertices, impossible. Therefore any triangle would have to lie entirely in the original v-vertices, but those induce G, which is triangle-free. Hence M(G) is triangle-free.

Chromatic number >= k+1: suppose M(G) has a k-coloring. Restrict it to the original vertices v_i; since chi(G)=k, this restriction must use all k colors. In particular, for each i, vertex v_i has some color c(v_i) in {1,...,k}.
Now consider the colors on u_i. By construction, u_i is adjacent to every neighbor of v_i in G. Therefore, in any proper coloring, u_i cannot use any color that appears on N_G(v_i) (the neighbors of v_i in G). In particular, if u_i were colored the same as v_i for all i, then w would be adjacent to all those colors and would require a new color; the standard argument is as follows:
Take a proper k-coloring of the v_i. For each i, color u_i with the same color as v_i. This is valid because u_i is adjacent only to neighbors of v_i, and no neighbor of v_i shares v_i's color. Then w is adjacent to all u_i and hence to vertices of all k colors, so w requires a (k+1)th color. This gives an explicit (k+1)-coloring of M(G), so chi(M(G)) <= k+1.
On the other hand, the previous paragraph shows that w cannot be colored with any of the k colors if the u_i exhaust all k colors, which occurs whenever the v_i use all k colors (they must). More formally: in any proper coloring of M(G), for each color class among the v_i, there is at least one v_i of that color, and its corresponding u_i must avoid that color on N(v_i) but may take it; the coloring described above shows k+1 colors suffice, and the adjacency of w to all u_i forces w to use a color not used by any u_i. Since the coloring above uses all k colors among u_i, w needs a new color, so chi(M(G)) >= k+1.
Therefore chi(M(G))=k+1.

Finally, taking G on h_3(k) vertices yields M(G) on 2 h_3(k)+1 vertices, proving h_3(k+1) <= 2 h_3(k)+1.
QED.

5) VERIFICATION

- Lemma 1013.1: checked termination of the vertex-deletion process (finite descent) and the coloring extension argument.
- Lemma 1013.2: checked the identity sum_{uv in E}(d(u)+d(v)) = sum_v d(v)^2: each incident edge contributes d(v) once, and there are d(v) such edges.
- Lemma 1013.4: triangle-freeness checked by case split on whether the triangle uses w or any u_i.
  For chromatic number, the explicit (k+1)-coloring is correct; the forcing of >=k+1 is the nontrivial part and relies on the fact that the original v_i require k colors and the w-u_i adjacency forces an additional color in the standard Mycielski argument.
- Computational sanity: for G=C5, a brute-force coloring routine confirms chi(C5)=3, chi(M(C5))=4, chi(M^2(C5))=5, and all are triangle-free.

6) FINAL

**UNRESOLVED**

(i) Strongest proved partial result.
- Fully proved general-purpose bounds:
  * Lower: h_3(k) >= 2(k-1) (Lemma 1013.3).
  * Upper recurrence: h_3(k+1) <= 2 h_3(k) + 1 via Mycielski (Lemma 1013.4), giving an explicit (but extremely weak) exponential upper bound.
- Verified exact small cases: h_3(1)=1, h_3(2)=2, h_3(3)=5; and explicit constructions for k=4,5 with n=11,23.

(ii) First gap (crisp).
Bridge the large gap between the trivial linear lower bound and the (log k)*k^2 upper bound stated in the problem file, and in particular prove an asymptotic formula for h_3(k) and the limit h_3(k+1)/h_3(k) -> 1.

(iii) Top 3 next moves.
1. Use the inequality chi >= n/alpha together with the best available lower bounds on alpha for triangle-free graphs (in terms of n and average degree) to re-derive the k^2 log k scale, but with fully proved constants.
2. Attempt to prove a near-subadditivity relation like h_3(k+l) <= h_3(k)+h_3(l)+o(h_3(k)+h_3(l)) by combining triangle-free graphs in a way that preserves triangle-freeness and roughly adds chromatic number.
3. Computation/structure: search for triangle-free k-chromatic graphs near the currently available upper bound size to see whether h_3(k+1)/h_3(k) is empirically close to 1 for moderate k, and identify the construction families responsible.

(iv) What a minimal counterexample would likely look like.
Failure of lim h_3(k+1)/h_3(k)=1 would mean there exists eps>0 and infinitely many k with h_3(k+1) >= (1+eps) h_3(k). A minimal such counterexample would likely come from a sharp "jump" phenomenon in the extremal triangle-free constructions (e.g., a structural barrier to increasing chromatic number by 1 without multiplying the vertex count by a fixed factor), which seems unlikely given the k^2 log k scale but would need to be detected via a rigidity theorem for near-optimal constructions.

