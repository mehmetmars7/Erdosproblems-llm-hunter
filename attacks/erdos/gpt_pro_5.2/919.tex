
1) FORMAL RESTATEMENT

Let \omega_2 denote the first ordinal of cardinality \aleph_2.
Let \omega_2^2 denote the ordinal product \omega_2\cdot\omega_2, which is order-isomorphic to \omega_2\times\omega_2 with the lexicographic order.
A subset X of this well-ordered set has “lesser type” than \omega_2^2 if its induced order type is an ordinal <\omega_2^2.

Ambiguity: the problem statement speaks of “subgraph whose vertices have a lesser type”.
We interpret this as: for every subset X\subseteq \omega_2^2 with order type <\omega_2^2, the induced subgraph G[X] has \chi(G[X])\le\aleph_0.

Question (A): Does there exist a graph G with vertex set \omega_2^2 and \chi(G)=\aleph_2 such that for every X\subseteq \omega_2^2 with order type <\omega_2^2 we have \chi(G[X])\le \aleph_0?

Question (B): Same but requiring \chi(G)=\aleph_1.


2) QUICK LITERATURE/CONTEXT CHECK

The problem statement records:
- Babai: if G is a graph on a well-ordered vertex set with \chi(G)\ge \aleph_0 then there is a subgraph on vertices of order type \omega with chromatic number \aleph_0.
- Erd\H{o}s–Hajnal constructed a graph on \omega_1^2 with chromatic number \aleph_1 such that every strictly smaller subgraph has chromatic number \le\aleph_0, using the product-order comparability graph: vertices (\alpha,\beta) and edges when (\alpha_1<\alpha_2 and \beta_1<\beta_2) (symmetrized).
- A similar construction yields a graph on \omega_2^2 with chromatic number \aleph_2 such that every smaller subgraph has chromatic number \le \aleph_1.

I do not use or assert any additional external results beyond what is stated in the problem file.


3) ATTACK PLAN

Proof-track (existence):
- Try to strengthen the Erd\H{o}s–Hajnal product-order construction to force countable chromatic number on all vertex sets of order type <\omega_2^2 (not just \le\aleph_1).
- Attempt to build G by recursion along the well-order, ensuring each initial segment (hence each smaller-type set) stays countably colorable, while the whole graph becomes \aleph_2-chromatic.

Disproof-track (nonexistence):
- Attempt to prove that if \chi(G)=\aleph_2 on \omega_2^2, then some smaller-type subset must already witness uncountable chromatic number, perhaps via a reflection/pressing-down argument on the lex order.

Best path here: I cannot complete either track. I therefore prove concrete properties of the Erd\H{o}s–Hajnal comparability graph (since it is the natural candidate) and isolate the gap to reach \le\aleph_0 on all smaller types.


4) WORK

Definition (Erd\H{o}s–Hajnal product-order graph).
Fix an infinite cardinal \kappa and consider vertex set V=\kappa\times\kappa.
Define an undirected graph H_\kappa on V by putting an edge between distinct vertices (\alpha_1,\beta_1) and (\alpha_2,\beta_2) iff either
(\alpha_1<\alpha_2 and \beta_1<\beta_2) or (\alpha_2<\alpha_1 and \beta_2<\beta_1).
Equivalently, there is an edge iff the two vertices are comparable in the strict product order on \kappa\times\kappa.

Lemma 919.1 (chromatic number of H_\kappa equals \kappa).
For every infinite cardinal \kappa, \chi(H_\kappa)=\kappa.

Proof.
Upper bound: Color each vertex (\alpha,\beta) by its first coordinate \alpha.
If two vertices share the same first coordinate \alpha, then neither can have \alpha_1<\alpha_2, so they are not adjacent.
Hence each color class is independent and this is a proper coloring using \kappa colors.
Thus \chi(H_\kappa)\le \kappa.

Lower bound: Consider the diagonal set D:=\{(\alpha,\alpha):\alpha<\kappa\}.
If (\alpha,\alpha) and (\alpha',\alpha') are distinct, then either \alpha<\alpha' and also \alpha<\alpha', hence they are adjacent.
So D induces a clique of size \kappa.
Any proper coloring of a clique of size \kappa uses at least \kappa colors, hence \chi(H_\kappa)\ge \kappa.
Combining bounds gives \chi(H_\kappa)=\kappa. \qed

Lemma 919.2 (countable-coordinate-bounded subgraphs are countably colorable).
Let \kappa be any cardinal and let S\subseteq \kappa\times\kappa.
If there exists a countable ordinal \gamma<\omega_1 such that S\subseteq \gamma\times\kappa, then \chi(H_\kappa[S])\le \aleph_0.
Similarly, if S\subseteq \kappa\times\gamma for some countable \gamma, then \chi(H_\kappa[S])\le \aleph_0.

Proof.
Assume S\subseteq \gamma\times\kappa with \gamma countable.
Define c:S\to\gamma by c(\alpha,\beta)=\alpha.
As in Lemma 919.1, vertices with the same first coordinate are non-adjacent, so each fiber c^{-1}(\alpha) is independent.
Since \gamma is countable, this gives a proper coloring using at most \aleph_0 colors.
The second statement (bounding the second coordinate) is proved by coloring with the second coordinate. \qed

FAST REALITY CHECK.
A finite analogue of H_\kappa on [m]\times[m] (m finite) has chromatic number exactly m:
- The diagonal gives a clique of size m, so \chi\ge m.
- Coloring by the first coordinate uses m colors, so \chi\le m.
This matches Lemma 919.1 in the finite case.


5) VERIFICATION

- Lemma 919.1: Verified that “color by first coordinate” is valid because adjacency requires strict inequality in the first coordinate.
- Lemma 919.2: Verified that the number of colors used is |\gamma|\le\aleph_0 and each color class is independent.
- The lemmas do not address the key “lesser type” condition; they only show countable chromaticity for a restricted family of smaller-type sets (those bounded by a countable ordinal in one coordinate).


6) FINAL

**UNRESOLVED**
(i) Strongest proved partial result: The natural candidate graph H_{\omega_2} (product-order comparability graph on \omega_2\times\omega_2) has \chi(H_{\omega_2})=\aleph_2 (Lemma 919.1). Moreover, any induced subgraph whose vertices lie in \gamma\times\omega_2 or \omega_2\times\gamma for some countable \gamma is countably colorable (Lemma 919.2).
(ii) First gap (crisp): Show (or refute) that there exists a graph on \omega_2^2 with chromatic number \aleph_2 for which every subset of vertices of order type <\omega_2^2 induces a countably colorable subgraph.
(iii) Top 3 next moves:
  1. Analyze the Erd\H{o}s–Hajnal construction more deeply: characterize which order types <\omega_2^2 already force uncountable chromatic number in H_{\omega_2}. This becomes a concrete test case for (A).
  2. Attempt to build a graph by recursion along the lex order on \omega_2^2, explicitly maintaining countable colorings on all initial segments while diagonalizing against all \aleph_0-colorings globally.
  3. Try to prove a reflection lemma: if \chi(G)=\aleph_2 on \omega_2^2 then there exists X of order type <\omega_2^2 with \chi(G[X])\ge\aleph_1. Identifying the first nontrivial case (e.g. order type \omega_2\cdot\omega) would already be progress.
(iv) Minimal counterexample structure: A positive example for (A) would be a graph that is globally \aleph_2-chromatic on the well-order \omega_2^2 but whose chromatic number “collapses” to countable on every proper initial-type subset. Any disproof must show that some smaller-type subset necessarily inherits uncountable chromaticity.


