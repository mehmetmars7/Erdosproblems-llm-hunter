\section*{Problem 368: Largest prime factor of $n(n+1)$}

\subsection*{FORMAL RESTATEMENT}
Let
\[
F(n) := \max\{p:\ p\text{ prime and }p\mid n(n+1)\},
\]
the largest prime divisor of $n(n+1)$.

The prompt asks ``How large is $F(n)$?'' and then records several known results (e.g. $F(n)\to\infty$, lower bounds, and conjectures).

\textbf{Ambiguity / minimal correction.} ``How large'' is not a proposition.
A precise mathematical statement explicitly present in the prompt is:

\medskip
\noindent\textbf{(\*)} \emph{$F(n)\to\infty$ as $n\to\infty$.}

I will give a complete proof of (\*) using St\o rmer's theorem on consecutive smooth numbers.

\subsection*{QUICK LITERATURE/CONTEXT CHECK}
The Erd\H{o}s-problems page for \#368 attributes $F(n)\to\infty$ to P\'olya (1918) and also notes St\o rmer's earlier work. A standard effective route is St\o rmer's theorem (1897) that for any finite set of primes $S$, there are only finitely many pairs of consecutive $S$-smooth integers; an expository proof is given by M.~Klazar (2010) via Pell equations.

\subsection*{ATTACK PLAN}
\begin{enumerate}
\item Fix a bound $B$ and let $S$ be the set of primes $\le B$.
\item Show: if $F(n)\le B$ then both $n$ and $n+1$ are $S$-smooth (all prime factors in $S$).
\item Apply St\o rmer's theorem to conclude there are only finitely many such $n$.
\item Conclude: for each fixed $B$ there exists $N(B)$ such that $n\ge N(B)\Rightarrow F(n)>B$, i.e. $F(n)\to\infty$.
\end{enumerate}

\subsection*{WORK (complete proof of (\*))}
\subsubsection*{1. Smooth numbers and reduction to St\o rmer}
Let $S$ be a finite set of primes. Call a positive integer \emph{$S$-smooth} if all its prime factors lie in $S$.

\paragraph{Observation 1.1.} If $F(n)\le B$ and $S:=\{p\text{ prime}:p\le B\}$, then both $n$ and $n+1$ are $S$-smooth.
\begin{proof}
If $F(n)\le B$ then every prime dividing $n(n+1)$ is $\le B$, hence lies in $S$.
Since $\gcd(n,n+1)=1$, any prime dividing $n$ (resp. $n+1$) also divides $n(n+1)$, so it lies in $S$.
\end{proof}

Thus, to prove $F(n)\to\infty$, it suffices to show:

\medskip
\noindent\textbf{Claim.} For each finite set of primes $S$, there are only finitely many $n$ such that both $n$ and $n+1$ are $S$-smooth.

This is (a special case of) St\o rmer's theorem. We now prove it.

\subsubsection*{2. A key finiteness lemma for Pell equations}
For $d\in\mathbb{N}$, consider the Pell equation
\[
 x^2-dy^2=1.
\tag{Pell}_d
\]
Say that $y$ is a \emph{$d$-number} if every prime factor of $y$ divides $d$.

\paragraph{Lemma 2.1 (structure of Pell solutions).}
Assume $d$ is not a perfect square and $(\mathrm{Pell})_d$ has at least one positive integer solution.
Let $(a_1,b_1)$ be the \emph{fundamental solution}, i.e. the solution with $a_1>1$ minimal.
Then every positive solution $(a_n,b_n)$ is given by
\[
 a_n+b_n\sqrt d = (a_1+b_1\sqrt d)^n\qquad(n\ge 1),
\]
and conversely each $n\ge 1$ produces a positive solution.
\begin{proof}
Work in the quadratic ring $\mathbb{Z}[\sqrt d]$.
If $(x,y)$ solves $(\mathrm{Pell})_d$, then $u:=x+y\sqrt d$ satisfies $N(u)=u\,\overline u=x^2-dy^2=1$, so $u$ is a unit of norm $1$.
If $u=x+y\sqrt d$ and $u'=x'+y'\sqrt d$ both have norm $1$, then $uu'$ also has norm $1$, and its coefficients are integers, so solutions form a multiplicative group.

Let $\varepsilon:=a_1+b_1\sqrt d>1$.
Then $\varepsilon^n$ has norm $1$ and equals $a_n+b_n\sqrt d$ with integers $a_n,b_n$ (by repeated multiplication), so each $n$ yields a solution.

Conversely, let $u=x+y\sqrt d>1$ be any solution.
Choose $n\in\mathbb{Z}$ such that $\varepsilon^n\le u<\varepsilon^{n+1}$.
Then $v:=u/\varepsilon^n$ is again a unit of norm $1$, and satisfies $1\le v<\varepsilon$.
Writing $v=r+s\sqrt d$ with $r,s\in\mathbb{Z}$ (closure under multiplication and inversion in this unit group), the inequalities $1\le v<\varepsilon$ force $s=0$ and $r=1$ (since $|r-s\sqrt d|=|\overline v|=1/v\le 1$ and $r,s$ are integers).
Hence $v=1$ and $u=\varepsilon^n$.
Therefore every solution is a power of the fundamental unit.
\end{proof}

\paragraph{Lemma 2.2 (St\o rmer's Pell finiteness).}
For fixed $d\in\mathbb{N}$, the Pell equation $(\mathrm{Pell})_d$ has \emph{at most one} positive integer solution $(x,y)$ for which $y$ is a $d$-number.
\begin{proof}
If $(\mathrm{Pell})_d$ has no positive solutions, there is nothing to prove.
Otherwise let $(a_1,b_1)$ be its fundamental solution and write all solutions as in Lemma 2.1.
Suppose for some $m\ge 1$ the solution $(a_m,b_m)$ has $b_m$ a $d$-number.
We will show $m=1$.

\smallskip
\noindent\emph{Step 1: basic coprimality.}
From $a_1^2-db_1^2=1$, any common divisor of $a_1$ and $d$ must divide $1$, so
$\gcd(a_1,d)=1$ and in particular $a_1>1$.

\smallskip
\noindent\emph{Step 2: divisibility in the $b_n$ sequence.}
For $n=kl$ we have
\[
 a_{kl}+b_{kl}\sqrt d = (a_k+b_k\sqrt d)^l.
\]
Expanding the right-hand side by the binomial theorem, every term contributing to the $\sqrt d$-coefficient contains a factor $b_k$, so $b_k\mid b_{kl}$.
In particular, taking $k=1$ shows $b_1\mid b_n$ for all $n$, so we may write
\[
 b_n=b_1c_n\qquad (c_n\in\mathbb{N}).
\]
Moreover, for any $l\ge 1$, dividing the binomial expansion for $(a_1+b_1\sqrt d)^l$ by $b_1$ yields the explicit identity
\[
 c_l = l a_1^{l-1} + \binom{l}{3} a_1^{l-3} b_1^2 d + \binom{l}{5} a_1^{l-5} b_1^4 d^2 + \cdots,
\tag{1}
\]
a sum over odd indices.

\smallskip
\noindent\emph{Step 3: prime divisors of $m$ lead to contradiction.}
Assume for contradiction that $m>1$.
Let $p$ be any prime divisor of $m$, and write $m=pr$.
Then $b_p\mid b_m$, hence $c_p=b_p/b_1\mid b_m$.
Since $b_m$ is a $d$-number, every prime factor of $c_p$ divides $d$, i.e. $c_p$ is a $d$-number.
We now show this is impossible for any prime $p$.

\smallskip
\noindent\underline{Case $p=2$.}
From (1) with $l=2$ we have $c_2=2a_1$.
If $c_2$ is a $d$-number, every prime factor of $a_1$ divides $d$, contradicting $\gcd(a_1,d)=1$ unless $a_1=1$.
But $a_1>1$, contradiction.

\smallskip
\noindent\underline{Case $p=3$.}
From (1) with $l=3$ we get
\[
 c_3 = 3a_1^2 + b_1^2 d.
\]
Using $a_1^2-db_1^2=1$ we rewrite $b_1^2 d=a_1^2-1$, hence
\[
 c_3 = 4a_1^2-1=(2a_1-1)(2a_1+1).
\]
Let $q$ be any prime divisor of $c_3$. Since $c_3$ is a $d$-number, $q\mid d$.
Reducing $c_3=3a_1^2+b_1^2d$ modulo $q$ gives $0\equiv 3a_1^2\pmod q$.
As $\gcd(a_1,q)=1$ (because $q\mid d$ and $\gcd(a_1,d)=1$), we conclude $q=3$.
So $c_3$ is a power of $3$, meaning both $2a_1-1$ and $2a_1+1$ are powers of $3$ differing by $2$.
The only positive powers of $3$ differing by $2$ are $1$ and $3$, which forces $a_1=1$, contradiction.

\smallskip
\noindent\underline{Case $p\ge 5$.}
Let $q$ be any prime divisor of $c_p$.
Again $c_p$ is a $d$-number, so $q\mid d$.
Reduce identity (1) modulo $q$ with $l=p$:
all terms after the first contain a factor $d$, hence vanish modulo $q$, so
\[
0\equiv c_p \equiv p a_1^{p-1}\pmod q.
\]
Since $\gcd(a_1,q)=1$, this implies $q\mid p$, so $q=p$.
Thus every prime factor of $c_p$ is $p$, so $c_p=p^r$ for some $r\ge 1$.
In particular $p\mid c_p$, hence $p\mid d$.
Now in (1) with $l=p$, every term after the first is divisible by $p^2$ because it has a factor $d$ (hence $p$) and a binomial coefficient $\binom{p}{j}$ (hence $p$).
Therefore $c_p > p a_1^{p-1}$ (since all summands are positive), and because $a_1\ge 2$ we have $p a_1^{p-1}\ge p\cdot 2^{p-1} > p$.
So $c_p>p$, contradicting $c_p=p^r$ with $r=1$, and also contradicting $r\ge 2$ because then $c_p\ge p^2$ would force $p^2\mid p a_1^{p-1}$ and hence $p\mid a_1$, again contradicting $\gcd(a_1,d)=1$.

\smallskip
\noindent In all cases we obtain a contradiction, so $m$ has no prime divisors, hence $m=1$.
Thus the only solution with $y$ a $d$-number is the fundamental one, proving uniqueness.
\end{proof}

\subsubsection*{3. St\o rmer's theorem for consecutive $S$-smooth integers}
\paragraph{Theorem 3.1 (St\o rmer; difference $1$ case).}
Let $S=\{p_1,\dots,p_r\}$ be a finite set of primes.
Then the equation
\[
 x-y=1
\]
has only finitely many solutions in positive $S$-smooth integers $x,y$.
In fact, there are at most $3^r$ such solutions.
\begin{proof}
If $2\notin S$, then every $S$-smooth integer is odd, so no two can differ by $1$.
So assume $2\in S$.

Let $y$ be $S$-smooth and set $a:=2y+1$.
If also $y+1$ is $S$-smooth then $4y(y+1)$ is $S$-smooth, and
\[
4y(y+1)=(2y+1)^2-1=a^2-1.
\]
So it suffices to bound the number of $S$-smooth integers of the form $a^2-1$.

Write $a^2-1=\prod_{i=1}^r p_i^{e_i}$.
Define integers $b_i\ge 0$ by:
\begin{itemize}
\item if $e_i$ is odd, set $b_i:=e_i-1$ (so $e_i-b_i=1$);
\item if $e_i\in\{0,2\}$, set $b_i:=0$ (so $e_i-b_i\in\{0,2\}$);
\item if $e_i\ge 4$ is even, set $b_i:=e_i-2$ (so $e_i-b_i=2$).
\end{itemize}
Then set
\[
 d:=\prod_{i=1}^r p_i^{e_i-b_i},\qquad b:=\prod_{i=1}^r p_i^{b_i/2}.
\]
By construction, $e_i-b_i\in\{0,1,2\}$, so there are at most $3^r$ possible values of $d$.
Also $a^2-1=db^2$, i.e. $a^2-d b^2=1$, and whenever $b_i>0$ we have $e_i-b_i=2$, so $p_i\mid d$.
Thus every prime dividing $b$ divides $d$, i.e. $b$ is a $d$-number.

For each fixed $d$, Lemma 2.2 shows there is at most one positive solution $(a,b)$ to $a^2-d b^2=1$ with $b$ a $d$-number.
Hence there are at most $3^r$ possible values of $a$, and therefore at most $3^r$ possible values of $y=(a-1)/2$.
This proves finiteness.
\end{proof}

\subsubsection*{4. Proof that $F(n)\to\infty$}
Fix $B\ge 2$ and set $S:=\{p\text{ prime}:p\le B\}$.
If $F(n)\le B$, Observation 1.1 implies $n$ and $n+1$ are both $S$-smooth.
By Theorem 3.1, there are only finitely many such $n$.
Therefore for each $B$ there exists $N(B)$ such that $n\ge N(B)\Rightarrow F(n)>B$.
This is exactly the statement that $F(n)\to\infty$ as $n\to\infty$.

\subsection*{VERIFICATION}
\begin{itemize}
\item The reduction $F(n)\le B\Rightarrow n,n+1\text{ are }S\text{-smooth}$ used only $\gcd(n,n+1)=1$.
\item Lemma 2.2 was proved by contradiction using only integer divisibility and the explicit binomial expansion identity (1).
\item The final quantifier structure matches the limit definition: for each fixed $B$ only finitely many exceptions.
\end{itemize}

\subsection*{META-CHECK (common failure modes)}
\begin{itemize}
\item \emph{Confusing ``unbounded'' with ``tends to infinity'':} we proved the stronger statement ``for each $B$ only finitely many $n$ have $F(n)\le B$'', which implies $F(n)\to\infty$.
\item \emph{Hidden use of deep theorems:} the argument uses only Pell-equation structure and elementary divisibility (no Baker theory, etc.).
\item \emph{Ring issues for Pell:} we only used closure of $\mathbb{Z}[\sqrt d]$ under multiplication and the norm map.
\end{itemize}

\subsection*{FINAL}
FINAL: LABEL: \textbf{FULL SOLUTION} \quad SUBLABEL: \textbf{FULL PROOF}.

\subsection*{COMPLETION ESTIMATE}
COMPLETION ESTIMATE: 100\%.

\hrule

\section*{Bibliographic pointers (non-exhaustive)}
\begin{thebibliography}{99}
\bibitem{KlazarStormer}
M.~Klazar, \emph{St\o rmer's solution of the unit equation $x-y=1$}, 2010,\newline
\texttt{https://kam.mff.cuni.cz/~klazar/stormer.pdf}.

\bibitem{Polya1918}
G.~P\'olya, \emph{Zur arithmetischen Untersuchung der Polynome}, Math. Z.\ \textbf{1} (1918), 143--148.

\bibitem{OEISA060355}
OEIS Foundation, \emph{A060355: Numbers $k$ such that $k$ and $k+1$ are powerful},\newline
\texttt{https://oeis.org/A060355}.

\bibitem{OEISA074399}
OEIS Foundation, \emph{A074399: $a(n)$ is the largest prime divisor of $n(n+1)$},\newline
\texttt{https://oeis.org/A074399}.
\end{thebibliography}

