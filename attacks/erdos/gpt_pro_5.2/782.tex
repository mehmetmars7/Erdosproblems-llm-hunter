\section*{Problem \#782 (Squares and quasi-progressions / Hilbert cubes)}
\addcontentsline{toc}{section}{Problem \#782 (Squares and quasi-progressions / Hilbert cubes)}

\subsection*{1) FORMAL RESTATEMENT}
\addcontentsline{toc}{subsection}{1) Formal restatement (\#782)}

Let $\mathcal{S}=\{m^2: m\in\mathbb{N}\}$ be the set of (positive) squares.

\medskip
\noindent\textbf{(A) Quasi-progressions.}
Fix an integer constant $C\ge 0$.  A finite increasing sequence $(x_1<\cdots<x_k)$ of integers is a \emph{$C$--quasi-progression} if there exists an integer $d\ge 1$ such that
\[
 d \le x_{i+1}-x_i \le d+C \qquad\text{for all } i=1,\dots,k-1.
\]
Question: \emph{Is there a fixed constant $C$ such that $\mathcal{S}$ contains $C$--quasi-progressions of arbitrarily large length $k$?}
(Equivalently: does $\sup\{k: \exists\ C\text{ fixed with a length-$k$ quasi-progression in }\mathcal{S}\}=\infty$?)

\medskip
\noindent\textbf{(B) Hilbert cubes in squares.}
A (finite) \emph{Hilbert cube} of dimension $t$ is a set of the form
\[
H(a; b_1,\dots,b_t)\;:=\;\Big\{a+\sum_{j=1}^t \varepsilon_j b_j : \varepsilon_j\in\{0,1\}\Big\},
\]
where $a\in\mathbb{Z}$ and $b_1,\dots,b_t\in\mathbb{N}$.
Question: \emph{Does $\mathcal{S}$ contain Hilbert cubes of arbitrarily large dimension $t$?}

The problem statement notes that an affirmative answer to (A) would imply an affirmative answer to (B).

\subsection*{2) QUICK LITERATURE/CONTEXT CHECK}
\addcontentsline{toc}{subsection}{2) Quick literature/context check (\#782)}

\begin{itemize}[leftmargin=2em]
\item Brown--Erd\H{o}s--Freedman (1990) introduced quasi-progressions (and related ``descending waves'') and asked whether the squares contain arbitrarily long quasi-progressions with bounded error $C$.
\item Solymosi (2007) discussed the related question of Hilbert cubes in the squares and conjectured that the squares do \emph{not} contain arbitrarily large Hilbert cubes.
\item Cilleruelo--Granville (2007) observed that Solymosi's conjecture (no arbitrarily large Hilbert cubes in squares) would follow from the Bombieri--Lang conjecture.
\item Unconditionally, there are nontrivial upper bounds on the maximum dimension of a Hilbert cube contained in $\mathcal{S}\cap[1,N]$.  In particular, Dietmann--Elsholtz (2015) prove a bound of the shape
\[
\dim\big(\mathcal{S}\cap[1,N]\big) \le 7\,\log\log N
\qquad\text{for all sufficiently large }N.
\]
This still allows the possibility of arbitrarily large cubes (since $\log\log N\to\infty$).
\end{itemize}

So as of these works, both questions remain open unconditionally.

\subsection*{3) ATTACK PLAN}
\addcontentsline{toc}{subsection}{3) Attack plan (\#782)}

\textbf{(A) Attempt a construction.}
Try to build many squares whose successive gaps are almost equal.
A heuristic route is to search for squares lying close to a long arithmetic progression.

\medskip
\noindent\textbf{(B) Attempt an obstruction.}
Try to show that bounded-gap quasi-progressions would force some Diophantine structure (e.g. many integer points on curves) that can be controlled.
For Hilbert cubes, one tries to translate the existence of a large cube into many solutions of systems of quadratic equations.

\subsection*{4) WORK}
\addcontentsline{toc}{subsection}{4) Work (\#782)}

\subsubsection*{4.1 Basic constraints and reformulations}
\paragraph{Gap identity.}
If $x_i=a_i^2\in\mathcal{S}$, then
\[
 x_{i+1}-x_i = a_{i+1}^2-a_i^2 = (a_{i+1}-a_i)(a_{i+1}+a_i).
\]
If all gaps lie in an interval $[d,d+C]$ with fixed $C$, then as $a_i$ grows, the factor $a_{i+1}+a_i$ grows, hence the integer factor $a_{i+1}-a_i$ must typically get very small (eventually $=1$), suggesting that long quasi-progressions might force $a_i$ to stay in a controlled range relative to $d$.
This intuition is not yet a proof.

\paragraph{A trivial non-constant-$C$ construction.}
Consecutive squares give a quasi-progression of length $k$ but with $C$ growing linearly in $k$:
\[
(m+1)^2-m^2 = 2m+1,
\]
so over $k$ consecutive squares the gaps range over an interval of length $2(k-1)$.
This shows that the question is genuinely about keeping $C$ \emph{independent of $k$}.

\subsubsection*{4.2 Hilbert cubes: what is known and what it would take}
\paragraph{Dimension in $[1,N]$.}
Let
\[
\dim\big(\mathcal{S}\cap[1,N]\big) := \max\Big\{ t:\exists a,b_1,\dots,b_t\text{ with }H(a;b_1,\dots,b_t)\subseteq \mathcal{S}\cap[1,N]\Big\}.
\]
Dietmann--Elsholtz show $\dim(\mathcal{S}\cap[1,N]) \ll \log\log N$.
Thus, if arbitrarily large cubes exist in the squares, their dimension must grow extremely slowly with the size of the squares involved.

\paragraph{Why Bombieri--Lang would rule out large cubes (informal).}
A $t$-dimensional cube in squares gives $2^t$ distinct square values with additive structure.  Fixing $a$ and the $b_i$, one can encode many of these relations as rational points on certain curves/surfaces of general type.  The Bombieri--Lang conjecture predicts strong uniform finiteness properties for rational points on such varieties, which would bound $t$ by an absolute constant.

\subsection*{5) VERIFICATION}
\addcontentsline{toc}{subsection}{5) Verification (\#782)}

\begin{itemize}[leftmargin=2em]
\item The ``consecutive squares'' check correctly shows that naive constructions do not keep $C$ fixed.
\item The stated $\log\log N$ upper bound for cube dimension in squares matches the cited Dietmann--Elsholtz theorem.
\item No step above claims an unconditional resolution; the reasoning is consistent with the current literature status.
\end{itemize}

\subsection*{6) FINAL}
\addcontentsline{toc}{subsection}{6) Final (\#782)}

\paragraph{LABEL: \textbf{UNRESOLVED}.}

\paragraph{Best partial results I can state cleanly.}
\begin{itemize}[leftmargin=2em]
\item There are strong unconditional upper bounds on the maximum dimension of a Hilbert cube inside $\mathcal{S}\cap[1,N]$, namely $\dim(\mathcal{S}\cap[1,N])\ll\log\log N$.
\item Conditional on Bombieri--Lang, the squares do not contain arbitrarily large Hilbert cubes (hence a negative answer to (B) and therefore to (A) as well).
\end{itemize}

\paragraph{Strongest heuristic.}
Random-model heuristics suggest that large additive subcubes inside the sparse set of squares should be rare; this supports Solymosi's conjecture that an absolute bound on cube dimension might hold.  On the other hand, the existing unconditional bounds still allow very slowly growing cube dimension, leaving both outcomes plausible.

\paragraph{Main obstacle.}
To rule out arbitrarily long quasi-progressions or cubes one needs uniform control over integer/rational points on families of quadratic varieties arising from additive constraints among squares.  Current unconditional tools (sum-product methods, bounds on solutions to $x^2+y^2=z^2+w^2$, etc.) are not yet strong enough to force a constant bound.

\paragraph{Next concrete steps.}
\begin{enumerate}[leftmargin=2em]
\item Search systematically for explicit cubes/quasi-progressions of the largest currently known sizes (even small improvements could suggest the right growth scale).
\item Try to sharpen the $\log\log N$ bound toward an absolute bound using improvements in additive combinatorics for polynomial images.
\item For quasi-progressions, attempt to prove that bounded-gap sequences of squares imply many rational points on a fixed genus-$\ge 2$ curve, which would allow Faltings-type finiteness arguments.
\end{enumerate}

\paragraph{COMPLETION: 25\%.}

%%%%%%%%%%%%%%%%%%%%%%%%%%%%%%%%%%%%%%%%%%%%%%%%%%%%%%%%%%%%%%%%%%%%%%%%%%%%%%
