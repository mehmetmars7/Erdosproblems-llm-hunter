

\noindent\textbf{FORMAL RESTATEMENT.}
Let $\epsilon>0$. Does there exist $N_0(\epsilon)$ such that for all integers $N\ge N_0(\epsilon)$ the following holds?

If $A\subseteq \{1,2,\dots,N\}$ and $|A|\ge \epsilon N$, then there exist \emph{distinct} $a,b,c\in A$ such that
\[
\operatorname{lcm}(a,b)=\operatorname{lcm}(b,c)=\operatorname{lcm}(a,c),
\]
where $\operatorname{lcm}(x,y)$ denotes the least common multiple.

\noindent\textbf{QUICK LITERATURE/CONTEXT CHECK.}
The problem statement (in the provided source file) notes:
(i) the analogous demand for \emph{four} elements with all pairwise lcm equal is false (attributed there to Erd\H{o}s),
(ii) there are sketched constructions avoiding the triple-lcm property with $|A|\gg (\log\log N)^{f(N)}\,N/\log N$ for some $f(N)\to\infty$, and
(iii) there is a sketched proof for very large densities ($\epsilon>221/225$).
In this write-up, I do \emph{not} use any external results beyond what is stated in the file, and I give self-contained arguments for the partial results below.

\noindent\textbf{ATTACK PLAN.}
1) Rephrase the condition $\operatorname{lcm}(a,b)=\operatorname{lcm}(a,c)=\operatorname{lcm}(b,c)$ in $p$-adic terms and obtain a clean structural parametrization.
2) Use the parametrization to search for forced small patterns inside dense $A$.
3) Do brute-force search for small $N$ to see how large a set can be while avoiding the configuration.

\noindent\textbf{WORK.}
\textbf{Fast reality check (small $N$).}
I brute-forced all subsets $A\subseteq\{1,\dots,N\}$ for $3\le N\le 20$ and computed the maximum size of a set avoiding any triple of distinct $a,b,c$ with equal pairwise lcm.
The maximum sizes found were:
\[
\begin{array}{c|cccccccccccccccccc}
N&3&4&5&6&7&8&9&10&11&12&13&14&15&16&17&18&19&20\\\hline
\max|A|&3&4&5&5&6&7&8&9&10&10&11&12&12&13&14&15&16&16
\end{array}
\]
For example, for $N=20$ an extremal avoiding set of size $16$ is
$\{1,4,5,7,8,9,10,11,12,13,14,15,16,17,18,19\}$.

\medskip
\textbf{Lemma 536.1 (prime-exponent characterization).}
Let $a,b,c\in\mathbb N$ be distinct. Write $v_p(x)$ for the $p$-adic valuation. Then
\[
\operatorname{lcm}(a,b)=\operatorname{lcm}(a,c)=\operatorname{lcm}(b,c)
\]
holds if and only if for every prime $p$ the maximum of $\{v_p(a),v_p(b),v_p(c)\}$ is attained by at least two of the three numbers.

\textbf{Proof.}
Let $L_{ab}=\operatorname{lcm}(a,b)$, etc. For a prime $p$,
\[
 v_p(L_{ab})=\max\{v_p(a),v_p(b)\},\quad
 v_p(L_{ac})=\max\{v_p(a),v_p(c)\},\quad
 v_p(L_{bc})=\max\{v_p(b),v_p(c)\}.
\]
Thus $L_{ab}=L_{ac}=L_{bc}$ if and only if for every prime $p$ the three maxima above are equal.
Let $\alpha=v_p(a)$, $\beta=v_p(b)$, $\gamma=v_p(c)$, and let $M=\max\{\alpha,\beta,\gamma\}$.
If $M$ is attained by only one of $\{\alpha,\beta,\gamma\}$, say $\alpha=M>\beta,\gamma$, then $\max\{\beta,\gamma\}<M$ and therefore $v_p(L_{bc})<v_p(L_{ab})$, so equality of lcms fails.
Conversely, if $M$ is attained by at least two among $\alpha,\beta,\gamma$, then every pair has maximum exponent $M$ and hence $v_p(L_{ab})=v_p(L_{ac})=v_p(L_{bc})=M$.
Since this holds for every prime $p$, the lemma follows. \qed

\medskip
\textbf{Lemma 536.2 (structural parametrization).}
For $a,b,c\in\mathbb N$ distinct, the condition
$\operatorname{lcm}(a,b)=\operatorname{lcm}(a,c)=\operatorname{lcm}(b,c)$
holds if and only if there exist integers $d,u,v,w\in\mathbb N$ with
\[
\gcd(u,v)=\gcd(u,w)=\gcd(v,w)=1
\]
(symmetric pairwise coprimality) such that, up to permuting $a,b,c$,
\[
(a,b,c)=(d\,u v,\ d\,u w,\ d\,v w).
\]
In that case the common value of the three pairwise lcms is $d\,u v w$.

\textbf{Proof.}
($\Leftarrow$) Assume $a=d u v$, $b=d u w$, $c=d v w$ with pairwise coprime $u,v,w$.
Then
\[
\operatorname{lcm}(a,b)=\operatorname{lcm}(d u v, d u w)=d u\,\operatorname{lcm}(v,w).
\]
Because $\gcd(v,w)=1$, $\operatorname{lcm}(v,w)=vw$, so $\operatorname{lcm}(a,b)=d u v w$.
Similarly, $\operatorname{lcm}(a,c)=d v\,\operatorname{lcm}(u,w)=d u v w$ and $\operatorname{lcm}(b,c)=d w\,\operatorname{lcm}(u,v)=d u v w$.

($\Rightarrow$) Assume the triple-lcm condition holds.
Let $d=\gcd(a,b,c)$ and write $a=d a_1$, $b=d b_1$, $c=d c_1$ so that $\gcd(a_1,b_1,c_1)=1$.
Using $\operatorname{lcm}(d x, d y)=d\,\operatorname{lcm}(x,y)$, the pairwise-lcm equality for $(a,b,c)$ is equivalent to the same equality for $(a_1,b_1,c_1)$.

Fix a prime $p$ and set $\alpha=v_p(a_1)$, $\beta=v_p(b_1)$, $\gamma=v_p(c_1)$.
Since $\gcd(a_1,b_1,c_1)=1$, at least one of $\alpha,\beta,\gamma$ is $0$.
By Lemma~536.1, the maximum of $\{\alpha,\beta,\gamma\}$ is attained at least twice.
Therefore for each prime $p$ there are only two possibilities:
(i) $\alpha=\beta=\gamma=0$ (so $p$ divides none of $a_1,b_1,c_1$), or
(ii) exactly two of $\alpha,\beta,\gamma$ are equal and positive, and the third is $0$.
In case (ii), $p$ divides exactly two of the numbers $a_1,b_1,c_1$ with the \emph{same} exponent.

Define
\begin{align*}
 u &= \prod_{p: v_p(a_1)>0,\ v_p(b_1)>0} p^{v_p(a_1)} ,\\
 v &= \prod_{p: v_p(a_1)>0,\ v_p(c_1)>0} p^{v_p(a_1)} ,\\
 w &= \prod_{p: v_p(b_1)>0,\ v_p(c_1)>0} p^{v_p(b_1)}.
\end{align*}
These products are over disjoint sets of primes because a prime cannot divide all three of $a_1,b_1,c_1$ (that would contradict $\gcd(a_1,b_1,c_1)=1$).
Hence $u,v,w$ are pairwise coprime.
Moreover, every prime dividing $a_1$ divides either $b_1$ or $c_1$ (again by the classification above), so $a_1$ factors exactly as the product of primes shared with $b_1$ (which contribute to $u$) and primes shared with $c_1$ (which contribute to $v$). Thus $a_1=uv$.
Similarly $b_1=uw$ and $c_1=vw$.
Therefore $(a,b,c)=(d u v, d u w, d v w)$.
Finally, the lcm computation in the first direction gives the common lcm value $d u v w$. \qed

\medskip
\textbf{Proposition 536.3 (a high-density sufficient condition).}
Let $A\subseteq\{1,\dots,N\}$ and set $M=\lfloor N/6\rfloor$.
If
\[
M>3\,(N-|A|),
\]
(equivalently, $N-|A|<M/3$), then there exist distinct $a,b,c\in A$ with equal pairwise lcm.

\textbf{Proof.}
Consider the parametrized family of triples
\[
(2d,3d,6d)\qquad (1\le d\le M).
\]
Each such triple satisfies $\operatorname{lcm}(2d,3d)=\operatorname{lcm}(2d,6d)=\operatorname{lcm}(3d,6d)=6d$.
So it suffices to show that for some $d\le M$ all three numbers $2d,3d,6d$ lie in $A$.

Define the bad sets
\[
B_2=\{d\in\{1,\dots,M\}:2d\notin A\},\quad
B_3=\{d\in\{1,\dots,M\}:3d\notin A\},\quad
B_6=\{d\in\{1,\dots,M\}:6d\notin A\}.
\]
If no $d$ works, then every $d\in\{1,\dots,M\}$ lies in $B_2\cup B_3\cup B_6$, hence
\[
M\le |B_2|+|B_3|+|B_6|.
\]
For each $t\in\{2,3,6\}$, the map $d\mapsto td$ is injective on $\{1,\dots,M\}$ and takes values in $\{1,\dots,N\}$. Therefore $|B_t|$ is at most the number of missing elements $N-|A|$.
Consequently,
\[
M\le |B_2|+|B_3|+|B_6|\le 3(N-|A|),
\]
contradicting the hypothesis $M>3(N-|A|)$.
Hence some $d\le M$ has $2d,3d,6d\in A$, giving the desired triple. \qed

\textbf{Remark.}
Since $M=\lfloor N/6\rfloor\sim N/6$, the condition $M>3(N-|A|)$ holds whenever $|A|>(17/18+o(1))N$.
\noindent\textbf{VERIFICATION.}
\emph{Checking Lemma 536.2 on examples.} If $u,v,w$ are pairwise coprime and $a=d u v$, $b=d u w$, $c=d v w$, then for any prime $p$ dividing $u$ we have $v_p(a)=v_p(b)=v_p(d)+v_p(u)$ and $v_p(c)=v_p(d)$, so the maximum exponent is attained twice. The same holds for primes dividing $v$ or $w$, and primes dividing $d$ appear in all three. This is consistent with Lemma 536.1.

\emph{Sanity check for Proposition 536.3.} The obstruction to finding some $(2d,3d,6d)$ fully inside $A$ is that too many of the multiples $2d,3d,6d$ are missing. The proof counts these missing-multiple obstructions and uses only injectivity of $d\mapsto td$.

\noindent\textbf{FINAL.}\;\textbf{UNRESOLVED.}
\begin{itemize}
\item[(i)] \textbf{Strongest proved partial result.} Writing $M=\lfloor N/6\rfloor$, if $M>3(N-|A|)$ (equivalently $N-|A|<M/3$; in particular this holds when $|A|>(17/18+o(1))N$), then $A$ contains a triple $\{2d,3d,6d\}$ and hence distinct $a,b,c\in A$ with equal pairwise lcm (Proposition~536.3). Also, Lemma~536.2 gives a complete structural characterization of all triples with pairwise-equal lcm.
\item[(ii)] \textbf{First gap (crisp statement).} Prove (or disprove) that for every fixed $\epsilon>0$ there exists $N_0(\epsilon)$ such that for all $N\ge N_0(\epsilon)$ every $A\subseteq\{1,\dots,N\}$ with $|A|\ge\epsilon N$ contains some triple $(d u v, d u w, d v w)$ with $\gcd(u,v)=\gcd(u,w)=\gcd(v,w)=1$.
\item[(iii)] \textbf{Top 3 next moves.}
  \begin{itemize}
  \item Prove a density increment statement for the pattern $(2d,3d,6d)$ (or other small $(uv,uw,vw)$ patterns) that works for densities far below $17/18$ by using many patterns simultaneously and a refined counting of missing multiples.
  \item Develop a hypergraph formulation: view each $d$ as defining a 3-uniform hyperedge $\{2d,3d,6d\}$ and attempt a removal/regularity argument to show that dense $A$ must hit some hyperedge completely.
  \item Computationally: for larger $N$ (beyond brute force), implement a heuristic/ILP search for large avoiding sets to guess whether extremal size is $N-o(N)$ or $o(N)$.
  \end{itemize}
\item[(iv)] \textbf{Minimal counterexample structure to look for.} Any counterexample for a fixed $\epsilon>0$ would have to avoid 
  \emph{all} parametrized triples $(d u v, d u w, d v w)$ with pairwise coprime $u,v,w$, in particular avoiding the infinite family $(2d,3d,6d)$; thus it must delete at least one element from each triple $(2d,3d,6d)$ for $d\le N/6$, and similarly from each triple $(uvd,uwd,vwd)$ for many small coprime $(u,v,w)$.
\end{itemize}


