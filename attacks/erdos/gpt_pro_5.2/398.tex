% Erdos Problem #398
% URL: https://www.erdosproblems.com/398

1) FORMAL RESTATEMENT

Consider integer solutions $(n,x)$ with $n\ge 1$ and $x\in\mathbb Z$ to
\[
 n! = x^2 - 1,\quad\text{equivalently}\quad n!+1=x^2.
\]

Question (Brocard--Ramanujan conjecture).
Are the only solutions in positive integers $n$ given by
\[
(n,x)\in\{(4,5),(5,11),(7,71)\}?
\]

Edge cases.
For $n=1$, $1!+1=2$ is not a square; for $n=2$, $2!+1=3$ is not a square.

2) QUICK LITERATURE/CONTEXT CHECK

I only restate what is in the problem statement.

The statement records:
- Erd\H{o}s and Graham described the conjecture as “almost certainly true” but intractable.
- Overholt proved there are only finitely many solutions assuming a weak form of ABC.
- There are no other solutions with $n<10^9$ (as reported by the statement; not reproduced here).

3) ATTACK PLAN

Proof-track:
- Obtain congruence restrictions on $x$ and/or $n$ using small moduli.
- Compute small $n$ to verify known solutions and search for new ones (finite computation only).

Disproof-track:
- Try to find a new solution by computation in a moderate range.

Chosen path: prove a couple of easy necessary congruence conditions and verify all $n\le 50$ by computation.

4) WORK

PHASE 1 — FAST REALITY CHECK (computed)

I checked all $1\le n\le 50$ by computing $n!+1$ and testing if it is a perfect square. The only solutions in this range are
\[
(4,5),\ (5,11),\ (7,71).
\]

Lemma 1 (parity constraint).
If $n\ge 2$ and $n!+1=x^2$, then $x$ is odd.

Proof.
For $n\ge 2$, $n!$ is even, so $n!+1$ is odd. If $x^2$ is odd, then $x$ must be odd. \qed

Lemma 2 (a mod $10$ constraint for $n\ge 5$).
If $n\ge 5$ and $n!+1=x^2$, then $x\equiv \pm 1\pmod{10}$.

Proof.
For $n\ge 5$, the factorial $n!$ is divisible by $10$ (it contains factors $2$ and $5$). Hence
\[
 x^2 = n!+1 \equiv 1 \pmod{10}.
\]
The quadratic residues modulo $10$ of odd integers are $1,5,9$ (check $1^2\equiv 1$, $3^2\equiv 9$, $5^2\equiv 5$, $7^2\equiv 9$, $9^2\equiv 1$ mod $10$). Therefore $x^2\equiv 1\pmod{10}$ forces $x\equiv 1$ or $9\pmod{10}$, i.e. $x\equiv \pm 1\pmod{10}$. \qed

5) VERIFICATION

- Lemma 1 matches the known solutions: $x=5,11,71$ are odd.
- Lemma 2 matches the known solutions for $n=5,7$: $11\equiv 1\pmod{10}$ and $71\equiv 1\pmod{10}$.
- Computation up to $n=50$ found no additional solutions.

6) FINAL

**UNRESOLVED**

(i) Strongest fully proved partial result obtained here.
- Verified computationally that no solutions exist for $n\le 50$ besides $n=4,5,7$.
- Proved two necessary conditions: for $n\ge 2$, $x$ must be odd (Lemma 1); for $n\ge 5$, $x\equiv \pm 1\pmod{10}$ (Lemma 2).

(ii) Exact first gap.
Prove that no further solutions $(n,x)$ exist beyond the known ones; equivalently, show $n!+1$ is never a square for $n\notin\{4,5,7\}$.

(iii) Top 3 next moves (concrete targets).
1. Develop stronger congruence obstructions (e.g. modulo primes $p\le n$) that force $x$ into an increasingly restrictive set of residue classes as $n$ grows.
2. Combine such congruences with explicit bounds (e.g. lower bounds for gaps between consecutive squares) to rule out large $n$.
3. Extend computation further in $n$ (using fast integer square tests) to strengthen numerical evidence and possibly discover new patterns.

(iv) What a minimal counterexample would likely look like.
A counterexample would be a smallest $n>7$ such that $n!+1$ is a perfect square. It would necessarily have $x$ odd and $x\equiv \pm 1\pmod{10}$, and would satisfy $x^2\equiv 1\pmod p$ for every prime $p\le n$.


