
1) FORMAL RESTATEMENT

Let $a_1<a_2<\cdots$ be a strictly increasing sequence of positive integers satisfying
\[
  \liminf_{n\to\infty} a_n^{1/2^n} > 1.
\]
Define the series
\[
  S:=\sum_{n=1}^\infty \frac{1}{a_n a_{n+1}}.
\]
Question: must $S$ be irrational?

2) QUICK LITERATURE/CONTEXT CHECK

The provided text says Erd\H{o}s noted that this is true if $a_n\to\infty$ "rapidly", but no precise
criterion is given there.

3) ATTACK PLAN

Proof track:
- Convert the liminf growth condition into an explicit double-exponential lower bound (Lemma 4.1).
- Seek a Cantor-series / Engel-expansion style criterion under additional assumptions (divisibility or
  superquadratic growth) (Lemma 4.2), to clarify what "rapidly" could mean.

Disproof track:
- Try to build sequences with liminf $>1$ but engineered arithmetic structure to make $S$ rational
  (e.g. via telescoping or controlled denominators).

4) WORK

FAST REALITY CHECK (example)

Take $a_n=2^{2^n}$ (integers, strictly increasing, and $a_n^{1/2^n}=2$ for all $n$).
Partial sums:
\[
\begin{array}{c|c}
N & \sum_{n=1}^N \frac{1}{a_n a_{n+1}}\\\hline
1 & 0.015625\\
2 & 0.015869140625\\
3 & 0.015869200229644775390625\\
4 & 0.0158692002296483281043\dots
\end{array}
\]
No obvious rational pattern is visible.

Lemma 4.1 (Liminf growth implies a uniform double-exponential lower bound)

If $\liminf\limits_{n\to\infty} a_n^{1/2^n} > 1$, then there exist constants $c>1$ and $N_0$ such that
\[
  a_n \ge c^{2^n}\quad\text{for all }n\ge N_0.
\]

Proof.
Let $L:=\liminf_{n\to\infty} a_n^{1/2^n}$. By hypothesis $L>1$.
Choose any $c$ with $1<c<L$.
By definition of liminf, there exists $N_0$ such that for all $n\ge N_0$,
$a_n^{1/2^n}\ge c$.
Raising both sides to the power $2^n$ gives $a_n\ge c^{2^n}$.
$\square$

Lemma 4.2 (One sufficient condition for irrationality under strong divisibility+growth)

Assume in addition that:
(a) $a_{n+1}$ is divisible by $a_n$ for all $n$, and
(b) $a_{n+1}\ge a_n^2$ for all sufficiently large $n$.
Then
\[
  S=\sum_{n=1}^\infty \frac{1}{a_n a_{n+1}}
\]
is irrational.

Proof.
Let $b_n:=a_n a_{n+1}$. By (a), $a_{n+1}=a_n q_n$ with integer $q_n\ge 2$ (since strictly increasing).
Then $a_{n+2}=a_{n+1}q_{n+1}=a_n q_n q_{n+1}$, so $a_n\mid a_{n+2}$.
Hence
\[
  b_{n+1}=a_{n+1}a_{n+2}
\]
is divisible by $a_n a_{n+1}=b_n$. Thus $b_n\mid b_{n+1}$ for all $n$.
Consequently, each partial sum
\[
  S_N:=\sum_{n=1}^N \frac{1}{b_n}
\]
is a rational with denominator dividing $b_N$ (indeed, because the common denominator of
$1/b_1,\dots,1/b_N$ divides $b_N$).
So we can write $S_N=p_N/b_N$ in lowest terms with $b_N\mid a_N a_{N+1}$.

Now estimate the tail using (b). For large $n$ we have $a_{n+1}\ge a_n^2$, hence
\[
  b_{n+1}=a_{n+1}a_{n+2} \ge a_{n+1}\cdot a_{n+1}^2 = a_{n+1}^3.
\]
Also
\[
  b_n^2=(a_n a_{n+1})^2 \le (a_{n+1}^{1/2} a_{n+1})^2 = a_{n+1}^3,
\]
because $a_{n+1}\ge a_n^2$ implies $a_n\le a_{n+1}^{1/2}$.
Therefore for all sufficiently large $n$,
\[
  b_{n+1} \ge b_n^2.
\]
Fix such an $N$ large enough that this holds for all $n\ge N$.
Then the tail satisfies
\begin{align*}
  0< S-S_N &= \sum_{n=N+1}^\infty \frac{1}{b_n}
  \le \frac{1}{b_{N+1}}\left(1+\frac{1}{b_{N+2}/b_{N+1}}+\frac{1}{b_{N+3}/b_{N+1}}+\cdots\right).
\end{align*}
Since $b_{m+1}\ge b_m^2\ge 2 b_m$ for large $m$, we have $b_{N+j}\ge b_{N+1}^{2^{j-1}}$ for $j\ge 1$.
Thus
\[
  S-S_N \le \frac{1}{b_{N+1}} + \frac{1}{b_{N+1}^2}+\frac{1}{b_{N+1}^4}+\cdots
  < \frac{1}{b_{N+1}}\left(1+\frac{1}{b_{N+1}}+\frac{1}{b_{N+1}^3}+\cdots\right)
  < \frac{2}{b_{N+1}},
\]
for $b_{N+1}\ge 2$.
Using $b_{N+1}\ge b_N^2$ gives
\[
  0< S-S_N < \frac{2}{b_N^2}.
\]

Now suppose, for contradiction, that $S$ is rational: $S=p/q$ in lowest terms.
Choose $N$ so large that $b_N>2q$ and the above error bound holds.
Then
\[
  0< \left|\frac{p}{q}-\frac{p_N}{b_N}\right| = |S-S_N| < \frac{2}{b_N^2}.
\]
But the difference between two distinct rationals with denominators $q$ and $b_N$ is at least
$1/(q b_N)$:
\[
  \left|\frac{p}{q}-\frac{p_N}{b_N}\right| \ge \frac{1}{q b_N}.
\]
Combining gives
\[
  \frac{1}{q b_N} < \frac{2}{b_N^2} \iff b_N < 2q,
\]
contradiction.
Therefore $S$ is irrational.
$\square$

Remark.
Lemma 4.2 shows one concrete meaning of "rapidly": nested denominators and superquadratic growth
force irrationality by very good rational approximations.
However, the original hypothesis $\liminf a_n^{1/2^n}>1$ does not imply the divisibility condition (a)
or the strong pointwise inequality (b).

5) VERIFICATION

- Lemma 4.1 is a direct unpacking of liminf.
- In Lemma 4.2, the key steps are:
  (1) $b_n\mid b_{n+1}$ from $a_n\mid a_{n+1}$,
  (2) $b_{n+1}\ge b_n^2$ from $a_{n+1}\ge a_n^2$,
  (3) lower bound $|p/q - p_N/b_N|\ge 1/(q b_N)$.
  Each is justified explicitly.

6) FINAL

**UNRESOLVED**

(i) Strongest proved partial result.
- The liminf condition implies a uniform lower bound $a_n\ge c^{2^n}$ for all large $n$ (Lemma 4.1).
- Under additional divisibility plus superquadratic growth, the series is irrational (Lemma 4.2).

(ii) First gap (crisp statement).
Show that the liminf condition alone forces irrationality of $\sum 1/(a_n a_{n+1})$, without assuming
any divisibility or pointwise growth inequality like $a_{n+1}\ge a_n^2$.

(iii) Top 3 next moves.
1. Try to prove that the liminf condition implies infinitely many indices $n$ with $a_{n+1}\ge a_n^2$;
   if true, attempt to use these indices to build good rational approximations.
2. Look for a Cantor-series type representation by rewriting the sum in terms of the ratios
   $q_n:=a_{n+1}/a_n$ (which are rational but not necessarily integral) and see whether irrationality
   follows from growth of $q_n$.
3. Attempt to construct (or rule out) a counterexample sequence engineered so that partial sums have
   denominators with a fixed prime set, potentially allowing rationality.

(iv) Minimal counterexample structure.
A counterexample would be a sequence with $a_n\approx c^{2^n}$ in size (to satisfy the liminf condition)
but with carefully arranged arithmetic so that the series admits massive cancellation in a fixed
rational denominator. Such a sequence would likely avoid having $a_{n+1}$ too large relative to $a_n$
for most $n$ (so that Lemma 4.2-type approximation cannot be triggered).


