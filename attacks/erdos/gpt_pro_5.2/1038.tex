% Erdos Problem #1038
% URL: https://www.erdosproblems.com/1038

Determine the infimum and supremum of\[\lvert \{ x\in \mathbb{R} : \lvert f(x)\rvert < 1\}\rvert\]as $f\in \mathbb{R}[x]$ ranges over all non-constant monic polynomials, all of whose roots are real and in the interval $[-1,1]$. A problem of Erd\H{o}s, Herzog, and Piranian \cite{EHP58}, who proved that the measure of the set in question is always at most $2\sqrt{2}$ under the assumption that all the roots are in $\{-1,1\}$, and conjecture this is the best possible upper bound. They also note that the infimum of the set in question is less than $2$, as witnessed by $f(x)=(x+1)(x-1)^m$ for $m\geq 3$. They further note that if the roots are restricted to $[-2,2]$ then the infimum is zero, as witnessed by a small perturbation of the Chebyshev polynomials. They further conjectured that, if the roots are restricted to $[-2,2]$, then\[\lvert \{ x\in \mathbb{R} : \lvert f(x)\rvert < 1\}\rvert\geq n^{-c}\]for an absolute constant $c>0$. This was proved by Pommerenke \cite{Po61}, who in fact showed that this set must contain an interval of width $\gg n^{-4}$. The current best known bounds (see the discussion in the comments) are\[1.519\approx 2^{4/3}-1\leq \inf \leq 1.835\cdots\]and\[\sup = 2\sqrt{2}\approx 2.828.\] References [EHP58] Erd\H{o}s, P. and Herzog, F. and Piranian, G., Metric properties of polynomials . J. Analyse Math. (1958), 125-148. [Po61] Pommerenke, Ch., On metric properties of complex polynomials . Michigan Math. J. (1961), 97-115.


% --- SOLUTION TO ERDOS PROBLEM #1038 ---

FORMAL RESTATEMENT
Let f be a non-constant monic real polynomial whose roots are all real and lie in [-1,1]. Define
  S(f) := { x in R : |f(x)| < 1 }
and let m(f) be the Lebesgue measure (total length) of S(f). Determine
  inf_f m(f)  and  sup_f m(f)
over all such f.

QUICK LITERATURE/CONTEXT CHECK
Only facts explicitly present in the problem statement are treated as established here. In particular:
* Under the extra restriction that all roots lie in {-1,1}, it is stated that m(f) <= 2*sqrt(2) and this upper bound is conjectured sharp.
* For roots restricted to [-2,2], an n^{-c} lower bound (and even an interval of width >> n^{-4}) is stated to be known.
* The problem statement records best-known numeric bounds for the infimum and that sup = 2*sqrt(2).

ATTACK PLAN
(1) Test explicit families f to produce lower bounds for the supremum and upper bounds for the infimum.
(2) For a given explicit f, reduce m(f) to locating the real roots of f(x)^2-1 and summing the lengths of the intervals on which the sign is negative.
(3) Use small-degree numerical checks as sanity.

WORK
FAST REALITY CHECK (small degrees; computed by a minimal script that samples roots on grids / random and computes m(f) via the real roots of f^2-1).
The script outputs:
* degree 1: always m(f)=2.
* degree 2: sampled min approx 2.000000, sampled max approx 2.828427 (achieved at roots -1,1, i.e. f=x^2-1).
* degree 3: sampled min approx 2.000000, sampled max approx 2.735895 (random search; not exhaustive).
Also for the family f(x)=(x+1)(x-1)^m:
* m=3: m(f) approx 1.984305
* m=4: m(f) approx 1.903210
* m=5: m(f) approx 1.876260
(these are numerical; the lemma below gives a rigorous strict inequality m(f)<2 for m=3).

Lemma 1038.1 (Erdos-lemniscate example reaches length 2*sqrt(2)).
For any integer m>=1, let f_m(x) = (x^2-1)^m. Then
  m(f_m) = 2*sqrt(2).

Proof.
We have |f_m(x)|<1 iff |x^2-1|^m <1 iff |x^2-1|<1. The inequality |x^2-1|<1 is equivalent to
  -1 < x^2-1 < 1,
which is equivalent to
  0 < x^2 < 2.
Thus S(f_m) = (-sqrt(2), sqrt(2)) with the two points x=0 and x=\pm sqrt(2) removed/added only at the boundary where |f_m|=1. These boundary points do not affect Lebesgue measure. Hence m(f_m)=2*sqrt(2).

Lemma 1038.2 (a concrete f with m(f)<2).
Let f(x) = (x+1)(x-1)^3. Then m(f) < 2.

Proof.
Write f(x) = (x+1)(x-1)^3 and define the set S = {x: |f(x)|<1}.
We analyze S on the three regions x>1, -1<=x<=1, and x<-1.

(1) Region x>1.  Put y=x-1>0. Then
  |f(x)| = (x+1)(x-1)^3 = (y+2) y^3.
Let y_+ be the unique positive solution to (y+2)y^3=1 (monotone increasing for y>0). Then S\cap(1,\infty) = (1, 1+y_+), and this contributes length y_+.
The equation is y^4+2y^3-1=0. Evaluate at the rationals 71/100 and 718/1000:
  q(y) := y^4+2y^3-1.
Direct computation gives q(71/100) < 0 and q(718/1000) > 0, so 71/100 < y_+ < 718/1000.
In particular, y_+ < 0.718.

(2) Region -1<=x<=1.  Put t=1-x in [0,2], so x=1-t and |x+1|=2-t. Then
  |f(x)| = (2-t) t^3.
Let t_* be the unique solution in (1,2) of (2-t)t^3=1; this is unique because h(t)=(2-t)t^3 satisfies h(1)=1, increases for t in (1,3/2), then decreases to 0 at t=2.
Then h(t)<1 holds for t in [0,1) \cup (t_*,2], i.e. x in (0,1] \cup [-1, 1-t_*).
So the contribution of S\cap[-1,1] has length
  length((0,1)) + length((-1, 1-t_*)) = 1 + (2-t_*).
The equation (2-t)t^3=1 rearranges to p(t)=t^4-2t^3+1=0. Compute p(1839/1000) < 0 and p(1840/1000) > 0; hence 1839/1000 < t_* < 1840/1000.
Therefore 2-t_* < 2-1839/1000 = 161/1000 = 0.161.
So S\cap[-1,1] has length < 1 + 0.161 = 1.161.

(3) Region x<-1.  Put t=1-x >2, so |x+1|=t-2 and |x-1|=t. Then
  |f(x)| = (t-2)t^3.
Let t_+>2 be the unique solution of (t-2)t^3=1. Then S\cap(-\infty,-1) = (1-t_+, -1), and this contributes length t_+-2.
The equation is r(t)=t^4-2t^3-1=0. Compute r(2106/1000) < 0 and r(2107/1000) > 0; hence 2.106 < t_+ < 2.107 and thus t_+-2 < 0.107.

Putting (1)(2)(3) together, we obtain
  m(f) = y_+ + (1 + (2-t_*)) + (t_+-2)
      < 0.718 + 1 + 0.161 + 0.107
      = 1.986.
In particular m(f) < 2.

VERIFICATION
* Lemma 1038.1 is an exact algebraic reduction.
* Lemma 1038.2 reduces the measure computation to three monotone boundary equations and brackets each boundary root by explicit sign evaluations at rationals.

FINAL
UNRESOLVED
(i) Strongest proved partial result here: explicit extremal-type example f_m=(x^2-1)^m gives m(f_m)=2*sqrt(2) (Lemma 1038.1), and an explicit example f(x)=(x+1)(x-1)^3 has m(f)<2 (Lemma 1038.2), showing the global infimum is strictly below 2.
(ii) First gap: determine the exact values of inf_f m(f) and sup_f m(f) under the full root constraint (all real roots in [-1,1]).
(iii) Top 3 next moves:
  1. Attempt to prove a universal upper bound m(f) <= 2*sqrt(2) without the restriction roots in {-1,1}, perhaps by comparison with a symmetrized polynomial or an extremal inequality on level sets.
  2. For the infimum, study candidate extremizers with many roots near 1 and a few near -1 (or vice versa), and attempt to turn the 1.5--1.8 numeric window into rigorous bounds by analytic inequalities.
  3. Computational exploration for fixed degree n: optimize m(f) over root locations (e.g. gradient methods on the roots) to guess extremizers and then attempt to certify via inequalities.
(iv) Minimal counterexample structure (if sup>2*sqrt(2) or inf smaller than expected): a sequence of monic polynomials with roots in [-1,1] whose |f|<1 set either stretches far outside [-sqrt(2),sqrt(2)] (for sup) or concentrates into very short neighbourhoods of roots with large gaps (for inf).

