% Erdos Problem #531
% URL: https://www.erdosproblems.com/531

Let $F(k)$ be the minimal $N$ such that if we two-colour $\{1,\ldots,N\}$ there is a set $A$ of size $k$ such that all subset sums $\sum_{a\in S}a$ (for $\emptyset\neq S\subseteq A$) are monochromatic. Estimate $F(k)$. The existence of $F(k)$ was established by Sanders and Folkman, and it also follows from Rado's theorem. It is commonly known as Folkman's theorem . Erd\H{o}s and Spencer \cite{ErSp89} proved that\[F(k) \geq 2^{ck^2/\log k}\]for some constant $c>0$. Balogh, Eberhrad, Narayanan, Treglown, and Wagner \cite{BENTW17} have improved this to\[F(k) \geq 2^{2^{k-1}/k}.\] References [BENTW17] Balogh, J\'{o}zsef and Eberhard, Sean and Narayanan, Bhargav and Treglown, Andrew and Wagner, Adam Zsolt, An improved lower bound for Folkman's theorem . Bull. Lond. Math. Soc. (2017), 745-747. [ErSp89] Erd\H{o}s, Paul and Spencer, Joel, Monochromatic sumsets . J. Combin. Theory Ser. A (1989), 162-163.

%Erdos problem 531

\subsection*{FORMAL RESTATEMENT}
Fix $k\ge 1$. For $N\ge 1$, a \emph{2-colouring} of $[N]:=\{1,2,\dots,N\}$ is a map $\chi:[N]\to\{0,1\}$. For a finite set $A\subseteq [N]$, define its nonempty subset-sum set
\[
\mathrm{FS}(A):=\Big\{\sum_{a\in S}a:\ \emptyset\ne S\subseteq A\Big\}\subseteq \mathbb N.
\]
We say $A$ is \emph{monochromatic-sum} for $\chi$ if
\begin{enumerate}
\item $\mathrm{FS}(A)\subseteq [N]$ (so every sum is assigned a colour), and
\item $\chi$ is constant on $\mathrm{FS}(A)$.
\end{enumerate}
Define $F(k)$ to be the minimal $N$ (if it exists) such that every 2-colouring of $[N]$ contains some $A\subseteq [N]$ with $|A|=k$ that is monochromatic-sum.

\emph{Ambiguity in the literal statement.} The original phrasing does not explicitly require $\mathrm{FS}(A)\subseteq [N]$, but since colours are only defined on $[N]$, the natural convention is to require it; otherwise the property is ill-posed. I adopt that convention.

\subsection*{QUICK LITERATURE/CONTEXT CHECK}
The statement records large lower bounds for $F(k)$ from \cite{ErSp89,BENTW17}. Per the integrity rule I do not add or use any additional external results.

\subsection*{ATTACK PLAN}
\emph{Proof-track ideas.} (i) Compute exact values for small $k$ by exhaustive search over all colourings (reality check). (ii) Prove structural reductions (monotonicity; necessary constraints on $A$ such as $\sum A\le N$).

\emph{Disproof-track ideas.} For any proposed explicit upper bound on $F(k)$, attempt to construct a colouring of $[N]$ avoiding monochromatic-sum sets of size $k$ using backtracking/SAT.

\subsection*{WORK}
\begin{lemma}[Necessary sum constraint]\label{lem:531-sum}
If $A\subseteq [N]$ satisfies $\mathrm{FS}(A)\subseteq [N]$, then $\sum_{a\in A} a \le N$.
\end{lemma}
\begin{proof}
The full sum $\sum_{a\in A}a$ is an element of $\mathrm{FS}(A)$ (take $S=A$). If $\mathrm{FS}(A)\subseteq [N]$, then in particular $\sum_{a\in A}a\le N$.
\end{proof}

\begin{proposition}[Exact value for $k=2$]\label{prop:531-F2}
Under the convention above (distinct elements in $A$), one has $F(2)=9$.
\end{proposition}
\begin{proof}
For $k=2$, a set $A=\{x,y\}$ with $x<y$ is monochromatic-sum iff $x,y,x+y\in [N]$ and $\chi(x)=\chi(y)=\chi(x+y)$. By Lemma~\ref{lem:531-sum}, necessarily $x+y\le N$.

\emph{Lower bound $F(2)>8$.}
Consider the colouring of $[8]$ given (in order $1,2,\dots,8$) by
\[
BBRBRRRB,
\]
where $B$ means colour $0$ and $R$ means colour $1$.
I checked \emph{all} pairs $1\le x<y\le 8$ with $x+y\le 8$ (there are finitely many) and verified that none satisfy $\chi(x)=\chi(y)=\chi(x+y)$ under this colouring. Hence there is a 2-colouring of $[8]$ with no monochromatic-sum $2$-set, so $F(2)\ge 9$.

\emph{Upper bound $F(2)\le 9$.}
I exhaustively checked all $2^9=512$ colourings of $[9]$; in every case there exists a pair $1\le x<y\le 9$ with $x+y\le 9$ and $\chi(x)=\chi(y)=\chi(x+y)$. Therefore every colouring of $[9]$ contains a monochromatic-sum set of size $2$, so $F(2)\le 9$.

Combining yields $F(2)=9$.
\end{proof}

\paragraph{FAST REALITY CHECK (exact computation for small parameters).}
The bounds in Proposition~\ref{prop:531-F2} were verified by exhaustive search over all colourings of $[N]$ for $N\le 9$ and all candidate pairs $\{x,y\}$ with $x+y\le N$.

For $k=3$, exhaustive search over all colourings becomes infeasible for large $N$ because there are $2^N$ colourings, but I did check the existence of \emph{one} obstruction colouring for each $N\le 25$; in particular,
\[
F(3)>25.
\]
One explicit witness colouring of $[25]$ avoiding monochromatic-sum $3$-sets is (in order $1,2,\dots,25$)
\[
RRBBRBRRRRBBBBBBBBBBBBBBB.
\]
This was verified by enumerating all triples $A=\{a<b<c\}\subseteq [25]$ with $a+b+c\le 25$ (Lemma~\ref{lem:531-sum}) and checking that the seven subset sums in $\mathrm{FS}(A)$ are not all the same colour.

\subsection*{VERIFICATION}
\begin{itemize}
\item Definition check: $\mathrm{FS}(A)$ uses distinct elements, so $k=2$ forbids using $x=x$; this is why $F(2)$ differs from the classical ``monochromatic $x+y=z$'' condition that allows $x=y$.
\item Proposition~\ref{prop:531-F2}: both the lower-bound witness on $[8]$ and the universal property on $[9]$ were established by exhaustive enumeration over finite search spaces.
\item For $F(3)>25$: the verification checks \emph{all} eligible triples for the displayed colouring, so it is exact for that $N$.
\end{itemize}

\subsection*{FINAL}
\textbf{**UNRESOLVED**}

(i) \emph{Strongest proved partial result.} Exact small-$k$ information: $F(2)=9$ (Proposition~\ref{prop:531-F2}) and an explicit computed lower bound $F(3)>25$ with a witness colouring. Additionally, any candidate $A$ must satisfy $\sum A\le N$ (Lemma~\ref{lem:531-sum}).

(ii) \emph{First gap (crisp statement).} Determine $F(3)$ (even within reasonable magnitude), or prove any explicit general upper bound on $F(k)$ as a function of $k$.

(iii) \emph{Top 3 next moves.}
1. Implement SAT/ILP with symmetry-breaking for $k=3$ to push explicit lower-bound colourings to much larger $N$, narrowing the range for $F(3)$.
2. Seek a finitary proof of existence of $F(k)$ with explicit quantitative bounds by encoding the problem as a finite system of partition-regular equations and tracking constants.
3. Attempt an iterative ``finite sums'' construction that produces a monochromatic $\mathrm{FS}(A)$ set while explicitly controlling $\max \mathrm{FS}(A)\le N$, yielding constructive upper bounds.

(iv) \emph{Minimal counterexample structure.} A colouring witnessing large $F(k)$ must avoid monochromaticity on \emph{every} $\mathrm{FS}(A)$ with $|A|=k$, forcing a dense pattern of colour alternation across many additive configurations; for $k=3$ this means avoiding monochromaticity on all $7$-term sets $\mathrm{FS}(\{a,b,c\})$ with $a+b+c\le N$.


