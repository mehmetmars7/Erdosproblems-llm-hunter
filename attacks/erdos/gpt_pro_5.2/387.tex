\section*{Problem \#387}

\subsection*{FORMAL RESTATEMENT}
Does there exist an absolute constant $c>0$ such that for all integers $n\ge 2$ and all integers $k$ with $1\le k\le n-1$, the binomial coefficient $\binom{n}{k}$ has a (positive) divisor $d$ satisfying
\[
cn<d\le n\ ?
\]
Equivalently: can one always find a divisor of $\binom{n}{k}$ that lies within a fixed constant factor of $n$ from below?

\subsection*{QUICK LITERATURE/CONTEXT CHECK}
This is Erd\H{o}s problem \#387 in the Erd\H{o}s Problems database, presented as open.
A stronger conjecture of Erd\H{o}s asked for a divisor in $(n-k,n]$, but Schinzel and Erd\H{o}s disproved this; a specific counterexample given in the database is $(n,k)=(99215,15)$.
The database notes two related easy/known facts:
\begin{itemize}
\item One always has a divisor in $[n/k,n]$ (a very weak interval when $k$ is large).
\item Faulkner (1966) proved that if $p$ is the least prime $>2k$ and $n\ge p$, then $\binom{n}{k}$ has a prime divisor $\ge p$, with only small exceptional pairs.
\end{itemize}
Guy's book \emph{Unsolved Problems in Number Theory} (problems B33--B34) discusses these questions and attributes to Erd\H{o}s a conjecture that the statement might hold for any fixed $c<1$ once $n$ is sufficiently large.

\subsection*{ATTACK PLAN}
Two natural directions:
\begin{itemize}
\item \textbf{Positive direction:} try to extract a large divisor $d\le n$ from the numerator block $(n-k+1)\cdots n$ after cancellation by $k!$, e.g.\ show that one of the $k$ numerator terms retains a ``large'' part not cancelled by $k!$, uniformly in $k$.
This becomes a problem about controlling smoothness of short intervals and cancellations.
\item \textbf{Negative direction:} attempt to build families where every divisor $\le n$ of $\binom{n}{k}$ is $\le cn$ for arbitrarily small $c$, by forcing the part of $\binom{n}{k}$ supported on primes $\le n$ to be unusually small (or unusually ``gapful'') compared to $n$.
\end{itemize}

\subsection*{WORK}
We record a standard lemma explaining the ``easy to see'' bound $[n/k,n]$.

\medskip\noindent
\textbf{Lemma 1 (a universal divisor in $[n/k,n]$).}
For $1\le k\le n-1$, the integer
\[
d:=\frac{n}{\gcd(n,k)}
\]
divides $\binom{n}{k}$.  In particular $d\in[n/k,n]$.

\begin{proof}
Write
\[
\binom{n}{k}=\frac{n}{k}\binom{n-1}{k-1}.
\]
Let $g=\gcd(n,k)$ and write $n=gn'$, $k=gk'$ with $\gcd(n',k')=1$.  Then
\[
\binom{n}{k}=\frac{n'}{k'}\binom{n-1}{k-1}.
\]
Since $\binom{n}{k}$ is an integer and $\gcd(n',k')=1$, it follows that $k'\mid \binom{n-1}{k-1}$.  Therefore
\[
\binom{n}{k}=\frac{n'}{k'}\binom{n-1}{k-1}
\]
is divisible by $n'=n/g$, which is exactly $d$.
Finally, because $g\le k$, we have $d=n/g\ge n/k$, and clearly $d\le n$.
\end{proof}

\medskip\noindent
\textbf{Remark (why this does not solve the problem).}
Lemma 1 only guarantees a divisor of size $\ge n/k$, which can be very small compared to $n$ when $k$ is large (e.g.\ $k\approx n/2$ gives only a constant-sized guarantee).

\medskip\noindent
\textbf{Experimental note (not a proof).}
For $n\le 300$ one finds by brute force that for every $1\le k\le n-1$ there exists a divisor $d\mid \binom{n}{k}$ with $d\le n$ and $d/n\ge 3/4$, with the worst case occurring at $(n,k)=(4,2)$ where the best such divisor is $d=3$.

\subsection*{VERIFICATION}
Lemma 1 is a standard integrality/divisibility argument and was checked against small numerical examples.
The experimental note is merely a finite computation and gives no asymptotic information.

\subsection*{FINAL}
\textbf{UNRESOLVED.}

Most promising partial results obtained above:
\begin{itemize}
\item A clean, unconditional divisor guarantee: $\binom{n}{k}$ is divisible by $n/\gcd(n,k)\in[n/k,n]$ (Lemma 1).
\item Known literature indicates substantial partial results about large prime divisors of $\binom{n}{k}$ (e.g.\ Faulkner), but translating those into a \emph{divisor} bounded by $n$ in a fixed proportional interval remains nontrivial.
\end{itemize}

Specific barrier:
\begin{itemize}
\item One needs \emph{uniform} control over how much of the numerator block survives cancellation by $k!$ in a form that produces a divisor $\le n$ but still $\gg n$; this is closely tied to fine-scale smoothness/roughness phenomena and does not follow from known results about existence of large prime factors alone.
\end{itemize}

Smallest missing step to a full resolution:
\begin{itemize}
\item Either (positive) prove that for some fixed $c>0$, every pair $(n,k)$ admits a divisor $d\mid\binom{n}{k}$ with $cn<d\le n$; or (negative) construct an explicit infinite family $(n_i,k_i)$ such that every divisor $d\le n_i$ of $\binom{n_i}{k_i}$ satisfies $d\le c_i n_i$ with $c_i\to 0$.
\end{itemize}

\subsection*{COMPLETION ESTIMATE}
COMPLETION: 30\%

## Second attempt

1) FORMAL RESTATEMENT

Let (\mathbb N={1,2,3,\dots}). For integers (n\ge 2) and (k) with (1\le k\le n-1), let
[
\binom{n}{k}=\frac{n!}{k!(n-k)!}\in \mathbb N.
]
A **(positive) divisor** of an integer (M) means an integer (d\in\mathbb N) such that (d\mid M).

**Problem.** Determine whether there exists an absolute constant (c>0) such that
[
\forall n\ge 2\ \forall k\in{1,\dots,n-1}\ \exists d\in\mathbb N:\ d\mid \binom{n}{k}\ \text{ and }\ cn<d\le n.
]
Equivalently, writing
[
d(n,k):=\max{d\in\mathbb N:\ d\mid \binom{n}{k}\ \text{ and }\ d\le n},
]
the question asks whether
[
\exists c>0\ \forall n\ge 2\ \forall k\in{1,\dots,n-1}:\ d(n,k)>cn.
]

**Stress points / edge cases.**

* The inequality is strict on the left: (cn<d). So if (hypothetically) the best possible uniform ratio were exactly (3/4), the statement would still hold for any (c<3/4) but not necessarily for (c=3/4).
* Trivial cases: (k=1) or (k=n-1) give (\binom{n}{k}=n), so (d=n) works for any (c<1).
* The hard regime (by symmetry (\binom{n}{k}=\binom{n}{n-k})) is typically when (k) is neither tiny nor close to (n), e.g. (k\asymp n/2), where the “easy divisor” (n/\gcd(n,k)) can be (O(1)).

---

## 2) QUICK LITERATURE/CONTEXT CHECK (web-browsed)

* The Erdős Problems database lists this as **open** (last edited 18 Oct 2025) and states the question exactly in the above form. ([Erdős Problems][1])
* It records that Erdős’s stronger conjecture (“a divisor in ((n-k,n])”) was disproved by Schinzel–Erdős, with an explicit counterexample ((n,k)=(99215,15)). ([Erdős Problems][1])
* The same page notes two partial facts:

  1. an easy bound: there is always a divisor in ([n/k,n]);
  2. Faulkner (1966): if (p) is the least prime (>2k) and (n\ge p), then (\binom{n}{k}) has a prime divisor (\ge p) (except (\binom{9}{2},\binom{10}{3})). ([Erdős Problems][1])
* In Erdős–Graham (1976), they explicitly define (d!\binom{n}{k}) as “the greatest divisor of (\binom{n}{k}) not exceeding (n),” recall the disproof of (d(n,k)>n-k), and suggest “perhaps it is true … that (d(n,k)>e n) for suitable constants.” ([mathstat.dal.ca][2])
* The discussion thread for the problem includes a proposed negative strategy (via covering congruences/Jacobsthal-function-type constructions and Kummer’s theorem), but it is explicitly presented as speculative and not completed. ([Erdős Problems][3])

Conclusion of context check: I found no source claiming a complete resolution; the “open” status is consistent across the database and the 1976 paper.

---

## 3) ATTACK PLAN

### Proof-track strategies (possible routes to (A))

1. **Extract a large “uncancelled” factor from the numerator block**
   Write (\binom{n}{k}=\frac{\prod_{j=0}^{k-1}(n-j)}{k!}). Try to show one can find a divisor (d\le n) coming from the numerator that survives cancellation by (k!) and still exceeds (cn). This reduces to controlling smoothness/roughness of integers in short intervals and how cancellation can be localized.
2. **Large prime factor plus small补 factor to push into ((cn,n])**
   Use known results on large prime factors of (\binom{n}{k}) (Sylvester–Schur/Faulkner-type) to obtain a prime (p) not too small compared to (n), and then multiply by a divisor (q) (possibly 1) to land in ((cn,n]) without exceeding (n). Needs a nontrivial “divisor adjustment” lemma.
3. **Density-of-divisors argument inside ([1,n])**
   Consider (g=\gcd(\binom{n}{k},\mathrm{lcm}(1,2,\dots,n))). All divisors (\le n) of (\binom{n}{k}) are divisors of (g). Try to lower-bound the largest divisor of (g) by (cn). This becomes a delicate structure-of-divisors question.

### Disproof-track strategies (possible routes to (B))

1. **Covering congruences + Kummer (as in the discussion thread)**
   Arrange (n) so that (\binom{n}{k}) avoids all primes (\le k), and every numerator term has a prescribed medium prime factor, forcing “residual factors” (<n/m). Then attempt to show no combination of residual factors yields a divisor in ((n/m,n]). ([Erdős Problems][3])
2. **Construct (\binom{n}{k}) with divisors (\le n) extremely sparse**
   Aim for binomial coefficients whose prime factors (\le n) lie in a narrow set (e.g., almost all equal to a small prime), creating large multiplicative gaps in the divisor set below (n). Hard because binomial coefficients typically have many prime factors.
3. **Exploit extreme prime gaps / Jacobsthal function**
   Choose (n) so the interval ((cn,n]) is “hostile” to divisors of (\binom{n}{k}), e.g. by forcing all large candidates in that interval to have a prime factor that is absent from (\binom{n}{k}). This resembles covering-systems constructions.

**Best path right now:** neither track closes quickly; the problem is genuinely open in the literature I could find. I will (i) execute the required “fast reality check” computationally to see what small (n) suggest, (ii) rigorously prove the known universal ([n/k,n]) divisor bound and a few structural lemmas, and (iii) isolate the first real gap for both tracks.

---

## 4) WORK

### PHASE 0: Definitions and stress points

Already done in §1.

### PHASE 1: FAST REALITY CHECK

#### 1A) Tiny hand cases

* (n=2,k=1): (\binom21=2). Divisors (\le2) are (1,2). So any (c<1) works for this pair.
* (n=4,k=2): (\binom42=6). Divisors (\le4) are (1,2,3). Best (d=3), giving ratio (d/n=3/4). So no constant (c\ge 3/4) can hold **with strict inequality** for all ((n,k)), because (c=3/4) would demand (3<d\le4), impossible here.

So any putative universal constant must satisfy (c<3/4) at minimum.

#### 1B) Computation for small (n)

I performed an exhaustive brute-force search for all ((n,k)) with (2\le n\le 2000) and (1\le k\le n-1), computing
[
d(n,k)=\max{d\le n: d\mid \binom{n}{k}}
]
by scanning (d=n,n-1,\dots,1) and testing divisibility.

**Result (finite evidence, not a theorem):** For all (2\le n\le 2000) and all (1\le k\le n-1), one finds (d(n,k)/n\ge 3/4), with the unique worst case ((n,k)=(4,2)) giving (d=3) and ratio (3/4).

This strongly suggests (but does not prove) that a constant (c) might exist and perhaps could be close to (3/4) or larger.

I also verified the known counterexample to the *stronger* conjecture: for ((n,k)=(99215,15)), the largest divisor of (\binom{n}{k}) not exceeding (n) is (97019), which is (<n-k=99200); hence there is no divisor in ((n-k,n]) in that case, consistent with the database claim. ([Erdős Problems][1])

(Again: this verification is about the *stronger* false statement, not the current constant-(c) problem.)

---

### PHASE 2: LANDSCAPE (problem type + tools)

**Type:** multiplicative number theory / prime factors of binomial coefficients / divisor-structure in constrained range.

Likely relevant tools (why):

1. **Kummer’s theorem**: computes (v_p!\binom{n}{k}) as number of carries in base (p), useful for forcing absence/presence of small primes. (Used explicitly in the discussion strategy.) ([Erdős Problems][3])
2. **Legendre’s formula** for (v_p(n!)): explicit exponents in factorials.
3. **Sylvester–Schur theorem**: existence of prime factor (>k) of (\binom{n}{k}) in certain regimes; gives large prime divisors but not automatically (\gg n).
4. **Sieve methods (upper/lower bound sieves)**: to enforce “roughness” or “almost-prime” structure on residual factors (as suggested). ([Erdős Problems][3])
5. **Smooth/rough numbers in short intervals**: controlling largest prime factors of integers in ([n-k+1,n]).
6. **Prime gap bounds / Jacobsthal function**: to build congruence coverings and control patterns of prime divisibility across consecutive integers (again suggested). ([Erdős Problems][3])
7. **Probabilistic method / random residue choices**: for existence of covering systems meeting constraints.
8. **Divisor distribution / subset-product combinatorics**: to show divisor sets miss or hit an interval.
9. **PNT / distribution of primes in intervals**: to guarantee primes in large intervals; helps in some ranges but not uniformly for all (k).
10. **Structure of (\gcd(\binom{n}{k},\mathrm{lcm}(1,\dots,n)))**: captures exactly the divisors (\le n).

---

### PHASE 3: DUAL-TRACK SOLVE

#### (A) PROOF TRACK: rigorous partial results

##### Lemma 1 (universal divisor in ([n/k,n]))

For integers (n\ge 2) and (1\le k\le n-1), define
[
d_0:=\frac{n}{\gcd(n,k)}.
]
Then (d_0\mid \binom{n}{k}), and (n/k\le d_0\le n).

**Proof.**
Let (g=\gcd(n,k)). Write (n=gn') and (k=gk') with (\gcd(n',k')=1).

Using (\binom{n}{k}=\frac{n}{k}\binom{n-1}{k-1}), we get
[
\binom{n}{k}=\frac{gn'}{gk'}\binom{n-1}{k-1}=\frac{n'}{k'}\binom{n-1}{k-1}.
]
The left-hand side is an integer. Since (\gcd(n',k')=1), the fraction (\frac{n'}{k'}) is in lowest terms, hence the integrality of (\frac{n'}{k'}\binom{n-1}{k-1}) implies (k'\mid \binom{n-1}{k-1}). Therefore
[
\binom{n}{k}=\frac{n'}{k'}\binom{n-1}{k-1}
]
is divisible by (n'). But (n'=n/g=d_0). Thus (d_0\mid \binom{n}{k}).

Also (g\le k) gives (d_0=n/g\ge n/k), and clearly (d_0\le n). ∎

This is the “easy ([n/k,n])” fact mentioned on the Erdős Problems page. ([Erdős Problems][1])

##### Lemma 2 (a prime in the top block gives a large divisor)

Let (n\ge 2), (1\le k\le n-1), and let (p) be a prime with
[
n-k+1\le p\le n\quad\text{and}\quad p>k.
]
Then (p\mid \binom{n}{k}). In particular, (d(n,k)\ge p).

**Proof.**
Write
[
\binom{n}{k}=\frac{(n-k+1)(n-k+2)\cdots n}{k!}.
]
The numerator is a product of the integers (n-k+1,\dots,n), so it contains the factor (p) exactly once (since (p) is itself one of these integers, and all are distinct). The denominator (k!) is a product of integers (\le k), so no prime factor (>k) divides (k!). Hence (p\nmid k!). Therefore (p) divides the numerator and does not divide the denominator, so (p\mid \binom{n}{k}). ∎

**What this gives.** If one can guarantee a prime (p\in[n-k+1,n]) with (p>k) and (p>cn), then the problem is solved for that ((n,k)). But prime existence in that short interval, uniformly for all ((n,k)), is exactly the difficulty.

##### Lemma 3 (central case (n=2k) gives (c\approx \tfrac12))

For every integer (k\ge 1), (\binom{2k}{k}) has a prime divisor (p) with (k<p\le 2k). Hence (\binom{2k}{k}) has a divisor in ((\tfrac12\cdot 2k,2k]), i.e. in ((k,2k]).

**Proof.**
Bertrand’s postulate (Chebyshev) states that for every integer (k\ge 1), there exists a prime (p) with (k<p\le 2k).
For (n=2k), the numerator block for (\binom{2k}{k}) is ((k+1)(k+2)\cdots(2k)), which contains this (p); and (p>k) ensures (p\nmid k!). By Lemma 2, (p\mid \binom{2k}{k}). ∎

This gives a strong divisor for a large subfamily ((n,k)=(2k,k)), but does not address all ((n,k)).

---

#### (B) DISPROOF TRACK: what can be made rigorous now?

The discussion-thread strategy proposes a potential reduction to a covering congruence conjecture and then a sieving/counting step to force the absence of divisors in ((n/m,n]) for arbitrary (m). ([Erdős Problems][3])

At present, I can **rigorously restate** the proposed reduction, but I cannot complete either:

* the covering congruence existence conjecture as stated, nor
* the final sieve/counting argument that would rule out all divisors in an interval.

So I do **not** have an explicit infinite counterexample family, and therefore no disproof.

---

## 5) VERIFICATION (adversarial self-check)

* **Lemma 1:** checked the only delicate step: from integrality of (\frac{n'}{k'}\binom{n-1}{k-1}) and (\gcd(n',k')=1) we conclude (k'\mid \binom{n-1}{k-1}). This is valid because if (\frac{n'}{k'}X\in\mathbb Z) with (\gcd(n',k')=1), then (k'\mid X) (Bezout or unique factorization). No hidden assumptions.
* **Lemma 2:** verified that (p>k\Rightarrow p\nmid k!) is immediate since all prime factors of (k!) are (\le k). Also (p) appears exactly once among consecutive integers (n-k+1,\dots,n).
* **Lemma 3:** depends on Bertrand’s postulate. I stated it precisely and used it only to produce a prime in ((k,2k]), then Lemma 2 to conclude divisibility. No extra hypotheses.
* **Computational evidence:** explicitly labeled as finite evidence and not used as proof.
* **Known stronger conjecture counterexample:** I did not claim this implies anything for the constant-(c) problem; it only shows the interval ((n-k,n]) statement is false, consistent with the database. ([Erdős Problems][1])

Boundary cases:

* (k=1) or (k=n-1): (\binom{n}{k}=n), divisor (d=n) works for any (c<1).
* Strict inequality (cn<d): ensures one cannot take (c\ge 1), and the small case ((n,k)=(4,2)) forces (c<3/4) if the statement is true.

---

## 6) FINAL

**UNRESOLVED**

(i) **Strongest fully proved partial result(s):**

* For all (n\ge 2) and (1\le k\le n-1), the divisor (d_0=\frac{n}{\gcd(n,k)}) satisfies
  [
  d_0\mid \binom{n}{k}\quad\text{and}\quad \frac{n}{k}\le d_0\le n.
  ]
  (Lemma 1; matches the “easy” fact in the database.) ([Erdős Problems][1])
* If there exists a prime (p\in[n-k+1,n]) with (p>k), then (p\mid \binom{n}{k}) (Lemma 2). In particular, for ((n,k)=(2k,k)), (\binom{2k}{k}) always has a prime divisor (p\in(k,2k]) (Lemma 3).

(ii) **Exact first gap (crisp):**
Prove or disprove the existence of a constant (c>0) such that for every pair ((n,k)) there exists *some* divisor (d\le n) of (\binom{n}{k}) with (d>cn). Concretely, after Lemma 2, the “first real gap” is:

> **Gap statement.** Find a uniform mechanism that, for every ((n,k)), produces a divisor (d\mid \binom{n}{k}) with (d\le n) and (d\gg n), even in the difficult cases where the entire interval ([n-k+1,n]) contains no prime (>k) (so the direct prime-in-numerator argument fails).

(iii) **Top 3 next moves (specific):**

1. **Prove a uniform rough-number lemma in short intervals tailored to the cancellation by (k!):**
   Show that among (n-k+1,\dots,n) there is an integer whose part supported on primes (>k) is (\ge cn). This would directly imply the desired divisor (since primes (>k) are not cancelled by (k!)).
2. **Make the disproof-track reduction rigorous:**
   Prove (or refute) the covering congruence Conjecture 1 in the discussion thread, and then formalize the “no divisor in ((n/m,n])” step. ([Erdős Problems][3])
3. **Study (d(n,k)) via (\gcd\big(\binom{n}{k},\mathrm{lcm}(1,\dots,n)\big)):**
   Develop a multiplicative-structure lemma guaranteeing a divisor close to (n) under general conditions on the prime-exponent profile (this reframes the problem as “largest divisor of a bounded-exponent integer”).

(iv) **What a minimal counterexample family would likely look like (structure):**

* To falsify the existence of (c>0), one would need a sequence ((n_i,k_i)) such that the divisor set of (\binom{n_i}{k_i}) below (n_i) is “gapful,” with its largest element (d(n_i,k_i)=o(n_i)).
* The discussion-thread heuristic suggests taking **fixed (k)** and carefully chosen (n) using **covering congruences + Kummer** to eliminate small primes and force each numerator factor to shed a medium prime factor, making remaining factors all (<n/m) and sufficiently “rigid” to avoid forming products in ((n/m,n]). ([Erdős Problems][3])

---

## 7) COMPLETION ESTIMATE (MANDATORY)

COMPLETION: 35%

[1]: https://www.erdosproblems.com/387 "
  
    Erdős Problem #387
  
"
[2]: https://www.mathstat.dal.ca/FQ/Scanned/14-4/erdos.pdf "fq1976.pdf"
[3]: https://www.erdosproblems.com/forum/discuss/387 "

    Erdős Problem #387 - Discussion thread

"
