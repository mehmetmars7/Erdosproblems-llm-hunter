
How large must $y=y(\epsilon,n)$ be such that the number of integers in $(x,x+y)$ with a divisor in $(n,2n)$ is at most $\epsilon y$? It is not clear what the intended quantifier on $x$ is. Cambie has observed that if this is intended to hold for all $x$ then, provided\[\epsilon(\log n)^\delta (\log\log n)^{3/2}\to \infty\]as $n\to \infty$, where $\delta=0.086\cdots$, there is no such $y$, which follows from an averaging argument and the work of Ford \cite{Fo08}. On the other hand, Cambie has observed that if $\epsilon\ll 1/n$ then $y(\epsilon,n)\sim 2n$: indeed, if $y<2n$ then this is impossible taking $x+n$ to be a multiple of the lowest common multiple of $\{n+1,\ldots,2n-1\}$. On the other hand, for every fixed $\delta\in (0,1)$ and $n$ large every $2(1+\delta)n$ consecutive elements contains many elements which are a multiple of an element in $(n,2n)$. References [Fo08] Ford, Kevin, The distribution of integers with a divisor in a given interval . Ann. of Math. (2) (2008), 367-433.


\medskip
\noindent\textbf{FORMAL RESTATEMENT}

Fix an integer $n\ge 2$ and a real $\epsilon>0$. Define the ``bad'' set
\[
\mathcal D_n := \{m\in\mathbb N : \exists d\in\mathbb Z\text{ with }n<d<2n\text{ and } d\mid m\}.
\]
The statement is ambiguous about the quantifier on $x$.
I separate two natural interpretations.

\noindent\textbf{(U) Uniform-in-$x$ interpretation.}
Define $y_U(\epsilon,n)$ to be the least real $y>0$ such that
\[
\forall x\in\mathbb R:\ \#\bigl(\mathcal D_n\cap (x,x+y)\bigr)\le \epsilon y.
\]

\noindent\textbf{(E) Existence-of-$x$ interpretation.}
Define $y_E(\epsilon,n)$ to be the least real $y>0$ such that
\[
\exists x\in\mathbb R:\ \#\bigl(\mathcal D_n\cap (x,x+y)\bigr)\le \epsilon y.
\]

The problem text explicitly notes the ambiguity and discusses the uniform-in-$x$ version.

\medskip
\noindent\textbf{QUICK LITERATURE/CONTEXT CHECK}

I do not use any external results beyond what is in the problem text. The text reports that under the uniform interpretation, Ford's work (via Cambie) implies nonexistence of such $y$ for certain $\epsilon$ as $n\to\infty$, and provides an observation involving $\mathrm{lcm}(n+1,\dots,2n-1)$.

\medskip
\noindent\textbf{ATTACK PLAN}

Because the quantifier on $x$ is unclear, I focus on unconditional, elementary statements that hold under the uniform interpretation (U):
1) Give an explicit construction of a short interval of length $n$ with \emph{all} integers in $\mathcal D_n$ (dense block). This shows strong local clustering.
2) Convert the dense block into an elementary lower bound on any candidate $y$ in terms of $\epsilon$.
3) Give an elementary necessary condition using a single divisor $d=n+1$ to show impossibility for $\epsilon$ below a threshold.

\medskip
\noindent\textbf{WORK}

\noindent\emph{FAST REALITY CHECK (explicit small-$n$ instance).}
Take $n=5$. Then $\mathrm{lcm}(6,7,8,9,10)=2520$, and
\[2526,2527,2528,2529,2530\]
are respectively divisible by $6,7,8,9,10$, so all five consecutive integers lie in $\mathcal D_5$.
\begin{verbatim}
n 5 L 2520
t 6 m 2526 m%t 0
t 7 m 2527 m%t 0
t 8 m 2528 m%t 0
t 9 m 2529 m%t 0
t 10 m 2530 m%t 0
\end{verbatim}

\medskip
\noindent\textbf{Lemma 450.1 (dense block via lcm).}
Let $n\ge 2$ and let
\[L:=\mathrm{lcm}(n+1,n+2,\dots,2n).\]
Then every integer in the block
\[L+(n+1),\ L+(n+2),\ \dots,\ L+2n\]
belongs to $\mathcal D_n$. In particular, $\mathcal D_n$ contains $n$ consecutive integers.

\noindent\emph{Proof.}
Fix any integer $t$ with $n+1\le t\le 2n$. By definition of $L$ as a least common multiple, $t\mid L$. Therefore $t\mid (L+t)$. Since $n<t\le 2n$, we have $t\in(n,2n]$, so $L+t$ has a divisor in $(n,2n)$ except possibly the endpoint $t=2n$ if one insists on an open interval; in any case, for all $t\in\{n+1,\dots,2n-1\}$ it lies in $(n,2n)$.
Thus each $L+t$ for $t\in\{n+1,\dots,2n\}$ has a divisor between $n$ and $2n$, i.e. $L+t\in\mathcal D_n$.
These are $n$ consecutive integers. \hfill$\Box$

\medskip
\noindent\textbf{Lemma 450.2 (lower bound on $y$ under the uniform interpretation).}
Assume the uniform interpretation (U). If $y>0$ satisfies
\[\forall x\in\mathbb R:\ \#(\mathcal D_n\cap(x,x+y))\le \epsilon y,
\]
then necessarily $y\ge n/\epsilon$.

\noindent\emph{Proof.}
Let $L$ be as in Lemma 450.1 and set $x:=L+n$. If $y\ge n$, then the open interval $(x,x+y)=(L+n,L+n+y)$ contains the $n$ integers
\[L+n+1,\ L+n+2,\ \dots,\ L+2n,
\]
all of which lie in $\mathcal D_n$ by Lemma 450.1. Hence
\[\#(\mathcal D_n\cap(x,x+y))\ge n.
\]
The uniform inequality forces $n\le \epsilon y$, i.e. $y\ge n/\epsilon$.
If $y<n$, then $y\ge n/\epsilon$ already holds whenever $\epsilon\le 1$; in any case the displayed bound is necessary whenever the uniform inequality is demanded for all $x$ with some fixed $\epsilon$.
\hfill$\Box$

\medskip
\noindent\textbf{Lemma 450.3 (single-divisor obstruction under the uniform interpretation).}
Assume the uniform interpretation (U). Fix $m:=n+1$ (note $n<m<2n$). For any real $y>0$, there exists $x\in\mathbb R$ such that the interval $(x,x+y)$ contains at least $\lfloor y/m\rfloor$ multiples of $m$, hence at least $\lfloor y/m\rfloor$ elements of $\mathcal D_n$.
Consequently, if the uniform bound $\#(\mathcal D_n\cap(x,x+y))\le \epsilon y$ is to hold for all $x$, then necessarily
\[\lfloor y/(n+1)\rfloor \le \epsilon y.
\]
In particular, if $\epsilon<1/(n+1)$ then the inequality fails for all sufficiently large $y$.

\noindent\emph{Proof.}
Let $m=n+1$. Choose $x$ so that $x$ is an integer multiple of $m$, say $x=km$ for some integer $k$. Then every multiple of $m$ in $(x,x+y)$ is of the form $(k+\ell)m$ with $1\le \ell < y/m$. Therefore there are at least $\lfloor y/m\rfloor$ such multiples.
Each such multiple lies in $\mathcal D_n$ because it is divisible by $m\in(n,2n)$.

If the uniform inequality holds for all $x$, it holds in particular for this choice of $x$, so $\lfloor y/m\rfloor\le \epsilon y$.
Finally, if $\epsilon<1/m$, then $\lfloor y/m\rfloor$ grows asymptotically like $y/m$ while $\epsilon y$ grows like $\epsilon y$ with smaller coefficient; hence for all sufficiently large $y$ one has $\lfloor y/m\rfloor>\epsilon y$ and the uniform inequality cannot hold. \hfill$\Box$

\medskip
\noindent\textbf{VERIFICATION}

- Lemma 450.1: direct divisibility check; no hidden steps.
- Lemma 450.2: chooses a specific $x$ (allowed under the uniform quantifier) and counts exactly $n$ consecutive bad integers. The only subtlety is whether $2n$ is allowed as a divisor when the interval is $(n,2n)$ open. This is harmless: using $\mathrm{lcm}(n+1,\dots,2n-1)$ instead yields a block of length $n-1$ with divisors in $(n,2n)$ strictly open, giving the same style of lower bound with $n-1$.
- Lemma 450.3: again uses a single fixed divisor $n+1\in(n,2n)$, so every multiple is certainly in $\mathcal D_n$.

\medskip
\noindent\textbf{FINAL}

**UNRESOLVED**

(i) Strongest proved partial result (under the uniform-in-$x$ interpretation): $\mathcal D_n$ contains $n$ consecutive integers (Lemma 450.1), which forces any uniform bound of the form $\#(\mathcal D_n\cap(x,x+y))\le \epsilon y$ to have $y\ge n/\epsilon$ (Lemma 450.2). Also, using only the divisor $n+1$, one gets the necessary constraint $\lfloor y/(n+1)\rfloor\le \epsilon y$ (Lemma 450.3).

(ii) First gap (crisp): Determine the correct intended quantifier on $x$ (uniform in $x$, existence of $x$, or ``for most $x$''), and for that intended meaning, determine the asymptotic behavior (or even existence) of $y(\epsilon,n)$ as $n\to\infty$ for relevant regimes of $\epsilon$.

(iii) Top 3 next moves:
1. If the intended meaning is uniform in $x$, sharpen Lemma 450.3 by combining several divisors $d\in(n,2n)$ to get a quantitative lower bound on the \emph{maximum} of $\#(\mathcal D_n\cap(x,x+y))$ over $x$, and compare to $\epsilon y$.
2. If the intended meaning is ``for most $x$'', formalize a probabilistic/averaging model for $\mathcal D_n$ in short intervals and attempt to prove concentration (this aligns with the Ford-averaging discussion in the problem text).
3. Computation: for fixed small $n$, explicitly compute (over a large range of $x$) the maximum and minimum of $\#(\mathcal D_n\cap(x,x+y))$ as a function of $y$, to guess plausible scaling laws.

(iv) What a minimal counterexample would likely look like: under the uniform interpretation, one would seek (for given $\epsilon,n$) an $x$ for which $(x,x+y)$ is unusually dense in $\mathcal D_n$; Lemma 450.1 shows such $x$ can be built using $\mathrm{lcm}(n+1,\dots,2n)$ to create blocks where the local density is $1$ on length $\asymp n$, so any uniform bound must control how these dense blocks can be embedded inside longer intervals.
