
% Erdős Problem #43

\subsection*{Erd\H{o}s Problem \#43}
\noindent\emph{Problem statement (from the problem file).}
If $A,B\subset [N]$ are two Sidon sets such that $(A-A)\cap(B-B)=\{0\}$ then is it true that
\[
 \binom{|A|}{2}+\binom{|B|}{2}\le \binom{f(N)}{2}+O(1),
\]
where $f(N)$ is the maximum possible size of a Sidon set in $[N]$?
If $|A|=|B|$ then can this bound be improved to
\[
\binom{|A|}{2}+\binom{|B|}{2}\le (1-c+o(1))\binom{f(N)}{2}
\]
for some constant $c>0$?

\paragraph{FORMAL RESTATEMENT.}
Fix $N\ge 1$. Let $f(N)$ denote the maximum size of a Sidon set in $[N]$.
Let $A,B\subseteq [N]$ be Sidon sets in $\mathbb{Z}$.
Assume $(A-A)\cap (B-B)=\{0\}$, where $A-A:=\{a-a':a,a'\in A\}$.
Question (1): does there exist an absolute constant $C$ such that for all $N$ and all such $A,B$,
\[
\binom{|A|}{2}+\binom{|B|}{2} \le \binom{f(N)}{2}+C?
\]
Question (2): assuming additionally $|A|=|B|$, does there exist an absolute $c>0$ such that along \emph{all} sufficiently large $N$,
\[
\binom{|A|}{2}+\binom{|B|}{2} \le (1-c+o(1))\binom{f(N)}{2}?
\]
(Here $o(1)$ denotes a term depending on $N$ that tends to $0$ as $N\to\infty$.)

\paragraph{QUICK LITERATURE/CONTEXT CHECK.}
The problem file states: (i) $f(N)\sim \sqrt{N}$, (ii) Tao has an argument proving the $|A|=|B|$ upper bound without the $-c$ improvement, and (iii) Barreto gave a negative answer to Question (2) (existence of infinitely many $N$ and examples with LHS $(1-o(1))\binom{f(N)}{2}$).
I do not assume any further results beyond these statements.

\paragraph{ATTACK PLAN.}
\emph{Disproof track (for Question (2)).} Construct an infinite family of pairs $A,B\subset [N]$ with $|A|=|B|$ and $(A-A)\cap(B-B)=\{0\}$ such that $\binom{|A|}{2}+\binom{|B|}{2} \sim \binom{f(N)}{2}$.
\emph{Proof track (partial, for Question (1) / and Tao's bound).} Prove a robust inequality of the form $|A|^2+|B|^2 \le N + O(\sqrt{N})$ by convolution/Cauchy--Schwarz.
\emph{Reality check.} Exhaustive search for small $N$ to see how large the gap
$\big(\binom{|A|}{2}+\binom{|B|}{2}\big)-\binom{f(N)}{2}$ can be.

\paragraph{WORK.}
\subparagraph{FAST REALITY CHECK (small $N$).}
A brute-force search over all Sidon subsets of $[N]$ for $2\le N\le 16$ found that
\[
\max_{A,B} \Big(\binom{|A|}{2}+\binom{|B|}{2}-\binom{f(N)}{2}\Big)
\in \{0,1,2,3\}\ \text{for}\ N\le 16,
\]
with the maximum difference equalling $3$ for $N\in\{10,11,15,16\}$.
In the equal-size regime, the maximum difference was $2$ for $N\in\{14,15,16\}$.
One explicit equal-size example at $N=8$ is
\[
A=\{1,2,5\},\quad B=\{1,3,8\},
\]
for which $A-A=\{0,\pm 1,\pm 3,\pm 4\}$, $B-B=\{0,\pm 2,\pm 5,\pm 7\}$ so the intersection is $\{0\}$ and
$\binom{|A|}{2}+\binom{|B|}{2}=6=\binom{f(8)}{2}$.

\subparagraph{Lemma 1 (nonzero differences are unique in a Sidon set).}
Let $S$ be a Sidon set in an abelian group $G$ (in particular $G=\mathbb{Z}$ or $\mathbb{Z}/m\mathbb{Z}$).
If $s_1-s_2=s_3-s_4$ and this common difference is nonzero, then $(s_1,s_2)=(s_3,s_4)$.

\emph{Proof.}
Assume $s_1-s_2=s_3-s_4\neq 0$. Rearranging gives $s_1+s_4=s_3+s_2$.
By the Sidon property, the multiset $\{s_1,s_4\}$ equals the multiset $\{s_3,s_2\}$.
There are two cases:
\begin{itemize}
\item If $s_1=s_3$ then necessarily $s_4=s_2$, hence $(s_1,s_2)=(s_3,s_4)$.
\item If $s_1=s_2$ then the common difference is $0$, contradicting the hypothesis.
\end{itemize}
Thus only the first case can occur. \qed

\subparagraph{Lemma 2 (Tao-type $\ell^2$ bound; gives the $1/\sqrt{2}$ upper bound when $|A|=|B|$).}
Let $A,B\subset [N]$ be Sidon and satisfy $(A-A)\cap(B-B)=\{0\}$.
Then for every integer $H\ge 1$,
\[
\|1_A*1_{[1,H]}\|_{\ell^2(\mathbb{Z})}^2+\|1_B*1_{[1,H]}\|_{\ell^2(\mathbb{Z})}^2
\le (|A|^2+|B|^2)H + H^2,
\]
and consequently, for all $H\ge 1$,
\[
(|A|^2+|B|^2)\,\frac{H^2}{N+H}\le (|A|^2+|B|^2)H + H^2.
\]
In particular, choosing $H=\lfloor \sqrt{N}\rfloor$ yields
\[
|A|^2+|B|^2 \le N + O(\sqrt{N}).
\]
If additionally $|A|=|B|$, then
\[
|A| \le \frac{1}{\sqrt{2}}\sqrt{N} + O(1).
\]

\emph{Proof.}
Write $f_A:=1_A$ and $g:=1_{[1,H]}$ as functions on $\mathbb{Z}$.
Define the difference-counting functions
\[
\Delta_A(n):=(f_A*\widetilde f_A)(n)=\sum_{x\in\mathbb{Z}} f_A(x)f_A(x-n)=|\{(a,a')\in A^2:a-a'=n\}|,
\]
and similarly $\Delta_B$. Here $\widetilde f_A(x):=f_A(-x)$.
Also define $w(n):=(g*\widetilde g)(n)=|\{(h,h')\in [1,H]^2:h-h'=n\}|$, so $w(0)=H$ and $\sum_n w(n)=H^2$.

\emph{Step 1: pointwise bound for $\Delta_A+\Delta_B$.}
Since $A$ is Sidon, by Lemma~1 every nonzero difference occurs for at most one ordered pair, hence for $n\neq 0$ we have $\Delta_A(n)\le 1$; likewise $\Delta_B(n)\le 1$.
The hypothesis $(A-A)\cap (B-B)=\{0\}$ implies that for $n\neq 0$, at most one of $\Delta_A(n),\Delta_B(n)$ can be $1$, hence
\[
\Delta_A(n)+\Delta_B(n)\le 1\quad (n\neq 0).
\]
At $n=0$ we have $\Delta_A(0)=|A|$ and $\Delta_B(0)=|B|$, so trivially
\[
\Delta_A(0)+\Delta_B(0)\le |A|^2+|B|^2.
\]
Combining gives the uniform bound
\[
\Delta_A(n)+\Delta_B(n)\le (|A|^2+|B|^2)1_{n=0}+1\quad (\forall n\in\mathbb{Z}).
\]

\emph{Step 2: correlation identity.}
Expanding squares gives
\begin{align*}
\|f_A*g\|_2^2
&= \sum_x \Big(\sum_y f_A(y)g(x-y)\Big)^2
 = \sum_x \sum_{y_1,y_2} f_A(y_1)f_A(y_2) g(x-y_1)g(x-y_2)\\
&= \sum_{y_1,y_2} f_A(y_1)f_A(y_2) \sum_x g(x-y_1)g(x-y_2).
\end{align*}
The inner sum is $(g*\widetilde g)(y_1-y_2)=w(y_1-y_2)$.
Thus
\[
\|f_A*g\|_2^2 = \sum_{y_1,y_2\in A} w(y_1-y_2) = \sum_{n\in\mathbb{Z}} \Delta_A(n)\,w(n).
\]
Similarly $\|f_B*g\|_2^2=\sum_n \Delta_B(n)w(n)$.
Summing and using the pointwise bound yields
\[
\|f_A*g\|_2^2+\|f_B*g\|_2^2
\le (|A|^2+|B|^2)w(0) + \sum_n w(n)
\le (|A|^2+|B|^2)H + H^2.
\]
This proves the first displayed inequality.

\emph{Step 3: Cauchy--Schwarz lower bound.}
The function $f_A*g=1_A*1_{[1,H]}$ is supported on $[2, N+H]$, hence on at most $N+H$ integers.
Also $\|f_A*g\|_1 = |A|H$.
By Cauchy--Schwarz,
\[
\|f_A*g\|_2^2 \ge \frac{\|f_A*g\|_1^2}{|\mathrm{supp}(f_A*g)|} \ge \frac{|A|^2 H^2}{N+H}.
\]
The same bound holds for $B$.
Adding gives the second displayed inequality.

\emph{Step 4: choice of $H$.}
Take $H=\lfloor \sqrt{N}\rfloor$.
A direct rearrangement of
$(|A|^2+|B|^2)\frac{H^2}{N+H}\le (|A|^2+|B|^2)H+H^2$
with $H\asymp \sqrt{N}$ yields $|A|^2+|B|^2\le N + O(\sqrt{N})$.
If $|A|=|B|=m$, then $2m^2\le N+O(\sqrt{N})$, hence $m\le \frac{1}{\sqrt{2}}\sqrt{N}+O(1)$. \qed

\subparagraph{Proposition 3 (explicit infinite family refuting Question (2)).}
For every odd prime power $q$, set
\[
M:=q^2-1,\qquad N:=\frac{M}{2}=\frac{q^2-1}{2}.
\]
Then there exist Sidon sets $A,B\subset [N]$ with $|A|=|B|$ and $(A-A)\cap (B-B)=\{0\}$ such that
\[
\binom{|A|}{2}+\binom{|B|}{2} = \frac{N}{2} + O(N^{3/4}).
\]
Consequently, using the fact $f(N)\sim \sqrt{N}$ from the problem file, one has
\[
\binom{|A|}{2}+\binom{|B|}{2} \ge (1-o(1))\binom{f(N)}{2}
\]
along the infinite sequence $N=(q^2-1)/2$, so no absolute $c>0$ can make the proposed improvement in Question (2) valid for all large $N$.

\emph{Proof (construction and verification).}
\emph{Step 0: a Sidon set in a cyclic group of order $M=q^2-1$.}
Let $\mathbb{F}:=\mathbb{F}_{q^2}$ and let $\mathbb{K}:=\mathbb{F}_q\subset\mathbb{F}$ be the unique subfield of size $q$.
The multiplicative group $\mathbb{F}^*$ is cyclic of order $M=q^2-1$; fix a generator $\theta\in\mathbb{F}^*$.
Pick an element $\alpha\in\mathbb{F}\setminus\mathbb{K}$.
For each $t\in\mathbb{K}$, the element $\alpha+t\in\mathbb{F}^*$, so there exists a unique residue $s(t)\in\mathbb{Z}/M\mathbb{Z}$ such that
\[
\theta^{s(t)}=\alpha+t.
\]
Define the set
\[
S:=\{s(t): t\in \mathbb{K}\}\subset \mathbb{Z}/M\mathbb{Z},\qquad |S|=q.
\]
\emph{Claim: $S$ is Sidon modulo $M$.}
Suppose $s(t_1)+s(t_2)\equiv s(t_3)+s(t_4)\pmod M$.
Applying $\theta^{(\cdot)}$ gives
\[
(\alpha+t_1)(\alpha+t_2)=\theta^{s(t_1)+s(t_2)}=\theta^{s(t_3)+s(t_4)}=(\alpha+t_3)(\alpha+t_4)
\]
in $\mathbb{F}$.
Expanding and cancelling $\alpha^2$ yields
\[
\alpha\,(t_1+t_2-t_3-t_4) + (t_1t_2-t_3t_4)=0.
\]
The coefficient of $\alpha$ lies in $\mathbb{K}$, and the constant term lies in $\mathbb{K}$.
Because $\alpha\notin\mathbb{K}$ and $[\mathbb{F}:\mathbb{K}]=2$, the elements $1,\alpha$ are linearly independent over $\mathbb{K}$, so both coefficients must vanish:
$t_1+t_2=t_3+t_4$ and $t_1t_2=t_3t_4$.
Thus $\{t_1,t_2\}$ and $\{t_3,t_4\}$ are the (multi)set of roots of the same quadratic polynomial $X^2-(t_1+t_2)X+t_1t_2$ over $\mathbb{K}$, implying $\{t_1,t_2\}=\{t_3,t_4\}$.
Hence $S$ is Sidon modulo $M$.

\emph{Step 1: split by parity and obtain large equal-size parts.}
Since $q$ is odd, $M=q^2-1$ is even, so parity is well-defined for integer representatives of residues mod $M$.
Write
\[
S_{\mathrm{even}}:=\{s\in S: s\equiv 0\pmod 2\},\qquad S_{\mathrm{odd}}:=\{s\in S: s\equiv 1\pmod 2\},
\]
with sizes $a:=|S_{\mathrm{even}}|$, $b:=|S_{\mathrm{odd}}|$ so $a+b=q$.
Any difference of two even residues (or two odd residues) is an even residue.
By Lemma~1 applied in $G=\mathbb{Z}/M\mathbb{Z}$, the directed differences from ordered pairs within $S_{\mathrm{even}}$ produce $a(a-1)$ distinct nonzero residues, all even; similarly $S_{\mathrm{odd}}$ produces $b(b-1)$ distinct nonzero even residues.
These two collections are disjoint: if an even difference $d\neq 0$ were produced both by an ordered even-even pair and an ordered odd-odd pair, Lemma~1 would force those ordered pairs to coincide, contradicting parity.
Therefore
\[
a(a-1)+b(b-1)\le \#\{\text{nonzero even residues mod }M\} = N-1.
\]
Compute
\[
a(a-1)+b(b-1)=a^2+b^2-(a+b)=\frac{(a+b)^2+(a-b)^2}{2}-q=\frac{q^2+(a-b)^2}{2}-q.
\]
Comparing with $N-1=(q^2-3)/2$ yields $(a-b)^2\le 2q-3$.
Thus
\[
\min\{a,b\}=\frac{q-|a-b|}{2}\ge \frac{q-\sqrt{2q-3}}{2}=\frac{q}{2}-O(\sqrt{q}).
\]
Let $m:=\min\{a,b\}$, and choose subsets $E\subseteq S_{\mathrm{even}}$, $O\subseteq S_{\mathrm{odd}}$ with $|E|=|O|=m$.

\emph{Step 2: descend to $[N]$ and verify properties.}
Choose representatives of residues so that $E\subset\{0,1,\dots,M-1\}$ consists of even integers and $O\subset\{0,1,\dots,M-1\}$ consists of odd integers.
Define
\[
A_0:=\{e/2: e\in E\}\subset \{0,1,\dots,N-1\},\qquad
B_0:=\{(o-1)/2: o\in O\}\subset \{0,1,\dots,N-1\},
\]
and finally set $A:=A_0+1$, $B:=B_0+1$, so $A,B\subset [N]$ and $|A|=|B|=m$.

\emph{Claim 1: $A$ and $B$ are Sidon in $\mathbb{Z}$.}
We prove it for $A$ (the proof for $B$ is identical).
Suppose $x_1+x_2=x_3+x_4$ with $x_i\in A_0$.
Write $x_i=e_i/2$ with $e_i\in E\subseteq S$.
Then $e_1+e_2=e_3+e_4$ in integers, hence also as a congruence mod $M$.
Since $S$ is Sidon modulo $M$, we get $\{e_1,e_2\}=\{e_3,e_4\}$, hence $\{x_1,x_2\}=\{x_3,x_4\}$.
Thus $A_0$, and hence $A$, is Sidon.

\emph{Claim 2: $(A-A)\cap(B-B)=\{0\}$.}
It suffices to prove $(A_0-A_0)\cap(B_0-B_0)=\{0\}$ since shifting does not change differences.
Let $d\in (A_0-A_0)\cap(B_0-B_0)$.
Then there exist $e_1,e_2\in E$ and $o_1,o_2\in O$ such that
\[
 d=\frac{e_1-e_2}{2}=\frac{o_1-o_2}{2}.
\]
Thus $e_1-e_2=o_1-o_2$ as integers, hence also modulo $M$.
If this common residue were nonzero, Lemma~1 applied to the Sidon set $S$ in $\mathbb{Z}/M\mathbb{Z}$ would force $(e_1,e_2)=(o_1,o_2)$, impossible because $e_i$ are even and $o_i$ are odd.
Hence the common residue is $0$, so $e_1=e_2$ and $o_1=o_2$, and therefore $d=0$.

\emph{Step 3: size estimate and comparison.}
Since $|A|=|B|=m$, we have
\[
\binom{|A|}{2}+\binom{|B|}{2}=2\binom{m}{2}=m(m-1).
\]
With $m=\frac{q}{2}-O(\sqrt{q})$ from Step 1, this gives
\[
 m(m-1)=\frac{q^2}{4}+O(q^{3/2})=\frac{2N+1}{4}+O(N^{3/4})=\frac{N}{2}+O(N^{3/4}).
\]
Using the asymptotic $f(N)\sim \sqrt{N}$ from the problem file,
\[
\binom{f(N)}{2}=\frac{f(N)(f(N)-1)}{2}=\left(\frac{1}{2}+o(1)\right)N.
\]
Thus along $N=(q^2-1)/2$,
$\binom{|A|}{2}+\binom{|B|}{2} \ge (1-o(1))\binom{f(N)}{2}$.
This contradicts the existence of any fixed $c>0$ in Question (2). \qed

\paragraph{VERIFICATION.}
\begin{itemize}
\item \emph{Quantifiers.} The construction in Proposition~3 produces examples for infinitely many $N$ (those of the form $(q^2-1)/2$ with $q$ odd prime power). This is enough to negate the existence of an absolute $c>0$ in Question (2).
\item \emph{Sidon conventions.} Lemma~1 is stated and proved for the Sidon property in an abelian group; it is used only to deduce uniqueness of nonzero directed differences and disjointness of difference collections.
\item \emph{Parity/doubling map.} Because $M=2N$, halving an even residue (resp. $(o-1)/2$ for odd $o$) produces a well-defined integer representative in $\{0,\dots,N-1\}$.
\item \emph{Boundary/small cases.} The explicit $N=8$ example verifies the phenomenon in the small regime; it does not by itself refute an asymptotic statement, but the infinite family does.
\end{itemize}

\paragraph{FINAL.} \textbf{UNRESOLVED.}
\begin{enumerate}
\item[(i)] \emph{Strongest proved partial result.} The Tao-type convolution argument (Lemma~2) gives $|A|^2+|B|^2\le N+O(\sqrt{N})$, hence for $|A|=|B|$ one has $|A|\le \frac{1}{\sqrt{2}}\sqrt{N}+O(1)$. Also, Question (2) is false: Proposition~3 gives infinitely many $N$ with $|A|=|B|$ and $\binom{|A|}{2}+\binom{|B|}{2}\ge (1-o(1))\binom{f(N)}{2}$.
\item[(ii)] \emph{First gap.} Prove or disprove Question (1): does there exist an absolute constant $C$ such that always $\binom{|A|}{2}+\binom{|B|}{2}\le \binom{f(N)}{2}+C$?
\item[(iii)] \emph{Top 3 next moves.}
(1) Attempt to adapt the Erd\H{o}s--Tur\'an upper bound proof for $f(N)$ to a two-colour/difference-disjoint setting to directly compare $\binom{|A|}{2}+\binom{|B|}{2}$ to $\binom{f(N)}{2}$ with an absolute error.
(2) Search computationally for moderate $N$ (e.g. $N\le 40$) for examples where $\binom{|A|}{2}+\binom{|B|}{2}-\binom{f(N)}{2}$ grows, to guess whether the $O(1)$ term might need to be larger (e.g. $O(\log N)$) or whether it plausibly stays bounded.
(3) Try to prove a stability statement: if $\binom{|A|}{2}+\binom{|B|}{2}$ is close to the conjectural maximum, then $A$ and $B$ must arise from (near-)optimal Sidon constructions with a parity split, as in Proposition~3.
\item[(iv)] \emph{Minimal counterexample structure.} A counterexample to Question (1) would be a sequence $N\to\infty$ with Sidon sets $A_N,B_N\subset [N]$, $(A_N-A_N)\cap(B_N-B_N)=\{0\}$, such that the excess
$\binom{|A_N|}{2}+\binom{|B_N|}{2}-\binom{f(N)}{2}$ tends to $+\infty$ (necessarily very slowly if it exists, given small-$N$ data).
\end{enumerate}

