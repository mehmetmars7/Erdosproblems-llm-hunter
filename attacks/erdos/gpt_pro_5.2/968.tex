% Erdos Problem #968

1) FORMAL RESTATEMENT

Let $p_n$ denote the $n$th prime number (with $p_1=2,p_2=3,\dots$) and define
\[
 u_n := \frac{p_n}{n}.
\]

Question 1: Does the set
\[
\{n\in\mathbb N: u_n < u_{n+1}\}
\]
have positive (natural) density, i.e. does
\[
\liminf_{N\to\infty} \frac{1}{N}\bigl|\{1\le n\le N: u_n<u_{n+1}\}\bigr| >0
\]
hold?

Question 2: Are there infinitely many solutions to $u_n<u_{n+1}<u_{n+2}$ and to $u_n>u_{n+1}>u_{n+2}$?

2) QUICK LITERATURE/CONTEXT CHECK

The problem text states (not reproved here) that Erd\H{o}s and Prachar proved
\[
\sum_{p_n<x} |u_{n+1}-u_n| \asymp (\log x)^2
\]
and that the set of $n$ with $u_n>u_{n+1}$ has positive density.

No additional external results are invoked below.

3) ATTACK PLAN

- Rewrite the inequalities $u_n<u_{n+1}$ in terms of prime gaps $g_n:=p_{n+1}-p_n$ to isolate the underlying arithmetic condition.
- Compute the frequency of $u_n<u_{n+1}$ for large initial segments to sanity-check the “positive density” question.

4) WORK

FAST REALITY CHECK (computation for initial segments):

For $N\in\{10,10^2,10^3,10^4,10^5,2\cdot 10^5,5\cdot 10^5,10^6\}$ I computed
\[
\#\{1\le n\le N: u_n<u_{n+1}\}
\]
exactly from the first $N+1$ primes.

\begin{verbatim}
N=10       count=6      ratio=0.600000
N=100      count=60     ratio=0.600000
N=1000     count=475    ratio=0.475000
N=10000    count=4463   ratio=0.446300
N=100000   count=41299  ratio=0.412990
N=200000   count=82697  ratio=0.413485
N=500000   count=201326 ratio=0.402652
N=1000000  count=406140 ratio=0.406140

For N=1,000,000: lt=406140, gt=593859, eq=0.
\end{verbatim}

These computations suggest the set $\{n:u_n<u_{n+1}\}$ has a positive density around $\approx 0.40$ (empirically, up to $10^6$).

Lemma 968.1 (Prime-gap formula for increments).
Let $g_n:=p_{n+1}-p_n$ be the $n$th prime gap. Then
\[
 u_{n+1}-u_n = \frac{g_n-u_n}{n+1}.
\]

Proof.
Compute directly:
\[
 u_{n+1}-u_n = \frac{p_{n+1}}{n+1}-\frac{p_n}{n}
=\frac{n p_{n+1}-(n+1)p_n}{n(n+1)}
=\frac{n(p_{n+1}-p_n)-p_n}{n(n+1)}.
\]
Since $g_n=p_{n+1}-p_n$ and $u_n=p_n/n$, the numerator equals $n g_n - n u_n = n(g_n-u_n)$. Dividing by $n(n+1)$ gives $(g_n-u_n)/(n+1)$. $\square$

Lemma 968.2 (Equivalent inequality conditions).
For each $n\ge 1$,
\[
 u_n<u_{n+1}\iff g_n>u_n,
\qquad
 u_n>u_{n+1}\iff g_n<u_n.
\]

Proof.
By Lemma 968.1, $u_{n+1}-u_n$ has the same sign as $g_n-u_n$ because the factor $1/(n+1)$ is positive. Therefore $u_{n+1}>u_n$ is equivalent to $g_n-u_n>0$, i.e. $g_n>u_n$, and similarly for the strict reverse inequality. $\square$

(Observation) The 3-term monotonicity patterns are equivalently:
\[
 u_n<u_{n+1}<u_{n+2}\iff (g_n>u_n)\ \text{and}\ (g_{n+1}>u_{n+1}),
\]
\[
 u_n>u_{n+1}>u_{n+2}\iff (g_n<u_n)\ \text{and}\ (g_{n+1}<u_{n+1}).
\]
This is immediate from Lemma 968.2 applied twice.

5) VERIFICATION

- Lemma 968.1 is pure algebra; I rechecked the numerator manipulation $n(p_{n+1}-p_n)-p_n = n(g_n-u_n)$ using $u_n=p_n/n$.
- Lemma 968.2 follows by sign comparison; no hidden cases since no equality was assumed.
- Computation: verified there were no ties $u_n=u_{n+1}$ for $n\le 10^6$ in the computed data.

6) FINAL

**UNRESOLVED**

(i) Strongest proved partial result: the exact identity $u_{n+1}-u_n=(g_n-u_n)/(n+1)$ (Lemma 968.1) and hence the equivalence $u_n<u_{n+1}\iff g_n>u_n$ (Lemma 968.2), plus the empirical density estimate $\approx 0.406$ for $n\le 10^6$.

(ii) First gap (crisp): prove a positive lower density of indices $n$ for which the prime gap satisfies $g_n>p_n/n$ (equivalently $u_n<u_{n+1}$), without invoking results not already stated in the problem.

(iii) Top 3 next moves:
1. Translate the Erd\H{o}s--Prachar statement about $\sum |u_{n+1}-u_n|$ into quantitative information about how often $|g_n-u_n|$ is large, using Lemma 968.1.
2. Extend computation to much larger $N$ and also count 3-term patterns $u_n<u_{n+1}<u_{n+2}$ and $u_n>u_{n+1}>u_{n+2}$ to guess whether they might be infinite.
3. Attempt to bound the frequency of $g_n>u_n$ from below using only very soft information (e.g. average-size comparisons between gaps and $u_n$ in ranges), making every step explicit.

(iv) Minimal counterexample structure (to “positive density”): a scenario where for every $\delta>0$ and all sufficiently large $N$, at most $\delta N$ indices $n\le N$ satisfy $g_n>p_n/n$; in other words, almost all prime gaps satisfy $g_n\le p_n/n$.


