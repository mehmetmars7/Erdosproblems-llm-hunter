

\medskip
\noindent\textbf{FORMAL RESTATEMENT}

\smallskip
\noindent
Let $n,k\in\mathbb{Z}$ with $1\le k\le n-1$ and $n\ge 2k\ge 2$. Let
\[
\binom{n}{k}=\frac{n!}{k!(n-k)!}\in\mathbb{Z}_{>1}.
\]
Define $\operatorname{lpf}(m)$ for an integer $m>1$ to be the least (smallest) prime divisor of $m$.

\smallskip
\noindent
\textbf{Claim/conjecture (as stated).}
There exists a finite set $E\subset\{(n,k)\in\mathbb{Z}^2: n\ge 2k\ge 2\}$ such that for all $(n,k)\notin E$,
\[
\operatorname{lpf}\!\left(\binom{n}{k}\right)\le \max\left(\frac{n}{k},\,k\right).
\]
Equivalently: for all but finitely many $(n,k)$ with $n\ge 2k\ge 2$, there exists a prime $p\le \max(n/k,k)$ such that $p\mid\binom{n}{k}$.

\medskip
\noindent\textbf{QUICK LITERATURE/CONTEXT CHECK}

\smallskip
\noindent
The problem text attributes the conjecture (with a specific finite exceptional list of $14$ binomial coefficients) to Erd\H{o}s--Lacampagne--Selfridge, and mentions stronger variants with $\max(n/k,\sqrt{k})$ or $\max(n/k,O(\log k))$ (again with finitely many exceptions). Per the integrity rule for this task, I do \emph{not} assert any literature statements beyond what is explicitly written in the problem file.

\medskip
\noindent\textbf{ATTACK PLAN}

\smallskip
\noindent
\begin{itemize}
\item \textbf{Proof track (partial).} Prove the inequality in infinite families (e.g. $k\mid n$, or when $n$ has a ``medium'' prime factor), and prove the conjecture completely for very small $k$.
\item \textbf{Disproof track.} Search systematically for violations $(n,k)$ with $n\ge 2k$ where every prime $\le\max(n/k,k)$ fails to divide $\binom{n}{k}$; attempt to find patterns that could generate infinitely many violations.
\item \textbf{Reality check.} Exhaustive computation for moderate ranges of $(n,k)$ to see whether violations match the $14$ listed exceptions.
\end{itemize}

\medskip
\noindent\textbf{WORK}

\smallskip
\noindent\textbf{Lemma 1094.1 (a divisibility identity when $k\mid n$).}
If $n=mk$ for some integer $m\ge 2$, then
\[
\frac{n}{k}=m\ \text{divides}\ \binom{n}{k}.
\]
In particular,
\[
\operatorname{lpf}\!\left(\binom{n}{k}\right)\le \operatorname{lpf}(m)\le m=\frac{n}{k}\le \max\left(\frac{n}{k},k\right).
\]

\smallskip
\noindent\emph{Proof.}
Use the standard identity
\[
\binom{n}{k}=\frac{n}{k}\binom{n-1}{k-1}.
\]
If $n=mk$ then $n/k=m\in\mathbb{Z}$, so the right-hand side is $m\binom{n-1}{k-1}$. Since $\binom{n-1}{k-1}\in\mathbb{Z}$, this shows $m\mid\binom{n}{k}$. Any prime divisor of $m$ is a prime divisor of $\binom{n}{k}$, hence $\operatorname{lpf}(\binom{n}{k})\le \operatorname{lpf}(m)\le m=n/k$. \qed

\smallskip
\noindent\textbf{Lemma 1094.2 (a prime factor of $n$ larger than $k$ survives cancellation).}
Let $p$ be a prime with $p\mid n$ and $p>k$. Then $p\mid\binom{n}{k}$. Consequently,
\[
\operatorname{lpf}\!\left(\binom{n}{k}\right)\le p.
\]

\smallskip
\noindent\emph{Proof.}
Write
\[
\binom{n}{k}=\frac{n(n-1)\cdots(n-k+1)}{k!}.
\]
The numerator is divisible by $p$ because $p\mid n$. The denominator $k!$ is not divisible by $p$ because $p>k$ and hence none of $1,2,\dots,k$ is a multiple of $p$. Therefore the reduced fraction still has a factor $p$; since the binomial coefficient is an integer, this means $p\mid\binom{n}{k}$. \qed

\smallskip
\noindent\textbf{Corollary 1094.3 (a sufficient condition for the desired inequality).}
If $n$ has a prime divisor $p$ with
\[
k<p\le \frac{n}{k},
\]
then $\operatorname{lpf}(\binom{n}{k})\le p\le n/k\le \max(n/k,k)$.

\smallskip
\noindent\emph{Proof.}
Lemma 1094.2 gives $p\mid\binom{n}{k}$, so the least prime factor is at most $p$. The displayed inequalities then give the numeric bound. \qed

\smallskip
\noindent\textbf{Lemma 1094.4 (complete verification for $k=2$).}
For every integer $n\ge 4$,
\[
\operatorname{lpf}\!\left(\binom{n}{2}\right)\le \frac{n}{2} = \max\left(\frac{n}{2},2\right).
\]

\smallskip
\noindent\emph{Proof.}
We have
\[
\binom{n}{2}=\frac{n(n-1)}{2}.
\]
If $n$ is even, write $n=2m$ with $m\ge 2$. Then $\binom{n}{2}=m(n-1)$ is divisible by $m$, so $\operatorname{lpf}(\binom{n}{2})\le \operatorname{lpf}(m)\le m=n/2$.

If $n$ is odd, write $n=2m+1$ with $m\ge 2$ (since $n\ge 5$ in this case). Then
\[
\binom{n}{2}=n\cdot\frac{n-1}{2}=(2m+1)\cdot m.
\]
Since $m\ge 2$, $\operatorname{lpf}(\binom{n}{2})\le \operatorname{lpf}(m)\le m=(n-1)/2 < n/2$.
This proves the claim for all $n\ge 4$. \qed

\smallskip
\noindent\textbf{FAST REALITY CHECK (exact computation for $n\le 400$).}

\smallskip
\noindent
I exhaustively checked all pairs $(n,k)$ with $2\le n\le 400$ and $1\le k\le \lfloor n/2\rfloor$. For each pair I computed $\operatorname{lpf}(\binom{n}{k})$ by checking divisibility by primes $p\le n$ using Legendre's formula for $v_p(n!)$.

\smallskip
\noindent
\emph{Result.} Exactly $14$ pairs violate
$\operatorname{lpf}(\binom{n}{k})\le\max(n/k,k)$, and they are precisely the $14$ binomial coefficients listed in the problem statement. The violating pairs and their least prime factors are:
\[
\begin{array}{r|r|r|r}
(n,k) & \max(n/k,k) & \operatorname{lpf}(\binom{n}{k}) & \binom{n}{k} \\\hline
(7,3) & 3 & 5 & 35\\
(13,4) & 4 & 5 & 715\\
(14,4) & 4 & 7 & 1001\\
(23,5) & 5 & 7 & 33649\\
(44,8) & 8 & 11 & 177232627\\
(46,10) & 10 & 11 & 4076350421\\
(47,10) & 10 & 11 & 5178066751\\
(47,11) & 11 & 13 & 17417133617\\
(62,6) & 62/6\approx 10.33 & 19 & 61474519\\
(74,10) & 10 & 11 & 718406958841\\
(94,10) & 10 & 11 & 9041256841903\\
(95,10) & 10 & 11 & 10104934117421\\
(241,16) & 16 & 17 & 3720625555021727122496771\\
(284,28) & 28 & 29 & 408428827130333363893145556340656087569
\end{array}
\]

\medskip
\noindent\textbf{VERIFICATION}

\smallskip
\noindent
\begin{itemize}
\item Lemma 1094.1 uses only the exact identity $\binom{n}{k}=\frac{n}{k}\binom{n-1}{k-1}$ and the integrality of binomial coefficients; no hidden assumptions.
\item Lemma 1094.2 checks cancellation carefully: the only way $p$ could cancel is if $p\mid k!$, which is impossible when $p>k$.
\item For Lemma 1094.4, the parity split covers all $n\ge 4$; the smallest odd case is $n=5$ and is included.
\item Computational check: for each $(n,k)$ the algorithm only needs primes $p\le n$, since every prime divisor of $\binom{n}{k}$ is $\le n$ (because $\binom{n}{k}\mid n!$). This makes the least-prime-factor search exhaustive.
\end{itemize}

\medskip
\noindent\textbf{UNRESOLVED}

\smallskip
\noindent
(i) \emph{Strongest proved partial result here.}
The conjectured inequality holds for all $(n,k)$ with $k\mid n$ (Lemma 1094.1), and more generally whenever $n$ has a prime factor $p$ with $k<p\le n/k$ (Corollary 1094.3). It also holds for all $n\ge 4$ when $k=2$ (Lemma 1094.4). Exhaustive computation verified that for $2\le n\le 400$ the only violations are exactly the $14$ listed exceptions.

\smallskip
\noindent
(ii) \emph{First gap (crisp).}
Prove that outside a finite exceptional set of pairs $(n,k)$ with $n\ge 2k\ge 2$, there exists \emph{some} prime $p\le \max(n/k,k)$ dividing $\binom{n}{k}$. Equivalently, rule out infinitely many pairs for which all primes $p\le \max(n/k,k)$ fail to divide $\binom{n}{k}$.

\smallskip
\noindent
(iii) \emph{Top 3 next moves.}
\begin{enumerate}
\item Use $p$-adic valuations (Legendre/Kummer/Lucas-type digit conditions) to translate ``$p\nmid\binom{n}{k}$'' into constraints on the base-$p$ digits of $n,k$, then try to show these constraints cannot hold simultaneously for all primes $p\le \max(n/k,k)$ except in finitely many cases.
\item Push computation further (e.g. $n$ up to several thousand) to look for any new violations beyond the known list and to search for structural patterns in near-violations.
\item Attack the regime $k\approx \sqrt{n}$, where the bound $\max(n/k,k)$ is smallest (about $\sqrt{n}$), and try to show that $\binom{n}{k}$ must have a prime factor $\le \sqrt{n}$ except in controlled exceptional configurations.
\end{enumerate}

\smallskip
\noindent
(iv) \emph{Minimal counterexample structure.}
An infinite counterexample family would require infinitely many pairs $(n,k)$ with $n\ge 2k$ such that $\binom{n}{k}$ is ``$B$-rough'' with $B:=\max(n/k,k)$, i.e. every prime $p\le B$ has $v_p\!\left(\binom{n}{k}\right)=0$. Since $B$ is minimized near $k\approx \sqrt{n}$, any infinite counterexample would most plausibly occur with $k$ comparable to $\sqrt{n}$ (rather than $k$ fixed or $k\mid n$, where Lemma 1094.1 gives the bound).

