\section*{Problem \#652}

\subsection*{1) FORMAL RESTATEMENT}

Let $n\ge 2$ and let $X=\{x_1,\dots,x_n\}\subset \mathbb{R}^2$ be a set of $n$ distinct points.
For a point $x\in X$ define
\[
R_X(x)\;:=\;\left\lvert \{\,\mathrm{dist}(x,y): y\in X\setminus\{x\}\,\} \right\rvert\in\{1,2,\dots,n-1\},
\]
the number of distinct distances from $x$ to the other points.
Order the points of $X$ so that
\[
R_X(x_{(1)})\le R_X(x_{(2)})\le \cdots \le R_X(x_{(n)}),
\]
where $x_{(k)}$ denotes a point attaining the $k$th smallest value (break ties arbitrarily).
Fix $k\in\mathbb{N}$.

Define
\[
\alpha_k\;:=\;\inf\Bigl\{\,\alpha>0:\exists N_0\ \forall n\ge N_0\ \exists X\subset\mathbb{R}^2,\ \left\lvert X \right\rvert=n,\ \text{with }R_X(x_{(k)})\le \alpha\sqrt n\Bigr\}.
\]
(Using $\le$ instead of $<$ is immaterial asymptotically and avoids the issue that a ``minimum'' need not be attained.)

\medskip
\noindent
\textbf{Question.} Is it true that $\alpha_k\to\infty$ as $k\to\infty$?

\subsection*{2) QUICK LITERATURE/CONTEXT CHECK}

\begin{itemize}[leftmargin=2.2em]
\item A construction of Elekes (``circle grid'') shows that for every fixed $k$ one can realize $R_X(x_{(k)})=O_k(\sqrt n)$, i.e. $\alpha_k<\infty$ for every fixed $k$.
\item The bipartite distinct-distances paper of Mathialagan (EJC 2021) contains a theorem (their Theorem~14) implying a 
\emph{lower bound} of order $\sqrt{k}\,\sqrt n$ for $R_X(x_{(k)})$ once $n$ is sufficiently large compared to $k$.
This yields $\alpha_k=\Omega(\sqrt{k})$, and in particular $\alpha_k\to\infty$.
\end{itemize}

\subsection*{3) ATTACK PLAN}

\begin{enumerate}[leftmargin=2.2em]
\item Let $X$ be an arbitrary $n$-point set and let $P$ be the subset of the $k$ points with smallest $R_X(\cdot)$, i.e. $P=\{x_{(1)},\dots,x_{(k)}\}$. Let $Q=X\setminus P$.
\item Apply the following bipartite theorem: for $\left\lvert P \right\rvert=m$ and $\left\lvert Q \right\rvert=N$ with $m\le N^{1/3}$, there exists $p\in P$ that determines $\Omega(\sqrt{mN})$ distinct distances to $Q$.
\item Since $p\in P$, we have $R_X(p)\le R_X(x_{(k)})$; but also $R_X(p)\ge R_Q(p)$ (distinct distances from $p$ to $Q$).
This will force $R_X(x_{(k)})\ge \Omega(\sqrt{k(n-k)})$.
\item Divide by $\sqrt n$ and choose $n$ large compared to $k^3$ to conclude $\alpha_k\ge c\sqrt{k}$ for an absolute constant $c>0$.
\end{enumerate}

\subsection*{4) WORK}

\paragraph{Step 1: A bipartite lower bound (quoted).}
We quote the following theorem from Mathialagan~\cite[Theorem~14]{Mathialagan2021}.

\begin{quote}
\textbf{Theorem (Mathialagan).}
Let $P,Q\subset\mathbb{R}^2$ be finite point sets with $\left\lvert P \right\rvert=m$ and $\left\lvert Q \right\rvert=n$ satisfying $2\le m\le n^{1/3}$.
Then there exists a point $p\in P$ such that the number of distinct distances from $p$ to points of $Q$ is $\Omega(\sqrt{mn})$.
\end{quote}

\noindent
Here the implied constant in $\Omega(\cdot)$ is absolute.
(For completeness: Mathialagan proves this via a crossing-number argument on a multigraph built from concentric circles centered at points of $P$ passing through points of $Q$; see \cite[pp.~6--7]{Mathialagan2021}.)

\paragraph{Step 2: Apply the bipartite theorem to the $k$ smallest $R$-values.}
Let $X\subset\mathbb{R}^2$ be any $n$-point set and write $x_{(1)},\dots,x_{(n)}$ for the ordering by $R_X(\cdot)$.
Fix $k\ge 2$ and set
\[
P:=\{x_{(1)},\dots,x_{(k)}\},\qquad Q:=X\setminus P.
\]
Then $\left\lvert P \right\rvert=k$ and $\left\lvert Q \right\rvert=n-k$.
Assume that
\begin{equation}\label{eq:mk_condition}
2\le k\le (n-k)^{1/3}.
\end{equation}
Under \eqref{eq:mk_condition}, Mathialagan's theorem yields a point $p\in P$ such that
\begin{equation}\label{eq:lower_bip}
\left\lvert \{\mathrm{dist}(p,q): q\in Q\} \right\rvert\;\ge\; c\sqrt{k(n-k)}
\end{equation}
for some absolute constant $c>0$.

But
\[
R_X(p)\;\ge\;\left\lvert \{\mathrm{dist}(p,q): q\in Q\} \right\rvert,
\]
because the set of distances from $p$ to $Q$ is a subset of the set of distances from $p$ to $X\setminus\{p\}$.
Therefore, from \eqref{eq:lower_bip} we get
\begin{equation}\label{eq:lower_Rp}
R_X(p)\ge c\sqrt{k(n-k)}.
\end{equation}
Since $p\in P$ and $P$ consists of the $k$ points with the smallest $R_X$-values,
\[
R_X(p)\le R_X(x_{(k)}).
\]
Combining this with \eqref{eq:lower_Rp} gives the universal bound
\begin{equation}\label{eq:lower_Rk}
R_X(x_{(k)})\ge c\sqrt{k(n-k)}\qquad\text{whenever }2\le k\le (n-k)^{1/3}.
\end{equation}

\paragraph{Step 3: Convert to a lower bound on $\alpha_k$.}
Fix $k\ge 2$ and choose $n$ sufficiently large so that $n-k\ge k^3$; equivalently $n\ge k^3+k$.
Then $k\le (n-k)^{1/3}$, so \eqref{eq:lower_Rk} applies and yields
\[
R_X(x_{(k)})\ge c\sqrt{k(n-k)}\ge c\sqrt{k}\,\sqrt{n-k}.
\]
For example, if $n\ge 2k^3$ then $n-k\ge n/2$ and we get
\begin{equation}\label{eq:lower_ratio}
\frac{R_X(x_{(k)})}{\sqrt n}\ge \frac{c}{\sqrt 2}\,\sqrt{k}\,.
\end{equation}
Since \eqref{eq:lower_ratio} holds for \emph{every} $n$-point set $X$ once $n\ge 2k^3$,
any constant $\alpha$ for which one can have $R_X(x_{(k)})\le \alpha\sqrt n$ for large $n$ must satisfy
$\alpha\ge (c/\sqrt2)\sqrt{k}$.
Therefore
\[
\alpha_k\ge c'\sqrt{k}\qquad\text{for some absolute }c'>0.
\]
In particular $\alpha_k\to\infty$ as $k\to\infty$.

\subsection*{5) VERIFICATION / FAST REALITY CHECK}

\begin{itemize}[leftmargin=2.2em]
\item \textbf{Sanity check for small $k$.}
For $k=2$ and large $n$, \eqref{eq:lower_Rk} gives $R_X(x_{(2)})\ge c\sqrt{2(n-2)}=\Omega(\sqrt n)$, consistent with the known phenomenon that one can force at least constant$\times\sqrt n$ distinct distances for the second-smallest $R$.
\item \textbf{Scaling.}
If $n\asymp k^3$, then \eqref{eq:lower_Rk} gives $R_X(x_{(k)})=\Omega(k^2)$ while $\sqrt n\asymp k^{3/2}$, so $R_X(x_{(k)})/\sqrt n=\Omega(\sqrt k)$, matching the claimed growth.
\item \textbf{Compatibility with Elekes.}
Elekes' construction yields $R_X(x_{(k)})\le C_k\sqrt n$ with $C_k$ depending on $k$; our lower bound shows $C_k$ must grow at least on the order of $\sqrt k$.
\end{itemize}

\subsection*{6) FINAL}

\textbf{FULL SOLUTION} \\
\textbf{PROVED}

\medskip
\noindent
Using Mathialagan's bipartite theorem (Theorem~14 in \cite{Mathialagan2021}), one obtains the universal bound
$R_X(x_{(k)})\ge c\sqrt{k(n-k)}$ for all $n$-point sets $X$ whenever $k\le (n-k)^{1/3}$.
Taking $n\ge 2k^3$ gives $R_X(x_{(k)})/\sqrt n\ge c'\sqrt{k}$.
Consequently $\alpha_k\ge c'\sqrt{k}$ for an absolute $c'>0$, and hence $\alpha_k\to\infty$ as $k\to\infty$.

\subsection*{7) COMPLETION ESTIMATE}

100\%.


%%%%%%%%%%%%%%%%%%%%%%%%%%%%%%%%%%%%%%%%%%%%%%%%%%%%%%%%%%%%%%%%%%%%%%%%%%%%%%%
