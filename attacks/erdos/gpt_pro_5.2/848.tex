\section*{1. Formal Restatement}

\paragraph{Definitions and conventions.}
Let $N\in\mathbb{N}$ with $N\ge 1$ and write
\[
[N]:=\{1,2,\dots,N\}.
\]
A positive integer $m$ is \emph{squarefree} if there is no prime $p$ such that $p^2\mid m$.
Equivalently, with the M\"obius function $\mu$, one has $m$ is squarefree $\Longleftrightarrow \mu(m)\neq 0$.

For a set $A\subseteq[N]$, define the property
\[
\mathbf{P}(A;N):\qquad \forall a,b\in A,\ \ ab+1\ \text{is not squarefree}.
\]
Note that the quantifier includes the diagonal $a=b$, hence $\mathbf{P}(A;N)$ implies that
\[
\forall a\in A,\ \ a^2+1 \ \text{is not squarefree}.
\]

Define the extremal function
\[
f(N):=\max\{|A|:\ A\subseteq[N]\ \text{and}\ \mathbf{P}(A;N)\}.
\]
Define the model sets
\[
A_7(N):=\{n\in[N]: n\equiv 7\pmod{25}\},\qquad
A_{18}(N):=\{n\in[N]: n\equiv 18\pmod{25}\}.
\]

\paragraph{Literal vs.\ corrected statement.}
The question can be read in two ways:

\begin{itemize}
\item \textbf{(L) Literal for all $N$:} For every $N\ge 1$, the maximum $f(N)$ is achieved by $A_7(N)$, i.e.\ $f(N)=|A_7(N)|$ (possibly with other maximizers).

\item \textbf{(C) Corrected (minimal) asymptotic version:} There exists $N_0$ such that for all $N\ge N_0$,
\[
f(N)=|A_7(N)|,
\]
and moreover any extremal set is contained in $A_7(N)$ or in $A_{18}(N)$.
\end{itemize}

In what follows we prove (C) completely.

\section*{2. Quick Literature/Context Check}

This is a problem of Erd\H{o}s--S\'{a}rk\"{o}zy. A proof of the asymptotic extremal result (for all sufficiently large $N$), together with a stability statement, is given by M.\ Sawhney (arXiv:2511.16072).

\section*{3. Attack Plan}

\begin{itemize}
\item Prove that $A_7(N)$ and $A_{18}(N)$ satisfy $\mathbf{P}(A;N)$ (lower bound construction).
\item Use the diagonal condition $a=b$ to force each $a\in A$ into residue classes modulo $p^2$ satisfying $a^2\equiv -1\pmod{p^2}$ for some prime $p$.
\item Combine a quantitative inclusion--exclusion sieve for unions of residue classes with a sieve for the condition $\mu(ab+1)=0$ in fixed arithmetic progressions.
\item Derive a stability theorem: any set $A$ with density close to $1/25$ must lie in $A_7(N)$ or $A_{18}(N)$, and cannot meet both.
\item Deduce the extremal bound $|A|\le |A_7(N)|$ for all sufficiently large $N$.
\end{itemize}

\section*{4. Work}

\subsection*{4.1. The construction works}

\textbf{Lemma 4.1.}
For every $N$, the sets $A_7(N)$ and $A_{18}(N)$ satisfy $\mathbf{P}(A;N)$.

\textit{Proof.}
If $a\equiv b\equiv 7\pmod{25}$, then $ab\equiv 49\equiv -1\pmod{25}$, hence $25\mid (ab+1)$ and $ab+1$ is not squarefree.
If $a\equiv b\equiv 18\pmod{25}$, then $18\equiv -7\pmod{25}$ so $ab\equiv (-7)^2\equiv -1\pmod{25}$, giving the same conclusion.
\qed

Thus $f(N)\ge |A_7(N)|$ always.

\subsection*{4.2. Congruences for $x^2\equiv -1\pmod{p^2}$}

\textbf{Lemma 4.2.}
For every integer $x$, one has $4\nmid (x^2+1)$.

\textit{Proof.}
If $x$ is even then $x^2\equiv 0\pmod 4$ so $x^2+1\equiv 1\pmod 4$.
If $x$ is odd then $x^2\equiv 1\pmod 4$ so $x^2+1\equiv 2\pmod 4$.
In either case $x^2+1\not\equiv 0\pmod 4$.
\qed

\textbf{Lemma 4.3.}
Let $p$ be an odd prime with $p\equiv 3\pmod 4$. Then $x^2\equiv -1\pmod p$ has no solutions, hence $x^2\equiv -1\pmod{p^2}$ has no solutions.

\textit{Proof.}
If $x^2\equiv -1\pmod p$, then raising to $(p-1)/2$ yields
\[
x^{p-1}\equiv (-1)^{(p-1)/2}\equiv -1\pmod p
\]
since $(p-1)/2$ is odd. But by Fermat, $x^{p-1}\equiv 1\pmod p$ for $p\nmid x$, contradiction. Thus no solution mod $p$, hence none mod $p^2$.
\qed

\textbf{Lemma 4.4.}
Let $p$ be an odd prime with $p\equiv 1\pmod 4$. Then the congruence $x^2\equiv -1\pmod{p^2}$ has exactly two solutions modulo $p^2$.

\textit{Proof.}
By Euler's criterion, $-1$ is a quadratic residue mod $p$ iff $(-1)^{(p-1)/2}\equiv 1\pmod p$, i.e.\ iff $(p-1)/2$ is even, i.e.\ iff $p\equiv 1\pmod 4$.
Thus there exist exactly two solutions $\pm u\pmod p$ to $u^2\equiv -1\pmod p$, with $p\nmid u$.

Fix such a solution $u\pmod p$. Seek a lift $v=u+kp$ with $k\pmod p$ such that $v^2\equiv -1\pmod{p^2}$.
Compute
\[
v^2+1=(u+kp)^2+1=(u^2+1)+2ukp+k^2p^2.
\]
Write $u^2+1=pm$ for some integer $m$. Modulo $p^2$, the condition is
\[
pm+2ukp\equiv 0\pmod{p^2}
\iff m+2uk\equiv 0\pmod p.
\]
Since $\gcd(2u,p)=1$, there is a unique $k\pmod p$ solving this, so $u$ lifts uniquely to a solution mod $p^2$.
The same holds for $-u$, producing exactly two distinct lifts mod $p^2$.
\qed

\textbf{Lemma 4.5.}
One has $25\mid (x^2+1)$ if and only if $x\equiv 7\pmod{25}$ or $x\equiv 18\pmod{25}$.

\textit{Proof.}
Directly check residues modulo $25$: $7^2=49\equiv -1\pmod{25}$ and $18^2=324\equiv -1\pmod{25}$, and these are the only residues with square $\equiv -1\pmod{25}$.
\qed

\subsection*{4.3. Sieve Lemma 1: unions of residue classes in a progression}

\textbf{Lemma 4.6 (Union sieve with power-saving error).}
Let $N\ge 3$. Let $P$ be a finite set of primes with $\max P\le N^{1/2}$.
Let $q\in\mathbb{N}$, fix $t\pmod q$, and for each $p\in P$ let $R_p\subseteq \mathbb{Z}/p^2\mathbb{Z}$ with $|R_p|\le 2$.
Assume $R_p=\varnothing$ whenever $\gcd(p,q)\ne 1$.
Define
\[
S:=\Bigl\{n\in[N]: n\equiv t\pmod q\ \text{and}\ \exists p\in P\ \text{with}\ n\bmod p^2\in R_p\Bigr\}.
\]
Then there is an absolute constant $C>0$ such that for all sufficiently large $N$,
\[
\left|\ |S|-\frac{N}{q}\left(1-\prod_{p\in P}\left(1-\frac{|R_p|}{p^2}\right)\right)\right|
\le C\frac{N}{\sqrt{\log N}}.
\]

\textit{Proof.}
Let $T:=\lfloor\sqrt{\log N}\rfloor$ and split $P=P_{\le T}\cup P_{>T}$.

\emph{Large primes.}
For $p\in P_{>T}$, each residue class mod $p^2$ hits $[N]$ at most $\lceil N/p^2\rceil\le N/p^2+1\le 2N/p^2$ (since $p^2\le N$).
Thus
\[
|S_{>T}|\le \sum_{p\in P_{>T}} |R_p|\cdot\frac{2N}{p^2}\le \sum_{p\in P_{>T}} \frac{4N}{p^2}
\le 4N\sum_{m>T}\frac{1}{m^2}\le \frac{4N}{T}.
\]

\emph{Small primes.}
For $p\in P_{\le T}$ set
\[
A_p:=\{n\in[N]: n\equiv t\pmod q,\ n\bmod p^2\in R_p\}.
\]
Then $S_{\le T}=\bigcup_{p\in P_{\le T}} A_p$.
By inclusion--exclusion,
\[
|S_{\le T}|=\sum_{\varnothing\ne S\subseteq P_{\le T}}(-1)^{|S|-1}\left|\bigcap_{p\in S}A_p\right|.
\]
Fix nonempty $S\subseteq P_{\le T}$ and set $M_S=q\prod_{p\in S}p^2$.
Because $R_p=\varnothing$ when $\gcd(p,q)\ne 1$, every $p\in S$ satisfies $\gcd(p,q)=1$ and the CRT applies.
The conditions $n\equiv t\pmod q$ and $n\bmod p^2\in R_p$ for all $p\in S$ define exactly $\prod_{p\in S}|R_p|$ residue classes modulo $M_S$.
For each such class $c\pmod{M_S}$,
\[
\left|\#\{n\in[N]:n\equiv c\!\!\!\pmod{M_S}\}-\frac{N}{M_S}\right|\le 1.
\]
Hence
\[
\left|\left|\bigcap_{p\in S}A_p\right|-\frac{N}{M_S}\prod_{p\in S}|R_p|\right|
\le \prod_{p\in S}|R_p|\le 2^{|S|}.
\]
Writing the corresponding error as $E_S$ yields
\[
\left|\bigcap_{p\in S}A_p\right|
=\frac{N}{q}\prod_{p\in S}\frac{|R_p|}{p^2}+E_S,\qquad |E_S|\le 2^{|S|}.
\]
Summing $E_S$ over all nonempty $S\subseteq P_{\le T}$ gives total error at most
\[
\sum_{\varnothing\ne S\subseteq P_{\le T}}2^{|S|}
=(1+2)^{|P_{\le T}|}-1=3^{|P_{\le T}|}-1\le 3^{\pi(T)}\le 3^T.
\]
Moreover $3^T=\exp((\log 3)\sqrt{\log N})=o\!\left(\frac{N}{\sqrt{\log N}}\right)$ as $N\to\infty$.
Thus for all sufficiently large $N$,
\[
|S_{\le T}|
=\frac{N}{q}\left(1-\prod_{p\in P_{\le T}}\left(1-\frac{|R_p|}{p^2}\right)\right)+O\!\left(\frac{N}{T}\right).
\]
Finally, completing the product from $P_{\le T}$ to all of $P$ changes the main term by at most
\[
\frac{N}{q}\sum_{p\in P_{>T}}\frac{|R_p|}{p^2}
\le N\cdot 2\sum_{m>T}\frac{1}{m^2}\ll \frac{N}{T}.
\]
Combining with $|S_{>T}|\ll N/T$ yields the claimed estimate with $T=\lfloor\sqrt{\log N}\rfloor$.
\qed

\subsection*{4.4. Sieve Lemma 2: counting $a$ with $\mu(ab+1)=0$ in a progression}

\textbf{Lemma 4.7 (Squarefree sieve in a progression).}
Let $N\ge 3$, let $q$ be a perfect square, fix $t\pmod q$, and let $b$ satisfy $1\le b\le N$.
Assume there is no prime $p$ with $p^2\mid q$ and $p^2\mid (bt+1)$.
Then there exists an absolute constant $C'>0$ such that for all sufficiently large $N$,
\[
\left|
\#\{a\in[N]: a\equiv t\!\!\!\pmod q,\ \mu(ab+1)=0\}
-\frac{N}{q}\left(1-\prod_{\substack{p\\(p,qb)=1}}\left(1-\frac{1}{p^2}\right)\right)
\right|
\le C'\frac{N}{\sqrt{\log N}}.
\]

\textit{Proof.}
Let
\[
S:=\{a\in[N]: a\equiv t\pmod q,\ \mu(ab+1)=0\}.
\]
If $\mu(ab+1)=0$, then there exists a prime $p$ with $p^2\mid (ab+1)$.
We first show any such $p$ satisfies $(p,qb)=1$:
\begin{itemize}
\item If $p\mid b$ then $ab+1\equiv 1\pmod p$, so $p\nmid (ab+1)$, contradiction.
\item If $p\mid q$, then since $q$ is a square we have $p^2\mid q$, and from $a\equiv t\pmod q$ we get $a\equiv t\pmod{p^2}$, hence
\[
ab+1\equiv bt+1\pmod{p^2}.
\]
By hypothesis $bt+1\not\equiv 0\pmod{p^2}$, so $p^2\nmid(ab+1)$, contradiction.
\end{itemize}
Thus $(p,qb)=1$.

Let $T:=\lfloor\sqrt{\log N}\rfloor$.
For each prime $p<T$ with $(p,qb)=1$, since $(p,b)=1$ the congruence $ab+1\equiv 0\pmod{p^2}$ is equivalent to
\[
a\equiv -b^{-1}\pmod{p^2},
\]
i.e.\ exactly one residue class mod $p^2$.
Apply Lemma~4.6 with $P=\{p<T: (p,qb)=1\}$, $|R_p|=1$, and the given progression $a\equiv t\pmod q$.
This yields
\[
\#\{a\in[N]: a\equiv t\!\!\!\pmod q,\ \exists p<T,\ (p,qb)=1,\ p^2\mid (ab+1)\}
=
\frac{N}{q}\left(1-\prod_{\substack{p<T\\(p,qb)=1}}\left(1-\frac{1}{p^2}\right)\right)
+O\!\left(\frac{N}{\sqrt{\log N}}\right).
\]
For primes $p\ge T$, the union bound gives a contribution
\[
\sum_{p\ge T}\left(\frac{N}{p^2}+1\right)
\le N\sum_{m\ge T}\frac{1}{m^2}+\pi(N)
\ll \frac{N}{T}+\pi(N).
\]
Using the classical bound $\pi(x)\ll x/\log x$ (Chebyshev), we have $\pi(N)=o(N/T)$ for $T=\sqrt{\log N}$, hence the contribution of $p\ge T$ is $O(N/\sqrt{\log N})$.
Finally, completing the truncated product to all primes with $(p,qb)=1$ changes the main term by $\ll \frac{N}{q}\sum_{p\ge T}p^{-2}\ll N/T$.
This proves the stated approximation.
\qed

\subsection*{4.5. A tail bound for prime-square products and explicit constants}

\textbf{Lemma 4.8 (Tail bound).}
Let $c\in(0,1]$ and let $L\ge 2c$. Then
\[
\prod_{p>L}\left(1-\frac{c}{p^2}\right)\ge \exp\!\left(-\frac{2c}{L}\right).
\]

\textit{Proof.}
For $0\le x\le 1/2$ one has $\log(1-x)\ge -2x$.
If $p>L\ge 2c$, then $c/p^2\le 1/2$, so
\[
\log\left(1-\frac{c}{p^2}\right)\ge -\frac{2c}{p^2}.
\]
Summing over primes $p>L$ and using $\sum_{p>L}p^{-2}\le \sum_{n>L}n^{-2}\le 1/L$ gives
\[
\log\prod_{p>L}\left(1-\frac{c}{p^2}\right)\ge -2c\sum_{p>L}\frac{1}{p^2}\ge -\frac{2c}{L},
\]
and exponentiating yields the claim.
\qed

\textbf{Lemma 4.9 (Explicit numerical lower bounds).}
Define the infinite products
\[
P_1:=\prod_{\substack{p\equiv 1\ (\mathrm{mod}\ 4)\\ p\ge 13}}\left(1-\frac{2}{p^2}\right),\qquad
P_2:=\prod_{p\ne 2,5}\left(1-\frac{1}{p^2}\right),\qquad
P_3:=\prod_{p\ne 5}\left(1-\frac{1}{p^2}\right).
\]
Then one has the rigorous bounds
\[
P_1\ge 0.9726153779,\qquad
P_2\ge 0.8443269877,\qquad
P_3\ge 0.6332452408.
\]

\textit{Proof.}
Fix $L=10^5$. Compute the partial products over primes $p\le L$ in each definition, and bound the tail $\prod_{p>L}(1-c/p^2)$ by Lemma~4.8 with $c=2$ (for $P_1$) and $c=1$ (for $P_2,P_3$).
This reduces each bound to a finite verification.
\qed

For later use, note the derived upper bounds:
\[
1-P_1\le 0.0273846221,\quad 1-P_2\le 0.1556730123,\quad 1-P_3\le 0.3667547592.
\]
Consequently,
\[
\frac{23}{25}(1-P_1)+\frac{2}{25}(1-P_2)\le 0.0376476933,
\]
\[
\frac{23}{50}(1-P_1)+\frac{1}{50}+\frac{1}{50}(1-P_2)\le 0.0357103864,
\]
\[
\frac{23}{50}(1-P_1)+\frac{1}{25}(1-P_3)+\frac{1}{25}(1-P_2)\le 0.0334940371,
\]
\[
\frac{2}{25}(1-P_3)\le 0.0293403808.
\]

\subsection*{4.6. Stability theorem}

\textbf{Theorem 4.10 (Stability).}
There exist constants $\eta>0$ and $N_0\in\mathbb{N}$ such that for all $N\ge N_0$, if $A\subseteq[N]$ satisfies $\mathbf{P}(A;N)$ and
\[
|A|\ge \left(\frac{1}{25}-\eta\right)N,
\]
then either $A\subseteq A_7(N)$ or $A\subseteq A_{18}(N)$.

\textit{Proof.}
Fix $\eta:=0.001$.
Let the implied constants in Lemmas~4.6 and 4.7 be $C,C'$, and choose $N_0$ large so that for all $N\ge N_0$,
\[
\frac{C}{\sqrt{\log N}}\le 10^{-4},\qquad \frac{C'}{\sqrt{\log N}}\le 10^{-4}.
\]
Fix $N\ge N_0$ and a set $A\subseteq[N]$ satisfying $\mathbf{P}(A;N)$.
Partition
\[
A_7:=\{a\in A: a\equiv 7\pmod{25}\},\quad
A_{18}:=\{a\in A: a\equiv 18\pmod{25}\},\quad
A^*:=A\setminus(A_7\cup A_{18}).
\]

\emph{Step 1: Elements of $A^*$ force a $p\equiv 1\pmod 4$, $p\ge 13$ square divisor of $a^2+1$.}
If $a\in A$, then taking $b=a$ in $\mathbf{P}(A;N)$ gives $\mu(a^2+1)=0$, so $p^2\mid a^2+1$ for some prime $p$.
By Lemma~4.2, $p\ne 2$.
By Lemma~4.3, $p\not\equiv 3\pmod 4$, hence $p\equiv 1\pmod 4$.
If $a\in A^*$, then by Lemma~4.5 we have $25\nmid a^2+1$, so $p\ne 5$.
Thus for each $a\in A^*$ there exists a prime $p\equiv 1\pmod 4$ with $p\ge 13$ such that $a^2\equiv -1\pmod{p^2}$.
By Lemma~4.4, for each such $p$ the congruence $x^2\equiv -1\pmod{p^2}$ has exactly two solutions mod $p^2$.

\emph{Step 2: $A^*$ contains no even integer.}
Assume there exists $b\in A^*$ with $2\mid b$.

\smallskip
\emph{(2a) Bound $|A^*|$.}
For each residue class $t\pmod{25}$ with $t\not\equiv 7,18\pmod{25}$, apply Lemma~4.6 with $q=25$, progression $n\equiv t\pmod{25}$, and primes
\[
P_N:=\{p\le N^{1/2}: p\equiv 1\pmod 4,\ p\ge 13\},
\]
taking $R_p$ to be the two solutions to $x^2\equiv -1\pmod{p^2}$ (so $|R_p|=2$).
Summing over the $23$ allowed residue classes gives
\[
\frac{|A^*|}{N}\le \frac{23}{25}\left(1-\prod_{p\in P_N}\left(1-\frac{2}{p^2}\right)\right)+10^{-4}.
\]
Since $\sum_{p>N^{1/2}}2/p^2\le 2\sum_{m>N^{1/2}}m^{-2}\le 2N^{-1/2}$, the product over $p\in P_N$ differs from $P_1$ by $o(1)$, hence for all sufficiently large $N$,
\[
\frac{|A^*|}{N}\le \frac{23}{25}(1-P_1)+2\cdot 10^{-4}.
\]

\smallskip
\emph{(2b) Bound $|A_7\cup A_{18}|$.}
Let $a\in A_7\cup A_{18}$. Since $b$ is even, $ab+1$ is odd, so $2\nmid(ab+1)$ and in particular $4\nmid(ab+1)$.
Also $25\nmid(ab+1)$: if $a\equiv 7\pmod{25}$ and $25\mid(7b+1)$ then $b\equiv 7\pmod{25}$, contradicting $b\in A^*$; similarly for $a\equiv 18$.
Thus any square prime divisor of $ab+1$ must be $p^2$ with $p\ne 2,5$.
Apply Lemma~4.7 with $q=25$ and $t\in\{7,18\}$ (the hypothesis $25\nmid(bt+1)$ was just verified) to obtain
\[
\frac{|A_7\cup A_{18}|}{N}
\le \frac{2}{25}\left(1-\prod_{p\ne 2,5}\left(1-\frac{1}{p^2}\right)\right)+10^{-4}
= \frac{2}{25}(1-P_2)+10^{-4}.
\]
Adding gives
\[
\frac{|A|}{N}\le \frac{23}{25}(1-P_1)+\frac{2}{25}(1-P_2)+3\cdot 10^{-4}.
\]
Using Lemma~4.9 and the derived bound
\[
\frac{23}{25}(1-P_1)+\frac{2}{25}(1-P_2)\le 0.0376476933,
\]
we obtain $|A|/N<0.038$, contradicting $|A|/N\ge 1/25-\eta=0.039$.
Hence $A^*$ contains no even integer, i.e.\ $A^*\subseteq\{\text{odd}\}$.

\emph{Step 3: $A^*=\varnothing$.}
Assume $A^*\ne\varnothing$ and pick an odd $b\in A^*$.

\smallskip
\emph{(3a) Bound $|A^*|$ using modulus $50$.}
Since $A^*$ is odd and avoids $7,18\pmod{25}$, it lies in $23$ residue classes mod $50$.
Applying Lemma~4.6 with $q=50$ and the same prime-square obstruction as in Step 2a gives
\[
\frac{|A^*|}{N}\le \frac{23}{50}(1-P_1)+2\cdot 10^{-4}.
\]

Now consider $A_7\cup A_{18}$ and split into cases.

\smallskip
\emph{Case 3.1: $A_7\cup A_{18}$ contains no even integer.}
Then all elements of $A_7\cup A_{18}$ are odd.
Thus $A_7$ lies in two residue classes mod $100$ (namely $7$ or $57$), and $A_{18}$ lies in two residue classes mod $100$ (namely $43$ or $93$).
Since $b$ is odd, exactly one of the two classes for $A_7$ yields $ab\equiv 3\pmod 4$ (hence $4\mid ab+1$), and similarly exactly one class for $A_{18}$ yields $4\mid ab+1$.
Therefore at most density $2\cdot (1/100)=1/50$ of $a$ can be covered by the automatic divisor $4$.
For the remaining two mod $100$ classes we have $4\nmid (ab+1)$, and also $25\nmid (ab+1)$ since $b\in A^*$, so any square divisor must be $p^2$ with $p\ne 2,5$.
Applying Lemma~4.7 with $q=100$ to those two classes gives an additional contribution at most
\[
2\cdot \frac{1}{100}(1-P_2)+10^{-4}=\frac{1}{50}(1-P_2)+10^{-4}.
\]
Hence
\[
\frac{|A_7\cup A_{18}|}{N}\le \frac{1}{50}+\frac{1}{50}(1-P_2)+10^{-4}.
\]
Combining with the bound on $|A^*|$ yields
\[
\frac{|A|}{N}\le \frac{23}{50}(1-P_1)+\frac{1}{50}+\frac{1}{50}(1-P_2)+3\cdot 10^{-4}.
\]
Using the explicit inequality
\[
\frac{23}{50}(1-P_1)+\frac{1}{50}+\frac{1}{50}(1-P_2)\le 0.0357103864,
\]
we get $|A|/N<0.036$, contradicting $|A|/N\ge 0.039$.
Thus Case 3.1 is impossible.

\smallskip
\emph{Case 3.2: $A_7\cup A_{18}$ contains an even integer.}
Without loss of generality pick an even $b'\in A_7$.
Apply Lemma~4.7 to bound $|A_7|$ using the odd $b\in A^*$, modulus $q=25$, and $t=7$.
Since $25\nmid (7b+1)$ (else $b\equiv 7\pmod{25}$), the lemma gives
\[
\frac{|A_7|}{N}\le \frac{1}{25}(1-P_3)+10^{-4}.
\]
Next bound $|A_{18}|$ using the even $b'\in A_7$, modulus $q=25$, and $t=18$.
Here $ab'+1$ is odd for all $a$, so $2^2\nmid (ab'+1)$, and also $25\nmid(18b'+1)$ since $b'\equiv 7\pmod{25}$ gives $18\cdot 7+1\equiv 2\pmod{25}$.
Thus the square divisor must come from $p\ne 2,5$, and Lemma~4.7 gives
\[
\frac{|A_{18}|}{N}\le \frac{1}{25}(1-P_2)+10^{-4}.
\]
Adding with the $|A^*|$ bound gives
\[
\frac{|A|}{N}\le \frac{23}{50}(1-P_1)+\frac{1}{25}(1-P_3)+\frac{1}{25}(1-P_2)+3\cdot 10^{-4}.
\]
Using the explicit inequality
\[
\frac{23}{50}(1-P_1)+\frac{1}{25}(1-P_3)+\frac{1}{25}(1-P_2)\le 0.0334940371,
\]
we get $|A|/N<0.034$, contradicting $|A|/N\ge 0.039$.
Thus Case 3.2 is also impossible.

\smallskip
Since both cases are impossible, the assumption $A^*\ne\varnothing$ is false, so $A^*=\varnothing$ and hence
\[
A\subseteq A_7(N)\cup A_{18}(N).
\]

\emph{Step 4: $A$ cannot meet both $A_7(N)$ and $A_{18}(N)$.}
Assume $A_7$ and $A_{18}$ are both nonempty; pick $b\in A_7$ and $b''\in A_{18}$.
Apply Lemma~4.7 with $q=25$, $t=7$, and $b''$.
Since $b''\equiv 18\pmod{25}$, we have $7b''+1\equiv 7\cdot 18+1=127\equiv 2\pmod{25}$, so $25\nmid(7b''+1)$, and the lemma yields
\[
\frac{|A_7|}{N}\le \frac{1}{25}(1-P_3)+10^{-4}.
\]
Similarly applying Lemma~4.7 with $q=25$, $t=18$, and $b\in A_7$ gives
\[
\frac{|A_{18}|}{N}\le \frac{1}{25}(1-P_3)+10^{-4}.
\]
Hence
\[
\frac{|A|}{N}\le \frac{2}{25}(1-P_3)+2\cdot 10^{-4}.
\]
Using $\frac{2}{25}(1-P_3)\le 0.0293403808$, we get $|A|/N<0.03$, contradicting $|A|/N\ge 0.039$.
Thus one of $A_7,A_{18}$ is empty, i.e.\ $A\subseteq A_7(N)$ or $A\subseteq A_{18}(N)$.

This completes the proof.
\qed

\subsection*{4.7. Extremal result for sufficiently large $N$}

\textbf{Corollary 4.11 (Extremal bound for large $N$).}
There exists $N_0$ such that for all $N\ge N_0$, every $A\subseteq[N]$ with $\mathbf{P}(A;N)$ satisfies
\[
|A|\le |A_7(N)|.
\]
Moreover $A_7(N)$ achieves equality, and $A_{18}(N)$ also achieves equality whenever $|A_{18}(N)|=|A_7(N)|$.

\textit{Proof.}
Let $\eta=0.001$ and $N_0$ be as in Theorem~4.10.
Fix $N\ge N_0$ and $A\subseteq[N]$ with $\mathbf{P}(A;N)$.

If $|A|\ge (1/25-\eta)N$, then Theorem~4.10 implies $A\subseteq A_7(N)$ or $A\subseteq A_{18}(N)$, hence $|A|\le \max(|A_7(N)|,|A_{18}(N)|)=|A_7(N)|$.

If $|A|< (1/25-\eta)N$, then for all sufficiently large $N$ we have $|A_7(N)|\ge \lfloor N/25\rfloor\ge N/25-1>(1/25-\eta)N$, so again $|A|\le |A_7(N)|$.

Finally, Lemma~4.1 shows $A_7(N)$ (and $A_{18}(N)$) satisfy $\mathbf{P}(A;N)$, hence achieve the bound.
\qed

\section*{5. Verification Notes}

\begin{itemize}
\item The diagonal condition $a=b$ is used essentially (to force $\mu(a^2+1)=0$ for all $a\in A$). If one removed the diagonal, the argument does not apply.
\item Every use of Lemma~4.7 checks the ``no forced square divisor from $q$'' hypothesis (for $q=25$ this amounts to verifying $25\nmid(bt+1)$; for $q=100$ we also separate the cases where $4\mid (ab+1)$).
\item The explicit constants in Lemma~4.9 reduce to finite computations plus the rigorous tail bound Lemma~4.8.
\end{itemize}

\section*{6. Final Answer (Corrected statement)}

For all sufficiently large $N$, the maximum size of $A\subseteq[N]$ such that $ab+1$ is never squarefree for all $a,b\in A$ is achieved by $A_7(N)=\{n\le N:\ n\equiv 7\pmod{25}\}$ (and also by $A_{18}(N)$ when it has the same size).

\section*{7. Completion Estimate}

COMPLETION: 100\%.
