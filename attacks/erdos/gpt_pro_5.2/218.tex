\section*{Problem \#218 (prime gaps: comparisons of consecutive gaps)}

\subsection*{1. Formal restatement}
Let $p_n$ denote the $n$th prime and set
\[
 d_n := p_{n+1}-p_n \qquad (n\ge 1).
\]
Define subsets of the positive integers:
\[
A:=\{n\ge 1: d_{n+1}\ge d_n\},\quad
B:=\{n\ge 1: d_{n+1}\le d_n\},\quad
E:=\{n\ge 1: d_{n+1}=d_n\}.
\]
For a set $S\subseteq\mathbb N$ the \emph{(natural) density} is
\[
\delta(S):=\lim_{N\to\infty}\frac{|S\cap\{1,2,\dots,N\}|}{N}
\]
if the limit exists.

The problem statement (as given) asserts:
\begin{enumerate}[label=(\alph*)]
\item $\delta(A)=\tfrac12$ and $\delta(B)=\tfrac12$;
\item $E$ is infinite (infinitely many $n$ with $d_{n+1}=d_n$);
\item (Erd\H{o}s, 1985) for every $k\ge 1$ there exists $n$ such that
\[d_n=d_{n+1}=\cdots=d_{n+k}.
\]
\end{enumerate}

\medskip
\noindent\textbf{Minimal clarifications / equivalences.}
\begin{itemize}
\item The sets $A$ and $B$ always satisfy $A\cup B=\mathbb N$ and $A\cap B=E$.
\item If both $\delta(A)$ and $\delta(B)$ exist and equal $\tfrac12$, then necessarily $\delta(E)=0$ (because $\delta(A)+\delta(B)=1+\delta(E)$).
\item The condition $d_n=d_{n+1}=\cdots=d_{n+k}$ is equivalent to saying that
$p_n,p_{n+1},\dots,p_{n+k+1}$ form an arithmetic progression (and in fact are \emph{consecutive} primes).
A short proof is given below.
\end{itemize}

\subsection*{2. Quick literature/context check (browsing available)}
The Erd\H{o}s Problems site lists this as \emph{OPEN} and records the formulation above, including the remark that the system $d_n=\cdots=d_{n+k}$ is equivalent to consecutive primes in arithmetic progression.\footnote{\url{https://www.erdosproblems.com/218}}

In the associated discussion thread (comments on the same page), T. Tao notes that it ``should be possible'' (via sieve methods) to show the set $E$ has density $0$; this is not presented there as a completed refereed proof.\footnote{\url{https://www.erdosproblems.com/218?forum=1}}

No published proof of the full density-$\tfrac12$ claims is supplied on that page, and I did not locate (in a quick browse) a definitive resolution; so I treat the main statements as open.

\subsection*{3. Attack plan}
A plausible route to the density claims would be:
\begin{enumerate}[label=(\roman*)]
\item Prove $\delta(E)=0$ (density zero of equal consecutive gaps). A natural idea is to bound the count of indices $n\le N$ with $d_{n+1}=d_n$ by an upper-bound sieve for prime constellations of type $\{0,d,2d\}$ (plus a treatment of large $d$ via a Markov/averaging argument).
\item Show that the ``strict'' sets $\{n: d_{n+1}>d_n\}$ and $\{n: d_{n+1}<d_n\}$ have equal density (or at least both have density $\tfrac12$ after removing $E$). This would likely require some distributional symmetry of pairs $(d_n,d_{n+1})$ which is far beyond current unconditional technology.
\item For the long-run conjecture $d_n=\cdots=d_{n+k}$ for all $k$, relate it to prime $k$-tuple / consecutive-primes-in-AP phenomena (stronger than Green--Tao, which gives APs of primes but not consecutiveness).
\end{enumerate}

\subsection*{4. Work}
\subsubsection*{4.1. Fast reality check (small $n$)}
The first primes are $2,3,5,7,11,13,17,19,\dots$ giving gaps
\[
(d_1,d_2,d_3,d_4,d_5,d_6,\dots)=(1,2,2,4,2,4,2,\dots).
\]
Then $d_2\ge d_1$ (true), $d_3\ge d_2$ (equality), and $d_4\ge d_3$ (strict). So $A$ is nonempty and $E$ already occurs at $n=2$.

\subsubsection*{4.2. Equivalence: runs of equal gaps $\Longleftrightarrow$ consecutive primes in AP}
\begin{lemma}
Fix $k\ge 1$. For an index $n\ge 1$, the following are equivalent:
\begin{enumerate}[label=(\alph*)]
\item $d_n=d_{n+1}=\cdots=d_{n+k}=d$ for some integer $d\ge 1$.
\item The primes $p_n,p_{n+1},\dots,p_{n+k+1}$ satisfy $p_{n+j}=p_n+j d$ for all $0\le j\le k+1$; i.e. they form an arithmetic progression of length $k+2$.
\end{enumerate}
In particular, they are consecutive primes lying in that progression.
\end{lemma}

\begin{proof}
(a)$\Rightarrow$(b): By definition $p_{n+1}=p_n+d$, and if $p_{n+j+1}-p_{n+j}=d$ for $0\le j\le k$, then summing gives
\[
 p_{n+m}=p_n+\sum_{j=0}^{m-1}(p_{n+j+1}-p_{n+j})=p_n+md
\]
for every $1\le m\le k+1$.

(b)$\Rightarrow$(a): If $p_{n+j}=p_n+j d$, then
\[
 d_{n+j}=p_{n+j+1}-p_{n+j}=(p_n+(j+1)d)-(p_n+j d)=d
\]
for $0\le j\le k$.
\end{proof}

\subsubsection*{4.3. A density identity (elementary)}
\begin{proposition}
Assume the natural densities $\delta(A)$ and $\delta(B)$ exist. Then $\delta(E)$ exists and
\[
\delta(A)+\delta(B)=1+\delta(E).
\]
In particular, if $\delta(A)=\delta(B)=\tfrac12$, then $\delta(E)=0$.
\end{proposition}

\begin{proof}
For every $n$, at least one of $d_{n+1}\ge d_n$ or $d_{n+1}\le d_n$ holds, so $A\cup B=\mathbb N$. Moreover, both hold iff $d_{n+1}=d_n$, so $A\cap B=E$.
Hence for each $N$,
\[
|A\cap[1,N]|+|B\cap[1,N]|=|[1,N]|+|E\cap[1,N]|=N+|E\cap[1,N]|.
\]
Divide by $N$ and let $N\to\infty$.
\end{proof}

\subsubsection*{4.4. Numerical experiment (finite data)}
I computed the first $10^6$ prime gaps and measured the frequencies of
\(d_{n+1}>d_n\), \(d_{n+1}<d_n\), and \(d_{n+1}=d_n\). The results (for comparisons among the first $N$ gaps) were:

\begin{center}
\begin{tabular}{r|ccc}
$N$ & $\frac{\#\{d_{n+1}>d_n\}}{N}$ & $\frac{\#\{d_{n+1}<d_n\}}{N}$ & $\frac{\#\{d_{n+1}=d_n\}}{N}$\\
\hline
$10^3$ & $0.480$ & $0.461$ & $0.059$\\
$10^4$ & $0.489$ & $0.462$ & $0.049$\\
$10^5$ & $0.482$ & $0.481$ & $0.038$\\
$5\cdot 10^5$ & $0.4837$ & $0.4826$ & $0.0337$\\
$10^6$ & $0.484183$ & $0.483839$ & $0.031978$\\
\end{tabular}
\end{center}

Equivalently, including equality, one finds
$\#\{d_{n+1}\ge d_n\}/N\approx 0.516$ and $\#\{d_{n+1}\le d_n\}/N\approx 0.516$ at $N=10^6$, with equality occurring about $3.2\%$ of the time in this range.
These computations are only heuristic evidence and do not address existence of densities.

\subsection*{5. Verification}
\begin{itemize}
\item The equivalence lemma is a direct telescoping-sum check and has no hidden cases.
\item The density identity uses only set algebra $A\cup B=\mathbb N$ and $A\cap B=E$.
\item The numeric table is consistent with the identity
\(\#\{\ge\}+\#\{\le\}=N+\#\{=\}\).
\end{itemize}

\subsection*{6. Final}
\begin{center}
\fbox{\parbox{0.92\linewidth}{\textbf{UNRESOLVED.}\\
I did not obtain a complete proof or counterexample to the density-$\tfrac12$ statements or to the infinitude of $E$, and I did not resolve the stronger ``arbitrarily long runs'' conjecture.\\[6pt]
\textbf{(i) Strongest fully proved partial result:} The identity $A\cup B=\mathbb N$, $A\cap B=E$ implies $\delta(A)+\delta(B)=1+\delta(E)$ whenever these densities exist; in particular, the simultaneous claims $\delta(A)=\delta(B)=\tfrac12$ force $\delta(E)=0$. Also, $d_n=\cdots=d_{n+k}$ is equivalent to $k+2$ consecutive primes in an arithmetic progression.\\[6pt]
\textbf{(ii) First gap that blocks completion:} Showing $\delta(E)=0$ (or even proving $E$ is infinite) appears to require nontrivial upper bounds for the count of consecutive 3-term prime arithmetic progressions (or related prime-constellation counts) in a range, beyond what I derived here.\\[6pt]
\textbf{(iii) Top 3 next moves:} (1) Try to implement the sieve/averaging strategy sketched in the forum discussion to prove $\delta(E)=0$; (2) look for any known results on sign changes/oscillation of consecutive prime gaps and whether they yield lower bounds for $\#\{d_{n+1}>d_n\}$ and $\#\{d_{n+1}<d_n\}$; (3) study ``chains of consecutive prime gaps'' literature for any distributional symmetry of $(d_n,d_{n+1})$.\\[6pt]
\textbf{(iv) What a minimal counterexample would look like:} To refute $\delta(A)=\tfrac12$ one would need to show a persistent bias in comparisons $d_{n+1}\ge d_n$ versus $d_{n+1}\le d_n$ that does not vanish as $n\to\infty$ (equivalently, a positive-density bias in strict inequalities after removing the equality set).}}
\end{center}

\subsection*{7. Completion estimate}
\textbf{10\%}. I can prove basic identities and equivalences and compute evidence, but the core density assertions seem to require deep, currently unavailable distributional control of consecutive prime gaps.


%%%%%%%%%%%%%%%%%%%%%%%%%%%%%%%%%%%%%%%%%%%%%%%%%%%%%%%%%%%%%%%%%%%%%%%%%%%%%%%
