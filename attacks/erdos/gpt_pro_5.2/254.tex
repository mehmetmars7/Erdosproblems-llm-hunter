\section*{Erd\H{o}s Problem \#254}

\subsection*{1) FORMAL RESTATEMENT}
Let $A\subseteq \mathbb{N}$. Write $A(x):=\left|A\cap[1,x]\right|$ for $x\ge 1$ (with $[1,x]$ understood as $\{n\in\mathbb{N}:1\le n\le x\}$).
Assume:
\begin{align}
A(2x)-A(x) &\xrightarrow[x\to\infty]{} \infty,\label{eq:dyadic_growth}\\
\sum_{n\in A} \{\theta n\} &= \infty\quad\text{for every }\theta\in(0,1),\label{eq:theta_div}
\end{align}
where $\{y\}$ denotes the distance from $y$ to the nearest integer (so $\{y\}\in[0,1/2]$).

Conclusion to prove:
There exists $N_0$ such that every integer $N\ge N_0$ can be expressed as a sum of \emph{distinct} elements of $A$.

\subsection*{2) QUICK LITERATURE/CONTEXT CHECK}
Cassels proved the conclusion under stronger hypotheses: instead of \eqref{eq:dyadic_growth} he assumed
\[
\frac{A(2x)-A(x)}{\log\log x}\to\infty,
\]
and instead of \eqref{eq:theta_div} he assumed $\sum_{n\in A} \{\theta n\}^2=\infty$ for every $\theta\in(0,1)$ \cite{Ca60}. The version stated above is listed as open in \cite{ErdosProblems254}.

\subsection*{3) ATTACK PLAN}
\textbf{Track A (prove):}
Attempt to extract from $A$ a ``complete'' increasing subsequence $(a_j)$ satisfying a coin-representation criterion such as $a_{j+1}\le 1+\sum_{i\le j}a_i$ after some point, or prove directly that the subset-sum set
\[
\Sigma(A):=\Big\{\sum_{a\in F} a\;:\;F\subseteq A\text{ finite}\Big\}
\]
contains all integers beyond some threshold. A Fourier-analytic approach is natural: if the exponential sums $\sum_{n\in A} e(\theta n)$ are small in aggregate, subset sums should spread.

\textbf{Track B (disprove):}
Try to build $A$ satisfying \eqref{eq:dyadic_growth}--\eqref{eq:theta_div} but for which $\Sigma(A)$ misses infinitely many integers (e.g. misses an arithmetic progression). Condition \eqref{eq:theta_div} blocks simple congruence obstructions (like $A\subseteq q\mathbb{N}$), so any counterexample would need a subtler obstruction.

\subsection*{4) WORK}
\subsubsection*{4.1 Immediate consequences of the hypotheses}
\paragraph{Infinitude and dyadic thickness.}
Condition \eqref{eq:dyadic_growth} implies that for every $M$ there exists $x_M$ such that for all $x\ge x_M$ the interval $(x,2x]$ contains at least $M$ elements of $A$. In particular $A$ is infinite.

\paragraph{No fixed modulus obstruction; in particular $\gcd(A)=1$.}
Fix an integer $q\ge2$ and take $\theta=1/q$. For $n\in\mathbb{N}$, the quantity $\{n/q\}$ equals $0$ exactly when $q\mid n$, and otherwise $\{n/q\}\ge 1/q$.
If $A$ contained only finitely many integers not divisible by $q$, then $\sum_{n\in A}\{n/q\}$ would be finite, contradicting \eqref{eq:theta_div}.
Therefore, for every $q\ge2$, the set $A$ contains infinitely many integers not divisible by $q$.
In particular, $\gcd(A)=1$.

\subsubsection*{4.2 Relation to Cassels' theorem}
Cassels' result \cite{Ca60} proves the desired completeness conclusion under two strengthened assumptions:
\begin{enumerate}
\item Dyadic growth strengthened from $A(2x)-A(x)\to\infty$ to $(A(2x)-A(x))/\log\log x\to\infty$.
\item Divergence strengthened from $\sum \{\theta n\}=\infty$ to $\sum \{\theta n\}^2=\infty$ for all $\theta\in(0,1)$.
\end{enumerate}
The implication ``$\sum \{\theta n\}=\infty\Rightarrow\sum \{\theta n\}^2=\infty$'' is false in general (bounded terms can have divergent sum but convergent sum of squares), so Cassels' hypotheses do not follow formally from \eqref{eq:dyadic_growth}--\eqref{eq:theta_div}.

\subsubsection*{4.3 Where a naive attempt stalls}
A standard sufficient condition for a sequence $1\le a_1<a_2<\cdots$ to be complete (every sufficiently large integer is a sum of distinct terms) is the ``greedy'' inequality
\[
 a_{j+1} \le 1+\sum_{i\le j} a_i\qquad\text{for all large }j,
\]
which ensures that consecutive representable intervals expand without gaps.
Condition \eqref{eq:dyadic_growth} guarantees many elements of $A$ in each dyadic interval, but does not obviously give control on the relative sizes needed to force such a greedy inequality for some subsequence.

Similarly, the divergence condition \eqref{eq:theta_div} suggests $A$ is not concentrated near multiples of $1/\theta$ for any fixed $\theta$, but converting this into uniform smallness of Fourier coefficients for $A$ (which would help show subset sums fill intervals) requires quantitative estimates, whereas \eqref{eq:theta_div} is qualitative.

\subsection*{5) VERIFICATION (adversarial check)}
\begin{itemize}
\item The deduction that $A$ has infinitely many elements not divisible by each $q\ge2$ is rigorous and uses only $\theta=1/q$.
\item The claim that Cassels' hypotheses are stronger and not implied by \eqref{eq:dyadic_growth}--\eqref{eq:theta_div} is justified by the general fact that divergence of $\sum x_n$ for bounded $x_n\ge0$ does not force divergence of $\sum x_n^2$.
\item No step here assumes equidistribution or any deep estimate; the point is precisely that such quantitative input is missing.
\end{itemize}

\subsection*{6) FINAL}
\textbf{UNRESOLVED.}
\begin{enumerate}
\item[(i)] \textbf{Most advanced partial result proved here.}
From the stated hypotheses one can deduce $\gcd(A)=1$ and that $A$ meets every dyadic interval $(x,2x]$ in arbitrarily large finite sets.
\item[(ii)] \textbf{Strongest obstruction / missing lemma.}
What is missing is a mechanism that converts the qualitative divergence condition \eqref{eq:theta_div} into a quantitative additive-structure conclusion about the subset-sum set $\Sigma(A)$ (for example, a bound forcing long intervals inside $\Sigma(A)$). Cassels' theorem achieves this using stronger growth and a square-divergence condition.
\item[(iii)] \textbf{Next concrete steps.}
One concrete direction is to attempt a ``quantitative upgrade'' of \eqref{eq:theta_div} to a lower bound on the average of $\{\theta n\}$ over $A\cap[1,x]$ for many $x$ (uniform in $\theta$), then feed that into Fourier analysis of subset sums. Another is to attempt to extract a complete subsequence from $A$ using the dyadic thickness to enforce a greedy-type inequality.
\item[(iv)] \textbf{Counterexample search description.}
A potential counterexample would be a set $A$ built in blocks so that $A(2x)-A(x)\to\infty$ and for each fixed $\theta$ the terms $\{\theta n\}$ do not decay too fast along $A$, yet the subset sums avoid infinitely many integers. Any construction must avoid congruence obstructions (ruled out by \eqref{eq:theta_div} at rational $\theta$), so the obstruction would likely be more subtle (e.g. structured gaps in the subset-sum semigroup).
\end{enumerate}

\subsection*{7) COMPLETION ESTIMATE}
$\mathbf{90\%}$ confidence that the deductions recorded here are correct; $\mathbf{0\%}$ confidence of a full resolution (the stated form is open).

\hrule

\begin{thebibliography}{99}

\bibitem{ErdosProblems254}
T.~F.~Bloom,
\emph{Erd\H{o}s Problem \#254},
\url{https://www.erdosproblems.com/254}
(accessed 2026-01-17).

\bibitem{Ca60}
J.~W.~S.~Cassels,
\emph{On the representation of integers as the sums of distinct summands taken from a fixed set},
Acta Sci. Math. (Szeged) (1960), 111--124.

\bibitem{Er52}
P.~Erd\H{o}s,
\emph{Problem 4493},
Amer. Math. Monthly (1952), 557--558.

\bibitem{Er58b}
P.~Erd\H{o}s,
\emph{Sur certaines s\'eries \`a valeur irrationnelle},
Enseign. Math. (2) (1958), 93--100.

\bibitem{Er88c}
P.~Erd\H{o}s,
\emph{On the irrationality of certain series: problems and results},
in \emph{New advances in transcendence theory} (Durham, 1986), (1988), 102--109.

\bibitem{FLC07}
J.~B.~Friedlander, F.~Luca, and M.~Stoiciu,
\emph{On the irrationality of a divisor function series},
Integers (2007).

\bibitem{Pr22}
K.~Pratt,
\emph{The irrationality of a divisor function series of Erd\H{o}s and Kac},
arXiv:2209.11124 (2022).

\bibitem{ScPu06}
J.~C.~Schlage-Puchta,
\emph{The irrationality of a number theoretical series},
Ramanujan J. (2006), 455--460.

\end{thebibliography}

