
\subsection*{FORMAL RESTATEMENT}
Let $\mathbb{N}=\{1,2,3,\dots\}$ and define
\[A := \left\{ \sum_{n\in S} n! \ :\  S\subset \mathbb{N}\ \text{finite}\right\} \subseteq \mathbb{Z}_{\ge 0},\]
where the empty sum is $0$.

Fix an integer $k\ge 2$.
\begin{itemize}
\item Question 1: Does $A$ contain only finitely many $k$th powers, i.e. integers of the form $m^k$ with $m\in\mathbb{Z}_{\ge 0}$?
\item Question 2: Does $A$ contain only finitely many \emph{powerful} numbers?
Here I use the standard convention that a (positive) integer $n$ is powerful (squarefull) if for every prime $p\mid n$ we have $p^2\mid n$.
\end{itemize}

\subsection*{QUICK LITERATURE/CONTEXT CHECK}
The problem text notes that it is open even whether there are infinitely many squares of the form $1+n!$. It also cites a result of Brindza and Erd\H{o}s: for any fixed $r$, if $n_1!+\cdots+n_r!$ is powerful then the smallest factorial index $n_1$ is bounded in terms of $r$.
I do not use any results not explicitly stated in the problem file.

\subsection*{ATTACK PLAN}
\emph{Proof-track strategies.}
\begin{itemize}
\item Use $p$-adic valuation: compare $v_p\left(\sum_{n\in S} n!\right)$ with constraints forced by being a $k$th power or powerful.
\item Work modulo $(m+1)!$: because all terms $n!$ with $n\ge m+1$ vanish mod $(m+1)!$, the residue of an element of $A$ mod $(m+1)!$ is controlled by finitely many ``digits'' $\in\{0,1\}$.
\item Attempt to show that if $\sum_{n\in S} n!$ is a large $k$th power then there must be a ``carry'' phenomenon in factorial base that is incompatible with having only $0/1$ digits.
\end{itemize}

\emph{Disproof-track strategies.}
\begin{itemize}
\item Search computationally for large squares/cubes/powerful values in $A$ and attempt to detect a pattern that could generate infinitely many.
\item Try to engineer identities of the shape $(x\pm y)^2 = \sum_{n\in S} n!$ by forcing the right-hand side to approximate a square with controlled error using factorial growth.
\end{itemize}

\subsection*{WORK}
\paragraph{FAST REALITY CHECK (computation).}
I enumerated all subset sums of $\{1!,2!,\dots,m!\}$ for $m\le 12$ (so $|A\cap[0,(13)! )|=2^{12}=4096$ elements) and checked for perfect powers and powerful (squarefull) numbers.

Up to $\sum_{n=1}^{12} n! = 522956313 < 13!$, the set $A$ contains exactly $15$ squares:
\[
0,1,9,25,121,144,729,841,5041,5184,45369,46225,363609,403225,3674889.
\]
Their $0/1$ factorial representations (as a set of indices $S$) are:
\begin{itemize}
\item $9=1!+2!+3!$;
\item $25=1!+4!$;
\item $121=1!+5!$;
\item $144=4!+5!$;
\item $729=1!+2!+3!+6!$;
\item $841=1!+5!+6!$;
\item $5041=1!+7!$;
\item $5184=4!+5!+7!$;
\item $45369=1!+2!+3!+7!+8!$;
\item $46225=1!+4!+5!+6!+7!+8!$;
\item $363609=1!+2!+3!+6!+9!$;
\item $403225=1!+4!+8!+9!$;
\item $3674889=1!+2!+3!+6!+7!+8!+10!$.
\end{itemize}
For cubes, up to the same bound $A$ contains exactly $5$ cubes:
\[0,1,8,27,729\]
with $8=2!+3!$, $27=1!+2!+4!$, and $729=1!+2!+3!+6!$.
For fifth powers up to the same bound, $A$ contains $0,1,32$ with $32=2!+3!+4!$.

These computations do not settle finiteness; they only show that there are already many small squares/powers in $A$.

\paragraph{Lemma 1108.1 (a basic factorial-sum inequality).}
For every integer $m\ge 2$,
\[\sum_{j=1}^{m-1} j! < m!.\]

\paragraph{Proof.}
For $1\le j\le m-1$ we have
\[(m-1)! = j!\cdot (j+1)(j+2)\cdots(m-1).\]
If $j\le m-2$ then each factor in $(j+1)(j+2)\cdots(m-1)$ is at least $2$, and there are $(m-1)-(j+1)+1 = m-j-1$ such factors. Therefore
\[j!\le \frac{(m-1)!}{2^{m-j-1}}\qquad (1\le j\le m-2),\]
and trivially $(m-1)!\le (m-1)!$ for $j=m-1$.
Summing gives
\[\sum_{j=1}^{m-1} j! \le (m-1)!\left(1+\sum_{t=1}^{m-2} 2^{-t}\right) < (m-1)!\left(1+\sum_{t=1}^{\infty} 2^{-t}\right) = 2(m-1)!.\]
If $m=2$ then $2(m-1)!=2=2!$ and the left-hand side equals $1! =1$, so the strict inequality holds.
If $m\ge 3$ then $2(m-1)!\le m(m-1)! = m!$, and since the inequality above is strict we get $\sum_{j=1}^{m-1} j! < m!$.
\hfill $\square$

\paragraph{Lemma 1108.2 (injectivity of $0/1$ factorial sums).}
If $S,T\subset\mathbb{N}$ are finite and
\[\sum_{n\in S} n! = \sum_{n\in T} n!,\]
then $S=T$. Equivalently, the map $S\mapsto \sum_{n\in S} n!$ from finite subsets of $\mathbb{N}$ to $\mathbb{Z}_{\ge 0}$ is injective.

\paragraph{Proof.}
Assume for contradiction that $S\ne T$. Let
\[m:=\max(S\triangle T),\]
where $S\triangle T$ is the symmetric difference.
Then exactly one of $S,T$ contains $m$. Without loss of generality assume $m\in S\setminus T$.
Rewrite the difference of the sums as
\[0=\sum_{n\in S} n! - \sum_{n\in T} n! = m! + \sum_{n\in S\cap\{1,\dots,m-1\}} n! - \sum_{n\in T\cap\{1,\dots,m-1\}} n!.\]
Thus
\[m! = \sum_{n\in T\cap\{1,\dots,m-1\}} n! - \sum_{n\in S\cap\{1,\dots,m-1\}} n!.
\]
Taking absolute values and using the triangle inequality gives
\[m! \le \sum_{n\in T\cap\{1,\dots,m-1\}} n! + \sum_{n\in S\cap\{1,\dots,m-1\}} n! \le 2\sum_{j=1}^{m-1} j!.
\]
By Lemma 1108.1, $\sum_{j=1}^{m-1} j! < m!$, so the right-hand side is strictly less than $2m!$.
However, we can argue more sharply: each of the two sums on the right is at most $\sum_{j=1}^{m-1} j! < m!$, hence their difference has absolute value strictly less than $m!$.
But the equation above asserts that this difference equals $m!$, a contradiction.
Therefore $S=T$. \hfill $\square$

\paragraph{Lemma 1108.3 (counting $A$ below $(m+1)!$; density zero).}
For every integer $m\ge 1$,
\[\left|A\cap\{0,1,\dots,(m+1)!-1\}\right| = 2^m.
\]
In particular, $A$ has natural density $0$ in $\mathbb{Z}_{\ge 0}$.

\paragraph{Proof.}
Any subset $S\subseteq\{1,2,\dots,m\}$ produces an element $\sum_{n\in S} n!\in A$.
By Lemma 1108.2 these $2^m$ sums are all distinct.
Also, each such sum is $< (m+1)!$ because it is at most $\sum_{n=1}^{m} n!$ and $\sum_{n=1}^{m} n! < (m+1)!$ follows from Lemma 1108.1 applied to $m+1$.
So we get at least $2^m$ elements of $A$ in $[0,(m+1)!-1]$.

Conversely, if $S$ contains an index $\ge m+1$ then $\sum_{n\in S} n!\ge (m+1)!$.
Therefore every element of $A$ lying in $[0,(m+1)!-1]$ must come from a subset of $\{1,\dots,m\}$, giving at most $2^m$ possibilities.
Hence the count is exactly $2^m$.

For density: take $X=(m+1)!-1$. Then
\[\frac{|A\cap[0,X]|}{X+1} = \frac{2^m}{(m+1)!} \to 0\qquad (m\to\infty),\]
so the natural density is $0$. \hfill $\square$

\subsection*{VERIFICATION}
\begin{itemize}
\item The definition of $A$ includes $0$ (empty subset). This contributes a single trivial $k$th power ($0^k$) and does not affect finiteness questions.
\item Lemma 1108.2 uses Lemma 1108.1 via the inequality $\sum_{j=1}^{m-1} j! < m!$; Lemma 1108.1 was proved with an explicit geometric-series bound.
\item The computation enumerated all $2^m$ subset sums for $m\le 12$, so within the range $[0,13!)$ the reported lists are exhaustive.
\end{itemize}

\subsection*{FINAL}
\textbf{UNRESOLVED}

(i) \textbf{Strongest proved partial result.}
Lemmas 1108.1--1108.3 give a complete injective ``$0/1$ factorial-base'' description of $A$ and show $|A\cap[0,(m+1)! )|=2^m$, hence $A$ has density $0$.

(ii) \textbf{First gap.}
Prove (or refute) that for each fixed $k\ge 2$, only finitely many elements of $A$ are perfect $k$th powers; and similarly, only finitely many elements of $A$ are powerful (squarefull) numbers.

(iii) \textbf{Top 3 next moves.}
1. Develop a $p$-adic obstruction: for a large element $a=\sum_{n\in S}n!$ with $m=\max S$, analyze $v_p(a)$ for primes $p\le m$ and compare with the constraints $v_p(a)\equiv 0\pmod{k}$ for $k$th powers.
2. Work modulo $(m+1)!$ and smaller factorial moduli to restrict possible residues of $k$th powers within the $0/1$ digit set; attempt to show that for large $m$ no residue class can match.
3. Extend the brute-force search to larger $m$ (e.g. $m\le 16$ gives $65536$ elements) to look for new squares/cubes/powerful numbers beyond those found for $m\le 12$, and record their factorial-digit patterns.

(iv) \textbf{What a minimal counterexample would likely look like.}
If there are infinitely many $k$th powers in $A$, then for arbitrarily large $m$ there exists a subset $S\subseteq\{1,\dots,m\}$ with $m\in S$ such that $\sum_{n\in S} n!$ lands exactly on a $k$th power. By Lemma 1108.2 this forces a highly rigid pattern of $0/1$ digits in factorial base; a ``minimal'' large counterexample would likely have $m\in S$ and then a carefully tuned choice of lower factorials to match the gap between consecutive $k$th powers near $m!$.


