
\noindent\textbf{1) FORMAL RESTATEMENT}

Fix an integer $k\ge 2$. For an integer $n\ge 2$, define
\[
D_n := (n,2n)\cap\mathbb{Z} = \{n+1,n+2,\dots,2n-1\}.
\]
Let
\[
A_{n,k}:=\{a\in\mathbb{Z}: n\le a\le n^k\ \text{and there exists }d\in D_n\text{ with }d\mid a\}.
\]
List the elements of $A_{n,k}$ in increasing order as $a_1<a_2<\cdots<a_t$. The quantity of interest is
\[
G(n,k):=\max_{1\le i<t}(a_{i+1}-a_i).
\]
Question: is $G(n,k)\le (\log n)^{O(1)}$ for fixed $k$ and all sufficiently large $n$?

\bigskip
\noindent\textbf{2) QUICK LITERATURE/CONTEXT CHECK}

No external results are stated in the problem file beyond a cross-reference to another problem. I do not use web sources.

\bigskip
\noindent\textbf{3) ATTACK PLAN}

\emph{Proof track:} interpret $A_{n,k}$ as a union of arithmetic progressions (multiples of each $d\in(n,2n)$) and try to show these progressions hit every interval of polylogarithmic length.

\emph{Disproof track:} attempt to construct an interval of length $L\gg (\log n)^C$ inside $[n,n^k]$ avoiding all multiples of every $d\in(n,2n)$.

Here I obtain a simple unconditional bound $G(n,k)\le n+1$ and provide computational evidence for much smaller gaps.

\bigskip
\noindent\textbf{4) WORK}

\noindent\textbf{Lemma 693.1 (Trivial linear gap bound).}
For every $n\ge 2$ and $k\ge 2$,
\[
G(n,k)\le n+1.
\]

\smallskip
\noindent\emph{Proof.}
Consider the fixed divisor $d=n+1\in D_n$. Every multiple of $d$ belongs to $A_{n,k}$ provided it lies in $[n,n^k]$. Consecutive multiples of $d$ differ by exactly $d=n+1$.

Now take any two consecutive elements $a_i<a_{i+1}$ of $A_{n,k}$. Suppose for contradiction that $a_{i+1}-a_i\ge n+2$. Then the integer interval $[a_i+1,a_i+(n+1)]$ has length $n+1$, so it contains an integer congruent to $0\pmod{n+1}$ (among any $n+1$ consecutive integers, exactly one is divisible by $n+1$). Let $m$ be that multiple of $n+1$.

Since $a_i<m<a_{i+1}$ and $m$ is divisible by $n+1\in(n,2n)$, we have $m\in A_{n,k}$, contradicting the assumption that $a_i,a_{i+1}$ are consecutive. Therefore $a_{i+1}-a_i\le n+1$ for all $i$, so $G(n,k)\le n+1$. \hfill$\square$

\medskip
\noindent\textbf{Lemma 693.2 (Crude size lower bound).}
For every $n\ge 2$ and $k\ge 2$,
\[
|A_{n,k}|\ \ge\ \Big\lfloor\frac{n^k}{n+1}\Big\rfloor-\Big\lfloor\frac{n-1}{n+1}\Big\rfloor\ \ge\ \Big\lfloor\frac{n^k}{n+1}\Big\rfloor-0.
\]
In particular $|A_{n,k}|\ge \lfloor n^{k-1}/2\rfloor$ for all $n\ge 2$.

\smallskip
\noindent\emph{Proof.}
All multiples of $n+1$ lying in $[n,n^k]$ are included in $A_{n,k}$ by definition, since $n+1\in(n,2n)$. The count of multiples of $n+1$ in $[n,n^k]$ is
\[
\#\{m(n+1): n\le m(n+1)\le n^k\}=\Big\lfloor\frac{n^k}{n+1}\Big\rfloor-\Big\lfloor\frac{n-1}{n+1}\Big\rfloor.
\]
This lower-bounds $|A_{n,k}|$. For $n\ge 2$, $n+1\le 2n$, hence $\frac{n^k}{n+1}\ge \frac{n^k}{2n}=\frac{1}{2}n^{k-1}$, so $\lfloor\frac{n^k}{n+1}\rfloor\ge \lfloor\frac{1}{2}n^{k-1}\rfloor$. \hfill$\square$

\medskip
\noindent\textbf{FAST REALITY CHECK (computed small cases).}
A local marking computation (for each $d\in(n,2n)$ mark its multiples in $[n,n^k]$) gives the following exact values of $G(n,k)$:
\[
\begin{array}{c|ccc}
(n,k) & (50,2) & (100,2) & (200,2)\\\hline
G(n,k) & 9 & 11 & 17
\end{array}
\qquad
\begin{array}{c|ccc}
(n,k) & (50,3) & (100,3) & (200,3)\\\hline
G(n,k) & 18 & 20 & 24
\end{array}
\]
In these tests, the maximal gap is much smaller than the unconditional bound $n+1$.

\bigskip
\noindent\textbf{5) VERIFICATION}

- Lemma 693.1 uses only the single progression of multiples of $n+1$, so it is valid for all $k\ge 2$ without any distributional assumptions.

- The computational check is exact for the stated $(n,k)$ because it exhaustively marks all $d\in\{n+1,\dots,2n-1\}$.

\bigskip
\noindent\textbf{6) FINAL}

\textbf{UNRESOLVED}

(i) \emph{Strongest proved partial result.} For all $n\ge 2$ and $k\ge 2$, the maximal gap satisfies $G(n,k)\le n+1$ (Lemma 693.1). Also $|A_{n,k}|\ge \lfloor n^k/(n+1)\rfloor$ (Lemma 693.2).

(ii) \emph{First gap (crisp).} Prove (or disprove) that for each fixed $k\ge 2$ there exists $C_k$ such that $G(n,k)\le (\log n)^{C_k}$ for all sufficiently large $n$.

(iii) \emph{Top 3 next moves.}
1. Model $A_{n,k}$ as a union of arithmetic progressions and prove a short-interval covering lemma: every interval of length $L=(\log n)^{C_k}$ in $[n,n^k]$ intersects some progression $d\mathbb{Z}$ with $d\in(n,2n)$.
2. Study the complement $[n,n^k]\setminus A_{n,k}$ as integers avoiding all divisors in $(n,2n)$; attempt a sieve upper bound on the number of such integers in intervals of length $L$.
3. Extend the computation to larger $(n,k)$ and track the growth rate of $G(n,k)$, looking for potential counterexample structure (e.g. unusually long runs of integers avoiding all $d\in(n,2n)$).

(iv) \emph{Minimal counterexample structure.} A counterexample to the polylog-gap conjecture would consist of a number $X\in[n,n^k]$ and a long interval $[X,X+L]$ (with $L\gg (\log n)^C$) such that no integer in this interval is divisible by any $d\in(n,2n)$. Equivalently, for every $d\in\{n+1,\dots,2n-1\}$, all residues of $X,X+1,\dots,X+L$ avoid $0\pmod d$ simultaneously.


