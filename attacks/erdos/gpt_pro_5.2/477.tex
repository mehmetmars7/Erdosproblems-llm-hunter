% Erdos Problem #477
% URL: https://www.erdosproblems.com/477

1) “FORMAL RESTATEMENT”

We seek a polynomial f: Z -> Z of degree at least 2 and a set A ⊆ Z such that every integer n has a unique decomposition

    n = a + b

with a ∈ A and b ∈ B, where B := f(Z) = { f(t) : t ∈ Z }.

Equivalently, the translates (A+b)_{b∈B} form a partition of Z.

2) “QUICK LITERATURE/CONTEXT CHECK”

A quick check of the Erdos Problems discussion page for #477 shows it is listed as OPEN (as of mid-Jan 2026). I do not use any external results beyond the problem statement.

3) “ATTACK PLAN”

Try to prove impossibility by deriving strong structural constraints from uniqueness: translates A+b must be pairwise disjoint, forcing (A-A) to avoid (B-B). For polynomial images B, the difference set B-B is often very large (e.g. for squares), which may force A to be extremely small.

Try to construct an example by choosing f with large congruence obstructions so that B-B is “thin”, then attempt to build A as a set of representatives.

4) “WORK”

Lemma 477.1 (difference-set obstruction).
Assume Z = A + B with unique representation, where B is any subset of Z. Then

    (A-A) ∩ (B-B) = {0}.

Proof.
Suppose there exist a1,a2 in A and b1,b2 in B with a1-a2 = b2-b1. Then a1+b1 = a2+b2. If (a1,b1)!=(a2,b2) this gives two distinct representations of the same integer, contradicting uniqueness. Hence the only way a1-a2 can equal b2-b1 is if both differences are 0, i.e. a1=a2 and b1=b2. Therefore the intersection of difference sets is exactly {0}.  QED.

Lemma 477.2 (no solution for f(n)=n^2).
Let B = { n^2 : n∈Z }. Then there is no set A ⊆ Z such that every integer has a unique representation n=a+b with a∈A, b∈B.

Proof.
Step 1: Characterize B-B. A classical fact: an integer d is a difference of two squares iff d is not congruent to 2 modulo 4. Indeed, d = x^2 - y^2 = (x-y)(x+y). If x,y have the same parity then d is divisible by 4; if they have opposite parity then both factors are odd and d is odd. Thus d cannot be 2 mod 4. Conversely, if d is odd, write d=(d+1)/2^2 - (d-1)/2^2 (difference of squares of consecutive integers). If d is divisible by 4, write d=(d/4+1)^2 - (d/4-1)^2 times 4, i.e. d=(u+v)(u-v) with u,v even; equivalently choose x=(d/4+1), y=(d/4-1) and scale by 2. In each case we represent d as a difference of squares.
Hence

    B-B = { d ∈ Z : d not ≡ 2 (mod 4) }.

Step 2: Apply Lemma 477.1. If Z = A + B uniquely, then (A-A) ∩ (B-B) = {0}. Since B-B contains every nonzero integer except those ≡2 (mod 4), it follows that every nonzero difference a1-a2 (a1!=a2) must be congruent to 2 mod 4.

Step 3: This forces A to have size at most 2. Fix a0 ∈ A. Any other a ∈ A\{a0} must satisfy a-a0 ≡2 (mod 4), i.e. a ≡ a0+2 (mod 4). But then two distinct such elements a,a' would both be ≡ a0+2 (mod 4), so a-a' ≡0 (mod 4), contradicting the requirement that nonzero differences are ≡2 (mod 4). Therefore A has at most 2 elements.

Step 4: A finite A cannot give a unique cover with the sparse set of squares. If |A|<=2, then A+B is contained in at most two translates of the square set, hence cannot cover all integers (e.g. the gaps between consecutive squares grow without bound, so two fixed translates cannot fill all gaps). Thus no such A exists.  QED.

Lemma 477.3 (a general “subgroup in B-B” obstruction).
Assume Z = A + B uniquely. If there exists an integer d>=2 such that dZ ⊆ (B-B), then A has size at most d (hence finite), and in particular no such decomposition can exist when B is a polynomial value set of degree >=2.

Proof.
If dZ ⊆ (B-B), then Lemma 477.1 implies (A-A) ∩ dZ = {0}. If A contained two distinct integers a1!=a2 with a1≡a2 (mod d), then a1-a2 would be a nonzero multiple of d, contradicting (A-A)∩dZ={0}. Hence A contains at most one representative from each residue class modulo d, so |A|<=d.
If B=f(Z) with deg f>=2, then B has density 0 in Z, and a finite union of translates of B cannot cover Z. Therefore the situation dZ ⊆ (B-B) cannot occur in a genuine solution; it is an obstruction in many concrete cases (as in Lemma 477.2 where 4Z ⊆ B-B).  QED.

FAST REALITY CHECK.
For f(n)=n^2, I checked by a short brute force that all residues mod 4 other than 2 occur among differences x^2-y^2 with |x|,|y|<=20, matching the classical characterization used in Lemma 477.2.

5) “VERIFICATION”

- Lemma 477.2 uses only elementary congruence facts about differences of squares and the uniqueness constraint from Lemma 477.1.
- The final step that two translates of squares cannot cover Z can be justified by the fact that the gap between consecutive squares is (k+1)^2-k^2 = 2k+1, which tends to infinity, so any fixed finite number of translates still leaves arbitrarily large uncovered intervals.

6) FINAL

UNRESOLVED

(i) Strongest proved partial result:
No such unique representation is possible when f(n)=n^2 (Lemma 477.2). More generally, any instance where (B-B) contains a full subgroup dZ forces A to be finite (Lemma 477.3), strongly suggesting nonexistence for many polynomial images.

(ii) First gap (crisp statement):
Either (a) construct some explicit polynomial f of degree >=2 and set A giving a unique decomposition of every integer as a+f(t), or (b) prove that for every integer polynomial f of degree >=2, no such set A exists.

(iii) Top 3 next moves (concrete):
1. For each degree-2 polynomial f(n)=an^2+bn+c, analyze B-B modulo small moduli (e.g. mod 4,8,12) to see whether B-B contains a nontrivial subgroup dZ; if so, Lemma 477.3 rules it out.
2. Try to prove a general theorem: for any integer polynomial f of degree>=2, the difference set f(Z)-f(Z) is “thick” enough (contains dZ for some d>=2) to force A finite.
3. Conversely, attempt a constructive approach: seek f for which f(Z) is contained in a sparse congruence class mod d but still hits every multiple of d; then try to build A as a system of residues mod d.

(iv) Minimal counterexample structure (if the answer is “yes”):
A concrete polynomial f of degree >=2 together with an explicit set A, such that the translates A+f(t) (t∈Z) are pairwise disjoint and their union is all of Z.


