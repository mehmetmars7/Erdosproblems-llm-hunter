
\subsection*{Erd\H{o}s problem \#334}

\noindent\textbf{1) FORMAL RESTATEMENT.}

\medskip
\noindent\textbf{Smoothness.}
For an integer $m\ge 1$ and a real $y\ge 2$, write $P^+(m)$ for the largest prime divisor of $m$ (with the convention $P^+(1)=1$). We say that $m$ is \emph{$y$--smooth} if $P^+(m)\le y$.

\medskip
\noindent\textbf{Target function.}
We seek the smallest possible function $f:\mathbb{N}\to[2,\infty)$ such that for every integer $n\ge 2$ there exist integers $a,b\ge 1$ with
\[
 n=a+b,\qquad P^+(a)\le f(n),\quad P^+(b)\le f(n).
\]
The question asks for the \emph{best} asymptotic growth of such an $f(n)$.

\medskip
\noindent\textbf{What ``best'' can mean.}
Pointwise, the optimal value is the minimum smoothness threshold that works for each fixed $n$. Asymptotically, one can ask for the smallest order of growth of $f(n)$ as $n\to\infty$ that guarantees a decomposition for every $n$.

\medskip
\noindent\textbf{2) QUICK LITERATURE/CONTEXT CHECK.}
The problem statement itself is the only allowed source of prior results. It does not assert a theorem-level bound, but it does ask whether $f(n)$ can be as small as $n^\varepsilon$ for each fixed $\varepsilon>0$, or whether $f(n)$ must be substantially larger.

\medskip
\noindent\textbf{3) ATTACK PLAN.}

\begin{itemize}
\item Reduce the problem to an explicit pointwise extremal function $g(n)$.
\item Prove unconditional elementary bounds on $g(n)$.
\item Run a small computation of $g(n)$ for sanity.
\item Identify the first genuine analytic gap: controlling $g(n)$ asymptotically requires nontrivial information about representations of $n$ as a sum of smooth numbers.
\end{itemize}

\medskip
\noindent\textbf{4) WORK.}

\medskip
\noindent\textbf{Lemma 334.1 (pointwise optimal threshold).}
Define
\[
 g(n) := \min_{1\le a\le n-1}\; \max\bigl(P^+(a),\,P^+(n-a)\bigr)\qquad (n\ge 2).
\]
Then:
\begin{enumerate}
\item[(a)] For each $n\ge 2$, the value $g(n)$ is the \emph{smallest} real number $y$ for which $n$ admits a decomposition $n=a+b$ with both $a$ and $b$ $y$--smooth.
\item[(b)] A function $f$ satisfies the problem requirement for all $n$ if and only if $f(n)\ge g(n)$ for all $n\ge 2$.
\end{enumerate}

\noindent\emph{Proof.}
(a) If $n=a+b$ with both $a$ and $b$ $y$--smooth, then $P^+(a)\le y$ and $P^+(b)\le y$, hence $\max(P^+(a),P^+(b))\le y$. Minimizing over all $a$ shows $g(n)\le y$. Conversely, by definition of $g(n)$ there exists some $a$ such that $\max(P^+(a),P^+(n-a))=g(n)$; then $a$ and $n-a$ are $g(n)$--smooth and sum to $n$.

(b) If $f$ works, then for each $n$ there exists $a$ with $\max(P^+(a),P^+(n-a))\le f(n)$, hence $g(n)\le f(n)$ by the defining minimum. Conversely if $f(n)\ge g(n)$, pick $a$ attaining the minimum in $g(n)$, and the corresponding $a,n-a$ are $f(n)$--smooth.
\hfill$\square$

\medskip
\noindent\textbf{Lemma 334.2 (elementary universal upper bounds).}
For all $n\ge 2$,
\[
 g(n)\le P^+(n-1)\le n-1.
\]
Moreover, for even $n=2m$ one has $g(n)\le P^+(m)$.

\noindent\emph{Proof.}
Take the decomposition $n=1+(n-1)$. Since $P^+(1)=1$ and $P^+(n-1)\le n-1$, we get
\[
 g(n)\le \max\bigl(P^+(1),P^+(n-1)\bigr)=P^+(n-1)\le n-1.
\]
If $n=2m$, then $n=m+m$ gives $g(2m)\le \max(P^+(m),P^+(m))=P^+(m)$.
\hfill$\square$

\medskip
\noindent\textbf{FAST REALITY CHECK (computed small cases).}
I computed $g(n)$ exactly by brute force for $n\le 10\,000$ using the definition in Lemma~334.1.
For $2\le n\le 20$ the values and one minimizing decomposition $(a,n-a)$ are:
\[
\begin{array}{c|ccccccccccccccccccc}
 n &2&3&4&5&6&7&8&9&10&11&12&13&14&15&16&17&18&19&20\\\hline
 g(n)&1&2&2&2&2&3&2&2&2&3&2&3&3&3&2&2&2&3&2\\
 (a,b)&(1,1)&(1,2)&(2,2)&(1,4)&(2,4)&(1,6)&(4,4)&(1,8)&(2,8)&(2,9)&(4,8)&(1,12)&(2,12)&(3,12)&(8,8)&(1,16)&(2,16)&(1,18)&(4,16)
\end{array}
\]
For larger ranges, the maximum observed values were:
\begin{itemize}
\item $\max_{2\le n\le 200} g(n)=7$ (attained at $n\in\{71,119,142,191\}$).
\item $\max_{2\le n\le 1\,000} g(n)=13$ (attained at $n\in\{479,958\}$).
\item $\max_{2\le n\le 10\,000} g(n)=19$ (attained at $n\in\{5711,8399,9239\}$).
\end{itemize}
These computations suggest $g(n)$ stays small for $n$ up to $10^4$, but this is far from an asymptotic proof.

\medskip
\noindent\textbf{5) VERIFICATION.}

\begin{itemize}
\item Lemma~334.1 is a direct unpacking of definitions.
\item Lemma~334.2 uses explicit decompositions $n=1+(n-1)$ and $2m=m+m$.
\item The computation for $g(n)$ is finite and exact for the tested range: for each $n$ it searches all $a\in\{1,\dots,n-1\}$ and evaluates $\max(P^+(a),P^+(n-a))$.
\end{itemize}

\medskip
\noindent\textbf{6) FINAL.}

\noindent\textbf{UNRESOLVED}

\smallskip
\noindent (i) \textbf{Strongest fully proved partial result obtained here.}
The problem reduces pointwise to bounding
\[
 g(n)=\min_{1\le a\le n-1}\max\bigl(P^+(a),P^+(n-a)\bigr)
\]
(Lemma~334.1), and we have the unconditional bound $g(n)\le P^+(n-1)\le n-1$ (Lemma~334.2). Computations up to $n=10\,000$ found $g(n)\le 19$.

\smallskip
\noindent (ii) \textbf{Exact first gap.}
We do not have a nontrivial asymptotic upper bound on $g(n)$ that holds for \emph{all} large $n$ (e.g., $g(n)\le n^\varepsilon$ for every fixed $\varepsilon>0$). Any such bound would require controlling representations of every integer $n$ as a sum of two $y$--smooth numbers with $y$ much smaller than $n$.

\smallskip
\noindent (iii) \textbf{Top 3 next moves (concrete targets).}
\begin{enumerate}
\item Prove a deterministic covering lemma of the form: for some explicit $y=y(n)$, every interval $[1,n]$ contains enough $y$--smooth numbers so that $\{a+b: a,b\ \text{$y$--smooth}\}$ covers $[1,n]$.
\item Try to show that for each residue class modulo a suitable primorial $\prod_{p\le y}p$, every large $n$ has a representation $n\equiv a+b\pmod{\prod_{p\le y}p}$ with $a,b$ chosen from a dense set of $y$--smooth residues.
\item Compute $g(n)$ to larger ranges (e.g. $10^6$) and record the maximizing $n$ and the structure of minimizing decompositions to guess a sharper conjecture for $g(n)$.
\end{enumerate}

\smallskip
\noindent (iv) \textbf{Minimal counterexample structure.}
A counterexample to a proposed bound $g(n)\le y(n)$ would be an integer $n$ such that for every $1\le a\le n-1$ at least one of $a$ or $n-a$ has a prime factor $>y(n)$. Equivalently, the complement of the $y(n)$--smooth numbers would have to be ``sum-dominating'' enough that it blocks every decomposition of $n$.


