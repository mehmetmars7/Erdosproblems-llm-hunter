% Erdos Problem #364
% Attempt for Erdos Problem #364
% Following PROMPT_STRATEGY.MD
% Tools/Constraints:
% - Web browsing available? YES
% - Computation available (Python/Sage/Mathematica)? YES

## 1) FORMAL RESTATEMENT

A positive integer \(n\) is **powerful** (a.k.a. **2-full**, **squareful**) if
\[
\forall\text{ primes }p:\ p\mid n\ \Rightarrow\ p^2\mid n.
\]

The problem asks:

> Does there exist \(N\in\mathbb N\) such that \(N,\,N+1,\,N+2\) are all powerful?

(Equivalently: do there exist three consecutive positive integers all powerful?)

Edge cases:
* \(1\) is powerful by the given definition (no prime divides 1).
* \(2\) is not powerful.

---

## 2) QUICK LITERATURE/CONTEXT CHECK

I only record what the provided problem text states (no additional external claims):

* There are infinitely many **pairs** of consecutive powerful numbers from solutions to
  \(x^2=8y^2+1\).
* The abc conjecture would imply only finitely many such **triples**.
* There are no **quadruples** of consecutive powerful numbers.
* A computational claim (OEIS A060355) is mentioned: no such triples with \(n<10^{22}\).

---

## 3) ATTACK PLAN

**Proof track (partial structure).**
1. Prove standard structural characterizations of powerful numbers.
2. Prove rigorously the infinite consecutive-pairs construction from Pell’s equation.
3. Prove the easy obstruction for 4 consecutive powerful numbers.

**Disproof track.**
Try to find a triple by computation in a modest range. (No triple found up to \(10^6\).)

---

## 4) WORK

### Phase 1: FAST REALITY CHECK (computation)

I brute-searched \(N\le 10^6\) and found **no** triple \((N,N+1,N+2)\) with all three powerful.

---

### Lemma 4.1 (Powerful \(\Leftrightarrow a^2 b^3\) with \(b\) squarefree)
A positive integer \(n\) is powerful if and only if it can be written as
\[
 n=a^2 b^3
\]
with \(a,b\in\mathbb N\) and \(b\) squarefree.

**Proof.**

\underline{(\(\Rightarrow\)).}
Write the prime factorization \(n=\prod_p p^{e_p}\) with each \(e_p\ge 0\). Powerfulness means:
whenever \(e_p>0\), we have \(e_p\ge 2\).

Define
\[
 b:=\prod_{p: e_p\text{ odd}} p\quad\text{(so }b\text{ is squarefree),}\qquad
 a:=\prod_p p^{(e_p-3\cdot\mathbf 1_{e_p\text{ odd}})/2}.
\]
This makes sense because:
* if \(e_p\) is even, the exponent is \(e_p/2\in\mathbb N_0\);
* if \(e_p\) is odd and \(e_p\ge 3\), then \((e_p-3)/2\in\mathbb N_0\).

Then for each prime \(p\), the exponent of \(p\) in \(a^2 b^3\) is
\(2\cdot\frac{e_p-3\cdot\mathbf 1_{e_p\text{ odd}}}{2} + 3\cdot\mathbf 1_{e_p\text{ odd}}=e_p\).
Hence \(n=a^2 b^3\).

\underline{(\(\Leftarrow\)).}
If \(n=a^2 b^3\), then for any prime \(p\mid n\), either \(p\mid a\) (then \(p^2\mid a^2\mid n\))
or \(p\mid b\) (then \(p^3\mid b^3\mid n\), so in particular \(p^2\mid n\)). Thus \(n\) is powerful.

This proves the equivalence. \(\square\)

---

### Lemma 4.2 (Infinite consecutive pairs from Pell)
Let \((x,y)\in\mathbb N^2\) satisfy the Pell equation
\[
 x^2-8y^2=1.
\]
Then \(8y^2\) and \(x^2\) are consecutive powerful numbers.

**Proof.** The Pell equation gives \(x^2=8y^2+1\), so \(x^2\) and \(8y^2\) differ by 1.

* \(x^2\) is a perfect square, hence for every prime \(p\mid x^2\) we have \(p^2\mid x^2\); thus \(x^2\)
  is powerful.
* \(8y^2 = 2^3\cdot y^2\). Any prime divisor of \(y\) occurs in \(y^2\) with exponent at least 2, and
  the prime 2 occurs with exponent 3. Hence every prime divisor of \(8y^2\) has exponent at least 2,
  so \(8y^2\) is powerful.

To see there are infinitely many positive integer solutions, start from the solution \((x_1,y_1)=(3,1)\)
(since \(3^2-8\cdot 1^2=1\)) and define recursively
\[
x_{t+1}=3x_t+8y_t,\qquad y_{t+1}=x_t+3y_t.
\]
If \(x_t^2-8y_t^2=1\), then
\[
x_{t+1}^2-8y_{t+1}^2=(3x_t+8y_t)^2-8(x_t+3y_t)^2=x_t^2-8y_t^2=1,
\]
so \((x_{t+1},y_{t+1})\) is also a solution. Moreover \(x_{t+1}=3x_t+8y_t>x_t\), hence the
solutions are all distinct and infinite in number.

Therefore there are infinitely many such consecutive powerful pairs. \(\square\)

---

### Lemma 4.3 (No four consecutive powerful numbers)
There do not exist four consecutive positive integers all of which are powerful.

**Proof.** Among any four consecutive integers, exactly one is congruent to \(2\pmod 4\).
Such an integer has 2-adic valuation exactly 1, i.e. it is divisible by 2 but not by 4.
Hence for that integer \(n\), the prime \(2\mid n\) but \(2^2\nmid n\), so \(n\) is not powerful.
\(\square\)

---

## 5) VERIFICATION

* Lemma 4.1: the construction of \(a,b\) assigns odd exponents to \(b\) (once each) and the remaining
  even part to \(a^2\); the exponent check is explicit prime-by-prime.
* Lemma 4.2: the only property used is “square implies powerful” and “\(y^2\) gives exponent \(\ge 2\)”
  plus the explicit exponent 3 of 2 in 8.
* Lemma 4.3: the mod-4 argument is airtight.
* Computation is only a sanity check; no conclusions beyond \(10^6\) are claimed.

---

## 6) FINAL

**UNRESOLVED**

(i) **Strongest fully proved partial result obtained here.**

* There are infinitely many pairs of consecutive powerful numbers of the form \((8y^2, x^2)\) coming
  from Pell solutions \(x^2-8y^2=1\) (Lemma 4.2).
* There are no four consecutive powerful numbers (Lemma 4.3).
* A brute-force search finds no triple \((N,N+1,N+2)\) of powerful numbers with \(N\le 10^6\).

(ii) **Exact first gap.**

Prove or disprove:
\[
\exists N\in\mathbb N:\ N,\ N+1,\ N+2\text{ are all powerful}.
\]

(iii) **Top 3 next moves (concrete targets).**

1. Use Lemma 4.1 to parametrize a hypothetical triple as
   \(N=a^2b^3\), \(N+1=c^2d^3\), \(N+2=e^2f^3\) with squarefree \(b,d,f\), and exploit gcd constraints
   \(\gcd(N,N+1)=\gcd(N+1,N+2)=1\).
2. Search for strong modular obstructions: for a powerful \(n\), \(n\equiv 0,1\pmod 4\); combine such
   restrictions across three consecutive integers to narrow cases.
3. If allowed to assume abc (as the problem text mentions), formalize the implication “finitely many
   triples” and try to obtain an unconditional weaker bound (e.g. growth gaps between powerful numbers).

(iv) **What a minimal counterexample would likely look like.**

If a triple exists, the smallest \(N\) must be very large (computational evidence suggests no small
examples). Structurally, by coprimality, the three powerful numbers would have disjoint prime sets
pairwise, forcing a delicate arrangement where each of \(N, N+1, N+2\) is composed entirely of prime
powers \(\ge 2\) yet adjacent numbers share no primes.


