% Erdos Problem #177
% Attempt for Erdos Problem #177
% Following PROMPT_STRATEGY.MD
% Tools/Constraints:
% - Web browsing available? YES (not used beyond the problem text; no extra literature claims).
% - Computation available (Python/Sage/Mathematica)? YES (not essential here).

PROBLEM

Find the smallest $h(d)$ such that the following holds. There exists a function $f:\mathbb{N}\to\{-1,1\}$ such that, for every $d\geq 1$,
\[\max_{P_d}\left\lvert \sum_{n\in P_d}f(n)\right\rvert\leq h(d),\]
where $P_d$ ranges over all finite arithmetic progressions with common difference $d$.

OUTPUT

## 1) FORMAL RESTATEMENT

We seek a function $h:\mathbb N\to\mathbb N$ of minimal growth such that there exists a single function $f:\mathbb N\to\{-1,1\}$ with the property:

For every $d\ge 1$,
\[
\Delta_f(d):=\sup\Bigl\{\Bigl|\sum_{n\in P} f(n)\Bigr| : P=\{a,a+d,\dots,a+(m-1)d\}\text{ for some }a,m\in\mathbb N\Bigr\}\le h(d).
\]
(Here $m$ ranges over all positive integers, so $P$ ranges over all finite arithmetic progressions of common difference $d$.)

The problem asks for the smallest possible such $h(d)$ (up to order of magnitude).

---

## 2) QUICK LITERATURE/CONTEXT CHECK

I will not use or claim any external results beyond what is explicitly written in the problem text.

The problem text reports:

* There exist constructions with $h(d)\ll d!$ (Cantor–Erd\H{o}s–Schreiber–Straus).
* Van der Waerden implies $h(d)\to\infty$.
* Beck obtained $h(d)\le d^{8+\varepsilon}$.
* Roth gives $h(d)\gg d^{1/2}$.

I do not re-prove these bounds.

---

## 3) ATTACK PLAN

1. Prove explicit bounds in simplified settings (e.g. controlling a single difference $d$).
2. Prove that $h$ cannot be bounded by an absolute constant if it works for all $d$ (using van der Waerden as allowed by the problem text).

---

## 4) WORK

### Lemma 4.1 (perfect control for a fixed difference $d$ with a tailored function)

Fix $d\ge 1$ and define
\[
f_d(n):=(-1)^{\left\lfloor\frac{n-1}{d}\right\rfloor}.
\]
Then for every finite arithmetic progression $P$ with common difference $d$,
\[
\Bigl|\sum_{n\in P} f_d(n)\Bigr|\le 1.
\]

**Proof.**
Let $P=\{a,a+d,a+2d,\dots,a+(m-1)d\}$.
For each $j\ge 0$,
\[
\left\lfloor\frac{(a+jd)-1}{d}\right\rfloor
=\left\lfloor\frac{a-1}{d}+j\right\rfloor
=\left\lfloor\frac{a-1}{d}\right\rfloor + j.
\]
Therefore
\[
f_d(a+jd)=(-1)^{\lfloor(a-1)/d\rfloor+j}=(-1)^{\lfloor(a-1)/d\rfloor}\cdot (-1)^j.
\]
So along the progression $P$ the values alternate in sign. Hence
\[
\sum_{n\in P} f_d(n)=(-1)^{\lfloor(a-1)/d\rfloor}\sum_{j=0}^{m-1} (-1)^j.
\]
The alternating sum $\sum_{j=0}^{m-1}(-1)^j$ equals $0$ if $m$ is even and equals $1$ if $m$ is odd.
Thus the total sum is in $\{0,\pm 1\}$ and has absolute value at most $1$.
$\square$

*Remark.* This shows that if one were allowed to choose $f$ depending on $d$, then the optimal bound for that single $d$ would be $h(d)=1$. The actual problem is the simultaneous requirement for all $d$ with one fixed $f$.

---

### Lemma 4.2 (no uniform constant bound across all differences)

There is no function $f:\mathbb N\to\{-1,1\}$ such that $\Delta_f(d)\le H$ for all $d$ with a fixed constant $H$.
Consequently any admissible $h(d)$ must be unbounded.

**Proof.**
Assume for contradiction that there exist $f$ and a constant $H$ such that for every $d$, every finite $d$-progression has sum of absolute value at most $H$.
Interpret $f$ as a 2-coloring of $\mathbb N$ (color $+1$ vs $-1$).

By van der Waerden’s theorem (as referenced in the problem text), for each integer $m>H$ there exists a monochromatic arithmetic progression $P$ of length $m$ in this 2-coloring.
On that progression, $f$ is constant, so
\[
\Bigl|\sum_{n\in P} f(n)\Bigr|=m>H,
\]
contradicting the assumed uniform bound.

Therefore no such uniform constant exists, so any admissible function $h(d)$ must satisfy $\sup_d h(d)=\infty$.
$\square$

---

## 5) VERIFICATION

* Lemma 4.1: the key identity $\lfloor((a+jd)-1)/d\rfloor=\lfloor(a-1)/d\rfloor+j$ is correct because $j$ is an integer.
* Lemma 4.2: the only external input is van der Waerden’s theorem, explicitly mentioned in the problem text.

---

## 6) FINAL

**UNRESOLVED**

(i) **Strongest fully proved partial result obtained here.**

I proved:

* For each fixed $d$ there exists a tailored $f_d$ with discrepancy at most $1$ on all $d$-progressions (Lemma 4.1).
* No single $f$ can have a uniform constant discrepancy bound across all differences $d$ (Lemma 4.2).

(ii) **Exact first gap.**

Construct a single function $f$ that simultaneously controls all $d$ with an explicit small-growth function $h(d)$, or prove lower bounds showing such control must be large (e.g. polynomial).

(iii) **Top 3 next moves (concrete targets).**

1. Try to combine periodic/block constructions at many scales (e.g. Chinese remainder style) to produce one $f$ with $h(d)$ poly$(d)$.
2. Prove a clean lower bound mechanism: show that any $f$ forces some difference $d$ to have discrepancy at least $\Omega(d^\alpha)$ for some $\alpha>0$.
3. For small $d$, compute (by search) the minimal achievable $\Delta_f(d)$ under constraints for multiple differences simultaneously, to guess the right growth.

(iv) **What a minimal counterexample would likely look like.**

A counterexample to “polynomial $h(d)$ is possible” would be a proof that any $f$ necessarily has very large discrepancy on progressions of some moderately large difference $d$ (perhaps using a density increment argument). Such a proof would likely show that controlling many differences forces a structured bias that creates long monochromatic APs.

