\section*{Erd\H{o}s Problem \#264}

\subsection*{1) FORMAL RESTATEMENT}

\begin{definition}[Type 3 irrationality sequence]
A sequence of integers $(a_n)_{n\ge 1}$ is a \emph{Type 3 irrationality sequence} if for every bounded sequence of integers $(b_n)_{n\ge 1}$ satisfying
\[
 b_n \neq 0 \ \text{and}\ a_n+b_n\neq 0 \quad\text{for all }n,
\]
the series $\sum_{n\ge 1} \frac{1}{a_n+b_n}$ converges and its sum is an irrational real number.
\end{definition}

The question asks whether the specific choices $a_n=2^n$ and $a_n=n!$ are Type 3 irrationality sequences.

\subsection*{2) QUICK LITERATURE/CONTEXT CHECK (only if browsing available)}

\begin{itemize}[leftmargin=2em]
\item Kova\v{c}--Tao (2024/2025) introduced a quantitative criterion implying a sequence is \emph{not} Type~3; in particular they show $a_n=2^n$ fails to be Type~3 (indeed, there exists bounded $b_n$ for which the sum is rational). They also show any strictly increasing $a_n$ with $\limsup a_{n+1}/a_n<\infty$ is not Type~3, so ``pure exponential'' growth cannot work.
\item In the opposite direction, Kova\v{c}--Tao prove that for any function $F$ with $F(n+1)/F(n)\to\infty$ there exists a Type~3 irrationality sequence with $a_n\sim F(n)$; in particular there exist Type~3 sequences asymptotic to $n!$, but this does not decide whether the exact sequence $a_n=n!$ itself is Type~3.
\end{itemize}

\subsection*{3) ATTACK PLAN}

\begin{enumerate}[leftmargin=2.2em]
\item Disprove the Type~3 property for $a_n=2^n$ by constructing (or at least proving existence of) a bounded integer sequence $b_n$ such that
\[\sum_{n\ge 1}\frac{1}{2^n+b_n}\in\mathbb{Q}.
\]
It suffices to produce one rational value.
\item Show the existence of such a $b_n$ by an ``overlapping-interval''/digit-expansion argument: allow $b_n\in\{1,2,3,4,5\}$, prove the set of achievable sums is an interval, then pick a rational point inside that interval.
\item For $a_n=n!$, summarize what is known and why the same interval method fails for fixed bounded $b_n$ (the gaps between $1/(n!+b)$ values dwarf the tail width). Record the open part.
\end{enumerate}

\subsection*{4) WORK (with full details)}

\subsubsection*{4.1. A concrete counterexample for $a_n=2^n$ (bounded perturbations can give a rational sum)}

Fix $B=5$. For $n\ge 1$ define the ``tail bounds''
\[
L_n := \sum_{k=n}^{\infty}\frac{1}{2^k+B},\qquad
U_n := \sum_{k=n}^{\infty}\frac{1}{2^k+1},
\]
so $0<L_n\le U_n<\infty$ and $L_n,U_n\to 0$ as $n\to\infty$.

For each $n\ge 1$ and each $b\in\{1,2,3,4,5\}$ set
\[
I_{n}(b):=\Bigl[\frac{1}{2^n+b}+L_{n+1},\ \frac{1}{2^n+b}+U_{n+1}\Bigr].
\]
Intuitively, if we choose the $n$th ``digit'' $b_n=b$, then the remaining tail sum lies in $[L_{n+1},U_{n+1}]$, so the total sum lies in $I_n(b)$.

\begin{lemma}[Overlap of digit intervals for $B=5$]
For every $n\ge 1$ the union $\bigcup_{b=1}^{5} I_n(b)$ is exactly the interval $[L_n,U_n]$.
\end{lemma}

\begin{proof}
First note the extreme endpoints match:
\[
\min I_n(5)=\frac{1}{2^n+5}+L_{n+1}=L_n,
\qquad
\max I_n(1)=\frac{1}{2^n+1}+U_{n+1}=U_n.
\]
Thus $\bigcup_{b=1}^{5} I_n(b)\subseteq [L_n,U_n]$ and the union reaches both endpoints.

It remains to show that consecutive intervals overlap, i.e.
\begin{equation}\label{eq:overlap_condition}
I_n(b)\cap I_n(b+1)\neq\varnothing\quad\text{for }b=1,2,3,4.
\end{equation}
Since $\frac{1}{2^n+b}$ decreases in $b$, the intervals are ordered from top to bottom, and \eqref{eq:overlap_condition} is equivalent to
\[
\frac{1}{2^n+(b+1)}+U_{n+1}\ge \frac{1}{2^n+b}+L_{n+1}
\iff
U_{n+1}-L_{n+1}\ge \Bigl(\frac{1}{2^n+b}-\frac{1}{2^n+b+1}\Bigr).
\]
The right-hand side is maximized at $b=1$, so it suffices to show
\begin{equation}\label{eq:key_ineq_target}
U_{n+1}-L_{n+1}\ \ge\ \frac{1}{(2^n+1)(2^n+2)}.
\end{equation}
Compute the tail width explicitly:
\[
U_{n+1}-L_{n+1}
=\sum_{k=n+1}^{\infty}\Bigl(\frac{1}{2^k+1}-\frac{1}{2^k+5}\Bigr)
=\sum_{k=n+1}^{\infty}\frac{4}{(2^k+1)(2^k+5)}.
\]
Fix $n$ and write $x:=2^{-(n+1)}\in(0,1/4]$. For every $k\ge n+1$ we have $2^{-k}\le x$, hence
\[
(2^k+1)(2^k+5)
=2^{2k}(1+2^{-k})(1+5\cdot 2^{-k})
\le 2^{2k}(1+x)(1+5x).
\]
Therefore
\[
\frac{4}{(2^k+1)(2^k+5)}\ge \frac{4}{2^{2k}(1+x)(1+5x)}=\frac{4^{1-k}}{(1+x)(1+5x)}.
\]
Summing this lower bound gives
\begin{equation}\label{eq:tailwidth_lb}
U_{n+1}-L_{n+1}
\ge \frac{1}{(1+x)(1+5x)}\sum_{k=n+1}^{\infty}4^{1-k}
=\frac{1}{(1+x)(1+5x)}\cdot \frac{4}{3}\cdot 4^{-n}.
\end{equation}
On the other hand,
\[
\frac{1}{(2^n+1)(2^n+2)}
=\frac{1}{2^{2n}(1+2^{-n})(1+2\cdot 2^{-n})}
=\frac{4^{-n}}{(1+2x)(1+4x)}
\qquad\bigl(\text{since }2^{-n}=2x\bigr).
\]
Thus \eqref{eq:key_ineq_target} will follow from
\[
\frac{4}{3}\cdot\frac{1}{(1+x)(1+5x)}\ \ge\ \frac{1}{(1+2x)(1+4x)}.
\]
Clearing denominators (all positive) reduces this to
\[
4(1+2x)(1+4x)\ \ge\ 3(1+x)(1+5x).
\]
Expanding both sides yields
\[
4(1+6x+8x^2)\ge 3(1+6x+5x^2)
\iff
1+6x+17x^2\ge 0,
\]
which is true for all $x\ge 0$. This proves \eqref{eq:key_ineq_target}, hence all overlaps \eqref{eq:overlap_condition}. Therefore the union $\bigcup_{b=1}^{5} I_n(b)$ is connected and spans from $L_n$ to $U_n$, i.e. it equals $[L_n,U_n]$.
\end{proof}

\begin{lemma}[Digit-by-digit construction]
Fix any real number $x\in[L_1,U_1]$. Then there exists a sequence $(b_n)_{n\ge 1}$ with each $b_n\in\{1,2,3,4,5\}$ such that
\[
 x=\sum_{n=1}^{\infty}\frac{1}{2^n+b_n}.
\]
\end{lemma}

\begin{proof}
We construct $b_n$ recursively. Set $r_1:=x$.

Assume $r_n\in[L_n,U_n]$ is defined. By the previous lemma,
\[
[L_n,U_n]=\bigcup_{b=1}^{5} I_n(b),
\]
so there exists at least one $b_n\in\{1,2,3,4,5\}$ such that $r_n\in I_n(b_n)$. Choose one such $b_n$ (for definiteness, the smallest such $b_n$). Define
\[
 r_{n+1}:=r_n-\frac{1}{2^n+b_n}.
\]
Because $r_n\in I_n(b_n)$, we have $r_{n+1}\in[L_{n+1},U_{n+1}]$ by the definition of $I_n(b_n)$. This closes the induction.

Iterating the relation $r_{n}=\frac{1}{2^n+b_n}+r_{n+1}$ gives for every $N\ge 1$,
\[
 x=r_1=\sum_{n=1}^{N}\frac{1}{2^n+b_n}+r_{N+1}.
\]
Finally $0\le r_{N+1}\le U_{N+1}\to 0$ as $N\to\infty$, so passing to the limit yields $x=\sum_{n\ge 1}\frac{1}{2^n+b_n}$.
\end{proof}

We now exhibit a rational $x\in[L_1,U_1]$.

\begin{lemma}[$\tfrac{3}{4}$ lies in the digit interval]
One has $\frac{3}{4}\in[L_1,U_1]$.
\end{lemma}

\begin{proof}
\emph{Lower endpoint:} since $2^n+5\ge 2^n$ for $n\ge 2$,
\[
L_1=\sum_{n=1}^{\infty}\frac{1}{2^n+5}
\le \frac{1}{7}+\sum_{n=2}^{\infty}\frac{1}{2^n}
=\frac{1}{7}+\frac{1}{2}=\frac{9}{14}<\frac{3}{4}.
\]

\emph{Upper endpoint:} clearly $U_1$ dominates any partial sum. Direct computation gives
\[
\sum_{n=1}^{7}\frac{1}{2^n+1}
=\frac{3559541}{4703985}
=\frac{3}{4}+\frac{126209}{18815940}
>\frac{3}{4}.
\]
Therefore $U_1>\frac{3}{4}$. Combining both inequalities yields $\frac{3}{4}\in[L_1,U_1]$.
\end{proof}

\begin{proposition}[Disproof for $a_n=2^n$]
Let $a_n=2^n$. Then $(a_n)$ is \emph{not} a Type~3 irrationality sequence.
\end{proposition}

\begin{proof}
By the previous lemma, $x=\tfrac{3}{4}$ lies in $[L_1,U_1]$. By the digit-by-digit construction lemma, there exists a bounded integer sequence $b_n\in\{1,2,3,4,5\}$ such that
\[
\sum_{n=1}^{\infty}\frac{1}{2^n+b_n}=\frac{3}{4}\in\mathbb{Q}.
\]
This is a direct violation of the Type~3 definition for $a_n=2^n$.
\end{proof}

\begin{remark}[An explicit initial segment]
Running the ``choose the smallest admissible digit'' rule produces, for example,
\[
(b_n)_{n=1}^{40}=1,1,1,5,4,3,5,5,5,3,5,2,3,2,5,2,4,4,3,2,3,5,4,5,3,3,3,5,2,2,2,2,2,2,2,3,2,4,4,3,\ldots
\]
which yields partial sums rapidly converging to $3/4$.
\end{remark}

\subsubsection*{4.2. The case $a_n=n!$}

The above interval-overlap method is tailored to exponential denominators: the ``digit gaps''
\[
\frac{1}{2^n+b}-\frac{1}{2^n+(b+1)}\asymp 4^{-n}
\]
are of the same order as the tail flexibility $U_{n+1}-L_{n+1}\asymp 4^{-n}$.

For $a_n=n!$, if we again restricted to a fixed bounded range $b\in\{1,\dots,B\}$, then the digit gaps are on the order of $1/(n!)^2$, but the tail flexibility is on the order of
\[
\sum_{k\ge n+1}\frac{1}{(k!)^2}\asymp \frac{1}{((n+1)!)^2}=\frac{1}{(n+1)^2}\cdot\frac{1}{(n!)^2},
\]
which is smaller by a factor $(n+1)^{-2}$. Thus no fixed $B$ can yield the kind of uniform overlap we used above.

At present, the exact Type~3 status of $a_n=n!$ is not settled by the argument above. What \emph{is} known (from the general existence theorem mentioned in the literature check) is that one can construct Type~3 sequences $(a_n)$ with $a_n\sim n!$ (even with $a_n=n!+o(n!)$), but this does not imply that the unperturbed factorial sequence itself has the Type~3 property.

\subsection*{5) VERIFICATION / SANITY CHECK}

\begin{itemize}[leftmargin=2em]
\item All series considered with denominators $2^n+b_n$ and bounded $b_n$ converge absolutely, since $2^n+b_n\asymp 2^n$.
\item The constructed $b_n$ satisfy the constraints $b_n\neq 0$ and $2^n+b_n\neq 0$ automatically (here $b_n\in\{1,2,3,4,5\}$).
\item The overlap inequality was reduced to an explicit polynomial inequality $1+6x+17x^2\ge 0$, so no hidden analytic assumptions remain.
\end{itemize}

\subsection*{6) FINAL}

\textbf{UNRESOLVED.}

\begin{itemize}[leftmargin=2em]
\item[(i)] \textbf{What is proved here:} $a_n=2^n$ is \emph{not} Type~3, by exhibiting a bounded choice of $b_n\in\{1,\dots,5\}$ for which $\sum_{n\ge 1}\frac{1}{2^n+b_n}=\tfrac{3}{4}\in\mathbb{Q}$.
\item[(ii)] \textbf{What remains open:} whether $a_n=n!$ is Type~3 (i.e., whether $\sum\frac{1}{n!+b_n}$ is irrational for every bounded integer perturbation $b_n\neq 0$ with $n!+b_n\neq 0$).
\item[(iii)] \textbf{Most plausible next step:} develop methods that exploit the arithmetic structure of $n!$ (e.g., strong divisibility and factorial-base expansions) to force irrationality for all bounded perturbations; the interval-overlap method above cannot work uniformly for fixed bounded $b_n$.
\item[(iv)] \textbf{Completion estimate:} resolving the factorial case likely requires a new invariant or approximation argument; at minimum, one would need either (a) a universal irrationality criterion that applies to all bounded perturbations of $n!$, or (b) an explicit bounded perturbation producing a rational sum.
\end{itemize}

\subsection*{7) COMPLETION ESTIMATE}

For the remaining $a_n=n!$ case, one would need to either:
\begin{enumerate}[leftmargin=2.2em]
\item produce an explicit bounded $b_n$ with $\sum 1/(n!+b_n)\in\mathbb{Q}$ (a counterexample), or
\item prove a uniform irrationality criterion that works against \emph{every} bounded perturbation of $n!$.
\end{enumerate}
Either direction appears to require substantially new ideas beyond the overlap/digit method used for $2^n$.

%%%%%%%%%%%%%%%%%%%%%%%%%%%%%%%%%%%%%%%%%%%%%%%%%%%%%%%%%%%%%%%%%%%%%%%%%%%%%%%
