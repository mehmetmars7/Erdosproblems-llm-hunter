
\noindent\textbf{FORMAL RESTATEMENT.}
Fix an integer $r\ge 2$ and $N\ge 2$.
Let $A\subseteq\{1,2,\dots,N\}$ satisfy the multiplicity condition:
for every integer $m$, the number of pairs $(p,a)$ with $p$ prime, $a\in A$, and $m=p a$ is at most $r$.
Equivalently,
\[
\forall m\in\mathbb N:\quad \bigl|\{p\text{ prime}: p\mid m\ \text{and}\ m/p\in A\}\bigr|\le r.
\]
Determine the best possible upper bound (in terms of $r$ and $N$) for
\[
\sum_{n\in A}\frac1n.
\]

\noindent\textbf{QUICK LITERATURE/CONTEXT CHECK.}
The problem statement itself includes Erd\H{o}s's double-counting argument yielding
\(\sum_{n\in A} \frac{1}{n} \ll r \frac{\log N}{\log\log N}\).
I re-derive the exact inequality behind that bound. I do not claim optimality.

\noindent\textbf{ATTACK PLAN.}
1) Re-derive Erd\H{o}s's inequality carefully and isolate the clean, unconditional form.
2) Produce explicit constructions of sets $A$ satisfying the multiplicity condition to get nontrivial lower bounds on the extremal harmonic sum.
3) Brute force for small $N$ to see what extremizers look like and what growth to expect.

\noindent\textbf{WORK.}
\textbf{Fast reality check (small $N$ brute force).}
For $N\le 18$ I brute-forced all subsets $A\subseteq\{1,\dots,N\}$ and computed the maximum possible value of $\sum_{n\in A}1/n$ subject to the multiplicity condition, for $r=1$ and $r=2$.
The maximizing sets and values found (rounded) were:
\[
\begin{array}{c|l|l}
N & r=1\text{ best }\sum 1/n & r=2\text{ best }\sum 1/n\\\hline
8 & 1.875\ \ (A=\{1,2,4,8\}) & 2.717857\ \ (A=\{1,\dots,8\})\\
12 & 1.986111\ (A=\{1,2,4,8,9\}) & 3.103211\ (A=\{1,\dots,12\})\\
15 & 1.986111\ (A=\{1,2,4,8,9\}) & 3.251562\ (A=\{1,\dots,14\})\\
18 & 1.986111\ (A=\{1,2,4,8,9\}) & 3.428441\ (A=\{1,\dots,14,16,17,18\})
\end{array}
\]
(These are exact for the searched range; they are not claimed to represent the asymptotic optimum.)

\medskip
\textbf{Lemma 538.1 (Erd\H{o}s double counting inequality).}
Let $A\subseteq\{1,\dots,N\}$ satisfy the multiplicity condition with parameter $r$.
Let $\mathcal P=\{p\text{ prime}: p\le N\}$.
Then
\[
\Bigl(\sum_{a\in A}\frac{1}{a}\Bigr)\Bigl(\sum_{p\in\mathcal P}\frac{1}{p}\Bigr)
\le r\sum_{m\le N^2}\frac{1}{m}.
\]
In particular,
\[
\sum_{a\in A}\frac{1}{a}\le r\,\frac{\sum_{m\le N^2}\frac{1}{m}}{\sum_{p\le N}\frac{1}{p}}.
\]

\textbf{Proof.}
Compute
\[
\Bigl(\sum_{a\in A}\frac{1}{a}\Bigr)\Bigl(\sum_{p\le N}\frac{1}{p}\Bigr)
=\sum_{a\in A}\sum_{p\le N}\frac{1}{ap}.
\]
For each ordered pair $(a,p)$, set $m=ap$. Since $a\le N$ and $p\le N$, we have $m\le N^2$.
Regrouping the double sum by the value of $m$ gives
\[
\sum_{a\in A}\sum_{p\le N}\frac{1}{ap}
=\sum_{m\le N^2}\frac{1}{m}\cdot \#\{(p,a): p\le N\text{ prime},\ a\in A,\ ap=m\}.
\]
By the hypothesis, for every $m$ the multiplicity term is at most $r$.
Therefore the whole sum is at most
\(\sum_{m\le N^2} (1/m)\cdot r\), proving the stated inequality.
Dividing by $\sum_{p\le N}1/p$ yields the second display. \qed

\medskip
\textbf{Lemma 538.2 (a family of admissible examples from $P$-smooth numbers).}
Let $P$ be any set of primes with $|P|=r$.
Define
\[
A_P:=\{a\in\{1,\dots,N\}: \text{every prime divisor of $a$ lies in $P$}\}.
\]
Then $A_P$ satisfies the multiplicity condition with parameter $r$.
Moreover,
\[
\sum_{a\in A_P}\frac1a \ge \prod_{p\in P}\frac{1}{1-1/p}\ -\ \text{(a nonnegative truncation error)}.
\]
In particular, for fixed $r$, the supremum of $\sum_{a\in A}1/a$ over admissible $A$ is at least a positive constant depending on $r$.

\textbf{Proof (multiplicity).}
Fix $m\in\mathbb N$.
If $m=pa$ with $a\in A_P$, then all prime divisors of $a$ lie in $P$, hence all prime divisors of $m$ other than $p$ lie in $P$.
If $p\notin P$, then $m$ can have no other prime divisor outside $P$ (otherwise $a=m/p$ would inherit it and fail to lie in $A_P$), so $p$ is uniquely determined as the unique prime divisor of $m$ outside $P$ (if it exists).
If $p\in P$, then different solutions correspond to different primes $p\in P$ dividing $m$ such that $m/p\le N$; there are at most $|P|=r$ such primes.
Thus for every $m$ the number of solutions is $\le r$.

\textbf{Proof (harmonic sum lower bound).}
The set of all positive integers with prime factors contained in $P$ has Euler product
\(
\sum_{a: \operatorname{supp}(a)\subseteq P} 1/a = \prod_{p\in P} (1+1/p+1/p^2+\cdots)=\prod_{p\in P} (1-1/p)^{-1}.
\)
The truncated set $A_P$ only omits terms with $a>N$, so its partial sum is at least the full Euler product minus a nonnegative tail. \qed

\noindent\textbf{VERIFICATION.}
\emph{Check Lemma 538.2 on a concrete collision.}
If $P=\{2,3\}$ and $A_P$ are the $\{2,3\}$-smooth numbers $\le N$, then for $m=2\cdot 3^t$ we have representations with $p=2$ and with $p=3$ (if $m/3\in A_P$), giving at most $2$ solutions, consistent with $r=2$.

\noindent\textbf{FINAL.}\;\textbf{UNRESOLVED.}
\begin{itemize}
\item[(i)] \textbf{Strongest proved partial result.} For every admissible $A$,
\[
\sum_{a\in A}\frac{1}{a}\le r\,\frac{\sum_{m\le N^2} \frac{1}{m}}{\sum_{p\le N}\frac{1}{p}}
\le r\,\frac{2\log N+1}{\sum_{p\le N}\frac{1}{p}},
\]
where the second inequality uses the elementary bound $\sum_{m\le N^2}1/m\le 1+\int_1^{N^2} dx/x = 1+2\log N$.
Also, the explicit admissible family $A_P$ of $P$-smooth numbers (Lemma~538.2) gives nontrivial lower bounds depending on $r$.
\item[(ii)] \textbf{First gap (crisp statement).} Determine the true asymptotic order (in $N$) of
\(\sup\{\sum_{a\in A} 1/a : A\subseteq[1,N]\text{ admissible for }(r,N)\}\)
for fixed $r$ (or describe the dependence when $r=r(N)$ grows).
\item[(iii)] \textbf{Top 3 next moves.}
  \begin{itemize}
  \item Try to show a stronger upper bound by exploiting the restriction on how many quotients $m/p$ can lie in $A$ for an $m$ with many prime factors, perhaps via a weighted sieve/entropy argument.
  \item Search for better constructions than $P$-smooth numbers that make $\sum_{a\in A}1/a$ grow with $N$ while keeping the multiplicity $\le r$.
  \item Computationally: implement integer programming/heuristic search for $N$ up to $50$ or $100$ to guess the growth rate and typical extremal structure.
  \end{itemize}
\item[(iv)] \textbf{Minimal counterexample structure to look for.} Any construction approaching the Erd\H{o}s upper bound would need many relatively small elements (to make the harmonic sum large) while ensuring that for each $m$ the set $\{m/p: p\mid m\}$ intersects $A$ in size $\le r$; in particular it must avoid including too many of the prime-divisor quotients of integers with many distinct prime factors.
\end{itemize}


