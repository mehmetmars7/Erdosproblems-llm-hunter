
\noindent\textbf{FORMAL RESTATEMENT.}
Let $u=\{u_1<u_2<\cdots\}$ be a strictly increasing sequence of integers with $u_i\ge 2$ such that
\begin{itemize}
\item $\gcd(u_i,u_j)=1$ for all $i\ne j$,
\item $\sum_{i=1}^{\infty}\frac{1}{u_i}<\infty$.
\end{itemize}
Let $A\subset\mathbb N$ be the set of positive integers not divisible by any $u_i$, and write it in increasing order as
$A=\{a_1<a_2<\cdots\}$.
For each real $x\ge 1$ define $t_x\in\mathbb N\cup\{0\}$ by
\[
U_{t_x}:=\prod_{i=1}^{t_x}u_i\le x < \prod_{i=1}^{t_x+1}u_i=U_{t_x}u_{t_x+1},
\]
(with the convention $U_0=1$).
Write
\[\delta:=\prod_{i=1}^{\infty}\Bigl(1-\frac{1}{u_i}\Bigr)\in(0,1],\qquad C:=\delta^{-1}.
\]
Call $u$ \emph{good} if for every $\varepsilon>0$ there exists $X_0(\varepsilon)$ such that for all $x\ge X_0(\varepsilon)$,
\[
\max_{a_k<x}(a_{k+1}-a_k) < (1+\varepsilon)\,t_x\,C.
\]
Question: does there exist a good sequence $u$ with $u_n<n^{O(1)}$? does there exist a good sequence with $u_n\le e^{o(n)}$?

\medskip
\noindent\textbf{QUICK LITERATURE/CONTEXT CHECK (from the problem file only).}
The problem text states:
(i) Erd\H{o}s proved existence of some good sequence with all $u_i$ prime.
(ii) Erd\H{o}s believed: no good sequence with $u_n<n^{O(1)}$, but yes with $u_n\le e^{o(n)}$.
(iii) ``An easy sieve argument'' gives, for any such $u$, the lower bound
\[
\max_{a_k<x}(a_{k+1}-a_k)>(1+o(1))\,t_x\,C.
\]
(iv) A strong form asks whether $u_i=p_i^2$ (prime squares) is good.
I do not use any external literature beyond these statements.

\medskip
\noindent\textbf{ATTACK PLAN.}
\begin{itemize}
\item (Structure) Prove exact periodic counting for the finite-prefix sieve by $u_1,\dots,u_m$ and deduce existence of density $\delta$ for $A$.
\item (Disproof track) Try to build long gaps (intervals with no elements of $A$) via CRT congruences; compare their length to the conjectured scale $t_xC$.
\item (Reality check) For the concrete sequence $u_i=p_i^2$, compute the maximal gap between squarefree numbers up to moderate $x$ and compare to $t_xC$.
\end{itemize}

\medskip
\noindent\textbf{WORK.}

\medskip
\noindent\textbf{Lemma 1 (exact finite-prefix count and periodicity).}
Fix $m\ge 1$ and set $U_m:=\prod_{i=1}^m u_i$ and
\[A^{(m)}:=\{n\in\mathbb N: u_i\nmid n\ \text{for all }1\le i\le m\}.
\]
Then for every integer $q\ge 0$ and $0\le r\le U_m$ we have
\[
|A^{(m)}\cap[1,qU_m+r]| = q\,U_m\prod_{i=1}^m\Bigl(1-\frac{1}{u_i}\Bigr) + |A^{(m)}\cap[1,r]|.
\]
In particular the natural density of $A^{(m)}$ exists and equals
\[\delta_m:=\prod_{i=1}^m\Bigl(1-\frac{1}{u_i}\Bigr),\qquad\text{i.e. }\frac{|A^{(m)}\cap[1,X]|}{X}=\delta_m+O\!\left(\frac{U_m}{X}\right).
\]

\noindent\emph{Proof.}
Membership of $n$ in $A^{(m)}$ depends only on $n\bmod U_m$ because each condition $u_i\nmid n$ depends only on $n\bmod u_i$, and $u_i\mid U_m$.
Thus the pattern of membership repeats with period $U_m$.
So to prove the first displayed identity it suffices to count the number of allowed residues in one full period $[1,U_m]$.

For each $i\le m$, among the $U_m$ residues modulo $U_m$ there are exactly $U_m/u_i$ residues divisible by $u_i$.
Because the $u_i$ are pairwise coprime, for any subset $S\subseteq\{1,\dots,m\}$ the number of residues divisible by all $u_i$ with $i\in S$ is
\[\frac{U_m}{\prod_{i\in S}u_i},
\]
since the divisibility condition is equivalently divisibility by $\prod_{i\in S}u_i$.
By inclusion--exclusion, the number of residues divisible by none of $u_1,\dots,u_m$ is
\[
U_m\sum_{S\subseteq[m]}(-1)^{|S|}\frac{1}{\prod_{i\in S}u_i}=U_m\prod_{i=1}^m\Bigl(1-\frac{1}{u_i}\Bigr)=U_m\delta_m.
\]
Therefore in each full block of length $U_m$ exactly $U_m\delta_m$ integers lie in $A^{(m)}$, which gives the claimed periodic counting formula and the density statement. \qed

\medskip
\noindent\textbf{Lemma 2 (existence and value of the density of $A$).}
The set $A$ has a natural density and
\[\lim_{X\to\infty}\frac{|A\cap[1,X]|}{X}=\delta:=\prod_{i=1}^{\infty}\Bigl(1-\frac{1}{u_i}\Bigr)\in(0,1).
\]

\noindent\emph{Proof.}
First, because $\sum_i \frac1{u_i}<\infty$, we have $\sum_i \frac1{u_i^2}<\infty$ as well.
Using the Taylor bound $\log(1-x)=-x+O(x^2)$ for $0<x\le 1/2$ and the fact that $1/u_i\to 0$, the series
\[\sum_{i=1}^{\infty}\log\Bigl(1-\frac1{u_i}\Bigr)
\]
converges, hence $\delta_m\to\delta\in(0,1)$.

Now fix $m\ge 1$. We have $A\subseteq A^{(m)}$.
If $n\in A^{(m)}\setminus A$, then $n$ is divisible by some $u_i$ with $i>m$.
Hence for every $X\ge 1$,
\[
|A^{(m)}\cap[1,X]|-|A\cap[1,X]|\le \sum_{i>m}\Bigl\lfloor\frac{X}{u_i}\Bigr\rfloor\le X\sum_{i>m}\frac1{u_i}.
\]
Divide by $X$ and let $X\to\infty$. By Lemma 1, $|A^{(m)}\cap[1,X]|/X\to\delta_m$, so
\[
0\le \delta_m-\limsup_{X\to\infty}\frac{|A\cap[1,X]|}{X} \le \sum_{i>m}\frac1{u_i}.
\]
Letting $m\to\infty$ forces the right-hand side to $0$ and $\delta_m\to\delta$, so
\[\limsup_{X\to\infty}\frac{|A\cap[1,X]|}{X}\ge \delta.
\]
On the other hand, since $A\subseteq A^{(m)}$,
\[\liminf_{X\to\infty}\frac{|A\cap[1,X]|}{X}\le \delta_m\quad\text{for every }m,
\]
so letting $m\to\infty$ gives $\liminf\le \delta$.
Combining yields $\liminf=\limsup=\delta$, i.e. the density exists and equals $\delta$. \qed

\medskip
\noindent\textbf{Lemma 3 (CRT gap of length $m$).}
For every $m\ge 1$ there exists an integer $N\ge 0$ such that
\[u_i\mid (N+i)\qquad\text{for all }1\le i\le m.
\]
Consequently, the interval $\{N+1,\dots,N+m\}$ contains no element of $A$, so for any $x\ge N+m$ one has
\[\max_{a_k<x}(a_{k+1}-a_k)\ge m.
\]

\noindent\emph{Proof.}
Consider the congruences $N\equiv -i\pmod{u_i}$ for $i=1,\dots,m$.
Because the moduli $u_i$ are pairwise coprime, the Chinese remainder theorem guarantees a simultaneous solution modulo $U_m=\prod_{i=1}^m u_i$.
Choose $N\in\{0,1,\dots,U_m-1\}$ satisfying these congruences.
Then $u_i\mid(N+i)$ for all $i\le m$.
Thus each integer in $\{N+1,\dots,N+m\}$ is divisible by at least one $u_i$, so none belongs to $A$.
If $x\ge N+m$, then among the increasing sequence $a_k$ there is some $a_{k}\le N$ and the next element $a_{k+1}\ge N+m+1$, implying $a_{k+1}-a_k\ge m+1\ge m$. \qed

\medskip
\noindent\textbf{FAST REALITY CHECK (prime squares $u_i=p_i^2$).}
For $u_i=p_i^2$, the set $A$ is exactly the squarefree positive integers and
$\delta=\prod_p(1-1/p^2)=6/\pi^2$, so $C=\pi^2/6\approx 1.644934$.
I computed by a direct sieve the maximum gap between consecutive squarefree numbers up to $x$ for several $x$.
For each $x$ I also computed $t_x$ from $(p_1\cdots p_{t_x})^2\le x < (p_1\cdots p_{t_x}p_{t_x+1})^2$.
The results were:
\[
\begin{array}{c|c|c|c}
 x & t_x & \max_{a_k\le x}(a_{k+1}-a_k) & \text{ratio }\frac{\max\text{ gap}}{t_x C}\\\hline
10^4 & 3 & 6 & 1.215854\dots\\
10^5 & 4 & 7 & 1.063872\dots\\
10^6 & 4 & 8 & 1.215854\dots\\
10^7 & 5 & 10 & 1.215854\dots
\end{array}
\]
(In each case the maximising gap occurred at the specific interval reported by the script; e.g. for $x=10^7$ the gap $10$ occurred between $8870023$ and $8870033$.)
These computations are only a sanity check at small $x$ and do not resolve goodness.

\medskip
\noindent\textbf{VERIFICATION.}
\begin{itemize}
\item Lemma 1: checked that inclusion--exclusion is valid because the $u_i$ are pairwise coprime, so intersections correspond to divisibility by the product.
\item Lemma 2: the only place convergence is used is (a) to ensure $\delta_m$ converges to a positive limit, and (b) to make the tail bound $\sum_{i>m}1/u_i$ small.
\item Lemma 3: CRT applies to the finite set of congruences; no hidden assumptions.
\item For the computation: for $u_i=p_i^2$ the identification of $A$ with squarefree integers is correct because an integer is squarefree iff it is not divisible by $p^2$ for any prime $p$.
\end{itemize}

\medskip
\noindent\textbf{FINAL: \textbf{UNRESOLVED}.}
(i) \emph{Strongest proved partial result here.} The set $A$ has positive density $\delta=\prod_i(1-1/u_i)$ (Lemma 2), and for each $m$ there exist gaps of length at least $m$ in $A$ produced by CRT (Lemma 3).

(ii) \emph{First gap (crisp).} For any explicit growth regime on $u_n$ (e.g. $u_n<n^{O(1)}$ or $u_n\le e^{o(n)}$), I do not have a proof or disproof of the upper bound
\[\max_{a_k<x}(a_{k+1}-a_k) < (1+\varepsilon)\,t_x\,\delta^{-1}\]
for all sufficiently large $x$.

(iii) \emph{Top 3 next moves.}
1. For the candidate $u_i=p_i^2$, attempt to bound maximal runs of non-squarefree numbers in terms of $t_x$ by constructing explicit coverings of intervals using many prime squares beyond the first $t_x$.
2. Try to prove (or disprove) that the extremal gap is achieved (up to $1+o(1)$ factors) by a CRT-type construction using the first $t_x$ moduli; this would convert goodness to an optimization problem over residue choices.
3. For sequences with $u_n\le e^{o(n)}$, model the sieve process as a random covering and attempt to prove a high-probability upper bound for the maximal uncovered gap using dependence estimates specific to coprime moduli.

(iv) \emph{Minimal counterexample structure.} A counterexample to ``$u$ is good'' would be a sequence $u$ with the stated hypotheses for which there exist arbitrarily large $x$ admitting an interval $[y,y+L]$ of length $L\gg t_x\,\delta^{-1}$ completely covered by the residue classes $0\bmod u_i$; equivalently, a highly non-random alignment of many moduli producing unusually long consecutive runs of excluded integers.


