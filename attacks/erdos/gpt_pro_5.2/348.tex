
## 1) FORMAL RESTATEMENT

### 1.1. Ambiguity in the word “complete”
The problem text itself notes two competing conventions:

* **Strong completeness:**
  \[
  \Sigma(A):=\Bigl\{\sum_{i\in B} a_i : B\subset \mathbb N\text{ finite}\Bigr\}=\mathbb N\cup\{0\}.
  \]
* **Weak completeness:** \(\Sigma(A)\) contains **all but finitely many** natural numbers, i.e.
  \[
  \exists N_0\ \forall N\ge N_0:\ N\in\Sigma(A).
  \]

The statement “Erd\H{o}s and Graham most likely meant the weaker notion” suggests the intended
interpretation is **weak completeness**.

I will therefore (i) give a precise treatment of both notions and (ii) prove rigorously that the
claimed examples (powers of 2; Fibonacci) satisfy the required deletion property in the **strong**
notion (hence also in the weak notion). The general classification for the weak notion remains open
(e.g. \(m=2,n=3\) is explicitly stated as unknown).

### 1.2. Conventions
* \(\mathbb N:=\{1,2,3,\dots\}\) and \(\mathbb N_0:=\{0,1,2,\dots\}\).
* A “sequence” \(A=\{a_1\le a_2\le\cdots\}\) is an infinite nondecreasing sequence of **positive** integers.
  (Duplicates are allowed; distinctness in subset sums is by **index**.)
* For a finite set of indices \(D\subset\mathbb N\), \(A\setminus D\) denotes the sequence obtained by
  removing the terms \(\{a_i: i\in D\}\) (keeping the remaining terms in nondecreasing order).

### 1.3. Property to classify
Fix integers \(0\le m<n\). We ask for which pairs \((m,n)\) there exists a complete sequence \(A\)
(say, in the weak sense unless otherwise stated) such that:

1. (**Robustness**) For every \(D\subset\mathbb N\) with \(|D|=m\), the sequence \(A\setminus D\) is complete.
2. (**Fragility**) For every \(D\subset\mathbb N\) with \(|D|=n\), the sequence \(A\setminus D\) is not complete.

Edge cases:
* \(m=0\) means simply “\(A\) is complete.”
* The “fragility” condition is monotone in \(n\) (proved below): if it holds for some \(n\), it holds
  for all \(n'\ge n\).

---

## 2) QUICK LITERATURE/CONTEXT CHECK

I do not use any external sources here; I only record what is already in the provided problem text:

* Powers of 2 give an example for \((m,n)=(0,1)\).
* Fibonacci \(1,1,2,3,5,\dots\) gives an example for \((m,n)=(1,2)\).
* \((m,n)=(2,3)\) is stated as not known (for the intended weak completeness).
* van Doorn proved that **no** such sequence exists for \(2\le m<n\) under the **strong** completeness
  definition \(\Sigma(A)=\mathbb N\) (as stated in the problem text).

---

## 3) ATTACK PLAN

**Proof track (partial results).**
1. Work in the strong completeness notion (it implies the weak notion).
2. Use a sharp necessary-and-sufficient condition for strong completeness of a nondecreasing
   positive sequence in terms of partial sums.
3. Verify the powers-of-2 and Fibonacci examples with deletion properties by checking this
   condition after deletions.

**Disproof/obstruction track.**
Try to derive constraints for \(m\ge 2\) (for the weak notion) from “gap creation” after deletions.
I did not succeed beyond what is already stated (\(m=2,n=3\) open).

---

## 4) WORK

### Phase 1: FAST REALITY CHECK (computation)

I tested the Fibonacci claims via the strong-completeness partial-sum criterion below.
Using the first 60 Fibonacci numbers \(1,1,2,3,\dots\):

* After deleting **any single** term among the first 30, the remaining sequence satisfies the
  strong-completeness inequalities through the first 50 remaining terms.
* After deleting **any pair** of terms among the first 30, the inequalities fail by (at most) the
  first 50 remaining terms.

(These are sanity checks only; the proofs below are exact and do not rely on computation.)

---

### Lemma 4.1 (Strong completeness criterion via partial sums)
Let \(A=(a_1\le a_2\le\cdots)\) be a nondecreasing sequence of positive integers, and define
\(S_k:=\sum_{i=1}^k a_i\).

**Claim.** The following are equivalent:

(A) \(A\) is **strongly complete**, i.e. every \(N\in\mathbb N_0\) is a sum of distinct terms of \(A\).

(B) \(a_1=1\) and for every \(k\ge 1\),
\[
 a_{k+1}\le S_k+1.
\]

**Proof.**

\underline{(B) \(\Rightarrow\) (A).}
We prove by induction on \(k\) that the set of subset sums of \(\{a_1,\dots,a_k\}\) contains the entire
interval \([0,S_k]\cap\mathbb N_0\).

Base \(k=1\): since \(a_1=1\), subset sums are \(\{0,1\}=[0,S_1]\).

Inductive step: assume \([0,S_k]\) is achievable using \(a_1,\dots,a_k\). Consider \(a_{k+1}\).
* Using no \(a_{k+1}\) we can make all values in \([0,S_k]\).
* Using \(a_{k+1}\) plus any subset sum of \(a_1,\dots,a_k\), we can make all values in
  \([a_{k+1}, a_{k+1}+S_k]\).

If \(a_{k+1}\le S_k+1\), then these two intervals overlap or touch:
\([0,S_k]\cup[a_{k+1},a_{k+1}+S_k]\supseteq [0,S_k+a_{k+1}]=[0,S_{k+1}]\).
Hence \([0,S_{k+1}]\) is achievable, completing the induction.

Since \(A\) is infinite with positive terms, \(S_k\to\infty\), so every \(N\in\mathbb N_0\) lies in
\([0,S_k]\) for some \(k\), hence is representable. Thus \(A\) is strongly complete.

\underline{(A) \(\Rightarrow\) (B).}
If \(a_1\ne 1\), then \(1\notin\Sigma(A)\), contradicting strong completeness, so \(a_1=1\).
Now fix \(k\ge 1\). Suppose for contradiction that \(a_{k+1}>S_k+1\).
Any subset sum using only \(a_1,\dots,a_k\) is \(\le S_k\), while any subset sum that uses
\(a_{k+1}\) is \(\ge a_{k+1}>S_k+1\). Therefore \(S_k+1\) is not representable, contradicting
strong completeness. Hence \(a_{k+1}\le S_k+1\).

This proves (A)\(\Leftrightarrow\)(B). \(\square\)

---

### Lemma 4.2 (Monotonicity of non-completeness under deletion)
Let \(A\) be any sequence of positive integers (duplicates allowed). If \(A\) is not complete
(strong or weak), then any subsequence obtained by deleting additional terms is also not complete.

**Proof.** Deleting terms can only remove available subset sums; i.e. if \(B\subseteq A\) as
multisets (or by indices), then \(\Sigma(B)\subseteq\Sigma(A)\). If \(\Sigma(A)\) misses infinitely
many integers (weak incompleteness) or misses some integer (strong incompleteness), then
\(\Sigma(B)\) also misses those integers. \(\square\)

**Consequence.** If a sequence satisfies the “fragility” condition for some \(n\), then it also
satisfies it for every \(n'\ge n\).

---

### Proposition 4.3 (Powers of 2 witness \(m=0,n=1\) in the strong sense)
Let \(A=(1,2,4,8,\dots)\) where \(a_i=2^{i-1}\).

1. \(A\) is strongly complete.
2. For every single deletion \(D=\{t\}\) (remove \(2^{t-1}\)), the sequence \(A\setminus D\) is not
   complete (hence the pair \((m,n)=(0,1)\) is achievable).

**Proof.**

1. (Strong completeness.) Here \(S_k=1+2+\cdots+2^{k-1}=2^k-1\), so \(a_{k+1}=2^k=S_k+1\).
   By Lemma 4.1, \(A\) is strongly complete.

2. (Deleting one term breaks completeness.) Fix \(t\ge 1\) and delete \(2^{t-1}\). Consider the number
   \(N:=2^{t-1}\). Any subset sum of the remaining powers of 2 is either
   * \(<2^{t-1}\) if it uses only smaller powers \(1,2,\dots,2^{t-2}\), since their sum is
     \(2^{t-1}-1\), or
   * \(\ge 2^t\) if it uses any larger power (since the smallest remaining larger power is \(2^t\)).

   Thus \(N\) cannot be represented, so \(A\setminus\{t\}\) is not strongly complete.

By Lemma 4.2, deleting any \(n\ge 1\) terms also yields a non-complete sequence. \(\square\)

---

### Proposition 4.4 (Fibonacci witnesses \(m=1,n=2\) in the strong sense)
Let \((F_k)_{k\ge 1}\) be Fibonacci numbers with \(F_1=1, F_2=1\), and \(F_{k+2}=F_{k+1}+F_k\).
Let \(A=(F_1,F_2,F_3,\dots)=(1,1,2,3,5,\dots)\).

We prove:

1. \(A\) is strongly complete.
2. After deleting any single term, the remaining sequence is still strongly complete.
3. After deleting any two terms, the remaining sequence is not strongly complete.

Together, (2) and (3) give the required property for \((m,n)=(1,2)\).

#### Identity used
For all \(k\ge 1\),
\[
\sum_{i=1}^k F_i = F_{k+2}-1.
\]
This is proved by induction: it is true for \(k=1\), and
\(\sum_{i=1}^{k+1}F_i=(F_{k+2}-1)+F_{k+1}=F_{k+3}-1\) using the recurrence.

#### (1) Strong completeness of Fibonacci
We verify Lemma 4.1(B). We have \(F_1=1\). For \(k\ge 1\), using the identity,
\[
S_k=\sum_{i=1}^k F_i = F_{k+2}-1\quad\Rightarrow\quad S_k+1=F_{k+2}.
\]
Since \(F_{k+1}<F_{k+2}\), we have \(F_{k+1}\le S_k+1\). Thus Lemma 4.1 implies \(A\) is strongly
complete.

#### (2) Deleting any one Fibonacci number preserves strong completeness
Fix \(t\ge 1\) and delete the single term \(F_t\). Let the remaining sequence be
\(B=(b_1\le b_2\le\cdots)\); note \(b_1=1\) always (because either \(F_1\) or \(F_2\) remains).

We verify Lemma 4.1(B) for \(B\). Consider any stage where the next element of \(B\) is \(F_{k+1}\).
There are two cases.

*If \(k<t\)*: then none of \(F_1,\dots,F_k\) have been deleted yet (since only \(F_t\) was deleted), so
the partial sum of previous terms equals \(\sum_{i=1}^k F_i = F_{k+2}-1\). Hence
\(F_{k+1}\le F_{k+2}= (F_{k+2}-1)+1\), so the inequality holds.

*If \(k\ge t\)*: then among \(F_1,\dots,F_k\) we are missing exactly \(F_t\). Thus the partial sum of
previous terms equals
\[
\sum_{i=1}^k F_i - F_t = (F_{k+2}-1)-F_t.
\]
Adding 1 gives \(F_{k+2}-F_t\). Using \(F_{k+2}=F_{k+1}+F_k\), the desired inequality
\(F_{k+1}\le (\text{previous sum})+1\) becomes
\[
F_{k+1}\le F_{k+2}-F_t = F_{k+1}+F_k-F_t.
\]
This holds because \(k\ge t\Rightarrow F_k\ge F_t\) (Fibonacci is nondecreasing). Therefore
Lemma 4.1 applies and \(B\) is strongly complete.

#### (3) Deleting any two Fibonacci numbers destroys strong completeness
Fix distinct indices \(1\le a<b\) and delete \(F_a\) and \(F_b\). Let \(C\) be the remaining sequence.
Consider the stage just before \(F_{b+1}\) in \(C\): all Fibonacci numbers \(F_1,\dots,F_b\) except
\(F_a,F_b\) are included among the previous terms.

Hence the partial sum of previous terms is
\[
\sum_{i=1}^b F_i - F_a - F_b = (F_{b+2}-1) - F_a - F_b.
\]
Adding 1 gives
\[
(F_{b+2}-1)-F_a-F_b +1 = F_{b+2}-F_a-F_b.
\]
Using \(F_{b+2}=F_{b+1}+F_b\), this equals
\[
F_{b+1}+F_b -F_a -F_b = F_{b+1}-F_a.
\]
Since \(F_a\ge 1\), we have \(F_{b+1}-F_a < F_{b+1}\), i.e.
\[
F_{b+1} > (\text{previous sum})+1.
\]
Thus the necessary condition in Lemma 4.1(B) fails, so \(C\) is not strongly complete.

This proves (1)–(3). \(\square\)

---

## 5) VERIFICATION

* Lemma 4.1: the forward direction uses a standard interval-cover induction; the backward direction
  isolates the specific missing number \(S_k+1\) when there is a “gap”. Quantifiers are correct.
* Proposition 4.3: after deleting \(2^{t-1}\), the target number \(2^{t-1}\) cannot be made because all
  sums of smaller remaining powers are \(\le 2^{t-1}-1\), and any use of a larger remaining power is
  \(\ge 2^t\).
* Proposition 4.4(3): the failure occurs exactly at the next Fibonacci number after the largest
  deleted index, so it does not depend on computational cutoffs.
* Edge case: deleting both \(F_1\) and \(F_2\) leaves \(2,3,5,\dots\); the proof still detects failure at
  \(F_3\) since the “previous sum” is 0.

---

## 6) FINAL

**UNRESOLVED**

(i) **Strongest fully proved partial result obtained here.**

* In the **strong** sense, powers of 2 give a complete sequence such that deleting any \(\ge 1\)
  element(s) destroys completeness (Proposition 4.3). In particular \((m,n)=(0,1)\) works (and by
  monotonicity also \((0,n)\) for any \(n\ge 1\)).
* In the **strong** sense, Fibonacci \(1,1,2,3,5,\dots\) remains complete after deleting any one term
  but becomes incomplete after deleting any two terms (Proposition 4.4). Thus \((m,n)=(1,2)\) works
  (and by monotonicity also \((1,n)\) for any \(n\ge 2\)).

(ii) **Exact first gap.**

Under the intended **weak** completeness notion (all but finitely many integers representable), the
problem asks for a classification of \((m,n)\). The first explicit open instance stated in the
problem text is:

> Does there exist a weakly complete sequence that remains weakly complete after deleting any
> \(m=2\) terms, but is not weakly complete after deleting any \(n=3\) terms?

(iii) **Top 3 next moves (concrete targets).**

1. **Translate weak completeness into a checkable inequality with “finite slack”:** find a necessary
   and sufficient criterion (or a usable sufficient criterion) analogous to Lemma 4.1 but allowing a
   bounded number of initial gaps.
2. **Attempt a “redundant Fibonacci” construction:** build a sequence satisfying a near-Fibonacci
   growth condition that is stable under 2 deletions but where any 3 deletions force a gap in the
   partial-sum inequality at some later scale.
3. **Computational search for \((m,n)=(2,3)\):** restrict to “structured” candidates (linear recurrences
   or block-repeated greedy sequences) and test the deletion property on large finite prefixes to
   discover patterns that suggest a proof.

(iv) **What a minimal counterexample would likely look like.**

If \((m,n)=(2,3)\) is impossible in the weak sense, a minimal obstruction would be a theorem of the
form:

> Any weakly complete sequence has a set of 3 terms whose removal still leaves infinitely many
> representable integers.

Equivalently, weak completeness would impose enough redundancy that some triple of deletions cannot
kill it. Such a result would need a structural description of weakly complete sequences beyond the
strong partial-sum criterion.


