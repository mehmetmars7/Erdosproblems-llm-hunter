% Erdos Problem #404
% URL: https://www.erdosproblems.com/404

1) FORMAL RESTATEMENT

Fix an integer $a\ge 1$ and a prime $p$.

For a finite strictly increasing sequence of integers
\[
a=a_1<a_2<\cdots<a_n,
\]
define the factorial sum
\[
S:=a_1!+a_2!+\cdots+a_n!.
\]

For each such pair $(a,p)$, define
\[
f(a,p):=\sup\{k\ge 0:\ \exists\ n\ge 1,\ \exists\ a=a_1<\cdots<a_n\text{ with }p^k\mid S\}.
\]
(So $f(a,p)=0$ means no such sum is divisible by $p$; if arbitrarily large $k$ occur we interpret $f(a,p)=\infty$.)

The problem asks:

(A) For which $(a,p)$ is $f(a,p)$ finite (i.e. there is an upper bound on achievable $k$)? How does $f(a,p)$ behave?

(B) Does there exist a prime $p$ and an infinite sequence $a_1<a_2<\cdots$ such that if $m_k$ is the exact $p$-adic valuation
\[
m_k:=v_p\Bigl(\sum_{i\le k} a_i!\Bigr),
\]
then $m_k\to\infty$ as $k\to\infty$?

Edge cases.
If $a=p-1$ then $a!$ is not divisible by $p$ but all $m!$ for $m\ge p$ are divisible by $p$.

2) QUICK LITERATURE/CONTEXT CHECK

I only restate what is recorded in the problem statement.

The statement records:
- Lin showed $f(2,2)\le 254$.

3) ATTACK PLAN

Proof-track (partial):
- Establish elementary constraints on $p$-adic valuations of factorials and simple sums of factorials.
- Identify pairs $(a,p)$ where $f(a,p)=0$ by a modulo $p$ obstruction.
- Compute small examples for sanity.

Disproof-track:
- Try to build explicit constructions forcing $v_p$ to grow (computational evidence only).

Chosen path: prove a few exact lemmas (including an explicit family with $f(a,p)=0$) and give brute-force lower bounds for small $(a,p)$.

4) WORK

PHASE 1 — FAST REALITY CHECK (computed)

For small $M$ I brute-forced over subsets of $\{a,a+1,\dots,M\}$ (always including $a$) to maximise $v_p\bigl(\sum i!\bigr)$.
These computations give lower bounds on $f(a,p)$.

Examples (all exact within the stated search cutoff):

- For $(a,p)=(1,2)$, the maximum $v_2$ found up to $M=20$ is $0$ (subset $\{1\}$), suggesting $f(1,2)=0$.
- For $(a,p)=(2,2)$, up to $M=18$ I found a subset $\{2,3,5,6,7,11,12,15,16,18\}$ with
\[
 v_2\Bigl(\sum_{i\in S} i!\Bigr)=17,
\]
so $f(2,2)\ge 17$.
- For $(a,p)=(2,3)$, up to $M=20$ the best $v_3$ was $0$ (subset $\{2\}$), suggesting $f(2,3)=0$.

Lemma 1 (elementary lower bound for factorial valuations).
For every prime $p$ and integer $m\ge 1$,
\[
 v_p(m!)\ge \Bigl\lfloor\frac{m}{p}\Bigr\rfloor.
\]
In particular, if $m\ge kp$ then $p^k\mid m!$.

Proof.
Among the factors $1,2,\dots,m$ there are exactly $\lfloor m/p\rfloor$ multiples of $p$, and each contributes at least one factor of $p$ to the product $m!$. Therefore the total exponent of $p$ in $m!$ is at least $\lfloor m/p\rfloor$. The final statement follows because $\lfloor m/p\rfloor\ge k$ when $m\ge kp$. \qed

Lemma 2 (a two-term valuation identity).
For every integer $a\ge 1$,
\[
 a!+(a+1)! = a!(a+2).
\]
Consequently, for every prime $p$,
\[
 v_p\bigl(a!+(a+1)!\bigr)=v_p(a!)+v_p(a+2).
\]

Proof.
Factor out $a!$:
\[
 a!+(a+1)! = a! + (a+1)a! = a!(1+a+1)=a!(a+2).
\]
Taking $p$-adic valuations and using $v_p(xy)=v_p(x)+v_p(y)$ gives the valuation formula. \qed

Lemma 3 (an explicit infinite family with $f(a,p)=0$).
For every prime $p$,
\[
 f(p-1,p)=0.
\]
Equivalently, if $a=p-1$ then no sum $a_1!+\cdots+a_n!$ with $a_1=a$ is divisible by $p$.

Proof.
Fix prime $p$ and set $a=p-1$.
Any admissible sequence has $a_1=p-1$ and all other terms (if any) satisfy $a_i\ge p$. Therefore each $a_i!$ for $i\ge 2$ is divisible by $p$. Hence modulo $p$ we have
\[
 a_1!+a_2!+\cdots+a_n! \equiv (p-1)! \pmod p.
\]
But $(p-1)!$ is a product of integers all strictly less than $p$, so $p\nmid (p-1)!$. Thus the sum is nonzero mod $p$, so it is not divisible by $p$. Therefore no power $p^k$ with $k\ge 1$ divides the sum, i.e. $f(p-1,p)=0$. \qed

Corollary 4.
The case $(a,p)=(2,3)$ satisfies $f(2,3)=0$ (since $2=3-1$), consistent with the computation.

5) VERIFICATION

- Lemma 1: for $p=2$, $v_2(5!)=3\ge\lfloor 5/2\rfloor=2$; the inequality is weak but always correct.
- Lemma 2: tested numerically for small $a$; the factorisation is exact.
- Lemma 3: for $p=2$, $a=1$ gives sums always odd, so $f(1,2)=0$; for $p=3$, $a=2$ gives sums always $\equiv 2\pmod 3$, so $f(2,3)=0$.

6) FINAL

**UNRESOLVED**

(i) Strongest fully proved partial result obtained here.
- Exact family with $f(p-1,p)=0$ for all primes $p$ (Lemma 3).
- General tools: $v_p(m!)\ge\lfloor m/p\rfloor$ (Lemma 1) and the exact identity $v_p(a!+(a+1)!)=v_p(a!)+v_p(a+2)$ (Lemma 2).
- Computational lower bounds: for example $f(2,2)\ge 17$ (from an explicit subset up to $18$).

(ii) Exact first gap.
Determine for any fixed pair $(a,p)$ with $a\not\equiv -1\pmod p$ whether $f(a,p)$ is finite or infinite; in particular, exhibit any $(a,p)$ for which one can prove $f(a,p)=\infty$ (unbounded $p$-adic valuation) or prove a nontrivial finite upper bound.

(iii) Top 3 next moves (concrete targets).
1. Develop a systematic $p$-adic “greedy” algorithm to choose additional factorial terms to cancel the low-order $p$-adic digits of the partial sum and thereby increase $v_p$.
2. For specific small pairs (e.g. $(a,p)=(2,2)$), attempt to prove unboundedness by constructing an explicit infinite sequence $a_1<a_2<\cdots$ with $v_p$ of partial sums strictly increasing.
3. Conversely, seek rigid obstructions (e.g. working modulo $p^{k+1}$ once $a!$ has exact valuation $k$ and later factorials are all multiples of $p^{k+1}$) that would force $f(a,p)$ finite in certain regimes.

(iv) What a minimal counterexample would likely look like.
A counterexample to the existence of any $(a,p)$ with $f(a,p)=\infty$ would be a proof that for every $a,p$ there exists a maximal $k$ such that $p^k\mid\sum a_i!$. Such a proof would likely show that once $v_p$ reaches a certain level, the residues contributed by the finitely many factorials with valuation exactly that level cannot be arranged to cancel modulo the next power.


