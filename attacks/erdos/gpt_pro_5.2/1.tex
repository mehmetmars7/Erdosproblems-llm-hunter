% Solutions / work-log generated by ChatGPT following PROMPT_STRATEGY.MD

% Erdős Problem #1
% URL: https://www.erdosproblems.com/1

\noindent\textbf{1) FORMAL RESTATEMENT.}

Let $n\ge 1$ and $N\ge 1$ be integers and let $A\subseteq\{1,\dots,N\}$ with $|A|=n$. For $S\subseteq A$ write
\[
\sigma(S)=\sum_{a\in S} a\in\mathbb{Z}_{\ge 0}.
\]
Assume that the map $S\mapsto\sigma(S)$ is injective on $\mathcal{P}(A)$ (equivalently, all $2^n$ subset sums are distinct). The conjectured conclusion is that
\[
N\gg 2^n,
\]
meaning: there exists an absolute constant $c>0$ such that for every such pair $(A,N)$ one has $N\ge c\,2^n$.

\medskip
\noindent\textbf{2) QUICK LITERATURE/CONTEXT CHECK.}

The problem statement itself records:

\begin{itemize}
\item a trivial bound $N\gg 2^n/n$ (from $2^n$ distinct integers lying in $[0,Nn)$);
\item an Erd\H{o}s--Moser bound $N\ge (\tfrac14-o(1))\,2^n/\sqrt{n}$;
\item improvements of the constant and a statement that Dubroff--Fox--Xu prove $N\ge \binom{n}{\lfloor n/2\rfloor}$.
\end{itemize}

Per the project integrity rules, in what follows I do \emph{not} assert any external results beyond what is written in the problem file; I only prove elementary consequences of the hypotheses and do small-case computations.

\medskip
\noindent\textbf{3) ATTACK PLAN.}

\begin{itemize}
\item \emph{Lower bounds from pigeonhole/volume.} Use that $2^n$ distinct subset sums must fit into an interval of integers controlled by $N$ (or by finer structural data such as how subset sums concentrate).
\item \emph{Additive-structure approach.} The injectivity of subset sums implies strong additive independence (no nontrivial $\{0,\pm1\}$-relation). Try to convert this into a strong lower bound on the maximum element.
\item \emph{Disproof search.} Attempt to construct families of such sets with $N=o(2^n)$; small-$n$ computations can suggest the growth of the minimal feasible $N$.
\end{itemize}

\medskip
\noindent\textbf{4) WORK.}

\noindent\textbf{Lemma 1 (trivial but sharp bookkeeping).}
Let $A\subseteq\{1,\dots,N\}$ with $|A|=n$ have distinct subset sums. Then
\[
\sum_{a\in A} a \;\ge\; 2^n-1,
\]
and consequently
\[
N\;\ge\;\frac{2^n-1}{n}.
\]

\noindent\emph{Proof.}
Let $T:=\sum_{a\in A} a$. Every subset sum $\sigma(S)$ is an integer in the interval $[0,T]$. Since there are exactly $2^n$ subsets of $A$ and the subset-sum map is injective, the set
\[
\Sigma:=\{\sigma(S): S\subseteq A\}
\]
has size $|\Sigma|=2^n$ and lies in $\{0,1,\dots,T\}$, a set of size $T+1$. Therefore $T+1\ge 2^n$, i.e. $T\ge 2^n-1$.

Because each $a\in A$ satisfies $a\le N$, we have $T\le nN$. Combining with $T\ge 2^n-1$ gives $nN\ge 2^n-1$, hence $N\ge (2^n-1)/n$. \hfill$\square$

\medskip
\noindent\textbf{Lemma 2 (pairwise sums are distinct; Sidon consequence).}
Let $A\subseteq\mathbb{Z}$ have distinct subset sums. Then all two-term sums $a+b$ with $a,b\in A$ and $a<b$ are distinct. Equivalently, if
\[
a_i+a_j=a_k+a_\ell\quad\text{with } i<j,\ k<\ell,
\]
then $\{i,j\}=\{k,\ell\}$.

\noindent\emph{Proof.}
Assume $a_i+a_j=a_k+a_\ell$ with $i<j$ and $k<\ell$. Consider the two 2-element subsets $S:=\{a_i,a_j\}$ and $T:=\{a_k,a_\ell\}$. We have $\sigma(S)=\sigma(T)$ by the displayed equality.

Since subset sums are distinct, $\sigma(S)=\sigma(T)$ forces $S=T$ as sets. In particular, the underlying index sets coincide, so $\{i,j\}=\{k,\ell\}$.

To be explicit about overlaps: if $i=k$ then the equality reduces to $a_j=a_\ell$, and since the elements of $A$ are distinct, this implies $j=\ell$, again giving $\{i,j\}=\{k,\ell\}$. The same reasoning applies to any single-index overlap. \hfill$\square$

\medskip
\noindent\textbf{Lemma 3 (a weak quadratic lower bound via distinct differences).}
Let $A\subseteq\{1,\dots,N\}$ have distinct subset sums and $|A|=n$. Then
\[
N\ge \frac{n(n-1)}{2}+1.
\]

\noindent\emph{Proof.}
Order $A$ as $a_1<a_2<\dots<a_n$. By Lemma 2, $A$ is Sidon in the sense that all pairwise sums $a_i+a_j$ (with $i<j$) are distinct. We claim all positive differences $a_j-a_i$ with $j>i$ are also distinct.

Suppose $a_j-a_i=a_\ell-a_k$ for some indices with $j>i$ and $\ell>k$. Rearranging gives
\[
a_j+a_k=a_\ell+a_i.
\]
By Lemma 2 (distinct pairwise sums), this implies $\{j,k\}=\{\ell,i\}$. Because $j>i$ and $\ell>k$, the only way the unordered pairs can agree is $j=\ell$ and $k=i$ (the alternative $j=i$ would contradict $j>i$). Hence $(i,j)=(k,\ell)$, proving distinctness of positive differences.

There are $\binom{n}{2}=n(n-1)/2$ distinct positive differences. Each difference satisfies $1\le a_j-a_i\le N-1$. Therefore the interval $\{1,2,\dots,N-1\}$ must contain at least $\binom{n}{2}$ distinct integers, i.e. $N-1\ge \binom{n}{2}$. This yields $N\ge \binom{n}{2}+1$. \hfill$\square$

\medskip
\noindent\textbf{FAST REALITY CHECK (small $n$).}

I ran an exhaustive backtracking search over $A\subseteq\{1,\dots,N\}$ for $n\le 6$ to find the \emph{minimal} $N$ for which such an $A$ exists, together with one witness $A$. The exact results found are:
\[
\begin{array}{c|c|l}
 n & \min N & \text{one witness }A\subseteq\{1,\dots,N\}\\\hline
 1 & 1 & \{1\}\\
 2 & 2 & \{1,2\}\\
 3 & 4 & \{1,2,4\}\\
 4 & 7 & \{3,5,6,7\}\\
 5 & 13 & \{3,6,11,12,13\}\\
 6 & 24 & \{11,17,20,22,23,24\}
\end{array}
\]
For each witness, the script verified that the set of subset sums has size $2^n$.

\medskip
\noindent\textbf{5) VERIFICATION.}

\begin{itemize}
\item Lemma 1: checked that the argument only uses injectivity and the integer interval $[0,T]$; no hidden assumptions.
\item Lemma 2: verified that any equality of two-term sums would give equality of subset sums of 2-element subsets, contradicting injectivity.
\item Lemma 3: verified the implication \,$a_j-a_i=a_\ell-a_k\Rightarrow a_j+a_k=a_\ell+a_i$\, and that Lemma 2 forces the unordered pair equality, from which $(i,j)=(k,\ell)$ follows because of the strict inequalities $j>i$, $\ell>k$.
\item Computation: for each listed witness $A$, a direct subset-sum enumeration confirms exactly $2^n$ distinct sums.
\end{itemize}

\medskip
\noindent\textbf{6) FINAL.}

\textbf{UNRESOLVED}

(i) \emph{Strongest proved partial result here.} If $A\subseteq\{1,\dots,N\}$ with $|A|=n$ has all subset sums distinct, then
\[N\ge \frac{2^n-1}{n}\qquad\text{and also}\qquad N\ge \frac{n(n-1)}{2}+1.\]
Additionally, an exhaustive search found the exact minima for $n\le 6$ listed in the FAST REALITY CHECK table.

(ii) \emph{First gap (crisp).} Prove (or disprove) the existence of an absolute $c>0$ such that every such $A\subseteq\{1,\dots,N\}$ satisfies $N\ge c\,2^n$.

(iii) \emph{Top 3 next moves (concrete).}
\begin{enumerate}
\item Prove a nontrivial \emph{anti-concentration} statement for subset sums of a dissociated set in $\mathbb{Z}$: for example, show that the $2^n$ subset sums must occupy many disjoint intervals of length comparable to $N$, forcing $N$ to be a positive fraction of $2^n$.
\item Search for explicit constructions with $N=o(2^n)$ (e.g., $N\asymp 2^n/\sqrt{n}$) by computer-guided heuristics: start from the minimal examples for $n\le 6$ and attempt greedy/ILP extensions to larger $n$ while tracking $N/2^n$.
\item Attempt to upgrade Lemma 1 by proving that many subset sums must lie in a short interval of length $O(N)$ (e.g., by pairing subsets of equal size, or by a median/quantile argument); any lower bound of the form ``at least $\alpha 2^n$ sums lie in an interval of length $\beta N$'' would give $N\ge (\alpha/\beta)2^n$.
\end{enumerate}

(iv) \emph{Minimal counterexample structure.} A disproof would require a sequence of pairs $(A_n,N_n)$ with $|A_n|=n$, $A_n\subseteq\{1,\dots,N_n\}$, all subset sums distinct, and $N_n/2^n\to 0$. Empirically for $n\le 6$ the minimizing sets are relatively ``top-heavy'' (many elements close to $N$), suggesting any counterexample sequence may have elements clustered near $N_n$ with finely tuned additive independence.

\bigskip

