
\noindent\textbf{1) FORMAL RESTATEMENT.}

Define
\[
A:=\left\{\text{odd }n\in\mathbb{N}: \nexists\text{ prime }p\text{ and integers }k,\ell\ge 0\text{ with } n=p+2^k+2^\ell\right\}.
\]
The question asks whether the \emph{upper density} of $A$ is positive:
\[
\overline{d}(A):=\limsup_{N\to\infty}\frac{|A\cap\{1,2,\dots,N\}|}{N} \;>\;0\ ?
\]

\medskip
\noindent\textbf{2) QUICK LITERATURE/CONTEXT CHECK.}

The problem text states (without providing proofs here) that $A$ is infinite (credited to Schinzel) and gives quantitative lower bounds on $|A\cap\{1,\dots,N\}|$ due to Crocker and Pan. I do not reprove those results. I only prove elementary observations about the representation condition and report computational data in a finite range.

\medskip
\noindent\textbf{3) ATTACK PLAN.}

\begin{itemize}
\item Use parity constraints to understand which $(p,2^k+2^\ell)$ combinations can produce an odd $n$.
\item Bound the number of possible offsets $2^k+2^\ell$ up to a cutoff to understand the ``degrees of freedom''.
\item Compute $A\cap[1,X]$ for a moderately large $X$ to see how early exceptions appear.
\end{itemize}

\medskip
\noindent\textbf{4) WORK.}

\noindent\textbf{Lemma 1 (parity classification for odd targets).}
Let $n$ be odd. If $n=p+2^k+2^\ell$ with $p$ prime and $k,\ell\ge 0$, then exactly one of the following holds:
\begin{enumerate}
\item $p=2$ and $2^k+2^\ell$ is odd (equivalently, exactly one of $k,\ell$ equals $0$);
\item $p$ is odd and $2^k+2^\ell$ is even (equivalently, either $k=\ell=0$ or both $k,\ell\ge 1$).
\end{enumerate}

\noindent\emph{Proof.}
Since $n$ is odd, $p$ and $2^k+2^\ell$ must have opposite parity.

If $p=2$ is even, then $2^k+2^\ell$ must be odd. A sum of two powers of $2$ is odd if and only if exactly one of the summands equals $1$ (i.e. exactly one of $k,\ell$ is $0$), because all other powers of $2$ are even.

If $p$ is an odd prime, then $2^k+2^\ell$ must be even. The sum $2^k+2^\ell$ is even precisely when $2^k$ and $2^\ell$ have the same parity, which occurs when either both are $1$ (i.e. $k=\ell=0$) or both are even (i.e. $k,\ell\ge 1$). \hfill$\square$

\medskip
\noindent\textbf{Lemma 2 (counting possible offsets).}
For $X\ge 1$, the number of distinct integers of the form $2^k+2^\ell$ with $k,\ell\ge 0$ and $2^k+2^\ell\le X$ is at most $(\lfloor\log_2 X\rfloor+1)^2$.

\noindent\emph{Proof.}
If $2^k\le X$ then $k\le \lfloor\log_2 X\rfloor$. Thus any pair $(k,\ell)$ that can contribute must satisfy
\[0\le k,\ell\le \lfloor\log_2 X\rfloor.
\]
There are $(\lfloor\log_2 X\rfloor+1)^2$ such ordered pairs, and each produces at most one value of $2^k+2^\ell$. The number of distinct values is therefore bounded above by the number of pairs. \hfill$\square$

\medskip
\noindent\textbf{Lemma 3 (a very common representation).}
If $n$ is odd and $n-2$ is an odd prime, then $n\notin A$; in fact
\[
n=(n-2)+2^0+2^0.
\]

\noindent\emph{Proof.}
If $n-2$ is prime, take $p=n-2$ and $k=\ell=0$. Then $2^0+2^0=2$, hence $p+2^k+2^\ell=(n-2)+2=n$. This exhibits a valid representation, so $n\notin A$. \hfill$\square$

\medskip
\noindent\textbf{FAST REALITY CHECK (computation).}

I computed $A\cap[1,X]$ for $X=20{,}000{,}000$ as follows:
\begin{itemize}
\item sieve all primes up to $X$;
\item precompute the set $\mathcal{S}_X:=\{2^k+2^\ell\le X: k,\ell\ge 0\}$ (there are $322$ distinct such offsets for $X=20{,}000{,}000$);
\item for each odd $n\le X$, test whether $n-s$ is prime for some $s\in\mathcal{S}_X$.
\end{itemize}
The exact result found was
\[
A\cap[1,20{,}000{,}000] = \{1,3\},
\]
so among the $10{,}000{,}000$ odd numbers up to $X$ there were exactly $2$ exceptions in this range.

\medskip
\noindent\textbf{5) VERIFICATION.}

\begin{itemize}
\item Lemma 1: parity is checked explicitly using that $2^0=1$ is odd and $2^k$ is even for $k\ge 1$.
\item Lemma 2: the only input is that $2^k\le X\iff k\le\lfloor\log_2 X\rfloor$.
\item Lemma 3: direct substitution $2^0+2^0=2$.
\item Computation: the representation test correctly required $n-s\ge 2$ and primality of $n-s$; offsets $s>X$ were not used because $s\le X$ by construction.
\end{itemize}

\medskip
\noindent\textbf{6) FINAL.}

\textbf{UNRESOLVED}

(i) \emph{Strongest proved partial result here.} Parity constraints force any representation of an odd $n$ to fall into the two cases of Lemma 1, and the number of admissible offsets $2^k+2^\ell\le X$ grows at most quadratically in $\log_2 X$ (Lemma 2). A computation up to $X=20{,}000{,}000$ found only $\{1,3\}$ in $A$ in that range.

(ii) \emph{First gap (crisp).} Decide whether $\overline d(A)>0$; equivalently, determine whether there exists $\delta>0$ and arbitrarily large $N$ with $|A\cap[1,N]|\ge \delta N$.

(iii) \emph{Top 3 next moves (concrete).}
\begin{enumerate}
\item Push the computation of $A\cap[1,X]$ to much larger $X$ (using a segmented sieve) to locate the first nontrivial exceptions beyond $3$ and estimate growth of $|A\cap[1,X]|$.
\item Study the problem in residue classes: for a fixed modulus $q$, analyze whether the set of representable numbers $\{p+2^k+2^\ell\}$ occupies all reduced classes mod $q$ (a ``covering'' heuristic mentioned in the problem text).
\item Attempt to bound from above the number of representable odds up to $X$ by combining (a) the number of offsets (Lemma 2) with (b) nontrivial uniform upper bounds on primes in shifted sets $\{n-s\}$; any bound of the form $|\{n\le X: n\text{ representable}\}|\le (1-\delta)X$ would imply $\overline d(A)\ge \delta$.
\end{enumerate}

(iv) \emph{Minimal counterexample structure.} If $\overline d(A)=0$, then for every $\delta>0$ there exists $N_0$ such that $|A\cap[1,N]|<\delta N$ for all $N\ge N_0$; in that regime almost every odd integer is representable as $p+2^k+2^\ell$. If instead $\overline d(A)>0$, then there exists $\delta>0$ and arbitrarily large $N$ for which a positive fraction of odd $n\le N$ avoid every prime shift by the sparse offset set $\{2^k+2^\ell\}$. Any proof must control how the shifted prime sets overlap as $k,\ell$ vary.

