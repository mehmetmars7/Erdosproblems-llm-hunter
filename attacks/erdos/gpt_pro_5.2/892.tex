


\noindent\textbf{1) FORMAL RESTATEMENT}

Let $(b_n)_{n\ge 1}$ be a strictly increasing sequence of positive integers.
We ask for conditions on $(b_n)$ that guarantee the existence of a strictly
increasing \emph{primitive} sequence $(a_n)_{n\ge 1}$ of integers with

\[ a_n \le C\, b_n \quad \text{for all } n\ge 1 \]

for some absolute constant $C\ge 1$ (depending on $b_\bullet$ but not on $n$).
Here \emph{primitive} means: for all $i<j$, $a_i\nmid a_j$.

The problem also asks, in particular, whether such an $(a_n)$ always exists under
an additional hypothesis on $(b_n)$:

\begin{quote}
(\*) There are no \emph{non-trivial} solutions to $(b_i,b_j)=b_k$.
\end{quote}

\emph{Ambiguity.} Since $b_i<b_j$ implies $(b_i,b_j)\le b_i$, any equality
$(b_i,b_j)=b_k$ forces $k\le i$. Thus the only “trivial” equalities that can
occur with $i<j$ are $(b_i,b_j)=b_i$ (equivalently $b_i\mid b_j$), with $k=i$.
A natural minimal interpretation of (\*) is therefore:

\[ \forall\, i<j,\; (b_i,b_j)\notin\{b_1,\dots,b_{i-1}\}. \]

I will keep this interpretation; all statements below that mention (\*) will
explicitly use only this meaning.

\medskip
\noindent\textbf{2) QUICK LITERATURE/CONTEXT CHECK}

The problem statement (input file) records two known necessary conditions on $b_n$:
\(\sum_n 1/(b_n\log b_n)<\infty\) (attributed there to Erd\H{o}s (1935)) and
\(\sum_{b_n<x}1/b_n = o(\log x/\sqrt{\log\log x})\) (attributed there to
Erd\H{o}s--S\'{a}rk\"{o}zy--Szemer\'{e}di (1967)).
No sufficient condition is given there, and the existence of a necessary and
sufficient condition is the open question.

\medskip
\noindent\textbf{3) ATTACK PLAN}

\emph{Proof track (partial results):}
\begin{itemize}
\item Derive clean necessary conditions on $(b_n)$ from the existence of a
primitive $(a_n)$ with $a_n\le Cb_n$.
\item Give at least one transparent sufficient condition on $(b_n)$ that allows
an explicit construction of a primitive $(a_n)$ with $a_n\le Cb_n$.
\end{itemize}

\emph{Disproof/construction track:}
\begin{itemize}
\item Test the “in particular” hypothesis (\*) on toy sequences: does it prevent
obvious obstructions (e.g. forcing $b_n$ to grow too slowly)?
\item Search small prefixes computationally for feasibility of a primitive
approximation under fixed $C$.
\end{itemize}

\medskip
\noindent\textbf{4) WORK}

\textbf{PHASE 1 — FAST REALITY CHECK (small prefixes).}

I ran a small backtracking search for finite prefixes $b_1,\dots,b_m$ asking
whether there exist strictly increasing integers $a_1<\dots<a_m$ with
$a_i\le C b_i$ and $a_i\nmid a_j$ for all $i<j$.

\begin{itemize}
\item For $b_n=n$ and $C=2$, solutions exist for $m\le 4$ (e.g.
$[2,3,5,7]$) but \emph{no} solution exists for $m=5$.
\item For $b_n=n$ and $C=3$, solutions exist up to $m=10$;
one solution is the prime list $[2,3,5,7,11,13,17,19,23,29]$.
\item For $b_n=2^n$ and $C=1$, solutions exist (again via primes) at least up to
$m=12$.
\end{itemize}

These are only sanity checks on prefixes; they do \emph{not} decide the infinite
question.

\medskip
\textbf{Problem-specific lemmas towards the classical necessary condition
$\sum 1/(b_n\log b_n)<\infty$.}

The next two lemmas are purely about primitive sets and are the key combinatorial
input in Erd\H{o}s’ 1935 argument bounding
$\sum_{a\in A} 1/(a\log a)$ for a primitive set $A$.

\medskip
\noindent\textbf{Lemma 892.1 (Disjointness of Erd\H{o}s L-sets).}
Let $a>1$ be an integer and write $P(a)$ for the largest prime factor of $a$.
Define
\[
L_a := \{ a\,m : m\in\mathbb{N},\ \text{every prime }p\mid m \text{ satisfies } p>P(a)\}.
\]
If $A$ is a primitive set of integers $>1$, then the sets $(L_a)_{a\in A}$ are
pairwise disjoint.

\emph{Proof.}
Assume for contradiction that $x\in L_a\cap L_{a'}$ for distinct $a,a'\in A$.
Then there exist $m,m'\in\mathbb{N}$ such that
\[ x = a m = a' m', \]
with every prime divisor of $m$ strictly larger than $P(a)$ and every prime
divisor of $m'$ strictly larger than $P(a')$.

Let $p:=P(a)$ and $q:=P(a')$. Without loss of generality assume $p\le q$.
Since $p\mid a$ and every prime divisor of $m$ is $>p$, the exact exponent of $p$
in the prime factorization of $x$ equals the exponent of $p$ in $a$.
Similarly, since $p\le q$ and every prime divisor of $m'$ is $>q\ge p$, the prime
$p$ cannot divide $m'$, so the exponent of $p$ in $x$ equals the exponent of $p$
in $a'$.
Hence $v_p(a)=v_p(a')$.

Now let $r$ be any prime $\le p$.
Because every prime divisor of $m$ is $>p\ge r$, the exponent of $r$ in $x$ equals
$v_r(a)$. Likewise, every prime divisor of $m'$ is $>q\ge p\ge r$, so the exponent
of $r$ in $x$ equals $v_r(a')$. Hence $v_r(a)=v_r(a')$ for every prime $r\le p$.

But all prime factors of $a$ are $\le p$ by definition of $p=P(a)$.
Therefore $a$ is exactly the product of $r^{v_r(a)}$ over primes $r\le p$, and the
same exponents occur in $a'$. It follows that $a\mid a'$.
Since $A$ is primitive and $a\ne a'$, this is impossible.
Thus $L_a\cap L_{a'}=\emptyset$.
\qed

\medskip
\noindent\textbf{Lemma 892.2 (Density of $L_a$).}
With $L_a$ as above, the natural density $d(L_a)$ exists and equals
\[
 d(L_a)=\frac{1}{a}\prod_{p\le P(a)}\Bigl(1-\frac{1}{p}\Bigr).
\]

\emph{Proof.}
Let $p_0:=P(a)$ and let $S:=\{p\ \text{prime}: p\le p_0\}$.
Define the set of $p_0$-\emph{rough} integers
\[
R_{p_0}:=\{m\in\mathbb{N}: \forall p\in S,\ p\nmid m\}.
\]
By inclusion--exclusion over the finite set $S$, the natural density of $R_{p_0}$
exists and equals
\[
 d(R_{p_0})=\prod_{p\in S}\Bigl(1-\frac{1}{p}\Bigr).
\]
(Indeed, for any finite set of primes $S$, the density of integers divisible by a
fixed squarefree modulus $d=\prod_{p\in T}p$ is $1/d$, and inclusion--exclusion
computes the density of those divisible by none of the primes in $S$.)

Now $L_a = a\,R_{p_0}$ is the image of $R_{p_0}$ under multiplication by $a$.
Scaling by $a$ multiplies natural densities by $1/a$:
\[
 d(L_a)=d(aR_{p_0})=\lim_{x\to\infty}\frac{|\{m\le x/a: m\in R_{p_0}\}|}{x}
      =\frac{1}{a} d(R_{p_0}).
\]
Substituting the expression for $d(R_{p_0})$ gives the stated formula.
\qed

\medskip
\noindent\textbf{Proposition 892.3 (Erd\H{o}s’ convergence mechanism; conditional on Mertens).}
Let $A$ be a primitive set of integers $>1$. Then
\[
\sum_{a\in A} \frac{1}{a\log a}<\infty.
\]

\emph{Proof.}
By Lemma~892.1 the sets $\{L_a\}_{a\in A}$ are disjoint, so
\[ \sum_{a\in A} d(L_a) \le 1. \]
By Lemma~892.2,
\[ \sum_{a\in A} \frac{1}{a}\prod_{p\le P(a)}\Bigl(1-\frac{1}{p}\Bigr) \le 1.
\]
A classical analytic estimate (Mertens’ product theorem) asserts that
\(\prod_{p\le y}(1-1/p) \asymp 1/\log y\) as $y\to\infty$.
In particular, there is an absolute constant $c>0$ and a threshold $y_0$ such that
for all $y\ge y_0$,
\[
\prod_{p\le y}\Bigl(1-\frac{1}{p}\Bigr) \ge \frac{c}{\log y}.
\]
Applying this with $y=P(a)$ for all sufficiently large $a\in A$ gives
\[
\sum_{a\in A,\ a\ \text{large}} \frac{c}{a\log P(a)} \le 1.
\]
Since $P(a)\le a$ we have $\log P(a)\le \log a$ and hence
$1/(a\log a)\le 1/(a\log P(a))$ for $a>1$. Therefore the tail
$\sum_{a\in A,\ a\ \text{large}} 1/(a\log a)$ is bounded by $(1/c)$, and adding
finitely many initial terms shows the full series converges.
\qed

\medskip
\noindent\textbf{Proposition 892.4 (Necessary condition on $b_n$ from existence of $a_n$).}
Assume there exist $C\ge 1$ and a primitive increasing sequence $(a_n)$ such that
$a_n\le C b_n$ for all $n$. Then
\[ \sum_{n\ge 1} \frac{1}{b_n\log b_n} <\infty. \]

\emph{Proof.}
Fix $C$ and let $n$ be large enough that $a_n\ge C^2e$.
Then $b_n\ge a_n/C\ge Ce$ and
\[ \log b_n \ge \log(a_n/C)=\log a_n -\log C \ge \tfrac12\log a_n. \]
Thus for all such $n$,
\[
\frac{1}{b_n\log b_n}
\le \frac{C}{a_n}\cdot\frac{2}{\log a_n}
= \frac{2C}{a_n\log a_n}.
\]
By Proposition~892.3 the series $\sum_n 1/(a_n\log a_n)$ converges, hence so does
the tail $\sum_{n\ge n_0} 1/(b_n\log b_n)$. Adding finitely many initial terms
finishes.
\qed

\medskip
\textbf{A transparent sufficient condition (one explicit construction).}

\noindent\textbf{Proposition 892.5 (Doubling-growth suffices).}
Assume $b_1\ge 2$ and $b_{n}\ge 2b_{n-1}$ for all $n\ge 2$.
Then there exists a primitive increasing sequence $(a_n)$ with $a_n\le b_n$ for
all $n$.

\emph{Proof.}
We will choose $a_n$ to be primes, one from each interval $(b_{n-1},b_n]$.
Since $b_n\ge 2b_{n-1}$, the interval $(b_{n-1},b_n]$ contains the sub-interval
$(b_{n-1},2b_{n-1}]$.
By Bertrand’s postulate, for every integer $x\ge 2$ there is a prime in
$(x,2x]$. Applying this with $x=b_{n-1}$ yields a prime $a_n\in(b_{n-1},2b_{n-1}]$.
In particular $a_n\le 2b_{n-1}\le b_n$, and $a_n>b_{n-1}\ge a_{n-1}$, so the
sequence is strictly increasing.
Distinct primes never divide each other, so $(a_n)$ is primitive.
\qed

\medskip
\noindent\textbf{5) VERIFICATION}

\begin{itemize}
\item Lemma~892.1: the only subtle point is handling the largest prime factor
$p=P(a)$ when comparing factorizations in $x=am=a'm'$. This is why $L_a$ was
defined using primes \emph{strictly} larger than $P(a)$; then primes $\le P(a)$
come only from $a$, making the divisibility conclusion $a\mid a'$ robust.
\item Lemma~892.2: inclusion--exclusion is exact since only finitely many primes
$\le P(a)$ are involved.
\item Proposition~892.3: the only external input is the lower bound in Mertens’
product theorem. All other steps are explicit. The comparison
$\log P(a)\le \log a$ is correct.
\item Proposition~892.4: the comparison of logs uses $\log a_n -\log C\ge \tfrac12\log a_n$
for $a_n\ge C^2$, which is correct.
\item Proposition~892.5: depends on Bertrand’s postulate; the rest is immediate.
\end{itemize}

\medskip
\noindent\textbf{6) FINAL}

\textbf{UNRESOLVED}

(i) \emph{Strongest proved partial result.}
If such a primitive $(a_n)$ with $a_n\le Cb_n$ exists, then necessarily
$\sum_n 1/(b_n\log b_n)<\infty$ (Proposition~892.4). Also, a simple sufficient
condition is doubling growth $b_n\ge 2b_{n-1}$, which guarantees a construction
with $a_n\le b_n$ (Proposition~892.5).

(ii) \emph{First gap (crisp).}
Give a necessary and sufficient condition on an arbitrary increasing integer
sequence $(b_n)$ that is equivalent to the existence of a primitive
$(a_n)$ with $a_n\le Cb_n$ for all $n$.
In particular, decide whether hypothesis (\*) alone implies existence.

(iii) \emph{Top 3 next moves.}
\begin{enumerate}
\item Prove (or disprove) that the two known necessary conditions listed in the
problem statement (the $\sum 1/(b_n\log b_n)$ condition and the upper-log-density
condition) are jointly sufficient.
\item Try to construct a counterexample sequence $(b_n)$ satisfying (\*) but
violating any plausible sufficient condition by forcing many divisibility
constraints on any nearby $a_n$.
\item Computationally: for candidate $b_n$ with (\*) and modest growth, attempt
SAT/backtracking search for longer prefixes and look for structural obstructions
(e.g. forced divisibility chains among all $a\le Cb_n$).
\end{enumerate}

(iv) \emph{Minimal counterexample structure (if false).}
A minimal counterexample to “(\*) implies existence” would likely have relatively
slow growth (to make primitivity hard) while being arranged so that gcds of pairs
avoid earlier terms. One expects many shared small prime factors across $b_n$
(creating divisibility pressure) but arranged so that each common gcd is not
itself among the earlier $b_k$.


