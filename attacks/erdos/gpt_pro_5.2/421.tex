
\subsection*{FORMAL RESTATEMENT}
\textbf{Definitions.}
\begin{itemize}
\item A strictly increasing sequence $(d_i)_{i\ge 1}$ of positive integers determines a set $D=\{d_1,d_2,\dots\}\subset\mathbb{N}$.
\item The (natural) density of $D$ is
\[
\delta(D)=\lim_{x\to\infty}\frac{\#(D\cap\{1,2,\dots,\lfloor x\rfloor\})}{\lfloor x\rfloor}
\]
when this limit exists.
\item For $1\le u\le v$, define the consecutive product
\[
P_{u,v}=\prod_{i=u}^v d_i.
\]
\end{itemize}

\textbf{Question.} Does there exist a sequence $1\le d_1<d_2<\cdots$ such that $\delta(\{d_i\})=1$ and all products $P_{u,v}$ are distinct (i.e. the map $(u,v)\mapsto P_{u,v}$ is injective on $\{(u,v):1\le u\le v\}$)?

\subsection*{QUICK LITERATURE/CONTEXT CHECK}
The extracted statement records that Selfridge constructed such a sequence with density $>1/e-\varepsilon$ for every $\varepsilon>0$, but the existence of density $1$ remains open.

\subsection*{ATTACK PLAN}
\textbf{Proof-track ideas (construct a density-$1$ example).}
\begin{itemize}
\item Attempt a greedy construction: include integers unless doing so creates a collision among interval products; then estimate the deletion rate.
\item Encode interval products uniquely using prime-factor ``markers'' that are forced to appear in exactly one interval; the challenge is to do this while keeping density $1$.
\end{itemize}
\textbf{Disproof-track ideas.}
\begin{itemize}
\item Show that density $1$ forces too many multiplicative relations among consecutive blocks, yielding collisions $P_{u,v}=P_{u',v'}$ by pigeonhole-type counting in the divisor lattice of $\prod_{i\le N} d_i$.
\item Prove that any such sequence must omit a positive proportion of integers because it must avoid including numbers equal to products of previous consecutive blocks.
\end{itemize}

\subsection*{WORK}
\subsubsection*{Fast reality check (finite search in $\{1,\dots,M\}$)}
For each $M\le 20$, I searched over all subsets $S\subseteq\{1,\dots,M\}$, taking $d_i$ to be the increasing enumeration of $S$, and checked whether all consecutive products are distinct. The maximum size found, together with an example achieving it, is:
\begin{verbatim}
M  max|S|  density=max|S|/M  example S (in increasing order)
 1    1        1.0000       [1]
 2    1        0.5000       [1]
 3    2        0.6667       [2, 3]
 4    3        0.7500       [2, 3, 4]
 5    4        0.8000       [2, 3, 4, 5]
 6    4        0.6667       [2, 3, 4, 5]
 7    5        0.7143       [2, 3, 4, 5, 7]
 8    6        0.7500       [2, 3, 4, 5, 7, 8]
 9    7        0.7778       [2, 3, 4, 5, 7, 8, 9]
10    8        0.8000       [2, 3, 4, 5, 7, 8, 9, 10]
11    9        0.8182       [2, 3, 4, 5, 7, 8, 9, 10, 11]
12    9        0.7500       [2, 3, 4, 5, 7, 8, 9, 10, 11]
13   10        0.7692       [2, 3, 4, 5, 7, 8, 9, 10, 11, 13]
14   11        0.7857       [2, 3, 4, 5, 7, 8, 9, 10, 11, 13, 14]
15   12        0.8000       [2, 3, 4, 5, 7, 8, 9, 10, 11, 13, 14, 15]
16   12        0.7500       [2, 3, 4, 5, 7, 8, 9, 10, 11, 13, 14, 15]
17   13        0.7647       [2, 3, 4, 5, 7, 8, 9, 10, 11, 13, 14, 15, 17]
18   14        0.7778       [2, 3, 4, 5, 7, 8, 9, 10, 11, 13, 14, 15, 17, 18]
19   15        0.7895       [2, 3, 4, 5, 7, 8, 9, 10, 11, 13, 14, 15, 17, 18, 19]
20   16        0.8000       [2, 3, 4, 7, 8, 9, 10, 11, 13, 14, 15, 16, 17, 18, 19, 20]
\end{verbatim}
These finite searches show that high density is possible in short initial segments, but also that certain integers (notably $1$ and various composite numbers) must be omitted to avoid collisions.

\subsubsection*{Lemma 421.1 (No term can equal $1$ in a nontrivial sequence)}
\textbf{Lemma.} Suppose $(d_i)_{i\ge 1}$ has the property that all products $P_{u,v}=\prod_{i=u}^v d_i$ are distinct. If the sequence has at least two terms, then $d_i\ne 1$ for every $i$.

\textbf{Proof.}
Assume for contradiction that $d_j=1$ for some $j$ and that there exists at least one index $j+1$ (i.e. the sequence has at least two terms and $j$ is not the last term).
Consider the two intervals $[j+1,j+1]$ and $[j,j+1]$.
Their products are
\[
P_{j+1,j+1}=d_{j+1},\qquad P_{j,j+1}=d_j d_{j+1}=1\cdot d_{j+1}=d_{j+1}.
\]
Thus $P_{j+1,j+1}=P_{j,j+1}$, contradicting the assumption that all interval products are distinct.
Therefore no such $j$ exists; i.e. no term equals $1$.
\hfill$\square$

\subsubsection*{Lemma 421.2 (Many divisors of the initial total product)}
\textbf{Lemma.} Suppose $(d_i)$ has distinct interval products. For each $N\ge 1$, let
\[
P_N=\prod_{i=1}^N d_i.
\]
Then
\[
\tau(P_N)\ge \frac{N(N+1)}{2}.
\]

\textbf{Proof.}
For each pair $(u,v)$ with $1\le u\le v\le N$, the interval product $P_{u,v}=\prod_{i=u}^v d_i$ divides $P_N$ because
\[
P_N=\Big(\prod_{i=1}^{u-1} d_i\Big)\cdot \Big(\prod_{i=u}^v d_i\Big)\cdot \Big(\prod_{i=v+1}^{N} d_i\Big),
\]
and the two outer products are integers.
Hence every $P_{u,v}$ is a positive divisor of $P_N$.

There are exactly $N(N+1)/2$ pairs $(u,v)$ with $1\le u\le v\le N$.
By hypothesis, the corresponding products $P_{u,v}$ are all distinct. Therefore $P_N$ has at least $N(N+1)/2$ distinct positive divisors, i.e. $\tau(P_N)\ge N(N+1)/2$.
\hfill$\square$

\subsubsection*{Lemma 421.3 (Density $1$ forces $d_n\sim n$)}
\textbf{Lemma.} Let $1\le d_1<d_2<\cdots$ be strictly increasing and let $D=\{d_i\}$. If $\delta(D)=1$, then
\[
\lim_{n\to\infty} \frac{d_n}{n}=1.
\]

\textbf{Proof.}
For each $n$, we have $\#(D\cap\{1,2,\dots,d_n\})=n$ by definition of $d_n$.
If $\delta(D)=1$, then by the definition of density,
\[
\lim_{x\to\infty} \frac{\#(D\cap\{1,2,\dots,\lfloor x\rfloor\})}{\lfloor x\rfloor}=1.
\]
Apply this along the subsequence $x=d_n$ (note $d_n\to\infty$): we obtain
\[
\lim_{n\to\infty} \frac{\#(D\cap\{1,2,\dots,d_n\})}{d_n}=\lim_{n\to\infty} \frac{n}{d_n}=1.
\]
Taking reciprocals yields $d_n/n\to 1$.
\hfill$\square$

\subsection*{VERIFICATION}
\begin{itemize}
\item Lemma~421.1 uses only the definition of distinct interval products and checks a concrete collision.
\item Lemma~421.2: every interval product divides $P_N$ by an explicit factorisation; distinctness gives distinct divisors.
\item Lemma~421.3: uses density along the subsequence $x=d_n$, which is legitimate since $d_n\to\infty$.
\end{itemize}

\subsection*{FINAL}
\textbf{UNRESOLVED}
\begin{enumerate}
\item[(i)] \textbf{Strongest proved partial result here.}
Necessary constraints on any solution sequence:
\begin{itemize}
\item No term can equal $1$ (Lemma~421.1).
\item The initial product $P_N$ must have at least $N(N+1)/2$ divisors (Lemma~421.2).
\item Density $1$ would force $d_n/n\to 1$ (Lemma~421.3).
\end{itemize}
Additionally, exhaustive search shows that for $M\le 20$ one can select subsets of $\{1,\dots,M\}$ of density about $0.8$ whose increasing enumerations satisfy the distinct-interval-product property.
\item[(ii)] \textbf{First gap (crisp).}
Decide whether there exists an infinite increasing sequence $(d_i)$ with $\delta(\{d_i\})=1$ such that all interval products $\prod_{u\le i\le v} d_i$ are distinct.
\item[(iii)] \textbf{Top 3 next moves.}
\begin{itemize}
\item Develop a quantified ``collision forcing'' lemma: show that if $D\cap\{1,\dots,M\}$ is too large then some collision among interval products must occur. Any such lemma with a bound tending to $1$ as $M\to\infty$ would disprove density $1$.
\item Explore greedy constructions computationally for large $M$ (e.g. $M=10^4$) to estimate the deletion rate needed to maintain distinct interval products, and look for patterns that might suggest an infinite density-$1$ construction.
\item Re-express collisions $P_{u,v}=P_{u',v'}$ as multiplicative relations among consecutive blocks and attempt to apply multiplicative-energy or entropy methods to high-density sets.
\end{itemize}
\item[(iv)] \textbf{What a minimal counterexample would likely look like.}
If density $1$ is impossible, a minimal obstruction would likely be that any set $D$ missing only $o(M)$ integers in $\{1,\dots,M\}$ forces the presence of some $d_j$ equal to a product of a previous consecutive block, creating an unavoidable collision between a length-$1$ interval and a longer interval. Such a counterexample would manifest as an explicit collision $\prod_{i=u}^v d_i=\prod_{i=u'}^{v'} d_i$ with $u'<u\le v<v'$ or similar nested structure.
\end{enumerate}

\section*{1) FORMAL RESTATEMENT}

\textbf{Conventions.}
Let $\mathbb N:=\{1,2,3,\dots\}$.

A \textbf{candidate} is a strictly increasing sequence of positive integers
\[
1\le d_1<d_2<\cdots,\qquad d_i\in\mathbb N,
\]
and we write $D:=\{d_i:i\in\mathbb N\}\subset\mathbb N$.

Define the \textbf{(natural/asymptotic) density} of $D$, if it exists, by
\[
\delta(D):=\lim_{N\to\infty}\frac{|D\cap\{1,2,\dots,N\}|}{N}.
\]
For integers $u,v\in\mathbb N$ with $1\le u\le v$, define the \textbf{consecutive block product}
\[
P_{u,v}:=\prod_{i=u}^v d_i\in\mathbb N.
\]

\textbf{Problem (\#421).}
Does there exist an infinite strictly increasing sequence $(d_i)_{i\ge1}$ such that
\begin{enumerate}
\item $\delta(D)=1$, and
\item (\textbf{distinct interval products}) for all $(u,v)\neq(u',v')$ with $1\le u\le v$ and $1\le u'\le v'$,
\[
P_{u,v}\neq P_{u',v'}\, ?
\]
\end{enumerate}

\textbf{Stress points / edge cases.}
\begin{itemize}
\item The sequence must be infinite (otherwise $\delta(D)=1$ cannot hold).
\item If any $d_j=1$ and there is a later term, then $P_{j,j+1}=P_{j+1,j+1}$, so any infinite solution must satisfy $d_i\ge2$ for all $i$.
\item The condition is global: it forbids \emph{all} multiplicative relations among consecutive blocks, including relations that can be \emph{created} by deleting terms.
\end{itemize}

\section*{2) QUICK LITERATURE/CONTEXT CHECK}

As of 2026-01-17, the Erd\H{o}s Problems database lists \#421 as \textbf{OPEN} and records that Selfridge constructed sequences with the desired ``distinct consecutive block products'' property and density $>1/e-\varepsilon$ for every $\varepsilon>0$, while the density-$1$ case is not marked as resolved there. A recent forum discussion proposes a route via a prefix-stable greedy construction plus determinant-method bounds for integral points on certain separated-variable curves; however, no fully published unconditional resolution was identified in that check.

\section*{3) ATTACK PLAN}

\textbf{Proof-track strategies.}
\begin{enumerate}
\item \textbf{Prefix-stable greedy construction.} Scan $n=2,3,4,\dots$ and include $n$ unless it creates a collision among consecutive products in the current sequence, in a way that guarantees later decisions cannot create new collisions among already-accepted terms. Goal: show the rejected set is $o(N)$ up to $N$.
\item \textbf{Dyadic deletion + local deficiency + Diophantine bounds.} Delete a density-$0$ set scale-by-scale while controlling local deletion counts, so only $X^{o(1)}$ ``gap patterns'' occur at scale $X$; then use determinant-method point counting to show only $o(X)$ collisions per dyadic block.
\item \textbf{Large prime factor / random deletion hybrid.} Prepare the sequence so each term has a large prime factor, then delete a sparse set randomly to destroy long-block collisions; use Diophantine tools for the $\log$-length bottleneck.
\end{enumerate}

\textbf{Disproof-track strategies.}
\begin{enumerate}
\item \textbf{Forced collision counting.} Attempt to show density $1$ implies too many multiplicative identities among consecutive blocks (pigeonhole/divisor-lattice arguments).
\item \textbf{Unavoidable product closure.} Attempt to show density $1$ forces inclusion of too many numbers that equal products of short consecutive blocks, making collisions inevitable.
\end{enumerate}

\textbf{Chosen direction.}
The strongest available ``hard'' input appears to be determinant-method point counting for plane curves, but a major independent obstacle is that deletion is not monotone: deleting terms can \emph{create} new collisions. Any density-$1$ construction must be \emph{prefix-stable} against that phenomenon.

\section*{4) WORK}

\subsection*{4.1 Basic necessary lemma: no $1$ in an infinite solution}

\textbf{Lemma 4.1.}
If $(d_i)_{i\ge1}$ is strictly increasing and has the distinct-interval-products property, and it has at least two terms, then $d_i\neq1$ for all $i$. In particular any infinite solution satisfies $d_i\ge2$ for all $i$.

\textbf{Proof.}
Suppose $d_j=1$ for some $j$ and that $d_{j+1}$ exists. Then
\[
P_{j+1,j+1}=d_{j+1},\qquad P_{j,j+1}=d_jd_{j+1}=1\cdot d_{j+1}=d_{j+1},
\]
so $P_{j+1,j+1}=P_{j,j+1}$ with $(j+1,j+1)\neq(j,j+1)$, contradicting injectivity. \hfill$\square$

\subsection*{4.2 Normal form for collisions: disjoint separated consecutive blocks}

A collision is an equality
\[
P_{u,v}=P_{u',v'}\qquad\text{with }(u,v)\neq(u',v').
\]
The key structural fact is that, since all $d_i\ge2$, one interval cannot properly contain the other; after cancellation one obtains an equality of products of two disjoint consecutive blocks.

\textbf{Lemma 4.2 (collision normal form).}
Assume $d_i\ge2$ for all $i$. If there exist $1\le u\le v$ and $1\le u'\le v'$ with $(u,v)\neq(u',v')$ such that
\[
\prod_{i=u}^v d_i=\prod_{i=u'}^{v'} d_i,
\]
then there exist integers $M,N,\ell,k$ with $M<N$, $\ell\ge1$, $k\ge0$ such that
\[
\boxed{\ \prod_{i=M-\ell+1}^{M} d_i\;=\;\prod_{i=N}^{N+k} d_i\ }\tag{NF}
\]
(i.e.\ equality of products of two disjoint consecutive blocks, left block ending at $M$, right block starting at $N$).

\textbf{Proof.}
Without loss of generality assume $u\le u'$. Consider cases.

\emph{Case 1: Disjoint, $v<u'$.}
Set $M:=v$, $N:=u'$, $\ell:=v-u+1$, $k:=v'-u'$. Then (NF) is exactly the original equality.

\emph{Case 2: Overlap, $u\le u'\le v$.}
Then
\[
\prod_{i=u}^v d_i=\prod_{i=u'}^{v'} d_i.
\]
Cancel the common factor $\prod_{i=u'}^{v} d_i$ (since $u'\le v$) to obtain
\[
\prod_{i=u}^{u'-1} d_i=\prod_{i=v+1}^{v'} d_i.\tag{1}
\]
Now $u'-1<v+1$, so the two blocks in (1) are disjoint and separated. Define $M:=u'-1$, $N:=v+1$, $\ell:=u'-u$, and $k:=v'-v-1$. Then (1) becomes (NF).

\emph{Case 3: Containment, $u\le u'$ and $v'\le v$.}
Then $[u',v']\subseteq[u,v]$, and cancellation gives
\[
\prod_{i=u}^{u'-1} d_i\cdot \prod_{i=v'+1}^{v} d_i=1.
\]
Each factor is $\ge2$, so the left-hand side is $\ge2$ unless both products are empty, i.e.\ unless $u=u'$ and $v=v'$, contradicting $(u,v)\neq(u',v')$. Hence containment cannot occur.

Thus in all possible cases a collision yields (NF). \hfill$\square$

\subsection*{4.3 Deleting numbers can create collisions (non-monotonicity)}

\textbf{Lemma 4.3 (non-monotonicity under deletion).}
There exists a finite strictly increasing sequence whose interval products are all distinct, but deleting a term from it creates an interval-product collision.

\textbf{Proof (explicit example).}
Take the set $A=\{2,4,6,12\}$ with enumeration $d_1=2,d_2=4,d_3=6,d_4=12$.
All interval products are:
\begin{itemize}
\item length $1$: $2,4,6,12$;
\item length $2$: $2\cdot4=8$, $4\cdot6=24$, $6\cdot12=72$;
\item length $3$: $2\cdot4\cdot6=48$, $4\cdot6\cdot12=288$;
\item length $4$: $2\cdot4\cdot6\cdot12=576$.
\end{itemize}
These $10$ values are pairwise distinct, so $A$ satisfies the distinct-interval-products property.

Now delete $4$ to obtain $A'=\{2,6,12\}$. Then
\[
P'_{1,2}=2\cdot 6=12=P'_{3,3},
\]
a collision. \hfill$\square$

\textbf{Consequence.}
Any deletion-based construction must establish a \emph{prefix-stability} invariant; it is not enough to hit collisions present at one stage, because new collisions can be created by the deletion itself.

\subsection*{4.4 Determinant-method input (polylog bounds when degree $\asymp \log B$)}

A standard Diophantine-geometric reduction rewrites a collision in normal form (NF) as an integer point on a separated-variable curve of the shape
\[
\prod_{i=0}^{\ell-1}(x-a_i)=\prod_{j=0}^{k}(y+b_j),
\]
where $a_0=b_0=0$ and typically $a_i,b_j\in\mathbb Z_{\ge0}$ encode local gaps.

One quantitative determinant-method theorem (Castryck--Cluckers--Dittmann--Nguyen, plane curve case) states:

\textbf{Theorem 4.4 (CCDN, plane curve case).}
Let $X\subset\mathbb A^2$ be an \emph{integral} affine curve of degree $d\ge2$ defined over $\mathbb Q$.
Let $N(X,B)$ be the number of integer points $(x,y)\in X(\mathbb Z)$ with $|x|\le B$ and $|y|\le B$.
Then
\[
N(X,B)\ \ll\ (d^3\log B+d^4)\,B^{1/d},
\]
with an absolute implied constant.

\textbf{Corollary 4.5.}
Fix constants $0<c\le C$. If $c\log B\le d\le C\log B$, then
\[
N(X,B)\ \ll\ (\log B)^5.
\]

\textbf{Proof.}
Since $d\ge c\log B$, we have
\[
B^{1/d}=\exp\!\Big(\frac{\log B}{d}\Big)\le \exp(1/c),
\]
so Theorem 4.4 yields $N(X,B)\ll (d^3\log B+d^4)\ll (\log B)^5$. \hfill$\square$

\subsection*{4.5 Line-component degeneracy is harmless in the collision region $x<y$, $x,y>0$}

Set
\[
f(x):=\prod_{i=0}^{\ell-1}(x-a_i),\qquad g(y):=\prod_{j=0}^{k}(y+b_j),
\]
with $a_0=b_0=0$ and $a_i,b_j\in\mathbb Z_{\ge0}$. Assume $\deg f=\deg g=d\ge1$ and both are monic.
Let
\[
F(x,y):=f(x)-g(y)\in\mathbb Z[x,y].
\]

\textbf{Lemma 4.6 (classification of linear factors).}
If $F(x,y)=f(x)-g(y)$ has a nonconstant linear factor over $\mathbb Q$, then there exists $t\in\mathbb Q$ such that either
\[
f(x)\equiv g(x+t)\quad\text{or}\quad f(x)\equiv g(-x+t)
\]
as polynomials in $\mathbb Q[x]$.

\textbf{Proof.}
Let $L(x,y)=ux+vy+w$ be a linear factor of $F$ with $(u,v)\neq(0,0)$.
Then $v\neq0$ (otherwise $L$ depends only on $x$, forcing $f$ to be constant).
Write the line as $y=\alpha x+\beta$ with $\alpha,\beta\in\mathbb Q$.
Divisibility implies
\[
0\equiv F(x,\alpha x+\beta)=f(x)-g(\alpha x+\beta)
\]
in $\mathbb Q[x]$, hence $f(x)\equiv g(\alpha x+\beta)$.
Comparing leading coefficients, since $f$ and $g$ are monic of degree $d$,
the leading coefficient of $g(\alpha x+\beta)$ is $\alpha^d$, so $\alpha^d=1$ in $\mathbb Q$.
Thus $\alpha=1$ or $\alpha=-1$. Writing $t:=\beta$ gives the claim. \hfill$\square$

\textbf{Lemma 4.7 (no relevant points from line components).}
Under the above assumptions, in either of the identities $f(x)\equiv g(x+t)$ or $f(x)\equiv g(-x+t)$ we obtain:
\begin{itemize}
\item If $f(x)\equiv g(x+t)$, then necessarily $t\le0$, so the line $y=x+t$ contains no integer point with $x<y$.
\item If $f(x)\equiv g(-x+t)$, then necessarily $t=0$, so the line $y=-x$ contains no integer point with $x,y>0$.
\end{itemize}

\textbf{Proof.}
If $f(x)\equiv g(x+t)$, then the root multisets agree:
roots of $f$ are $\{a_i\}\subseteq\mathbb Z_{\ge0}$ and include $0=a_0$;
roots of $g(x+t)$ are $\{-t-b_j\}$ and include $-t$ (since $b_0=0$).
Thus $-t$ is a root of $f$, so $-t\ge0$, i.e.\ $t\le0$, and $y=x+t\le x$, excluding $x<y$.

If $f(x)\equiv g(-x+t)$, then roots of $g(-x+t)$ are $\{t+b_j\}\subseteq t+\mathbb Z_{\ge0}$ and include $t$.
Since $0$ is a root of $f$, we need $0=t+b_j$ for some $j$, hence $t\le0$.
But also $t$ itself is a root of $f$, so $t\ge0$. Therefore $t=0$, and $y=-x$ has no solutions with $x,y>0$. \hfill$\square$

\textbf{Conclusion.}
Any affine line component of $F(x,y)=0$ yields no points in the collision-relevant region $x<y$ with $x,y>0$.

\subsection*{4.6 Precise point where the full solution breaks}

The determinant-method bounds in \S4.4 are strong enough to bound integer points on \emph{a fixed} integral curve of degree $d$.
However, turning this into a density-$1$ set $D$ with \emph{all} consecutive block products distinct requires a construction argument that is robust under deletions/rejections. Lemma 4.3 shows naive deletion can create new collisions.

A crisp formulation of the missing step is:

\medskip
\noindent\textbf{GAP (construction stability problem).}
Construct (or prove existence of) an infinite set $D\subset\mathbb N$ of density $1$ via an explicit selection/deletion rule such that, for every cutoff $N$, there are no collisions in normal form (NF) whose maximum index/value lies $\le N$, and such that the number of rejected integers $\le N$ is $o(N)$.
In particular, one must prove a \emph{prefix-stability} invariant that prevents the phenomenon of Lemma 4.3 at all scales.
\medskip

\section*{5) VERIFICATION}

\begin{itemize}
\item Lemma 4.1: checked by explicit collision $P_{j,j+1}=P_{j+1,j+1}$ if $d_j=1$ and a later term exists.
\item Lemma 4.2: all interval-pair configurations split into disjoint, overlap, containment; cancellation is explicit; containment is ruled out using $d_i\ge2$.
\item Lemma 4.3: explicit finite example verified by listing all interval products; deletion creates a concrete collision.
\item Lemmas 4.6--4.7: linear factor $\Rightarrow$ affine parametrization $y=\alpha x+\beta$; monicity forces $\alpha\in\{\pm1\}$; root-multiset arguments with $a_i,b_j\ge0$ exclude any collision-region integer points.
\item No claim of a full construction is made; the stated GAP is genuine and consistent with non-monotonicity under deletion.
\end{itemize}

\section*{6) FINAL}

\textbf{UNRESOLVED}

\begin{enumerate}
\item[(i)] \textbf{Strongest fully proved partial result obtained here.}
\begin{itemize}
\item Any collision can be reduced to the disjoint-block normal form (NF) (Lemma 4.2).
\item The ``no-collision'' property is not monotone under deletion (Lemma 4.3).
\item For the separated-variable curve $f(x)-g(y)=0$ with nonnegative gap roots, any affine line component yields no collision-relevant points with $x<y$, $x,y>0$ (Lemmas 4.6--4.7).
\item Determinant-method bounds give explicit uniform point counts $N(X,B)\ll(d^3\log B+d^4)B^{1/d}$ for integral plane curves of degree $d$; in particular for $d\asymp\log B$ this is polylogarithmic in $B$ (Theorem 4.4 and Corollary 4.5).
\end{itemize}

\item[(ii)] \textbf{First gap (single crisp statement).}
Provide a fully rigorous, prefix-stable construction of a density-$1$ increasing sequence $(d_i)$ with distinct consecutive block products; in particular, prove that the deletion/rejection mechanism cannot create new collisions below already-processed scales (cf.\ Lemma 4.3).

\item[(iii)] \textbf{Top 3 next moves.}
\begin{itemize}
\item Design and prove a genuinely prefix-stable selection rule (an invariant ``no collisions with maximum $\le n$'' maintained at each step).
\item Prove a quantitative bound on the number of realized local gap patterns at each dyadic scale under that invariant.
\item Combine the pattern bound with Theorem 4.4 on each (integral) component and a controlled degeneracy classification to show only $o(X)$ total collision points per dyadic range $[X,2X]$.
\end{itemize}

\item[(iv)] \textbf{Likely structure of a minimal obstruction (if density $1$ is impossible).}
A forcing lemma would show that any $D\subset\mathbb N$ with $|[1,N]\setminus D|=o(N)$ must contain a short normal-form collision (NF) with maximum $\le N$, likely involving very short blocks (e.g.\ length $2$ vs length $1$), since those are hardest to eliminate without deleting a positive proportion.
\end{enumerate}

\section*{7) COMPLETION ESTIMATE}

\noindent COMPLETION: 45\%.
