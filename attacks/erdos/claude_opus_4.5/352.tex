
OUTPUT
1) FORMAL RESTATEMENT
Let $A\subseteq\mathbb{R}^2$ be Lebesgue measurable and let $\lambda_2(A)$ be its area.
The question asks whether there exists a constant $c>0$ such that
\[
\forall A\subseteq\mathbb{R}^2 \text{ measurable},\ \lambda_2(A)\ge c\ \Rightarrow\ \exists x,y,z\in A\ \text{with}\ \text{Area}(\triangle xyz)=1.
\]
We interpret "triangle of area 1" as a non-degenerate triangle with exact area 1.

2) QUICK LITERATURE/CONTEXT CHECK
Browsing is not available. I only record what the problem statement itself claims:
- Erd\H{o}s proved the statement for infinite-measure sets and for unbounded positive-measure sets.
- The conjectured sharp threshold is $c=4\pi/\sqrt{27}$, motivated by disks of radius $<2\cdot 3^{-3/4}$.
- Freiling--Mauldin proved an outer-measure $>4\pi/\sqrt{27}$ result for area $>1$ triangles,
  and proved the conjecture for compact convex sets and for unions of $\le 3$ such sets.
These claims are not re-proved here.

3) ATTACK PLAN
Proof strategies:
1. Reduce to structured sets (finite unions of convex bodies) and use continuity of area in a
   sliding-vertex argument.
2. Use density/translation arguments (Steinhaus-type) to pass from measure to geometric patterns.
3. Explore extremal configurations: disks appear to be the barrier, so show disk-threshold suffices.

Disproof strategies:
1. Construct a measurable set of area just above $4\pi/\sqrt{27}$ with no area-1 triangles.
2. Attempt fractal or highly disconnected sets to evade geometric continuity arguments.

Chosen path: Prove elementary geometric lemmas about disks and triangle areas (a rigorous lower bound
for sets containing a disk) and isolate the first gap to the full measurable case.

4) WORK
Lemma 1 (Maximum triangle area in a disk).
Let $B_R$ be a closed disk of radius $R$. Any triangle with vertices in $B_R$ has area at most
\[\frac{3\sqrt{3}}{4}R^2.\]

Proof.
Any triangle inscribed in a circle of radius $R$ has area
\[\text{Area} = 2R^2\sin A\sin B\sin C,\]
where $A+B+C=\pi$. The product $\sin A\sin B\sin C$ is maximized when $A=B=C=\pi/3$,
so the maximum area is $2R^2(\sqrt{3}/2)^3 = \frac{3\sqrt{3}}{4}R^2$. ∎

Lemma 2 (Exact area 1 inside a large disk).
If a set $A$ contains a closed disk of radius $R\ge 2\cdot 3^{-3/4}$, then $A$ contains a
triangle of area exactly $1$.

Proof.
Let $C$ be the circle of radius $R$. The equilateral triangle inscribed in $C$ has area
$\frac{3\sqrt{3}}{4}R^2\ge 1$. Fix two distinct points on $C$ and let the third vertex move
continuously along $C$. The triangle area varies continuously and takes values arbitrarily close to
$0$ (as the third point approaches one of the fixed points) and at least $1$ (at the equilateral
position). By the intermediate value theorem, some position gives area exactly $1$. ∎

Lemma 2a (Disk threshold calculation).
Let $R_*=2\cdot 3^{-3/4}$. Then the disk of radius $R_*$ has area
\[
\pi R_*^2=\frac{4\pi}{\sqrt{27}},
\]
and its maximum triangle area equals $1$.

Proof.
Compute $R_*^2=4\cdot 3^{-3/2}=4/\sqrt{27}$. Hence the disk area is
$\pi R_*^2=4\pi/\sqrt{27}$. By Lemma 1, the maximal triangle area in the disk is
$(3\sqrt{3}/4)R_*^2=(3\sqrt{3}/4)\cdot 4/\sqrt{27}=1$. ∎

Lemma 2b (Area values inside a disk form an interval).
Let $B_R$ be a closed disk. Then for any $t$ with $0\le t\le \frac{3\sqrt{3}}{4}R^2$,
there is a triangle with vertices in $B_R$ of area exactly $t$.

Proof.
Fix two distinct points on the boundary of $B_R$. As the third vertex moves continuously along
the boundary circle, the triangle area varies continuously from $0$ (degenerate position) up to
the maximal inscribed area. The intermediate value theorem gives all intermediate values. ∎

Lemma 3 (Disk obstruction at the conjectured threshold).
If $R<2\cdot 3^{-3/4}$, then no triangle in the disk of radius $R$ has area $1$.

Proof.
By Lemma 1, the maximum possible area is $\frac{3\sqrt{3}}{4}R^2<1$. ∎

These lemmas show the conjectured constant is the correct threshold for sets that contain a full disk,
but they do not address general measurable sets.

5) VERIFICATION
Edge cases:
- Degenerate triangles (area 0) are excluded by definition.
- Lemma 2 uses continuity of the area as a function of the third vertex on a circle.
- Lemma 2b uses the same continuity argument to show all intermediate areas occur in a disk.

Quantifier checks:
- Lemma 2 requires the disk to be contained in $A$; it does not apply to arbitrary measurable sets
  of the same measure without a disk.

6) FINAL
**UNRESOLVED**

(i) Strongest fully proved partial result:
If $A$ contains a disk of radius $R\ge 2\cdot 3^{-3/4}$, then $A$ contains a triangle of area 1
(Lemma 2). Disks with radius $R<2\cdot 3^{-3/4}$ contain no such triangle (Lemma 3). The threshold
area $4\pi/\sqrt{27}$ and maximal inscribed triangle area are computed explicitly (Lemma 2a), and
all areas up to the maximum are realized inside a disk (Lemma 2b).

(ii) First gap:
Extend the disk-based argument to arbitrary measurable sets of area $>4\pi/\sqrt{27}$ without
assuming any convexity or interior ball condition.

(iii) Top 3 next moves:
1. Prove the conjecture for unions of finitely many convex bodies for all $n$ (extend the $n\le 3$
   result reported in the statement).
2. Show that the set of achievable triangle areas inside a positive-measure set is an interval.
3. Develop a reduction from general measurable sets to structured approximations with controlled area.

(iv) Minimal counterexample structure (if false):
A bounded measurable set with area just above $4\pi/\sqrt{27}$ that is highly disconnected (no large
interior balls), engineered to avoid all exact area-1 triangles while still having large measure.
