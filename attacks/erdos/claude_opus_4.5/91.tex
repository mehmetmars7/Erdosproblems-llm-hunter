% Attempt for Erdős Problem #91
% Following PROMPT_STRATEGY.MD
% Tools/Constraints:
% - Web browsing available? NO (restricted)
% - Computation available (Python/Sage/Mathematica)? YES, not used

OUTPUT
1) FORMAL RESTATEMENT
For a finite set $A\subset\mathbb{R}^2$, let
\[
d(A)=\bigl|\{\|x-y\|: x,y\in A,\ x\ne y\}\bigr|
\]
be the number of distinct pairwise distances. Define
\[\delta(n)=\min\{d(A): A\subset\mathbb{R}^2,\ |A|=n\}.\]
Let $\mathcal{M}_n$ be the family of minimizers $A$ with $|A|=n$ and $d(A)=\delta(n)$.
Two sets are similar if one is obtained from the other by a Euclidean similarity
(translation, rotation, reflection, uniform scaling).

Claim: There exists $N_0$ such that for all $n\ge N_0$ there are $A,B\in\mathcal{M}_n$ with $A$ not similar to $B$.

2) QUICK LITERATURE/CONTEXT CHECK
Browsing is not available. I only record what the problem statement itself claims:
- $n=3$: equilateral triangle is the unique minimizer.
- $n=4$: square and two equilateral triangles sharing an edge are two non-similar minimizers.
- $n=5$: regular pentagon is the unique minimizer (Kov\'acs).
- $6\le n\le 9$: at least two non-similar minimizers exist (Erd\H{o}s).
These claims are not re-proved here except for the $n=3,4$ cases below.

3) ATTACK PLAN
Proof strategies:
1. Show that for large $n$ the minimizer set $\mathcal{M}_n$ is non-rigid and has multiple
   combinatorial types.
2. Construct two explicit non-similar minimizers for all large $n$.
3. Use stability ideas: near-minimizers come in families, forcing non-uniqueness.

Disproof strategies:
1. Show that all minimizers are similar for infinitely many $n$ (unlikely).
2. Find a rigidity theorem that forces a unique combinatorial type for large $n$.

Chosen path: Prove the small cases $n=3$ and $n=4$ rigorously and isolate the gap for large $n$.

4) WORK
Lemma 1 (Three points).
For $n=3$, $\delta(3)=1$, and the unique minimizer up to similarity is the equilateral triangle.

Proof.
Any three non-collinear points determine at least one distance. If not all three distances are
 equal, then $d(A)\ge 2$. Thus $\delta(3)=1$, achieved only when all three pairwise distances are
 equal, i.e., the triangle is equilateral. ∎

Lemma 2 (Equidistant sets in the plane are small).
If all pairwise distances among points of $A\subset\mathbb{R}^2$ are equal, then $|A|\le 3$.

Proof.
Choose distinct $a,b,c\in A$; then $\triangle abc$ is equilateral. Any $x\in A$ must satisfy
$\|x-a\|=\|x-b\|=\|x-c\|$. The locus $\|x-a\|=\|x-b\|$ is a perpendicular bisector line, and the
 intersection of two such bisectors is a single point. Hence no fourth point exists. ∎

Lemma 3 (Four points).
For $n=4$, $\delta(4)=2$, and there are at least two non-similar minimizers: the square and the
"double equilateral triangle" (two equilateral triangles sharing an edge).

Proof.
By Lemma 2, four points cannot have all pairwise distances equal, so $\delta(4)\ge 2$.
Both the square and the double-equilateral configuration realize exactly two distances, so
$\delta(4)=2$ and there are at least two non-similar minimizers. ∎

These are the only fully verified cases in this attempt.

5) VERIFICATION
Edge cases:
- $n=3$ handled in Lemma 1.
- $n=4$ handled in Lemma 3.

Quantifiers:
The large-$n$ statement remains open; the lemmas only verify small-$n$ instances.

6) FINAL
**UNRESOLVED**

(i) Strongest fully proved partial result:
For $n=3$, the equilateral triangle is the unique minimizer (Lemma 1). For $n=4$, $\delta(4)=2$ and
there are at least two non-similar minimizers (Lemma 3).

(ii) First gap:
Construct or prove the existence of two non-similar minimizers for all sufficiently large $n$.

(iii) Top 3 next moves:
1. Identify structural families of minimizers for large $n$ and prove they are not all similar.
2. Establish stability results showing multiple combinatorial types achieve $\delta(n)$ for large $n$.
3. Prove monotonicity/extension results: if non-uniqueness holds for some $n$, propagate it to larger $n$.

(iv) Minimal counterexample structure (if statement is false):
Infinitely many $n$ for which all minimizers are similar, forced by a rigid geometric structure
(e.g., a unique lattice-like arrangement).
