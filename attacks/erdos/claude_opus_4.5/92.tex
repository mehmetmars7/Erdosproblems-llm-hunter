% Attempt for Erdős Problem #92
% Following PROMPT_STRATEGY.MD
% Tools/Constraints:
% - Web browsing available? NO (restricted)
% - Computation available (Python/Sage/Mathematica)? YES, not used

OUTPUT
1) FORMAL RESTATEMENT
For a finite set $A\subset\mathbb{R}^2$ with $|A|=n$, define
\[g_A(x)=\max_{r>0} |\{y\in A: \|x-y\|=r\}|.\]
Let
\[f(n)=\max_{|A|=n}\ \min_{x\in A} g_A(x).\]
The questions are:
(Q1) Is $f(n)\le n^{o(1)}$?
(Q2) Is $f(n)\le n^{C/\log\log n}$ for some constant $C>0$?

2) QUICK LITERATURE/CONTEXT CHECK
Browsing is not available. I only record what the problem statement itself claims:
- Lattice points give a lower bound $f(n) > n^{c/\log\log n}$.
- Upper bounds: $f(n)\ll n^{1/2}$ (trivial), $f(n)\ll n^{2/5}$ (Pach--Sharir), and
  $f(n)\ll n^{4/11}$ (Janzer--Janzer--Methuku--Tardos via Hunter).
- Fishburn showed $f(6)=3$ and $f(8)=4$ (as reported).
These are not re-proved here.

3) ATTACK PLAN
Proof strategies:
1. Translate the condition "many points equidistant from each vertex" into a dense point-circle
   incidence problem and apply incidence bounds.
2. Use energy arguments to convert per-vertex multiplicity into global distance counts.
3. Search for structural rigidity in configurations that maximize the minimum multiplicity.

Disproof strategies:
1. Construct algebraic or lattice-based sets with unusually large $g_A(x)$ for every $x$.
2. Use product sets or grid-like configurations to force high multiplicities.

Chosen path: Prove only elementary bounds and small-case examples, and record the first gap.

4) WORK
Lemma 1 (Trivial upper bound).
For any $n\ge 2$, $f(n)\le n-1$.

Proof.
For any $A$ and any $x\in A$, there are only $n-1$ other points, so $g_A(x)\le n-1$.
Taking the minimum over $x$ and the maximum over $A$ gives $f(n)\le n-1$. ∎

Lemma 2 (Uniform lower bound $\ge 2$ for $n\ge 3$).
For all $n\ge 3$, $f(n)\ge 2$.

Proof.
Take $A$ to be the vertices of a regular $n$-gon. For any vertex $x$, the two adjacent vertices
are at the same distance from $x$, so $g_A(x)\ge 2$. Thus $\min_{x\in A} g_A(x)\ge 2$, and hence
$f(n)\ge 2$. ∎

Lemma 3 (Exact value for $n=2$ and $n=3$).
We have $f(2)=1$ and $f(3)=2$.

Proof.
For $n=2$, each point has exactly one other point at any distance, so $f(2)=1$.
For $n=3$, Lemma 2 gives $f(3)\ge 2$, and the upper bound $f(3)\le 2$ follows from Lemma 1. ∎

These results do not approach the conjectured upper bounds in (Q1)--(Q2), but they are rigorous.

5) VERIFICATION
Edge cases:
- $n=1$: vacuous, but $f(1)=0$ by convention (no other points).
- $n=2$: Lemma 3 handles the exact value.

Quantifier checks:
The claims in this attempt are unconditional and elementary; the asymptotic questions (Q1),(Q2)
remain open.

6) FINAL
**UNRESOLVED**

(i) Strongest fully proved partial result:
Elementary bounds $2\le f(n)\le n-1$ for $n\ge 3$, and exact values $f(2)=1$, $f(3)=2$.

(ii) First gap:
Any nontrivial subpolynomial upper bound (such as $n^{o(1)}$) requires incidence geometry or
other deep methods beyond this attempt.

(iii) Top 3 next moves:
1. Reproduce a point-circle incidence bound strong enough to yield $f(n)\ll n^{2/5}$.
2. Rework the lattice-point construction to provide a clean self-contained lower bound
   $f(n)\ge n^{c/\log\log n}$.
3. Explore whether the minimum-over-vertices condition forces new constraints beyond
   the unit-distance problem.

(iv) Minimal counterexample structure (if (Q1) is false):
A family of configurations where every vertex lies on a large circle containing
$n^{\epsilon}$ points for some fixed $\epsilon>0$, uniformly over all vertices.
