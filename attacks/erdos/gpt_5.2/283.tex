% Attempt for Erdos Problem #283
% Following PROMPT_STRATEGY.MD
% Tools/Constraints:
% - Web browsing available? NO (restricted)
% - Computation available? YES (local Python), used only to verify the example
% GPT 5.2

OUTPUT
1) FORMAL RESTATEMENT
Let $p\in\mathbb{Z}[x]$ be a polynomial with positive leading coefficient and with no fixed divisor on $\{1,2,3,\dots\}$:
there is no integer $d\ge 2$ such that $d\mid p(n)$ for every $n\ge 1$.

Question: Is it true that there exists $M_0$ (depending on $p$) such that for every integer $m\ge M_0$
there exist integers $k\ge 1$ and $1\le n_1<\cdots<n_k$ with
\[1=\sum_{i=1}^k \frac1{n_i}\qquad\text{and}\qquad m=\sum_{i=1}^k p(n_i)?\]

2) QUICK LITERATURE/CONTEXT CHECK
Browsing is not available. The statement contains several claimed results for special $p$ and a verified example for $p(x)=x^2$ and $m=200$. I only use the example as a sanity check and do not rely on the other literature claims.

3) ATTACK PLAN
Proof track:
1. Build flexible families of Egyptian fraction decompositions of $1$ with distinct denominators.
2. Track how $\sum p(n_i)$ changes under local “splitting” moves that preserve $\sum 1/n_i$.
3. Try to obtain an “additive basis” property for achievable $m$ values via many small adjustable moves.

Disproof track:
1. Look for congruence obstructions on $m=\sum p(n_i)$ induced by the constraint $\sum 1/n_i=1$.
2. Try small-degree polynomials $p$ and search computationally for missing residue classes of achievable $m$.

Chosen path: prove that achievable $m$ are unbounded for every such $p$ (but not that all large $m$ occur).

4) WORK
Lemma 1 (Infinitely many Egyptian fraction decompositions of $1$ with distinct denominators).
Starting from $1=\frac12+\frac13+\frac16$, one can generate infinitely many identities
\[1=\sum_{i=1}^k \frac1{n_i}\]
with strictly increasing $n_1<\cdots<n_k$ by repeatedly applying the splitting identity
\[\frac1{n}=\frac1{n+1}+\frac1{n(n+1)}.\]

Proof.
The splitting identity is verified by bringing to a common denominator:
\[\frac1{n+1}+\frac1{n(n+1)}=\frac{n}{n(n+1)}+\frac1{n(n+1)}=\frac{n+1}{n(n+1)}=\frac1n.\]
Starting from the 3-term decomposition $1=\frac12+\frac13+\frac16$, pick the largest denominator $n$ in the current decomposition and replace $1/n$ by $1/(n+1)+1/(n(n+1))$.
Both new denominators exceed $n$, hence exceed all previous denominators; thus denominators remain distinct and increasing.
This increases the number of terms by $1$ and preserves the sum $1$.
Iterating yields infinitely many distinct decompositions. \qed

Lemma 2 (Achievable $m$ are unbounded for every $p$ with positive leading coefficient).
Let $p\in\mathbb{Z}[x]$ have positive leading coefficient. Then there exist solutions to the reciprocal equation
\[1=\sum_{i=1}^k \frac1{n_i}\]
with distinct $n_i$ for which the corresponding value $m=\sum_i p(n_i)$ is arbitrarily large.

Proof.
By Lemma 1 there are infinitely many decompositions of $1$ with distinct denominators, obtained by repeatedly splitting the current largest denominator.
Consider one splitting step applied to a term $1/n$, replacing it by $1/(n+1)+1/(n(n+1))$.
This changes the $p$-sum by
\[\Delta(n)=p(n+1)+p(n(n+1)) - p(n).\]
Since $p$ has positive leading coefficient, $p(t)\to +\infty$ as $t\to+\infty$, and in particular $p(n(n+1))\to+\infty$ as $n\to\infty$.
Therefore $\Delta(n)\to+\infty$ as $n\to\infty$, so for all sufficiently large $n$ we have $\Delta(n)>0$.
Along the iterative splitting process, the largest denominator tends to infinity, so eventually we apply a split at some $n$ with $\Delta(n)>0$, strictly increasing $m$.
Repeating thereafter produces decompositions with $m$ growing without bound. \qed

FAST REALITY CHECK (given example).
For $n_i\in\{2,4,6,12\}$,
\[\frac12+\frac14+\frac16+\frac1{12}=1\quad\text{and}\quad 2^2+4^2+6^2+12^2=200.\]
I verified both equalities exactly by computation.

5) VERIFICATION
- Lemma 1’s splitting operation preserves the reciprocal sum and strictly increases denominators, so it never creates repeats.
- Lemma 2 uses only that $p(n(n+1))$ dominates and tends to $+\infty$; thus it shows unboundedness of achievable $m$ but not coverage of all large $m$.
- The example check is arithmetic only.

6) FINAL
**UNRESOLVED**

(i) Strongest fully proved partial result:
For any integer polynomial $p$ with positive leading coefficient, there exist solutions with $m=\sum_i p(n_i)$ arbitrarily large (Lemma 2). In particular the set of achievable $m$ is infinite and unbounded.

(ii) First gap:
Upgrade “infinitely many (unbounded) achievable $m$” to “all sufficiently large $m$ are achievable”; concretely, show that for some $M_0$ every $m\ge M_0$ can be hit by choosing an Egyptian fraction decomposition of $1$ whose $p$-sum equals $m$.

(iii) Top 3 next moves:
1. Develop local moves on decompositions of $1$ that change $\sum p(n_i)$ by controlled \emph{small} increments (e.g. a finite generating set of $\Delta$-values with gcd $1$).
2. For specific $p$ (say $p(x)=x^2$), compute achievable $m$ up to a moderate bound and look for modular obstructions or a stabilization phenomenon.
3. Combine “many degrees of freedom” in Egyptian fraction expansions with additive number theory to represent every large $m$ (e.g. show the set of achievable $\sum p(n_i)$ contains a long interval).

(iv) Minimal counterexample structure (if the conjecture were false):
A polynomial $p$ as above and an infinite sequence of integers $m_j\to\infty$ such that no Egyptian fraction decomposition of $1$ with distinct denominators yields $\sum p(n_i)=m_j$; such a failure would likely come from a hidden congruence obstruction linking $\sum 1/n_i=1$ to $\sum p(n_i)$.

