% Attempt for Erdos Problem #307
% Following PROMPT_STRATEGY.MD
% Tools/Constraints:
% - Web browsing available? NO (restricted)
% - Computation available? YES (local Python), used
% GPT 5.2

OUTPUT
1) FORMAL RESTATEMENT
Question (prime version): do there exist finite nonempty sets of primes $P,Q$ such that
\[1=\left(\sum_{p\in P}\frac1p\right)\left(\sum_{q\in Q}\frac1q\right)?\]

Question (weakened coprime version): if we allow $P,Q$ to be finite sets of pairwise coprime integers (not necessarily prime),
does such an identity exist? What if additionally $1\notin P\cup Q$?

2) QUICK LITERATURE/CONTEXT CHECK
Browsing is not available. The statement gives explicit examples for the weakened version (allowing $1$) and claims simple necessary conditions for the prime version. I verify those necessary conditions and the examples.

3) ATTACK PLAN
Proof track:
1. Prove structural constraints on sums of reciprocals of primes (exact reduced denominator).
2. Deduce disjointness of $P$ and $Q$ in the prime version and a lower bound on $|P\cup Q|$ from $\sum 1/p\ge 2$.
3. Attempt to use these constraints to narrow the Diophantine search space (computationally, if feasible).

Disproof track:
1. Try to show no solutions exist by analyzing prime-adic valuations in the reduced forms of the two sums.

Chosen path: prove the easy necessary conditions and do small numeric sanity checks; the existence question remains open here.

4) WORK
Lemma 1 (Reduced form of a sum of reciprocal primes).
Let $P$ be a finite set of primes and set
\[x=\sum_{p\in P}\frac1p.\]
Then
\[x=\frac{A}{\prod_{p\in P}p}\]
for some integer $A$, and moreover $\gcd\!\left(A,\prod_{p\in P}p\right)=1$ (so the denominator in lowest terms is exactly $\prod_{p\in P}p$).

Proof.
Let $D=\prod_{p\in P}p$. Then
\[x=\frac{1}{D}\sum_{p\in P}\frac{D}{p}=\frac{A}{D}\]
with $A=\sum_{p\in P} D/p\in\mathbb{Z}$.
Fix $p_0\in P$. Reduce $A$ modulo $p_0$.
For $p\ne p_0$, the term $D/p$ is divisible by $p_0$ (since it still contains the factor $p_0$), hence $D/p\equiv 0\pmod{p_0}$.
For $p=p_0$, the term $D/p_0=\prod_{p\in P\setminus\{p_0\}}p$ is not divisible by $p_0$, so $A\not\equiv 0\pmod{p_0}$.
Thus no prime in $P$ divides $A$, i.e. $\gcd(A,D)=1$. \qed

Lemma 2 (Prime version forces disjointness and a size lower bound).
If $P,Q$ are finite sets of primes and
\[1=\left(\sum_{p\in P}\frac1p\right)\left(\sum_{q\in Q}\frac1q\right),\]
then:
(a) $P\cap Q=\varnothing$.
(b) $\sum_{r\in P\cup Q}\frac1r\ge 2$ and hence $|P\cup Q|\ge 59$.

Proof.
Let $x=\sum_{p\in P}1/p$ and $y=\sum_{q\in Q}1/q$, so $xy=1$.
By Lemma 1, $x=A/D_P$ and $y=B/D_Q$ in lowest terms with $D_P=\prod_{p\in P}p$ and $D_Q=\prod_{q\in Q}q$.
If some prime $p$ lay in $P\cap Q$, then $p\mid D_P$ and $p\mid D_Q$ while $p\nmid A,B$.
Thus $p^2\mid D_P D_Q$ and $p\nmid AB$, so the reduced form of $xy=AB/(D_P D_Q)$ would have denominator divisible by $p^2$, contradicting $xy=1$.
Hence $P\cap Q=\varnothing$.

For (b), since $x,y>0$ and $xy=1$, AM--GM gives $x+y\ge 2\sqrt{xy}=2$.
But $x+y=\sum_{r\in P\cup Q}1/r$, giving the stated inequality.
Finally, by exact computation (see FAST REALITY CHECK), the sum of reciprocals of the first $58$ primes is $<2$ while for the first $59$ primes it is $>2$.
Therefore any set of primes with reciprocal sum $\ge 2$ has size at least $59$. \qed

FAST REALITY CHECK (computation).
- Verified the weakened examples:
\[(1+1/5)(1/2+1/3)=(6/5)(5/6)=1,\qquad (1+1/41)(1/2+1/3+1/7)=(42/41)(41/42)=1.\]
- Checked prime-reciprocal thresholds exactly:
\[\sum_{i=1}^{58}\frac{1}{p_i}<2<\sum_{i=1}^{59}\frac{1}{p_i}\]
where $p_i$ is the $i$th prime. Hence Lemma 2 gives $|P\cup Q|\ge 59$ (note: the problem statement says $\ge 60$, but the threshold is reached at $59$).

5) VERIFICATION
- Lemma 1’s “no cancellation” claim is checked prime-by-prime using a single congruence class.
- Lemma 2(a) uses Lemma 1 to force denominators squarefree and uncancellable; the $p^2$ obstruction is exact.
- Lemma 2(b) is purely AM--GM plus a computed threshold.

6) FINAL
**UNRESOLVED**

(i) Strongest fully proved partial result:
If a solution with prime sets exists, then $P$ and $Q$ must be disjoint and $\sum_{r\in P\cup Q}1/r\ge 2$, forcing $|P\cup Q|\ge 59$ (Lemmas 1–2).

(ii) First gap:
Construct explicit prime sets $P,Q$ satisfying the product identity, or prove that no such prime sets can exist.

(iii) Top 3 next moves:
1. Use Lemma 1 to rewrite the problem as an explicit Diophantine equation $AB=D_P D_Q$ with coprimality constraints and attempt to derive contradictions modulo carefully chosen primes.
2. Perform a guided computational search over candidate sets of primes (likely huge, but one might impose additional structure such as bounding the largest prime or forcing near-equality of $x$ and $y$).
3. Analyze the weakened “pairwise coprime” version with $1\notin P\cup Q$ first; any construction there might be adaptable to primes via splitting identities.

(iv) Minimal counterexample structure (if a solution exists):
A solution must involve at least $59$ distinct primes split into two disjoint groups $P,Q$, with $\sum_{p\in P}1/p$ and $\sum_{q\in Q}1/q$ being reduced fractions whose numerators multiply to the product of all primes used.

