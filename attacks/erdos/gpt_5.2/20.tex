
OUTPUT
1) FORMAL RESTATEMENT
Fix integers $n\ge 1$ and $k\ge 2$. A \emph{$k$-sunflower} (or $\Delta$-system) is a collection of $k$ \emph{distinct}
sets $S_1,\dots,S_k$ such that there exists a set $K$ (the \emph{kernel}) with
$S_i\cap S_j = K$ for all $i\ne j$.
An \emph{$n$-uniform} family is a family of sets each of size $n$.
Define $f(n,k)$ to be the least integer such that every $n$-uniform family $\mathcal F$ with
$|\mathcal F|\ge f(n,k)$ contains a $k$-sunflower.
Question: for each fixed $k$, is there a constant $c_k>0$ such that $f(n,k) < c_k^n$ for all $n$?

2) QUICK LITERATURE/CONTEXT CHECK
Browsing is not available. I only record what the problem statement itself claims about known bounds.
I have not verified those results here.

3) ATTACK PLAN
Proof track:
1. Prove exponential-in-$n$ bounds on $f(n,k)$ by combinatorial compression or entropy methods.
2. For fixed $k$ (e.g. $k=3$), refine known arguments to remove extra $\log n$ factors in the base.

Disproof track:
1. Construct large $n$-uniform families with no $k$-sunflower and size superexponential in $n$.
2. Search for algebraic constructions (finite geometries) that avoid sunflowers.

Chosen path: prove basic properties and exact small cases that can be checked elementarily.

4) WORK
Lemma 1 (Trivial case $k=2$).
For every $n\ge 1$, $f(n,2)=2$.

Proof.
Any two distinct sets $S_1,S_2$ form a $2$-sunflower with kernel $K=S_1\cap S_2$.
Thus any $n$-uniform family of size $2$ contains a $2$-sunflower, so $f(n,2)\le 2$.
Also $f(n,2)>1$ because a family of size $1$ contains no $2$-sunflower. Hence $f(n,2)=2$. \qed

Lemma 2 (Exact value for $n=1$).
For every $k\ge 2$, $f(1,k)=k$.

Proof.
A $1$-uniform family is a set of distinct singletons. Any $k$ distinct singletons have pairwise intersection $\emptyset$,
so they form a $k$-sunflower with kernel $\emptyset$. Hence $f(1,k)\le k$.
On the other hand, a family of size $k-1$ contains no $k$-sunflower. Thus $f(1,k)=k$. \qed

Lemma 3 (Monotonicity in $k$).
For fixed $n$, the function $k\mapsto f(n,k)$ is nondecreasing.

Proof.
If every family of size at least $M$ contains a $(k+1)$-sunflower, then it also contains a $k$-sunflower
(by discarding one petal). Thus the minimal threshold for $(k+1)$ cannot be smaller than that for $k$. \qed

5) VERIFICATION
- Lemma 1 uses only the definition of a $2$-sunflower.
- Lemma 2 depends on the convention that the family has no duplicate sets.
- Lemma 3 is immediate from containment of configurations.

6) FINAL
**UNRESOLVED**

(i) Strongest fully proved partial result:
Exact values in trivial regimes: $f(n,2)=2$ and $f(1,k)=k$, plus monotonicity in $k$.

(ii) First gap:
No nontrivial upper bound of the form $f(n,k)\le c_k^n$ is proved here for $k\ge 3$.

(iii) Top 3 next moves:
1. Implement an explicit proof of the Erd\H{o}s--Rado bound and then track constants to compare with $c_k^n$.
2. Study the modern proofs (ALWZ-style) and isolate the step responsible for the $\log n$ factor in the base.
3. Attempt lower-bound constructions to test sharpness for small $k$ (especially $k=3$).

(iv) Minimal counterexample structure (if $f(n,k)<c_k^n$ were false):
A family of $n$-sets of size larger than $c^n$ for every constant $c$, yet avoiding any $k$-sunflower,
likely relying on rigid algebraic structure preventing repeated intersections.
