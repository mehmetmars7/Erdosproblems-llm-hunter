
OUTPUT
1) FORMAL RESTATEMENT
Definitions/Conventions.
- We work in the Euclidean plane. The \emph{unit square} is $[0,1]\times[0,1]$.
- A \emph{square} means a (possibly rotated) closed square region with positive side length.
- ``No common interior point'' is interpreted as: the interiors of the $n$ squares are pairwise disjoint, i.e. no point lies in the interior of two distinct squares.
  (Equivalently, the interiors form a packing.)

For $n\ge 1$, define $f(n)$ to be the supremum (or maximum, if attained) of
\[\sum_{i=1}^n s_i\]
over all collections of $n$ squares contained in the unit square with pairwise disjoint interiors, where $s_i$ is the side length of the $i$th square.

Question.
Is it true that for every integer $k\ge 1$,
\[f(k^2+1)=k?\]
Also: for which $n$ does $f(n+1)=f(n)$ hold?

2) QUICK LITERATURE/CONTEXT CHECK
Web browsing is not available in this session.
I only use information stated in the problem text (e.g. that $f(k^2)=k$ is trivial by Cauchy--Schwarz, and that the conjecture $f(k^2+1)=k$ is open).

3) ATTACK PLAN
Proof track ideas:
1. Use area: disjoint interiors imply $\sum s_i^2\le 1$, giving strong universal upper bounds on $\sum s_i$.
2. Try to sharpen the area-only bound in the special case $n=k^2+1$ by exploiting that $k^2$ squares already tile optimally, leaving no area slack.

Disproof track ideas:
1. Try to build packings with $k^2+1$ squares whose side-length sum exceeds $k$ (the text mentions many lower bounds of this type for $k^2+2c$ and $k^2+2c+1$).

Chosen path in this attempt: establish the area/Cauchy--Schwarz upper bound rigorously and derive the exact value $f(k^2)=k$, and record the simple lower bound $f(k^2+1)\ge k$.

4) WORK
Lemma 1 (Area bound).
If $n$ squares of side lengths $s_1,\dots,s_n$ are placed inside the unit square with pairwise disjoint interiors, then
\[\sum_{i=1}^n s_i^2 \le 1.\]

Proof.
The area of the $i$th square is $s_i^2$.
Since the interiors are pairwise disjoint and all squares lie inside the unit square (area $1$), the sum of their areas is at most $1$:
\[\sum_{i=1}^n s_i^2 \le 1.\qedhere\]

Lemma 2 (Cauchy--Schwarz upper bound on total side length).
Under the same hypotheses,
\[\sum_{i=1}^n s_i \le \sqrt{n}.\]
In particular, $f(n)\le \sqrt n$ for all $n$.

Proof.
By Cauchy--Schwarz,
\[\left(\sum_{i=1}^n s_i\right)^2 \le \left(\sum_{i=1}^n 1^2\right)\left(\sum_{i=1}^n s_i^2\right)=n\sum_{i=1}^n s_i^2.\]
By Lemma 1, $\sum s_i^2\le 1$, hence $(\sum s_i)^2\le n$, i.e. $\sum s_i\le \sqrt n$.
Taking the supremum over all packings shows $f(n)\le \sqrt n$. \qed

Lemma 3 (Exact value $f(k^2)=k$).
For every integer $k\ge 1$,
\[f(k^2)=k.\]

Proof.
Upper bound: apply Lemma 2 with $n=k^2$ to get $f(k^2)\le \sqrt{k^2}=k$.

Lower bound: partition the unit square into a $k\times k$ grid of $k^2$ axis-parallel squares, each of side length $1/k$.
These squares have disjoint interiors and are contained in the unit square, and their side lengths sum to $k^2\cdot(1/k)=k$.
Thus $f(k^2)\ge k$.

Combining both bounds gives $f(k^2)=k$. \qed

Lemma 4 (Simple lower bound for $k^2+1$ squares).
For every integer $k\ge 1$,
\[f(k^2+1)\ge k.\]

Proof.
Start from the $k^2$-square grid packing from Lemma 3, with each square of side length $1/k$.
Remove one of these squares and replace it by two axis-parallel squares of side length $1/(2k)$ placed inside the removed square (e.g. side-by-side).
This yields $(k^2-1)+2=k^2+1$ squares with disjoint interiors in the unit square.
The total side-length sum is
\[(k^2-1)\cdot\frac1k + 2\cdot\frac1{2k} = \frac{k^2-1}{k} + \frac{1}{k}=k.\]
Therefore $f(k^2+1)\ge k$. \qed

Lemma 5 (Monotonicity).
The function $f(n)$ is nondecreasing: $f(n+1)\ge f(n)$ for all $n\ge 1$.

Proof.
Take any packing of $n$ squares in the unit square with total side-length sum within $\varepsilon>0$ of $f(n)$.
Place an additional square of side length $\delta>0$ (very small) in a corner region of the unit square not intersecting the interiors of the existing squares; this is possible by choosing $\delta$ small enough.
Then we obtain a packing of $n+1$ squares with total side-length sum increased by $\delta$.
Letting $\delta\downarrow 0$ and then $\varepsilon\downarrow 0$ shows $f(n+1)\ge f(n)$. \qed

FAST REALITY CHECK
- For $k=1$, Lemma 3 gives $f(1)=1$ and Lemma 4 gives $f(2)\ge 1$; Lemma 2 gives $f(2)\le \sqrt2$ (so $f(2)=1$ is plausible, as stated in the problem text).
- For $k=2$, Lemma 3 gives $f(4)=2$ and Lemma 4 gives $f(5)\ge 2$; Lemma 2 gives $f(5)\le \sqrt5\approx 2.236$ (consistent with $f(5)=2$ stated in the text).

5) VERIFICATION
- Lemmas 1--2 use only disjointness of interiors and total area $\le 1$.
- Lemma 3 is tight because the grid construction saturates the area bound with equal squares, where Cauchy--Schwarz has equality.
- Lemma 4 uses an explicit packing modification and preserves disjoint interiors by nesting inside one square.

6) FINAL
**UNRESOLVED**

(i) Strongest fully proved partial result:
I proved the universal upper bound $f(n)\le \sqrt n$ and the exact value $f(k^2)=k$ (Lemmas 2--3), and the easy lower bound $f(k^2+1)\ge k$ (Lemma 4).

(ii) First gap:
Prove the matching upper bound $f(k^2+1)\le k$ (or construct a packing with sum $>k$).
Area/Cauchy--Schwarz alone only gives $f(k^2+1)\le \sqrt{k^2+1}=k+O(1/k)$, which is too weak.

(iii) Top 3 next moves (concrete):
1. Seek a refinement of Lemma 1 that uses not only area but also boundary/packing constraints (e.g. by integrating projections) to upper bound $\sum s_i$ when $n=k^2+1$.
2. Analyze equality cases of the Cauchy--Schwarz bound: for $n=k^2+1$, equality is impossible, so quantify the inevitable slack in $\sum s_i$ given $\sum s_i^2\le 1$ and geometric packing constraints.
3. Attempt to optimize explicit constructions for $k^2+1$ squares (beyond the ``split one square'' construction) and see whether any arrangement can push the sum above $k$ for small $k$.

(iv) What a minimal counterexample would likely look like (structure/parameters):
For some fixed $k$, a packing of $k^2+1$ squares in the unit square with total side-length sum $k+\delta$ for some $\delta>0$,
meaning the packing uses non-equal square sizes in a way that beats the naive ``grid + split'' construction while still respecting the tight area constraint.

