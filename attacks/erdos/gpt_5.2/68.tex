% Attempt for Erdos Problem #68
% Following PROMPT_STRATEGY.MD
% Tools/Constraints:
% - Web browsing available? NO (restricted)
% - Computation available? YES (local Python), used for numerical sanity check
% GPT 5.2

OUTPUT
1) FORMAL RESTATEMENT
Definitions/Conventions.
Let
\[S:=\sum_{n=2}^{\infty}\frac{1}{n!-1},\]
which converges absolutely.

Question.
Is $S$ irrational?

2) QUICK LITERATURE/CONTEXT CHECK
Web browsing is not available in this session.
From the problem text only:
- Weisenberg observed the identity $S=\sum_{k\ge 1}\sum_{n\ge 2}(n!)^{-k}$.
- Erd\H{o}s noted related series $\sum 1/(n!+t)$ should be transcendental.
I do not claim any further known results.

3) ATTACK PLAN
Proof track ideas:
1. Try to adapt standard irrationality arguments for rapidly convergent series: assume $S=p/q$, multiply by a carefully chosen integer (e.g. $N!$ or an lcm) and derive a contradiction by bounding the fractional part.
2. Use the geometric-series expansion $1/(n!-1)=\sum_{k\ge 1} (n!)^{-k}$ to rewrite $S$ in a form amenable to factorial-base expansions.

Disproof track ideas:
1. Produce an exact rational closed form by telescoping or expressing $1/(n!-1)$ as a difference of rational terms with factorial denominators (seems unlikely).

Chosen path in this attempt: prove two basic, fully rigorous identities/bounds and compute a high-precision numerical value to support sanity, but I do not reach irrationality.

4) WORK
Lemma 1 (Geometric-series expansion).
For every integer $n\ge 2$,
\[\frac{1}{n!-1}=\sum_{k=1}^{\infty}\frac{1}{(n!)^k}.\]

Proof.
For $n\ge 2$, we have $n!\ge 2$, so $|1/n!|<1$.
Compute
\[\frac{1}{n!-1}=\frac{1}{n!\left(1-\frac{1}{n!}\right)}=\frac{1}{n!}\cdot \frac{1}{1-\frac{1}{n!}}.\]
Since $|1/n!|<1$, the geometric series converges:
\[\frac{1}{1-\frac{1}{n!}}=\sum_{k=0}^{\infty}\left(\frac{1}{n!}\right)^k.\]
Multiplying by $1/n!$ gives
\[\frac{1}{n!-1}=\sum_{k=0}^{\infty}\frac{1}{(n!)^{k+1}}=\sum_{k=1}^{\infty}\frac{1}{(n!)^k}.\qed\]

Lemma 2 (A concrete tail bound).
For every $N\ge 2$,
\[\sum_{n=N+1}^{\infty}\frac{1}{n!-1}\ <\ \frac{2(N+2)}{(N+1)!(N+1)}.\]

Proof.
For $n\ge 2$ we have $n!-1\ge \tfrac12 n!$, hence
\[\frac{1}{n!-1}\le \frac{2}{n!}.\]
Therefore
\[\sum_{n=N+1}^{\infty}\frac{1}{n!-1}\le 2\sum_{n=N+1}^{\infty}\frac{1}{n!}.\]
Now factor out $(N+1)!$:
\[\sum_{n=N+1}^{\infty}\frac{1}{n!}=\frac{1}{(N+1)!}\sum_{j=0}^{\infty}\frac{1}{(N+2)(N+3)\cdots(N+1+j)}.\]
For each $j\ge 1$, the denominator is at least $(N+2)^j$, hence
\[\sum_{j=0}^{\infty}\frac{1}{(N+2)(N+3)\cdots(N+1+j)}\le 1+\sum_{j=1}^{\infty}\frac{1}{(N+2)^j}=\frac{1}{1-\frac{1}{N+2}}=\frac{N+2}{N+1}.\]
Thus
\[\sum_{n=N+1}^{\infty}\frac{1}{n!}\le \frac{1}{(N+1)!}\cdot\frac{N+2}{N+1},\]
and multiplying by $2$ gives the stated tail bound. \qed

FAST REALITY CHECK (numerics).
Using high precision arithmetic, the partial sum up to $n=20$ is
\[\sum_{n=2}^{20}\frac{1}{n!-1}\approx 1.2534987556999534716228579572195523287580046561.\]
Lemma 2 with $N=20$ gives a tail bound
\[\sum_{n\ge 21}\frac{1}{n!-1}<4.11\times 10^{-20}.\]
So the displayed decimal is correct to at least $19$ decimal places.

5) VERIFICATION
- Lemma 1 is a direct geometric-series manipulation and is valid because $|1/n!|<1$ for $n\ge 2$.
- Lemma 2 uses only the inequality $n!-1\ge \tfrac12 n!$ and a simple domination of the factorial tail by a geometric series.
- The numerical value is consistent with the rigorous tail bound.

6) FINAL
**UNRESOLVED**

(i) Strongest fully proved partial result:
I proved the exact expansion $\frac{1}{n!-1}=\sum_{k\ge 1}(n!)^{-k}$ (Lemma 1) and a very strong explicit tail bound (Lemma 2),
which yields high-precision control of the value of the series from modest partial sums.

(ii) First gap:
Turn the rapid convergence and factorial structure into an irrationality proof: show that no rational $p/q$ can match the value because
after multiplying by a suitable integer, the fractional part cannot be both nonzero and arbitrarily small.

(iii) Top 3 next moves (concrete):
1. Attempt a rational-contradiction argument: assume $S=p/q$, choose $M$ divisible by many $(n!-1)$, and bound $|MS-\mathrm{round}(MS)|$ from above (via Lemma 2) and below (via a congruence obstruction).
2. Explore factorial-base expansions using Lemma 1: rewrite $S$ as $\sum_{n,k}(n!)^{-k}$ and try to show that the factorial digits cannot eventually become periodic (a standard signature of rationality).
3. Investigate whether known irrationality criteria for series of rational terms (e.g. Engel/L\"uroth-type expansions) can be adapted to denominators $n!-1$.

(iv) What a minimal counterexample would likely look like (structure/parameters):
If $S$ were rational, then there would exist integers $p,q$ such that for all large $N$,
the quantity $q\sum_{n=N+1}^{\infty}\frac{1}{n!-1}$ would have to cancel exactly the fractional part of $q\sum_{n=2}^{N}\frac{1}{n!-1}$.
Given Lemma 2 makes the tail extremely small, this would force extraordinarily rigid congruence relations among the partial sums mod $1$,
suggesting that any rationality proof would need a hidden telescoping/congruence structure in $(n!-1)^{-1}$ that is not apparent.

