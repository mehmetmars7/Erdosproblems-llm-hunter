% Attempt for Erdos Problem #36
% Following PROMPT_STRATEGY.MD
% Tools/Constraints:
% - Web browsing available? NO (restricted)
% - Computation available? YES (local Python), not used
% GPT 5.2

OUTPUT
1) FORMAL RESTATEMENT
Let $N\ge 1$ and let $A\sqcup B=\{1,\dots,2N\}$ be a partition with $|A|=|B|=N$.
For each integer $x$, define
\[r(x):=|\{(a,b)\in A\times B: a-b=x\}|.
\]
Problem: find the optimal constant $c>0$ such that for all sufficiently large $N$ and all such partitions,
there exists an $x$ with $r(x)\ge cN$.

2) QUICK LITERATURE/CONTEXT CHECK
Browsing is not available. I only record what the problem statement itself claims.
I have not verified those results here.

3) ATTACK PLAN
Proof track:
1. Use averaging: $\sum_x r(x)=|A||B|=N^2$ and there are $4N-1$ possible differences.
2. Improve the average bound using structural information about $A,B$ (beyond this attempt).

Disproof track:
1. Construct explicit partitions that keep every $r(x)$ small (interval-type examples).

Chosen path: prove the trivial $c\ge 1/4$ lower bound and verify the interval example gives $c\le 1/2$.

4) WORK
Lemma 1 (Sum of representation counts).
For any partition $A\sqcup B=\{1,\dots,2N\}$ with $|A|=|B|=N$,
\[\sum_{x=-(2N-1)}^{2N-1} r(x)=N^2.
\]

Proof.
Each ordered pair $(a,b)\in A\times B$ contributes exactly once to the sum, at $x=a-b$.
There are $|A||B|=N^2$ such pairs. \qed

Corollary 2 (Trivial lower bound $c\ge 1/4$).
For every such partition, there exists $x$ with
\[r(x)\ge \frac{N^2}{4N-1} > \frac{N}{4}.
\]

Proof.
There are exactly $4N-1$ integers in the range $[-(2N-1),\dots,2N-1]$.
By Lemma 1, the average value of $r(x)$ over this range is $N^2/(4N-1)$, so the maximum is at least the average.
\qed

Example 3 (An upper bound $c\le 1/2$ from an interval partition).
Assume $N$ is even and take $A=\{N/2+1,\dots,3N/2\}$ and $B$ its complement in $\{1,\dots,2N\}$.
Then $\max_x r(x)\le N/2$, so any universal $c$ must satisfy $c\le 1/2$.

Proof.
Write $B=B_1\sqcup B_2$ where $B_1=\{1,\dots,N/2\}$ and $B_2=\{3N/2+1,\dots,2N\}$.
For fixed $x$, the condition $a-b=x$ determines $a=b+x$.
Thus
\[
r(x)=\bigl|(B_1+x)\cap A\bigr|+\bigl|(B_2+x)\cap A\bigr|.
\]
Now $B_1+x=[1+x,N/2+x]$ and $B_2+x=[3N/2+1+x,2N+x]$ are intervals of length $N/2$ whose distance apart is
\[
(3N/2+1+x)-(N/2+x)=N+1.
\]
Any interval that meets both $B_1+x$ and $B_2+x$ must have length at least $N+2$.
Since $A=[N/2+1,3N/2]$ has length exactly $N$, it cannot intersect both $B_1+x$ and $B_2+x$.
Therefore at most one of the two intersection terms is nonzero, and the nonzero one is bounded by $|B_i|=N/2$.
Hence $r(x)\le N/2$ for all $x$. \qed

5) VERIFICATION
- Corollary 2 is a direct pigeonhole argument.
- Example 3: the final inequality $r(x)\le N/2$ relies on bounding solutions by the size of the relevant $b$-interval.

6) FINAL
**UNRESOLVED**

(i) Strongest fully proved partial result:
A universal bound $c\ge 1/4$ (Corollary 2) and an explicit construction giving $c\le 1/2$ (Example 3).

(ii) First gap:
No improvement beyond the averaging bound is obtained.

(iii) Top 3 next moves:
1. Tighten Example 3 by computing $\max_x r(x)$ exactly and compare to known best upper bounds.
2. Develop a structural inequality relating the distribution of $A$ and $B$ to the maximum of $r(x)$.
3. Explore Fourier analysis on $\mathbb Z$ (or $\mathbb Z/(2N)\mathbb Z$) to bound the maximum overlap.

(iv) Minimal counterexample structure (if $c$ were smaller than current lower bounds):
A partition where the difference multiset $A-B$ is unusually "flat", i.e. all multiplicities $r(x)$ are close to $N/4$.
