% Attempt for Erdos Problem #65
% Following PROMPT_STRATEGY.MD
% Tools/Constraints:
% - Web browsing available? NO (restricted)
% - Computation available? YES (local), used only for tiny numeric sanity check
% GPT 5.2

OUTPUT
1) FORMAL RESTATEMENT
Definitions/Conventions.
- $G$ is a finite simple graph with $n$ vertices and $kn$ edges (so average degree $2k$).
- Let $\{a_1<a_2<\cdots<a_t\}$ be the set of \emph{distinct} cycle lengths occurring in $G$.
- Define
\[S(G):=\sum_{i=1}^t \frac{1}{a_i}.\]

Questions.
1. Is it always true that $S(G)\ge c\log k$ for some absolute constant $c>0$ when $k$ is large?
2. Among graphs with $n$ vertices and $kn$ edges, is $S(G)$ minimized (up to $O(1)$) by a complete bipartite graph?

2) QUICK LITERATURE/CONTEXT CHECK
Web browsing is not available in this session.
From the problem text only:
- GKS84 proved $S(G)\gg \log k$ (so Question 1 is resolved).
- Liu--Montgomery proved an asymptotically sharp lower bound $S(G)\ge (\tfrac12-o(1))\log k$.
Thus the remaining open part is the minimization question (Question 2).

3) ATTACK PLAN
Proof track ideas for Question 2:
1. Compute $S(G)$ explicitly for complete bipartite graphs and compare to the sharp $(1/2)\log k$ lower bound; show it matches in order.
2. Attempt a stability/minimization argument: if $S(G)$ is close to minimal, then $G$ must be close to bipartite and close to complete bipartite.
3. Try extremal transformations that reduce $S(G)$ while preserving $n$ and $kn$ edges (e.g. edge shifting) and see whether they converge to complete bipartite graphs.

Disproof track ideas:
1. Construct a graph with $kn$ edges whose cycle-length set is unusually sparse (few lengths, all large), giving a smaller reciprocal sum than any complete bipartite graph with the same parameters.
2. Search among highly structured graphs (blow-ups of sparse graphs, random regular bipartite graphs) to see whether their cycle-length sets reduce $S(G)$.

Chosen path in this attempt: fully analyze $S(G)$ for complete bipartite graphs and relate it to $k$, providing evidence for the minimization conjecture.

4) WORK
Lemma 1 (Cycle lengths in $K_{s,t}$).
Let $K_{s,t}$ be the complete bipartite graph with parts of sizes $s$ and $t$.
Then $K_{s,t}$ contains a cycle of length $2r$ for every integer $r$ with $2\le r\le \min(s,t)$, and it contains no other cycle lengths.

Proof.
Every cycle in a bipartite graph has even length, so all cycle lengths in $K_{s,t}$ are even.

Fix $r$ with $2\le r\le \min(s,t)$. Choose distinct vertices $x_1,\dots,x_r$ in the $s$-part and distinct vertices $y_1,\dots,y_r$ in the $t$-part.
Since the graph is complete bipartite, all edges $x_i y_j$ exist.
Then
\[x_1-y_1-x_2-y_2-\cdots-x_r-y_r-x_1\]
is a cycle of length $2r$.

Conversely, any cycle in $K_{s,t}$ alternates between the two parts and therefore uses the same number of vertices from each part.
So a cycle of length $2r$ uses $r$ vertices from each part, which forces $r\le \min(s,t)$.
Thus no cycle length $2r$ with $r>\min(s,t)$ can occur. \qed

Lemma 2 (Reciprocal sum for $K_{s,t}$).
Let $m:=\min(s,t)$. Then
\[S(K_{s,t})=\sum_{r=2}^{m}\frac{1}{2r}=\frac12\left(H_m-1\right),\]
where $H_m=\sum_{r=1}^m \frac1r$ is the $m$th harmonic number.

Proof.
By Lemma 1, the set of cycle lengths in $K_{s,t}$ is exactly $\{4,6,8,\dots,2m\}$, i.e. $\{2r:2\le r\le m\}$.
Therefore
\[S(K_{s,t})=\sum_{r=2}^{m}\frac{1}{2r}=\frac12\left(\sum_{r=1}^{m}\frac1r-1\right)=\frac12(H_m-1).\qed\]

Lemma 3 (Relating $m=\min(s,t)$ to $k$ when $e(K_{s,t})=kn$).
Let $n=s+t$ and $e(K_{s,t})=st=kn$. Then $\min(s,t)\ge k$.

Proof.
Assume without loss of generality that $s\le t$, so $\min(s,t)=s$.
We have
\[k=\frac{st}{s+t}=\frac{1}{\frac1s+\frac1t}.\]
Since $\frac1s+\frac1t\ge \frac1s$, it follows that
\[k=\frac{1}{\frac1s+\frac1t}\le \frac{1}{1/s}=s.\]
Thus $s\ge k$, as claimed. \qed

Corollary 4 (A lower bound for $S(K_{s,t})$ in terms of $k$).
If $K_{s,t}$ has $n$ vertices and $kn$ edges, then
\[S(K_{s,t})\ge \frac12\left(\log k - 1\right)\]
for all $k\ge 2$.

Proof.
Let $m=\min(s,t)$. By Lemma 3, $m\ge k$.
Also $H_m\ge \int_1^{m+1}\frac{dx}{x}=\log(m+1)\ge \log m$.
Therefore by Lemma 2,
\[S(K_{s,t})=\frac12(H_m-1)\ge \frac12(\log m - 1)\ge \frac12(\log k - 1).\qed\]

FAST REALITY CHECK (tiny numeric example).
Take $K_{50,50}$: then $n=100$, edges $=2500$, so $k=25$.
Here $m=\min(s,t)=50$, so Lemma 2 gives
\[S(K_{50,50})=\tfrac12(H_{50}-1)\approx \tfrac12(4.499\ldots-1)=1.749\ldots\]
while $\tfrac12\log k=\tfrac12\log 25\approx 1.609$.
So the explicit value is consistent with Corollary 4.

5) VERIFICATION
- Lemma 1 constructs all even cycle lengths up to $2\min(s,t)$ and rules out longer cycles by counting distinct vertices per part.
- Lemma 3 is just the inequality between the harmonic mean and the minimum of two numbers.
- Corollary 4 uses only the standard bound $H_m\ge \log m$.

6) FINAL
**UNRESOLVED**

(i) Strongest fully proved partial result:
For complete bipartite graphs with $kn$ edges, I computed the exact reciprocal sum
$S(K_{s,t})=\frac12(H_{\min(s,t)}-1)$ and proved the lower bound $S(K_{s,t})\ge \frac12(\log k-1)$ (Lemmas 1--3, Corollary 4),
showing that complete bipartite graphs match the known $(1/2)\log k$ scale.

(ii) First gap:
Prove that among \emph{all} $n$-vertex graphs with $kn$ edges, the minimum of $S(G)$ is attained (up to $O(1)$) by a complete bipartite graph.
I did not produce a transformation or inequality comparing an arbitrary $G$ to $K_{s,t}$.

(iii) Top 3 next moves (concrete):
1. Formulate an extremal optimization problem for the cycle-length indicator set $\{a_i\}$ under constraints implied by $kn$ edges, and show complete bipartite graphs minimize the resulting reciprocal sum.
2. Prove a stability statement: if $S(G)\le (\tfrac12+o(1))\log k$, then $G$ must be close (in edit distance) to a complete bipartite graph.
3. Search for candidate minimizers other than complete bipartite (e.g. sparse random bipartite regular graphs) and compute/estimate their cycle-length sets to see whether $S(G)$ can drop below the complete-bipartite benchmark.

(iv) What a minimal counterexample would likely look like (structure/parameters):
A family of graphs $G$ with $kn$ edges whose cycle-length set is sparser (or shifted to longer lengths) than that of any complete bipartite graph with the same $k$,
so that $\sum 1/a_i$ falls below the complete-bipartite value by an unbounded amount as $k\to\infty$.

