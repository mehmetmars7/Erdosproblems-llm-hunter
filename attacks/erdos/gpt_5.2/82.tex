
OUTPUT
1) FORMAL RESTATEMENT
Definitions/Conventions.
- A (finite) graph $H$ is \emph{regular} if all vertices of $H$ have the same degree (within $H$).
- For $n\ge 1$, define $F(n)$ as the largest integer such that every graph on $n$ vertices contains an \emph{induced} regular subgraph on at least $F(n)$ vertices.
Equivalently,
\[F(n):=\min_{|V(G)|=n}\ \max\{|V(H)|:\ H \text{ is an induced regular subgraph of }G\}.\]

Conjecture.
Prove that
\[\frac{F(n)}{\log n}\to\infty\quad\text{as }n\to\infty.\]

2) QUICK LITERATURE/CONTEXT CHECK
Web browsing is not available in this session.
From the problem text only:
- Ramsey yields $F(n)\gg\log n$.
- Known upper bounds are around $n^{1/2+o(1)}$, improved by AKS.
- Equivalent formulation via $G(n)$ is given.
I do not re-prove the deep AKS bound here.

3) ATTACK PLAN
Proof track ideas:
1. Start from the trivial regular induced subgraphs (cliques and independent sets) and try to push beyond Ramsey.
2. Use degree-sequence pigeonhole principles: a large induced subgraph with a very narrow degree range can sometimes be made regular by deleting few vertices.
3. Work with the inverse function $G(t)$ and attempt to show $G(t)\le 2^{o(t)}$ via a probabilistic or iterative method.

Disproof track ideas:
1. Construct graphs where all induced regular subgraphs are small, aiming to show $F(n)=O(\log n)$.
The problem text indicates this is believed false.

Chosen path in this attempt: prove (i) the Ramsey lower bound in a clean form, and (ii) the equivalence between the $F$-formulation and the subexponential bound on $G$.

4) WORK
Lemma 1 (Ramsey lower bound gives $F(n)\ge c\log n$).
For every $n\ge 2$,
\[F(n)\ge \frac12\log_2 n.\]

Proof.
Let $R(k)$ be the diagonal Ramsey number: the least $m$ such that every graph on $m$ vertices contains a clique of size $k$ or an independent set of size $k$.
It is standard (and follows from the classical recursion $R(k,k)\le 2R(k-1,k)$) that
\[R(k)\le 4^{k-1}\qquad(k\ge 2).\tag{*}\]
I sketch the recursion proof: one shows $R(s,t)\le R(s-1,t)+R(s,t-1)$ by considering a pivot vertex and its red/blue neighborhoods, and then deduces
$R(k,k)\le \binom{2k-2}{k-1}<4^{k-1}$ by induction on $s+t$.

Now let $G$ be any graph on $n$ vertices, and choose $k:=\left\lfloor \frac12\log_2 n\right\rfloor$.
Then $4^{k-1}\le 4^{\frac12\log_2 n}=n$, so by (*) we have $R(k)\le n$.
Therefore $G$ contains either a clique or an independent set on $k$ vertices.
Both a clique and an independent set are regular graphs (degrees $k-1$ and $0$ respectively), and they are induced subgraphs on their vertex sets.
Hence every $n$-vertex graph contains an induced regular subgraph of size at least $k\ge \frac12\log_2 n-1$.
Absorbing the $-1$ into the constant for $n\ge 2$ yields $F(n)\ge \frac12\log_2 n$. \qed

Lemma 2 (Equivalence with the inverse function $G$).
Define $G(t)$ to be the least integer $m$ such that every graph on $m$ vertices contains an induced regular subgraph on at least $t$ vertices.
Then the following are equivalent:
(a) $\displaystyle \frac{F(n)}{\log n}\to\infty$;
(b) $\displaystyle \log G(t)=o(t)$, i.e. $G(t)\le 2^{o(t)}$.

Proof.
Write the inverse relationship (valid for all $n,t$):
\[F(n)\ge t \quad\Longleftrightarrow\quad G(t)\le n.\tag{1}\]
This is immediate from the definitions: $F(n)\ge t$ means every $n$-vertex graph has an induced regular subgraph on $\ge t$ vertices, so the minimal $m$ with this property satisfies $G(t)\le n$, and conversely.

(a)$\Rightarrow$(b).
Fix $\varepsilon>0$. By (a), for all sufficiently large $n$ one has
\[F(n)\ge \frac{1}{\varepsilon}\log n.\tag{2}\]
Let $t$ be large and set $n:=\left\lceil e^{\varepsilon t}\right\rceil$ (so $\log n\ge \varepsilon t$ and $n$ is large enough for (2)).
Then (2) gives $F(n)\ge \frac{1}{\varepsilon}\log n \ge t$, and hence by (1) we have $G(t)\le n\le e^{\varepsilon t}+1$.
Taking logs yields $\log G(t)\le \varepsilon t + o(t)$.
Since $\varepsilon>0$ was arbitrary, this implies $\log G(t)=o(t)$.

(b)$\Rightarrow$(a).
Fix $\varepsilon>0$. By (b), for all sufficiently large $t$ one has
\[\log G(t)\le \varepsilon t,\qquad\text{equivalently }G(t)\le e^{\varepsilon t}.\tag{3}\]
Let $n$ be large and set $t:=\left\lfloor \frac{1}{\varepsilon}\log n\right\rfloor$; for $n$ large this $t$ satisfies (3).
Then $G(t)\le e^{\varepsilon t}\le n$, so by (1) we get $F(n)\ge t\ge \frac{1}{\varepsilon}\log n-1$.
Since $\varepsilon>0$ is arbitrary, $F(n)/\log n\to\infty$. \qed

FAST REALITY CHECK
The problem text reports $F(5)=3$ and $F(7)=4$ and that Ramsey gives $F(n)\gg\log n$.
Lemma 1 recovers such a Ramsey-type lower bound with an explicit constant.

5) VERIFICATION
- Lemma 1 uses only the classical explicit bound $R(k)\le 4^{k-1}$ and the observation that cliques/independent sets are regular induced subgraphs.
- Lemma 2's inversion step is the key: setting $t=\lfloor \frac{1}{\varepsilon}\log n\rfloor$ and using $\log G(t)\le \varepsilon t$ forces $G(t)\le n$ and hence $F(n)\ge t$.

6) FINAL
**UNRESOLVED**

(i) Strongest fully proved partial result:
I proved a clean Ramsey lower bound $F(n)\ge \tfrac12\log_2 n$ (Lemma 1) and proved that the conjecture $F(n)/\log n\to\infty$ is equivalent to $\log G(t)=o(t)$, i.e. $G(t)\le 2^{o(t)}$ (Lemma 2).

(ii) First gap:
Prove any super-Ramsey growth of induced regular subgraphs, i.e. show $F(n)$ grows faster than $c\log n$ for every fixed $c$.
Equivalently, prove $\log G(t)=o(t)$ with an explicit rate.

(iii) Top 3 next moves (concrete):
1. Try to show that every large graph contains an induced subgraph with at most $O(1)$ distinct degrees and size $\gg \log n$, then prune it to a regular induced subgraph.
2. Investigate whether probabilistic or iterative deletion processes can force large induced subgraphs with very concentrated degree sequences.
3. For upper bounds, attempt to construct graphs where induced regular subgraphs are provably small, to understand the true scale of $F(n)$.

(iv) What a minimal counterexample would likely look like (structure/parameters):
A sequence of graphs $G_n$ on $n$ vertices such that every induced regular subgraph has size $O(\log n)$ (or at most $C\log n$ for some fixed $C$),
showing the Ramsey lower bound is essentially tight. (This is believed not to exist, per the conjecture.)
