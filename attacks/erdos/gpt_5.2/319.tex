% Attempt for Erdos Problem #319
% Following PROMPT_STRATEGY.MD
% Tools/Constraints:
% - Web browsing available? NO (restricted)
% - Computation available? YES (local Python), used
% GPT 5.2

OUTPUT
1) FORMAL RESTATEMENT
For $N\ge 1$, define $M(N)$ to be the maximum size of a set $A\subseteq\{1,\dots,N\}$ for which there exists a sign function
\[\delta:A\to\{-1,1\}\]
such that
\[\sum_{n\in A}\frac{\delta(n)}{n}=0,\tag{$\ast$}\]
but for every nonempty proper subset $A'\subsetneq A$,
\[\sum_{n\in A'}\frac{\delta(n)}{n}\ne 0.\tag{$\ast\ast$}\]
Question: estimate $M(N)$.

2) QUICK LITERATURE/CONTEXT CHECK
Browsing is not available. The problem statement sketches a route to a large lower bound assuming existence of a set $B$ with $\sum_{b\in B}1/b=1$ in a short interval; I do not reprove that existence. I focus on structural lemmas and small-$N$ computation.

3) ATTACK PLAN
Proof track:
1. Give a general “lifting lemma”: if $B$ is a minimal Egyptian-fraction representation of $1$, then $A=B\cup\{1\}$ with signs $(+,-)$ satisfies $(\ast)$ and $(\ast\ast)$.
2. Provide explicit small constructions and verify minimality directly.
3. Compute $M(N)$ exactly for small $N$ to sanity-check.

Disproof track:
1. Try to prove an upper bound by showing that minimality $(\ast\ast)$ forces large denominators or forbids dense sets.

Chosen path: establish the lifting lemma + concrete examples + small-$N$ computation.

4) WORK
Lemma 1 (From minimal Egyptian fraction to a minimal signed zero-sum).
Let $B$ be a finite set of integers $>1$ such that
\[\sum_{b\in B}\frac1b=1,\]
and no proper nonempty subset $B'\subsetneq B$ satisfies $\sum_{b\in B'}1/b=1$.
Let $A=B\cup\{1\}$ and define $\delta(1)=1$ and $\delta(b)=-1$ for $b\in B$.
Then $(A,\delta)$ satisfies $(\ast)$ and $(\ast\ast)$.

Proof.
We have
\[\sum_{n\in A}\frac{\delta(n)}{n}=1-\sum_{b\in B}\frac1b=0,\]
so $(\ast)$ holds.
Now let $A'\subsetneq A$ be nonempty.
If $1\notin A'$, then all signs in $A'$ are $-1$ and $\sum_{n\in A'}\delta(n)/n<0$, so it is not $0$.
If $1\in A'$, write $A'=\{1\}\cup B'$ with $B'\subseteq B$.
Then
\[\sum_{n\in A'}\frac{\delta(n)}{n}=1-\sum_{b\in B'}\frac1b.\]
If $B'=B$, then $A'=A$, excluded by assumption.
If $B'\subsetneq B$, then by hypothesis $\sum_{b\in B'}1/b\ne 1$, so the displayed sum is nonzero.
Thus $(\ast\ast)$ holds. \qed

Lemma 2 (Explicit example for all $N\ge 6$).
Let $A=\{1,2,3,6\}$ and define $\delta(1)=1$ and $\delta(n)=-1$ for $n\in\{2,3,6\}$.
Then $(A,\delta)$ satisfies $(\ast)$ and $(\ast\ast)$.

Proof.
We compute
\[\sum_{n\in A}\frac{\delta(n)}{n}=1-\frac12-\frac13-\frac16=0.\]
Now check proper nonempty subsets $A'\subsetneq A$.
If $1\notin A'$, the sum is negative, hence nonzero.
If $1\in A'$, then $A'=\{1\}\cup B'$ with $B'\subsetneq\{2,3,6\}$.
The possible sums $\sum_{b\in B'}1/b$ are in $\{0,1/2,1/3,1/6,5/6,2/3\}$, none equal to $1$.
Therefore $1-\sum_{b\in B'}1/b\ne 0$ for all such $B'$, giving $(\ast\ast)$. \qed

FAST REALITY CHECK (computation).
Exhaustive search for $N\le 12$ (over all $A\subseteq\{1,\dots,N\}$ and all signings $\delta$) gives:
- $M(N)=0$ for $2\le N\le 5$ (no examples).
- $M(N)=4$ for $6\le N\le 11$, achieved by $(A,\delta)=(\{1,2,3,6\},(1,-1,-1,-1))$ as in Lemma 2.
- $M(12)=6$, achieved for example by $A=\{1,2,3,4,6,12\}$ with signs $(1,-1,-1,-1,1,-1)$.

5) VERIFICATION
- Lemma 1 reduces $(\ast\ast)$ to the minimality of an Egyptian-fraction decomposition of $1$ inside $B$.
- Lemma 2 checks minimality by an explicit finite enumeration of subset sums.
- The computation is small-$N$ only; large-$N$ behavior is the core difficulty.

6) FINAL
**UNRESOLVED**

(i) Strongest fully proved partial result:
Any minimal Egyptian-fraction decomposition of $1$ yields a minimal signed zero-sum set by adjoining $1$ with opposite sign (Lemma 1). In particular, $\{1,2,3,6\}$ gives a size-$4$ example for all $N\ge 6$ (Lemma 2). Computation shows $M(12)=6$ and lists small-$N$ values.

(ii) First gap:
Construct large sets $A\subseteq[1,N]$ and signs $\delta$ satisfying $(\ast)$ and the strong minimality condition $(\ast\ast)$, with $|A|$ growing linearly in $N$ (or prove that this is impossible).

(iii) Top 3 next moves:
1. Search for large \emph{minimal} Egyptian fraction decompositions of $1$ inside short intervals $[cN,N]$, since Lemma 1 would immediately convert these into large $A$.
2. Prove structural necessary conditions for minimality $(\ast\ast)$ (e.g. constraints on gcds, divisibility patterns, or spacing of denominators) that could yield upper bounds on $M(N)$.
3. Compute $M(N)$ for larger $N$ with optimized meet-in-the-middle to see whether growth appears linear or much smaller.

(iv) Minimal counterexample structure (if linear growth were false):
For all large $N$, every signed zero-sum set $A\subseteq[1,N]$ would contain a nontrivial proper zero-sum subconfiguration, preventing $(\ast\ast)$; equivalently, there would be no large minimal Egyptian fraction decompositions of $1$ inside $[1,N]$.

