% Attempt for Erdos Problem #10
% Following PROMPT_STRATEGY.MD
% Tools/Constraints:
% - Web browsing available? NO (restricted)
% - Computation available? YES (local Python), not used
% GPT 5.2

OUTPUT
1) FORMAL RESTATEMENT
Does there exist an integer $k\ge 0$ such that every integer $n$ can be written as
\[n=p+2^{a_1}+\cdots+2^{a_t},\qquad 0\le t\le k,\ a_i\ge 0,\ p\text{ prime}?\]

2) QUICK LITERATURE/CONTEXT CHECK
Browsing is not available. I only record what the problem statement itself claims.
I have not verified those results here.

3) ATTACK PLAN
Proof track:
1. Sieve to show large density of representable integers for a fixed $k$.
2. Combine parity obstructions with small-modulus obstructions to show any fixed $k$ fails.

Disproof track:
1. Exhibit infinite families of integers not representable with small $k$.
2. Search for local congruence obstructions mod $m$ for all $k$.

Chosen path: prove elementary necessary conditions and a baseline representation with variable $k$.

4) WORK
Lemma 1 (No fixed $k=0$).
There is no $k=0$ solution: not every integer is prime.

Proof.
If $k=0$, the representation is $n=p$. Since composite integers exist, not every $n$ is representable. \qed

Lemma 2 (No fixed $k=1$).
There is no $k=1$ solution: not every integer is of the form $p+2^a$.

Proof.
Consider $n=22$. For $a\ge 0$ with $2^a\le 22$, we have
\[
22-2^a\in\{21,20,18,14,6\},
\]
all of which are composite. Hence there is no prime $p$ and exponent $a$ with $22=p+2^a$, so $k=1$ cannot work. \qed

Lemma 3 (Variable $k$ always suffices).
Every integer $n\ge 2$ can be written as $n=2+\sum_{j=1}^t 2^{a_j}$ for some $t$ (depending on $n$).

Proof.
Write $n-2$ in binary as a sum of distinct powers of $2$. Then
$n=2+\sum 2^{a_j}$, with $p=2$ prime. \qed

5) VERIFICATION
- Lemma 2 is an explicit finite check for $n=22$.
- Lemma 3 is the binary expansion identity.

6) FINAL
**UNRESOLVED**

(i) Strongest fully proved partial result:
No fixed $k\le 1$ can work (Lemmas 1--2), while a variable $k$ (depending on $n$) always works via binary expansion.

(ii) First gap:
No argument here settles whether any fixed $k\ge 2$ suffices for all integers.

(iii) Top 3 next moves:
1. Search for congruence obstructions mod $m$ that rule out representation with a fixed $k$.
2. Use sieve methods to estimate the density of representable integers for a fixed $k$.
3. Attempt computational exploration for small $k$ to identify systematic obstructions.

(iv) Minimal counterexample structure (if the conjecture were false):
An infinite set of integers, likely in specific residue classes modulo large powers of 2 or small odd moduli,
that avoid all representations with at most $k$ powers of 2.
