
OUTPUT
1) FORMAL RESTATEMENT
Definitions/Conventions.
- Graphs are simple, undirected, and may be infinite.
- A graph is \emph{bipartite} iff it contains no odd cycle.
- For a finite graph $H$, define its \emph{edge-bipartization number}
\[\mathrm{ebip}(H):=\min\{|F|: F\subseteq E(H)\ \text{and}\ H-F\ \text{is bipartite}\}.\]

Question.
Given any function $f:\mathbb N\to\mathbb N$ with $f(n)\to\infty$, does there exist a graph $G$ with infinite chromatic number (i.e. $\chi(G)$ is not finite) such that
for every $n\in\mathbb N$ and every finite subgraph $H\subseteq G$ with $|V(H)|=n$, one has
\[\mathrm{ebip}(H)\le f(n)?\]

2) QUICK LITERATURE/CONTEXT CHECK
Web browsing is not available in this session.
From the problem text only:
- The conjecture is attributed to Erd\H{o}s--Hajnal--Szemer\'{e}di.
- R\"{o}dl proved existence for $f(n)=\varepsilon n$ (and for hypergraphs).
- The problem is open even for $f(n)=\sqrt n$.
- The analogous statement fails for graphs of chromatic number $\aleph_1$.
I do not re-derive these results here.

3) ATTACK PLAN
Proof track ideas:
1. Reinterpret $\mathrm{ebip}(H)$ as a minimum edge set hitting all odd cycles (proved below), then try to build an infinite-chromatic graph where every small subgraph has a tiny odd-cycle edge transversal.
2. Attempt to modify probabilistic high-chromatic constructions (Erd\H{o}s high-girth graphs) so that small subgraphs have very few odd cycles, allowing deletion of $f(n)$ edges.

Disproof track ideas:
1. Try to prove a universal lower bound: in any infinite-chromatic graph, some $n$-vertex subgraph must have $\mathrm{ebip}(H)\ge g(n)$ for a function $g(n)\to\infty$ faster than some candidate $f(n)$.

Chosen path in this attempt: prove two problem-specific lemmas clarifying what ``delete $f(n)$ edges to make bipartite'' means combinatorially, and what it forbids in $n$-vertex subgraphs.

4) WORK
Lemma 1 (Edge-bipartization equals odd-cycle edge transversal).
For a finite graph $H$, $\mathrm{ebip}(H)$ equals the minimum size of a set of edges $F\subseteq E(H)$ that intersects every odd cycle in $H$.

Proof.
First, if $F\subseteq E(H)$ is such that $H-F$ is bipartite, then $H-F$ contains no odd cycle.
Therefore every odd cycle $C$ of $H$ must use at least one deleted edge, i.e. $E(C)\cap F\ne\emptyset$.
So any feasible edge-deletion set is an odd-cycle edge transversal.

Conversely, suppose $F\subseteq E(H)$ intersects every odd cycle of $H$. Then $H-F$ has no odd cycle.
It is standard that a graph is bipartite iff it has no odd cycle (e.g. by BFS 2-coloring and checking parity along cycles).
Thus $H-F$ is bipartite, so $F$ is feasible for $\mathrm{ebip}(H)$.

Taking minima over $|F|$ in the two equivalent formulations shows equality. \qed

Lemma 2 (Packing obstruction: many edge-disjoint odd cycles force large $\mathrm{ebip}$).
Let $H$ be a finite graph, and suppose $H$ contains $t$ pairwise edge-disjoint odd cycles.
Then $\mathrm{ebip}(H)\ge t$.

Proof.
Let $C_1,\dots,C_t$ be edge-disjoint odd cycles.
By Lemma 1, any set of edges $F$ whose deletion makes $H$ bipartite must intersect every odd cycle, hence must intersect each $C_i$ in at least one edge.
Since the $C_i$ are edge-disjoint, $F$ must contain at least one distinct edge from each cycle, so $|F|\ge t$.
Therefore $\mathrm{ebip}(H)\ge t$. \qed

FAST REALITY CHECK
- For a triangle $K_3$, $\mathrm{ebip}(K_3)=1$ (delete any one edge).
- For a disjoint union of $t$ triangles, $\mathrm{ebip}=t$ by Lemma 2 (the $t$ triangles are edge-disjoint odd cycles).
So controlling $\mathrm{ebip}(H)$ on all $n$-vertex subgraphs is a strong restriction: it forbids packing many edge-disjoint odd cycles into small vertex sets.

5) VERIFICATION
- Lemma 1 uses only the characterization ``bipartite iff no odd cycle'' and is logically tight in both directions.
- Lemma 2 is a direct consequence of Lemma 1: any transversal must hit each edge-disjoint cycle.

6) FINAL
**UNRESOLVED**

(i) Strongest fully proved partial result:
The local requirement ``delete at most $f(n)$ edges to make bipartite'' is exactly the statement that every $n$-vertex subgraph has an odd-cycle edge transversal of size $\le f(n)$ (Lemma 1),
and in particular no $n$-vertex subgraph can contain more than $f(n)$ edge-disjoint odd cycles (Lemma 2).

(ii) First gap:
Construct (or rule out) an infinite-chromatic graph $G$ such that every $n$-vertex subgraph has $\mathrm{ebip}\le f(n)$ for a slowly growing $f$ (e.g. $f(n)=\sqrt n$).
The lemmas above give obstructions, but I did not find a construction meeting them.

(iii) Top 3 next moves (concrete):
1. Search for a construction strategy that guarantees ``few edge-disjoint odd cycles in every small subgraph'' while allowing global chromatic number to diverge (e.g. by gluing gadgets along sparse cutsets).
2. Attempt a probabilistic existence proof tailored to the local transversal condition: choose a random sparse graph and bound, for each $n$, the probability that some $n$-vertex subgraph contains $>f(n)$ edge-disjoint odd cycles.
3. Try to prove a lower bound on $\mathrm{ebip}(H)$ in terms of $\chi(H)$ that holds uniformly for all finite subgraphs $H$; any such bound would convert the problem into an explicit inequality relating $f(n)$ and possible chromatic growth.

(iv) What a minimal counterexample would likely look like (structure/parameters):
If the answer is negative for some slowly growing $f$, then for every infinite-chromatic graph $G$ there should exist arbitrarily large $n$ and an $n$-vertex subgraph $H\subseteq G$ containing
many edge-disjoint odd cycles (or another certificate forcing large odd-cycle transversals), with $\mathrm{ebip}(H)>f(n)$.

