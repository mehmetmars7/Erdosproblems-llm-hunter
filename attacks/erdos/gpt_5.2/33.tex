% Attempt for Erdos Problem #33
% Following PROMPT_STRATEGY.MD
% Tools/Constraints:
% - Web browsing available? NO (restricted)
% - Computation available? YES (local Python), not used
% GPT 5.2

OUTPUT
1) FORMAL RESTATEMENT
Let $A\subseteq\mathbb N$ be such that every sufficiently large integer $m$ can be written as
\[m=n^2+a\quad\text{for some integer $n\ge 0$ and some $a\in A$.}\]
Equivalently, $A$ is an additive complement of the squares.
Question: what is the smallest possible value of
\[\limsup_{N\to\infty} \frac{|A\cap[1,N]|}{\sqrt N}?\]
And must one have $\liminf_{N\to\infty} |A\cap[1,N]|/\sqrt N > 1$?

2) QUICK LITERATURE/CONTEXT CHECK
Browsing is not available. I only record what the problem statement itself claims.
I have not verified those results here.

3) ATTACK PLAN
Proof track:
1. Use covering arguments: each $a\in A$ covers the translate $a+\{0,1,4,\dots\}$.
2. Apply density/packing bounds to force $|A\cap[1,N]|$ to be at least a constant times $\sqrt N$.

Disproof track:
1. Construct $A$ explicitly by choosing residues in intervals so that $A+\{n^2\}$ covers all large integers.

Chosen path: prove a simple unconditional lower bound $\limsup |A\cap[1,N]|/\sqrt N \ge 1$.

4) WORK
Lemma 1 (Basic counting lower bound).
Assume that for all sufficiently large $M$, every integer in $[1,M]$ is representable as $n^2+a$ with $a\in A$.
Then for all sufficiently large $M$,
\[|A\cap[1,M]|\ge \frac{M}{\lfloor\sqrt M\rfloor+1}.
\]
In particular,
\[\limsup_{M\to\infty} \frac{|A\cap[1,M]|}{\sqrt M} \ge 1.
\]

Proof.
Fix $M$ large. For each $a\in A\cap[1,M]$, the numbers in $[1,M]$ representable as $n^2+a$ are those with
$n^2\le M-a$, so there are at most $\lfloor\sqrt M\rfloor+1$ choices of $n$ (including $n=0$).
Hence each $a$ covers at most $\lfloor\sqrt M\rfloor+1$ integers in $[1,M]$.
By the union bound, the total number of covered integers in $[1,M]$ is at most $|A\cap[1,M]|(\lfloor\sqrt M\rfloor+1)$.
Since by hypothesis all $M$ integers are covered for large $M$, we get
$|A\cap[1,M]|(\lfloor\sqrt M\rfloor+1)\ge M$.
Dividing by $\sqrt M$ and taking limsup yields the second claim. \qed

5) VERIFICATION
- Lemma 1 is a coarse counting argument; it ignores overlaps but remains valid.
- The hypothesis was strengthened to covering all integers up to $M$; the original problem only requires covering all sufficiently large integers,
  but the limsup lower bound still follows by applying the argument on large intervals.

6) FINAL
**UNRESOLVED**

(i) Strongest fully proved partial result:
Any additive complement of the squares must satisfy
$\limsup_{N\to\infty} |A\cap[1,N]|/\sqrt N\ge 1$ (Lemma 1).

(ii) First gap:
No method here improves the constant $1$ to any value $>1$ such as $4/\pi$, nor addresses minimization of the limsup.

(iii) Top 3 next moves:
1. Refine Lemma 1 by exploiting overlaps between square-translates to improve the constant.
2. Study known constructions (as mentioned in the problem statement) and compute their limsup constants.
3. Investigate Fourier/geometry-of-numbers approaches to additive complements of polynomial sets.

(iv) Minimal counterexample structure (if $\liminf >1$ were false):
A complement set $A$ with $|A\cap[1,N]|\le (1+o(1))\sqrt N$ along a sequence of $N$, yet still covering all large integers by square shifts.
