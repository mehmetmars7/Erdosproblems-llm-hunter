% Attempt for Erdos Problem #78
% Following PROMPT_STRATEGY.MD
% Tools/Constraints:
% - Web browsing available? NO (restricted)
% - Computation available? YES (local), not needed
% GPT 5.2

OUTPUT
1) FORMAL RESTATEMENT
Definitions/Conventions.
- $R(k)$ is the diagonal Ramsey number.
- For a finite graph $G$, let $\omega(G)$ be its clique number and $\alpha(G)$ its independence number.

Goal (constructive lower bound).
Give an explicit (deterministic, efficiently describable) family of graphs $G_k$ on $n_k$ vertices such that
\[\omega(G_k)<k\quad\text{and}\quad \alpha(G_k)<k,\]
with $n_k\ge C^k$ for some constant $C>1$.
Equivalently, prove constructively that $R(k)>C^k$ for some $C>1$.

2) QUICK LITERATURE/CONTEXT CHECK
Web browsing is not available in this session.
The problem text notes nonconstructive probabilistic lower bounds and partial constructive results (Cohen, Li).
I do not reproduce those deep constructions here; I focus on elementary equivalences and a standard amplification operation that any explicit construction could use.

3) ATTACK PLAN
Proof track ideas:
1. Re-express the problem in terms of explicit Ramsey graphs: graphs with both $\omega(G)$ and $\alpha(G)$ at most $c\log n$.
2. Identify closure properties: how to combine smaller Ramsey graphs to create larger ones while controlling $\omega$ and $\alpha$ (lexicographic products).
3. Search for an explicit base graph with $\omega,\alpha=O(\log n)$; this is the core open challenge.

Disproof track ideas:
1. Disprove by showing that no explicit construction can beat subexponential in $k$ (unlikely and not formalized by current information).

Chosen path in this attempt: prove two problem-specific lemmas (equivalence and amplification) that make the objective precise and modular.

4) WORK
Lemma 1 (Ramsey-number equivalence with clique/independence avoidance).
Let $k\ge 2$ and $n\ge 1$.
Then $R(k)>n$ if and only if there exists a (simple) graph $G$ on $n$ vertices such that
\[\omega(G)<k\quad\text{and}\quad \alpha(G)<k.\]

Proof.
($\Rightarrow$) If $R(k)>n$, then by definition there exists a 2-coloring of the edges of $K_n$ with no monochromatic $K_k$.
Let $G$ be the graph whose edges are the red edges.
If $G$ contained a clique of size $k$, then those edges would form a red $K_k$, contradiction.
Thus $\omega(G)<k$.
If $G$ contained an independent set of size $k$, then all edges among that vertex set are blue, giving a blue $K_k$, contradiction.
Thus $\alpha(G)<k$.

($\Leftarrow$) Conversely, if there exists a graph $G$ on $n$ vertices with $\omega(G)<k$ and $\alpha(G)<k$, color edges of $K_n$ red if they are edges of $G$ and blue otherwise.
Then there is no red $K_k$ (since $\omega(G)<k$) and no blue $K_k$ (since a blue $K_k$ would be an independent set of size $k$ in $G$).
Thus this coloring avoids monochromatic $K_k$, so $R(k)>n$. \qed

Lemma 2 (Lexicographic product amplifies Ramsey graphs).
For graphs $G$ and $H$, define the lexicographic product $G\circ H$ as the graph on $V(G)\times V(H)$ where
$(g,h)$ is adjacent to $(g',h')$ iff either $gg'\in E(G)$, or $g=g'$ and $hh'\in E(H)$.
Then
\[\omega(G\circ H)=\omega(G)\,\omega(H)\qquad\text{and}\qquad \alpha(G\circ H)=\alpha(G)\,\alpha(H).\]
In particular, if $\omega(G)<k$ and $\alpha(G)<k$ and $\omega(H)<\ell$ and $\alpha(H)<\ell$, then
$\omega(G\circ H)<k\ell$ and $\alpha(G\circ H)<k\ell$, while $|V(G\circ H)|=|V(G)|\,|V(H)|$.

Proof.
Let $C\subseteq V(G)\times V(H)$ be a clique in $G\circ H$.
Project $C$ to $V(G)$: let $S:=\{g\in V(G): \exists h\ (g,h)\in C\}$.
If $g\ne g'$ are in $S$, then for any $(g,h),(g',h')\in C$ we must have $(g,h)$ adjacent to $(g',h')$, which can only happen if $gg'\in E(G)$ (since $g\ne g'$).
Thus $S$ is a clique in $G$, so $|S|\le \omega(G)$.
For each fixed $g\in S$, the fiber $C_g:=\{h:(g,h)\in C\}$ must be a clique in $H$ (because adjacency inside the same $g$ fiber is exactly adjacency in $H$).
So $|C_g|\le \omega(H)$.
Therefore
\[|C|=\sum_{g\in S} |C_g|\le |S|\cdot \omega(H)\le \omega(G)\omega(H),\]
showing $\omega(G\circ H)\le \omega(G)\omega(H)$.
Equality is achieved by taking a clique $S$ in $G$ of size $\omega(G)$ and, for each $g\in S$, taking a clique in $H$ of size $\omega(H)$; their product set is a clique in $G\circ H$.
Hence $\omega(G\circ H)=\omega(G)\omega(H)$.

For independence sets, let $I\subseteq V(G)\times V(H)$ be independent.
If $g\ne g'$ are both in the projection $S:=\{g:\exists h\ (g,h)\in I\}$, then for any $(g,h)\in I$ and $(g',h')\in I$ we must have no adjacency, which forces $gg'\notin E(G)$.
Thus $S$ is an independent set in $G$, so $|S|\le \alpha(G)$.
Within each fiber $g$, the set $I_g:=\{h:(g,h)\in I\}$ must be independent in $H$, so $|I_g|\le \alpha(H)$.
Thus $|I|\le \alpha(G)\alpha(H)$, and equality is achieved by taking $S$ an independent set of size $\alpha(G)$ and in each fiber an independent set of size $\alpha(H)$.
Therefore $\alpha(G\circ H)=\alpha(G)\alpha(H)$.
The final statement follows immediately. \qed

FAST REALITY CHECK
Lemma 1 explains the equivalence in the problem statement:
an explicit lower bound $R(k)>C^k$ is equivalent to an explicit family of graphs on $n$ vertices with $\omega,\alpha = O(\log n)$.
Lemma 2 shows a standard way to combine constructions: if one can explicitly build a graph on $N$ vertices with $\omega,\alpha \le t$,
then the $m$-fold lexicographic power has $N^m$ vertices and $\omega,\alpha \le t^m$.
This modularizes the task: the bottleneck is producing a base graph with $t$ comparable to $O(\log N)$.

5) VERIFICATION
- Lemma 1 is a direct translation between 2-colorings and graphs/complements.
- Lemma 2 is checked by projecting cliques/independent sets to $G$ and fibers to $H$, and showing both upper and lower bounds match.

6) FINAL
**UNRESOLVED**

(i) Strongest fully proved partial result:
I proved the exact equivalence between constructive Ramsey lower bounds and explicit graphs with simultaneously small clique and independence numbers (Lemma 1),
and proved an amplification operation (lexicographic product) that multiplies both clique and independence numbers while multiplying vertex counts (Lemma 2).

(ii) First gap:
Provide an explicit graph family on $n$ vertices with $\omega(G),\alpha(G)\le c\log n$ for some absolute $c>0$.
That is exactly the missing ingredient needed (via Lemma 1) to get $R(k)>C^k$ constructively.

(iii) Top 3 next moves (concrete):
1. Pick a candidate explicit pseudorandom graph family (e.g. based on finite fields) and attempt to prove $\omega,\alpha=O(\log n)$ using only accessible exponential-sum bounds.
2. Use Lemma 2 to amplify any explicit construction that achieves $\omega,\alpha\le (\log n)^C$ and see whether iterating/bootstrapping can reduce the exponent $C$ (it typically does not, but may yield optimized constants).
3. Recast the goal as constructing explicit 2-colorings of $K_n$ with no monochromatic clique/independent set larger than $c\log n$, and explore derandomization tools (extractors/dispersers) that could certify this property.

(iv) What a minimal counterexample would likely look like (structure/parameters):
A complexity-theoretic barrier showing that any explicit family of graphs describable within a given resource bound must contain a clique or independent set of size $\ge (\log n)^{1+\delta}$,
which would rule out achieving $O(\log n)$ and hence would prevent any constructive $R(k)>C^k$ result. (No such barrier is established here.)

