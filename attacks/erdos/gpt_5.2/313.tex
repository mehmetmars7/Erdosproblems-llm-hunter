% Attempt for Erdos Problem #313
% Following PROMPT_STRATEGY.MD
% Tools/Constraints:
% - Web browsing available? NO (restricted)
% - Computation available? YES (local Python), used
% GPT 5.2

OUTPUT
1) FORMAL RESTATEMENT
Question: Are there infinitely many integers $m\ge 2$ for which there exist distinct primes $p_1<\cdots<p_k$ satisfying
\[\sum_{i=1}^k \frac1{p_i}=1-\frac1m?\tag{$\ast$}\]

Equivalently, are there infinitely many squarefree integers $m$ such that
\[\sum_{p\mid m}\frac1p + \frac1m = 1?\tag{$\ast\ast$}\]

2) QUICK LITERATURE/CONTEXT CHECK
Browsing is not available. The statement asserts that $m$ must equal the product of the primes appearing, and mentions “8 known” primary pseudoperfect numbers; I do not use the OEIS claim, but I verify the “$m=\prod p_i$” implication and compute examples up to $200{,}000$.

3) ATTACK PLAN
Proof track:
1. Prove the structural fact that $(\ast)$ forces $m=\prod p_i$.
2. Rephrase the problem as the infinitude of integers $m$ satisfying $(\ast\ast)$ (primary pseudoperfect numbers).
3. Compute all such $m$ up to a moderate bound to sanity-check known examples and see growth.

Disproof track:
1. Look for congruence obstructions or growth constraints that could force finiteness.

Chosen path: rigorous reformulation + computation; infinitude remains open here.

4) WORK
Lemma 1 (Sum of reciprocals of distinct primes is reduced with squarefree denominator).
Let $p_1,\dots,p_k$ be distinct primes and set $P=\prod_{i=1}^k p_i$.
Then
\[\sum_{i=1}^k \frac1{p_i} = \frac{A}{P}\]
for an integer $A$ with $\gcd(A,P)=1$.
Consequently,
\[1-\sum_{i=1}^k \frac1{p_i} = \frac{P-A}{P}\]
is also reduced (since $\gcd(P-A,P)=\gcd(A,P)=1$).

Proof.
As in Problem \#307’s Lemma 1: write $\sum 1/p_i = (1/P)\sum (P/p_i)$.
For a fixed $p_j$, reduce the numerator modulo $p_j$; all terms except $P/p_j$ are divisible by $p_j$, while $P/p_j$ is not, so the numerator is nonzero mod $p_j$.
Thus no $p_j$ divides the numerator, giving $\gcd(A,P)=1$.
Finally $\gcd(P-A,P)=\gcd(A,P)=1$ by the identity $\gcd(P-A,P)=\gcd(A,P)$. \qed

Proposition 2 (In any solution, $m=\prod p_i$).
If distinct primes $p_1<\cdots<p_k$ and an integer $m\ge 2$ satisfy $(\ast)$, then
\[m=p_1p_2\cdots p_k.\]
In particular, for a given $m$ there is at most one such solution.

Proof.
Let $P=\prod_{i=1}^k p_i$ and write $\sum_{i=1}^k 1/p_i = A/P$ in lowest terms using Lemma 1.
Then $(\ast)$ becomes
\[\frac{A}{P}=1-\frac1m=\frac{m-1}{m},\]
so
\[\frac{1}{m}=1-\frac{A}{P}=\frac{P-A}{P}.\]
By Lemma 1, $(P-A)/P$ is already in lowest terms with denominator $P$, whereas $1/m$ in lowest terms has denominator $m$.
Therefore $m=P$. \qed

FAST REALITY CHECK (computation).
I searched for integers $m\le 200{,}000$ satisfying $(\ast\ast)$ by factoring squarefree $m$ and checking
\[\sum_{p\mid m}\frac1p+\frac1m=1.\]
Found exactly five:
\[m=2,\ 6,\ 42,\ 1806,\ 47058.\]
These include the examples $6$ and $42$ in the problem statement.

5) VERIFICATION
- Lemma 1’s coprimality check is done prime-by-prime and prevents cancellation in the denominator.
- Proposition 2 reduces $(\ast)$ to an equality of reduced fractions; the conclusion $m=P$ is forced.
- The computation is finite-range only and does not address infinitude.

6) FINAL
**UNRESOLVED**

(i) Strongest fully proved partial result:
Any solution to $(\ast)$ necessarily has $m=\prod_i p_i$ (Proposition 2), hence solutions are in bijection with integers $m$ satisfying $(\ast\ast)$. Computation finds $m=2,6,42,1806,47058$ up to $200{,}000$.

(ii) First gap:
Prove infinitude (or finiteness) of integers $m$ satisfying $\sum_{p\mid m}1/p+1/m=1$.

(iii) Top 3 next moves:
1. Derive recursive constructions: given one solution $m$, try to generate a larger $m'$ satisfying $(\ast\ast)$ by adjoining a new prime and solving the resulting equation.
2. Prove constraints on prime factorization of $m$ from $(\ast\ast)$ (e.g. bounds on the largest prime factor or on the number of prime factors).
3. Push computation to much larger ranges to see whether solutions become extremely sparse and to guess growth rates.

(iv) Minimal counterexample structure (if infinitude failed):
There would exist a largest primary pseudoperfect number $m_{\max}$; equivalently, beyond some point no squarefree $m$ would satisfy $\sum_{p\mid m}1/p = 1-1/m$, suggesting strong and unexplained rigidity in the Diophantine condition $(\ast\ast)$.

