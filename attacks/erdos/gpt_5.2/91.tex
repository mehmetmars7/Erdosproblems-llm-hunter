
OUTPUT
1) FORMAL RESTATEMENT
Definitions/Conventions.
For a finite set $A\subset\mathbb R^2$, define the number of distinct distances
\[D(A):=\big|\{|x-y|: x,y\in A,\ x\ne y\}\big|.\]
Let
\[D_{\min}(n):=\min\{D(A): A\subset\mathbb R^2,\ |A|=n\}.\]
Say that $A$ is an \emph{extremizer} if $|A|=n$ and $D(A)=D_{\min}(n)$.
Two sets are \emph{similar} if one can be obtained from the other by a similarity transformation (composition of isometry and scaling).

Question.
For large $n$, prove that there exist at least two non-similar extremizers $A$ for $D_{\min}(n)$.

2) QUICK LITERATURE/CONTEXT CHECK
Web browsing is not available in this session.
I will not use external results about $D_{\min}(n)$ beyond the small-$n$ examples stated in the problem text.

3) ATTACK PLAN
Proof track ideas:
1. Understand small-$n$ extremizers and extract a mechanism for nonuniqueness (e.g. different combinatorial distance graphs).
2. Attempt to show that for large $n$, extremizers must be ``highly structured'' but not uniquely so, by comparing regular polygons vs. other configurations.

Disproof track ideas:
1. Try to show uniqueness of extremizers for infinitely many $n$; the problem statement suggests this is false for large $n$.

Chosen path in this attempt: establish the $n=4$ nonuniqueness rigorously as a model case by proving (i) $D_{\min}(4)=2$ and (ii) there are at least two non-similar extremizers achieving it.

4) WORK
Lemma 1 (Four points determine at least two distinct distances).
For any set $A\subset\mathbb R^2$ with $|A|=4$, one has $D(A)\ge 2$.

Proof.
Assume for contradiction that $D(A)=1$.
Then all pairwise distances between distinct points in $A$ are equal to some value $d>0$; i.e. $A$ is a set of four pairwise equidistant points.
Pick three points $x,y,z\in A$.
Since $|x-y|=|y-z|=|z-x|=d$, the triangle $xyz$ is equilateral.
Let $O$ be the circumcenter of triangle $xyz$; in an equilateral triangle, the circumcenter is unique and satisfies $|O-x|=|O-y|=|O-z|=d/\sqrt3$.

Let $w$ be the fourth point in $A$.
Because $|w-x|=|w-y|=d$, the point $w$ lies on the perpendicular bisector of segment $xy$.
Similarly, $w$ lies on the perpendicular bisector of $xz$.
The intersection of these two perpendicular bisectors is unique and equals the circumcenter $O$.
Therefore $w=O$.
But then $|w-x|=|O-x|=d/\sqrt3\ne d$, contradiction.
Hence $D(A)\ge 2$. \qed

Lemma 2 (Two non-similar extremizers for $n=4$).
There exist two non-similar 4-point sets $A,B\subset\mathbb R^2$ with $D(A)=D(B)=2$.

Proof.
Let $A$ be the vertices of a square of side length $1$. Then the pairwise distances among vertices are either $1$ (sides) or $\sqrt2$ (diagonals), so $D(A)=2$.

Let $B$ be the vertices of a rhombus with side length $1$ and acute angle $60^\circ$ (equivalently, two equilateral triangles sharing an edge).
Then distances among its vertices are either $1$ (edges) or $\sqrt3$ (the longer diagonal), so again $D(B)=2$.

By Lemma 1, any 4-point set has at least 2 distinct distances, so $D_{\min}(4)=2$, and both $A$ and $B$ are extremizers.
They are not similar because in a square the ratio of diagonal to side is $\sqrt2$, while in the $60^\circ$ rhombus the longer diagonal to side ratio is $\sqrt3$; similarity preserves this ratio.
Thus $A$ and $B$ are non-similar extremizers. \qed

FAST REALITY CHECK
The problem statement says: $n=3$ has a unique extremizer (equilateral), $n=4$ has at least two, $n=5$ is unique (regular pentagon).
Lemma 2 confirms the $n=4$ nonuniqueness with a complete proof.

5) VERIFICATION
- Lemma 1's key fact is that the intersection of perpendicular bisectors of $xy$ and $xz$ is unique; this is standard Euclidean geometry.
- Lemma 2 checks the distance sets explicitly and uses diagonal/side ratio as a similarity invariant.

6) FINAL
**UNRESOLVED**

(i) Strongest fully proved partial result:
I proved that $D_{\min}(4)=2$ and that there are at least two non-similar extremizers achieving it (Lemmas 1--2).

(ii) First gap:
Extend beyond small $n$ and prove that for all sufficiently large $n$ there are at least two non-similar extremizers for $D_{\min}(n)$.

(iii) Top 3 next moves (concrete):
1. For each $n$, characterize the distance graph structure of extremizers (which distances occur with which multiplicities) and look for incompatible structures that can both realize $D_{\min}(n)$.
2. For $6\le n\le 9$ (where Erd\H{o}s claimed nonuniqueness), write down explicit non-similar candidate extremizers and verify they achieve the same minimal number of distances.
3. Investigate whether known near-extremal constructions for the distinct distances problem (e.g. subsets of grids, regular polygons) can be perturbed in two inequivalent ways without increasing $D(A)$.

(iv) What a minimal counterexample would likely look like (structure/parameters):
An infinite sequence $n_i\to\infty$ for which the extremizer is unique up to similarity, i.e. any $A$ with $|A|=n_i$ and $D(A)=D_{\min}(n_i)$ must be similar to a single canonical configuration.

