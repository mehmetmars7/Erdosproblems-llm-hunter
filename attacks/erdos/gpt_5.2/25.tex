% Attempt for Erdos Problem #25
% Following PROMPT_STRATEGY.MD
% Tools/Constraints:
% - Web browsing available? NO (restricted)
% - Computation available? YES (local Python), not used
% GPT 5.2

OUTPUT
1) FORMAL RESTATEMENT
Let $n_1<n_2<\cdots$ be positive integers and for each $i$ fix a residue class $a_i\pmod{n_i}$.
Define
\[A:=\{n\in\mathbb N: \forall i,\ (n<n_i)\ \text{or}\ (n\not\equiv a_i\pmod{n_i})\}.\]
Equivalently, for each $i$ we forbid the congruence class $a_i\pmod{n_i}$ for all integers $n\ge n_i$.
Question: must the logarithmic density
\[\delta(A):=\lim_{N\to\infty}\frac{1}{\log N}\sum_{\substack{n\le N\\n\in A}}\frac{1}{n}
\]
exist (i.e. does the limit exist, not just limsup/liminf)?

2) QUICK LITERATURE/CONTEXT CHECK
Browsing is not available.

3) ATTACK PLAN
Proof track:
1. Identify conditions on the moduli $(n_i)$ under which the forbidden classes behave like a classical sieve,
   for which logarithmic density might exist.
2. Reduce to the periodic case when all $n_i$ divide a common modulus.

Disproof track:
1. Try to choose moduli and residues so that the logarithmic density oscillates between two values.
2. Use rapidly growing moduli to "turn on" and "turn off" large chunks of integers in logarithmic scale.

Chosen path: prove existence of logarithmic density in an elementary periodic special case.

4) WORK
Definition.
For a set $S\subseteq\mathbb N$, define its logarithmic density (if it exists) by
\[\delta(S):=\lim_{N\to\infty}\frac{1}{\log N}\sum_{\substack{n\le N\\n\in S}}\frac{1}{n}.
\]

Lemma 1 (Finite restriction case).
If only finitely many residue classes are forbidden (i.e. the sequence $(n_i)$ is finite), then $A$ is eventually periodic
and in particular has a logarithmic density.

Proof.
Let the forbidden classes be $a_1\pmod{n_1},\dots,a_t\pmod{n_t}$ and set $L=\mathrm{lcm}(n_1,\dots,n_t)$.
For $n\ge n_t$, membership in $A$ depends only on the residue of $n\bmod L$.
Thus for sufficiently large $n$, $A$ is a union of residue classes modulo $L$.
Any union of residue classes modulo $L$ has a natural density, hence also a logarithmic density (equal to that natural density).
\qed

Lemma 2 (Periodic sets have logarithmic density).
If $S\subseteq\mathbb N$ is a union of residue classes modulo some $L$, then $\delta(S)$ exists.

Proof.
Write $S$ as a disjoint union of residue classes $r\pmod L$.
For each fixed residue class $r\pmod L$, one has
\[\sum_{\substack{n\le N\\n\equiv r\ (\mathrm{mod}\ L)}}\frac{1}{n} = \frac{1}{L}\log N + O(1)
\]
(as $N\to\infty$), by comparison with the harmonic series on an arithmetic progression.
Summing over finitely many residue classes gives
\[\sum_{\substack{n\le N\\n\in S}}\frac{1}{n} = \frac{|R|}{L}\log N + O(1),\]
where $R\subseteq\{0,\dots,L-1\}$ is the set of allowed residues. Dividing by $\log N$ and letting $N\to\infty$ yields
$\delta(S)=|R|/L$. \qed

5) VERIFICATION
- Lemma 2 uses the standard estimate $\sum_{m\le M} 1/(Lm+r)= (1/L)\log M + O(1)$, which follows by integral comparison.
- These lemmas do not address the genuinely infinite sequence case.

6) FINAL
**UNRESOLVED**

(i) Strongest fully proved partial result:
If the forbidden congruences are finite (or more generally if $A$ is eventually periodic), then the logarithmic density exists
(Lemmas 1--2).

(ii) First gap:
No argument here handles arbitrary infinite sequences $(n_i,a_i)$ where periodicity fails.

(iii) Top 3 next moves:
1. Try to construct a counterexample where the logarithmic density oscillates by activating constraints at carefully chosen scales.
2. Identify sieve-type sufficient conditions (e.g. on overlaps or on $\sum 1/n_i$) implying existence of logarithmic density.
3. Compare with known results on logarithmic densities in classical sieves and with related Problem [486].

(iv) Minimal counterexample structure (if existence were false):
A choice of moduli $n_i$ growing so fast that each new restriction removes a fresh chunk of integers in a way that causes
the weighted harmonic sums to oscillate between two accumulation points.
