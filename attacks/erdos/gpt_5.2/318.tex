% Attempt for Erdos Problem #318
% Following PROMPT_STRATEGY.MD
% Tools/Constraints:
% - Web browsing available? NO (restricted)
% - Computation available? YES (local Python), used (tiny check)
% GPT 5.2

OUTPUT
1) FORMAL RESTATEMENT
Let $A\subseteq\mathbb{N}$ be infinite and let $f:A\to\{-1,1\}$ be non-constant.
Say that $(A,f)$ has the “zero-sum” property if there exists a finite nonempty $S\subset A$ such that
\[\sum_{n\in S}\frac{f(n)}{n}=0.\tag{$\ast$}\]

Question 1: If $A$ is an infinite arithmetic progression, must $(A,f)$ have the zero-sum property for every non-constant $f$?

Question 2: If $A$ has positive density, must $(A,f)$ have the zero-sum property for every non-constant $f$? (Statement says: no.)

Question 3: If $A=\{k^2:k\ge 2\}$ (squares excluding $1$), must $(A,f)$ have the zero-sum property?

2) QUICK LITERATURE/CONTEXT CHECK
Browsing is not available. The statement itself reports that Question 1 is known true, Question 2 is false (with an explicit structural reason), and Question 3 is open. I do not rely on the reported proofs; I give self-contained proofs of the easy counterexample mechanisms mentioned.

3) ATTACK PLAN
Proof track:
1. Prove the “one even number” obstruction for positive-density sets directly.
2. Prove why including $1$ makes the squares version trivially false (as stated) without using advanced constants.
3. For the open “squares excluding 1” case, try small computational experiments for finite truncations to see whether zero-sum subsets tend to exist.

Disproof track:
1. Attempt to adapt the “one even number” obstruction to the squares-excluding-1 set (it does not apply directly since there are infinitely many even squares).

Chosen path: prove the two explicit obstruction lemmas; no progress on the open squares-excluding-1 case.

4) WORK
Lemma 1 (Positive density can fail: sets with exactly one even element).
Let $A\subseteq\mathbb{N}$ contain exactly one even number $e$ and all other elements of $A$ be odd.
Define $f:A\to\{-1,1\}$ by $f(e)=-1$ and $f(n)=1$ for all odd $n\in A$.
Then there is no finite nonempty $S\subset A$ with $\sum_{n\in S} f(n)/n=0$.

Proof.
Let $S\subset A$ be nonempty.
If $e\notin S$, then $\sum_{n\in S} f(n)/n=\sum_{n\in S} 1/n>0$, so it is not $0$.
If $e\in S$, write
\[\sum_{n\in S}\frac{f(n)}{n}=\Bigl(\sum_{\substack{n\in S\\ n\ \mathrm{odd}}}\frac1n\Bigr)-\frac1e.\]
The odd-part sum is a rational number whose reduced denominator is odd: indeed, its denominator divides the lcm of finitely many odd integers, hence is odd, and reduction cannot introduce a factor $2$.
But $1/e$ in lowest terms has even denominator (since $e$ is even).
Therefore the equality “odd-denominator rational $= 1/e$” is impossible, so the displayed expression cannot be $0$. \qed

Lemma 2 (Why $1$ must be excluded from the squares case).
Let $A=\{1\}\cup\{k^2:k\ge 2\}$ and define $f(1)=1$ and $f(k^2)=-1$ for all $k\ge 2$.
Then $f$ is non-constant and there is no finite nonempty $S\subset A$ satisfying $(\ast)$.

Proof.
First observe the strict inequality
\[\sum_{k=2}^{\infty}\frac{1}{k^2}<\sum_{k=2}^{\infty}\frac{1}{k(k-1)}=\sum_{k=2}^{\infty}\left(\frac{1}{k-1}-\frac{1}{k}\right)=1.\tag{$\dagger$}\]
Now let $S\subset A$ be nonempty.
If $1\notin S$, then $\sum_{n\in S} f(n)/n$ is a negative sum of terms $-1/k^2$, so it is $<0$.
If $1\in S$, then
\[\sum_{n\in S}\frac{f(n)}{n}=1-\sum_{\substack{k^2\in S\\ k\ge 2}}\frac{1}{k^2}\ge 1-\sum_{k=2}^{\infty}\frac{1}{k^2}>0\]
by $(\dagger)$. In both cases the sum is nonzero. \qed

FAST REALITY CHECK (computation).
Numerically,
\[\sum_{k=2}^{20}\frac{1}{k^2}\approx 0.59616<1,\]
consistent with the strict bound $(\dagger)$ used in Lemma 2.

5) VERIFICATION
- Lemma 1 is purely 2-adic/denominator parity: any odd-denominator rational cannot equal $1/e$ for even $e$.
- Lemma 2 avoids using $\pi^2/6$ by a strict termwise comparison to a telescoping series.
- These lemmas do not address the main arithmetic progression case (claimed solved in the statement) or the open “squares excluding 1” case.

6) FINAL
**UNRESOLVED**

(i) Strongest fully proved partial result:
The property fails for some positive-density sets: any set with exactly one even number admits a nonconstant $f$ with no zero-sum subset (Lemma 1). Also, including $1$ makes the squares version trivially false via an explicit $f$ (Lemma 2).

(ii) First gap:
For $A=\{k^2:k\ge 2\}$ (squares excluding $1$), determine whether every nonconstant $f:A\to\{-1,1\}$ admits a finite zero-sum subset.

(iii) Top 3 next moves:
1. For the squares-excluding-1 case, run targeted searches for small finite subsets $A\cap[1,M]$ and adversarial sign patterns $f$ to look for candidate obstructions.
2. Try to adapt the known (claimed) arithmetic progression argument to sparse sets like squares, identifying exactly where it fails and what additional input would be needed.
3. Investigate necessary conditions for failure: characterize sign patterns $f$ for which all finite signed reciprocal sums stay away from $0$ (e.g. by bounding possible ranges of partial sums).

(iv) Minimal counterexample structure (if the squares-excluding-1 statement were false):
A specific nonconstant sign assignment $f$ on the squares $\{k^2:k\ge 2\}$ such that every finite signed reciprocal sum $\sum_{k^2\in S} f(k^2)/k^2$ is nonzero; any such $f$ would have to defeat cancellation across infinitely many even and odd square denominators simultaneously.

