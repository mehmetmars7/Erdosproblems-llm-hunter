
OUTPUT
1) FORMAL RESTATEMENT
Let $p$ be a prime. Call $p$ a \emph{cluster prime} if for every even integer $n$ with $2\le n\le p-3$ there exist primes
$q_1,q_2\le p$ such that $n=q_1-q_2$.
Question: are there infinitely many cluster primes?

2) QUICK LITERATURE/CONTEXT CHECK
Browsing is not available. I only record what the problem statement itself claims about known results.
I have not verified those results here.

3) ATTACK PLAN
Proof track:
1. Rephrase the condition as: for each even $n\le p-3$ there exists a prime pair $(q,q+n)$ with $q,q+n\le p$.
2. Use distribution of prime pairs (Hardy--Littlewood heuristics / sieve) to show such pairs exist for all even $n$ up to $p$.

Disproof track:
1. For a given $p$, search for an even $n\le p-3$ such that no prime pair with gap $n$ lies below $p$.
2. Try to show that typical primes $p$ fail for some medium-sized $n$ due to missing prime gaps.

Chosen path: establish equivalent formulations and verify small examples by hand.

4) WORK
Lemma 1 (Equivalent "prime-pair" formulation).
Let $p$ be prime. The following are equivalent:
(a) For every even $n$ with $2\le n\le p-3$ there exist primes $q_1,q_2\le p$ with $n=q_1-q_2$.
(b) For every even $n$ with $2\le n\le p-3$ there exists a prime $q\le p-n$ such that $q+n$ is prime.

Proof.
(a)$\Rightarrow$(b): given $n=q_1-q_2$ with $q_1,q_2\le p$, set $q=q_2$ so $q\le p-n$ and $q+n=q_1$ is prime.
(b)$\Rightarrow$(a): given $q$ as in (b), set $q_2=q$ and $q_1=q+n$.
\qed

Lemma 2 (The top even $n=p-3$ is automatic for $p\ge 5$).
If $p\ge 5$ is prime, then $p-3$ can be written as a difference of two primes $\le p$.

Proof.
$p-3=p-3$ with $p$ prime and $3$ prime, and both are $\le p$ when $p\ge 5$. \qed

Example 3 (Small cluster primes).
$p=5,7,11$ are cluster primes.

Proof.
For $p=5$, the only even $n\le 2$ is $n=2=5-3$.
For $p=7$, the even $n\le 4$ are $2=5-3$ and $4=7-3$.
For $p=11$, the even $n\le 8$ are $2=5-3$, $4=7-3$, $6=11-5$, $8=11-3$.
All differences use primes $\le p$. \qed

5) VERIFICATION
- Lemma 1 is a direct change of variables.
- Lemma 2 checks the boundary case $n=p-3$.
- Example 3 is a finite explicit verification.

6) FINAL
**UNRESOLVED**

(i) Strongest fully proved partial result:
The cluster-prime condition is equivalent to requiring a prime pair $(q,q+n)$ below $p$ for every even $n\le p-3$ (Lemma 1),
and the maximal even value $n=p-3$ is always representable for $p\ge 5$ (Lemma 2).

(ii) First gap:
No argument here produces infinitely many primes $p$ for which \emph{all} even gaps $2\le n\le p-3$ occur between primes $\le p$.

(iii) Top 3 next moves:
1. Run computation to list cluster primes up to a cutoff and locate the first failures (the statement mentions $97$).
2. Use sieve/prime-pair estimates to lower-bound the number of prime pairs with each even gap $n$ below $p$.
3. Seek a probabilistic model to estimate the chance that all even gaps up to $p$ appear among primes $\le p$.

(iv) Minimal counterexample structure (if infinitude were false):
A mechanism forcing, for all sufficiently large primes $p$, the existence of some even $n\le p-3$ such that
no prime pair of gap $n$ occurs with both primes $\le p$.
