% Attempt for Erdos Problem #23
% Following PROMPT_STRATEGY.MD
% Tools/Constraints:
% - Web browsing available? NO (restricted)
% - Computation available? YES (local Python), not used
% GPT 5.2

OUTPUT
1) FORMAL RESTATEMENT
Let $n\ge 1$. Let $G=(V,E)$ be a triangle-free graph on $|V|=5n$ vertices.
Question: does there always exist a set of edges $F\subseteq E$ with $|F|\le n^2$ such that $G\setminus F$ is bipartite?
Equivalently, is the \emph{bipartite edge deletion number} of every triangle-free $G$ on $5n$ vertices at most $n^2$?

2) QUICK LITERATURE/CONTEXT CHECK
Browsing is not available. I only record what the problem statement itself claims.
I have not verified those results here.

3) ATTACK PLAN
Proof track:
1. Relate "delete edges to make bipartite" to finding a large cut (max-cut), since edges across a cut form a bipartite subgraph.
2. Use structural properties of triangle-free graphs (e.g. Mantel's bound $e(G)\le |V|^2/4$) to bound deletions.
3. Improve constants using stability/regularity or semi-definite methods (beyond this attempt).

Disproof track:
1. Analyze extremal triangle-free graphs (blow-ups of $C_5$) and compute how many edges must be removed.
2. Try to construct denser triangle-free graphs where every cut misses many edges.

Chosen path: prove a baseline constant via max-cut and verify the blow-up lower bound is plausible.

4) WORK
Lemma 1 (Max-cut gives a universal $1/2$-deletion bound).
For any finite graph $G=(V,E)$, there exists a bipartite subgraph with at least $|E|/2$ edges.
Equivalently, one can delete at most $|E|/2$ edges to make $G$ bipartite.

Proof.
Choose a random bipartition $V=X\sqcup Y$ by putting each vertex independently into $X$ with probability $1/2$.
Each edge crosses the cut with probability $1/2$, so the expected number of crossing edges is $|E|/2$.
Hence there exists a cut with at least $|E|/2$ crossing edges. Keeping those edges yields a bipartite graph. \qed

Lemma 2 (Mantel bound for triangle-free graphs).
If $G$ is triangle-free on $N$ vertices, then $|E(G)|\le N^2/4$.

Proof.
For any edge $uv$, triangle-freeness implies $N(u)\cap N(v)=\emptyset$, so $\deg(u)+\deg(v)\le N$.
Summing $\deg(u)+\deg(v)\le N$ over all edges and noting that the left-hand side equals $\sum_{w\in V} \deg(w)^2$,
we obtain
\[\sum_{w\in V}\deg(w)^2 \le N|E|.\]
By Cauchy--Schwarz, $(\sum_w \deg(w))^2 \le N\sum_w \deg(w)^2 \le N\cdot N|E|=N^2|E|$.
Since $\sum_w \deg(w)=2|E|$, this gives $4|E|^2\le N^2|E|$, hence $|E|\le N^2/4$. \qed

Corollary 3 (A fully proved but weak constant for Problem 23).
Every triangle-free graph on $5n$ vertices can be made bipartite by deleting at most $\frac{25}{8}n^2$ edges.

Proof.
By Lemma 2, $|E|\le (5n)^2/4=25n^2/4$. By Lemma 1, delete at most $|E|/2\le 25n^2/8$ edges. \qed

Lemma 4 (Blow-up of $C_5$ needs at least $n^2$ deletions).
Let $G$ be the blow-up of $C_5$ with vertex classes $V_0,\dots,V_4$ each of size $n$ and complete bipartite edges
between $V_i$ and $V_{i+1}$ (indices mod $5$). Then any bipartite subgraph of $G$ has at most $4n^2$ edges.
In particular, at least $n^2$ edges must be deleted to make $G$ bipartite.

Proof.
Let $X\sqcup Y$ be any bipartition of $V(G)$, and let $x_i=|V_i\cap X|$.
For the complete bipartite graph between $V_i$ and $V_{i+1}$, the number of edges \emph{crossing} the cut is
$x_i(n-x_{i+1})+(n-x_i)x_{i+1}$, so the number of \emph{monochromatic} edges is
$x_ix_{i+1}+(n-x_i)(n-x_{i+1})$.
Let $T$ be the total number of monochromatic edges over the five complete bipartite blocks.
Choose independent uniformly random vertices $v_i\in V_i$ and consider the $5$-cycle
$v_0v_1v_2v_3v_4v_0$ (all these edges exist in $G$). Any $2$-coloring of a $5$-cycle has at least one monochromatic edge,
so this random $5$-cycle has at least one monochromatic edge deterministically, hence its expected number of
monochromatic edges is at least $1$.
On the other hand, the probability that the edge $v_iv_{i+1}$ is monochromatic equals
$\bigl(x_ix_{i+1}+(n-x_i)(n-x_{i+1})\bigr)/n^2$.
Summing over the five edges shows $\mathbb E[\#\text{monochromatic edges in the cycle}]=T/n^2$.
Therefore $T/n^2\ge 1$, i.e. $T\ge n^2$.
To obtain a bipartite subgraph from the cut $X\sqcup Y$, one must delete all monochromatic edges, so at least $n^2$ edges
must be deleted for every cut, and hence for every bipartite subgraph.
Since $G$ has $5n^2$ edges total, any bipartite subgraph has at most $5n^2-n^2=4n^2$ edges. \qed

5) VERIFICATION
- Lemma 2: the key identity $\sum_{uv\in E}(\deg u+\deg v)=\sum_w \deg(w)^2$ holds because each incident edge contributes $\deg(w)$.
- Lemma 4: the probabilistic step is only linearity of expectation; the "at least one monochromatic edge" fact holds because
  an odd cycle cannot be properly $2$-colored.

6) FINAL
**UNRESOLVED**

(i) Strongest fully proved partial result:
A universal (weak) deletion bound $\le 25n^2/8$ (Corollary 3), and a rigorous lower bound example showing
$n^2$ deletions can be necessary (Lemma 4).

(ii) First gap:
No method here improves the constant from $25/8$ down to $1$ (or even close to $1$).

(iii) Top 3 next moves:
1. Study strengthened max-cut lower bounds specific to triangle-free graphs.
2. Use stability of triangle-free extremal graphs to show near-extremal graphs are close to bipartite or close to a $C_5$ blow-up.
3. Implement the known best constants from the literature in a self-contained argument.

(iv) Minimal counterexample structure (if the $n^2$ bound were false):
A triangle-free graph on $5n$ vertices in which every cut misses more than $n^2$ edges (i.e. every bipartite subgraph
has fewer than $|E|-n^2$ edges), likely a dense pseudorandom triangle-free construction.
