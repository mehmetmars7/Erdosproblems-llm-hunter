
OUTPUT
1) FORMAL RESTATEMENT
For a finite interval $I=[a,b]\subset\mathbb{N}$ define $S(I)=\sum_{n=a}^{b}1/n$.
The problem asks: for all sufficiently large integers $k$, do there exist $k$ intervals $I_1,\dots,I_k$ such that:
- each $I_i$ has length at least $2$ ($|I_i|\ge 2$),
- the intervals are pairwise disjoint and not adjacent (if $I_i=[a_i,b_i]$ and $I_j=[a_j,b_j]$ with $b_i<a_j$ then $b_i+1<a_j$),
- and
\[1=\sum_{i=1}^k S(I_i)?\]

2) QUICK LITERATURE/CONTEXT CHECK
Browsing is not available. The statement gives an explicit example for $2$ using five disjoint non-adjacent intervals of lengths $\ge 2$.

3) ATTACK PLAN
Proof track:
1. Use integral bounds $S([a,b])\approx \log(b/a)$ to design intervals whose contributions add to $1$.
2. Build an exact identity by starting from an Egyptian fraction expansion of $1$ and grouping denominators into separated blocks of consecutive integers.
3. Attempt inductive “refinement” moves that replace a unit fraction by a sum over a short interval, increasing the number of intervals.

Disproof track:
1. Try to show a structural obstruction: disjoint non-adjacent intervals have sums with restricted denominators making $1$ impossible for large $k$.

Chosen path: prove useful analytic bounds and rule out the trivial small-$k$ cases; computationally no solutions found for small search ranges.

4) WORK
Lemma 1 (Integral bounds for a harmonic interval).
Let $1\le a\le b$. Then
\[\int_{a}^{b+1}\frac{dx}{x}\le \sum_{n=a}^{b}\frac1n \le \int_{a-1}^{b}\frac{dx}{x}\quad\text{(for }a\ge 2\text{)}.\]
Equivalently,
\[\log\Bigl(\frac{b+1}{a}\Bigr)\le S([a,b])\le \log\Bigl(\frac{b}{a-1}\Bigr)\quad(a\ge 2).\]

Proof.
The function $x\mapsto 1/x$ is decreasing.
For the lower bound, for each integer $n\in[a,b]$ we have $\int_{n}^{n+1}\frac{dx}{x}\le 1/n$; summing gives
\[\int_{a}^{b+1}\frac{dx}{x}\le \sum_{n=a}^{b}\frac1n.\]
For the upper bound (when $a\ge 2$), we have $1/n\le \int_{n-1}^{n}\frac{dx}{x}$; summing gives
\[\sum_{n=a}^{b}\frac1n\le \int_{a-1}^{b}\frac{dx}{x}.\]
Evaluating the integrals gives the logarithmic forms. \qed

Lemma 2 (No single interval of length $\ge 2$ sums to $1$).
There do not exist integers $1\le a<b$ such that $\sum_{n=a}^{b}1/n=1$.
Hence there is no solution to the problem for $k=1$.

Proof.
This is exactly Lemma 1 from Problem \#287’s analysis (reproved here in brief):
If $a=1$ then the sum is $>1$.
If $a\ge 2$ and $a>b/2$, then $\sum_{n=a}^{b}1/n\le (b-a+1)/a<1$.
If $a\le b/2$, choose a prime $p\in(b/2,b]$ (Bertrand). Then $p$ appears exactly once among $a,a+1,\dots,b$,
and clearing denominators shows the sum is not an integer, hence not $1$. \qed

FAST REALITY CHECK (computation).
I brute-forced with endpoint bounds:
- For $k=2$, searching all disjoint non-adjacent interval pairs with endpoints $\le 300$ and lengths $\ge 2$, I found no solutions to $S(I_1)+S(I_2)=1$.
- For $k=3$, searching all disjoint non-adjacent interval triples with endpoints $\le 120$ and lengths $\ge 2$, I found no solutions to $S(I_1)+S(I_2)+S(I_3)=1$.

5) VERIFICATION
- Lemma 1 is a standard integral comparison and provides a useful heuristic: $S([a,b])$ is close to $\log(b/a)$ when $a$ is large.
- Lemma 2 uses a prime modulus obstruction and rules out $k=1$ rigorously.
- The computations are limited-range searches and do not speak to large $k$ behavior.

6) FINAL
**UNRESOLVED**

(i) Strongest fully proved partial result:
Harmonic interval sums satisfy tight logarithmic bounds (Lemma 1), and $k=1$ is impossible under the “length $\ge 2$” restriction (Lemma 2). Small-range brute force finds no solutions for $k=2$ with endpoints $\le 300$ or for $k=3$ with endpoints $\le 120$.

(ii) First gap:
Construct, for arbitrarily large $k$, a decomposition $1=\sum_{i=1}^k S(I_i)$ into disjoint non-adjacent intervals of length $\ge 2$, or prove such decompositions do not exist.

(iii) Top 3 next moves:
1. Try to adapt known Egyptian fraction constructions to produce denominators in consecutive \emph{blocks} separated by gaps, then group each block as an interval.
2. Use Lemma 1 to design approximate solutions via $\sum \log((b_i+1)/a_i)\approx 1$ and then attempt exact correction by local adjustments.
3. Extend the computational search using heuristic/backtracking methods targeting large $k$ with large denominators (where Lemma 1 suggests flexibility).

(iv) Minimal counterexample structure (if the statement were false):
Even for very large $k$, every collection of $k$ disjoint non-adjacent intervals of length $\ge 2$ would have total sum either always $<1$ or always $>1$, never hitting $1$; a “minimal” obstruction would likely come from unavoidable prime-denominator effects preventing exact equality.

