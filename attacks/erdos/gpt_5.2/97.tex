
OUTPUT
1) FORMAL RESTATEMENT
Definitions/Conventions.
Let $A\subset\mathbb R^2$ be the vertex set of a convex polygon with $|A|=n$.
For $x\in A$, define
\[g_A(x):=\max_{r>0}\big|\{y\in A\setminus\{x\}: |x-y|=r\}\big|.\]

Question.
Is it true that every convex polygon vertex set $A$ contains a vertex $x$ with
\[g_A(x)\le 3?\]
(Equivalently: no distance from $x$ occurs with multiplicity $\ge 4$ among the other vertices.)

2) QUICK LITERATURE/CONTEXT CHECK
Web browsing is not available in this session.
From the statement only:
- The stronger conjecture $g_A(x)\le 2$ for some vertex is false (Danzer example).
- There are examples where every vertex has three equidistant vertices (Fishburn--Reeds).
I do not reproduce those constructions.

3) ATTACK PLAN
Proof track ideas:
1. Use convexity and circle intersection constraints to limit how many vertices can lie on a circle centered at a vertex.
2. Attempt to show that large multiplicities at every vertex force global structural constraints that contradict convexity.

Disproof track ideas:
1. Try to extend Danzer/Fishburn--Reeds style constructions to produce 4 equidistant vertices from every vertex, which would refute the conjecture.

Chosen path in this attempt: prove two small, fully rigorous lemmas: (i) regular polygons satisfy a much stronger bound $g_A(x)\le 2$, and (ii) the inductive implication to Problem \#96.

4) WORK
Lemma 1 (Regular polygons have $g_A(x)\le 2$).
Let $A$ be the vertex set of a regular $n$-gon ($n\ge 3$).
Then for every vertex $x\in A$, one has
\[g_A(x)\le 2.\]

Proof.
Fix a vertex $x$.
Distances from $x$ to other vertices depend only on the step distance along the cycle: the distance from $x$ to the vertex $t$ steps clockwise equals the distance to the vertex $t$ steps counterclockwise.
Thus for each $t\in\{1,2,\dots,\lfloor (n-1)/2\rfloor\}$, there are exactly two vertices at that distance from $x$.
If $n$ is even, there is one additional vertex opposite $x$ at step $n/2$, giving a distance that occurs once.
Therefore the maximum multiplicity of any distance from $x$ is $2$. \qed

Lemma 2 (If $g_A(x)\le k$ holds at some vertex, then unit distances are linear).
Fix $k\ge 1$.
Assume that every convex $n$-gon $A$ contains a vertex $x$ with $g_A(x)\le k$.
Then every convex $n$-gon $A$ satisfies
\[U(A)\le kn,\]
where $U(A)$ is the number of unit-distance pairs among vertices.

Proof.
We induct on $n$.
For $n\le 2$, $U(A)=0\le kn$.
Assume $n\ge 3$ and the claim holds for smaller sizes.
Let $A$ be a convex $n$-gon, and choose $x\in A$ with $g_A(x)\le k$ (by the assumption).
Let $A':=A\setminus\{x\}$; this is again the vertex set of a convex $(n-1)$-gon.

Every unit-distance pair in $A$ is either contained in $A'$ or involves $x$.
By induction, $U(A')\le k(n-1)$.
The number of unit-distance neighbors of $x$ is at most $g_A(x)\le k$ (since radius $1$ is one of the radii counted by $g_A(x)$).
Therefore $U(A)\le U(A')+k\le k(n-1)+k=kn$. \qed

FAST REALITY CHECK
The conjecture asks for existence of a vertex with $g_A(x)\le 3$ in \emph{every} convex polygon.
Regular polygons satisfy an even stronger property (Lemma 1), so any counterexample must be highly non-uniform.

5) VERIFICATION
- Lemma 1 uses only rotational symmetry of the regular polygon and the pairing of vertices at equal step distance.
- Lemma 2 is purely inductive and does not assume any deep geometry.

6) FINAL
**UNRESOLVED**

(i) Strongest fully proved partial result:
Regular polygons satisfy $g_A(x)\le 2$ for every vertex $x$ (Lemma 1), and any universal bound $g_A(x)\le k$ would imply a linear unit-distance bound $U(A)\le kn$ for convex polygons (Lemma 2).

(ii) First gap:
Prove that \emph{every} convex polygon has at least one vertex with $g_A(x)\le 3$, or construct a convex polygon where every vertex has four (or more) equidistant vertices.

(iii) Top 3 next moves (concrete):
1. Try to show that if every vertex has $g_A(x)\ge 4$, then the polygon must contain many concyclic vertex subsets centered at vertices, forcing a global contradiction with convexity.
2. Attempt to adapt known constructions with $g_A(x)=3$ at every vertex (Danzer/Fishburn--Reeds) to push to $4$.
3. Identify weaker variants (existence of some absolute $k$) and attempt to prove them, as they already yield $O(n)$ unit-distance bounds via Lemma 2.

(iv) What a minimal counterexample would likely look like (structure/parameters):
A convex $n$-gon in which for every vertex $x$, there exists a radius $r_x$ such that at least four other vertices lie on the circle centered at $x$ of radius $r_x$,
yet these circles and chords fit together without violating convexity.
