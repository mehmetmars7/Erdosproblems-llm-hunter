% Attempt for Erdos Problem #96
% Following PROMPT_STRATEGY.MD
% Tools/Constraints:
% - Web browsing available? NO (restricted)
% - Computation available? YES (local), not used
% GPT 5.2

OUTPUT
1) FORMAL RESTATEMENT
Definitions/Conventions.
Let $A\subset\mathbb R^2$ be the vertex set of a convex polygon with $|A|=n$.
Let
\[U(A):=\big|\{\{x,y\}\subset A:\ |x-y|=1\}\big|\]
be the number of unit-distance pairs among vertices.

Conjecture.
There exists an absolute constant $C$ such that for all $n$ and all convex $n$-gons $A$,
\[U(A)\le Cn.\]
Stronger conjecture: $U(A)\le 2n+O(1)$.

2) QUICK LITERATURE/CONTEXT CHECK
Web browsing is not available in this session.
The statement lists known bounds $O(n\log n)$ and the best known explicit bound $n\log_2 n+4n$.
I do not re-prove those here.

3) ATTACK PLAN
Proof track ideas:
1. Use convexity to strengthen point-circle incidence bounds for unit distances (this is what the known $O(n\log n)$ proofs do).
2. Use Problem \#97-type equidistance restrictions plus induction to get linear bounds.

Disproof track ideas:
1. Construct convex polygons with $\omega(n)$ unit distances; the statement suggests the best known constructions are linear ($2n-7$).

Chosen path in this attempt: provide two rigorous lemmas: an ``easy'' $O(n^{3/2})$ upper bound (valid for all point sets), and the precise implication from Problem \#97 to linear unit-distance bounds by induction.

4) WORK
Lemma 1 (Easy bound: $U(A)=O(n^{3/2})$ for any point set).
For any set $A$ of $n$ distinct points in the plane (in particular, for vertices of a convex polygon),
\[U(A)\le n^{3/2}+n.\]

Proof.
Build a bipartite graph $B$ with left part $L=A$ (centers) and right part $R=A$ (points), joining $x\in L$ to $y\in R$ iff $|x-y|=1$.
Then $e(B)=2U(A)$.
As in Problem \#90, any two distinct circles intersect in at most two points, so any two vertices of $L$ have at most two common neighbors in $R$.
Let $d(r)$ be degrees on the right.
Then $P:=\sum_{r\in R}\binom{d(r)}{2}$ counts unordered 2-paths through $R$ and satisfies $P\le 2\binom{n}{2}<n^2$.
Also $P\ge \frac12\cdot\frac{e(B)^2}{n}-\frac12 e(B)$ by Cauchy--Schwarz as before.
Thus $\frac{e(B)^2}{2n}-\frac{e(B)}{2}\le n^2$, giving $e(B)\le 2n^{3/2}+2n$ and hence $U(A)\le n^{3/2}+n$. \qed

Lemma 2 (How equidistance control implies a linear unit-distance bound).
Fix an integer $k\ge 1$.
Assume the following statement holds for convex polygons:
\[
(\star_k)\quad\text{Every convex }n\text{-gon }A\subset\mathbb R^2\text{ contains a vertex }x\in A\text{ such that }g_A(x)\le k,
\]
where $g_A(x):=\max_{r>0}|\{y\in A\setminus\{x\}: |x-y|=r\}|$.
Then every convex $n$-gon $A$ satisfies
\[U(A)\le k\,n.\]

Proof.
We induct on $n$.
For $n\le 2$, $U(A)=0\le kn$.
Assume $n\ge 3$ and the claim holds for smaller sizes.
Let $A$ be a convex $n$-gon, and choose $x\in A$ satisfying $g_A(x)\le k$ (by $(\star_k)$).
Let $A':=A\setminus\{x\}$; this is a convex $(n-1)$-gon after relabeling the cyclic order.

Any unit-distance pair in $A$ is either entirely within $A'$ or involves $x$.
By induction, $U(A')\le k(n-1)$.
Also, the number of unit-distance neighbors of $x$ is at most $g_A(x)\le k$ (since distance $1$ is one of the radii counted by $g_A(x)$).
Therefore
\[U(A)\le U(A') + k \le k(n-1)+k=kn.\qed\]

FAST REALITY CHECK
Lemma 2 formalizes the statement's remark: a positive answer to Problem \#97 (for $k=3$ or better) would immediately imply a linear $O(n)$ bound for unit distances in convex polygons.

5) VERIFICATION
- Lemma 2 uses only induction and the definition of $g_A(x)$; convexity is used only to ensure $A\setminus\{x\}$ is again realizable as a convex polygon vertex set.

6) FINAL
**UNRESOLVED**

(i) Strongest fully proved partial result:
I proved a general upper bound $U(A)\le n^{3/2}+n$ (Lemma 1) and a clean implication $(\star_k)\Rightarrow U(A)\le kn$ for convex polygons (Lemma 2).

(ii) First gap:
Prove $(\star_k)$ for some absolute constant $k$ (Problem \#97 is essentially the case $k=3$), or otherwise exploit convexity directly to reach $O(n)$ without such an equidistance lemma.

(iii) Top 3 next moves (concrete):
1. Attempt to prove $(\star_3)$ (or some fixed $k$) for convex polygons by geometric constraints on repeated distances from a vertex.
2. Study the known $O(n\log n)$ proofs and identify the bottleneck that prevents improving to $O(n)$.
3. Search for convex constructions with $\gg n$ unit distances to test sharpness (current best known is $2n-7$ per statement).

(iv) What a minimal counterexample would likely look like (structure/parameters):
A family of convex $n$-gons where every vertex has at least $k+1$ equidistant vertices for some fixed $k$ (so $(\star_k)$ fails),
and where the unit-distance graph remains dense enough to yield $\omega(n)$ unit-distance pairs.
