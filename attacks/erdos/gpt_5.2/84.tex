% Attempt for Erdos Problem #84
% Following PROMPT_STRATEGY.MD
% Tools/Constraints:
% - Web browsing available? NO (restricted)
% - Computation available? YES (local), not used
% GPT 5.2

OUTPUT
1) FORMAL RESTATEMENT
Definitions/Conventions.
- Let $G$ be a finite simple graph on vertex set $[n]=\{1,\dots,n\}$.
- Define the \emph{cycle set} of $G$:
\[C(G):=\{\ell\in\{3,4,\dots,n\}:\ G\text{ contains a cycle of length }\ell\}.\]
- Define
\[f(n):=\big|\{C(G):\ G\text{ is a graph on }n\text{ vertices}\}\big|,\]
the number of distinct cycle sets realizable by $n$-vertex graphs.

Tasks.
1. Prove $f(n)=o(2^n)$ (i.e. $f(n)/2^n\to 0$).
2. Prove $f(n)/2^{n/2}\to\infty$.

2) QUICK LITERATURE/CONTEXT CHECK
Web browsing is not available in this session.
From the problem text only:
- The first task is solved (Verstra\"{e}te; improved by Nenadov) with very strong quantitative bounds.
- The second task remains conjectural; Erd\H{o}s--Faudree showed $f(n)>2^{n/2}$.
I do not re-prove Verstra\"{e}te/Nenadov here.

3) ATTACK PLAN
Proof track ideas:
1. Build explicit families of graphs whose cycle sets can be prescribed in many ways, to get lower bounds on $f(n)$.
2. Find structural constraints on cycle sets to reduce the number of possibilities and prove $f(n)=o(2^n)$.

Disproof track ideas:
1. For the $f(n)/2^{n/2}\to\infty$ conjecture, attempt to show only $2^{\Theta(n)}$ cycle sets are possible (which is true), but with exponent at least $1/2$ (nontrivial).

Chosen path in this attempt: prove two basic bounds from first principles: the trivial upper bound, and a constructive lower bound using disjoint unions of cycles (yielding $2^{\Theta(\sqrt n)}$ realized cycle sets). This does not solve either requested asymptotic, but it is fully rigorous progress.

4) WORK
Lemma 1 (Trivial upper bound).
For every $n\ge 3$,
\[f(n)\le 2^{n-2}.\]

Proof.
Every cycle set is a subset of $\{3,4,\dots,n\}$, which has $n-2$ elements.
There are exactly $2^{n-2}$ subsets, so $f(n)$ (the number of realizable subsets) is at most $2^{n-2}$. \qed

Lemma 2 (A constructive lower bound via disjoint unions of cycles).
Let $t\ge 1$ and suppose
\[\sum_{\ell=3}^{t+2} \ell \le n.\tag{*}\]
Then $f(n)\ge 2^{t}$.

Proof.
For each subset $S\subseteq \{3,4,\dots,t+2\}$, construct a graph $G_S$ on $n$ vertices as follows:
- for each $\ell\in S$, include a cycle $C_\ell$ as a connected component;
- include the remaining vertices (if any) as isolated vertices (no edges).
Because components are disjoint, every cycle of $G_S$ lies entirely within one component cycle $C_\ell$.
Thus the cycle lengths occurring in $G_S$ are exactly the set $S$.
Therefore the cycle set map $S\mapsto C(G_S)$ is injective on the $2^t$ subsets of $\{3,\dots,t+2\}$, proving $f(n)\ge 2^t$. \qed

Corollary 3 (Explicit growth $f(n)\ge 2^{\Omega(\sqrt n)}$).
For all sufficiently large $n$,
\[f(n)\ge 2^{\lfloor \sqrt{2n}\rfloor-3}.\]

Proof.
Take $t:=\lfloor \sqrt{2n}\rfloor-3$.
Then
\[\sum_{\ell=3}^{t+2}\ell = \sum_{\ell=3}^{\lfloor \sqrt{2n}\rfloor-1}\ell \le \sum_{\ell=1}^{\lfloor \sqrt{2n}\rfloor-1}\ell
 = \frac{(\lfloor \sqrt{2n}\rfloor-1)\lfloor \sqrt{2n}\rfloor}{2}\le n\]
for $n$ large, so condition (*) holds and Lemma 2 gives $f(n)\ge 2^t$. \qed

FAST REALITY CHECK
The corollary gives only $f(n)\ge 2^{\Theta(\sqrt n)}$, which is far weaker than the conjectured $f(n)\gg 2^{n/2}$ and far weaker than the known lower bound $f(n)>2^{n/2}$ stated in the problem text.
So the real challenge is to build graphs whose cycle sets can encode linearly many bits without accidentally introducing many extra cycle lengths.

5) VERIFICATION
- Lemma 2 relies crucially on disjoint union: disconnected components prevent creation of cycles that mix lengths.
- Corollary 3 checks the vertex budget using a crude sum-of-integers bound.

6) FINAL
**UNRESOLVED**

(i) Strongest fully proved partial result:
I proved the trivial upper bound $f(n)\le 2^{n-2}$ (Lemma 1) and a constructive lower bound $f(n)\ge 2^{\Omega(\sqrt n)}$ (Corollary 3).

(ii) First gap:
Prove a meaningful \emph{exponential-in-$n$} lower bound beyond $2^{o(n)}$ (such as the known $2^{n/2}$), or re-prove the deep upper bound $f(n)=o(2^n)$.

(iii) Top 3 next moves (concrete):
1. Try to reconstruct Erd\H{o}s--Faudree's proof that $f(n)>2^{n/2}$ by explicitly building a family of graphs where cycle sets vary in $\Theta(n)$ independent ways.
2. Investigate structural constraints on cycle sets (e.g. closure properties, forbidden patterns) that could yield a direct proof that $f(n)=2^{(1-o(1))n}$ is impossible, aiming toward $f(n)=o(2^n)$ without heavy machinery.
3. For the lower-bound direction, design bounded-degree ``cycle-length selector'' gadgets on $O(1)$ vertices that toggle presence/absence of a particular length while keeping other lengths controlled.

(iv) What a minimal counterexample would likely look like (structure/parameters):
To refute $f(n)/2^{n/2}\to\infty$, one would need $f(n)\le C 2^{n/2}$ along an infinite sequence of $n$,
meaning that cycle sets have at most $n/2+O(1)$ bits of freedom; this would require proving a strong global constraint that prevents encoding more than $n/2$ independent choices in cycle-length patterns.

