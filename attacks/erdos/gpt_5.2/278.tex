
OUTPUT
1) FORMAL RESTATEMENT
Fix integers $2\le n_1<\cdots<n_r$. For each $i$ choose a residue class $a_i\pmod{n_i}$ and let
\[U(a_1,\dots,a_r)=\bigcup_{i=1}^r \{m\in\mathbb{Z}: m\equiv a_i\!\!\!\pmod{n_i}\}\]
be the set of integers covered by at least one congruence.
Since $U$ is periodic (period dividing $L=\mathrm{lcm}(n_1,\dots,n_r)$), it has a natural density
\[\mathbf{d}(U)=\lim_{N\to\infty}\frac{|U\cap[-N,N]|}{2N+1}=\frac{|U\bmod L|}{L}.\]

Question 1: Determine/estimate $\max_{(a_i)} \mathbf{d}(U)$.

Question 2: Is $\min_{(a_i)} \mathbf{d}(U)$ attained when all $a_i$ are equal (e.g. all $0$)?

2) QUICK LITERATURE/CONTEXT CHECK
Browsing is not available. I only record what the problem statement itself claims: Simpson observed a general lower bound (an inclusion--exclusion expression in lcms) which is achieved when all $a_i$ are equal, settling Question 2. I do not re-prove the full general statement here; I prove special cases and sanity-check computationally.

3) ATTACK PLAN
Proof track:
1. Reduce density computations to counting covered residues modulo $L=\mathrm{lcm}(n_i)$.
2. Prove Question 2 in tractable cases: $r=2$ and the pairwise coprime case (where density is independent of residues).
3. For Question 1, look for constructions maximizing coverage by making overlaps as small as possible; compare to the trivial upper bound $\sum 1/n_i$.

Disproof track:
1. For Question 2, attempt to find small moduli where a nontrivial choice of residues gives smaller union than the aligned choice; brute force small examples.

Chosen path: rigorous reductions + special-case proofs; brute force confirms the “all equal” choice minimizes density in tested examples.

4) WORK
Lemma 1 (Density via one period).
Let $L=\mathrm{lcm}(n_1,\dots,n_r)$. Then $U$ is periodic modulo $L$ and
\[\mathbf{d}(U)=\frac{|\{x\in\{0,\dots,L-1\}: x\equiv a_i\pmod{n_i}\ \text{for some }i\}|}{L}.\]

Proof.
Membership in $U$ depends only on the residue class of $m$ modulo each $n_i$, hence only on $m\bmod L$.
Therefore $U$ is a union of residue classes modulo $L$. The natural density of any union of residue classes modulo $L$
is exactly the proportion of residues modulo $L$ it occupies, which is the displayed formula. \qed

Lemma 2 (Pairwise coprime moduli: density is independent of residues).
Assume $\gcd(n_i,n_j)=1$ for all $i\ne j$. Then for any residue choices $(a_i)$,
\[\mathbf{d}(U)=1-\prod_{i=1}^r\Bigl(1-\frac1{n_i}\Bigr).\]
In particular, the minimum equals the maximum and is achieved when all $a_i$ are equal.

Proof.
Let $L=\prod_{i=1}^r n_i$ (pairwise coprime).
By Lemma 1 it suffices to count residues $x\bmod L$ that avoid all congruences $x\equiv a_i\pmod{n_i}$.
For a fixed $i$, there are exactly $n_i-1$ choices of $x\bmod n_i$ that avoid $a_i$.
By the Chinese Remainder Theorem, choices modulo different $n_i$ combine independently, so the number of residues modulo $L$
that avoid all $r$ congruences is $\prod_{i=1}^r (n_i-1)$.
Thus the uncovered density is $\prod_{i=1}^r (n_i-1)/n_i$, and the covered density is the stated complement. \qed

Lemma 3 (Two moduli: aligned residues minimize the union).
Let $r=2$ with moduli $n_1,n_2$ and residues $a_1\pmod{n_1}$, $a_2\pmod{n_2}$.
Let $g=\gcd(n_1,n_2)$ and $M=\mathrm{lcm}(n_1,n_2)$. Then
\[\mathbf{d}(U)=\frac1{n_1}+\frac1{n_2}-\delta,\]
where $\delta$ equals $1/M$ if $a_1\equiv a_2\pmod g$ (the congruences are compatible) and equals $0$ otherwise.
Hence the minimum density is attained when $a_1\equiv a_2\pmod g$ (in particular when $a_1=a_2$ as integers).

Proof.
By Lemma 1, work modulo $M$.
Each congruence class has density $1/n_i$.
Their intersection is the set of $x$ satisfying $x\equiv a_1\pmod{n_1}$ and $x\equiv a_2\pmod{n_2}$, which is either empty or
a single congruence class modulo $M$ (Chinese remainder for non-coprime moduli), hence has density $0$ or $1/M$.
Inclusion--exclusion for two sets gives the formula.
Since the intersection term is subtracted, the union is minimized when the intersection is nonempty (maximal), i.e. when $a_1\equiv a_2\pmod g$. \qed

FAST REALITY CHECK (computation).
For $(n_i)=(2,3,4,6)$ I exhaustively checked all residue choices and found:
minimum density $=2/3$, achieved at $(a_i)=(0,0,0,0)$, and maximum density $=11/12$ (e.g. at $(0,0,1,1)$).
This matches the inclusion--exclusion value in the problem statement for the aligned choice.

5) VERIFICATION
- Lemma 1 is a standard periodicity reduction; the limit defining density exists because of periodicity.
- Lemma 2 uses CRT in a counting form; no probabilistic independence assumption is needed.
- Lemma 3 carefully distinguishes the compatible vs incompatible case via $\gcd(n_1,n_2)$.
- The computation is a finite check modulo $L$; it does not prove Question 2 in general.

6) FINAL
**UNRESOLVED**

(i) Strongest fully proved partial result:
Question 2 holds for $r=2$ (Lemma 3) and for pairwise coprime moduli (Lemma 2). Density computation reduces to counting residues modulo $L=\mathrm{lcm}(n_i)$ (Lemma 1).

(ii) First gap:
Prove for \emph{arbitrary} moduli $n_1<\dots<n_r$ that the minimum of $\mathbf{d}(U)$ over residue choices is attained when all $a_i$ are equal, i.e. establish Simpson’s general inequality in full.

(iii) Top 3 next moves:
1. Attempt a general “compression/shifting” argument on $\mathbb{Z}/L\mathbb{Z}$ showing unions of cosets of subgroups are minimized when the cosets have nonempty intersection.
2. Develop an exact formula for $\max_{(a_i)}\mathbf{d}(U)$ (Question 1), at least in terms of the divisor lattice of the $n_i$.
3. Extend the brute-force verification to larger random sets $A$ to look for extremal patterns that might guide the proof of Question 1/2.

(iv) Minimal counterexample structure (if Question 2 were false):
A smallest set of moduli $(n_i)$ and residue choice $(a_i)$ such that $\mathbf{d}(U)$ is \emph{smaller} than in the aligned case $(a_i\equiv a)$; this would require a delicate pattern of forced higher-order overlaps reducing the union beyond what alignment produces.

