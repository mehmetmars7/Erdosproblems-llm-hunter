% Attempt for Erdos Problem #14
% Following PROMPT_STRATEGY.MD
% Tools/Constraints:
% - Web browsing available? NO (restricted)
% - Computation available? YES (local Python), not used
% GPT 5.2

OUTPUT
1) FORMAL RESTATEMENT
Let $A\subseteq\mathbb N$. For each integer $n$, consider the number of representations
\[n=a+b,\qquad a,b\in A,\ a\le b.\]
Let $B$ be the set of $n$ for which there is exactly one such representation. Questions:
(a) Is it true that for all $\epsilon>0$ and large $N$,
$|\{1,\dots,N\}\setminus B|\gg_\epsilon N^{1/2-\epsilon}$?
(b) Can it happen that $|\{1,\dots,N\}\setminus B|=o(N^{1/2})$?

2) QUICK LITERATURE/CONTEXT CHECK
Browsing is not available. I only record what the problem statement itself claims.
I have not verified those results here.

3) ATTACK PLAN
Proof track:
1. Relate unique representation counts to additive energy of $A$.
2. Use sumset growth to force many collisions in $A+A$.

Disproof track:
1. Construct very thin $A$ (Sidon-type) to maximize uniqueness.
2. Construct dense $A$ to force many multiple representations and large complement.

Chosen path: prove elementary bounds and work out a concrete example.

4) WORK
Lemma 1 (Trivial bound via number of pairs).
Let $A_N=A\cap\{1,\dots,N\}$ and $m=|A_N|$. Then
\[|B\cap\{1,\dots,N\}|\le \binom{m+1}{2}.
\]

Proof.
Each unordered pair $(a,b)$ with $a\le b$ gives at most one integer $n=a+b$ with a unique representation.
There are $\binom{m+1}{2}$ such pairs. \qed

Lemma 2 (Unique sums for an interval).
Let $A=\{1,2,\dots,m\}$ with $m\ge 3$. Then the integers with a unique representation as $a+b$ with $a,b\in A$ are
exactly $2,3,2m-1,2m$.

Proof.
For $s\le m+1$, the number of pairs $(a,b)$ with $a\le b$ and $a+b=s$ equals $\lfloor s/2\rfloor$:
indeed $a$ can be any integer from $1$ to $\lfloor s/2\rfloor$, and then $b=s-a$ automatically satisfies $a\le b$ and $b\le m$.
Thus $s$ has a unique representation iff $\lfloor s/2\rfloor=1$, i.e., $s=2$ or $3$.
For $s>m+1$, the number of representations of $s$ equals that of $2m+2-s$ by the map $(a,b)\mapsto(m+1-b, m+1-a)$,
so unique sums are also $2m-1$ and $2m$. \qed

Corollary 3 (Large complement example).
For $A=\{1,\dots,m\}$ and $N\ge 2m$, we have
$|\{1,\dots,N\}\setminus B|\ge N-4$.

Proof.
By Lemma 2, $B\cap\{1,\dots,2m\}$ has size $4$. All other integers up to $N$ lie outside $B$. \qed

5) VERIFICATION
- Lemma 2 uses an explicit count of representations in an interval, which can be checked directly.
- Corollary 3 applies only to the specific choice $A=\{1,\dots,m\}$.

6) FINAL
**UNRESOLVED**

(i) Strongest fully proved partial result:
For any $A$, $|B\cap[1,N]|\le \binom{|A\cap[1,N]|+1}{2}$, and for $A=\{1,\dots,m\}$ we have $|B\cap[1,2m]|=4$.

(ii) First gap:
No general lower bound on $|\{1,\dots,N\}\setminus B|$ of order $N^{1/2}$ is proved.

(iii) Top 3 next moves:
1. Relate $|\{1,\dots,N\}\setminus B|$ to additive energy and use sumset estimates.
2. Study Sidon-type constructions where $A+A$ has mostly unique sums.
3. Optimize between dense and sparse $A$ to minimize the complement size.

(iv) Minimal counterexample structure (if the lower bound were false):
A family of sets $A$ whose sumset $A+A$ has unusually many unique representations, forcing
$|\{1,\dots,N\}\setminus B|=o(N^{1/2})$.
