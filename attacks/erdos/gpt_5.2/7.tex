
OUTPUT
1) FORMAL RESTATEMENT
A covering system is a finite collection of congruence classes
\[\{x\equiv a_i \pmod{n_i}\}_{i=1}^k\]
that covers all integers (every $x\in\mathbb Z$ satisfies at least one congruence). The question is:
Does there exist a covering system in which all moduli $n_i$ are odd? A stronger version asks all $n_i$ odd and squarefree.

2) QUICK LITERATURE/CONTEXT CHECK
Browsing is not available. I only record what the problem statement itself claims about known results.
I have not verified those results here.

3) ATTACK PLAN
Proof track:
1. Use density arguments to show odd moduli cannot cover all residues modulo a carefully chosen modulus.
2. Apply local obstructions modulo small primes (2,3,5) and use the Chinese remainder theorem.

Disproof track:
1. Try to build an explicit covering system with odd moduli using CRT and greedy coverage.
2. Search computationally over odd moduli with bounded lcm.

Chosen path: establish necessary density and modular constraints that any covering system must satisfy.

4) WORK
Lemma 1 (Density bound).
If congruence classes $x\equiv a_i \pmod{n_i}$ cover all integers, then
\[\sum_{i=1}^k \frac{1}{n_i} \ge 1.\]

Proof.
Each class has natural density $1/n_i$. The union of classes has density at most the sum of densities,
so $1\le \sum_i 1/n_i$. \qed

Lemma 2 (Reduction to a finite modulus).
Let $L=\operatorname{lcm}(n_1,\dots,n_k)$. The system is a covering system iff every residue class modulo $L$
obeys at least one congruence $x\equiv a_i\pmod{n_i}$.

Proof.
If all residue classes modulo $L$ are covered, then every integer is covered by periodicity.
Conversely, if some residue class modulo $L$ is uncovered, then all integers in that class are uncovered. \qed

Lemma 3 (Odd moduli give no parity restriction).
If all $n_i$ are odd, then each congruence class contains both even and odd integers.

Proof.
For odd $n_i$, the map $x\mapsto x+n_i$ toggles parity, so each residue class mod $n_i$ contains both parities. \qed

5) VERIFICATION
- Lemma 1 uses only the union bound for densities.
- Lemma 2 is a standard lcm-periodicity argument.
- Lemma 3 confirms there is no immediate parity obstruction from odd moduli alone.

6) FINAL
**UNRESOLVED**

(i) Strongest fully proved partial result:
Any covering system must satisfy $\sum_i 1/n_i\ge 1$, and the problem reduces to covering all residues modulo
$L=\mathrm{lcm}(n_i)$.

(ii) First gap:
No argument here rules out an odd-modulus covering system or constructs one.

(iii) Top 3 next moves:
1. Use Lemma 2 to search for a covering of $\mathbb Z/L\mathbb Z$ with all odd moduli and small $L$.
2. Investigate local obstructions modulo $3,5,7$ and attempt to force a modulus divisible by 2 or 3.
3. Encode the problem as an integer program using the density constraint from Lemma 1.

(iv) Minimal counterexample structure (if the conjecture were false):
A finite set of odd moduli whose lcm $L$ is highly composite (likely divisible by many odd primes) and a choice of
residues covering all $L$ classes while keeping $\sum 1/n_i$ just above 1.
