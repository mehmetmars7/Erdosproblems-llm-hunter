% Attempt for Erdos Problem #107
% Following PROMPT_STRATEGY.MD
% Tools/Constraints:
% - Web browsing available? NO (restricted)
% - Computation available? YES (local), not used
% GPT 5.2

OUTPUT
1) FORMAL RESTATEMENT
Definitions/Conventions.
- A finite set $P\subset\mathbb R^2$ is in \emph{general position} if no three points are collinear.
- For $n\ge 3$, a subset $Q\subseteq P$ of size $n$ \emph{forms the vertices of a convex $n$-gon} if the $n$ points of $Q$ are in convex position, i.e. every point of $Q$ is a vertex of $\mathrm{conv}(Q)$.
- Define $f(n)$ to be the least integer $N$ such that every set $P\subset\mathbb R^2$ of $N$ points in general position contains an $n$-point subset in convex position.

Conjecture (Erd\H{o}s--Szekeres).
Prove that for all $n\ge 3$,
\[f(n)=2^{n-2}+1.\]

2) QUICK LITERATURE/CONTEXT CHECK
Web browsing is not available in this session.
From the problem statement:
- The exact values $f(4)=5$ and $f(5)=9$ are known.
- Erd\H{o}s--Szekeres proved $2^{n-2}+1\le f(n)\le \binom{2n-4}{n-2}+1$.
- Stronger upper bounds are known, but the exact formula is open in general.
I do not use the deep upper bounds here.

3) ATTACK PLAN
Proof track ideas:
1. Try to prove the conjectured exact formula by understanding extremal configurations with no convex $n$-gon and showing they have size at most $2^{n-2}$.
2. For small $n$, prove exact values directly as a reality check and as a base for any induction attempt.

Disproof track ideas:
1. Try to construct, for some $n$, a set of size $2^{n-2}+1$ in general position with no convex $n$-gon. Any such construction would refute the conjecture.

Chosen path in this attempt: prove two basic, fully rigorous facts: monotonicity of $f(n)$, and the exact value $f(4)=5$ (the first nontrivial case).

4) WORK
Lemma 1 (Monotonicity).
For all $n\ge 3$, one has $f(n+1)\ge f(n)$.

Proof.
Let $N:=f(n+1)$. By definition, every set of $N$ points in general position contains a subset of $n+1$ points in convex position.
Any subset of vertices of a convex polygon is again in convex position (because removing vertices cannot create an interior point among the remaining ones).
Therefore, the same $N$ points also contain $n$ points in convex position.
Thus $N$ is admissible for $f(n)$, so $f(n)\le N=f(n+1)$. \qed

Lemma 2 (Exact value $f(4)=5$).
Every set of $5$ points in $\mathbb R^2$ in general position contains $4$ points in convex position, and there exists a set of $4$ points in general position with no $4$ points in convex position.
Hence $f(4)=5$.

Proof.
Lower bound $f(4)\ge 5$: take three noncollinear points $A,B,C$ and a fourth point $D$ strictly inside the triangle $\triangle ABC$.
Then $A,B,C,D$ are in general position, but $D$ is not a vertex of the convex hull of $\{A,B,C,D\}$, so these four points are not in convex position.
Thus $f(4)>4$, i.e. $f(4)\ge 5$.

Upper bound $f(4)\le 5$: let $P$ be any set of $5$ points in general position.
If at least $4$ points of $P$ lie on the boundary of $\mathrm{conv}(P)$, then those $4$ boundary points are in convex position and we are done.
So assume $\mathrm{conv}(P)$ has at most $3$ vertices, meaning at least one point of $P$ lies strictly inside the convex hull of the others.

Consider any straight-line triangulation of the point set $P$ (a planar subdivision of $\mathrm{conv}(P)$ into triangles whose vertices are points of $P$ and whose edges are noncrossing segments between points of $P$).
Such a triangulation exists for any finite set of points in general position.
Because $P$ is not all on the boundary of its convex hull, the triangulation has at least one \emph{interior} edge $ab$ (an edge not on the boundary of $\mathrm{conv}(P)$).
This interior edge is incident to exactly two triangles in the triangulation, say triangles $abc$ and $abd$ with $c\ne d$.

Since $ab$ is an interior edge, the triangles $abc$ and $abd$ lie on opposite sides of the line through $a$ and $b$.
Equivalently, the points $c$ and $d$ lie in opposite open half-planes determined by the line $ab$.
In particular, the segment $\overline{cd}$ intersects the segment $\overline{ab}$ at an interior point of both segments.
Whenever two segments $\overline{ab}$ and $\overline{cd}$ cross (and no three of $a,b,c,d$ are collinear), the four endpoints $a,b,c,d$ are in convex position: they are the vertices of a convex quadrilateral with diagonals $\overline{ab}$ and $\overline{cd}$.
Thus $\{a,b,c,d\}\subseteq P$ is a 4-point subset in convex position.

This proves every $5$-point general-position set contains a convex quadrilateral, i.e. $f(4)\le 5$.

Combining both bounds yields $f(4)=5$. \qed

FAST REALITY CHECK
- The conjectured formula gives $f(3)=2^{1}+1=3$ and $f(4)=2^{2}+1=5$, matching the trivial triangle case and Lemma 2.
- The statement reports $f(5)=2^{3}+1=9$, consistent with the conjecture but not proved here.

5) VERIFICATION
- Lemma 1 uses only the fact that subsets of vertices of a convex polygon remain in convex position.
- In Lemma 2, the key geometric step is: if a line through two interior points intersects two sides of a triangle, then two triangle vertices lie on opposite sides of that line; then crossing diagonals imply convex position of four points. No collinearity assumptions ensure the crossing criterion is valid.

6) FINAL
**UNRESOLVED**

(i) Strongest fully proved partial result:
I proved monotonicity of $f(n)$ (Lemma 1) and established the exact value $f(4)=5$ with a complete proof (Lemma 2).

(ii) First gap:
Prove $f(n)\le 2^{n-2}+1$ for general $n$ (or find a counterexample for some $n$).
Beyond $n=5$, known proofs require substantially deeper combinatorial geometry than the small-case argument in Lemma 2.

(iii) Top 3 next moves (concrete):
1. Prove the next open-looking case(s) directly (e.g. re-derive $f(5)=9$) as a stepping stone and to understand extremal configurations.
2. Study the structure of known lower-bound constructions of size $2^{n-2}$ with no convex $n$-gon (as mentioned in the statement) and attempt to show they are extremal.
3. Try to connect the conjecture to an induction on $n$ by identifying a canonical ``splitting vertex'' in a point set that reduces the problem to smaller $n$.

(iv) What a minimal counterexample would likely look like (structure/parameters):
For some $n\ge 6$, a set of $2^{n-2}+1$ points in general position with no $n$ points in convex position; such a configuration would have to be highly structured and likely resemble known extremal constructions for smaller $n$.
