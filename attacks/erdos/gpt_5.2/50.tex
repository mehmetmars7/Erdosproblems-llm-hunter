% Attempt for Erdos Problem #50
% Following PROMPT_STRATEGY.MD
% Tools/Constraints:
% - Web browsing available? NO (restricted)
% - Computation available? YES (local Python), used for sanity checks
% GPT 5.2

OUTPUT
1) FORMAL RESTATEMENT
Definitions/Conventions.
- $\phi(n)$ is Euler's totient function: $\phi(n)=|\{1\le k\le n:\gcd(k,n)=1\}|$.
- For a set $S\subset\mathbb N$, its (natural) density is
\[d(S):=\lim_{N\to\infty}\frac{|S\cap\{1,\dots,N\}|}{N},\]
provided the limit exists.

Given (Schoenberg): for each $c\in[0,1]$, the density
\[f(c):=d\big(\{n\in\mathbb N:\phi(n)<cn\}\big)\]
exists.

Question.
Is it true that there does not exist any $x\in(0,1)$ such that the derivative $f'(x)$ exists and $f'(x)>0$?

2) QUICK LITERATURE/CONTEXT CHECK
Web browsing is not available in this session.
From the problem text only:
- Schoenberg proved existence of the densities $f(c)$ for all $c\in[0,1]$.
- Erd\H{o}s proved the distribution function $f$ is purely singular.
I do not re-derive these results here.

3) ATTACK PLAN
Proof track ideas:
1. Understand the ``random variable'' $X(n):=\phi(n)/n=\prod_{p\mid n}(1-1/p)$ (prime-factor model) and try to show that
   the pushforward density measure is singular.
2. Strengthen singularity to the pointwise statement ``no positive finite derivative anywhere'' by showing that at every $x$,
   the increment $f(x+h)-f(x)$ is $o(h)$ as $h\downarrow 0$.
3. Try to isolate a Cantor-type structure in the set of possible values of $\phi(n)/n$ and relate it to the distribution.

Disproof track ideas:
1. Search numerically for an $x$ where finite differences suggest a stable positive slope.
2. If found, attempt to turn that into a rigorous lower bound $f(x+h)-f(x)\ge c_0h$ along a sequence $h\to 0$.

Chosen path in this attempt: prove two clean structural lemmas about $\phi(n)/n$ and the sets $\{n:\phi(n)<cn\}$, and run a finite-$N$ sanity check.

4) WORK
Lemma 1 (Prime-divisor formula and monotonicity under divisibility).
For every $n\ge 1$,
\[\frac{\phi(n)}{n}=\prod_{p\mid n}\left(1-\frac1p\right),\]
where the product is over distinct primes dividing $n$. In particular, if $m\mid n$ then
\[\frac{\phi(n)}{n}\le \frac{\phi(m)}{m}.\]

Proof.
Write the prime factorization $n=\prod_{i=1}^r p_i^{k_i}$ with distinct primes $p_i$.
It is standard (and follows from multiplicativity of $\phi$ and $\phi(p^k)=p^k-p^{k-1}$) that
\[\phi(n)=\prod_{i=1}^r \phi(p_i^{k_i})=\prod_{i=1}^r\big(p_i^{k_i}-p_i^{k_i-1}\big)=\prod_{i=1}^r p_i^{k_i}\left(1-\frac1{p_i}\right).\]
Dividing by $n=\prod_{i=1}^r p_i^{k_i}$ gives the claimed formula.

If $m\mid n$, then every prime dividing $m$ also divides $n$, so the set of primes in the product for $\phi(m)/m$
is a subset of the primes in the product for $\phi(n)/n$. Since each factor $(1-1/p)\in(0,1)$, adding more factors can only decrease the product.
Thus $\phi(n)/n\le \phi(m)/m$. \qed

Lemma 2 (Euler: $\sum_{p}1/p$ diverges).
The sum of reciprocals of the primes diverges:
\[\sum_{p\ \mathrm{prime}}\frac1p=+\infty.\]

Proof.
Let $H_N:=\sum_{n=1}^N \frac1n$ be the $N$th harmonic sum, which satisfies $H_N\to\infty$ as $N\to\infty$.
Consider the Euler product over primes $\le N$:
\[\prod_{p\le N}\left(1+\frac1p+\frac1{p^2}+\cdots\right)=\prod_{p\le N}\frac1{1-1/p}.\]
Expanding the product on the left produces a term $1/n$ for every integer $n$ whose prime factors are all $\le N$.
In particular, every $1\le n\le N$ has all prime factors $\le N$, so all terms $1/n$ for $n\le N$ appear in the expansion with coefficient $1$.
Therefore
\[H_N=\sum_{n=1}^N\frac1n \le \prod_{p\le N}\frac1{1-1/p}.\]
Taking logarithms gives
\[\log H_N \le \sum_{p\le N} -\log\left(1-\frac1p\right).\]
Using the inequality $-\log(1-x)\le x+x^2+x^3+\cdots=\frac{x}{1-x}$ valid for $0<x<1$, with $x=1/p$, yields
\[-\log\left(1-\frac1p\right)\le \frac{1/p}{1-1/p}=\frac1{p-1}\le \frac2p\qquad(p\ge 2).\]
Hence
\[\log H_N \le 2\sum_{p\le N}\frac1p.\]
Since $\log H_N\to\infty$, the right-hand side must also tend to $\infty$, which forces $\sum_{p\le N}1/p\to\infty$.
This proves divergence of $\sum_p 1/p$. \qed

Lemma 3 (For every $c>0$, the set $\{n:\phi(n)<cn\}$ has positive lower density).
Fix $c\in(0,1)$. Then there exists a squarefree integer $m\ge 1$ such that:
1. $\phi(m)<cm$, and
2. every multiple of $m$ lies in $\{n:\phi(n)<cn\}$.
In particular, the (lower) asymptotic density of $\{n:\phi(n)<cn\}$ is at least $1/m$.

Proof.
By Lemma 2, the partial sums $\sum_{p\le y}1/p$ can be made arbitrarily large. Choose $y$ such that
\[\sum_{p\le y}\frac1p > \log\left(\frac1c\right).\]
Let $m:=\prod_{p\le y} p$ (squarefree by construction). Then by Lemma 1,
\[\frac{\phi(m)}{m}=\prod_{p\le y}\left(1-\frac1p\right).\]
Using $\log(1-u)\le -u$ for $u\in(0,1)$ and exponentiating gives
\[\prod_{p\le y}\left(1-\frac1p\right)\le \exp\left(-\sum_{p\le y}\frac1p\right) < \exp\left(-\log\left(\frac1c\right)\right)=c.\]
Thus $\phi(m)/m < c$, i.e. $\phi(m)<cm$.

Now let $n$ be any multiple of $m$. Then $m\mid n$, so by Lemma 1,
\[\frac{\phi(n)}{n}\le \frac{\phi(m)}{m}<c,\]
equivalently $\phi(n)<cn$. Hence every multiple of $m$ lies in the desired set.
The set of multiples of $m$ has natural density $1/m$, so the lower density of $\{n:\phi(n)<cn\}$ is at least $1/m$. \qed

FAST REALITY CHECK (local computation; finite $N$).
I computed $\phi(n)/n$ for $1\le n\le 300{,}000$ and approximated
\[f_N(c):=\frac1N\left|\left\{1\le n\le N:\frac{\phi(n)}{n}<c\right\}\right|.\]
Sample values:
\[
\begin{array}{c|c}
c & f_{300000}(c)\\\hline
0.25 & 0.011453\\
0.35 & 0.178160\\
0.50 & 0.511133\\
0.70 & 0.678287\\
0.90 & 0.786890
\end{array}
\]
Finite-difference slopes (with $h=0.01$) suggested strong variation near $c=0.5$, but this is only exploratory and not stable evidence about $f'(x)$.

5) VERIFICATION
- Lemma 1: the formula for $\phi(n)$ and the prime-divisor product are standard and the monotonicity under divisibility follows directly.
- Lemma 2: the key inequality is $H_N \le \prod_{p\le N}(1-1/p)^{-1}$, justified because every $n\le N$ uses primes $\le N$.
  The bound $-\log(1-1/p)\le 2/p$ for $p\ge 2$ is valid.
- Lemma 3: the only nontrivial input is Lemma 2 and $\log(1-u)\le -u$.

6) FINAL
**UNRESOLVED**

(i) Strongest fully proved partial result:
I proved structural facts about the sets $\{n:\phi(n)<cn\}$:
the ratio $\phi(n)/n$ depends only on the prime divisors of $n$ and is monotone under divisibility (Lemma 1),
and for every $c>0$ the set $\{n:\phi(n)<cn\}$ contains a full arithmetic progression of multiples of some $m$ and hence has positive lower density (Lemma 3).

(ii) First gap:
Show (or refute) that for this particular distribution function $f$, there is no point $x$ where the derivative exists and is positive.
I do not know how to upgrade ``purely singular'' (as stated in the problem text) to this pointwise derivative exclusion.

(iii) Top 3 next moves (concrete):
1. Reconstruct Schoenberg's proof of existence of $f(c)$ and track any quantitative estimates on $f(x+h)-f(x)$.
2. Translate Erd\H{o}s' ``purely singular'' claim into a precise measure-theoretic statement for the limiting distribution of $\phi(n)/n$, then check whether it implies absence of points with positive finite derivative.
3. Use computation to look for candidate $x$ with stable positive slopes at shrinking $h$ (e.g. $h=10^{-2},10^{-3},10^{-4}$) and, if any appear, attempt to prove a matching lower bound on the density of $n$ with $\phi(n)/n\in[x,x+h]$.

(iv) What a minimal counterexample would likely look like (structure/parameters):
A point $x$ and a sequence $h_k\downarrow 0$ such that the set of $n$ with $\phi(n)/n\in[x,x+h_k)$ has density at least $c_0h_k$ for some fixed $c_0>0$,
creating a genuinely linear growth in $f$ near $x$.

