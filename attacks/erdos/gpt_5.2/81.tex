
OUTPUT
1) FORMAL RESTATEMENT
Definitions/Conventions.
- A \emph{clique} in a graph is a complete subgraph (on some vertex set $K$).
- An \emph{edge-partition into cliques} is a family of cliques $K_1,\dots,K_t$ such that the sets of edges $E(K_i)$ are pairwise disjoint and
\[E(G)=\bigsqcup_{i=1}^t E(K_i).\]
Let $\mathrm{cp}(G)$ be the minimum possible such $t$ (the clique-partition number).
- A graph $G$ is \emph{chordal} if it has no induced cycle of length $\ge 4$.
- A graph $G$ is \emph{split} if its vertices can be partitioned as $V=C\sqcup I$ where $C$ is a clique and $I$ is an independent set.
Every split graph is chordal.

Question.
Is it true that every chordal graph $G$ on $n$ vertices satisfies
\[\mathrm{cp}(G)\le \frac{n^2}{6}+O(n)?\]

2) QUICK LITERATURE/CONTEXT CHECK
Web browsing is not available in this session.
From the problem text only:
- EOZ93 proved an upper bound $(1/4-\varepsilon)n^2$ for chordal graphs.
- CEO94 proved $\frac{3}{16}n^2+O(n)$ for split graphs.
- The split example ``clique of size $n/3$ joined to independent set of size $2n/3$'' shows $\mathrm{cp}(G)\ge n^2/6+O(n)$ can be necessary.
I do not use any results beyond what I prove below.

3) ATTACK PLAN
Proof track ideas:
1. Understand the extremal split example and compute its clique-partition number exactly; this clarifies why $n^2/6$ appears.
2. Attempt to exploit chordal structure (perfect elimination orderings / clique trees) to build clique partitions near the split-graph lower bound.

Disproof track ideas:
1. Search for chordal graphs whose clique-partition number exceeds $n^2/6$ by a fixed fraction, potentially via dense overlap patterns of maximal cliques.

Chosen path in this attempt: completely analyze the split example from the statement, proving it indeed forces $\mathrm{cp}(G)=n^2/6+O(n)$, by matching lower and upper bounds.

4) WORK
Consider the following split graph $G=G(a,b)$:
- Vertex set $V=C\sqcup I$ with $|C|=a$, $|I|=b$.
- $C$ induces a clique (all edges inside $C$).
- $I$ induces an independent set (no edges inside $I$).
- All edges between $C$ and $I$ are present (complete bipartite between $C$ and $I$).

This is exactly the example described in the problem statement when $a=\lfloor n/3\rfloor$ and $b=\lceil 2n/3\rceil$.

Lemma 1 (Lower bound for $\mathrm{cp}(G(a,b))$).
For the split graph $G(a,b)$ defined above,
\[\mathrm{cp}(G(a,b))\ \ge\ ab-\binom{a}{2}.\]

Proof.
Let $K_1,\dots,K_t$ be any clique-partition of $E(G(a,b))$.
Since $I$ is independent, every clique $K_j$ contains at most one vertex from $I$.
Let $J$ be the set of indices $j$ such that $K_j$ contains a vertex of $I$.
For $j\in J$, write
\[s_j:=|K_j\cap C|\in\{1,2,\dots,a\}.\]
Then $K_j$ covers exactly $s_j$ cross-edges between its unique vertex in $I$ and the $s_j$ vertices of $C$.
Because the cross-edges between $C$ and $I$ form a complete bipartite graph with $ab$ edges and must be partitioned,
we have
\[\sum_{j\in J} s_j = ab.\tag{1}\]

Also, for each $j\in J$, the clique $K_j$ contains all edges inside $K_j\cap C$, contributing $\binom{s_j}{2}$ edges of the clique $C$.
Because the partition is edge-disjoint and the total number of edges inside $C$ is $\binom{a}{2}$, we have
\[\sum_{j\in J} \binom{s_j}{2} \ \le\ \binom{a}{2}.\tag{2}\]

Finally, note that for every integer $s\ge 1$,
\[s \le 1+\binom{s}{2},\]
since $\binom{s}{2}=0$ for $s=1$ and $\binom{s}{2}\ge s-1$ for $s\ge 2$.
Summing this inequality over $j\in J$ and using (1) gives
\[ab=\sum_{j\in J} s_j \le \sum_{j\in J}\left(1+\binom{s_j}{2}\right)=|J|+\sum_{j\in J}\binom{s_j}{2}\le |J|+\binom{a}{2}\]
by (2). Therefore
\[|J|\ge ab-\binom{a}{2}.\]
Since $t\ge |J|$, we conclude $\mathrm{cp}(G(a,b))\ge ab-\binom{a}{2}$. \qed

Lemma 2 (Matching upper bound when $b\ge a$).
Assume $b\ge a$. Then
\[\mathrm{cp}(G(a,b))\ \le\ ab-\binom{a}{2}.\]

Proof.
We explicitly construct a clique partition with exactly $ab-\binom{a}{2}$ cliques.

Step 1: partition the edges inside $C$ into matchings.
The complete graph $K_a$ on vertex set $C$ admits a proper edge-coloring with at most $a$ colors such that each color class is a matching:
- if $a$ is even, $K_a$ decomposes into $a-1$ perfect matchings;
- if $a$ is odd, $K_a$ decomposes into $a$ matchings (classical ``round-robin'' construction).
In all cases, we obtain matchings $M_1,\dots,M_r$ whose disjoint union is $E(C)$, with $r\le a$.

Since $b\ge a\ge r$, choose distinct vertices $i_1,\dots,i_r\in I$.

Step 2: use each matching edge to form a triangle clique.
For each $\ell\in\{1,\dots,r\}$ and each edge $\{u,v\}\in M_\ell$, define the clique (triangle)
\[K_{\ell,\{u,v\}}:=\{i_\ell,u,v\}.\]
This is a clique because $u,v\in C$ are adjacent and $i_\ell$ is adjacent to all of $C$.
Its edge set consists of exactly three edges: $uv$ (inside $C$) and the two cross edges $i_\ell u$ and $i_\ell v$.

Because the matchings are edge-disjoint, the edges $uv$ covered by these triangles are disjoint and cover all inside-$C$ edges.
Also, within a fixed matching $M_\ell$, the edges are vertex-disjoint, so the cross edges $i_\ell u$ and $i_\ell v$ used are all distinct.
Across different $\ell$, the $i_\ell$ are distinct, so cross edges also do not overlap.
Thus these triangle cliques are edge-disjoint.

Total triangles created: $|E(C)|=\binom{a}{2}$, and they cover:
- all $\binom{a}{2}$ edges of $C$,
- and $2\binom{a}{2}$ cross edges between $C$ and $I$.

Step 3: cover remaining cross edges individually.
There are $ab$ cross edges total, so the number of cross edges not yet covered is
\[ab-2\binom{a}{2}.\]
Cover each remaining cross edge by a 2-vertex clique (an edge).

Total cliques in this partition:
\[\binom{a}{2} + \big(ab-2\binom{a}{2}\big)=ab-\binom{a}{2}.\]
This proves $\mathrm{cp}(G(a,b))\le ab-\binom{a}{2}$ when $b\ge a$. \qed

Corollary 3 (The split example forces $n^2/6+O(n)$ cliques).
Let $n$ be divisible by $3$ and take $a=n/3$, $b=2n/3$. Then $G(a,b)$ is split (hence chordal) and
\[\mathrm{cp}(G(a,b)) = ab-\binom{a}{2} = \frac{n^2}{6}+\frac{n}{6}.\]
In particular, $\mathrm{cp}(G(a,b))\ge n^2/6+O(n)$, matching the statement's lower-bound example.

Proof.
Here $b=2n/3\ge a=n/3$, so Lemmas 1--2 give equality $\mathrm{cp}(G(a,b))=ab-\binom{a}{2}$.
Compute:
\[ab=\frac{n^2}{9}\cdot 2=\frac{2n^2}{9},\qquad \binom{a}{2}=\binom{n/3}{2}=\frac{n^2}{18}-\frac{n}{6}.\]
Subtracting yields $\frac{n^2}{6}+\frac{n}{6}$. \qed

FAST REALITY CHECK
The extremal split example is not only a lower-bound witness: Lemmas 1--2 show its clique-partition number can be computed exactly, and equals $n^2/6+O(n)$.

5) VERIFICATION
- Lemma 1 uses only: (i) each clique hits at most one vertex of $I$; (ii) disjointness of inside-$C$ edges; and (iii) the inequality $s\le 1+\binom{s}{2}$.
- Lemma 2 relies on existence of a proper edge-coloring of $K_a$ into matchings; this is standard and easy to construct explicitly (round-robin).
- Corollary 3 checks arithmetic and shows the constant $1/6$ arises from ``cross edges minus inside edges''.

6) FINAL
**UNRESOLVED**

(i) Strongest fully proved partial result:
I proved that the canonical split example in the statement has
\[\mathrm{cp}(G)=\frac{n^2}{6}+O(n)\]
indeed, exactly $\frac{n^2}{6}+\frac{n}{6}$ when $n$ is divisible by $3$ (Corollary 3).

(ii) First gap:
Extend from this split-graph analysis to \emph{all} chordal graphs: prove or refute that every chordal $n$-vertex graph satisfies
$\mathrm{cp}(G)\le n^2/6+O(n)$.

(iii) Top 3 next moves (concrete):
1. Use chordal structure (perfect elimination order / clique tree) to decompose the edge set into a controlled number of split-like pieces and attempt to apply Lemma 1-style accounting.
2. Search for chordal graphs with many overlapping maximal cliques and attempt to lower bound $\mathrm{cp}(G)$ above $n^2/6$ by adapting the ``cross edges vs inside edges'' argument.
3. For special subclasses (interval graphs, block graphs, split graphs), attempt to compute $\mathrm{cp}(G)$ up to $O(n)$ and see whether $n^2/6$ is the correct universal leading term.

(iv) What a minimal counterexample would likely look like (structure/parameters):
A chordal graph formed by gluing many large cliques along large overlaps, arranged so that any clique partition is forced to use many cliques because edges are ``shared'' between different maximal cliques in a way that blocks the efficient triangle-based packing used in Lemma 2.

