
OUTPUT
1) FORMAL RESTATEMENT
Let $a/b\in\mathbb{Q}_{>0}$ be written in lowest terms with $b$ squarefree.
Question: does there exist an integer $k\ge 1$ and integers
\[1<n_1<\cdots<n_k\]
such that each $n_i$ is the product of two \emph{distinct} primes and
\[\frac{a}{b}=\sum_{i=1}^k \frac1{n_i}?\]

2) QUICK LITERATURE/CONTEXT CHECK
Browsing is not available. The statement mentions a proved analogue when denominators have three distinct prime factors (for $b=1$). I do not use this result.

3) ATTACK PLAN
Proof track:
1. Identify necessary conditions forced by the restriction “each $n_i$ is a product of two distinct primes”.
2. Find explicit families of representable rationals to sanity-check feasibility.
3. Try to build a general constructive scheme for arbitrary squarefree $b$ (likely by combining small identities).

Disproof track:
1. Attempt to find a squarefree $b$ and rational $a/b$ that cannot be represented, via local congruence/denominator obstructions.

Chosen path: prove a clean necessary condition (squarefree denominator) and provide explicit infinite families of representable rationals; general case remains open.

4) WORK
Lemma 1 (Squarefree denominator is necessary).
Let $n_1,\dots,n_k$ be squarefree integers. Then the reduced denominator of $\sum_{i=1}^k 1/n_i$ is squarefree.
In particular, if each $n_i$ is the product of two distinct primes, then any representable rational $a/b$ must have squarefree $b$.

Proof.
Let $L=\mathrm{lcm}(n_1,\dots,n_k)$. Since each $n_i$ is squarefree, $L$ is squarefree.
Write
\[\sum_{i=1}^k \frac1{n_i}=\frac{1}{L}\sum_{i=1}^k \frac{L}{n_i}=\frac{A}{L}\]
with $A\in\mathbb{Z}$. Reducing $A/L$ can only cancel prime factors of $L$; since $L$ is squarefree, the reduced denominator remains squarefree. \qed

Lemma 2 (A 3-term semiprime identity from three primes).
Let $p,q,r$ be distinct primes. Then
\[\frac{1}{pq}+\frac{1}{pr}+\frac{1}{qr}=\frac{p+q+r}{pqr}.\]
In particular, the rational $(p+q+r)/(pqr)$ has an Egyptian fraction decomposition using three denominators, each a product of two distinct primes.

Proof.
Bring to the common denominator $pqr$:
\[\frac{1}{pq}+\frac{1}{pr}+\frac{1}{qr}=\frac{r}{pqr}+\frac{q}{pqr}+\frac{p}{pqr}=\frac{p+q+r}{pqr}.\qed\]

Example.
Taking $(p,q,r)=(2,3,5)$ gives
\[\frac{1}{6}+\frac{1}{10}+\frac{1}{15}=\frac{2+3+5}{30}=\frac{10}{30}=\frac{1}{3}.\]

FAST REALITY CHECK (computation).
Searching among semiprimes $n=pq\le 200$ (with $p\ne q$ prime) and allowing up to $6$ terms:
- Found $1/3 = 1/6+1/10+1/15$.
- Found no representation of $1/2$, $2/3$, or $1$ using $\le 6$ such denominators $\le 200$ (this is only a bounded search).

5) VERIFICATION
- Lemma 1 uses only that $\mathrm{lcm}$ of squarefree integers is squarefree.
- Lemma 2 is an exact identity; denominators are pairwise products of distinct primes as required.
- The computational search is not evidence of impossibility, only a small sanity check.

6) FINAL
**UNRESOLVED**

(i) Strongest fully proved partial result:
Any sum of reciprocals of squarefree denominators has squarefree reduced denominator (Lemma 1), and there is an explicit infinite family of representable rationals $(p+q+r)/(pqr)$ via three semiprime denominators (Lemma 2), including $1/3=1/6+1/10+1/15$.

(ii) First gap:
Given an \emph{arbitrary} reduced fraction $a/b$ with $b$ squarefree, construct (or rule out) a decomposition into reciprocals of products of two distinct primes.

(iii) Top 3 next moves:
1. Develop “local moves” that preserve the semiprime property while refining denominators (analogues of the standard splitting $1/n=1/(n+1)+1/(n(n+1))$, which does not preserve semiprimes).
2. For each squarefree $b$, computationally test representability of all $a/b$ for small $b$ with a search over semiprimes up to a large cutoff, to guess obstructions or universal constructions.
3. Try to represent $1/b$ first, then scale/patch to get general $a/b$ (since sums add), perhaps via identities like Lemma 2 glued together.

(iv) Minimal counterexample structure (if the answer were “no”):
A smallest squarefree $b$ and some $a$ with $\gcd(a,b)=1$ such that $a/b$ cannot be written as a finite sum of reciprocals of semiprimes with two distinct prime factors; any obstruction would have to go beyond the squarefree-denominator condition (Lemma 1).

