% Attempt for Erdos Problem #86
% Following PROMPT_STRATEGY.MD
% Tools/Constraints:
% - Web browsing available? NO (restricted)
% - Computation available? YES (local Python), used for tiny $n$
% GPT 5.2

OUTPUT
1) FORMAL RESTATEMENT
Definitions/Conventions.
- The $n$-dimensional hypercube graph $Q_n$ has vertex set $\{0,1\}^n$ and an edge between two vertices iff they differ in exactly one coordinate.
- Let $f(n)$ be the maximum number of edges in a $C_4$-free subgraph of $Q_n$.

Conjecture.
Prove that
\[f(n)\le \left(\frac12+o(1)\right)\,|E(Q_n)|=\left(\frac12+o(1)\right)n2^{n-1}.\]

2) QUICK LITERATURE/CONTEXT CHECK
Web browsing is not available in this session.
The statement lists several known bounds; I do not re-prove them.
I will prove only a simple general upper bound ($3/4$ of edges) from first principles.

3) ATTACK PLAN
Proof track ideas:
1. Use the fact that every $C_4$ in $Q_n$ is exactly a 2-dimensional face; translate $C_4$-freeness into a per-face constraint and double count.
2. Seek sharper inequalities by coupling constraints across overlapping faces, beyond the naive ``at most 3 edges per face''.

Disproof track ideas:
1. Attempt to construct $C_4$-free subgraphs with edge density strictly larger than $1/2$ asymptotically.
The problem text suggests such constructions exist up to $1/2+c/\sqrt n$ but not more.

Chosen path in this attempt: prove two exact combinatorial identities for squares in $Q_n$ and derive a clean $3/4$ upper bound for $f(n)$; include a small-$n$ check where this bound is tight.

4) WORK
Lemma 1 (Every $C_4$ in $Q_n$ is a 2-face, and counts of faces/containment).
In $Q_n$:
1. Every 4-cycle is contained in a unique 2-dimensional face (specified by choosing two coordinates to vary and fixing the other $n-2$ coordinates).
2. The number of 2-faces (and hence the number of $C_4$ subgraphs) is
\[\binom{n}{2}\,2^{n-2}.\]
3. Every edge of $Q_n$ lies in exactly $n-1$ distinct 2-faces.

Proof.
(1) A length-4 cycle in the hypercube flips coordinates along its edges. Returning to the start after 4 steps means each coordinate is flipped an even number of times.
In a simple 4-cycle, exactly two coordinates are flipped twice (and all others zero times), so the cycle lies in the subcube where only those two coordinates vary, i.e. a 2-face.
Uniqueness follows because the set of coordinates flipped along the cycle is determined by the cycle.

(2) Choose an unordered pair of coordinates $\{i,j\}$ to vary ($\binom{n}{2}$ choices), and choose values of the remaining $n-2$ coordinates ($2^{n-2}$ choices). This determines a unique 2-face with 4 vertices and 4 edges, i.e. one $C_4$.

(3) Fix an edge corresponding to flipping coordinate $i$. To specify a 2-face containing it, choose a second coordinate $j\ne i$ to vary; there are $n-1$ choices, and each yields a distinct face. \qed

Lemma 2 (A $3/4$ upper bound by double counting faces).
Let $H$ be a $C_4$-free subgraph of $Q_n$ with $e(H)$ edges. Then
\[e(H)\le \frac34\, n2^{n-1}.\]

Proof.
By Lemma 1(1), a $C_4$ in $Q_n$ is exactly the event that all 4 edges of some 2-face are present.
So if $H$ is $C_4$-free, then in every 2-face, $H$ contains at most $3$ of the $4$ edges.

Count incidences between edges of $H$ and 2-faces of $Q_n$ that contain them.
By Lemma 1(3), each edge lies in exactly $n-1$ faces, so the number of incidences equals $(n-1)e(H)$.
On the other hand, by the per-face bound, each face contributes at most $3$ incidences.
There are $\binom{n}{2}2^{n-2}$ faces by Lemma 1(2), so total incidences are at most $3\binom{n}{2}2^{n-2}$.
Hence
\[(n-1)e(H)\le 3\binom{n}{2}2^{n-2} = 3\cdot \frac{n(n-1)}{2}\cdot 2^{n-2}.\]
Cancel $n-1$ to obtain $e(H)\le \frac{3n}{2}2^{n-2}=\frac34 n2^{n-1}$. \qed

FAST REALITY CHECK (exact for $n=2,3$).
By exhaustive search:
- For $Q_2$ (4 edges, one square), the maximum $C_4$-free subgraph has 3 edges, i.e. fraction $3/4$.
- For $Q_3$ (12 edges, 6 squares), the maximum $C_4$-free subgraph has 9 edges, again fraction $3/4$.
So Lemma 2 is tight for small $n$, but it is far from the conjectured asymptotic $1/2$.

5) VERIFICATION
- Lemma 1(1) relies on the ``coordinate-flip parity'' characterization of walks in the hypercube.
- Lemma 2 is a clean incidence double count: each selected edge meets $n-1$ faces; each face can contribute at most 3 selected edges.
- The small-$n$ computation confirms correctness and shows the bound can be attained for $n=2,3$.

6) FINAL
**UNRESOLVED**

(i) Strongest fully proved partial result:
I proved a universal upper bound
\[f(n)\le \frac34\,n2^{n-1}\]
for all $n$ (Lemma 2), via face double counting.

(ii) First gap:
Improve the constant from $3/4$ down toward the conjectured $1/2$.
This requires exploiting dependencies between overlapping 2-faces, not just the per-face constraint.

(iii) Top 3 next moves (concrete):
1. Strengthen the face-counting argument by showing that if many faces have 3 edges present, then some $C_4$ must appear due to overlap constraints.
2. Search for a linear-algebraic formulation (treating edges as variables and squares as constraints) to prove a global density bound for $C_4$-free subgraphs.
3. Compute exact $f(n)$ for $n=4,5$ via ILP/SAT to see whether the maximum density begins to drift toward $1/2$ beyond the small-$n$ tightness at $3/4$.

(iv) What a minimal counterexample would likely look like (structure/parameters):
A family of $C_4$-free subgraphs of $Q_n$ with edge density bounded away from $1/2$ (say $>0.55$) for infinitely many $n$,
forcing $f(n)$ to exceed $(1/2+o(1))n2^{n-1}$.

