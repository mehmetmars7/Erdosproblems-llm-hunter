
OUTPUT
1) FORMAL RESTATEMENT
Fix $k\ge 3$. For each prime $p$ choose a residue class $a_p\in\{0,1,\dots,p-1\}$.
Define the covered set
\[C=\{n\in\mathbb{Z}: \exists\ \text{prime }p,\ \exists t\in\mathbb{Z},\ t\ge k,\ \text{such that }n=a_p+tp\}.\]
Equivalently, for each prime $p$ we cover all integers $n\equiv a_p\pmod p$ with $n\ge a_p+kp$.

Question: can one choose $(a_p)_{p\ \mathrm{prime}}$ so that $\mathbb{Z}\setminus C$ is finite (i.e. all sufficiently large integers are covered)?

2) QUICK LITERATURE/CONTEXT CHECK
Browsing is not available. I only record what the problem statement itself claims:
for $k=1$ or $k=2$ the statement holds for any $A$ with $\sum_{n\in A}1/n=\infty$; and a more general “prime-replaced-by-$A$” version may hold under additional density/divergence conditions. I do not verify those claims here.

3) ATTACK PLAN
Proof track:
1. Separate composites vs primes: show composites are easy to cover for any fixed $k$; the real difficulty is covering primes.
2. Try to build $(a_p)$ to target primes via congruences $q\equiv a_p\pmod p$ with $p\le q/k$.
3. Use probabilistic choices of residues with a second-moment/Borel--Cantelli style argument (if feasible) to get cofinite coverage.

Disproof track:
1. For a given selection rule $(a_p)$, attempt to construct infinitely many uncovered integers (in particular primes) by Chinese remainder constraints.

Chosen path: prove the “composites are easy” reduction rigorously; the prime-covering core remains open.

4) WORK
Lemma 1 (Trivial case $k=1$ for primes).
For $k=1$, choosing $a_p\equiv 0\pmod p$ for every prime $p$ covers every integer $n\ge 2$.

Proof.
If $n\ge 2$ is composite, pick a prime divisor $p\mid n$. Then $n=0+(n/p)\,p$ with $t=n/p\ge 1$.
If $n\ge 2$ is prime, take $p=n$; then $n=0+1\cdot n$ with $t=1\ge 1$.
Thus every $n\ge 2$ is of the form $a_p+tp$ with $t\ge 1$ under this choice. \qed

Lemma 2 (For general $k$, composites are eventually covered by $a_p\equiv 0$).
Fix $k\ge 2$ and choose $a_p\equiv 0\pmod p$ for every prime $p$.
Then every composite integer $n\ge k^2$ is covered, i.e. can be written as $0+tp$ with a prime $p$ and $t\ge k$.

Proof.
Let $n\ge k^2$ be composite, and let $p$ be its smallest prime factor. Then $p\le \sqrt{n}$, so
\[t:=\frac{n}{p}\ge \sqrt{n}\ge k.\]
Since $p$ is prime and $n=0+tp$, the integer $n$ is covered with this $p$ and $t\ge k$. \qed

Corollary 3 (Reduction to covering primes).
For any fixed $k\ge 2$, with the choice $a_p\equiv 0$ all but finitely many \emph{composite} integers are covered.
Thus, for $k\ge 3$, the main difficulty is to cover all sufficiently large \emph{primes} $q$ by congruences
\[q\equiv a_p\pmod p\quad\text{with some prime }p\le q/k.\]

FAST REALITY CHECK (computation, $k=3$ and $a_p\equiv 0$).
With $k=3$ and the choice $a_p\equiv 0$ for all primes, I checked all $n\le 200$:
the uncovered integers are
\[[1,2,3,4,5,7,11,13,17,19,23,29,31,37,41,43,47,53,59,61,67,71,73,79,83,89,97,101,103,107,109,113,127,131,137,139,149,151,157,163,167,173,179,181,191,193,197,199],\]
and every composite $n\ge 9$ is covered (no exceptions up to $200$), consistent with Lemma 2.

5) VERIFICATION
- Lemma 2 uses only the inequality “smallest prime factor $\le \sqrt n$”.
- Note that for primes $q$, the choice $a_p\equiv 0$ never represents $q$ with $t\ge 2$ (since $q$ has no smaller prime divisor), so the reduction in Corollary 3 is genuine.
- Edge cases: Lemma 2 only claims coverage for $n\ge k^2$; finitely many smaller composites may fail.

6) FINAL
**UNRESOLVED**

(i) Strongest fully proved partial result:
With $a_p\equiv 0$ for all primes, every composite $n\ge k^2$ is covered for any fixed $k\ge 2$ (Lemma 2). Hence any obstruction for $k\ge 3$ must come from primes.

(ii) First gap:
Construct $(a_p)$ so that every sufficiently large \emph{prime} $q$ satisfies $q\equiv a_p\pmod p$ for some prime $p\le q/k$ (equivalently $q=a_p+tp$ with $t\ge k$).

(iii) Top 3 next moves:
1. For $k=3$, attempt an explicit residue rule (e.g. via targeting $q$ using a divisor of $q-1$ or $q+1$) and test coverage of primes computationally up to a large bound.
2. Try a probabilistic method: choose $a_p$ uniformly at random and study whether the uncovered primes are almost surely finite (a strong Borel--Cantelli statement).
3. For a given partial choice of $(a_p)$ on small primes, use CRT to search for structured infinite families of primes forced to avoid all chosen classes, as a potential disproof mechanism.

(iv) Minimal counterexample structure (if the statement were false):
An infinite set of primes $q$ that avoid all selected residue classes $a_p\pmod p$ for primes $p\le q/k$; any such obstruction would have to be “self-sieving” across many moduli simultaneously.
