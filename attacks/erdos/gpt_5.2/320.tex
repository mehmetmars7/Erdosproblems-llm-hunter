
OUTPUT
1) FORMAL RESTATEMENT
For $N\ge 1$, define
\[S(N)=\left|\left\{\sum_{n\in A}\frac1n:\ A\subseteq\{1,\dots,N\}\right\}\right|,\]
the number of distinct rational values attained by sums of reciprocals of subsets of $\{1,\dots,N\}$.
Question: estimate the growth of $S(N)$.

2) QUICK LITERATURE/CONTEXT CHECK
Browsing is not available. The problem statement lists sophisticated iterated-log bounds from the literature; I do not rederive them. I prove basic structural bounds and compute exact $S(N)$ for small $N$.

3) ATTACK PLAN
Proof track:
1. Give easy upper bounds from the common denominator $L_N=\mathrm{lcm}(1,\dots,N)$ and the range $[0,H_N]$.
2. Give a clean exponential-in-$\pi(N)$ lower bound by restricting to subsets of primes and showing their reciprocal sums are all distinct.
3. Compute exact $S(N)$ for small $N$ to see collisions and growth.

Disproof track:
1. Not applicable; the task is to estimate growth, not a yes/no statement.

Chosen path: elementary bounds + computation.

4) WORK
Lemma 1 (Prime-subset sums are all distinct).
Let $P_N$ be the set of primes $\le N$. Then the map
\[A\subseteq P_N\ \longmapsto\ \sum_{p\in A}\frac1p\]
is injective. Consequently,
\[S(N)\ge 2^{\pi(N)}.\]

Proof.
Suppose $A,B\subseteq P_N$ and $\sum_{p\in A}1/p=\sum_{p\in B}1/p$.
Let $p_0$ be the smallest prime in the symmetric difference $A\triangle B$ (so $p_0\in A\setminus B$ or vice versa).
Let $D=\prod_{p\in A\cup B} p$.
Multiply the equality by $D$ to get
\[\sum_{p\in A}\frac{D}{p}=\sum_{p\in B}\frac{D}{p}.\]
Reduce modulo $p_0$.
For any prime $p\ne p_0$, $D/p$ is divisible by $p_0$ (since $p_0\mid D$ and $p\ne p_0$), hence contributes $0$ modulo $p_0$.
The lone term $D/p_0$ appears on exactly one side (because $p_0$ is in exactly one of $A,B$), and $D/p_0$ is not divisible by $p_0$.
Thus the two sides cannot be congruent modulo $p_0$, contradiction.
Therefore $A=B$, proving injectivity and yielding $2^{\pi(N)}$ distinct sums. \qed

Lemma 2 (A crude upper bound via $L_N$).
Let $L_N=\mathrm{lcm}(1,2,\dots,N)$ and $H_N=\sum_{n=1}^N 1/n$.
Then every subset sum $\sum_{n\in A}1/n$ can be written as $m/L_N$ with an integer $m\in[0,H_N L_N]$.
Hence
\[S(N)\le \lfloor H_N L_N\rfloor+1.\]

Proof.
As in Problem \#311 Lemma 1:
\[\sum_{n\in A}\frac1n=\frac{1}{L_N}\sum_{n\in A}\frac{L_N}{n}=\frac{m}{L_N}\]
with $m\in\mathbb{Z}_{\ge 0}$. Also $\sum_{n\in A}1/n\le H_N$, so $m\le H_N L_N$.
There are at most $\lfloor H_N L_N\rfloor+1$ integers $m$ in this range. \qed

FAST REALITY CHECK (computation).
Exact values of $S(N)$ for $1\le N\le 15$:
\[2,4,8,16,32,52,104,208,416,832,1664,1856,3712,7424,9664.\]
Collisions begin at $N=6$ (since $S(6)=52<2^6$), reflecting identities like $1=1/2+1/3+1/6$.

5) VERIFICATION
- Lemma 1 is a clean modular uniqueness argument using the smallest differing prime.
- Lemma 2 is purely a denominator/range count and is very weak asymptotically but correct.
- The computation is exact for $N\le 15$ and shows the first collisions at $N=6$.

6) FINAL
**UNRESOLVED**

(i) Strongest fully proved partial result:
$S(N)\ge 2^{\pi(N)}$ by injectivity on prime subsets (Lemma 1), and $S(N)\le \lfloor H_N L_N\rfloor+1$ with $L_N=\mathrm{lcm}(1,\dots,N)$ (Lemma 2). Exact $S(N)$ for $N\le 15$ is listed in the reality check.

(ii) First gap:
Bridge the gap between the elementary exponential-in-$\pi(N)$ lower bound and meaningful upper bounds, toward the iterated-log-type growth rates suggested by the literature bounds in the statement.

(iii) Top 3 next moves:
1. Prove stronger lower bounds by selecting structured subfamilies of denominators with provably distinct subset sums (beyond just primes).
2. Develop collision-counting upper bounds: quantify how many equalities $\sum_{n\in A}1/n=\sum_{n\in B}1/n$ can occur, perhaps via denominator size control.
3. Extend exact computation of $S(N)$ to larger $N$ using integer numerators mod $L_N$ to observe growth and collision structure.

(iv) Minimal counterexample structure (if a conjectured lower bound were false):
Large families of distinct subsets $A\subseteq[1,N]$ producing the same rational sum, implying massive collision structure beyond those coming from classical Egyptian fraction identities.

