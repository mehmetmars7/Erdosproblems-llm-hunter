
OUTPUT
1) FORMAL RESTATEMENT
Let $A\subseteq\mathbb N$ be infinite and satisfy property (P):
there do not exist distinct $a,b,c\in A$ with $b,c>a$ and $a\mid(b+c)$.
Questions:
(a) Does there exist such an $A$ with $\liminf_{N\to\infty} |A\cap[1,N]|/N^{1/2} > 0$?
(b) Does there exist $c>0$ such that for every such $A$ there are infinitely many $N$ with
$|A\cap[1,N]|<N^{1-c}$?
(c) Must $\sum_{n\in A}1/n$ converge?

2) QUICK LITERATURE/CONTEXT CHECK
Browsing is not available. I only record what the problem statement itself claims.
I have not verified those results here.

3) ATTACK PLAN
Proof track:
1. Translate property (P) into restrictions on residues modulo each $a\in A$.
2. Use these restrictions to bound local densities of $A$ in intervals.

Disproof track:
1. Construct dense examples via Chinese remainder theorem and rapidly growing blocks.
2. Explore sets of squares or prime powers to avoid divisibility conditions.

Chosen path: derive basic residue-class restrictions implied by (P).

4) WORK
Lemma 1 (Residue class restriction).
Fix $a\in A$. For any residue class $r\pmod a$, if $A$ contains an element $b> a$ with $b\equiv r\pmod a$
then $A$ contains no element $c> a$ with $c\equiv -r\pmod a$.

Proof.
If such $b,c$ existed, then $b+c\equiv r+(-r)\equiv 0\pmod a$, so $a\mid (b+c)$, contradicting property (P). \qed

Lemma 2 (At most one large multiple of $a$).
Fix $a\in A$. Then there is at most one element of $A$ greater than $a$ that is divisible by $a$.

Proof.
If $b> a$ and $c> a$ are distinct multiples of $a$, then $a\mid (b+c)$, contradicting (P). \qed

Corollary 3 (Local density constraint modulo $a$).
Fix $a\in A$ odd. Among the $a$ residue classes modulo $a$, $A\cap(a,\infty)$ can occupy at most $(a-1)/2$ nonzero
classes (one from each pair $\{r,-r\}$) and at most one element in the $0$ class.

Proof.
Pairs $\{r,-r\}$ are disjoint for $r\ne 0$ when $a$ is odd; apply Lemma 1 and Lemma 2. \qed

5) VERIFICATION
- Lemma 1 directly encodes the prohibition $a\mid(b+c)$.
- Lemma 2 is the special case $r=0$.
- Corollary 3 is a counting statement about residue classes, not a global density bound.

6) FINAL
**UNRESOLVED**

(i) Strongest fully proved partial result:
Property (P) forces strong residue-class exclusions modulo each $a\in A$ (Lemma 1 and Corollary 3).

(ii) First gap:
These residue restrictions do not yet yield any global density bounds such as $|A\cap[1,N]|\ll N^{1-c}$.

(iii) Top 3 next moves:
1. Quantify how the residue restrictions for multiple $a\in A$ interact to limit density.
2. Build explicit constructions via CRT to test lower bounds on $|A\cap[1,N]|$.
3. Analyze the reciprocal sum $\sum_{n\in A}1/n$ via partial summation and local density constraints.

(iv) Minimal counterexample structure (if the conjectured bounds were false):
A set $A$ with many elements but arranged so that for each $a\in A$, all larger elements of $A$ lie in a single
half-system of residues modulo $a$, avoiding opposite residue pairs.
