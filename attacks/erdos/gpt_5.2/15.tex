
OUTPUT
1) FORMAL RESTATEMENT
Let $p_n$ be the $n$-th prime. The main question is whether the alternating series
\[\sum_{n=1}^\infty (-1)^n \frac{n}{p_n}\]
converges. Additional conjectures in the statement concern two alternating series involving $p_{n+1}-p_n$.

2) QUICK LITERATURE/CONTEXT CHECK
Browsing is not available. I only record what the problem statement itself claims.
I have not verified those results here.

3) ATTACK PLAN
Proof track:
1. Show $n/p_n$ is eventually decreasing to $0$, then apply the alternating series test.
2. Use analytic information on $p_n$ (e.g., PNT) to justify monotonicity.

Disproof track:
1. Seek oscillations in $n/p_n$ that prevent convergence of the alternating sum.

Chosen path: state and prove a conditional convergence result via the alternating series test.

4) WORK
Lemma 1 (Alternating series test).
If $b_n\ge 0$ is eventually decreasing and $b_n\to 0$, then the series $\sum_{n\ge 1} (-1)^n b_n$ converges.

Proof.
This is the Leibniz alternating series test: partial sums form a Cauchy sequence by monotone alternating bounds. \qed

Lemma 2 (Conditional convergence criterion for the Erd\H{o}s series).
If $n/p_n$ is eventually decreasing and tends to $0$, then
\[\sum_{n=1}^\infty (-1)^n \frac{n}{p_n}\]
converges.

Proof.
Apply Lemma 1 with $b_n=n/p_n$. \qed

Lemma 3 (Bounded gaps force divergence of the positive series).
If there exist infinitely many $n$ with $p_{n+1}-p_n\le B$ for some fixed $B$, then
\[\sum_{n=1}^\infty \frac{1}{p_{n+1}-p_n}\]
diverges.

Proof.
Infinitely many terms are at least $1/B$, so the series of positive terms cannot converge. \qed

5) VERIFICATION
- Lemma 1 is standard and requires only monotonicity and $b_n\to 0$.
- Lemma 2 is conditional and does not assert the required monotonicity.
- Lemma 3 uses only a comparison test with a divergent sum of constants.

6) FINAL
**UNRESOLVED**

(i) Strongest fully proved partial result:
If $n/p_n$ is eventually decreasing to $0$, then the main alternating series converges (Lemma 2).

(ii) First gap:
No unconditional proof that $n/p_n$ decreases to $0$ (or enough regularity to apply Dirichlet-type tests).

(iii) Top 3 next moves:
1. Prove quantitative bounds on $p_n$ strong enough to show $n/p_n\to 0$ and eventual monotonicity.
2. Explore summation by parts using information on $\pi(x)$ or prime gaps.
3. Investigate conditional results under PNT or Hardy-Littlewood conjectures.

(iv) Minimal counterexample structure (if convergence were false):
A sequence where $n/p_n$ does not tend to $0$ or has oscillations preventing alternating cancellation,
forcing partial sums to diverge or fail to converge.
