% Attempt for Erdos Problem #274
% Following PROMPT_STRATEGY.MD
% Tools/Constraints:
% - Web browsing available? NO (restricted)
% - Computation available? YES (local Python), used
% GPT 5.2

OUTPUT
1) FORMAL RESTATEMENT
Let $G$ be a group. An \emph{exact covering} (or \emph{partition}) of $G$ by cosets is a finite family of left cosets
\[a_1G_1,\dots,a_kG_k\qquad (G_i\le G,\ a_i\in G)\]
such that these cosets are pairwise disjoint and their union is $G$.

The \emph{Herzog--Sch\"onheim conjecture} (as stated here) is:
if the indices $[G:G_i]$ are all distinct, then $\{a_iG_i\}_{i=1}^k$ cannot form a partition of $G$.

The original question in the first sentence can be read as asking whether such a partition can exist with cosets
of different cardinalities $|a_iG_i|=|G_i|$ (in the finite case this is equivalent to having different indices).
We exclude the trivial $k=1$ case.

2) QUICK LITERATURE/CONTEXT CHECK
Browsing is not available. I only record what the problem statement itself claims:
the conjecture is proved when all $G_i$ are subnormal, hence in particular for abelian groups; and it is verified
for all groups of size $<1440$. I have not checked these results here.

3) ATTACK PLAN
Proof track:
1. For finite cyclic groups (equivalently exact covers of $\mathbb{Z}/N\mathbb{Z}$ by congruence classes), use roots of unity to force repetition of the largest index.
2. Try to extend the same character-sum argument to abelian groups via 1-dimensional characters.
3. For general finite groups, attempt a representation-theoretic version: apply an irreducible character to the coset partition and isolate the largest index.

Disproof track:
1. Search for small finite-group counterexamples by brute force over subgroup lattices (if any exist, the smallest one would be very informative).
2. Try semidirect products where subgroups fail to be subnormal, aiming to engineer a partition with distinct indices.

Chosen path: prove clean necessary conditions and a complete proof for cyclic groups; no general counterexample found.

4) WORK
Lemma 1 (Counting constraint in the finite case).
Let $G$ be a \emph{finite} group and suppose
\[G=\bigsqcup_{i=1}^k a_iG_i\]
is a disjoint union of left cosets of subgroups $G_i\le G$. Then
\[\sum_{i=1}^k \frac{1}{[G:G_i]}=1.\]

Proof.
Each coset $a_iG_i$ has cardinality $|G_i|$. Disjointness gives
\[|G|=\sum_{i=1}^k |a_iG_i|=\sum_{i=1}^k |G_i|.\]
Divide both sides by $|G|$ and use $|G_i|/|G|=1/[G:G_i]$. \qed

Proposition 2 (Herzog--Sch\"onheim for finite cyclic groups).
Let $G=\mathbb{Z}/N\mathbb{Z}$ be a finite cyclic group. Suppose $G$ is partitioned into finitely many cosets of subgroups,
equivalently into congruence classes
\[G=\bigsqcup_{i=1}^k \{x\in \mathbb{Z}/N\mathbb{Z}: x\equiv a_i \!\!\!\pmod{d_i}\}\]
where each $d_i\mid N$ and the coset has index $d_i=[G:G_i]$. Then the largest index occurs at least twice.
In particular, the indices cannot be pairwise distinct.

Proof.
Let $m=\max_i d_i$. Since each $d_i\mid N$, also $m\mid N$.
Let $\zeta$ be a primitive $m$th root of unity, so $\zeta^m=1$ and $\zeta\ne 1$. Then $\zeta^N=1$ and hence
\[\sum_{x=0}^{N-1}\zeta^x=0.\tag{$\ast$}\]

Write each congruence class in $\{0,1,\dots,N-1\}$ representatives. For a fixed modulus $d\mid N$ and residue $a$,
the corresponding set in $\{0,\dots,N-1\}$ is $\{a+td:0\le t\le N/d-1\}$ and
\[\sum_{\substack{0\le x\le N-1\\ x\equiv a\!\!\!\!\pmod d}}\zeta^x
=\sum_{t=0}^{N/d-1}\zeta^{a+td}
=\zeta^a\sum_{t=0}^{N/d-1}(\zeta^d)^t.\tag{$\ast\ast$}\]

If $d<m$ then $\zeta^d\ne 1$ (since $\zeta$ has exact order $m$ and $0<d<m$), while $(\zeta^d)^{N/d}=\zeta^N=1$.
Therefore the geometric series in $(\ast\ast)$ sums to $0$, so every congruence class with modulus $d<m$ contributes $0$
to the sum $(\ast)$.

If $d=m$ then $\zeta^d=\zeta^m=1$, so $(\ast\ast)$ equals $\zeta^a\cdot (N/m)$.
Using that the congruence classes partition $\{0,\dots,N-1\}$, we can rewrite $(\ast)$ as
\[0=\sum_{x=0}^{N-1}\zeta^x=\sum_{i=1}^k \sum_{\substack{0\le x\le N-1\\ x\equiv a_i\!\!\!\!\pmod {d_i}}}\zeta^x
=\frac{N}{m}\sum_{i:\, d_i=m}\zeta^{a_i}.\]
Since $N/m\ne 0$, we get $\sum_{i:\, d_i=m}\zeta^{a_i}=0$.
If there were exactly one index with $d_i=m$, this sum would be a single nonzero complex number $\zeta^{a_i}$, contradiction.
Hence at least two of the $d_i$ equal $m$. \qed

FAST REALITY CHECK (computation).
I brute-forced cyclic groups $\mathbb{Z}/n\mathbb{Z}$ for $2\le n\le 80$ and found no partition into cosets of subgroups with pairwise distinct indices.
(Program output: “No partitions with distinct indices found for cyclic groups $\\mathbb{Z}/n\\mathbb{Z}$, $n\\le 80$”.)

5) VERIFICATION
- Lemma 1 uses only finiteness and disjointness; no normality assumptions are needed.
- In Proposition 2, the key points are: $m\mid N$ so $\zeta^N=1$; for $d<m$ we have $\zeta^d\ne 1$; and each coset sum is a full geometric progression.
- Edge cases: $k=1$ is excluded; if a coset has index $1$ it is all of $G$ and cannot be disjoint from others.

6) FINAL
**UNRESOLVED**

(i) Strongest fully proved partial result:
For finite cyclic groups $\mathbb{Z}/N\mathbb{Z}$, any coset partition forces the largest index to repeat (Proposition 2). In any finite group partition, the indices satisfy $\sum_i 1/[G:G_i]=1$ (Lemma 1).

(ii) First gap:
Extend the “largest index repeats” argument from cyclic (or abelian) groups to \emph{all} finite groups, i.e. show that any partition $G=\bigsqcup a_iG_i$ forces two indices $[G:G_i]$ to be equal.

(iii) Top 3 next moves:
1. Try a representation-theoretic analogue of Proposition 2: evaluate $\sum_{g\in G}\chi(g)$ for a suitable irreducible character $\chi$ and decompose it over the coset partition.
2. Reduce to finite quotients: if an infinite group counterexample existed, look for a finite quotient inheriting a coset partition with distinct indices.
3. Implement a targeted search for small non-abelian groups by enumerating subgroup cosets and testing partitions (beyond the cyclic sanity check).

(iv) Minimal counterexample structure (if the conjecture were false):
A smallest finite counterexample would be a non-abelian group (since abelian is claimed settled) of order at least $1440$ (per the statement), with a coset partition in which all subgroup indices are distinct and (likely) none of the subgroups are subnormal.

