% Attempt for Erdos Problem #39
% Following PROMPT_STRATEGY.MD
% Tools/Constraints:
% - Web browsing available? NO (restricted)
% - Computation available? YES (local Python), used for small sanity checks
% GPT 5.2

OUTPUT
1) FORMAL RESTATEMENT
A set $A\subseteq\mathbb N$ is a Sidon set (a $B_2$-set) if for all $a,b,c,d\in A$,
\[a+b=c+d\implies \{a,b\}=\{c,d\}.\]
Let $A(N):=|A\cap\{1,\dots,N\}|$.
Question: does there exist an infinite Sidon set $A$ such that for every $\varepsilon>0$ there is a constant $c(\varepsilon)>0$ with
\[A(N)\ge c(\varepsilon)\,N^{1/2-\varepsilon}\quad\text{for all }N\ge 1?
\]

2) QUICK LITERATURE/CONTEXT CHECK
Web browsing is not available. I only quote claims already present in the problem statement.

3) ATTACK PLAN
Proof track:
1. Construct explicit infinite Sidon sets by concatenating well-separated finite Sidon blocks.
2. Use probabilistic constructions on each block and a range-separation argument to preserve the Sidon property globally.

Disproof track:
1. Try to show any infinite Sidon set must have large gaps forcing $A(N)$ to dip below $N^{1/2-\varepsilon}$ infinitely often.

Chosen path: give a fully proved "trivial" construction with exponent $1/3$ (matching the statement), and record elementary bounds.

4) WORK
Lemma 1 (Sidon implies distinct positive differences).
Let $A=\{a_1<a_2<\cdots<a_m\}\subseteq\{1,\dots,N\}$ be Sidon. Then all positive differences $a_j-a_i$ ($1\le i<j\le m$) are distinct.
In particular,
\[\binom{m}{2}\le N-1\quad\text{and hence}\quad m<\sqrt{2N}+1.
\]

Proof.
Suppose $a_j-a_i=a_{j'}-a_{i'}$ with $i<j$ and $i'<j'$. Then $a_j+a_{i'}=a_{j'}+a_i$.
By the Sidon property, $\{a_j,a_{i'}\}=\{a_{j'},a_i\}$.
Since $a_j>a_i$ and $a_{j'}>a_{i'}$, we must have $a_j=a_{j'}$ and $a_i=a_{i'}$, hence the differences are distinct.
All differences lie in $\{1,2,\dots,N-1\}$, which has size $N-1$, so $\binom{m}{2}\le N-1$.
Solving $m(m-1)/2\le N-1$ gives $m<\sqrt{2N}+1$. \qed

Lemma 2 (Finite Sidon sets of size $\gg N^{1/3}$ exist).
For every integer $N\ge 1$ there exists a Sidon set $S\subseteq\{1,\dots,N\}$ with
\[|S|\ge \frac{1}{5}N^{1/3}.
\]

Proof.
Let $p:=\frac{1}{4}N^{-2/3}$ and form a random subset $R\subseteq\{1,\dots,N\}$ by including each integer independently with probability $p$.
Then $\mathbb E|R|=pN=\frac{1}{4}N^{1/3}$.

Call an ordered quadruple $(a,b,c,d)\in\{1,\dots,N\}^4$ \emph{bad} if $a+b=c+d$ but $(a,b)$ is not a permutation of $(c,d)$.
If $R$ contains all four entries of a bad quadruple, then $R$ is not Sidon.
The number of bad quadruples is at most $N^3$: choose $(a,b,c)$ freely (at most $N^3$ choices) and then $d=a+b-c$ is determined;
we discard choices with $d\notin\{1,\dots,N\}$, so this is an upper bound.
Each fixed quadruple is contained in $R$ with probability $p^4$, so the expected number $X$ of bad quadruples contained in $R$ satisfies
\[\mathbb E[X]\le N^3 p^4 = N^3\cdot \frac{1}{4^4}N^{-8/3} = \frac{1}{256}N^{1/3}.
\]

Starting from $R$, we delete elements one-by-one while there remains any bad quadruple contained in the current set.
Each deletion removes at least one bad quadruple, so the total number of deletions is at most $X$.
Let $S$ be the resulting set. Then $S$ is Sidon and
\[|S|\ge |R|-X.
\]
Taking expectations gives
\[\mathbb E|S|\ge \mathbb E|R|-\mathbb E[X] \ge \left(\frac{1}{4}-\frac{1}{256}\right)N^{1/3}=\frac{63}{256}N^{1/3} > \frac{1}{5}N^{1/3}.
\]
Therefore there exists a realization with $|S|\ge \frac{1}{5}N^{1/3}$. \qed

Lemma 3 (An infinite Sidon set with uniform $N^{1/3}$ growth).
There exists an infinite Sidon set $A\subseteq\mathbb N$ and a constant $c>0$ such that for all $N\ge 1$,
\[|A\cap\{1,\dots,N\}|\ge c\,N^{1/3}.
\]

Proof.
For each $k\ge 1$, set $L_k:=4^k$ and let $S_k\subseteq\{1,\dots,4^{k-1}\}$ be a Sidon set with
$|S_k|\ge \frac{1}{5}\,(4^{k-1})^{1/3}$, which exists by Lemma 2.
Define the translated block
\[A_k:=\{L_k+s:s\in S_k\}\subseteq \{L_k+1,\dots,L_k+4^{k-1}\}.
\]
Let $A:=\bigcup_{k\ge 1} A_k$.

We claim $A$ is Sidon.
Fix $k$. Any sum of two elements from blocks strictly below $k$ is at most
\[2\max(A_1\cup\cdots\cup A_{k-1}) < 2\left(L_{k-1}+4^{k-2}\right)=2\left(4^{k-1}+4^{k-2}\right)=\frac{5}{2}4^{k-1} < 4^k=L_k.
\]
Thus any sum of two elements from blocks $<k$ is $<L_k$.
On the other hand, any sum involving at least one element of $A_k$ is at least $L_k+1+1 > L_k$.
So a sum of two elements from blocks $<k$ cannot equal a sum involving an element of block $k$.
Similarly, if a sum involves an element of $A_k$ but not of $A_{k'}$ for $k'>k$, then that sum is $<2L_{k+1}$ while any sum involving an
element of $A_{k+1}$ is $>L_{k+1}=4L_k$, hence the block index of the largest summand is determined by the size of the sum.
Therefore any additive collision $x_1+x_2=y_1+y_2$ in $A$ must occur within a single block $A_k$.
But each $A_k$ is Sidon (translation preserves the Sidon property), so the collision is trivial.
Hence $A$ is Sidon.

For the growth bound, for $N\ge L_k$ we have $A_1\cup\cdots\cup A_k\subseteq[1,N]$ and therefore
\[|A\cap[1,N]|\ge |A_k|=|S_k|\ge \frac{1}{5}(4^{k-1})^{1/3} = \frac{1}{5\cdot 4^{1/3}}\,L_k^{1/3} \ge \frac{1}{5\cdot 4^{1/3}}\,N^{1/3}.
\]
So one may take $c:=\frac{1}{5\cdot 4^{1/3}}$. \qed

FAST REALITY CHECK (local computation, greedy Sidon sequence).
I computed the first 25 terms of the standard greedy Sidon sequence (smallest possible next element) as:
\[1,2,4,8,13,21,31,45,66,81,97,123,148,182,204,252,290,361,401,475,565,593,662,775,822.\]
For this prefix: $|A\cap[1,100]|=11$ and $|A\cap[1,1000]|=27$. This is only a sanity check, not a proof of any asymptotic.

5) VERIFICATION
- Lemma 1 is a standard Sidon $\Rightarrow$ distinct differences argument.
- Lemma 2 is a probabilistic existence proof with an explicit deletion algorithm; all steps are justified.
- Lemma 3's key point is range separation of pairwise sums across blocks.

6) FINAL
**UNRESOLVED**

(i) Strongest fully proved partial result:
A fully explicit (though not optimal) infinite Sidon set exists with uniform lower bound $|A\cap[1,N]|\ge cN^{1/3}$ (Lemma 3),
and every Sidon subset of $[1,N]$ has size $<\sqrt{2N}+1$ (Lemma 1).

(ii) First gap:
Upgrading the exponent from $1/3$ toward $1/2-\varepsilon$ requires substantially denser finite Sidon blocks (or different ideas);
no argument here produces an infinite Sidon set with $A(N)\gg_\varepsilon N^{1/2-\varepsilon}$.

(iii) Top 3 next moves (concrete):
1. Replace Lemma 2's $N^{1/3}$ construction with denser finite Sidon constructions and then re-run the block-separation scheme.
2. Attempt to formalize the known $\liminf A(N)/N^{1/2}=0$ phenomenon via a self-contained combinatorial argument.
3. Explore whether controlling $1_A*1_A$ uniformly (as mentioned in the statement) can be reconciled with growth $N^{1/2-\varepsilon}$.

(iv) Minimal counterexample structure (if the answer is negative):
A proof would need to show that every infinite Sidon set has infinitely many scales $N$ where $A(N)$ is much smaller than $N^{1/2-\varepsilon}$,
likely via a density-decrement or gap-growth mechanism intrinsic to the Sidon constraint.
