
OUTPUT
1) FORMAL RESTATEMENT
Definitions/Conventions.
- Write $\mathfrak c:=|{\mathbb R}|$ for the cardinality of the continuum, and identify it with its initial ordinal.
- For a set $X$ and integer $r\ge 1$, write $[X]^r$ for the family of $r$-element subsets of $X$.
- For cardinals/ordinals $\kappa,\alpha,\gamma$ and $r\ge 1$, the partition relation
\[\kappa \to (\alpha,\gamma)^r_2\]
means: for every coloring $c:[\kappa]^r\to\{0,1\}$, there exists either
  (i) $A\subseteq \kappa$ of order type $\alpha$ such that $c$ is constantly $0$ on $[A]^r$, or
  (ii) $B\subseteq \kappa$ of order type $\gamma$ such that $c$ is constantly $1$ on $[B]^r$.
Here, when $\gamma=n\in\mathbb N$, ``order type $n$'' just means ``a subset of size $n$''.

Question.
Fix a countable ordinal $\beta$ and an integer $n$ with $2\le n<\omega$. Is it true that
\[\mathfrak c \to (\beta,n)^3_2?\]

2) QUICK LITERATURE/CONTEXT CHECK
Web browsing is not available in this session.
From the problem text only:
- Erd\H{o}s and Rado proved $\mathfrak c\to(\omega+n,4)^3_2$ for $2\le n<\omega$.
I do not reproduce that proof here.

3) ATTACK PLAN
Proof track ideas:
1. Reduce to the first nontrivial finite case $n=4$ by monotonicity in the second parameter (proved below), then attempt to extend the known $(\omega+n,4)$ case to all countable $\beta$.
2. Try to build an increasing $\omega_1$-sequence inside $\mathfrak c$ via a transfinite recursion and apply canonical partition theorems.

Disproof track ideas:
1. Try to construct a 2-coloring of $[\mathfrak c]^3$ with no 1-homogeneous $n$-set for some fixed $n\ge 4$ and no 0-homogeneous set of order type $\beta$.
2. Identify which additional axioms (CH, MA, large cardinals) might affect $\mathfrak c\to(\beta,n)^3_2$.

Chosen path in this attempt: isolate trivial cases and monotonicity reductions (fully proved), to clarify the first genuinely open parameter regime.

4) WORK
Lemma 1 (Triviality for $n=2$).
For any infinite cardinal $\kappa$ and any ordinal $\beta$ with $\beta\le \kappa$, one has
\[\kappa\to(\beta,2)^3_2.\]

Proof.
Let $c:[\kappa]^3\to\{0,1\}$ be any coloring. Take any two-element subset $B\subseteq\kappa$.
Then $[B]^3=\emptyset$, so $c$ is vacuously constant $1$ on $[B]^3$.
Thus the second alternative in the definition of $\kappa\to(\beta,2)^3_2$ always holds (provided $\kappa$ has at least two elements, which is true for infinite $\kappa$). \qed

Lemma 2 (Triviality for $n=3$).
For any infinite cardinal $\kappa$ and any ordinal $\beta$ with $\beta\le \kappa$, one has
\[\kappa\to(\beta,3)^3_2.\]

Proof.
Let $c:[\kappa]^3\to\{0,1\}$ be any coloring.
If there exists a triple $B\in[\kappa]^3$ with $c(B)=1$, then $B$ itself witnesses the second alternative, since $[B]^3=\{B\}$.
Otherwise, $c$ is identically $0$ on $[\kappa]^3$. In that case, choose any subset $A\subseteq \kappa$ of order type $\beta$ (possible since $\beta\le\kappa$),
and then $c$ is constantly $0$ on $[A]^3$. So one of the two alternatives always holds. \qed

Lemma 3 (Monotonicity in the finite parameter).
Fix $\kappa,\beta$ and integers $3\le m\le n$. If
\[\kappa\to(\beta,n)^3_2,\]
then
\[\kappa\to(\beta,m)^3_2.\]

Proof.
Let $c:[\kappa]^3\to\{0,1\}$ be any coloring.
Apply $\kappa\to(\beta,n)^3_2$ to obtain either:
(i) a $0$-homogeneous set $A$ of order type $\beta$, in which case we are done; or
(ii) a $1$-homogeneous set $B$ of size $n$ (so $c$ is constantly $1$ on $[B]^3$).
In case (ii), choose any subset $B'\subseteq B$ of size $m$.
Then $[B']^3\subseteq[B]^3$, so $c$ is also constantly $1$ on $[B']^3$, giving the second alternative for $(\beta,m)^3_2$.
Thus $\kappa\to(\beta,m)^3_2$ holds. \qed

FAST REALITY CHECK
The above shows the statement is vacuous for $n=2$ and $n=3$.
So the first nontrivial finite case is $n=4$, and the general question for finite $n$ reduces (by Lemma 3) to understanding the hardest values of $n$.

5) VERIFICATION
- Lemmas 1--2 rely only on the fact that $[B]^3$ is empty for $|B|<3$ and a singleton for $|B|=3$.
- Lemma 3 is a direct subset argument: $1$-homogeneity is inherited by subsets.

6) FINAL
**UNRESOLVED**

(i) Strongest fully proved partial result:
The relation $\kappa\to(\beta,n)^3_2$ is trivial for $n\in\{2,3\}$ (Lemmas 1--2), and is monotone in $n$ (Lemma 3).

(ii) First gap:
Handle the first genuinely nontrivial case: for fixed countable $\beta$, prove or refute
\[\mathfrak c\to(\beta,4)^3_2.\]
The known result in the text only covers $\beta=\omega+n$ with $n=4$.

(iii) Top 3 next moves (concrete):
1. Specialize to $\beta=\omega_1$ (or to $\beta=\omega^\omega$) and try to either extend the Erd\H{o}s--Rado method or construct an explicit bad coloring of $[\mathfrak c]^3$.
2. Determine which set-theoretic axioms affect the truth of $\mathfrak c\to(\beta,n)^3_2$; attempt to prove consistency/independence results for some $\beta,n$.
3. Search for a reduction to canonical colorings on $[\mathfrak c]^3$ (e.g. via a definable well-order of the reals) and analyze homogeneous sets of prescribed order types.

(iv) What a minimal counterexample would likely look like (structure/parameters):
A 2-coloring $c:[\mathfrak c]^3\to\{0,1\}$ such that every $1$-homogeneous set has size $\le n-1$ (no blue $n$-set),
and every $0$-homogeneous set has order type strictly less than the target countable $\beta$.

