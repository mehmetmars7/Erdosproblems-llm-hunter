
OUTPUT
1) FORMAL RESTATEMENT
Let $A\subseteq\mathbb N$ with
\[\sum_{n\in A}\frac{1}{n}=\infty.\]
Question: For every integer $k\ge 3$, does $A$ contain a nontrivial $k$-term arithmetic progression,
i.e., distinct $a,a+d,\dots,a+(k-1)d\in A$ with $d\ge 1$? Equivalently, does $A$ contain arbitrarily long arithmetic progressions?

2) QUICK LITERATURE/CONTEXT CHECK
Browsing is not available. I only record what the problem statement itself claims about known bounds on $r_k(N)$.
I have not verified those results here.

3) ATTACK PLAN
Proof track:
1. Connect divergent reciprocal sum to lower bounds on $|A\cap[1,N]|$ via summation by parts.
2. Show that any sufficiently strong upper bound on $r_k(N)$ forces convergence of $\sum_{n\in A}1/n$ for $k$-AP-free sets.
3. Reduce the conjecture to proving a density bound of the form $r_k(N) \ll N/(\log N)(\log\log N)^2$.

Disproof track:
1. Try to construct $k$-AP-free sets with divergent harmonic series (e.g., structured thin sets).
2. Explore Behrend-type constructions and check their reciprocal sums.

Chosen path: establish the conditional implication from a bound on $r_k(N)$ to the conjecture.

4) WORK
Let $A(x)=|A\cap\{1,\dots,\lfloor x\rfloor\}|$.

Lemma 1 (summation by parts bound).
For any $X\ge 1$,
\[\sum_{\substack{n\in A\\n\le X}}\frac{1}{n}=\frac{A(X)}{X}+\sum_{m=1}^{\lfloor X\rfloor-1} A(m)\Bigl(\frac{1}{m}-\frac{1}{m+1}\Bigr).\]
In particular,
\[\sum_{\substack{n\in A\\n\le X}}\frac{1}{n}\le \frac{A(X)}{X}+\int_1^X \frac{A(t)}{t^2}\,dt.\]

Proof.
Write the indicator of membership as $1_A(n)=A(n)-A(n-1)$ and apply summation by parts:
\[\sum_{n\le X} \frac{1_A(n)}{n} = \frac{A(X)}{X} + \sum_{m=1}^{\lfloor X\rfloor-1} A(m)\Bigl(\frac{1}{m}-\frac{1}{m+1}\Bigr).\]
The integral bound follows since $\frac{1}{m}-\frac{1}{m+1} \le \int_m^{m+1} t^{-2}dt$. \qed

Lemma 2 (a sufficient counting bound forces convergence).
If there exists $X_0$ and a constant $C$ such that for all $x\ge X_0$,
\[A(x)\le \frac{C x}{(\log x)(\log\log x)^2},\]
then $\sum_{n\in A} 1/n$ converges.

Proof.
Insert the bound into Lemma 1. The term $A(X)/X$ is $O(1/(\log X)(\log\log X)^2)$.
The integral is bounded by
\[C\int_{X_0}^\infty \frac{1}{t\log t (\log\log t)^2}\,dt<\infty\]
by the integral test. Hence the partial sums are bounded, so the series converges. \qed

Proposition 3 (conditional reduction to $r_k(N)$).
Suppose for some fixed $k\ge 3$ there is a constant $C_k$ such that for all large $N$,
\[r_k(N)\le \frac{C_k N}{(\log N)(\log\log N)^2}.\]
Then every set $A\subseteq\mathbb N$ with $\sum_{n\in A}1/n=\infty$ contains a $k$-term arithmetic progression.

Proof.
If $A$ contained no $k$-term AP, then $A(x)\le r_k(x)$ for all $x$, and Lemma 2 would imply
$\sum_{n\in A}1/n$ converges, a contradiction. \qed

5) VERIFICATION
- Lemma 1 is standard summation by parts; the inequality uses monotonicity of $t\mapsto 1/t$.
- Lemma 2 uses only the integral test and does not assume any external theorems on $r_k$.
- Proposition 3 is purely conditional and does not assert the bound on $r_k$.

6) FINAL
**UNRESOLVED**

(i) Strongest fully proved partial result:
A quantitative bound of the form $r_k(N)\ll N/(\log N)(\log\log N)^2$ would imply the conjecture
(Proposition 3). This conditional implication is proved here.

(ii) First gap:
No unconditional upper bound on $r_k(N)$ strong enough to trigger Lemma 2 has been proved here.

(iii) Top 3 next moves:
1. Work towards proving $r_k(N)\ll N/(\log N)(\log\log N)^2$ for fixed $k$ using density increment machinery.
2. Attempt a direct construction of $k$-AP-free sets with divergent reciprocal sum (potential counterexample).
3. Refine the summation-by-parts approach to use weaker upper bounds on $r_k(N)$.

(iv) Minimal counterexample structure (if the conjecture were false):
A $k$-AP-free set $A$ with counting function $A(x)$ close to $x/(\log x)$ (or denser) so that
$\sum_{n\in A} 1/n$ diverges despite avoiding long arithmetic progressions.
