
OUTPUT
1) FORMAL RESTATEMENT
Definitions/Conventions.
- Graphs may be infinite.
- The chromatic number $\chi(G)$ of a graph $G$ is the least cardinal $\kappa$ such that $V(G)$ can be colored with $\kappa$ colors so that adjacent vertices receive different colors.
- A graph $H$ is a subgraph of $G$ if there exists an injective map $\varphi:V(H)\to V(G)$ that sends edges of $H$ to edges of $G$ (not necessarily induced).

Question (two graphs).
If $\chi(G_1)=\chi(G_2)=\aleph_1$, must there exist a graph $G$ with $\chi(G)=4$ (or even $\chi(G)=\aleph_0$) such that
$G$ is a subgraph of both $G_1$ and $G_2$?

Question (finite collections).
More generally, for any finite collection $G_1,\dots,G_t$ with $\chi(G_i)=\aleph_1$, must there exist a common subgraph $H$ of chromatic number $4$ (or $\aleph_0$)?

2) QUICK LITERATURE/CONTEXT CHECK
Web browsing is not available in this session.
From the problem text only:
- EHS74 proved that every $\aleph_1$-chromatic graph contains all sufficiently large odd cycles.
- Erd\H{o}s speculated that every $\aleph_1$-chromatic graph contains all 4-chromatic graphs of sufficiently large girth.
I do not re-derive these deep results here.

3) ATTACK PLAN
Proof track ideas:
1. Use compactness-type principles to relate uncountable chromatic number to existence of finite high-chromatic subgraphs, then try to force the \emph{same} finite 4-chromatic graph to appear in both $G_1$ and $G_2$.
2. Use the stated EHS74 theorem as motivation: if both graphs contain ``many'' odd cycles, can one upgrade common 3-chromatic subgraphs to common 4-chromatic ones by attaching gadgets?
3. For finite collections, try a pigeonhole argument over the (countably many) isomorphism types of finite 4-chromatic graphs.

Disproof track ideas:
1. Attempt to construct two $\aleph_1$-chromatic graphs with incompatible finite subgraph spectra, so that no 4-chromatic (or countably chromatic) graph embeds into both.
2. In particular, try to build $G_1,G_2$ with disjoint sets of ``forbidden'' finite subgraphs of chromatic number $4$.

Chosen path in this attempt: prove two unconditional lemmas about extracting finite high-chromatic subgraphs from infinite graphs (de Bruijn--Erd\H{o}s compactness), which gives partial progress but not the common-subgraph conclusion.

4) WORK
Lemma 1 (de Bruijn--Erd\H{o}s compactness for $k$-colorings).
Fix a positive integer $k$.
If every finite induced subgraph of a graph $G$ is $k$-colorable, then $G$ is $k$-colorable.

Proof.
Let $V$ be the vertex set of $G$.
Consider the set $X:=\{1,2,\dots,k\}^V$ of all functions $c:V\to\{1,\dots,k\}$.
Give $\{1,\dots,k\}$ the discrete topology (which is compact because it is finite), and give $X$ the product topology.
By Tychonoff's theorem, $X$ is compact.

For each edge $uv\in E(G)$, define the set
\[F_{uv}:=\{c\in X:\ c(u)\ne c(v)\}.\]
This set is closed in $X$ because it is the preimage of the closed set $\{(i,j)\in[k]^2:i\ne j\}$ under the continuous projection map $c\mapsto (c(u),c(v))$.

Now take any finite set of edges $E_0\subseteq E(G)$, and let $H$ be the finite subgraph of $G$ with vertex set consisting of endpoints of edges in $E_0$ and edge set $E_0$.
By hypothesis, $H$ is $k$-colorable, so there exists a function $c_H:V(H)\to[k]$ proper on all edges in $E_0$.
Extend $c_H$ arbitrarily to a function $c:V\to[k]$ (define colors on $V\setminus V(H)$ arbitrarily). Then $c\in \bigcap_{uv\in E_0} F_{uv}$.
Therefore, for every finite $E_0$, the intersection $\bigcap_{uv\in E_0} F_{uv}$ is nonempty.
Equivalently, the family of closed sets $\{F_{uv}: uv\in E(G)\}$ has the finite intersection property.

By compactness of $X$, the total intersection $\bigcap_{uv\in E(G)} F_{uv}$ is nonempty.
Any $c$ in this intersection is a proper $k$-coloring of $G$, proving that $\chi(G)\le k$.
\qed

Lemma 2 (Finite obstruction to $k$-colorability).
Let $k$ be a positive integer. If $\chi(G)>k$, then $G$ contains a \emph{finite} subgraph $H$ with $\chi(H)>k$.
In particular, if $\chi(G)=\aleph_1$, then $G$ contains a finite subgraph with chromatic number at least $4$.

Proof.
Take the contrapositive of Lemma 1: if every finite induced subgraph of $G$ is $k$-colorable, then $G$ is $k$-colorable.
Thus if $G$ is not $k$-colorable, there must exist some finite induced subgraph that is not $k$-colorable, i.e. has chromatic number $>k$.

Apply this with $k=3$: if $\chi(G)=\aleph_1$, then in particular $\chi(G)>3$, so $G$ contains a finite subgraph $H$ with $\chi(H)\ge 4$. \qed

FAST REALITY CHECK
No computation is relevant here (this is a set-theoretic/infinite-graph problem).
As a sanity check on definitions: every graph contains $K_1$ (chromatic number $1$) as a common subgraph, so the content is entirely in forcing a common subgraph of \emph{higher} chromatic number.

5) VERIFICATION
- Lemma 1 is standard and uses only compactness of $[k]^V$ and the finite intersection property; the proof explicitly constructs the closed sets $F_{uv}$.
- Lemma 2 is the contrapositive application of Lemma 1.
- These lemmas show that each $\aleph_1$-chromatic graph contains some finite 4-chromatic subgraph, but they do not address whether \emph{the same} 4-chromatic graph embeds into \emph{both} $G_1$ and $G_2$.

6) FINAL
**UNRESOLVED**

(i) Strongest fully proved partial result:
Every $\aleph_1$-chromatic graph contains some finite 4-chromatic subgraph (Lemma 2).

(ii) First gap:
Given two graphs $G_1,G_2$ of chromatic number $\aleph_1$, prove the existence of a \emph{single} graph $H$ of chromatic number $4$ (or $\aleph_0$) that embeds as a subgraph of both.
Lemma 2 only produces (potentially different) finite 4-chromatic subgraphs $H_1\subseteq G_1$ and $H_2\subseteq G_2$.

(iii) Top 3 next moves (concrete):
1. Use that there are only countably many isomorphism types of finite 4-chromatic graphs: try to prove that every $\aleph_1$-chromatic graph contains \emph{infinitely many} distinct finite 4-chromatic subgraphs, then seek an overlap argument for two graphs.
2. Explore whether the EHS74 statement about containing all sufficiently large odd cycles can be combined with a fixed 4-chromatic ``gadget'' to force a common 4-chromatic subgraph.
3. Attempt to build a counterexample pair $(G_1,G_2)$ by ensuring their finite 4-chromatic subgraph spectra are disjoint; clarify which set-theoretic principles (choice/CH/etc.) would be needed.

(iv) What a minimal counterexample would likely look like (structure/parameters):
Two $\aleph_1$-chromatic graphs $G_1,G_2$ whose sets of finite 4-chromatic subgraphs (up to isomorphism) are disjoint, so that no finite 4-chromatic graph embeds into both,
and such that no countably chromatic infinite graph can embed into both either (e.g. by incompatible forbidden configurations).

