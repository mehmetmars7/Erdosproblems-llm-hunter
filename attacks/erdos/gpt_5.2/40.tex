% Attempt for Erdos Problem #40
% Following PROMPT_STRATEGY.MD
% Tools/Constraints:
% - Web browsing available? NO (restricted)
% - Computation available? YES (local Python), used for small sanity checks
% GPT 5.2

OUTPUT
1) FORMAL RESTATEMENT
Let $A\subseteq\mathbb N$ and define the (ordered) representation function
\[r_A(n):=(1_A*1_A)(n)=|\{(a,b)\in A^2: a+b=n\}|.\]
Let $A(N):=|A\cap\{1,\dots,N\}|$.
Given a function $g(N)\to\infty$, the question asks for which $g$ the following implication holds:
if there exists $c>0$ such that for all sufficiently large $N$,
\[A(N)\ge c\,\frac{N^{1/2}}{g(N)},\]
then necessarily
\[\limsup_{n\to\infty} r_A(n)=\infty.
\]

2) QUICK LITERATURE/CONTEXT CHECK
Web browsing is not available. I only note the stated relation to Problem [28].

3) ATTACK PLAN
Proof track:
1. Use the identity $\sum_{n\le 2N} r_{A\cap[1,N]}(n)=A(N)^2$ to force large values when $A(N)$ is bigger than $\sqrt N$.
2. If $A(N)$ is only $\asymp \sqrt N / g(N)$, attempt to prove unboundedness via additional structure (not done here).

Disproof track:
1. Use Sidon sets (bounded representation function) as counterexamples whenever they are dense enough to satisfy the hypothesis.

Chosen path: prove a strong sufficient condition (density spikes) and a rigorous negative range of $g$ using a Sidon construction.

4) WORK
Lemma 1 (Averaging identity for $r_A$).
Let $A_N:=A\cap\{1,\dots,N\}$ and $m:=|A_N|$. Then
\[\sum_{n=2}^{2N} r_{A_N}(n)=m^2.
\]
In particular there exists $n\in\{2,\dots,2N\}$ with
\[r_A(n)\ge r_{A_N}(n)\ge \frac{m^2}{2N-1}.
\]

Proof.
Each ordered pair $(a,b)\in A_N^2$ contributes exactly once to the sum, namely to $r_{A_N}(a+b)$.
Thus the sum equals $|A_N|^2=m^2$. The maximum is at least the average $m^2/(2N-1)$.
Finally $r_A(n)\ge r_{A_N}(n)$ because $A_N\subseteq A$. \qed

Corollary 2 (A sufficient condition for $\limsup r_A(n)=\infty$).
If $A(N)\ge N^{1/2}h(N)$ for infinitely many $N$ with $h(N)\to\infty$, then $\limsup_{n\to\infty} r_A(n)=\infty$.

Proof.
Apply Lemma 1 along those $N$ to get some $n\le 2N$ with
$r_A(n)\ge A(N)^2/(2N-1)\gg h(N)^2\to\infty$. \qed

Lemma 3 (Sidon sets have uniformly bounded representation function).
If $A$ is Sidon, then for every $n$ one has $r_A(n)\le 2$.

Proof.
If $r_A(n)\ge 3$, then there exist two distinct ordered pairs $(a,b)\ne(c,d)$ in $A^2$ with $a+b=c+d=n$.
If $(a,b)$ is not a permutation of $(c,d)$, this contradicts the Sidon property.
If it is a permutation, the only distinct ordered pairs are $(a,b)$ and $(b,a)$ with $a\ne b$, so $r_A(n)=2$.
Hence $r_A(n)\le 2$ for all $n$. \qed

Lemma 4 (Existence of an infinite Sidon set with $N^{1/3}$ growth).
There exists an infinite Sidon set $A\subseteq\mathbb N$ and a constant $c>0$ such that for all $N\ge 1$,
\[|A\cap[1,N]|\ge c\,N^{1/3}.
\]

Proof.
Step 1: finite blocks. We claim that for every integer $M\ge 1$ there exists a Sidon set $S\subseteq\{1,\dots,M\}$ with
$|S|\ge \frac{1}{5}M^{1/3}$.

To prove the claim, set $p:=\frac{1}{4}M^{-2/3}$ and choose a random subset $R\subseteq\{1,\dots,M\}$ by including each element
independently with probability $p$. Then $\mathbb E|R|=pM=\frac{1}{4}M^{1/3}$.
Call an ordered quadruple $(a,b,c,d)\in\{1,\dots,M\}^4$ \emph{bad} if $a+b=c+d$ but $(a,b)$ is not a permutation of $(c,d)$.
The number of bad quadruples is at most $M^3$: choose $(a,b,c)$ freely and set $d=a+b-c$.
Let $X$ be the number of bad quadruples contained in $R$. Then
\[\mathbb E[X]\le M^3 p^4 = M^3\cdot \frac{1}{4^4}M^{-8/3} = \frac{1}{256}M^{1/3}.
\]
Starting from $R$, repeatedly delete elements while the current set contains any bad quadruple.
Each deletion removes at least one bad quadruple, so the number of deletions is at most $X$.
Let $S$ be the final set. Then $S$ is Sidon and $|S|\ge |R|-X$.
Taking expectations yields
\[\mathbb E|S|\ge \left(\frac{1}{4}-\frac{1}{256}\right)M^{1/3}=\frac{63}{256}M^{1/3} > \frac{1}{5}M^{1/3}.
\]
So some realization satisfies $|S|\ge \frac15 M^{1/3}$, proving the claim.

Step 2: infinite union of separated blocks.
For each $k\ge 1$, let $L_k:=4^k$ and choose a Sidon set $S_k\subseteq\{1,\dots,4^{k-1}\}$ with
$|S_k|\ge \frac15(4^{k-1})^{1/3}$ (by the claim with $M=4^{k-1}$).
Define the translated block $A_k:=\{L_k+s:s\in S_k\}\subseteq\{L_k+1,\dots,L_k+4^{k-1}\}$ and set $A:=\bigcup_{k\ge 1}A_k$.

We check $A$ is Sidon.
Fix $k$. Any sum of two elements from blocks strictly below $k$ is less than
$2(L_{k-1}+4^{k-2})=2(4^{k-1}+4^{k-2})=\frac52 4^{k-1} < 4^k=L_k$.
Any sum involving at least one element of $A_k$ is greater than $L_k$.
Hence no sum of two elements from blocks $<k$ can equal a sum involving block $k$.
Therefore any collision $x_1+x_2=y_1+y_2$ in $A$ must occur within a single block $A_k$.
But each $A_k$ is Sidon (translation preserves Sidon), so the collision is trivial.
Thus $A$ is Sidon.

Finally, for $N\ge L_k$ we have $A_k\subseteq[1,N]$, so
\[|A\cap[1,N]|\ge |A_k|=|S_k|\ge \frac15 (4^{k-1})^{1/3} = \frac{1}{5\cdot 4^{1/3}}\,L_k^{1/3} \ge \frac{1}{5\cdot 4^{1/3}}\,N^{1/3}.
\]
So the lemma holds with $c:=\frac{1}{5\cdot 4^{1/3}}$. \qed

Proposition 5 (Negative range for $g$).
Let $g:\mathbb N\to(0,\infty)$ satisfy $g(N)\ge C N^{1/6}$ for all sufficiently large $N$.
Then the implication in the problem statement is false for this $g$.

Proof.
Let $A$ be the Sidon set from Lemma 4. Then $A(N)\ge cN^{1/3}$ for all $N$.
For large $N$ we have
\[cN^{1/3} \ge c\,\frac{N^{1/2}}{C N^{1/6}} = \frac{c}{C}\,\frac{N^{1/2}}{g(N)},\]
so $A(N)\gg N^{1/2}/g(N)$ holds.
But by Lemma 3, $\limsup r_A(n)\le 2$, so the conclusion $\limsup r_A(n)=\infty$ fails.
\qed

FAST REALITY CHECK (local computation).
For the first 25 terms of the greedy Sidon sequence
$1,2,4,8,13,21,31,45,66,81,97,123,148,182,204,252,290,361,401,475,565,593,662,775,822$,
I computed all ordered sums and found $\max_n r(n)=2$, consistent with Lemma 3.

5) VERIFICATION
- Lemma 1 is exact.
- Lemma 4 is fully proved and self-contained.
- Proposition 5 is an explicit counterexample range.

6) FINAL
**UNRESOLVED**

(i) Strongest fully proved partial result:
If $g(N)\ge C N^{1/6}$ eventually, then the implication is false (Proposition 5).
Also, if $A(N)\ge N^{1/2}h(N)$ infinitely often with $h(N)\to\infty$, then $\limsup r_A(n)=\infty$ (Corollary 2).

(ii) First gap:
Determine the threshold behavior for slower-growing $g$ (e.g. polylogarithmic), where Sidon-type counterexamples are not known
to be dense enough.

(iii) Top 3 next moves:
1. Improve the best explicit lower bound for infinite Sidon sets (problem #39) to push the negative range below $N^{1/6}$.
2. Prove a positive result for some explicit class of $g(N)\to\infty$ (e.g. $g(N)=\log N$).
3. Investigate whether sets with $A(N)\gg \sqrt N/\mathrm{polylog}(N)$ must have large additive energy forcing large $\max r_A(n)$.

(iv) Minimal counterexample structure (if the implication fails for very slow $g$):
An infinite set with near-$\sqrt N$ growth but uniformly bounded $r_A(n)$, i.e. an extremely dense infinite Sidon-type set.
