
OUTPUT
1) FORMAL RESTATEMENT
Let $A\subseteq\mathbb N$ be infinite and let $1_A$ be its indicator.
Define the representation function
\[r_A(n):=(1_A*1_A)(n)=|\{(a,b)\in A\times A: a+b=n\}|\]
(ordered representations).
Assume that $A+A$ contains all but finitely many integers (i.e. $\exists N_0$ such that every $n\ge N_0$ has $r_A(n)\ge 1$).
Conjecture: $\limsup_{n\to\infty} r_A(n)=\infty$. (Stronger: $\limsup r_A(n)/\log n>0$.)

2) QUICK LITERATURE/CONTEXT CHECK
Browsing is not available.

3) ATTACK PLAN
Proof track:
1. Use averaging: $\sum_{n\le 2N} r_A(n)=|A\cap[1,N]|^2$ to force large values of $r_A$ when $A$ is denser than $\sqrt N$.
2. Attempt to show any additive basis of order $2$ must have density spikes beyond $\sqrt N$.

Disproof track:
1. Try to build an additive basis $A$ with $|A\cap[1,N]|\asymp \sqrt N$ and uniformly bounded $r_A(n)$.

Chosen path: prove the averaging lemma and a conditional unboundedness result.

4) WORK
Lemma 1 (Representation sum identity).
Let $A_N:=A\cap\{1,\dots,N\}$ and $m(N):=|A_N|$. Then
\[\sum_{n=2}^{2N} r_{A_N}(n)=m(N)^2.
\]

Proof.
Each ordered pair $(a,b)\in A_N\times A_N$ contributes exactly once to the sum, namely to $r_{A_N}(a+b)$.
Summing over all $n$ counts all ordered pairs, giving $m(N)^2$. \qed

Corollary 2 (A pigeonhole lower bound).
There exists $n\in\{2,\dots,2N\}$ with
\[r_{A_N}(n)\ge \frac{m(N)^2}{2N-1}.
\]

Proof.
Average value of $r_{A_N}$ on $\{2,\dots,2N\}$ equals $m(N)^2/(2N-1)$ by Lemma 1, so the maximum is at least the average.
\qed

Proposition 3 (Conditional unboundedness from density).
If $m(N)\ge N^{1/2+\varepsilon}$ for infinitely many $N$, then $\limsup_{n\to\infty} r_A(n)=\infty$.

Proof.
Apply Corollary 2 along those $N$:
$r_{A_N}(n)\ge m(N)^2/(2N-1) \gg N^{2\varepsilon}$, which tends to infinity. Since $r_A(n)\ge r_{A_N}(n)$,
we get unboundedness of $r_A$. \qed

5) VERIFICATION
- Lemma 1 is purely double counting.
- Proposition 3 is conditional and does not use the hypothesis $A+A$ cofinite.

6) FINAL
**UNRESOLVED**

(i) Strongest fully proved partial result:
If $A$ has growth $|A\cap[1,N]|\ge N^{1/2+\varepsilon}$ infinitely often, then $r_A(n)$ is unbounded (Proposition 3).

(ii) First gap:
The cofinite-sumset hypothesis $A+A$ contains all large integers does not, by itself, force such density spikes,
and no argument here proves unboundedness when $|A\cap[1,N]|\asymp \sqrt N$.

(iii) Top 3 next moves:
1. Derive lower bounds on $|A\cap[1,N]|$ (or its fluctuations) from the assumption that $A+A$ is cofinite.
2. Study structured near-minimal bases of order 2 (e.g. greedy bases) and estimate their $r_A(n)$.
3. Attempt to adapt Fourier/additive energy methods to show bounded $r_A$ forces $A+A$ to miss infinitely many integers.

(iv) Minimal counterexample structure (if the conjecture were false):
An additive basis of order 2 with $|A\cap[1,N]|\asymp \sqrt N$ and with representation function $r_A(n)$ uniformly bounded.
