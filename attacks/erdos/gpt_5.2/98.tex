
OUTPUT
1) FORMAL RESTATEMENT
Definitions/Conventions.
Let $A\subset\mathbb R^2$ be a set of $n$ points such that:
- no three points of $A$ are collinear;
- no four points of $A$ are concyclic.
Let $D(A)$ be the number of distinct distances determined by pairs of points in $A$.
Define
\[h(n):=\min\{D(A): A\subset\mathbb R^2,\ |A|=n,\ A\text{ satisfies the two nondegeneracy conditions}\}.\]

Question.
Is it true that
\[\frac{h(n)}{n}\to\infty?\]

2) QUICK LITERATURE/CONTEXT CHECK
Web browsing is not available in this session.
I only use the statement as given; I will derive a simple linear lower bound from the ``no four concyclic'' condition.

3) ATTACK PLAN
Proof track ideas:
1. Use the nondegeneracy conditions to bound how often a single distance can repeat (especially from a fixed point), forcing many distinct distances.
2. Try to upgrade the linear lower bound to superlinear using global incidence constraints (hard).

Disproof track ideas:
1. Construct configurations satisfying the constraints but with only $O(n)$ distinct distances; any such construction would refute $h(n)/n\to\infty$.

Chosen path in this attempt: prove two elementary lemmas showing a linear lower bound $h(n)\ge (n-1)/3$ coming directly from the ``no four on a circle'' condition.

4) WORK
Lemma 1 (No four concyclic $\Rightarrow$ each point has $\ge (n-1)/3$ distinct distances).
Let $A$ satisfy ``no four points on a circle''.
Fix $x\in A$ and let $d(x)$ be the number of distinct distances from $x$ to other points:
\[d(x):=\big|\{|x-y|: y\in A\setminus\{x\}\}\big|.\]
Then
\[d(x)\ge \frac{n-1}{3}.\]

Proof.
For each radius $r>0$, the set $\{y\in A: |x-y|=r\}$ lies on the circle centered at $x$ of radius $r$.
By the hypothesis ``no four on a circle'', any circle contains at most $3$ points of $A$.
Therefore, for each $r$, there are at most $3$ points $y\in A$ at distance $r$ from $x$.
The $n-1$ points in $A\setminus\{x\}$ are partitioned into distance classes, each of size at most $3$.
Hence the number of classes is at least $(n-1)/3$, i.e. $d(x)\ge (n-1)/3$. \qed

Lemma 2 (A linear lower bound on $h(n)$).
For all $n\ge 2$,
\[h(n)\ge \frac{n-1}{3}.\]

Proof.
The total number of distinct distances $D(A)$ is at least $d(x)$ for any fixed $x\in A$ (since $D(A)$ includes all distances from $x$).
By Lemma 1, $d(x)\ge (n-1)/3$, so $D(A)\ge (n-1)/3$ for every admissible $A$.
Taking the minimum over $A$ gives $h(n)\ge (n-1)/3$. \qed

FAST REALITY CHECK
Lemma 2 gives only a linear lower bound $h(n)\ge n/3$.
The question asks whether $h(n)$ is \emph{superlinear}, i.e. $h(n)/n\to\infty$.
So the main difficulty is to prove that under these nondegeneracy constraints, many distances must appear (much more than linearly many).

5) VERIFICATION
- Lemma 1 uses only the fact that points at equal distance from $x$ lie on a circle centered at $x$, and the hypothesis that no circle contains 4 points of $A$.
- Corollary 2 correctly observes $D(A)\ge d(x)$ for any $x$.

6) FINAL
**UNRESOLVED**

(i) Strongest fully proved partial result:
I proved the linear lower bound $h(n)\ge (n-1)/3$ (Lemma 2), using only the ``no four concyclic'' condition.

(ii) First gap:
Upgrade the linear bound to superlinear, i.e. prove that $h(n)/n\to\infty$.
The local bound in Lemma 1 caps multiplicities around a \emph{fixed} center but does not control global repetition patterns across different centers.

(iii) Top 3 next moves (concrete):
1. Use energy/Cauchy--Schwarz: relate $D(A)$ to the number of distance-equality quadruples and attempt to bound those quadruples using the nondegeneracy assumptions.
2. Study explicit constructions satisfying the constraints (variants of grids with perturbations to avoid collinearities/concyclicities) and estimate their distinct-distance counts to see whether $O(n)$ might still be achievable.
3. Try to leverage the ``no three collinear'' condition together with circle intersection bounds to obtain stronger incidence estimates than in the unrestricted distinct distances problem.

(iv) What a minimal counterexample would likely look like (structure/parameters):
A family of sets $A_n$ with $|A_n|=n$ satisfying the two nondegeneracy conditions but with only $O(n)$ distinct distances,
meaning that despite low multiplicities on any one circle, the same distances repeat across many different centers in a coordinated way.
