% Attempt for Erdos Problem #77
% Following PROMPT_STRATEGY.MD
% Tools/Constraints:
% - Web browsing available? NO (restricted)
% - Computation available? YES (local), not needed
% GPT 5.2

OUTPUT
1) FORMAL RESTATEMENT
Definitions/Conventions.
- For $k\ge 2$, the diagonal Ramsey number $R(k)$ is the least integer $n$ such that every 2-coloring of the edges of $K_n$
  contains a monochromatic copy of $K_k$.
Equivalently, $R(k)$ is the least $n$ such that every graph on $n$ vertices contains either a clique of size $k$ or an independent set of size $k$.

Question.
Does the limit
\[\lim_{k\to\infty} R(k)^{1/k}\]
exist? If it exists, determine its value.

2) QUICK LITERATURE/CONTEXT CHECK
Web browsing is not available in this session.
The problem statement lists a long history of upper bounds and notes Erd\H{o}s proved $\sqrt2\le\liminf\le\limsup\le 4$.
I will prove the classical $\sqrt2$ lower bound on $\liminf$ and the $4$ upper bound on $\limsup$ from scratch below, but I will not address the deeper improvements.

3) ATTACK PLAN
Proof track ideas:
1. Prove subadditivity-type properties for $\log R(k)$ (if possible) to force existence of $\lim \frac{1}{k}\log R(k)$ by Fekete's lemma.
2. Improve upper/lower bounds on $\liminf,\limsup$.

Disproof track ideas:
1. Exhibit oscillation in $R(k)^{1/k}$ by proving different exponential rates along subsequences (no such phenomenon is known).

Chosen path in this attempt: establish two core inequalities with full proofs:
an exponential lower bound via the probabilistic method and an exponential upper bound via the classic Ramsey recursion.

4) WORK
Lemma 1 (Probabilistic lower bound: $R(k)$ is at least $\approx 2^{k/2}$).
For every integer $k\ge 6$,
\[R(k) > 2^{k/2}.\]
Consequently,
\[\liminf_{k\to\infty} R(k)^{1/k} \ge \sqrt{2}.\]

Proof.
Fix $k\ge 6$ and set $n:=\lfloor 2^{k/2}\rfloor$.
Consider a random red/blue coloring of the edges of $K_n$ where each edge is independently red with probability $1/2$ and blue with probability $1/2$.

For any fixed set $S$ of $k$ vertices, the probability that $S$ spans a red $K_k$ is $2^{-\binom{k}{2}}$, since all $\binom{k}{2}$ edges must be red.
Similarly, the probability that $S$ spans a blue $K_k$ is $2^{-\binom{k}{2}}$.
Thus the probability that $S$ spans a monochromatic $K_k$ (either color) is $2\cdot 2^{-\binom{k}{2}}$.

Let $X$ be the number of monochromatic $K_k$ subgraphs in the random coloring.
By linearity of expectation,
\[\mathbb E[X]=\binom{n}{k}\cdot 2\cdot 2^{-\binom{k}{2}}.\]
Using the crude bound $\binom{n}{k}\le \frac{n^k}{k!}$, we get
\[\mathbb E[X]\le \frac{n^k}{k!}\cdot 2\cdot 2^{-\binom{k}{2}}.\]
Since $n\le 2^{k/2}$, this gives
\[\mathbb E[X]\le \frac{(2^{k/2})^k}{k!}\cdot 2\cdot 2^{-\binom{k}{2}}
=\frac{2^{k^2/2}}{k!}\cdot 2\cdot 2^{-(k^2-k)/2}
=\frac{2^{1+k/2}}{k!}.\]
For $k\ge 6$, one has $k!\ge 720$ and $2^{1+k/2}\le 2^{1+3}=16$, so $\mathbb E[X]\le 16/720<1$.
Therefore there exists at least one coloring with $X=0$, i.e. with no monochromatic $K_k$.
This means $R(k)>n\ge 2^{k/2}-1$, and in particular $R(k)>2^{k/2}$ for all $k\ge 6$.

Taking $k$th roots and letting $k\to\infty$ gives $\liminf_{k\to\infty} R(k)^{1/k}\ge \sqrt2$. \qed

Lemma 2 (Classic recursion and exponential upper bound).
Let $R(s,t)$ denote the off-diagonal Ramsey number (least $n$ such that every 2-coloring of $E(K_n)$ contains a red $K_s$ or a blue $K_t$).
Then for integers $s,t\ge 2$,
\[R(s,t)\le R(s-1,t)+R(s,t-1).\]
Consequently, for $k\ge 2$,
\[R(k)=R(k,k)\le \binom{2k-2}{k-1} < 4^{k-1},\]
and hence
\[\limsup_{k\to\infty} R(k)^{1/k}\le 4.\]

Proof.
We first prove the recursion. Let
\[N:=R(s-1,t)+R(s,t-1).\]
Consider any red/blue coloring of the edges of $K_N$, and fix a vertex $v$.
Let $R$ be the set of vertices joined to $v$ by a red edge, and $B$ the set joined to $v$ by a blue edge.
Then $|R|+|B|=N-1$.
If $|R|\ge R(s-1,t)$, then by definition of $R(s-1,t)$ the induced coloring on $K_R$ contains either a red $K_{s-1}$ or a blue $K_t$.
In the first case, adding $v$ produces a red $K_s$; in the second case we already have a blue $K_t$.
Similarly, if $|B|\ge R(s,t-1)$, then inside $B$ there is either a red $K_s$ or a blue $K_{t-1}$; in the second case adding $v$ gives a blue $K_t$.

Since $|R|+|B|=N-1=(R(s-1,t)-1)+(R(s,t-1)-1)+1$, at least one of the inequalities
$|R|\ge R(s-1,t)$ or $|B|\ge R(s,t-1)$ must hold.
Thus every coloring of $K_N$ contains a red $K_s$ or a blue $K_t$, proving $R(s,t)\le N$.

Now prove the binomial bound by induction on $s+t$.
The base cases $R(1,t)=R(s,1)=1$ satisfy $1=\binom{s+t-2}{s-1}$ when one parameter is $1$.
Assuming $R(s-1,t)\le \binom{s+t-3}{s-2}$ and $R(s,t-1)\le \binom{s+t-3}{s-1}$, the recursion gives
\[R(s,t)\le \binom{s+t-3}{s-2}+\binom{s+t-3}{s-1}=\binom{s+t-2}{s-1}\]
by Pascal's identity.
Setting $s=t=k$ yields $R(k)\le \binom{2k-2}{k-1}$.
Finally, $\binom{2k-2}{k-1}\le 2^{2k-2}=4^{k-1}$, giving $\limsup R(k)^{1/k}\le 4$. \qed

FAST REALITY CHECK
For small $k$, one knows exact values only for a few $k$, and brute-force computation grows infeasible quickly.
The two lemmas above give the classic exponential window
\[\sqrt2\ \le\ \liminf R(k)^{1/k}\ \le\ \limsup R(k)^{1/k}\ \le\ 4,\]
matching the statement in the problem text (at a coarse level).

5) VERIFICATION
- Lemma 1: the expectation bound is explicit and uses only $\binom{n}{k}\le n^k/k!$ and $k!\ge 720$ for $k\ge 6$.
- Lemma 2: the recursion argument with a pivot vertex is standard; the induction to $\binom{2k-2}{k-1}$ uses only Pascal's identity.

6) FINAL
**UNRESOLVED**

(i) Strongest fully proved partial result:
I proved from scratch the classical exponential bounds
\[2^{k/2} < R(k) < 4^k\quad\text{(up to constant factors)}\]
for all large $k$, yielding $\sqrt2\le \liminf R(k)^{1/k}$ and $\limsup R(k)^{1/k}\le 4$ (Lemmas 1--2).

(ii) First gap:
Prove that $R(k)^{1/k}$ actually converges, i.e. show
\[\liminf_{k\to\infty} R(k)^{1/k}=\limsup_{k\to\infty} R(k)^{1/k}.\]
The standard recursion gives upper bounds but does not provide a subadditivity property strong enough to force existence of the limit.

(iii) Top 3 next moves (concrete):
1. Seek an inequality relating $R(k+\ell)$ to $R(k)$ and $R(\ell)$ that is close enough to multiplicativity/submultiplicativity to imply existence of $\lim \frac{1}{k}\log R(k)$ via an (almost) subadditive lemma.
2. Improve the exponential upper bound constant (known improvements exist per the problem text) and see whether the methods can also yield structural control that might imply convergence.
3. Investigate whether the lexicographic/tensor product constructions for lower bounds give any supermultiplicativity properties for the inverse Ramsey function that could translate into convergence information.

(iv) What a minimal counterexample would likely look like (structure/parameters):
A scenario where along two subsequences $k_i$ and $k'_j$, the ratios $\frac{1}{k_i}\log R(k_i)$ and $\frac{1}{k'_j}\log R(k'_j)$ converge to different limits,
forcing $R(k)^{1/k}$ to oscillate between two distinct constants.

