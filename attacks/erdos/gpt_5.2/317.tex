
OUTPUT
1) FORMAL RESTATEMENT
For each $n\ge 1$ and coefficients $\delta_1,\dots,\delta_n\in\{-1,0,1\}$ define
\[S(\delta)=\sum_{k=1}^n \frac{\delta_k}{k}\in\mathbb{Q}.\]

Question 1: Does there exist an absolute constant $c>0$ such that for every $n$ there exists $\delta$ with
\[0<|S(\delta)|<\frac{c}{2^n}?\]

Question 2: Let $L_n=\mathrm{lcm}(1,2,\dots,n)$. Is it true that for all sufficiently large $n$ and all $\delta$,
if $S(\delta)\ne 0$ then
\[|S(\delta)|>\frac{1}{L_n}?\]
(Non-strict $\ge 1/L_n$ is trivial; the issue is excluding equality $|S(\delta)|=1/L_n$.)

2) QUICK LITERATURE/CONTEXT CHECK
Browsing is not available. The statement mentions known weaker bounds and heuristics; I do not use them.

3) ATTACK PLAN
Proof track:
1. Rewrite $S(\delta)$ with common denominator $L_n$ and translate Question 2 into an integer representability problem.
2. For Question 1, compute minimal nonzero $|S(\delta)|$ for small $n$ and compare to $2^{-n}$.

Disproof track:
1. Try to construct $\delta$ giving $|S(\delta)|=1/L_n$ for infinitely many $n$ (would refute Question 2).

Chosen path: exact denominator lemmas + small-$n$ computation.

4) WORK
Lemma 1 (Common denominator bound).
Let $L_n=\mathrm{lcm}(1,2,\dots,n)$. For any $\delta\in\{-1,0,1\}^n$,
\[S(\delta)=\frac{A}{L_n}\]
for some integer $A$.
In particular, if $S(\delta)\ne 0$ then
\[|S(\delta)|\ge \frac{1}{L_n}.\]

Proof.
Write
\[S(\delta)=\frac{1}{L_n}\sum_{k=1}^n \delta_k\frac{L_n}{k}\]
and note each $L_n/k\in\mathbb{Z}$. The sum is an integer $A$.
If $A\ne 0$ then $|A|\ge 1$, giving the inequality. \qed

Lemma 2 (Reformulation of the “strict” question).
For $n\ge 1$, the equality $|S(\delta)|=1/L_n$ holds for some $\delta\in\{-1,0,1\}^n$ if and only if there exist $\delta_k\in\{-1,0,1\}$ such that
\[\sum_{k=1}^n \delta_k\frac{L_n}{k}=\pm 1.\tag{$\ast$}\]
Therefore Question 2 is equivalent to: for all sufficiently large $n$, the integer $\pm 1$ is not representable in the form $(\ast)$.

Proof.
By Lemma 1, $S(\delta)=A/L_n$ with $A=\sum \delta_k(L_n/k)\in\mathbb{Z}$.
Then $|S(\delta)|=1/L_n$ is equivalent to $|A|=1$, i.e. $A=\pm 1$, which is exactly $(\ast)$. \qed

FAST REALITY CHECK (computation).
For each $n\le 12$ I computed the minimum of $|S(\delta)|$ over all $\delta\in\{-1,0,1\}^n$ with $S(\delta)\ne 0$:
\[n=1:1;\ n=2:1/2;\ n=3:1/6;\ n=4:1/12;\ n=5:1/30;\ n=6:1/30;\ n=7:4/420;\ n=8:7/840;\]
\[n=9:4/2520;\ n=10:3/2520;\ n=11:12/27720;\ n=12:12/27720.\]
In particular, equality $|S|=1/L_n$ occurs for $n=1,2,3,4$ (e.g. $1/2-1/3-1/4=-1/12$ for $n=4$) but not for $5\le n\le 12$ in this search.
Also, the ratios $\min|S(\delta)|\cdot 2^n$ for $5\le n\le 12$ lie between $\approx 0.81$ and $\approx 2.13$.

5) VERIFICATION
- Lemma 1 is exact and accounts for the “obvious” non-strict inequality in Question 2.
- Lemma 2 correctly identifies the strictness problem with representability of $\pm 1$ by the integer weights $L_n/k$.
- The computation is exhaustive only up to $n=12$; it does not settle either question.

6) FINAL
**UNRESOLVED**

(i) Strongest fully proved partial result:
Every nonzero signed harmonic sum has size at least $1/L_n$ (Lemma 1), and the strictness question is equivalent to the nonexistence of integer representations $\sum \delta_k(L_n/k)=\pm 1$ for large $n$ (Lemma 2). Exhaustive computation shows equality occurs for $n\le 4$ but not for $5\le n\le 12$.

(ii) First gap:
Either (a) prove that for all sufficiently large $n$ the representation $\sum \delta_k(L_n/k)=\pm 1$ is impossible, or (b) exhibit infinitely many $n$ and explicit $\delta$ achieving it (refuting Question 2). For Question 1, prove existence of $\delta$ with $|S(\delta)|\ll 2^{-n}$ uniformly in $n$.

(iii) Top 3 next moves:
1. Study the integer set $\{L_n/k:1\le k\le n\}$ as a restricted signed subset-sum problem and try to exclude $\pm 1$ using congruences modulo carefully chosen primes.
2. Push exhaustive computation of the minimum nonzero $|S(\delta)|$ to larger $n$ using meet-in-the-middle and modular pruning, to guess the true scale (compare to $2^{-n}$ and $1/L_n$).
3. For Question 1, attempt a probabilistic construction of $\delta$ (random $\pm 1$ on a carefully selected subset of indices) and bound the minimum achievable value by a pigeonhole/entropy argument.

(iv) Minimal counterexample structure (if Question 2 were false):
Infinitely many $n$ for which there exists $\delta\in\{-1,0,1\}^n$ with $\sum_{k=1}^n \delta_k(L_n/k)=\pm 1$, i.e. a signed subset of the integers $L_n/k$ summing exactly to $\pm 1$.

