
OUTPUT
1) FORMAL RESTATEMENT
Definitions/Conventions.
Given a set $A\subset\mathbb R^2$ of $n$ distinct points, let
\[U(A):=\big|\{\{x,y\}\subset A:\ |x-y|=1\}\big|\]
be the number of unordered unit-distance pairs in $A$.

Question.
Is it true that
\[U(A)\ \le\ n^{1+O(1/\log\log n)}\]
for every $n$ and every $A$ of size $n$?

2) QUICK LITERATURE/CONTEXT CHECK
Web browsing is not available in this session.
The statement says:
- easy bound $U(A)=O(n^{3/2})$,
- best known $U(A)=O(n^{4/3})$ (SST),
and conjectured $n^{1+o(1)}$.
I will prove the easy $O(n^{3/2})$ bound from scratch.

3) ATTACK PLAN
Proof track ideas:
1. Use incidence bounds between points and circles to upper bound the number of unit distances.
2. Improve beyond $n^{4/3}$ requires Euclidean-specific tools (per statement); that is out of reach here.

Disproof track ideas:
1. Construct point sets with many unit distances; lattice points give $\Omega(n)$.

Chosen path in this attempt: (i) prove the classical $O(n^{3/2})$ bound via a K\H{o}v\'ari--S\'os--Tur\'an argument, and (ii) give an explicit $\Omega(n)$ construction from the integer grid.

4) WORK
Lemma 1 (An $O(n^{3/2})$ upper bound for unit distances).
For every set $A\subset\mathbb R^2$ of $n$ points,
\[U(A)\le n^{3/2}+n.\]

Proof.
Build a bipartite graph $B$ with left part $L=A$ (centers) and right part $R=A$ (points), and connect $x\in L$ to $y\in R$ iff $|x-y|=1$.
Then each unordered unit-distance pair $\{x,y\}$ contributes two directed edges $(x,y)$ and $(y,x)$, so
\[e(B)=2U(A).\tag{1}\]

Two circles in the plane intersect in at most two points.
Therefore, for two distinct centers $x\ne x'$, the unit circles around them have at most two common points, meaning $x$ and $x'$ have at most two common neighbors in $R$.
Equivalently, $B$ is $K_{2,3}$-free with the two-vertex side in $L$.

We now bound $e(B)$ directly.
Let $d(r)$ be the degree of a right vertex $r\in R$.
Count length-2 paths $\ell-r-\ell'$ with $\ell\ne \ell'$ and $r\in R$:
\[P:=\sum_{r\in R}\binom{d(r)}{2}.\]
For a fixed unordered pair $\{\ell,\ell'\}\subseteq L$, the number of common neighbors of $\ell$ and $\ell'$ in $R$ is at most $2$ (else we would have three points lying on the intersection of two unit circles).
Thus $P\le 2\binom{n}{2}<n^2$.

On the other hand,
\[P=\sum_{r}\binom{d(r)}{2}=\frac12\sum_r d(r)^2-\frac12\sum_r d(r)\ge \frac12\cdot\frac{e(B)^2}{n}-\frac12 e(B)\]
by Cauchy--Schwarz ($\sum_r d(r)^2\ge e(B)^2/n$) and $\sum_r d(r)=e(B)$.
Hence
\[\frac{e(B)^2}{2n}-\frac{e(B)}{2}\le n^2.\]
Solving this quadratic inequality gives $e(B)\le 2n^{3/2}+2n$ (for instance, $(e(B)-n/2)^2\le 2n^3+n^2/4\le (2n^{3/2}+n)^2$).
Combine with (1) to get $U(A)=e(B)/2\le n^{3/2}+n$. \qed

Lemma 2 (Grid lower bound $\Omega(n)$).
Let $m\ge 2$ and let $A$ be the $m\times m$ integer grid
\[A:=\{(i,j): i,j\in\{1,2,\dots,m\}\},\]
so $|A|=n=m^2$.
Then
\[U(A)=2m(m-1)=2n-2\sqrt n.\]

Proof.
Unit distances in the integer grid occur exactly between horizontally adjacent points and vertically adjacent points.
There are $m$ rows, each with $m-1$ horizontal unit edges, giving $m(m-1)$ pairs.
There are $m$ columns, each with $m-1$ vertical unit edges, giving another $m(m-1)$ pairs.
No other pairs are at Euclidean distance $1$ since the squared distance between two distinct integer lattice points is an integer and equals $1$ only for these axis-adjacent pairs.
Thus $U(A)=2m(m-1)=2m^2-2m=2n-2\sqrt n$. \qed

FAST REALITY CHECK
Lemma 1 matches the ``easy'' $O(n^{3/2})$ bound stated in the problem text.
Lemma 2 shows $U(A)=\Theta(n)$ is achievable, so any upper bound of the form $n^{1+o(1)}$ would be asymptotically tight up to subpolynomial factors.

5) VERIFICATION
- Lemma 1 reduces unit distances to incidences between points and unit circles and uses only the geometric fact ``two circles meet in at most 2 points'' plus KST.
- Lemma 2 counts unit edges explicitly and checks there are no diagonal unit distances in the integer grid.

6) FINAL
**UNRESOLVED**

(i) Strongest fully proved partial result:
I proved the classical easy bound $U(A)\le n^{3/2}+n$ (Lemma 1) and the grid construction with $U(A)=2n-2\sqrt n$ (Lemma 2).

(ii) First gap:
Improve the exponent $3/2$ toward $1+o(1)$, i.e. exploit Euclidean-specific structure beyond the circle intersection bound.

(iii) Top 3 next moves (concrete):
1. Replace the $K_{2,3}$-free extremal argument by a stronger incidence theorem (Szemer\'{e}di--Trotter) for point-circle incidences to reach exponents below $3/2$.
2. Identify which step in the SST84 proof uses Euclidean geometry in an essential way and see whether it can be pushed below $4/3$.
3. Explore additional restrictions (e.g. convex position, no three collinear) that yield better bounds (see Problem \#96 and \#92).

(iv) What a minimal counterexample would likely look like (structure/parameters):
A family of point sets $A_n$ with $|A_n|=n$ such that $U(A_n)\ge n^{1+\delta}$ for some fixed $\delta>0$,
which would disprove the conjectured $n^{1+o(1)}$ upper bound.
