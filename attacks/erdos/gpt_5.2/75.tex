% Attempt for Erdos Problem #75
% Following PROMPT_STRATEGY.MD
% Tools/Constraints:
% - Web browsing available? NO (restricted)
% - Computation available? YES (local), not needed
% GPT 5.2

OUTPUT
1) FORMAL RESTATEMENT
Definitions/Conventions.
- Graphs may be infinite.
- $\chi(G)$ is the chromatic number (least cardinal of a proper vertex coloring).
- For a graph $H$, $\alpha(H)$ is the independence number (largest size of an independent set).
- ``$H$ is a subgraph on $n$ vertices'' means: $|V(H)|=n$ and $H$ is (not necessarily induced) a subgraph of $G$.

Question.
Does there exist a graph $G$ with $\chi(G)=\aleph_1$ such that for every $\varepsilon>0$ there exists $n_0(\varepsilon)$ with the property:
for every integer $n\ge n_0(\varepsilon)$ and every subgraph $H\subseteq G$ with $|V(H)|=n$,
\[\alpha(H) > n^{1-\varepsilon}?\]

2) QUICK LITERATURE/CONTEXT CHECK
Web browsing is not available in this session.
I only use the statement as given; I do not rely on external results beyond elementary graph theory inequalities proved below.

3) ATTACK PLAN
Proof track ideas:
1. Translate the local independent-set requirement into local sparsity bounds on subgraphs (average degree must be $n^{o(1)}$), then attempt to construct an $\aleph_1$-chromatic graph that is locally extremely sparse.
2. Explore set-theoretic constructions of graphs of chromatic number $\aleph_1$ with strong anti-Ramsey properties.

Disproof track ideas:
1. Prove that any $\aleph_1$-chromatic graph necessarily contains finite subgraphs with relatively small independence number (e.g. bounded by $n^{1-c}$ for some $c>0$), contradicting the requirement for all $\varepsilon$.

Chosen path in this attempt: derive sharp necessary conditions on finite subgraphs $H$ that would follow from the inequality $\alpha(H)>n^{1-\varepsilon}$, to clarify what such a $G$ must look like.

4) WORK
Lemma 1 (Greedy bound from maximum degree).
Let $H$ be a finite graph on $n$ vertices with maximum degree $\Delta(H)=\Delta$.
Then
\[\alpha(H)\ge \frac{n}{\Delta+1}.\]

Proof.
Run the greedy algorithm: pick any vertex $v_1$, add it to an independent set, and delete $v_1$ and all its neighbors from the graph.
Repeat until no vertices remain.
At each step, we remove at most $\Delta+1$ vertices (the chosen vertex and its at most $\Delta$ neighbors), and we add exactly one vertex to the independent set.
Thus after $t$ steps we have removed at most $t(\Delta+1)$ vertices. To remove all $n$ vertices we need $t(\Delta+1)\ge n$, hence $t\ge n/(\Delta+1)$.
The algorithm produces an independent set of size $t$, so $\alpha(H)\ge n/(\Delta+1)$. \qed

Lemma 2 (Greedy bound from average degree; local sparsity consequence).
Let $H$ be a finite graph on $n$ vertices with $m$ edges and average degree $d_{\mathrm{avg}}=2m/n$.
Then
\[\alpha(H)\ge \frac{n}{d_{\mathrm{avg}}+1}=\frac{n^2}{2m+n}.\]
In particular, if $\alpha(H)>n^{1-\varepsilon}$ then $m<\tfrac12 n^{1+\varepsilon}$ for all sufficiently large $n$.

Proof.
At each step of the greedy algorithm, choose a vertex of degree at most the current average degree.
Since the average degree is $d_{\mathrm{avg}}$, there exists a vertex of degree $\le d_{\mathrm{avg}}$.
Removing such a vertex and its neighbors deletes at most $d_{\mathrm{avg}}+1$ vertices per chosen independent vertex.
By the same counting as in Lemma 1, this yields an independent set of size at least $n/(d_{\mathrm{avg}}+1)$.
The expression $n/(d_{\mathrm{avg}}+1)$ equals $n^2/(2m+n)$.

For the consequence: if $\alpha(H)>n^{1-\varepsilon}$, then in particular
\[\frac{n^2}{2m+n}\ge \alpha(H) > n^{1-\varepsilon}.\]
Rearranging gives $n/(2m+n)>n^{-\varepsilon}$, i.e. $2m+n < n^{1+\varepsilon}$, hence $m<\tfrac12(n^{1+\varepsilon}-n)$.
For large $n$ this implies $m<\tfrac12 n^{1+\varepsilon}$. \qed

Lemma 3 (Large independent sets imply small chromatic number of finite subgraphs).
Let $H$ be a finite graph on $n$ vertices with $\alpha(H)\ge n^{1-\varepsilon}$.
Then
\[\chi(H)\le n^{\varepsilon}.\]

Proof.
Iteratively remove an independent set of maximum size from the current graph and assign it a new color.
At each step, the current graph has at most $n$ vertices, so the hypothesis $\alpha(\cdot)\ge (\#V)^{1-\varepsilon}$ is not automatically inherited.
However, we only need a crude upper bound: since $\alpha(H)\ge n^{1-\varepsilon}$, the first color class removes at least $n^{1-\varepsilon}$ vertices.
Then the remaining graph has at most $n-n^{1-\varepsilon}$ vertices; apply the same argument with the trivial bound $\alpha(\cdot)\ge 1$ to continue.

A simpler direct argument avoids iteration: by definition, $\chi(H)\ge n/\alpha(H)$ because a proper coloring partitions $V(H)$ into $\chi(H)$ independent sets, each of size at most $\alpha(H)$.
Thus
\[\chi(H)\le \frac{n}{\alpha(H)}\le \frac{n}{n^{1-\varepsilon}}=n^{\varepsilon}.\qed\]

FAST REALITY CHECK
The condition in the problem would force every large $n$-vertex subgraph $H$ to be extremely sparse:
by Lemma 2, $e(H)=O(n^{1+\varepsilon})$ for every fixed $\varepsilon>0$ (hence $e(H)=n^{1+o(1)}$).
Also, by Lemma 3, every such finite subgraph would have very small chromatic number $\chi(H)\le n^{\varepsilon}$ for every fixed $\varepsilon>0$.
So the challenge is to reconcile these strong local constraints with a global chromatic number of $\aleph_1$.

5) VERIFICATION
- Lemma 1 is a standard greedy packing argument.
- Lemma 2 uses the same greedy idea but with a degree $\le$ average degree vertex; the algebraic rearrangement is checked.
- Lemma 3 uses only the inequality $\chi(H)\ge n/\alpha(H)$ coming from partitioning into independent sets.

6) FINAL
**UNRESOLVED**

(i) Strongest fully proved partial result:
Any graph satisfying the problem's local condition must have all large $n$-vertex subgraphs extremely sparse: $e(H)<\tfrac12 n^{1+\varepsilon}$ for every fixed $\varepsilon>0$ (Lemma 2),
and in particular such subgraphs have small chromatic number $\chi(H)\le n^{\varepsilon}$ (Lemma 3).

(ii) First gap:
Construct an $\aleph_1$-chromatic graph that is locally this sparse (and hence locally has huge independent sets), or prove that any $\aleph_1$-chromatic graph must contain finite subgraphs violating these bounds.

(iii) Top 3 next moves (concrete):
1. Investigate known $\aleph_1$-chromatic graph constructions (e.g. ladder systems, comparability graphs, shift graphs) and compute/estimate $\alpha(H)$ for their finite subgraphs.
2. Try to prove a universal finite-subgraph theorem for $\aleph_1$-chromatic graphs (analogous to the fact they contain all large odd cycles) that forces some finite subgraph with $\alpha(H)\le n^{1-c}$ for some fixed $c>0$.
3. Attempt a forcing-style construction (if allowed by set theory) where $\chi(G)=\aleph_1$ but finite subgraphs are all locally sparse; determine which axioms are needed.

(iv) What a minimal counterexample would likely look like (structure/parameters):
An $\aleph_1$-chromatic graph $G$ together with a sequence of finite subgraphs $H_n\subseteq G$ on $n$ vertices
with $\alpha(H_n)\le n^{1-\varepsilon_0}$ for some fixed $\varepsilon_0>0$ (violating the requirement for that $\varepsilon_0$).

