
OUTPUT
1) FORMAL RESTATEMENT
A positive integer $m$ is \emph{practical} if every integer $1\le n < m$ can be written as a sum of distinct divisors of $m$.
If $m$ is practical, define $h(m)$ to be the least integer $H$ such that every $1\le n < m$ can be written as a sum of at most
$H$ distinct divisors of $m$.
Questions (as stated):
(a) Are there infinitely many practical $m$ with $h(m) < (\log\log m)^{O(1)}$?
(b) Is it true that $h(n!)<n^{o(1)}$? Or even $h(n!)<(\log n)^{O(1)}$?

2) QUICK LITERATURE/CONTEXT CHECK
Browsing is not available. I only record what the problem statement itself claims.
I have not verified those results here.

3) ATTACK PLAN
Proof track:
1. For explicit families of practical numbers, compute/estimate $h(m)$ by a greedy algorithm.
2. For $m=n!$, exploit the density of divisors to represent all $<m$ using few divisors.

Disproof track:
1. Produce lower bounds on $h(m)$ for large classes of practical $m$ (e.g. via information-theoretic counting).
2. Attempt to show $h(n!)$ must grow at least like a power of $\log n$.

Chosen path: work out a fully explicit family ($m=2^k$) and its exact $h(m)$.

4) WORK
Lemma 1 ($2^k$ is practical).
For every integer $k\ge 1$, the number $m=2^k$ is practical.

Proof.
The divisors of $2^k$ are $1,2,4,\dots,2^k$. Every integer $0\le n<2^k$ has a binary expansion
$n=\sum_{j=0}^{k-1} \epsilon_j 2^j$ with $\epsilon_j\in\{0,1\}$, i.e. a sum of distinct divisors of $2^k$.
Thus $2^k$ is practical. \qed

Lemma 2 (Upper bound for $h(2^k)$).
For $m=2^k$, one has $h(m)\le k$.

Proof.
By Lemma 1, every $n<2^k$ is a sum of distinct powers of $2$ less than $2^k$, and there are at most $k$ such powers.
So $k$ divisors always suffice. \qed

Lemma 3 (Lower bound for $h(2^k)$).
For $m=2^k$, one has $h(m)\ge k$.

Proof.
The integer $2^k-1$ has binary expansion with all $k$ lower bits equal to $1$:
$2^k-1=1+2+\cdots+2^{k-1}$. Any sum of distinct divisors of $2^k$ uses distinct powers of $2$.
To obtain $2^k-1$ one must use each of $1,2,\dots,2^{k-1}$, hence at least $k$ divisors.
Thus $h(2^k)\ge k$. \qed

Corollary 4 (Exact value).
For $m=2^k$, $h(m)=k$.

Proof.
Combine Lemma 2 and Lemma 3. \qed

5) VERIFICATION
- Lemmas 1--3 are purely about binary expansion and use no external results.
- Edge cases: $k=1$ gives $m=2$, and indeed $h(2)=1$.

6) FINAL
**UNRESOLVED**

(i) Strongest fully proved partial result:
An explicit infinite family of practical numbers has exact growth $h(2^k)=k=\log_2(2^k)$ (Corollary 4).

(ii) First gap:
No argument here produces practical numbers with $h(m)$ as small as a polylogarithm of $\log m$,
or controls $h(n!)$ beyond the trivial constructions.

(iii) Top 3 next moves:
1. Prove general lower bounds on $h(m)$ in terms of divisor structure (e.g. via counting possible subset sums).
2. For $m=n!$, design a constructive algorithm representing all $<m$ using very few divisors and prove its correctness.
3. Search for families of practical numbers with exceptionally dense divisors (e.g. highly composite numbers) and estimate $h(m)$.

(iv) Minimal counterexample structure (if $h(n!)<(\log n)^{O(1)}$ were false):
A sequence $n_j$ for which there exists some $t<n_j!$ that requires \emph{many} divisors in any subset-sum representation,
forcing $h(n_j!)$ to grow faster than any power of $\log n_j$.
