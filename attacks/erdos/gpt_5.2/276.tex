
OUTPUT
1) FORMAL RESTATEMENT
Fix integers $a_0,a_1\in\mathbb{Z}$ and define $(a_n)_{n\ge 0}$ by the Lucas/Fibonacci recurrence
\[a_{n+2}=a_{n+1}+a_n\qquad(n\ge 0).\]
Call the sequence \emph{composite} if every $a_n$ is a composite integer (in particular $|a_n|>1$ and $a_n$ is not prime).
The condition “no integer has a common factor with every term” means
\[\gcd(a_0,a_1,a_2,\dots)=1.\]

The first (well-posed) question is: does there exist such a sequence?
The problem statement says Graham exhibited explicit $a_0,a_1$ producing one.

The second question (“without an underlying system of covering congruences responsible”) is not formally defined in the statement; see (ii) in FINAL for a precise blocker and a minimal candidate formalization.

2) QUICK LITERATURE/CONTEXT CHECK
Browsing is not available. I only record what the statement claims: Graham constructed an explicit composite Lucas sequence with $\gcd$ of all terms equal to $1$, using covering systems; and there is a conjectural construction “not explained by a covering system”. I have not verified the literature claims here.

3) ATTACK PLAN
Proof track:
1. Prove structural facts: describe $\gcd$ of the whole sequence; periodicity modulo $m$; forced divisibility patterns.
2. For a proposed “no covering system responsible” formalization, try to show it is satisfied (or violated) by known constructions.

Disproof track:
1. If a strong formalization is adopted, try to show every composite Lucas sequence necessarily admits a finite covering mechanism (using periodicity mod primes).

Chosen path: establish rigorous invariants and do a concrete sanity check on Graham’s explicit initial values; the “no covering system” part remains ill-defined.

4) WORK
Lemma 1 (Constant gcd of consecutive terms).
Let $(a_n)$ satisfy $a_{n+2}=a_{n+1}+a_n$. Then for all $n\ge 0$,
\[\gcd(a_n,a_{n+1})=\gcd(a_0,a_1).\]

Proof.
For every $n\ge 0$,
\[\gcd(a_{n+1},a_{n+2})=\gcd(a_{n+1},a_{n+1}+a_n)=\gcd(a_{n+1},a_n).\]
Thus $\gcd(a_n,a_{n+1})$ is constant in $n$ by induction. \qed

Lemma 2 (Common divisor of all terms).
Let $d=\gcd(a_0,a_1)$. Then $d\mid a_n$ for every $n\ge 0$. Moreover, if an integer $q$ divides every $a_n$, then $q\mid d$.
Consequently,
\[\gcd(a_0,a_1,a_2,\dots)=\gcd(a_0,a_1).\]

Proof.
If $d\mid a_0,a_1$ then by the recurrence $d\mid a_2=a_1+a_0$, and inductively $d\mid a_n$ for all $n$.
Conversely, if $q\mid a_n$ for all $n$ then in particular $q\mid a_0$ and $q\mid a_1$, so $q\mid \gcd(a_0,a_1)=d$.
This characterizes the gcd of the whole sequence as $d$. \qed

Lemma 3 (Pure periodicity modulo $m$).
Fix $m\ge 2$. The sequence of pairs $(a_n\bmod m,\ a_{n+1}\bmod m)$ is purely periodic, with period at most $m^2$.
In particular, $(a_n\bmod m)$ is periodic with period at most $m^2$.

Proof.
Work in $(\mathbb{Z}/m\mathbb{Z})^2$ and define the update map
\[T(x,y)=(y,x+y)\pmod m.\]
Then $T$ is bijective with inverse $T^{-1}(x,y)=(y-x,x)\pmod m$.
Therefore the orbit of the initial pair $(a_0\bmod m,\ a_1\bmod m)$ under $T$ lies in a finite set of size $m^2$ and, since $T$ is bijective, must be a cycle (no preperiod).
Hence the pair sequence is periodic with some period $\le m^2$. \qed

FAST REALITY CHECK (computation, Graham’s explicit $a_0,a_1$).
With
\[a_0 = 1786772701928802632268715130455793,\quad a_1 = 1059683225053915111058165141686995,\]
I computed $\gcd(a_0,a_1)=1$, so Lemma 2 confirms the “no common factor with every term” condition is equivalent to this gcd being $1$.
For the first few terms, a quick trial division search found small prime factors for $a_1,\dots,a_9$ (e.g. $3\mid a_1$, $2\mid a_2$, $7\mid a_3$, $17\mid a_4$); for $a_0$ no prime factor $\le 1000$ was found (I did not factor $a_0$).

5) VERIFICATION
- Lemmas 1–2 are exact equalities about gcd/divisibility and use only the recurrence.
- Lemma 3 relies on invertibility of the state update map modulo $m$; the explicit inverse is checked directly.
- The computational check only verifies $\gcd(a_0,a_1)=1$ and small factors for early terms; it does not verify “all $a_n$ composite”.

6) FINAL
**UNRESOLVED**

(i) Strongest fully proved partial result:
Any common divisor of all terms equals $\gcd(a_0,a_1)$ (Lemma 2), and the sequence is purely periodic modulo every $m$ with period $\le m^2$ (Lemma 3). For Graham’s stated initial values, $\gcd(a_0,a_1)=1$.

(ii) First gap:
The phrase “without an underlying system of covering congruences responsible” is undefined. A minimal formalization I cannot justify from the statement alone is:
there does not exist a \emph{finite} set of congruences $n\equiv r_j\pmod{M_j}$ together with primes $p_j$ such that for every $n\ge 0$ there is some $j$ with $n\equiv r_j\pmod{M_j}$ and $p_j\mid a_n$.
Under any such formalization, I do not have a proof or disproof.

(iii) Top 3 next moves:
1. Fix an explicit formal definition of “covering system responsible” (finite vs infinite cover; whether primes may repeat; whether one covers all $n$ or only all large $n$).
2. For a proposed Lucas sequence, compute $(a_n\bmod p)$ periods for many small primes $p$ and attempt to \emph{algorithmically} extract a finite covering system if one exists.
3. Search for smaller initial conditions $(a_0,a_1)$ producing long runs of composite terms with $\gcd(a_0,a_1)=1$, to look for patterns contradicting (or supporting) any covering-system hypothesis.

(iv) Minimal counterexample structure (if the “no covering system” conjecture were false):
Every composite Lucas sequence with $\gcd(a_0,a_1)=1$ would admit a finite list of congruence classes of indices $n$ whose union is all of $\mathbb{N}$, each class forcing divisibility of $a_n$ by a predetermined prime. Such a certificate would exploit the periodicity in Lemma 3 in an essentially finite way.

