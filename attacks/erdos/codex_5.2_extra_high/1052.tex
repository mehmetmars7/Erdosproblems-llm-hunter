
OUTPUT
1) FORMAL RESTATEMENT
Let $n\ge 1$ be an integer. A divisor $d\mid n$ is **unitary** if $\gcd(d,n/d)=1$. Let
\[
U(n) := \{d \mid n : \gcd(d,n/d)=1\}
\]
be the set of unitary divisors, and let
\[
\sigma^*(n) := \sum_{d\in U(n)} d
\]
be the unitary divisor sum. Define $n$ to be **unitary perfect** if
\[
\sum_{\substack{d\in U(n) \\ d<n}} d = n.
\]
Equivalently (see Lemma 1 below), $n$ is unitary perfect iff $\sigma^*(n)=2n$.

The open question is whether the set
\[
\{n\in \mathbb{N} : n \text{ is unitary perfect}\}
\]
is finite.

2) QUICK LITERATURE/CONTEXT CHECK
Browsing is not available. I only record what the problem statement itself claims:
- No odd unitary perfect numbers exist.
- Five examples are known: $6,60,90,87360,146361946186458562560000$.
I do not verify those claims here.

3) ATTACK PLAN
Proof track:
1. Characterize unitary divisors via prime factorization and derive the product formula for $\sigma^*(n)$.
2. Use the product formula to prove structural constraints (e.g., no odd unitary perfect numbers).
3. Classify small-parameter cases (e.g., exactly two distinct primes) by direct algebra.

Disproof track:
1. Attempt to construct an infinite family solving $\sigma^*(n)=2n$.
2. Seek parameterized solutions with fixed number of primes or fixed $2$-adic valuation.

Chosen path: provide fully proved structural lemmas and a complete classification when $n$ has exactly two prime factors;
leave finiteness open.

4) WORK
Lemma 1 (Proper vs. full unitary sum).
For $n\ge 1$, let $U(n)$ be the set of unitary divisors and $\sigma^*(n)=\sum_{d\in U(n)} d$. Then
\[
\sum_{\substack{d\in U(n) \\ d<n}} d = n \quad\Longleftrightarrow\quad \sigma^*(n)=2n.
\]

Proof.
Since $n\mid n$ and $\gcd(n,n/n)=\gcd(n,1)=1$, we have $n\in U(n)$. Hence
\[
\sigma^*(n) = n + \sum_{\substack{d\in U(n) \\ d<n}} d.
\]
Therefore the proper-unitary sum equals $n$ iff $\sigma^*(n)=2n$. ∎

Lemma 2 (Unitary divisors of a prime power).
If $n=p^a$ with $p$ prime and $a\ge 1$, then $U(n)=\{1,p^a\}$.

Proof.
Every divisor of $p^a$ has the form $p^k$ with $0\le k\le a$. We have
$\gcd(p^k,p^{a-k})=p^{\min(k,a-k)}$. This equals $1$ iff $\min(k,a-k)=0$, i.e.,
$k=0$ or $k=a$. Thus the only unitary divisors are $1$ and $p^a$. ∎

Corollary 3 (No prime power is unitary perfect).
If $n=p^a$ with $a\ge 1$, then $n$ is not unitary perfect.

Proof.
By Lemma 2, the proper unitary divisors sum to $1$, so Lemma 1 gives
$\sigma^*(n)=1+p^a\ne 2p^a$ for $p^a>1$. Hence $n$ is not unitary perfect. ∎

Lemma 4 (Unitary divisors via prime factorization).
Let $n=\prod_{i=1}^k p_i^{a_i}$ be the prime factorization with distinct primes $p_i$.
A divisor $d\mid n$ is unitary iff for each $i$, the exponent of $p_i$ in $d$ is either $0$ or $a_i$.
Consequently,
\[
U(n)=\Bigl\{\prod_{i\in S} p_i^{a_i} : S\subseteq \{1,\dots,k\}\Bigr\}.
\]

Proof.
Write $d=\prod p_i^{b_i}$ with $0\le b_i\le a_i$. Then
$\gcd(d,n/d)=\prod p_i^{\min(b_i,a_i-b_i)}$. This equals $1$ iff
$\min(b_i,a_i-b_i)=0$ for all $i$, i.e., $b_i\in\{0,a_i\}$ for each $i$.
This gives the stated characterization. ∎

Lemma 5 (Product formula for $\sigma^*$).
With the notation of Lemma 4,
\[
\sigma^*(n)=\prod_{i=1}^k (1+p_i^{a_i}).
\]

Proof.
By Lemma 4, unitary divisors correspond bijectively to subsets $S$ of $\{1,\dots,k\}$,
with divisor $d_S=\prod_{i\in S} p_i^{a_i}$. Summing over all subsets gives
\[
\sigma^*(n)=\sum_{S} \prod_{i\in S} p_i^{a_i} = \prod_{i=1}^k (1+p_i^{a_i}),
\]
by the distributive expansion of the product. ∎

Lemma 6 (No odd unitary perfect numbers).
If $n$ is unitary perfect, then $n$ is even.

Proof.
Suppose $n$ is odd and write $n=\prod_{i=1}^k p_i^{a_i}$ with odd primes $p_i$.
Then each factor $1+p_i^{a_i}$ is even, so by Lemma 5, $\sigma^*(n)$ is divisible by $2^k$.
If $k\ge 2$, then $\sigma^*(n)$ is divisible by $4$, but $\sigma^*(n)=2n$ has only a single factor of $2$.
Hence $k=1$, so $n=p^a$ is an odd prime power, contradicting Corollary 3. ∎

Lemma 7 (Two-prime classification).
If $n$ has exactly two distinct prime factors and is unitary perfect, then $n=6$.

Proof.
Let $n=p^a q^b$ with distinct primes $p,q$ and $a,b\ge 1$. By Lemma 4,
$U(n)=\{1,p^a,q^b,p^a q^b\}$, so the proper-unitary sum is $1+p^a+q^b$.
Unitary perfectness gives
\[
p^a q^b = 1+p^a+q^b.
\]
Rearrange:
\[(p^a-1)(q^b-1)=2.
\]
Since $p^a-1$ and $q^b-1$ are positive integers, the only possibility is
$(p^a-1,q^b-1)=(1,2)$ or $(2,1)$. Thus $(p^a,q^b)=(2,3)$ or $(3,2)$, so $n=6$. ∎

Lemma 8 (Multiplicativity of $\sigma^*$).
If $\gcd(a,b)=1$, then $\sigma^*(ab)=\sigma^*(a)\sigma^*(b)$.

Proof.
Write the prime factorization of $ab$ as the disjoint union of the prime factors of $a$ and $b$.
By Lemma 5, $\sigma^*(ab)$ is the product over all prime powers dividing $ab$ of $(1+p^e)$.
Since $\gcd(a,b)=1$, this product splits into the product for $a$ and the product for $b$,
which equals $\sigma^*(a)\sigma^*(b)$. ∎

Lemma 9 (2-adic budget constraint).
Let $n$ be unitary perfect and write $n=2^m \prod_{i=1}^r p_i^{a_i}$ with odd primes $p_i$.
Then
\[
\sum_{i=1}^r \nu_2(1+p_i^{a_i}) = m+1,
\]
and in particular $r \le m+1$.

Proof.
By Lemma 5 and multiplicativity (Lemma 8),
\[
\sigma^*(n)=(1+2^m)\prod_{i=1}^r (1+p_i^{a_i}).
\]
Since $m\ge 1$ (Lemma 6), the factor $1+2^m$ is odd, hence
\[
\nu_2(\sigma^*(n)) = \sum_{i=1}^r \nu_2(1+p_i^{a_i}).
\]
Unitary perfectness gives $\sigma^*(n)=2n$, so $\nu_2(\sigma^*(n))=\nu_2(2n)=m+1$.
Each odd $p_i$ satisfies $1+p_i^{a_i}$ even, so each summand is at least $1$, yielding $r\le m+1$. ∎

Lemma 10 (Normalized product identity).
With $n=2^m \prod_{i=1}^r p_i^{a_i}$ as above, unitary perfectness is equivalent to
\[
\prod_{i=1}^r \left(1+\frac{1}{p_i^{a_i}}\right)=\frac{2^{m+1}}{1+2^m}.
\]

Proof.
Divide the identity $\sigma^*(n)=2n$ by $n$ and use Lemma 5 to get
\[
\frac{\sigma^*(n)}{n} = \frac{1+2^m}{2^m}\prod_{i=1}^r \left(1+\frac{1}{p_i^{a_i}}\right)=2.
\]
Rearranging gives the stated formula. ∎

Lemma 11 (Odd part divisible by $1+2^m$).
Let $n=2^m \prod_{i=1}^r p_i^{a_i}$ be unitary perfect. Then $1+2^m$ divides
the odd part $\prod_{i=1}^r p_i^{a_i}$.

Proof.
From Lemma 10,
\[
(1+2^m)\prod_{i=1}^r (p_i^{a_i}+1)=2^{m+1}\prod_{i=1}^r p_i^{a_i}.
\]
Since $\gcd(1+2^m,2^{m+1})=1$, the odd factor $1+2^m$ must divide
$\prod_{i=1}^r p_i^{a_i}$. ∎

Corollary 11a (If $m$ is odd, then $3\mid n$).
If $n$ is unitary perfect and $m=\nu_2(n)$ is odd, then $3$ divides $n$.

Proof.
If $m$ is odd, then $2^m\equiv 2 \pmod{3}$, so $1+2^m\equiv 0 \pmod{3}$.
By Lemma 11, $1+2^m$ divides the odd part of $n$, hence $3\mid n$. ∎

Lemma 12 (One odd prime factor case).
If $n=2^m p^a$ with $p$ odd prime is unitary perfect, then $n=6$.

Proof.
Lemma 10 gives
\[
1+\frac{1}{p^a}=\frac{2^{m+1}}{1+2^m},
\]
hence
\[
p^a=\frac{1+2^m}{2^m-1}=1+\frac{2}{2^m-1}.
\]
Thus $2^m-1$ divides $2$, so $2^m-1\in\{1,2\}$ and $m=1$.
Then $p^a=3$, giving $n=2\cdot 3=6$. ∎

Lemma 13 (Case $m=1$ with two odd primes).
If $n=2\cdot 3^a q^b$ with $q$ an odd prime $q\ne 3$ is unitary perfect, then $n=90$.

Proof.
Lemma 10 with $m=1$ gives
\[
\left(1+\frac{1}{3^a}\right)\left(1+\frac{1}{q^b}\right)=\frac{4}{3},
\]
so
\[
(3^a+1)(q^b+1)=4\cdot 3^{a-1} q^b. \tag{*}
\]
Since $\gcd(3^a+1,3)=1$, the factor $3^a+1$ divides $4q^b$, hence
$3^a+1=2^t q^b$ with $t\in\{1,2\}$ (because $3^a+1$ is even).
Then $(*)$ becomes $2^t(q^b+1)=4\cdot 3^{a-1}$, so
$q^b+1=2^{2-t}3^{a-1}$.

If $a$ is even, then $3^a+1\equiv 2\pmod{4}$, so $t=1$ and
$(3^a+1)/2=q^b=2\cdot 3^{a-1}-1$.
Equating gives $3^a=9$, hence $a=2$ and $q^b=5$.

If $a$ is odd, then $3^a+1\equiv 0\pmod{4}$, so $t=2$ and
$(3^a+1)/4=q^b=3^{a-1}-1$, which forces $3^a=15$, impossible.
Thus $a=2$, $q=5$, $b=1$, and $n=2\cdot 3^2\cdot 5=90$. ∎

Lemma 14 (Case $m=2$ with two odd primes).
If $n=2^2\cdot 5^a q^b$ with $q$ an odd prime $q\ne 5$ is unitary perfect, then $n=60$.

Proof.
By Lemma 11, $1+2^2=5$ divides the odd part, so one odd prime is $5$.
Lemma 10 with $m=2$ gives
\[
\left(1+\frac{1}{5^a}\right)\left(1+\frac{1}{q^b}\right)=\frac{8}{5},
\]
hence
\[
(5^a+1)(q^b+1)=8\cdot 5^{a-1} q^b. \tag{**}
\]
Since $\gcd(5^a+1,5)=1$, we have $5^a+1\mid 8q^b$, so
$5^a+1=2^t q^b$ for some $t\le 3$. But $5^a\equiv 1\pmod{4}$, so
$5^a+1\equiv 2\pmod{4}$ and $t=1$. Then $(**)$ gives
$2(q^b+1)=8\cdot 5^{a-1}$, so $q^b+1=4\cdot 5^{a-1}$.
Equating $q^b=(5^a+1)/2$ yields $5^a=5$, hence $a=1$ and $q^b=3$.
Thus $n=2^2\cdot 5\cdot 3=60$. ∎

Lemma 15 (Case $m=3$ with two odd primes).
If $n=2^3\cdot 3^a q^b$ with $q$ an odd prime $q\ne 3$ is unitary perfect, then no such $n$ exists.

Proof.
Lemma 10 with $m=3$ gives
\[
\left(1+\frac{1}{3^a}\right)\left(1+\frac{1}{q^b}\right)=\frac{16}{9},
\]
so
\[
(3^a+1)(q^b+1)=16\cdot 3^{a-2} q^b. \tag{***}
\]
If $a=1$, then the right-hand side equals $16q^b/3$, which is not an integer unless $q=3$,
contrary to $q\ne 3$. Hence $a\ge 2$.

Since $\gcd(3^a+1,3)=1$, the factor $3^a+1$ divides $16q^b$, so
$3^a+1=2^t q^b$ for some $t\le 4$. Cancelling $q^b$ in $(***)$ yields
$2^t(q^b+1)=16\cdot 3^{a-2}$, so $q^b+1=2^{4-t}3^{a-2}$.
Substituting back gives
\[
3^a+1=2^t(2^{4-t}3^{a-2}-1)=16\cdot 3^{a-2}-2^t,
\]
so
\[
2^t+1=7\cdot 3^{a-2}. \tag{****}
\]
If $a$ is even, then $3^{a-2}\equiv 1\pmod{8}$ and the right-hand side is $7\pmod{8}$,
but $2^t+1\equiv 1,3,5\pmod{8}$, impossible. If $a$ is odd, then $3^{a-2}\equiv 3\pmod{8}$
so the right-hand side is $5\pmod{8}$, forcing $2^t+1\equiv 5\pmod{8}$ and thus $t=2$.
Then $(****)$ gives $7\cdot 3^{a-2}=5$, impossible. Therefore no such $n$ exists. ∎

Lemma 16 (Case $m=4$ with two odd primes).
If $n=2^4\cdot 17^a q^b$ with $q$ an odd prime $q\ne 17$ is unitary perfect, then no such $n$ exists.

Proof.
Lemma 10 with $m=4$ gives
\[
\left(1+\frac{1}{17^a}\right)\left(1+\frac{1}{q^b}\right)=\frac{32}{17},
\]
so
\[
(17^a+1)(q^b+1)=32\cdot 17^{a-1} q^b. \tag{*****}
\]
Since $\gcd(17^a+1,17)=1$, we have $17^a+1\mid 32 q^b$, hence
$17^a+1=2^t q^b$ for some $t\le 5$. Cancelling $q^b$ in $(*****)$ yields
$2^t(q^b+1)=32\cdot 17^{a-1}$, so $q^b+1=2^{5-t}17^{a-1}$.
Substituting back gives
\[
17^a+1=2^t(2^{5-t}17^{a-1}-1)=32\cdot 17^{a-1}-2^t,
\]
so
\[
2^t+1=15\cdot 17^{a-1}. \tag{******}
\]
Since $17^{a-1}\equiv 1\pmod{8}$, the right-hand side is $7\pmod{8}$, while
$2^t+1\equiv 1,3,5\pmod{8}$ for $t\ge 1$. This is impossible, so no such $n$ exists. ∎

Example (checked by hand):
For $n=6$, the unitary divisors are $1,2,3,6$; the proper sum is $1+2+3=6$, hence $6$ is unitary perfect.
This uses no computation beyond listing divisors.

5) VERIFICATION
Edge cases:
- $n=1$: $U(1)=\{1\}$, proper sum $=0\ne 1$, so $1$ is not unitary perfect.
- $n=p$ prime: Corollary 3 applies (proper sum $=1$).
- $n=6$: verified above as unitary perfect.

Quantifiers:
All lemmas apply to $n\ge 1$ with explicit prime factorizations where used.
Lemma 6 relies only on Lemma 5 and parity, Lemma 7 uses only algebra on two primes, and
Lemmas 8--16 follow from Lemma 5 plus basic properties of $\nu_2$, parity, and coprime factorizations.

No outside theorems beyond unique factorization and basic divisor facts are used.

6) FINAL
**UNRESOLVED**

(i) Strongest fully proved partial result:
No odd unitary perfect numbers exist (Lemma 6), no prime power is unitary perfect (Corollary 3),
and if a unitary perfect number has exactly two distinct primes then it is $6$ (Lemma 7).
We also have structural constraints from Lemmas 9--11 (2-adic budget, normalized product identity,
and $1+2^m$ dividing the odd part), and explicit classifications in small cases:
if $n=2^m p^a$ then $n=6$ (Lemma 12), if $m=1$ and there are exactly two odd primes then $n=90$
(Lemma 13), if $m=2$ and there are exactly two odd primes then $n=60$ (Lemma 14), and
there are no solutions with exactly two odd primes when $m=3$ or $m=4$ (Lemmas 15--16).

(ii) First gap:
A global bound on either the number of distinct odd prime factors of a unitary perfect number or its
$2$-adic valuation; such a bound would imply finiteness via known strategies and is not achieved here.

(iii) Top 3 next moves:
1. Prove a uniform bound on the number of distinct odd prime factors of a unitary perfect number.
2. Prove a uniform bound on $\nu_2(n)$ for unitary perfect $n$.
3. Extend the fixed-$m$ two-odd-prime analysis beyond $m=4$ using Lemma 11 and modular constraints.

(iv) Minimal counterexample structure (if finiteness is false):
An infinite sequence of even unitary perfect numbers with growing numbers of distinct odd primes
and rapidly increasing $2$-adic valuation, consistent with the product formula constraints.
