% Erdos problem #564
%
FORMAL RESTATEMENT

Let $R_3(n)$ be the least $m$ such that every red/blue colouring of the edges of the complete $3$-uniform hypergraph on $m$ vertices contains a monochromatic copy of the complete $3$-uniform hypergraph on $n$ vertices.
The question asks whether there exists an absolute constant $c>0$ such that
\[
R_3(n) \ge 2^{2^{c n}} \qquad\text{for all sufficiently large }n.
\]

QUICK LITERATURE/CONTEXT CHECK

The file states that Erd\H{o}s--Hajnal--Rado (1965) proved bounds
$2^{c n^2} < R_3(n) < 2^{2^{n}}$ for some constant $c>0$.
I will not use any additional results beyond what is proved here.

ATTACK PLAN

Proof track:
Use the stepping-down lemma from Problem 562 to re-derive a double-exponential upper bound $R_3(n)\le 2^{2^{O(n)}}$.

Disproof/construction track:
Use the probabilistic method to obtain the classical lower bound $R_3(n) > 2^{c n^2}$.
Compare this with the conjectured $2^{2^{c n}}$ lower bound; note the gap is exactly the open difficulty.

WORK

Lemma 564.1 (double-exponential upper bound, proved from stepping-down).
There exists an absolute constant $C>0$ such that for all $n\ge 3$,
\[
R_3(n) \le 2^{2^{C n}}.
\]

Proof.
Apply Corollary 562.2 with $r=3$. Then with $t=R_2(n-1)$,
\[
R_3(n) \le (t+1) 2^{\binom{t}{2}} + 1 \le 2^{t} 2^{t^2/2} 2 = 2^{t^2/2+t+1}.
\]
Using the elementary graph Ramsey upper bound $R_2(n-1)\le 4^{n-1}=2^{2n-2}$,
we have $t\le 2^{2n}$ for $n\ge 2$.
Therefore $t^2/2+t+1 \le 2^{4n}$ for all $n$ large enough, hence
\[
R_3(n) \le 2^{2^{4n}}.
\]
This is of the claimed form with $C=4$. \hfill$\square$

Lemma 564.2 (probabilistic lower bound).
There exists an absolute constant $c>0$ such that for all $n\ge 3$,
\[
R_3(n) > 2^{c n^2}.
\]

Proof.
This is the case $r=3$ of Lemma 562.4. For completeness, set $m=\lfloor 2^{c n^2}\rfloor$.
For a random $2$-colouring of triples on $m$ vertices, the expected number of monochromatic $n$-sets is
$\binom{m}{n}2^{1-\binom{n}{3}}$.
Using $\binom{m}{n}\le (em/n)^n$ and $\binom{n}{3}\ge n^3/48$ for $n\ge 6$, one checks that for sufficiently small fixed $c>0$ this expectation is $<1$ for all large $n$.
Hence some colouring has no monochromatic $K_n^{(3)}$, so $R_3(n)>m$. \hfill$\square$

VERIFICATION

(1) The proof of Lemma 564.1 only uses inequalities valid for $n\ge 3$; finitely many small $n$ are irrelevant to the asymptotic.

(2) Fast reality check by computation (first-moment threshold for $r=3$):
Largest $m$ such that $\binom{m}{n}2^{1-\binom{n}{3}}<1$ is
$m=5$ for $n=4$, $m=11$ for $n=5$, $m=29$ for $n=6$, $m=100$ for $n=7$, $m=445$ for $n=8$.
These are consistent with the proved lower bound form $2^{c n^2}$ but far below the conjectured $2^{2^{c n}}$.

FINAL

UNRESOLVED

(i) Strongest proved partial result here: there exist absolute constants $c,C>0$ such that
\[
2^{c n^2} < R_3(n) \le 2^{2^{C n}}.
\]
(The upper bound is re-derived via the stepping-down lemma; the lower bound via the probabilistic method.)

(ii) First gap (crisp): prove or disprove the existence of $c>0$ with
\[
R_3(n) \ge 2^{2^{c n}} \quad\text{for all large }n.
\]

(iii) Top 3 next moves:
(1) Attempt explicit algebraic or geometric $2$-colourings of triples with $m=2^{2^{c n}}$ and prove they avoid monochromatic $K_n^{(3)}$.
(2) Strengthen the random-colouring argument using dependencies/containers to push beyond $2^{c n^2}$.
(3) Computationally: for small $n$ (say $n=5,6,7$), search for best known lower bounds by SAT/ILP to identify patterns of extremal colourings.

(iv) Minimal counterexample structure: if the conjectured bound fails, then for every $c>0$ there are arbitrarily large $n$ with a $2$-colouring of triples on $m=2^{2^{c n}}$ vertices having no monochromatic $K_n^{(3)}$.
Such a construction would have to exhibit long-range structure well beyond what random colourings provide (since random colourings already fail around $m=2^{O(n^2)}$).


