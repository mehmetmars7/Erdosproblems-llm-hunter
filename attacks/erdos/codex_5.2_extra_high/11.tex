% Attempt for Erdos Problem #11
% Following PROMPT_STRATEGY.MD
% Tools/Constraints:
% - Web browsing available? NO (restricted)
% - Computation available? YES (local Python), not used
% Codex 5.2 extra high

OUTPUT
1) FORMAL RESTATEMENT
A positive integer is squarefree if it is not divisible by $p^2$ for any prime $p$.
Question: For every odd integer $n$, does there exist $a\ge 0$ and a squarefree integer $s$ such that
\[n=s+2^a?\]

2) QUICK LITERATURE/CONTEXT CHECK
Browsing is not available. I only record what the problem statement itself claims.
I have not verified those results here.

3) ATTACK PLAN
Proof track:
1. Use sieve methods to show that for each odd $n$, some $n-2^a$ avoids square factors.
2. Relate to distribution of squarefree numbers in arithmetic progressions $n-2^a$.

Disproof track:
1. Try to construct an odd integer $n$ such that every $n-2^a$ is divisible by a square.
2. Study congruence obstructions mod $p^2$ simultaneously for many primes.

Chosen path: establish elementary infinite families of representable odd integers.

4) WORK
Lemma 1 (Immediate representation via $2^0$).
If $n$ is odd and $n-1$ is squarefree, then $n$ is representable.

Proof.
Take $a=0$ so $2^a=1$ and set $s=n-1$. Then $n=s+1$ with $s$ squarefree by hypothesis. \qed

Lemma 2 (Infinitely many representable odd integers).
There are infinitely many odd integers $n$ that can be written as a squarefree number plus a power of $2$.

Proof.
There are infinitely many odd squarefree numbers, e.g., all odd primes.
Let $s$ be any even squarefree number; then $n=s+1$ is odd and representable by Lemma 1.
There are infinitely many even squarefree numbers of the form $2q$ with $q$ an odd prime. \qed

Lemma 3 (Trivial inclusion for any $a$).
If $n-2^a$ is squarefree for some $a\ge 0$, then $n$ is representable.

Proof.
This is the defining condition. \qed

5) VERIFICATION
- Lemma 2 uses the infinitude of primes to produce infinitely many squarefree numbers.
- Lemma 3 is tautological but correctly states the sufficient condition.

6) FINAL
**UNRESOLVED**

(i) Strongest fully proved partial result:
Infinitely many odd integers are representable (Lemma 2).

(ii) First gap:
No method here guarantees representation for every odd integer.

(iii) Top 3 next moves:
1. Apply a sieve to show that for each odd $n$ there exists $a$ with $n-2^a$ squarefree.
2. Study obstructions modulo $p^2$ for multiple primes simultaneously.
3. Use probabilistic heuristics to estimate the density of squarefree values in the set $\{n-2^a\}$.

(iv) Minimal counterexample structure (if the conjecture were false):
An odd integer $n$ such that for every $a\ge 0$, the number $n-2^a$ is divisible by some square $p^2$,
implying a system of congruences $n\equiv 2^a \pmod{p^2}$ covering all $a$.
