% Erdos Problem #809
%
\noindent\textbf{1) FORMAL RESTATEMENT.}

Fix an integer $k\ge 3$. For each integer $n\ge 1$, let
\[
 m(n):=\left\lfloor \frac{n^2}{4}\right\rfloor+1.
\]
Define $F_k(n)$ to be the minimal integer $r\ge 1$ such that there exist
\begin{itemize}
\item a simple graph $G$ on $n$ vertices with exactly $m(n)$ edges, and
\item an edge-coloring $\chi: E(G)\to [r]:=\{1,\dots,r\}$
\end{itemize}
with the property:

\emph{(Rainbow $C_{2k+1}$ property)} For every subgraph $H\subseteq G$ isomorphic to the cycle $C_{2k+1}$, the $2k+1$ edges of $H$ receive pairwise distinct colors under $\chi$.

Question: Is it true that
\[
F_k(n)\sim \frac{n^2}{8}\qquad (n\to\infty)?
\]

\medskip
\noindent\textbf{2) QUICK LITERATURE/CONTEXT CHECK.}

The problem text states that Burr--Erd\H{o}s--Graham--S\'os proved $F_k(n)\gg n^2$ (no proof given here). I do not use any additional literature.

\medskip
\noindent\textbf{3) ATTACK PLAN.}

\emph{Proof (upper bound) strategies.}
\begin{itemize}
\item Construct a specific $n$-vertex graph with $m(n)$ edges and an explicit coloring using about $n^2/8$ colors such that every $C_{2k+1}$ is rainbow.
\item Start from an extremal bipartite graph $K_{\lfloor n/2\rfloor,\lceil n/2\rceil}$ (which has $\lfloor n^2/4\rfloor$ edges) and add one edge; understand the resulting family of $C_{2k+1}$'s and design a near-optimal coloring for those cycles.
\end{itemize}

\emph{Disproof/lower bound strategies.}
\begin{itemize}
\item Prove that in any such $G$ with $m(n)$ edges, any coloring with $o(n^2)$ colors forces a repeated color on some $C_{2k+1}$ by a counting/double-counting argument over cycles.
\item Seek a graph $G$ with $m(n)$ edges and relatively few (or specially structured) $C_{2k+1}$'s, potentially allowing a coloring with $o(n^2)$ colors, which would refute the conjectured constant $1/8$.
\end{itemize}

I did not obtain a proof or counterexample; below are fully proved elementary bounds and structural observations.

\medskip
\noindent\textbf{4) WORK.}

\textbf{Lemma 809.1 (trivial universal upper bound).}
For every $k\ge 3$ and $n\ge 1$,
\[
F_k(n)\le m(n)=\left\lfloor \frac{n^2}{4}\right\rfloor+1.
\]

\emph{Proof.}
Take any graph $G$ on $n$ vertices with $m(n)$ edges (such graphs exist for all $n\ge 3$). Color every edge with a distinct color. Then every subgraph, in particular every $C_{2k+1}$, has all edges different colors. This uses exactly $m(n)$ colors, so $F_k(n)\le m(n)$.
\qed

\textbf{Lemma 809.2 (odd cycles in a near-bipartite extremal graph).}
Let $a,b\ge 1$ and let $G$ be obtained from the complete bipartite graph $K_{a,b}$ (with bipartition $A\cup B$) by adding one extra edge $uv$ inside $A$.
Then for every integer $j$ with
\[
1\le j\le \min\{a-1,b\},
\]
$G$ contains a cycle of length $2j+1$.

\emph{Proof.}
Fix such a $j$. Choose distinct vertices
\[
 u=v_0,\ v=v_{2j}\in A,
\]
and choose further distinct vertices $v_2,v_4,\dots,v_{2j-2}\in A\setminus\{u,v\}$ (this is possible because $A$ has size $a$ and we need $j-1\le a-2$ additional vertices, i.e. $j\le a-1$).
Also choose distinct vertices $v_1,v_3,\dots,v_{2j-1}\in B$ (possible because we need $j\le b$ vertices in $B$).

Now consider the cyclic sequence
\[
 v_0(=u),\ v_1,\ v_2,\ v_3,\dots,\ v_{2j-1},\ v_{2j}(=v),\ v_0.
\]
For each even index $2i$, $v_{2i}\in A$ and $v_{2i+1}\in B$, so the edge $v_{2i}v_{2i+1}$ lies in $K_{a,b}$. Similarly $v_{2i+1}v_{2i+2}$ lies in $K_{a,b}$.
Finally, the closing edge $v_{2j}v_0$ is exactly the added edge $uv$ inside $A$.
All vertices in the sequence are distinct by construction, so these edges form a simple cycle of length $2j+1$.
\qed

\textbf{FAST REALITY CHECK (tiny instances by computation).}

Two basic sanity checks:
\begin{itemize}
\item If $n<2k+1$ then no graph on $n$ vertices contains a $C_{2k+1}$, so the rainbow-cycle condition is vacuous and $F_k(n)=1$.
\item Even when $n=2k+1$, it can happen that $F_k(n)=1$ because one can choose $G$ with $m(n)$ edges but no $C_{2k+1}$.
\end{itemize}

Concrete example for $k=3$ and $n=7$: here $m(7)=\lfloor 49/4\rfloor+1=13$.
Let $G$ be the disjoint union of an isolated vertex and a $6$-vertex graph formed from $K_6$ by deleting two edges. Then $|E(G)|=13$ and $G$ has no $C_7$ (because every component has at most $6$ vertices).
A direct programmatic check confirms ``has $C_7$? False'' for this $G$. Therefore $F_3(7)=1$.

\medskip
\noindent\textbf{5) VERIFICATION.}

\begin{itemize}
\item Lemma 809.1 is immediate; unique-edge coloring always enforces rainbow-ness on any fixed cycle.
\item Lemma 809.2: verified that the constructed cycle uses only edges from $K_{a,b}$ except for the single added intra-$A$ edge, and that all vertices are distinct.
\item The $k=3,n=7$ sanity-check graph is disconnected with maximum component size $6$, so it cannot contain a $7$-cycle; the computation confirms this.
\end{itemize}

\medskip
\noindent\textbf{6) FINAL.} \textbf{UNRESOLVED.}

\begin{enumerate}
\item[(i)] \emph{Strongest proved partial result here:} the trivial upper bound $F_k(n)\le \lfloor n^2/4\rfloor+1$ (Lemma 809.1) and a structural fact that the natural extremal ``$K_{\lfloor n/2\rfloor,\lceil n/2\rceil}$ plus one edge'' construction contains many odd cycles of all lengths $3,5,\dots,2\min\{\lfloor n/2\rfloor-1,\lceil n/2\rceil\}+1$ (Lemma 809.2).
\item[(ii)] \emph{First gap (crisp):} either construct, for each large $n$, a graph with $m(n)$ edges admitting a rainbow-$C_{2k+1}$ edge-coloring using $(1/8+o(1))n^2$ colors, or prove that any such coloring requires at least $(1/8-o(1))n^2$ colors.
\item[(iii)] \emph{Top 3 next moves:}
  \begin{enumerate}
  \item For the candidate graph $K_{\lfloor n/2\rfloor,\lceil n/2\rceil}$ plus one intra-part edge, explicitly characterize all $C_{2k+1}$ cycles and attempt to color the bipartite edges so that no two edges of the same color can lie on any such cycle.
  \item Prove a lower bound by counting $C_{2k+1}$'s and showing that if a color class contains too many edges then two of them must co-occur on some $C_{2k+1}$.
  \item Run exact search for the smallest nontrivial instances (e.g. $n=2k+2$ or $2k+3$) using SAT/ILP to determine $F_k(n)$ and to guess extremal structures.
  \end{enumerate}
\item[(iv)] \emph{What a minimal counterexample would likely look like:} if the asymptotic constant $1/8$ is false, then there should exist graphs with $m(n)$ edges for which the family of $C_{2k+1}$'s is structured so that edges can be grouped into color classes of average size substantially larger than $2$ without creating two same-colored edges on a $C_{2k+1}$; equivalently, a graph where the ``co-occurrence graph'' on edges (joining two edges if they lie on a common $C_{2k+1}$) has unusually small chromatic number.
\end{enumerate}


