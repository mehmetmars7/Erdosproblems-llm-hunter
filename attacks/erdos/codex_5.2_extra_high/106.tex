% Erdos Problem #106
% Attempt for Erdos Problem #106
% Following PROMPT_STRATEGY.MD
% Tools/Constraints:
% - Web browsing available? YES (not used; only facts explicitly stated in the problem text)
% - Computation available (Python)? YES (not used)
%

\section*{Erd\H{o}s Problem \#106}

\subsection*{1) FORMAL RESTATEMENT}
Let $Q=[0,1]^2$ be the unit square.  Place $n$ squares $S_1,\dots,S_n\subset Q$ (rotation allowed) such that 
\[
\operatorname{int}(S_i)\cap \operatorname{int}(S_j)=\varnothing\quad (i\neq j).
\]
Write $s_i>0$ for the side length of $S_i$ and define
\[
 f(n):=\sup \Big\{\sum_{i=1}^n s_i:\; S_1,\dots,S_n\subset Q\text{ squares with pairwise disjoint interiors}\Big\}.
\]

\paragraph{Ambiguity check.}
The phrase ``no common interior point'' could mean either
(i) pairwise interior-disjoint (as assumed above), or
(ii) only that $\bigcap_{i=1}^n \operatorname{int}(S_i)=\varnothing$.
The problem text states ``it is trivial from Cauchy--Schwarz that $f(k^2)=k$'', which is true under (i) via the area bound $\sum s_i^2\le 1$, but is false under (ii) (since squares could overlap heavily). Therefore the intended meaning is (i).

\paragraph{Main question.}
For integers $k\ge1$, is it true that
\[
 f(k^2+1)=k\ ?
\]

\subsection*{2) QUICK LITERATURE/CONTEXT CHECK}
I only record what the problem text itself states:
\begin{itemize}
\item Erd\H{o}s proved $f(2)=1$.
\item Newman proved (as reported) that $f(5)=2$.
\item The inequality $f(k^2)=k$ follows from Cauchy--Schwarz.
\item Hal\'asz gives constructions implying various lower bounds for $f(k^2+t)$.
\item A parallel-axis variant $g(n)$ (all squares axis-parallel) is determined exactly in the cited 2024 preprint.
\end{itemize}
No other results are assumed here.

\subsection*{3) ATTACK PLAN}
\begin{itemize}
\item \textbf{Proof track (for $f(k^2+1)=k$):} strengthen the area/Cauchy--Schwarz argument by incorporating an additional geometric constraint (e.g. width, boundary length, or a packing/covering inequality) that rules out total side sum $>k$ when only $k^2+1$ squares are present.
\item \textbf{Disproof track:} try to beat $k$ by a construction that replaces several $1/k$-tiles by many smaller squares whose side lengths sum to more than the removed tile lengths.
\end{itemize}
In this writeup I establish the standard universal bounds and constructions; the conjectured matching upper bound remains the first gap.

\subsection*{4) WORK}
\paragraph{Lemma 106.1 (area bound).}
For any admissible configuration of $n$ squares,
\[
\sum_{i=1}^n s_i^2\le 1.
\]
\emph{Proof.}
The interiors are pairwise disjoint, hence the total area is the sum of the areas:
$\sum_i \mathrm{area}(S_i)=\sum_i s_i^2$.  Since each $S_i\subset Q$, this total area is at most $\mathrm{area}(Q)=1$. \qed

\paragraph{Lemma 106.2 (Cauchy--Schwarz upper bound).}
For all $n\ge1$,
\[
 f(n)\le \sqrt{n}.
\]
\emph{Proof.}
By Cauchy--Schwarz and Lemma~106.1,
\[
\sum_{i=1}^n s_i \le \sqrt{n\sum_{i=1}^n s_i^2}\le \sqrt{n\cdot 1}=\sqrt{n}.
\]
Taking the supremum over all admissible configurations yields $f(n)\le\sqrt{n}$. \qed

\paragraph{Corollary 106.3 (exact value at perfect squares).}
For every integer $k\ge1$,
\[
 f(k^2)=k.
\]
\emph{Proof.}
Lemma~106.2 gives $f(k^2)\le\sqrt{k^2}=k$.  Conversely, partition $Q$ into a $k\times k$ grid of $k^2$ congruent squares of side $1/k$.  Then $\sum s_i = k^2\cdot (1/k)=k$, showing $f(k^2)\ge k$. \qed

\paragraph{Lemma 106.4 (standard lower bound for $k^2+1$).}
For every integer $k\ge1$,
\[
 f(k^2+1)\ge k.
\]
\emph{Proof.}
Partition $Q$ into a $k\times k$ grid of $k^2$ tiles of side $1/k$.  Keep $k^2-1$ of these tiles as squares of side $1/k$.  In the remaining tile (a $1/k\times 1/k$ square), place two disjoint squares of side $1/(2k)$, e.g. one in the lower-left corner and one in the upper-right corner of that tile.  This gives $(k^2-1)+2=k^2+1$ squares with total side sum
\[
(k^2-1)\cdot\frac1k + 2\cdot\frac1{2k}=\frac{k^2-1}{k}+\frac1k = k.
\]
Hence $f(k^2+1)\ge k$. \qed

\paragraph{Lemma 106.5 (monotonicity of $f$).}
The function $f(n)$ is nondecreasing in $n$: for every $n\ge1$,
\[
 f(n+1)\ge f(n).
\]
\emph{Proof.}
Fix $\varepsilon>0$ and choose an admissible configuration of $n$ squares with total side sum $>f(n)-\varepsilon$.  Shrink each square about its center by a factor $(1-\delta)$, where $\delta>0$ is small enough that the new total side sum is $>(f(n)-\varepsilon)-\varepsilon$.
Shrinking preserves interior-disjointness and keeps all squares inside $Q$.  The total area strictly decreases, so the complement of the union of the shrunken closed squares is a nonempty open subset of $Q$.  Any nonempty open subset of $\mathbb R^2$ contains a (small) square; place an additional square $S_{n+1}$ in that open region.  The resulting $(n+1)$-square configuration has total side sum $>f(n)-2\varepsilon$.  Letting $\varepsilon\to0$ yields $f(n+1)\ge f(n)$. \qed

\subsection*{5) VERIFICATION (FAST REALITY CHECK)}
\begin{itemize}
\item $n=1$: clearly $f(1)=1$ (take $S_1=Q$). Lemma~106.2 gives $f(1)\le1$.
\item $n=2$: Lemma~106.2 gives $f(2)\le\sqrt2\approx1.414$; the problem text states Erd\H{o}s proved $f(2)=1$.
\item $n=4$ ($k=2$): Corollary~106.3 gives $f(4)=2$.
\item $n=5$ ($k=2$): Lemma~106.4 gives $f(5)\ge2$; the problem text states $f(5)=2$.
\end{itemize}
All inequalities above are consistent with these small cases.

\subsection*{6) FINAL}
\textbf{UNRESOLVED.}

(i) \emph{Strongest fully proved partial result obtained here.}
For all $n$, $f(n)\le\sqrt{n}$ (Lemma~106.2); for all $k$, $f(k^2)=k$ (Corollary~106.3) and $f(k^2+1)\ge k$ (Lemma~106.4).

(ii) \emph{Exact first gap.}
To prove $f(k^2+1)=k$, it suffices to prove the missing upper bound
\[
 f(k^2+1)\le k\qquad(k\ge1),
\]
which is stronger than the universal bound $f(k^2+1)\le\sqrt{k^2+1}$.

(iii) \emph{Top 3 next moves (concrete targets).}
\begin{enumerate}
\item Prove a geometric inequality upgrading $\sum s_i^2\le1$ by adding a second constraint (e.g. involving total boundary length or projection widths) that forces $\sum s_i\le k$ when $n\le k^2+1$.
\item Classify ``near-extremal'' configurations for $f(k^2)$ and show any attempt to insert one more square forces a loss in total side sum.
\item Search (by computer) for $k=2,3$ for configurations of $k^2+1$ squares with sum $>k$ to test plausibility and identify extremal patterns.
\end{enumerate}

(iv) \emph{Minimal counterexample structure.}
A counterexample for some $k$ would be a packing of $k^2+1$ interior-disjoint squares in $Q$ with $\sum s_i>k$.  Such a configuration cannot rely on many extremely tiny squares (which add little to the sum) and would need a highly non-grid-like arrangement of moderately sized squares.


