% Attempt for Erdos Problem #1
% Following PROMPT_STRATEGY.MD
% Tools/Constraints:
% - Web browsing available? NO (restricted)
% - Computation available? YES (local Python), not used
% Codex 5.2 extra high

OUTPUT
1) FORMAL RESTATEMENT
Let $n\ge 1$ and let $A\subseteq\{1,\dots,N\}$ with $|A|=n$. For each subset $S\subseteq A$ define
\[\sigma(S)=\sum_{a\in S} a.\]
Assume the map $S\mapsto \sigma(S)$ is injective on $\mathcal P(A)$ (all $2^n$ subset sums are distinct).
Conjecture: there exist absolute constants $c>0$ and $n_0$ such that for all $n\ge n_0$ and all such $A$,
\[N\ge c\,2^n.\]

2) QUICK LITERATURE/CONTEXT CHECK
Browsing is not available. I only record what the problem statement itself claims:
Erd\H{o}s--Moser proved $N\ge (1/4-o(1))2^n/\sqrt n$, later improved to the exact bound
$N\ge \binom{n}{\lfloor n/2\rfloor}$. I have not verified these results here.

3) ATTACK PLAN
Proof track:
1. Use additive combinatorics/entropy to show most subset sums concentrate, forcing $N$ large.
2. Apply Sperner-type or isoperimetric bounds on the Boolean cube after mapping sums to $[0,nN]$.
3. Investigate stability of the powers-of-two construction to identify extremal structure.

Disproof track:
1. Search for constructions with $N=o(2^n)$ and distinct subset sums (e.g., mixed radix).
2. Attempt probabilistic constructions with large gaps between elements.

Chosen path: provide elementary lower bounds and constructions only.

4) WORK
Lemma 1 (Trivial lower bound).
If all subset sums of $A\subseteq\{1,\dots,N\}$ are distinct and $|A|=n$, then
\[N\ge \frac{2^n-1}{n}.\]

Proof.
There are $2^n$ distinct subset sums. Each sum lies in $[0, nN]$ since all elements are at most $N$.
Thus $nN+1\ge 2^n$, giving $N\ge (2^n-1)/n$. \qed

Lemma 2 (Powers of two construction).
Let $A=\{1,2,4,\dots,2^{n-1}\}$. Then all subset sums are distinct and $A\subseteq\{1,\dots,2^{n-1}\}$.

Proof.
Every subset sum of $A$ has a unique binary expansion using digits $0$ or $1$, hence distinct.
The largest element is $2^{n-1}$, so $A\subseteq\{1,\dots,2^{n-1}\}$. \qed

Corollary 3 (Order of magnitude is tight up to constants).
There exist examples with $N=2^{n-1}$, so any universal lower bound must be of the form
$N\ge c 2^n$ with $c\le 1/2$.

5) VERIFICATION
- Lemma 1 uses only the bound $\max \sigma(S)\le nN$ and injectivity.
- Lemma 2 uses uniqueness of binary expansions.
- Edge cases: $n=1$ gives $N\ge 1$; the construction gives $A=\{1\}$.

6) FINAL
**UNRESOLVED**

(i) Strongest fully proved partial result:
$N\ge (2^n-1)/n$, and the powers-of-two example shows $N$ can be as small as $2^{n-1}$.

(ii) First gap:
No method here improves the trivial factor $1/n$ in the lower bound toward a constant.

(iii) Top 3 next moves:
1. Implement the Dubroff--Fox--Xu argument in a self-contained way (likely via compression on the Boolean cube).
2. Prove a sharp inequality on the distribution of subset sums (entropy or rearrangement approach).
3. Analyze extremal structure to show near-equality forces a near-powers-of-two set.

(iv) Minimal counterexample structure (if the conjecture were false):
A family of sets $A_n$ with $|A_n|=n$ and $A_n\subseteq\{1,\dots,o(2^n)\}$ but still having all subset sums distinct;
such a family would likely require highly sparse, rapidly growing elements with delicate additive structure.
