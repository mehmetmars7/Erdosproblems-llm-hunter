
1) FORMAL RESTATEMENT

Fix a (simple, undirected) finite graph $G$. For two graphs $A,B$, write $R(A,B)$ for the least integer $N$ such that every red/blue colouring of the edges of the complete graph $K_N$ contains either a red copy of $A$ or a blue copy of $B$.

Assume that $G$ satisfies the following two growth conditions.

(A) There exists a constant $c_T(G)>0$ such that for every $n\ge 1$ and every tree $T_n$ on $n$ vertices,
\[
R(G,T_n) \le c_T(G)\,n.
\]

(B) There exists a constant $c_K(G)>0$ such that for every $n\ge 1$,
\[
R(G,K_n) \le c_K(G)\,n^2.
\]

Question: Does there exist a constant $c(G)>0$ such that for every graph $H$ with $m=e(H)$ edges and with no isolated vertices,
\[
R(G,H) \le c(G)\,m?
\]
(Equivalently: under (A) and (B), is $G$ \emph{Ramsey size linear}?)

2) QUICK LITERATURE/CONTEXT CHECK

Only the definitions and claims explicitly present in the provided problem file are treated as ``given''. (No external theorems are invoked as black boxes.)

3) ATTACK PLAN

If the statement is true, a natural strategy is:

- Use (A) to deduce that in any colouring of $K_N$ with no red $G$, the blue graph is ``tree-universal'' up to order about $N/c_T(G)$.

- Use (B) to deduce that in any colouring of $K_N$ with no red $G$, the blue graph contains a clique of order about $\sqrt{N/c_K(G)}$.

- Try to embed a general $m$-edge graph $H$ into such a blue graph using a decomposition of $H$ into a tree-like backbone plus additional edges that can be routed inside the forced clique.

If the statement is false, one should look for a graph $G$ satisfying (A) and (B) but for which there exists a sparse (many-vertex) graph $H$ with $m$ edges and $R(G,H)\gg m$.

4) WORK

\emph{Fast reality check / sanity checks.}

- If $G=K_2$ (a single edge), then $R(K_2,T_n)=n$ for every tree $T_n$, and $R(K_2,K_n)=n$. For any $H$ with no isolated vertices and $m=e(H)$, we have
\[
R(K_2,H)=|V(H)|\le 2m
\]
(see Lemma~568.1 below), so $K_2$ is Ramsey size linear.

- Condition (B) alone implies a quadratic-in-$m$ bound for all $H$ (Lemma~568.2). Condition (A) alone implies a linear-in-$m$ bound for forests (Lemma~568.3). The hard part is controlling arbitrary extra edges beyond a forest.

\medskip

\textbf{Lemma 568.1 (vertex bound from ``no isolated vertices'').}
If $H$ is a graph with $m=e(H)$ edges and no isolated vertices, then
\[
|V(H)| \le 2m.
\]

\textbf{Proof.}
Let $n=|V(H)|$. By the handshake lemma, the sum of degrees is
\[
\sum_{v\in V(H)} \deg_H(v)=2m.
\]
If $H$ has no isolated vertices then $\deg_H(v)\ge 1$ for every $v$, hence
\[
2m=\sum_{v\in V(H)} \deg_H(v) \ge \sum_{v\in V(H)} 1 = n.
\]
So $n\le 2m$ as claimed. \qed

\medskip

\textbf{Lemma 568.2 (quadratic bound from (B)).}
Assume (B): $R(G,K_n)\le c_K(G)n^2$ for all $n$. Then for every graph $H$ with $m=e(H)$ edges and no isolated vertices,
\[
R(G,H) \le 4c_K(G)\,m^2.
\]

\textbf{Proof.}
Let $n=|V(H)|$. Since $H$ is a subgraph of the complete graph $K_n$, any blue copy of $K_n$ contains a blue copy of $H$. Therefore, by monotonicity in the second argument,
\[
R(G,H) \le R(G,K_n).
\]
By (B), $R(G,K_n)\le c_K(G)n^2$. By Lemma~568.1, $n\le 2m$, hence
\[
R(G,H) \le c_K(G)n^2 \le c_K(G)(2m)^2 = 4c_K(G)m^2.
\]
\qed

\medskip

\textbf{Lemma 568.3 (linear bound for forests from (A)).}
Assume (A): $R(G,T_n)\le c_T(G)n$ for all $n$ and all trees $T_n$ on $n$ vertices. Let $F$ be any forest with $m=e(F)$ edges and no isolated vertices. Then
\[
R(G,F) \le 2c_T(G)\,m.
\]

\textbf{Proof.}
Let $n=|V(F)|$. By Lemma~568.1, $n\le 2m$.

We first construct a tree $T$ on the same vertex set $V(F)$ such that $F\subseteq T$ as graphs. Write the components of $F$ as $F_1,\dots,F_r$ with vertex sets $V_1,\dots,V_r$ (so $r\ge 1$ and the $V_i$ partition $V(F)$). For $i=1,\dots,r-1$ choose vertices $x_i\in V_i$ and $y_i\in V_{i+1}$ and add the edge $x_iy_i$. Let $T$ be the graph obtained from $F$ by adding these $r-1$ edges. Then $T$ is connected (it links the components in a chain) and acyclic (adding an edge between distinct components cannot create a cycle). Hence $T$ is a tree on $n$ vertices, and clearly $F\subseteq T$.

Now use monotonicity in the second argument again: since $F\subseteq T$,
\[
R(G,F) \le R(G,T).
\]
By (A), $R(G,T)\le c_T(G)\,n \le c_T(G)\,(2m)=2c_T(G)m$.
\qed

\medskip

\textbf{Remark (cliques also give linear-in-$m$ bounds).}
For $H=K_t$, we have $m=\binom t2$ and (B) gives $R(G,K_t)\le c_K(G)t^2\le 2c_K(G)\,m$ for all $t\ge 2$.

5) VERIFICATION

- Lemma~568.1 uses only the handshake lemma and the assumption ``no isolated vertices''.

- Monotonicity used in Lemma~568.2 and Lemma~568.3: if $H\subseteq H'$, then every blue $H'$ contains a blue $H$, so $R(G,H)\le R(G,H')$.

- Lemma~568.3 explicitly constructs $T$ and checks connectivity and acyclicity.

6) FINAL

**UNRESOLVED**

(i) Strongest proved partial result here: under (B) we always have the quadratic size bound
\[
R(G,H)\le 4c_K(G)m^2\qquad (H\text{ has }m\text{ edges and no isolated vertices}),
\]
and under (A) we have the linear bound $R(G,F)\le 2c_T(G)m$ for every forest $F$ with $m$ edges and no isolated vertices.

(ii) First gap (crisp): show how to use the forced presence of \emph{every} $n$-vertex tree (for $n\asymp N$) together with a forced blue clique of size $\asymp\sqrt N$ to embed an \emph{arbitrary} $m$-edge graph $H$ when $N\asymp m$.

(iii) Top 3 next moves:

1. Prove an embedding lemma of the form: if a graph $B$ on $N$ vertices is universal for all trees on $\ge \alpha N$ vertices and contains a clique of size $\ge \beta\sqrt N$, then $B$ contains every $m$-edge graph $H$ with no isolated vertices for $N\ge C(\alpha,\beta)m$.

2. Try to reduce general $H$ to a bounded number of ``tree + dense core'' pieces (e.g. via finding a small vertex cover or large matching) and show each piece can be embedded using (A) and (B).

3. For a potential disproof, search for a candidate $G$ satisfying (A) and (B) and construct $H$ with $m$ edges but with large unavoidable obstruction to appearing in complements of $G$-free graphs (e.g. $H$ with very large ``tree-universality'' requirements but small $m$).

(iv) Minimal counterexample structure (if false): a fixed graph $G$ satisfying (A) and (B) together with a family $H_m$ with $e(H_m)=m$, no isolated vertices, and colourings of $K_{c m}$ with no red $G$ and no blue $H_m$ for arbitrarily large $m$. Such an $H_m$ would have to avoid the simple ``forest'' and ``clique'' mechanisms captured by Lemmas~568.2--568.3.


