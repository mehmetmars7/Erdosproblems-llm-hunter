% Attempt for Erdős Problem #96
% Following PROMPT_STRATEGY.MD
% Tools/Constraints:
% - Web browsing available? NO (restricted)
% - Computation available? YES (local Python), not used
% Codex 5.2 extra high

OUTPUT
1) FORMAL RESTATEMENT
Let $A \subset \mathbb{R}^2$ be a set of $n$ distinct points in convex position (i.e., the points are the vertices of a
convex $n$-gon; in particular no three are collinear). Let
\[
U(A) := \bigl|\{\{x,y\} \subset A : x \ne y,\ \|x-y\|=1\}\bigr|
\]
be the number of unordered pairs at unit distance. The conjecture is:
there exist constants $C>0$ and $n_0$ such that for all $n \ge n_0$ and all such $A$,
\[
U(A) \le C n.
\]

2) QUICK LITERATURE/CONTEXT CHECK
Browsing is not available. I only record what the problem statement itself claims:
F\"{u}redi proved an $O(n\log n)$ upper bound; Brass--Pach gave a short proof; Aggarwal improved the constant to
$n\log_2 n + 4n$; Edelsbrunner--Hajnal constructed examples with $2n-7$ unit distances.
I have not verified these results here.

3) ATTACK PLAN
Proof track:
1. Use combinatorial geometry on convex $n$-gons to bound repeated distances (incidence bounds on chords).
2. Encode unit distances as edges of a geometric graph; apply crossing-number or forbidden-configuration arguments.
3. Exploit convexity to limit the number of times a fixed distance can occur on parallel diagonals.

Disproof track:
1. Search for explicit convex configurations with superlinear unit distances.
2. Consider cyclic (all points on a circle) configurations with many equal chords.
3. Use perturbations of grids onto convex curves to preserve many unit distances.

Chosen path: No complete proof or counterexample is currently reachable with the tools used here.
I provide fully proved elementary bounds, including a sharp $O(n)$ bound for cyclic polygons.

4) WORK
Lemma 1 (Cyclic case: $U(A)\le n$).
Assume all points of $A$ lie on a common circle. Then $U(A)\le n$.

Proof.
Fix $x\in A$. Points $y\in A$ with $\|x-y\|=1$ lie in the intersection of two circles:
the circumcircle of $A$ and the circle centered at $x$ of radius $1$. These circles are distinct
(since $x$ lies on the circumcircle but is its center only if the radius were $0$), so they
intersect in at most two points. Thus there are at most two such $y$. Hence each vertex has degree at most $2$
in the unit-distance graph on $A$, and
\[
2U(A) = \sum_{x\in A} \deg(x) \le 2n,
\]
which implies $U(A)\le n$. ∎

Lemma 2 (Regular $n$-gon attains $U(A)=n$).
There exists a convex $n$-gon with exactly $n$ unit distances.

Proof.
Take a regular $n$-gon and scale it so its side length is $1$. The distance between vertices
separated by $k$ edges is $2R\sin(k\pi/n)$, where $R$ is the circumradius. This is strictly
increasing in $k$ for $1 \le k \le \lfloor n/2\rfloor$, hence only adjacent vertices are at
distance $1$. Thus $U(A)=n$ (the $n$ edges). ∎

Lemma 3 (Unit-length edges are at most $n$).
Let $A$ be the vertex set of a convex $n$-gon. Then the number of unit-distance pairs that are polygon edges is at most $n$.

Proof.
A convex $n$-gon has exactly $n$ edges. Each edge is a pair of vertices. Thus, at most $n$ unordered pairs can be edges, and
therefore at most $n$ pairs can be unit-length edges. ∎

Lemma 4 (Trivial global bound).
For any $A$ with $|A|=n$, one always has $U(A) \le \binom{n}{2}$.

Proof.
There are only $\binom{n}{2}$ unordered pairs of distinct points in total. ∎

Lemma 5 (Reduction to fixed-distance multiplicity).
Assume there is a function $f(n)$ such that for every convex $n$-gon $A$ and every real $t>0$,
the number of unordered pairs $\{x,y\}\subset A$ with $\|x-y\|=t$ is at most $f(n)$. Then
for every such $A$, $U(A)\le f(n)$.

Proof.
Apply the assumed bound with $t=1$. By definition, $U(A)$ counts the pairs at distance $1$. ∎

Conditional corollary (literature claim, not proved here).
If one accepts the stated best-known bound for a fixed distance in a convex $n$-gon,
namely $f(n)=n\log_2 n+4n$, then $U(A)\le n\log_2 n+4n$.

These bounds are far weaker than the conjectured $O(n)$ total bound for unit distances in general convex position, but they
show the conjecture holds with constant $1$ in the cyclic case and that $U(A)=n$ is achievable.

5) VERIFICATION
Edge cases:
- $n=2$: $U(A)=1$ is possible (two points at distance 1).
- $n=3$: $U(A)$ can be $3$ (equilateral triangle).
- $n=4$: in a square of side 1, $U(A)=4$ (edges only).

Quantifiers:
Lemma 1 assumes all points are cyclic; the two-circle intersection argument bounds degree by $2$ and
the circles are distinct because the vertex is not the circumcenter.
Lemma 2 is constructive for a regular $n$-gon and uses strict monotonicity of chord length by step.
Lemma 3 counts only edges; it does not claim any control on unit diagonals. This is deliberate and correct.
Lemma 4 is purely combinatorial and holds for all finite $A$.
Lemma 5 is a logical reduction; it does not assert the existence of such an $f(n)$.
The conditional corollary depends on the literature claim and is not a proof here.

No unverified theorems are used in the proofs given.

6) FINAL
**UNRESOLVED**

(i) Strongest fully proved partial result:
If all $n$ points lie on a common circle, then $U(A)\le n$ (Lemma 1), and this is sharp because a
regular $n$-gon has $U(A)=n$ (Lemma 2). In general, unit-length edges are at most $n$ (Lemma 3), and
trivially $U(A) \le \binom{n}{2}$ for all $A$ (Lemma 4). The logical reduction from fixed-distance
multiplicity to $U(A)$ is immediate (Lemma 5).

(ii) First gap:
A nontrivial upper bound on unit-distance diagonals in a convex (not necessarily cyclic) $n$-gon; equivalently, a uniform
upper bound on the multiplicity of a fixed distance among vertices of a convex $n$-gon.

(iii) Top 3 next moves:
1. Formalize a bound on the number of equal-length diagonals in a convex polygon (fixed-distance multiplicity).
2. Use crossing-number or forbidden-configuration arguments for unit-distance graphs on convex position.
3. Reproduce an $O(n\log n)$ bound (F\"{u}redi / Brass--Pach) in a self-contained way.

(iv) Minimal counterexample structure (if the conjecture is false):
A sequence of convex $n$-gons with $U(A) \gg n\log n$ (superlinear) unit distances, likely involving many repeated chord
lengths on a cyclic configuration with carefully chosen arc spacings.
