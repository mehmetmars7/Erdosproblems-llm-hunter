
OUTPUT
1) FORMAL RESTATEMENT
Let $p_n$ be the $n$-th prime and define
\[a_n := \frac{p_{n+1}-p_n}{\log n}.\]
Let $S$ be the set of limit points of the real sequence $(a_n)$ in the extended real line $[0,\infty]$.
Question: Is $S=[0,\infty]$? Equivalently, for each $C\ge 0$ does there exist an infinite subsequence $n_i$
with $a_{n_i}\to C$, and is $\infty$ a limit point as well?

2) QUICK LITERATURE/CONTEXT CHECK
Browsing is not available. I only record what the problem statement itself claims about known results on $S$.
I have not verified those results here.

3) ATTACK PLAN
Proof track:
1. Relate $\log n$ and $\log p_n$ to simplify normalization.
2. Use sieve estimates to produce clusters of small gaps and occasional large gaps.

Disproof track:
1. Attempt to show $S$ has gaps by establishing oscillation bounds on $a_n$.

Chosen path: provide elementary structural facts about limit points and normalization.

4) WORK
Lemma 1 (Nonnegativity).
For all $n$, $a_n\ge 0$. Hence $S\subseteq[0,\infty]$.

Proof.
$p_{n+1}>p_n$ so $p_{n+1}-p_n>0$, and $\log n>0$ for $n\ge 2$. \qed

Lemma 2 (Closedness of limit points).
For any real sequence $(x_n)$, the set of limit points is closed in $\mathbb R$ (and in $[0,\infty]$ under the
extended topology).

Proof.
Let $L$ be the set of limit points of $(x_n)$. Suppose $y$ is a limit point of $L$. Then there exists a sequence
$y_m\in L$ with $y_m\to y$. For each $m$, choose a subsequence $(x_{n_{m,j}})_j$ with $x_{n_{m,j}}\to y_m$.
A diagonal argument extracts a subsequence of $(x_n)$ converging to $y$. Hence $y\in L$. \qed

Lemma 3 (Comparison with $\log p_n$).
For all $n\ge 2$,
\[\frac{p_{n+1}-p_n}{\log p_n}\le \frac{p_{n+1}-p_n}{\log n}.\]

Proof.
Since $p_n\ge n+1$, we have $\log p_n\ge \log n$, so dividing by a larger denominator yields the inequality. \qed

5) VERIFICATION
- Lemma 2 uses only subsequences and diagonalization; no prime facts.
- Lemma 3 uses the trivial bound $p_n\ge n+1$.

6) FINAL
**UNRESOLVED**

(i) Strongest fully proved partial result:
The set of limit points $S$ is closed, and the normalized gaps are nonnegative.

(ii) First gap:
No method here proves the existence of any specific nontrivial limit point (e.g., $0$ or $\infty$).

(iii) Top 3 next moves:
1. Incorporate sieve estimates to produce subsequences of small and large prime gaps.
2. Relate normalization by $\log n$ and $\log p_n$ more sharply using explicit bounds for $p_n$.
3. Formalize a density argument for the set of limit points (e.g., positive measure results).

(iv) Minimal counterexample structure (if the conjecture were false):
A scenario where $a_n$ avoids an interval, i.e., the normalized gaps are forced to lie in a closed proper subset
of $[0,\infty]$, requiring systematic constraints on prime gaps not compatible with known heuristics.
